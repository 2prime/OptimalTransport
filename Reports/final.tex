% !TeX encoding = UTF-8 Unicode
% !TeX program = LuaLaTeX
% !TeX spellcheck = LaTeX

% Author : lzh (merge from hjk, jzy)
% Description : Convex Optimization Project Final Report

\documentclass[english]{pkupaper}

\usepackage[paper, si]{def}
\usepackage{pgf}
\usepackage{algorithm}
\usepackage{algorithmic}
\usepackage{multirow}
\usepackage{makecell}

\addbibresource{reference.bib}

\newcommand{\cuniversity}{Peking University}
\newcommand{\cthesisname}{Convex Optimization}
\newcommand{\titlemark}{Convex Optimization Project Final Report}

\begin{document}

\DeclareRobustCommand{\authorthing}{%
\begin{tabular}{ccc}%
侯霁开\thanks{The authors are arranged lexicographically.} & 贾泽宇\thanksmark{1} & 李知含\thanksmark{1}\\%
1600010681 & 1600010603 & 1600010653%
\end{tabular}%
}
\title{\titlemark}
\author{\authorthing}

\maketitle

\section{Introduction}

Optimal transport (OT) problems are an important series of problems considering the minimal cost of transportation, receiving increasing attention from the community of applied mathematics. The discrete form of optimal trnapsort problems fall into a kind of network flow problems, where the graph is restricted to be a bipartite graph and flow restrictions are freed. Cost related to a specific metric, solutions to these problems characterize the deformation between two probability distributions, and therefore the objective, called Wasserstein metric, is useful in many fields including medical image processing (feature identification), geometric learning (registration and segmentation), machine learning (feature extraction and alignment), computer vision (classification), computer graphics (blend between shapes) and even deep learning (generative adversarial networks). However, as a emerging research area, the main challenge for optimal transport problems is a lack of fast and efficient algorithm. Although there are several discrete algorithms \parencite{Bertsekas1992} \parencite{Schrieber2017} for these problems, the vast space of admissible transportation plans introduces great difficulty to computation.

In this report, we explain and report our implementation of several algorithms to discrete optimal transport problems. We include the best algorithms and answers to given questions in this report, and leave some algorithms with deficiency in either efficiency or precision in the supplementary materials. The codes and raw datium, together with the Git repository, are affliated with this report. Please refer to \verb"Readme.md" for more details about the repository.

\section{Problem statement}

The standard formulation of optimal transport are derived from couplings. \parencite{Villani2009} That is, let $ \rbr{ \mathcal{X}, \mu } $ and $ \rbr{ \mathcal{Y}, \nu } $ be two probability spaces, and a probability distribution $\pi$ on $ \mathcal{X} \times \mathcal{Y} $ is called \emph{coupling} if $ \opproj_{\mathcal{X}} \rbr{\pi} = \mu $ and $ \opproj_{\mathcal{Y}} \rbr{\pi} \nu $. An optimal transport between $ \rbr{ \mathcal{X}, \mu } $ and $ \rbr{ \mathcal{Y}, \nu } $, or an optimal coupling, is a coupling minimize
\begin{equation}
\intl{ \mathcal{X} \times \mathcal{Y} }{ c \rbr{ x, y } \sd \pi \rbr{ x, y } }.
\end{equation}

Optimal transport problems can be categorized according to the discreteness of $\mu$ and $\nu$. In this report, we only consider discrete optimal tranport problems, where the two distributions are distributions of finite weighted points.

A discrete optimal transport problem can be formulated into a linear program as
\begin{equation} \label{Eq:StdLP}
\begin{array}{ll}
\mtx{minimize} & \sume{i}{1}{m}{\sume{j}{1}{n}{ c_{ i j } s_{ i j } }}, \\
\mtx{subject to} & \sume{j}{1}{n}{s_{ i j }} = \mu_i, \crbr{ i = 1, 2, \cdots, m } \\
& \sume{i}{1}{n}{s_{ i j }} = \nu_j, \crbr{ j = 1, 2, \cdots n } \\
& s \succeq 0,
\end{array}
\end{equation}
where $c$ stands for the cost and $s$ for the transportation plan, while $\mu$ and $\nu$ are restrictions. Note that we always suppose $ c \succeq 0 $, $ \mu \succeq 0 $, $ \nu \succeq 0 $ and $ \sume{i}{1}{m}{\mu_i} = \sume{j}{1}{n}{\nu_j} = 1 $ implicitly. From realistic background, $c$ is always valued the squared Euclidean distanced or some other norms. Note that there are $ m n $ variables in this formulation, and this leads to intensive computation.

The dual problem of \eqref{Eq:StdLP} can be written as 
\begin{equation} \label{Eq:Dual}
\begin{array}{ll}
\mtx{maximize} & \sume{i}{1}{m}{ \mu_i \lambda_i } + \sume{j}{1}{n}{ \nu_j \eta_j }, \\
\mtx{subject to} & c_{ i j } - \lambda_i - \eta_j \ge 0. \crbr{ i = 1, 2, \cdots, m; j = 1, 2, \cdots, n }
\end{array}
\end{equation}
Although this formulation only emploies $ m + n $ variables, there are still challenges including the recovery of $s$ from $\mu$ and $\nu$ and the great number of constraints.

Figure \ref{Fig:EgDisOT} shows an example of discrete optimal transport, where the dots represents sources and targets, and arrows represents transportations. (Tiny transportations are left out in the figure)

\begin{figure}
\centering \scalebox{0.65}{%% Creator: Matplotlib, PGF backend
%%
%% To include the figure in your LaTeX document, write
%%   \input{<filename>.pgf}
%%
%% Make sure the required packages are loaded in your preamble
%%   \usepackage{pgf}
%%
%% Figures using additional raster images can only be included by \input if
%% they are in the same directory as the main LaTeX file. For loading figures
%% from other directories you can use the `import` package
%%   \usepackage{import}
%% and then include the figures with
%%   \import{<path to file>}{<filename>.pgf}
%%
%% Matplotlib used the following preamble
%%   \usepackage{fontspec}
%%
\begingroup%
\makeatletter%
\begin{pgfpicture}%
\pgfpathrectangle{\pgfpointorigin}{\pgfqpoint{6.400000in}{4.800000in}}%
\pgfusepath{use as bounding box, clip}%
\begin{pgfscope}%
\pgfsetbuttcap%
\pgfsetmiterjoin%
\definecolor{currentfill}{rgb}{1.000000,1.000000,1.000000}%
\pgfsetfillcolor{currentfill}%
\pgfsetlinewidth{0.000000pt}%
\definecolor{currentstroke}{rgb}{1.000000,1.000000,1.000000}%
\pgfsetstrokecolor{currentstroke}%
\pgfsetdash{}{0pt}%
\pgfpathmoveto{\pgfqpoint{0.000000in}{0.000000in}}%
\pgfpathlineto{\pgfqpoint{6.400000in}{0.000000in}}%
\pgfpathlineto{\pgfqpoint{6.400000in}{4.800000in}}%
\pgfpathlineto{\pgfqpoint{0.000000in}{4.800000in}}%
\pgfpathclose%
\pgfusepath{fill}%
\end{pgfscope}%
\begin{pgfscope}%
\pgfsetbuttcap%
\pgfsetmiterjoin%
\definecolor{currentfill}{rgb}{1.000000,1.000000,1.000000}%
\pgfsetfillcolor{currentfill}%
\pgfsetlinewidth{0.000000pt}%
\definecolor{currentstroke}{rgb}{0.000000,0.000000,0.000000}%
\pgfsetstrokecolor{currentstroke}%
\pgfsetstrokeopacity{0.000000}%
\pgfsetdash{}{0pt}%
\pgfpathmoveto{\pgfqpoint{0.800000in}{0.528000in}}%
\pgfpathlineto{\pgfqpoint{4.768000in}{0.528000in}}%
\pgfpathlineto{\pgfqpoint{4.768000in}{4.224000in}}%
\pgfpathlineto{\pgfqpoint{0.800000in}{4.224000in}}%
\pgfpathclose%
\pgfusepath{fill}%
\end{pgfscope}%
\begin{pgfscope}%
\pgfpathrectangle{\pgfqpoint{0.800000in}{0.528000in}}{\pgfqpoint{3.968000in}{3.696000in}} %
\pgfusepath{clip}%
\pgfsetbuttcap%
\pgfsetroundjoin%
\definecolor{currentfill}{rgb}{0.121569,0.466667,0.705882}%
\pgfsetfillcolor{currentfill}%
\pgfsetlinewidth{1.003750pt}%
\definecolor{currentstroke}{rgb}{0.121569,0.466667,0.705882}%
\pgfsetstrokecolor{currentstroke}%
\pgfsetdash{}{0pt}%
\pgfpathmoveto{\pgfqpoint{2.086759in}{1.005673in}}%
\pgfpathcurveto{\pgfqpoint{2.098689in}{1.005673in}}{\pgfqpoint{2.110131in}{1.010412in}}{\pgfqpoint{2.118567in}{1.018848in}}%
\pgfpathcurveto{\pgfqpoint{2.127002in}{1.027283in}}{\pgfqpoint{2.131742in}{1.038726in}}{\pgfqpoint{2.131742in}{1.050655in}}%
\pgfpathcurveto{\pgfqpoint{2.131742in}{1.062585in}}{\pgfqpoint{2.127002in}{1.074027in}}{\pgfqpoint{2.118567in}{1.082463in}}%
\pgfpathcurveto{\pgfqpoint{2.110131in}{1.090898in}}{\pgfqpoint{2.098689in}{1.095638in}}{\pgfqpoint{2.086759in}{1.095638in}}%
\pgfpathcurveto{\pgfqpoint{2.074830in}{1.095638in}}{\pgfqpoint{2.063387in}{1.090898in}}{\pgfqpoint{2.054952in}{1.082463in}}%
\pgfpathcurveto{\pgfqpoint{2.046516in}{1.074027in}}{\pgfqpoint{2.041777in}{1.062585in}}{\pgfqpoint{2.041777in}{1.050655in}}%
\pgfpathcurveto{\pgfqpoint{2.041777in}{1.038726in}}{\pgfqpoint{2.046516in}{1.027283in}}{\pgfqpoint{2.054952in}{1.018848in}}%
\pgfpathcurveto{\pgfqpoint{2.063387in}{1.010412in}}{\pgfqpoint{2.074830in}{1.005673in}}{\pgfqpoint{2.086759in}{1.005673in}}%
\pgfpathclose%
\pgfusepath{stroke,fill}%
\end{pgfscope}%
\begin{pgfscope}%
\pgfpathrectangle{\pgfqpoint{0.800000in}{0.528000in}}{\pgfqpoint{3.968000in}{3.696000in}} %
\pgfusepath{clip}%
\pgfsetbuttcap%
\pgfsetroundjoin%
\definecolor{currentfill}{rgb}{0.121569,0.466667,0.705882}%
\pgfsetfillcolor{currentfill}%
\pgfsetlinewidth{1.003750pt}%
\definecolor{currentstroke}{rgb}{0.121569,0.466667,0.705882}%
\pgfsetstrokecolor{currentstroke}%
\pgfsetdash{}{0pt}%
\pgfpathmoveto{\pgfqpoint{2.328967in}{1.019994in}}%
\pgfpathcurveto{\pgfqpoint{2.337099in}{1.019994in}}{\pgfqpoint{2.344898in}{1.023225in}}{\pgfqpoint{2.350648in}{1.028975in}}%
\pgfpathcurveto{\pgfqpoint{2.356398in}{1.034724in}}{\pgfqpoint{2.359629in}{1.042524in}}{\pgfqpoint{2.359629in}{1.050655in}}%
\pgfpathcurveto{\pgfqpoint{2.359629in}{1.058787in}}{\pgfqpoint{2.356398in}{1.066586in}}{\pgfqpoint{2.350648in}{1.072336in}}%
\pgfpathcurveto{\pgfqpoint{2.344898in}{1.078086in}}{\pgfqpoint{2.337099in}{1.081317in}}{\pgfqpoint{2.328967in}{1.081317in}}%
\pgfpathcurveto{\pgfqpoint{2.320836in}{1.081317in}}{\pgfqpoint{2.313036in}{1.078086in}}{\pgfqpoint{2.307286in}{1.072336in}}%
\pgfpathcurveto{\pgfqpoint{2.301537in}{1.066586in}}{\pgfqpoint{2.298306in}{1.058787in}}{\pgfqpoint{2.298306in}{1.050655in}}%
\pgfpathcurveto{\pgfqpoint{2.298306in}{1.042524in}}{\pgfqpoint{2.301537in}{1.034724in}}{\pgfqpoint{2.307286in}{1.028975in}}%
\pgfpathcurveto{\pgfqpoint{2.313036in}{1.023225in}}{\pgfqpoint{2.320836in}{1.019994in}}{\pgfqpoint{2.328967in}{1.019994in}}%
\pgfpathclose%
\pgfusepath{stroke,fill}%
\end{pgfscope}%
\begin{pgfscope}%
\pgfpathrectangle{\pgfqpoint{0.800000in}{0.528000in}}{\pgfqpoint{3.968000in}{3.696000in}} %
\pgfusepath{clip}%
\pgfsetbuttcap%
\pgfsetroundjoin%
\definecolor{currentfill}{rgb}{0.121569,0.466667,0.705882}%
\pgfsetfillcolor{currentfill}%
\pgfsetlinewidth{1.003750pt}%
\definecolor{currentstroke}{rgb}{0.121569,0.466667,0.705882}%
\pgfsetstrokecolor{currentstroke}%
\pgfsetdash{}{0pt}%
\pgfpathmoveto{\pgfqpoint{2.571175in}{1.023686in}}%
\pgfpathcurveto{\pgfqpoint{2.578328in}{1.023686in}}{\pgfqpoint{2.585188in}{1.026528in}}{\pgfqpoint{2.590245in}{1.031585in}}%
\pgfpathcurveto{\pgfqpoint{2.595303in}{1.036643in}}{\pgfqpoint{2.598144in}{1.043503in}}{\pgfqpoint{2.598144in}{1.050655in}}%
\pgfpathcurveto{\pgfqpoint{2.598144in}{1.057808in}}{\pgfqpoint{2.595303in}{1.064668in}}{\pgfqpoint{2.590245in}{1.069725in}}%
\pgfpathcurveto{\pgfqpoint{2.585188in}{1.074783in}}{\pgfqpoint{2.578328in}{1.077624in}}{\pgfqpoint{2.571175in}{1.077624in}}%
\pgfpathcurveto{\pgfqpoint{2.564023in}{1.077624in}}{\pgfqpoint{2.557163in}{1.074783in}}{\pgfqpoint{2.552105in}{1.069725in}}%
\pgfpathcurveto{\pgfqpoint{2.547048in}{1.064668in}}{\pgfqpoint{2.544206in}{1.057808in}}{\pgfqpoint{2.544206in}{1.050655in}}%
\pgfpathcurveto{\pgfqpoint{2.544206in}{1.043503in}}{\pgfqpoint{2.547048in}{1.036643in}}{\pgfqpoint{2.552105in}{1.031585in}}%
\pgfpathcurveto{\pgfqpoint{2.557163in}{1.026528in}}{\pgfqpoint{2.564023in}{1.023686in}}{\pgfqpoint{2.571175in}{1.023686in}}%
\pgfpathclose%
\pgfusepath{stroke,fill}%
\end{pgfscope}%
\begin{pgfscope}%
\pgfpathrectangle{\pgfqpoint{0.800000in}{0.528000in}}{\pgfqpoint{3.968000in}{3.696000in}} %
\pgfusepath{clip}%
\pgfsetbuttcap%
\pgfsetroundjoin%
\definecolor{currentfill}{rgb}{0.121569,0.466667,0.705882}%
\pgfsetfillcolor{currentfill}%
\pgfsetlinewidth{1.003750pt}%
\definecolor{currentstroke}{rgb}{0.121569,0.466667,0.705882}%
\pgfsetstrokecolor{currentstroke}%
\pgfsetdash{}{0pt}%
\pgfpathmoveto{\pgfqpoint{2.813383in}{1.001523in}}%
\pgfpathcurveto{\pgfqpoint{2.826414in}{1.001523in}}{\pgfqpoint{2.838912in}{1.006700in}}{\pgfqpoint{2.848125in}{1.015914in}}%
\pgfpathcurveto{\pgfqpoint{2.857339in}{1.025127in}}{\pgfqpoint{2.862516in}{1.037625in}}{\pgfqpoint{2.862516in}{1.050655in}}%
\pgfpathcurveto{\pgfqpoint{2.862516in}{1.063685in}}{\pgfqpoint{2.857339in}{1.076184in}}{\pgfqpoint{2.848125in}{1.085397in}}%
\pgfpathcurveto{\pgfqpoint{2.838912in}{1.094611in}}{\pgfqpoint{2.826414in}{1.099788in}}{\pgfqpoint{2.813383in}{1.099788in}}%
\pgfpathcurveto{\pgfqpoint{2.800353in}{1.099788in}}{\pgfqpoint{2.787855in}{1.094611in}}{\pgfqpoint{2.778642in}{1.085397in}}%
\pgfpathcurveto{\pgfqpoint{2.769428in}{1.076184in}}{\pgfqpoint{2.764251in}{1.063685in}}{\pgfqpoint{2.764251in}{1.050655in}}%
\pgfpathcurveto{\pgfqpoint{2.764251in}{1.037625in}}{\pgfqpoint{2.769428in}{1.025127in}}{\pgfqpoint{2.778642in}{1.015914in}}%
\pgfpathcurveto{\pgfqpoint{2.787855in}{1.006700in}}{\pgfqpoint{2.800353in}{1.001523in}}{\pgfqpoint{2.813383in}{1.001523in}}%
\pgfpathclose%
\pgfusepath{stroke,fill}%
\end{pgfscope}%
\begin{pgfscope}%
\pgfpathrectangle{\pgfqpoint{0.800000in}{0.528000in}}{\pgfqpoint{3.968000in}{3.696000in}} %
\pgfusepath{clip}%
\pgfsetbuttcap%
\pgfsetroundjoin%
\definecolor{currentfill}{rgb}{0.121569,0.466667,0.705882}%
\pgfsetfillcolor{currentfill}%
\pgfsetlinewidth{1.003750pt}%
\definecolor{currentstroke}{rgb}{0.121569,0.466667,0.705882}%
\pgfsetstrokecolor{currentstroke}%
\pgfsetdash{}{0pt}%
\pgfpathmoveto{\pgfqpoint{3.055592in}{1.016692in}}%
\pgfpathcurveto{\pgfqpoint{3.064599in}{1.016692in}}{\pgfqpoint{3.073238in}{1.020271in}}{\pgfqpoint{3.079607in}{1.026640in}}%
\pgfpathcurveto{\pgfqpoint{3.085976in}{1.033009in}}{\pgfqpoint{3.089555in}{1.041648in}}{\pgfqpoint{3.089555in}{1.050655in}}%
\pgfpathcurveto{\pgfqpoint{3.089555in}{1.059663in}}{\pgfqpoint{3.085976in}{1.068302in}}{\pgfqpoint{3.079607in}{1.074671in}}%
\pgfpathcurveto{\pgfqpoint{3.073238in}{1.081040in}}{\pgfqpoint{3.064599in}{1.084619in}}{\pgfqpoint{3.055592in}{1.084619in}}%
\pgfpathcurveto{\pgfqpoint{3.046584in}{1.084619in}}{\pgfqpoint{3.037945in}{1.081040in}}{\pgfqpoint{3.031576in}{1.074671in}}%
\pgfpathcurveto{\pgfqpoint{3.025207in}{1.068302in}}{\pgfqpoint{3.021628in}{1.059663in}}{\pgfqpoint{3.021628in}{1.050655in}}%
\pgfpathcurveto{\pgfqpoint{3.021628in}{1.041648in}}{\pgfqpoint{3.025207in}{1.033009in}}{\pgfqpoint{3.031576in}{1.026640in}}%
\pgfpathcurveto{\pgfqpoint{3.037945in}{1.020271in}}{\pgfqpoint{3.046584in}{1.016692in}}{\pgfqpoint{3.055592in}{1.016692in}}%
\pgfpathclose%
\pgfusepath{stroke,fill}%
\end{pgfscope}%
\begin{pgfscope}%
\pgfpathrectangle{\pgfqpoint{0.800000in}{0.528000in}}{\pgfqpoint{3.968000in}{3.696000in}} %
\pgfusepath{clip}%
\pgfsetbuttcap%
\pgfsetroundjoin%
\definecolor{currentfill}{rgb}{0.121569,0.466667,0.705882}%
\pgfsetfillcolor{currentfill}%
\pgfsetlinewidth{1.003750pt}%
\definecolor{currentstroke}{rgb}{0.121569,0.466667,0.705882}%
\pgfsetstrokecolor{currentstroke}%
\pgfsetdash{}{0pt}%
\pgfpathmoveto{\pgfqpoint{2.086759in}{1.174677in}}%
\pgfpathcurveto{\pgfqpoint{2.100281in}{1.174677in}}{\pgfqpoint{2.113252in}{1.180050in}}{\pgfqpoint{2.122813in}{1.189611in}}%
\pgfpathcurveto{\pgfqpoint{2.132375in}{1.199173in}}{\pgfqpoint{2.137747in}{1.212143in}}{\pgfqpoint{2.137747in}{1.225665in}}%
\pgfpathcurveto{\pgfqpoint{2.137747in}{1.239188in}}{\pgfqpoint{2.132375in}{1.252158in}}{\pgfqpoint{2.122813in}{1.261719in}}%
\pgfpathcurveto{\pgfqpoint{2.113252in}{1.271281in}}{\pgfqpoint{2.100281in}{1.276653in}}{\pgfqpoint{2.086759in}{1.276653in}}%
\pgfpathcurveto{\pgfqpoint{2.073237in}{1.276653in}}{\pgfqpoint{2.060267in}{1.271281in}}{\pgfqpoint{2.050705in}{1.261719in}}%
\pgfpathcurveto{\pgfqpoint{2.041143in}{1.252158in}}{\pgfqpoint{2.035771in}{1.239188in}}{\pgfqpoint{2.035771in}{1.225665in}}%
\pgfpathcurveto{\pgfqpoint{2.035771in}{1.212143in}}{\pgfqpoint{2.041143in}{1.199173in}}{\pgfqpoint{2.050705in}{1.189611in}}%
\pgfpathcurveto{\pgfqpoint{2.060267in}{1.180050in}}{\pgfqpoint{2.073237in}{1.174677in}}{\pgfqpoint{2.086759in}{1.174677in}}%
\pgfpathclose%
\pgfusepath{stroke,fill}%
\end{pgfscope}%
\begin{pgfscope}%
\pgfpathrectangle{\pgfqpoint{0.800000in}{0.528000in}}{\pgfqpoint{3.968000in}{3.696000in}} %
\pgfusepath{clip}%
\pgfsetbuttcap%
\pgfsetroundjoin%
\definecolor{currentfill}{rgb}{0.121569,0.466667,0.705882}%
\pgfsetfillcolor{currentfill}%
\pgfsetlinewidth{1.003750pt}%
\definecolor{currentstroke}{rgb}{0.121569,0.466667,0.705882}%
\pgfsetstrokecolor{currentstroke}%
\pgfsetdash{}{0pt}%
\pgfpathmoveto{\pgfqpoint{2.328967in}{1.183385in}}%
\pgfpathcurveto{\pgfqpoint{2.340180in}{1.183385in}}{\pgfqpoint{2.350935in}{1.187840in}}{\pgfqpoint{2.358864in}{1.195768in}}%
\pgfpathcurveto{\pgfqpoint{2.366793in}{1.203697in}}{\pgfqpoint{2.371248in}{1.214452in}}{\pgfqpoint{2.371248in}{1.225665in}}%
\pgfpathcurveto{\pgfqpoint{2.371248in}{1.236878in}}{\pgfqpoint{2.366793in}{1.247634in}}{\pgfqpoint{2.358864in}{1.255562in}}%
\pgfpathcurveto{\pgfqpoint{2.350935in}{1.263491in}}{\pgfqpoint{2.340180in}{1.267946in}}{\pgfqpoint{2.328967in}{1.267946in}}%
\pgfpathcurveto{\pgfqpoint{2.317754in}{1.267946in}}{\pgfqpoint{2.306999in}{1.263491in}}{\pgfqpoint{2.299070in}{1.255562in}}%
\pgfpathcurveto{\pgfqpoint{2.291141in}{1.247634in}}{\pgfqpoint{2.286687in}{1.236878in}}{\pgfqpoint{2.286687in}{1.225665in}}%
\pgfpathcurveto{\pgfqpoint{2.286687in}{1.214452in}}{\pgfqpoint{2.291141in}{1.203697in}}{\pgfqpoint{2.299070in}{1.195768in}}%
\pgfpathcurveto{\pgfqpoint{2.306999in}{1.187840in}}{\pgfqpoint{2.317754in}{1.183385in}}{\pgfqpoint{2.328967in}{1.183385in}}%
\pgfpathclose%
\pgfusepath{stroke,fill}%
\end{pgfscope}%
\begin{pgfscope}%
\pgfpathrectangle{\pgfqpoint{0.800000in}{0.528000in}}{\pgfqpoint{3.968000in}{3.696000in}} %
\pgfusepath{clip}%
\pgfsetbuttcap%
\pgfsetroundjoin%
\definecolor{currentfill}{rgb}{0.121569,0.466667,0.705882}%
\pgfsetfillcolor{currentfill}%
\pgfsetlinewidth{1.003750pt}%
\definecolor{currentstroke}{rgb}{0.121569,0.466667,0.705882}%
\pgfsetstrokecolor{currentstroke}%
\pgfsetdash{}{0pt}%
\pgfpathmoveto{\pgfqpoint{2.571175in}{1.184736in}}%
\pgfpathcurveto{\pgfqpoint{2.582030in}{1.184736in}}{\pgfqpoint{2.592441in}{1.189049in}}{\pgfqpoint{2.600116in}{1.196724in}}%
\pgfpathcurveto{\pgfqpoint{2.607792in}{1.204400in}}{\pgfqpoint{2.612104in}{1.214811in}}{\pgfqpoint{2.612104in}{1.225665in}}%
\pgfpathcurveto{\pgfqpoint{2.612104in}{1.236520in}}{\pgfqpoint{2.607792in}{1.246931in}}{\pgfqpoint{2.600116in}{1.254606in}}%
\pgfpathcurveto{\pgfqpoint{2.592441in}{1.262282in}}{\pgfqpoint{2.582030in}{1.266594in}}{\pgfqpoint{2.571175in}{1.266594in}}%
\pgfpathcurveto{\pgfqpoint{2.560321in}{1.266594in}}{\pgfqpoint{2.549910in}{1.262282in}}{\pgfqpoint{2.542234in}{1.254606in}}%
\pgfpathcurveto{\pgfqpoint{2.534559in}{1.246931in}}{\pgfqpoint{2.530246in}{1.236520in}}{\pgfqpoint{2.530246in}{1.225665in}}%
\pgfpathcurveto{\pgfqpoint{2.530246in}{1.214811in}}{\pgfqpoint{2.534559in}{1.204400in}}{\pgfqpoint{2.542234in}{1.196724in}}%
\pgfpathcurveto{\pgfqpoint{2.549910in}{1.189049in}}{\pgfqpoint{2.560321in}{1.184736in}}{\pgfqpoint{2.571175in}{1.184736in}}%
\pgfpathclose%
\pgfusepath{stroke,fill}%
\end{pgfscope}%
\begin{pgfscope}%
\pgfpathrectangle{\pgfqpoint{0.800000in}{0.528000in}}{\pgfqpoint{3.968000in}{3.696000in}} %
\pgfusepath{clip}%
\pgfsetbuttcap%
\pgfsetroundjoin%
\definecolor{currentfill}{rgb}{0.121569,0.466667,0.705882}%
\pgfsetfillcolor{currentfill}%
\pgfsetlinewidth{1.003750pt}%
\definecolor{currentstroke}{rgb}{0.121569,0.466667,0.705882}%
\pgfsetstrokecolor{currentstroke}%
\pgfsetdash{}{0pt}%
\pgfpathmoveto{\pgfqpoint{2.813383in}{1.208082in}}%
\pgfpathcurveto{\pgfqpoint{2.818047in}{1.208082in}}{\pgfqpoint{2.822520in}{1.209934in}}{\pgfqpoint{2.825817in}{1.213232in}}%
\pgfpathcurveto{\pgfqpoint{2.829114in}{1.216529in}}{\pgfqpoint{2.830967in}{1.221002in}}{\pgfqpoint{2.830967in}{1.225665in}}%
\pgfpathcurveto{\pgfqpoint{2.830967in}{1.230329in}}{\pgfqpoint{2.829114in}{1.234802in}}{\pgfqpoint{2.825817in}{1.238099in}}%
\pgfpathcurveto{\pgfqpoint{2.822520in}{1.241396in}}{\pgfqpoint{2.818047in}{1.243249in}}{\pgfqpoint{2.813383in}{1.243249in}}%
\pgfpathcurveto{\pgfqpoint{2.808720in}{1.243249in}}{\pgfqpoint{2.804247in}{1.241396in}}{\pgfqpoint{2.800950in}{1.238099in}}%
\pgfpathcurveto{\pgfqpoint{2.797653in}{1.234802in}}{\pgfqpoint{2.795800in}{1.230329in}}{\pgfqpoint{2.795800in}{1.225665in}}%
\pgfpathcurveto{\pgfqpoint{2.795800in}{1.221002in}}{\pgfqpoint{2.797653in}{1.216529in}}{\pgfqpoint{2.800950in}{1.213232in}}%
\pgfpathcurveto{\pgfqpoint{2.804247in}{1.209934in}}{\pgfqpoint{2.808720in}{1.208082in}}{\pgfqpoint{2.813383in}{1.208082in}}%
\pgfpathclose%
\pgfusepath{stroke,fill}%
\end{pgfscope}%
\begin{pgfscope}%
\pgfpathrectangle{\pgfqpoint{0.800000in}{0.528000in}}{\pgfqpoint{3.968000in}{3.696000in}} %
\pgfusepath{clip}%
\pgfsetbuttcap%
\pgfsetroundjoin%
\definecolor{currentfill}{rgb}{0.121569,0.466667,0.705882}%
\pgfsetfillcolor{currentfill}%
\pgfsetlinewidth{1.003750pt}%
\definecolor{currentstroke}{rgb}{0.121569,0.466667,0.705882}%
\pgfsetstrokecolor{currentstroke}%
\pgfsetdash{}{0pt}%
\pgfpathmoveto{\pgfqpoint{3.055592in}{1.175085in}}%
\pgfpathcurveto{\pgfqpoint{3.069006in}{1.175085in}}{\pgfqpoint{3.081872in}{1.180414in}}{\pgfqpoint{3.091358in}{1.189899in}}%
\pgfpathcurveto{\pgfqpoint{3.100843in}{1.199385in}}{\pgfqpoint{3.106172in}{1.212251in}}{\pgfqpoint{3.106172in}{1.225665in}}%
\pgfpathcurveto{\pgfqpoint{3.106172in}{1.239080in}}{\pgfqpoint{3.100843in}{1.251946in}}{\pgfqpoint{3.091358in}{1.261431in}}%
\pgfpathcurveto{\pgfqpoint{3.081872in}{1.270917in}}{\pgfqpoint{3.069006in}{1.276246in}}{\pgfqpoint{3.055592in}{1.276246in}}%
\pgfpathcurveto{\pgfqpoint{3.042177in}{1.276246in}}{\pgfqpoint{3.029311in}{1.270917in}}{\pgfqpoint{3.019826in}{1.261431in}}%
\pgfpathcurveto{\pgfqpoint{3.010340in}{1.251946in}}{\pgfqpoint{3.005011in}{1.239080in}}{\pgfqpoint{3.005011in}{1.225665in}}%
\pgfpathcurveto{\pgfqpoint{3.005011in}{1.212251in}}{\pgfqpoint{3.010340in}{1.199385in}}{\pgfqpoint{3.019826in}{1.189899in}}%
\pgfpathcurveto{\pgfqpoint{3.029311in}{1.180414in}}{\pgfqpoint{3.042177in}{1.175085in}}{\pgfqpoint{3.055592in}{1.175085in}}%
\pgfpathclose%
\pgfusepath{stroke,fill}%
\end{pgfscope}%
\begin{pgfscope}%
\pgfpathrectangle{\pgfqpoint{0.800000in}{0.528000in}}{\pgfqpoint{3.968000in}{3.696000in}} %
\pgfusepath{clip}%
\pgfsetbuttcap%
\pgfsetroundjoin%
\definecolor{currentfill}{rgb}{0.121569,0.466667,0.705882}%
\pgfsetfillcolor{currentfill}%
\pgfsetlinewidth{1.003750pt}%
\definecolor{currentstroke}{rgb}{0.121569,0.466667,0.705882}%
\pgfsetstrokecolor{currentstroke}%
\pgfsetdash{}{0pt}%
\pgfpathmoveto{\pgfqpoint{2.086759in}{1.365857in}}%
\pgfpathcurveto{\pgfqpoint{2.095993in}{1.365857in}}{\pgfqpoint{2.104850in}{1.369526in}}{\pgfqpoint{2.111379in}{1.376055in}}%
\pgfpathcurveto{\pgfqpoint{2.117909in}{1.382585in}}{\pgfqpoint{2.121577in}{1.391441in}}{\pgfqpoint{2.121577in}{1.400675in}}%
\pgfpathcurveto{\pgfqpoint{2.121577in}{1.409909in}}{\pgfqpoint{2.117909in}{1.418766in}}{\pgfqpoint{2.111379in}{1.425295in}}%
\pgfpathcurveto{\pgfqpoint{2.104850in}{1.431825in}}{\pgfqpoint{2.095993in}{1.435493in}}{\pgfqpoint{2.086759in}{1.435493in}}%
\pgfpathcurveto{\pgfqpoint{2.077525in}{1.435493in}}{\pgfqpoint{2.068668in}{1.431825in}}{\pgfqpoint{2.062139in}{1.425295in}}%
\pgfpathcurveto{\pgfqpoint{2.055610in}{1.418766in}}{\pgfqpoint{2.051941in}{1.409909in}}{\pgfqpoint{2.051941in}{1.400675in}}%
\pgfpathcurveto{\pgfqpoint{2.051941in}{1.391441in}}{\pgfqpoint{2.055610in}{1.382585in}}{\pgfqpoint{2.062139in}{1.376055in}}%
\pgfpathcurveto{\pgfqpoint{2.068668in}{1.369526in}}{\pgfqpoint{2.077525in}{1.365857in}}{\pgfqpoint{2.086759in}{1.365857in}}%
\pgfpathclose%
\pgfusepath{stroke,fill}%
\end{pgfscope}%
\begin{pgfscope}%
\pgfpathrectangle{\pgfqpoint{0.800000in}{0.528000in}}{\pgfqpoint{3.968000in}{3.696000in}} %
\pgfusepath{clip}%
\pgfsetbuttcap%
\pgfsetroundjoin%
\definecolor{currentfill}{rgb}{0.121569,0.466667,0.705882}%
\pgfsetfillcolor{currentfill}%
\pgfsetlinewidth{1.003750pt}%
\definecolor{currentstroke}{rgb}{0.121569,0.466667,0.705882}%
\pgfsetstrokecolor{currentstroke}%
\pgfsetdash{}{0pt}%
\pgfpathmoveto{\pgfqpoint{2.328967in}{1.361198in}}%
\pgfpathcurveto{\pgfqpoint{2.339437in}{1.361198in}}{\pgfqpoint{2.349479in}{1.365357in}}{\pgfqpoint{2.356882in}{1.372760in}}%
\pgfpathcurveto{\pgfqpoint{2.364285in}{1.380164in}}{\pgfqpoint{2.368445in}{1.390206in}}{\pgfqpoint{2.368445in}{1.400675in}}%
\pgfpathcurveto{\pgfqpoint{2.368445in}{1.411145in}}{\pgfqpoint{2.364285in}{1.421187in}}{\pgfqpoint{2.356882in}{1.428590in}}%
\pgfpathcurveto{\pgfqpoint{2.349479in}{1.435993in}}{\pgfqpoint{2.339437in}{1.440153in}}{\pgfqpoint{2.328967in}{1.440153in}}%
\pgfpathcurveto{\pgfqpoint{2.318498in}{1.440153in}}{\pgfqpoint{2.308455in}{1.435993in}}{\pgfqpoint{2.301052in}{1.428590in}}%
\pgfpathcurveto{\pgfqpoint{2.293649in}{1.421187in}}{\pgfqpoint{2.289490in}{1.411145in}}{\pgfqpoint{2.289490in}{1.400675in}}%
\pgfpathcurveto{\pgfqpoint{2.289490in}{1.390206in}}{\pgfqpoint{2.293649in}{1.380164in}}{\pgfqpoint{2.301052in}{1.372760in}}%
\pgfpathcurveto{\pgfqpoint{2.308455in}{1.365357in}}{\pgfqpoint{2.318498in}{1.361198in}}{\pgfqpoint{2.328967in}{1.361198in}}%
\pgfpathclose%
\pgfusepath{stroke,fill}%
\end{pgfscope}%
\begin{pgfscope}%
\pgfpathrectangle{\pgfqpoint{0.800000in}{0.528000in}}{\pgfqpoint{3.968000in}{3.696000in}} %
\pgfusepath{clip}%
\pgfsetbuttcap%
\pgfsetroundjoin%
\definecolor{currentfill}{rgb}{0.121569,0.466667,0.705882}%
\pgfsetfillcolor{currentfill}%
\pgfsetlinewidth{1.003750pt}%
\definecolor{currentstroke}{rgb}{0.121569,0.466667,0.705882}%
\pgfsetstrokecolor{currentstroke}%
\pgfsetdash{}{0pt}%
\pgfpathmoveto{\pgfqpoint{2.571175in}{1.367513in}}%
\pgfpathcurveto{\pgfqpoint{2.579970in}{1.367513in}}{\pgfqpoint{2.588406in}{1.371007in}}{\pgfqpoint{2.594625in}{1.377226in}}%
\pgfpathcurveto{\pgfqpoint{2.600843in}{1.383445in}}{\pgfqpoint{2.604338in}{1.391881in}}{\pgfqpoint{2.604338in}{1.400675in}}%
\pgfpathcurveto{\pgfqpoint{2.604338in}{1.409470in}}{\pgfqpoint{2.600843in}{1.417906in}}{\pgfqpoint{2.594625in}{1.424125in}}%
\pgfpathcurveto{\pgfqpoint{2.588406in}{1.430343in}}{\pgfqpoint{2.579970in}{1.433838in}}{\pgfqpoint{2.571175in}{1.433838in}}%
\pgfpathcurveto{\pgfqpoint{2.562381in}{1.433838in}}{\pgfqpoint{2.553945in}{1.430343in}}{\pgfqpoint{2.547726in}{1.424125in}}%
\pgfpathcurveto{\pgfqpoint{2.541507in}{1.417906in}}{\pgfqpoint{2.538013in}{1.409470in}}{\pgfqpoint{2.538013in}{1.400675in}}%
\pgfpathcurveto{\pgfqpoint{2.538013in}{1.391881in}}{\pgfqpoint{2.541507in}{1.383445in}}{\pgfqpoint{2.547726in}{1.377226in}}%
\pgfpathcurveto{\pgfqpoint{2.553945in}{1.371007in}}{\pgfqpoint{2.562381in}{1.367513in}}{\pgfqpoint{2.571175in}{1.367513in}}%
\pgfpathclose%
\pgfusepath{stroke,fill}%
\end{pgfscope}%
\begin{pgfscope}%
\pgfpathrectangle{\pgfqpoint{0.800000in}{0.528000in}}{\pgfqpoint{3.968000in}{3.696000in}} %
\pgfusepath{clip}%
\pgfsetbuttcap%
\pgfsetroundjoin%
\definecolor{currentfill}{rgb}{0.121569,0.466667,0.705882}%
\pgfsetfillcolor{currentfill}%
\pgfsetlinewidth{1.003750pt}%
\definecolor{currentstroke}{rgb}{0.121569,0.466667,0.705882}%
\pgfsetstrokecolor{currentstroke}%
\pgfsetdash{}{0pt}%
\pgfpathmoveto{\pgfqpoint{2.813383in}{1.375403in}}%
\pgfpathcurveto{\pgfqpoint{2.820086in}{1.375403in}}{\pgfqpoint{2.826514in}{1.378066in}}{\pgfqpoint{2.831254in}{1.382805in}}%
\pgfpathcurveto{\pgfqpoint{2.835993in}{1.387544in}}{\pgfqpoint{2.838656in}{1.393973in}}{\pgfqpoint{2.838656in}{1.400675in}}%
\pgfpathcurveto{\pgfqpoint{2.838656in}{1.407378in}}{\pgfqpoint{2.835993in}{1.413806in}}{\pgfqpoint{2.831254in}{1.418545in}}%
\pgfpathcurveto{\pgfqpoint{2.826514in}{1.423285in}}{\pgfqpoint{2.820086in}{1.425947in}}{\pgfqpoint{2.813383in}{1.425947in}}%
\pgfpathcurveto{\pgfqpoint{2.806681in}{1.425947in}}{\pgfqpoint{2.800253in}{1.423285in}}{\pgfqpoint{2.795513in}{1.418545in}}%
\pgfpathcurveto{\pgfqpoint{2.790774in}{1.413806in}}{\pgfqpoint{2.788111in}{1.407378in}}{\pgfqpoint{2.788111in}{1.400675in}}%
\pgfpathcurveto{\pgfqpoint{2.788111in}{1.393973in}}{\pgfqpoint{2.790774in}{1.387544in}}{\pgfqpoint{2.795513in}{1.382805in}}%
\pgfpathcurveto{\pgfqpoint{2.800253in}{1.378066in}}{\pgfqpoint{2.806681in}{1.375403in}}{\pgfqpoint{2.813383in}{1.375403in}}%
\pgfpathclose%
\pgfusepath{stroke,fill}%
\end{pgfscope}%
\begin{pgfscope}%
\pgfpathrectangle{\pgfqpoint{0.800000in}{0.528000in}}{\pgfqpoint{3.968000in}{3.696000in}} %
\pgfusepath{clip}%
\pgfsetbuttcap%
\pgfsetroundjoin%
\definecolor{currentfill}{rgb}{0.121569,0.466667,0.705882}%
\pgfsetfillcolor{currentfill}%
\pgfsetlinewidth{1.003750pt}%
\definecolor{currentstroke}{rgb}{0.121569,0.466667,0.705882}%
\pgfsetstrokecolor{currentstroke}%
\pgfsetdash{}{0pt}%
\pgfpathmoveto{\pgfqpoint{3.055592in}{1.351338in}}%
\pgfpathcurveto{\pgfqpoint{3.068676in}{1.351338in}}{\pgfqpoint{3.081226in}{1.356536in}}{\pgfqpoint{3.090478in}{1.365789in}}%
\pgfpathcurveto{\pgfqpoint{3.099731in}{1.375041in}}{\pgfqpoint{3.104929in}{1.387591in}}{\pgfqpoint{3.104929in}{1.400675in}}%
\pgfpathcurveto{\pgfqpoint{3.104929in}{1.413760in}}{\pgfqpoint{3.099731in}{1.426310in}}{\pgfqpoint{3.090478in}{1.435562in}}%
\pgfpathcurveto{\pgfqpoint{3.081226in}{1.444814in}}{\pgfqpoint{3.068676in}{1.450013in}}{\pgfqpoint{3.055592in}{1.450013in}}%
\pgfpathcurveto{\pgfqpoint{3.042507in}{1.450013in}}{\pgfqpoint{3.029957in}{1.444814in}}{\pgfqpoint{3.020705in}{1.435562in}}%
\pgfpathcurveto{\pgfqpoint{3.011453in}{1.426310in}}{\pgfqpoint{3.006254in}{1.413760in}}{\pgfqpoint{3.006254in}{1.400675in}}%
\pgfpathcurveto{\pgfqpoint{3.006254in}{1.387591in}}{\pgfqpoint{3.011453in}{1.375041in}}{\pgfqpoint{3.020705in}{1.365789in}}%
\pgfpathcurveto{\pgfqpoint{3.029957in}{1.356536in}}{\pgfqpoint{3.042507in}{1.351338in}}{\pgfqpoint{3.055592in}{1.351338in}}%
\pgfpathclose%
\pgfusepath{stroke,fill}%
\end{pgfscope}%
\begin{pgfscope}%
\pgfpathrectangle{\pgfqpoint{0.800000in}{0.528000in}}{\pgfqpoint{3.968000in}{3.696000in}} %
\pgfusepath{clip}%
\pgfsetbuttcap%
\pgfsetroundjoin%
\definecolor{currentfill}{rgb}{0.121569,0.466667,0.705882}%
\pgfsetfillcolor{currentfill}%
\pgfsetlinewidth{1.003750pt}%
\definecolor{currentstroke}{rgb}{0.121569,0.466667,0.705882}%
\pgfsetstrokecolor{currentstroke}%
\pgfsetdash{}{0pt}%
\pgfpathmoveto{\pgfqpoint{2.086759in}{1.536369in}}%
\pgfpathcurveto{\pgfqpoint{2.097186in}{1.536369in}}{\pgfqpoint{2.107187in}{1.540511in}}{\pgfqpoint{2.114560in}{1.547884in}}%
\pgfpathcurveto{\pgfqpoint{2.121933in}{1.555257in}}{\pgfqpoint{2.126076in}{1.565258in}}{\pgfqpoint{2.126076in}{1.575685in}}%
\pgfpathcurveto{\pgfqpoint{2.126076in}{1.586112in}}{\pgfqpoint{2.121933in}{1.596113in}}{\pgfqpoint{2.114560in}{1.603486in}}%
\pgfpathcurveto{\pgfqpoint{2.107187in}{1.610859in}}{\pgfqpoint{2.097186in}{1.615002in}}{\pgfqpoint{2.086759in}{1.615002in}}%
\pgfpathcurveto{\pgfqpoint{2.076332in}{1.615002in}}{\pgfqpoint{2.066331in}{1.610859in}}{\pgfqpoint{2.058958in}{1.603486in}}%
\pgfpathcurveto{\pgfqpoint{2.051585in}{1.596113in}}{\pgfqpoint{2.047443in}{1.586112in}}{\pgfqpoint{2.047443in}{1.575685in}}%
\pgfpathcurveto{\pgfqpoint{2.047443in}{1.565258in}}{\pgfqpoint{2.051585in}{1.555257in}}{\pgfqpoint{2.058958in}{1.547884in}}%
\pgfpathcurveto{\pgfqpoint{2.066331in}{1.540511in}}{\pgfqpoint{2.076332in}{1.536369in}}{\pgfqpoint{2.086759in}{1.536369in}}%
\pgfpathclose%
\pgfusepath{stroke,fill}%
\end{pgfscope}%
\begin{pgfscope}%
\pgfpathrectangle{\pgfqpoint{0.800000in}{0.528000in}}{\pgfqpoint{3.968000in}{3.696000in}} %
\pgfusepath{clip}%
\pgfsetbuttcap%
\pgfsetroundjoin%
\definecolor{currentfill}{rgb}{0.121569,0.466667,0.705882}%
\pgfsetfillcolor{currentfill}%
\pgfsetlinewidth{1.003750pt}%
\definecolor{currentstroke}{rgb}{0.121569,0.466667,0.705882}%
\pgfsetstrokecolor{currentstroke}%
\pgfsetdash{}{0pt}%
\pgfpathmoveto{\pgfqpoint{2.328967in}{1.572904in}}%
\pgfpathcurveto{\pgfqpoint{2.329705in}{1.572904in}}{\pgfqpoint{2.330412in}{1.573197in}}{\pgfqpoint{2.330934in}{1.573719in}}%
\pgfpathcurveto{\pgfqpoint{2.331455in}{1.574240in}}{\pgfqpoint{2.331748in}{1.574948in}}{\pgfqpoint{2.331748in}{1.575685in}}%
\pgfpathcurveto{\pgfqpoint{2.331748in}{1.576423in}}{\pgfqpoint{2.331455in}{1.577130in}}{\pgfqpoint{2.330934in}{1.577652in}}%
\pgfpathcurveto{\pgfqpoint{2.330412in}{1.578173in}}{\pgfqpoint{2.329705in}{1.578466in}}{\pgfqpoint{2.328967in}{1.578466in}}%
\pgfpathcurveto{\pgfqpoint{2.328230in}{1.578466in}}{\pgfqpoint{2.327522in}{1.578173in}}{\pgfqpoint{2.327001in}{1.577652in}}%
\pgfpathcurveto{\pgfqpoint{2.326479in}{1.577130in}}{\pgfqpoint{2.326186in}{1.576423in}}{\pgfqpoint{2.326186in}{1.575685in}}%
\pgfpathcurveto{\pgfqpoint{2.326186in}{1.574948in}}{\pgfqpoint{2.326479in}{1.574240in}}{\pgfqpoint{2.327001in}{1.573719in}}%
\pgfpathcurveto{\pgfqpoint{2.327522in}{1.573197in}}{\pgfqpoint{2.328230in}{1.572904in}}{\pgfqpoint{2.328967in}{1.572904in}}%
\pgfpathclose%
\pgfusepath{stroke,fill}%
\end{pgfscope}%
\begin{pgfscope}%
\pgfpathrectangle{\pgfqpoint{0.800000in}{0.528000in}}{\pgfqpoint{3.968000in}{3.696000in}} %
\pgfusepath{clip}%
\pgfsetbuttcap%
\pgfsetroundjoin%
\definecolor{currentfill}{rgb}{0.121569,0.466667,0.705882}%
\pgfsetfillcolor{currentfill}%
\pgfsetlinewidth{1.003750pt}%
\definecolor{currentstroke}{rgb}{0.121569,0.466667,0.705882}%
\pgfsetstrokecolor{currentstroke}%
\pgfsetdash{}{0pt}%
\pgfpathmoveto{\pgfqpoint{2.571175in}{1.534906in}}%
\pgfpathcurveto{\pgfqpoint{2.581990in}{1.534906in}}{\pgfqpoint{2.592363in}{1.539203in}}{\pgfqpoint{2.600010in}{1.546850in}}%
\pgfpathcurveto{\pgfqpoint{2.607657in}{1.554497in}}{\pgfqpoint{2.611954in}{1.564871in}}{\pgfqpoint{2.611954in}{1.575685in}}%
\pgfpathcurveto{\pgfqpoint{2.611954in}{1.586500in}}{\pgfqpoint{2.607657in}{1.596873in}}{\pgfqpoint{2.600010in}{1.604520in}}%
\pgfpathcurveto{\pgfqpoint{2.592363in}{1.612167in}}{\pgfqpoint{2.581990in}{1.616464in}}{\pgfqpoint{2.571175in}{1.616464in}}%
\pgfpathcurveto{\pgfqpoint{2.560361in}{1.616464in}}{\pgfqpoint{2.549987in}{1.612167in}}{\pgfqpoint{2.542340in}{1.604520in}}%
\pgfpathcurveto{\pgfqpoint{2.534693in}{1.596873in}}{\pgfqpoint{2.530397in}{1.586500in}}{\pgfqpoint{2.530397in}{1.575685in}}%
\pgfpathcurveto{\pgfqpoint{2.530397in}{1.564871in}}{\pgfqpoint{2.534693in}{1.554497in}}{\pgfqpoint{2.542340in}{1.546850in}}%
\pgfpathcurveto{\pgfqpoint{2.549987in}{1.539203in}}{\pgfqpoint{2.560361in}{1.534906in}}{\pgfqpoint{2.571175in}{1.534906in}}%
\pgfpathclose%
\pgfusepath{stroke,fill}%
\end{pgfscope}%
\begin{pgfscope}%
\pgfpathrectangle{\pgfqpoint{0.800000in}{0.528000in}}{\pgfqpoint{3.968000in}{3.696000in}} %
\pgfusepath{clip}%
\pgfsetbuttcap%
\pgfsetroundjoin%
\definecolor{currentfill}{rgb}{0.121569,0.466667,0.705882}%
\pgfsetfillcolor{currentfill}%
\pgfsetlinewidth{1.003750pt}%
\definecolor{currentstroke}{rgb}{0.121569,0.466667,0.705882}%
\pgfsetstrokecolor{currentstroke}%
\pgfsetdash{}{0pt}%
\pgfpathmoveto{\pgfqpoint{2.813383in}{1.546018in}}%
\pgfpathcurveto{\pgfqpoint{2.821251in}{1.546018in}}{\pgfqpoint{2.828798in}{1.549144in}}{\pgfqpoint{2.834361in}{1.554707in}}%
\pgfpathcurveto{\pgfqpoint{2.839925in}{1.560271in}}{\pgfqpoint{2.843051in}{1.567817in}}{\pgfqpoint{2.843051in}{1.575685in}}%
\pgfpathcurveto{\pgfqpoint{2.843051in}{1.583553in}}{\pgfqpoint{2.839925in}{1.591100in}}{\pgfqpoint{2.834361in}{1.596663in}}%
\pgfpathcurveto{\pgfqpoint{2.828798in}{1.602227in}}{\pgfqpoint{2.821251in}{1.605353in}}{\pgfqpoint{2.813383in}{1.605353in}}%
\pgfpathcurveto{\pgfqpoint{2.805516in}{1.605353in}}{\pgfqpoint{2.797969in}{1.602227in}}{\pgfqpoint{2.792405in}{1.596663in}}%
\pgfpathcurveto{\pgfqpoint{2.786842in}{1.591100in}}{\pgfqpoint{2.783716in}{1.583553in}}{\pgfqpoint{2.783716in}{1.575685in}}%
\pgfpathcurveto{\pgfqpoint{2.783716in}{1.567817in}}{\pgfqpoint{2.786842in}{1.560271in}}{\pgfqpoint{2.792405in}{1.554707in}}%
\pgfpathcurveto{\pgfqpoint{2.797969in}{1.549144in}}{\pgfqpoint{2.805516in}{1.546018in}}{\pgfqpoint{2.813383in}{1.546018in}}%
\pgfpathclose%
\pgfusepath{stroke,fill}%
\end{pgfscope}%
\begin{pgfscope}%
\pgfpathrectangle{\pgfqpoint{0.800000in}{0.528000in}}{\pgfqpoint{3.968000in}{3.696000in}} %
\pgfusepath{clip}%
\pgfsetbuttcap%
\pgfsetroundjoin%
\definecolor{currentfill}{rgb}{0.121569,0.466667,0.705882}%
\pgfsetfillcolor{currentfill}%
\pgfsetlinewidth{1.003750pt}%
\definecolor{currentstroke}{rgb}{0.121569,0.466667,0.705882}%
\pgfsetstrokecolor{currentstroke}%
\pgfsetdash{}{0pt}%
\pgfpathmoveto{\pgfqpoint{3.055592in}{1.538000in}}%
\pgfpathcurveto{\pgfqpoint{3.065586in}{1.538000in}}{\pgfqpoint{3.075172in}{1.541971in}}{\pgfqpoint{3.082239in}{1.549038in}}%
\pgfpathcurveto{\pgfqpoint{3.089306in}{1.556105in}}{\pgfqpoint{3.093277in}{1.565691in}}{\pgfqpoint{3.093277in}{1.575685in}}%
\pgfpathcurveto{\pgfqpoint{3.093277in}{1.585680in}}{\pgfqpoint{3.089306in}{1.595266in}}{\pgfqpoint{3.082239in}{1.602333in}}%
\pgfpathcurveto{\pgfqpoint{3.075172in}{1.609400in}}{\pgfqpoint{3.065586in}{1.613370in}}{\pgfqpoint{3.055592in}{1.613370in}}%
\pgfpathcurveto{\pgfqpoint{3.045597in}{1.613370in}}{\pgfqpoint{3.036011in}{1.609400in}}{\pgfqpoint{3.028944in}{1.602333in}}%
\pgfpathcurveto{\pgfqpoint{3.021877in}{1.595266in}}{\pgfqpoint{3.017906in}{1.585680in}}{\pgfqpoint{3.017906in}{1.575685in}}%
\pgfpathcurveto{\pgfqpoint{3.017906in}{1.565691in}}{\pgfqpoint{3.021877in}{1.556105in}}{\pgfqpoint{3.028944in}{1.549038in}}%
\pgfpathcurveto{\pgfqpoint{3.036011in}{1.541971in}}{\pgfqpoint{3.045597in}{1.538000in}}{\pgfqpoint{3.055592in}{1.538000in}}%
\pgfpathclose%
\pgfusepath{stroke,fill}%
\end{pgfscope}%
\begin{pgfscope}%
\pgfpathrectangle{\pgfqpoint{0.800000in}{0.528000in}}{\pgfqpoint{3.968000in}{3.696000in}} %
\pgfusepath{clip}%
\pgfsetbuttcap%
\pgfsetroundjoin%
\definecolor{currentfill}{rgb}{0.121569,0.466667,0.705882}%
\pgfsetfillcolor{currentfill}%
\pgfsetlinewidth{1.003750pt}%
\definecolor{currentstroke}{rgb}{0.121569,0.466667,0.705882}%
\pgfsetstrokecolor{currentstroke}%
\pgfsetdash{}{0pt}%
\pgfpathmoveto{\pgfqpoint{2.086759in}{1.701836in}}%
\pgfpathcurveto{\pgfqpoint{2.099717in}{1.701836in}}{\pgfqpoint{2.112145in}{1.706984in}}{\pgfqpoint{2.121308in}{1.716147in}}%
\pgfpathcurveto{\pgfqpoint{2.130470in}{1.725309in}}{\pgfqpoint{2.135618in}{1.737738in}}{\pgfqpoint{2.135618in}{1.750695in}}%
\pgfpathcurveto{\pgfqpoint{2.135618in}{1.763653in}}{\pgfqpoint{2.130470in}{1.776081in}}{\pgfqpoint{2.121308in}{1.785244in}}%
\pgfpathcurveto{\pgfqpoint{2.112145in}{1.794406in}}{\pgfqpoint{2.099717in}{1.799554in}}{\pgfqpoint{2.086759in}{1.799554in}}%
\pgfpathcurveto{\pgfqpoint{2.073802in}{1.799554in}}{\pgfqpoint{2.061373in}{1.794406in}}{\pgfqpoint{2.052211in}{1.785244in}}%
\pgfpathcurveto{\pgfqpoint{2.043048in}{1.776081in}}{\pgfqpoint{2.037900in}{1.763653in}}{\pgfqpoint{2.037900in}{1.750695in}}%
\pgfpathcurveto{\pgfqpoint{2.037900in}{1.737738in}}{\pgfqpoint{2.043048in}{1.725309in}}{\pgfqpoint{2.052211in}{1.716147in}}%
\pgfpathcurveto{\pgfqpoint{2.061373in}{1.706984in}}{\pgfqpoint{2.073802in}{1.701836in}}{\pgfqpoint{2.086759in}{1.701836in}}%
\pgfpathclose%
\pgfusepath{stroke,fill}%
\end{pgfscope}%
\begin{pgfscope}%
\pgfpathrectangle{\pgfqpoint{0.800000in}{0.528000in}}{\pgfqpoint{3.968000in}{3.696000in}} %
\pgfusepath{clip}%
\pgfsetbuttcap%
\pgfsetroundjoin%
\definecolor{currentfill}{rgb}{0.121569,0.466667,0.705882}%
\pgfsetfillcolor{currentfill}%
\pgfsetlinewidth{1.003750pt}%
\definecolor{currentstroke}{rgb}{0.121569,0.466667,0.705882}%
\pgfsetstrokecolor{currentstroke}%
\pgfsetdash{}{0pt}%
\pgfpathmoveto{\pgfqpoint{2.328967in}{1.719668in}}%
\pgfpathcurveto{\pgfqpoint{2.337196in}{1.719668in}}{\pgfqpoint{2.345088in}{1.722937in}}{\pgfqpoint{2.350907in}{1.728756in}}%
\pgfpathcurveto{\pgfqpoint{2.356725in}{1.734574in}}{\pgfqpoint{2.359994in}{1.742467in}}{\pgfqpoint{2.359994in}{1.750695in}}%
\pgfpathcurveto{\pgfqpoint{2.359994in}{1.758924in}}{\pgfqpoint{2.356725in}{1.766816in}}{\pgfqpoint{2.350907in}{1.772635in}}%
\pgfpathcurveto{\pgfqpoint{2.345088in}{1.778453in}}{\pgfqpoint{2.337196in}{1.781722in}}{\pgfqpoint{2.328967in}{1.781722in}}%
\pgfpathcurveto{\pgfqpoint{2.320739in}{1.781722in}}{\pgfqpoint{2.312846in}{1.778453in}}{\pgfqpoint{2.307028in}{1.772635in}}%
\pgfpathcurveto{\pgfqpoint{2.301209in}{1.766816in}}{\pgfqpoint{2.297940in}{1.758924in}}{\pgfqpoint{2.297940in}{1.750695in}}%
\pgfpathcurveto{\pgfqpoint{2.297940in}{1.742467in}}{\pgfqpoint{2.301209in}{1.734574in}}{\pgfqpoint{2.307028in}{1.728756in}}%
\pgfpathcurveto{\pgfqpoint{2.312846in}{1.722937in}}{\pgfqpoint{2.320739in}{1.719668in}}{\pgfqpoint{2.328967in}{1.719668in}}%
\pgfpathclose%
\pgfusepath{stroke,fill}%
\end{pgfscope}%
\begin{pgfscope}%
\pgfpathrectangle{\pgfqpoint{0.800000in}{0.528000in}}{\pgfqpoint{3.968000in}{3.696000in}} %
\pgfusepath{clip}%
\pgfsetbuttcap%
\pgfsetroundjoin%
\definecolor{currentfill}{rgb}{0.121569,0.466667,0.705882}%
\pgfsetfillcolor{currentfill}%
\pgfsetlinewidth{1.003750pt}%
\definecolor{currentstroke}{rgb}{0.121569,0.466667,0.705882}%
\pgfsetstrokecolor{currentstroke}%
\pgfsetdash{}{0pt}%
\pgfpathmoveto{\pgfqpoint{2.571175in}{1.701217in}}%
\pgfpathcurveto{\pgfqpoint{2.584297in}{1.701217in}}{\pgfqpoint{2.596883in}{1.706431in}}{\pgfqpoint{2.606162in}{1.715709in}}%
\pgfpathcurveto{\pgfqpoint{2.615440in}{1.724987in}}{\pgfqpoint{2.620653in}{1.737574in}}{\pgfqpoint{2.620653in}{1.750695in}}%
\pgfpathcurveto{\pgfqpoint{2.620653in}{1.763817in}}{\pgfqpoint{2.615440in}{1.776403in}}{\pgfqpoint{2.606162in}{1.785681in}}%
\pgfpathcurveto{\pgfqpoint{2.596883in}{1.794960in}}{\pgfqpoint{2.584297in}{1.800173in}}{\pgfqpoint{2.571175in}{1.800173in}}%
\pgfpathcurveto{\pgfqpoint{2.558054in}{1.800173in}}{\pgfqpoint{2.545468in}{1.794960in}}{\pgfqpoint{2.536189in}{1.785681in}}%
\pgfpathcurveto{\pgfqpoint{2.526911in}{1.776403in}}{\pgfqpoint{2.521697in}{1.763817in}}{\pgfqpoint{2.521697in}{1.750695in}}%
\pgfpathcurveto{\pgfqpoint{2.521697in}{1.737574in}}{\pgfqpoint{2.526911in}{1.724987in}}{\pgfqpoint{2.536189in}{1.715709in}}%
\pgfpathcurveto{\pgfqpoint{2.545468in}{1.706431in}}{\pgfqpoint{2.558054in}{1.701217in}}{\pgfqpoint{2.571175in}{1.701217in}}%
\pgfpathclose%
\pgfusepath{stroke,fill}%
\end{pgfscope}%
\begin{pgfscope}%
\pgfpathrectangle{\pgfqpoint{0.800000in}{0.528000in}}{\pgfqpoint{3.968000in}{3.696000in}} %
\pgfusepath{clip}%
\pgfsetbuttcap%
\pgfsetroundjoin%
\definecolor{currentfill}{rgb}{0.121569,0.466667,0.705882}%
\pgfsetfillcolor{currentfill}%
\pgfsetlinewidth{1.003750pt}%
\definecolor{currentstroke}{rgb}{0.121569,0.466667,0.705882}%
\pgfsetstrokecolor{currentstroke}%
\pgfsetdash{}{0pt}%
\pgfpathmoveto{\pgfqpoint{2.813383in}{1.709712in}}%
\pgfpathcurveto{\pgfqpoint{2.824252in}{1.709712in}}{\pgfqpoint{2.834678in}{1.714030in}}{\pgfqpoint{2.842363in}{1.721715in}}%
\pgfpathcurveto{\pgfqpoint{2.850049in}{1.729401in}}{\pgfqpoint{2.854367in}{1.739826in}}{\pgfqpoint{2.854367in}{1.750695in}}%
\pgfpathcurveto{\pgfqpoint{2.854367in}{1.761564in}}{\pgfqpoint{2.850049in}{1.771990in}}{\pgfqpoint{2.842363in}{1.779675in}}%
\pgfpathcurveto{\pgfqpoint{2.834678in}{1.787361in}}{\pgfqpoint{2.824252in}{1.791679in}}{\pgfqpoint{2.813383in}{1.791679in}}%
\pgfpathcurveto{\pgfqpoint{2.802514in}{1.791679in}}{\pgfqpoint{2.792089in}{1.787361in}}{\pgfqpoint{2.784404in}{1.779675in}}%
\pgfpathcurveto{\pgfqpoint{2.776718in}{1.771990in}}{\pgfqpoint{2.772400in}{1.761564in}}{\pgfqpoint{2.772400in}{1.750695in}}%
\pgfpathcurveto{\pgfqpoint{2.772400in}{1.739826in}}{\pgfqpoint{2.776718in}{1.729401in}}{\pgfqpoint{2.784404in}{1.721715in}}%
\pgfpathcurveto{\pgfqpoint{2.792089in}{1.714030in}}{\pgfqpoint{2.802514in}{1.709712in}}{\pgfqpoint{2.813383in}{1.709712in}}%
\pgfpathclose%
\pgfusepath{stroke,fill}%
\end{pgfscope}%
\begin{pgfscope}%
\pgfpathrectangle{\pgfqpoint{0.800000in}{0.528000in}}{\pgfqpoint{3.968000in}{3.696000in}} %
\pgfusepath{clip}%
\pgfsetbuttcap%
\pgfsetroundjoin%
\definecolor{currentfill}{rgb}{0.121569,0.466667,0.705882}%
\pgfsetfillcolor{currentfill}%
\pgfsetlinewidth{1.003750pt}%
\definecolor{currentstroke}{rgb}{0.121569,0.466667,0.705882}%
\pgfsetstrokecolor{currentstroke}%
\pgfsetdash{}{0pt}%
\pgfpathmoveto{\pgfqpoint{3.055592in}{1.744166in}}%
\pgfpathcurveto{\pgfqpoint{3.057323in}{1.744166in}}{\pgfqpoint{3.058984in}{1.744854in}}{\pgfqpoint{3.060208in}{1.746078in}}%
\pgfpathcurveto{\pgfqpoint{3.061433in}{1.747303in}}{\pgfqpoint{3.062121in}{1.748964in}}{\pgfqpoint{3.062121in}{1.750695in}}%
\pgfpathcurveto{\pgfqpoint{3.062121in}{1.752427in}}{\pgfqpoint{3.061433in}{1.754088in}}{\pgfqpoint{3.060208in}{1.755312in}}%
\pgfpathcurveto{\pgfqpoint{3.058984in}{1.756537in}}{\pgfqpoint{3.057323in}{1.757224in}}{\pgfqpoint{3.055592in}{1.757224in}}%
\pgfpathcurveto{\pgfqpoint{3.053860in}{1.757224in}}{\pgfqpoint{3.052199in}{1.756537in}}{\pgfqpoint{3.050975in}{1.755312in}}%
\pgfpathcurveto{\pgfqpoint{3.049750in}{1.754088in}}{\pgfqpoint{3.049062in}{1.752427in}}{\pgfqpoint{3.049062in}{1.750695in}}%
\pgfpathcurveto{\pgfqpoint{3.049062in}{1.748964in}}{\pgfqpoint{3.049750in}{1.747303in}}{\pgfqpoint{3.050975in}{1.746078in}}%
\pgfpathcurveto{\pgfqpoint{3.052199in}{1.744854in}}{\pgfqpoint{3.053860in}{1.744166in}}{\pgfqpoint{3.055592in}{1.744166in}}%
\pgfpathclose%
\pgfusepath{stroke,fill}%
\end{pgfscope}%
\begin{pgfscope}%
\pgfpathrectangle{\pgfqpoint{0.800000in}{0.528000in}}{\pgfqpoint{3.968000in}{3.696000in}} %
\pgfusepath{clip}%
\pgfsetbuttcap%
\pgfsetroundjoin%
\definecolor{currentfill}{rgb}{1.000000,0.498039,0.054902}%
\pgfsetfillcolor{currentfill}%
\pgfsetlinewidth{1.003750pt}%
\definecolor{currentstroke}{rgb}{1.000000,0.498039,0.054902}%
\pgfsetstrokecolor{currentstroke}%
\pgfsetdash{}{0pt}%
\pgfpathmoveto{\pgfqpoint{1.843526in}{1.163430in}}%
\pgfpathcurveto{\pgfqpoint{1.857107in}{1.163430in}}{\pgfqpoint{1.870133in}{1.168826in}}{\pgfqpoint{1.879735in}{1.178428in}}%
\pgfpathcurveto{\pgfqpoint{1.889338in}{1.188031in}}{\pgfqpoint{1.894734in}{1.201057in}}{\pgfqpoint{1.894734in}{1.214638in}}%
\pgfpathcurveto{\pgfqpoint{1.894734in}{1.228218in}}{\pgfqpoint{1.889338in}{1.241244in}}{\pgfqpoint{1.879735in}{1.250847in}}%
\pgfpathcurveto{\pgfqpoint{1.870133in}{1.260449in}}{\pgfqpoint{1.857107in}{1.265845in}}{\pgfqpoint{1.843526in}{1.265845in}}%
\pgfpathcurveto{\pgfqpoint{1.829946in}{1.265845in}}{\pgfqpoint{1.816920in}{1.260449in}}{\pgfqpoint{1.807317in}{1.250847in}}%
\pgfpathcurveto{\pgfqpoint{1.797714in}{1.241244in}}{\pgfqpoint{1.792319in}{1.228218in}}{\pgfqpoint{1.792319in}{1.214638in}}%
\pgfpathcurveto{\pgfqpoint{1.792319in}{1.201057in}}{\pgfqpoint{1.797714in}{1.188031in}}{\pgfqpoint{1.807317in}{1.178428in}}%
\pgfpathcurveto{\pgfqpoint{1.816920in}{1.168826in}}{\pgfqpoint{1.829946in}{1.163430in}}{\pgfqpoint{1.843526in}{1.163430in}}%
\pgfpathclose%
\pgfusepath{stroke,fill}%
\end{pgfscope}%
\begin{pgfscope}%
\pgfpathrectangle{\pgfqpoint{0.800000in}{0.528000in}}{\pgfqpoint{3.968000in}{3.696000in}} %
\pgfusepath{clip}%
\pgfsetbuttcap%
\pgfsetroundjoin%
\definecolor{currentfill}{rgb}{1.000000,0.498039,0.054902}%
\pgfsetfillcolor{currentfill}%
\pgfsetlinewidth{1.003750pt}%
\definecolor{currentstroke}{rgb}{1.000000,0.498039,0.054902}%
\pgfsetstrokecolor{currentstroke}%
\pgfsetdash{}{0pt}%
\pgfpathmoveto{\pgfqpoint{4.380342in}{2.599906in}}%
\pgfpathcurveto{\pgfqpoint{4.392051in}{2.599906in}}{\pgfqpoint{4.403282in}{2.604558in}}{\pgfqpoint{4.411561in}{2.612838in}}%
\pgfpathcurveto{\pgfqpoint{4.419840in}{2.621117in}}{\pgfqpoint{4.424492in}{2.632347in}}{\pgfqpoint{4.424492in}{2.644056in}}%
\pgfpathcurveto{\pgfqpoint{4.424492in}{2.655765in}}{\pgfqpoint{4.419840in}{2.666996in}}{\pgfqpoint{4.411561in}{2.675275in}}%
\pgfpathcurveto{\pgfqpoint{4.403282in}{2.683554in}}{\pgfqpoint{4.392051in}{2.688206in}}{\pgfqpoint{4.380342in}{2.688206in}}%
\pgfpathcurveto{\pgfqpoint{4.368634in}{2.688206in}}{\pgfqpoint{4.357403in}{2.683554in}}{\pgfqpoint{4.349124in}{2.675275in}}%
\pgfpathcurveto{\pgfqpoint{4.340844in}{2.666996in}}{\pgfqpoint{4.336192in}{2.655765in}}{\pgfqpoint{4.336192in}{2.644056in}}%
\pgfpathcurveto{\pgfqpoint{4.336192in}{2.632347in}}{\pgfqpoint{4.340844in}{2.621117in}}{\pgfqpoint{4.349124in}{2.612838in}}%
\pgfpathcurveto{\pgfqpoint{4.357403in}{2.604558in}}{\pgfqpoint{4.368634in}{2.599906in}}{\pgfqpoint{4.380342in}{2.599906in}}%
\pgfpathclose%
\pgfusepath{stroke,fill}%
\end{pgfscope}%
\begin{pgfscope}%
\pgfpathrectangle{\pgfqpoint{0.800000in}{0.528000in}}{\pgfqpoint{3.968000in}{3.696000in}} %
\pgfusepath{clip}%
\pgfsetbuttcap%
\pgfsetroundjoin%
\definecolor{currentfill}{rgb}{1.000000,0.498039,0.054902}%
\pgfsetfillcolor{currentfill}%
\pgfsetlinewidth{1.003750pt}%
\definecolor{currentstroke}{rgb}{1.000000,0.498039,0.054902}%
\pgfsetstrokecolor{currentstroke}%
\pgfsetdash{}{0pt}%
\pgfpathmoveto{\pgfqpoint{3.662692in}{0.997406in}}%
\pgfpathcurveto{\pgfqpoint{3.676759in}{0.997406in}}{\pgfqpoint{3.690252in}{1.002995in}}{\pgfqpoint{3.700200in}{1.012942in}}%
\pgfpathcurveto{\pgfqpoint{3.710147in}{1.022890in}}{\pgfqpoint{3.715736in}{1.036383in}}{\pgfqpoint{3.715736in}{1.050450in}}%
\pgfpathcurveto{\pgfqpoint{3.715736in}{1.064518in}}{\pgfqpoint{3.710147in}{1.078011in}}{\pgfqpoint{3.700200in}{1.087958in}}%
\pgfpathcurveto{\pgfqpoint{3.690252in}{1.097906in}}{\pgfqpoint{3.676759in}{1.103495in}}{\pgfqpoint{3.662692in}{1.103495in}}%
\pgfpathcurveto{\pgfqpoint{3.648624in}{1.103495in}}{\pgfqpoint{3.635131in}{1.097906in}}{\pgfqpoint{3.625183in}{1.087958in}}%
\pgfpathcurveto{\pgfqpoint{3.615236in}{1.078011in}}{\pgfqpoint{3.609647in}{1.064518in}}{\pgfqpoint{3.609647in}{1.050450in}}%
\pgfpathcurveto{\pgfqpoint{3.609647in}{1.036383in}}{\pgfqpoint{3.615236in}{1.022890in}}{\pgfqpoint{3.625183in}{1.012942in}}%
\pgfpathcurveto{\pgfqpoint{3.635131in}{1.002995in}}{\pgfqpoint{3.648624in}{0.997406in}}{\pgfqpoint{3.662692in}{0.997406in}}%
\pgfpathclose%
\pgfusepath{stroke,fill}%
\end{pgfscope}%
\begin{pgfscope}%
\pgfpathrectangle{\pgfqpoint{0.800000in}{0.528000in}}{\pgfqpoint{3.968000in}{3.696000in}} %
\pgfusepath{clip}%
\pgfsetbuttcap%
\pgfsetroundjoin%
\definecolor{currentfill}{rgb}{1.000000,0.498039,0.054902}%
\pgfsetfillcolor{currentfill}%
\pgfsetlinewidth{1.003750pt}%
\definecolor{currentstroke}{rgb}{1.000000,0.498039,0.054902}%
\pgfsetstrokecolor{currentstroke}%
\pgfsetdash{}{0pt}%
\pgfpathmoveto{\pgfqpoint{1.807677in}{1.035908in}}%
\pgfpathcurveto{\pgfqpoint{1.813525in}{1.035908in}}{\pgfqpoint{1.819134in}{1.038231in}}{\pgfqpoint{1.823269in}{1.042366in}}%
\pgfpathcurveto{\pgfqpoint{1.827404in}{1.046501in}}{\pgfqpoint{1.829728in}{1.052110in}}{\pgfqpoint{1.829728in}{1.057958in}}%
\pgfpathcurveto{\pgfqpoint{1.829728in}{1.063806in}}{\pgfqpoint{1.827404in}{1.069415in}}{\pgfqpoint{1.823269in}{1.073550in}}%
\pgfpathcurveto{\pgfqpoint{1.819134in}{1.077685in}}{\pgfqpoint{1.813525in}{1.080008in}}{\pgfqpoint{1.807677in}{1.080008in}}%
\pgfpathcurveto{\pgfqpoint{1.801830in}{1.080008in}}{\pgfqpoint{1.796220in}{1.077685in}}{\pgfqpoint{1.792085in}{1.073550in}}%
\pgfpathcurveto{\pgfqpoint{1.787950in}{1.069415in}}{\pgfqpoint{1.785627in}{1.063806in}}{\pgfqpoint{1.785627in}{1.057958in}}%
\pgfpathcurveto{\pgfqpoint{1.785627in}{1.052110in}}{\pgfqpoint{1.787950in}{1.046501in}}{\pgfqpoint{1.792085in}{1.042366in}}%
\pgfpathcurveto{\pgfqpoint{1.796220in}{1.038231in}}{\pgfqpoint{1.801830in}{1.035908in}}{\pgfqpoint{1.807677in}{1.035908in}}%
\pgfpathclose%
\pgfusepath{stroke,fill}%
\end{pgfscope}%
\begin{pgfscope}%
\pgfpathrectangle{\pgfqpoint{0.800000in}{0.528000in}}{\pgfqpoint{3.968000in}{3.696000in}} %
\pgfusepath{clip}%
\pgfsetbuttcap%
\pgfsetroundjoin%
\definecolor{currentfill}{rgb}{1.000000,0.498039,0.054902}%
\pgfsetfillcolor{currentfill}%
\pgfsetlinewidth{1.003750pt}%
\definecolor{currentstroke}{rgb}{1.000000,0.498039,0.054902}%
\pgfsetstrokecolor{currentstroke}%
\pgfsetdash{}{0pt}%
\pgfpathmoveto{\pgfqpoint{2.993089in}{2.342702in}}%
\pgfpathcurveto{\pgfqpoint{2.998305in}{2.342702in}}{\pgfqpoint{3.003309in}{2.344774in}}{\pgfqpoint{3.006997in}{2.348463in}}%
\pgfpathcurveto{\pgfqpoint{3.010686in}{2.352152in}}{\pgfqpoint{3.012759in}{2.357155in}}{\pgfqpoint{3.012759in}{2.362372in}}%
\pgfpathcurveto{\pgfqpoint{3.012759in}{2.367588in}}{\pgfqpoint{3.010686in}{2.372592in}}{\pgfqpoint{3.006997in}{2.376280in}}%
\pgfpathcurveto{\pgfqpoint{3.003309in}{2.379969in}}{\pgfqpoint{2.998305in}{2.382041in}}{\pgfqpoint{2.993089in}{2.382041in}}%
\pgfpathcurveto{\pgfqpoint{2.987872in}{2.382041in}}{\pgfqpoint{2.982869in}{2.379969in}}{\pgfqpoint{2.979180in}{2.376280in}}%
\pgfpathcurveto{\pgfqpoint{2.975492in}{2.372592in}}{\pgfqpoint{2.973419in}{2.367588in}}{\pgfqpoint{2.973419in}{2.362372in}}%
\pgfpathcurveto{\pgfqpoint{2.973419in}{2.357155in}}{\pgfqpoint{2.975492in}{2.352152in}}{\pgfqpoint{2.979180in}{2.348463in}}%
\pgfpathcurveto{\pgfqpoint{2.982869in}{2.344774in}}{\pgfqpoint{2.987872in}{2.342702in}}{\pgfqpoint{2.993089in}{2.342702in}}%
\pgfpathclose%
\pgfusepath{stroke,fill}%
\end{pgfscope}%
\begin{pgfscope}%
\pgfpathrectangle{\pgfqpoint{0.800000in}{0.528000in}}{\pgfqpoint{3.968000in}{3.696000in}} %
\pgfusepath{clip}%
\pgfsetbuttcap%
\pgfsetroundjoin%
\definecolor{currentfill}{rgb}{1.000000,0.498039,0.054902}%
\pgfsetfillcolor{currentfill}%
\pgfsetlinewidth{1.003750pt}%
\definecolor{currentstroke}{rgb}{1.000000,0.498039,0.054902}%
\pgfsetstrokecolor{currentstroke}%
\pgfsetdash{}{0pt}%
\pgfpathmoveto{\pgfqpoint{2.890706in}{0.652585in}}%
\pgfpathcurveto{\pgfqpoint{2.904309in}{0.652585in}}{\pgfqpoint{2.917357in}{0.657990in}}{\pgfqpoint{2.926976in}{0.667609in}}%
\pgfpathcurveto{\pgfqpoint{2.936595in}{0.677228in}}{\pgfqpoint{2.942000in}{0.690276in}}{\pgfqpoint{2.942000in}{0.703879in}}%
\pgfpathcurveto{\pgfqpoint{2.942000in}{0.717483in}}{\pgfqpoint{2.936595in}{0.730531in}}{\pgfqpoint{2.926976in}{0.740150in}}%
\pgfpathcurveto{\pgfqpoint{2.917357in}{0.749769in}}{\pgfqpoint{2.904309in}{0.755174in}}{\pgfqpoint{2.890706in}{0.755174in}}%
\pgfpathcurveto{\pgfqpoint{2.877102in}{0.755174in}}{\pgfqpoint{2.864054in}{0.749769in}}{\pgfqpoint{2.854435in}{0.740150in}}%
\pgfpathcurveto{\pgfqpoint{2.844816in}{0.730531in}}{\pgfqpoint{2.839411in}{0.717483in}}{\pgfqpoint{2.839411in}{0.703879in}}%
\pgfpathcurveto{\pgfqpoint{2.839411in}{0.690276in}}{\pgfqpoint{2.844816in}{0.677228in}}{\pgfqpoint{2.854435in}{0.667609in}}%
\pgfpathcurveto{\pgfqpoint{2.864054in}{0.657990in}}{\pgfqpoint{2.877102in}{0.652585in}}{\pgfqpoint{2.890706in}{0.652585in}}%
\pgfpathclose%
\pgfusepath{stroke,fill}%
\end{pgfscope}%
\begin{pgfscope}%
\pgfpathrectangle{\pgfqpoint{0.800000in}{0.528000in}}{\pgfqpoint{3.968000in}{3.696000in}} %
\pgfusepath{clip}%
\pgfsetbuttcap%
\pgfsetroundjoin%
\definecolor{currentfill}{rgb}{1.000000,0.498039,0.054902}%
\pgfsetfillcolor{currentfill}%
\pgfsetlinewidth{1.003750pt}%
\definecolor{currentstroke}{rgb}{1.000000,0.498039,0.054902}%
\pgfsetstrokecolor{currentstroke}%
\pgfsetdash{}{0pt}%
\pgfpathmoveto{\pgfqpoint{3.747397in}{2.745211in}}%
\pgfpathcurveto{\pgfqpoint{3.759156in}{2.745211in}}{\pgfqpoint{3.770434in}{2.749883in}}{\pgfqpoint{3.778749in}{2.758197in}}%
\pgfpathcurveto{\pgfqpoint{3.787064in}{2.766512in}}{\pgfqpoint{3.791736in}{2.777791in}}{\pgfqpoint{3.791736in}{2.789549in}}%
\pgfpathcurveto{\pgfqpoint{3.791736in}{2.801308in}}{\pgfqpoint{3.787064in}{2.812587in}}{\pgfqpoint{3.778749in}{2.820901in}}%
\pgfpathcurveto{\pgfqpoint{3.770434in}{2.829216in}}{\pgfqpoint{3.759156in}{2.833888in}}{\pgfqpoint{3.747397in}{2.833888in}}%
\pgfpathcurveto{\pgfqpoint{3.735638in}{2.833888in}}{\pgfqpoint{3.724359in}{2.829216in}}{\pgfqpoint{3.716045in}{2.820901in}}%
\pgfpathcurveto{\pgfqpoint{3.707730in}{2.812587in}}{\pgfqpoint{3.703058in}{2.801308in}}{\pgfqpoint{3.703058in}{2.789549in}}%
\pgfpathcurveto{\pgfqpoint{3.703058in}{2.777791in}}{\pgfqpoint{3.707730in}{2.766512in}}{\pgfqpoint{3.716045in}{2.758197in}}%
\pgfpathcurveto{\pgfqpoint{3.724359in}{2.749883in}}{\pgfqpoint{3.735638in}{2.745211in}}{\pgfqpoint{3.747397in}{2.745211in}}%
\pgfpathclose%
\pgfusepath{stroke,fill}%
\end{pgfscope}%
\begin{pgfscope}%
\pgfpathrectangle{\pgfqpoint{0.800000in}{0.528000in}}{\pgfqpoint{3.968000in}{3.696000in}} %
\pgfusepath{clip}%
\pgfsetbuttcap%
\pgfsetroundjoin%
\definecolor{currentfill}{rgb}{1.000000,0.498039,0.054902}%
\pgfsetfillcolor{currentfill}%
\pgfsetlinewidth{1.003750pt}%
\definecolor{currentstroke}{rgb}{1.000000,0.498039,0.054902}%
\pgfsetstrokecolor{currentstroke}%
\pgfsetdash{}{0pt}%
\pgfpathmoveto{\pgfqpoint{2.653614in}{1.359187in}}%
\pgfpathcurveto{\pgfqpoint{2.657233in}{1.359187in}}{\pgfqpoint{2.660704in}{1.360624in}}{\pgfqpoint{2.663263in}{1.363183in}}%
\pgfpathcurveto{\pgfqpoint{2.665822in}{1.365742in}}{\pgfqpoint{2.667260in}{1.369213in}}{\pgfqpoint{2.667260in}{1.372832in}}%
\pgfpathcurveto{\pgfqpoint{2.667260in}{1.376451in}}{\pgfqpoint{2.665822in}{1.379922in}}{\pgfqpoint{2.663263in}{1.382481in}}%
\pgfpathcurveto{\pgfqpoint{2.660704in}{1.385040in}}{\pgfqpoint{2.657233in}{1.386478in}}{\pgfqpoint{2.653614in}{1.386478in}}%
\pgfpathcurveto{\pgfqpoint{2.649995in}{1.386478in}}{\pgfqpoint{2.646524in}{1.385040in}}{\pgfqpoint{2.643965in}{1.382481in}}%
\pgfpathcurveto{\pgfqpoint{2.641406in}{1.379922in}}{\pgfqpoint{2.639968in}{1.376451in}}{\pgfqpoint{2.639968in}{1.372832in}}%
\pgfpathcurveto{\pgfqpoint{2.639968in}{1.369213in}}{\pgfqpoint{2.641406in}{1.365742in}}{\pgfqpoint{2.643965in}{1.363183in}}%
\pgfpathcurveto{\pgfqpoint{2.646524in}{1.360624in}}{\pgfqpoint{2.649995in}{1.359187in}}{\pgfqpoint{2.653614in}{1.359187in}}%
\pgfpathclose%
\pgfusepath{stroke,fill}%
\end{pgfscope}%
\begin{pgfscope}%
\pgfpathrectangle{\pgfqpoint{0.800000in}{0.528000in}}{\pgfqpoint{3.968000in}{3.696000in}} %
\pgfusepath{clip}%
\pgfsetbuttcap%
\pgfsetroundjoin%
\definecolor{currentfill}{rgb}{1.000000,0.498039,0.054902}%
\pgfsetfillcolor{currentfill}%
\pgfsetlinewidth{1.003750pt}%
\definecolor{currentstroke}{rgb}{1.000000,0.498039,0.054902}%
\pgfsetstrokecolor{currentstroke}%
\pgfsetdash{}{0pt}%
\pgfpathmoveto{\pgfqpoint{3.889098in}{2.413667in}}%
\pgfpathcurveto{\pgfqpoint{3.901342in}{2.413667in}}{\pgfqpoint{3.913085in}{2.418532in}}{\pgfqpoint{3.921743in}{2.427189in}}%
\pgfpathcurveto{\pgfqpoint{3.930400in}{2.435847in}}{\pgfqpoint{3.935265in}{2.447590in}}{\pgfqpoint{3.935265in}{2.459834in}}%
\pgfpathcurveto{\pgfqpoint{3.935265in}{2.472077in}}{\pgfqpoint{3.930400in}{2.483821in}}{\pgfqpoint{3.921743in}{2.492479in}}%
\pgfpathcurveto{\pgfqpoint{3.913085in}{2.501136in}}{\pgfqpoint{3.901342in}{2.506001in}}{\pgfqpoint{3.889098in}{2.506001in}}%
\pgfpathcurveto{\pgfqpoint{3.876854in}{2.506001in}}{\pgfqpoint{3.865111in}{2.501136in}}{\pgfqpoint{3.856453in}{2.492479in}}%
\pgfpathcurveto{\pgfqpoint{3.847796in}{2.483821in}}{\pgfqpoint{3.842931in}{2.472077in}}{\pgfqpoint{3.842931in}{2.459834in}}%
\pgfpathcurveto{\pgfqpoint{3.842931in}{2.447590in}}{\pgfqpoint{3.847796in}{2.435847in}}{\pgfqpoint{3.856453in}{2.427189in}}%
\pgfpathcurveto{\pgfqpoint{3.865111in}{2.418532in}}{\pgfqpoint{3.876854in}{2.413667in}}{\pgfqpoint{3.889098in}{2.413667in}}%
\pgfpathclose%
\pgfusepath{stroke,fill}%
\end{pgfscope}%
\begin{pgfscope}%
\pgfpathrectangle{\pgfqpoint{0.800000in}{0.528000in}}{\pgfqpoint{3.968000in}{3.696000in}} %
\pgfusepath{clip}%
\pgfsetbuttcap%
\pgfsetroundjoin%
\definecolor{currentfill}{rgb}{1.000000,0.498039,0.054902}%
\pgfsetfillcolor{currentfill}%
\pgfsetlinewidth{1.003750pt}%
\definecolor{currentstroke}{rgb}{1.000000,0.498039,0.054902}%
\pgfsetstrokecolor{currentstroke}%
\pgfsetdash{}{0pt}%
\pgfpathmoveto{\pgfqpoint{2.779516in}{2.890138in}}%
\pgfpathcurveto{\pgfqpoint{2.791746in}{2.890138in}}{\pgfqpoint{2.803478in}{2.894997in}}{\pgfqpoint{2.812126in}{2.903645in}}%
\pgfpathcurveto{\pgfqpoint{2.820775in}{2.912294in}}{\pgfqpoint{2.825634in}{2.924025in}}{\pgfqpoint{2.825634in}{2.936256in}}%
\pgfpathcurveto{\pgfqpoint{2.825634in}{2.948487in}}{\pgfqpoint{2.820775in}{2.960218in}}{\pgfqpoint{2.812126in}{2.968866in}}%
\pgfpathcurveto{\pgfqpoint{2.803478in}{2.977515in}}{\pgfqpoint{2.791746in}{2.982374in}}{\pgfqpoint{2.779516in}{2.982374in}}%
\pgfpathcurveto{\pgfqpoint{2.767285in}{2.982374in}}{\pgfqpoint{2.755554in}{2.977515in}}{\pgfqpoint{2.746905in}{2.968866in}}%
\pgfpathcurveto{\pgfqpoint{2.738257in}{2.960218in}}{\pgfqpoint{2.733398in}{2.948487in}}{\pgfqpoint{2.733398in}{2.936256in}}%
\pgfpathcurveto{\pgfqpoint{2.733398in}{2.924025in}}{\pgfqpoint{2.738257in}{2.912294in}}{\pgfqpoint{2.746905in}{2.903645in}}%
\pgfpathcurveto{\pgfqpoint{2.755554in}{2.894997in}}{\pgfqpoint{2.767285in}{2.890138in}}{\pgfqpoint{2.779516in}{2.890138in}}%
\pgfpathclose%
\pgfusepath{stroke,fill}%
\end{pgfscope}%
\begin{pgfscope}%
\pgfpathrectangle{\pgfqpoint{0.800000in}{0.528000in}}{\pgfqpoint{3.968000in}{3.696000in}} %
\pgfusepath{clip}%
\pgfsetbuttcap%
\pgfsetroundjoin%
\definecolor{currentfill}{rgb}{1.000000,0.498039,0.054902}%
\pgfsetfillcolor{currentfill}%
\pgfsetlinewidth{1.003750pt}%
\definecolor{currentstroke}{rgb}{1.000000,0.498039,0.054902}%
\pgfsetstrokecolor{currentstroke}%
\pgfsetdash{}{0pt}%
\pgfpathmoveto{\pgfqpoint{3.637863in}{3.997131in}}%
\pgfpathcurveto{\pgfqpoint{3.651396in}{3.997131in}}{\pgfqpoint{3.664377in}{4.002508in}}{\pgfqpoint{3.673947in}{4.012077in}}%
\pgfpathcurveto{\pgfqpoint{3.683516in}{4.021647in}}{\pgfqpoint{3.688893in}{4.034628in}}{\pgfqpoint{3.688893in}{4.048161in}}%
\pgfpathcurveto{\pgfqpoint{3.688893in}{4.061695in}}{\pgfqpoint{3.683516in}{4.074676in}}{\pgfqpoint{3.673947in}{4.084245in}}%
\pgfpathcurveto{\pgfqpoint{3.664377in}{4.093815in}}{\pgfqpoint{3.651396in}{4.099192in}}{\pgfqpoint{3.637863in}{4.099192in}}%
\pgfpathcurveto{\pgfqpoint{3.624329in}{4.099192in}}{\pgfqpoint{3.611348in}{4.093815in}}{\pgfqpoint{3.601779in}{4.084245in}}%
\pgfpathcurveto{\pgfqpoint{3.592209in}{4.074676in}}{\pgfqpoint{3.586832in}{4.061695in}}{\pgfqpoint{3.586832in}{4.048161in}}%
\pgfpathcurveto{\pgfqpoint{3.586832in}{4.034628in}}{\pgfqpoint{3.592209in}{4.021647in}}{\pgfqpoint{3.601779in}{4.012077in}}%
\pgfpathcurveto{\pgfqpoint{3.611348in}{4.002508in}}{\pgfqpoint{3.624329in}{3.997131in}}{\pgfqpoint{3.637863in}{3.997131in}}%
\pgfpathclose%
\pgfusepath{stroke,fill}%
\end{pgfscope}%
\begin{pgfscope}%
\pgfpathrectangle{\pgfqpoint{0.800000in}{0.528000in}}{\pgfqpoint{3.968000in}{3.696000in}} %
\pgfusepath{clip}%
\pgfsetbuttcap%
\pgfsetroundjoin%
\definecolor{currentfill}{rgb}{1.000000,0.498039,0.054902}%
\pgfsetfillcolor{currentfill}%
\pgfsetlinewidth{1.003750pt}%
\definecolor{currentstroke}{rgb}{1.000000,0.498039,0.054902}%
\pgfsetstrokecolor{currentstroke}%
\pgfsetdash{}{0pt}%
\pgfpathmoveto{\pgfqpoint{1.926844in}{2.205060in}}%
\pgfpathcurveto{\pgfqpoint{1.938726in}{2.205060in}}{\pgfqpoint{1.950123in}{2.209781in}}{\pgfqpoint{1.958525in}{2.218183in}}%
\pgfpathcurveto{\pgfqpoint{1.966927in}{2.226585in}}{\pgfqpoint{1.971648in}{2.237982in}}{\pgfqpoint{1.971648in}{2.249864in}}%
\pgfpathcurveto{\pgfqpoint{1.971648in}{2.261747in}}{\pgfqpoint{1.966927in}{2.273144in}}{\pgfqpoint{1.958525in}{2.281546in}}%
\pgfpathcurveto{\pgfqpoint{1.950123in}{2.289948in}}{\pgfqpoint{1.938726in}{2.294669in}}{\pgfqpoint{1.926844in}{2.294669in}}%
\pgfpathcurveto{\pgfqpoint{1.914962in}{2.294669in}}{\pgfqpoint{1.903565in}{2.289948in}}{\pgfqpoint{1.895163in}{2.281546in}}%
\pgfpathcurveto{\pgfqpoint{1.886761in}{2.273144in}}{\pgfqpoint{1.882040in}{2.261747in}}{\pgfqpoint{1.882040in}{2.249864in}}%
\pgfpathcurveto{\pgfqpoint{1.882040in}{2.237982in}}{\pgfqpoint{1.886761in}{2.226585in}}{\pgfqpoint{1.895163in}{2.218183in}}%
\pgfpathcurveto{\pgfqpoint{1.903565in}{2.209781in}}{\pgfqpoint{1.914962in}{2.205060in}}{\pgfqpoint{1.926844in}{2.205060in}}%
\pgfpathclose%
\pgfusepath{stroke,fill}%
\end{pgfscope}%
\begin{pgfscope}%
\pgfpathrectangle{\pgfqpoint{0.800000in}{0.528000in}}{\pgfqpoint{3.968000in}{3.696000in}} %
\pgfusepath{clip}%
\pgfsetbuttcap%
\pgfsetroundjoin%
\definecolor{currentfill}{rgb}{1.000000,0.498039,0.054902}%
\pgfsetfillcolor{currentfill}%
\pgfsetlinewidth{1.003750pt}%
\definecolor{currentstroke}{rgb}{1.000000,0.498039,0.054902}%
\pgfsetstrokecolor{currentstroke}%
\pgfsetdash{}{0pt}%
\pgfpathmoveto{\pgfqpoint{2.645896in}{1.712695in}}%
\pgfpathcurveto{\pgfqpoint{2.650862in}{1.712695in}}{\pgfqpoint{2.655625in}{1.714668in}}{\pgfqpoint{2.659136in}{1.718179in}}%
\pgfpathcurveto{\pgfqpoint{2.662648in}{1.721691in}}{\pgfqpoint{2.664621in}{1.726454in}}{\pgfqpoint{2.664621in}{1.731420in}}%
\pgfpathcurveto{\pgfqpoint{2.664621in}{1.736385in}}{\pgfqpoint{2.662648in}{1.741148in}}{\pgfqpoint{2.659136in}{1.744660in}}%
\pgfpathcurveto{\pgfqpoint{2.655625in}{1.748171in}}{\pgfqpoint{2.650862in}{1.750144in}}{\pgfqpoint{2.645896in}{1.750144in}}%
\pgfpathcurveto{\pgfqpoint{2.640931in}{1.750144in}}{\pgfqpoint{2.636168in}{1.748171in}}{\pgfqpoint{2.632656in}{1.744660in}}%
\pgfpathcurveto{\pgfqpoint{2.629145in}{1.741148in}}{\pgfqpoint{2.627172in}{1.736385in}}{\pgfqpoint{2.627172in}{1.731420in}}%
\pgfpathcurveto{\pgfqpoint{2.627172in}{1.726454in}}{\pgfqpoint{2.629145in}{1.721691in}}{\pgfqpoint{2.632656in}{1.718179in}}%
\pgfpathcurveto{\pgfqpoint{2.636168in}{1.714668in}}{\pgfqpoint{2.640931in}{1.712695in}}{\pgfqpoint{2.645896in}{1.712695in}}%
\pgfpathclose%
\pgfusepath{stroke,fill}%
\end{pgfscope}%
\begin{pgfscope}%
\pgfpathrectangle{\pgfqpoint{0.800000in}{0.528000in}}{\pgfqpoint{3.968000in}{3.696000in}} %
\pgfusepath{clip}%
\pgfsetbuttcap%
\pgfsetroundjoin%
\definecolor{currentfill}{rgb}{1.000000,0.498039,0.054902}%
\pgfsetfillcolor{currentfill}%
\pgfsetlinewidth{1.003750pt}%
\definecolor{currentstroke}{rgb}{1.000000,0.498039,0.054902}%
\pgfsetstrokecolor{currentstroke}%
\pgfsetdash{}{0pt}%
\pgfpathmoveto{\pgfqpoint{3.872209in}{2.389467in}}%
\pgfpathcurveto{\pgfqpoint{3.874195in}{2.389467in}}{\pgfqpoint{3.876101in}{2.390256in}}{\pgfqpoint{3.877505in}{2.391660in}}%
\pgfpathcurveto{\pgfqpoint{3.878909in}{2.393064in}}{\pgfqpoint{3.879698in}{2.394970in}}{\pgfqpoint{3.879698in}{2.396956in}}%
\pgfpathcurveto{\pgfqpoint{3.879698in}{2.398942in}}{\pgfqpoint{3.878909in}{2.400847in}}{\pgfqpoint{3.877505in}{2.402251in}}%
\pgfpathcurveto{\pgfqpoint{3.876101in}{2.403656in}}{\pgfqpoint{3.874195in}{2.404445in}}{\pgfqpoint{3.872209in}{2.404445in}}%
\pgfpathcurveto{\pgfqpoint{3.870223in}{2.404445in}}{\pgfqpoint{3.868318in}{2.403656in}}{\pgfqpoint{3.866914in}{2.402251in}}%
\pgfpathcurveto{\pgfqpoint{3.865509in}{2.400847in}}{\pgfqpoint{3.864720in}{2.398942in}}{\pgfqpoint{3.864720in}{2.396956in}}%
\pgfpathcurveto{\pgfqpoint{3.864720in}{2.394970in}}{\pgfqpoint{3.865509in}{2.393064in}}{\pgfqpoint{3.866914in}{2.391660in}}%
\pgfpathcurveto{\pgfqpoint{3.868318in}{2.390256in}}{\pgfqpoint{3.870223in}{2.389467in}}{\pgfqpoint{3.872209in}{2.389467in}}%
\pgfpathclose%
\pgfusepath{stroke,fill}%
\end{pgfscope}%
\begin{pgfscope}%
\pgfpathrectangle{\pgfqpoint{0.800000in}{0.528000in}}{\pgfqpoint{3.968000in}{3.696000in}} %
\pgfusepath{clip}%
\pgfsetbuttcap%
\pgfsetroundjoin%
\definecolor{currentfill}{rgb}{1.000000,0.498039,0.054902}%
\pgfsetfillcolor{currentfill}%
\pgfsetlinewidth{1.003750pt}%
\definecolor{currentstroke}{rgb}{1.000000,0.498039,0.054902}%
\pgfsetstrokecolor{currentstroke}%
\pgfsetdash{}{0pt}%
\pgfpathmoveto{\pgfqpoint{3.136514in}{3.139704in}}%
\pgfpathcurveto{\pgfqpoint{3.138794in}{3.139704in}}{\pgfqpoint{3.140982in}{3.140610in}}{\pgfqpoint{3.142594in}{3.142223in}}%
\pgfpathcurveto{\pgfqpoint{3.144207in}{3.143835in}}{\pgfqpoint{3.145113in}{3.146023in}}{\pgfqpoint{3.145113in}{3.148303in}}%
\pgfpathcurveto{\pgfqpoint{3.145113in}{3.150584in}}{\pgfqpoint{3.144207in}{3.152771in}}{\pgfqpoint{3.142594in}{3.154384in}}%
\pgfpathcurveto{\pgfqpoint{3.140982in}{3.155997in}}{\pgfqpoint{3.138794in}{3.156903in}}{\pgfqpoint{3.136514in}{3.156903in}}%
\pgfpathcurveto{\pgfqpoint{3.134233in}{3.156903in}}{\pgfqpoint{3.132045in}{3.155997in}}{\pgfqpoint{3.130433in}{3.154384in}}%
\pgfpathcurveto{\pgfqpoint{3.128820in}{3.152771in}}{\pgfqpoint{3.127914in}{3.150584in}}{\pgfqpoint{3.127914in}{3.148303in}}%
\pgfpathcurveto{\pgfqpoint{3.127914in}{3.146023in}}{\pgfqpoint{3.128820in}{3.143835in}}{\pgfqpoint{3.130433in}{3.142223in}}%
\pgfpathcurveto{\pgfqpoint{3.132045in}{3.140610in}}{\pgfqpoint{3.134233in}{3.139704in}}{\pgfqpoint{3.136514in}{3.139704in}}%
\pgfpathclose%
\pgfusepath{stroke,fill}%
\end{pgfscope}%
\begin{pgfscope}%
\pgfpathrectangle{\pgfqpoint{0.800000in}{0.528000in}}{\pgfqpoint{3.968000in}{3.696000in}} %
\pgfusepath{clip}%
\pgfsetbuttcap%
\pgfsetroundjoin%
\definecolor{currentfill}{rgb}{1.000000,0.498039,0.054902}%
\pgfsetfillcolor{currentfill}%
\pgfsetlinewidth{1.003750pt}%
\definecolor{currentstroke}{rgb}{1.000000,0.498039,0.054902}%
\pgfsetstrokecolor{currentstroke}%
\pgfsetdash{}{0pt}%
\pgfpathmoveto{\pgfqpoint{4.142346in}{3.054875in}}%
\pgfpathcurveto{\pgfqpoint{4.144716in}{3.054875in}}{\pgfqpoint{4.146989in}{3.055817in}}{\pgfqpoint{4.148665in}{3.057493in}}%
\pgfpathcurveto{\pgfqpoint{4.150341in}{3.059168in}}{\pgfqpoint{4.151283in}{3.061442in}}{\pgfqpoint{4.151283in}{3.063812in}}%
\pgfpathcurveto{\pgfqpoint{4.151283in}{3.066182in}}{\pgfqpoint{4.150341in}{3.068455in}}{\pgfqpoint{4.148665in}{3.070131in}}%
\pgfpathcurveto{\pgfqpoint{4.146989in}{3.071806in}}{\pgfqpoint{4.144716in}{3.072748in}}{\pgfqpoint{4.142346in}{3.072748in}}%
\pgfpathcurveto{\pgfqpoint{4.139976in}{3.072748in}}{\pgfqpoint{4.137703in}{3.071806in}}{\pgfqpoint{4.136027in}{3.070131in}}%
\pgfpathcurveto{\pgfqpoint{4.134351in}{3.068455in}}{\pgfqpoint{4.133410in}{3.066182in}}{\pgfqpoint{4.133410in}{3.063812in}}%
\pgfpathcurveto{\pgfqpoint{4.133410in}{3.061442in}}{\pgfqpoint{4.134351in}{3.059168in}}{\pgfqpoint{4.136027in}{3.057493in}}%
\pgfpathcurveto{\pgfqpoint{4.137703in}{3.055817in}}{\pgfqpoint{4.139976in}{3.054875in}}{\pgfqpoint{4.142346in}{3.054875in}}%
\pgfpathclose%
\pgfusepath{stroke,fill}%
\end{pgfscope}%
\begin{pgfscope}%
\pgfpathrectangle{\pgfqpoint{0.800000in}{0.528000in}}{\pgfqpoint{3.968000in}{3.696000in}} %
\pgfusepath{clip}%
\pgfsetbuttcap%
\pgfsetroundjoin%
\definecolor{currentfill}{rgb}{1.000000,0.498039,0.054902}%
\pgfsetfillcolor{currentfill}%
\pgfsetlinewidth{1.003750pt}%
\definecolor{currentstroke}{rgb}{1.000000,0.498039,0.054902}%
\pgfsetstrokecolor{currentstroke}%
\pgfsetdash{}{0pt}%
\pgfpathmoveto{\pgfqpoint{4.409186in}{2.837846in}}%
\pgfpathcurveto{\pgfqpoint{4.416176in}{2.837846in}}{\pgfqpoint{4.422880in}{2.840623in}}{\pgfqpoint{4.427822in}{2.845565in}}%
\pgfpathcurveto{\pgfqpoint{4.432765in}{2.850508in}}{\pgfqpoint{4.435542in}{2.857212in}}{\pgfqpoint{4.435542in}{2.864201in}}%
\pgfpathcurveto{\pgfqpoint{4.435542in}{2.871191in}}{\pgfqpoint{4.432765in}{2.877895in}}{\pgfqpoint{4.427822in}{2.882838in}}%
\pgfpathcurveto{\pgfqpoint{4.422880in}{2.887780in}}{\pgfqpoint{4.416176in}{2.890557in}}{\pgfqpoint{4.409186in}{2.890557in}}%
\pgfpathcurveto{\pgfqpoint{4.402196in}{2.890557in}}{\pgfqpoint{4.395492in}{2.887780in}}{\pgfqpoint{4.390550in}{2.882838in}}%
\pgfpathcurveto{\pgfqpoint{4.385607in}{2.877895in}}{\pgfqpoint{4.382830in}{2.871191in}}{\pgfqpoint{4.382830in}{2.864201in}}%
\pgfpathcurveto{\pgfqpoint{4.382830in}{2.857212in}}{\pgfqpoint{4.385607in}{2.850508in}}{\pgfqpoint{4.390550in}{2.845565in}}%
\pgfpathcurveto{\pgfqpoint{4.395492in}{2.840623in}}{\pgfqpoint{4.402196in}{2.837846in}}{\pgfqpoint{4.409186in}{2.837846in}}%
\pgfpathclose%
\pgfusepath{stroke,fill}%
\end{pgfscope}%
\begin{pgfscope}%
\pgfpathrectangle{\pgfqpoint{0.800000in}{0.528000in}}{\pgfqpoint{3.968000in}{3.696000in}} %
\pgfusepath{clip}%
\pgfsetbuttcap%
\pgfsetroundjoin%
\definecolor{currentfill}{rgb}{1.000000,0.498039,0.054902}%
\pgfsetfillcolor{currentfill}%
\pgfsetlinewidth{1.003750pt}%
\definecolor{currentstroke}{rgb}{1.000000,0.498039,0.054902}%
\pgfsetstrokecolor{currentstroke}%
\pgfsetdash{}{0pt}%
\pgfpathmoveto{\pgfqpoint{3.888362in}{3.167131in}}%
\pgfpathcurveto{\pgfqpoint{3.901425in}{3.167131in}}{\pgfqpoint{3.913955in}{3.172321in}}{\pgfqpoint{3.923193in}{3.181559in}}%
\pgfpathcurveto{\pgfqpoint{3.932430in}{3.190796in}}{\pgfqpoint{3.937620in}{3.203326in}}{\pgfqpoint{3.937620in}{3.216390in}}%
\pgfpathcurveto{\pgfqpoint{3.937620in}{3.229453in}}{\pgfqpoint{3.932430in}{3.241983in}}{\pgfqpoint{3.923193in}{3.251220in}}%
\pgfpathcurveto{\pgfqpoint{3.913955in}{3.260458in}}{\pgfqpoint{3.901425in}{3.265648in}}{\pgfqpoint{3.888362in}{3.265648in}}%
\pgfpathcurveto{\pgfqpoint{3.875298in}{3.265648in}}{\pgfqpoint{3.862768in}{3.260458in}}{\pgfqpoint{3.853531in}{3.251220in}}%
\pgfpathcurveto{\pgfqpoint{3.844294in}{3.241983in}}{\pgfqpoint{3.839104in}{3.229453in}}{\pgfqpoint{3.839104in}{3.216390in}}%
\pgfpathcurveto{\pgfqpoint{3.839104in}{3.203326in}}{\pgfqpoint{3.844294in}{3.190796in}}{\pgfqpoint{3.853531in}{3.181559in}}%
\pgfpathcurveto{\pgfqpoint{3.862768in}{3.172321in}}{\pgfqpoint{3.875298in}{3.167131in}}{\pgfqpoint{3.888362in}{3.167131in}}%
\pgfpathclose%
\pgfusepath{stroke,fill}%
\end{pgfscope}%
\begin{pgfscope}%
\pgfpathrectangle{\pgfqpoint{0.800000in}{0.528000in}}{\pgfqpoint{3.968000in}{3.696000in}} %
\pgfusepath{clip}%
\pgfsetbuttcap%
\pgfsetroundjoin%
\definecolor{currentfill}{rgb}{1.000000,0.498039,0.054902}%
\pgfsetfillcolor{currentfill}%
\pgfsetlinewidth{1.003750pt}%
\definecolor{currentstroke}{rgb}{1.000000,0.498039,0.054902}%
\pgfsetstrokecolor{currentstroke}%
\pgfsetdash{}{0pt}%
\pgfpathmoveto{\pgfqpoint{4.206578in}{3.623538in}}%
\pgfpathcurveto{\pgfqpoint{4.216918in}{3.623538in}}{\pgfqpoint{4.226836in}{3.627646in}}{\pgfqpoint{4.234148in}{3.634958in}}%
\pgfpathcurveto{\pgfqpoint{4.241460in}{3.642269in}}{\pgfqpoint{4.245568in}{3.652187in}}{\pgfqpoint{4.245568in}{3.662528in}}%
\pgfpathcurveto{\pgfqpoint{4.245568in}{3.672868in}}{\pgfqpoint{4.241460in}{3.682786in}}{\pgfqpoint{4.234148in}{3.690097in}}%
\pgfpathcurveto{\pgfqpoint{4.226836in}{3.697409in}}{\pgfqpoint{4.216918in}{3.701517in}}{\pgfqpoint{4.206578in}{3.701517in}}%
\pgfpathcurveto{\pgfqpoint{4.196238in}{3.701517in}}{\pgfqpoint{4.186320in}{3.697409in}}{\pgfqpoint{4.179008in}{3.690097in}}%
\pgfpathcurveto{\pgfqpoint{4.171697in}{3.682786in}}{\pgfqpoint{4.167589in}{3.672868in}}{\pgfqpoint{4.167589in}{3.662528in}}%
\pgfpathcurveto{\pgfqpoint{4.167589in}{3.652187in}}{\pgfqpoint{4.171697in}{3.642269in}}{\pgfqpoint{4.179008in}{3.634958in}}%
\pgfpathcurveto{\pgfqpoint{4.186320in}{3.627646in}}{\pgfqpoint{4.196238in}{3.623538in}}{\pgfqpoint{4.206578in}{3.623538in}}%
\pgfpathclose%
\pgfusepath{stroke,fill}%
\end{pgfscope}%
\begin{pgfscope}%
\pgfpathrectangle{\pgfqpoint{0.800000in}{0.528000in}}{\pgfqpoint{3.968000in}{3.696000in}} %
\pgfusepath{clip}%
\pgfsetbuttcap%
\pgfsetroundjoin%
\definecolor{currentfill}{rgb}{1.000000,0.498039,0.054902}%
\pgfsetfillcolor{currentfill}%
\pgfsetlinewidth{1.003750pt}%
\definecolor{currentstroke}{rgb}{1.000000,0.498039,0.054902}%
\pgfsetstrokecolor{currentstroke}%
\pgfsetdash{}{0pt}%
\pgfpathmoveto{\pgfqpoint{0.986586in}{1.740597in}}%
\pgfpathcurveto{\pgfqpoint{0.997059in}{1.740597in}}{\pgfqpoint{1.007105in}{1.744758in}}{\pgfqpoint{1.014511in}{1.752164in}}%
\pgfpathcurveto{\pgfqpoint{1.021917in}{1.759570in}}{\pgfqpoint{1.026078in}{1.769616in}}{\pgfqpoint{1.026078in}{1.780089in}}%
\pgfpathcurveto{\pgfqpoint{1.026078in}{1.790563in}}{\pgfqpoint{1.021917in}{1.800609in}}{\pgfqpoint{1.014511in}{1.808015in}}%
\pgfpathcurveto{\pgfqpoint{1.007105in}{1.815421in}}{\pgfqpoint{0.997059in}{1.819582in}}{\pgfqpoint{0.986586in}{1.819582in}}%
\pgfpathcurveto{\pgfqpoint{0.976112in}{1.819582in}}{\pgfqpoint{0.966066in}{1.815421in}}{\pgfqpoint{0.958660in}{1.808015in}}%
\pgfpathcurveto{\pgfqpoint{0.951254in}{1.800609in}}{\pgfqpoint{0.947093in}{1.790563in}}{\pgfqpoint{0.947093in}{1.780089in}}%
\pgfpathcurveto{\pgfqpoint{0.947093in}{1.769616in}}{\pgfqpoint{0.951254in}{1.759570in}}{\pgfqpoint{0.958660in}{1.752164in}}%
\pgfpathcurveto{\pgfqpoint{0.966066in}{1.744758in}}{\pgfqpoint{0.976112in}{1.740597in}}{\pgfqpoint{0.986586in}{1.740597in}}%
\pgfpathclose%
\pgfusepath{stroke,fill}%
\end{pgfscope}%
\begin{pgfscope}%
\pgfpathrectangle{\pgfqpoint{0.800000in}{0.528000in}}{\pgfqpoint{3.968000in}{3.696000in}} %
\pgfusepath{clip}%
\pgfsetbuttcap%
\pgfsetroundjoin%
\definecolor{currentfill}{rgb}{1.000000,0.498039,0.054902}%
\pgfsetfillcolor{currentfill}%
\pgfsetlinewidth{1.003750pt}%
\definecolor{currentstroke}{rgb}{1.000000,0.498039,0.054902}%
\pgfsetstrokecolor{currentstroke}%
\pgfsetdash{}{0pt}%
\pgfpathmoveto{\pgfqpoint{3.915369in}{2.421526in}}%
\pgfpathcurveto{\pgfqpoint{3.928295in}{2.421526in}}{\pgfqpoint{3.940693in}{2.426662in}}{\pgfqpoint{3.949833in}{2.435802in}}%
\pgfpathcurveto{\pgfqpoint{3.958973in}{2.444942in}}{\pgfqpoint{3.964109in}{2.457340in}}{\pgfqpoint{3.964109in}{2.470266in}}%
\pgfpathcurveto{\pgfqpoint{3.964109in}{2.483192in}}{\pgfqpoint{3.958973in}{2.495591in}}{\pgfqpoint{3.949833in}{2.504731in}}%
\pgfpathcurveto{\pgfqpoint{3.940693in}{2.513871in}}{\pgfqpoint{3.928295in}{2.519006in}}{\pgfqpoint{3.915369in}{2.519006in}}%
\pgfpathcurveto{\pgfqpoint{3.902443in}{2.519006in}}{\pgfqpoint{3.890044in}{2.513871in}}{\pgfqpoint{3.880904in}{2.504731in}}%
\pgfpathcurveto{\pgfqpoint{3.871764in}{2.495591in}}{\pgfqpoint{3.866628in}{2.483192in}}{\pgfqpoint{3.866628in}{2.470266in}}%
\pgfpathcurveto{\pgfqpoint{3.866628in}{2.457340in}}{\pgfqpoint{3.871764in}{2.444942in}}{\pgfqpoint{3.880904in}{2.435802in}}%
\pgfpathcurveto{\pgfqpoint{3.890044in}{2.426662in}}{\pgfqpoint{3.902443in}{2.421526in}}{\pgfqpoint{3.915369in}{2.421526in}}%
\pgfpathclose%
\pgfusepath{stroke,fill}%
\end{pgfscope}%
\begin{pgfscope}%
\pgfpathrectangle{\pgfqpoint{0.800000in}{0.528000in}}{\pgfqpoint{3.968000in}{3.696000in}} %
\pgfusepath{clip}%
\pgfsetbuttcap%
\pgfsetroundjoin%
\definecolor{currentfill}{rgb}{1.000000,0.498039,0.054902}%
\pgfsetfillcolor{currentfill}%
\pgfsetlinewidth{1.003750pt}%
\definecolor{currentstroke}{rgb}{1.000000,0.498039,0.054902}%
\pgfsetstrokecolor{currentstroke}%
\pgfsetdash{}{0pt}%
\pgfpathmoveto{\pgfqpoint{2.935494in}{4.001232in}}%
\pgfpathcurveto{\pgfqpoint{2.940457in}{4.001232in}}{\pgfqpoint{2.945219in}{4.003204in}}{\pgfqpoint{2.948729in}{4.006714in}}%
\pgfpathcurveto{\pgfqpoint{2.952238in}{4.010224in}}{\pgfqpoint{2.954211in}{4.014985in}}{\pgfqpoint{2.954211in}{4.019949in}}%
\pgfpathcurveto{\pgfqpoint{2.954211in}{4.024913in}}{\pgfqpoint{2.952238in}{4.029674in}}{\pgfqpoint{2.948729in}{4.033184in}}%
\pgfpathcurveto{\pgfqpoint{2.945219in}{4.036694in}}{\pgfqpoint{2.940457in}{4.038666in}}{\pgfqpoint{2.935494in}{4.038666in}}%
\pgfpathcurveto{\pgfqpoint{2.930530in}{4.038666in}}{\pgfqpoint{2.925769in}{4.036694in}}{\pgfqpoint{2.922259in}{4.033184in}}%
\pgfpathcurveto{\pgfqpoint{2.918749in}{4.029674in}}{\pgfqpoint{2.916777in}{4.024913in}}{\pgfqpoint{2.916777in}{4.019949in}}%
\pgfpathcurveto{\pgfqpoint{2.916777in}{4.014985in}}{\pgfqpoint{2.918749in}{4.010224in}}{\pgfqpoint{2.922259in}{4.006714in}}%
\pgfpathcurveto{\pgfqpoint{2.925769in}{4.003204in}}{\pgfqpoint{2.930530in}{4.001232in}}{\pgfqpoint{2.935494in}{4.001232in}}%
\pgfpathclose%
\pgfusepath{stroke,fill}%
\end{pgfscope}%
\begin{pgfscope}%
\pgfpathrectangle{\pgfqpoint{0.800000in}{0.528000in}}{\pgfqpoint{3.968000in}{3.696000in}} %
\pgfusepath{clip}%
\pgfsetbuttcap%
\pgfsetroundjoin%
\definecolor{currentfill}{rgb}{1.000000,0.498039,0.054902}%
\pgfsetfillcolor{currentfill}%
\pgfsetlinewidth{1.003750pt}%
\definecolor{currentstroke}{rgb}{1.000000,0.498039,0.054902}%
\pgfsetstrokecolor{currentstroke}%
\pgfsetdash{}{0pt}%
\pgfpathmoveto{\pgfqpoint{3.509123in}{2.367980in}}%
\pgfpathcurveto{\pgfqpoint{3.516566in}{2.367980in}}{\pgfqpoint{3.523705in}{2.370938in}}{\pgfqpoint{3.528968in}{2.376201in}}%
\pgfpathcurveto{\pgfqpoint{3.534231in}{2.381464in}}{\pgfqpoint{3.537188in}{2.388603in}}{\pgfqpoint{3.537188in}{2.396046in}}%
\pgfpathcurveto{\pgfqpoint{3.537188in}{2.403489in}}{\pgfqpoint{3.534231in}{2.410628in}}{\pgfqpoint{3.528968in}{2.415891in}}%
\pgfpathcurveto{\pgfqpoint{3.523705in}{2.421154in}}{\pgfqpoint{3.516566in}{2.424111in}}{\pgfqpoint{3.509123in}{2.424111in}}%
\pgfpathcurveto{\pgfqpoint{3.501680in}{2.424111in}}{\pgfqpoint{3.494541in}{2.421154in}}{\pgfqpoint{3.489278in}{2.415891in}}%
\pgfpathcurveto{\pgfqpoint{3.484015in}{2.410628in}}{\pgfqpoint{3.481058in}{2.403489in}}{\pgfqpoint{3.481058in}{2.396046in}}%
\pgfpathcurveto{\pgfqpoint{3.481058in}{2.388603in}}{\pgfqpoint{3.484015in}{2.381464in}}{\pgfqpoint{3.489278in}{2.376201in}}%
\pgfpathcurveto{\pgfqpoint{3.494541in}{2.370938in}}{\pgfqpoint{3.501680in}{2.367980in}}{\pgfqpoint{3.509123in}{2.367980in}}%
\pgfpathclose%
\pgfusepath{stroke,fill}%
\end{pgfscope}%
\begin{pgfscope}%
\pgfpathrectangle{\pgfqpoint{0.800000in}{0.528000in}}{\pgfqpoint{3.968000in}{3.696000in}} %
\pgfusepath{clip}%
\pgfsetbuttcap%
\pgfsetroundjoin%
\definecolor{currentfill}{rgb}{1.000000,0.498039,0.054902}%
\pgfsetfillcolor{currentfill}%
\pgfsetlinewidth{1.003750pt}%
\definecolor{currentstroke}{rgb}{1.000000,0.498039,0.054902}%
\pgfsetstrokecolor{currentstroke}%
\pgfsetdash{}{0pt}%
\pgfpathmoveto{\pgfqpoint{1.704718in}{3.446833in}}%
\pgfpathcurveto{\pgfqpoint{1.715499in}{3.446833in}}{\pgfqpoint{1.725840in}{3.451117in}}{\pgfqpoint{1.733463in}{3.458740in}}%
\pgfpathcurveto{\pgfqpoint{1.741087in}{3.466363in}}{\pgfqpoint{1.745370in}{3.476704in}}{\pgfqpoint{1.745370in}{3.487485in}}%
\pgfpathcurveto{\pgfqpoint{1.745370in}{3.498266in}}{\pgfqpoint{1.741087in}{3.508607in}}{\pgfqpoint{1.733463in}{3.516231in}}%
\pgfpathcurveto{\pgfqpoint{1.725840in}{3.523854in}}{\pgfqpoint{1.715499in}{3.528137in}}{\pgfqpoint{1.704718in}{3.528137in}}%
\pgfpathcurveto{\pgfqpoint{1.693937in}{3.528137in}}{\pgfqpoint{1.683596in}{3.523854in}}{\pgfqpoint{1.675973in}{3.516231in}}%
\pgfpathcurveto{\pgfqpoint{1.668349in}{3.508607in}}{\pgfqpoint{1.664066in}{3.498266in}}{\pgfqpoint{1.664066in}{3.487485in}}%
\pgfpathcurveto{\pgfqpoint{1.664066in}{3.476704in}}{\pgfqpoint{1.668349in}{3.466363in}}{\pgfqpoint{1.675973in}{3.458740in}}%
\pgfpathcurveto{\pgfqpoint{1.683596in}{3.451117in}}{\pgfqpoint{1.693937in}{3.446833in}}{\pgfqpoint{1.704718in}{3.446833in}}%
\pgfpathclose%
\pgfusepath{stroke,fill}%
\end{pgfscope}%
\begin{pgfscope}%
\pgfpathrectangle{\pgfqpoint{0.800000in}{0.528000in}}{\pgfqpoint{3.968000in}{3.696000in}} %
\pgfusepath{clip}%
\pgfsetbuttcap%
\pgfsetroundjoin%
\definecolor{currentfill}{rgb}{1.000000,0.498039,0.054902}%
\pgfsetfillcolor{currentfill}%
\pgfsetlinewidth{1.003750pt}%
\definecolor{currentstroke}{rgb}{1.000000,0.498039,0.054902}%
\pgfsetstrokecolor{currentstroke}%
\pgfsetdash{}{0pt}%
\pgfpathmoveto{\pgfqpoint{2.676945in}{2.839931in}}%
\pgfpathcurveto{\pgfqpoint{2.690816in}{2.839931in}}{\pgfqpoint{2.704120in}{2.845442in}}{\pgfqpoint{2.713928in}{2.855250in}}%
\pgfpathcurveto{\pgfqpoint{2.723736in}{2.865058in}}{\pgfqpoint{2.729247in}{2.878363in}}{\pgfqpoint{2.729247in}{2.892233in}}%
\pgfpathcurveto{\pgfqpoint{2.729247in}{2.906104in}}{\pgfqpoint{2.723736in}{2.919408in}}{\pgfqpoint{2.713928in}{2.929216in}}%
\pgfpathcurveto{\pgfqpoint{2.704120in}{2.939024in}}{\pgfqpoint{2.690816in}{2.944535in}}{\pgfqpoint{2.676945in}{2.944535in}}%
\pgfpathcurveto{\pgfqpoint{2.663074in}{2.944535in}}{\pgfqpoint{2.649770in}{2.939024in}}{\pgfqpoint{2.639962in}{2.929216in}}%
\pgfpathcurveto{\pgfqpoint{2.630154in}{2.919408in}}{\pgfqpoint{2.624643in}{2.906104in}}{\pgfqpoint{2.624643in}{2.892233in}}%
\pgfpathcurveto{\pgfqpoint{2.624643in}{2.878363in}}{\pgfqpoint{2.630154in}{2.865058in}}{\pgfqpoint{2.639962in}{2.855250in}}%
\pgfpathcurveto{\pgfqpoint{2.649770in}{2.845442in}}{\pgfqpoint{2.663074in}{2.839931in}}{\pgfqpoint{2.676945in}{2.839931in}}%
\pgfpathclose%
\pgfusepath{stroke,fill}%
\end{pgfscope}%
\begin{pgfscope}%
\pgfpathrectangle{\pgfqpoint{0.800000in}{0.528000in}}{\pgfqpoint{3.968000in}{3.696000in}} %
\pgfusepath{clip}%
\pgfsetbuttcap%
\pgfsetroundjoin%
\definecolor{currentfill}{rgb}{1.000000,0.498039,0.054902}%
\pgfsetfillcolor{currentfill}%
\pgfsetlinewidth{1.003750pt}%
\definecolor{currentstroke}{rgb}{1.000000,0.498039,0.054902}%
\pgfsetstrokecolor{currentstroke}%
\pgfsetdash{}{0pt}%
\pgfpathmoveto{\pgfqpoint{3.075174in}{2.291874in}}%
\pgfpathcurveto{\pgfqpoint{3.085725in}{2.291874in}}{\pgfqpoint{3.095845in}{2.296066in}}{\pgfqpoint{3.103306in}{2.303526in}}%
\pgfpathcurveto{\pgfqpoint{3.110767in}{2.310987in}}{\pgfqpoint{3.114959in}{2.321107in}}{\pgfqpoint{3.114959in}{2.331658in}}%
\pgfpathcurveto{\pgfqpoint{3.114959in}{2.342209in}}{\pgfqpoint{3.110767in}{2.352330in}}{\pgfqpoint{3.103306in}{2.359790in}}%
\pgfpathcurveto{\pgfqpoint{3.095845in}{2.367251in}}{\pgfqpoint{3.085725in}{2.371443in}}{\pgfqpoint{3.075174in}{2.371443in}}%
\pgfpathcurveto{\pgfqpoint{3.064623in}{2.371443in}}{\pgfqpoint{3.054503in}{2.367251in}}{\pgfqpoint{3.047042in}{2.359790in}}%
\pgfpathcurveto{\pgfqpoint{3.039581in}{2.352330in}}{\pgfqpoint{3.035389in}{2.342209in}}{\pgfqpoint{3.035389in}{2.331658in}}%
\pgfpathcurveto{\pgfqpoint{3.035389in}{2.321107in}}{\pgfqpoint{3.039581in}{2.310987in}}{\pgfqpoint{3.047042in}{2.303526in}}%
\pgfpathcurveto{\pgfqpoint{3.054503in}{2.296066in}}{\pgfqpoint{3.064623in}{2.291874in}}{\pgfqpoint{3.075174in}{2.291874in}}%
\pgfpathclose%
\pgfusepath{stroke,fill}%
\end{pgfscope}%
\begin{pgfscope}%
\pgfpathrectangle{\pgfqpoint{0.800000in}{0.528000in}}{\pgfqpoint{3.968000in}{3.696000in}} %
\pgfusepath{clip}%
\pgfsetbuttcap%
\pgfsetroundjoin%
\definecolor{currentfill}{rgb}{1.000000,0.498039,0.054902}%
\pgfsetfillcolor{currentfill}%
\pgfsetlinewidth{1.003750pt}%
\definecolor{currentstroke}{rgb}{1.000000,0.498039,0.054902}%
\pgfsetstrokecolor{currentstroke}%
\pgfsetdash{}{0pt}%
\pgfpathmoveto{\pgfqpoint{4.586494in}{3.418968in}}%
\pgfpathcurveto{\pgfqpoint{4.588417in}{3.418968in}}{\pgfqpoint{4.590262in}{3.419733in}}{\pgfqpoint{4.591622in}{3.421093in}}%
\pgfpathcurveto{\pgfqpoint{4.592983in}{3.422453in}}{\pgfqpoint{4.593747in}{3.424298in}}{\pgfqpoint{4.593747in}{3.426222in}}%
\pgfpathcurveto{\pgfqpoint{4.593747in}{3.428145in}}{\pgfqpoint{4.592983in}{3.429990in}}{\pgfqpoint{4.591622in}{3.431350in}}%
\pgfpathcurveto{\pgfqpoint{4.590262in}{3.432711in}}{\pgfqpoint{4.588417in}{3.433475in}}{\pgfqpoint{4.586494in}{3.433475in}}%
\pgfpathcurveto{\pgfqpoint{4.584570in}{3.433475in}}{\pgfqpoint{4.582725in}{3.432711in}}{\pgfqpoint{4.581365in}{3.431350in}}%
\pgfpathcurveto{\pgfqpoint{4.580005in}{3.429990in}}{\pgfqpoint{4.579240in}{3.428145in}}{\pgfqpoint{4.579240in}{3.426222in}}%
\pgfpathcurveto{\pgfqpoint{4.579240in}{3.424298in}}{\pgfqpoint{4.580005in}{3.422453in}}{\pgfqpoint{4.581365in}{3.421093in}}%
\pgfpathcurveto{\pgfqpoint{4.582725in}{3.419733in}}{\pgfqpoint{4.584570in}{3.418968in}}{\pgfqpoint{4.586494in}{3.418968in}}%
\pgfpathclose%
\pgfusepath{stroke,fill}%
\end{pgfscope}%
\begin{pgfscope}%
\pgfpathrectangle{\pgfqpoint{0.800000in}{0.528000in}}{\pgfqpoint{3.968000in}{3.696000in}} %
\pgfusepath{clip}%
\pgfsetbuttcap%
\pgfsetroundjoin%
\definecolor{currentfill}{rgb}{1.000000,0.498039,0.054902}%
\pgfsetfillcolor{currentfill}%
\pgfsetlinewidth{1.003750pt}%
\definecolor{currentstroke}{rgb}{1.000000,0.498039,0.054902}%
\pgfsetstrokecolor{currentstroke}%
\pgfsetdash{}{0pt}%
\pgfpathmoveto{\pgfqpoint{3.040570in}{3.080689in}}%
\pgfpathcurveto{\pgfqpoint{3.053174in}{3.080689in}}{\pgfqpoint{3.065264in}{3.085696in}}{\pgfqpoint{3.074177in}{3.094609in}}%
\pgfpathcurveto{\pgfqpoint{3.083089in}{3.103521in}}{\pgfqpoint{3.088097in}{3.115611in}}{\pgfqpoint{3.088097in}{3.128215in}}%
\pgfpathcurveto{\pgfqpoint{3.088097in}{3.140820in}}{\pgfqpoint{3.083089in}{3.152909in}}{\pgfqpoint{3.074177in}{3.161822in}}%
\pgfpathcurveto{\pgfqpoint{3.065264in}{3.170735in}}{\pgfqpoint{3.053174in}{3.175742in}}{\pgfqpoint{3.040570in}{3.175742in}}%
\pgfpathcurveto{\pgfqpoint{3.027966in}{3.175742in}}{\pgfqpoint{3.015876in}{3.170735in}}{\pgfqpoint{3.006963in}{3.161822in}}%
\pgfpathcurveto{\pgfqpoint{2.998051in}{3.152909in}}{\pgfqpoint{2.993043in}{3.140820in}}{\pgfqpoint{2.993043in}{3.128215in}}%
\pgfpathcurveto{\pgfqpoint{2.993043in}{3.115611in}}{\pgfqpoint{2.998051in}{3.103521in}}{\pgfqpoint{3.006963in}{3.094609in}}%
\pgfpathcurveto{\pgfqpoint{3.015876in}{3.085696in}}{\pgfqpoint{3.027966in}{3.080689in}}{\pgfqpoint{3.040570in}{3.080689in}}%
\pgfpathclose%
\pgfusepath{stroke,fill}%
\end{pgfscope}%
\begin{pgfscope}%
\pgfpathrectangle{\pgfqpoint{0.800000in}{0.528000in}}{\pgfqpoint{3.968000in}{3.696000in}} %
\pgfusepath{clip}%
\pgfsetbuttcap%
\pgfsetroundjoin%
\definecolor{currentfill}{rgb}{1.000000,0.498039,0.054902}%
\pgfsetfillcolor{currentfill}%
\pgfsetlinewidth{1.003750pt}%
\definecolor{currentstroke}{rgb}{1.000000,0.498039,0.054902}%
\pgfsetstrokecolor{currentstroke}%
\pgfsetdash{}{0pt}%
\pgfpathmoveto{\pgfqpoint{1.732975in}{2.459198in}}%
\pgfpathcurveto{\pgfqpoint{1.739775in}{2.459198in}}{\pgfqpoint{1.746296in}{2.461900in}}{\pgfqpoint{1.751104in}{2.466707in}}%
\pgfpathcurveto{\pgfqpoint{1.755912in}{2.471515in}}{\pgfqpoint{1.758613in}{2.478037in}}{\pgfqpoint{1.758613in}{2.484836in}}%
\pgfpathcurveto{\pgfqpoint{1.758613in}{2.491635in}}{\pgfqpoint{1.755912in}{2.498157in}}{\pgfqpoint{1.751104in}{2.502964in}}%
\pgfpathcurveto{\pgfqpoint{1.746296in}{2.507772in}}{\pgfqpoint{1.739775in}{2.510473in}}{\pgfqpoint{1.732975in}{2.510473in}}%
\pgfpathcurveto{\pgfqpoint{1.726176in}{2.510473in}}{\pgfqpoint{1.719655in}{2.507772in}}{\pgfqpoint{1.714847in}{2.502964in}}%
\pgfpathcurveto{\pgfqpoint{1.710039in}{2.498157in}}{\pgfqpoint{1.707338in}{2.491635in}}{\pgfqpoint{1.707338in}{2.484836in}}%
\pgfpathcurveto{\pgfqpoint{1.707338in}{2.478037in}}{\pgfqpoint{1.710039in}{2.471515in}}{\pgfqpoint{1.714847in}{2.466707in}}%
\pgfpathcurveto{\pgfqpoint{1.719655in}{2.461900in}}{\pgfqpoint{1.726176in}{2.459198in}}{\pgfqpoint{1.732975in}{2.459198in}}%
\pgfpathclose%
\pgfusepath{stroke,fill}%
\end{pgfscope}%
\begin{pgfscope}%
\pgfpathrectangle{\pgfqpoint{0.800000in}{0.528000in}}{\pgfqpoint{3.968000in}{3.696000in}} %
\pgfusepath{clip}%
\pgfsetbuttcap%
\pgfsetroundjoin%
\definecolor{currentfill}{rgb}{1.000000,0.498039,0.054902}%
\pgfsetfillcolor{currentfill}%
\pgfsetlinewidth{1.003750pt}%
\definecolor{currentstroke}{rgb}{1.000000,0.498039,0.054902}%
\pgfsetstrokecolor{currentstroke}%
\pgfsetdash{}{0pt}%
\pgfpathmoveto{\pgfqpoint{2.341203in}{2.836380in}}%
\pgfpathcurveto{\pgfqpoint{2.353858in}{2.836380in}}{\pgfqpoint{2.365997in}{2.841408in}}{\pgfqpoint{2.374945in}{2.850357in}}%
\pgfpathcurveto{\pgfqpoint{2.383894in}{2.859305in}}{\pgfqpoint{2.388922in}{2.871444in}}{\pgfqpoint{2.388922in}{2.884099in}}%
\pgfpathcurveto{\pgfqpoint{2.388922in}{2.896754in}}{\pgfqpoint{2.383894in}{2.908892in}}{\pgfqpoint{2.374945in}{2.917841in}}%
\pgfpathcurveto{\pgfqpoint{2.365997in}{2.926789in}}{\pgfqpoint{2.353858in}{2.931817in}}{\pgfqpoint{2.341203in}{2.931817in}}%
\pgfpathcurveto{\pgfqpoint{2.328548in}{2.931817in}}{\pgfqpoint{2.316409in}{2.926789in}}{\pgfqpoint{2.307461in}{2.917841in}}%
\pgfpathcurveto{\pgfqpoint{2.298512in}{2.908892in}}{\pgfqpoint{2.293484in}{2.896754in}}{\pgfqpoint{2.293484in}{2.884099in}}%
\pgfpathcurveto{\pgfqpoint{2.293484in}{2.871444in}}{\pgfqpoint{2.298512in}{2.859305in}}{\pgfqpoint{2.307461in}{2.850357in}}%
\pgfpathcurveto{\pgfqpoint{2.316409in}{2.841408in}}{\pgfqpoint{2.328548in}{2.836380in}}{\pgfqpoint{2.341203in}{2.836380in}}%
\pgfpathclose%
\pgfusepath{stroke,fill}%
\end{pgfscope}%
\begin{pgfscope}%
\pgfpathrectangle{\pgfqpoint{0.800000in}{0.528000in}}{\pgfqpoint{3.968000in}{3.696000in}} %
\pgfusepath{clip}%
\pgfsetbuttcap%
\pgfsetroundjoin%
\definecolor{currentfill}{rgb}{0.274149,0.751988,0.436601}%
\pgfsetfillcolor{currentfill}%
\pgfsetlinewidth{0.000000pt}%
\definecolor{currentstroke}{rgb}{0.000000,0.000000,0.000000}%
\pgfsetstrokecolor{currentstroke}%
\pgfsetdash{}{0pt}%
\pgfpathmoveto{\pgfqpoint{2.078441in}{1.038317in}}%
\pgfpathlineto{\pgfqpoint{1.946250in}{1.127438in}}%
\pgfpathlineto{\pgfqpoint{1.941952in}{1.094444in}}%
\pgfpathlineto{\pgfqpoint{1.843526in}{1.214638in}}%
\pgfpathlineto{\pgfqpoint{1.991860in}{1.168472in}}%
\pgfpathlineto{\pgfqpoint{1.962886in}{1.152114in}}%
\pgfpathlineto{\pgfqpoint{2.095077in}{1.062993in}}%
\pgfpathlineto{\pgfqpoint{2.078441in}{1.038317in}}%
\pgfusepath{fill}%
\end{pgfscope}%
\begin{pgfscope}%
\pgfpathrectangle{\pgfqpoint{0.800000in}{0.528000in}}{\pgfqpoint{3.968000in}{3.696000in}} %
\pgfusepath{clip}%
\pgfsetbuttcap%
\pgfsetroundjoin%
\definecolor{currentfill}{rgb}{0.281924,0.089666,0.412415}%
\pgfsetfillcolor{currentfill}%
\pgfsetlinewidth{0.000000pt}%
\definecolor{currentstroke}{rgb}{0.000000,0.000000,0.000000}%
\pgfsetstrokecolor{currentstroke}%
\pgfsetdash{}{0pt}%
\pgfpathmoveto{\pgfqpoint{2.086370in}{1.035781in}}%
\pgfpathlineto{\pgfqpoint{1.941162in}{1.039580in}}%
\pgfpathlineto{\pgfqpoint{1.955259in}{1.009441in}}%
\pgfpathlineto{\pgfqpoint{1.807677in}{1.057958in}}%
\pgfpathlineto{\pgfqpoint{1.957594in}{1.098690in}}%
\pgfpathlineto{\pgfqpoint{1.941941in}{1.069330in}}%
\pgfpathlineto{\pgfqpoint{2.087148in}{1.065530in}}%
\pgfpathlineto{\pgfqpoint{2.086370in}{1.035781in}}%
\pgfusepath{fill}%
\end{pgfscope}%
\begin{pgfscope}%
\pgfpathrectangle{\pgfqpoint{0.800000in}{0.528000in}}{\pgfqpoint{3.968000in}{3.696000in}} %
\pgfusepath{clip}%
\pgfsetbuttcap%
\pgfsetroundjoin%
\definecolor{currentfill}{rgb}{0.274128,0.199721,0.498911}%
\pgfsetfillcolor{currentfill}%
\pgfsetlinewidth{0.000000pt}%
\definecolor{currentstroke}{rgb}{0.000000,0.000000,0.000000}%
\pgfsetstrokecolor{currentstroke}%
\pgfsetdash{}{0pt}%
\pgfpathmoveto{\pgfqpoint{2.324205in}{1.036558in}}%
\pgfpathlineto{\pgfqpoint{1.965641in}{1.157681in}}%
\pgfpathlineto{\pgfqpoint{1.970214in}{1.124724in}}%
\pgfpathlineto{\pgfqpoint{1.843526in}{1.214638in}}%
\pgfpathlineto{\pgfqpoint{1.998787in}{1.209309in}}%
\pgfpathlineto{\pgfqpoint{1.975165in}{1.185876in}}%
\pgfpathlineto{\pgfqpoint{2.333729in}{1.064753in}}%
\pgfpathlineto{\pgfqpoint{2.324205in}{1.036558in}}%
\pgfusepath{fill}%
\end{pgfscope}%
\begin{pgfscope}%
\pgfpathrectangle{\pgfqpoint{0.800000in}{0.528000in}}{\pgfqpoint{3.968000in}{3.696000in}} %
\pgfusepath{clip}%
\pgfsetbuttcap%
\pgfsetroundjoin%
\definecolor{currentfill}{rgb}{0.239346,0.300855,0.540844}%
\pgfsetfillcolor{currentfill}%
\pgfsetlinewidth{0.000000pt}%
\definecolor{currentstroke}{rgb}{0.000000,0.000000,0.000000}%
\pgfsetstrokecolor{currentstroke}%
\pgfsetdash{}{0pt}%
\pgfpathmoveto{\pgfqpoint{2.582118in}{1.060739in}}%
\pgfpathlineto{\pgfqpoint{2.810901in}{0.812448in}}%
\pgfpathlineto{\pgfqpoint{2.822703in}{0.843557in}}%
\pgfpathlineto{\pgfqpoint{2.890706in}{0.703879in}}%
\pgfpathlineto{\pgfqpoint{2.757046in}{0.783059in}}%
\pgfpathlineto{\pgfqpoint{2.789015in}{0.792282in}}%
\pgfpathlineto{\pgfqpoint{2.560233in}{1.040572in}}%
\pgfpathlineto{\pgfqpoint{2.582118in}{1.060739in}}%
\pgfusepath{fill}%
\end{pgfscope}%
\begin{pgfscope}%
\pgfpathrectangle{\pgfqpoint{0.800000in}{0.528000in}}{\pgfqpoint{3.968000in}{3.696000in}} %
\pgfusepath{clip}%
\pgfsetbuttcap%
\pgfsetroundjoin%
\definecolor{currentfill}{rgb}{0.135066,0.544853,0.554029}%
\pgfsetfillcolor{currentfill}%
\pgfsetlinewidth{0.000000pt}%
\definecolor{currentstroke}{rgb}{0.000000,0.000000,0.000000}%
\pgfsetstrokecolor{currentstroke}%
\pgfsetdash{}{0pt}%
\pgfpathmoveto{\pgfqpoint{2.813387in}{1.065535in}}%
\pgfpathlineto{\pgfqpoint{3.528775in}{1.065363in}}%
\pgfpathlineto{\pgfqpoint{3.513902in}{1.095126in}}%
\pgfpathlineto{\pgfqpoint{3.662692in}{1.050450in}}%
\pgfpathlineto{\pgfqpoint{3.513881in}{1.005846in}}%
\pgfpathlineto{\pgfqpoint{3.528768in}{1.035603in}}%
\pgfpathlineto{\pgfqpoint{2.813380in}{1.035775in}}%
\pgfpathlineto{\pgfqpoint{2.813387in}{1.065535in}}%
\pgfusepath{fill}%
\end{pgfscope}%
\begin{pgfscope}%
\pgfpathrectangle{\pgfqpoint{0.800000in}{0.528000in}}{\pgfqpoint{3.968000in}{3.696000in}} %
\pgfusepath{clip}%
\pgfsetbuttcap%
\pgfsetroundjoin%
\definecolor{currentfill}{rgb}{0.129933,0.559582,0.551864}%
\pgfsetfillcolor{currentfill}%
\pgfsetlinewidth{0.000000pt}%
\definecolor{currentstroke}{rgb}{0.000000,0.000000,0.000000}%
\pgfsetstrokecolor{currentstroke}%
\pgfsetdash{}{0pt}%
\pgfpathmoveto{\pgfqpoint{2.827907in}{1.053894in}}%
\pgfpathlineto{\pgfqpoint{2.876084in}{0.837828in}}%
\pgfpathlineto{\pgfqpoint{2.901892in}{0.858828in}}%
\pgfpathlineto{\pgfqpoint{2.890706in}{0.703879in}}%
\pgfpathlineto{\pgfqpoint{2.814752in}{0.839398in}}%
\pgfpathlineto{\pgfqpoint{2.847037in}{0.831351in}}%
\pgfpathlineto{\pgfqpoint{2.798860in}{1.047417in}}%
\pgfpathlineto{\pgfqpoint{2.827907in}{1.053894in}}%
\pgfusepath{fill}%
\end{pgfscope}%
\begin{pgfscope}%
\pgfpathrectangle{\pgfqpoint{0.800000in}{0.528000in}}{\pgfqpoint{3.968000in}{3.696000in}} %
\pgfusepath{clip}%
\pgfsetbuttcap%
\pgfsetroundjoin%
\definecolor{currentfill}{rgb}{0.141935,0.526453,0.555991}%
\pgfsetfillcolor{currentfill}%
\pgfsetlinewidth{0.000000pt}%
\definecolor{currentstroke}{rgb}{0.000000,0.000000,0.000000}%
\pgfsetstrokecolor{currentstroke}%
\pgfsetdash{}{0pt}%
\pgfpathmoveto{\pgfqpoint{3.055597in}{1.065535in}}%
\pgfpathlineto{\pgfqpoint{3.528777in}{1.065376in}}%
\pgfpathlineto{\pgfqpoint{3.513907in}{1.095141in}}%
\pgfpathlineto{\pgfqpoint{3.662692in}{1.050450in}}%
\pgfpathlineto{\pgfqpoint{3.513876in}{1.005861in}}%
\pgfpathlineto{\pgfqpoint{3.528766in}{1.035616in}}%
\pgfpathlineto{\pgfqpoint{3.055587in}{1.035775in}}%
\pgfpathlineto{\pgfqpoint{3.055597in}{1.065535in}}%
\pgfusepath{fill}%
\end{pgfscope}%
\begin{pgfscope}%
\pgfpathrectangle{\pgfqpoint{0.800000in}{0.528000in}}{\pgfqpoint{3.968000in}{3.696000in}} %
\pgfusepath{clip}%
\pgfsetbuttcap%
\pgfsetroundjoin%
\definecolor{currentfill}{rgb}{0.120092,0.600104,0.542530}%
\pgfsetfillcolor{currentfill}%
\pgfsetlinewidth{0.000000pt}%
\definecolor{currentstroke}{rgb}{0.000000,0.000000,0.000000}%
\pgfsetstrokecolor{currentstroke}%
\pgfsetdash{}{0pt}%
\pgfpathmoveto{\pgfqpoint{2.072057in}{1.223370in}}%
\pgfpathlineto{\pgfqpoint{1.932802in}{2.115252in}}%
\pgfpathlineto{\pgfqpoint{1.905693in}{2.095959in}}%
\pgfpathlineto{\pgfqpoint{1.926844in}{2.249864in}}%
\pgfpathlineto{\pgfqpoint{1.993905in}{2.109732in}}%
\pgfpathlineto{\pgfqpoint{1.962205in}{2.119843in}}%
\pgfpathlineto{\pgfqpoint{2.101461in}{1.227961in}}%
\pgfpathlineto{\pgfqpoint{2.072057in}{1.223370in}}%
\pgfusepath{fill}%
\end{pgfscope}%
\begin{pgfscope}%
\pgfpathrectangle{\pgfqpoint{0.800000in}{0.528000in}}{\pgfqpoint{3.968000in}{3.696000in}} %
\pgfusepath{clip}%
\pgfsetbuttcap%
\pgfsetroundjoin%
\definecolor{currentfill}{rgb}{0.120092,0.600104,0.542530}%
\pgfsetfillcolor{currentfill}%
\pgfsetlinewidth{0.000000pt}%
\definecolor{currentstroke}{rgb}{0.000000,0.000000,0.000000}%
\pgfsetstrokecolor{currentstroke}%
\pgfsetdash{}{0pt}%
\pgfpathmoveto{\pgfqpoint{2.080063in}{1.212377in}}%
\pgfpathlineto{\pgfqpoint{1.099482in}{1.706533in}}%
\pgfpathlineto{\pgfqpoint{1.099377in}{1.673261in}}%
\pgfpathlineto{\pgfqpoint{0.986586in}{1.780089in}}%
\pgfpathlineto{\pgfqpoint{1.139556in}{1.752989in}}%
\pgfpathlineto{\pgfqpoint{1.112875in}{1.733110in}}%
\pgfpathlineto{\pgfqpoint{2.093456in}{1.238953in}}%
\pgfpathlineto{\pgfqpoint{2.080063in}{1.212377in}}%
\pgfusepath{fill}%
\end{pgfscope}%
\begin{pgfscope}%
\pgfpathrectangle{\pgfqpoint{0.800000in}{0.528000in}}{\pgfqpoint{3.968000in}{3.696000in}} %
\pgfusepath{clip}%
\pgfsetbuttcap%
\pgfsetroundjoin%
\definecolor{currentfill}{rgb}{0.272594,0.025563,0.353093}%
\pgfsetfillcolor{currentfill}%
\pgfsetlinewidth{0.000000pt}%
\definecolor{currentstroke}{rgb}{0.000000,0.000000,0.000000}%
\pgfsetstrokecolor{currentstroke}%
\pgfsetdash{}{0pt}%
\pgfpathmoveto{\pgfqpoint{2.316119in}{1.233172in}}%
\pgfpathlineto{\pgfqpoint{2.912683in}{2.254247in}}%
\pgfpathlineto{\pgfqpoint{2.879481in}{2.256412in}}%
\pgfpathlineto{\pgfqpoint{2.993089in}{2.362372in}}%
\pgfpathlineto{\pgfqpoint{2.956569in}{2.211373in}}%
\pgfpathlineto{\pgfqpoint{2.938379in}{2.239234in}}%
\pgfpathlineto{\pgfqpoint{2.341815in}{1.218159in}}%
\pgfpathlineto{\pgfqpoint{2.316119in}{1.233172in}}%
\pgfusepath{fill}%
\end{pgfscope}%
\begin{pgfscope}%
\pgfpathrectangle{\pgfqpoint{0.800000in}{0.528000in}}{\pgfqpoint{3.968000in}{3.696000in}} %
\pgfusepath{clip}%
\pgfsetbuttcap%
\pgfsetroundjoin%
\definecolor{currentfill}{rgb}{0.279566,0.067836,0.391917}%
\pgfsetfillcolor{currentfill}%
\pgfsetlinewidth{0.000000pt}%
\definecolor{currentstroke}{rgb}{0.000000,0.000000,0.000000}%
\pgfsetstrokecolor{currentstroke}%
\pgfsetdash{}{0pt}%
\pgfpathmoveto{\pgfqpoint{2.315117in}{1.220227in}}%
\pgfpathlineto{\pgfqpoint{1.961936in}{2.119770in}}%
\pgfpathlineto{\pgfqpoint{1.939673in}{2.095043in}}%
\pgfpathlineto{\pgfqpoint{1.926844in}{2.249864in}}%
\pgfpathlineto{\pgfqpoint{2.022777in}{2.127672in}}%
\pgfpathlineto{\pgfqpoint{1.989638in}{2.130646in}}%
\pgfpathlineto{\pgfqpoint{2.342818in}{1.231103in}}%
\pgfpathlineto{\pgfqpoint{2.315117in}{1.220227in}}%
\pgfusepath{fill}%
\end{pgfscope}%
\begin{pgfscope}%
\pgfpathrectangle{\pgfqpoint{0.800000in}{0.528000in}}{\pgfqpoint{3.968000in}{3.696000in}} %
\pgfusepath{clip}%
\pgfsetbuttcap%
\pgfsetroundjoin%
\definecolor{currentfill}{rgb}{0.267004,0.004874,0.329415}%
\pgfsetfillcolor{currentfill}%
\pgfsetlinewidth{0.000000pt}%
\definecolor{currentstroke}{rgb}{0.000000,0.000000,0.000000}%
\pgfsetstrokecolor{currentstroke}%
\pgfsetdash{}{0pt}%
\pgfpathmoveto{\pgfqpoint{2.316358in}{1.233567in}}%
\pgfpathlineto{\pgfqpoint{2.562176in}{1.625841in}}%
\pgfpathlineto{\pgfqpoint{2.529057in}{1.629035in}}%
\pgfpathlineto{\pgfqpoint{2.645896in}{1.731420in}}%
\pgfpathlineto{\pgfqpoint{2.604710in}{1.581627in}}%
\pgfpathlineto{\pgfqpoint{2.587393in}{1.610038in}}%
\pgfpathlineto{\pgfqpoint{2.341576in}{1.217764in}}%
\pgfpathlineto{\pgfqpoint{2.316358in}{1.233567in}}%
\pgfusepath{fill}%
\end{pgfscope}%
\begin{pgfscope}%
\pgfpathrectangle{\pgfqpoint{0.800000in}{0.528000in}}{\pgfqpoint{3.968000in}{3.696000in}} %
\pgfusepath{clip}%
\pgfsetbuttcap%
\pgfsetroundjoin%
\definecolor{currentfill}{rgb}{0.227802,0.326594,0.546532}%
\pgfsetfillcolor{currentfill}%
\pgfsetlinewidth{0.000000pt}%
\definecolor{currentstroke}{rgb}{0.000000,0.000000,0.000000}%
\pgfsetstrokecolor{currentstroke}%
\pgfsetdash{}{0pt}%
\pgfpathmoveto{\pgfqpoint{2.316632in}{1.233988in}}%
\pgfpathlineto{\pgfqpoint{2.987938in}{2.228965in}}%
\pgfpathlineto{\pgfqpoint{2.954945in}{2.233275in}}%
\pgfpathlineto{\pgfqpoint{3.075174in}{2.331658in}}%
\pgfpathlineto{\pgfqpoint{3.028955in}{2.183341in}}%
\pgfpathlineto{\pgfqpoint{3.012608in}{2.212321in}}%
\pgfpathlineto{\pgfqpoint{2.341302in}{1.217343in}}%
\pgfpathlineto{\pgfqpoint{2.316632in}{1.233988in}}%
\pgfusepath{fill}%
\end{pgfscope}%
\begin{pgfscope}%
\pgfpathrectangle{\pgfqpoint{0.800000in}{0.528000in}}{\pgfqpoint{3.968000in}{3.696000in}} %
\pgfusepath{clip}%
\pgfsetbuttcap%
\pgfsetroundjoin%
\definecolor{currentfill}{rgb}{0.282910,0.105393,0.426902}%
\pgfsetfillcolor{currentfill}%
\pgfsetlinewidth{0.000000pt}%
\definecolor{currentstroke}{rgb}{0.000000,0.000000,0.000000}%
\pgfsetstrokecolor{currentstroke}%
\pgfsetdash{}{0pt}%
\pgfpathmoveto{\pgfqpoint{2.561004in}{1.236527in}}%
\pgfpathlineto{\pgfqpoint{3.781176in}{2.379156in}}%
\pgfpathlineto{\pgfqpoint{3.749973in}{2.390708in}}%
\pgfpathlineto{\pgfqpoint{3.889098in}{2.459834in}}%
\pgfpathlineto{\pgfqpoint{3.810999in}{2.325540in}}%
\pgfpathlineto{\pgfqpoint{3.801518in}{2.357434in}}%
\pgfpathlineto{\pgfqpoint{2.581346in}{1.214804in}}%
\pgfpathlineto{\pgfqpoint{2.561004in}{1.236527in}}%
\pgfusepath{fill}%
\end{pgfscope}%
\begin{pgfscope}%
\pgfpathrectangle{\pgfqpoint{0.800000in}{0.528000in}}{\pgfqpoint{3.968000in}{3.696000in}} %
\pgfusepath{clip}%
\pgfsetbuttcap%
\pgfsetroundjoin%
\definecolor{currentfill}{rgb}{0.257322,0.256130,0.526563}%
\pgfsetfillcolor{currentfill}%
\pgfsetlinewidth{0.000000pt}%
\definecolor{currentstroke}{rgb}{0.000000,0.000000,0.000000}%
\pgfsetstrokecolor{currentstroke}%
\pgfsetdash{}{0pt}%
\pgfpathmoveto{\pgfqpoint{2.559564in}{1.234971in}}%
\pgfpathlineto{\pgfqpoint{3.413763in}{2.300849in}}%
\pgfpathlineto{\pgfqpoint{3.381235in}{2.307848in}}%
\pgfpathlineto{\pgfqpoint{3.509123in}{2.396046in}}%
\pgfpathlineto{\pgfqpoint{3.450903in}{2.252016in}}%
\pgfpathlineto{\pgfqpoint{3.436985in}{2.282238in}}%
\pgfpathlineto{\pgfqpoint{2.582787in}{1.216360in}}%
\pgfpathlineto{\pgfqpoint{2.559564in}{1.234971in}}%
\pgfusepath{fill}%
\end{pgfscope}%
\begin{pgfscope}%
\pgfpathrectangle{\pgfqpoint{0.800000in}{0.528000in}}{\pgfqpoint{3.968000in}{3.696000in}} %
\pgfusepath{clip}%
\pgfsetbuttcap%
\pgfsetroundjoin%
\definecolor{currentfill}{rgb}{0.277134,0.185228,0.489898}%
\pgfsetfillcolor{currentfill}%
\pgfsetlinewidth{0.000000pt}%
\definecolor{currentstroke}{rgb}{0.000000,0.000000,0.000000}%
\pgfsetstrokecolor{currentstroke}%
\pgfsetdash{}{0pt}%
\pgfpathmoveto{\pgfqpoint{2.557635in}{1.231836in}}%
\pgfpathlineto{\pgfqpoint{3.006101in}{2.215965in}}%
\pgfpathlineto{\pgfqpoint{2.972850in}{2.214766in}}%
\pgfpathlineto{\pgfqpoint{3.075174in}{2.331658in}}%
\pgfpathlineto{\pgfqpoint{3.054092in}{2.177744in}}%
\pgfpathlineto{\pgfqpoint{3.033181in}{2.203625in}}%
\pgfpathlineto{\pgfqpoint{2.584716in}{1.219495in}}%
\pgfpathlineto{\pgfqpoint{2.557635in}{1.231836in}}%
\pgfusepath{fill}%
\end{pgfscope}%
\begin{pgfscope}%
\pgfpathrectangle{\pgfqpoint{0.800000in}{0.528000in}}{\pgfqpoint{3.968000in}{3.696000in}} %
\pgfusepath{clip}%
\pgfsetbuttcap%
\pgfsetroundjoin%
\definecolor{currentfill}{rgb}{0.269944,0.014625,0.341379}%
\pgfsetfillcolor{currentfill}%
\pgfsetlinewidth{0.000000pt}%
\definecolor{currentstroke}{rgb}{0.000000,0.000000,0.000000}%
\pgfsetstrokecolor{currentstroke}%
\pgfsetdash{}{0pt}%
\pgfpathmoveto{\pgfqpoint{2.802166in}{1.235442in}}%
\pgfpathlineto{\pgfqpoint{3.789888in}{2.368656in}}%
\pgfpathlineto{\pgfqpoint{3.757677in}{2.376993in}}%
\pgfpathlineto{\pgfqpoint{3.889098in}{2.459834in}}%
\pgfpathlineto{\pgfqpoint{3.824980in}{2.318331in}}%
\pgfpathlineto{\pgfqpoint{3.812322in}{2.349102in}}%
\pgfpathlineto{\pgfqpoint{2.824601in}{1.215888in}}%
\pgfpathlineto{\pgfqpoint{2.802166in}{1.235442in}}%
\pgfusepath{fill}%
\end{pgfscope}%
\begin{pgfscope}%
\pgfpathrectangle{\pgfqpoint{0.800000in}{0.528000in}}{\pgfqpoint{3.968000in}{3.696000in}} %
\pgfusepath{clip}%
\pgfsetbuttcap%
\pgfsetroundjoin%
\definecolor{currentfill}{rgb}{0.165117,0.467423,0.558141}%
\pgfsetfillcolor{currentfill}%
\pgfsetlinewidth{0.000000pt}%
\definecolor{currentstroke}{rgb}{0.000000,0.000000,0.000000}%
\pgfsetstrokecolor{currentstroke}%
\pgfsetdash{}{0pt}%
\pgfpathmoveto{\pgfqpoint{3.043260in}{1.233993in}}%
\pgfpathlineto{\pgfqpoint{3.801815in}{2.357181in}}%
\pgfpathlineto{\pgfqpoint{3.768824in}{2.361506in}}%
\pgfpathlineto{\pgfqpoint{3.889098in}{2.459834in}}%
\pgfpathlineto{\pgfqpoint{3.842812in}{2.311538in}}%
\pgfpathlineto{\pgfqpoint{3.826477in}{2.340525in}}%
\pgfpathlineto{\pgfqpoint{3.067923in}{1.217337in}}%
\pgfpathlineto{\pgfqpoint{3.043260in}{1.233993in}}%
\pgfusepath{fill}%
\end{pgfscope}%
\begin{pgfscope}%
\pgfpathrectangle{\pgfqpoint{0.800000in}{0.528000in}}{\pgfqpoint{3.968000in}{3.696000in}} %
\pgfusepath{clip}%
\pgfsetbuttcap%
\pgfsetroundjoin%
\definecolor{currentfill}{rgb}{0.191090,0.708366,0.482284}%
\pgfsetfillcolor{currentfill}%
\pgfsetlinewidth{0.000000pt}%
\definecolor{currentstroke}{rgb}{0.000000,0.000000,0.000000}%
\pgfsetstrokecolor{currentstroke}%
\pgfsetdash{}{0pt}%
\pgfpathmoveto{\pgfqpoint{3.043349in}{1.234123in}}%
\pgfpathlineto{\pgfqpoint{3.827009in}{2.368538in}}%
\pgfpathlineto{\pgfqpoint{3.794066in}{2.373210in}}%
\pgfpathlineto{\pgfqpoint{3.915369in}{2.470266in}}%
\pgfpathlineto{\pgfqpoint{3.867523in}{2.322466in}}%
\pgfpathlineto{\pgfqpoint{3.851495in}{2.351623in}}%
\pgfpathlineto{\pgfqpoint{3.067834in}{1.217208in}}%
\pgfpathlineto{\pgfqpoint{3.043349in}{1.234123in}}%
\pgfusepath{fill}%
\end{pgfscope}%
\begin{pgfscope}%
\pgfpathrectangle{\pgfqpoint{0.800000in}{0.528000in}}{\pgfqpoint{3.968000in}{3.696000in}} %
\pgfusepath{clip}%
\pgfsetbuttcap%
\pgfsetroundjoin%
\definecolor{currentfill}{rgb}{0.277134,0.185228,0.489898}%
\pgfsetfillcolor{currentfill}%
\pgfsetlinewidth{0.000000pt}%
\definecolor{currentstroke}{rgb}{0.000000,0.000000,0.000000}%
\pgfsetstrokecolor{currentstroke}%
\pgfsetdash{}{0pt}%
\pgfpathmoveto{\pgfqpoint{2.072613in}{1.396059in}}%
\pgfpathlineto{\pgfqpoint{1.760374in}{2.352907in}}%
\pgfpathlineto{\pgfqpoint{1.736699in}{2.329529in}}%
\pgfpathlineto{\pgfqpoint{1.732975in}{2.484836in}}%
\pgfpathlineto{\pgfqpoint{1.821574in}{2.357225in}}%
\pgfpathlineto{\pgfqpoint{1.788666in}{2.362139in}}%
\pgfpathlineto{\pgfqpoint{2.100905in}{1.405291in}}%
\pgfpathlineto{\pgfqpoint{2.072613in}{1.396059in}}%
\pgfusepath{fill}%
\end{pgfscope}%
\begin{pgfscope}%
\pgfpathrectangle{\pgfqpoint{0.800000in}{0.528000in}}{\pgfqpoint{3.968000in}{3.696000in}} %
\pgfusepath{clip}%
\pgfsetbuttcap%
\pgfsetroundjoin%
\definecolor{currentfill}{rgb}{0.255645,0.260703,0.528312}%
\pgfsetfillcolor{currentfill}%
\pgfsetlinewidth{0.000000pt}%
\definecolor{currentstroke}{rgb}{0.000000,0.000000,0.000000}%
\pgfsetstrokecolor{currentstroke}%
\pgfsetdash{}{0pt}%
\pgfpathmoveto{\pgfqpoint{2.072093in}{1.403191in}}%
\pgfpathlineto{\pgfqpoint{2.303897in}{2.754622in}}%
\pgfpathlineto{\pgfqpoint{2.272050in}{2.744987in}}%
\pgfpathlineto{\pgfqpoint{2.341203in}{2.884099in}}%
\pgfpathlineto{\pgfqpoint{2.360045in}{2.729894in}}%
\pgfpathlineto{\pgfqpoint{2.333229in}{2.749591in}}%
\pgfpathlineto{\pgfqpoint{2.101425in}{1.398160in}}%
\pgfpathlineto{\pgfqpoint{2.072093in}{1.403191in}}%
\pgfusepath{fill}%
\end{pgfscope}%
\begin{pgfscope}%
\pgfpathrectangle{\pgfqpoint{0.800000in}{0.528000in}}{\pgfqpoint{3.968000in}{3.696000in}} %
\pgfusepath{clip}%
\pgfsetbuttcap%
\pgfsetroundjoin%
\definecolor{currentfill}{rgb}{0.220124,0.725509,0.466226}%
\pgfsetfillcolor{currentfill}%
\pgfsetlinewidth{0.000000pt}%
\definecolor{currentstroke}{rgb}{0.000000,0.000000,0.000000}%
\pgfsetstrokecolor{currentstroke}%
\pgfsetdash{}{0pt}%
\pgfpathmoveto{\pgfqpoint{2.314476in}{1.404056in}}%
\pgfpathlineto{\pgfqpoint{2.632028in}{2.765196in}}%
\pgfpathlineto{\pgfqpoint{2.599665in}{2.757467in}}%
\pgfpathlineto{\pgfqpoint{2.676945in}{2.892233in}}%
\pgfpathlineto{\pgfqpoint{2.686611in}{2.737183in}}%
\pgfpathlineto{\pgfqpoint{2.661010in}{2.758435in}}%
\pgfpathlineto{\pgfqpoint{2.343458in}{1.397295in}}%
\pgfpathlineto{\pgfqpoint{2.314476in}{1.404056in}}%
\pgfusepath{fill}%
\end{pgfscope}%
\begin{pgfscope}%
\pgfpathrectangle{\pgfqpoint{0.800000in}{0.528000in}}{\pgfqpoint{3.968000in}{3.696000in}} %
\pgfusepath{clip}%
\pgfsetbuttcap%
\pgfsetroundjoin%
\definecolor{currentfill}{rgb}{0.267968,0.223549,0.512008}%
\pgfsetfillcolor{currentfill}%
\pgfsetlinewidth{0.000000pt}%
\definecolor{currentstroke}{rgb}{0.000000,0.000000,0.000000}%
\pgfsetstrokecolor{currentstroke}%
\pgfsetdash{}{0pt}%
\pgfpathmoveto{\pgfqpoint{2.559820in}{1.410292in}}%
\pgfpathlineto{\pgfqpoint{3.649494in}{2.696970in}}%
\pgfpathlineto{\pgfqpoint{3.617167in}{2.704848in}}%
\pgfpathlineto{\pgfqpoint{3.747397in}{2.789549in}}%
\pgfpathlineto{\pgfqpoint{3.685297in}{2.647149in}}%
\pgfpathlineto{\pgfqpoint{3.672204in}{2.677737in}}%
\pgfpathlineto{\pgfqpoint{2.582530in}{1.391059in}}%
\pgfpathlineto{\pgfqpoint{2.559820in}{1.410292in}}%
\pgfusepath{fill}%
\end{pgfscope}%
\begin{pgfscope}%
\pgfpathrectangle{\pgfqpoint{0.800000in}{0.528000in}}{\pgfqpoint{3.968000in}{3.696000in}} %
\pgfusepath{clip}%
\pgfsetbuttcap%
\pgfsetroundjoin%
\definecolor{currentfill}{rgb}{0.281887,0.150881,0.465405}%
\pgfsetfillcolor{currentfill}%
\pgfsetlinewidth{0.000000pt}%
\definecolor{currentstroke}{rgb}{0.000000,0.000000,0.000000}%
\pgfsetstrokecolor{currentstroke}%
\pgfsetdash{}{0pt}%
\pgfpathmoveto{\pgfqpoint{2.556430in}{1.402676in}}%
\pgfpathlineto{\pgfqpoint{2.746766in}{2.805552in}}%
\pgfpathlineto{\pgfqpoint{2.715276in}{2.794808in}}%
\pgfpathlineto{\pgfqpoint{2.779516in}{2.936256in}}%
\pgfpathlineto{\pgfqpoint{2.803745in}{2.782805in}}%
\pgfpathlineto{\pgfqpoint{2.776256in}{2.801551in}}%
\pgfpathlineto{\pgfqpoint{2.585920in}{1.398675in}}%
\pgfpathlineto{\pgfqpoint{2.556430in}{1.402676in}}%
\pgfusepath{fill}%
\end{pgfscope}%
\begin{pgfscope}%
\pgfpathrectangle{\pgfqpoint{0.800000in}{0.528000in}}{\pgfqpoint{3.968000in}{3.696000in}} %
\pgfusepath{clip}%
\pgfsetbuttcap%
\pgfsetroundjoin%
\definecolor{currentfill}{rgb}{0.260571,0.246922,0.522828}%
\pgfsetfillcolor{currentfill}%
\pgfsetlinewidth{0.000000pt}%
\definecolor{currentstroke}{rgb}{0.000000,0.000000,0.000000}%
\pgfsetstrokecolor{currentstroke}%
\pgfsetdash{}{0pt}%
\pgfpathmoveto{\pgfqpoint{2.801036in}{1.408979in}}%
\pgfpathlineto{\pgfqpoint{3.660316in}{2.686725in}}%
\pgfpathlineto{\pgfqpoint{3.627317in}{2.690985in}}%
\pgfpathlineto{\pgfqpoint{3.747397in}{2.789549in}}%
\pgfpathlineto{\pgfqpoint{3.701403in}{2.641162in}}%
\pgfpathlineto{\pgfqpoint{3.685011in}{2.670117in}}%
\pgfpathlineto{\pgfqpoint{2.825731in}{1.392372in}}%
\pgfpathlineto{\pgfqpoint{2.801036in}{1.408979in}}%
\pgfusepath{fill}%
\end{pgfscope}%
\begin{pgfscope}%
\pgfpathrectangle{\pgfqpoint{0.800000in}{0.528000in}}{\pgfqpoint{3.968000in}{3.696000in}} %
\pgfusepath{clip}%
\pgfsetbuttcap%
\pgfsetroundjoin%
\definecolor{currentfill}{rgb}{0.281477,0.755203,0.432552}%
\pgfsetfillcolor{currentfill}%
\pgfsetlinewidth{0.000000pt}%
\definecolor{currentstroke}{rgb}{0.000000,0.000000,0.000000}%
\pgfsetstrokecolor{currentstroke}%
\pgfsetdash{}{0pt}%
\pgfpathmoveto{\pgfqpoint{3.045408in}{1.411525in}}%
\pgfpathlineto{\pgfqpoint{4.272512in}{2.563256in}}%
\pgfpathlineto{\pgfqpoint{4.241296in}{2.574773in}}%
\pgfpathlineto{\pgfqpoint{4.380342in}{2.644056in}}%
\pgfpathlineto{\pgfqpoint{4.302395in}{2.509674in}}%
\pgfpathlineto{\pgfqpoint{4.292878in}{2.541557in}}%
\pgfpathlineto{\pgfqpoint{3.065775in}{1.389826in}}%
\pgfpathlineto{\pgfqpoint{3.045408in}{1.411525in}}%
\pgfusepath{fill}%
\end{pgfscope}%
\begin{pgfscope}%
\pgfpathrectangle{\pgfqpoint{0.800000in}{0.528000in}}{\pgfqpoint{3.968000in}{3.696000in}} %
\pgfusepath{clip}%
\pgfsetbuttcap%
\pgfsetroundjoin%
\definecolor{currentfill}{rgb}{0.276022,0.044167,0.370164}%
\pgfsetfillcolor{currentfill}%
\pgfsetlinewidth{0.000000pt}%
\definecolor{currentstroke}{rgb}{0.000000,0.000000,0.000000}%
\pgfsetstrokecolor{currentstroke}%
\pgfsetdash{}{0pt}%
\pgfpathmoveto{\pgfqpoint{3.044668in}{1.410779in}}%
\pgfpathlineto{\pgfqpoint{4.307331in}{2.775989in}}%
\pgfpathlineto{\pgfqpoint{4.275379in}{2.785272in}}%
\pgfpathlineto{\pgfqpoint{4.409186in}{2.864201in}}%
\pgfpathlineto{\pgfqpoint{4.340923in}{2.724651in}}%
\pgfpathlineto{\pgfqpoint{4.329179in}{2.755782in}}%
\pgfpathlineto{\pgfqpoint{3.066516in}{1.390572in}}%
\pgfpathlineto{\pgfqpoint{3.044668in}{1.410779in}}%
\pgfusepath{fill}%
\end{pgfscope}%
\begin{pgfscope}%
\pgfpathrectangle{\pgfqpoint{0.800000in}{0.528000in}}{\pgfqpoint{3.968000in}{3.696000in}} %
\pgfusepath{clip}%
\pgfsetbuttcap%
\pgfsetroundjoin%
\definecolor{currentfill}{rgb}{0.283187,0.125848,0.444960}%
\pgfsetfillcolor{currentfill}%
\pgfsetlinewidth{0.000000pt}%
\definecolor{currentstroke}{rgb}{0.000000,0.000000,0.000000}%
\pgfsetstrokecolor{currentstroke}%
\pgfsetdash{}{0pt}%
\pgfpathmoveto{\pgfqpoint{3.043994in}{1.409998in}}%
\pgfpathlineto{\pgfqpoint{3.819868in}{2.375210in}}%
\pgfpathlineto{\pgfqpoint{3.787350in}{2.382258in}}%
\pgfpathlineto{\pgfqpoint{3.915369in}{2.470266in}}%
\pgfpathlineto{\pgfqpoint{3.856936in}{2.326323in}}%
\pgfpathlineto{\pgfqpoint{3.843063in}{2.356565in}}%
\pgfpathlineto{\pgfqpoint{3.067189in}{1.391353in}}%
\pgfpathlineto{\pgfqpoint{3.043994in}{1.409998in}}%
\pgfusepath{fill}%
\end{pgfscope}%
\begin{pgfscope}%
\pgfpathrectangle{\pgfqpoint{0.800000in}{0.528000in}}{\pgfqpoint{3.968000in}{3.696000in}} %
\pgfusepath{clip}%
\pgfsetbuttcap%
\pgfsetroundjoin%
\definecolor{currentfill}{rgb}{0.269944,0.014625,0.341379}%
\pgfsetfillcolor{currentfill}%
\pgfsetlinewidth{0.000000pt}%
\definecolor{currentstroke}{rgb}{0.000000,0.000000,0.000000}%
\pgfsetstrokecolor{currentstroke}%
\pgfsetdash{}{0pt}%
\pgfpathmoveto{\pgfqpoint{2.073181in}{1.581772in}}%
\pgfpathlineto{\pgfqpoint{2.608585in}{2.776117in}}%
\pgfpathlineto{\pgfqpoint{2.575342in}{2.774713in}}%
\pgfpathlineto{\pgfqpoint{2.676945in}{2.892233in}}%
\pgfpathlineto{\pgfqpoint{2.656811in}{2.738192in}}%
\pgfpathlineto{\pgfqpoint{2.635742in}{2.763944in}}%
\pgfpathlineto{\pgfqpoint{2.100337in}{1.569598in}}%
\pgfpathlineto{\pgfqpoint{2.073181in}{1.581772in}}%
\pgfusepath{fill}%
\end{pgfscope}%
\begin{pgfscope}%
\pgfpathrectangle{\pgfqpoint{0.800000in}{0.528000in}}{\pgfqpoint{3.968000in}{3.696000in}} %
\pgfusepath{clip}%
\pgfsetbuttcap%
\pgfsetroundjoin%
\definecolor{currentfill}{rgb}{0.127568,0.566949,0.550556}%
\pgfsetfillcolor{currentfill}%
\pgfsetlinewidth{0.000000pt}%
\definecolor{currentstroke}{rgb}{0.000000,0.000000,0.000000}%
\pgfsetstrokecolor{currentstroke}%
\pgfsetdash{}{0pt}%
\pgfpathmoveto{\pgfqpoint{2.072153in}{1.578526in}}%
\pgfpathlineto{\pgfqpoint{2.301032in}{2.755482in}}%
\pgfpathlineto{\pgfqpoint{2.268979in}{2.746556in}}%
\pgfpathlineto{\pgfqpoint{2.341203in}{2.884099in}}%
\pgfpathlineto{\pgfqpoint{2.356618in}{2.729514in}}%
\pgfpathlineto{\pgfqpoint{2.330245in}{2.749801in}}%
\pgfpathlineto{\pgfqpoint{2.101365in}{1.572845in}}%
\pgfpathlineto{\pgfqpoint{2.072153in}{1.578526in}}%
\pgfusepath{fill}%
\end{pgfscope}%
\begin{pgfscope}%
\pgfpathrectangle{\pgfqpoint{0.800000in}{0.528000in}}{\pgfqpoint{3.968000in}{3.696000in}} %
\pgfusepath{clip}%
\pgfsetbuttcap%
\pgfsetroundjoin%
\definecolor{currentfill}{rgb}{0.253935,0.265254,0.529983}%
\pgfsetfillcolor{currentfill}%
\pgfsetlinewidth{0.000000pt}%
\definecolor{currentstroke}{rgb}{0.000000,0.000000,0.000000}%
\pgfsetstrokecolor{currentstroke}%
\pgfsetdash{}{0pt}%
\pgfpathmoveto{\pgfqpoint{2.556467in}{1.577938in}}%
\pgfpathlineto{\pgfqpoint{2.744537in}{2.806131in}}%
\pgfpathlineto{\pgfqpoint{2.712867in}{2.795927in}}%
\pgfpathlineto{\pgfqpoint{2.779516in}{2.936256in}}%
\pgfpathlineto{\pgfqpoint{2.801119in}{2.782414in}}%
\pgfpathlineto{\pgfqpoint{2.773954in}{2.801627in}}%
\pgfpathlineto{\pgfqpoint{2.585884in}{1.573433in}}%
\pgfpathlineto{\pgfqpoint{2.556467in}{1.577938in}}%
\pgfusepath{fill}%
\end{pgfscope}%
\begin{pgfscope}%
\pgfpathrectangle{\pgfqpoint{0.800000in}{0.528000in}}{\pgfqpoint{3.968000in}{3.696000in}} %
\pgfusepath{clip}%
\pgfsetbuttcap%
\pgfsetroundjoin%
\definecolor{currentfill}{rgb}{0.174274,0.445044,0.557792}%
\pgfsetfillcolor{currentfill}%
\pgfsetlinewidth{0.000000pt}%
\definecolor{currentstroke}{rgb}{0.000000,0.000000,0.000000}%
\pgfsetstrokecolor{currentstroke}%
\pgfsetdash{}{0pt}%
\pgfpathmoveto{\pgfqpoint{2.556932in}{1.579992in}}%
\pgfpathlineto{\pgfqpoint{2.987570in}{3.004333in}}%
\pgfpathlineto{\pgfqpoint{2.954777in}{2.998702in}}%
\pgfpathlineto{\pgfqpoint{3.040570in}{3.128215in}}%
\pgfpathlineto{\pgfqpoint{3.040236in}{2.972864in}}%
\pgfpathlineto{\pgfqpoint{3.016056in}{2.995720in}}%
\pgfpathlineto{\pgfqpoint{2.585419in}{1.571379in}}%
\pgfpathlineto{\pgfqpoint{2.556932in}{1.579992in}}%
\pgfusepath{fill}%
\end{pgfscope}%
\begin{pgfscope}%
\pgfpathrectangle{\pgfqpoint{0.800000in}{0.528000in}}{\pgfqpoint{3.968000in}{3.696000in}} %
\pgfusepath{clip}%
\pgfsetbuttcap%
\pgfsetroundjoin%
\definecolor{currentfill}{rgb}{0.281446,0.084320,0.407414}%
\pgfsetfillcolor{currentfill}%
\pgfsetlinewidth{0.000000pt}%
\definecolor{currentstroke}{rgb}{0.000000,0.000000,0.000000}%
\pgfsetstrokecolor{currentstroke}%
\pgfsetdash{}{0pt}%
\pgfpathmoveto{\pgfqpoint{2.801591in}{1.584759in}}%
\pgfpathlineto{\pgfqpoint{3.653937in}{2.692487in}}%
\pgfpathlineto{\pgfqpoint{3.621276in}{2.698842in}}%
\pgfpathlineto{\pgfqpoint{3.747397in}{2.789549in}}%
\pgfpathlineto{\pgfqpoint{3.692034in}{2.644397in}}%
\pgfpathlineto{\pgfqpoint{3.677522in}{2.674338in}}%
\pgfpathlineto{\pgfqpoint{2.825176in}{1.566611in}}%
\pgfpathlineto{\pgfqpoint{2.801591in}{1.584759in}}%
\pgfusepath{fill}%
\end{pgfscope}%
\begin{pgfscope}%
\pgfpathrectangle{\pgfqpoint{0.800000in}{0.528000in}}{\pgfqpoint{3.968000in}{3.696000in}} %
\pgfusepath{clip}%
\pgfsetbuttcap%
\pgfsetroundjoin%
\definecolor{currentfill}{rgb}{0.282290,0.145912,0.461510}%
\pgfsetfillcolor{currentfill}%
\pgfsetlinewidth{0.000000pt}%
\definecolor{currentstroke}{rgb}{0.000000,0.000000,0.000000}%
\pgfsetstrokecolor{currentstroke}%
\pgfsetdash{}{0pt}%
\pgfpathmoveto{\pgfqpoint{2.800937in}{1.583840in}}%
\pgfpathlineto{\pgfqpoint{3.802522in}{3.112527in}}%
\pgfpathlineto{\pgfqpoint{3.769474in}{3.116390in}}%
\pgfpathlineto{\pgfqpoint{3.888362in}{3.216390in}}%
\pgfpathlineto{\pgfqpoint{3.844153in}{3.067461in}}%
\pgfpathlineto{\pgfqpoint{3.827415in}{3.096217in}}%
\pgfpathlineto{\pgfqpoint{2.825830in}{1.567530in}}%
\pgfpathlineto{\pgfqpoint{2.800937in}{1.583840in}}%
\pgfusepath{fill}%
\end{pgfscope}%
\begin{pgfscope}%
\pgfpathrectangle{\pgfqpoint{0.800000in}{0.528000in}}{\pgfqpoint{3.968000in}{3.696000in}} %
\pgfusepath{clip}%
\pgfsetbuttcap%
\pgfsetroundjoin%
\definecolor{currentfill}{rgb}{0.156270,0.489624,0.557936}%
\pgfsetfillcolor{currentfill}%
\pgfsetlinewidth{0.000000pt}%
\definecolor{currentstroke}{rgb}{0.000000,0.000000,0.000000}%
\pgfsetstrokecolor{currentstroke}%
\pgfsetdash{}{0pt}%
\pgfpathmoveto{\pgfqpoint{3.042323in}{1.582420in}}%
\pgfpathlineto{\pgfqpoint{3.814480in}{3.103706in}}%
\pgfpathlineto{\pgfqpoint{3.781208in}{3.103907in}}%
\pgfpathlineto{\pgfqpoint{3.888362in}{3.216390in}}%
\pgfpathlineto{\pgfqpoint{3.860820in}{3.063499in}}%
\pgfpathlineto{\pgfqpoint{3.841018in}{3.090237in}}%
\pgfpathlineto{\pgfqpoint{3.068860in}{1.568951in}}%
\pgfpathlineto{\pgfqpoint{3.042323in}{1.582420in}}%
\pgfusepath{fill}%
\end{pgfscope}%
\begin{pgfscope}%
\pgfpathrectangle{\pgfqpoint{0.800000in}{0.528000in}}{\pgfqpoint{3.968000in}{3.696000in}} %
\pgfusepath{clip}%
\pgfsetbuttcap%
\pgfsetroundjoin%
\definecolor{currentfill}{rgb}{0.267968,0.223549,0.512008}%
\pgfsetfillcolor{currentfill}%
\pgfsetlinewidth{0.000000pt}%
\definecolor{currentstroke}{rgb}{0.000000,0.000000,0.000000}%
\pgfsetstrokecolor{currentstroke}%
\pgfsetdash{}{0pt}%
\pgfpathmoveto{\pgfqpoint{2.073912in}{1.758202in}}%
\pgfpathlineto{\pgfqpoint{2.699104in}{2.828136in}}%
\pgfpathlineto{\pgfqpoint{2.665902in}{2.830303in}}%
\pgfpathlineto{\pgfqpoint{2.779516in}{2.936256in}}%
\pgfpathlineto{\pgfqpoint{2.742987in}{2.785260in}}%
\pgfpathlineto{\pgfqpoint{2.724799in}{2.813122in}}%
\pgfpathlineto{\pgfqpoint{2.099607in}{1.743188in}}%
\pgfpathlineto{\pgfqpoint{2.073912in}{1.758202in}}%
\pgfusepath{fill}%
\end{pgfscope}%
\begin{pgfscope}%
\pgfpathrectangle{\pgfqpoint{0.800000in}{0.528000in}}{\pgfqpoint{3.968000in}{3.696000in}} %
\pgfusepath{clip}%
\pgfsetbuttcap%
\pgfsetroundjoin%
\definecolor{currentfill}{rgb}{0.123444,0.636809,0.528763}%
\pgfsetfillcolor{currentfill}%
\pgfsetlinewidth{0.000000pt}%
\definecolor{currentstroke}{rgb}{0.000000,0.000000,0.000000}%
\pgfsetstrokecolor{currentstroke}%
\pgfsetdash{}{0pt}%
\pgfpathmoveto{\pgfqpoint{2.072227in}{1.747499in}}%
\pgfpathlineto{\pgfqpoint{1.718956in}{3.353495in}}%
\pgfpathlineto{\pgfqpoint{1.693087in}{3.332569in}}%
\pgfpathlineto{\pgfqpoint{1.704718in}{3.487485in}}%
\pgfpathlineto{\pgfqpoint{1.780283in}{3.351750in}}%
\pgfpathlineto{\pgfqpoint{1.748021in}{3.359889in}}%
\pgfpathlineto{\pgfqpoint{2.101292in}{1.753892in}}%
\pgfpathlineto{\pgfqpoint{2.072227in}{1.747499in}}%
\pgfusepath{fill}%
\end{pgfscope}%
\begin{pgfscope}%
\pgfpathrectangle{\pgfqpoint{0.800000in}{0.528000in}}{\pgfqpoint{3.968000in}{3.696000in}} %
\pgfusepath{clip}%
\pgfsetbuttcap%
\pgfsetroundjoin%
\definecolor{currentfill}{rgb}{0.281924,0.089666,0.412415}%
\pgfsetfillcolor{currentfill}%
\pgfsetlinewidth{0.000000pt}%
\definecolor{currentstroke}{rgb}{0.000000,0.000000,0.000000}%
\pgfsetstrokecolor{currentstroke}%
\pgfsetdash{}{0pt}%
\pgfpathmoveto{\pgfqpoint{2.073541in}{1.757529in}}%
\pgfpathlineto{\pgfqpoint{2.602223in}{2.780106in}}%
\pgfpathlineto{\pgfqpoint{2.568953in}{2.780555in}}%
\pgfpathlineto{\pgfqpoint{2.676945in}{2.892233in}}%
\pgfpathlineto{\pgfqpoint{2.648261in}{2.739553in}}%
\pgfpathlineto{\pgfqpoint{2.628659in}{2.766438in}}%
\pgfpathlineto{\pgfqpoint{2.099977in}{1.743861in}}%
\pgfpathlineto{\pgfqpoint{2.073541in}{1.757529in}}%
\pgfusepath{fill}%
\end{pgfscope}%
\begin{pgfscope}%
\pgfpathrectangle{\pgfqpoint{0.800000in}{0.528000in}}{\pgfqpoint{3.968000in}{3.696000in}} %
\pgfusepath{clip}%
\pgfsetbuttcap%
\pgfsetroundjoin%
\definecolor{currentfill}{rgb}{0.267004,0.004874,0.329415}%
\pgfsetfillcolor{currentfill}%
\pgfsetlinewidth{0.000000pt}%
\definecolor{currentstroke}{rgb}{0.000000,0.000000,0.000000}%
\pgfsetstrokecolor{currentstroke}%
\pgfsetdash{}{0pt}%
\pgfpathmoveto{\pgfqpoint{2.314592in}{1.754537in}}%
\pgfpathlineto{\pgfqpoint{2.886538in}{3.894413in}}%
\pgfpathlineto{\pgfqpoint{2.853945in}{3.887722in}}%
\pgfpathlineto{\pgfqpoint{2.935494in}{4.019949in}}%
\pgfpathlineto{\pgfqpoint{2.940197in}{3.864668in}}%
\pgfpathlineto{\pgfqpoint{2.915289in}{3.886728in}}%
\pgfpathlineto{\pgfqpoint{2.343343in}{1.746853in}}%
\pgfpathlineto{\pgfqpoint{2.314592in}{1.754537in}}%
\pgfusepath{fill}%
\end{pgfscope}%
\begin{pgfscope}%
\pgfpathrectangle{\pgfqpoint{0.800000in}{0.528000in}}{\pgfqpoint{3.968000in}{3.696000in}} %
\pgfusepath{clip}%
\pgfsetbuttcap%
\pgfsetroundjoin%
\definecolor{currentfill}{rgb}{0.253935,0.265254,0.529983}%
\pgfsetfillcolor{currentfill}%
\pgfsetlinewidth{0.000000pt}%
\definecolor{currentstroke}{rgb}{0.000000,0.000000,0.000000}%
\pgfsetstrokecolor{currentstroke}%
\pgfsetdash{}{0pt}%
\pgfpathmoveto{\pgfqpoint{2.315747in}{1.757525in}}%
\pgfpathlineto{\pgfqpoint{2.965886in}{3.016063in}}%
\pgfpathlineto{\pgfqpoint{2.932616in}{3.016501in}}%
\pgfpathlineto{\pgfqpoint{3.040570in}{3.128215in}}%
\pgfpathlineto{\pgfqpoint{3.011937in}{2.975525in}}%
\pgfpathlineto{\pgfqpoint{2.992326in}{3.002404in}}%
\pgfpathlineto{\pgfqpoint{2.342187in}{1.743866in}}%
\pgfpathlineto{\pgfqpoint{2.315747in}{1.757525in}}%
\pgfusepath{fill}%
\end{pgfscope}%
\begin{pgfscope}%
\pgfpathrectangle{\pgfqpoint{0.800000in}{0.528000in}}{\pgfqpoint{3.968000in}{3.696000in}} %
\pgfusepath{clip}%
\pgfsetbuttcap%
\pgfsetroundjoin%
\definecolor{currentfill}{rgb}{0.993248,0.906157,0.143936}%
\pgfsetfillcolor{currentfill}%
\pgfsetlinewidth{0.000000pt}%
\definecolor{currentstroke}{rgb}{0.000000,0.000000,0.000000}%
\pgfsetstrokecolor{currentstroke}%
\pgfsetdash{}{0pt}%
\pgfpathmoveto{\pgfqpoint{2.557679in}{1.756961in}}%
\pgfpathlineto{\pgfqpoint{3.567971in}{3.932961in}}%
\pgfpathlineto{\pgfqpoint{3.534712in}{3.931997in}}%
\pgfpathlineto{\pgfqpoint{3.637863in}{4.048161in}}%
\pgfpathlineto{\pgfqpoint{3.615690in}{3.894400in}}%
\pgfpathlineto{\pgfqpoint{3.594963in}{3.920429in}}%
\pgfpathlineto{\pgfqpoint{2.584672in}{1.744429in}}%
\pgfpathlineto{\pgfqpoint{2.557679in}{1.756961in}}%
\pgfusepath{fill}%
\end{pgfscope}%
\begin{pgfscope}%
\pgfpathrectangle{\pgfqpoint{0.800000in}{0.528000in}}{\pgfqpoint{3.968000in}{3.696000in}} %
\pgfusepath{clip}%
\pgfsetbuttcap%
\pgfsetroundjoin%
\definecolor{currentfill}{rgb}{0.283187,0.125848,0.444960}%
\pgfsetfillcolor{currentfill}%
\pgfsetlinewidth{0.000000pt}%
\definecolor{currentstroke}{rgb}{0.000000,0.000000,0.000000}%
\pgfsetstrokecolor{currentstroke}%
\pgfsetdash{}{0pt}%
\pgfpathmoveto{\pgfqpoint{2.801385in}{1.759495in}}%
\pgfpathlineto{\pgfqpoint{3.797161in}{3.117201in}}%
\pgfpathlineto{\pgfqpoint{3.764363in}{3.122802in}}%
\pgfpathlineto{\pgfqpoint{3.888362in}{3.216390in}}%
\pgfpathlineto{\pgfqpoint{3.836356in}{3.070001in}}%
\pgfpathlineto{\pgfqpoint{3.821159in}{3.099600in}}%
\pgfpathlineto{\pgfqpoint{2.825382in}{1.741895in}}%
\pgfpathlineto{\pgfqpoint{2.801385in}{1.759495in}}%
\pgfusepath{fill}%
\end{pgfscope}%
\begin{pgfscope}%
\pgfpathrectangle{\pgfqpoint{0.800000in}{0.528000in}}{\pgfqpoint{3.968000in}{3.696000in}} %
\pgfusepath{clip}%
\pgfsetbuttcap%
\pgfsetroundjoin%
\definecolor{currentfill}{rgb}{0.128729,0.563265,0.551229}%
\pgfsetfillcolor{currentfill}%
\pgfsetlinewidth{0.000000pt}%
\definecolor{currentstroke}{rgb}{0.000000,0.000000,0.000000}%
\pgfsetstrokecolor{currentstroke}%
\pgfsetdash{}{0pt}%
\pgfpathmoveto{\pgfqpoint{2.801358in}{1.759459in}}%
\pgfpathlineto{\pgfqpoint{4.115682in}{3.563060in}}%
\pgfpathlineto{\pgfqpoint{4.082867in}{3.568561in}}%
\pgfpathlineto{\pgfqpoint{4.206578in}{3.662528in}}%
\pgfpathlineto{\pgfqpoint{4.155021in}{3.515980in}}%
\pgfpathlineto{\pgfqpoint{4.139733in}{3.545533in}}%
\pgfpathlineto{\pgfqpoint{2.825409in}{1.741932in}}%
\pgfpathlineto{\pgfqpoint{2.801358in}{1.759459in}}%
\pgfusepath{fill}%
\end{pgfscope}%
\begin{pgfscope}%
\pgfsetbuttcap%
\pgfsetroundjoin%
\definecolor{currentfill}{rgb}{0.000000,0.000000,0.000000}%
\pgfsetfillcolor{currentfill}%
\pgfsetlinewidth{0.803000pt}%
\definecolor{currentstroke}{rgb}{0.000000,0.000000,0.000000}%
\pgfsetstrokecolor{currentstroke}%
\pgfsetdash{}{0pt}%
\pgfsys@defobject{currentmarker}{\pgfqpoint{0.000000in}{-0.048611in}}{\pgfqpoint{0.000000in}{0.000000in}}{%
\pgfpathmoveto{\pgfqpoint{0.000000in}{0.000000in}}%
\pgfpathlineto{\pgfqpoint{0.000000in}{-0.048611in}}%
\pgfusepath{stroke,fill}%
}%
\begin{pgfscope}%
\pgfsys@transformshift{1.360135in}{0.528000in}%
\pgfsys@useobject{currentmarker}{}%
\end{pgfscope}%
\end{pgfscope}%
\begin{pgfscope}%
\pgftext[x=1.360135in,y=0.430778in,,top]{\rmfamily\fontsize{10.000000}{12.000000}\selectfont \(\displaystyle -0.5\)}%
\end{pgfscope}%
\begin{pgfscope}%
\pgfsetbuttcap%
\pgfsetroundjoin%
\definecolor{currentfill}{rgb}{0.000000,0.000000,0.000000}%
\pgfsetfillcolor{currentfill}%
\pgfsetlinewidth{0.803000pt}%
\definecolor{currentstroke}{rgb}{0.000000,0.000000,0.000000}%
\pgfsetstrokecolor{currentstroke}%
\pgfsetdash{}{0pt}%
\pgfsys@defobject{currentmarker}{\pgfqpoint{0.000000in}{-0.048611in}}{\pgfqpoint{0.000000in}{0.000000in}}{%
\pgfpathmoveto{\pgfqpoint{0.000000in}{0.000000in}}%
\pgfpathlineto{\pgfqpoint{0.000000in}{-0.048611in}}%
\pgfusepath{stroke,fill}%
}%
\begin{pgfscope}%
\pgfsys@transformshift{1.965655in}{0.528000in}%
\pgfsys@useobject{currentmarker}{}%
\end{pgfscope}%
\end{pgfscope}%
\begin{pgfscope}%
\pgftext[x=1.965655in,y=0.430778in,,top]{\rmfamily\fontsize{10.000000}{12.000000}\selectfont \(\displaystyle 0.0\)}%
\end{pgfscope}%
\begin{pgfscope}%
\pgfsetbuttcap%
\pgfsetroundjoin%
\definecolor{currentfill}{rgb}{0.000000,0.000000,0.000000}%
\pgfsetfillcolor{currentfill}%
\pgfsetlinewidth{0.803000pt}%
\definecolor{currentstroke}{rgb}{0.000000,0.000000,0.000000}%
\pgfsetstrokecolor{currentstroke}%
\pgfsetdash{}{0pt}%
\pgfsys@defobject{currentmarker}{\pgfqpoint{0.000000in}{-0.048611in}}{\pgfqpoint{0.000000in}{0.000000in}}{%
\pgfpathmoveto{\pgfqpoint{0.000000in}{0.000000in}}%
\pgfpathlineto{\pgfqpoint{0.000000in}{-0.048611in}}%
\pgfusepath{stroke,fill}%
}%
\begin{pgfscope}%
\pgfsys@transformshift{2.571175in}{0.528000in}%
\pgfsys@useobject{currentmarker}{}%
\end{pgfscope}%
\end{pgfscope}%
\begin{pgfscope}%
\pgftext[x=2.571175in,y=0.430778in,,top]{\rmfamily\fontsize{10.000000}{12.000000}\selectfont \(\displaystyle 0.5\)}%
\end{pgfscope}%
\begin{pgfscope}%
\pgfsetbuttcap%
\pgfsetroundjoin%
\definecolor{currentfill}{rgb}{0.000000,0.000000,0.000000}%
\pgfsetfillcolor{currentfill}%
\pgfsetlinewidth{0.803000pt}%
\definecolor{currentstroke}{rgb}{0.000000,0.000000,0.000000}%
\pgfsetstrokecolor{currentstroke}%
\pgfsetdash{}{0pt}%
\pgfsys@defobject{currentmarker}{\pgfqpoint{0.000000in}{-0.048611in}}{\pgfqpoint{0.000000in}{0.000000in}}{%
\pgfpathmoveto{\pgfqpoint{0.000000in}{0.000000in}}%
\pgfpathlineto{\pgfqpoint{0.000000in}{-0.048611in}}%
\pgfusepath{stroke,fill}%
}%
\begin{pgfscope}%
\pgfsys@transformshift{3.176696in}{0.528000in}%
\pgfsys@useobject{currentmarker}{}%
\end{pgfscope}%
\end{pgfscope}%
\begin{pgfscope}%
\pgftext[x=3.176696in,y=0.430778in,,top]{\rmfamily\fontsize{10.000000}{12.000000}\selectfont \(\displaystyle 1.0\)}%
\end{pgfscope}%
\begin{pgfscope}%
\pgfsetbuttcap%
\pgfsetroundjoin%
\definecolor{currentfill}{rgb}{0.000000,0.000000,0.000000}%
\pgfsetfillcolor{currentfill}%
\pgfsetlinewidth{0.803000pt}%
\definecolor{currentstroke}{rgb}{0.000000,0.000000,0.000000}%
\pgfsetstrokecolor{currentstroke}%
\pgfsetdash{}{0pt}%
\pgfsys@defobject{currentmarker}{\pgfqpoint{0.000000in}{-0.048611in}}{\pgfqpoint{0.000000in}{0.000000in}}{%
\pgfpathmoveto{\pgfqpoint{0.000000in}{0.000000in}}%
\pgfpathlineto{\pgfqpoint{0.000000in}{-0.048611in}}%
\pgfusepath{stroke,fill}%
}%
\begin{pgfscope}%
\pgfsys@transformshift{3.782216in}{0.528000in}%
\pgfsys@useobject{currentmarker}{}%
\end{pgfscope}%
\end{pgfscope}%
\begin{pgfscope}%
\pgftext[x=3.782216in,y=0.430778in,,top]{\rmfamily\fontsize{10.000000}{12.000000}\selectfont \(\displaystyle 1.5\)}%
\end{pgfscope}%
\begin{pgfscope}%
\pgfsetbuttcap%
\pgfsetroundjoin%
\definecolor{currentfill}{rgb}{0.000000,0.000000,0.000000}%
\pgfsetfillcolor{currentfill}%
\pgfsetlinewidth{0.803000pt}%
\definecolor{currentstroke}{rgb}{0.000000,0.000000,0.000000}%
\pgfsetstrokecolor{currentstroke}%
\pgfsetdash{}{0pt}%
\pgfsys@defobject{currentmarker}{\pgfqpoint{0.000000in}{-0.048611in}}{\pgfqpoint{0.000000in}{0.000000in}}{%
\pgfpathmoveto{\pgfqpoint{0.000000in}{0.000000in}}%
\pgfpathlineto{\pgfqpoint{0.000000in}{-0.048611in}}%
\pgfusepath{stroke,fill}%
}%
\begin{pgfscope}%
\pgfsys@transformshift{4.387736in}{0.528000in}%
\pgfsys@useobject{currentmarker}{}%
\end{pgfscope}%
\end{pgfscope}%
\begin{pgfscope}%
\pgftext[x=4.387736in,y=0.430778in,,top]{\rmfamily\fontsize{10.000000}{12.000000}\selectfont \(\displaystyle 2.0\)}%
\end{pgfscope}%
\begin{pgfscope}%
\pgfsetbuttcap%
\pgfsetroundjoin%
\definecolor{currentfill}{rgb}{0.000000,0.000000,0.000000}%
\pgfsetfillcolor{currentfill}%
\pgfsetlinewidth{0.803000pt}%
\definecolor{currentstroke}{rgb}{0.000000,0.000000,0.000000}%
\pgfsetstrokecolor{currentstroke}%
\pgfsetdash{}{0pt}%
\pgfsys@defobject{currentmarker}{\pgfqpoint{-0.048611in}{0.000000in}}{\pgfqpoint{0.000000in}{0.000000in}}{%
\pgfpathmoveto{\pgfqpoint{0.000000in}{0.000000in}}%
\pgfpathlineto{\pgfqpoint{-0.048611in}{0.000000in}}%
\pgfusepath{stroke,fill}%
}%
\begin{pgfscope}%
\pgfsys@transformshift{0.800000in}{0.525626in}%
\pgfsys@useobject{currentmarker}{}%
\end{pgfscope}%
\end{pgfscope}%
\begin{pgfscope}%
\pgftext[x=0.417283in,y=0.477431in,left,base]{\rmfamily\fontsize{10.000000}{12.000000}\selectfont \(\displaystyle -0.5\)}%
\end{pgfscope}%
\begin{pgfscope}%
\pgfsetbuttcap%
\pgfsetroundjoin%
\definecolor{currentfill}{rgb}{0.000000,0.000000,0.000000}%
\pgfsetfillcolor{currentfill}%
\pgfsetlinewidth{0.803000pt}%
\definecolor{currentstroke}{rgb}{0.000000,0.000000,0.000000}%
\pgfsetstrokecolor{currentstroke}%
\pgfsetdash{}{0pt}%
\pgfsys@defobject{currentmarker}{\pgfqpoint{-0.048611in}{0.000000in}}{\pgfqpoint{0.000000in}{0.000000in}}{%
\pgfpathmoveto{\pgfqpoint{0.000000in}{0.000000in}}%
\pgfpathlineto{\pgfqpoint{-0.048611in}{0.000000in}}%
\pgfusepath{stroke,fill}%
}%
\begin{pgfscope}%
\pgfsys@transformshift{0.800000in}{0.963150in}%
\pgfsys@useobject{currentmarker}{}%
\end{pgfscope}%
\end{pgfscope}%
\begin{pgfscope}%
\pgftext[x=0.525308in,y=0.914956in,left,base]{\rmfamily\fontsize{10.000000}{12.000000}\selectfont \(\displaystyle 0.0\)}%
\end{pgfscope}%
\begin{pgfscope}%
\pgfsetbuttcap%
\pgfsetroundjoin%
\definecolor{currentfill}{rgb}{0.000000,0.000000,0.000000}%
\pgfsetfillcolor{currentfill}%
\pgfsetlinewidth{0.803000pt}%
\definecolor{currentstroke}{rgb}{0.000000,0.000000,0.000000}%
\pgfsetstrokecolor{currentstroke}%
\pgfsetdash{}{0pt}%
\pgfsys@defobject{currentmarker}{\pgfqpoint{-0.048611in}{0.000000in}}{\pgfqpoint{0.000000in}{0.000000in}}{%
\pgfpathmoveto{\pgfqpoint{0.000000in}{0.000000in}}%
\pgfpathlineto{\pgfqpoint{-0.048611in}{0.000000in}}%
\pgfusepath{stroke,fill}%
}%
\begin{pgfscope}%
\pgfsys@transformshift{0.800000in}{1.400675in}%
\pgfsys@useobject{currentmarker}{}%
\end{pgfscope}%
\end{pgfscope}%
\begin{pgfscope}%
\pgftext[x=0.525308in,y=1.352481in,left,base]{\rmfamily\fontsize{10.000000}{12.000000}\selectfont \(\displaystyle 0.5\)}%
\end{pgfscope}%
\begin{pgfscope}%
\pgfsetbuttcap%
\pgfsetroundjoin%
\definecolor{currentfill}{rgb}{0.000000,0.000000,0.000000}%
\pgfsetfillcolor{currentfill}%
\pgfsetlinewidth{0.803000pt}%
\definecolor{currentstroke}{rgb}{0.000000,0.000000,0.000000}%
\pgfsetstrokecolor{currentstroke}%
\pgfsetdash{}{0pt}%
\pgfsys@defobject{currentmarker}{\pgfqpoint{-0.048611in}{0.000000in}}{\pgfqpoint{0.000000in}{0.000000in}}{%
\pgfpathmoveto{\pgfqpoint{0.000000in}{0.000000in}}%
\pgfpathlineto{\pgfqpoint{-0.048611in}{0.000000in}}%
\pgfusepath{stroke,fill}%
}%
\begin{pgfscope}%
\pgfsys@transformshift{0.800000in}{1.838200in}%
\pgfsys@useobject{currentmarker}{}%
\end{pgfscope}%
\end{pgfscope}%
\begin{pgfscope}%
\pgftext[x=0.525308in,y=1.790006in,left,base]{\rmfamily\fontsize{10.000000}{12.000000}\selectfont \(\displaystyle 1.0\)}%
\end{pgfscope}%
\begin{pgfscope}%
\pgfsetbuttcap%
\pgfsetroundjoin%
\definecolor{currentfill}{rgb}{0.000000,0.000000,0.000000}%
\pgfsetfillcolor{currentfill}%
\pgfsetlinewidth{0.803000pt}%
\definecolor{currentstroke}{rgb}{0.000000,0.000000,0.000000}%
\pgfsetstrokecolor{currentstroke}%
\pgfsetdash{}{0pt}%
\pgfsys@defobject{currentmarker}{\pgfqpoint{-0.048611in}{0.000000in}}{\pgfqpoint{0.000000in}{0.000000in}}{%
\pgfpathmoveto{\pgfqpoint{0.000000in}{0.000000in}}%
\pgfpathlineto{\pgfqpoint{-0.048611in}{0.000000in}}%
\pgfusepath{stroke,fill}%
}%
\begin{pgfscope}%
\pgfsys@transformshift{0.800000in}{2.275725in}%
\pgfsys@useobject{currentmarker}{}%
\end{pgfscope}%
\end{pgfscope}%
\begin{pgfscope}%
\pgftext[x=0.525308in,y=2.227531in,left,base]{\rmfamily\fontsize{10.000000}{12.000000}\selectfont \(\displaystyle 1.5\)}%
\end{pgfscope}%
\begin{pgfscope}%
\pgfsetbuttcap%
\pgfsetroundjoin%
\definecolor{currentfill}{rgb}{0.000000,0.000000,0.000000}%
\pgfsetfillcolor{currentfill}%
\pgfsetlinewidth{0.803000pt}%
\definecolor{currentstroke}{rgb}{0.000000,0.000000,0.000000}%
\pgfsetstrokecolor{currentstroke}%
\pgfsetdash{}{0pt}%
\pgfsys@defobject{currentmarker}{\pgfqpoint{-0.048611in}{0.000000in}}{\pgfqpoint{0.000000in}{0.000000in}}{%
\pgfpathmoveto{\pgfqpoint{0.000000in}{0.000000in}}%
\pgfpathlineto{\pgfqpoint{-0.048611in}{0.000000in}}%
\pgfusepath{stroke,fill}%
}%
\begin{pgfscope}%
\pgfsys@transformshift{0.800000in}{2.713250in}%
\pgfsys@useobject{currentmarker}{}%
\end{pgfscope}%
\end{pgfscope}%
\begin{pgfscope}%
\pgftext[x=0.525308in,y=2.665056in,left,base]{\rmfamily\fontsize{10.000000}{12.000000}\selectfont \(\displaystyle 2.0\)}%
\end{pgfscope}%
\begin{pgfscope}%
\pgfsetbuttcap%
\pgfsetroundjoin%
\definecolor{currentfill}{rgb}{0.000000,0.000000,0.000000}%
\pgfsetfillcolor{currentfill}%
\pgfsetlinewidth{0.803000pt}%
\definecolor{currentstroke}{rgb}{0.000000,0.000000,0.000000}%
\pgfsetstrokecolor{currentstroke}%
\pgfsetdash{}{0pt}%
\pgfsys@defobject{currentmarker}{\pgfqpoint{-0.048611in}{0.000000in}}{\pgfqpoint{0.000000in}{0.000000in}}{%
\pgfpathmoveto{\pgfqpoint{0.000000in}{0.000000in}}%
\pgfpathlineto{\pgfqpoint{-0.048611in}{0.000000in}}%
\pgfusepath{stroke,fill}%
}%
\begin{pgfscope}%
\pgfsys@transformshift{0.800000in}{3.150775in}%
\pgfsys@useobject{currentmarker}{}%
\end{pgfscope}%
\end{pgfscope}%
\begin{pgfscope}%
\pgftext[x=0.525308in,y=3.102580in,left,base]{\rmfamily\fontsize{10.000000}{12.000000}\selectfont \(\displaystyle 2.5\)}%
\end{pgfscope}%
\begin{pgfscope}%
\pgfsetbuttcap%
\pgfsetroundjoin%
\definecolor{currentfill}{rgb}{0.000000,0.000000,0.000000}%
\pgfsetfillcolor{currentfill}%
\pgfsetlinewidth{0.803000pt}%
\definecolor{currentstroke}{rgb}{0.000000,0.000000,0.000000}%
\pgfsetstrokecolor{currentstroke}%
\pgfsetdash{}{0pt}%
\pgfsys@defobject{currentmarker}{\pgfqpoint{-0.048611in}{0.000000in}}{\pgfqpoint{0.000000in}{0.000000in}}{%
\pgfpathmoveto{\pgfqpoint{0.000000in}{0.000000in}}%
\pgfpathlineto{\pgfqpoint{-0.048611in}{0.000000in}}%
\pgfusepath{stroke,fill}%
}%
\begin{pgfscope}%
\pgfsys@transformshift{0.800000in}{3.588300in}%
\pgfsys@useobject{currentmarker}{}%
\end{pgfscope}%
\end{pgfscope}%
\begin{pgfscope}%
\pgftext[x=0.525308in,y=3.540105in,left,base]{\rmfamily\fontsize{10.000000}{12.000000}\selectfont \(\displaystyle 3.0\)}%
\end{pgfscope}%
\begin{pgfscope}%
\pgfsetbuttcap%
\pgfsetroundjoin%
\definecolor{currentfill}{rgb}{0.000000,0.000000,0.000000}%
\pgfsetfillcolor{currentfill}%
\pgfsetlinewidth{0.803000pt}%
\definecolor{currentstroke}{rgb}{0.000000,0.000000,0.000000}%
\pgfsetstrokecolor{currentstroke}%
\pgfsetdash{}{0pt}%
\pgfsys@defobject{currentmarker}{\pgfqpoint{-0.048611in}{0.000000in}}{\pgfqpoint{0.000000in}{0.000000in}}{%
\pgfpathmoveto{\pgfqpoint{0.000000in}{0.000000in}}%
\pgfpathlineto{\pgfqpoint{-0.048611in}{0.000000in}}%
\pgfusepath{stroke,fill}%
}%
\begin{pgfscope}%
\pgfsys@transformshift{0.800000in}{4.025825in}%
\pgfsys@useobject{currentmarker}{}%
\end{pgfscope}%
\end{pgfscope}%
\begin{pgfscope}%
\pgftext[x=0.525308in,y=3.977630in,left,base]{\rmfamily\fontsize{10.000000}{12.000000}\selectfont \(\displaystyle 3.5\)}%
\end{pgfscope}%
\begin{pgfscope}%
\pgfsetrectcap%
\pgfsetmiterjoin%
\pgfsetlinewidth{0.803000pt}%
\definecolor{currentstroke}{rgb}{0.000000,0.000000,0.000000}%
\pgfsetstrokecolor{currentstroke}%
\pgfsetdash{}{0pt}%
\pgfpathmoveto{\pgfqpoint{0.800000in}{0.528000in}}%
\pgfpathlineto{\pgfqpoint{0.800000in}{4.224000in}}%
\pgfusepath{stroke}%
\end{pgfscope}%
\begin{pgfscope}%
\pgfsetrectcap%
\pgfsetmiterjoin%
\pgfsetlinewidth{0.803000pt}%
\definecolor{currentstroke}{rgb}{0.000000,0.000000,0.000000}%
\pgfsetstrokecolor{currentstroke}%
\pgfsetdash{}{0pt}%
\pgfpathmoveto{\pgfqpoint{4.768000in}{0.528000in}}%
\pgfpathlineto{\pgfqpoint{4.768000in}{4.224000in}}%
\pgfusepath{stroke}%
\end{pgfscope}%
\begin{pgfscope}%
\pgfsetrectcap%
\pgfsetmiterjoin%
\pgfsetlinewidth{0.803000pt}%
\definecolor{currentstroke}{rgb}{0.000000,0.000000,0.000000}%
\pgfsetstrokecolor{currentstroke}%
\pgfsetdash{}{0pt}%
\pgfpathmoveto{\pgfqpoint{0.800000in}{0.528000in}}%
\pgfpathlineto{\pgfqpoint{4.768000in}{0.528000in}}%
\pgfusepath{stroke}%
\end{pgfscope}%
\begin{pgfscope}%
\pgfsetrectcap%
\pgfsetmiterjoin%
\pgfsetlinewidth{0.803000pt}%
\definecolor{currentstroke}{rgb}{0.000000,0.000000,0.000000}%
\pgfsetstrokecolor{currentstroke}%
\pgfsetdash{}{0pt}%
\pgfpathmoveto{\pgfqpoint{0.800000in}{4.224000in}}%
\pgfpathlineto{\pgfqpoint{4.768000in}{4.224000in}}%
\pgfusepath{stroke}%
\end{pgfscope}%
\begin{pgfscope}%
\pgfpathrectangle{\pgfqpoint{5.016000in}{0.528000in}}{\pgfqpoint{0.184800in}{3.696000in}} %
\pgfusepath{clip}%
\pgfsetbuttcap%
\pgfsetmiterjoin%
\definecolor{currentfill}{rgb}{1.000000,1.000000,1.000000}%
\pgfsetfillcolor{currentfill}%
\pgfsetlinewidth{0.010037pt}%
\definecolor{currentstroke}{rgb}{1.000000,1.000000,1.000000}%
\pgfsetstrokecolor{currentstroke}%
\pgfsetdash{}{0pt}%
\pgfpathmoveto{\pgfqpoint{5.016000in}{0.528000in}}%
\pgfpathlineto{\pgfqpoint{5.016000in}{0.542438in}}%
\pgfpathlineto{\pgfqpoint{5.016000in}{4.209562in}}%
\pgfpathlineto{\pgfqpoint{5.016000in}{4.224000in}}%
\pgfpathlineto{\pgfqpoint{5.200800in}{4.224000in}}%
\pgfpathlineto{\pgfqpoint{5.200800in}{4.209562in}}%
\pgfpathlineto{\pgfqpoint{5.200800in}{0.542438in}}%
\pgfpathlineto{\pgfqpoint{5.200800in}{0.528000in}}%
\pgfpathclose%
\pgfusepath{stroke,fill}%
\end{pgfscope}%
\begin{pgfscope}%
\pgfsys@transformshift{5.020000in}{0.530000in}%
\pgftext[left,bottom]{\pgfimage[interpolate=true,width=0.180000in,height=3.690000in]{Figure-0001-20180109-004132-022723-img0.png}}%
\end{pgfscope}%
\begin{pgfscope}%
\pgfsetbuttcap%
\pgfsetroundjoin%
\definecolor{currentfill}{rgb}{0.000000,0.000000,0.000000}%
\pgfsetfillcolor{currentfill}%
\pgfsetlinewidth{0.803000pt}%
\definecolor{currentstroke}{rgb}{0.000000,0.000000,0.000000}%
\pgfsetstrokecolor{currentstroke}%
\pgfsetdash{}{0pt}%
\pgfsys@defobject{currentmarker}{\pgfqpoint{0.000000in}{0.000000in}}{\pgfqpoint{0.048611in}{0.000000in}}{%
\pgfpathmoveto{\pgfqpoint{0.000000in}{0.000000in}}%
\pgfpathlineto{\pgfqpoint{0.048611in}{0.000000in}}%
\pgfusepath{stroke,fill}%
}%
\begin{pgfscope}%
\pgfsys@transformshift{5.200800in}{0.665263in}%
\pgfsys@useobject{currentmarker}{}%
\end{pgfscope}%
\end{pgfscope}%
\begin{pgfscope}%
\pgftext[x=5.298022in,y=0.617069in,left,base]{\rmfamily\fontsize{10.000000}{12.000000}\selectfont \(\displaystyle 0.01\)}%
\end{pgfscope}%
\begin{pgfscope}%
\pgfsetbuttcap%
\pgfsetroundjoin%
\definecolor{currentfill}{rgb}{0.000000,0.000000,0.000000}%
\pgfsetfillcolor{currentfill}%
\pgfsetlinewidth{0.803000pt}%
\definecolor{currentstroke}{rgb}{0.000000,0.000000,0.000000}%
\pgfsetstrokecolor{currentstroke}%
\pgfsetdash{}{0pt}%
\pgfsys@defobject{currentmarker}{\pgfqpoint{0.000000in}{0.000000in}}{\pgfqpoint{0.048611in}{0.000000in}}{%
\pgfpathmoveto{\pgfqpoint{0.000000in}{0.000000in}}%
\pgfpathlineto{\pgfqpoint{0.048611in}{0.000000in}}%
\pgfusepath{stroke,fill}%
}%
\begin{pgfscope}%
\pgfsys@transformshift{5.200800in}{1.377032in}%
\pgfsys@useobject{currentmarker}{}%
\end{pgfscope}%
\end{pgfscope}%
\begin{pgfscope}%
\pgftext[x=5.298022in,y=1.328838in,left,base]{\rmfamily\fontsize{10.000000}{12.000000}\selectfont \(\displaystyle 0.02\)}%
\end{pgfscope}%
\begin{pgfscope}%
\pgfsetbuttcap%
\pgfsetroundjoin%
\definecolor{currentfill}{rgb}{0.000000,0.000000,0.000000}%
\pgfsetfillcolor{currentfill}%
\pgfsetlinewidth{0.803000pt}%
\definecolor{currentstroke}{rgb}{0.000000,0.000000,0.000000}%
\pgfsetstrokecolor{currentstroke}%
\pgfsetdash{}{0pt}%
\pgfsys@defobject{currentmarker}{\pgfqpoint{0.000000in}{0.000000in}}{\pgfqpoint{0.048611in}{0.000000in}}{%
\pgfpathmoveto{\pgfqpoint{0.000000in}{0.000000in}}%
\pgfpathlineto{\pgfqpoint{0.048611in}{0.000000in}}%
\pgfusepath{stroke,fill}%
}%
\begin{pgfscope}%
\pgfsys@transformshift{5.200800in}{2.088801in}%
\pgfsys@useobject{currentmarker}{}%
\end{pgfscope}%
\end{pgfscope}%
\begin{pgfscope}%
\pgftext[x=5.298022in,y=2.040606in,left,base]{\rmfamily\fontsize{10.000000}{12.000000}\selectfont \(\displaystyle 0.03\)}%
\end{pgfscope}%
\begin{pgfscope}%
\pgfsetbuttcap%
\pgfsetroundjoin%
\definecolor{currentfill}{rgb}{0.000000,0.000000,0.000000}%
\pgfsetfillcolor{currentfill}%
\pgfsetlinewidth{0.803000pt}%
\definecolor{currentstroke}{rgb}{0.000000,0.000000,0.000000}%
\pgfsetstrokecolor{currentstroke}%
\pgfsetdash{}{0pt}%
\pgfsys@defobject{currentmarker}{\pgfqpoint{0.000000in}{0.000000in}}{\pgfqpoint{0.048611in}{0.000000in}}{%
\pgfpathmoveto{\pgfqpoint{0.000000in}{0.000000in}}%
\pgfpathlineto{\pgfqpoint{0.048611in}{0.000000in}}%
\pgfusepath{stroke,fill}%
}%
\begin{pgfscope}%
\pgfsys@transformshift{5.200800in}{2.800570in}%
\pgfsys@useobject{currentmarker}{}%
\end{pgfscope}%
\end{pgfscope}%
\begin{pgfscope}%
\pgftext[x=5.298022in,y=2.752375in,left,base]{\rmfamily\fontsize{10.000000}{12.000000}\selectfont \(\displaystyle 0.04\)}%
\end{pgfscope}%
\begin{pgfscope}%
\pgfsetbuttcap%
\pgfsetroundjoin%
\definecolor{currentfill}{rgb}{0.000000,0.000000,0.000000}%
\pgfsetfillcolor{currentfill}%
\pgfsetlinewidth{0.803000pt}%
\definecolor{currentstroke}{rgb}{0.000000,0.000000,0.000000}%
\pgfsetstrokecolor{currentstroke}%
\pgfsetdash{}{0pt}%
\pgfsys@defobject{currentmarker}{\pgfqpoint{0.000000in}{0.000000in}}{\pgfqpoint{0.048611in}{0.000000in}}{%
\pgfpathmoveto{\pgfqpoint{0.000000in}{0.000000in}}%
\pgfpathlineto{\pgfqpoint{0.048611in}{0.000000in}}%
\pgfusepath{stroke,fill}%
}%
\begin{pgfscope}%
\pgfsys@transformshift{5.200800in}{3.512338in}%
\pgfsys@useobject{currentmarker}{}%
\end{pgfscope}%
\end{pgfscope}%
\begin{pgfscope}%
\pgftext[x=5.298022in,y=3.464144in,left,base]{\rmfamily\fontsize{10.000000}{12.000000}\selectfont \(\displaystyle 0.05\)}%
\end{pgfscope}%
\begin{pgfscope}%
\pgfsetbuttcap%
\pgfsetmiterjoin%
\pgfsetlinewidth{0.803000pt}%
\definecolor{currentstroke}{rgb}{0.000000,0.000000,0.000000}%
\pgfsetstrokecolor{currentstroke}%
\pgfsetdash{}{0pt}%
\pgfpathmoveto{\pgfqpoint{5.016000in}{0.528000in}}%
\pgfpathlineto{\pgfqpoint{5.016000in}{0.542438in}}%
\pgfpathlineto{\pgfqpoint{5.016000in}{4.209562in}}%
\pgfpathlineto{\pgfqpoint{5.016000in}{4.224000in}}%
\pgfpathlineto{\pgfqpoint{5.200800in}{4.224000in}}%
\pgfpathlineto{\pgfqpoint{5.200800in}{4.209562in}}%
\pgfpathlineto{\pgfqpoint{5.200800in}{0.542438in}}%
\pgfpathlineto{\pgfqpoint{5.200800in}{0.528000in}}%
\pgfpathclose%
\pgfusepath{stroke}%
\end{pgfscope}%
\end{pgfpicture}%
\makeatother%
\endgroup%
}
\caption{An example of discrete optimal transport} \label{Fig:EgDisOT}
\end{figure}

\section{Calling solvers}

According to \textbf{Question 1}, we fist solve discrete optimal transport problems by directly calling solvers MOSEK and gurobi.

Note that simplex methods are a series of methods specialized for linear programs, and therefore simplex methods are generally faster and more precise than interior point methods. However, because of special structure of this problem (the number of constraints are always fewer than that of variables), the performance of simlex methods and the interior point methods are rather close. More information can be find in Section \ref{Sec:NumRes}.

\section{First-order methods} \label{Sec:FOM}

According to \textbf{Question 2}, we implemented several first-order methods, including an ADMM for the primal problem, and a fast operator splitting method with penalty functions. Furthermore, we propose an new algorithm by combing these two algorithms.

\subsection{ADMM for the primal problem}

We first implement an alternative direction method of multipliers (ADMM) according to a reformulation of \eqref{Eq:StdLP} as
\begin{equation} \label{Eq:ADMMPrimal}
\begin{array}{ll}
\mtx{minimize} & \sume{i}{1}{m}{\sume{j}{1}{n}{ c_{ i j } s_{ i j } }} + \iota_+ \rbr{\widetilde{s}}, \\
\mtx{subject to} & \sume{j}{1}{n}{s_{ i j }} = \mu_i, \crbr{ i = 1, 2, \cdots, m } \\
& \sume{i}{1}{n}{s_{ i j }} = \nu_j, \crbr{ j = 1, 2, \cdots n } \\
& s = \widetilde{s},
\end{array}
\end{equation}
where $\iota_+$ are indicator of $ \Rset^{ m \times n }_+ $. The augmented Lagragian is
\begin{equation}
\begin{aligned}
L_{\rho} \rbr{ s, \widetilde{s}, \lambda, \eta, e } &= \sume{i}{1}{n}{\sume{j}{1}{m}{ c_{ i j } s_{ i j } }} + \iota_+ \rbr{\widetilde{s}} \\
&+ \sume{i}{1}{n}{ \lambda_i \rbr{ \mu_i - \sume{j}{1}{m}{s_{ i j }} } } + \sume{j}{1}{m}{ \eta_j \rbr{ \nu_j - \sume{i}{1}{n}{s_{ i j }} } } + \sume{i}{1}{n}{\sume{j}{1}{m}{ e_{ i j } \rbr{ s_{ i j } - \widetilde{s}_{ i j } } }} \\
&+ \frac{\rho}{2} \sume{i}{1}{n}{\rbr{ \mu_i - \sume{j}{1}{m}{s_{ i j }} }^2} + \frac{\rho}{2} \sume{j}{1}{m}{\rbr{ \nu_j - \sume{i}{1}{n}{s_{ i j }} }^2} + \frac{\rho}{2} \sume{i}{1}{n}{\sume{j}{1}{m}{\rbr{ s_{ i j } - \widetilde{s}_{ i j } }^2}}. \\
\end{aligned}
\end{equation}
One of the minimization steps can be realized explicitly by
\begin{equation} \label{Eq:ProjStep}
\argmin_{\widetilde{s}} L_{\rho} \rbr{ s, \widetilde{s}, \lambda, \eta, e } = \rbr{ s - \frac{1}{\rho} e }_+
\end{equation}
where $\rbr{\cdot}_+$ is the projection to $ \Rset^{ m \times n }_+ $, and another can be derived from
\begin{gather} \label{Eq:LinSys}
\sume{k}{1}{n}{s_{ i k }} + \sume{k}{1}{m}{s_{ k j }} + s_{ i j } = \frac{1}{\rho} \rbr{ e_{ i j } + \lambda_i + \eta_j - c_{ i j } } + \mu_i + \nu_j + \widetilde{s}_{ i j } \equiv r_{ i j },
\end{gather}
which can be solved directly by its special structure. To be exact, the solution can be written explicitly as
\begin{equation}
s_{ i j } = r_{ i j } - \frac{1}{ n + 1 } \sume{k}{1}{n}{\rbr{ r_{ i k } - \frac{1}{ m + n + 1 } \sume{l}{1}{m}{r_{ l k }} }} - \frac{1}{ m + 1 } \sume{k}{1}{m}{\rbr{ r_{ k j } - \frac{1}{ m + n + 1 } \sume{l}{1}{n}{r_{ k l }} }}.
\end{equation}
The algorithm is shown in Algorithm \ref{Alg:ADMMPrimal}.

\begin{algorithm}
\caption{ADMM for the primal problem}
\label{Alg:ADMMPrimal}
\begin{algorithmic}
\REQUIRE $\mu$, $\nu$, $c$, step size $\rho$, scale factor $\alpha$
\STATE $ t \slar 0 $
\STATE $ s^{\rbr{t}}, \widetilde{s}^{\rbr{t}}, e^{\rbr{t}} \slar 0 $, $ \lambda^{\rbr{t}} \slar 0 $, $ \eta^{\rbr{t}} \slar 0 $
\WHILE{not converges}
\STATE $ s^{\rbr{ t + 1 }} \slar \argmin_s L_{\rho} \rbr{ s, \widetilde{s}^{\rbr{s}}, \lambda^{\rbr{s}}, \eta^{\rbr{s}}, e^{\rbr{s}} } $
\STATE $ \widetilde{s}^{\rbr{ t + 1 }} \slar \argmin_{\widetilde{s}} L_{\rho} \rbr{ s^{\rbr{ t + 1 }}, \widetilde{s}, \lambda^{\rbr{s}}, \eta^{\rbr{s}}, e^{\rbr{s}} } $
\STATE $ \lambda^{\rbr{ t + 1 }}_i \slar \lambda^{\rbr{t}}_i + \alpha \rho \rbr{ \mu_i - \sume{j}{1}{m}{s_{ i j }} } $
\STATE $ \eta^{\rbr{ t + 1 }}_j \slar \eta^{\rbr{t}}_j + \alpha \rho \rbr{ \nu_i - \sume{i}{1}{n}{s_{ i j }} } $
\STATE $ e^{\rbr{ t + 1 }} \slar e^{\rbr{t}} + \alpha \rho \rbr{ s - \widetilde{s} } $
\STATE $ t \slar t + 1 $
\ENDWHILE
\end{algorithmic}
\end{algorithm}

\subsection{Fast operator splitting method with penalty functions}

We then implement an accelerated operator splitting method using penalty functions. With penalty functions, the objective function can be written as $ f \rbr{s} + g \rbr{s} $, where
\begin{gather}
f \rbr{s} = \sume{i}{1}{n}{\sume{j}{1}{m}{ c_{ i j } s_{ i j } }} + \pi_1 \sume{i}{1}{n}{\rbr{ \mu_i - \sume{j}{1}{m}{s_{ i j }} }^2} + \pi_2 \sume{j}{1}{m}{\rbr{ \nu_j - \sume{i}{1}{n}{s_{ i j }} }^2} \\
g \rbr{s} = \pi_0 \sume{i}{1}{n}{\sume{j}{1}{m}{\rbr{s_{ i j }}_-}}
\end{gather}
and $\pi_1$, $\pi_2$ and $\pi_0$ are penalty factors.
We have found that $ \opprox_{ t g } $ can be achieved by shrinking the negative part, that is
\begin{equation}
\opprox_{ t g } \rbr{s} = \max \cbr{ s, 0 } + \min \cbr{ s + t \pi_0, 0 },
\end{equation}
and $ \opprox_{ t f } $ can be derived by  a linear system with special structure (similar to \eqref{Eq:LinSys}), which is
\begin{equation}
\begin{split}
\opprox_{ t f } \rbr{s}_{ i j } &= r_{ i j } \\
&- \frac{ \pi_1 t }{ \pi_1 t n + 1 } \sume{k}{1}{n}{\rbr{ r_{ i k } - \frac{ \pi_2 t }{ \pi_2 t m + \pi_1 t n + 1 } \sume{l}{1}{m}{r_{ l k }} }} \\
&- \frac{ \pi_2 t }{ \pi_2 t m + 1 } \sume{k}{1}{m}{\rbr{ r_{ k j } - \frac{ \pi_1 t }{ \pi_1 t m + \pi_2 t n + 1 } \sume{l}{1}{n}{r_{ k l }} }}
\end{split}
\end{equation}
to be exact. We then adopt Peaceman-Rachford splitting scheme to solve this problem. Furthermore, we use Nesterov momentum in this algorithm to boost its convergence, because part of the penalty terms are quadratic. This algorithm is listed as Algorithm \ref{Alg:GradPrimal}.

\begin{algorithm}
\caption{Operator splitting fast proximal gradient method using penalty functions}
\label{Alg:GradPrimal}
\begin{algorithmic}
\REQUIRE $\mu$, $\nu$, $c$, step size $l$, penalty factors $\pi_1$, $\pi_2$, $\pi_0$.
\STATE $ t \slar 0 $
\STATE $ s^{\rbr{-1}}, s^{\rbr{t}} \slar 0 $
\WHILE{not converges}
\STATE $ s' = s^{\rbr{t}} + \frac{ t - 1 }{ t + 2 } \rbr{ s^{\rbr{t}} - s^{\rbr{ t - 1 }} } $
\STATE $ s^{\rbr{ t + 1 }} \slar \opprox_{ l g } \rbr{ \opprox_{ l f } \rbr{s'} } $
\STATE $ t \slar t + 1 $
\ENDWHILE
\end{algorithmic}
\end{algorithm}

\subsection{Proposed combined algorithm}

According to some numerical experiments, we have found that Algorithm \ref{Alg:ADMMPrimal} suffers from difficult satisfication of constraints, while Algorithm \ref{Alg:GradPrimal} suffers from slow and hard convergence. Figure \ref{Fig:Loss} shows the evolution of error of $\mu$ and $\nu$, (see Section \ref{Sec:NumRes} for details) where the error fails in reaching $10^{-4}$ after about 10000 iterations in \SI{44}{\second}, while MOSEK solver only takes about \SI{1}{\second}. The deficiency of Peaceman-rachford splitting scheme may account for the hard convergence of Algorithm \ref{Alg:GradPrimal}.

\begin{figure}
\centering \scalebox{0.65}{%% Creator: Matplotlib, PGF backend
%%
%% To include the figure in your LaTeX document, write
%%   \input{<filename>.pgf}
%%
%% Make sure the required packages are loaded in your preamble
%%   \usepackage{pgf}
%%
%% Figures using additional raster images can only be included by \input if
%% they are in the same directory as the main LaTeX file. For loading figures
%% from other directories you can use the `import` package
%%   \usepackage{import}
%% and then include the figures with
%%   \import{<path to file>}{<filename>.pgf}
%%
%% Matplotlib used the following preamble
%%   \usepackage{fontspec}
%%
\begingroup%
\makeatletter%
\begin{pgfpicture}%
\pgfpathrectangle{\pgfpointorigin}{\pgfqpoint{6.400000in}{4.800000in}}%
\pgfusepath{use as bounding box, clip}%
\begin{pgfscope}%
\pgfsetbuttcap%
\pgfsetmiterjoin%
\definecolor{currentfill}{rgb}{1.000000,1.000000,1.000000}%
\pgfsetfillcolor{currentfill}%
\pgfsetlinewidth{0.000000pt}%
\definecolor{currentstroke}{rgb}{1.000000,1.000000,1.000000}%
\pgfsetstrokecolor{currentstroke}%
\pgfsetdash{}{0pt}%
\pgfpathmoveto{\pgfqpoint{0.000000in}{0.000000in}}%
\pgfpathlineto{\pgfqpoint{6.400000in}{0.000000in}}%
\pgfpathlineto{\pgfqpoint{6.400000in}{4.800000in}}%
\pgfpathlineto{\pgfqpoint{0.000000in}{4.800000in}}%
\pgfpathclose%
\pgfusepath{fill}%
\end{pgfscope}%
\begin{pgfscope}%
\pgfsetbuttcap%
\pgfsetmiterjoin%
\definecolor{currentfill}{rgb}{1.000000,1.000000,1.000000}%
\pgfsetfillcolor{currentfill}%
\pgfsetlinewidth{0.000000pt}%
\definecolor{currentstroke}{rgb}{0.000000,0.000000,0.000000}%
\pgfsetstrokecolor{currentstroke}%
\pgfsetstrokeopacity{0.000000}%
\pgfsetdash{}{0pt}%
\pgfpathmoveto{\pgfqpoint{0.800000in}{0.528000in}}%
\pgfpathlineto{\pgfqpoint{5.760000in}{0.528000in}}%
\pgfpathlineto{\pgfqpoint{5.760000in}{4.224000in}}%
\pgfpathlineto{\pgfqpoint{0.800000in}{4.224000in}}%
\pgfpathclose%
\pgfusepath{fill}%
\end{pgfscope}%
\begin{pgfscope}%
\pgfsetbuttcap%
\pgfsetroundjoin%
\definecolor{currentfill}{rgb}{0.000000,0.000000,0.000000}%
\pgfsetfillcolor{currentfill}%
\pgfsetlinewidth{0.803000pt}%
\definecolor{currentstroke}{rgb}{0.000000,0.000000,0.000000}%
\pgfsetstrokecolor{currentstroke}%
\pgfsetdash{}{0pt}%
\pgfsys@defobject{currentmarker}{\pgfqpoint{0.000000in}{-0.048611in}}{\pgfqpoint{0.000000in}{0.000000in}}{%
\pgfpathmoveto{\pgfqpoint{0.000000in}{0.000000in}}%
\pgfpathlineto{\pgfqpoint{0.000000in}{-0.048611in}}%
\pgfusepath{stroke,fill}%
}%
\begin{pgfscope}%
\pgfsys@transformshift{1.025455in}{0.528000in}%
\pgfsys@useobject{currentmarker}{}%
\end{pgfscope}%
\end{pgfscope}%
\begin{pgfscope}%
\pgftext[x=1.025455in,y=0.430778in,,top]{\rmfamily\fontsize{10.000000}{12.000000}\selectfont \(\displaystyle 0\)}%
\end{pgfscope}%
\begin{pgfscope}%
\pgfsetbuttcap%
\pgfsetroundjoin%
\definecolor{currentfill}{rgb}{0.000000,0.000000,0.000000}%
\pgfsetfillcolor{currentfill}%
\pgfsetlinewidth{0.803000pt}%
\definecolor{currentstroke}{rgb}{0.000000,0.000000,0.000000}%
\pgfsetstrokecolor{currentstroke}%
\pgfsetdash{}{0pt}%
\pgfsys@defobject{currentmarker}{\pgfqpoint{0.000000in}{-0.048611in}}{\pgfqpoint{0.000000in}{0.000000in}}{%
\pgfpathmoveto{\pgfqpoint{0.000000in}{0.000000in}}%
\pgfpathlineto{\pgfqpoint{0.000000in}{-0.048611in}}%
\pgfusepath{stroke,fill}%
}%
\begin{pgfscope}%
\pgfsys@transformshift{1.927363in}{0.528000in}%
\pgfsys@useobject{currentmarker}{}%
\end{pgfscope}%
\end{pgfscope}%
\begin{pgfscope}%
\pgftext[x=1.927363in,y=0.430778in,,top]{\rmfamily\fontsize{10.000000}{12.000000}\selectfont \(\displaystyle 2000\)}%
\end{pgfscope}%
\begin{pgfscope}%
\pgfsetbuttcap%
\pgfsetroundjoin%
\definecolor{currentfill}{rgb}{0.000000,0.000000,0.000000}%
\pgfsetfillcolor{currentfill}%
\pgfsetlinewidth{0.803000pt}%
\definecolor{currentstroke}{rgb}{0.000000,0.000000,0.000000}%
\pgfsetstrokecolor{currentstroke}%
\pgfsetdash{}{0pt}%
\pgfsys@defobject{currentmarker}{\pgfqpoint{0.000000in}{-0.048611in}}{\pgfqpoint{0.000000in}{0.000000in}}{%
\pgfpathmoveto{\pgfqpoint{0.000000in}{0.000000in}}%
\pgfpathlineto{\pgfqpoint{0.000000in}{-0.048611in}}%
\pgfusepath{stroke,fill}%
}%
\begin{pgfscope}%
\pgfsys@transformshift{2.829271in}{0.528000in}%
\pgfsys@useobject{currentmarker}{}%
\end{pgfscope}%
\end{pgfscope}%
\begin{pgfscope}%
\pgftext[x=2.829271in,y=0.430778in,,top]{\rmfamily\fontsize{10.000000}{12.000000}\selectfont \(\displaystyle 4000\)}%
\end{pgfscope}%
\begin{pgfscope}%
\pgfsetbuttcap%
\pgfsetroundjoin%
\definecolor{currentfill}{rgb}{0.000000,0.000000,0.000000}%
\pgfsetfillcolor{currentfill}%
\pgfsetlinewidth{0.803000pt}%
\definecolor{currentstroke}{rgb}{0.000000,0.000000,0.000000}%
\pgfsetstrokecolor{currentstroke}%
\pgfsetdash{}{0pt}%
\pgfsys@defobject{currentmarker}{\pgfqpoint{0.000000in}{-0.048611in}}{\pgfqpoint{0.000000in}{0.000000in}}{%
\pgfpathmoveto{\pgfqpoint{0.000000in}{0.000000in}}%
\pgfpathlineto{\pgfqpoint{0.000000in}{-0.048611in}}%
\pgfusepath{stroke,fill}%
}%
\begin{pgfscope}%
\pgfsys@transformshift{3.731180in}{0.528000in}%
\pgfsys@useobject{currentmarker}{}%
\end{pgfscope}%
\end{pgfscope}%
\begin{pgfscope}%
\pgftext[x=3.731180in,y=0.430778in,,top]{\rmfamily\fontsize{10.000000}{12.000000}\selectfont \(\displaystyle 6000\)}%
\end{pgfscope}%
\begin{pgfscope}%
\pgfsetbuttcap%
\pgfsetroundjoin%
\definecolor{currentfill}{rgb}{0.000000,0.000000,0.000000}%
\pgfsetfillcolor{currentfill}%
\pgfsetlinewidth{0.803000pt}%
\definecolor{currentstroke}{rgb}{0.000000,0.000000,0.000000}%
\pgfsetstrokecolor{currentstroke}%
\pgfsetdash{}{0pt}%
\pgfsys@defobject{currentmarker}{\pgfqpoint{0.000000in}{-0.048611in}}{\pgfqpoint{0.000000in}{0.000000in}}{%
\pgfpathmoveto{\pgfqpoint{0.000000in}{0.000000in}}%
\pgfpathlineto{\pgfqpoint{0.000000in}{-0.048611in}}%
\pgfusepath{stroke,fill}%
}%
\begin{pgfscope}%
\pgfsys@transformshift{4.633088in}{0.528000in}%
\pgfsys@useobject{currentmarker}{}%
\end{pgfscope}%
\end{pgfscope}%
\begin{pgfscope}%
\pgftext[x=4.633088in,y=0.430778in,,top]{\rmfamily\fontsize{10.000000}{12.000000}\selectfont \(\displaystyle 8000\)}%
\end{pgfscope}%
\begin{pgfscope}%
\pgfsetbuttcap%
\pgfsetroundjoin%
\definecolor{currentfill}{rgb}{0.000000,0.000000,0.000000}%
\pgfsetfillcolor{currentfill}%
\pgfsetlinewidth{0.803000pt}%
\definecolor{currentstroke}{rgb}{0.000000,0.000000,0.000000}%
\pgfsetstrokecolor{currentstroke}%
\pgfsetdash{}{0pt}%
\pgfsys@defobject{currentmarker}{\pgfqpoint{0.000000in}{-0.048611in}}{\pgfqpoint{0.000000in}{0.000000in}}{%
\pgfpathmoveto{\pgfqpoint{0.000000in}{0.000000in}}%
\pgfpathlineto{\pgfqpoint{0.000000in}{-0.048611in}}%
\pgfusepath{stroke,fill}%
}%
\begin{pgfscope}%
\pgfsys@transformshift{5.534996in}{0.528000in}%
\pgfsys@useobject{currentmarker}{}%
\end{pgfscope}%
\end{pgfscope}%
\begin{pgfscope}%
\pgftext[x=5.534996in,y=0.430778in,,top]{\rmfamily\fontsize{10.000000}{12.000000}\selectfont \(\displaystyle 10000\)}%
\end{pgfscope}%
\begin{pgfscope}%
\pgfsetbuttcap%
\pgfsetroundjoin%
\definecolor{currentfill}{rgb}{0.000000,0.000000,0.000000}%
\pgfsetfillcolor{currentfill}%
\pgfsetlinewidth{0.803000pt}%
\definecolor{currentstroke}{rgb}{0.000000,0.000000,0.000000}%
\pgfsetstrokecolor{currentstroke}%
\pgfsetdash{}{0pt}%
\pgfsys@defobject{currentmarker}{\pgfqpoint{-0.048611in}{0.000000in}}{\pgfqpoint{0.000000in}{0.000000in}}{%
\pgfpathmoveto{\pgfqpoint{0.000000in}{0.000000in}}%
\pgfpathlineto{\pgfqpoint{-0.048611in}{0.000000in}}%
\pgfusepath{stroke,fill}%
}%
\begin{pgfscope}%
\pgfsys@transformshift{0.800000in}{0.659676in}%
\pgfsys@useobject{currentmarker}{}%
\end{pgfscope}%
\end{pgfscope}%
\begin{pgfscope}%
\pgftext[x=0.414775in,y=0.611482in,left,base]{\rmfamily\fontsize{10.000000}{12.000000}\selectfont \(\displaystyle 10^{-4}\)}%
\end{pgfscope}%
\begin{pgfscope}%
\pgfsetbuttcap%
\pgfsetroundjoin%
\definecolor{currentfill}{rgb}{0.000000,0.000000,0.000000}%
\pgfsetfillcolor{currentfill}%
\pgfsetlinewidth{0.803000pt}%
\definecolor{currentstroke}{rgb}{0.000000,0.000000,0.000000}%
\pgfsetstrokecolor{currentstroke}%
\pgfsetdash{}{0pt}%
\pgfsys@defobject{currentmarker}{\pgfqpoint{-0.048611in}{0.000000in}}{\pgfqpoint{0.000000in}{0.000000in}}{%
\pgfpathmoveto{\pgfqpoint{0.000000in}{0.000000in}}%
\pgfpathlineto{\pgfqpoint{-0.048611in}{0.000000in}}%
\pgfusepath{stroke,fill}%
}%
\begin{pgfscope}%
\pgfsys@transformshift{0.800000in}{1.211618in}%
\pgfsys@useobject{currentmarker}{}%
\end{pgfscope}%
\end{pgfscope}%
\begin{pgfscope}%
\pgftext[x=0.414775in,y=1.163423in,left,base]{\rmfamily\fontsize{10.000000}{12.000000}\selectfont \(\displaystyle 10^{-3}\)}%
\end{pgfscope}%
\begin{pgfscope}%
\pgfsetbuttcap%
\pgfsetroundjoin%
\definecolor{currentfill}{rgb}{0.000000,0.000000,0.000000}%
\pgfsetfillcolor{currentfill}%
\pgfsetlinewidth{0.803000pt}%
\definecolor{currentstroke}{rgb}{0.000000,0.000000,0.000000}%
\pgfsetstrokecolor{currentstroke}%
\pgfsetdash{}{0pt}%
\pgfsys@defobject{currentmarker}{\pgfqpoint{-0.048611in}{0.000000in}}{\pgfqpoint{0.000000in}{0.000000in}}{%
\pgfpathmoveto{\pgfqpoint{0.000000in}{0.000000in}}%
\pgfpathlineto{\pgfqpoint{-0.048611in}{0.000000in}}%
\pgfusepath{stroke,fill}%
}%
\begin{pgfscope}%
\pgfsys@transformshift{0.800000in}{1.763560in}%
\pgfsys@useobject{currentmarker}{}%
\end{pgfscope}%
\end{pgfscope}%
\begin{pgfscope}%
\pgftext[x=0.414775in,y=1.715365in,left,base]{\rmfamily\fontsize{10.000000}{12.000000}\selectfont \(\displaystyle 10^{-2}\)}%
\end{pgfscope}%
\begin{pgfscope}%
\pgfsetbuttcap%
\pgfsetroundjoin%
\definecolor{currentfill}{rgb}{0.000000,0.000000,0.000000}%
\pgfsetfillcolor{currentfill}%
\pgfsetlinewidth{0.803000pt}%
\definecolor{currentstroke}{rgb}{0.000000,0.000000,0.000000}%
\pgfsetstrokecolor{currentstroke}%
\pgfsetdash{}{0pt}%
\pgfsys@defobject{currentmarker}{\pgfqpoint{-0.048611in}{0.000000in}}{\pgfqpoint{0.000000in}{0.000000in}}{%
\pgfpathmoveto{\pgfqpoint{0.000000in}{0.000000in}}%
\pgfpathlineto{\pgfqpoint{-0.048611in}{0.000000in}}%
\pgfusepath{stroke,fill}%
}%
\begin{pgfscope}%
\pgfsys@transformshift{0.800000in}{2.315501in}%
\pgfsys@useobject{currentmarker}{}%
\end{pgfscope}%
\end{pgfscope}%
\begin{pgfscope}%
\pgftext[x=0.414775in,y=2.267307in,left,base]{\rmfamily\fontsize{10.000000}{12.000000}\selectfont \(\displaystyle 10^{-1}\)}%
\end{pgfscope}%
\begin{pgfscope}%
\pgfsetbuttcap%
\pgfsetroundjoin%
\definecolor{currentfill}{rgb}{0.000000,0.000000,0.000000}%
\pgfsetfillcolor{currentfill}%
\pgfsetlinewidth{0.803000pt}%
\definecolor{currentstroke}{rgb}{0.000000,0.000000,0.000000}%
\pgfsetstrokecolor{currentstroke}%
\pgfsetdash{}{0pt}%
\pgfsys@defobject{currentmarker}{\pgfqpoint{-0.048611in}{0.000000in}}{\pgfqpoint{0.000000in}{0.000000in}}{%
\pgfpathmoveto{\pgfqpoint{0.000000in}{0.000000in}}%
\pgfpathlineto{\pgfqpoint{-0.048611in}{0.000000in}}%
\pgfusepath{stroke,fill}%
}%
\begin{pgfscope}%
\pgfsys@transformshift{0.800000in}{2.867443in}%
\pgfsys@useobject{currentmarker}{}%
\end{pgfscope}%
\end{pgfscope}%
\begin{pgfscope}%
\pgftext[x=0.501581in,y=2.819249in,left,base]{\rmfamily\fontsize{10.000000}{12.000000}\selectfont \(\displaystyle 10^{0}\)}%
\end{pgfscope}%
\begin{pgfscope}%
\pgfsetbuttcap%
\pgfsetroundjoin%
\definecolor{currentfill}{rgb}{0.000000,0.000000,0.000000}%
\pgfsetfillcolor{currentfill}%
\pgfsetlinewidth{0.803000pt}%
\definecolor{currentstroke}{rgb}{0.000000,0.000000,0.000000}%
\pgfsetstrokecolor{currentstroke}%
\pgfsetdash{}{0pt}%
\pgfsys@defobject{currentmarker}{\pgfqpoint{-0.048611in}{0.000000in}}{\pgfqpoint{0.000000in}{0.000000in}}{%
\pgfpathmoveto{\pgfqpoint{0.000000in}{0.000000in}}%
\pgfpathlineto{\pgfqpoint{-0.048611in}{0.000000in}}%
\pgfusepath{stroke,fill}%
}%
\begin{pgfscope}%
\pgfsys@transformshift{0.800000in}{3.419385in}%
\pgfsys@useobject{currentmarker}{}%
\end{pgfscope}%
\end{pgfscope}%
\begin{pgfscope}%
\pgftext[x=0.501581in,y=3.371190in,left,base]{\rmfamily\fontsize{10.000000}{12.000000}\selectfont \(\displaystyle 10^{1}\)}%
\end{pgfscope}%
\begin{pgfscope}%
\pgfsetbuttcap%
\pgfsetroundjoin%
\definecolor{currentfill}{rgb}{0.000000,0.000000,0.000000}%
\pgfsetfillcolor{currentfill}%
\pgfsetlinewidth{0.803000pt}%
\definecolor{currentstroke}{rgb}{0.000000,0.000000,0.000000}%
\pgfsetstrokecolor{currentstroke}%
\pgfsetdash{}{0pt}%
\pgfsys@defobject{currentmarker}{\pgfqpoint{-0.048611in}{0.000000in}}{\pgfqpoint{0.000000in}{0.000000in}}{%
\pgfpathmoveto{\pgfqpoint{0.000000in}{0.000000in}}%
\pgfpathlineto{\pgfqpoint{-0.048611in}{0.000000in}}%
\pgfusepath{stroke,fill}%
}%
\begin{pgfscope}%
\pgfsys@transformshift{0.800000in}{3.971327in}%
\pgfsys@useobject{currentmarker}{}%
\end{pgfscope}%
\end{pgfscope}%
\begin{pgfscope}%
\pgftext[x=0.501581in,y=3.923132in,left,base]{\rmfamily\fontsize{10.000000}{12.000000}\selectfont \(\displaystyle 10^{2}\)}%
\end{pgfscope}%
\begin{pgfscope}%
\pgfsetbuttcap%
\pgfsetroundjoin%
\definecolor{currentfill}{rgb}{0.000000,0.000000,0.000000}%
\pgfsetfillcolor{currentfill}%
\pgfsetlinewidth{0.602250pt}%
\definecolor{currentstroke}{rgb}{0.000000,0.000000,0.000000}%
\pgfsetstrokecolor{currentstroke}%
\pgfsetdash{}{0pt}%
\pgfsys@defobject{currentmarker}{\pgfqpoint{-0.027778in}{0.000000in}}{\pgfqpoint{0.000000in}{0.000000in}}{%
\pgfpathmoveto{\pgfqpoint{0.000000in}{0.000000in}}%
\pgfpathlineto{\pgfqpoint{-0.027778in}{0.000000in}}%
\pgfusepath{stroke,fill}%
}%
\begin{pgfscope}%
\pgfsys@transformshift{0.800000in}{0.537229in}%
\pgfsys@useobject{currentmarker}{}%
\end{pgfscope}%
\end{pgfscope}%
\begin{pgfscope}%
\pgfsetbuttcap%
\pgfsetroundjoin%
\definecolor{currentfill}{rgb}{0.000000,0.000000,0.000000}%
\pgfsetfillcolor{currentfill}%
\pgfsetlinewidth{0.602250pt}%
\definecolor{currentstroke}{rgb}{0.000000,0.000000,0.000000}%
\pgfsetstrokecolor{currentstroke}%
\pgfsetdash{}{0pt}%
\pgfsys@defobject{currentmarker}{\pgfqpoint{-0.027778in}{0.000000in}}{\pgfqpoint{0.000000in}{0.000000in}}{%
\pgfpathmoveto{\pgfqpoint{0.000000in}{0.000000in}}%
\pgfpathlineto{\pgfqpoint{-0.027778in}{0.000000in}}%
\pgfusepath{stroke,fill}%
}%
\begin{pgfscope}%
\pgfsys@transformshift{0.800000in}{0.574179in}%
\pgfsys@useobject{currentmarker}{}%
\end{pgfscope}%
\end{pgfscope}%
\begin{pgfscope}%
\pgfsetbuttcap%
\pgfsetroundjoin%
\definecolor{currentfill}{rgb}{0.000000,0.000000,0.000000}%
\pgfsetfillcolor{currentfill}%
\pgfsetlinewidth{0.602250pt}%
\definecolor{currentstroke}{rgb}{0.000000,0.000000,0.000000}%
\pgfsetstrokecolor{currentstroke}%
\pgfsetdash{}{0pt}%
\pgfsys@defobject{currentmarker}{\pgfqpoint{-0.027778in}{0.000000in}}{\pgfqpoint{0.000000in}{0.000000in}}{%
\pgfpathmoveto{\pgfqpoint{0.000000in}{0.000000in}}%
\pgfpathlineto{\pgfqpoint{-0.027778in}{0.000000in}}%
\pgfusepath{stroke,fill}%
}%
\begin{pgfscope}%
\pgfsys@transformshift{0.800000in}{0.606187in}%
\pgfsys@useobject{currentmarker}{}%
\end{pgfscope}%
\end{pgfscope}%
\begin{pgfscope}%
\pgfsetbuttcap%
\pgfsetroundjoin%
\definecolor{currentfill}{rgb}{0.000000,0.000000,0.000000}%
\pgfsetfillcolor{currentfill}%
\pgfsetlinewidth{0.602250pt}%
\definecolor{currentstroke}{rgb}{0.000000,0.000000,0.000000}%
\pgfsetstrokecolor{currentstroke}%
\pgfsetdash{}{0pt}%
\pgfsys@defobject{currentmarker}{\pgfqpoint{-0.027778in}{0.000000in}}{\pgfqpoint{0.000000in}{0.000000in}}{%
\pgfpathmoveto{\pgfqpoint{0.000000in}{0.000000in}}%
\pgfpathlineto{\pgfqpoint{-0.027778in}{0.000000in}}%
\pgfusepath{stroke,fill}%
}%
\begin{pgfscope}%
\pgfsys@transformshift{0.800000in}{0.634421in}%
\pgfsys@useobject{currentmarker}{}%
\end{pgfscope}%
\end{pgfscope}%
\begin{pgfscope}%
\pgfsetbuttcap%
\pgfsetroundjoin%
\definecolor{currentfill}{rgb}{0.000000,0.000000,0.000000}%
\pgfsetfillcolor{currentfill}%
\pgfsetlinewidth{0.602250pt}%
\definecolor{currentstroke}{rgb}{0.000000,0.000000,0.000000}%
\pgfsetstrokecolor{currentstroke}%
\pgfsetdash{}{0pt}%
\pgfsys@defobject{currentmarker}{\pgfqpoint{-0.027778in}{0.000000in}}{\pgfqpoint{0.000000in}{0.000000in}}{%
\pgfpathmoveto{\pgfqpoint{0.000000in}{0.000000in}}%
\pgfpathlineto{\pgfqpoint{-0.027778in}{0.000000in}}%
\pgfusepath{stroke,fill}%
}%
\begin{pgfscope}%
\pgfsys@transformshift{0.800000in}{0.825827in}%
\pgfsys@useobject{currentmarker}{}%
\end{pgfscope}%
\end{pgfscope}%
\begin{pgfscope}%
\pgfsetbuttcap%
\pgfsetroundjoin%
\definecolor{currentfill}{rgb}{0.000000,0.000000,0.000000}%
\pgfsetfillcolor{currentfill}%
\pgfsetlinewidth{0.602250pt}%
\definecolor{currentstroke}{rgb}{0.000000,0.000000,0.000000}%
\pgfsetstrokecolor{currentstroke}%
\pgfsetdash{}{0pt}%
\pgfsys@defobject{currentmarker}{\pgfqpoint{-0.027778in}{0.000000in}}{\pgfqpoint{0.000000in}{0.000000in}}{%
\pgfpathmoveto{\pgfqpoint{0.000000in}{0.000000in}}%
\pgfpathlineto{\pgfqpoint{-0.027778in}{0.000000in}}%
\pgfusepath{stroke,fill}%
}%
\begin{pgfscope}%
\pgfsys@transformshift{0.800000in}{0.923019in}%
\pgfsys@useobject{currentmarker}{}%
\end{pgfscope}%
\end{pgfscope}%
\begin{pgfscope}%
\pgfsetbuttcap%
\pgfsetroundjoin%
\definecolor{currentfill}{rgb}{0.000000,0.000000,0.000000}%
\pgfsetfillcolor{currentfill}%
\pgfsetlinewidth{0.602250pt}%
\definecolor{currentstroke}{rgb}{0.000000,0.000000,0.000000}%
\pgfsetstrokecolor{currentstroke}%
\pgfsetdash{}{0pt}%
\pgfsys@defobject{currentmarker}{\pgfqpoint{-0.027778in}{0.000000in}}{\pgfqpoint{0.000000in}{0.000000in}}{%
\pgfpathmoveto{\pgfqpoint{0.000000in}{0.000000in}}%
\pgfpathlineto{\pgfqpoint{-0.027778in}{0.000000in}}%
\pgfusepath{stroke,fill}%
}%
\begin{pgfscope}%
\pgfsys@transformshift{0.800000in}{0.991978in}%
\pgfsys@useobject{currentmarker}{}%
\end{pgfscope}%
\end{pgfscope}%
\begin{pgfscope}%
\pgfsetbuttcap%
\pgfsetroundjoin%
\definecolor{currentfill}{rgb}{0.000000,0.000000,0.000000}%
\pgfsetfillcolor{currentfill}%
\pgfsetlinewidth{0.602250pt}%
\definecolor{currentstroke}{rgb}{0.000000,0.000000,0.000000}%
\pgfsetstrokecolor{currentstroke}%
\pgfsetdash{}{0pt}%
\pgfsys@defobject{currentmarker}{\pgfqpoint{-0.027778in}{0.000000in}}{\pgfqpoint{0.000000in}{0.000000in}}{%
\pgfpathmoveto{\pgfqpoint{0.000000in}{0.000000in}}%
\pgfpathlineto{\pgfqpoint{-0.027778in}{0.000000in}}%
\pgfusepath{stroke,fill}%
}%
\begin{pgfscope}%
\pgfsys@transformshift{0.800000in}{1.045467in}%
\pgfsys@useobject{currentmarker}{}%
\end{pgfscope}%
\end{pgfscope}%
\begin{pgfscope}%
\pgfsetbuttcap%
\pgfsetroundjoin%
\definecolor{currentfill}{rgb}{0.000000,0.000000,0.000000}%
\pgfsetfillcolor{currentfill}%
\pgfsetlinewidth{0.602250pt}%
\definecolor{currentstroke}{rgb}{0.000000,0.000000,0.000000}%
\pgfsetstrokecolor{currentstroke}%
\pgfsetdash{}{0pt}%
\pgfsys@defobject{currentmarker}{\pgfqpoint{-0.027778in}{0.000000in}}{\pgfqpoint{0.000000in}{0.000000in}}{%
\pgfpathmoveto{\pgfqpoint{0.000000in}{0.000000in}}%
\pgfpathlineto{\pgfqpoint{-0.027778in}{0.000000in}}%
\pgfusepath{stroke,fill}%
}%
\begin{pgfscope}%
\pgfsys@transformshift{0.800000in}{1.089170in}%
\pgfsys@useobject{currentmarker}{}%
\end{pgfscope}%
\end{pgfscope}%
\begin{pgfscope}%
\pgfsetbuttcap%
\pgfsetroundjoin%
\definecolor{currentfill}{rgb}{0.000000,0.000000,0.000000}%
\pgfsetfillcolor{currentfill}%
\pgfsetlinewidth{0.602250pt}%
\definecolor{currentstroke}{rgb}{0.000000,0.000000,0.000000}%
\pgfsetstrokecolor{currentstroke}%
\pgfsetdash{}{0pt}%
\pgfsys@defobject{currentmarker}{\pgfqpoint{-0.027778in}{0.000000in}}{\pgfqpoint{0.000000in}{0.000000in}}{%
\pgfpathmoveto{\pgfqpoint{0.000000in}{0.000000in}}%
\pgfpathlineto{\pgfqpoint{-0.027778in}{0.000000in}}%
\pgfusepath{stroke,fill}%
}%
\begin{pgfscope}%
\pgfsys@transformshift{0.800000in}{1.126121in}%
\pgfsys@useobject{currentmarker}{}%
\end{pgfscope}%
\end{pgfscope}%
\begin{pgfscope}%
\pgfsetbuttcap%
\pgfsetroundjoin%
\definecolor{currentfill}{rgb}{0.000000,0.000000,0.000000}%
\pgfsetfillcolor{currentfill}%
\pgfsetlinewidth{0.602250pt}%
\definecolor{currentstroke}{rgb}{0.000000,0.000000,0.000000}%
\pgfsetstrokecolor{currentstroke}%
\pgfsetdash{}{0pt}%
\pgfsys@defobject{currentmarker}{\pgfqpoint{-0.027778in}{0.000000in}}{\pgfqpoint{0.000000in}{0.000000in}}{%
\pgfpathmoveto{\pgfqpoint{0.000000in}{0.000000in}}%
\pgfpathlineto{\pgfqpoint{-0.027778in}{0.000000in}}%
\pgfusepath{stroke,fill}%
}%
\begin{pgfscope}%
\pgfsys@transformshift{0.800000in}{1.158129in}%
\pgfsys@useobject{currentmarker}{}%
\end{pgfscope}%
\end{pgfscope}%
\begin{pgfscope}%
\pgfsetbuttcap%
\pgfsetroundjoin%
\definecolor{currentfill}{rgb}{0.000000,0.000000,0.000000}%
\pgfsetfillcolor{currentfill}%
\pgfsetlinewidth{0.602250pt}%
\definecolor{currentstroke}{rgb}{0.000000,0.000000,0.000000}%
\pgfsetstrokecolor{currentstroke}%
\pgfsetdash{}{0pt}%
\pgfsys@defobject{currentmarker}{\pgfqpoint{-0.027778in}{0.000000in}}{\pgfqpoint{0.000000in}{0.000000in}}{%
\pgfpathmoveto{\pgfqpoint{0.000000in}{0.000000in}}%
\pgfpathlineto{\pgfqpoint{-0.027778in}{0.000000in}}%
\pgfusepath{stroke,fill}%
}%
\begin{pgfscope}%
\pgfsys@transformshift{0.800000in}{1.186362in}%
\pgfsys@useobject{currentmarker}{}%
\end{pgfscope}%
\end{pgfscope}%
\begin{pgfscope}%
\pgfsetbuttcap%
\pgfsetroundjoin%
\definecolor{currentfill}{rgb}{0.000000,0.000000,0.000000}%
\pgfsetfillcolor{currentfill}%
\pgfsetlinewidth{0.602250pt}%
\definecolor{currentstroke}{rgb}{0.000000,0.000000,0.000000}%
\pgfsetstrokecolor{currentstroke}%
\pgfsetdash{}{0pt}%
\pgfsys@defobject{currentmarker}{\pgfqpoint{-0.027778in}{0.000000in}}{\pgfqpoint{0.000000in}{0.000000in}}{%
\pgfpathmoveto{\pgfqpoint{0.000000in}{0.000000in}}%
\pgfpathlineto{\pgfqpoint{-0.027778in}{0.000000in}}%
\pgfusepath{stroke,fill}%
}%
\begin{pgfscope}%
\pgfsys@transformshift{0.800000in}{1.377769in}%
\pgfsys@useobject{currentmarker}{}%
\end{pgfscope}%
\end{pgfscope}%
\begin{pgfscope}%
\pgfsetbuttcap%
\pgfsetroundjoin%
\definecolor{currentfill}{rgb}{0.000000,0.000000,0.000000}%
\pgfsetfillcolor{currentfill}%
\pgfsetlinewidth{0.602250pt}%
\definecolor{currentstroke}{rgb}{0.000000,0.000000,0.000000}%
\pgfsetstrokecolor{currentstroke}%
\pgfsetdash{}{0pt}%
\pgfsys@defobject{currentmarker}{\pgfqpoint{-0.027778in}{0.000000in}}{\pgfqpoint{0.000000in}{0.000000in}}{%
\pgfpathmoveto{\pgfqpoint{0.000000in}{0.000000in}}%
\pgfpathlineto{\pgfqpoint{-0.027778in}{0.000000in}}%
\pgfusepath{stroke,fill}%
}%
\begin{pgfscope}%
\pgfsys@transformshift{0.800000in}{1.474961in}%
\pgfsys@useobject{currentmarker}{}%
\end{pgfscope}%
\end{pgfscope}%
\begin{pgfscope}%
\pgfsetbuttcap%
\pgfsetroundjoin%
\definecolor{currentfill}{rgb}{0.000000,0.000000,0.000000}%
\pgfsetfillcolor{currentfill}%
\pgfsetlinewidth{0.602250pt}%
\definecolor{currentstroke}{rgb}{0.000000,0.000000,0.000000}%
\pgfsetstrokecolor{currentstroke}%
\pgfsetdash{}{0pt}%
\pgfsys@defobject{currentmarker}{\pgfqpoint{-0.027778in}{0.000000in}}{\pgfqpoint{0.000000in}{0.000000in}}{%
\pgfpathmoveto{\pgfqpoint{0.000000in}{0.000000in}}%
\pgfpathlineto{\pgfqpoint{-0.027778in}{0.000000in}}%
\pgfusepath{stroke,fill}%
}%
\begin{pgfscope}%
\pgfsys@transformshift{0.800000in}{1.543920in}%
\pgfsys@useobject{currentmarker}{}%
\end{pgfscope}%
\end{pgfscope}%
\begin{pgfscope}%
\pgfsetbuttcap%
\pgfsetroundjoin%
\definecolor{currentfill}{rgb}{0.000000,0.000000,0.000000}%
\pgfsetfillcolor{currentfill}%
\pgfsetlinewidth{0.602250pt}%
\definecolor{currentstroke}{rgb}{0.000000,0.000000,0.000000}%
\pgfsetstrokecolor{currentstroke}%
\pgfsetdash{}{0pt}%
\pgfsys@defobject{currentmarker}{\pgfqpoint{-0.027778in}{0.000000in}}{\pgfqpoint{0.000000in}{0.000000in}}{%
\pgfpathmoveto{\pgfqpoint{0.000000in}{0.000000in}}%
\pgfpathlineto{\pgfqpoint{-0.027778in}{0.000000in}}%
\pgfusepath{stroke,fill}%
}%
\begin{pgfscope}%
\pgfsys@transformshift{0.800000in}{1.597409in}%
\pgfsys@useobject{currentmarker}{}%
\end{pgfscope}%
\end{pgfscope}%
\begin{pgfscope}%
\pgfsetbuttcap%
\pgfsetroundjoin%
\definecolor{currentfill}{rgb}{0.000000,0.000000,0.000000}%
\pgfsetfillcolor{currentfill}%
\pgfsetlinewidth{0.602250pt}%
\definecolor{currentstroke}{rgb}{0.000000,0.000000,0.000000}%
\pgfsetstrokecolor{currentstroke}%
\pgfsetdash{}{0pt}%
\pgfsys@defobject{currentmarker}{\pgfqpoint{-0.027778in}{0.000000in}}{\pgfqpoint{0.000000in}{0.000000in}}{%
\pgfpathmoveto{\pgfqpoint{0.000000in}{0.000000in}}%
\pgfpathlineto{\pgfqpoint{-0.027778in}{0.000000in}}%
\pgfusepath{stroke,fill}%
}%
\begin{pgfscope}%
\pgfsys@transformshift{0.800000in}{1.641112in}%
\pgfsys@useobject{currentmarker}{}%
\end{pgfscope}%
\end{pgfscope}%
\begin{pgfscope}%
\pgfsetbuttcap%
\pgfsetroundjoin%
\definecolor{currentfill}{rgb}{0.000000,0.000000,0.000000}%
\pgfsetfillcolor{currentfill}%
\pgfsetlinewidth{0.602250pt}%
\definecolor{currentstroke}{rgb}{0.000000,0.000000,0.000000}%
\pgfsetstrokecolor{currentstroke}%
\pgfsetdash{}{0pt}%
\pgfsys@defobject{currentmarker}{\pgfqpoint{-0.027778in}{0.000000in}}{\pgfqpoint{0.000000in}{0.000000in}}{%
\pgfpathmoveto{\pgfqpoint{0.000000in}{0.000000in}}%
\pgfpathlineto{\pgfqpoint{-0.027778in}{0.000000in}}%
\pgfusepath{stroke,fill}%
}%
\begin{pgfscope}%
\pgfsys@transformshift{0.800000in}{1.678063in}%
\pgfsys@useobject{currentmarker}{}%
\end{pgfscope}%
\end{pgfscope}%
\begin{pgfscope}%
\pgfsetbuttcap%
\pgfsetroundjoin%
\definecolor{currentfill}{rgb}{0.000000,0.000000,0.000000}%
\pgfsetfillcolor{currentfill}%
\pgfsetlinewidth{0.602250pt}%
\definecolor{currentstroke}{rgb}{0.000000,0.000000,0.000000}%
\pgfsetstrokecolor{currentstroke}%
\pgfsetdash{}{0pt}%
\pgfsys@defobject{currentmarker}{\pgfqpoint{-0.027778in}{0.000000in}}{\pgfqpoint{0.000000in}{0.000000in}}{%
\pgfpathmoveto{\pgfqpoint{0.000000in}{0.000000in}}%
\pgfpathlineto{\pgfqpoint{-0.027778in}{0.000000in}}%
\pgfusepath{stroke,fill}%
}%
\begin{pgfscope}%
\pgfsys@transformshift{0.800000in}{1.710071in}%
\pgfsys@useobject{currentmarker}{}%
\end{pgfscope}%
\end{pgfscope}%
\begin{pgfscope}%
\pgfsetbuttcap%
\pgfsetroundjoin%
\definecolor{currentfill}{rgb}{0.000000,0.000000,0.000000}%
\pgfsetfillcolor{currentfill}%
\pgfsetlinewidth{0.602250pt}%
\definecolor{currentstroke}{rgb}{0.000000,0.000000,0.000000}%
\pgfsetstrokecolor{currentstroke}%
\pgfsetdash{}{0pt}%
\pgfsys@defobject{currentmarker}{\pgfqpoint{-0.027778in}{0.000000in}}{\pgfqpoint{0.000000in}{0.000000in}}{%
\pgfpathmoveto{\pgfqpoint{0.000000in}{0.000000in}}%
\pgfpathlineto{\pgfqpoint{-0.027778in}{0.000000in}}%
\pgfusepath{stroke,fill}%
}%
\begin{pgfscope}%
\pgfsys@transformshift{0.800000in}{1.738304in}%
\pgfsys@useobject{currentmarker}{}%
\end{pgfscope}%
\end{pgfscope}%
\begin{pgfscope}%
\pgfsetbuttcap%
\pgfsetroundjoin%
\definecolor{currentfill}{rgb}{0.000000,0.000000,0.000000}%
\pgfsetfillcolor{currentfill}%
\pgfsetlinewidth{0.602250pt}%
\definecolor{currentstroke}{rgb}{0.000000,0.000000,0.000000}%
\pgfsetstrokecolor{currentstroke}%
\pgfsetdash{}{0pt}%
\pgfsys@defobject{currentmarker}{\pgfqpoint{-0.027778in}{0.000000in}}{\pgfqpoint{0.000000in}{0.000000in}}{%
\pgfpathmoveto{\pgfqpoint{0.000000in}{0.000000in}}%
\pgfpathlineto{\pgfqpoint{-0.027778in}{0.000000in}}%
\pgfusepath{stroke,fill}%
}%
\begin{pgfscope}%
\pgfsys@transformshift{0.800000in}{1.929711in}%
\pgfsys@useobject{currentmarker}{}%
\end{pgfscope}%
\end{pgfscope}%
\begin{pgfscope}%
\pgfsetbuttcap%
\pgfsetroundjoin%
\definecolor{currentfill}{rgb}{0.000000,0.000000,0.000000}%
\pgfsetfillcolor{currentfill}%
\pgfsetlinewidth{0.602250pt}%
\definecolor{currentstroke}{rgb}{0.000000,0.000000,0.000000}%
\pgfsetstrokecolor{currentstroke}%
\pgfsetdash{}{0pt}%
\pgfsys@defobject{currentmarker}{\pgfqpoint{-0.027778in}{0.000000in}}{\pgfqpoint{0.000000in}{0.000000in}}{%
\pgfpathmoveto{\pgfqpoint{0.000000in}{0.000000in}}%
\pgfpathlineto{\pgfqpoint{-0.027778in}{0.000000in}}%
\pgfusepath{stroke,fill}%
}%
\begin{pgfscope}%
\pgfsys@transformshift{0.800000in}{2.026903in}%
\pgfsys@useobject{currentmarker}{}%
\end{pgfscope}%
\end{pgfscope}%
\begin{pgfscope}%
\pgfsetbuttcap%
\pgfsetroundjoin%
\definecolor{currentfill}{rgb}{0.000000,0.000000,0.000000}%
\pgfsetfillcolor{currentfill}%
\pgfsetlinewidth{0.602250pt}%
\definecolor{currentstroke}{rgb}{0.000000,0.000000,0.000000}%
\pgfsetstrokecolor{currentstroke}%
\pgfsetdash{}{0pt}%
\pgfsys@defobject{currentmarker}{\pgfqpoint{-0.027778in}{0.000000in}}{\pgfqpoint{0.000000in}{0.000000in}}{%
\pgfpathmoveto{\pgfqpoint{0.000000in}{0.000000in}}%
\pgfpathlineto{\pgfqpoint{-0.027778in}{0.000000in}}%
\pgfusepath{stroke,fill}%
}%
\begin{pgfscope}%
\pgfsys@transformshift{0.800000in}{2.095862in}%
\pgfsys@useobject{currentmarker}{}%
\end{pgfscope}%
\end{pgfscope}%
\begin{pgfscope}%
\pgfsetbuttcap%
\pgfsetroundjoin%
\definecolor{currentfill}{rgb}{0.000000,0.000000,0.000000}%
\pgfsetfillcolor{currentfill}%
\pgfsetlinewidth{0.602250pt}%
\definecolor{currentstroke}{rgb}{0.000000,0.000000,0.000000}%
\pgfsetstrokecolor{currentstroke}%
\pgfsetdash{}{0pt}%
\pgfsys@defobject{currentmarker}{\pgfqpoint{-0.027778in}{0.000000in}}{\pgfqpoint{0.000000in}{0.000000in}}{%
\pgfpathmoveto{\pgfqpoint{0.000000in}{0.000000in}}%
\pgfpathlineto{\pgfqpoint{-0.027778in}{0.000000in}}%
\pgfusepath{stroke,fill}%
}%
\begin{pgfscope}%
\pgfsys@transformshift{0.800000in}{2.149350in}%
\pgfsys@useobject{currentmarker}{}%
\end{pgfscope}%
\end{pgfscope}%
\begin{pgfscope}%
\pgfsetbuttcap%
\pgfsetroundjoin%
\definecolor{currentfill}{rgb}{0.000000,0.000000,0.000000}%
\pgfsetfillcolor{currentfill}%
\pgfsetlinewidth{0.602250pt}%
\definecolor{currentstroke}{rgb}{0.000000,0.000000,0.000000}%
\pgfsetstrokecolor{currentstroke}%
\pgfsetdash{}{0pt}%
\pgfsys@defobject{currentmarker}{\pgfqpoint{-0.027778in}{0.000000in}}{\pgfqpoint{0.000000in}{0.000000in}}{%
\pgfpathmoveto{\pgfqpoint{0.000000in}{0.000000in}}%
\pgfpathlineto{\pgfqpoint{-0.027778in}{0.000000in}}%
\pgfusepath{stroke,fill}%
}%
\begin{pgfscope}%
\pgfsys@transformshift{0.800000in}{2.193054in}%
\pgfsys@useobject{currentmarker}{}%
\end{pgfscope}%
\end{pgfscope}%
\begin{pgfscope}%
\pgfsetbuttcap%
\pgfsetroundjoin%
\definecolor{currentfill}{rgb}{0.000000,0.000000,0.000000}%
\pgfsetfillcolor{currentfill}%
\pgfsetlinewidth{0.602250pt}%
\definecolor{currentstroke}{rgb}{0.000000,0.000000,0.000000}%
\pgfsetstrokecolor{currentstroke}%
\pgfsetdash{}{0pt}%
\pgfsys@defobject{currentmarker}{\pgfqpoint{-0.027778in}{0.000000in}}{\pgfqpoint{0.000000in}{0.000000in}}{%
\pgfpathmoveto{\pgfqpoint{0.000000in}{0.000000in}}%
\pgfpathlineto{\pgfqpoint{-0.027778in}{0.000000in}}%
\pgfusepath{stroke,fill}%
}%
\begin{pgfscope}%
\pgfsys@transformshift{0.800000in}{2.230004in}%
\pgfsys@useobject{currentmarker}{}%
\end{pgfscope}%
\end{pgfscope}%
\begin{pgfscope}%
\pgfsetbuttcap%
\pgfsetroundjoin%
\definecolor{currentfill}{rgb}{0.000000,0.000000,0.000000}%
\pgfsetfillcolor{currentfill}%
\pgfsetlinewidth{0.602250pt}%
\definecolor{currentstroke}{rgb}{0.000000,0.000000,0.000000}%
\pgfsetstrokecolor{currentstroke}%
\pgfsetdash{}{0pt}%
\pgfsys@defobject{currentmarker}{\pgfqpoint{-0.027778in}{0.000000in}}{\pgfqpoint{0.000000in}{0.000000in}}{%
\pgfpathmoveto{\pgfqpoint{0.000000in}{0.000000in}}%
\pgfpathlineto{\pgfqpoint{-0.027778in}{0.000000in}}%
\pgfusepath{stroke,fill}%
}%
\begin{pgfscope}%
\pgfsys@transformshift{0.800000in}{2.262013in}%
\pgfsys@useobject{currentmarker}{}%
\end{pgfscope}%
\end{pgfscope}%
\begin{pgfscope}%
\pgfsetbuttcap%
\pgfsetroundjoin%
\definecolor{currentfill}{rgb}{0.000000,0.000000,0.000000}%
\pgfsetfillcolor{currentfill}%
\pgfsetlinewidth{0.602250pt}%
\definecolor{currentstroke}{rgb}{0.000000,0.000000,0.000000}%
\pgfsetstrokecolor{currentstroke}%
\pgfsetdash{}{0pt}%
\pgfsys@defobject{currentmarker}{\pgfqpoint{-0.027778in}{0.000000in}}{\pgfqpoint{0.000000in}{0.000000in}}{%
\pgfpathmoveto{\pgfqpoint{0.000000in}{0.000000in}}%
\pgfpathlineto{\pgfqpoint{-0.027778in}{0.000000in}}%
\pgfusepath{stroke,fill}%
}%
\begin{pgfscope}%
\pgfsys@transformshift{0.800000in}{2.290246in}%
\pgfsys@useobject{currentmarker}{}%
\end{pgfscope}%
\end{pgfscope}%
\begin{pgfscope}%
\pgfsetbuttcap%
\pgfsetroundjoin%
\definecolor{currentfill}{rgb}{0.000000,0.000000,0.000000}%
\pgfsetfillcolor{currentfill}%
\pgfsetlinewidth{0.602250pt}%
\definecolor{currentstroke}{rgb}{0.000000,0.000000,0.000000}%
\pgfsetstrokecolor{currentstroke}%
\pgfsetdash{}{0pt}%
\pgfsys@defobject{currentmarker}{\pgfqpoint{-0.027778in}{0.000000in}}{\pgfqpoint{0.000000in}{0.000000in}}{%
\pgfpathmoveto{\pgfqpoint{0.000000in}{0.000000in}}%
\pgfpathlineto{\pgfqpoint{-0.027778in}{0.000000in}}%
\pgfusepath{stroke,fill}%
}%
\begin{pgfscope}%
\pgfsys@transformshift{0.800000in}{2.481652in}%
\pgfsys@useobject{currentmarker}{}%
\end{pgfscope}%
\end{pgfscope}%
\begin{pgfscope}%
\pgfsetbuttcap%
\pgfsetroundjoin%
\definecolor{currentfill}{rgb}{0.000000,0.000000,0.000000}%
\pgfsetfillcolor{currentfill}%
\pgfsetlinewidth{0.602250pt}%
\definecolor{currentstroke}{rgb}{0.000000,0.000000,0.000000}%
\pgfsetstrokecolor{currentstroke}%
\pgfsetdash{}{0pt}%
\pgfsys@defobject{currentmarker}{\pgfqpoint{-0.027778in}{0.000000in}}{\pgfqpoint{0.000000in}{0.000000in}}{%
\pgfpathmoveto{\pgfqpoint{0.000000in}{0.000000in}}%
\pgfpathlineto{\pgfqpoint{-0.027778in}{0.000000in}}%
\pgfusepath{stroke,fill}%
}%
\begin{pgfscope}%
\pgfsys@transformshift{0.800000in}{2.578844in}%
\pgfsys@useobject{currentmarker}{}%
\end{pgfscope}%
\end{pgfscope}%
\begin{pgfscope}%
\pgfsetbuttcap%
\pgfsetroundjoin%
\definecolor{currentfill}{rgb}{0.000000,0.000000,0.000000}%
\pgfsetfillcolor{currentfill}%
\pgfsetlinewidth{0.602250pt}%
\definecolor{currentstroke}{rgb}{0.000000,0.000000,0.000000}%
\pgfsetstrokecolor{currentstroke}%
\pgfsetdash{}{0pt}%
\pgfsys@defobject{currentmarker}{\pgfqpoint{-0.027778in}{0.000000in}}{\pgfqpoint{0.000000in}{0.000000in}}{%
\pgfpathmoveto{\pgfqpoint{0.000000in}{0.000000in}}%
\pgfpathlineto{\pgfqpoint{-0.027778in}{0.000000in}}%
\pgfusepath{stroke,fill}%
}%
\begin{pgfscope}%
\pgfsys@transformshift{0.800000in}{2.647803in}%
\pgfsys@useobject{currentmarker}{}%
\end{pgfscope}%
\end{pgfscope}%
\begin{pgfscope}%
\pgfsetbuttcap%
\pgfsetroundjoin%
\definecolor{currentfill}{rgb}{0.000000,0.000000,0.000000}%
\pgfsetfillcolor{currentfill}%
\pgfsetlinewidth{0.602250pt}%
\definecolor{currentstroke}{rgb}{0.000000,0.000000,0.000000}%
\pgfsetstrokecolor{currentstroke}%
\pgfsetdash{}{0pt}%
\pgfsys@defobject{currentmarker}{\pgfqpoint{-0.027778in}{0.000000in}}{\pgfqpoint{0.000000in}{0.000000in}}{%
\pgfpathmoveto{\pgfqpoint{0.000000in}{0.000000in}}%
\pgfpathlineto{\pgfqpoint{-0.027778in}{0.000000in}}%
\pgfusepath{stroke,fill}%
}%
\begin{pgfscope}%
\pgfsys@transformshift{0.800000in}{2.701292in}%
\pgfsys@useobject{currentmarker}{}%
\end{pgfscope}%
\end{pgfscope}%
\begin{pgfscope}%
\pgfsetbuttcap%
\pgfsetroundjoin%
\definecolor{currentfill}{rgb}{0.000000,0.000000,0.000000}%
\pgfsetfillcolor{currentfill}%
\pgfsetlinewidth{0.602250pt}%
\definecolor{currentstroke}{rgb}{0.000000,0.000000,0.000000}%
\pgfsetstrokecolor{currentstroke}%
\pgfsetdash{}{0pt}%
\pgfsys@defobject{currentmarker}{\pgfqpoint{-0.027778in}{0.000000in}}{\pgfqpoint{0.000000in}{0.000000in}}{%
\pgfpathmoveto{\pgfqpoint{0.000000in}{0.000000in}}%
\pgfpathlineto{\pgfqpoint{-0.027778in}{0.000000in}}%
\pgfusepath{stroke,fill}%
}%
\begin{pgfscope}%
\pgfsys@transformshift{0.800000in}{2.744995in}%
\pgfsys@useobject{currentmarker}{}%
\end{pgfscope}%
\end{pgfscope}%
\begin{pgfscope}%
\pgfsetbuttcap%
\pgfsetroundjoin%
\definecolor{currentfill}{rgb}{0.000000,0.000000,0.000000}%
\pgfsetfillcolor{currentfill}%
\pgfsetlinewidth{0.602250pt}%
\definecolor{currentstroke}{rgb}{0.000000,0.000000,0.000000}%
\pgfsetstrokecolor{currentstroke}%
\pgfsetdash{}{0pt}%
\pgfsys@defobject{currentmarker}{\pgfqpoint{-0.027778in}{0.000000in}}{\pgfqpoint{0.000000in}{0.000000in}}{%
\pgfpathmoveto{\pgfqpoint{0.000000in}{0.000000in}}%
\pgfpathlineto{\pgfqpoint{-0.027778in}{0.000000in}}%
\pgfusepath{stroke,fill}%
}%
\begin{pgfscope}%
\pgfsys@transformshift{0.800000in}{2.781946in}%
\pgfsys@useobject{currentmarker}{}%
\end{pgfscope}%
\end{pgfscope}%
\begin{pgfscope}%
\pgfsetbuttcap%
\pgfsetroundjoin%
\definecolor{currentfill}{rgb}{0.000000,0.000000,0.000000}%
\pgfsetfillcolor{currentfill}%
\pgfsetlinewidth{0.602250pt}%
\definecolor{currentstroke}{rgb}{0.000000,0.000000,0.000000}%
\pgfsetstrokecolor{currentstroke}%
\pgfsetdash{}{0pt}%
\pgfsys@defobject{currentmarker}{\pgfqpoint{-0.027778in}{0.000000in}}{\pgfqpoint{0.000000in}{0.000000in}}{%
\pgfpathmoveto{\pgfqpoint{0.000000in}{0.000000in}}%
\pgfpathlineto{\pgfqpoint{-0.027778in}{0.000000in}}%
\pgfusepath{stroke,fill}%
}%
\begin{pgfscope}%
\pgfsys@transformshift{0.800000in}{2.813954in}%
\pgfsys@useobject{currentmarker}{}%
\end{pgfscope}%
\end{pgfscope}%
\begin{pgfscope}%
\pgfsetbuttcap%
\pgfsetroundjoin%
\definecolor{currentfill}{rgb}{0.000000,0.000000,0.000000}%
\pgfsetfillcolor{currentfill}%
\pgfsetlinewidth{0.602250pt}%
\definecolor{currentstroke}{rgb}{0.000000,0.000000,0.000000}%
\pgfsetstrokecolor{currentstroke}%
\pgfsetdash{}{0pt}%
\pgfsys@defobject{currentmarker}{\pgfqpoint{-0.027778in}{0.000000in}}{\pgfqpoint{0.000000in}{0.000000in}}{%
\pgfpathmoveto{\pgfqpoint{0.000000in}{0.000000in}}%
\pgfpathlineto{\pgfqpoint{-0.027778in}{0.000000in}}%
\pgfusepath{stroke,fill}%
}%
\begin{pgfscope}%
\pgfsys@transformshift{0.800000in}{2.842188in}%
\pgfsys@useobject{currentmarker}{}%
\end{pgfscope}%
\end{pgfscope}%
\begin{pgfscope}%
\pgfsetbuttcap%
\pgfsetroundjoin%
\definecolor{currentfill}{rgb}{0.000000,0.000000,0.000000}%
\pgfsetfillcolor{currentfill}%
\pgfsetlinewidth{0.602250pt}%
\definecolor{currentstroke}{rgb}{0.000000,0.000000,0.000000}%
\pgfsetstrokecolor{currentstroke}%
\pgfsetdash{}{0pt}%
\pgfsys@defobject{currentmarker}{\pgfqpoint{-0.027778in}{0.000000in}}{\pgfqpoint{0.000000in}{0.000000in}}{%
\pgfpathmoveto{\pgfqpoint{0.000000in}{0.000000in}}%
\pgfpathlineto{\pgfqpoint{-0.027778in}{0.000000in}}%
\pgfusepath{stroke,fill}%
}%
\begin{pgfscope}%
\pgfsys@transformshift{0.800000in}{3.033594in}%
\pgfsys@useobject{currentmarker}{}%
\end{pgfscope}%
\end{pgfscope}%
\begin{pgfscope}%
\pgfsetbuttcap%
\pgfsetroundjoin%
\definecolor{currentfill}{rgb}{0.000000,0.000000,0.000000}%
\pgfsetfillcolor{currentfill}%
\pgfsetlinewidth{0.602250pt}%
\definecolor{currentstroke}{rgb}{0.000000,0.000000,0.000000}%
\pgfsetstrokecolor{currentstroke}%
\pgfsetdash{}{0pt}%
\pgfsys@defobject{currentmarker}{\pgfqpoint{-0.027778in}{0.000000in}}{\pgfqpoint{0.000000in}{0.000000in}}{%
\pgfpathmoveto{\pgfqpoint{0.000000in}{0.000000in}}%
\pgfpathlineto{\pgfqpoint{-0.027778in}{0.000000in}}%
\pgfusepath{stroke,fill}%
}%
\begin{pgfscope}%
\pgfsys@transformshift{0.800000in}{3.130786in}%
\pgfsys@useobject{currentmarker}{}%
\end{pgfscope}%
\end{pgfscope}%
\begin{pgfscope}%
\pgfsetbuttcap%
\pgfsetroundjoin%
\definecolor{currentfill}{rgb}{0.000000,0.000000,0.000000}%
\pgfsetfillcolor{currentfill}%
\pgfsetlinewidth{0.602250pt}%
\definecolor{currentstroke}{rgb}{0.000000,0.000000,0.000000}%
\pgfsetstrokecolor{currentstroke}%
\pgfsetdash{}{0pt}%
\pgfsys@defobject{currentmarker}{\pgfqpoint{-0.027778in}{0.000000in}}{\pgfqpoint{0.000000in}{0.000000in}}{%
\pgfpathmoveto{\pgfqpoint{0.000000in}{0.000000in}}%
\pgfpathlineto{\pgfqpoint{-0.027778in}{0.000000in}}%
\pgfusepath{stroke,fill}%
}%
\begin{pgfscope}%
\pgfsys@transformshift{0.800000in}{3.199745in}%
\pgfsys@useobject{currentmarker}{}%
\end{pgfscope}%
\end{pgfscope}%
\begin{pgfscope}%
\pgfsetbuttcap%
\pgfsetroundjoin%
\definecolor{currentfill}{rgb}{0.000000,0.000000,0.000000}%
\pgfsetfillcolor{currentfill}%
\pgfsetlinewidth{0.602250pt}%
\definecolor{currentstroke}{rgb}{0.000000,0.000000,0.000000}%
\pgfsetstrokecolor{currentstroke}%
\pgfsetdash{}{0pt}%
\pgfsys@defobject{currentmarker}{\pgfqpoint{-0.027778in}{0.000000in}}{\pgfqpoint{0.000000in}{0.000000in}}{%
\pgfpathmoveto{\pgfqpoint{0.000000in}{0.000000in}}%
\pgfpathlineto{\pgfqpoint{-0.027778in}{0.000000in}}%
\pgfusepath{stroke,fill}%
}%
\begin{pgfscope}%
\pgfsys@transformshift{0.800000in}{3.253234in}%
\pgfsys@useobject{currentmarker}{}%
\end{pgfscope}%
\end{pgfscope}%
\begin{pgfscope}%
\pgfsetbuttcap%
\pgfsetroundjoin%
\definecolor{currentfill}{rgb}{0.000000,0.000000,0.000000}%
\pgfsetfillcolor{currentfill}%
\pgfsetlinewidth{0.602250pt}%
\definecolor{currentstroke}{rgb}{0.000000,0.000000,0.000000}%
\pgfsetstrokecolor{currentstroke}%
\pgfsetdash{}{0pt}%
\pgfsys@defobject{currentmarker}{\pgfqpoint{-0.027778in}{0.000000in}}{\pgfqpoint{0.000000in}{0.000000in}}{%
\pgfpathmoveto{\pgfqpoint{0.000000in}{0.000000in}}%
\pgfpathlineto{\pgfqpoint{-0.027778in}{0.000000in}}%
\pgfusepath{stroke,fill}%
}%
\begin{pgfscope}%
\pgfsys@transformshift{0.800000in}{3.296937in}%
\pgfsys@useobject{currentmarker}{}%
\end{pgfscope}%
\end{pgfscope}%
\begin{pgfscope}%
\pgfsetbuttcap%
\pgfsetroundjoin%
\definecolor{currentfill}{rgb}{0.000000,0.000000,0.000000}%
\pgfsetfillcolor{currentfill}%
\pgfsetlinewidth{0.602250pt}%
\definecolor{currentstroke}{rgb}{0.000000,0.000000,0.000000}%
\pgfsetstrokecolor{currentstroke}%
\pgfsetdash{}{0pt}%
\pgfsys@defobject{currentmarker}{\pgfqpoint{-0.027778in}{0.000000in}}{\pgfqpoint{0.000000in}{0.000000in}}{%
\pgfpathmoveto{\pgfqpoint{0.000000in}{0.000000in}}%
\pgfpathlineto{\pgfqpoint{-0.027778in}{0.000000in}}%
\pgfusepath{stroke,fill}%
}%
\begin{pgfscope}%
\pgfsys@transformshift{0.800000in}{3.333888in}%
\pgfsys@useobject{currentmarker}{}%
\end{pgfscope}%
\end{pgfscope}%
\begin{pgfscope}%
\pgfsetbuttcap%
\pgfsetroundjoin%
\definecolor{currentfill}{rgb}{0.000000,0.000000,0.000000}%
\pgfsetfillcolor{currentfill}%
\pgfsetlinewidth{0.602250pt}%
\definecolor{currentstroke}{rgb}{0.000000,0.000000,0.000000}%
\pgfsetstrokecolor{currentstroke}%
\pgfsetdash{}{0pt}%
\pgfsys@defobject{currentmarker}{\pgfqpoint{-0.027778in}{0.000000in}}{\pgfqpoint{0.000000in}{0.000000in}}{%
\pgfpathmoveto{\pgfqpoint{0.000000in}{0.000000in}}%
\pgfpathlineto{\pgfqpoint{-0.027778in}{0.000000in}}%
\pgfusepath{stroke,fill}%
}%
\begin{pgfscope}%
\pgfsys@transformshift{0.800000in}{3.365896in}%
\pgfsys@useobject{currentmarker}{}%
\end{pgfscope}%
\end{pgfscope}%
\begin{pgfscope}%
\pgfsetbuttcap%
\pgfsetroundjoin%
\definecolor{currentfill}{rgb}{0.000000,0.000000,0.000000}%
\pgfsetfillcolor{currentfill}%
\pgfsetlinewidth{0.602250pt}%
\definecolor{currentstroke}{rgb}{0.000000,0.000000,0.000000}%
\pgfsetstrokecolor{currentstroke}%
\pgfsetdash{}{0pt}%
\pgfsys@defobject{currentmarker}{\pgfqpoint{-0.027778in}{0.000000in}}{\pgfqpoint{0.000000in}{0.000000in}}{%
\pgfpathmoveto{\pgfqpoint{0.000000in}{0.000000in}}%
\pgfpathlineto{\pgfqpoint{-0.027778in}{0.000000in}}%
\pgfusepath{stroke,fill}%
}%
\begin{pgfscope}%
\pgfsys@transformshift{0.800000in}{3.394129in}%
\pgfsys@useobject{currentmarker}{}%
\end{pgfscope}%
\end{pgfscope}%
\begin{pgfscope}%
\pgfsetbuttcap%
\pgfsetroundjoin%
\definecolor{currentfill}{rgb}{0.000000,0.000000,0.000000}%
\pgfsetfillcolor{currentfill}%
\pgfsetlinewidth{0.602250pt}%
\definecolor{currentstroke}{rgb}{0.000000,0.000000,0.000000}%
\pgfsetstrokecolor{currentstroke}%
\pgfsetdash{}{0pt}%
\pgfsys@defobject{currentmarker}{\pgfqpoint{-0.027778in}{0.000000in}}{\pgfqpoint{0.000000in}{0.000000in}}{%
\pgfpathmoveto{\pgfqpoint{0.000000in}{0.000000in}}%
\pgfpathlineto{\pgfqpoint{-0.027778in}{0.000000in}}%
\pgfusepath{stroke,fill}%
}%
\begin{pgfscope}%
\pgfsys@transformshift{0.800000in}{3.585536in}%
\pgfsys@useobject{currentmarker}{}%
\end{pgfscope}%
\end{pgfscope}%
\begin{pgfscope}%
\pgfsetbuttcap%
\pgfsetroundjoin%
\definecolor{currentfill}{rgb}{0.000000,0.000000,0.000000}%
\pgfsetfillcolor{currentfill}%
\pgfsetlinewidth{0.602250pt}%
\definecolor{currentstroke}{rgb}{0.000000,0.000000,0.000000}%
\pgfsetstrokecolor{currentstroke}%
\pgfsetdash{}{0pt}%
\pgfsys@defobject{currentmarker}{\pgfqpoint{-0.027778in}{0.000000in}}{\pgfqpoint{0.000000in}{0.000000in}}{%
\pgfpathmoveto{\pgfqpoint{0.000000in}{0.000000in}}%
\pgfpathlineto{\pgfqpoint{-0.027778in}{0.000000in}}%
\pgfusepath{stroke,fill}%
}%
\begin{pgfscope}%
\pgfsys@transformshift{0.800000in}{3.682728in}%
\pgfsys@useobject{currentmarker}{}%
\end{pgfscope}%
\end{pgfscope}%
\begin{pgfscope}%
\pgfsetbuttcap%
\pgfsetroundjoin%
\definecolor{currentfill}{rgb}{0.000000,0.000000,0.000000}%
\pgfsetfillcolor{currentfill}%
\pgfsetlinewidth{0.602250pt}%
\definecolor{currentstroke}{rgb}{0.000000,0.000000,0.000000}%
\pgfsetstrokecolor{currentstroke}%
\pgfsetdash{}{0pt}%
\pgfsys@defobject{currentmarker}{\pgfqpoint{-0.027778in}{0.000000in}}{\pgfqpoint{0.000000in}{0.000000in}}{%
\pgfpathmoveto{\pgfqpoint{0.000000in}{0.000000in}}%
\pgfpathlineto{\pgfqpoint{-0.027778in}{0.000000in}}%
\pgfusepath{stroke,fill}%
}%
\begin{pgfscope}%
\pgfsys@transformshift{0.800000in}{3.751687in}%
\pgfsys@useobject{currentmarker}{}%
\end{pgfscope}%
\end{pgfscope}%
\begin{pgfscope}%
\pgfsetbuttcap%
\pgfsetroundjoin%
\definecolor{currentfill}{rgb}{0.000000,0.000000,0.000000}%
\pgfsetfillcolor{currentfill}%
\pgfsetlinewidth{0.602250pt}%
\definecolor{currentstroke}{rgb}{0.000000,0.000000,0.000000}%
\pgfsetstrokecolor{currentstroke}%
\pgfsetdash{}{0pt}%
\pgfsys@defobject{currentmarker}{\pgfqpoint{-0.027778in}{0.000000in}}{\pgfqpoint{0.000000in}{0.000000in}}{%
\pgfpathmoveto{\pgfqpoint{0.000000in}{0.000000in}}%
\pgfpathlineto{\pgfqpoint{-0.027778in}{0.000000in}}%
\pgfusepath{stroke,fill}%
}%
\begin{pgfscope}%
\pgfsys@transformshift{0.800000in}{3.805176in}%
\pgfsys@useobject{currentmarker}{}%
\end{pgfscope}%
\end{pgfscope}%
\begin{pgfscope}%
\pgfsetbuttcap%
\pgfsetroundjoin%
\definecolor{currentfill}{rgb}{0.000000,0.000000,0.000000}%
\pgfsetfillcolor{currentfill}%
\pgfsetlinewidth{0.602250pt}%
\definecolor{currentstroke}{rgb}{0.000000,0.000000,0.000000}%
\pgfsetstrokecolor{currentstroke}%
\pgfsetdash{}{0pt}%
\pgfsys@defobject{currentmarker}{\pgfqpoint{-0.027778in}{0.000000in}}{\pgfqpoint{0.000000in}{0.000000in}}{%
\pgfpathmoveto{\pgfqpoint{0.000000in}{0.000000in}}%
\pgfpathlineto{\pgfqpoint{-0.027778in}{0.000000in}}%
\pgfusepath{stroke,fill}%
}%
\begin{pgfscope}%
\pgfsys@transformshift{0.800000in}{3.848879in}%
\pgfsys@useobject{currentmarker}{}%
\end{pgfscope}%
\end{pgfscope}%
\begin{pgfscope}%
\pgfsetbuttcap%
\pgfsetroundjoin%
\definecolor{currentfill}{rgb}{0.000000,0.000000,0.000000}%
\pgfsetfillcolor{currentfill}%
\pgfsetlinewidth{0.602250pt}%
\definecolor{currentstroke}{rgb}{0.000000,0.000000,0.000000}%
\pgfsetstrokecolor{currentstroke}%
\pgfsetdash{}{0pt}%
\pgfsys@defobject{currentmarker}{\pgfqpoint{-0.027778in}{0.000000in}}{\pgfqpoint{0.000000in}{0.000000in}}{%
\pgfpathmoveto{\pgfqpoint{0.000000in}{0.000000in}}%
\pgfpathlineto{\pgfqpoint{-0.027778in}{0.000000in}}%
\pgfusepath{stroke,fill}%
}%
\begin{pgfscope}%
\pgfsys@transformshift{0.800000in}{3.885830in}%
\pgfsys@useobject{currentmarker}{}%
\end{pgfscope}%
\end{pgfscope}%
\begin{pgfscope}%
\pgfsetbuttcap%
\pgfsetroundjoin%
\definecolor{currentfill}{rgb}{0.000000,0.000000,0.000000}%
\pgfsetfillcolor{currentfill}%
\pgfsetlinewidth{0.602250pt}%
\definecolor{currentstroke}{rgb}{0.000000,0.000000,0.000000}%
\pgfsetstrokecolor{currentstroke}%
\pgfsetdash{}{0pt}%
\pgfsys@defobject{currentmarker}{\pgfqpoint{-0.027778in}{0.000000in}}{\pgfqpoint{0.000000in}{0.000000in}}{%
\pgfpathmoveto{\pgfqpoint{0.000000in}{0.000000in}}%
\pgfpathlineto{\pgfqpoint{-0.027778in}{0.000000in}}%
\pgfusepath{stroke,fill}%
}%
\begin{pgfscope}%
\pgfsys@transformshift{0.800000in}{3.917838in}%
\pgfsys@useobject{currentmarker}{}%
\end{pgfscope}%
\end{pgfscope}%
\begin{pgfscope}%
\pgfsetbuttcap%
\pgfsetroundjoin%
\definecolor{currentfill}{rgb}{0.000000,0.000000,0.000000}%
\pgfsetfillcolor{currentfill}%
\pgfsetlinewidth{0.602250pt}%
\definecolor{currentstroke}{rgb}{0.000000,0.000000,0.000000}%
\pgfsetstrokecolor{currentstroke}%
\pgfsetdash{}{0pt}%
\pgfsys@defobject{currentmarker}{\pgfqpoint{-0.027778in}{0.000000in}}{\pgfqpoint{0.000000in}{0.000000in}}{%
\pgfpathmoveto{\pgfqpoint{0.000000in}{0.000000in}}%
\pgfpathlineto{\pgfqpoint{-0.027778in}{0.000000in}}%
\pgfusepath{stroke,fill}%
}%
\begin{pgfscope}%
\pgfsys@transformshift{0.800000in}{3.946071in}%
\pgfsys@useobject{currentmarker}{}%
\end{pgfscope}%
\end{pgfscope}%
\begin{pgfscope}%
\pgfsetbuttcap%
\pgfsetroundjoin%
\definecolor{currentfill}{rgb}{0.000000,0.000000,0.000000}%
\pgfsetfillcolor{currentfill}%
\pgfsetlinewidth{0.602250pt}%
\definecolor{currentstroke}{rgb}{0.000000,0.000000,0.000000}%
\pgfsetstrokecolor{currentstroke}%
\pgfsetdash{}{0pt}%
\pgfsys@defobject{currentmarker}{\pgfqpoint{-0.027778in}{0.000000in}}{\pgfqpoint{0.000000in}{0.000000in}}{%
\pgfpathmoveto{\pgfqpoint{0.000000in}{0.000000in}}%
\pgfpathlineto{\pgfqpoint{-0.027778in}{0.000000in}}%
\pgfusepath{stroke,fill}%
}%
\begin{pgfscope}%
\pgfsys@transformshift{0.800000in}{4.137478in}%
\pgfsys@useobject{currentmarker}{}%
\end{pgfscope}%
\end{pgfscope}%
\begin{pgfscope}%
\pgfpathrectangle{\pgfqpoint{0.800000in}{0.528000in}}{\pgfqpoint{4.960000in}{3.696000in}} %
\pgfusepath{clip}%
\pgfsetrectcap%
\pgfsetroundjoin%
\pgfsetlinewidth{1.505625pt}%
\definecolor{currentstroke}{rgb}{0.121569,0.466667,0.705882}%
\pgfsetstrokecolor{currentstroke}%
\pgfsetdash{}{0pt}%
\pgfpathmoveto{\pgfqpoint{1.025455in}{3.928923in}}%
\pgfpathlineto{\pgfqpoint{1.026356in}{3.662321in}}%
\pgfpathlineto{\pgfqpoint{1.028611in}{2.854900in}}%
\pgfpathlineto{\pgfqpoint{1.030415in}{2.867216in}}%
\pgfpathlineto{\pgfqpoint{1.030866in}{2.867051in}}%
\pgfpathlineto{\pgfqpoint{1.032219in}{2.865360in}}%
\pgfpathlineto{\pgfqpoint{1.034925in}{2.856050in}}%
\pgfpathlineto{\pgfqpoint{1.037179in}{2.843986in}}%
\pgfpathlineto{\pgfqpoint{1.046198in}{2.770183in}}%
\pgfpathlineto{\pgfqpoint{1.048453in}{2.731404in}}%
\pgfpathlineto{\pgfqpoint{1.051610in}{2.671865in}}%
\pgfpathlineto{\pgfqpoint{1.054316in}{2.598537in}}%
\pgfpathlineto{\pgfqpoint{1.059727in}{2.494234in}}%
\pgfpathlineto{\pgfqpoint{1.062433in}{2.461511in}}%
\pgfpathlineto{\pgfqpoint{1.068746in}{2.342177in}}%
\pgfpathlineto{\pgfqpoint{1.079118in}{2.222866in}}%
\pgfpathlineto{\pgfqpoint{1.082275in}{2.205745in}}%
\pgfpathlineto{\pgfqpoint{1.093098in}{2.110413in}}%
\pgfpathlineto{\pgfqpoint{1.097156in}{2.093573in}}%
\pgfpathlineto{\pgfqpoint{1.101666in}{2.060171in}}%
\pgfpathlineto{\pgfqpoint{1.103019in}{2.060985in}}%
\pgfpathlineto{\pgfqpoint{1.105724in}{2.054361in}}%
\pgfpathlineto{\pgfqpoint{1.109783in}{2.031055in}}%
\pgfpathlineto{\pgfqpoint{1.113842in}{2.013848in}}%
\pgfpathlineto{\pgfqpoint{1.115645in}{2.008239in}}%
\pgfpathlineto{\pgfqpoint{1.119704in}{1.985464in}}%
\pgfpathlineto{\pgfqpoint{1.120606in}{1.985804in}}%
\pgfpathlineto{\pgfqpoint{1.123763in}{1.995838in}}%
\pgfpathlineto{\pgfqpoint{1.124664in}{1.997389in}}%
\pgfpathlineto{\pgfqpoint{1.125115in}{1.996706in}}%
\pgfpathlineto{\pgfqpoint{1.127370in}{1.981303in}}%
\pgfpathlineto{\pgfqpoint{1.130978in}{1.954617in}}%
\pgfpathlineto{\pgfqpoint{1.133684in}{1.945722in}}%
\pgfpathlineto{\pgfqpoint{1.138193in}{1.956070in}}%
\pgfpathlineto{\pgfqpoint{1.138644in}{1.955515in}}%
\pgfpathlineto{\pgfqpoint{1.140448in}{1.943223in}}%
\pgfpathlineto{\pgfqpoint{1.144506in}{1.911067in}}%
\pgfpathlineto{\pgfqpoint{1.144957in}{1.910699in}}%
\pgfpathlineto{\pgfqpoint{1.145408in}{1.911439in}}%
\pgfpathlineto{\pgfqpoint{1.148114in}{1.913564in}}%
\pgfpathlineto{\pgfqpoint{1.149467in}{1.913653in}}%
\pgfpathlineto{\pgfqpoint{1.157133in}{1.894023in}}%
\pgfpathlineto{\pgfqpoint{1.159388in}{1.887440in}}%
\pgfpathlineto{\pgfqpoint{1.162996in}{1.861645in}}%
\pgfpathlineto{\pgfqpoint{1.163897in}{1.862680in}}%
\pgfpathlineto{\pgfqpoint{1.167054in}{1.865788in}}%
\pgfpathlineto{\pgfqpoint{1.168407in}{1.867348in}}%
\pgfpathlineto{\pgfqpoint{1.170662in}{1.860758in}}%
\pgfpathlineto{\pgfqpoint{1.175622in}{1.838548in}}%
\pgfpathlineto{\pgfqpoint{1.176073in}{1.838690in}}%
\pgfpathlineto{\pgfqpoint{1.177426in}{1.838795in}}%
\pgfpathlineto{\pgfqpoint{1.177877in}{1.838198in}}%
\pgfpathlineto{\pgfqpoint{1.178328in}{1.838790in}}%
\pgfpathlineto{\pgfqpoint{1.180132in}{1.841017in}}%
\pgfpathlineto{\pgfqpoint{1.181485in}{1.838742in}}%
\pgfpathlineto{\pgfqpoint{1.188249in}{1.820831in}}%
\pgfpathlineto{\pgfqpoint{1.190955in}{1.826406in}}%
\pgfpathlineto{\pgfqpoint{1.191857in}{1.824507in}}%
\pgfpathlineto{\pgfqpoint{1.194111in}{1.811178in}}%
\pgfpathlineto{\pgfqpoint{1.196817in}{1.793771in}}%
\pgfpathlineto{\pgfqpoint{1.199974in}{1.792710in}}%
\pgfpathlineto{\pgfqpoint{1.203130in}{1.797759in}}%
\pgfpathlineto{\pgfqpoint{1.204934in}{1.801736in}}%
\pgfpathlineto{\pgfqpoint{1.206287in}{1.801169in}}%
\pgfpathlineto{\pgfqpoint{1.208091in}{1.795443in}}%
\pgfpathlineto{\pgfqpoint{1.211248in}{1.784354in}}%
\pgfpathlineto{\pgfqpoint{1.213051in}{1.782335in}}%
\pgfpathlineto{\pgfqpoint{1.213953in}{1.781249in}}%
\pgfpathlineto{\pgfqpoint{1.218463in}{1.769861in}}%
\pgfpathlineto{\pgfqpoint{1.227031in}{1.758766in}}%
\pgfpathlineto{\pgfqpoint{1.229286in}{1.754564in}}%
\pgfpathlineto{\pgfqpoint{1.230639in}{1.751661in}}%
\pgfpathlineto{\pgfqpoint{1.235148in}{1.738041in}}%
\pgfpathlineto{\pgfqpoint{1.237854in}{1.732292in}}%
\pgfpathlineto{\pgfqpoint{1.239207in}{1.729682in}}%
\pgfpathlineto{\pgfqpoint{1.239658in}{1.729977in}}%
\pgfpathlineto{\pgfqpoint{1.241913in}{1.729975in}}%
\pgfpathlineto{\pgfqpoint{1.245971in}{1.718047in}}%
\pgfpathlineto{\pgfqpoint{1.249579in}{1.707127in}}%
\pgfpathlineto{\pgfqpoint{1.250481in}{1.705703in}}%
\pgfpathlineto{\pgfqpoint{1.250932in}{1.705977in}}%
\pgfpathlineto{\pgfqpoint{1.254088in}{1.714426in}}%
\pgfpathlineto{\pgfqpoint{1.257245in}{1.722512in}}%
\pgfpathlineto{\pgfqpoint{1.258598in}{1.722073in}}%
\pgfpathlineto{\pgfqpoint{1.259951in}{1.717833in}}%
\pgfpathlineto{\pgfqpoint{1.263558in}{1.706529in}}%
\pgfpathlineto{\pgfqpoint{1.266264in}{1.704223in}}%
\pgfpathlineto{\pgfqpoint{1.267166in}{1.704858in}}%
\pgfpathlineto{\pgfqpoint{1.269872in}{1.710957in}}%
\pgfpathlineto{\pgfqpoint{1.271225in}{1.710575in}}%
\pgfpathlineto{\pgfqpoint{1.273028in}{1.707827in}}%
\pgfpathlineto{\pgfqpoint{1.275734in}{1.697695in}}%
\pgfpathlineto{\pgfqpoint{1.278440in}{1.685823in}}%
\pgfpathlineto{\pgfqpoint{1.282047in}{1.679042in}}%
\pgfpathlineto{\pgfqpoint{1.283400in}{1.677022in}}%
\pgfpathlineto{\pgfqpoint{1.283851in}{1.677283in}}%
\pgfpathlineto{\pgfqpoint{1.286106in}{1.679349in}}%
\pgfpathlineto{\pgfqpoint{1.287910in}{1.677073in}}%
\pgfpathlineto{\pgfqpoint{1.288812in}{1.675776in}}%
\pgfpathlineto{\pgfqpoint{1.289263in}{1.676551in}}%
\pgfpathlineto{\pgfqpoint{1.291518in}{1.679732in}}%
\pgfpathlineto{\pgfqpoint{1.291968in}{1.679309in}}%
\pgfpathlineto{\pgfqpoint{1.293772in}{1.673405in}}%
\pgfpathlineto{\pgfqpoint{1.296929in}{1.661936in}}%
\pgfpathlineto{\pgfqpoint{1.298282in}{1.663131in}}%
\pgfpathlineto{\pgfqpoint{1.303693in}{1.679851in}}%
\pgfpathlineto{\pgfqpoint{1.305046in}{1.679817in}}%
\pgfpathlineto{\pgfqpoint{1.306399in}{1.677520in}}%
\pgfpathlineto{\pgfqpoint{1.311359in}{1.660217in}}%
\pgfpathlineto{\pgfqpoint{1.314065in}{1.651359in}}%
\pgfpathlineto{\pgfqpoint{1.316771in}{1.647539in}}%
\pgfpathlineto{\pgfqpoint{1.321731in}{1.646966in}}%
\pgfpathlineto{\pgfqpoint{1.324888in}{1.650826in}}%
\pgfpathlineto{\pgfqpoint{1.326241in}{1.648389in}}%
\pgfpathlineto{\pgfqpoint{1.332103in}{1.631469in}}%
\pgfpathlineto{\pgfqpoint{1.333456in}{1.629268in}}%
\pgfpathlineto{\pgfqpoint{1.336162in}{1.631091in}}%
\pgfpathlineto{\pgfqpoint{1.337515in}{1.629460in}}%
\pgfpathlineto{\pgfqpoint{1.340672in}{1.623379in}}%
\pgfpathlineto{\pgfqpoint{1.342024in}{1.622070in}}%
\pgfpathlineto{\pgfqpoint{1.345632in}{1.615860in}}%
\pgfpathlineto{\pgfqpoint{1.347887in}{1.606915in}}%
\pgfpathlineto{\pgfqpoint{1.350593in}{1.597877in}}%
\pgfpathlineto{\pgfqpoint{1.353749in}{1.594573in}}%
\pgfpathlineto{\pgfqpoint{1.356906in}{1.592186in}}%
\pgfpathlineto{\pgfqpoint{1.360063in}{1.584872in}}%
\pgfpathlineto{\pgfqpoint{1.360514in}{1.585115in}}%
\pgfpathlineto{\pgfqpoint{1.364121in}{1.587308in}}%
\pgfpathlineto{\pgfqpoint{1.366827in}{1.591297in}}%
\pgfpathlineto{\pgfqpoint{1.367729in}{1.591062in}}%
\pgfpathlineto{\pgfqpoint{1.369533in}{1.585869in}}%
\pgfpathlineto{\pgfqpoint{1.377199in}{1.556460in}}%
\pgfpathlineto{\pgfqpoint{1.378101in}{1.557216in}}%
\pgfpathlineto{\pgfqpoint{1.380355in}{1.559475in}}%
\pgfpathlineto{\pgfqpoint{1.382159in}{1.559462in}}%
\pgfpathlineto{\pgfqpoint{1.383963in}{1.555911in}}%
\pgfpathlineto{\pgfqpoint{1.387120in}{1.548739in}}%
\pgfpathlineto{\pgfqpoint{1.387571in}{1.548933in}}%
\pgfpathlineto{\pgfqpoint{1.389826in}{1.549636in}}%
\pgfpathlineto{\pgfqpoint{1.394786in}{1.542148in}}%
\pgfpathlineto{\pgfqpoint{1.397943in}{1.541095in}}%
\pgfpathlineto{\pgfqpoint{1.399296in}{1.541847in}}%
\pgfpathlineto{\pgfqpoint{1.401099in}{1.538329in}}%
\pgfpathlineto{\pgfqpoint{1.404707in}{1.530158in}}%
\pgfpathlineto{\pgfqpoint{1.409217in}{1.530211in}}%
\pgfpathlineto{\pgfqpoint{1.411471in}{1.532979in}}%
\pgfpathlineto{\pgfqpoint{1.415079in}{1.539113in}}%
\pgfpathlineto{\pgfqpoint{1.417334in}{1.537985in}}%
\pgfpathlineto{\pgfqpoint{1.418236in}{1.538791in}}%
\pgfpathlineto{\pgfqpoint{1.418687in}{1.538324in}}%
\pgfpathlineto{\pgfqpoint{1.422294in}{1.528578in}}%
\pgfpathlineto{\pgfqpoint{1.428608in}{1.505596in}}%
\pgfpathlineto{\pgfqpoint{1.429509in}{1.504035in}}%
\pgfpathlineto{\pgfqpoint{1.429960in}{1.504231in}}%
\pgfpathlineto{\pgfqpoint{1.432215in}{1.505474in}}%
\pgfpathlineto{\pgfqpoint{1.434470in}{1.507408in}}%
\pgfpathlineto{\pgfqpoint{1.438980in}{1.508453in}}%
\pgfpathlineto{\pgfqpoint{1.440332in}{1.507906in}}%
\pgfpathlineto{\pgfqpoint{1.441234in}{1.508206in}}%
\pgfpathlineto{\pgfqpoint{1.443940in}{1.512009in}}%
\pgfpathlineto{\pgfqpoint{1.446646in}{1.516382in}}%
\pgfpathlineto{\pgfqpoint{1.447999in}{1.515914in}}%
\pgfpathlineto{\pgfqpoint{1.451155in}{1.510789in}}%
\pgfpathlineto{\pgfqpoint{1.453410in}{1.510337in}}%
\pgfpathlineto{\pgfqpoint{1.455665in}{1.512412in}}%
\pgfpathlineto{\pgfqpoint{1.458371in}{1.514351in}}%
\pgfpathlineto{\pgfqpoint{1.465135in}{1.512367in}}%
\pgfpathlineto{\pgfqpoint{1.467390in}{1.513322in}}%
\pgfpathlineto{\pgfqpoint{1.469193in}{1.509098in}}%
\pgfpathlineto{\pgfqpoint{1.473252in}{1.495938in}}%
\pgfpathlineto{\pgfqpoint{1.476860in}{1.493717in}}%
\pgfpathlineto{\pgfqpoint{1.478664in}{1.493566in}}%
\pgfpathlineto{\pgfqpoint{1.482722in}{1.486502in}}%
\pgfpathlineto{\pgfqpoint{1.487683in}{1.476240in}}%
\pgfpathlineto{\pgfqpoint{1.498055in}{1.487149in}}%
\pgfpathlineto{\pgfqpoint{1.499858in}{1.486734in}}%
\pgfpathlineto{\pgfqpoint{1.501211in}{1.486835in}}%
\pgfpathlineto{\pgfqpoint{1.502564in}{1.486292in}}%
\pgfpathlineto{\pgfqpoint{1.505270in}{1.484346in}}%
\pgfpathlineto{\pgfqpoint{1.507976in}{1.483426in}}%
\pgfpathlineto{\pgfqpoint{1.513387in}{1.481231in}}%
\pgfpathlineto{\pgfqpoint{1.517897in}{1.484226in}}%
\pgfpathlineto{\pgfqpoint{1.521504in}{1.480723in}}%
\pgfpathlineto{\pgfqpoint{1.523759in}{1.478304in}}%
\pgfpathlineto{\pgfqpoint{1.528268in}{1.487672in}}%
\pgfpathlineto{\pgfqpoint{1.529621in}{1.487282in}}%
\pgfpathlineto{\pgfqpoint{1.537288in}{1.478225in}}%
\pgfpathlineto{\pgfqpoint{1.545405in}{1.459340in}}%
\pgfpathlineto{\pgfqpoint{1.548110in}{1.456047in}}%
\pgfpathlineto{\pgfqpoint{1.554424in}{1.466766in}}%
\pgfpathlineto{\pgfqpoint{1.557130in}{1.466616in}}%
\pgfpathlineto{\pgfqpoint{1.558482in}{1.463159in}}%
\pgfpathlineto{\pgfqpoint{1.562992in}{1.448822in}}%
\pgfpathlineto{\pgfqpoint{1.567501in}{1.437419in}}%
\pgfpathlineto{\pgfqpoint{1.570658in}{1.433474in}}%
\pgfpathlineto{\pgfqpoint{1.571109in}{1.433869in}}%
\pgfpathlineto{\pgfqpoint{1.572913in}{1.437884in}}%
\pgfpathlineto{\pgfqpoint{1.578775in}{1.449194in}}%
\pgfpathlineto{\pgfqpoint{1.584638in}{1.444011in}}%
\pgfpathlineto{\pgfqpoint{1.592304in}{1.427904in}}%
\pgfpathlineto{\pgfqpoint{1.592755in}{1.428173in}}%
\pgfpathlineto{\pgfqpoint{1.594108in}{1.429093in}}%
\pgfpathlineto{\pgfqpoint{1.596363in}{1.431550in}}%
\pgfpathlineto{\pgfqpoint{1.599519in}{1.433155in}}%
\pgfpathlineto{\pgfqpoint{1.602676in}{1.433252in}}%
\pgfpathlineto{\pgfqpoint{1.604480in}{1.429635in}}%
\pgfpathlineto{\pgfqpoint{1.608989in}{1.421062in}}%
\pgfpathlineto{\pgfqpoint{1.610793in}{1.421943in}}%
\pgfpathlineto{\pgfqpoint{1.613048in}{1.424194in}}%
\pgfpathlineto{\pgfqpoint{1.618459in}{1.429254in}}%
\pgfpathlineto{\pgfqpoint{1.620714in}{1.429295in}}%
\pgfpathlineto{\pgfqpoint{1.622518in}{1.429320in}}%
\pgfpathlineto{\pgfqpoint{1.624322in}{1.427599in}}%
\pgfpathlineto{\pgfqpoint{1.627478in}{1.423014in}}%
\pgfpathlineto{\pgfqpoint{1.635145in}{1.417675in}}%
\pgfpathlineto{\pgfqpoint{1.637850in}{1.419802in}}%
\pgfpathlineto{\pgfqpoint{1.645066in}{1.427149in}}%
\pgfpathlineto{\pgfqpoint{1.646869in}{1.424881in}}%
\pgfpathlineto{\pgfqpoint{1.651830in}{1.417880in}}%
\pgfpathlineto{\pgfqpoint{1.653183in}{1.417988in}}%
\pgfpathlineto{\pgfqpoint{1.654536in}{1.418822in}}%
\pgfpathlineto{\pgfqpoint{1.657241in}{1.421892in}}%
\pgfpathlineto{\pgfqpoint{1.662202in}{1.424030in}}%
\pgfpathlineto{\pgfqpoint{1.664908in}{1.425596in}}%
\pgfpathlineto{\pgfqpoint{1.667613in}{1.423356in}}%
\pgfpathlineto{\pgfqpoint{1.670319in}{1.416185in}}%
\pgfpathlineto{\pgfqpoint{1.672574in}{1.412406in}}%
\pgfpathlineto{\pgfqpoint{1.676632in}{1.416009in}}%
\pgfpathlineto{\pgfqpoint{1.680691in}{1.421371in}}%
\pgfpathlineto{\pgfqpoint{1.682495in}{1.422244in}}%
\pgfpathlineto{\pgfqpoint{1.684299in}{1.418758in}}%
\pgfpathlineto{\pgfqpoint{1.688808in}{1.408103in}}%
\pgfpathlineto{\pgfqpoint{1.692416in}{1.408389in}}%
\pgfpathlineto{\pgfqpoint{1.694220in}{1.411104in}}%
\pgfpathlineto{\pgfqpoint{1.696474in}{1.413667in}}%
\pgfpathlineto{\pgfqpoint{1.698278in}{1.411877in}}%
\pgfpathlineto{\pgfqpoint{1.709101in}{1.398883in}}%
\pgfpathlineto{\pgfqpoint{1.710454in}{1.398727in}}%
\pgfpathlineto{\pgfqpoint{1.715414in}{1.395380in}}%
\pgfpathlineto{\pgfqpoint{1.723081in}{1.410953in}}%
\pgfpathlineto{\pgfqpoint{1.725786in}{1.414359in}}%
\pgfpathlineto{\pgfqpoint{1.728943in}{1.412951in}}%
\pgfpathlineto{\pgfqpoint{1.731649in}{1.410582in}}%
\pgfpathlineto{\pgfqpoint{1.736609in}{1.398313in}}%
\pgfpathlineto{\pgfqpoint{1.741570in}{1.389378in}}%
\pgfpathlineto{\pgfqpoint{1.743374in}{1.389472in}}%
\pgfpathlineto{\pgfqpoint{1.745628in}{1.390165in}}%
\pgfpathlineto{\pgfqpoint{1.746981in}{1.390084in}}%
\pgfpathlineto{\pgfqpoint{1.748785in}{1.387234in}}%
\pgfpathlineto{\pgfqpoint{1.750589in}{1.384550in}}%
\pgfpathlineto{\pgfqpoint{1.753295in}{1.385680in}}%
\pgfpathlineto{\pgfqpoint{1.761412in}{1.388778in}}%
\pgfpathlineto{\pgfqpoint{1.762765in}{1.386320in}}%
\pgfpathlineto{\pgfqpoint{1.771333in}{1.364001in}}%
\pgfpathlineto{\pgfqpoint{1.773137in}{1.366556in}}%
\pgfpathlineto{\pgfqpoint{1.776744in}{1.373259in}}%
\pgfpathlineto{\pgfqpoint{1.779450in}{1.375099in}}%
\pgfpathlineto{\pgfqpoint{1.782607in}{1.373917in}}%
\pgfpathlineto{\pgfqpoint{1.788469in}{1.366990in}}%
\pgfpathlineto{\pgfqpoint{1.791175in}{1.363938in}}%
\pgfpathlineto{\pgfqpoint{1.798841in}{1.354050in}}%
\pgfpathlineto{\pgfqpoint{1.802449in}{1.353252in}}%
\pgfpathlineto{\pgfqpoint{1.804703in}{1.353302in}}%
\pgfpathlineto{\pgfqpoint{1.808762in}{1.354861in}}%
\pgfpathlineto{\pgfqpoint{1.811017in}{1.354118in}}%
\pgfpathlineto{\pgfqpoint{1.814173in}{1.350708in}}%
\pgfpathlineto{\pgfqpoint{1.819134in}{1.346944in}}%
\pgfpathlineto{\pgfqpoint{1.824996in}{1.342867in}}%
\pgfpathlineto{\pgfqpoint{1.828153in}{1.339426in}}%
\pgfpathlineto{\pgfqpoint{1.829506in}{1.340008in}}%
\pgfpathlineto{\pgfqpoint{1.831310in}{1.340178in}}%
\pgfpathlineto{\pgfqpoint{1.837172in}{1.337509in}}%
\pgfpathlineto{\pgfqpoint{1.848897in}{1.345373in}}%
\pgfpathlineto{\pgfqpoint{1.851152in}{1.346417in}}%
\pgfpathlineto{\pgfqpoint{1.860171in}{1.357405in}}%
\pgfpathlineto{\pgfqpoint{1.861975in}{1.356541in}}%
\pgfpathlineto{\pgfqpoint{1.864680in}{1.351983in}}%
\pgfpathlineto{\pgfqpoint{1.872347in}{1.338242in}}%
\pgfpathlineto{\pgfqpoint{1.875954in}{1.336239in}}%
\pgfpathlineto{\pgfqpoint{1.882718in}{1.334265in}}%
\pgfpathlineto{\pgfqpoint{1.884973in}{1.329839in}}%
\pgfpathlineto{\pgfqpoint{1.889934in}{1.316342in}}%
\pgfpathlineto{\pgfqpoint{1.890836in}{1.316821in}}%
\pgfpathlineto{\pgfqpoint{1.894443in}{1.325675in}}%
\pgfpathlineto{\pgfqpoint{1.897149in}{1.331187in}}%
\pgfpathlineto{\pgfqpoint{1.900306in}{1.332698in}}%
\pgfpathlineto{\pgfqpoint{1.905266in}{1.335189in}}%
\pgfpathlineto{\pgfqpoint{1.908423in}{1.337244in}}%
\pgfpathlineto{\pgfqpoint{1.910678in}{1.334431in}}%
\pgfpathlineto{\pgfqpoint{1.915187in}{1.324525in}}%
\pgfpathlineto{\pgfqpoint{1.917893in}{1.319923in}}%
\pgfpathlineto{\pgfqpoint{1.925559in}{1.309186in}}%
\pgfpathlineto{\pgfqpoint{1.927814in}{1.310274in}}%
\pgfpathlineto{\pgfqpoint{1.931422in}{1.313127in}}%
\pgfpathlineto{\pgfqpoint{1.934127in}{1.314784in}}%
\pgfpathlineto{\pgfqpoint{1.937284in}{1.312803in}}%
\pgfpathlineto{\pgfqpoint{1.939990in}{1.313294in}}%
\pgfpathlineto{\pgfqpoint{1.943146in}{1.313274in}}%
\pgfpathlineto{\pgfqpoint{1.945401in}{1.312557in}}%
\pgfpathlineto{\pgfqpoint{1.948107in}{1.311975in}}%
\pgfpathlineto{\pgfqpoint{1.951263in}{1.307676in}}%
\pgfpathlineto{\pgfqpoint{1.953518in}{1.305151in}}%
\pgfpathlineto{\pgfqpoint{1.954871in}{1.305668in}}%
\pgfpathlineto{\pgfqpoint{1.958028in}{1.307585in}}%
\pgfpathlineto{\pgfqpoint{1.959832in}{1.305690in}}%
\pgfpathlineto{\pgfqpoint{1.962988in}{1.301005in}}%
\pgfpathlineto{\pgfqpoint{1.966596in}{1.300240in}}%
\pgfpathlineto{\pgfqpoint{1.969302in}{1.302297in}}%
\pgfpathlineto{\pgfqpoint{1.972458in}{1.305867in}}%
\pgfpathlineto{\pgfqpoint{1.975164in}{1.307020in}}%
\pgfpathlineto{\pgfqpoint{1.977419in}{1.311545in}}%
\pgfpathlineto{\pgfqpoint{1.979223in}{1.313230in}}%
\pgfpathlineto{\pgfqpoint{1.983732in}{1.310511in}}%
\pgfpathlineto{\pgfqpoint{1.990497in}{1.300635in}}%
\pgfpathlineto{\pgfqpoint{1.994104in}{1.300112in}}%
\pgfpathlineto{\pgfqpoint{1.995908in}{1.296717in}}%
\pgfpathlineto{\pgfqpoint{2.001319in}{1.281341in}}%
\pgfpathlineto{\pgfqpoint{2.003574in}{1.279358in}}%
\pgfpathlineto{\pgfqpoint{2.004927in}{1.280425in}}%
\pgfpathlineto{\pgfqpoint{2.008986in}{1.287132in}}%
\pgfpathlineto{\pgfqpoint{2.013946in}{1.293431in}}%
\pgfpathlineto{\pgfqpoint{2.018456in}{1.290505in}}%
\pgfpathlineto{\pgfqpoint{2.022514in}{1.285362in}}%
\pgfpathlineto{\pgfqpoint{2.024769in}{1.285607in}}%
\pgfpathlineto{\pgfqpoint{2.028828in}{1.292611in}}%
\pgfpathlineto{\pgfqpoint{2.030631in}{1.293740in}}%
\pgfpathlineto{\pgfqpoint{2.033788in}{1.290403in}}%
\pgfpathlineto{\pgfqpoint{2.036043in}{1.285097in}}%
\pgfpathlineto{\pgfqpoint{2.039651in}{1.278037in}}%
\pgfpathlineto{\pgfqpoint{2.045964in}{1.270949in}}%
\pgfpathlineto{\pgfqpoint{2.049121in}{1.267452in}}%
\pgfpathlineto{\pgfqpoint{2.051375in}{1.269352in}}%
\pgfpathlineto{\pgfqpoint{2.055885in}{1.274069in}}%
\pgfpathlineto{\pgfqpoint{2.060394in}{1.270216in}}%
\pgfpathlineto{\pgfqpoint{2.064453in}{1.275737in}}%
\pgfpathlineto{\pgfqpoint{2.067159in}{1.276226in}}%
\pgfpathlineto{\pgfqpoint{2.070315in}{1.274471in}}%
\pgfpathlineto{\pgfqpoint{2.076629in}{1.264693in}}%
\pgfpathlineto{\pgfqpoint{2.081138in}{1.265317in}}%
\pgfpathlineto{\pgfqpoint{2.083393in}{1.265130in}}%
\pgfpathlineto{\pgfqpoint{2.086099in}{1.264130in}}%
\pgfpathlineto{\pgfqpoint{2.097824in}{1.265399in}}%
\pgfpathlineto{\pgfqpoint{2.101882in}{1.263554in}}%
\pgfpathlineto{\pgfqpoint{2.104137in}{1.262510in}}%
\pgfpathlineto{\pgfqpoint{2.112705in}{1.256553in}}%
\pgfpathlineto{\pgfqpoint{2.124430in}{1.244876in}}%
\pgfpathlineto{\pgfqpoint{2.126685in}{1.245365in}}%
\pgfpathlineto{\pgfqpoint{2.132547in}{1.243482in}}%
\pgfpathlineto{\pgfqpoint{2.137508in}{1.245359in}}%
\pgfpathlineto{\pgfqpoint{2.144723in}{1.260612in}}%
\pgfpathlineto{\pgfqpoint{2.149232in}{1.258494in}}%
\pgfpathlineto{\pgfqpoint{2.151938in}{1.256903in}}%
\pgfpathlineto{\pgfqpoint{2.154193in}{1.255918in}}%
\pgfpathlineto{\pgfqpoint{2.156899in}{1.252687in}}%
\pgfpathlineto{\pgfqpoint{2.159604in}{1.249692in}}%
\pgfpathlineto{\pgfqpoint{2.167271in}{1.247243in}}%
\pgfpathlineto{\pgfqpoint{2.170427in}{1.246556in}}%
\pgfpathlineto{\pgfqpoint{2.171780in}{1.248346in}}%
\pgfpathlineto{\pgfqpoint{2.178093in}{1.260354in}}%
\pgfpathlineto{\pgfqpoint{2.181250in}{1.259286in}}%
\pgfpathlineto{\pgfqpoint{2.184407in}{1.254302in}}%
\pgfpathlineto{\pgfqpoint{2.193877in}{1.238966in}}%
\pgfpathlineto{\pgfqpoint{2.199288in}{1.242336in}}%
\pgfpathlineto{\pgfqpoint{2.201543in}{1.242951in}}%
\pgfpathlineto{\pgfqpoint{2.205151in}{1.239589in}}%
\pgfpathlineto{\pgfqpoint{2.214170in}{1.234543in}}%
\pgfpathlineto{\pgfqpoint{2.216876in}{1.233047in}}%
\pgfpathlineto{\pgfqpoint{2.218679in}{1.232782in}}%
\pgfpathlineto{\pgfqpoint{2.223189in}{1.234518in}}%
\pgfpathlineto{\pgfqpoint{2.231306in}{1.232549in}}%
\pgfpathlineto{\pgfqpoint{2.233110in}{1.232174in}}%
\pgfpathlineto{\pgfqpoint{2.234914in}{1.232609in}}%
\pgfpathlineto{\pgfqpoint{2.239874in}{1.233488in}}%
\pgfpathlineto{\pgfqpoint{2.243482in}{1.230189in}}%
\pgfpathlineto{\pgfqpoint{2.248442in}{1.225010in}}%
\pgfpathlineto{\pgfqpoint{2.249795in}{1.224265in}}%
\pgfpathlineto{\pgfqpoint{2.252952in}{1.221362in}}%
\pgfpathlineto{\pgfqpoint{2.255207in}{1.221935in}}%
\pgfpathlineto{\pgfqpoint{2.259265in}{1.224153in}}%
\pgfpathlineto{\pgfqpoint{2.262873in}{1.223517in}}%
\pgfpathlineto{\pgfqpoint{2.265579in}{1.223036in}}%
\pgfpathlineto{\pgfqpoint{2.271441in}{1.220519in}}%
\pgfpathlineto{\pgfqpoint{2.274598in}{1.220153in}}%
\pgfpathlineto{\pgfqpoint{2.276401in}{1.221024in}}%
\pgfpathlineto{\pgfqpoint{2.279558in}{1.223055in}}%
\pgfpathlineto{\pgfqpoint{2.282264in}{1.221712in}}%
\pgfpathlineto{\pgfqpoint{2.285421in}{1.219760in}}%
\pgfpathlineto{\pgfqpoint{2.288577in}{1.220970in}}%
\pgfpathlineto{\pgfqpoint{2.289930in}{1.220437in}}%
\pgfpathlineto{\pgfqpoint{2.296243in}{1.213765in}}%
\pgfpathlineto{\pgfqpoint{2.298498in}{1.212757in}}%
\pgfpathlineto{\pgfqpoint{2.303459in}{1.212172in}}%
\pgfpathlineto{\pgfqpoint{2.305713in}{1.212808in}}%
\pgfpathlineto{\pgfqpoint{2.307968in}{1.209849in}}%
\pgfpathlineto{\pgfqpoint{2.314733in}{1.196802in}}%
\pgfpathlineto{\pgfqpoint{2.316085in}{1.195417in}}%
\pgfpathlineto{\pgfqpoint{2.316536in}{1.195850in}}%
\pgfpathlineto{\pgfqpoint{2.321046in}{1.202125in}}%
\pgfpathlineto{\pgfqpoint{2.323301in}{1.203660in}}%
\pgfpathlineto{\pgfqpoint{2.326908in}{1.203042in}}%
\pgfpathlineto{\pgfqpoint{2.330065in}{1.200172in}}%
\pgfpathlineto{\pgfqpoint{2.332771in}{1.200545in}}%
\pgfpathlineto{\pgfqpoint{2.334124in}{1.199527in}}%
\pgfpathlineto{\pgfqpoint{2.339535in}{1.192684in}}%
\pgfpathlineto{\pgfqpoint{2.342241in}{1.194665in}}%
\pgfpathlineto{\pgfqpoint{2.346750in}{1.200333in}}%
\pgfpathlineto{\pgfqpoint{2.352162in}{1.206960in}}%
\pgfpathlineto{\pgfqpoint{2.359828in}{1.205295in}}%
\pgfpathlineto{\pgfqpoint{2.362534in}{1.204608in}}%
\pgfpathlineto{\pgfqpoint{2.367043in}{1.204724in}}%
\pgfpathlineto{\pgfqpoint{2.369298in}{1.203395in}}%
\pgfpathlineto{\pgfqpoint{2.371553in}{1.202935in}}%
\pgfpathlineto{\pgfqpoint{2.382376in}{1.193056in}}%
\pgfpathlineto{\pgfqpoint{2.386434in}{1.194504in}}%
\pgfpathlineto{\pgfqpoint{2.388689in}{1.189696in}}%
\pgfpathlineto{\pgfqpoint{2.393199in}{1.178328in}}%
\pgfpathlineto{\pgfqpoint{2.395453in}{1.182042in}}%
\pgfpathlineto{\pgfqpoint{2.399963in}{1.190351in}}%
\pgfpathlineto{\pgfqpoint{2.404021in}{1.195486in}}%
\pgfpathlineto{\pgfqpoint{2.412139in}{1.193779in}}%
\pgfpathlineto{\pgfqpoint{2.414844in}{1.186427in}}%
\pgfpathlineto{\pgfqpoint{2.419354in}{1.173997in}}%
\pgfpathlineto{\pgfqpoint{2.420707in}{1.173150in}}%
\pgfpathlineto{\pgfqpoint{2.422511in}{1.175129in}}%
\pgfpathlineto{\pgfqpoint{2.429275in}{1.188883in}}%
\pgfpathlineto{\pgfqpoint{2.431079in}{1.190469in}}%
\pgfpathlineto{\pgfqpoint{2.433784in}{1.190482in}}%
\pgfpathlineto{\pgfqpoint{2.439647in}{1.184376in}}%
\pgfpathlineto{\pgfqpoint{2.442804in}{1.176483in}}%
\pgfpathlineto{\pgfqpoint{2.450019in}{1.159015in}}%
\pgfpathlineto{\pgfqpoint{2.452725in}{1.155912in}}%
\pgfpathlineto{\pgfqpoint{2.456783in}{1.154884in}}%
\pgfpathlineto{\pgfqpoint{2.459038in}{1.157823in}}%
\pgfpathlineto{\pgfqpoint{2.462195in}{1.163620in}}%
\pgfpathlineto{\pgfqpoint{2.464900in}{1.165625in}}%
\pgfpathlineto{\pgfqpoint{2.467155in}{1.164110in}}%
\pgfpathlineto{\pgfqpoint{2.470312in}{1.161429in}}%
\pgfpathlineto{\pgfqpoint{2.475723in}{1.163884in}}%
\pgfpathlineto{\pgfqpoint{2.478429in}{1.163150in}}%
\pgfpathlineto{\pgfqpoint{2.489703in}{1.154424in}}%
\pgfpathlineto{\pgfqpoint{2.491507in}{1.152369in}}%
\pgfpathlineto{\pgfqpoint{2.492859in}{1.153249in}}%
\pgfpathlineto{\pgfqpoint{2.497820in}{1.166766in}}%
\pgfpathlineto{\pgfqpoint{2.501879in}{1.177643in}}%
\pgfpathlineto{\pgfqpoint{2.505035in}{1.178176in}}%
\pgfpathlineto{\pgfqpoint{2.523524in}{1.162784in}}%
\pgfpathlineto{\pgfqpoint{2.525779in}{1.160753in}}%
\pgfpathlineto{\pgfqpoint{2.527583in}{1.162306in}}%
\pgfpathlineto{\pgfqpoint{2.529387in}{1.164352in}}%
\pgfpathlineto{\pgfqpoint{2.537504in}{1.177855in}}%
\pgfpathlineto{\pgfqpoint{2.539308in}{1.177148in}}%
\pgfpathlineto{\pgfqpoint{2.544268in}{1.173340in}}%
\pgfpathlineto{\pgfqpoint{2.546523in}{1.171673in}}%
\pgfpathlineto{\pgfqpoint{2.548778in}{1.167168in}}%
\pgfpathlineto{\pgfqpoint{2.557346in}{1.148487in}}%
\pgfpathlineto{\pgfqpoint{2.563208in}{1.151174in}}%
\pgfpathlineto{\pgfqpoint{2.565463in}{1.149982in}}%
\pgfpathlineto{\pgfqpoint{2.567718in}{1.145699in}}%
\pgfpathlineto{\pgfqpoint{2.570875in}{1.139834in}}%
\pgfpathlineto{\pgfqpoint{2.573580in}{1.138911in}}%
\pgfpathlineto{\pgfqpoint{2.587109in}{1.143819in}}%
\pgfpathlineto{\pgfqpoint{2.590266in}{1.147461in}}%
\pgfpathlineto{\pgfqpoint{2.595677in}{1.149960in}}%
\pgfpathlineto{\pgfqpoint{2.603343in}{1.150294in}}%
\pgfpathlineto{\pgfqpoint{2.605598in}{1.148648in}}%
\pgfpathlineto{\pgfqpoint{2.608755in}{1.146336in}}%
\pgfpathlineto{\pgfqpoint{2.612813in}{1.146537in}}%
\pgfpathlineto{\pgfqpoint{2.615519in}{1.143141in}}%
\pgfpathlineto{\pgfqpoint{2.620930in}{1.135231in}}%
\pgfpathlineto{\pgfqpoint{2.624538in}{1.136087in}}%
\pgfpathlineto{\pgfqpoint{2.627244in}{1.138247in}}%
\pgfpathlineto{\pgfqpoint{2.630851in}{1.138748in}}%
\pgfpathlineto{\pgfqpoint{2.634459in}{1.140122in}}%
\pgfpathlineto{\pgfqpoint{2.637616in}{1.140931in}}%
\pgfpathlineto{\pgfqpoint{2.638969in}{1.140628in}}%
\pgfpathlineto{\pgfqpoint{2.642125in}{1.135520in}}%
\pgfpathlineto{\pgfqpoint{2.647086in}{1.129609in}}%
\pgfpathlineto{\pgfqpoint{2.648439in}{1.131400in}}%
\pgfpathlineto{\pgfqpoint{2.651144in}{1.135757in}}%
\pgfpathlineto{\pgfqpoint{2.653850in}{1.136603in}}%
\pgfpathlineto{\pgfqpoint{2.657007in}{1.134092in}}%
\pgfpathlineto{\pgfqpoint{2.661967in}{1.129613in}}%
\pgfpathlineto{\pgfqpoint{2.666928in}{1.128315in}}%
\pgfpathlineto{\pgfqpoint{2.669183in}{1.127079in}}%
\pgfpathlineto{\pgfqpoint{2.670535in}{1.125978in}}%
\pgfpathlineto{\pgfqpoint{2.679555in}{1.117006in}}%
\pgfpathlineto{\pgfqpoint{2.681358in}{1.118108in}}%
\pgfpathlineto{\pgfqpoint{2.685868in}{1.122623in}}%
\pgfpathlineto{\pgfqpoint{2.687672in}{1.121669in}}%
\pgfpathlineto{\pgfqpoint{2.701200in}{1.105535in}}%
\pgfpathlineto{\pgfqpoint{2.703455in}{1.106502in}}%
\pgfpathlineto{\pgfqpoint{2.707063in}{1.112099in}}%
\pgfpathlineto{\pgfqpoint{2.709317in}{1.117507in}}%
\pgfpathlineto{\pgfqpoint{2.713827in}{1.126993in}}%
\pgfpathlineto{\pgfqpoint{2.718788in}{1.132315in}}%
\pgfpathlineto{\pgfqpoint{2.721493in}{1.133660in}}%
\pgfpathlineto{\pgfqpoint{2.724650in}{1.131394in}}%
\pgfpathlineto{\pgfqpoint{2.735924in}{1.118327in}}%
\pgfpathlineto{\pgfqpoint{2.739080in}{1.118454in}}%
\pgfpathlineto{\pgfqpoint{2.741786in}{1.120092in}}%
\pgfpathlineto{\pgfqpoint{2.745394in}{1.123870in}}%
\pgfpathlineto{\pgfqpoint{2.748550in}{1.122807in}}%
\pgfpathlineto{\pgfqpoint{2.751707in}{1.118190in}}%
\pgfpathlineto{\pgfqpoint{2.758021in}{1.106888in}}%
\pgfpathlineto{\pgfqpoint{2.762530in}{1.096420in}}%
\pgfpathlineto{\pgfqpoint{2.765687in}{1.094156in}}%
\pgfpathlineto{\pgfqpoint{2.773353in}{1.092931in}}%
\pgfpathlineto{\pgfqpoint{2.778313in}{1.091515in}}%
\pgfpathlineto{\pgfqpoint{2.780568in}{1.091913in}}%
\pgfpathlineto{\pgfqpoint{2.785078in}{1.093216in}}%
\pgfpathlineto{\pgfqpoint{2.789136in}{1.087433in}}%
\pgfpathlineto{\pgfqpoint{2.791391in}{1.085331in}}%
\pgfpathlineto{\pgfqpoint{2.794548in}{1.086062in}}%
\pgfpathlineto{\pgfqpoint{2.797704in}{1.086281in}}%
\pgfpathlineto{\pgfqpoint{2.801312in}{1.087623in}}%
\pgfpathlineto{\pgfqpoint{2.804018in}{1.088671in}}%
\pgfpathlineto{\pgfqpoint{2.805822in}{1.089073in}}%
\pgfpathlineto{\pgfqpoint{2.807625in}{1.089053in}}%
\pgfpathlineto{\pgfqpoint{2.810782in}{1.089712in}}%
\pgfpathlineto{\pgfqpoint{2.814390in}{1.091216in}}%
\pgfpathlineto{\pgfqpoint{2.824762in}{1.086987in}}%
\pgfpathlineto{\pgfqpoint{2.826566in}{1.088108in}}%
\pgfpathlineto{\pgfqpoint{2.828820in}{1.091158in}}%
\pgfpathlineto{\pgfqpoint{2.833781in}{1.096249in}}%
\pgfpathlineto{\pgfqpoint{2.836487in}{1.096132in}}%
\pgfpathlineto{\pgfqpoint{2.838741in}{1.093040in}}%
\pgfpathlineto{\pgfqpoint{2.844604in}{1.082254in}}%
\pgfpathlineto{\pgfqpoint{2.848662in}{1.077486in}}%
\pgfpathlineto{\pgfqpoint{2.856329in}{1.078344in}}%
\pgfpathlineto{\pgfqpoint{2.865348in}{1.076290in}}%
\pgfpathlineto{\pgfqpoint{2.868053in}{1.079601in}}%
\pgfpathlineto{\pgfqpoint{2.871210in}{1.082868in}}%
\pgfpathlineto{\pgfqpoint{2.873465in}{1.082833in}}%
\pgfpathlineto{\pgfqpoint{2.876171in}{1.080917in}}%
\pgfpathlineto{\pgfqpoint{2.879327in}{1.080257in}}%
\pgfpathlineto{\pgfqpoint{2.883386in}{1.081511in}}%
\pgfpathlineto{\pgfqpoint{2.885190in}{1.081547in}}%
\pgfpathlineto{\pgfqpoint{2.890601in}{1.076172in}}%
\pgfpathlineto{\pgfqpoint{2.900071in}{1.057271in}}%
\pgfpathlineto{\pgfqpoint{2.902326in}{1.058727in}}%
\pgfpathlineto{\pgfqpoint{2.908639in}{1.068296in}}%
\pgfpathlineto{\pgfqpoint{2.927579in}{1.074019in}}%
\pgfpathlineto{\pgfqpoint{2.933893in}{1.077349in}}%
\pgfpathlineto{\pgfqpoint{2.938853in}{1.072047in}}%
\pgfpathlineto{\pgfqpoint{2.946068in}{1.069958in}}%
\pgfpathlineto{\pgfqpoint{2.956440in}{1.071861in}}%
\pgfpathlineto{\pgfqpoint{2.965009in}{1.069717in}}%
\pgfpathlineto{\pgfqpoint{2.970871in}{1.061823in}}%
\pgfpathlineto{\pgfqpoint{2.974028in}{1.062025in}}%
\pgfpathlineto{\pgfqpoint{2.978537in}{1.067146in}}%
\pgfpathlineto{\pgfqpoint{2.981694in}{1.069530in}}%
\pgfpathlineto{\pgfqpoint{2.983949in}{1.068574in}}%
\pgfpathlineto{\pgfqpoint{2.989811in}{1.056789in}}%
\pgfpathlineto{\pgfqpoint{2.999281in}{1.031745in}}%
\pgfpathlineto{\pgfqpoint{3.002889in}{1.025530in}}%
\pgfpathlineto{\pgfqpoint{3.004692in}{1.025802in}}%
\pgfpathlineto{\pgfqpoint{3.006947in}{1.028315in}}%
\pgfpathlineto{\pgfqpoint{3.013712in}{1.041581in}}%
\pgfpathlineto{\pgfqpoint{3.015966in}{1.042713in}}%
\pgfpathlineto{\pgfqpoint{3.017770in}{1.041686in}}%
\pgfpathlineto{\pgfqpoint{3.020927in}{1.039123in}}%
\pgfpathlineto{\pgfqpoint{3.023633in}{1.041974in}}%
\pgfpathlineto{\pgfqpoint{3.029044in}{1.048923in}}%
\pgfpathlineto{\pgfqpoint{3.037161in}{1.052768in}}%
\pgfpathlineto{\pgfqpoint{3.038514in}{1.052729in}}%
\pgfpathlineto{\pgfqpoint{3.042573in}{1.055599in}}%
\pgfpathlineto{\pgfqpoint{3.045729in}{1.054109in}}%
\pgfpathlineto{\pgfqpoint{3.053396in}{1.046622in}}%
\pgfpathlineto{\pgfqpoint{3.056552in}{1.048222in}}%
\pgfpathlineto{\pgfqpoint{3.066022in}{1.046311in}}%
\pgfpathlineto{\pgfqpoint{3.074139in}{1.035374in}}%
\pgfpathlineto{\pgfqpoint{3.077747in}{1.037834in}}%
\pgfpathlineto{\pgfqpoint{3.087217in}{1.048027in}}%
\pgfpathlineto{\pgfqpoint{3.089472in}{1.046441in}}%
\pgfpathlineto{\pgfqpoint{3.095334in}{1.036234in}}%
\pgfpathlineto{\pgfqpoint{3.098491in}{1.037057in}}%
\pgfpathlineto{\pgfqpoint{3.104353in}{1.037354in}}%
\pgfpathlineto{\pgfqpoint{3.110667in}{1.034461in}}%
\pgfpathlineto{\pgfqpoint{3.112471in}{1.036674in}}%
\pgfpathlineto{\pgfqpoint{3.121490in}{1.048704in}}%
\pgfpathlineto{\pgfqpoint{3.123293in}{1.048020in}}%
\pgfpathlineto{\pgfqpoint{3.129156in}{1.043378in}}%
\pgfpathlineto{\pgfqpoint{3.132763in}{1.042062in}}%
\pgfpathlineto{\pgfqpoint{3.142684in}{1.044060in}}%
\pgfpathlineto{\pgfqpoint{3.147194in}{1.044232in}}%
\pgfpathlineto{\pgfqpoint{3.148998in}{1.044093in}}%
\pgfpathlineto{\pgfqpoint{3.155311in}{1.038398in}}%
\pgfpathlineto{\pgfqpoint{3.165232in}{1.032933in}}%
\pgfpathlineto{\pgfqpoint{3.166585in}{1.035029in}}%
\pgfpathlineto{\pgfqpoint{3.169742in}{1.039988in}}%
\pgfpathlineto{\pgfqpoint{3.172898in}{1.041557in}}%
\pgfpathlineto{\pgfqpoint{3.180114in}{1.038366in}}%
\pgfpathlineto{\pgfqpoint{3.184623in}{1.032718in}}%
\pgfpathlineto{\pgfqpoint{3.192740in}{1.020396in}}%
\pgfpathlineto{\pgfqpoint{3.193191in}{1.020724in}}%
\pgfpathlineto{\pgfqpoint{3.197701in}{1.022643in}}%
\pgfpathlineto{\pgfqpoint{3.203563in}{1.028645in}}%
\pgfpathlineto{\pgfqpoint{3.211229in}{1.027615in}}%
\pgfpathlineto{\pgfqpoint{3.214386in}{1.024325in}}%
\pgfpathlineto{\pgfqpoint{3.216190in}{1.022664in}}%
\pgfpathlineto{\pgfqpoint{3.219347in}{1.018984in}}%
\pgfpathlineto{\pgfqpoint{3.222052in}{1.020080in}}%
\pgfpathlineto{\pgfqpoint{3.224307in}{1.019906in}}%
\pgfpathlineto{\pgfqpoint{3.227464in}{1.018155in}}%
\pgfpathlineto{\pgfqpoint{3.233777in}{1.013352in}}%
\pgfpathlineto{\pgfqpoint{3.235581in}{1.013702in}}%
\pgfpathlineto{\pgfqpoint{3.239189in}{1.015416in}}%
\pgfpathlineto{\pgfqpoint{3.243698in}{1.013839in}}%
\pgfpathlineto{\pgfqpoint{3.249561in}{1.019560in}}%
\pgfpathlineto{\pgfqpoint{3.251364in}{1.018500in}}%
\pgfpathlineto{\pgfqpoint{3.253168in}{1.016449in}}%
\pgfpathlineto{\pgfqpoint{3.253619in}{1.016747in}}%
\pgfpathlineto{\pgfqpoint{3.256325in}{1.016462in}}%
\pgfpathlineto{\pgfqpoint{3.259031in}{1.016014in}}%
\pgfpathlineto{\pgfqpoint{3.265795in}{1.009920in}}%
\pgfpathlineto{\pgfqpoint{3.271206in}{1.013337in}}%
\pgfpathlineto{\pgfqpoint{3.273912in}{1.011091in}}%
\pgfpathlineto{\pgfqpoint{3.276167in}{1.010479in}}%
\pgfpathlineto{\pgfqpoint{3.278873in}{1.009731in}}%
\pgfpathlineto{\pgfqpoint{3.286088in}{1.005142in}}%
\pgfpathlineto{\pgfqpoint{3.288794in}{0.998377in}}%
\pgfpathlineto{\pgfqpoint{3.294205in}{0.984560in}}%
\pgfpathlineto{\pgfqpoint{3.296911in}{0.980857in}}%
\pgfpathlineto{\pgfqpoint{3.299166in}{0.980892in}}%
\pgfpathlineto{\pgfqpoint{3.304577in}{0.987513in}}%
\pgfpathlineto{\pgfqpoint{3.311341in}{0.996001in}}%
\pgfpathlineto{\pgfqpoint{3.316753in}{0.994908in}}%
\pgfpathlineto{\pgfqpoint{3.328478in}{0.988904in}}%
\pgfpathlineto{\pgfqpoint{3.332536in}{0.985753in}}%
\pgfpathlineto{\pgfqpoint{3.335693in}{0.986257in}}%
\pgfpathlineto{\pgfqpoint{3.338399in}{0.987943in}}%
\pgfpathlineto{\pgfqpoint{3.342908in}{0.992586in}}%
\pgfpathlineto{\pgfqpoint{3.349221in}{0.992008in}}%
\pgfpathlineto{\pgfqpoint{3.353731in}{0.996608in}}%
\pgfpathlineto{\pgfqpoint{3.357790in}{0.998454in}}%
\pgfpathlineto{\pgfqpoint{3.359593in}{0.997222in}}%
\pgfpathlineto{\pgfqpoint{3.364554in}{0.991702in}}%
\pgfpathlineto{\pgfqpoint{3.368162in}{0.987684in}}%
\pgfpathlineto{\pgfqpoint{3.372220in}{0.984834in}}%
\pgfpathlineto{\pgfqpoint{3.376730in}{0.985847in}}%
\pgfpathlineto{\pgfqpoint{3.379886in}{0.981990in}}%
\pgfpathlineto{\pgfqpoint{3.385298in}{0.975975in}}%
\pgfpathlineto{\pgfqpoint{3.391611in}{0.976984in}}%
\pgfpathlineto{\pgfqpoint{3.394317in}{0.975497in}}%
\pgfpathlineto{\pgfqpoint{3.399277in}{0.975569in}}%
\pgfpathlineto{\pgfqpoint{3.408296in}{0.985605in}}%
\pgfpathlineto{\pgfqpoint{3.412355in}{0.986622in}}%
\pgfpathlineto{\pgfqpoint{3.415061in}{0.986683in}}%
\pgfpathlineto{\pgfqpoint{3.418217in}{0.985965in}}%
\pgfpathlineto{\pgfqpoint{3.422276in}{0.985978in}}%
\pgfpathlineto{\pgfqpoint{3.425433in}{0.987034in}}%
\pgfpathlineto{\pgfqpoint{3.436707in}{0.979271in}}%
\pgfpathlineto{\pgfqpoint{3.441216in}{0.972158in}}%
\pgfpathlineto{\pgfqpoint{3.445275in}{0.968415in}}%
\pgfpathlineto{\pgfqpoint{3.451137in}{0.958305in}}%
\pgfpathlineto{\pgfqpoint{3.453843in}{0.958400in}}%
\pgfpathlineto{\pgfqpoint{3.457450in}{0.960394in}}%
\pgfpathlineto{\pgfqpoint{3.460607in}{0.959962in}}%
\pgfpathlineto{\pgfqpoint{3.464666in}{0.958972in}}%
\pgfpathlineto{\pgfqpoint{3.468273in}{0.957810in}}%
\pgfpathlineto{\pgfqpoint{3.473685in}{0.960838in}}%
\pgfpathlineto{\pgfqpoint{3.475940in}{0.961909in}}%
\pgfpathlineto{\pgfqpoint{3.481802in}{0.961891in}}%
\pgfpathlineto{\pgfqpoint{3.487664in}{0.972413in}}%
\pgfpathlineto{\pgfqpoint{3.490370in}{0.973515in}}%
\pgfpathlineto{\pgfqpoint{3.493978in}{0.973988in}}%
\pgfpathlineto{\pgfqpoint{3.497134in}{0.973864in}}%
\pgfpathlineto{\pgfqpoint{3.504350in}{0.968937in}}%
\pgfpathlineto{\pgfqpoint{3.510663in}{0.956311in}}%
\pgfpathlineto{\pgfqpoint{3.513820in}{0.951544in}}%
\pgfpathlineto{\pgfqpoint{3.516976in}{0.951504in}}%
\pgfpathlineto{\pgfqpoint{3.520133in}{0.952722in}}%
\pgfpathlineto{\pgfqpoint{3.522388in}{0.950487in}}%
\pgfpathlineto{\pgfqpoint{3.525996in}{0.945010in}}%
\pgfpathlineto{\pgfqpoint{3.534113in}{0.933978in}}%
\pgfpathlineto{\pgfqpoint{3.540877in}{0.930248in}}%
\pgfpathlineto{\pgfqpoint{3.550798in}{0.939203in}}%
\pgfpathlineto{\pgfqpoint{3.555308in}{0.946651in}}%
\pgfpathlineto{\pgfqpoint{3.560719in}{0.951141in}}%
\pgfpathlineto{\pgfqpoint{3.568385in}{0.959073in}}%
\pgfpathlineto{\pgfqpoint{3.569738in}{0.959445in}}%
\pgfpathlineto{\pgfqpoint{3.572895in}{0.960979in}}%
\pgfpathlineto{\pgfqpoint{3.579659in}{0.958292in}}%
\pgfpathlineto{\pgfqpoint{3.583267in}{0.956002in}}%
\pgfpathlineto{\pgfqpoint{3.587325in}{0.956621in}}%
\pgfpathlineto{\pgfqpoint{3.590031in}{0.957180in}}%
\pgfpathlineto{\pgfqpoint{3.594090in}{0.960676in}}%
\pgfpathlineto{\pgfqpoint{3.595893in}{0.959435in}}%
\pgfpathlineto{\pgfqpoint{3.600403in}{0.955808in}}%
\pgfpathlineto{\pgfqpoint{3.603109in}{0.956483in}}%
\pgfpathlineto{\pgfqpoint{3.608069in}{0.959006in}}%
\pgfpathlineto{\pgfqpoint{3.611226in}{0.957013in}}%
\pgfpathlineto{\pgfqpoint{3.616186in}{0.947962in}}%
\pgfpathlineto{\pgfqpoint{3.622500in}{0.933866in}}%
\pgfpathlineto{\pgfqpoint{3.629264in}{0.932521in}}%
\pgfpathlineto{\pgfqpoint{3.634675in}{0.939217in}}%
\pgfpathlineto{\pgfqpoint{3.637381in}{0.941808in}}%
\pgfpathlineto{\pgfqpoint{3.640538in}{0.940654in}}%
\pgfpathlineto{\pgfqpoint{3.647302in}{0.936104in}}%
\pgfpathlineto{\pgfqpoint{3.652263in}{0.937337in}}%
\pgfpathlineto{\pgfqpoint{3.660380in}{0.931438in}}%
\pgfpathlineto{\pgfqpoint{3.663086in}{0.932359in}}%
\pgfpathlineto{\pgfqpoint{3.667595in}{0.933888in}}%
\pgfpathlineto{\pgfqpoint{3.676163in}{0.927805in}}%
\pgfpathlineto{\pgfqpoint{3.680222in}{0.923247in}}%
\pgfpathlineto{\pgfqpoint{3.686986in}{0.915734in}}%
\pgfpathlineto{\pgfqpoint{3.690594in}{0.912374in}}%
\pgfpathlineto{\pgfqpoint{3.697358in}{0.906979in}}%
\pgfpathlineto{\pgfqpoint{3.698711in}{0.907806in}}%
\pgfpathlineto{\pgfqpoint{3.702319in}{0.909846in}}%
\pgfpathlineto{\pgfqpoint{3.705926in}{0.904946in}}%
\pgfpathlineto{\pgfqpoint{3.712691in}{0.896946in}}%
\pgfpathlineto{\pgfqpoint{3.723062in}{0.902358in}}%
\pgfpathlineto{\pgfqpoint{3.725768in}{0.907026in}}%
\pgfpathlineto{\pgfqpoint{3.729827in}{0.917073in}}%
\pgfpathlineto{\pgfqpoint{3.734336in}{0.925721in}}%
\pgfpathlineto{\pgfqpoint{3.742003in}{0.933661in}}%
\pgfpathlineto{\pgfqpoint{3.743806in}{0.932405in}}%
\pgfpathlineto{\pgfqpoint{3.746061in}{0.928544in}}%
\pgfpathlineto{\pgfqpoint{3.754629in}{0.909126in}}%
\pgfpathlineto{\pgfqpoint{3.755982in}{0.909349in}}%
\pgfpathlineto{\pgfqpoint{3.764099in}{0.919856in}}%
\pgfpathlineto{\pgfqpoint{3.767256in}{0.916479in}}%
\pgfpathlineto{\pgfqpoint{3.772216in}{0.910401in}}%
\pgfpathlineto{\pgfqpoint{3.774922in}{0.909318in}}%
\pgfpathlineto{\pgfqpoint{3.777628in}{0.909576in}}%
\pgfpathlineto{\pgfqpoint{3.796568in}{0.897195in}}%
\pgfpathlineto{\pgfqpoint{3.805587in}{0.881826in}}%
\pgfpathlineto{\pgfqpoint{3.807842in}{0.881362in}}%
\pgfpathlineto{\pgfqpoint{3.809646in}{0.883921in}}%
\pgfpathlineto{\pgfqpoint{3.817763in}{0.898658in}}%
\pgfpathlineto{\pgfqpoint{3.822272in}{0.902440in}}%
\pgfpathlineto{\pgfqpoint{3.825429in}{0.906581in}}%
\pgfpathlineto{\pgfqpoint{3.831742in}{0.913060in}}%
\pgfpathlineto{\pgfqpoint{3.834899in}{0.916258in}}%
\pgfpathlineto{\pgfqpoint{3.837605in}{0.915659in}}%
\pgfpathlineto{\pgfqpoint{3.842114in}{0.913725in}}%
\pgfpathlineto{\pgfqpoint{3.845271in}{0.913782in}}%
\pgfpathlineto{\pgfqpoint{3.847526in}{0.912033in}}%
\pgfpathlineto{\pgfqpoint{3.849781in}{0.907202in}}%
\pgfpathlineto{\pgfqpoint{3.854741in}{0.895993in}}%
\pgfpathlineto{\pgfqpoint{3.859251in}{0.891471in}}%
\pgfpathlineto{\pgfqpoint{3.870525in}{0.888988in}}%
\pgfpathlineto{\pgfqpoint{3.882249in}{0.881907in}}%
\pgfpathlineto{\pgfqpoint{3.890817in}{0.873838in}}%
\pgfpathlineto{\pgfqpoint{3.895327in}{0.872931in}}%
\pgfpathlineto{\pgfqpoint{3.912914in}{0.872001in}}%
\pgfpathlineto{\pgfqpoint{3.915169in}{0.875844in}}%
\pgfpathlineto{\pgfqpoint{3.920580in}{0.883993in}}%
\pgfpathlineto{\pgfqpoint{3.923286in}{0.885290in}}%
\pgfpathlineto{\pgfqpoint{3.925992in}{0.885766in}}%
\pgfpathlineto{\pgfqpoint{3.930952in}{0.884328in}}%
\pgfpathlineto{\pgfqpoint{3.934109in}{0.885257in}}%
\pgfpathlineto{\pgfqpoint{3.936815in}{0.887735in}}%
\pgfpathlineto{\pgfqpoint{3.943579in}{0.893225in}}%
\pgfpathlineto{\pgfqpoint{3.947638in}{0.889528in}}%
\pgfpathlineto{\pgfqpoint{3.962519in}{0.874397in}}%
\pgfpathlineto{\pgfqpoint{3.965676in}{0.874418in}}%
\pgfpathlineto{\pgfqpoint{3.969734in}{0.875778in}}%
\pgfpathlineto{\pgfqpoint{3.972891in}{0.873484in}}%
\pgfpathlineto{\pgfqpoint{3.978754in}{0.868924in}}%
\pgfpathlineto{\pgfqpoint{3.984616in}{0.869173in}}%
\pgfpathlineto{\pgfqpoint{3.990929in}{0.875451in}}%
\pgfpathlineto{\pgfqpoint{3.998145in}{0.882833in}}%
\pgfpathlineto{\pgfqpoint{4.001301in}{0.881660in}}%
\pgfpathlineto{\pgfqpoint{4.002654in}{0.882004in}}%
\pgfpathlineto{\pgfqpoint{4.012575in}{0.886847in}}%
\pgfpathlineto{\pgfqpoint{4.015281in}{0.884433in}}%
\pgfpathlineto{\pgfqpoint{4.021143in}{0.876567in}}%
\pgfpathlineto{\pgfqpoint{4.024751in}{0.875674in}}%
\pgfpathlineto{\pgfqpoint{4.027006in}{0.874810in}}%
\pgfpathlineto{\pgfqpoint{4.032417in}{0.870425in}}%
\pgfpathlineto{\pgfqpoint{4.036025in}{0.868736in}}%
\pgfpathlineto{\pgfqpoint{4.040083in}{0.866445in}}%
\pgfpathlineto{\pgfqpoint{4.048200in}{0.873030in}}%
\pgfpathlineto{\pgfqpoint{4.050004in}{0.871536in}}%
\pgfpathlineto{\pgfqpoint{4.056769in}{0.859366in}}%
\pgfpathlineto{\pgfqpoint{4.059474in}{0.856823in}}%
\pgfpathlineto{\pgfqpoint{4.063533in}{0.857521in}}%
\pgfpathlineto{\pgfqpoint{4.065788in}{0.856985in}}%
\pgfpathlineto{\pgfqpoint{4.067591in}{0.856562in}}%
\pgfpathlineto{\pgfqpoint{4.077512in}{0.861705in}}%
\pgfpathlineto{\pgfqpoint{4.080218in}{0.860387in}}%
\pgfpathlineto{\pgfqpoint{4.083375in}{0.857078in}}%
\pgfpathlineto{\pgfqpoint{4.090139in}{0.846249in}}%
\pgfpathlineto{\pgfqpoint{4.093747in}{0.844925in}}%
\pgfpathlineto{\pgfqpoint{4.096453in}{0.846273in}}%
\pgfpathlineto{\pgfqpoint{4.101413in}{0.852634in}}%
\pgfpathlineto{\pgfqpoint{4.108628in}{0.855719in}}%
\pgfpathlineto{\pgfqpoint{4.111334in}{0.854265in}}%
\pgfpathlineto{\pgfqpoint{4.115393in}{0.852198in}}%
\pgfpathlineto{\pgfqpoint{4.119451in}{0.854752in}}%
\pgfpathlineto{\pgfqpoint{4.126666in}{0.859523in}}%
\pgfpathlineto{\pgfqpoint{4.130274in}{0.856994in}}%
\pgfpathlineto{\pgfqpoint{4.136137in}{0.851937in}}%
\pgfpathlineto{\pgfqpoint{4.141097in}{0.844429in}}%
\pgfpathlineto{\pgfqpoint{4.151018in}{0.838713in}}%
\pgfpathlineto{\pgfqpoint{4.156880in}{0.832430in}}%
\pgfpathlineto{\pgfqpoint{4.159135in}{0.831308in}}%
\pgfpathlineto{\pgfqpoint{4.164547in}{0.832428in}}%
\pgfpathlineto{\pgfqpoint{4.168605in}{0.836726in}}%
\pgfpathlineto{\pgfqpoint{4.174017in}{0.845607in}}%
\pgfpathlineto{\pgfqpoint{4.178977in}{0.852934in}}%
\pgfpathlineto{\pgfqpoint{4.183036in}{0.856432in}}%
\pgfpathlineto{\pgfqpoint{4.186192in}{0.857642in}}%
\pgfpathlineto{\pgfqpoint{4.189349in}{0.854983in}}%
\pgfpathlineto{\pgfqpoint{4.198368in}{0.842103in}}%
\pgfpathlineto{\pgfqpoint{4.201525in}{0.841568in}}%
\pgfpathlineto{\pgfqpoint{4.207387in}{0.838597in}}%
\pgfpathlineto{\pgfqpoint{4.210093in}{0.837653in}}%
\pgfpathlineto{\pgfqpoint{4.217308in}{0.831535in}}%
\pgfpathlineto{\pgfqpoint{4.219112in}{0.832638in}}%
\pgfpathlineto{\pgfqpoint{4.221818in}{0.834297in}}%
\pgfpathlineto{\pgfqpoint{4.229033in}{0.832615in}}%
\pgfpathlineto{\pgfqpoint{4.233543in}{0.827876in}}%
\pgfpathlineto{\pgfqpoint{4.236699in}{0.824693in}}%
\pgfpathlineto{\pgfqpoint{4.251581in}{0.824846in}}%
\pgfpathlineto{\pgfqpoint{4.257894in}{0.832832in}}%
\pgfpathlineto{\pgfqpoint{4.268266in}{0.847829in}}%
\pgfpathlineto{\pgfqpoint{4.272776in}{0.849982in}}%
\pgfpathlineto{\pgfqpoint{4.276383in}{0.848132in}}%
\pgfpathlineto{\pgfqpoint{4.279089in}{0.845527in}}%
\pgfpathlineto{\pgfqpoint{4.287657in}{0.835911in}}%
\pgfpathlineto{\pgfqpoint{4.290814in}{0.832786in}}%
\pgfpathlineto{\pgfqpoint{4.293520in}{0.830855in}}%
\pgfpathlineto{\pgfqpoint{4.302539in}{0.827551in}}%
\pgfpathlineto{\pgfqpoint{4.306146in}{0.825785in}}%
\pgfpathlineto{\pgfqpoint{4.311558in}{0.824907in}}%
\pgfpathlineto{\pgfqpoint{4.316067in}{0.827876in}}%
\pgfpathlineto{\pgfqpoint{4.319224in}{0.830042in}}%
\pgfpathlineto{\pgfqpoint{4.325988in}{0.829532in}}%
\pgfpathlineto{\pgfqpoint{4.328694in}{0.827160in}}%
\pgfpathlineto{\pgfqpoint{4.333204in}{0.821874in}}%
\pgfpathlineto{\pgfqpoint{4.338615in}{0.818320in}}%
\pgfpathlineto{\pgfqpoint{4.341772in}{0.817508in}}%
\pgfpathlineto{\pgfqpoint{4.345830in}{0.817307in}}%
\pgfpathlineto{\pgfqpoint{4.348085in}{0.820451in}}%
\pgfpathlineto{\pgfqpoint{4.361163in}{0.843109in}}%
\pgfpathlineto{\pgfqpoint{4.368378in}{0.847060in}}%
\pgfpathlineto{\pgfqpoint{4.371986in}{0.845884in}}%
\pgfpathlineto{\pgfqpoint{4.374691in}{0.844010in}}%
\pgfpathlineto{\pgfqpoint{4.378299in}{0.837991in}}%
\pgfpathlineto{\pgfqpoint{4.385965in}{0.823410in}}%
\pgfpathlineto{\pgfqpoint{4.391377in}{0.817417in}}%
\pgfpathlineto{\pgfqpoint{4.395886in}{0.815958in}}%
\pgfpathlineto{\pgfqpoint{4.401749in}{0.814782in}}%
\pgfpathlineto{\pgfqpoint{4.404454in}{0.817567in}}%
\pgfpathlineto{\pgfqpoint{4.415277in}{0.829443in}}%
\pgfpathlineto{\pgfqpoint{4.419787in}{0.829572in}}%
\pgfpathlineto{\pgfqpoint{4.423845in}{0.827914in}}%
\pgfpathlineto{\pgfqpoint{4.427453in}{0.826405in}}%
\pgfpathlineto{\pgfqpoint{4.440080in}{0.824364in}}%
\pgfpathlineto{\pgfqpoint{4.446844in}{0.829615in}}%
\pgfpathlineto{\pgfqpoint{4.450452in}{0.828887in}}%
\pgfpathlineto{\pgfqpoint{4.459020in}{0.825301in}}%
\pgfpathlineto{\pgfqpoint{4.467137in}{0.828085in}}%
\pgfpathlineto{\pgfqpoint{4.469392in}{0.826151in}}%
\pgfpathlineto{\pgfqpoint{4.475705in}{0.815670in}}%
\pgfpathlineto{\pgfqpoint{4.481567in}{0.806485in}}%
\pgfpathlineto{\pgfqpoint{4.491037in}{0.795438in}}%
\pgfpathlineto{\pgfqpoint{4.499155in}{0.797860in}}%
\pgfpathlineto{\pgfqpoint{4.501860in}{0.800275in}}%
\pgfpathlineto{\pgfqpoint{4.505468in}{0.806432in}}%
\pgfpathlineto{\pgfqpoint{4.514036in}{0.821098in}}%
\pgfpathlineto{\pgfqpoint{4.525761in}{0.828578in}}%
\pgfpathlineto{\pgfqpoint{4.534780in}{0.821975in}}%
\pgfpathlineto{\pgfqpoint{4.543799in}{0.809888in}}%
\pgfpathlineto{\pgfqpoint{4.553720in}{0.802846in}}%
\pgfpathlineto{\pgfqpoint{4.560484in}{0.805429in}}%
\pgfpathlineto{\pgfqpoint{4.564092in}{0.806781in}}%
\pgfpathlineto{\pgfqpoint{4.569954in}{0.808912in}}%
\pgfpathlineto{\pgfqpoint{4.575817in}{0.816142in}}%
\pgfpathlineto{\pgfqpoint{4.578523in}{0.817282in}}%
\pgfpathlineto{\pgfqpoint{4.581679in}{0.815228in}}%
\pgfpathlineto{\pgfqpoint{4.585287in}{0.811207in}}%
\pgfpathlineto{\pgfqpoint{4.590698in}{0.806311in}}%
\pgfpathlineto{\pgfqpoint{4.593855in}{0.806206in}}%
\pgfpathlineto{\pgfqpoint{4.596561in}{0.804466in}}%
\pgfpathlineto{\pgfqpoint{4.604678in}{0.799034in}}%
\pgfpathlineto{\pgfqpoint{4.610540in}{0.797546in}}%
\pgfpathlineto{\pgfqpoint{4.614599in}{0.795108in}}%
\pgfpathlineto{\pgfqpoint{4.617305in}{0.795765in}}%
\pgfpathlineto{\pgfqpoint{4.620010in}{0.793267in}}%
\pgfpathlineto{\pgfqpoint{4.627677in}{0.786853in}}%
\pgfpathlineto{\pgfqpoint{4.630382in}{0.786097in}}%
\pgfpathlineto{\pgfqpoint{4.634441in}{0.785752in}}%
\pgfpathlineto{\pgfqpoint{4.642558in}{0.783468in}}%
\pgfpathlineto{\pgfqpoint{4.644813in}{0.783358in}}%
\pgfpathlineto{\pgfqpoint{4.652479in}{0.789323in}}%
\pgfpathlineto{\pgfqpoint{4.659694in}{0.795308in}}%
\pgfpathlineto{\pgfqpoint{4.665557in}{0.800471in}}%
\pgfpathlineto{\pgfqpoint{4.670066in}{0.802128in}}%
\pgfpathlineto{\pgfqpoint{4.674576in}{0.802904in}}%
\pgfpathlineto{\pgfqpoint{4.679085in}{0.800093in}}%
\pgfpathlineto{\pgfqpoint{4.682693in}{0.802326in}}%
\pgfpathlineto{\pgfqpoint{4.690810in}{0.812859in}}%
\pgfpathlineto{\pgfqpoint{4.694418in}{0.813448in}}%
\pgfpathlineto{\pgfqpoint{4.703888in}{0.813813in}}%
\pgfpathlineto{\pgfqpoint{4.710201in}{0.810718in}}%
\pgfpathlineto{\pgfqpoint{4.714711in}{0.808254in}}%
\pgfpathlineto{\pgfqpoint{4.718769in}{0.806609in}}%
\pgfpathlineto{\pgfqpoint{4.722377in}{0.805661in}}%
\pgfpathlineto{\pgfqpoint{4.725083in}{0.803776in}}%
\pgfpathlineto{\pgfqpoint{4.728239in}{0.802646in}}%
\pgfpathlineto{\pgfqpoint{4.734553in}{0.803508in}}%
\pgfpathlineto{\pgfqpoint{4.736807in}{0.802487in}}%
\pgfpathlineto{\pgfqpoint{4.739062in}{0.802074in}}%
\pgfpathlineto{\pgfqpoint{4.746728in}{0.804451in}}%
\pgfpathlineto{\pgfqpoint{4.760257in}{0.802503in}}%
\pgfpathlineto{\pgfqpoint{4.763414in}{0.801618in}}%
\pgfpathlineto{\pgfqpoint{4.765669in}{0.800042in}}%
\pgfpathlineto{\pgfqpoint{4.773335in}{0.794348in}}%
\pgfpathlineto{\pgfqpoint{4.778295in}{0.793048in}}%
\pgfpathlineto{\pgfqpoint{4.785060in}{0.795219in}}%
\pgfpathlineto{\pgfqpoint{4.789569in}{0.793449in}}%
\pgfpathlineto{\pgfqpoint{4.801294in}{0.787169in}}%
\pgfpathlineto{\pgfqpoint{4.809411in}{0.788490in}}%
\pgfpathlineto{\pgfqpoint{4.816175in}{0.787531in}}%
\pgfpathlineto{\pgfqpoint{4.820234in}{0.784736in}}%
\pgfpathlineto{\pgfqpoint{4.826998in}{0.778905in}}%
\pgfpathlineto{\pgfqpoint{4.829704in}{0.776666in}}%
\pgfpathlineto{\pgfqpoint{4.832861in}{0.774595in}}%
\pgfpathlineto{\pgfqpoint{4.835116in}{0.774581in}}%
\pgfpathlineto{\pgfqpoint{4.843233in}{0.776636in}}%
\pgfpathlineto{\pgfqpoint{4.855408in}{0.776673in}}%
\pgfpathlineto{\pgfqpoint{4.865329in}{0.768517in}}%
\pgfpathlineto{\pgfqpoint{4.869388in}{0.762310in}}%
\pgfpathlineto{\pgfqpoint{4.872545in}{0.760907in}}%
\pgfpathlineto{\pgfqpoint{4.875250in}{0.762450in}}%
\pgfpathlineto{\pgfqpoint{4.882466in}{0.768542in}}%
\pgfpathlineto{\pgfqpoint{4.886073in}{0.772522in}}%
\pgfpathlineto{\pgfqpoint{4.894641in}{0.782015in}}%
\pgfpathlineto{\pgfqpoint{4.899602in}{0.783805in}}%
\pgfpathlineto{\pgfqpoint{4.903661in}{0.785224in}}%
\pgfpathlineto{\pgfqpoint{4.905915in}{0.785355in}}%
\pgfpathlineto{\pgfqpoint{4.910425in}{0.782746in}}%
\pgfpathlineto{\pgfqpoint{4.914934in}{0.774666in}}%
\pgfpathlineto{\pgfqpoint{4.919444in}{0.768238in}}%
\pgfpathlineto{\pgfqpoint{4.922601in}{0.766791in}}%
\pgfpathlineto{\pgfqpoint{4.925306in}{0.765616in}}%
\pgfpathlineto{\pgfqpoint{4.932071in}{0.763917in}}%
\pgfpathlineto{\pgfqpoint{4.939737in}{0.761275in}}%
\pgfpathlineto{\pgfqpoint{4.944697in}{0.760156in}}%
\pgfpathlineto{\pgfqpoint{4.954167in}{0.763331in}}%
\pgfpathlineto{\pgfqpoint{4.957324in}{0.764252in}}%
\pgfpathlineto{\pgfqpoint{4.962736in}{0.764506in}}%
\pgfpathlineto{\pgfqpoint{4.965892in}{0.764010in}}%
\pgfpathlineto{\pgfqpoint{4.971304in}{0.762954in}}%
\pgfpathlineto{\pgfqpoint{4.975362in}{0.759384in}}%
\pgfpathlineto{\pgfqpoint{4.984381in}{0.749005in}}%
\pgfpathlineto{\pgfqpoint{4.991146in}{0.747037in}}%
\pgfpathlineto{\pgfqpoint{4.999714in}{0.747349in}}%
\pgfpathlineto{\pgfqpoint{5.002420in}{0.749498in}}%
\pgfpathlineto{\pgfqpoint{5.009184in}{0.754372in}}%
\pgfpathlineto{\pgfqpoint{5.012791in}{0.756787in}}%
\pgfpathlineto{\pgfqpoint{5.018654in}{0.761400in}}%
\pgfpathlineto{\pgfqpoint{5.022261in}{0.761681in}}%
\pgfpathlineto{\pgfqpoint{5.032182in}{0.758510in}}%
\pgfpathlineto{\pgfqpoint{5.034888in}{0.759766in}}%
\pgfpathlineto{\pgfqpoint{5.037143in}{0.763421in}}%
\pgfpathlineto{\pgfqpoint{5.044358in}{0.775694in}}%
\pgfpathlineto{\pgfqpoint{5.047966in}{0.775668in}}%
\pgfpathlineto{\pgfqpoint{5.056534in}{0.772726in}}%
\pgfpathlineto{\pgfqpoint{5.061044in}{0.767282in}}%
\pgfpathlineto{\pgfqpoint{5.065553in}{0.758390in}}%
\pgfpathlineto{\pgfqpoint{5.069612in}{0.752135in}}%
\pgfpathlineto{\pgfqpoint{5.071416in}{0.751310in}}%
\pgfpathlineto{\pgfqpoint{5.074121in}{0.752244in}}%
\pgfpathlineto{\pgfqpoint{5.080886in}{0.757222in}}%
\pgfpathlineto{\pgfqpoint{5.083591in}{0.756998in}}%
\pgfpathlineto{\pgfqpoint{5.090356in}{0.752364in}}%
\pgfpathlineto{\pgfqpoint{5.095767in}{0.752028in}}%
\pgfpathlineto{\pgfqpoint{5.106590in}{0.759536in}}%
\pgfpathlineto{\pgfqpoint{5.115609in}{0.762366in}}%
\pgfpathlineto{\pgfqpoint{5.120119in}{0.767848in}}%
\pgfpathlineto{\pgfqpoint{5.123726in}{0.770627in}}%
\pgfpathlineto{\pgfqpoint{5.126883in}{0.769672in}}%
\pgfpathlineto{\pgfqpoint{5.133647in}{0.767515in}}%
\pgfpathlineto{\pgfqpoint{5.138608in}{0.770996in}}%
\pgfpathlineto{\pgfqpoint{5.143568in}{0.775293in}}%
\pgfpathlineto{\pgfqpoint{5.147627in}{0.774588in}}%
\pgfpathlineto{\pgfqpoint{5.154842in}{0.769068in}}%
\pgfpathlineto{\pgfqpoint{5.169724in}{0.747693in}}%
\pgfpathlineto{\pgfqpoint{5.177841in}{0.733384in}}%
\pgfpathlineto{\pgfqpoint{5.180095in}{0.732618in}}%
\pgfpathlineto{\pgfqpoint{5.185056in}{0.732269in}}%
\pgfpathlineto{\pgfqpoint{5.190467in}{0.729735in}}%
\pgfpathlineto{\pgfqpoint{5.198134in}{0.729598in}}%
\pgfpathlineto{\pgfqpoint{5.202643in}{0.730682in}}%
\pgfpathlineto{\pgfqpoint{5.206702in}{0.729957in}}%
\pgfpathlineto{\pgfqpoint{5.211662in}{0.724982in}}%
\pgfpathlineto{\pgfqpoint{5.215721in}{0.719753in}}%
\pgfpathlineto{\pgfqpoint{5.219328in}{0.719809in}}%
\pgfpathlineto{\pgfqpoint{5.222485in}{0.721017in}}%
\pgfpathlineto{\pgfqpoint{5.227897in}{0.726964in}}%
\pgfpathlineto{\pgfqpoint{5.235112in}{0.736148in}}%
\pgfpathlineto{\pgfqpoint{5.241425in}{0.741520in}}%
\pgfpathlineto{\pgfqpoint{5.245484in}{0.746484in}}%
\pgfpathlineto{\pgfqpoint{5.250444in}{0.748608in}}%
\pgfpathlineto{\pgfqpoint{5.253150in}{0.748699in}}%
\pgfpathlineto{\pgfqpoint{5.257209in}{0.744619in}}%
\pgfpathlineto{\pgfqpoint{5.273894in}{0.719582in}}%
\pgfpathlineto{\pgfqpoint{5.280207in}{0.714508in}}%
\pgfpathlineto{\pgfqpoint{5.283815in}{0.713842in}}%
\pgfpathlineto{\pgfqpoint{5.286070in}{0.717813in}}%
\pgfpathlineto{\pgfqpoint{5.292383in}{0.728212in}}%
\pgfpathlineto{\pgfqpoint{5.296893in}{0.732072in}}%
\pgfpathlineto{\pgfqpoint{5.303657in}{0.734722in}}%
\pgfpathlineto{\pgfqpoint{5.306363in}{0.732518in}}%
\pgfpathlineto{\pgfqpoint{5.311323in}{0.729846in}}%
\pgfpathlineto{\pgfqpoint{5.314931in}{0.731304in}}%
\pgfpathlineto{\pgfqpoint{5.320342in}{0.731772in}}%
\pgfpathlineto{\pgfqpoint{5.323499in}{0.732214in}}%
\pgfpathlineto{\pgfqpoint{5.326656in}{0.734095in}}%
\pgfpathlineto{\pgfqpoint{5.333420in}{0.740946in}}%
\pgfpathlineto{\pgfqpoint{5.342890in}{0.745594in}}%
\pgfpathlineto{\pgfqpoint{5.347850in}{0.744566in}}%
\pgfpathlineto{\pgfqpoint{5.352360in}{0.742376in}}%
\pgfpathlineto{\pgfqpoint{5.367692in}{0.735156in}}%
\pgfpathlineto{\pgfqpoint{5.371300in}{0.732888in}}%
\pgfpathlineto{\pgfqpoint{5.374908in}{0.729070in}}%
\pgfpathlineto{\pgfqpoint{5.381221in}{0.721993in}}%
\pgfpathlineto{\pgfqpoint{5.384378in}{0.720939in}}%
\pgfpathlineto{\pgfqpoint{5.388887in}{0.721855in}}%
\pgfpathlineto{\pgfqpoint{5.400612in}{0.732549in}}%
\pgfpathlineto{\pgfqpoint{5.407376in}{0.735279in}}%
\pgfpathlineto{\pgfqpoint{5.411435in}{0.741233in}}%
\pgfpathlineto{\pgfqpoint{5.417748in}{0.750290in}}%
\pgfpathlineto{\pgfqpoint{5.424513in}{0.755588in}}%
\pgfpathlineto{\pgfqpoint{5.431728in}{0.756577in}}%
\pgfpathlineto{\pgfqpoint{5.438492in}{0.756232in}}%
\pgfpathlineto{\pgfqpoint{5.450668in}{0.747732in}}%
\pgfpathlineto{\pgfqpoint{5.461491in}{0.745359in}}%
\pgfpathlineto{\pgfqpoint{5.464197in}{0.746182in}}%
\pgfpathlineto{\pgfqpoint{5.467353in}{0.748804in}}%
\pgfpathlineto{\pgfqpoint{5.470961in}{0.751350in}}%
\pgfpathlineto{\pgfqpoint{5.480431in}{0.755394in}}%
\pgfpathlineto{\pgfqpoint{5.488548in}{0.757978in}}%
\pgfpathlineto{\pgfqpoint{5.494411in}{0.758419in}}%
\pgfpathlineto{\pgfqpoint{5.498469in}{0.756396in}}%
\pgfpathlineto{\pgfqpoint{5.503430in}{0.753869in}}%
\pgfpathlineto{\pgfqpoint{5.510645in}{0.746996in}}%
\pgfpathlineto{\pgfqpoint{5.514703in}{0.745194in}}%
\pgfpathlineto{\pgfqpoint{5.522370in}{0.744400in}}%
\pgfpathlineto{\pgfqpoint{5.525977in}{0.742715in}}%
\pgfpathlineto{\pgfqpoint{5.530487in}{0.741810in}}%
\pgfpathlineto{\pgfqpoint{5.534545in}{0.738841in}}%
\pgfpathlineto{\pgfqpoint{5.534545in}{0.738841in}}%
\pgfusepath{stroke}%
\end{pgfscope}%
\begin{pgfscope}%
\pgfpathrectangle{\pgfqpoint{0.800000in}{0.528000in}}{\pgfqpoint{4.960000in}{3.696000in}} %
\pgfusepath{clip}%
\pgfsetrectcap%
\pgfsetroundjoin%
\pgfsetlinewidth{1.505625pt}%
\definecolor{currentstroke}{rgb}{1.000000,0.498039,0.054902}%
\pgfsetstrokecolor{currentstroke}%
\pgfsetdash{}{0pt}%
\pgfpathmoveto{\pgfqpoint{1.025455in}{3.928923in}}%
\pgfpathlineto{\pgfqpoint{1.026356in}{3.662321in}}%
\pgfpathlineto{\pgfqpoint{1.028611in}{2.850564in}}%
\pgfpathlineto{\pgfqpoint{1.029964in}{2.867250in}}%
\pgfpathlineto{\pgfqpoint{1.030415in}{2.867216in}}%
\pgfpathlineto{\pgfqpoint{1.031768in}{2.866181in}}%
\pgfpathlineto{\pgfqpoint{1.033572in}{2.861142in}}%
\pgfpathlineto{\pgfqpoint{1.037630in}{2.839851in}}%
\pgfpathlineto{\pgfqpoint{1.044395in}{2.781108in}}%
\pgfpathlineto{\pgfqpoint{1.047551in}{2.727872in}}%
\pgfpathlineto{\pgfqpoint{1.057472in}{2.495314in}}%
\pgfpathlineto{\pgfqpoint{1.058374in}{2.485640in}}%
\pgfpathlineto{\pgfqpoint{1.065589in}{2.363410in}}%
\pgfpathlineto{\pgfqpoint{1.067393in}{2.352749in}}%
\pgfpathlineto{\pgfqpoint{1.074158in}{2.280414in}}%
\pgfpathlineto{\pgfqpoint{1.075060in}{2.276201in}}%
\pgfpathlineto{\pgfqpoint{1.087235in}{2.153694in}}%
\pgfpathlineto{\pgfqpoint{1.088137in}{2.150790in}}%
\pgfpathlineto{\pgfqpoint{1.088588in}{2.150910in}}%
\pgfpathlineto{\pgfqpoint{1.089039in}{2.151362in}}%
\pgfpathlineto{\pgfqpoint{1.090843in}{2.141080in}}%
\pgfpathlineto{\pgfqpoint{1.094901in}{2.089562in}}%
\pgfpathlineto{\pgfqpoint{1.095352in}{2.090494in}}%
\pgfpathlineto{\pgfqpoint{1.095803in}{2.089789in}}%
\pgfpathlineto{\pgfqpoint{1.100313in}{2.063405in}}%
\pgfpathlineto{\pgfqpoint{1.100764in}{2.063513in}}%
\pgfpathlineto{\pgfqpoint{1.104822in}{2.051548in}}%
\pgfpathlineto{\pgfqpoint{1.112038in}{2.024875in}}%
\pgfpathlineto{\pgfqpoint{1.113391in}{2.025360in}}%
\pgfpathlineto{\pgfqpoint{1.118351in}{2.006567in}}%
\pgfpathlineto{\pgfqpoint{1.120606in}{1.999383in}}%
\pgfpathlineto{\pgfqpoint{1.121959in}{1.994950in}}%
\pgfpathlineto{\pgfqpoint{1.124664in}{1.980123in}}%
\pgfpathlineto{\pgfqpoint{1.126919in}{1.971422in}}%
\pgfpathlineto{\pgfqpoint{1.127370in}{1.971370in}}%
\pgfpathlineto{\pgfqpoint{1.130527in}{1.977015in}}%
\pgfpathlineto{\pgfqpoint{1.132331in}{1.973962in}}%
\pgfpathlineto{\pgfqpoint{1.133684in}{1.968795in}}%
\pgfpathlineto{\pgfqpoint{1.142252in}{1.928711in}}%
\pgfpathlineto{\pgfqpoint{1.145408in}{1.918243in}}%
\pgfpathlineto{\pgfqpoint{1.145859in}{1.918652in}}%
\pgfpathlineto{\pgfqpoint{1.147212in}{1.919567in}}%
\pgfpathlineto{\pgfqpoint{1.149918in}{1.915361in}}%
\pgfpathlineto{\pgfqpoint{1.158035in}{1.876476in}}%
\pgfpathlineto{\pgfqpoint{1.158486in}{1.876949in}}%
\pgfpathlineto{\pgfqpoint{1.161643in}{1.881457in}}%
\pgfpathlineto{\pgfqpoint{1.167054in}{1.869052in}}%
\pgfpathlineto{\pgfqpoint{1.172015in}{1.852270in}}%
\pgfpathlineto{\pgfqpoint{1.175622in}{1.838330in}}%
\pgfpathlineto{\pgfqpoint{1.176524in}{1.837618in}}%
\pgfpathlineto{\pgfqpoint{1.179681in}{1.828111in}}%
\pgfpathlineto{\pgfqpoint{1.183289in}{1.830030in}}%
\pgfpathlineto{\pgfqpoint{1.184641in}{1.828352in}}%
\pgfpathlineto{\pgfqpoint{1.185543in}{1.826663in}}%
\pgfpathlineto{\pgfqpoint{1.185994in}{1.827022in}}%
\pgfpathlineto{\pgfqpoint{1.186896in}{1.826483in}}%
\pgfpathlineto{\pgfqpoint{1.188249in}{1.822587in}}%
\pgfpathlineto{\pgfqpoint{1.193210in}{1.806594in}}%
\pgfpathlineto{\pgfqpoint{1.193660in}{1.807116in}}%
\pgfpathlineto{\pgfqpoint{1.194562in}{1.807612in}}%
\pgfpathlineto{\pgfqpoint{1.204934in}{1.794600in}}%
\pgfpathlineto{\pgfqpoint{1.205836in}{1.793659in}}%
\pgfpathlineto{\pgfqpoint{1.209444in}{1.784272in}}%
\pgfpathlineto{\pgfqpoint{1.210346in}{1.785389in}}%
\pgfpathlineto{\pgfqpoint{1.212601in}{1.786592in}}%
\pgfpathlineto{\pgfqpoint{1.215306in}{1.790698in}}%
\pgfpathlineto{\pgfqpoint{1.216659in}{1.790927in}}%
\pgfpathlineto{\pgfqpoint{1.218914in}{1.782170in}}%
\pgfpathlineto{\pgfqpoint{1.223423in}{1.759265in}}%
\pgfpathlineto{\pgfqpoint{1.226129in}{1.758015in}}%
\pgfpathlineto{\pgfqpoint{1.227933in}{1.757498in}}%
\pgfpathlineto{\pgfqpoint{1.229737in}{1.757119in}}%
\pgfpathlineto{\pgfqpoint{1.241462in}{1.734029in}}%
\pgfpathlineto{\pgfqpoint{1.243265in}{1.732016in}}%
\pgfpathlineto{\pgfqpoint{1.247775in}{1.717634in}}%
\pgfpathlineto{\pgfqpoint{1.248677in}{1.718412in}}%
\pgfpathlineto{\pgfqpoint{1.249579in}{1.720035in}}%
\pgfpathlineto{\pgfqpoint{1.250030in}{1.719774in}}%
\pgfpathlineto{\pgfqpoint{1.251383in}{1.720171in}}%
\pgfpathlineto{\pgfqpoint{1.252735in}{1.719878in}}%
\pgfpathlineto{\pgfqpoint{1.256794in}{1.710143in}}%
\pgfpathlineto{\pgfqpoint{1.258598in}{1.707855in}}%
\pgfpathlineto{\pgfqpoint{1.262205in}{1.714228in}}%
\pgfpathlineto{\pgfqpoint{1.262656in}{1.713885in}}%
\pgfpathlineto{\pgfqpoint{1.264460in}{1.711945in}}%
\pgfpathlineto{\pgfqpoint{1.271225in}{1.689006in}}%
\pgfpathlineto{\pgfqpoint{1.273479in}{1.686622in}}%
\pgfpathlineto{\pgfqpoint{1.279342in}{1.692007in}}%
\pgfpathlineto{\pgfqpoint{1.280244in}{1.691682in}}%
\pgfpathlineto{\pgfqpoint{1.283400in}{1.687228in}}%
\pgfpathlineto{\pgfqpoint{1.285655in}{1.683364in}}%
\pgfpathlineto{\pgfqpoint{1.287910in}{1.676755in}}%
\pgfpathlineto{\pgfqpoint{1.289714in}{1.673186in}}%
\pgfpathlineto{\pgfqpoint{1.291067in}{1.672397in}}%
\pgfpathlineto{\pgfqpoint{1.292870in}{1.668532in}}%
\pgfpathlineto{\pgfqpoint{1.293321in}{1.669157in}}%
\pgfpathlineto{\pgfqpoint{1.296478in}{1.673090in}}%
\pgfpathlineto{\pgfqpoint{1.298282in}{1.673682in}}%
\pgfpathlineto{\pgfqpoint{1.301889in}{1.669090in}}%
\pgfpathlineto{\pgfqpoint{1.305046in}{1.664362in}}%
\pgfpathlineto{\pgfqpoint{1.306850in}{1.661432in}}%
\pgfpathlineto{\pgfqpoint{1.309105in}{1.659775in}}%
\pgfpathlineto{\pgfqpoint{1.310909in}{1.657562in}}%
\pgfpathlineto{\pgfqpoint{1.312712in}{1.654576in}}%
\pgfpathlineto{\pgfqpoint{1.314516in}{1.655345in}}%
\pgfpathlineto{\pgfqpoint{1.315418in}{1.653605in}}%
\pgfpathlineto{\pgfqpoint{1.319477in}{1.642439in}}%
\pgfpathlineto{\pgfqpoint{1.319928in}{1.643321in}}%
\pgfpathlineto{\pgfqpoint{1.324888in}{1.652526in}}%
\pgfpathlineto{\pgfqpoint{1.326692in}{1.652762in}}%
\pgfpathlineto{\pgfqpoint{1.328045in}{1.649521in}}%
\pgfpathlineto{\pgfqpoint{1.330751in}{1.634530in}}%
\pgfpathlineto{\pgfqpoint{1.333456in}{1.625337in}}%
\pgfpathlineto{\pgfqpoint{1.335260in}{1.623411in}}%
\pgfpathlineto{\pgfqpoint{1.337064in}{1.624280in}}%
\pgfpathlineto{\pgfqpoint{1.340672in}{1.620516in}}%
\pgfpathlineto{\pgfqpoint{1.342475in}{1.618181in}}%
\pgfpathlineto{\pgfqpoint{1.348789in}{1.611538in}}%
\pgfpathlineto{\pgfqpoint{1.351043in}{1.611388in}}%
\pgfpathlineto{\pgfqpoint{1.352847in}{1.607662in}}%
\pgfpathlineto{\pgfqpoint{1.356004in}{1.598174in}}%
\pgfpathlineto{\pgfqpoint{1.366827in}{1.575147in}}%
\pgfpathlineto{\pgfqpoint{1.370885in}{1.563575in}}%
\pgfpathlineto{\pgfqpoint{1.374493in}{1.561634in}}%
\pgfpathlineto{\pgfqpoint{1.378101in}{1.570357in}}%
\pgfpathlineto{\pgfqpoint{1.379003in}{1.568868in}}%
\pgfpathlineto{\pgfqpoint{1.382610in}{1.561439in}}%
\pgfpathlineto{\pgfqpoint{1.384414in}{1.558815in}}%
\pgfpathlineto{\pgfqpoint{1.390276in}{1.544776in}}%
\pgfpathlineto{\pgfqpoint{1.392531in}{1.542736in}}%
\pgfpathlineto{\pgfqpoint{1.395688in}{1.537625in}}%
\pgfpathlineto{\pgfqpoint{1.397492in}{1.542847in}}%
\pgfpathlineto{\pgfqpoint{1.400197in}{1.548651in}}%
\pgfpathlineto{\pgfqpoint{1.401550in}{1.548924in}}%
\pgfpathlineto{\pgfqpoint{1.403805in}{1.545483in}}%
\pgfpathlineto{\pgfqpoint{1.405609in}{1.540544in}}%
\pgfpathlineto{\pgfqpoint{1.409217in}{1.525714in}}%
\pgfpathlineto{\pgfqpoint{1.409668in}{1.525857in}}%
\pgfpathlineto{\pgfqpoint{1.411471in}{1.526157in}}%
\pgfpathlineto{\pgfqpoint{1.413726in}{1.525924in}}%
\pgfpathlineto{\pgfqpoint{1.419138in}{1.526976in}}%
\pgfpathlineto{\pgfqpoint{1.421843in}{1.522890in}}%
\pgfpathlineto{\pgfqpoint{1.424098in}{1.522742in}}%
\pgfpathlineto{\pgfqpoint{1.425902in}{1.518977in}}%
\pgfpathlineto{\pgfqpoint{1.429960in}{1.506692in}}%
\pgfpathlineto{\pgfqpoint{1.433568in}{1.497939in}}%
\pgfpathlineto{\pgfqpoint{1.434019in}{1.498498in}}%
\pgfpathlineto{\pgfqpoint{1.441685in}{1.505173in}}%
\pgfpathlineto{\pgfqpoint{1.443038in}{1.506943in}}%
\pgfpathlineto{\pgfqpoint{1.447548in}{1.515567in}}%
\pgfpathlineto{\pgfqpoint{1.451606in}{1.517275in}}%
\pgfpathlineto{\pgfqpoint{1.457469in}{1.513176in}}%
\pgfpathlineto{\pgfqpoint{1.459723in}{1.517724in}}%
\pgfpathlineto{\pgfqpoint{1.461527in}{1.521413in}}%
\pgfpathlineto{\pgfqpoint{1.463782in}{1.520148in}}%
\pgfpathlineto{\pgfqpoint{1.466488in}{1.516890in}}%
\pgfpathlineto{\pgfqpoint{1.468743in}{1.514169in}}%
\pgfpathlineto{\pgfqpoint{1.481369in}{1.483116in}}%
\pgfpathlineto{\pgfqpoint{1.482722in}{1.484456in}}%
\pgfpathlineto{\pgfqpoint{1.484526in}{1.486722in}}%
\pgfpathlineto{\pgfqpoint{1.492192in}{1.484166in}}%
\pgfpathlineto{\pgfqpoint{1.493545in}{1.483635in}}%
\pgfpathlineto{\pgfqpoint{1.495800in}{1.482261in}}%
\pgfpathlineto{\pgfqpoint{1.499858in}{1.480491in}}%
\pgfpathlineto{\pgfqpoint{1.503466in}{1.483018in}}%
\pgfpathlineto{\pgfqpoint{1.506623in}{1.486091in}}%
\pgfpathlineto{\pgfqpoint{1.507976in}{1.486631in}}%
\pgfpathlineto{\pgfqpoint{1.509328in}{1.485589in}}%
\pgfpathlineto{\pgfqpoint{1.515642in}{1.475079in}}%
\pgfpathlineto{\pgfqpoint{1.517446in}{1.475706in}}%
\pgfpathlineto{\pgfqpoint{1.520602in}{1.478735in}}%
\pgfpathlineto{\pgfqpoint{1.526014in}{1.482140in}}%
\pgfpathlineto{\pgfqpoint{1.531425in}{1.479096in}}%
\pgfpathlineto{\pgfqpoint{1.535484in}{1.474176in}}%
\pgfpathlineto{\pgfqpoint{1.538640in}{1.470351in}}%
\pgfpathlineto{\pgfqpoint{1.540895in}{1.465342in}}%
\pgfpathlineto{\pgfqpoint{1.548110in}{1.450525in}}%
\pgfpathlineto{\pgfqpoint{1.550816in}{1.445409in}}%
\pgfpathlineto{\pgfqpoint{1.551718in}{1.444689in}}%
\pgfpathlineto{\pgfqpoint{1.552169in}{1.445076in}}%
\pgfpathlineto{\pgfqpoint{1.556228in}{1.448027in}}%
\pgfpathlineto{\pgfqpoint{1.558031in}{1.448790in}}%
\pgfpathlineto{\pgfqpoint{1.559835in}{1.449182in}}%
\pgfpathlineto{\pgfqpoint{1.561188in}{1.449538in}}%
\pgfpathlineto{\pgfqpoint{1.562541in}{1.449490in}}%
\pgfpathlineto{\pgfqpoint{1.564345in}{1.448940in}}%
\pgfpathlineto{\pgfqpoint{1.566149in}{1.446041in}}%
\pgfpathlineto{\pgfqpoint{1.569756in}{1.437109in}}%
\pgfpathlineto{\pgfqpoint{1.570658in}{1.437322in}}%
\pgfpathlineto{\pgfqpoint{1.574266in}{1.444500in}}%
\pgfpathlineto{\pgfqpoint{1.577422in}{1.445661in}}%
\pgfpathlineto{\pgfqpoint{1.580128in}{1.440415in}}%
\pgfpathlineto{\pgfqpoint{1.582383in}{1.437075in}}%
\pgfpathlineto{\pgfqpoint{1.585991in}{1.433286in}}%
\pgfpathlineto{\pgfqpoint{1.591853in}{1.425127in}}%
\pgfpathlineto{\pgfqpoint{1.595010in}{1.418112in}}%
\pgfpathlineto{\pgfqpoint{1.596813in}{1.418292in}}%
\pgfpathlineto{\pgfqpoint{1.599970in}{1.421760in}}%
\pgfpathlineto{\pgfqpoint{1.604931in}{1.430579in}}%
\pgfpathlineto{\pgfqpoint{1.605382in}{1.430253in}}%
\pgfpathlineto{\pgfqpoint{1.614401in}{1.419546in}}%
\pgfpathlineto{\pgfqpoint{1.616205in}{1.419408in}}%
\pgfpathlineto{\pgfqpoint{1.622067in}{1.421882in}}%
\pgfpathlineto{\pgfqpoint{1.624773in}{1.419888in}}%
\pgfpathlineto{\pgfqpoint{1.627027in}{1.418871in}}%
\pgfpathlineto{\pgfqpoint{1.632890in}{1.416129in}}%
\pgfpathlineto{\pgfqpoint{1.634694in}{1.416927in}}%
\pgfpathlineto{\pgfqpoint{1.642811in}{1.415030in}}%
\pgfpathlineto{\pgfqpoint{1.644615in}{1.415042in}}%
\pgfpathlineto{\pgfqpoint{1.648222in}{1.418371in}}%
\pgfpathlineto{\pgfqpoint{1.650928in}{1.420329in}}%
\pgfpathlineto{\pgfqpoint{1.654085in}{1.422332in}}%
\pgfpathlineto{\pgfqpoint{1.660398in}{1.425351in}}%
\pgfpathlineto{\pgfqpoint{1.670770in}{1.419867in}}%
\pgfpathlineto{\pgfqpoint{1.672123in}{1.419998in}}%
\pgfpathlineto{\pgfqpoint{1.678887in}{1.411478in}}%
\pgfpathlineto{\pgfqpoint{1.680691in}{1.411097in}}%
\pgfpathlineto{\pgfqpoint{1.682495in}{1.411851in}}%
\pgfpathlineto{\pgfqpoint{1.684750in}{1.410097in}}%
\pgfpathlineto{\pgfqpoint{1.687906in}{1.403158in}}%
\pgfpathlineto{\pgfqpoint{1.688808in}{1.401640in}}%
\pgfpathlineto{\pgfqpoint{1.689259in}{1.402131in}}%
\pgfpathlineto{\pgfqpoint{1.691965in}{1.404048in}}%
\pgfpathlineto{\pgfqpoint{1.695122in}{1.405106in}}%
\pgfpathlineto{\pgfqpoint{1.702788in}{1.401384in}}%
\pgfpathlineto{\pgfqpoint{1.706846in}{1.399502in}}%
\pgfpathlineto{\pgfqpoint{1.709552in}{1.396816in}}%
\pgfpathlineto{\pgfqpoint{1.710905in}{1.399478in}}%
\pgfpathlineto{\pgfqpoint{1.713160in}{1.402473in}}%
\pgfpathlineto{\pgfqpoint{1.719473in}{1.402253in}}%
\pgfpathlineto{\pgfqpoint{1.721728in}{1.404021in}}%
\pgfpathlineto{\pgfqpoint{1.723532in}{1.403245in}}%
\pgfpathlineto{\pgfqpoint{1.731198in}{1.395910in}}%
\pgfpathlineto{\pgfqpoint{1.737060in}{1.395152in}}%
\pgfpathlineto{\pgfqpoint{1.741570in}{1.389830in}}%
\pgfpathlineto{\pgfqpoint{1.746530in}{1.378473in}}%
\pgfpathlineto{\pgfqpoint{1.746981in}{1.378844in}}%
\pgfpathlineto{\pgfqpoint{1.748785in}{1.381260in}}%
\pgfpathlineto{\pgfqpoint{1.751491in}{1.385808in}}%
\pgfpathlineto{\pgfqpoint{1.753295in}{1.385656in}}%
\pgfpathlineto{\pgfqpoint{1.756902in}{1.379722in}}%
\pgfpathlineto{\pgfqpoint{1.762765in}{1.368147in}}%
\pgfpathlineto{\pgfqpoint{1.765019in}{1.367070in}}%
\pgfpathlineto{\pgfqpoint{1.766823in}{1.367262in}}%
\pgfpathlineto{\pgfqpoint{1.769529in}{1.368704in}}%
\pgfpathlineto{\pgfqpoint{1.771784in}{1.366630in}}%
\pgfpathlineto{\pgfqpoint{1.774038in}{1.364862in}}%
\pgfpathlineto{\pgfqpoint{1.775842in}{1.363472in}}%
\pgfpathlineto{\pgfqpoint{1.778097in}{1.363005in}}%
\pgfpathlineto{\pgfqpoint{1.779901in}{1.363256in}}%
\pgfpathlineto{\pgfqpoint{1.783959in}{1.361652in}}%
\pgfpathlineto{\pgfqpoint{1.785763in}{1.363650in}}%
\pgfpathlineto{\pgfqpoint{1.788920in}{1.369022in}}%
\pgfpathlineto{\pgfqpoint{1.789371in}{1.368731in}}%
\pgfpathlineto{\pgfqpoint{1.791175in}{1.364434in}}%
\pgfpathlineto{\pgfqpoint{1.794782in}{1.354953in}}%
\pgfpathlineto{\pgfqpoint{1.801096in}{1.345458in}}%
\pgfpathlineto{\pgfqpoint{1.804703in}{1.342743in}}%
\pgfpathlineto{\pgfqpoint{1.807860in}{1.340764in}}%
\pgfpathlineto{\pgfqpoint{1.811017in}{1.336098in}}%
\pgfpathlineto{\pgfqpoint{1.815526in}{1.334538in}}%
\pgfpathlineto{\pgfqpoint{1.822742in}{1.336608in}}%
\pgfpathlineto{\pgfqpoint{1.825447in}{1.342424in}}%
\pgfpathlineto{\pgfqpoint{1.827251in}{1.342204in}}%
\pgfpathlineto{\pgfqpoint{1.830859in}{1.339797in}}%
\pgfpathlineto{\pgfqpoint{1.838074in}{1.334681in}}%
\pgfpathlineto{\pgfqpoint{1.840329in}{1.334856in}}%
\pgfpathlineto{\pgfqpoint{1.846191in}{1.335960in}}%
\pgfpathlineto{\pgfqpoint{1.848446in}{1.340031in}}%
\pgfpathlineto{\pgfqpoint{1.852955in}{1.351596in}}%
\pgfpathlineto{\pgfqpoint{1.855661in}{1.352652in}}%
\pgfpathlineto{\pgfqpoint{1.857916in}{1.350572in}}%
\pgfpathlineto{\pgfqpoint{1.862426in}{1.341028in}}%
\pgfpathlineto{\pgfqpoint{1.867386in}{1.329849in}}%
\pgfpathlineto{\pgfqpoint{1.870092in}{1.328550in}}%
\pgfpathlineto{\pgfqpoint{1.878209in}{1.322918in}}%
\pgfpathlineto{\pgfqpoint{1.883620in}{1.328219in}}%
\pgfpathlineto{\pgfqpoint{1.886326in}{1.326698in}}%
\pgfpathlineto{\pgfqpoint{1.891287in}{1.313296in}}%
\pgfpathlineto{\pgfqpoint{1.892188in}{1.314037in}}%
\pgfpathlineto{\pgfqpoint{1.896698in}{1.320584in}}%
\pgfpathlineto{\pgfqpoint{1.901208in}{1.331296in}}%
\pgfpathlineto{\pgfqpoint{1.902560in}{1.330813in}}%
\pgfpathlineto{\pgfqpoint{1.916089in}{1.311763in}}%
\pgfpathlineto{\pgfqpoint{1.916540in}{1.312192in}}%
\pgfpathlineto{\pgfqpoint{1.917893in}{1.312176in}}%
\pgfpathlineto{\pgfqpoint{1.925559in}{1.307342in}}%
\pgfpathlineto{\pgfqpoint{1.929618in}{1.311778in}}%
\pgfpathlineto{\pgfqpoint{1.932323in}{1.308294in}}%
\pgfpathlineto{\pgfqpoint{1.933225in}{1.309165in}}%
\pgfpathlineto{\pgfqpoint{1.936382in}{1.310816in}}%
\pgfpathlineto{\pgfqpoint{1.939539in}{1.311628in}}%
\pgfpathlineto{\pgfqpoint{1.947205in}{1.304930in}}%
\pgfpathlineto{\pgfqpoint{1.953969in}{1.298341in}}%
\pgfpathlineto{\pgfqpoint{1.956224in}{1.296123in}}%
\pgfpathlineto{\pgfqpoint{1.959832in}{1.291268in}}%
\pgfpathlineto{\pgfqpoint{1.965243in}{1.296611in}}%
\pgfpathlineto{\pgfqpoint{1.970655in}{1.298011in}}%
\pgfpathlineto{\pgfqpoint{1.972458in}{1.296259in}}%
\pgfpathlineto{\pgfqpoint{1.976517in}{1.291822in}}%
\pgfpathlineto{\pgfqpoint{1.978772in}{1.293307in}}%
\pgfpathlineto{\pgfqpoint{1.981026in}{1.299269in}}%
\pgfpathlineto{\pgfqpoint{1.984183in}{1.305754in}}%
\pgfpathlineto{\pgfqpoint{1.985987in}{1.306062in}}%
\pgfpathlineto{\pgfqpoint{1.990046in}{1.298758in}}%
\pgfpathlineto{\pgfqpoint{1.993653in}{1.288991in}}%
\pgfpathlineto{\pgfqpoint{1.997712in}{1.282175in}}%
\pgfpathlineto{\pgfqpoint{2.000868in}{1.281838in}}%
\pgfpathlineto{\pgfqpoint{2.004025in}{1.285013in}}%
\pgfpathlineto{\pgfqpoint{2.006280in}{1.287125in}}%
\pgfpathlineto{\pgfqpoint{2.007633in}{1.285333in}}%
\pgfpathlineto{\pgfqpoint{2.013044in}{1.270830in}}%
\pgfpathlineto{\pgfqpoint{2.013495in}{1.271256in}}%
\pgfpathlineto{\pgfqpoint{2.015750in}{1.276345in}}%
\pgfpathlineto{\pgfqpoint{2.020710in}{1.286663in}}%
\pgfpathlineto{\pgfqpoint{2.023416in}{1.285893in}}%
\pgfpathlineto{\pgfqpoint{2.028828in}{1.278367in}}%
\pgfpathlineto{\pgfqpoint{2.036043in}{1.279389in}}%
\pgfpathlineto{\pgfqpoint{2.040552in}{1.275450in}}%
\pgfpathlineto{\pgfqpoint{2.043258in}{1.268205in}}%
\pgfpathlineto{\pgfqpoint{2.049121in}{1.253344in}}%
\pgfpathlineto{\pgfqpoint{2.053179in}{1.249600in}}%
\pgfpathlineto{\pgfqpoint{2.056336in}{1.254283in}}%
\pgfpathlineto{\pgfqpoint{2.061747in}{1.264536in}}%
\pgfpathlineto{\pgfqpoint{2.064002in}{1.264866in}}%
\pgfpathlineto{\pgfqpoint{2.075276in}{1.257919in}}%
\pgfpathlineto{\pgfqpoint{2.077531in}{1.256863in}}%
\pgfpathlineto{\pgfqpoint{2.079785in}{1.256270in}}%
\pgfpathlineto{\pgfqpoint{2.082040in}{1.254633in}}%
\pgfpathlineto{\pgfqpoint{2.083844in}{1.253915in}}%
\pgfpathlineto{\pgfqpoint{2.086099in}{1.254119in}}%
\pgfpathlineto{\pgfqpoint{2.088354in}{1.255457in}}%
\pgfpathlineto{\pgfqpoint{2.090157in}{1.259388in}}%
\pgfpathlineto{\pgfqpoint{2.094667in}{1.270183in}}%
\pgfpathlineto{\pgfqpoint{2.096471in}{1.270070in}}%
\pgfpathlineto{\pgfqpoint{2.098726in}{1.267493in}}%
\pgfpathlineto{\pgfqpoint{2.103686in}{1.257634in}}%
\pgfpathlineto{\pgfqpoint{2.108196in}{1.245026in}}%
\pgfpathlineto{\pgfqpoint{2.113156in}{1.235392in}}%
\pgfpathlineto{\pgfqpoint{2.118567in}{1.238999in}}%
\pgfpathlineto{\pgfqpoint{2.119920in}{1.238378in}}%
\pgfpathlineto{\pgfqpoint{2.125332in}{1.230065in}}%
\pgfpathlineto{\pgfqpoint{2.127136in}{1.229942in}}%
\pgfpathlineto{\pgfqpoint{2.147880in}{1.253214in}}%
\pgfpathlineto{\pgfqpoint{2.150134in}{1.250873in}}%
\pgfpathlineto{\pgfqpoint{2.153742in}{1.241410in}}%
\pgfpathlineto{\pgfqpoint{2.155997in}{1.238514in}}%
\pgfpathlineto{\pgfqpoint{2.166369in}{1.241314in}}%
\pgfpathlineto{\pgfqpoint{2.170427in}{1.242727in}}%
\pgfpathlineto{\pgfqpoint{2.174035in}{1.248038in}}%
\pgfpathlineto{\pgfqpoint{2.176290in}{1.246367in}}%
\pgfpathlineto{\pgfqpoint{2.185309in}{1.237625in}}%
\pgfpathlineto{\pgfqpoint{2.187113in}{1.235960in}}%
\pgfpathlineto{\pgfqpoint{2.189818in}{1.232955in}}%
\pgfpathlineto{\pgfqpoint{2.193426in}{1.232752in}}%
\pgfpathlineto{\pgfqpoint{2.194779in}{1.232471in}}%
\pgfpathlineto{\pgfqpoint{2.198386in}{1.229248in}}%
\pgfpathlineto{\pgfqpoint{2.206053in}{1.226969in}}%
\pgfpathlineto{\pgfqpoint{2.211013in}{1.220620in}}%
\pgfpathlineto{\pgfqpoint{2.215072in}{1.223247in}}%
\pgfpathlineto{\pgfqpoint{2.220032in}{1.227721in}}%
\pgfpathlineto{\pgfqpoint{2.220483in}{1.227242in}}%
\pgfpathlineto{\pgfqpoint{2.230404in}{1.215258in}}%
\pgfpathlineto{\pgfqpoint{2.234012in}{1.215581in}}%
\pgfpathlineto{\pgfqpoint{2.241678in}{1.224557in}}%
\pgfpathlineto{\pgfqpoint{2.243482in}{1.221723in}}%
\pgfpathlineto{\pgfqpoint{2.252501in}{1.201976in}}%
\pgfpathlineto{\pgfqpoint{2.254756in}{1.201723in}}%
\pgfpathlineto{\pgfqpoint{2.258814in}{1.205584in}}%
\pgfpathlineto{\pgfqpoint{2.265128in}{1.212944in}}%
\pgfpathlineto{\pgfqpoint{2.268284in}{1.212444in}}%
\pgfpathlineto{\pgfqpoint{2.275951in}{1.205309in}}%
\pgfpathlineto{\pgfqpoint{2.277303in}{1.206594in}}%
\pgfpathlineto{\pgfqpoint{2.280460in}{1.210757in}}%
\pgfpathlineto{\pgfqpoint{2.282715in}{1.210616in}}%
\pgfpathlineto{\pgfqpoint{2.284519in}{1.210032in}}%
\pgfpathlineto{\pgfqpoint{2.290832in}{1.211068in}}%
\pgfpathlineto{\pgfqpoint{2.294891in}{1.206647in}}%
\pgfpathlineto{\pgfqpoint{2.302106in}{1.193351in}}%
\pgfpathlineto{\pgfqpoint{2.303910in}{1.194454in}}%
\pgfpathlineto{\pgfqpoint{2.311125in}{1.199735in}}%
\pgfpathlineto{\pgfqpoint{2.312929in}{1.197562in}}%
\pgfpathlineto{\pgfqpoint{2.317438in}{1.188006in}}%
\pgfpathlineto{\pgfqpoint{2.321948in}{1.189963in}}%
\pgfpathlineto{\pgfqpoint{2.325555in}{1.193329in}}%
\pgfpathlineto{\pgfqpoint{2.327359in}{1.192350in}}%
\pgfpathlineto{\pgfqpoint{2.333673in}{1.183998in}}%
\pgfpathlineto{\pgfqpoint{2.335927in}{1.185375in}}%
\pgfpathlineto{\pgfqpoint{2.337280in}{1.186947in}}%
\pgfpathlineto{\pgfqpoint{2.340888in}{1.190672in}}%
\pgfpathlineto{\pgfqpoint{2.342692in}{1.190207in}}%
\pgfpathlineto{\pgfqpoint{2.344946in}{1.188636in}}%
\pgfpathlineto{\pgfqpoint{2.347652in}{1.188884in}}%
\pgfpathlineto{\pgfqpoint{2.349907in}{1.187446in}}%
\pgfpathlineto{\pgfqpoint{2.352162in}{1.191111in}}%
\pgfpathlineto{\pgfqpoint{2.356671in}{1.199121in}}%
\pgfpathlineto{\pgfqpoint{2.358024in}{1.198478in}}%
\pgfpathlineto{\pgfqpoint{2.359828in}{1.194814in}}%
\pgfpathlineto{\pgfqpoint{2.364338in}{1.182509in}}%
\pgfpathlineto{\pgfqpoint{2.366141in}{1.181021in}}%
\pgfpathlineto{\pgfqpoint{2.367945in}{1.182569in}}%
\pgfpathlineto{\pgfqpoint{2.369298in}{1.182953in}}%
\pgfpathlineto{\pgfqpoint{2.373808in}{1.180234in}}%
\pgfpathlineto{\pgfqpoint{2.375611in}{1.180263in}}%
\pgfpathlineto{\pgfqpoint{2.380121in}{1.176798in}}%
\pgfpathlineto{\pgfqpoint{2.382827in}{1.178726in}}%
\pgfpathlineto{\pgfqpoint{2.385081in}{1.176961in}}%
\pgfpathlineto{\pgfqpoint{2.388689in}{1.172765in}}%
\pgfpathlineto{\pgfqpoint{2.394551in}{1.171251in}}%
\pgfpathlineto{\pgfqpoint{2.399061in}{1.165248in}}%
\pgfpathlineto{\pgfqpoint{2.401316in}{1.170074in}}%
\pgfpathlineto{\pgfqpoint{2.406727in}{1.181437in}}%
\pgfpathlineto{\pgfqpoint{2.408982in}{1.181815in}}%
\pgfpathlineto{\pgfqpoint{2.410786in}{1.180212in}}%
\pgfpathlineto{\pgfqpoint{2.415746in}{1.174220in}}%
\pgfpathlineto{\pgfqpoint{2.418452in}{1.173321in}}%
\pgfpathlineto{\pgfqpoint{2.427020in}{1.164555in}}%
\pgfpathlineto{\pgfqpoint{2.430628in}{1.170495in}}%
\pgfpathlineto{\pgfqpoint{2.434686in}{1.178235in}}%
\pgfpathlineto{\pgfqpoint{2.436490in}{1.178722in}}%
\pgfpathlineto{\pgfqpoint{2.438745in}{1.174758in}}%
\pgfpathlineto{\pgfqpoint{2.443255in}{1.155630in}}%
\pgfpathlineto{\pgfqpoint{2.446862in}{1.140998in}}%
\pgfpathlineto{\pgfqpoint{2.448215in}{1.139567in}}%
\pgfpathlineto{\pgfqpoint{2.450470in}{1.141915in}}%
\pgfpathlineto{\pgfqpoint{2.455430in}{1.147854in}}%
\pgfpathlineto{\pgfqpoint{2.466253in}{1.142159in}}%
\pgfpathlineto{\pgfqpoint{2.477076in}{1.142039in}}%
\pgfpathlineto{\pgfqpoint{2.479331in}{1.137782in}}%
\pgfpathlineto{\pgfqpoint{2.483389in}{1.131959in}}%
\pgfpathlineto{\pgfqpoint{2.485644in}{1.131369in}}%
\pgfpathlineto{\pgfqpoint{2.486997in}{1.132754in}}%
\pgfpathlineto{\pgfqpoint{2.491958in}{1.143267in}}%
\pgfpathlineto{\pgfqpoint{2.497369in}{1.145170in}}%
\pgfpathlineto{\pgfqpoint{2.499624in}{1.147917in}}%
\pgfpathlineto{\pgfqpoint{2.506388in}{1.157708in}}%
\pgfpathlineto{\pgfqpoint{2.508192in}{1.158064in}}%
\pgfpathlineto{\pgfqpoint{2.526230in}{1.150491in}}%
\pgfpathlineto{\pgfqpoint{2.529387in}{1.147103in}}%
\pgfpathlineto{\pgfqpoint{2.531642in}{1.146704in}}%
\pgfpathlineto{\pgfqpoint{2.534347in}{1.147627in}}%
\pgfpathlineto{\pgfqpoint{2.543366in}{1.155498in}}%
\pgfpathlineto{\pgfqpoint{2.551484in}{1.146555in}}%
\pgfpathlineto{\pgfqpoint{2.554189in}{1.146302in}}%
\pgfpathlineto{\pgfqpoint{2.571776in}{1.125354in}}%
\pgfpathlineto{\pgfqpoint{2.578090in}{1.122661in}}%
\pgfpathlineto{\pgfqpoint{2.580345in}{1.122457in}}%
\pgfpathlineto{\pgfqpoint{2.582599in}{1.126021in}}%
\pgfpathlineto{\pgfqpoint{2.586207in}{1.132300in}}%
\pgfpathlineto{\pgfqpoint{2.589815in}{1.134458in}}%
\pgfpathlineto{\pgfqpoint{2.591618in}{1.133245in}}%
\pgfpathlineto{\pgfqpoint{2.597030in}{1.128379in}}%
\pgfpathlineto{\pgfqpoint{2.601990in}{1.129801in}}%
\pgfpathlineto{\pgfqpoint{2.606049in}{1.127997in}}%
\pgfpathlineto{\pgfqpoint{2.608755in}{1.126294in}}%
\pgfpathlineto{\pgfqpoint{2.611911in}{1.127444in}}%
\pgfpathlineto{\pgfqpoint{2.613715in}{1.128243in}}%
\pgfpathlineto{\pgfqpoint{2.619578in}{1.125660in}}%
\pgfpathlineto{\pgfqpoint{2.620930in}{1.126168in}}%
\pgfpathlineto{\pgfqpoint{2.623185in}{1.126400in}}%
\pgfpathlineto{\pgfqpoint{2.626342in}{1.123096in}}%
\pgfpathlineto{\pgfqpoint{2.629499in}{1.119702in}}%
\pgfpathlineto{\pgfqpoint{2.634008in}{1.117127in}}%
\pgfpathlineto{\pgfqpoint{2.637165in}{1.119630in}}%
\pgfpathlineto{\pgfqpoint{2.640772in}{1.123234in}}%
\pgfpathlineto{\pgfqpoint{2.644380in}{1.125050in}}%
\pgfpathlineto{\pgfqpoint{2.647086in}{1.123630in}}%
\pgfpathlineto{\pgfqpoint{2.655203in}{1.114752in}}%
\pgfpathlineto{\pgfqpoint{2.658811in}{1.113563in}}%
\pgfpathlineto{\pgfqpoint{2.664222in}{1.110626in}}%
\pgfpathlineto{\pgfqpoint{2.666477in}{1.111715in}}%
\pgfpathlineto{\pgfqpoint{2.671888in}{1.108540in}}%
\pgfpathlineto{\pgfqpoint{2.674143in}{1.103843in}}%
\pgfpathlineto{\pgfqpoint{2.679104in}{1.095261in}}%
\pgfpathlineto{\pgfqpoint{2.680907in}{1.094559in}}%
\pgfpathlineto{\pgfqpoint{2.682711in}{1.096660in}}%
\pgfpathlineto{\pgfqpoint{2.689475in}{1.109008in}}%
\pgfpathlineto{\pgfqpoint{2.692632in}{1.108913in}}%
\pgfpathlineto{\pgfqpoint{2.694436in}{1.107236in}}%
\pgfpathlineto{\pgfqpoint{2.698495in}{1.101023in}}%
\pgfpathlineto{\pgfqpoint{2.700749in}{1.099092in}}%
\pgfpathlineto{\pgfqpoint{2.703455in}{1.096613in}}%
\pgfpathlineto{\pgfqpoint{2.704808in}{1.097332in}}%
\pgfpathlineto{\pgfqpoint{2.708416in}{1.105488in}}%
\pgfpathlineto{\pgfqpoint{2.714278in}{1.114317in}}%
\pgfpathlineto{\pgfqpoint{2.718337in}{1.118410in}}%
\pgfpathlineto{\pgfqpoint{2.721042in}{1.116565in}}%
\pgfpathlineto{\pgfqpoint{2.731414in}{1.106884in}}%
\pgfpathlineto{\pgfqpoint{2.734120in}{1.107573in}}%
\pgfpathlineto{\pgfqpoint{2.736826in}{1.109655in}}%
\pgfpathlineto{\pgfqpoint{2.740884in}{1.110872in}}%
\pgfpathlineto{\pgfqpoint{2.745845in}{1.112313in}}%
\pgfpathlineto{\pgfqpoint{2.749903in}{1.108296in}}%
\pgfpathlineto{\pgfqpoint{2.757119in}{1.092571in}}%
\pgfpathlineto{\pgfqpoint{2.762981in}{1.084265in}}%
\pgfpathlineto{\pgfqpoint{2.767942in}{1.081486in}}%
\pgfpathlineto{\pgfqpoint{2.770647in}{1.083038in}}%
\pgfpathlineto{\pgfqpoint{2.774255in}{1.084844in}}%
\pgfpathlineto{\pgfqpoint{2.776059in}{1.083918in}}%
\pgfpathlineto{\pgfqpoint{2.778313in}{1.082690in}}%
\pgfpathlineto{\pgfqpoint{2.782823in}{1.082586in}}%
\pgfpathlineto{\pgfqpoint{2.786431in}{1.077945in}}%
\pgfpathlineto{\pgfqpoint{2.788685in}{1.074613in}}%
\pgfpathlineto{\pgfqpoint{2.792293in}{1.067863in}}%
\pgfpathlineto{\pgfqpoint{2.796352in}{1.063611in}}%
\pgfpathlineto{\pgfqpoint{2.797704in}{1.063954in}}%
\pgfpathlineto{\pgfqpoint{2.801312in}{1.067006in}}%
\pgfpathlineto{\pgfqpoint{2.805371in}{1.068886in}}%
\pgfpathlineto{\pgfqpoint{2.808978in}{1.073465in}}%
\pgfpathlineto{\pgfqpoint{2.811233in}{1.073960in}}%
\pgfpathlineto{\pgfqpoint{2.813939in}{1.075260in}}%
\pgfpathlineto{\pgfqpoint{2.816645in}{1.076933in}}%
\pgfpathlineto{\pgfqpoint{2.818448in}{1.075821in}}%
\pgfpathlineto{\pgfqpoint{2.821605in}{1.078910in}}%
\pgfpathlineto{\pgfqpoint{2.828369in}{1.086012in}}%
\pgfpathlineto{\pgfqpoint{2.833781in}{1.079335in}}%
\pgfpathlineto{\pgfqpoint{2.843702in}{1.064476in}}%
\pgfpathlineto{\pgfqpoint{2.845506in}{1.063311in}}%
\pgfpathlineto{\pgfqpoint{2.850466in}{1.058357in}}%
\pgfpathlineto{\pgfqpoint{2.853172in}{1.056990in}}%
\pgfpathlineto{\pgfqpoint{2.857681in}{1.061697in}}%
\pgfpathlineto{\pgfqpoint{2.859936in}{1.064690in}}%
\pgfpathlineto{\pgfqpoint{2.862191in}{1.064773in}}%
\pgfpathlineto{\pgfqpoint{2.864446in}{1.063888in}}%
\pgfpathlineto{\pgfqpoint{2.867602in}{1.061890in}}%
\pgfpathlineto{\pgfqpoint{2.873916in}{1.060020in}}%
\pgfpathlineto{\pgfqpoint{2.875720in}{1.061624in}}%
\pgfpathlineto{\pgfqpoint{2.878876in}{1.063546in}}%
\pgfpathlineto{\pgfqpoint{2.881131in}{1.065310in}}%
\pgfpathlineto{\pgfqpoint{2.884739in}{1.068632in}}%
\pgfpathlineto{\pgfqpoint{2.887444in}{1.066933in}}%
\pgfpathlineto{\pgfqpoint{2.889699in}{1.065001in}}%
\pgfpathlineto{\pgfqpoint{2.892405in}{1.065110in}}%
\pgfpathlineto{\pgfqpoint{2.894660in}{1.062819in}}%
\pgfpathlineto{\pgfqpoint{2.900071in}{1.049704in}}%
\pgfpathlineto{\pgfqpoint{2.905032in}{1.041215in}}%
\pgfpathlineto{\pgfqpoint{2.907286in}{1.041943in}}%
\pgfpathlineto{\pgfqpoint{2.909541in}{1.045486in}}%
\pgfpathlineto{\pgfqpoint{2.913600in}{1.055711in}}%
\pgfpathlineto{\pgfqpoint{2.916305in}{1.057350in}}%
\pgfpathlineto{\pgfqpoint{2.917658in}{1.058346in}}%
\pgfpathlineto{\pgfqpoint{2.921717in}{1.059818in}}%
\pgfpathlineto{\pgfqpoint{2.926677in}{1.058075in}}%
\pgfpathlineto{\pgfqpoint{2.931187in}{1.057168in}}%
\pgfpathlineto{\pgfqpoint{2.936147in}{1.053378in}}%
\pgfpathlineto{\pgfqpoint{2.946519in}{1.045021in}}%
\pgfpathlineto{\pgfqpoint{2.951931in}{1.043398in}}%
\pgfpathlineto{\pgfqpoint{2.956440in}{1.047998in}}%
\pgfpathlineto{\pgfqpoint{2.959146in}{1.050418in}}%
\pgfpathlineto{\pgfqpoint{2.963205in}{1.049540in}}%
\pgfpathlineto{\pgfqpoint{2.967714in}{1.043074in}}%
\pgfpathlineto{\pgfqpoint{2.972224in}{1.036067in}}%
\pgfpathlineto{\pgfqpoint{2.974028in}{1.035813in}}%
\pgfpathlineto{\pgfqpoint{2.983047in}{1.041822in}}%
\pgfpathlineto{\pgfqpoint{2.989360in}{1.035683in}}%
\pgfpathlineto{\pgfqpoint{2.992066in}{1.028676in}}%
\pgfpathlineto{\pgfqpoint{3.000634in}{1.006672in}}%
\pgfpathlineto{\pgfqpoint{3.001987in}{1.005701in}}%
\pgfpathlineto{\pgfqpoint{3.003791in}{1.004928in}}%
\pgfpathlineto{\pgfqpoint{3.005594in}{1.006595in}}%
\pgfpathlineto{\pgfqpoint{3.015966in}{1.018789in}}%
\pgfpathlineto{\pgfqpoint{3.022731in}{1.016458in}}%
\pgfpathlineto{\pgfqpoint{3.024534in}{1.019114in}}%
\pgfpathlineto{\pgfqpoint{3.032652in}{1.036616in}}%
\pgfpathlineto{\pgfqpoint{3.034455in}{1.036516in}}%
\pgfpathlineto{\pgfqpoint{3.041220in}{1.035007in}}%
\pgfpathlineto{\pgfqpoint{3.044376in}{1.031865in}}%
\pgfpathlineto{\pgfqpoint{3.047984in}{1.031039in}}%
\pgfpathlineto{\pgfqpoint{3.051141in}{1.028709in}}%
\pgfpathlineto{\pgfqpoint{3.052945in}{1.028404in}}%
\pgfpathlineto{\pgfqpoint{3.056101in}{1.026321in}}%
\pgfpathlineto{\pgfqpoint{3.060160in}{1.025848in}}%
\pgfpathlineto{\pgfqpoint{3.062415in}{1.025583in}}%
\pgfpathlineto{\pgfqpoint{3.066473in}{1.022266in}}%
\pgfpathlineto{\pgfqpoint{3.068277in}{1.021520in}}%
\pgfpathlineto{\pgfqpoint{3.074590in}{1.020967in}}%
\pgfpathlineto{\pgfqpoint{3.077296in}{1.023214in}}%
\pgfpathlineto{\pgfqpoint{3.079100in}{1.023746in}}%
\pgfpathlineto{\pgfqpoint{3.082708in}{1.025830in}}%
\pgfpathlineto{\pgfqpoint{3.087217in}{1.019499in}}%
\pgfpathlineto{\pgfqpoint{3.091276in}{1.014883in}}%
\pgfpathlineto{\pgfqpoint{3.098040in}{1.015972in}}%
\pgfpathlineto{\pgfqpoint{3.100746in}{1.013619in}}%
\pgfpathlineto{\pgfqpoint{3.102550in}{1.015279in}}%
\pgfpathlineto{\pgfqpoint{3.106157in}{1.019108in}}%
\pgfpathlineto{\pgfqpoint{3.107510in}{1.019124in}}%
\pgfpathlineto{\pgfqpoint{3.117882in}{1.028629in}}%
\pgfpathlineto{\pgfqpoint{3.120137in}{1.027364in}}%
\pgfpathlineto{\pgfqpoint{3.121490in}{1.027060in}}%
\pgfpathlineto{\pgfqpoint{3.123293in}{1.029058in}}%
\pgfpathlineto{\pgfqpoint{3.127352in}{1.032975in}}%
\pgfpathlineto{\pgfqpoint{3.130509in}{1.035163in}}%
\pgfpathlineto{\pgfqpoint{3.133214in}{1.032407in}}%
\pgfpathlineto{\pgfqpoint{3.136822in}{1.022638in}}%
\pgfpathlineto{\pgfqpoint{3.139528in}{1.018689in}}%
\pgfpathlineto{\pgfqpoint{3.141783in}{1.019420in}}%
\pgfpathlineto{\pgfqpoint{3.146292in}{1.021168in}}%
\pgfpathlineto{\pgfqpoint{3.151704in}{1.018446in}}%
\pgfpathlineto{\pgfqpoint{3.154409in}{1.013288in}}%
\pgfpathlineto{\pgfqpoint{3.156664in}{1.010361in}}%
\pgfpathlineto{\pgfqpoint{3.160272in}{1.010522in}}%
\pgfpathlineto{\pgfqpoint{3.177859in}{1.018980in}}%
\pgfpathlineto{\pgfqpoint{3.179663in}{1.017619in}}%
\pgfpathlineto{\pgfqpoint{3.183270in}{1.013064in}}%
\pgfpathlineto{\pgfqpoint{3.185525in}{1.009480in}}%
\pgfpathlineto{\pgfqpoint{3.188682in}{1.004175in}}%
\pgfpathlineto{\pgfqpoint{3.190486in}{1.003857in}}%
\pgfpathlineto{\pgfqpoint{3.192289in}{1.006767in}}%
\pgfpathlineto{\pgfqpoint{3.197250in}{1.015579in}}%
\pgfpathlineto{\pgfqpoint{3.199054in}{1.015148in}}%
\pgfpathlineto{\pgfqpoint{3.201759in}{1.010725in}}%
\pgfpathlineto{\pgfqpoint{3.205818in}{1.005067in}}%
\pgfpathlineto{\pgfqpoint{3.208975in}{1.003978in}}%
\pgfpathlineto{\pgfqpoint{3.213033in}{1.006185in}}%
\pgfpathlineto{\pgfqpoint{3.217092in}{1.007801in}}%
\pgfpathlineto{\pgfqpoint{3.219798in}{1.009249in}}%
\pgfpathlineto{\pgfqpoint{3.221601in}{1.007511in}}%
\pgfpathlineto{\pgfqpoint{3.229268in}{0.994730in}}%
\pgfpathlineto{\pgfqpoint{3.230621in}{0.995543in}}%
\pgfpathlineto{\pgfqpoint{3.240992in}{1.003150in}}%
\pgfpathlineto{\pgfqpoint{3.243698in}{1.001789in}}%
\pgfpathlineto{\pgfqpoint{3.247306in}{1.003591in}}%
\pgfpathlineto{\pgfqpoint{3.254070in}{1.000787in}}%
\pgfpathlineto{\pgfqpoint{3.264893in}{0.981169in}}%
\pgfpathlineto{\pgfqpoint{3.265344in}{0.981615in}}%
\pgfpathlineto{\pgfqpoint{3.268952in}{0.986906in}}%
\pgfpathlineto{\pgfqpoint{3.273912in}{0.995445in}}%
\pgfpathlineto{\pgfqpoint{3.279324in}{0.996010in}}%
\pgfpathlineto{\pgfqpoint{3.282931in}{0.990708in}}%
\pgfpathlineto{\pgfqpoint{3.292401in}{0.973285in}}%
\pgfpathlineto{\pgfqpoint{3.296911in}{0.965675in}}%
\pgfpathlineto{\pgfqpoint{3.300518in}{0.962059in}}%
\pgfpathlineto{\pgfqpoint{3.303224in}{0.962102in}}%
\pgfpathlineto{\pgfqpoint{3.307283in}{0.964428in}}%
\pgfpathlineto{\pgfqpoint{3.311792in}{0.969470in}}%
\pgfpathlineto{\pgfqpoint{3.315400in}{0.970357in}}%
\pgfpathlineto{\pgfqpoint{3.323517in}{0.964580in}}%
\pgfpathlineto{\pgfqpoint{3.326674in}{0.966446in}}%
\pgfpathlineto{\pgfqpoint{3.329379in}{0.968449in}}%
\pgfpathlineto{\pgfqpoint{3.330732in}{0.969712in}}%
\pgfpathlineto{\pgfqpoint{3.336144in}{0.976036in}}%
\pgfpathlineto{\pgfqpoint{3.338399in}{0.976006in}}%
\pgfpathlineto{\pgfqpoint{3.341104in}{0.975655in}}%
\pgfpathlineto{\pgfqpoint{3.345614in}{0.979331in}}%
\pgfpathlineto{\pgfqpoint{3.348771in}{0.980191in}}%
\pgfpathlineto{\pgfqpoint{3.351025in}{0.979265in}}%
\pgfpathlineto{\pgfqpoint{3.353280in}{0.978799in}}%
\pgfpathlineto{\pgfqpoint{3.360044in}{0.973629in}}%
\pgfpathlineto{\pgfqpoint{3.363652in}{0.974197in}}%
\pgfpathlineto{\pgfqpoint{3.368162in}{0.971571in}}%
\pgfpathlineto{\pgfqpoint{3.372671in}{0.963429in}}%
\pgfpathlineto{\pgfqpoint{3.374926in}{0.961577in}}%
\pgfpathlineto{\pgfqpoint{3.381690in}{0.965272in}}%
\pgfpathlineto{\pgfqpoint{3.383494in}{0.964385in}}%
\pgfpathlineto{\pgfqpoint{3.391160in}{0.960390in}}%
\pgfpathlineto{\pgfqpoint{3.393866in}{0.957901in}}%
\pgfpathlineto{\pgfqpoint{3.398375in}{0.956570in}}%
\pgfpathlineto{\pgfqpoint{3.400630in}{0.956103in}}%
\pgfpathlineto{\pgfqpoint{3.405591in}{0.959515in}}%
\pgfpathlineto{\pgfqpoint{3.411453in}{0.960669in}}%
\pgfpathlineto{\pgfqpoint{3.414159in}{0.962147in}}%
\pgfpathlineto{\pgfqpoint{3.417316in}{0.966731in}}%
\pgfpathlineto{\pgfqpoint{3.424982in}{0.973132in}}%
\pgfpathlineto{\pgfqpoint{3.426335in}{0.971901in}}%
\pgfpathlineto{\pgfqpoint{3.437608in}{0.954309in}}%
\pgfpathlineto{\pgfqpoint{3.440765in}{0.950625in}}%
\pgfpathlineto{\pgfqpoint{3.445726in}{0.943967in}}%
\pgfpathlineto{\pgfqpoint{3.448882in}{0.941554in}}%
\pgfpathlineto{\pgfqpoint{3.454745in}{0.936295in}}%
\pgfpathlineto{\pgfqpoint{3.457000in}{0.935258in}}%
\pgfpathlineto{\pgfqpoint{3.458352in}{0.935868in}}%
\pgfpathlineto{\pgfqpoint{3.461509in}{0.936967in}}%
\pgfpathlineto{\pgfqpoint{3.466921in}{0.935867in}}%
\pgfpathlineto{\pgfqpoint{3.468724in}{0.936669in}}%
\pgfpathlineto{\pgfqpoint{3.471881in}{0.937142in}}%
\pgfpathlineto{\pgfqpoint{3.475038in}{0.937536in}}%
\pgfpathlineto{\pgfqpoint{3.482704in}{0.939994in}}%
\pgfpathlineto{\pgfqpoint{3.484508in}{0.940610in}}%
\pgfpathlineto{\pgfqpoint{3.487213in}{0.943535in}}%
\pgfpathlineto{\pgfqpoint{3.493076in}{0.952758in}}%
\pgfpathlineto{\pgfqpoint{3.495782in}{0.949702in}}%
\pgfpathlineto{\pgfqpoint{3.503899in}{0.938483in}}%
\pgfpathlineto{\pgfqpoint{3.507957in}{0.932870in}}%
\pgfpathlineto{\pgfqpoint{3.521486in}{0.922328in}}%
\pgfpathlineto{\pgfqpoint{3.525996in}{0.921863in}}%
\pgfpathlineto{\pgfqpoint{3.529152in}{0.923439in}}%
\pgfpathlineto{\pgfqpoint{3.532309in}{0.920115in}}%
\pgfpathlineto{\pgfqpoint{3.539975in}{0.908040in}}%
\pgfpathlineto{\pgfqpoint{3.545387in}{0.910655in}}%
\pgfpathlineto{\pgfqpoint{3.549896in}{0.916310in}}%
\pgfpathlineto{\pgfqpoint{3.554406in}{0.929768in}}%
\pgfpathlineto{\pgfqpoint{3.558013in}{0.937297in}}%
\pgfpathlineto{\pgfqpoint{3.561621in}{0.937821in}}%
\pgfpathlineto{\pgfqpoint{3.563876in}{0.937591in}}%
\pgfpathlineto{\pgfqpoint{3.569287in}{0.940752in}}%
\pgfpathlineto{\pgfqpoint{3.576051in}{0.942827in}}%
\pgfpathlineto{\pgfqpoint{3.582365in}{0.941206in}}%
\pgfpathlineto{\pgfqpoint{3.584620in}{0.941326in}}%
\pgfpathlineto{\pgfqpoint{3.593188in}{0.935539in}}%
\pgfpathlineto{\pgfqpoint{3.595893in}{0.934806in}}%
\pgfpathlineto{\pgfqpoint{3.599501in}{0.933056in}}%
\pgfpathlineto{\pgfqpoint{3.603109in}{0.934968in}}%
\pgfpathlineto{\pgfqpoint{3.608520in}{0.940426in}}%
\pgfpathlineto{\pgfqpoint{3.610775in}{0.940822in}}%
\pgfpathlineto{\pgfqpoint{3.613481in}{0.939248in}}%
\pgfpathlineto{\pgfqpoint{3.622951in}{0.921754in}}%
\pgfpathlineto{\pgfqpoint{3.626558in}{0.914493in}}%
\pgfpathlineto{\pgfqpoint{3.630617in}{0.911644in}}%
\pgfpathlineto{\pgfqpoint{3.633774in}{0.912809in}}%
\pgfpathlineto{\pgfqpoint{3.637832in}{0.915540in}}%
\pgfpathlineto{\pgfqpoint{3.645498in}{0.915058in}}%
\pgfpathlineto{\pgfqpoint{3.653616in}{0.910059in}}%
\pgfpathlineto{\pgfqpoint{3.656772in}{0.912365in}}%
\pgfpathlineto{\pgfqpoint{3.661282in}{0.916578in}}%
\pgfpathlineto{\pgfqpoint{3.663086in}{0.915342in}}%
\pgfpathlineto{\pgfqpoint{3.671654in}{0.902049in}}%
\pgfpathlineto{\pgfqpoint{3.680222in}{0.902354in}}%
\pgfpathlineto{\pgfqpoint{3.682026in}{0.900802in}}%
\pgfpathlineto{\pgfqpoint{3.688790in}{0.892122in}}%
\pgfpathlineto{\pgfqpoint{3.699613in}{0.892186in}}%
\pgfpathlineto{\pgfqpoint{3.702319in}{0.892932in}}%
\pgfpathlineto{\pgfqpoint{3.704122in}{0.891991in}}%
\pgfpathlineto{\pgfqpoint{3.705926in}{0.891724in}}%
\pgfpathlineto{\pgfqpoint{3.707730in}{0.891597in}}%
\pgfpathlineto{\pgfqpoint{3.717651in}{0.881900in}}%
\pgfpathlineto{\pgfqpoint{3.721259in}{0.885912in}}%
\pgfpathlineto{\pgfqpoint{3.728925in}{0.896573in}}%
\pgfpathlineto{\pgfqpoint{3.732082in}{0.901353in}}%
\pgfpathlineto{\pgfqpoint{3.733885in}{0.903667in}}%
\pgfpathlineto{\pgfqpoint{3.737493in}{0.908926in}}%
\pgfpathlineto{\pgfqpoint{3.740199in}{0.908650in}}%
\pgfpathlineto{\pgfqpoint{3.742454in}{0.905899in}}%
\pgfpathlineto{\pgfqpoint{3.749218in}{0.896316in}}%
\pgfpathlineto{\pgfqpoint{3.754178in}{0.897341in}}%
\pgfpathlineto{\pgfqpoint{3.757335in}{0.898064in}}%
\pgfpathlineto{\pgfqpoint{3.761845in}{0.895797in}}%
\pgfpathlineto{\pgfqpoint{3.773118in}{0.894047in}}%
\pgfpathlineto{\pgfqpoint{3.775824in}{0.889790in}}%
\pgfpathlineto{\pgfqpoint{3.782137in}{0.878188in}}%
\pgfpathlineto{\pgfqpoint{3.784392in}{0.877962in}}%
\pgfpathlineto{\pgfqpoint{3.786196in}{0.879091in}}%
\pgfpathlineto{\pgfqpoint{3.792058in}{0.884685in}}%
\pgfpathlineto{\pgfqpoint{3.796568in}{0.883326in}}%
\pgfpathlineto{\pgfqpoint{3.800627in}{0.878050in}}%
\pgfpathlineto{\pgfqpoint{3.806038in}{0.868001in}}%
\pgfpathlineto{\pgfqpoint{3.810999in}{0.866534in}}%
\pgfpathlineto{\pgfqpoint{3.813704in}{0.869823in}}%
\pgfpathlineto{\pgfqpoint{3.819567in}{0.880677in}}%
\pgfpathlineto{\pgfqpoint{3.824076in}{0.882789in}}%
\pgfpathlineto{\pgfqpoint{3.836252in}{0.899799in}}%
\pgfpathlineto{\pgfqpoint{3.838958in}{0.897857in}}%
\pgfpathlineto{\pgfqpoint{3.841212in}{0.897214in}}%
\pgfpathlineto{\pgfqpoint{3.843467in}{0.895356in}}%
\pgfpathlineto{\pgfqpoint{3.847977in}{0.888703in}}%
\pgfpathlineto{\pgfqpoint{3.858800in}{0.871626in}}%
\pgfpathlineto{\pgfqpoint{3.861956in}{0.869598in}}%
\pgfpathlineto{\pgfqpoint{3.868721in}{0.865856in}}%
\pgfpathlineto{\pgfqpoint{3.874583in}{0.863077in}}%
\pgfpathlineto{\pgfqpoint{3.891268in}{0.851780in}}%
\pgfpathlineto{\pgfqpoint{3.894425in}{0.851797in}}%
\pgfpathlineto{\pgfqpoint{3.900287in}{0.852859in}}%
\pgfpathlineto{\pgfqpoint{3.902542in}{0.851576in}}%
\pgfpathlineto{\pgfqpoint{3.906601in}{0.848670in}}%
\pgfpathlineto{\pgfqpoint{3.909307in}{0.852505in}}%
\pgfpathlineto{\pgfqpoint{3.915620in}{0.863527in}}%
\pgfpathlineto{\pgfqpoint{3.918326in}{0.863692in}}%
\pgfpathlineto{\pgfqpoint{3.924188in}{0.863891in}}%
\pgfpathlineto{\pgfqpoint{3.927345in}{0.865037in}}%
\pgfpathlineto{\pgfqpoint{3.930952in}{0.864605in}}%
\pgfpathlineto{\pgfqpoint{3.934560in}{0.866792in}}%
\pgfpathlineto{\pgfqpoint{3.937266in}{0.867466in}}%
\pgfpathlineto{\pgfqpoint{3.939520in}{0.866992in}}%
\pgfpathlineto{\pgfqpoint{3.941775in}{0.866077in}}%
\pgfpathlineto{\pgfqpoint{3.948540in}{0.866830in}}%
\pgfpathlineto{\pgfqpoint{3.950794in}{0.865541in}}%
\pgfpathlineto{\pgfqpoint{3.962970in}{0.850749in}}%
\pgfpathlineto{\pgfqpoint{3.964774in}{0.851351in}}%
\pgfpathlineto{\pgfqpoint{3.969283in}{0.851946in}}%
\pgfpathlineto{\pgfqpoint{3.976950in}{0.850873in}}%
\pgfpathlineto{\pgfqpoint{3.982361in}{0.846032in}}%
\pgfpathlineto{\pgfqpoint{3.985067in}{0.846878in}}%
\pgfpathlineto{\pgfqpoint{3.987773in}{0.849499in}}%
\pgfpathlineto{\pgfqpoint{3.998595in}{0.865646in}}%
\pgfpathlineto{\pgfqpoint{4.004909in}{0.873015in}}%
\pgfpathlineto{\pgfqpoint{4.013026in}{0.869913in}}%
\pgfpathlineto{\pgfqpoint{4.016183in}{0.864171in}}%
\pgfpathlineto{\pgfqpoint{4.020241in}{0.856674in}}%
\pgfpathlineto{\pgfqpoint{4.023849in}{0.854444in}}%
\pgfpathlineto{\pgfqpoint{4.031515in}{0.851697in}}%
\pgfpathlineto{\pgfqpoint{4.041436in}{0.851661in}}%
\pgfpathlineto{\pgfqpoint{4.043240in}{0.851814in}}%
\pgfpathlineto{\pgfqpoint{4.045946in}{0.854432in}}%
\pgfpathlineto{\pgfqpoint{4.049102in}{0.857553in}}%
\pgfpathlineto{\pgfqpoint{4.051357in}{0.856613in}}%
\pgfpathlineto{\pgfqpoint{4.054063in}{0.851750in}}%
\pgfpathlineto{\pgfqpoint{4.058572in}{0.842566in}}%
\pgfpathlineto{\pgfqpoint{4.064435in}{0.841434in}}%
\pgfpathlineto{\pgfqpoint{4.071199in}{0.844690in}}%
\pgfpathlineto{\pgfqpoint{4.076611in}{0.841870in}}%
\pgfpathlineto{\pgfqpoint{4.089237in}{0.827985in}}%
\pgfpathlineto{\pgfqpoint{4.093296in}{0.822498in}}%
\pgfpathlineto{\pgfqpoint{4.095551in}{0.822358in}}%
\pgfpathlineto{\pgfqpoint{4.097805in}{0.824131in}}%
\pgfpathlineto{\pgfqpoint{4.105923in}{0.834135in}}%
\pgfpathlineto{\pgfqpoint{4.110432in}{0.832086in}}%
\pgfpathlineto{\pgfqpoint{4.114040in}{0.829807in}}%
\pgfpathlineto{\pgfqpoint{4.115393in}{0.830846in}}%
\pgfpathlineto{\pgfqpoint{4.119902in}{0.835198in}}%
\pgfpathlineto{\pgfqpoint{4.124863in}{0.837718in}}%
\pgfpathlineto{\pgfqpoint{4.129823in}{0.835406in}}%
\pgfpathlineto{\pgfqpoint{4.139744in}{0.822629in}}%
\pgfpathlineto{\pgfqpoint{4.142901in}{0.818522in}}%
\pgfpathlineto{\pgfqpoint{4.149665in}{0.813291in}}%
\pgfpathlineto{\pgfqpoint{4.158233in}{0.809814in}}%
\pgfpathlineto{\pgfqpoint{4.161390in}{0.808006in}}%
\pgfpathlineto{\pgfqpoint{4.165449in}{0.809580in}}%
\pgfpathlineto{\pgfqpoint{4.169507in}{0.814106in}}%
\pgfpathlineto{\pgfqpoint{4.171311in}{0.817427in}}%
\pgfpathlineto{\pgfqpoint{4.177624in}{0.831286in}}%
\pgfpathlineto{\pgfqpoint{4.183036in}{0.839498in}}%
\pgfpathlineto{\pgfqpoint{4.186643in}{0.842542in}}%
\pgfpathlineto{\pgfqpoint{4.188898in}{0.841016in}}%
\pgfpathlineto{\pgfqpoint{4.192506in}{0.835084in}}%
\pgfpathlineto{\pgfqpoint{4.201074in}{0.821595in}}%
\pgfpathlineto{\pgfqpoint{4.204231in}{0.817864in}}%
\pgfpathlineto{\pgfqpoint{4.207387in}{0.814877in}}%
\pgfpathlineto{\pgfqpoint{4.211446in}{0.814381in}}%
\pgfpathlineto{\pgfqpoint{4.216406in}{0.817914in}}%
\pgfpathlineto{\pgfqpoint{4.222269in}{0.814708in}}%
\pgfpathlineto{\pgfqpoint{4.226327in}{0.812636in}}%
\pgfpathlineto{\pgfqpoint{4.233092in}{0.812189in}}%
\pgfpathlineto{\pgfqpoint{4.235797in}{0.810117in}}%
\pgfpathlineto{\pgfqpoint{4.242562in}{0.799946in}}%
\pgfpathlineto{\pgfqpoint{4.245267in}{0.800661in}}%
\pgfpathlineto{\pgfqpoint{4.248875in}{0.803000in}}%
\pgfpathlineto{\pgfqpoint{4.253385in}{0.807331in}}%
\pgfpathlineto{\pgfqpoint{4.268266in}{0.828825in}}%
\pgfpathlineto{\pgfqpoint{4.270972in}{0.828802in}}%
\pgfpathlineto{\pgfqpoint{4.274579in}{0.824565in}}%
\pgfpathlineto{\pgfqpoint{4.277736in}{0.822429in}}%
\pgfpathlineto{\pgfqpoint{4.279991in}{0.821464in}}%
\pgfpathlineto{\pgfqpoint{4.281795in}{0.821319in}}%
\pgfpathlineto{\pgfqpoint{4.285402in}{0.821139in}}%
\pgfpathlineto{\pgfqpoint{4.294421in}{0.817417in}}%
\pgfpathlineto{\pgfqpoint{4.297127in}{0.816386in}}%
\pgfpathlineto{\pgfqpoint{4.299833in}{0.813209in}}%
\pgfpathlineto{\pgfqpoint{4.307499in}{0.803469in}}%
\pgfpathlineto{\pgfqpoint{4.309754in}{0.804863in}}%
\pgfpathlineto{\pgfqpoint{4.313362in}{0.809577in}}%
\pgfpathlineto{\pgfqpoint{4.316067in}{0.812355in}}%
\pgfpathlineto{\pgfqpoint{4.318773in}{0.812524in}}%
\pgfpathlineto{\pgfqpoint{4.321479in}{0.811888in}}%
\pgfpathlineto{\pgfqpoint{4.326439in}{0.814080in}}%
\pgfpathlineto{\pgfqpoint{4.330047in}{0.812032in}}%
\pgfpathlineto{\pgfqpoint{4.338615in}{0.806159in}}%
\pgfpathlineto{\pgfqpoint{4.343575in}{0.805781in}}%
\pgfpathlineto{\pgfqpoint{4.349889in}{0.812611in}}%
\pgfpathlineto{\pgfqpoint{4.364770in}{0.829572in}}%
\pgfpathlineto{\pgfqpoint{4.369280in}{0.829242in}}%
\pgfpathlineto{\pgfqpoint{4.372887in}{0.827321in}}%
\pgfpathlineto{\pgfqpoint{4.376495in}{0.823489in}}%
\pgfpathlineto{\pgfqpoint{4.386867in}{0.810116in}}%
\pgfpathlineto{\pgfqpoint{4.393631in}{0.803727in}}%
\pgfpathlineto{\pgfqpoint{4.399945in}{0.799108in}}%
\pgfpathlineto{\pgfqpoint{4.402199in}{0.800225in}}%
\pgfpathlineto{\pgfqpoint{4.414826in}{0.815435in}}%
\pgfpathlineto{\pgfqpoint{4.422943in}{0.817297in}}%
\pgfpathlineto{\pgfqpoint{4.425649in}{0.814758in}}%
\pgfpathlineto{\pgfqpoint{4.432864in}{0.806283in}}%
\pgfpathlineto{\pgfqpoint{4.440080in}{0.809656in}}%
\pgfpathlineto{\pgfqpoint{4.442785in}{0.811027in}}%
\pgfpathlineto{\pgfqpoint{4.449099in}{0.809092in}}%
\pgfpathlineto{\pgfqpoint{4.451353in}{0.808191in}}%
\pgfpathlineto{\pgfqpoint{4.463980in}{0.811237in}}%
\pgfpathlineto{\pgfqpoint{4.471646in}{0.808064in}}%
\pgfpathlineto{\pgfqpoint{4.475254in}{0.800411in}}%
\pgfpathlineto{\pgfqpoint{4.478411in}{0.795020in}}%
\pgfpathlineto{\pgfqpoint{4.482018in}{0.790305in}}%
\pgfpathlineto{\pgfqpoint{4.490587in}{0.778240in}}%
\pgfpathlineto{\pgfqpoint{4.491939in}{0.778382in}}%
\pgfpathlineto{\pgfqpoint{4.500057in}{0.783273in}}%
\pgfpathlineto{\pgfqpoint{4.503664in}{0.785763in}}%
\pgfpathlineto{\pgfqpoint{4.514487in}{0.800331in}}%
\pgfpathlineto{\pgfqpoint{4.518546in}{0.804114in}}%
\pgfpathlineto{\pgfqpoint{4.523055in}{0.804010in}}%
\pgfpathlineto{\pgfqpoint{4.530270in}{0.804133in}}%
\pgfpathlineto{\pgfqpoint{4.534329in}{0.804117in}}%
\pgfpathlineto{\pgfqpoint{4.538839in}{0.800393in}}%
\pgfpathlineto{\pgfqpoint{4.543799in}{0.794531in}}%
\pgfpathlineto{\pgfqpoint{4.549211in}{0.787660in}}%
\pgfpathlineto{\pgfqpoint{4.553269in}{0.784708in}}%
\pgfpathlineto{\pgfqpoint{4.559132in}{0.784259in}}%
\pgfpathlineto{\pgfqpoint{4.562739in}{0.786550in}}%
\pgfpathlineto{\pgfqpoint{4.570856in}{0.797051in}}%
\pgfpathlineto{\pgfqpoint{4.577621in}{0.797326in}}%
\pgfpathlineto{\pgfqpoint{4.584836in}{0.793079in}}%
\pgfpathlineto{\pgfqpoint{4.592502in}{0.780773in}}%
\pgfpathlineto{\pgfqpoint{4.595659in}{0.777638in}}%
\pgfpathlineto{\pgfqpoint{4.606031in}{0.774285in}}%
\pgfpathlineto{\pgfqpoint{4.612344in}{0.776689in}}%
\pgfpathlineto{\pgfqpoint{4.615501in}{0.779598in}}%
\pgfpathlineto{\pgfqpoint{4.617756in}{0.779169in}}%
\pgfpathlineto{\pgfqpoint{4.624069in}{0.773016in}}%
\pgfpathlineto{\pgfqpoint{4.630382in}{0.766833in}}%
\pgfpathlineto{\pgfqpoint{4.642107in}{0.774522in}}%
\pgfpathlineto{\pgfqpoint{4.647519in}{0.774789in}}%
\pgfpathlineto{\pgfqpoint{4.653832in}{0.778615in}}%
\pgfpathlineto{\pgfqpoint{4.657440in}{0.779547in}}%
\pgfpathlineto{\pgfqpoint{4.662400in}{0.781982in}}%
\pgfpathlineto{\pgfqpoint{4.668713in}{0.784825in}}%
\pgfpathlineto{\pgfqpoint{4.683595in}{0.787171in}}%
\pgfpathlineto{\pgfqpoint{4.687653in}{0.792310in}}%
\pgfpathlineto{\pgfqpoint{4.692614in}{0.799499in}}%
\pgfpathlineto{\pgfqpoint{4.695771in}{0.800170in}}%
\pgfpathlineto{\pgfqpoint{4.705241in}{0.800941in}}%
\pgfpathlineto{\pgfqpoint{4.716064in}{0.791592in}}%
\pgfpathlineto{\pgfqpoint{4.721475in}{0.789505in}}%
\pgfpathlineto{\pgfqpoint{4.729141in}{0.784166in}}%
\pgfpathlineto{\pgfqpoint{4.731847in}{0.784774in}}%
\pgfpathlineto{\pgfqpoint{4.735455in}{0.784970in}}%
\pgfpathlineto{\pgfqpoint{4.739513in}{0.783393in}}%
\pgfpathlineto{\pgfqpoint{4.742219in}{0.784892in}}%
\pgfpathlineto{\pgfqpoint{4.758002in}{0.791793in}}%
\pgfpathlineto{\pgfqpoint{4.760257in}{0.790346in}}%
\pgfpathlineto{\pgfqpoint{4.764767in}{0.783780in}}%
\pgfpathlineto{\pgfqpoint{4.768825in}{0.775111in}}%
\pgfpathlineto{\pgfqpoint{4.771982in}{0.770674in}}%
\pgfpathlineto{\pgfqpoint{4.776041in}{0.768723in}}%
\pgfpathlineto{\pgfqpoint{4.778295in}{0.768805in}}%
\pgfpathlineto{\pgfqpoint{4.783707in}{0.772643in}}%
\pgfpathlineto{\pgfqpoint{4.788216in}{0.777369in}}%
\pgfpathlineto{\pgfqpoint{4.792275in}{0.778549in}}%
\pgfpathlineto{\pgfqpoint{4.796333in}{0.774957in}}%
\pgfpathlineto{\pgfqpoint{4.801294in}{0.771549in}}%
\pgfpathlineto{\pgfqpoint{4.811215in}{0.773296in}}%
\pgfpathlineto{\pgfqpoint{4.831057in}{0.759880in}}%
\pgfpathlineto{\pgfqpoint{4.835566in}{0.757165in}}%
\pgfpathlineto{\pgfqpoint{4.839625in}{0.756878in}}%
\pgfpathlineto{\pgfqpoint{4.845938in}{0.758172in}}%
\pgfpathlineto{\pgfqpoint{4.849546in}{0.759973in}}%
\pgfpathlineto{\pgfqpoint{4.852703in}{0.758554in}}%
\pgfpathlineto{\pgfqpoint{4.858565in}{0.755293in}}%
\pgfpathlineto{\pgfqpoint{4.861722in}{0.752948in}}%
\pgfpathlineto{\pgfqpoint{4.869388in}{0.744691in}}%
\pgfpathlineto{\pgfqpoint{4.873447in}{0.745428in}}%
\pgfpathlineto{\pgfqpoint{4.879309in}{0.746782in}}%
\pgfpathlineto{\pgfqpoint{4.883819in}{0.750580in}}%
\pgfpathlineto{\pgfqpoint{4.886524in}{0.753782in}}%
\pgfpathlineto{\pgfqpoint{4.891485in}{0.759460in}}%
\pgfpathlineto{\pgfqpoint{4.905464in}{0.765187in}}%
\pgfpathlineto{\pgfqpoint{4.908170in}{0.763608in}}%
\pgfpathlineto{\pgfqpoint{4.918542in}{0.751454in}}%
\pgfpathlineto{\pgfqpoint{4.922601in}{0.746330in}}%
\pgfpathlineto{\pgfqpoint{4.925306in}{0.745186in}}%
\pgfpathlineto{\pgfqpoint{4.931169in}{0.744665in}}%
\pgfpathlineto{\pgfqpoint{4.945599in}{0.741084in}}%
\pgfpathlineto{\pgfqpoint{4.949658in}{0.743360in}}%
\pgfpathlineto{\pgfqpoint{4.955971in}{0.744393in}}%
\pgfpathlineto{\pgfqpoint{4.960030in}{0.747379in}}%
\pgfpathlineto{\pgfqpoint{4.963637in}{0.746667in}}%
\pgfpathlineto{\pgfqpoint{4.971755in}{0.741403in}}%
\pgfpathlineto{\pgfqpoint{4.976715in}{0.735078in}}%
\pgfpathlineto{\pgfqpoint{4.983930in}{0.725412in}}%
\pgfpathlineto{\pgfqpoint{4.987087in}{0.725481in}}%
\pgfpathlineto{\pgfqpoint{4.996557in}{0.728610in}}%
\pgfpathlineto{\pgfqpoint{5.001518in}{0.731835in}}%
\pgfpathlineto{\pgfqpoint{5.014595in}{0.738983in}}%
\pgfpathlineto{\pgfqpoint{5.023163in}{0.738364in}}%
\pgfpathlineto{\pgfqpoint{5.026320in}{0.743040in}}%
\pgfpathlineto{\pgfqpoint{5.029477in}{0.746591in}}%
\pgfpathlineto{\pgfqpoint{5.032633in}{0.746699in}}%
\pgfpathlineto{\pgfqpoint{5.034888in}{0.746930in}}%
\pgfpathlineto{\pgfqpoint{5.038045in}{0.749739in}}%
\pgfpathlineto{\pgfqpoint{5.044809in}{0.756464in}}%
\pgfpathlineto{\pgfqpoint{5.049319in}{0.758113in}}%
\pgfpathlineto{\pgfqpoint{5.054279in}{0.755789in}}%
\pgfpathlineto{\pgfqpoint{5.065553in}{0.745573in}}%
\pgfpathlineto{\pgfqpoint{5.071866in}{0.741890in}}%
\pgfpathlineto{\pgfqpoint{5.079533in}{0.736833in}}%
\pgfpathlineto{\pgfqpoint{5.083591in}{0.735531in}}%
\pgfpathlineto{\pgfqpoint{5.087650in}{0.732148in}}%
\pgfpathlineto{\pgfqpoint{5.091257in}{0.729838in}}%
\pgfpathlineto{\pgfqpoint{5.093963in}{0.729524in}}%
\pgfpathlineto{\pgfqpoint{5.104786in}{0.737896in}}%
\pgfpathlineto{\pgfqpoint{5.108394in}{0.737478in}}%
\pgfpathlineto{\pgfqpoint{5.110198in}{0.738881in}}%
\pgfpathlineto{\pgfqpoint{5.124628in}{0.753080in}}%
\pgfpathlineto{\pgfqpoint{5.131392in}{0.754679in}}%
\pgfpathlineto{\pgfqpoint{5.134549in}{0.756094in}}%
\pgfpathlineto{\pgfqpoint{5.136353in}{0.757404in}}%
\pgfpathlineto{\pgfqpoint{5.147627in}{0.763496in}}%
\pgfpathlineto{\pgfqpoint{5.150332in}{0.761890in}}%
\pgfpathlineto{\pgfqpoint{5.154391in}{0.756049in}}%
\pgfpathlineto{\pgfqpoint{5.168371in}{0.729048in}}%
\pgfpathlineto{\pgfqpoint{5.174684in}{0.716978in}}%
\pgfpathlineto{\pgfqpoint{5.176488in}{0.716415in}}%
\pgfpathlineto{\pgfqpoint{5.179645in}{0.716573in}}%
\pgfpathlineto{\pgfqpoint{5.181899in}{0.715679in}}%
\pgfpathlineto{\pgfqpoint{5.189566in}{0.710435in}}%
\pgfpathlineto{\pgfqpoint{5.195428in}{0.710489in}}%
\pgfpathlineto{\pgfqpoint{5.198585in}{0.709464in}}%
\pgfpathlineto{\pgfqpoint{5.200388in}{0.708346in}}%
\pgfpathlineto{\pgfqpoint{5.203094in}{0.707211in}}%
\pgfpathlineto{\pgfqpoint{5.212113in}{0.704936in}}%
\pgfpathlineto{\pgfqpoint{5.215721in}{0.704990in}}%
\pgfpathlineto{\pgfqpoint{5.220681in}{0.700007in}}%
\pgfpathlineto{\pgfqpoint{5.223387in}{0.698065in}}%
\pgfpathlineto{\pgfqpoint{5.227446in}{0.699083in}}%
\pgfpathlineto{\pgfqpoint{5.229249in}{0.701528in}}%
\pgfpathlineto{\pgfqpoint{5.234210in}{0.709789in}}%
\pgfpathlineto{\pgfqpoint{5.239621in}{0.715422in}}%
\pgfpathlineto{\pgfqpoint{5.245033in}{0.724855in}}%
\pgfpathlineto{\pgfqpoint{5.251797in}{0.730826in}}%
\pgfpathlineto{\pgfqpoint{5.254954in}{0.730504in}}%
\pgfpathlineto{\pgfqpoint{5.257660in}{0.727144in}}%
\pgfpathlineto{\pgfqpoint{5.261267in}{0.720770in}}%
\pgfpathlineto{\pgfqpoint{5.272541in}{0.696236in}}%
\pgfpathlineto{\pgfqpoint{5.276149in}{0.696008in}}%
\pgfpathlineto{\pgfqpoint{5.279756in}{0.696384in}}%
\pgfpathlineto{\pgfqpoint{5.282913in}{0.698342in}}%
\pgfpathlineto{\pgfqpoint{5.287874in}{0.705540in}}%
\pgfpathlineto{\pgfqpoint{5.295991in}{0.717632in}}%
\pgfpathlineto{\pgfqpoint{5.299147in}{0.716826in}}%
\pgfpathlineto{\pgfqpoint{5.301853in}{0.716169in}}%
\pgfpathlineto{\pgfqpoint{5.304559in}{0.714652in}}%
\pgfpathlineto{\pgfqpoint{5.314931in}{0.706815in}}%
\pgfpathlineto{\pgfqpoint{5.319891in}{0.711300in}}%
\pgfpathlineto{\pgfqpoint{5.328910in}{0.722696in}}%
\pgfpathlineto{\pgfqpoint{5.332969in}{0.725640in}}%
\pgfpathlineto{\pgfqpoint{5.338380in}{0.725897in}}%
\pgfpathlineto{\pgfqpoint{5.349203in}{0.725014in}}%
\pgfpathlineto{\pgfqpoint{5.351458in}{0.726198in}}%
\pgfpathlineto{\pgfqpoint{5.354164in}{0.724674in}}%
\pgfpathlineto{\pgfqpoint{5.359575in}{0.721780in}}%
\pgfpathlineto{\pgfqpoint{5.363634in}{0.720528in}}%
\pgfpathlineto{\pgfqpoint{5.368594in}{0.720022in}}%
\pgfpathlineto{\pgfqpoint{5.372202in}{0.719900in}}%
\pgfpathlineto{\pgfqpoint{5.374457in}{0.717543in}}%
\pgfpathlineto{\pgfqpoint{5.382574in}{0.707289in}}%
\pgfpathlineto{\pgfqpoint{5.385280in}{0.706212in}}%
\pgfpathlineto{\pgfqpoint{5.397455in}{0.707893in}}%
\pgfpathlineto{\pgfqpoint{5.411435in}{0.722681in}}%
\pgfpathlineto{\pgfqpoint{5.419552in}{0.733269in}}%
\pgfpathlineto{\pgfqpoint{5.433532in}{0.746698in}}%
\pgfpathlineto{\pgfqpoint{5.436237in}{0.746166in}}%
\pgfpathlineto{\pgfqpoint{5.446609in}{0.739692in}}%
\pgfpathlineto{\pgfqpoint{5.452021in}{0.734077in}}%
\pgfpathlineto{\pgfqpoint{5.457432in}{0.732969in}}%
\pgfpathlineto{\pgfqpoint{5.463746in}{0.734664in}}%
\pgfpathlineto{\pgfqpoint{5.471412in}{0.738816in}}%
\pgfpathlineto{\pgfqpoint{5.488999in}{0.743310in}}%
\pgfpathlineto{\pgfqpoint{5.492156in}{0.744411in}}%
\pgfpathlineto{\pgfqpoint{5.494861in}{0.743544in}}%
\pgfpathlineto{\pgfqpoint{5.500724in}{0.740584in}}%
\pgfpathlineto{\pgfqpoint{5.504782in}{0.738076in}}%
\pgfpathlineto{\pgfqpoint{5.513802in}{0.733862in}}%
\pgfpathlineto{\pgfqpoint{5.516958in}{0.732133in}}%
\pgfpathlineto{\pgfqpoint{5.527330in}{0.723362in}}%
\pgfpathlineto{\pgfqpoint{5.534545in}{0.715745in}}%
\pgfpathlineto{\pgfqpoint{5.534545in}{0.715745in}}%
\pgfusepath{stroke}%
\end{pgfscope}%
\begin{pgfscope}%
\pgfpathrectangle{\pgfqpoint{0.800000in}{0.528000in}}{\pgfqpoint{4.960000in}{3.696000in}} %
\pgfusepath{clip}%
\pgfsetrectcap%
\pgfsetroundjoin%
\pgfsetlinewidth{1.505625pt}%
\definecolor{currentstroke}{rgb}{0.172549,0.627451,0.172549}%
\pgfsetstrokecolor{currentstroke}%
\pgfsetdash{}{0pt}%
\pgfpathmoveto{\pgfqpoint{1.025455in}{4.056000in}}%
\pgfpathlineto{\pgfqpoint{1.032670in}{2.867932in}}%
\pgfpathlineto{\pgfqpoint{1.034474in}{2.863608in}}%
\pgfpathlineto{\pgfqpoint{1.035826in}{2.862543in}}%
\pgfpathlineto{\pgfqpoint{1.041689in}{2.846835in}}%
\pgfpathlineto{\pgfqpoint{1.048002in}{2.811706in}}%
\pgfpathlineto{\pgfqpoint{1.052963in}{2.746777in}}%
\pgfpathlineto{\pgfqpoint{1.057021in}{2.678198in}}%
\pgfpathlineto{\pgfqpoint{1.060629in}{2.613099in}}%
\pgfpathlineto{\pgfqpoint{1.070550in}{2.455189in}}%
\pgfpathlineto{\pgfqpoint{1.073256in}{2.433600in}}%
\pgfpathlineto{\pgfqpoint{1.075961in}{2.397809in}}%
\pgfpathlineto{\pgfqpoint{1.080471in}{2.356612in}}%
\pgfpathlineto{\pgfqpoint{1.082275in}{2.344733in}}%
\pgfpathlineto{\pgfqpoint{1.086784in}{2.304288in}}%
\pgfpathlineto{\pgfqpoint{1.087686in}{2.299389in}}%
\pgfpathlineto{\pgfqpoint{1.089490in}{2.288987in}}%
\pgfpathlineto{\pgfqpoint{1.094000in}{2.247115in}}%
\pgfpathlineto{\pgfqpoint{1.095352in}{2.243478in}}%
\pgfpathlineto{\pgfqpoint{1.097156in}{2.235232in}}%
\pgfpathlineto{\pgfqpoint{1.101215in}{2.203241in}}%
\pgfpathlineto{\pgfqpoint{1.102117in}{2.202345in}}%
\pgfpathlineto{\pgfqpoint{1.112489in}{2.163339in}}%
\pgfpathlineto{\pgfqpoint{1.112940in}{2.162757in}}%
\pgfpathlineto{\pgfqpoint{1.113391in}{2.163153in}}%
\pgfpathlineto{\pgfqpoint{1.113842in}{2.163506in}}%
\pgfpathlineto{\pgfqpoint{1.121057in}{2.129528in}}%
\pgfpathlineto{\pgfqpoint{1.122410in}{2.128965in}}%
\pgfpathlineto{\pgfqpoint{1.122861in}{2.128892in}}%
\pgfpathlineto{\pgfqpoint{1.126919in}{2.115147in}}%
\pgfpathlineto{\pgfqpoint{1.130076in}{2.101767in}}%
\pgfpathlineto{\pgfqpoint{1.130527in}{2.101415in}}%
\pgfpathlineto{\pgfqpoint{1.130978in}{2.102687in}}%
\pgfpathlineto{\pgfqpoint{1.131429in}{2.101171in}}%
\pgfpathlineto{\pgfqpoint{1.132331in}{2.099233in}}%
\pgfpathlineto{\pgfqpoint{1.135487in}{2.089908in}}%
\pgfpathlineto{\pgfqpoint{1.138644in}{2.084684in}}%
\pgfpathlineto{\pgfqpoint{1.139546in}{2.082034in}}%
\pgfpathlineto{\pgfqpoint{1.146761in}{2.055146in}}%
\pgfpathlineto{\pgfqpoint{1.147212in}{2.056079in}}%
\pgfpathlineto{\pgfqpoint{1.147663in}{2.057389in}}%
\pgfpathlineto{\pgfqpoint{1.148114in}{2.056623in}}%
\pgfpathlineto{\pgfqpoint{1.150820in}{2.048653in}}%
\pgfpathlineto{\pgfqpoint{1.158937in}{2.023891in}}%
\pgfpathlineto{\pgfqpoint{1.159388in}{2.023853in}}%
\pgfpathlineto{\pgfqpoint{1.163897in}{2.005707in}}%
\pgfpathlineto{\pgfqpoint{1.166152in}{2.002172in}}%
\pgfpathlineto{\pgfqpoint{1.166603in}{2.002354in}}%
\pgfpathlineto{\pgfqpoint{1.167956in}{2.001666in}}%
\pgfpathlineto{\pgfqpoint{1.168858in}{2.000394in}}%
\pgfpathlineto{\pgfqpoint{1.177426in}{1.972810in}}%
\pgfpathlineto{\pgfqpoint{1.177877in}{1.973282in}}%
\pgfpathlineto{\pgfqpoint{1.178328in}{1.972559in}}%
\pgfpathlineto{\pgfqpoint{1.178779in}{1.974144in}}%
\pgfpathlineto{\pgfqpoint{1.179681in}{1.973286in}}%
\pgfpathlineto{\pgfqpoint{1.180583in}{1.972031in}}%
\pgfpathlineto{\pgfqpoint{1.183289in}{1.967114in}}%
\pgfpathlineto{\pgfqpoint{1.185543in}{1.962948in}}%
\pgfpathlineto{\pgfqpoint{1.186445in}{1.962339in}}%
\pgfpathlineto{\pgfqpoint{1.186896in}{1.963190in}}%
\pgfpathlineto{\pgfqpoint{1.187347in}{1.962334in}}%
\pgfpathlineto{\pgfqpoint{1.189151in}{1.960521in}}%
\pgfpathlineto{\pgfqpoint{1.189602in}{1.961086in}}%
\pgfpathlineto{\pgfqpoint{1.190504in}{1.961380in}}%
\pgfpathlineto{\pgfqpoint{1.198621in}{1.935489in}}%
\pgfpathlineto{\pgfqpoint{1.199072in}{1.935837in}}%
\pgfpathlineto{\pgfqpoint{1.199523in}{1.935085in}}%
\pgfpathlineto{\pgfqpoint{1.201327in}{1.933182in}}%
\pgfpathlineto{\pgfqpoint{1.201778in}{1.940266in}}%
\pgfpathlineto{\pgfqpoint{1.202229in}{1.935665in}}%
\pgfpathlineto{\pgfqpoint{1.203130in}{1.934173in}}%
\pgfpathlineto{\pgfqpoint{1.203581in}{1.934855in}}%
\pgfpathlineto{\pgfqpoint{1.208091in}{1.926901in}}%
\pgfpathlineto{\pgfqpoint{1.210346in}{1.922857in}}%
\pgfpathlineto{\pgfqpoint{1.212150in}{1.921429in}}%
\pgfpathlineto{\pgfqpoint{1.215757in}{1.919963in}}%
\pgfpathlineto{\pgfqpoint{1.217110in}{1.916445in}}%
\pgfpathlineto{\pgfqpoint{1.217561in}{1.918428in}}%
\pgfpathlineto{\pgfqpoint{1.218012in}{1.916586in}}%
\pgfpathlineto{\pgfqpoint{1.220267in}{1.908543in}}%
\pgfpathlineto{\pgfqpoint{1.222071in}{1.903786in}}%
\pgfpathlineto{\pgfqpoint{1.222522in}{1.903910in}}%
\pgfpathlineto{\pgfqpoint{1.224325in}{1.900343in}}%
\pgfpathlineto{\pgfqpoint{1.225678in}{1.899824in}}%
\pgfpathlineto{\pgfqpoint{1.227482in}{1.899904in}}%
\pgfpathlineto{\pgfqpoint{1.231541in}{1.889019in}}%
\pgfpathlineto{\pgfqpoint{1.231992in}{1.889307in}}%
\pgfpathlineto{\pgfqpoint{1.233795in}{1.883473in}}%
\pgfpathlineto{\pgfqpoint{1.236501in}{1.876354in}}%
\pgfpathlineto{\pgfqpoint{1.238756in}{1.871371in}}%
\pgfpathlineto{\pgfqpoint{1.245069in}{1.859818in}}%
\pgfpathlineto{\pgfqpoint{1.245971in}{1.860623in}}%
\pgfpathlineto{\pgfqpoint{1.247775in}{1.856418in}}%
\pgfpathlineto{\pgfqpoint{1.251383in}{1.853452in}}%
\pgfpathlineto{\pgfqpoint{1.253186in}{1.855647in}}%
\pgfpathlineto{\pgfqpoint{1.253637in}{1.854453in}}%
\pgfpathlineto{\pgfqpoint{1.254539in}{1.855181in}}%
\pgfpathlineto{\pgfqpoint{1.254990in}{1.857149in}}%
\pgfpathlineto{\pgfqpoint{1.255892in}{1.856645in}}%
\pgfpathlineto{\pgfqpoint{1.258147in}{1.856017in}}%
\pgfpathlineto{\pgfqpoint{1.265362in}{1.844137in}}%
\pgfpathlineto{\pgfqpoint{1.267617in}{1.840128in}}%
\pgfpathlineto{\pgfqpoint{1.268970in}{1.839038in}}%
\pgfpathlineto{\pgfqpoint{1.269421in}{1.840435in}}%
\pgfpathlineto{\pgfqpoint{1.269872in}{1.839466in}}%
\pgfpathlineto{\pgfqpoint{1.272126in}{1.838351in}}%
\pgfpathlineto{\pgfqpoint{1.276636in}{1.830146in}}%
\pgfpathlineto{\pgfqpoint{1.277087in}{1.831011in}}%
\pgfpathlineto{\pgfqpoint{1.278891in}{1.825885in}}%
\pgfpathlineto{\pgfqpoint{1.281146in}{1.823326in}}%
\pgfpathlineto{\pgfqpoint{1.281597in}{1.824104in}}%
\pgfpathlineto{\pgfqpoint{1.282498in}{1.823499in}}%
\pgfpathlineto{\pgfqpoint{1.283400in}{1.822327in}}%
\pgfpathlineto{\pgfqpoint{1.283851in}{1.822686in}}%
\pgfpathlineto{\pgfqpoint{1.284302in}{1.823618in}}%
\pgfpathlineto{\pgfqpoint{1.284753in}{1.822078in}}%
\pgfpathlineto{\pgfqpoint{1.290165in}{1.815942in}}%
\pgfpathlineto{\pgfqpoint{1.291067in}{1.814930in}}%
\pgfpathlineto{\pgfqpoint{1.295125in}{1.811496in}}%
\pgfpathlineto{\pgfqpoint{1.295576in}{1.812784in}}%
\pgfpathlineto{\pgfqpoint{1.296027in}{1.811368in}}%
\pgfpathlineto{\pgfqpoint{1.297380in}{1.812662in}}%
\pgfpathlineto{\pgfqpoint{1.298282in}{1.811823in}}%
\pgfpathlineto{\pgfqpoint{1.298733in}{1.812215in}}%
\pgfpathlineto{\pgfqpoint{1.301439in}{1.810669in}}%
\pgfpathlineto{\pgfqpoint{1.302791in}{1.809651in}}%
\pgfpathlineto{\pgfqpoint{1.306850in}{1.805568in}}%
\pgfpathlineto{\pgfqpoint{1.308203in}{1.802086in}}%
\pgfpathlineto{\pgfqpoint{1.308654in}{1.803354in}}%
\pgfpathlineto{\pgfqpoint{1.309105in}{1.801376in}}%
\pgfpathlineto{\pgfqpoint{1.315418in}{1.787133in}}%
\pgfpathlineto{\pgfqpoint{1.315869in}{1.789901in}}%
\pgfpathlineto{\pgfqpoint{1.316320in}{1.786972in}}%
\pgfpathlineto{\pgfqpoint{1.318124in}{1.782683in}}%
\pgfpathlineto{\pgfqpoint{1.319026in}{1.782545in}}%
\pgfpathlineto{\pgfqpoint{1.319477in}{1.783561in}}%
\pgfpathlineto{\pgfqpoint{1.319928in}{1.783240in}}%
\pgfpathlineto{\pgfqpoint{1.321731in}{1.782456in}}%
\pgfpathlineto{\pgfqpoint{1.323986in}{1.783228in}}%
\pgfpathlineto{\pgfqpoint{1.326241in}{1.779958in}}%
\pgfpathlineto{\pgfqpoint{1.327594in}{1.778658in}}%
\pgfpathlineto{\pgfqpoint{1.331201in}{1.769003in}}%
\pgfpathlineto{\pgfqpoint{1.333005in}{1.765323in}}%
\pgfpathlineto{\pgfqpoint{1.335260in}{1.761877in}}%
\pgfpathlineto{\pgfqpoint{1.337064in}{1.759284in}}%
\pgfpathlineto{\pgfqpoint{1.337515in}{1.760048in}}%
\pgfpathlineto{\pgfqpoint{1.338868in}{1.757069in}}%
\pgfpathlineto{\pgfqpoint{1.339319in}{1.760635in}}%
\pgfpathlineto{\pgfqpoint{1.339770in}{1.757790in}}%
\pgfpathlineto{\pgfqpoint{1.341122in}{1.756132in}}%
\pgfpathlineto{\pgfqpoint{1.341573in}{1.756842in}}%
\pgfpathlineto{\pgfqpoint{1.343828in}{1.755721in}}%
\pgfpathlineto{\pgfqpoint{1.346083in}{1.751708in}}%
\pgfpathlineto{\pgfqpoint{1.350142in}{1.739613in}}%
\pgfpathlineto{\pgfqpoint{1.352847in}{1.735903in}}%
\pgfpathlineto{\pgfqpoint{1.353298in}{1.737322in}}%
\pgfpathlineto{\pgfqpoint{1.353749in}{1.736116in}}%
\pgfpathlineto{\pgfqpoint{1.356906in}{1.729920in}}%
\pgfpathlineto{\pgfqpoint{1.360514in}{1.724934in}}%
\pgfpathlineto{\pgfqpoint{1.365474in}{1.723729in}}%
\pgfpathlineto{\pgfqpoint{1.376297in}{1.703423in}}%
\pgfpathlineto{\pgfqpoint{1.379454in}{1.700781in}}%
\pgfpathlineto{\pgfqpoint{1.379905in}{1.701201in}}%
\pgfpathlineto{\pgfqpoint{1.380355in}{1.700149in}}%
\pgfpathlineto{\pgfqpoint{1.382159in}{1.698472in}}%
\pgfpathlineto{\pgfqpoint{1.382610in}{1.699015in}}%
\pgfpathlineto{\pgfqpoint{1.384865in}{1.694496in}}%
\pgfpathlineto{\pgfqpoint{1.395237in}{1.680059in}}%
\pgfpathlineto{\pgfqpoint{1.395688in}{1.680305in}}%
\pgfpathlineto{\pgfqpoint{1.396139in}{1.683878in}}%
\pgfpathlineto{\pgfqpoint{1.396590in}{1.681322in}}%
\pgfpathlineto{\pgfqpoint{1.399747in}{1.683758in}}%
\pgfpathlineto{\pgfqpoint{1.401550in}{1.682869in}}%
\pgfpathlineto{\pgfqpoint{1.402001in}{1.684943in}}%
\pgfpathlineto{\pgfqpoint{1.402452in}{1.681915in}}%
\pgfpathlineto{\pgfqpoint{1.410569in}{1.667918in}}%
\pgfpathlineto{\pgfqpoint{1.412373in}{1.669570in}}%
\pgfpathlineto{\pgfqpoint{1.414628in}{1.671469in}}%
\pgfpathlineto{\pgfqpoint{1.420490in}{1.670643in}}%
\pgfpathlineto{\pgfqpoint{1.423196in}{1.664892in}}%
\pgfpathlineto{\pgfqpoint{1.427255in}{1.656986in}}%
\pgfpathlineto{\pgfqpoint{1.430411in}{1.651463in}}%
\pgfpathlineto{\pgfqpoint{1.433568in}{1.650030in}}%
\pgfpathlineto{\pgfqpoint{1.434019in}{1.650710in}}%
\pgfpathlineto{\pgfqpoint{1.434921in}{1.648757in}}%
\pgfpathlineto{\pgfqpoint{1.435372in}{1.649283in}}%
\pgfpathlineto{\pgfqpoint{1.437627in}{1.648597in}}%
\pgfpathlineto{\pgfqpoint{1.438078in}{1.652584in}}%
\pgfpathlineto{\pgfqpoint{1.438529in}{1.650573in}}%
\pgfpathlineto{\pgfqpoint{1.438980in}{1.649878in}}%
\pgfpathlineto{\pgfqpoint{1.439430in}{1.650240in}}%
\pgfpathlineto{\pgfqpoint{1.442587in}{1.656387in}}%
\pgfpathlineto{\pgfqpoint{1.443038in}{1.655374in}}%
\pgfpathlineto{\pgfqpoint{1.443489in}{1.655355in}}%
\pgfpathlineto{\pgfqpoint{1.443940in}{1.657153in}}%
\pgfpathlineto{\pgfqpoint{1.444842in}{1.656333in}}%
\pgfpathlineto{\pgfqpoint{1.450253in}{1.652224in}}%
\pgfpathlineto{\pgfqpoint{1.455214in}{1.649276in}}%
\pgfpathlineto{\pgfqpoint{1.455665in}{1.649734in}}%
\pgfpathlineto{\pgfqpoint{1.456116in}{1.648944in}}%
\pgfpathlineto{\pgfqpoint{1.462880in}{1.647655in}}%
\pgfpathlineto{\pgfqpoint{1.463331in}{1.649309in}}%
\pgfpathlineto{\pgfqpoint{1.464233in}{1.648924in}}%
\pgfpathlineto{\pgfqpoint{1.466488in}{1.649718in}}%
\pgfpathlineto{\pgfqpoint{1.467390in}{1.650687in}}%
\pgfpathlineto{\pgfqpoint{1.467841in}{1.650066in}}%
\pgfpathlineto{\pgfqpoint{1.470095in}{1.647799in}}%
\pgfpathlineto{\pgfqpoint{1.473703in}{1.640843in}}%
\pgfpathlineto{\pgfqpoint{1.476409in}{1.634840in}}%
\pgfpathlineto{\pgfqpoint{1.480467in}{1.629404in}}%
\pgfpathlineto{\pgfqpoint{1.480918in}{1.630977in}}%
\pgfpathlineto{\pgfqpoint{1.481369in}{1.628620in}}%
\pgfpathlineto{\pgfqpoint{1.486781in}{1.621457in}}%
\pgfpathlineto{\pgfqpoint{1.487232in}{1.622505in}}%
\pgfpathlineto{\pgfqpoint{1.489035in}{1.618811in}}%
\pgfpathlineto{\pgfqpoint{1.491741in}{1.618096in}}%
\pgfpathlineto{\pgfqpoint{1.494447in}{1.618695in}}%
\pgfpathlineto{\pgfqpoint{1.495800in}{1.619554in}}%
\pgfpathlineto{\pgfqpoint{1.497604in}{1.619007in}}%
\pgfpathlineto{\pgfqpoint{1.499858in}{1.619923in}}%
\pgfpathlineto{\pgfqpoint{1.502564in}{1.621624in}}%
\pgfpathlineto{\pgfqpoint{1.507976in}{1.625022in}}%
\pgfpathlineto{\pgfqpoint{1.509779in}{1.624246in}}%
\pgfpathlineto{\pgfqpoint{1.516093in}{1.620774in}}%
\pgfpathlineto{\pgfqpoint{1.521955in}{1.619346in}}%
\pgfpathlineto{\pgfqpoint{1.527367in}{1.622125in}}%
\pgfpathlineto{\pgfqpoint{1.528268in}{1.623551in}}%
\pgfpathlineto{\pgfqpoint{1.528719in}{1.623144in}}%
\pgfpathlineto{\pgfqpoint{1.530974in}{1.623838in}}%
\pgfpathlineto{\pgfqpoint{1.531425in}{1.624964in}}%
\pgfpathlineto{\pgfqpoint{1.532327in}{1.624319in}}%
\pgfpathlineto{\pgfqpoint{1.533680in}{1.624572in}}%
\pgfpathlineto{\pgfqpoint{1.539993in}{1.613166in}}%
\pgfpathlineto{\pgfqpoint{1.545856in}{1.598037in}}%
\pgfpathlineto{\pgfqpoint{1.549914in}{1.596628in}}%
\pgfpathlineto{\pgfqpoint{1.552169in}{1.598791in}}%
\pgfpathlineto{\pgfqpoint{1.555777in}{1.600018in}}%
\pgfpathlineto{\pgfqpoint{1.560737in}{1.595561in}}%
\pgfpathlineto{\pgfqpoint{1.569756in}{1.572661in}}%
\pgfpathlineto{\pgfqpoint{1.570207in}{1.575115in}}%
\pgfpathlineto{\pgfqpoint{1.570658in}{1.573023in}}%
\pgfpathlineto{\pgfqpoint{1.575168in}{1.581591in}}%
\pgfpathlineto{\pgfqpoint{1.576972in}{1.584494in}}%
\pgfpathlineto{\pgfqpoint{1.578775in}{1.584660in}}%
\pgfpathlineto{\pgfqpoint{1.581932in}{1.582364in}}%
\pgfpathlineto{\pgfqpoint{1.586442in}{1.574855in}}%
\pgfpathlineto{\pgfqpoint{1.592755in}{1.567737in}}%
\pgfpathlineto{\pgfqpoint{1.595912in}{1.566455in}}%
\pgfpathlineto{\pgfqpoint{1.597264in}{1.566540in}}%
\pgfpathlineto{\pgfqpoint{1.597715in}{1.567282in}}%
\pgfpathlineto{\pgfqpoint{1.598166in}{1.566503in}}%
\pgfpathlineto{\pgfqpoint{1.603127in}{1.565614in}}%
\pgfpathlineto{\pgfqpoint{1.607185in}{1.560361in}}%
\pgfpathlineto{\pgfqpoint{1.610342in}{1.555072in}}%
\pgfpathlineto{\pgfqpoint{1.612597in}{1.554226in}}%
\pgfpathlineto{\pgfqpoint{1.614401in}{1.554809in}}%
\pgfpathlineto{\pgfqpoint{1.623420in}{1.564744in}}%
\pgfpathlineto{\pgfqpoint{1.626576in}{1.561049in}}%
\pgfpathlineto{\pgfqpoint{1.628380in}{1.558531in}}%
\pgfpathlineto{\pgfqpoint{1.632890in}{1.553657in}}%
\pgfpathlineto{\pgfqpoint{1.635145in}{1.553407in}}%
\pgfpathlineto{\pgfqpoint{1.638301in}{1.555548in}}%
\pgfpathlineto{\pgfqpoint{1.642360in}{1.558913in}}%
\pgfpathlineto{\pgfqpoint{1.656339in}{1.569196in}}%
\pgfpathlineto{\pgfqpoint{1.659045in}{1.569345in}}%
\pgfpathlineto{\pgfqpoint{1.660849in}{1.569296in}}%
\pgfpathlineto{\pgfqpoint{1.662202in}{1.570472in}}%
\pgfpathlineto{\pgfqpoint{1.664908in}{1.570097in}}%
\pgfpathlineto{\pgfqpoint{1.667613in}{1.568721in}}%
\pgfpathlineto{\pgfqpoint{1.675280in}{1.559908in}}%
\pgfpathlineto{\pgfqpoint{1.677083in}{1.559544in}}%
\pgfpathlineto{\pgfqpoint{1.677534in}{1.560563in}}%
\pgfpathlineto{\pgfqpoint{1.677985in}{1.559555in}}%
\pgfpathlineto{\pgfqpoint{1.682495in}{1.557879in}}%
\pgfpathlineto{\pgfqpoint{1.682946in}{1.560526in}}%
\pgfpathlineto{\pgfqpoint{1.683397in}{1.557468in}}%
\pgfpathlineto{\pgfqpoint{1.688357in}{1.551371in}}%
\pgfpathlineto{\pgfqpoint{1.690161in}{1.550836in}}%
\pgfpathlineto{\pgfqpoint{1.690612in}{1.551633in}}%
\pgfpathlineto{\pgfqpoint{1.691063in}{1.550650in}}%
\pgfpathlineto{\pgfqpoint{1.692867in}{1.550733in}}%
\pgfpathlineto{\pgfqpoint{1.698278in}{1.552831in}}%
\pgfpathlineto{\pgfqpoint{1.704592in}{1.548263in}}%
\pgfpathlineto{\pgfqpoint{1.710003in}{1.543763in}}%
\pgfpathlineto{\pgfqpoint{1.714062in}{1.543330in}}%
\pgfpathlineto{\pgfqpoint{1.728041in}{1.544201in}}%
\pgfpathlineto{\pgfqpoint{1.734355in}{1.536253in}}%
\pgfpathlineto{\pgfqpoint{1.737962in}{1.529557in}}%
\pgfpathlineto{\pgfqpoint{1.739315in}{1.528380in}}%
\pgfpathlineto{\pgfqpoint{1.742021in}{1.526021in}}%
\pgfpathlineto{\pgfqpoint{1.744276in}{1.524868in}}%
\pgfpathlineto{\pgfqpoint{1.744726in}{1.526529in}}%
\pgfpathlineto{\pgfqpoint{1.745177in}{1.525156in}}%
\pgfpathlineto{\pgfqpoint{1.747432in}{1.523845in}}%
\pgfpathlineto{\pgfqpoint{1.753295in}{1.520304in}}%
\pgfpathlineto{\pgfqpoint{1.765470in}{1.503530in}}%
\pgfpathlineto{\pgfqpoint{1.769529in}{1.497675in}}%
\pgfpathlineto{\pgfqpoint{1.774038in}{1.496757in}}%
\pgfpathlineto{\pgfqpoint{1.774940in}{1.496773in}}%
\pgfpathlineto{\pgfqpoint{1.775842in}{1.497819in}}%
\pgfpathlineto{\pgfqpoint{1.776293in}{1.497362in}}%
\pgfpathlineto{\pgfqpoint{1.778548in}{1.497963in}}%
\pgfpathlineto{\pgfqpoint{1.781705in}{1.500616in}}%
\pgfpathlineto{\pgfqpoint{1.783959in}{1.503198in}}%
\pgfpathlineto{\pgfqpoint{1.788920in}{1.505802in}}%
\pgfpathlineto{\pgfqpoint{1.791175in}{1.504132in}}%
\pgfpathlineto{\pgfqpoint{1.800194in}{1.494402in}}%
\pgfpathlineto{\pgfqpoint{1.801547in}{1.495884in}}%
\pgfpathlineto{\pgfqpoint{1.801998in}{1.494359in}}%
\pgfpathlineto{\pgfqpoint{1.803801in}{1.493588in}}%
\pgfpathlineto{\pgfqpoint{1.808311in}{1.492919in}}%
\pgfpathlineto{\pgfqpoint{1.812821in}{1.488946in}}%
\pgfpathlineto{\pgfqpoint{1.816428in}{1.485483in}}%
\pgfpathlineto{\pgfqpoint{1.816879in}{1.486338in}}%
\pgfpathlineto{\pgfqpoint{1.817330in}{1.485259in}}%
\pgfpathlineto{\pgfqpoint{1.820036in}{1.483951in}}%
\pgfpathlineto{\pgfqpoint{1.822742in}{1.484443in}}%
\pgfpathlineto{\pgfqpoint{1.825447in}{1.485757in}}%
\pgfpathlineto{\pgfqpoint{1.829957in}{1.486221in}}%
\pgfpathlineto{\pgfqpoint{1.839427in}{1.484820in}}%
\pgfpathlineto{\pgfqpoint{1.841231in}{1.485392in}}%
\pgfpathlineto{\pgfqpoint{1.845740in}{1.486280in}}%
\pgfpathlineto{\pgfqpoint{1.846191in}{1.487703in}}%
\pgfpathlineto{\pgfqpoint{1.847093in}{1.486872in}}%
\pgfpathlineto{\pgfqpoint{1.851603in}{1.490664in}}%
\pgfpathlineto{\pgfqpoint{1.853406in}{1.492250in}}%
\pgfpathlineto{\pgfqpoint{1.853857in}{1.491852in}}%
\pgfpathlineto{\pgfqpoint{1.857014in}{1.490452in}}%
\pgfpathlineto{\pgfqpoint{1.861524in}{1.484288in}}%
\pgfpathlineto{\pgfqpoint{1.865582in}{1.477538in}}%
\pgfpathlineto{\pgfqpoint{1.868739in}{1.474515in}}%
\pgfpathlineto{\pgfqpoint{1.869190in}{1.476267in}}%
\pgfpathlineto{\pgfqpoint{1.869641in}{1.474669in}}%
\pgfpathlineto{\pgfqpoint{1.871445in}{1.473349in}}%
\pgfpathlineto{\pgfqpoint{1.874150in}{1.472486in}}%
\pgfpathlineto{\pgfqpoint{1.877758in}{1.469754in}}%
\pgfpathlineto{\pgfqpoint{1.878209in}{1.470613in}}%
\pgfpathlineto{\pgfqpoint{1.878660in}{1.469374in}}%
\pgfpathlineto{\pgfqpoint{1.881366in}{1.467653in}}%
\pgfpathlineto{\pgfqpoint{1.881817in}{1.468373in}}%
\pgfpathlineto{\pgfqpoint{1.882267in}{1.467027in}}%
\pgfpathlineto{\pgfqpoint{1.892639in}{1.460516in}}%
\pgfpathlineto{\pgfqpoint{1.894894in}{1.462953in}}%
\pgfpathlineto{\pgfqpoint{1.901659in}{1.471563in}}%
\pgfpathlineto{\pgfqpoint{1.904364in}{1.470555in}}%
\pgfpathlineto{\pgfqpoint{1.908874in}{1.465491in}}%
\pgfpathlineto{\pgfqpoint{1.911580in}{1.460238in}}%
\pgfpathlineto{\pgfqpoint{1.914736in}{1.455185in}}%
\pgfpathlineto{\pgfqpoint{1.915187in}{1.455711in}}%
\pgfpathlineto{\pgfqpoint{1.917893in}{1.451006in}}%
\pgfpathlineto{\pgfqpoint{1.923304in}{1.449176in}}%
\pgfpathlineto{\pgfqpoint{1.925108in}{1.449901in}}%
\pgfpathlineto{\pgfqpoint{1.925559in}{1.451803in}}%
\pgfpathlineto{\pgfqpoint{1.926461in}{1.451091in}}%
\pgfpathlineto{\pgfqpoint{1.928716in}{1.452489in}}%
\pgfpathlineto{\pgfqpoint{1.929618in}{1.452991in}}%
\pgfpathlineto{\pgfqpoint{1.930069in}{1.455677in}}%
\pgfpathlineto{\pgfqpoint{1.930971in}{1.454416in}}%
\pgfpathlineto{\pgfqpoint{1.937735in}{1.455950in}}%
\pgfpathlineto{\pgfqpoint{1.945401in}{1.451358in}}%
\pgfpathlineto{\pgfqpoint{1.948107in}{1.448457in}}%
\pgfpathlineto{\pgfqpoint{1.949009in}{1.447849in}}%
\pgfpathlineto{\pgfqpoint{1.958028in}{1.439310in}}%
\pgfpathlineto{\pgfqpoint{1.963890in}{1.438775in}}%
\pgfpathlineto{\pgfqpoint{1.965243in}{1.438951in}}%
\pgfpathlineto{\pgfqpoint{1.972458in}{1.440556in}}%
\pgfpathlineto{\pgfqpoint{1.981477in}{1.444952in}}%
\pgfpathlineto{\pgfqpoint{1.984634in}{1.444720in}}%
\pgfpathlineto{\pgfqpoint{1.990046in}{1.434703in}}%
\pgfpathlineto{\pgfqpoint{1.993653in}{1.428704in}}%
\pgfpathlineto{\pgfqpoint{1.994104in}{1.429481in}}%
\pgfpathlineto{\pgfqpoint{1.995908in}{1.426122in}}%
\pgfpathlineto{\pgfqpoint{2.003123in}{1.421550in}}%
\pgfpathlineto{\pgfqpoint{2.004927in}{1.422044in}}%
\pgfpathlineto{\pgfqpoint{2.007633in}{1.421784in}}%
\pgfpathlineto{\pgfqpoint{2.008986in}{1.421443in}}%
\pgfpathlineto{\pgfqpoint{2.015299in}{1.419455in}}%
\pgfpathlineto{\pgfqpoint{2.017103in}{1.420101in}}%
\pgfpathlineto{\pgfqpoint{2.027926in}{1.423259in}}%
\pgfpathlineto{\pgfqpoint{2.032886in}{1.422260in}}%
\pgfpathlineto{\pgfqpoint{2.036043in}{1.418867in}}%
\pgfpathlineto{\pgfqpoint{2.041003in}{1.409539in}}%
\pgfpathlineto{\pgfqpoint{2.046415in}{1.400970in}}%
\pgfpathlineto{\pgfqpoint{2.047768in}{1.399469in}}%
\pgfpathlineto{\pgfqpoint{2.050473in}{1.397566in}}%
\pgfpathlineto{\pgfqpoint{2.062649in}{1.407002in}}%
\pgfpathlineto{\pgfqpoint{2.064453in}{1.407243in}}%
\pgfpathlineto{\pgfqpoint{2.069413in}{1.408476in}}%
\pgfpathlineto{\pgfqpoint{2.081589in}{1.404167in}}%
\pgfpathlineto{\pgfqpoint{2.091510in}{1.406661in}}%
\pgfpathlineto{\pgfqpoint{2.094667in}{1.405380in}}%
\pgfpathlineto{\pgfqpoint{2.098726in}{1.402699in}}%
\pgfpathlineto{\pgfqpoint{2.103235in}{1.395659in}}%
\pgfpathlineto{\pgfqpoint{2.107294in}{1.390136in}}%
\pgfpathlineto{\pgfqpoint{2.110450in}{1.388813in}}%
\pgfpathlineto{\pgfqpoint{2.112705in}{1.388601in}}%
\pgfpathlineto{\pgfqpoint{2.118117in}{1.386769in}}%
\pgfpathlineto{\pgfqpoint{2.128939in}{1.377129in}}%
\pgfpathlineto{\pgfqpoint{2.129390in}{1.378525in}}%
\pgfpathlineto{\pgfqpoint{2.129841in}{1.377715in}}%
\pgfpathlineto{\pgfqpoint{2.131645in}{1.376227in}}%
\pgfpathlineto{\pgfqpoint{2.134351in}{1.377277in}}%
\pgfpathlineto{\pgfqpoint{2.136606in}{1.380010in}}%
\pgfpathlineto{\pgfqpoint{2.139311in}{1.384196in}}%
\pgfpathlineto{\pgfqpoint{2.144272in}{1.395274in}}%
\pgfpathlineto{\pgfqpoint{2.145174in}{1.393497in}}%
\pgfpathlineto{\pgfqpoint{2.149683in}{1.393269in}}%
\pgfpathlineto{\pgfqpoint{2.155546in}{1.388939in}}%
\pgfpathlineto{\pgfqpoint{2.155997in}{1.389320in}}%
\pgfpathlineto{\pgfqpoint{2.156448in}{1.388646in}}%
\pgfpathlineto{\pgfqpoint{2.159153in}{1.387904in}}%
\pgfpathlineto{\pgfqpoint{2.164114in}{1.385909in}}%
\pgfpathlineto{\pgfqpoint{2.164565in}{1.385640in}}%
\pgfpathlineto{\pgfqpoint{2.165016in}{1.387108in}}%
\pgfpathlineto{\pgfqpoint{2.165467in}{1.385197in}}%
\pgfpathlineto{\pgfqpoint{2.169525in}{1.385838in}}%
\pgfpathlineto{\pgfqpoint{2.176741in}{1.392078in}}%
\pgfpathlineto{\pgfqpoint{2.179897in}{1.391220in}}%
\pgfpathlineto{\pgfqpoint{2.188014in}{1.384528in}}%
\pgfpathlineto{\pgfqpoint{2.188465in}{1.386853in}}%
\pgfpathlineto{\pgfqpoint{2.188916in}{1.384528in}}%
\pgfpathlineto{\pgfqpoint{2.190269in}{1.383090in}}%
\pgfpathlineto{\pgfqpoint{2.190720in}{1.384396in}}%
\pgfpathlineto{\pgfqpoint{2.191171in}{1.383053in}}%
\pgfpathlineto{\pgfqpoint{2.193877in}{1.381849in}}%
\pgfpathlineto{\pgfqpoint{2.197034in}{1.380700in}}%
\pgfpathlineto{\pgfqpoint{2.201543in}{1.382036in}}%
\pgfpathlineto{\pgfqpoint{2.204249in}{1.382596in}}%
\pgfpathlineto{\pgfqpoint{2.205602in}{1.381974in}}%
\pgfpathlineto{\pgfqpoint{2.206053in}{1.383685in}}%
\pgfpathlineto{\pgfqpoint{2.206504in}{1.382028in}}%
\pgfpathlineto{\pgfqpoint{2.209209in}{1.379494in}}%
\pgfpathlineto{\pgfqpoint{2.217326in}{1.376021in}}%
\pgfpathlineto{\pgfqpoint{2.218679in}{1.375734in}}%
\pgfpathlineto{\pgfqpoint{2.226796in}{1.372854in}}%
\pgfpathlineto{\pgfqpoint{2.239423in}{1.374700in}}%
\pgfpathlineto{\pgfqpoint{2.239874in}{1.374584in}}%
\pgfpathlineto{\pgfqpoint{2.240325in}{1.375725in}}%
\pgfpathlineto{\pgfqpoint{2.240776in}{1.374621in}}%
\pgfpathlineto{\pgfqpoint{2.253854in}{1.356588in}}%
\pgfpathlineto{\pgfqpoint{2.257461in}{1.357253in}}%
\pgfpathlineto{\pgfqpoint{2.261069in}{1.360311in}}%
\pgfpathlineto{\pgfqpoint{2.266931in}{1.365951in}}%
\pgfpathlineto{\pgfqpoint{2.270539in}{1.365403in}}%
\pgfpathlineto{\pgfqpoint{2.283617in}{1.356017in}}%
\pgfpathlineto{\pgfqpoint{2.284068in}{1.355993in}}%
\pgfpathlineto{\pgfqpoint{2.284519in}{1.357602in}}%
\pgfpathlineto{\pgfqpoint{2.284970in}{1.356702in}}%
\pgfpathlineto{\pgfqpoint{2.286773in}{1.356770in}}%
\pgfpathlineto{\pgfqpoint{2.290832in}{1.357532in}}%
\pgfpathlineto{\pgfqpoint{2.292636in}{1.356458in}}%
\pgfpathlineto{\pgfqpoint{2.295792in}{1.352778in}}%
\pgfpathlineto{\pgfqpoint{2.298498in}{1.348156in}}%
\pgfpathlineto{\pgfqpoint{2.302106in}{1.343344in}}%
\pgfpathlineto{\pgfqpoint{2.305713in}{1.341226in}}%
\pgfpathlineto{\pgfqpoint{2.312027in}{1.338933in}}%
\pgfpathlineto{\pgfqpoint{2.316536in}{1.333994in}}%
\pgfpathlineto{\pgfqpoint{2.317889in}{1.334053in}}%
\pgfpathlineto{\pgfqpoint{2.319242in}{1.332862in}}%
\pgfpathlineto{\pgfqpoint{2.321046in}{1.333529in}}%
\pgfpathlineto{\pgfqpoint{2.330065in}{1.338876in}}%
\pgfpathlineto{\pgfqpoint{2.337731in}{1.337034in}}%
\pgfpathlineto{\pgfqpoint{2.341339in}{1.335414in}}%
\pgfpathlineto{\pgfqpoint{2.345397in}{1.334823in}}%
\pgfpathlineto{\pgfqpoint{2.348103in}{1.336669in}}%
\pgfpathlineto{\pgfqpoint{2.354867in}{1.342041in}}%
\pgfpathlineto{\pgfqpoint{2.358024in}{1.341384in}}%
\pgfpathlineto{\pgfqpoint{2.362985in}{1.337251in}}%
\pgfpathlineto{\pgfqpoint{2.367494in}{1.334640in}}%
\pgfpathlineto{\pgfqpoint{2.372455in}{1.335892in}}%
\pgfpathlineto{\pgfqpoint{2.376513in}{1.335968in}}%
\pgfpathlineto{\pgfqpoint{2.378768in}{1.334939in}}%
\pgfpathlineto{\pgfqpoint{2.379219in}{1.336490in}}%
\pgfpathlineto{\pgfqpoint{2.379670in}{1.335094in}}%
\pgfpathlineto{\pgfqpoint{2.385081in}{1.328463in}}%
\pgfpathlineto{\pgfqpoint{2.393650in}{1.319329in}}%
\pgfpathlineto{\pgfqpoint{2.397708in}{1.321975in}}%
\pgfpathlineto{\pgfqpoint{2.404923in}{1.329276in}}%
\pgfpathlineto{\pgfqpoint{2.405374in}{1.328821in}}%
\pgfpathlineto{\pgfqpoint{2.408080in}{1.328636in}}%
\pgfpathlineto{\pgfqpoint{2.410786in}{1.326799in}}%
\pgfpathlineto{\pgfqpoint{2.416648in}{1.316313in}}%
\pgfpathlineto{\pgfqpoint{2.420707in}{1.311073in}}%
\pgfpathlineto{\pgfqpoint{2.423413in}{1.311165in}}%
\pgfpathlineto{\pgfqpoint{2.425667in}{1.313299in}}%
\pgfpathlineto{\pgfqpoint{2.435137in}{1.325246in}}%
\pgfpathlineto{\pgfqpoint{2.438294in}{1.323820in}}%
\pgfpathlineto{\pgfqpoint{2.441000in}{1.319329in}}%
\pgfpathlineto{\pgfqpoint{2.445960in}{1.304734in}}%
\pgfpathlineto{\pgfqpoint{2.449117in}{1.296945in}}%
\pgfpathlineto{\pgfqpoint{2.449568in}{1.297912in}}%
\pgfpathlineto{\pgfqpoint{2.451372in}{1.294542in}}%
\pgfpathlineto{\pgfqpoint{2.454979in}{1.295438in}}%
\pgfpathlineto{\pgfqpoint{2.464900in}{1.298873in}}%
\pgfpathlineto{\pgfqpoint{2.471214in}{1.301120in}}%
\pgfpathlineto{\pgfqpoint{2.476174in}{1.302537in}}%
\pgfpathlineto{\pgfqpoint{2.478880in}{1.300499in}}%
\pgfpathlineto{\pgfqpoint{2.488801in}{1.288966in}}%
\pgfpathlineto{\pgfqpoint{2.490154in}{1.289215in}}%
\pgfpathlineto{\pgfqpoint{2.491507in}{1.290401in}}%
\pgfpathlineto{\pgfqpoint{2.494212in}{1.293977in}}%
\pgfpathlineto{\pgfqpoint{2.500526in}{1.303153in}}%
\pgfpathlineto{\pgfqpoint{2.507741in}{1.303296in}}%
\pgfpathlineto{\pgfqpoint{2.509545in}{1.302459in}}%
\pgfpathlineto{\pgfqpoint{2.521270in}{1.301052in}}%
\pgfpathlineto{\pgfqpoint{2.526230in}{1.297999in}}%
\pgfpathlineto{\pgfqpoint{2.528936in}{1.298556in}}%
\pgfpathlineto{\pgfqpoint{2.531191in}{1.300700in}}%
\pgfpathlineto{\pgfqpoint{2.541112in}{1.312310in}}%
\pgfpathlineto{\pgfqpoint{2.550131in}{1.300924in}}%
\pgfpathlineto{\pgfqpoint{2.550582in}{1.301147in}}%
\pgfpathlineto{\pgfqpoint{2.552836in}{1.297385in}}%
\pgfpathlineto{\pgfqpoint{2.556895in}{1.294825in}}%
\pgfpathlineto{\pgfqpoint{2.557797in}{1.294519in}}%
\pgfpathlineto{\pgfqpoint{2.568169in}{1.283750in}}%
\pgfpathlineto{\pgfqpoint{2.572227in}{1.280079in}}%
\pgfpathlineto{\pgfqpoint{2.575384in}{1.280249in}}%
\pgfpathlineto{\pgfqpoint{2.579894in}{1.282940in}}%
\pgfpathlineto{\pgfqpoint{2.588913in}{1.290100in}}%
\pgfpathlineto{\pgfqpoint{2.593873in}{1.291063in}}%
\pgfpathlineto{\pgfqpoint{2.600187in}{1.289366in}}%
\pgfpathlineto{\pgfqpoint{2.601990in}{1.288142in}}%
\pgfpathlineto{\pgfqpoint{2.602441in}{1.288506in}}%
\pgfpathlineto{\pgfqpoint{2.602892in}{1.287872in}}%
\pgfpathlineto{\pgfqpoint{2.613264in}{1.279104in}}%
\pgfpathlineto{\pgfqpoint{2.620029in}{1.273284in}}%
\pgfpathlineto{\pgfqpoint{2.622734in}{1.273025in}}%
\pgfpathlineto{\pgfqpoint{2.623185in}{1.273860in}}%
\pgfpathlineto{\pgfqpoint{2.623636in}{1.273319in}}%
\pgfpathlineto{\pgfqpoint{2.626342in}{1.273328in}}%
\pgfpathlineto{\pgfqpoint{2.632655in}{1.273451in}}%
\pgfpathlineto{\pgfqpoint{2.634910in}{1.273122in}}%
\pgfpathlineto{\pgfqpoint{2.635361in}{1.274469in}}%
\pgfpathlineto{\pgfqpoint{2.635812in}{1.273263in}}%
\pgfpathlineto{\pgfqpoint{2.637616in}{1.272544in}}%
\pgfpathlineto{\pgfqpoint{2.638067in}{1.273279in}}%
\pgfpathlineto{\pgfqpoint{2.638518in}{1.272737in}}%
\pgfpathlineto{\pgfqpoint{2.651595in}{1.267446in}}%
\pgfpathlineto{\pgfqpoint{2.661065in}{1.268371in}}%
\pgfpathlineto{\pgfqpoint{2.663320in}{1.267131in}}%
\pgfpathlineto{\pgfqpoint{2.663771in}{1.267504in}}%
\pgfpathlineto{\pgfqpoint{2.664222in}{1.266869in}}%
\pgfpathlineto{\pgfqpoint{2.672339in}{1.259698in}}%
\pgfpathlineto{\pgfqpoint{2.678653in}{1.254035in}}%
\pgfpathlineto{\pgfqpoint{2.680907in}{1.255285in}}%
\pgfpathlineto{\pgfqpoint{2.687672in}{1.259486in}}%
\pgfpathlineto{\pgfqpoint{2.689926in}{1.256818in}}%
\pgfpathlineto{\pgfqpoint{2.695789in}{1.244364in}}%
\pgfpathlineto{\pgfqpoint{2.700298in}{1.237298in}}%
\pgfpathlineto{\pgfqpoint{2.701651in}{1.236562in}}%
\pgfpathlineto{\pgfqpoint{2.703906in}{1.236790in}}%
\pgfpathlineto{\pgfqpoint{2.706612in}{1.239954in}}%
\pgfpathlineto{\pgfqpoint{2.712474in}{1.252295in}}%
\pgfpathlineto{\pgfqpoint{2.717886in}{1.261341in}}%
\pgfpathlineto{\pgfqpoint{2.720591in}{1.262267in}}%
\pgfpathlineto{\pgfqpoint{2.723297in}{1.260287in}}%
\pgfpathlineto{\pgfqpoint{2.730061in}{1.254257in}}%
\pgfpathlineto{\pgfqpoint{2.733669in}{1.254222in}}%
\pgfpathlineto{\pgfqpoint{2.735022in}{1.254694in}}%
\pgfpathlineto{\pgfqpoint{2.742237in}{1.257130in}}%
\pgfpathlineto{\pgfqpoint{2.742688in}{1.257808in}}%
\pgfpathlineto{\pgfqpoint{2.743590in}{1.257193in}}%
\pgfpathlineto{\pgfqpoint{2.748550in}{1.256130in}}%
\pgfpathlineto{\pgfqpoint{2.753060in}{1.251611in}}%
\pgfpathlineto{\pgfqpoint{2.760726in}{1.238753in}}%
\pgfpathlineto{\pgfqpoint{2.767040in}{1.229476in}}%
\pgfpathlineto{\pgfqpoint{2.771098in}{1.227012in}}%
\pgfpathlineto{\pgfqpoint{2.774255in}{1.227054in}}%
\pgfpathlineto{\pgfqpoint{2.782372in}{1.228948in}}%
\pgfpathlineto{\pgfqpoint{2.787784in}{1.225992in}}%
\pgfpathlineto{\pgfqpoint{2.788685in}{1.225296in}}%
\pgfpathlineto{\pgfqpoint{2.789136in}{1.226884in}}%
\pgfpathlineto{\pgfqpoint{2.789587in}{1.225166in}}%
\pgfpathlineto{\pgfqpoint{2.792293in}{1.222973in}}%
\pgfpathlineto{\pgfqpoint{2.795901in}{1.222755in}}%
\pgfpathlineto{\pgfqpoint{2.803116in}{1.225519in}}%
\pgfpathlineto{\pgfqpoint{2.804469in}{1.225658in}}%
\pgfpathlineto{\pgfqpoint{2.821154in}{1.224018in}}%
\pgfpathlineto{\pgfqpoint{2.831075in}{1.229784in}}%
\pgfpathlineto{\pgfqpoint{2.835134in}{1.228328in}}%
\pgfpathlineto{\pgfqpoint{2.840545in}{1.224445in}}%
\pgfpathlineto{\pgfqpoint{2.847309in}{1.214075in}}%
\pgfpathlineto{\pgfqpoint{2.851819in}{1.208827in}}%
\pgfpathlineto{\pgfqpoint{2.855427in}{1.209499in}}%
\pgfpathlineto{\pgfqpoint{2.857681in}{1.211669in}}%
\pgfpathlineto{\pgfqpoint{2.858132in}{1.214272in}}%
\pgfpathlineto{\pgfqpoint{2.859034in}{1.213447in}}%
\pgfpathlineto{\pgfqpoint{2.865348in}{1.216852in}}%
\pgfpathlineto{\pgfqpoint{2.870308in}{1.214549in}}%
\pgfpathlineto{\pgfqpoint{2.874367in}{1.212769in}}%
\pgfpathlineto{\pgfqpoint{2.874818in}{1.214147in}}%
\pgfpathlineto{\pgfqpoint{2.875269in}{1.212444in}}%
\pgfpathlineto{\pgfqpoint{2.877072in}{1.212201in}}%
\pgfpathlineto{\pgfqpoint{2.885641in}{1.211500in}}%
\pgfpathlineto{\pgfqpoint{2.887444in}{1.210326in}}%
\pgfpathlineto{\pgfqpoint{2.887895in}{1.210852in}}%
\pgfpathlineto{\pgfqpoint{2.888346in}{1.210088in}}%
\pgfpathlineto{\pgfqpoint{2.894660in}{1.201861in}}%
\pgfpathlineto{\pgfqpoint{2.900071in}{1.194453in}}%
\pgfpathlineto{\pgfqpoint{2.902326in}{1.194087in}}%
\pgfpathlineto{\pgfqpoint{2.908188in}{1.199814in}}%
\pgfpathlineto{\pgfqpoint{2.914953in}{1.208135in}}%
\pgfpathlineto{\pgfqpoint{2.918560in}{1.208873in}}%
\pgfpathlineto{\pgfqpoint{2.926226in}{1.207607in}}%
\pgfpathlineto{\pgfqpoint{2.936598in}{1.209946in}}%
\pgfpathlineto{\pgfqpoint{2.954186in}{1.203919in}}%
\pgfpathlineto{\pgfqpoint{2.956891in}{1.205280in}}%
\pgfpathlineto{\pgfqpoint{2.961401in}{1.205792in}}%
\pgfpathlineto{\pgfqpoint{2.965009in}{1.203010in}}%
\pgfpathlineto{\pgfqpoint{2.970871in}{1.198623in}}%
\pgfpathlineto{\pgfqpoint{2.981243in}{1.202692in}}%
\pgfpathlineto{\pgfqpoint{2.983949in}{1.200964in}}%
\pgfpathlineto{\pgfqpoint{2.987556in}{1.195353in}}%
\pgfpathlineto{\pgfqpoint{2.993419in}{1.181233in}}%
\pgfpathlineto{\pgfqpoint{2.999732in}{1.167139in}}%
\pgfpathlineto{\pgfqpoint{3.001987in}{1.165100in}}%
\pgfpathlineto{\pgfqpoint{3.002438in}{1.165745in}}%
\pgfpathlineto{\pgfqpoint{3.002889in}{1.164991in}}%
\pgfpathlineto{\pgfqpoint{3.004692in}{1.164831in}}%
\pgfpathlineto{\pgfqpoint{3.007849in}{1.167690in}}%
\pgfpathlineto{\pgfqpoint{3.011006in}{1.172455in}}%
\pgfpathlineto{\pgfqpoint{3.011457in}{1.176094in}}%
\pgfpathlineto{\pgfqpoint{3.012359in}{1.174882in}}%
\pgfpathlineto{\pgfqpoint{3.018221in}{1.183473in}}%
\pgfpathlineto{\pgfqpoint{3.022731in}{1.185228in}}%
\pgfpathlineto{\pgfqpoint{3.029495in}{1.187883in}}%
\pgfpathlineto{\pgfqpoint{3.029946in}{1.189307in}}%
\pgfpathlineto{\pgfqpoint{3.030848in}{1.188902in}}%
\pgfpathlineto{\pgfqpoint{3.038965in}{1.194483in}}%
\pgfpathlineto{\pgfqpoint{3.044376in}{1.194812in}}%
\pgfpathlineto{\pgfqpoint{3.049788in}{1.191499in}}%
\pgfpathlineto{\pgfqpoint{3.059258in}{1.184129in}}%
\pgfpathlineto{\pgfqpoint{3.059709in}{1.184680in}}%
\pgfpathlineto{\pgfqpoint{3.061513in}{1.181844in}}%
\pgfpathlineto{\pgfqpoint{3.067375in}{1.175125in}}%
\pgfpathlineto{\pgfqpoint{3.070983in}{1.172323in}}%
\pgfpathlineto{\pgfqpoint{3.073238in}{1.172501in}}%
\pgfpathlineto{\pgfqpoint{3.076845in}{1.175180in}}%
\pgfpathlineto{\pgfqpoint{3.084060in}{1.180464in}}%
\pgfpathlineto{\pgfqpoint{3.087668in}{1.179079in}}%
\pgfpathlineto{\pgfqpoint{3.097589in}{1.173361in}}%
\pgfpathlineto{\pgfqpoint{3.098040in}{1.174304in}}%
\pgfpathlineto{\pgfqpoint{3.098491in}{1.173224in}}%
\pgfpathlineto{\pgfqpoint{3.101648in}{1.171233in}}%
\pgfpathlineto{\pgfqpoint{3.107510in}{1.169866in}}%
\pgfpathlineto{\pgfqpoint{3.110667in}{1.171390in}}%
\pgfpathlineto{\pgfqpoint{3.114725in}{1.176394in}}%
\pgfpathlineto{\pgfqpoint{3.117882in}{1.180396in}}%
\pgfpathlineto{\pgfqpoint{3.121941in}{1.182890in}}%
\pgfpathlineto{\pgfqpoint{3.127803in}{1.183044in}}%
\pgfpathlineto{\pgfqpoint{3.132313in}{1.180932in}}%
\pgfpathlineto{\pgfqpoint{3.139528in}{1.176806in}}%
\pgfpathlineto{\pgfqpoint{3.147194in}{1.174282in}}%
\pgfpathlineto{\pgfqpoint{3.147645in}{1.175881in}}%
\pgfpathlineto{\pgfqpoint{3.148096in}{1.173937in}}%
\pgfpathlineto{\pgfqpoint{3.155311in}{1.167681in}}%
\pgfpathlineto{\pgfqpoint{3.163428in}{1.159941in}}%
\pgfpathlineto{\pgfqpoint{3.165683in}{1.160793in}}%
\pgfpathlineto{\pgfqpoint{3.167487in}{1.162753in}}%
\pgfpathlineto{\pgfqpoint{3.173349in}{1.168506in}}%
\pgfpathlineto{\pgfqpoint{3.173800in}{1.168575in}}%
\pgfpathlineto{\pgfqpoint{3.174251in}{1.169983in}}%
\pgfpathlineto{\pgfqpoint{3.174702in}{1.169110in}}%
\pgfpathlineto{\pgfqpoint{3.181467in}{1.164763in}}%
\pgfpathlineto{\pgfqpoint{3.186878in}{1.157239in}}%
\pgfpathlineto{\pgfqpoint{3.190486in}{1.155018in}}%
\pgfpathlineto{\pgfqpoint{3.193191in}{1.156137in}}%
\pgfpathlineto{\pgfqpoint{3.200407in}{1.161722in}}%
\pgfpathlineto{\pgfqpoint{3.202661in}{1.159827in}}%
\pgfpathlineto{\pgfqpoint{3.214837in}{1.155401in}}%
\pgfpathlineto{\pgfqpoint{3.217092in}{1.155197in}}%
\pgfpathlineto{\pgfqpoint{3.217543in}{1.156612in}}%
\pgfpathlineto{\pgfqpoint{3.217994in}{1.155635in}}%
\pgfpathlineto{\pgfqpoint{3.220249in}{1.155236in}}%
\pgfpathlineto{\pgfqpoint{3.223856in}{1.154310in}}%
\pgfpathlineto{\pgfqpoint{3.231522in}{1.151206in}}%
\pgfpathlineto{\pgfqpoint{3.246855in}{1.149654in}}%
\pgfpathlineto{\pgfqpoint{3.248208in}{1.149674in}}%
\pgfpathlineto{\pgfqpoint{3.257678in}{1.149250in}}%
\pgfpathlineto{\pgfqpoint{3.266697in}{1.145314in}}%
\pgfpathlineto{\pgfqpoint{3.270755in}{1.146741in}}%
\pgfpathlineto{\pgfqpoint{3.272108in}{1.147329in}}%
\pgfpathlineto{\pgfqpoint{3.276618in}{1.147757in}}%
\pgfpathlineto{\pgfqpoint{3.281578in}{1.144167in}}%
\pgfpathlineto{\pgfqpoint{3.286088in}{1.137573in}}%
\pgfpathlineto{\pgfqpoint{3.293303in}{1.126745in}}%
\pgfpathlineto{\pgfqpoint{3.295107in}{1.125517in}}%
\pgfpathlineto{\pgfqpoint{3.295558in}{1.126351in}}%
\pgfpathlineto{\pgfqpoint{3.296009in}{1.125314in}}%
\pgfpathlineto{\pgfqpoint{3.298715in}{1.124857in}}%
\pgfpathlineto{\pgfqpoint{3.304577in}{1.126094in}}%
\pgfpathlineto{\pgfqpoint{3.314047in}{1.129614in}}%
\pgfpathlineto{\pgfqpoint{3.317204in}{1.128683in}}%
\pgfpathlineto{\pgfqpoint{3.326223in}{1.122406in}}%
\pgfpathlineto{\pgfqpoint{3.326674in}{1.123141in}}%
\pgfpathlineto{\pgfqpoint{3.327125in}{1.122144in}}%
\pgfpathlineto{\pgfqpoint{3.329830in}{1.120534in}}%
\pgfpathlineto{\pgfqpoint{3.332987in}{1.121548in}}%
\pgfpathlineto{\pgfqpoint{3.339300in}{1.123803in}}%
\pgfpathlineto{\pgfqpoint{3.347869in}{1.126167in}}%
\pgfpathlineto{\pgfqpoint{3.353731in}{1.128193in}}%
\pgfpathlineto{\pgfqpoint{3.366809in}{1.128404in}}%
\pgfpathlineto{\pgfqpoint{3.369965in}{1.124819in}}%
\pgfpathlineto{\pgfqpoint{3.370416in}{1.124307in}}%
\pgfpathlineto{\pgfqpoint{3.370867in}{1.125977in}}%
\pgfpathlineto{\pgfqpoint{3.371318in}{1.124120in}}%
\pgfpathlineto{\pgfqpoint{3.374024in}{1.120869in}}%
\pgfpathlineto{\pgfqpoint{3.385298in}{1.112302in}}%
\pgfpathlineto{\pgfqpoint{3.398375in}{1.109218in}}%
\pgfpathlineto{\pgfqpoint{3.401532in}{1.110605in}}%
\pgfpathlineto{\pgfqpoint{3.411002in}{1.117492in}}%
\pgfpathlineto{\pgfqpoint{3.417316in}{1.121093in}}%
\pgfpathlineto{\pgfqpoint{3.422276in}{1.122611in}}%
\pgfpathlineto{\pgfqpoint{3.425433in}{1.121834in}}%
\pgfpathlineto{\pgfqpoint{3.429491in}{1.117474in}}%
\pgfpathlineto{\pgfqpoint{3.452941in}{1.088620in}}%
\pgfpathlineto{\pgfqpoint{3.453843in}{1.088472in}}%
\pgfpathlineto{\pgfqpoint{3.454294in}{1.089693in}}%
\pgfpathlineto{\pgfqpoint{3.454745in}{1.088855in}}%
\pgfpathlineto{\pgfqpoint{3.457000in}{1.089224in}}%
\pgfpathlineto{\pgfqpoint{3.457450in}{1.092141in}}%
\pgfpathlineto{\pgfqpoint{3.457901in}{1.090215in}}%
\pgfpathlineto{\pgfqpoint{3.459705in}{1.090906in}}%
\pgfpathlineto{\pgfqpoint{3.463313in}{1.092215in}}%
\pgfpathlineto{\pgfqpoint{3.466019in}{1.091211in}}%
\pgfpathlineto{\pgfqpoint{3.468724in}{1.088769in}}%
\pgfpathlineto{\pgfqpoint{3.472783in}{1.087369in}}%
\pgfpathlineto{\pgfqpoint{3.476391in}{1.088584in}}%
\pgfpathlineto{\pgfqpoint{3.480900in}{1.092764in}}%
\pgfpathlineto{\pgfqpoint{3.489919in}{1.104428in}}%
\pgfpathlineto{\pgfqpoint{3.492625in}{1.105468in}}%
\pgfpathlineto{\pgfqpoint{3.495782in}{1.104329in}}%
\pgfpathlineto{\pgfqpoint{3.501644in}{1.098928in}}%
\pgfpathlineto{\pgfqpoint{3.529152in}{1.070834in}}%
\pgfpathlineto{\pgfqpoint{3.532309in}{1.069290in}}%
\pgfpathlineto{\pgfqpoint{3.541328in}{1.065235in}}%
\pgfpathlineto{\pgfqpoint{3.548994in}{1.068932in}}%
\pgfpathlineto{\pgfqpoint{3.551249in}{1.071034in}}%
\pgfpathlineto{\pgfqpoint{3.553955in}{1.074437in}}%
\pgfpathlineto{\pgfqpoint{3.569738in}{1.093846in}}%
\pgfpathlineto{\pgfqpoint{3.573797in}{1.095897in}}%
\pgfpathlineto{\pgfqpoint{3.577404in}{1.094857in}}%
\pgfpathlineto{\pgfqpoint{3.596795in}{1.084787in}}%
\pgfpathlineto{\pgfqpoint{3.605363in}{1.084715in}}%
\pgfpathlineto{\pgfqpoint{3.610324in}{1.083966in}}%
\pgfpathlineto{\pgfqpoint{3.613481in}{1.081121in}}%
\pgfpathlineto{\pgfqpoint{3.625656in}{1.066202in}}%
\pgfpathlineto{\pgfqpoint{3.629264in}{1.066814in}}%
\pgfpathlineto{\pgfqpoint{3.634675in}{1.070163in}}%
\pgfpathlineto{\pgfqpoint{3.639636in}{1.072583in}}%
\pgfpathlineto{\pgfqpoint{3.644146in}{1.071972in}}%
\pgfpathlineto{\pgfqpoint{3.650459in}{1.071090in}}%
\pgfpathlineto{\pgfqpoint{3.651812in}{1.071071in}}%
\pgfpathlineto{\pgfqpoint{3.663987in}{1.068692in}}%
\pgfpathlineto{\pgfqpoint{3.669399in}{1.062589in}}%
\pgfpathlineto{\pgfqpoint{3.676163in}{1.055591in}}%
\pgfpathlineto{\pgfqpoint{3.678418in}{1.054877in}}%
\pgfpathlineto{\pgfqpoint{3.678869in}{1.055648in}}%
\pgfpathlineto{\pgfqpoint{3.679320in}{1.055089in}}%
\pgfpathlineto{\pgfqpoint{3.682026in}{1.055127in}}%
\pgfpathlineto{\pgfqpoint{3.686084in}{1.054388in}}%
\pgfpathlineto{\pgfqpoint{3.688790in}{1.051623in}}%
\pgfpathlineto{\pgfqpoint{3.693750in}{1.046271in}}%
\pgfpathlineto{\pgfqpoint{3.694201in}{1.047410in}}%
\pgfpathlineto{\pgfqpoint{3.694652in}{1.046404in}}%
\pgfpathlineto{\pgfqpoint{3.696907in}{1.045648in}}%
\pgfpathlineto{\pgfqpoint{3.701417in}{1.044434in}}%
\pgfpathlineto{\pgfqpoint{3.705475in}{1.040490in}}%
\pgfpathlineto{\pgfqpoint{3.710887in}{1.031393in}}%
\pgfpathlineto{\pgfqpoint{3.716298in}{1.023052in}}%
\pgfpathlineto{\pgfqpoint{3.719004in}{1.022352in}}%
\pgfpathlineto{\pgfqpoint{3.721259in}{1.024184in}}%
\pgfpathlineto{\pgfqpoint{3.725768in}{1.033280in}}%
\pgfpathlineto{\pgfqpoint{3.736591in}{1.057788in}}%
\pgfpathlineto{\pgfqpoint{3.739297in}{1.059621in}}%
\pgfpathlineto{\pgfqpoint{3.742003in}{1.058648in}}%
\pgfpathlineto{\pgfqpoint{3.746512in}{1.053559in}}%
\pgfpathlineto{\pgfqpoint{3.752375in}{1.048002in}}%
\pgfpathlineto{\pgfqpoint{3.755080in}{1.046866in}}%
\pgfpathlineto{\pgfqpoint{3.757786in}{1.047462in}}%
\pgfpathlineto{\pgfqpoint{3.759139in}{1.047659in}}%
\pgfpathlineto{\pgfqpoint{3.765903in}{1.047968in}}%
\pgfpathlineto{\pgfqpoint{3.783941in}{1.044617in}}%
\pgfpathlineto{\pgfqpoint{3.788451in}{1.044909in}}%
\pgfpathlineto{\pgfqpoint{3.791157in}{1.042790in}}%
\pgfpathlineto{\pgfqpoint{3.794764in}{1.036516in}}%
\pgfpathlineto{\pgfqpoint{3.804685in}{1.016945in}}%
\pgfpathlineto{\pgfqpoint{3.807391in}{1.016508in}}%
\pgfpathlineto{\pgfqpoint{3.807842in}{1.018523in}}%
\pgfpathlineto{\pgfqpoint{3.808744in}{1.017695in}}%
\pgfpathlineto{\pgfqpoint{3.810999in}{1.020435in}}%
\pgfpathlineto{\pgfqpoint{3.815508in}{1.028085in}}%
\pgfpathlineto{\pgfqpoint{3.824076in}{1.040916in}}%
\pgfpathlineto{\pgfqpoint{3.829037in}{1.045391in}}%
\pgfpathlineto{\pgfqpoint{3.838958in}{1.045214in}}%
\pgfpathlineto{\pgfqpoint{3.843918in}{1.043225in}}%
\pgfpathlineto{\pgfqpoint{3.848879in}{1.038303in}}%
\pgfpathlineto{\pgfqpoint{3.864662in}{1.020539in}}%
\pgfpathlineto{\pgfqpoint{3.869623in}{1.018362in}}%
\pgfpathlineto{\pgfqpoint{3.880445in}{1.014804in}}%
\pgfpathlineto{\pgfqpoint{3.885857in}{1.009787in}}%
\pgfpathlineto{\pgfqpoint{3.894425in}{1.001601in}}%
\pgfpathlineto{\pgfqpoint{3.898484in}{1.001028in}}%
\pgfpathlineto{\pgfqpoint{3.898935in}{1.002110in}}%
\pgfpathlineto{\pgfqpoint{3.899386in}{1.001136in}}%
\pgfpathlineto{\pgfqpoint{3.903444in}{1.003120in}}%
\pgfpathlineto{\pgfqpoint{3.909307in}{1.008094in}}%
\pgfpathlineto{\pgfqpoint{3.915169in}{1.017353in}}%
\pgfpathlineto{\pgfqpoint{3.922835in}{1.027331in}}%
\pgfpathlineto{\pgfqpoint{3.928247in}{1.030634in}}%
\pgfpathlineto{\pgfqpoint{3.934109in}{1.031184in}}%
\pgfpathlineto{\pgfqpoint{3.939070in}{1.031484in}}%
\pgfpathlineto{\pgfqpoint{3.939520in}{1.032695in}}%
\pgfpathlineto{\pgfqpoint{3.939971in}{1.031991in}}%
\pgfpathlineto{\pgfqpoint{3.942677in}{1.031727in}}%
\pgfpathlineto{\pgfqpoint{3.946736in}{1.030642in}}%
\pgfpathlineto{\pgfqpoint{3.950794in}{1.026950in}}%
\pgfpathlineto{\pgfqpoint{3.963421in}{1.012471in}}%
\pgfpathlineto{\pgfqpoint{3.970185in}{1.010701in}}%
\pgfpathlineto{\pgfqpoint{3.975597in}{1.007692in}}%
\pgfpathlineto{\pgfqpoint{3.982361in}{1.003983in}}%
\pgfpathlineto{\pgfqpoint{3.986871in}{1.003973in}}%
\pgfpathlineto{\pgfqpoint{3.990027in}{1.005635in}}%
\pgfpathlineto{\pgfqpoint{3.991831in}{1.007685in}}%
\pgfpathlineto{\pgfqpoint{3.993635in}{1.009165in}}%
\pgfpathlineto{\pgfqpoint{4.009418in}{1.026270in}}%
\pgfpathlineto{\pgfqpoint{4.012124in}{1.025391in}}%
\pgfpathlineto{\pgfqpoint{4.016183in}{1.020579in}}%
\pgfpathlineto{\pgfqpoint{4.030613in}{1.001517in}}%
\pgfpathlineto{\pgfqpoint{4.033770in}{1.001074in}}%
\pgfpathlineto{\pgfqpoint{4.038730in}{1.003125in}}%
\pgfpathlineto{\pgfqpoint{4.040985in}{1.004605in}}%
\pgfpathlineto{\pgfqpoint{4.048200in}{1.008670in}}%
\pgfpathlineto{\pgfqpoint{4.050906in}{1.007911in}}%
\pgfpathlineto{\pgfqpoint{4.054514in}{1.003913in}}%
\pgfpathlineto{\pgfqpoint{4.057220in}{0.999991in}}%
\pgfpathlineto{\pgfqpoint{4.057670in}{1.003425in}}%
\pgfpathlineto{\pgfqpoint{4.058121in}{1.000315in}}%
\pgfpathlineto{\pgfqpoint{4.060376in}{0.996837in}}%
\pgfpathlineto{\pgfqpoint{4.069395in}{0.991916in}}%
\pgfpathlineto{\pgfqpoint{4.073003in}{0.992018in}}%
\pgfpathlineto{\pgfqpoint{4.074356in}{0.992355in}}%
\pgfpathlineto{\pgfqpoint{4.082022in}{0.992162in}}%
\pgfpathlineto{\pgfqpoint{4.092394in}{0.985983in}}%
\pgfpathlineto{\pgfqpoint{4.093296in}{0.985691in}}%
\pgfpathlineto{\pgfqpoint{4.093747in}{0.987188in}}%
\pgfpathlineto{\pgfqpoint{4.094198in}{0.986139in}}%
\pgfpathlineto{\pgfqpoint{4.096002in}{0.985587in}}%
\pgfpathlineto{\pgfqpoint{4.105472in}{0.987990in}}%
\pgfpathlineto{\pgfqpoint{4.111334in}{0.988244in}}%
\pgfpathlineto{\pgfqpoint{4.117196in}{0.988866in}}%
\pgfpathlineto{\pgfqpoint{4.128019in}{0.991499in}}%
\pgfpathlineto{\pgfqpoint{4.131627in}{0.989880in}}%
\pgfpathlineto{\pgfqpoint{4.136137in}{0.985490in}}%
\pgfpathlineto{\pgfqpoint{4.148763in}{0.969003in}}%
\pgfpathlineto{\pgfqpoint{4.159135in}{0.962244in}}%
\pgfpathlineto{\pgfqpoint{4.162743in}{0.962998in}}%
\pgfpathlineto{\pgfqpoint{4.173566in}{0.979631in}}%
\pgfpathlineto{\pgfqpoint{4.179428in}{0.988731in}}%
\pgfpathlineto{\pgfqpoint{4.182585in}{0.990309in}}%
\pgfpathlineto{\pgfqpoint{4.185741in}{0.989326in}}%
\pgfpathlineto{\pgfqpoint{4.191153in}{0.983778in}}%
\pgfpathlineto{\pgfqpoint{4.194310in}{0.978590in}}%
\pgfpathlineto{\pgfqpoint{4.199270in}{0.973624in}}%
\pgfpathlineto{\pgfqpoint{4.203780in}{0.972211in}}%
\pgfpathlineto{\pgfqpoint{4.218661in}{0.970294in}}%
\pgfpathlineto{\pgfqpoint{4.240758in}{0.957773in}}%
\pgfpathlineto{\pgfqpoint{4.246620in}{0.954554in}}%
\pgfpathlineto{\pgfqpoint{4.249326in}{0.955379in}}%
\pgfpathlineto{\pgfqpoint{4.252483in}{0.958796in}}%
\pgfpathlineto{\pgfqpoint{4.261051in}{0.973264in}}%
\pgfpathlineto{\pgfqpoint{4.267364in}{0.981907in}}%
\pgfpathlineto{\pgfqpoint{4.271423in}{0.983820in}}%
\pgfpathlineto{\pgfqpoint{4.275481in}{0.983279in}}%
\pgfpathlineto{\pgfqpoint{4.281344in}{0.979937in}}%
\pgfpathlineto{\pgfqpoint{4.297578in}{0.968828in}}%
\pgfpathlineto{\pgfqpoint{4.306146in}{0.963821in}}%
\pgfpathlineto{\pgfqpoint{4.314714in}{0.959710in}}%
\pgfpathlineto{\pgfqpoint{4.321028in}{0.959157in}}%
\pgfpathlineto{\pgfqpoint{4.329596in}{0.958592in}}%
\pgfpathlineto{\pgfqpoint{4.342223in}{0.955918in}}%
\pgfpathlineto{\pgfqpoint{4.346281in}{0.958222in}}%
\pgfpathlineto{\pgfqpoint{4.351242in}{0.964176in}}%
\pgfpathlineto{\pgfqpoint{4.362966in}{0.979333in}}%
\pgfpathlineto{\pgfqpoint{4.367476in}{0.981402in}}%
\pgfpathlineto{\pgfqpoint{4.371535in}{0.980733in}}%
\pgfpathlineto{\pgfqpoint{4.375593in}{0.977559in}}%
\pgfpathlineto{\pgfqpoint{4.381456in}{0.969430in}}%
\pgfpathlineto{\pgfqpoint{4.391828in}{0.955033in}}%
\pgfpathlineto{\pgfqpoint{4.396337in}{0.952166in}}%
\pgfpathlineto{\pgfqpoint{4.398141in}{0.952058in}}%
\pgfpathlineto{\pgfqpoint{4.401749in}{0.953499in}}%
\pgfpathlineto{\pgfqpoint{4.407611in}{0.958833in}}%
\pgfpathlineto{\pgfqpoint{4.415277in}{0.965021in}}%
\pgfpathlineto{\pgfqpoint{4.420238in}{0.966482in}}%
\pgfpathlineto{\pgfqpoint{4.424747in}{0.965563in}}%
\pgfpathlineto{\pgfqpoint{4.433315in}{0.960528in}}%
\pgfpathlineto{\pgfqpoint{4.441433in}{0.956349in}}%
\pgfpathlineto{\pgfqpoint{4.453157in}{0.954378in}}%
\pgfpathlineto{\pgfqpoint{4.469843in}{0.953952in}}%
\pgfpathlineto{\pgfqpoint{4.473450in}{0.950442in}}%
\pgfpathlineto{\pgfqpoint{4.484724in}{0.933670in}}%
\pgfpathlineto{\pgfqpoint{4.488332in}{0.929849in}}%
\pgfpathlineto{\pgfqpoint{4.491488in}{0.929661in}}%
\pgfpathlineto{\pgfqpoint{4.495547in}{0.931963in}}%
\pgfpathlineto{\pgfqpoint{4.500958in}{0.937769in}}%
\pgfpathlineto{\pgfqpoint{4.514036in}{0.953121in}}%
\pgfpathlineto{\pgfqpoint{4.520349in}{0.956934in}}%
\pgfpathlineto{\pgfqpoint{4.524859in}{0.957182in}}%
\pgfpathlineto{\pgfqpoint{4.529369in}{0.955387in}}%
\pgfpathlineto{\pgfqpoint{4.534780in}{0.950969in}}%
\pgfpathlineto{\pgfqpoint{4.552367in}{0.933642in}}%
\pgfpathlineto{\pgfqpoint{4.555975in}{0.933586in}}%
\pgfpathlineto{\pgfqpoint{4.569503in}{0.936080in}}%
\pgfpathlineto{\pgfqpoint{4.577170in}{0.938227in}}%
\pgfpathlineto{\pgfqpoint{4.585738in}{0.938930in}}%
\pgfpathlineto{\pgfqpoint{4.592502in}{0.938435in}}%
\pgfpathlineto{\pgfqpoint{4.592953in}{0.939805in}}%
\pgfpathlineto{\pgfqpoint{4.593404in}{0.938612in}}%
\pgfpathlineto{\pgfqpoint{4.607384in}{0.938989in}}%
\pgfpathlineto{\pgfqpoint{4.620912in}{0.934086in}}%
\pgfpathlineto{\pgfqpoint{4.630833in}{0.925187in}}%
\pgfpathlineto{\pgfqpoint{4.636245in}{0.921660in}}%
\pgfpathlineto{\pgfqpoint{4.645264in}{0.918271in}}%
\pgfpathlineto{\pgfqpoint{4.649773in}{0.919880in}}%
\pgfpathlineto{\pgfqpoint{4.667812in}{0.929322in}}%
\pgfpathlineto{\pgfqpoint{4.675027in}{0.928931in}}%
\pgfpathlineto{\pgfqpoint{4.675478in}{0.930716in}}%
\pgfpathlineto{\pgfqpoint{4.675929in}{0.929793in}}%
\pgfpathlineto{\pgfqpoint{4.677282in}{0.929818in}}%
\pgfpathlineto{\pgfqpoint{4.683144in}{0.934694in}}%
\pgfpathlineto{\pgfqpoint{4.683595in}{0.939726in}}%
\pgfpathlineto{\pgfqpoint{4.684046in}{0.936793in}}%
\pgfpathlineto{\pgfqpoint{4.685399in}{0.937543in}}%
\pgfpathlineto{\pgfqpoint{4.693065in}{0.946272in}}%
\pgfpathlineto{\pgfqpoint{4.697124in}{0.946825in}}%
\pgfpathlineto{\pgfqpoint{4.702084in}{0.945263in}}%
\pgfpathlineto{\pgfqpoint{4.711103in}{0.939476in}}%
\pgfpathlineto{\pgfqpoint{4.721926in}{0.931806in}}%
\pgfpathlineto{\pgfqpoint{4.728690in}{0.930482in}}%
\pgfpathlineto{\pgfqpoint{4.738160in}{0.932996in}}%
\pgfpathlineto{\pgfqpoint{4.743572in}{0.936454in}}%
\pgfpathlineto{\pgfqpoint{4.750336in}{0.940284in}}%
\pgfpathlineto{\pgfqpoint{4.755297in}{0.940593in}}%
\pgfpathlineto{\pgfqpoint{4.759806in}{0.938706in}}%
\pgfpathlineto{\pgfqpoint{4.773335in}{0.931106in}}%
\pgfpathlineto{\pgfqpoint{4.781452in}{0.928914in}}%
\pgfpathlineto{\pgfqpoint{4.792726in}{0.928969in}}%
\pgfpathlineto{\pgfqpoint{4.802196in}{0.922005in}}%
\pgfpathlineto{\pgfqpoint{4.809862in}{0.918308in}}%
\pgfpathlineto{\pgfqpoint{4.819332in}{0.913915in}}%
\pgfpathlineto{\pgfqpoint{4.824293in}{0.912384in}}%
\pgfpathlineto{\pgfqpoint{4.824744in}{0.914050in}}%
\pgfpathlineto{\pgfqpoint{4.825195in}{0.913022in}}%
\pgfpathlineto{\pgfqpoint{4.826998in}{0.912698in}}%
\pgfpathlineto{\pgfqpoint{4.849997in}{0.917259in}}%
\pgfpathlineto{\pgfqpoint{4.854056in}{0.913896in}}%
\pgfpathlineto{\pgfqpoint{4.859467in}{0.905960in}}%
\pgfpathlineto{\pgfqpoint{4.866231in}{0.896784in}}%
\pgfpathlineto{\pgfqpoint{4.869839in}{0.895323in}}%
\pgfpathlineto{\pgfqpoint{4.872996in}{0.895846in}}%
\pgfpathlineto{\pgfqpoint{4.874799in}{0.896893in}}%
\pgfpathlineto{\pgfqpoint{4.880211in}{0.901953in}}%
\pgfpathlineto{\pgfqpoint{4.889681in}{0.910582in}}%
\pgfpathlineto{\pgfqpoint{4.896445in}{0.913443in}}%
\pgfpathlineto{\pgfqpoint{4.910876in}{0.917622in}}%
\pgfpathlineto{\pgfqpoint{4.915385in}{0.915368in}}%
\pgfpathlineto{\pgfqpoint{4.928012in}{0.907500in}}%
\pgfpathlineto{\pgfqpoint{4.928914in}{0.907229in}}%
\pgfpathlineto{\pgfqpoint{4.941090in}{0.899723in}}%
\pgfpathlineto{\pgfqpoint{4.945148in}{0.899275in}}%
\pgfpathlineto{\pgfqpoint{4.950560in}{0.901519in}}%
\pgfpathlineto{\pgfqpoint{4.956873in}{0.904049in}}%
\pgfpathlineto{\pgfqpoint{4.967696in}{0.903842in}}%
\pgfpathlineto{\pgfqpoint{4.971755in}{0.901959in}}%
\pgfpathlineto{\pgfqpoint{4.976715in}{0.897098in}}%
\pgfpathlineto{\pgfqpoint{4.990244in}{0.881419in}}%
\pgfpathlineto{\pgfqpoint{4.992499in}{0.881651in}}%
\pgfpathlineto{\pgfqpoint{4.996557in}{0.884668in}}%
\pgfpathlineto{\pgfqpoint{5.010988in}{0.897842in}}%
\pgfpathlineto{\pgfqpoint{5.016399in}{0.898873in}}%
\pgfpathlineto{\pgfqpoint{5.027222in}{0.899244in}}%
\pgfpathlineto{\pgfqpoint{5.033535in}{0.902548in}}%
\pgfpathlineto{\pgfqpoint{5.043005in}{0.907783in}}%
\pgfpathlineto{\pgfqpoint{5.047064in}{0.907582in}}%
\pgfpathlineto{\pgfqpoint{5.051123in}{0.904942in}}%
\pgfpathlineto{\pgfqpoint{5.056985in}{0.897947in}}%
\pgfpathlineto{\pgfqpoint{5.063298in}{0.888994in}}%
\pgfpathlineto{\pgfqpoint{5.063749in}{0.892540in}}%
\pgfpathlineto{\pgfqpoint{5.064200in}{0.889479in}}%
\pgfpathlineto{\pgfqpoint{5.066004in}{0.886555in}}%
\pgfpathlineto{\pgfqpoint{5.069612in}{0.885798in}}%
\pgfpathlineto{\pgfqpoint{5.075925in}{0.887112in}}%
\pgfpathlineto{\pgfqpoint{5.084944in}{0.889416in}}%
\pgfpathlineto{\pgfqpoint{5.087650in}{0.890093in}}%
\pgfpathlineto{\pgfqpoint{5.089003in}{0.890896in}}%
\pgfpathlineto{\pgfqpoint{5.093512in}{0.894069in}}%
\pgfpathlineto{\pgfqpoint{5.107492in}{0.905855in}}%
\pgfpathlineto{\pgfqpoint{5.119217in}{0.909466in}}%
\pgfpathlineto{\pgfqpoint{5.126883in}{0.911900in}}%
\pgfpathlineto{\pgfqpoint{5.128236in}{0.911788in}}%
\pgfpathlineto{\pgfqpoint{5.128687in}{0.913071in}}%
\pgfpathlineto{\pgfqpoint{5.129138in}{0.911621in}}%
\pgfpathlineto{\pgfqpoint{5.142666in}{0.908091in}}%
\pgfpathlineto{\pgfqpoint{5.149882in}{0.901368in}}%
\pgfpathlineto{\pgfqpoint{5.156195in}{0.896677in}}%
\pgfpathlineto{\pgfqpoint{5.164312in}{0.891288in}}%
\pgfpathlineto{\pgfqpoint{5.179194in}{0.878504in}}%
\pgfpathlineto{\pgfqpoint{5.197232in}{0.876868in}}%
\pgfpathlineto{\pgfqpoint{5.207604in}{0.870305in}}%
\pgfpathlineto{\pgfqpoint{5.219328in}{0.861456in}}%
\pgfpathlineto{\pgfqpoint{5.223387in}{0.861612in}}%
\pgfpathlineto{\pgfqpoint{5.227897in}{0.864417in}}%
\pgfpathlineto{\pgfqpoint{5.249993in}{0.881968in}}%
\pgfpathlineto{\pgfqpoint{5.253601in}{0.881746in}}%
\pgfpathlineto{\pgfqpoint{5.256758in}{0.879229in}}%
\pgfpathlineto{\pgfqpoint{5.261267in}{0.872168in}}%
\pgfpathlineto{\pgfqpoint{5.271639in}{0.854517in}}%
\pgfpathlineto{\pgfqpoint{5.274796in}{0.853203in}}%
\pgfpathlineto{\pgfqpoint{5.277953in}{0.854144in}}%
\pgfpathlineto{\pgfqpoint{5.282913in}{0.858375in}}%
\pgfpathlineto{\pgfqpoint{5.299598in}{0.873088in}}%
\pgfpathlineto{\pgfqpoint{5.305010in}{0.873370in}}%
\pgfpathlineto{\pgfqpoint{5.313578in}{0.873390in}}%
\pgfpathlineto{\pgfqpoint{5.332067in}{0.879557in}}%
\pgfpathlineto{\pgfqpoint{5.345596in}{0.886420in}}%
\pgfpathlineto{\pgfqpoint{5.354164in}{0.887132in}}%
\pgfpathlineto{\pgfqpoint{5.362281in}{0.884827in}}%
\pgfpathlineto{\pgfqpoint{5.367241in}{0.881434in}}%
\pgfpathlineto{\pgfqpoint{5.372653in}{0.874529in}}%
\pgfpathlineto{\pgfqpoint{5.381672in}{0.861677in}}%
\pgfpathlineto{\pgfqpoint{5.384829in}{0.861761in}}%
\pgfpathlineto{\pgfqpoint{5.389789in}{0.864582in}}%
\pgfpathlineto{\pgfqpoint{5.402867in}{0.872825in}}%
\pgfpathlineto{\pgfqpoint{5.409180in}{0.873042in}}%
\pgfpathlineto{\pgfqpoint{5.414592in}{0.873839in}}%
\pgfpathlineto{\pgfqpoint{5.419101in}{0.877136in}}%
\pgfpathlineto{\pgfqpoint{5.429924in}{0.885924in}}%
\pgfpathlineto{\pgfqpoint{5.440296in}{0.891080in}}%
\pgfpathlineto{\pgfqpoint{5.444355in}{0.889989in}}%
\pgfpathlineto{\pgfqpoint{5.455628in}{0.885168in}}%
\pgfpathlineto{\pgfqpoint{5.460589in}{0.885992in}}%
\pgfpathlineto{\pgfqpoint{5.466451in}{0.889556in}}%
\pgfpathlineto{\pgfqpoint{5.475020in}{0.894550in}}%
\pgfpathlineto{\pgfqpoint{5.488097in}{0.898513in}}%
\pgfpathlineto{\pgfqpoint{5.493058in}{0.897215in}}%
\pgfpathlineto{\pgfqpoint{5.500273in}{0.892597in}}%
\pgfpathlineto{\pgfqpoint{5.511998in}{0.884295in}}%
\pgfpathlineto{\pgfqpoint{5.519664in}{0.881920in}}%
\pgfpathlineto{\pgfqpoint{5.526879in}{0.878793in}}%
\pgfpathlineto{\pgfqpoint{5.534545in}{0.873015in}}%
\pgfpathlineto{\pgfqpoint{5.534545in}{0.873015in}}%
\pgfusepath{stroke}%
\end{pgfscope}%
\begin{pgfscope}%
\pgfsetrectcap%
\pgfsetmiterjoin%
\pgfsetlinewidth{0.803000pt}%
\definecolor{currentstroke}{rgb}{0.000000,0.000000,0.000000}%
\pgfsetstrokecolor{currentstroke}%
\pgfsetdash{}{0pt}%
\pgfpathmoveto{\pgfqpoint{0.800000in}{0.528000in}}%
\pgfpathlineto{\pgfqpoint{0.800000in}{4.224000in}}%
\pgfusepath{stroke}%
\end{pgfscope}%
\begin{pgfscope}%
\pgfsetrectcap%
\pgfsetmiterjoin%
\pgfsetlinewidth{0.803000pt}%
\definecolor{currentstroke}{rgb}{0.000000,0.000000,0.000000}%
\pgfsetstrokecolor{currentstroke}%
\pgfsetdash{}{0pt}%
\pgfpathmoveto{\pgfqpoint{5.760000in}{0.528000in}}%
\pgfpathlineto{\pgfqpoint{5.760000in}{4.224000in}}%
\pgfusepath{stroke}%
\end{pgfscope}%
\begin{pgfscope}%
\pgfsetrectcap%
\pgfsetmiterjoin%
\pgfsetlinewidth{0.803000pt}%
\definecolor{currentstroke}{rgb}{0.000000,0.000000,0.000000}%
\pgfsetstrokecolor{currentstroke}%
\pgfsetdash{}{0pt}%
\pgfpathmoveto{\pgfqpoint{0.800000in}{0.528000in}}%
\pgfpathlineto{\pgfqpoint{5.760000in}{0.528000in}}%
\pgfusepath{stroke}%
\end{pgfscope}%
\begin{pgfscope}%
\pgfsetrectcap%
\pgfsetmiterjoin%
\pgfsetlinewidth{0.803000pt}%
\definecolor{currentstroke}{rgb}{0.000000,0.000000,0.000000}%
\pgfsetstrokecolor{currentstroke}%
\pgfsetdash{}{0pt}%
\pgfpathmoveto{\pgfqpoint{0.800000in}{4.224000in}}%
\pgfpathlineto{\pgfqpoint{5.760000in}{4.224000in}}%
\pgfusepath{stroke}%
\end{pgfscope}%
\begin{pgfscope}%
\pgfsetbuttcap%
\pgfsetmiterjoin%
\definecolor{currentfill}{rgb}{1.000000,1.000000,1.000000}%
\pgfsetfillcolor{currentfill}%
\pgfsetfillopacity{0.800000}%
\pgfsetlinewidth{1.003750pt}%
\definecolor{currentstroke}{rgb}{0.800000,0.800000,0.800000}%
\pgfsetstrokecolor{currentstroke}%
\pgfsetstrokeopacity{0.800000}%
\pgfsetdash{}{0pt}%
\pgfpathmoveto{\pgfqpoint{4.497500in}{3.532056in}}%
\pgfpathlineto{\pgfqpoint{5.662778in}{3.532056in}}%
\pgfpathquadraticcurveto{\pgfqpoint{5.690556in}{3.532056in}}{\pgfqpoint{5.690556in}{3.559834in}}%
\pgfpathlineto{\pgfqpoint{5.690556in}{4.126778in}}%
\pgfpathquadraticcurveto{\pgfqpoint{5.690556in}{4.154556in}}{\pgfqpoint{5.662778in}{4.154556in}}%
\pgfpathlineto{\pgfqpoint{4.497500in}{4.154556in}}%
\pgfpathquadraticcurveto{\pgfqpoint{4.469722in}{4.154556in}}{\pgfqpoint{4.469722in}{4.126778in}}%
\pgfpathlineto{\pgfqpoint{4.469722in}{3.559834in}}%
\pgfpathquadraticcurveto{\pgfqpoint{4.469722in}{3.532056in}}{\pgfqpoint{4.497500in}{3.532056in}}%
\pgfpathclose%
\pgfusepath{stroke,fill}%
\end{pgfscope}%
\begin{pgfscope}%
\pgfsetrectcap%
\pgfsetroundjoin%
\pgfsetlinewidth{1.505625pt}%
\definecolor{currentstroke}{rgb}{0.121569,0.466667,0.705882}%
\pgfsetstrokecolor{currentstroke}%
\pgfsetdash{}{0pt}%
\pgfpathmoveto{\pgfqpoint{4.525278in}{4.050389in}}%
\pgfpathlineto{\pgfqpoint{4.803055in}{4.050389in}}%
\pgfusepath{stroke}%
\end{pgfscope}%
\begin{pgfscope}%
\pgftext[x=4.914167in,y=4.001778in,left,base]{\rmfamily\fontsize{10.000000}{12.000000}\selectfont Error of mu}%
\end{pgfscope}%
\begin{pgfscope}%
\pgfsetrectcap%
\pgfsetroundjoin%
\pgfsetlinewidth{1.505625pt}%
\definecolor{currentstroke}{rgb}{1.000000,0.498039,0.054902}%
\pgfsetstrokecolor{currentstroke}%
\pgfsetdash{}{0pt}%
\pgfpathmoveto{\pgfqpoint{4.525278in}{3.856778in}}%
\pgfpathlineto{\pgfqpoint{4.803055in}{3.856778in}}%
\pgfusepath{stroke}%
\end{pgfscope}%
\begin{pgfscope}%
\pgftext[x=4.914167in,y=3.808167in,left,base]{\rmfamily\fontsize{10.000000}{12.000000}\selectfont Error of nu}%
\end{pgfscope}%
\begin{pgfscope}%
\pgfsetrectcap%
\pgfsetroundjoin%
\pgfsetlinewidth{1.505625pt}%
\definecolor{currentstroke}{rgb}{0.172549,0.627451,0.172549}%
\pgfsetstrokecolor{currentstroke}%
\pgfsetdash{}{0pt}%
\pgfpathmoveto{\pgfqpoint{4.525278in}{3.663167in}}%
\pgfpathlineto{\pgfqpoint{4.803055in}{3.663167in}}%
\pgfusepath{stroke}%
\end{pgfscope}%
\begin{pgfscope}%
\pgftext[x=4.914167in,y=3.614556in,left,base]{\rmfamily\fontsize{10.000000}{12.000000}\selectfont Error of s}%
\end{pgfscope}%
\end{pgfpicture}%
\makeatother%
\endgroup%
} 
\caption{Error curve with respect to iterations} \label{Fig:Loss}
\end{figure}

Inspired by this phenomenon, we propose a new Algorithm \textbf{(Algorithm \hypertarget{EAlg:12}{1+2})} by combing these two algorithms: we first perform Algorithm \ref{Alg:ADMMPrimal} and stop it when the constraints are not satisfied very well, (generally when the error of $\mu$ and $\nu$ reaches $10^{-3}$), and then perform Algorithm \ref{Alg:GradPrimal} to decrease the the error of $\mu$ and $\nu$. Experiments tell Algorithm \hyperlink{EAlg:12}{1+2} have better efficiency and precision than the two original algorithms.

\section{Transportation simplex method} \label{Sec:TS}

According to \textbf{Question 3}, we implement the transportation simplex method mentioned in \parencite{Schrieber2017}. This method is similar to the ordiniary simplex method for linear programs. However, it utilizes the advantages of optimal transport problems, that is, the graph of transportation is bipartite.

\subsection{Description}

In the transportation simplex method, each solution to the problem corresponds to a weighted bipartite graph $G$ as
\begin{equation}
	G=(V, E) \qquad V=V_1\cup V_2
\end{equation}
where vertices set $V_1$ contains all the sources, and vertices set $V_2$ contains all the targets. For one feasible solution $s_{ij}$ satisfing all the constraints, we can construct the graph as follows: 
\begin{equation}
\begin{gathered}
		E = \cbr{(i, j)\mvert i\in V_1, j\in V_2, s_{ij}>0}\\
		w_{ij}=s_{ij}
\end{gathered}
\end{equation}
where $w_{ij}$ represents the weight of edge $(i, j)$.

Suppose $s_{ij}$ is an optimal solution, we may discover some properties of $G$:
\begin{partlist}
	\item $G$ does not have a loop;
	\item $G$ is bipartite, which means that for all $ e\in E$, $e$ must be some $ (i, j) $ with $ i\in V_1, j\in V_2$;
	\item For all $i\in V_1$, $\sum_{j\in N(i)} w_{ij} = \mu_i$;
	\item For all $j\in V_2$, $\sum_{i\in N(j)} w_{ij} = \nu_j$.
\end{partlist}
The last three property will be satisfied as long as $s$ is feasible.

In total, the transportation simplex method can be divided into two stages:
\begin{partlist}
	\item Find a basis solution satisfying all the constraints;
	\item Update the basis solution until it is optimal.
\end{partlist}
Details of each part are included in the following two subsections.

\subsection{Search of basis solution}

To search for a feasible solution, first we choose a vertex $i$ in $V_1$ and a vertex $j$ in $V_2$ and assign the maximal possible value transporting from $i$ to $j$. We add a weighted edge with the maximal value from $i$ to $j$  and abandon one of them which has become zero. Next another vertex is chosen from $U$ if $i$ is abandoned or $V$ if $j$ is abandoned. Previous process is repeated until all the vertices are abandoned.

By this approach, we obtain a feasible solution. We can use this solution as the basis solution.

If the graph $G$ we build based on $\pi_{ij}$ is not connected, we need to add some edges in $E$ to until $G$ is a tree. In each addition of edge, we need to make sure that $G$ is still acylic.

\subsection{Update of solution}

Given a solution $s_{ij}$, first we calculate the dual variable $u_i$ (corresponding to $\mu_i$) and $v_j$ (corresponding to $\nu_i$). The way to find dual variable is based on
\begin{equation}
	u_i+v_j=0
\end{equation}
where $ \rbr{ i, j } \in E $.

Note that if $G$ is a tree, all the $u_i$ and $v_j$ are fixed as long as the initial value (one of $u_i$ or $v_j$) is fixed. Consequently,  $ u_i+v_j$ is a fixed value for all $ i, j $, no matter how the initial value is choosed. Thus we can calculate $c_{ij}-u_i-v_j$ without ambiguity.

If the dual variables of solution $s_{ij}$ satisfy
\begin{equation}
	c_{ij}-u_i-v_j\ge 0
\end{equation}
where $ i = 1, 2, \cdots, m $ and $ j = 1, 2, \cdots, n $, then our solution is optimal.

If there exists $i,j$ not satisfying the previous inequality, we can find $i, j$ so that $c_{ij}-\mu_{i}-\nu{j}$ is has a largest negative part. Then, we add a path from $i$ to $j$ in $G$. In this way, an even-length loop is created in the graph. We shift the maximal amount of mass possible along this cycle, that is, the mass is alternatively added to and subtracted from consecutive transports. After shifting, we may remove the edge whose weight vanishes, and we get a new transportation graph.

We use this update until $c_{ij}-u_i-v_j\ge0$ for all $i,j$. Then we tell the solution corresponding to the graph is the optimal solution.

During our implementation, we find that the description of this method in \parencite{Luenberger2008} and \parencite{Schrieber2017} is not precise enough. In the reference, a process to find $i, j$ so that $c_{ij}-\mu_{i}-\nu_{j}$ is ``the most negative'' is mentioned. However, we encounter some special cases that multiple minima exist. In this case, the original algorithm may fail to find a loop, which leads to a fatal failure. To tackle this problem, we propose a slight modification by record the deleted edge during the update process and remove it from the search list. Note that this slight change would not interfere the convergence of this algorithm, since there is no transportation along the edge as long as it has been deleted. However, this modification is critical to accuracy and stability.

\subsection{Discussion}

The algorithm is listed as Algorithm \ref{Alg:TS}.

\begin{algorithm}
	\caption{Transportation simplex method} \label{Alg:TS}
	\begin{algorithmic}
		\REQUIRE $m$, $n$, $\mu$, $\nu$, $c$
		\STATE \textbf{Step1:} 
		\STATE Find a basis solution $\mathit{sol}$
		\STATE Build graph $G = (V, E)$ based on $\mathit{sol}$
		\STATE \textbf{Step2:} 
		\STATE Calculate the dual variable $u_i$ and $v_j$ corresponding to $\mu_i$, and $\nu_j$ respectively
		\IF{there exists some $i, j$, such that $ c_{ij}-mu_i-nu_j < 0$}
			\STATE Add $(i, j)$ in $E$ and find a loop $l$ in G
			\STATE Update weight of edges in $l$ so that the transportation in $l$ is optimal
			\STATE Delete vanished edge in $E$
			\STATE Update the solution $\mathit{sol}$
			\STATE Run \textbf{Step2} again
		\ENDIF
		\STATE Translate $G$ and $\mathit{sol}$ into the transport plan
	\end{algorithmic}
\end{algorithm}

Unlike other optimization method, simplex method has one distinguishing character. That is, this method could always find the optimal solution, while some other continuous approximation algorithm could only converge to the optimal solution after iterations, which introduces a hard trandoff between time and precision. In our numerical experiments, the unique factor influencing the accuracy of the solution is found to be the machine precision.

Furthermore, in each update iteration, the loss will always decrease. Figure \ref{Fig:CPO} shows the changing loss in iterations, where the loss decrease quickly with stability.

\begin{figure}
\centering
\scalebox{0.65}{%% Creator: Matplotlib, PGF backend
%%
%% To include the figure in your LaTeX document, write
%%   \input{<filename>.pgf}
%%
%% Make sure the required packages are loaded in your preamble
%%   \usepackage{pgf}
%%
%% Figures using additional raster images can only be included by \input if
%% they are in the same directory as the main LaTeX file. For loading figures
%% from other directories you can use the `import` package
%%   \usepackage{import}
%% and then include the figures with
%%   \import{<path to file>}{<filename>.pgf}
%%
%% Matplotlib used the following preamble
%%   \usepackage{fontspec}
%%
\begingroup%
\makeatletter%
\begin{pgfpicture}%
\pgfpathrectangle{\pgfpointorigin}{\pgfqpoint{6.400000in}{4.800000in}}%
\pgfusepath{use as bounding box, clip}%
\begin{pgfscope}%
\pgfsetbuttcap%
\pgfsetmiterjoin%
\definecolor{currentfill}{rgb}{1.000000,1.000000,1.000000}%
\pgfsetfillcolor{currentfill}%
\pgfsetlinewidth{0.000000pt}%
\definecolor{currentstroke}{rgb}{1.000000,1.000000,1.000000}%
\pgfsetstrokecolor{currentstroke}%
\pgfsetdash{}{0pt}%
\pgfpathmoveto{\pgfqpoint{0.000000in}{0.000000in}}%
\pgfpathlineto{\pgfqpoint{6.400000in}{0.000000in}}%
\pgfpathlineto{\pgfqpoint{6.400000in}{4.800000in}}%
\pgfpathlineto{\pgfqpoint{0.000000in}{4.800000in}}%
\pgfpathclose%
\pgfusepath{fill}%
\end{pgfscope}%
\begin{pgfscope}%
\pgfsetbuttcap%
\pgfsetmiterjoin%
\definecolor{currentfill}{rgb}{1.000000,1.000000,1.000000}%
\pgfsetfillcolor{currentfill}%
\pgfsetlinewidth{0.000000pt}%
\definecolor{currentstroke}{rgb}{0.000000,0.000000,0.000000}%
\pgfsetstrokecolor{currentstroke}%
\pgfsetstrokeopacity{0.000000}%
\pgfsetdash{}{0pt}%
\pgfpathmoveto{\pgfqpoint{0.800000in}{0.528000in}}%
\pgfpathlineto{\pgfqpoint{5.760000in}{0.528000in}}%
\pgfpathlineto{\pgfqpoint{5.760000in}{4.224000in}}%
\pgfpathlineto{\pgfqpoint{0.800000in}{4.224000in}}%
\pgfpathclose%
\pgfusepath{fill}%
\end{pgfscope}%
\begin{pgfscope}%
\pgfsetbuttcap%
\pgfsetroundjoin%
\definecolor{currentfill}{rgb}{0.000000,0.000000,0.000000}%
\pgfsetfillcolor{currentfill}%
\pgfsetlinewidth{0.803000pt}%
\definecolor{currentstroke}{rgb}{0.000000,0.000000,0.000000}%
\pgfsetstrokecolor{currentstroke}%
\pgfsetdash{}{0pt}%
\pgfsys@defobject{currentmarker}{\pgfqpoint{0.000000in}{-0.048611in}}{\pgfqpoint{0.000000in}{0.000000in}}{%
\pgfpathmoveto{\pgfqpoint{0.000000in}{0.000000in}}%
\pgfpathlineto{\pgfqpoint{0.000000in}{-0.048611in}}%
\pgfusepath{stroke,fill}%
}%
\begin{pgfscope}%
\pgfsys@transformshift{1.025455in}{0.528000in}%
\pgfsys@useobject{currentmarker}{}%
\end{pgfscope}%
\end{pgfscope}%
\begin{pgfscope}%
\pgftext[x=1.025455in,y=0.430778in,,top]{\rmfamily\fontsize{10.000000}{12.000000}\selectfont \(\displaystyle 0\)}%
\end{pgfscope}%
\begin{pgfscope}%
\pgfsetbuttcap%
\pgfsetroundjoin%
\definecolor{currentfill}{rgb}{0.000000,0.000000,0.000000}%
\pgfsetfillcolor{currentfill}%
\pgfsetlinewidth{0.803000pt}%
\definecolor{currentstroke}{rgb}{0.000000,0.000000,0.000000}%
\pgfsetstrokecolor{currentstroke}%
\pgfsetdash{}{0pt}%
\pgfsys@defobject{currentmarker}{\pgfqpoint{0.000000in}{-0.048611in}}{\pgfqpoint{0.000000in}{0.000000in}}{%
\pgfpathmoveto{\pgfqpoint{0.000000in}{0.000000in}}%
\pgfpathlineto{\pgfqpoint{0.000000in}{-0.048611in}}%
\pgfusepath{stroke,fill}%
}%
\begin{pgfscope}%
\pgfsys@transformshift{2.064895in}{0.528000in}%
\pgfsys@useobject{currentmarker}{}%
\end{pgfscope}%
\end{pgfscope}%
\begin{pgfscope}%
\pgftext[x=2.064895in,y=0.430778in,,top]{\rmfamily\fontsize{10.000000}{12.000000}\selectfont \(\displaystyle 500\)}%
\end{pgfscope}%
\begin{pgfscope}%
\pgfsetbuttcap%
\pgfsetroundjoin%
\definecolor{currentfill}{rgb}{0.000000,0.000000,0.000000}%
\pgfsetfillcolor{currentfill}%
\pgfsetlinewidth{0.803000pt}%
\definecolor{currentstroke}{rgb}{0.000000,0.000000,0.000000}%
\pgfsetstrokecolor{currentstroke}%
\pgfsetdash{}{0pt}%
\pgfsys@defobject{currentmarker}{\pgfqpoint{0.000000in}{-0.048611in}}{\pgfqpoint{0.000000in}{0.000000in}}{%
\pgfpathmoveto{\pgfqpoint{0.000000in}{0.000000in}}%
\pgfpathlineto{\pgfqpoint{0.000000in}{-0.048611in}}%
\pgfusepath{stroke,fill}%
}%
\begin{pgfscope}%
\pgfsys@transformshift{3.104335in}{0.528000in}%
\pgfsys@useobject{currentmarker}{}%
\end{pgfscope}%
\end{pgfscope}%
\begin{pgfscope}%
\pgftext[x=3.104335in,y=0.430778in,,top]{\rmfamily\fontsize{10.000000}{12.000000}\selectfont \(\displaystyle 1000\)}%
\end{pgfscope}%
\begin{pgfscope}%
\pgfsetbuttcap%
\pgfsetroundjoin%
\definecolor{currentfill}{rgb}{0.000000,0.000000,0.000000}%
\pgfsetfillcolor{currentfill}%
\pgfsetlinewidth{0.803000pt}%
\definecolor{currentstroke}{rgb}{0.000000,0.000000,0.000000}%
\pgfsetstrokecolor{currentstroke}%
\pgfsetdash{}{0pt}%
\pgfsys@defobject{currentmarker}{\pgfqpoint{0.000000in}{-0.048611in}}{\pgfqpoint{0.000000in}{0.000000in}}{%
\pgfpathmoveto{\pgfqpoint{0.000000in}{0.000000in}}%
\pgfpathlineto{\pgfqpoint{0.000000in}{-0.048611in}}%
\pgfusepath{stroke,fill}%
}%
\begin{pgfscope}%
\pgfsys@transformshift{4.143775in}{0.528000in}%
\pgfsys@useobject{currentmarker}{}%
\end{pgfscope}%
\end{pgfscope}%
\begin{pgfscope}%
\pgftext[x=4.143775in,y=0.430778in,,top]{\rmfamily\fontsize{10.000000}{12.000000}\selectfont \(\displaystyle 1500\)}%
\end{pgfscope}%
\begin{pgfscope}%
\pgfsetbuttcap%
\pgfsetroundjoin%
\definecolor{currentfill}{rgb}{0.000000,0.000000,0.000000}%
\pgfsetfillcolor{currentfill}%
\pgfsetlinewidth{0.803000pt}%
\definecolor{currentstroke}{rgb}{0.000000,0.000000,0.000000}%
\pgfsetstrokecolor{currentstroke}%
\pgfsetdash{}{0pt}%
\pgfsys@defobject{currentmarker}{\pgfqpoint{0.000000in}{-0.048611in}}{\pgfqpoint{0.000000in}{0.000000in}}{%
\pgfpathmoveto{\pgfqpoint{0.000000in}{0.000000in}}%
\pgfpathlineto{\pgfqpoint{0.000000in}{-0.048611in}}%
\pgfusepath{stroke,fill}%
}%
\begin{pgfscope}%
\pgfsys@transformshift{5.183215in}{0.528000in}%
\pgfsys@useobject{currentmarker}{}%
\end{pgfscope}%
\end{pgfscope}%
\begin{pgfscope}%
\pgftext[x=5.183215in,y=0.430778in,,top]{\rmfamily\fontsize{10.000000}{12.000000}\selectfont \(\displaystyle 2000\)}%
\end{pgfscope}%
\begin{pgfscope}%
\pgfsetbuttcap%
\pgfsetroundjoin%
\definecolor{currentfill}{rgb}{0.000000,0.000000,0.000000}%
\pgfsetfillcolor{currentfill}%
\pgfsetlinewidth{0.803000pt}%
\definecolor{currentstroke}{rgb}{0.000000,0.000000,0.000000}%
\pgfsetstrokecolor{currentstroke}%
\pgfsetdash{}{0pt}%
\pgfsys@defobject{currentmarker}{\pgfqpoint{-0.048611in}{0.000000in}}{\pgfqpoint{0.000000in}{0.000000in}}{%
\pgfpathmoveto{\pgfqpoint{0.000000in}{0.000000in}}%
\pgfpathlineto{\pgfqpoint{-0.048611in}{0.000000in}}%
\pgfusepath{stroke,fill}%
}%
\begin{pgfscope}%
\pgfsys@transformshift{0.800000in}{0.715990in}%
\pgfsys@useobject{currentmarker}{}%
\end{pgfscope}%
\end{pgfscope}%
\begin{pgfscope}%
\pgftext[x=0.455863in,y=0.667796in,left,base]{\rmfamily\fontsize{10.000000}{12.000000}\selectfont \(\displaystyle 1.00\)}%
\end{pgfscope}%
\begin{pgfscope}%
\pgfsetbuttcap%
\pgfsetroundjoin%
\definecolor{currentfill}{rgb}{0.000000,0.000000,0.000000}%
\pgfsetfillcolor{currentfill}%
\pgfsetlinewidth{0.803000pt}%
\definecolor{currentstroke}{rgb}{0.000000,0.000000,0.000000}%
\pgfsetstrokecolor{currentstroke}%
\pgfsetdash{}{0pt}%
\pgfsys@defobject{currentmarker}{\pgfqpoint{-0.048611in}{0.000000in}}{\pgfqpoint{0.000000in}{0.000000in}}{%
\pgfpathmoveto{\pgfqpoint{0.000000in}{0.000000in}}%
\pgfpathlineto{\pgfqpoint{-0.048611in}{0.000000in}}%
\pgfusepath{stroke,fill}%
}%
\begin{pgfscope}%
\pgfsys@transformshift{0.800000in}{1.214783in}%
\pgfsys@useobject{currentmarker}{}%
\end{pgfscope}%
\end{pgfscope}%
\begin{pgfscope}%
\pgftext[x=0.455863in,y=1.166588in,left,base]{\rmfamily\fontsize{10.000000}{12.000000}\selectfont \(\displaystyle 1.25\)}%
\end{pgfscope}%
\begin{pgfscope}%
\pgfsetbuttcap%
\pgfsetroundjoin%
\definecolor{currentfill}{rgb}{0.000000,0.000000,0.000000}%
\pgfsetfillcolor{currentfill}%
\pgfsetlinewidth{0.803000pt}%
\definecolor{currentstroke}{rgb}{0.000000,0.000000,0.000000}%
\pgfsetstrokecolor{currentstroke}%
\pgfsetdash{}{0pt}%
\pgfsys@defobject{currentmarker}{\pgfqpoint{-0.048611in}{0.000000in}}{\pgfqpoint{0.000000in}{0.000000in}}{%
\pgfpathmoveto{\pgfqpoint{0.000000in}{0.000000in}}%
\pgfpathlineto{\pgfqpoint{-0.048611in}{0.000000in}}%
\pgfusepath{stroke,fill}%
}%
\begin{pgfscope}%
\pgfsys@transformshift{0.800000in}{1.713575in}%
\pgfsys@useobject{currentmarker}{}%
\end{pgfscope}%
\end{pgfscope}%
\begin{pgfscope}%
\pgftext[x=0.455863in,y=1.665381in,left,base]{\rmfamily\fontsize{10.000000}{12.000000}\selectfont \(\displaystyle 1.50\)}%
\end{pgfscope}%
\begin{pgfscope}%
\pgfsetbuttcap%
\pgfsetroundjoin%
\definecolor{currentfill}{rgb}{0.000000,0.000000,0.000000}%
\pgfsetfillcolor{currentfill}%
\pgfsetlinewidth{0.803000pt}%
\definecolor{currentstroke}{rgb}{0.000000,0.000000,0.000000}%
\pgfsetstrokecolor{currentstroke}%
\pgfsetdash{}{0pt}%
\pgfsys@defobject{currentmarker}{\pgfqpoint{-0.048611in}{0.000000in}}{\pgfqpoint{0.000000in}{0.000000in}}{%
\pgfpathmoveto{\pgfqpoint{0.000000in}{0.000000in}}%
\pgfpathlineto{\pgfqpoint{-0.048611in}{0.000000in}}%
\pgfusepath{stroke,fill}%
}%
\begin{pgfscope}%
\pgfsys@transformshift{0.800000in}{2.212367in}%
\pgfsys@useobject{currentmarker}{}%
\end{pgfscope}%
\end{pgfscope}%
\begin{pgfscope}%
\pgftext[x=0.455863in,y=2.164173in,left,base]{\rmfamily\fontsize{10.000000}{12.000000}\selectfont \(\displaystyle 1.75\)}%
\end{pgfscope}%
\begin{pgfscope}%
\pgfsetbuttcap%
\pgfsetroundjoin%
\definecolor{currentfill}{rgb}{0.000000,0.000000,0.000000}%
\pgfsetfillcolor{currentfill}%
\pgfsetlinewidth{0.803000pt}%
\definecolor{currentstroke}{rgb}{0.000000,0.000000,0.000000}%
\pgfsetstrokecolor{currentstroke}%
\pgfsetdash{}{0pt}%
\pgfsys@defobject{currentmarker}{\pgfqpoint{-0.048611in}{0.000000in}}{\pgfqpoint{0.000000in}{0.000000in}}{%
\pgfpathmoveto{\pgfqpoint{0.000000in}{0.000000in}}%
\pgfpathlineto{\pgfqpoint{-0.048611in}{0.000000in}}%
\pgfusepath{stroke,fill}%
}%
\begin{pgfscope}%
\pgfsys@transformshift{0.800000in}{2.711160in}%
\pgfsys@useobject{currentmarker}{}%
\end{pgfscope}%
\end{pgfscope}%
\begin{pgfscope}%
\pgftext[x=0.455863in,y=2.662965in,left,base]{\rmfamily\fontsize{10.000000}{12.000000}\selectfont \(\displaystyle 2.00\)}%
\end{pgfscope}%
\begin{pgfscope}%
\pgfsetbuttcap%
\pgfsetroundjoin%
\definecolor{currentfill}{rgb}{0.000000,0.000000,0.000000}%
\pgfsetfillcolor{currentfill}%
\pgfsetlinewidth{0.803000pt}%
\definecolor{currentstroke}{rgb}{0.000000,0.000000,0.000000}%
\pgfsetstrokecolor{currentstroke}%
\pgfsetdash{}{0pt}%
\pgfsys@defobject{currentmarker}{\pgfqpoint{-0.048611in}{0.000000in}}{\pgfqpoint{0.000000in}{0.000000in}}{%
\pgfpathmoveto{\pgfqpoint{0.000000in}{0.000000in}}%
\pgfpathlineto{\pgfqpoint{-0.048611in}{0.000000in}}%
\pgfusepath{stroke,fill}%
}%
\begin{pgfscope}%
\pgfsys@transformshift{0.800000in}{3.209952in}%
\pgfsys@useobject{currentmarker}{}%
\end{pgfscope}%
\end{pgfscope}%
\begin{pgfscope}%
\pgftext[x=0.455863in,y=3.161758in,left,base]{\rmfamily\fontsize{10.000000}{12.000000}\selectfont \(\displaystyle 2.25\)}%
\end{pgfscope}%
\begin{pgfscope}%
\pgfsetbuttcap%
\pgfsetroundjoin%
\definecolor{currentfill}{rgb}{0.000000,0.000000,0.000000}%
\pgfsetfillcolor{currentfill}%
\pgfsetlinewidth{0.803000pt}%
\definecolor{currentstroke}{rgb}{0.000000,0.000000,0.000000}%
\pgfsetstrokecolor{currentstroke}%
\pgfsetdash{}{0pt}%
\pgfsys@defobject{currentmarker}{\pgfqpoint{-0.048611in}{0.000000in}}{\pgfqpoint{0.000000in}{0.000000in}}{%
\pgfpathmoveto{\pgfqpoint{0.000000in}{0.000000in}}%
\pgfpathlineto{\pgfqpoint{-0.048611in}{0.000000in}}%
\pgfusepath{stroke,fill}%
}%
\begin{pgfscope}%
\pgfsys@transformshift{0.800000in}{3.708744in}%
\pgfsys@useobject{currentmarker}{}%
\end{pgfscope}%
\end{pgfscope}%
\begin{pgfscope}%
\pgftext[x=0.455863in,y=3.660550in,left,base]{\rmfamily\fontsize{10.000000}{12.000000}\selectfont \(\displaystyle 2.50\)}%
\end{pgfscope}%
\begin{pgfscope}%
\pgfsetbuttcap%
\pgfsetroundjoin%
\definecolor{currentfill}{rgb}{0.000000,0.000000,0.000000}%
\pgfsetfillcolor{currentfill}%
\pgfsetlinewidth{0.803000pt}%
\definecolor{currentstroke}{rgb}{0.000000,0.000000,0.000000}%
\pgfsetstrokecolor{currentstroke}%
\pgfsetdash{}{0pt}%
\pgfsys@defobject{currentmarker}{\pgfqpoint{-0.048611in}{0.000000in}}{\pgfqpoint{0.000000in}{0.000000in}}{%
\pgfpathmoveto{\pgfqpoint{0.000000in}{0.000000in}}%
\pgfpathlineto{\pgfqpoint{-0.048611in}{0.000000in}}%
\pgfusepath{stroke,fill}%
}%
\begin{pgfscope}%
\pgfsys@transformshift{0.800000in}{4.207537in}%
\pgfsys@useobject{currentmarker}{}%
\end{pgfscope}%
\end{pgfscope}%
\begin{pgfscope}%
\pgftext[x=0.455863in,y=4.159342in,left,base]{\rmfamily\fontsize{10.000000}{12.000000}\selectfont \(\displaystyle 2.75\)}%
\end{pgfscope}%
\begin{pgfscope}%
\pgfpathrectangle{\pgfqpoint{0.800000in}{0.528000in}}{\pgfqpoint{4.960000in}{3.696000in}} %
\pgfusepath{clip}%
\pgfsetrectcap%
\pgfsetroundjoin%
\pgfsetlinewidth{1.505625pt}%
\definecolor{currentstroke}{rgb}{0.121569,0.466667,0.705882}%
\pgfsetstrokecolor{currentstroke}%
\pgfsetdash{}{0pt}%
\pgfpathmoveto{\pgfqpoint{1.025455in}{4.056000in}}%
\pgfpathlineto{\pgfqpoint{1.027533in}{4.052022in}}%
\pgfpathlineto{\pgfqpoint{1.035849in}{4.019722in}}%
\pgfpathlineto{\pgfqpoint{1.046243in}{3.992374in}}%
\pgfpathlineto{\pgfqpoint{1.054559in}{3.942322in}}%
\pgfpathlineto{\pgfqpoint{1.058717in}{3.902095in}}%
\pgfpathlineto{\pgfqpoint{1.062874in}{3.888657in}}%
\pgfpathlineto{\pgfqpoint{1.067032in}{3.868320in}}%
\pgfpathlineto{\pgfqpoint{1.069111in}{3.861542in}}%
\pgfpathlineto{\pgfqpoint{1.071190in}{3.792290in}}%
\pgfpathlineto{\pgfqpoint{1.077427in}{3.757840in}}%
\pgfpathlineto{\pgfqpoint{1.081584in}{3.744247in}}%
\pgfpathlineto{\pgfqpoint{1.083663in}{3.709045in}}%
\pgfpathlineto{\pgfqpoint{1.087821in}{3.687647in}}%
\pgfpathlineto{\pgfqpoint{1.089900in}{3.680345in}}%
\pgfpathlineto{\pgfqpoint{1.094058in}{3.643964in}}%
\pgfpathlineto{\pgfqpoint{1.096136in}{3.636743in}}%
\pgfpathlineto{\pgfqpoint{1.098215in}{3.622366in}}%
\pgfpathlineto{\pgfqpoint{1.102373in}{3.543723in}}%
\pgfpathlineto{\pgfqpoint{1.104452in}{3.522490in}}%
\pgfpathlineto{\pgfqpoint{1.110689in}{3.500797in}}%
\pgfpathlineto{\pgfqpoint{1.112768in}{3.472550in}}%
\pgfpathlineto{\pgfqpoint{1.114846in}{3.465497in}}%
\pgfpathlineto{\pgfqpoint{1.116925in}{3.415763in}}%
\pgfpathlineto{\pgfqpoint{1.125241in}{3.366708in}}%
\pgfpathlineto{\pgfqpoint{1.127320in}{3.359713in}}%
\pgfpathlineto{\pgfqpoint{1.133556in}{3.289191in}}%
\pgfpathlineto{\pgfqpoint{1.135635in}{3.282048in}}%
\pgfpathlineto{\pgfqpoint{1.137714in}{3.260644in}}%
\pgfpathlineto{\pgfqpoint{1.148108in}{3.225308in}}%
\pgfpathlineto{\pgfqpoint{1.150187in}{3.204615in}}%
\pgfpathlineto{\pgfqpoint{1.154345in}{3.191198in}}%
\pgfpathlineto{\pgfqpoint{1.158503in}{3.156996in}}%
\pgfpathlineto{\pgfqpoint{1.160582in}{3.150231in}}%
\pgfpathlineto{\pgfqpoint{1.164740in}{3.130173in}}%
\pgfpathlineto{\pgfqpoint{1.170976in}{3.044849in}}%
\pgfpathlineto{\pgfqpoint{1.173055in}{3.032221in}}%
\pgfpathlineto{\pgfqpoint{1.177213in}{2.967475in}}%
\pgfpathlineto{\pgfqpoint{1.187607in}{2.933848in}}%
\pgfpathlineto{\pgfqpoint{1.191765in}{2.913216in}}%
\pgfpathlineto{\pgfqpoint{1.198002in}{2.892678in}}%
\pgfpathlineto{\pgfqpoint{1.206317in}{2.843750in}}%
\pgfpathlineto{\pgfqpoint{1.218790in}{2.801736in}}%
\pgfpathlineto{\pgfqpoint{1.220869in}{2.767439in}}%
\pgfpathlineto{\pgfqpoint{1.222948in}{2.760604in}}%
\pgfpathlineto{\pgfqpoint{1.225027in}{2.740220in}}%
\pgfpathlineto{\pgfqpoint{1.231264in}{2.720102in}}%
\pgfpathlineto{\pgfqpoint{1.235421in}{2.700517in}}%
\pgfpathlineto{\pgfqpoint{1.237500in}{2.693956in}}%
\pgfpathlineto{\pgfqpoint{1.245816in}{2.648719in}}%
\pgfpathlineto{\pgfqpoint{1.254131in}{2.623058in}}%
\pgfpathlineto{\pgfqpoint{1.256210in}{2.603456in}}%
\pgfpathlineto{\pgfqpoint{1.260368in}{2.590502in}}%
\pgfpathlineto{\pgfqpoint{1.266605in}{2.558673in}}%
\pgfpathlineto{\pgfqpoint{1.270762in}{2.539994in}}%
\pgfpathlineto{\pgfqpoint{1.276999in}{2.521757in}}%
\pgfpathlineto{\pgfqpoint{1.281157in}{2.504132in}}%
\pgfpathlineto{\pgfqpoint{1.283236in}{2.498354in}}%
\pgfpathlineto{\pgfqpoint{1.285315in}{2.481079in}}%
\pgfpathlineto{\pgfqpoint{1.293630in}{2.459088in}}%
\pgfpathlineto{\pgfqpoint{1.299867in}{2.432372in}}%
\pgfpathlineto{\pgfqpoint{1.301946in}{2.427167in}}%
\pgfpathlineto{\pgfqpoint{1.308182in}{2.401477in}}%
\pgfpathlineto{\pgfqpoint{1.312340in}{2.391203in}}%
\pgfpathlineto{\pgfqpoint{1.314419in}{2.365803in}}%
\pgfpathlineto{\pgfqpoint{1.316498in}{2.360747in}}%
\pgfpathlineto{\pgfqpoint{1.322734in}{2.330841in}}%
\pgfpathlineto{\pgfqpoint{1.324813in}{2.325969in}}%
\pgfpathlineto{\pgfqpoint{1.328971in}{2.311591in}}%
\pgfpathlineto{\pgfqpoint{1.333129in}{2.302049in}}%
\pgfpathlineto{\pgfqpoint{1.337287in}{2.287933in}}%
\pgfpathlineto{\pgfqpoint{1.339365in}{2.283293in}}%
\pgfpathlineto{\pgfqpoint{1.343523in}{2.260471in}}%
\pgfpathlineto{\pgfqpoint{1.345602in}{2.219930in}}%
\pgfpathlineto{\pgfqpoint{1.358075in}{2.166583in}}%
\pgfpathlineto{\pgfqpoint{1.360154in}{2.162178in}}%
\pgfpathlineto{\pgfqpoint{1.376785in}{2.092939in}}%
\pgfpathlineto{\pgfqpoint{1.385101in}{2.076186in}}%
\pgfpathlineto{\pgfqpoint{1.387180in}{2.059734in}}%
\pgfpathlineto{\pgfqpoint{1.393416in}{2.047741in}}%
\pgfpathlineto{\pgfqpoint{1.399653in}{2.028056in}}%
\pgfpathlineto{\pgfqpoint{1.403811in}{2.020380in}}%
\pgfpathlineto{\pgfqpoint{1.407968in}{2.008458in}}%
\pgfpathlineto{\pgfqpoint{1.412126in}{2.000563in}}%
\pgfpathlineto{\pgfqpoint{1.418363in}{1.981549in}}%
\pgfpathlineto{\pgfqpoint{1.422521in}{1.974030in}}%
\pgfpathlineto{\pgfqpoint{1.428757in}{1.955284in}}%
\pgfpathlineto{\pgfqpoint{1.432915in}{1.947862in}}%
\pgfpathlineto{\pgfqpoint{1.437073in}{1.936915in}}%
\pgfpathlineto{\pgfqpoint{1.441231in}{1.929704in}}%
\pgfpathlineto{\pgfqpoint{1.449546in}{1.905003in}}%
\pgfpathlineto{\pgfqpoint{1.451625in}{1.901541in}}%
\pgfpathlineto{\pgfqpoint{1.457862in}{1.884775in}}%
\pgfpathlineto{\pgfqpoint{1.459940in}{1.881433in}}%
\pgfpathlineto{\pgfqpoint{1.462019in}{1.871420in}}%
\pgfpathlineto{\pgfqpoint{1.480729in}{1.840490in}}%
\pgfpathlineto{\pgfqpoint{1.491124in}{1.812332in}}%
\pgfpathlineto{\pgfqpoint{1.495281in}{1.806958in}}%
\pgfpathlineto{\pgfqpoint{1.503597in}{1.788559in}}%
\pgfpathlineto{\pgfqpoint{1.509834in}{1.780791in}}%
\pgfpathlineto{\pgfqpoint{1.520228in}{1.752973in}}%
\pgfpathlineto{\pgfqpoint{1.526465in}{1.733358in}}%
\pgfpathlineto{\pgfqpoint{1.534780in}{1.714226in}}%
\pgfpathlineto{\pgfqpoint{1.536859in}{1.711863in}}%
\pgfpathlineto{\pgfqpoint{1.538938in}{1.700238in}}%
\pgfpathlineto{\pgfqpoint{1.543096in}{1.693317in}}%
\pgfpathlineto{\pgfqpoint{1.545175in}{1.691024in}}%
\pgfpathlineto{\pgfqpoint{1.551411in}{1.671520in}}%
\pgfpathlineto{\pgfqpoint{1.555569in}{1.667262in}}%
\pgfpathlineto{\pgfqpoint{1.559727in}{1.656692in}}%
\pgfpathlineto{\pgfqpoint{1.561806in}{1.654606in}}%
\pgfpathlineto{\pgfqpoint{1.563884in}{1.648354in}}%
\pgfpathlineto{\pgfqpoint{1.565963in}{1.633694in}}%
\pgfpathlineto{\pgfqpoint{1.570121in}{1.629534in}}%
\pgfpathlineto{\pgfqpoint{1.572200in}{1.623373in}}%
\pgfpathlineto{\pgfqpoint{1.574279in}{1.621154in}}%
\pgfpathlineto{\pgfqpoint{1.576358in}{1.614536in}}%
\pgfpathlineto{\pgfqpoint{1.578437in}{1.612339in}}%
\pgfpathlineto{\pgfqpoint{1.580516in}{1.605721in}}%
\pgfpathlineto{\pgfqpoint{1.582594in}{1.603396in}}%
\pgfpathlineto{\pgfqpoint{1.592989in}{1.580690in}}%
\pgfpathlineto{\pgfqpoint{1.597147in}{1.576246in}}%
\pgfpathlineto{\pgfqpoint{1.603383in}{1.560893in}}%
\pgfpathlineto{\pgfqpoint{1.605462in}{1.547634in}}%
\pgfpathlineto{\pgfqpoint{1.607541in}{1.545394in}}%
\pgfpathlineto{\pgfqpoint{1.613778in}{1.534383in}}%
\pgfpathlineto{\pgfqpoint{1.620014in}{1.519736in}}%
\pgfpathlineto{\pgfqpoint{1.626251in}{1.504458in}}%
\pgfpathlineto{\pgfqpoint{1.628330in}{1.502588in}}%
\pgfpathlineto{\pgfqpoint{1.636645in}{1.489583in}}%
\pgfpathlineto{\pgfqpoint{1.644961in}{1.482278in}}%
\pgfpathlineto{\pgfqpoint{1.647040in}{1.474978in}}%
\pgfpathlineto{\pgfqpoint{1.649119in}{1.473141in}}%
\pgfpathlineto{\pgfqpoint{1.653276in}{1.467663in}}%
\pgfpathlineto{\pgfqpoint{1.661592in}{1.460930in}}%
\pgfpathlineto{\pgfqpoint{1.663671in}{1.456095in}}%
\pgfpathlineto{\pgfqpoint{1.665750in}{1.454547in}}%
\pgfpathlineto{\pgfqpoint{1.671986in}{1.446869in}}%
\pgfpathlineto{\pgfqpoint{1.686538in}{1.436081in}}%
\pgfpathlineto{\pgfqpoint{1.688617in}{1.431571in}}%
\pgfpathlineto{\pgfqpoint{1.690696in}{1.424059in}}%
\pgfpathlineto{\pgfqpoint{1.694854in}{1.421079in}}%
\pgfpathlineto{\pgfqpoint{1.696933in}{1.416678in}}%
\pgfpathlineto{\pgfqpoint{1.703169in}{1.412306in}}%
\pgfpathlineto{\pgfqpoint{1.707327in}{1.407982in}}%
\pgfpathlineto{\pgfqpoint{1.711485in}{1.405117in}}%
\pgfpathlineto{\pgfqpoint{1.717722in}{1.395255in}}%
\pgfpathlineto{\pgfqpoint{1.719800in}{1.389736in}}%
\pgfpathlineto{\pgfqpoint{1.732274in}{1.380210in}}%
\pgfpathlineto{\pgfqpoint{1.736432in}{1.374821in}}%
\pgfpathlineto{\pgfqpoint{1.738510in}{1.369460in}}%
\pgfpathlineto{\pgfqpoint{1.740589in}{1.368128in}}%
\pgfpathlineto{\pgfqpoint{1.744747in}{1.360390in}}%
\pgfpathlineto{\pgfqpoint{1.750984in}{1.354059in}}%
\pgfpathlineto{\pgfqpoint{1.753063in}{1.352814in}}%
\pgfpathlineto{\pgfqpoint{1.755141in}{1.342849in}}%
\pgfpathlineto{\pgfqpoint{1.763457in}{1.336684in}}%
\pgfpathlineto{\pgfqpoint{1.780088in}{1.318533in}}%
\pgfpathlineto{\pgfqpoint{1.792561in}{1.311332in}}%
\pgfpathlineto{\pgfqpoint{1.796719in}{1.305430in}}%
\pgfpathlineto{\pgfqpoint{1.800877in}{1.303128in}}%
\pgfpathlineto{\pgfqpoint{1.805035in}{1.296538in}}%
\pgfpathlineto{\pgfqpoint{1.813350in}{1.291132in}}%
\pgfpathlineto{\pgfqpoint{1.815429in}{1.286897in}}%
\pgfpathlineto{\pgfqpoint{1.823744in}{1.282677in}}%
\pgfpathlineto{\pgfqpoint{1.825823in}{1.279635in}}%
\pgfpathlineto{\pgfqpoint{1.827902in}{1.274557in}}%
\pgfpathlineto{\pgfqpoint{1.832060in}{1.271544in}}%
\pgfpathlineto{\pgfqpoint{1.834139in}{1.266660in}}%
\pgfpathlineto{\pgfqpoint{1.838297in}{1.263756in}}%
\pgfpathlineto{\pgfqpoint{1.840376in}{1.260882in}}%
\pgfpathlineto{\pgfqpoint{1.842454in}{1.251318in}}%
\pgfpathlineto{\pgfqpoint{1.844533in}{1.250376in}}%
\pgfpathlineto{\pgfqpoint{1.846612in}{1.247592in}}%
\pgfpathlineto{\pgfqpoint{1.848691in}{1.246667in}}%
\pgfpathlineto{\pgfqpoint{1.850770in}{1.240253in}}%
\pgfpathlineto{\pgfqpoint{1.852849in}{1.238440in}}%
\pgfpathlineto{\pgfqpoint{1.854928in}{1.233908in}}%
\pgfpathlineto{\pgfqpoint{1.863243in}{1.229417in}}%
\pgfpathlineto{\pgfqpoint{1.865322in}{1.227670in}}%
\pgfpathlineto{\pgfqpoint{1.867401in}{1.222469in}}%
\pgfpathlineto{\pgfqpoint{1.869480in}{1.221600in}}%
\pgfpathlineto{\pgfqpoint{1.871559in}{1.216383in}}%
\pgfpathlineto{\pgfqpoint{1.879874in}{1.213012in}}%
\pgfpathlineto{\pgfqpoint{1.886111in}{1.206655in}}%
\pgfpathlineto{\pgfqpoint{1.890269in}{1.204380in}}%
\pgfpathlineto{\pgfqpoint{1.894426in}{1.201311in}}%
\pgfpathlineto{\pgfqpoint{1.902742in}{1.198293in}}%
\pgfpathlineto{\pgfqpoint{1.904821in}{1.195312in}}%
\pgfpathlineto{\pgfqpoint{1.911057in}{1.192257in}}%
\pgfpathlineto{\pgfqpoint{1.915215in}{1.189232in}}%
\pgfpathlineto{\pgfqpoint{1.919373in}{1.186975in}}%
\pgfpathlineto{\pgfqpoint{1.923531in}{1.185475in}}%
\pgfpathlineto{\pgfqpoint{1.925610in}{1.181848in}}%
\pgfpathlineto{\pgfqpoint{1.929767in}{1.179586in}}%
\pgfpathlineto{\pgfqpoint{1.933925in}{1.177257in}}%
\pgfpathlineto{\pgfqpoint{1.940162in}{1.174871in}}%
\pgfpathlineto{\pgfqpoint{1.942241in}{1.171690in}}%
\pgfpathlineto{\pgfqpoint{1.944320in}{1.170897in}}%
\pgfpathlineto{\pgfqpoint{1.948477in}{1.163936in}}%
\pgfpathlineto{\pgfqpoint{1.956793in}{1.161044in}}%
\pgfpathlineto{\pgfqpoint{1.965108in}{1.148907in}}%
\pgfpathlineto{\pgfqpoint{1.971345in}{1.146743in}}%
\pgfpathlineto{\pgfqpoint{1.973424in}{1.143136in}}%
\pgfpathlineto{\pgfqpoint{1.975503in}{1.136164in}}%
\pgfpathlineto{\pgfqpoint{1.981739in}{1.132752in}}%
\pgfpathlineto{\pgfqpoint{1.983818in}{1.132082in}}%
\pgfpathlineto{\pgfqpoint{1.987976in}{1.129432in}}%
\pgfpathlineto{\pgfqpoint{1.996292in}{1.123793in}}%
\pgfpathlineto{\pgfqpoint{1.998370in}{1.120699in}}%
\pgfpathlineto{\pgfqpoint{2.000449in}{1.120072in}}%
\pgfpathlineto{\pgfqpoint{2.006686in}{1.113364in}}%
\pgfpathlineto{\pgfqpoint{2.008765in}{1.112768in}}%
\pgfpathlineto{\pgfqpoint{2.010844in}{1.109227in}}%
\pgfpathlineto{\pgfqpoint{2.012923in}{1.108638in}}%
\pgfpathlineto{\pgfqpoint{2.017080in}{1.105108in}}%
\pgfpathlineto{\pgfqpoint{2.021238in}{1.102175in}}%
\pgfpathlineto{\pgfqpoint{2.023317in}{1.101570in}}%
\pgfpathlineto{\pgfqpoint{2.025396in}{1.098552in}}%
\pgfpathlineto{\pgfqpoint{2.029554in}{1.097351in}}%
\pgfpathlineto{\pgfqpoint{2.033711in}{1.093772in}}%
\pgfpathlineto{\pgfqpoint{2.039948in}{1.089084in}}%
\pgfpathlineto{\pgfqpoint{2.042027in}{1.087337in}}%
\pgfpathlineto{\pgfqpoint{2.044106in}{1.082641in}}%
\pgfpathlineto{\pgfqpoint{2.052421in}{1.079814in}}%
\pgfpathlineto{\pgfqpoint{2.071131in}{1.065865in}}%
\pgfpathlineto{\pgfqpoint{2.075289in}{1.064760in}}%
\pgfpathlineto{\pgfqpoint{2.077368in}{1.060359in}}%
\pgfpathlineto{\pgfqpoint{2.087762in}{1.057689in}}%
\pgfpathlineto{\pgfqpoint{2.089841in}{1.055214in}}%
\pgfpathlineto{\pgfqpoint{2.100236in}{1.051734in}}%
\pgfpathlineto{\pgfqpoint{2.118945in}{1.044248in}}%
\pgfpathlineto{\pgfqpoint{2.123103in}{1.043134in}}%
\pgfpathlineto{\pgfqpoint{2.125182in}{1.040191in}}%
\pgfpathlineto{\pgfqpoint{2.148050in}{1.034597in}}%
\pgfpathlineto{\pgfqpoint{2.156365in}{1.030049in}}%
\pgfpathlineto{\pgfqpoint{2.164681in}{1.027848in}}%
\pgfpathlineto{\pgfqpoint{2.175075in}{1.021991in}}%
\pgfpathlineto{\pgfqpoint{2.185470in}{1.019077in}}%
\pgfpathlineto{\pgfqpoint{2.187549in}{1.018165in}}%
\pgfpathlineto{\pgfqpoint{2.189627in}{1.014062in}}%
\pgfpathlineto{\pgfqpoint{2.195864in}{1.011811in}}%
\pgfpathlineto{\pgfqpoint{2.200022in}{1.009531in}}%
\pgfpathlineto{\pgfqpoint{2.210416in}{0.997464in}}%
\pgfpathlineto{\pgfqpoint{2.214574in}{0.993964in}}%
\pgfpathlineto{\pgfqpoint{2.220811in}{0.991955in}}%
\pgfpathlineto{\pgfqpoint{2.224968in}{0.987969in}}%
\pgfpathlineto{\pgfqpoint{2.235363in}{0.984049in}}%
\pgfpathlineto{\pgfqpoint{2.239521in}{0.982615in}}%
\pgfpathlineto{\pgfqpoint{2.243678in}{0.979300in}}%
\pgfpathlineto{\pgfqpoint{2.251994in}{0.976934in}}%
\pgfpathlineto{\pgfqpoint{2.260309in}{0.970244in}}%
\pgfpathlineto{\pgfqpoint{2.264467in}{0.967028in}}%
\pgfpathlineto{\pgfqpoint{2.270704in}{0.964811in}}%
\pgfpathlineto{\pgfqpoint{2.285256in}{0.958883in}}%
\pgfpathlineto{\pgfqpoint{2.287335in}{0.956416in}}%
\pgfpathlineto{\pgfqpoint{2.297729in}{0.953324in}}%
\pgfpathlineto{\pgfqpoint{2.301887in}{0.951795in}}%
\pgfpathlineto{\pgfqpoint{2.306045in}{0.950008in}}%
\pgfpathlineto{\pgfqpoint{2.322676in}{0.945088in}}%
\pgfpathlineto{\pgfqpoint{2.324755in}{0.944760in}}%
\pgfpathlineto{\pgfqpoint{2.328912in}{0.942798in}}%
\pgfpathlineto{\pgfqpoint{2.330991in}{0.942480in}}%
\pgfpathlineto{\pgfqpoint{2.333070in}{0.940573in}}%
\pgfpathlineto{\pgfqpoint{2.341386in}{0.938994in}}%
\pgfpathlineto{\pgfqpoint{2.343465in}{0.937183in}}%
\pgfpathlineto{\pgfqpoint{2.355938in}{0.934385in}}%
\pgfpathlineto{\pgfqpoint{2.372569in}{0.929061in}}%
\pgfpathlineto{\pgfqpoint{2.378805in}{0.927629in}}%
\pgfpathlineto{\pgfqpoint{2.382963in}{0.925685in}}%
\pgfpathlineto{\pgfqpoint{2.393358in}{0.923659in}}%
\pgfpathlineto{\pgfqpoint{2.399594in}{0.920536in}}%
\pgfpathlineto{\pgfqpoint{2.409989in}{0.918837in}}%
\pgfpathlineto{\pgfqpoint{2.416225in}{0.916824in}}%
\pgfpathlineto{\pgfqpoint{2.426620in}{0.913634in}}%
\pgfpathlineto{\pgfqpoint{2.430777in}{0.912421in}}%
\pgfpathlineto{\pgfqpoint{2.441172in}{0.910535in}}%
\pgfpathlineto{\pgfqpoint{2.468197in}{0.900568in}}%
\pgfpathlineto{\pgfqpoint{2.470276in}{0.898255in}}%
\pgfpathlineto{\pgfqpoint{2.478592in}{0.896247in}}%
\pgfpathlineto{\pgfqpoint{2.484828in}{0.893649in}}%
\pgfpathlineto{\pgfqpoint{2.488986in}{0.892143in}}%
\pgfpathlineto{\pgfqpoint{2.491065in}{0.890030in}}%
\pgfpathlineto{\pgfqpoint{2.503538in}{0.886564in}}%
\pgfpathlineto{\pgfqpoint{2.511854in}{0.882942in}}%
\pgfpathlineto{\pgfqpoint{2.513933in}{0.882565in}}%
\pgfpathlineto{\pgfqpoint{2.516012in}{0.878694in}}%
\pgfpathlineto{\pgfqpoint{2.526406in}{0.875253in}}%
\pgfpathlineto{\pgfqpoint{2.534721in}{0.873826in}}%
\pgfpathlineto{\pgfqpoint{2.538879in}{0.871791in}}%
\pgfpathlineto{\pgfqpoint{2.559668in}{0.867601in}}%
\pgfpathlineto{\pgfqpoint{2.565905in}{0.865446in}}%
\pgfpathlineto{\pgfqpoint{2.586693in}{0.861533in}}%
\pgfpathlineto{\pgfqpoint{2.595009in}{0.859606in}}%
\pgfpathlineto{\pgfqpoint{2.630350in}{0.852060in}}%
\pgfpathlineto{\pgfqpoint{2.646981in}{0.849934in}}%
\pgfpathlineto{\pgfqpoint{2.657375in}{0.847366in}}%
\pgfpathlineto{\pgfqpoint{2.665691in}{0.846384in}}%
\pgfpathlineto{\pgfqpoint{2.671928in}{0.844961in}}%
\pgfpathlineto{\pgfqpoint{2.680243in}{0.843431in}}%
\pgfpathlineto{\pgfqpoint{2.711426in}{0.838249in}}%
\pgfpathlineto{\pgfqpoint{2.717663in}{0.836815in}}%
\pgfpathlineto{\pgfqpoint{2.730136in}{0.835022in}}%
\pgfpathlineto{\pgfqpoint{2.736373in}{0.833567in}}%
\pgfpathlineto{\pgfqpoint{2.782108in}{0.826514in}}%
\pgfpathlineto{\pgfqpoint{2.786266in}{0.825148in}}%
\pgfpathlineto{\pgfqpoint{2.800818in}{0.823360in}}%
\pgfpathlineto{\pgfqpoint{2.898526in}{0.808608in}}%
\pgfpathlineto{\pgfqpoint{2.935945in}{0.805416in}}%
\pgfpathlineto{\pgfqpoint{2.952576in}{0.803758in}}%
\pgfpathlineto{\pgfqpoint{2.971286in}{0.801616in}}%
\pgfpathlineto{\pgfqpoint{3.012864in}{0.797261in}}%
\pgfpathlineto{\pgfqpoint{3.054442in}{0.792550in}}%
\pgfpathlineto{\pgfqpoint{3.152149in}{0.781417in}}%
\pgfpathlineto{\pgfqpoint{3.237383in}{0.769971in}}%
\pgfpathlineto{\pgfqpoint{3.291434in}{0.763719in}}%
\pgfpathlineto{\pgfqpoint{3.339248in}{0.760167in}}%
\pgfpathlineto{\pgfqpoint{3.347564in}{0.758945in}}%
\pgfpathlineto{\pgfqpoint{3.353800in}{0.757628in}}%
\pgfpathlineto{\pgfqpoint{3.474375in}{0.748852in}}%
\pgfpathlineto{\pgfqpoint{3.540899in}{0.746574in}}%
\pgfpathlineto{\pgfqpoint{3.570004in}{0.744843in}}%
\pgfpathlineto{\pgfqpoint{3.628212in}{0.741010in}}%
\pgfpathlineto{\pgfqpoint{3.646922in}{0.739827in}}%
\pgfpathlineto{\pgfqpoint{4.012805in}{0.722623in}}%
\pgfpathlineto{\pgfqpoint{4.050225in}{0.721245in}}%
\pgfpathlineto{\pgfqpoint{4.285139in}{0.712455in}}%
\pgfpathlineto{\pgfqpoint{4.310085in}{0.711659in}}%
\pgfpathlineto{\pgfqpoint{4.468080in}{0.707732in}}%
\pgfpathlineto{\pgfqpoint{4.572024in}{0.705996in}}%
\pgfpathlineto{\pgfqpoint{4.994037in}{0.699185in}}%
\pgfpathlineto{\pgfqpoint{5.534545in}{0.696000in}}%
\pgfpathlineto{\pgfqpoint{5.534545in}{0.696000in}}%
\pgfusepath{stroke}%
\end{pgfscope}%
\begin{pgfscope}%
\pgfsetrectcap%
\pgfsetmiterjoin%
\pgfsetlinewidth{0.803000pt}%
\definecolor{currentstroke}{rgb}{0.000000,0.000000,0.000000}%
\pgfsetstrokecolor{currentstroke}%
\pgfsetdash{}{0pt}%
\pgfpathmoveto{\pgfqpoint{0.800000in}{0.528000in}}%
\pgfpathlineto{\pgfqpoint{0.800000in}{4.224000in}}%
\pgfusepath{stroke}%
\end{pgfscope}%
\begin{pgfscope}%
\pgfsetrectcap%
\pgfsetmiterjoin%
\pgfsetlinewidth{0.803000pt}%
\definecolor{currentstroke}{rgb}{0.000000,0.000000,0.000000}%
\pgfsetstrokecolor{currentstroke}%
\pgfsetdash{}{0pt}%
\pgfpathmoveto{\pgfqpoint{5.760000in}{0.528000in}}%
\pgfpathlineto{\pgfqpoint{5.760000in}{4.224000in}}%
\pgfusepath{stroke}%
\end{pgfscope}%
\begin{pgfscope}%
\pgfsetrectcap%
\pgfsetmiterjoin%
\pgfsetlinewidth{0.803000pt}%
\definecolor{currentstroke}{rgb}{0.000000,0.000000,0.000000}%
\pgfsetstrokecolor{currentstroke}%
\pgfsetdash{}{0pt}%
\pgfpathmoveto{\pgfqpoint{0.800000in}{0.528000in}}%
\pgfpathlineto{\pgfqpoint{5.760000in}{0.528000in}}%
\pgfusepath{stroke}%
\end{pgfscope}%
\begin{pgfscope}%
\pgfsetrectcap%
\pgfsetmiterjoin%
\pgfsetlinewidth{0.803000pt}%
\definecolor{currentstroke}{rgb}{0.000000,0.000000,0.000000}%
\pgfsetstrokecolor{currentstroke}%
\pgfsetdash{}{0pt}%
\pgfpathmoveto{\pgfqpoint{0.800000in}{4.224000in}}%
\pgfpathlineto{\pgfqpoint{5.760000in}{4.224000in}}%
\pgfusepath{stroke}%
\end{pgfscope}%
\begin{pgfscope}%
\pgfsetbuttcap%
\pgfsetmiterjoin%
\definecolor{currentfill}{rgb}{1.000000,1.000000,1.000000}%
\pgfsetfillcolor{currentfill}%
\pgfsetfillopacity{0.800000}%
\pgfsetlinewidth{1.003750pt}%
\definecolor{currentstroke}{rgb}{0.800000,0.800000,0.800000}%
\pgfsetstrokecolor{currentstroke}%
\pgfsetstrokeopacity{0.800000}%
\pgfsetdash{}{0pt}%
\pgfpathmoveto{\pgfqpoint{4.448194in}{3.919278in}}%
\pgfpathlineto{\pgfqpoint{5.662778in}{3.919278in}}%
\pgfpathquadraticcurveto{\pgfqpoint{5.690556in}{3.919278in}}{\pgfqpoint{5.690556in}{3.947056in}}%
\pgfpathlineto{\pgfqpoint{5.690556in}{4.126778in}}%
\pgfpathquadraticcurveto{\pgfqpoint{5.690556in}{4.154556in}}{\pgfqpoint{5.662778in}{4.154556in}}%
\pgfpathlineto{\pgfqpoint{4.448194in}{4.154556in}}%
\pgfpathquadraticcurveto{\pgfqpoint{4.420417in}{4.154556in}}{\pgfqpoint{4.420417in}{4.126778in}}%
\pgfpathlineto{\pgfqpoint{4.420417in}{3.947056in}}%
\pgfpathquadraticcurveto{\pgfqpoint{4.420417in}{3.919278in}}{\pgfqpoint{4.448194in}{3.919278in}}%
\pgfpathclose%
\pgfusepath{stroke,fill}%
\end{pgfscope}%
\begin{pgfscope}%
\pgfsetrectcap%
\pgfsetroundjoin%
\pgfsetlinewidth{1.505625pt}%
\definecolor{currentstroke}{rgb}{0.121569,0.466667,0.705882}%
\pgfsetstrokecolor{currentstroke}%
\pgfsetdash{}{0pt}%
\pgfpathmoveto{\pgfqpoint{4.475972in}{4.050389in}}%
\pgfpathlineto{\pgfqpoint{4.753750in}{4.050389in}}%
\pgfusepath{stroke}%
\end{pgfscope}%
\begin{pgfscope}%
\pgftext[x=4.864861in,y=4.001778in,left,base]{\rmfamily\fontsize{10.000000}{12.000000}\selectfont Primal value}%
\end{pgfscope}%
\end{pgfpicture}%
\makeatother%
\endgroup%
}
\caption{Curve of primal objective} \label{Fig:CPO}
\end{figure}

\section{Multiscale strategies} \label{Sec:MS}

According to \textbf{Question 4}, we implement the multiscale method in \parencite{Gerber2017}.

\subsection{Description}

From the linear program formulation \eqref{Eq:StdLP} of optimal transport problems, the rank of linear constraints 
\begin{gather}
\sum_{j=1}^n \pi_{ij} = \mu_i \crbr{ i=1,\dots,m }\\
\sum_{i=1}^m \pi_{ij} = \mu_j \crbr{ j=1,\dots,n-1}
\end{gather}
is exactly $m+n-1$. This implies that the number of non-zero elements in the solution is $n+m-1$ in non-degenerate cases. In other words, most transport paths does not exist in the optimal transport plan. 
However, as a standard LP Problem, it has $m n $ variables, which is far outweigh the number of non-zero elements of the solution.

One intuition is that we may kick absolute impossible paths off and solve the problem focused on the set of possible paths. Furthermore, we want the scale of set of possible paths to be $O(m+n)$ then we won't waste too much time solving the problem on impossible paths.

We may specify the meaning of possible path and impossible path then. Real optimal transport problems always have a good geometry structure like point cloud or image and the coefficient $c_{ij}$ is always a function of the distance between points or pixels. As a result, paths between points with long distance always does not exist in the optimal transport plan. Due to the good geometry structure, we can regard the transport plan as transport between ``big'' points which are aggregation of points with short distance.

Multiscale strategies can be understood by this intuition. We aggregate small points and solve the optimal transport problem between big points to decide whether the path between small points is possible. To introduce details of multiscale strategies, we first explain the meaning of \emph{coarsening} and \emph{propagating}. In \parencite{Gerber2017} there is another concept \emph{refining}. Actually it is a way to make ``propagating'' better.

A way of coarsening is to create a chain connecting the scales from fine to coarse in a multiscale fashion:
\begin{equation}
(X, \mu) = (X_J, \mu_J)\rightarrow(X_{J-1},\mu_{J-1})\rightarrow\cdots\rightarrow(X_1,\mu_1)\rightarrow\cdots\rightarrow(X_0,\mu_0)
\end{equation}
Note that $|X_j|$ decrease  as the scale decreases, and the discrete measure $\mu_j$ is a coarsification of $\mu$ at scale $j$, with $\opsupp(\mu_j) = X_j$. Similarly for $Y$ and $\nu$ we obtain the chain:
\begin{equation}
(Y, \nu) = (Y_J, \nu_J)\rightarrow(Y_{J-1},\nu_{J-1})\rightarrow\cdots\rightarrow(Y_1,\nu_1)\rightarrow\cdots\rightarrow(Y_0,\nu_0)
\end{equation}
Then we define the solution at scale $j$ $s_j$ as correspond optimal transport plan between $\mu_j$ and $\nu_j$.

Propagating is a way using the solution of optimal transport problem between $\mu_j$ and $\nu_j$ to decide which path in scale $j+1$ is possible, thus giving a warm start to the problem at scale $j+1$ and reducing the number of variables.

\subsection{Implementation Details}

As mentioned before, one of the significant advantages of multiscale strategy is reducing the amount of variables. Therefore, we record only the possible transportation paths, in order to save all non-zeros paths instead of saving them in a sparse matrix.

\subsubsection{Coarsening}

We implement the algorithm and test it on point clouds and images. Approaches to coarsening on these two data sets are be slightly different. We implement coarsening on images directly by downsampling, that is, combine $k\times k$ pixels into one pixel. In practical experiments, we choose $k =2$.

There are various ways of coarsening on cloud points, among which are two common ways. The first is to apply K-means algorithm to the point cloud, then combine the points in the same cluster. However, we infer that using K-means here is unreasonable. In most data sets, there are no explicit cluster structure, so the output of K-means may be heavily relevant on the choice of initial point, which indicates that K-means is not capable for this task.

Conditioned this, we choose the second way that we simply divide the points by a grid. We replace the point cloud in a coordinate grid with size $N\times N$, and combine the points in the same square. That's the first step. Next, we can divide the point combined with a new gird. Noticed that we get a $N\times N$ ``dot matrix'' after the first step, we can use either the coarsening method of image to deal with the matrix or division. In both methods, we choose a point in each square with shortest distance to the barycenter of points lying the square as a representative point. We update a new cost matrix $c_j$ for the squares using the cost between the representative points. Figure \ref{Fig:Div} shows an example of such sub-sampling.

\begin{figure}
\centering
\scalebox{0.4}{%% Creator: Matplotlib, PGF backend
%%
%% To include the figure in your LaTeX document, write
%%   \input{<filename>.pgf}
%%
%% Make sure the required packages are loaded in your preamble
%%   \usepackage{pgf}
%%
%% Figures using additional raster images can only be included by \input if
%% they are in the same directory as the main LaTeX file. For loading figures
%% from other directories you can use the `import` package
%%   \usepackage{import}
%% and then include the figures with
%%   \import{<path to file>}{<filename>.pgf}
%%
%% Matplotlib used the following preamble
%%   \usepackage{fontspec}
%%
\begingroup%
\makeatletter%
\begin{pgfpicture}%
\pgfpathrectangle{\pgfpointorigin}{\pgfqpoint{6.400000in}{4.800000in}}%
\pgfusepath{use as bounding box, clip}%
\begin{pgfscope}%
\pgfsetbuttcap%
\pgfsetmiterjoin%
\definecolor{currentfill}{rgb}{1.000000,1.000000,1.000000}%
\pgfsetfillcolor{currentfill}%
\pgfsetlinewidth{0.000000pt}%
\definecolor{currentstroke}{rgb}{1.000000,1.000000,1.000000}%
\pgfsetstrokecolor{currentstroke}%
\pgfsetdash{}{0pt}%
\pgfpathmoveto{\pgfqpoint{0.000000in}{0.000000in}}%
\pgfpathlineto{\pgfqpoint{6.400000in}{0.000000in}}%
\pgfpathlineto{\pgfqpoint{6.400000in}{4.800000in}}%
\pgfpathlineto{\pgfqpoint{0.000000in}{4.800000in}}%
\pgfpathclose%
\pgfusepath{fill}%
\end{pgfscope}%
\begin{pgfscope}%
\pgfsetbuttcap%
\pgfsetmiterjoin%
\definecolor{currentfill}{rgb}{1.000000,1.000000,1.000000}%
\pgfsetfillcolor{currentfill}%
\pgfsetlinewidth{0.000000pt}%
\definecolor{currentstroke}{rgb}{0.000000,0.000000,0.000000}%
\pgfsetstrokecolor{currentstroke}%
\pgfsetstrokeopacity{0.000000}%
\pgfsetdash{}{0pt}%
\pgfpathmoveto{\pgfqpoint{0.800000in}{0.528000in}}%
\pgfpathlineto{\pgfqpoint{5.760000in}{0.528000in}}%
\pgfpathlineto{\pgfqpoint{5.760000in}{4.224000in}}%
\pgfpathlineto{\pgfqpoint{0.800000in}{4.224000in}}%
\pgfpathclose%
\pgfusepath{fill}%
\end{pgfscope}%
\begin{pgfscope}%
\pgfpathrectangle{\pgfqpoint{0.800000in}{0.528000in}}{\pgfqpoint{4.960000in}{3.696000in}} %
\pgfusepath{clip}%
\pgfsetbuttcap%
\pgfsetroundjoin%
\definecolor{currentfill}{rgb}{0.121569,0.466667,0.705882}%
\pgfsetfillcolor{currentfill}%
\pgfsetlinewidth{1.003750pt}%
\definecolor{currentstroke}{rgb}{0.121569,0.466667,0.705882}%
\pgfsetstrokecolor{currentstroke}%
\pgfsetdash{}{0pt}%
\pgfpathmoveto{\pgfqpoint{1.505368in}{3.151009in}}%
\pgfpathcurveto{\pgfqpoint{1.515455in}{3.151009in}}{\pgfqpoint{1.525131in}{3.155016in}}{\pgfqpoint{1.532264in}{3.162149in}}%
\pgfpathcurveto{\pgfqpoint{1.539397in}{3.169282in}}{\pgfqpoint{1.543404in}{3.178957in}}{\pgfqpoint{1.543404in}{3.189045in}}%
\pgfpathcurveto{\pgfqpoint{1.543404in}{3.199132in}}{\pgfqpoint{1.539397in}{3.208808in}}{\pgfqpoint{1.532264in}{3.215941in}}%
\pgfpathcurveto{\pgfqpoint{1.525131in}{3.223073in}}{\pgfqpoint{1.515455in}{3.227081in}}{\pgfqpoint{1.505368in}{3.227081in}}%
\pgfpathcurveto{\pgfqpoint{1.495281in}{3.227081in}}{\pgfqpoint{1.485605in}{3.223073in}}{\pgfqpoint{1.478472in}{3.215941in}}%
\pgfpathcurveto{\pgfqpoint{1.471339in}{3.208808in}}{\pgfqpoint{1.467332in}{3.199132in}}{\pgfqpoint{1.467332in}{3.189045in}}%
\pgfpathcurveto{\pgfqpoint{1.467332in}{3.178957in}}{\pgfqpoint{1.471339in}{3.169282in}}{\pgfqpoint{1.478472in}{3.162149in}}%
\pgfpathcurveto{\pgfqpoint{1.485605in}{3.155016in}}{\pgfqpoint{1.495281in}{3.151009in}}{\pgfqpoint{1.505368in}{3.151009in}}%
\pgfpathclose%
\pgfusepath{stroke,fill}%
\end{pgfscope}%
\begin{pgfscope}%
\pgfpathrectangle{\pgfqpoint{0.800000in}{0.528000in}}{\pgfqpoint{4.960000in}{3.696000in}} %
\pgfusepath{clip}%
\pgfsetbuttcap%
\pgfsetroundjoin%
\definecolor{currentfill}{rgb}{0.121569,0.466667,0.705882}%
\pgfsetfillcolor{currentfill}%
\pgfsetlinewidth{1.003750pt}%
\definecolor{currentstroke}{rgb}{0.121569,0.466667,0.705882}%
\pgfsetstrokecolor{currentstroke}%
\pgfsetdash{}{0pt}%
\pgfpathmoveto{\pgfqpoint{2.532352in}{0.968801in}}%
\pgfpathcurveto{\pgfqpoint{2.542440in}{0.968801in}}{\pgfqpoint{2.552115in}{0.972809in}}{\pgfqpoint{2.559248in}{0.979941in}}%
\pgfpathcurveto{\pgfqpoint{2.566381in}{0.987074in}}{\pgfqpoint{2.570389in}{0.996750in}}{\pgfqpoint{2.570389in}{1.006837in}}%
\pgfpathcurveto{\pgfqpoint{2.570389in}{1.016924in}}{\pgfqpoint{2.566381in}{1.026600in}}{\pgfqpoint{2.559248in}{1.033733in}}%
\pgfpathcurveto{\pgfqpoint{2.552115in}{1.040866in}}{\pgfqpoint{2.542440in}{1.044873in}}{\pgfqpoint{2.532352in}{1.044873in}}%
\pgfpathcurveto{\pgfqpoint{2.522265in}{1.044873in}}{\pgfqpoint{2.512590in}{1.040866in}}{\pgfqpoint{2.505457in}{1.033733in}}%
\pgfpathcurveto{\pgfqpoint{2.498324in}{1.026600in}}{\pgfqpoint{2.494316in}{1.016924in}}{\pgfqpoint{2.494316in}{1.006837in}}%
\pgfpathcurveto{\pgfqpoint{2.494316in}{0.996750in}}{\pgfqpoint{2.498324in}{0.987074in}}{\pgfqpoint{2.505457in}{0.979941in}}%
\pgfpathcurveto{\pgfqpoint{2.512590in}{0.972809in}}{\pgfqpoint{2.522265in}{0.968801in}}{\pgfqpoint{2.532352in}{0.968801in}}%
\pgfpathclose%
\pgfusepath{stroke,fill}%
\end{pgfscope}%
\begin{pgfscope}%
\pgfpathrectangle{\pgfqpoint{0.800000in}{0.528000in}}{\pgfqpoint{4.960000in}{3.696000in}} %
\pgfusepath{clip}%
\pgfsetbuttcap%
\pgfsetroundjoin%
\definecolor{currentfill}{rgb}{0.121569,0.466667,0.705882}%
\pgfsetfillcolor{currentfill}%
\pgfsetlinewidth{1.003750pt}%
\definecolor{currentstroke}{rgb}{0.121569,0.466667,0.705882}%
\pgfsetstrokecolor{currentstroke}%
\pgfsetdash{}{0pt}%
\pgfpathmoveto{\pgfqpoint{5.130812in}{2.559872in}}%
\pgfpathcurveto{\pgfqpoint{5.140900in}{2.559872in}}{\pgfqpoint{5.150575in}{2.563879in}}{\pgfqpoint{5.157708in}{2.571012in}}%
\pgfpathcurveto{\pgfqpoint{5.164841in}{2.578145in}}{\pgfqpoint{5.168849in}{2.587821in}}{\pgfqpoint{5.168849in}{2.597908in}}%
\pgfpathcurveto{\pgfqpoint{5.168849in}{2.607995in}}{\pgfqpoint{5.164841in}{2.617671in}}{\pgfqpoint{5.157708in}{2.624804in}}%
\pgfpathcurveto{\pgfqpoint{5.150575in}{2.631936in}}{\pgfqpoint{5.140900in}{2.635944in}}{\pgfqpoint{5.130812in}{2.635944in}}%
\pgfpathcurveto{\pgfqpoint{5.120725in}{2.635944in}}{\pgfqpoint{5.111050in}{2.631936in}}{\pgfqpoint{5.103917in}{2.624804in}}%
\pgfpathcurveto{\pgfqpoint{5.096784in}{2.617671in}}{\pgfqpoint{5.092776in}{2.607995in}}{\pgfqpoint{5.092776in}{2.597908in}}%
\pgfpathcurveto{\pgfqpoint{5.092776in}{2.587821in}}{\pgfqpoint{5.096784in}{2.578145in}}{\pgfqpoint{5.103917in}{2.571012in}}%
\pgfpathcurveto{\pgfqpoint{5.111050in}{2.563879in}}{\pgfqpoint{5.120725in}{2.559872in}}{\pgfqpoint{5.130812in}{2.559872in}}%
\pgfpathclose%
\pgfusepath{stroke,fill}%
\end{pgfscope}%
\begin{pgfscope}%
\pgfpathrectangle{\pgfqpoint{0.800000in}{0.528000in}}{\pgfqpoint{4.960000in}{3.696000in}} %
\pgfusepath{clip}%
\pgfsetbuttcap%
\pgfsetroundjoin%
\definecolor{currentfill}{rgb}{0.121569,0.466667,0.705882}%
\pgfsetfillcolor{currentfill}%
\pgfsetlinewidth{1.003750pt}%
\definecolor{currentstroke}{rgb}{0.121569,0.466667,0.705882}%
\pgfsetstrokecolor{currentstroke}%
\pgfsetdash{}{0pt}%
\pgfpathmoveto{\pgfqpoint{2.659607in}{3.769311in}}%
\pgfpathcurveto{\pgfqpoint{2.669695in}{3.769311in}}{\pgfqpoint{2.679370in}{3.773318in}}{\pgfqpoint{2.686503in}{3.780451in}}%
\pgfpathcurveto{\pgfqpoint{2.693636in}{3.787584in}}{\pgfqpoint{2.697644in}{3.797260in}}{\pgfqpoint{2.697644in}{3.807347in}}%
\pgfpathcurveto{\pgfqpoint{2.697644in}{3.817434in}}{\pgfqpoint{2.693636in}{3.827110in}}{\pgfqpoint{2.686503in}{3.834243in}}%
\pgfpathcurveto{\pgfqpoint{2.679370in}{3.841376in}}{\pgfqpoint{2.669695in}{3.845383in}}{\pgfqpoint{2.659607in}{3.845383in}}%
\pgfpathcurveto{\pgfqpoint{2.649520in}{3.845383in}}{\pgfqpoint{2.639844in}{3.841376in}}{\pgfqpoint{2.632712in}{3.834243in}}%
\pgfpathcurveto{\pgfqpoint{2.625579in}{3.827110in}}{\pgfqpoint{2.621571in}{3.817434in}}{\pgfqpoint{2.621571in}{3.807347in}}%
\pgfpathcurveto{\pgfqpoint{2.621571in}{3.797260in}}{\pgfqpoint{2.625579in}{3.787584in}}{\pgfqpoint{2.632712in}{3.780451in}}%
\pgfpathcurveto{\pgfqpoint{2.639844in}{3.773318in}}{\pgfqpoint{2.649520in}{3.769311in}}{\pgfqpoint{2.659607in}{3.769311in}}%
\pgfpathclose%
\pgfusepath{stroke,fill}%
\end{pgfscope}%
\begin{pgfscope}%
\pgfpathrectangle{\pgfqpoint{0.800000in}{0.528000in}}{\pgfqpoint{4.960000in}{3.696000in}} %
\pgfusepath{clip}%
\pgfsetbuttcap%
\pgfsetroundjoin%
\definecolor{currentfill}{rgb}{0.121569,0.466667,0.705882}%
\pgfsetfillcolor{currentfill}%
\pgfsetlinewidth{1.003750pt}%
\definecolor{currentstroke}{rgb}{0.121569,0.466667,0.705882}%
\pgfsetstrokecolor{currentstroke}%
\pgfsetdash{}{0pt}%
\pgfpathmoveto{\pgfqpoint{4.419064in}{3.651235in}}%
\pgfpathcurveto{\pgfqpoint{4.429151in}{3.651235in}}{\pgfqpoint{4.438827in}{3.655243in}}{\pgfqpoint{4.445959in}{3.662376in}}%
\pgfpathcurveto{\pgfqpoint{4.453092in}{3.669509in}}{\pgfqpoint{4.457100in}{3.679184in}}{\pgfqpoint{4.457100in}{3.689271in}}%
\pgfpathcurveto{\pgfqpoint{4.457100in}{3.699359in}}{\pgfqpoint{4.453092in}{3.709034in}}{\pgfqpoint{4.445959in}{3.716167in}}%
\pgfpathcurveto{\pgfqpoint{4.438827in}{3.723300in}}{\pgfqpoint{4.429151in}{3.727308in}}{\pgfqpoint{4.419064in}{3.727308in}}%
\pgfpathcurveto{\pgfqpoint{4.408976in}{3.727308in}}{\pgfqpoint{4.399301in}{3.723300in}}{\pgfqpoint{4.392168in}{3.716167in}}%
\pgfpathcurveto{\pgfqpoint{4.385035in}{3.709034in}}{\pgfqpoint{4.381027in}{3.699359in}}{\pgfqpoint{4.381027in}{3.689271in}}%
\pgfpathcurveto{\pgfqpoint{4.381027in}{3.679184in}}{\pgfqpoint{4.385035in}{3.669509in}}{\pgfqpoint{4.392168in}{3.662376in}}%
\pgfpathcurveto{\pgfqpoint{4.399301in}{3.655243in}}{\pgfqpoint{4.408976in}{3.651235in}}{\pgfqpoint{4.419064in}{3.651235in}}%
\pgfpathclose%
\pgfusepath{stroke,fill}%
\end{pgfscope}%
\begin{pgfscope}%
\pgfpathrectangle{\pgfqpoint{0.800000in}{0.528000in}}{\pgfqpoint{4.960000in}{3.696000in}} %
\pgfusepath{clip}%
\pgfsetbuttcap%
\pgfsetroundjoin%
\definecolor{currentfill}{rgb}{0.121569,0.466667,0.705882}%
\pgfsetfillcolor{currentfill}%
\pgfsetlinewidth{1.003750pt}%
\definecolor{currentstroke}{rgb}{0.121569,0.466667,0.705882}%
\pgfsetstrokecolor{currentstroke}%
\pgfsetdash{}{0pt}%
\pgfpathmoveto{\pgfqpoint{4.955618in}{3.186522in}}%
\pgfpathcurveto{\pgfqpoint{4.965706in}{3.186522in}}{\pgfqpoint{4.975381in}{3.190530in}}{\pgfqpoint{4.982514in}{3.197663in}}%
\pgfpathcurveto{\pgfqpoint{4.989647in}{3.204796in}}{\pgfqpoint{4.993655in}{3.214471in}}{\pgfqpoint{4.993655in}{3.224559in}}%
\pgfpathcurveto{\pgfqpoint{4.993655in}{3.234646in}}{\pgfqpoint{4.989647in}{3.244321in}}{\pgfqpoint{4.982514in}{3.251454in}}%
\pgfpathcurveto{\pgfqpoint{4.975381in}{3.258587in}}{\pgfqpoint{4.965706in}{3.262595in}}{\pgfqpoint{4.955618in}{3.262595in}}%
\pgfpathcurveto{\pgfqpoint{4.945531in}{3.262595in}}{\pgfqpoint{4.935856in}{3.258587in}}{\pgfqpoint{4.928723in}{3.251454in}}%
\pgfpathcurveto{\pgfqpoint{4.921590in}{3.244321in}}{\pgfqpoint{4.917582in}{3.234646in}}{\pgfqpoint{4.917582in}{3.224559in}}%
\pgfpathcurveto{\pgfqpoint{4.917582in}{3.214471in}}{\pgfqpoint{4.921590in}{3.204796in}}{\pgfqpoint{4.928723in}{3.197663in}}%
\pgfpathcurveto{\pgfqpoint{4.935856in}{3.190530in}}{\pgfqpoint{4.945531in}{3.186522in}}{\pgfqpoint{4.955618in}{3.186522in}}%
\pgfpathclose%
\pgfusepath{stroke,fill}%
\end{pgfscope}%
\begin{pgfscope}%
\pgfpathrectangle{\pgfqpoint{0.800000in}{0.528000in}}{\pgfqpoint{4.960000in}{3.696000in}} %
\pgfusepath{clip}%
\pgfsetbuttcap%
\pgfsetroundjoin%
\definecolor{currentfill}{rgb}{0.121569,0.466667,0.705882}%
\pgfsetfillcolor{currentfill}%
\pgfsetlinewidth{1.003750pt}%
\definecolor{currentstroke}{rgb}{0.121569,0.466667,0.705882}%
\pgfsetstrokecolor{currentstroke}%
\pgfsetdash{}{0pt}%
\pgfpathmoveto{\pgfqpoint{3.929336in}{3.630764in}}%
\pgfpathcurveto{\pgfqpoint{3.939423in}{3.630764in}}{\pgfqpoint{3.949099in}{3.634771in}}{\pgfqpoint{3.956232in}{3.641904in}}%
\pgfpathcurveto{\pgfqpoint{3.963364in}{3.649037in}}{\pgfqpoint{3.967372in}{3.658713in}}{\pgfqpoint{3.967372in}{3.668800in}}%
\pgfpathcurveto{\pgfqpoint{3.967372in}{3.678887in}}{\pgfqpoint{3.963364in}{3.688563in}}{\pgfqpoint{3.956232in}{3.695696in}}%
\pgfpathcurveto{\pgfqpoint{3.949099in}{3.702829in}}{\pgfqpoint{3.939423in}{3.706836in}}{\pgfqpoint{3.929336in}{3.706836in}}%
\pgfpathcurveto{\pgfqpoint{3.919249in}{3.706836in}}{\pgfqpoint{3.909573in}{3.702829in}}{\pgfqpoint{3.902440in}{3.695696in}}%
\pgfpathcurveto{\pgfqpoint{3.895307in}{3.688563in}}{\pgfqpoint{3.891300in}{3.678887in}}{\pgfqpoint{3.891300in}{3.668800in}}%
\pgfpathcurveto{\pgfqpoint{3.891300in}{3.658713in}}{\pgfqpoint{3.895307in}{3.649037in}}{\pgfqpoint{3.902440in}{3.641904in}}%
\pgfpathcurveto{\pgfqpoint{3.909573in}{3.634771in}}{\pgfqpoint{3.919249in}{3.630764in}}{\pgfqpoint{3.929336in}{3.630764in}}%
\pgfpathclose%
\pgfusepath{stroke,fill}%
\end{pgfscope}%
\begin{pgfscope}%
\pgfpathrectangle{\pgfqpoint{0.800000in}{0.528000in}}{\pgfqpoint{4.960000in}{3.696000in}} %
\pgfusepath{clip}%
\pgfsetbuttcap%
\pgfsetroundjoin%
\definecolor{currentfill}{rgb}{0.121569,0.466667,0.705882}%
\pgfsetfillcolor{currentfill}%
\pgfsetlinewidth{1.003750pt}%
\definecolor{currentstroke}{rgb}{0.121569,0.466667,0.705882}%
\pgfsetstrokecolor{currentstroke}%
\pgfsetdash{}{0pt}%
\pgfpathmoveto{\pgfqpoint{2.033586in}{3.654595in}}%
\pgfpathcurveto{\pgfqpoint{2.043674in}{3.654595in}}{\pgfqpoint{2.053349in}{3.658603in}}{\pgfqpoint{2.060482in}{3.665736in}}%
\pgfpathcurveto{\pgfqpoint{2.067615in}{3.672869in}}{\pgfqpoint{2.071623in}{3.682544in}}{\pgfqpoint{2.071623in}{3.692631in}}%
\pgfpathcurveto{\pgfqpoint{2.071623in}{3.702719in}}{\pgfqpoint{2.067615in}{3.712394in}}{\pgfqpoint{2.060482in}{3.719527in}}%
\pgfpathcurveto{\pgfqpoint{2.053349in}{3.726660in}}{\pgfqpoint{2.043674in}{3.730668in}}{\pgfqpoint{2.033586in}{3.730668in}}%
\pgfpathcurveto{\pgfqpoint{2.023499in}{3.730668in}}{\pgfqpoint{2.013824in}{3.726660in}}{\pgfqpoint{2.006691in}{3.719527in}}%
\pgfpathcurveto{\pgfqpoint{1.999558in}{3.712394in}}{\pgfqpoint{1.995550in}{3.702719in}}{\pgfqpoint{1.995550in}{3.692631in}}%
\pgfpathcurveto{\pgfqpoint{1.995550in}{3.682544in}}{\pgfqpoint{1.999558in}{3.672869in}}{\pgfqpoint{2.006691in}{3.665736in}}%
\pgfpathcurveto{\pgfqpoint{2.013824in}{3.658603in}}{\pgfqpoint{2.023499in}{3.654595in}}{\pgfqpoint{2.033586in}{3.654595in}}%
\pgfpathclose%
\pgfusepath{stroke,fill}%
\end{pgfscope}%
\begin{pgfscope}%
\pgfpathrectangle{\pgfqpoint{0.800000in}{0.528000in}}{\pgfqpoint{4.960000in}{3.696000in}} %
\pgfusepath{clip}%
\pgfsetbuttcap%
\pgfsetroundjoin%
\definecolor{currentfill}{rgb}{0.121569,0.466667,0.705882}%
\pgfsetfillcolor{currentfill}%
\pgfsetlinewidth{1.003750pt}%
\definecolor{currentstroke}{rgb}{0.121569,0.466667,0.705882}%
\pgfsetstrokecolor{currentstroke}%
\pgfsetdash{}{0pt}%
\pgfpathmoveto{\pgfqpoint{1.626868in}{3.352555in}}%
\pgfpathcurveto{\pgfqpoint{1.636955in}{3.352555in}}{\pgfqpoint{1.646631in}{3.356562in}}{\pgfqpoint{1.653763in}{3.363695in}}%
\pgfpathcurveto{\pgfqpoint{1.660896in}{3.370828in}}{\pgfqpoint{1.664904in}{3.380504in}}{\pgfqpoint{1.664904in}{3.390591in}}%
\pgfpathcurveto{\pgfqpoint{1.664904in}{3.400678in}}{\pgfqpoint{1.660896in}{3.410354in}}{\pgfqpoint{1.653763in}{3.417487in}}%
\pgfpathcurveto{\pgfqpoint{1.646631in}{3.424620in}}{\pgfqpoint{1.636955in}{3.428627in}}{\pgfqpoint{1.626868in}{3.428627in}}%
\pgfpathcurveto{\pgfqpoint{1.616780in}{3.428627in}}{\pgfqpoint{1.607105in}{3.424620in}}{\pgfqpoint{1.599972in}{3.417487in}}%
\pgfpathcurveto{\pgfqpoint{1.592839in}{3.410354in}}{\pgfqpoint{1.588831in}{3.400678in}}{\pgfqpoint{1.588831in}{3.390591in}}%
\pgfpathcurveto{\pgfqpoint{1.588831in}{3.380504in}}{\pgfqpoint{1.592839in}{3.370828in}}{\pgfqpoint{1.599972in}{3.363695in}}%
\pgfpathcurveto{\pgfqpoint{1.607105in}{3.356562in}}{\pgfqpoint{1.616780in}{3.352555in}}{\pgfqpoint{1.626868in}{3.352555in}}%
\pgfpathclose%
\pgfusepath{stroke,fill}%
\end{pgfscope}%
\begin{pgfscope}%
\pgfpathrectangle{\pgfqpoint{0.800000in}{0.528000in}}{\pgfqpoint{4.960000in}{3.696000in}} %
\pgfusepath{clip}%
\pgfsetbuttcap%
\pgfsetroundjoin%
\definecolor{currentfill}{rgb}{0.121569,0.466667,0.705882}%
\pgfsetfillcolor{currentfill}%
\pgfsetlinewidth{1.003750pt}%
\definecolor{currentstroke}{rgb}{0.121569,0.466667,0.705882}%
\pgfsetstrokecolor{currentstroke}%
\pgfsetdash{}{0pt}%
\pgfpathmoveto{\pgfqpoint{1.313296in}{2.203542in}}%
\pgfpathcurveto{\pgfqpoint{1.323383in}{2.203542in}}{\pgfqpoint{1.333059in}{2.207550in}}{\pgfqpoint{1.340192in}{2.214683in}}%
\pgfpathcurveto{\pgfqpoint{1.347324in}{2.221815in}}{\pgfqpoint{1.351332in}{2.231491in}}{\pgfqpoint{1.351332in}{2.241578in}}%
\pgfpathcurveto{\pgfqpoint{1.351332in}{2.251666in}}{\pgfqpoint{1.347324in}{2.261341in}}{\pgfqpoint{1.340192in}{2.268474in}}%
\pgfpathcurveto{\pgfqpoint{1.333059in}{2.275607in}}{\pgfqpoint{1.323383in}{2.279615in}}{\pgfqpoint{1.313296in}{2.279615in}}%
\pgfpathcurveto{\pgfqpoint{1.303209in}{2.279615in}}{\pgfqpoint{1.293533in}{2.275607in}}{\pgfqpoint{1.286400in}{2.268474in}}%
\pgfpathcurveto{\pgfqpoint{1.279267in}{2.261341in}}{\pgfqpoint{1.275260in}{2.251666in}}{\pgfqpoint{1.275260in}{2.241578in}}%
\pgfpathcurveto{\pgfqpoint{1.275260in}{2.231491in}}{\pgfqpoint{1.279267in}{2.221815in}}{\pgfqpoint{1.286400in}{2.214683in}}%
\pgfpathcurveto{\pgfqpoint{1.293533in}{2.207550in}}{\pgfqpoint{1.303209in}{2.203542in}}{\pgfqpoint{1.313296in}{2.203542in}}%
\pgfpathclose%
\pgfusepath{stroke,fill}%
\end{pgfscope}%
\begin{pgfscope}%
\pgfpathrectangle{\pgfqpoint{0.800000in}{0.528000in}}{\pgfqpoint{4.960000in}{3.696000in}} %
\pgfusepath{clip}%
\pgfsetbuttcap%
\pgfsetroundjoin%
\definecolor{currentfill}{rgb}{0.121569,0.466667,0.705882}%
\pgfsetfillcolor{currentfill}%
\pgfsetlinewidth{1.003750pt}%
\definecolor{currentstroke}{rgb}{0.121569,0.466667,0.705882}%
\pgfsetstrokecolor{currentstroke}%
\pgfsetdash{}{0pt}%
\pgfpathmoveto{\pgfqpoint{1.475511in}{3.117864in}}%
\pgfpathcurveto{\pgfqpoint{1.485598in}{3.117864in}}{\pgfqpoint{1.495274in}{3.121871in}}{\pgfqpoint{1.502407in}{3.129004in}}%
\pgfpathcurveto{\pgfqpoint{1.509539in}{3.136137in}}{\pgfqpoint{1.513547in}{3.145813in}}{\pgfqpoint{1.513547in}{3.155900in}}%
\pgfpathcurveto{\pgfqpoint{1.513547in}{3.165987in}}{\pgfqpoint{1.509539in}{3.175663in}}{\pgfqpoint{1.502407in}{3.182796in}}%
\pgfpathcurveto{\pgfqpoint{1.495274in}{3.189928in}}{\pgfqpoint{1.485598in}{3.193936in}}{\pgfqpoint{1.475511in}{3.193936in}}%
\pgfpathcurveto{\pgfqpoint{1.465424in}{3.193936in}}{\pgfqpoint{1.455748in}{3.189928in}}{\pgfqpoint{1.448615in}{3.182796in}}%
\pgfpathcurveto{\pgfqpoint{1.441482in}{3.175663in}}{\pgfqpoint{1.437475in}{3.165987in}}{\pgfqpoint{1.437475in}{3.155900in}}%
\pgfpathcurveto{\pgfqpoint{1.437475in}{3.145813in}}{\pgfqpoint{1.441482in}{3.136137in}}{\pgfqpoint{1.448615in}{3.129004in}}%
\pgfpathcurveto{\pgfqpoint{1.455748in}{3.121871in}}{\pgfqpoint{1.465424in}{3.117864in}}{\pgfqpoint{1.475511in}{3.117864in}}%
\pgfpathclose%
\pgfusepath{stroke,fill}%
\end{pgfscope}%
\begin{pgfscope}%
\pgfpathrectangle{\pgfqpoint{0.800000in}{0.528000in}}{\pgfqpoint{4.960000in}{3.696000in}} %
\pgfusepath{clip}%
\pgfsetbuttcap%
\pgfsetroundjoin%
\definecolor{currentfill}{rgb}{0.121569,0.466667,0.705882}%
\pgfsetfillcolor{currentfill}%
\pgfsetlinewidth{1.003750pt}%
\definecolor{currentstroke}{rgb}{0.121569,0.466667,0.705882}%
\pgfsetstrokecolor{currentstroke}%
\pgfsetdash{}{0pt}%
\pgfpathmoveto{\pgfqpoint{2.414835in}{1.022220in}}%
\pgfpathcurveto{\pgfqpoint{2.424923in}{1.022220in}}{\pgfqpoint{2.434598in}{1.026227in}}{\pgfqpoint{2.441731in}{1.033360in}}%
\pgfpathcurveto{\pgfqpoint{2.448864in}{1.040493in}}{\pgfqpoint{2.452872in}{1.050169in}}{\pgfqpoint{2.452872in}{1.060256in}}%
\pgfpathcurveto{\pgfqpoint{2.452872in}{1.070343in}}{\pgfqpoint{2.448864in}{1.080019in}}{\pgfqpoint{2.441731in}{1.087152in}}%
\pgfpathcurveto{\pgfqpoint{2.434598in}{1.094284in}}{\pgfqpoint{2.424923in}{1.098292in}}{\pgfqpoint{2.414835in}{1.098292in}}%
\pgfpathcurveto{\pgfqpoint{2.404748in}{1.098292in}}{\pgfqpoint{2.395072in}{1.094284in}}{\pgfqpoint{2.387940in}{1.087152in}}%
\pgfpathcurveto{\pgfqpoint{2.380807in}{1.080019in}}{\pgfqpoint{2.376799in}{1.070343in}}{\pgfqpoint{2.376799in}{1.060256in}}%
\pgfpathcurveto{\pgfqpoint{2.376799in}{1.050169in}}{\pgfqpoint{2.380807in}{1.040493in}}{\pgfqpoint{2.387940in}{1.033360in}}%
\pgfpathcurveto{\pgfqpoint{2.395072in}{1.026227in}}{\pgfqpoint{2.404748in}{1.022220in}}{\pgfqpoint{2.414835in}{1.022220in}}%
\pgfpathclose%
\pgfusepath{stroke,fill}%
\end{pgfscope}%
\begin{pgfscope}%
\pgfpathrectangle{\pgfqpoint{0.800000in}{0.528000in}}{\pgfqpoint{4.960000in}{3.696000in}} %
\pgfusepath{clip}%
\pgfsetbuttcap%
\pgfsetroundjoin%
\definecolor{currentfill}{rgb}{0.121569,0.466667,0.705882}%
\pgfsetfillcolor{currentfill}%
\pgfsetlinewidth{1.003750pt}%
\definecolor{currentstroke}{rgb}{0.121569,0.466667,0.705882}%
\pgfsetstrokecolor{currentstroke}%
\pgfsetdash{}{0pt}%
\pgfpathmoveto{\pgfqpoint{3.652380in}{3.875540in}}%
\pgfpathcurveto{\pgfqpoint{3.662467in}{3.875540in}}{\pgfqpoint{3.672142in}{3.879548in}}{\pgfqpoint{3.679275in}{3.886681in}}%
\pgfpathcurveto{\pgfqpoint{3.686408in}{3.893814in}}{\pgfqpoint{3.690416in}{3.903489in}}{\pgfqpoint{3.690416in}{3.913577in}}%
\pgfpathcurveto{\pgfqpoint{3.690416in}{3.923664in}}{\pgfqpoint{3.686408in}{3.933339in}}{\pgfqpoint{3.679275in}{3.940472in}}%
\pgfpathcurveto{\pgfqpoint{3.672142in}{3.947605in}}{\pgfqpoint{3.662467in}{3.951613in}}{\pgfqpoint{3.652380in}{3.951613in}}%
\pgfpathcurveto{\pgfqpoint{3.642292in}{3.951613in}}{\pgfqpoint{3.632617in}{3.947605in}}{\pgfqpoint{3.625484in}{3.940472in}}%
\pgfpathcurveto{\pgfqpoint{3.618351in}{3.933339in}}{\pgfqpoint{3.614343in}{3.923664in}}{\pgfqpoint{3.614343in}{3.913577in}}%
\pgfpathcurveto{\pgfqpoint{3.614343in}{3.903489in}}{\pgfqpoint{3.618351in}{3.893814in}}{\pgfqpoint{3.625484in}{3.886681in}}%
\pgfpathcurveto{\pgfqpoint{3.632617in}{3.879548in}}{\pgfqpoint{3.642292in}{3.875540in}}{\pgfqpoint{3.652380in}{3.875540in}}%
\pgfpathclose%
\pgfusepath{stroke,fill}%
\end{pgfscope}%
\begin{pgfscope}%
\pgfpathrectangle{\pgfqpoint{0.800000in}{0.528000in}}{\pgfqpoint{4.960000in}{3.696000in}} %
\pgfusepath{clip}%
\pgfsetbuttcap%
\pgfsetroundjoin%
\definecolor{currentfill}{rgb}{0.121569,0.466667,0.705882}%
\pgfsetfillcolor{currentfill}%
\pgfsetlinewidth{1.003750pt}%
\definecolor{currentstroke}{rgb}{0.121569,0.466667,0.705882}%
\pgfsetstrokecolor{currentstroke}%
\pgfsetdash{}{0pt}%
\pgfpathmoveto{\pgfqpoint{4.671799in}{1.438134in}}%
\pgfpathcurveto{\pgfqpoint{4.681886in}{1.438134in}}{\pgfqpoint{4.691562in}{1.442142in}}{\pgfqpoint{4.698695in}{1.449275in}}%
\pgfpathcurveto{\pgfqpoint{4.705827in}{1.456408in}}{\pgfqpoint{4.709835in}{1.466083in}}{\pgfqpoint{4.709835in}{1.476171in}}%
\pgfpathcurveto{\pgfqpoint{4.709835in}{1.486258in}}{\pgfqpoint{4.705827in}{1.495933in}}{\pgfqpoint{4.698695in}{1.503066in}}%
\pgfpathcurveto{\pgfqpoint{4.691562in}{1.510199in}}{\pgfqpoint{4.681886in}{1.514207in}}{\pgfqpoint{4.671799in}{1.514207in}}%
\pgfpathcurveto{\pgfqpoint{4.661711in}{1.514207in}}{\pgfqpoint{4.652036in}{1.510199in}}{\pgfqpoint{4.644903in}{1.503066in}}%
\pgfpathcurveto{\pgfqpoint{4.637770in}{1.495933in}}{\pgfqpoint{4.633763in}{1.486258in}}{\pgfqpoint{4.633763in}{1.476171in}}%
\pgfpathcurveto{\pgfqpoint{4.633763in}{1.466083in}}{\pgfqpoint{4.637770in}{1.456408in}}{\pgfqpoint{4.644903in}{1.449275in}}%
\pgfpathcurveto{\pgfqpoint{4.652036in}{1.442142in}}{\pgfqpoint{4.661711in}{1.438134in}}{\pgfqpoint{4.671799in}{1.438134in}}%
\pgfpathclose%
\pgfusepath{stroke,fill}%
\end{pgfscope}%
\begin{pgfscope}%
\pgfpathrectangle{\pgfqpoint{0.800000in}{0.528000in}}{\pgfqpoint{4.960000in}{3.696000in}} %
\pgfusepath{clip}%
\pgfsetbuttcap%
\pgfsetroundjoin%
\definecolor{currentfill}{rgb}{0.121569,0.466667,0.705882}%
\pgfsetfillcolor{currentfill}%
\pgfsetlinewidth{1.003750pt}%
\definecolor{currentstroke}{rgb}{0.121569,0.466667,0.705882}%
\pgfsetstrokecolor{currentstroke}%
\pgfsetdash{}{0pt}%
\pgfpathmoveto{\pgfqpoint{5.161628in}{2.699858in}}%
\pgfpathcurveto{\pgfqpoint{5.171716in}{2.699858in}}{\pgfqpoint{5.181391in}{2.703866in}}{\pgfqpoint{5.188524in}{2.710999in}}%
\pgfpathcurveto{\pgfqpoint{5.195657in}{2.718132in}}{\pgfqpoint{5.199665in}{2.727807in}}{\pgfqpoint{5.199665in}{2.737895in}}%
\pgfpathcurveto{\pgfqpoint{5.199665in}{2.747982in}}{\pgfqpoint{5.195657in}{2.757658in}}{\pgfqpoint{5.188524in}{2.764790in}}%
\pgfpathcurveto{\pgfqpoint{5.181391in}{2.771923in}}{\pgfqpoint{5.171716in}{2.775931in}}{\pgfqpoint{5.161628in}{2.775931in}}%
\pgfpathcurveto{\pgfqpoint{5.151541in}{2.775931in}}{\pgfqpoint{5.141865in}{2.771923in}}{\pgfqpoint{5.134733in}{2.764790in}}%
\pgfpathcurveto{\pgfqpoint{5.127600in}{2.757658in}}{\pgfqpoint{5.123592in}{2.747982in}}{\pgfqpoint{5.123592in}{2.737895in}}%
\pgfpathcurveto{\pgfqpoint{5.123592in}{2.727807in}}{\pgfqpoint{5.127600in}{2.718132in}}{\pgfqpoint{5.134733in}{2.710999in}}%
\pgfpathcurveto{\pgfqpoint{5.141865in}{2.703866in}}{\pgfqpoint{5.151541in}{2.699858in}}{\pgfqpoint{5.161628in}{2.699858in}}%
\pgfpathclose%
\pgfusepath{stroke,fill}%
\end{pgfscope}%
\begin{pgfscope}%
\pgfpathrectangle{\pgfqpoint{0.800000in}{0.528000in}}{\pgfqpoint{4.960000in}{3.696000in}} %
\pgfusepath{clip}%
\pgfsetbuttcap%
\pgfsetroundjoin%
\definecolor{currentfill}{rgb}{0.121569,0.466667,0.705882}%
\pgfsetfillcolor{currentfill}%
\pgfsetlinewidth{1.003750pt}%
\definecolor{currentstroke}{rgb}{0.121569,0.466667,0.705882}%
\pgfsetstrokecolor{currentstroke}%
\pgfsetdash{}{0pt}%
\pgfpathmoveto{\pgfqpoint{2.391282in}{1.137261in}}%
\pgfpathcurveto{\pgfqpoint{2.401370in}{1.137261in}}{\pgfqpoint{2.411045in}{1.141269in}}{\pgfqpoint{2.418178in}{1.148402in}}%
\pgfpathcurveto{\pgfqpoint{2.425311in}{1.155535in}}{\pgfqpoint{2.429319in}{1.165210in}}{\pgfqpoint{2.429319in}{1.175298in}}%
\pgfpathcurveto{\pgfqpoint{2.429319in}{1.185385in}}{\pgfqpoint{2.425311in}{1.195060in}}{\pgfqpoint{2.418178in}{1.202193in}}%
\pgfpathcurveto{\pgfqpoint{2.411045in}{1.209326in}}{\pgfqpoint{2.401370in}{1.213334in}}{\pgfqpoint{2.391282in}{1.213334in}}%
\pgfpathcurveto{\pgfqpoint{2.381195in}{1.213334in}}{\pgfqpoint{2.371519in}{1.209326in}}{\pgfqpoint{2.364387in}{1.202193in}}%
\pgfpathcurveto{\pgfqpoint{2.357254in}{1.195060in}}{\pgfqpoint{2.353246in}{1.185385in}}{\pgfqpoint{2.353246in}{1.175298in}}%
\pgfpathcurveto{\pgfqpoint{2.353246in}{1.165210in}}{\pgfqpoint{2.357254in}{1.155535in}}{\pgfqpoint{2.364387in}{1.148402in}}%
\pgfpathcurveto{\pgfqpoint{2.371519in}{1.141269in}}{\pgfqpoint{2.381195in}{1.137261in}}{\pgfqpoint{2.391282in}{1.137261in}}%
\pgfpathclose%
\pgfusepath{stroke,fill}%
\end{pgfscope}%
\begin{pgfscope}%
\pgfpathrectangle{\pgfqpoint{0.800000in}{0.528000in}}{\pgfqpoint{4.960000in}{3.696000in}} %
\pgfusepath{clip}%
\pgfsetbuttcap%
\pgfsetroundjoin%
\definecolor{currentfill}{rgb}{0.121569,0.466667,0.705882}%
\pgfsetfillcolor{currentfill}%
\pgfsetlinewidth{1.003750pt}%
\definecolor{currentstroke}{rgb}{0.121569,0.466667,0.705882}%
\pgfsetstrokecolor{currentstroke}%
\pgfsetdash{}{0pt}%
\pgfpathmoveto{\pgfqpoint{1.627810in}{3.263651in}}%
\pgfpathcurveto{\pgfqpoint{1.637898in}{3.263651in}}{\pgfqpoint{1.647573in}{3.267659in}}{\pgfqpoint{1.654706in}{3.274792in}}%
\pgfpathcurveto{\pgfqpoint{1.661839in}{3.281925in}}{\pgfqpoint{1.665846in}{3.291600in}}{\pgfqpoint{1.665846in}{3.301687in}}%
\pgfpathcurveto{\pgfqpoint{1.665846in}{3.311775in}}{\pgfqpoint{1.661839in}{3.321450in}}{\pgfqpoint{1.654706in}{3.328583in}}%
\pgfpathcurveto{\pgfqpoint{1.647573in}{3.335716in}}{\pgfqpoint{1.637898in}{3.339724in}}{\pgfqpoint{1.627810in}{3.339724in}}%
\pgfpathcurveto{\pgfqpoint{1.617723in}{3.339724in}}{\pgfqpoint{1.608047in}{3.335716in}}{\pgfqpoint{1.600914in}{3.328583in}}%
\pgfpathcurveto{\pgfqpoint{1.593782in}{3.321450in}}{\pgfqpoint{1.589774in}{3.311775in}}{\pgfqpoint{1.589774in}{3.301687in}}%
\pgfpathcurveto{\pgfqpoint{1.589774in}{3.291600in}}{\pgfqpoint{1.593782in}{3.281925in}}{\pgfqpoint{1.600914in}{3.274792in}}%
\pgfpathcurveto{\pgfqpoint{1.608047in}{3.267659in}}{\pgfqpoint{1.617723in}{3.263651in}}{\pgfqpoint{1.627810in}{3.263651in}}%
\pgfpathclose%
\pgfusepath{stroke,fill}%
\end{pgfscope}%
\begin{pgfscope}%
\pgfpathrectangle{\pgfqpoint{0.800000in}{0.528000in}}{\pgfqpoint{4.960000in}{3.696000in}} %
\pgfusepath{clip}%
\pgfsetbuttcap%
\pgfsetroundjoin%
\definecolor{currentfill}{rgb}{0.121569,0.466667,0.705882}%
\pgfsetfillcolor{currentfill}%
\pgfsetlinewidth{1.003750pt}%
\definecolor{currentstroke}{rgb}{0.121569,0.466667,0.705882}%
\pgfsetstrokecolor{currentstroke}%
\pgfsetdash{}{0pt}%
\pgfpathmoveto{\pgfqpoint{1.356502in}{1.935398in}}%
\pgfpathcurveto{\pgfqpoint{1.366589in}{1.935398in}}{\pgfqpoint{1.376265in}{1.939406in}}{\pgfqpoint{1.383398in}{1.946539in}}%
\pgfpathcurveto{\pgfqpoint{1.390530in}{1.953672in}}{\pgfqpoint{1.394538in}{1.963347in}}{\pgfqpoint{1.394538in}{1.973435in}}%
\pgfpathcurveto{\pgfqpoint{1.394538in}{1.983522in}}{\pgfqpoint{1.390530in}{1.993198in}}{\pgfqpoint{1.383398in}{2.000331in}}%
\pgfpathcurveto{\pgfqpoint{1.376265in}{2.007463in}}{\pgfqpoint{1.366589in}{2.011471in}}{\pgfqpoint{1.356502in}{2.011471in}}%
\pgfpathcurveto{\pgfqpoint{1.346414in}{2.011471in}}{\pgfqpoint{1.336739in}{2.007463in}}{\pgfqpoint{1.329606in}{2.000331in}}%
\pgfpathcurveto{\pgfqpoint{1.322473in}{1.993198in}}{\pgfqpoint{1.318466in}{1.983522in}}{\pgfqpoint{1.318466in}{1.973435in}}%
\pgfpathcurveto{\pgfqpoint{1.318466in}{1.963347in}}{\pgfqpoint{1.322473in}{1.953672in}}{\pgfqpoint{1.329606in}{1.946539in}}%
\pgfpathcurveto{\pgfqpoint{1.336739in}{1.939406in}}{\pgfqpoint{1.346414in}{1.935398in}}{\pgfqpoint{1.356502in}{1.935398in}}%
\pgfpathclose%
\pgfusepath{stroke,fill}%
\end{pgfscope}%
\begin{pgfscope}%
\pgfpathrectangle{\pgfqpoint{0.800000in}{0.528000in}}{\pgfqpoint{4.960000in}{3.696000in}} %
\pgfusepath{clip}%
\pgfsetbuttcap%
\pgfsetroundjoin%
\definecolor{currentfill}{rgb}{0.121569,0.466667,0.705882}%
\pgfsetfillcolor{currentfill}%
\pgfsetlinewidth{1.003750pt}%
\definecolor{currentstroke}{rgb}{0.121569,0.466667,0.705882}%
\pgfsetstrokecolor{currentstroke}%
\pgfsetdash{}{0pt}%
\pgfpathmoveto{\pgfqpoint{4.396374in}{3.741129in}}%
\pgfpathcurveto{\pgfqpoint{4.406461in}{3.741129in}}{\pgfqpoint{4.416136in}{3.745136in}}{\pgfqpoint{4.423269in}{3.752269in}}%
\pgfpathcurveto{\pgfqpoint{4.430402in}{3.759402in}}{\pgfqpoint{4.434410in}{3.769078in}}{\pgfqpoint{4.434410in}{3.779165in}}%
\pgfpathcurveto{\pgfqpoint{4.434410in}{3.789252in}}{\pgfqpoint{4.430402in}{3.798928in}}{\pgfqpoint{4.423269in}{3.806061in}}%
\pgfpathcurveto{\pgfqpoint{4.416136in}{3.813193in}}{\pgfqpoint{4.406461in}{3.817201in}}{\pgfqpoint{4.396374in}{3.817201in}}%
\pgfpathcurveto{\pgfqpoint{4.386286in}{3.817201in}}{\pgfqpoint{4.376611in}{3.813193in}}{\pgfqpoint{4.369478in}{3.806061in}}%
\pgfpathcurveto{\pgfqpoint{4.362345in}{3.798928in}}{\pgfqpoint{4.358337in}{3.789252in}}{\pgfqpoint{4.358337in}{3.779165in}}%
\pgfpathcurveto{\pgfqpoint{4.358337in}{3.769078in}}{\pgfqpoint{4.362345in}{3.759402in}}{\pgfqpoint{4.369478in}{3.752269in}}%
\pgfpathcurveto{\pgfqpoint{4.376611in}{3.745136in}}{\pgfqpoint{4.386286in}{3.741129in}}{\pgfqpoint{4.396374in}{3.741129in}}%
\pgfpathclose%
\pgfusepath{stroke,fill}%
\end{pgfscope}%
\begin{pgfscope}%
\pgfpathrectangle{\pgfqpoint{0.800000in}{0.528000in}}{\pgfqpoint{4.960000in}{3.696000in}} %
\pgfusepath{clip}%
\pgfsetbuttcap%
\pgfsetroundjoin%
\definecolor{currentfill}{rgb}{0.121569,0.466667,0.705882}%
\pgfsetfillcolor{currentfill}%
\pgfsetlinewidth{1.003750pt}%
\definecolor{currentstroke}{rgb}{0.121569,0.466667,0.705882}%
\pgfsetstrokecolor{currentstroke}%
\pgfsetdash{}{0pt}%
\pgfpathmoveto{\pgfqpoint{3.732524in}{3.914454in}}%
\pgfpathcurveto{\pgfqpoint{3.742611in}{3.914454in}}{\pgfqpoint{3.752287in}{3.918462in}}{\pgfqpoint{3.759420in}{3.925595in}}%
\pgfpathcurveto{\pgfqpoint{3.766552in}{3.932727in}}{\pgfqpoint{3.770560in}{3.942403in}}{\pgfqpoint{3.770560in}{3.952490in}}%
\pgfpathcurveto{\pgfqpoint{3.770560in}{3.962578in}}{\pgfqpoint{3.766552in}{3.972253in}}{\pgfqpoint{3.759420in}{3.979386in}}%
\pgfpathcurveto{\pgfqpoint{3.752287in}{3.986519in}}{\pgfqpoint{3.742611in}{3.990527in}}{\pgfqpoint{3.732524in}{3.990527in}}%
\pgfpathcurveto{\pgfqpoint{3.722437in}{3.990527in}}{\pgfqpoint{3.712761in}{3.986519in}}{\pgfqpoint{3.705628in}{3.979386in}}%
\pgfpathcurveto{\pgfqpoint{3.698495in}{3.972253in}}{\pgfqpoint{3.694488in}{3.962578in}}{\pgfqpoint{3.694488in}{3.952490in}}%
\pgfpathcurveto{\pgfqpoint{3.694488in}{3.942403in}}{\pgfqpoint{3.698495in}{3.932727in}}{\pgfqpoint{3.705628in}{3.925595in}}%
\pgfpathcurveto{\pgfqpoint{3.712761in}{3.918462in}}{\pgfqpoint{3.722437in}{3.914454in}}{\pgfqpoint{3.732524in}{3.914454in}}%
\pgfpathclose%
\pgfusepath{stroke,fill}%
\end{pgfscope}%
\begin{pgfscope}%
\pgfpathrectangle{\pgfqpoint{0.800000in}{0.528000in}}{\pgfqpoint{4.960000in}{3.696000in}} %
\pgfusepath{clip}%
\pgfsetbuttcap%
\pgfsetroundjoin%
\definecolor{currentfill}{rgb}{0.121569,0.466667,0.705882}%
\pgfsetfillcolor{currentfill}%
\pgfsetlinewidth{1.003750pt}%
\definecolor{currentstroke}{rgb}{0.121569,0.466667,0.705882}%
\pgfsetstrokecolor{currentstroke}%
\pgfsetdash{}{0pt}%
\pgfpathmoveto{\pgfqpoint{3.868210in}{1.108001in}}%
\pgfpathcurveto{\pgfqpoint{3.878297in}{1.108001in}}{\pgfqpoint{3.887973in}{1.112008in}}{\pgfqpoint{3.895106in}{1.119141in}}%
\pgfpathcurveto{\pgfqpoint{3.902238in}{1.126274in}}{\pgfqpoint{3.906246in}{1.135950in}}{\pgfqpoint{3.906246in}{1.146037in}}%
\pgfpathcurveto{\pgfqpoint{3.906246in}{1.156124in}}{\pgfqpoint{3.902238in}{1.165800in}}{\pgfqpoint{3.895106in}{1.172933in}}%
\pgfpathcurveto{\pgfqpoint{3.887973in}{1.180065in}}{\pgfqpoint{3.878297in}{1.184073in}}{\pgfqpoint{3.868210in}{1.184073in}}%
\pgfpathcurveto{\pgfqpoint{3.858122in}{1.184073in}}{\pgfqpoint{3.848447in}{1.180065in}}{\pgfqpoint{3.841314in}{1.172933in}}%
\pgfpathcurveto{\pgfqpoint{3.834181in}{1.165800in}}{\pgfqpoint{3.830174in}{1.156124in}}{\pgfqpoint{3.830174in}{1.146037in}}%
\pgfpathcurveto{\pgfqpoint{3.830174in}{1.135950in}}{\pgfqpoint{3.834181in}{1.126274in}}{\pgfqpoint{3.841314in}{1.119141in}}%
\pgfpathcurveto{\pgfqpoint{3.848447in}{1.112008in}}{\pgfqpoint{3.858122in}{1.108001in}}{\pgfqpoint{3.868210in}{1.108001in}}%
\pgfpathclose%
\pgfusepath{stroke,fill}%
\end{pgfscope}%
\begin{pgfscope}%
\pgfpathrectangle{\pgfqpoint{0.800000in}{0.528000in}}{\pgfqpoint{4.960000in}{3.696000in}} %
\pgfusepath{clip}%
\pgfsetbuttcap%
\pgfsetroundjoin%
\definecolor{currentfill}{rgb}{0.121569,0.466667,0.705882}%
\pgfsetfillcolor{currentfill}%
\pgfsetlinewidth{1.003750pt}%
\definecolor{currentstroke}{rgb}{0.121569,0.466667,0.705882}%
\pgfsetstrokecolor{currentstroke}%
\pgfsetdash{}{0pt}%
\pgfpathmoveto{\pgfqpoint{5.145334in}{1.966178in}}%
\pgfpathcurveto{\pgfqpoint{5.155422in}{1.966178in}}{\pgfqpoint{5.165097in}{1.970186in}}{\pgfqpoint{5.172230in}{1.977318in}}%
\pgfpathcurveto{\pgfqpoint{5.179363in}{1.984451in}}{\pgfqpoint{5.183371in}{1.994127in}}{\pgfqpoint{5.183371in}{2.004214in}}%
\pgfpathcurveto{\pgfqpoint{5.183371in}{2.014302in}}{\pgfqpoint{5.179363in}{2.023977in}}{\pgfqpoint{5.172230in}{2.031110in}}%
\pgfpathcurveto{\pgfqpoint{5.165097in}{2.038243in}}{\pgfqpoint{5.155422in}{2.042250in}}{\pgfqpoint{5.145334in}{2.042250in}}%
\pgfpathcurveto{\pgfqpoint{5.135247in}{2.042250in}}{\pgfqpoint{5.125572in}{2.038243in}}{\pgfqpoint{5.118439in}{2.031110in}}%
\pgfpathcurveto{\pgfqpoint{5.111306in}{2.023977in}}{\pgfqpoint{5.107298in}{2.014302in}}{\pgfqpoint{5.107298in}{2.004214in}}%
\pgfpathcurveto{\pgfqpoint{5.107298in}{1.994127in}}{\pgfqpoint{5.111306in}{1.984451in}}{\pgfqpoint{5.118439in}{1.977318in}}%
\pgfpathcurveto{\pgfqpoint{5.125572in}{1.970186in}}{\pgfqpoint{5.135247in}{1.966178in}}{\pgfqpoint{5.145334in}{1.966178in}}%
\pgfpathclose%
\pgfusepath{stroke,fill}%
\end{pgfscope}%
\begin{pgfscope}%
\pgfpathrectangle{\pgfqpoint{0.800000in}{0.528000in}}{\pgfqpoint{4.960000in}{3.696000in}} %
\pgfusepath{clip}%
\pgfsetbuttcap%
\pgfsetroundjoin%
\definecolor{currentfill}{rgb}{0.121569,0.466667,0.705882}%
\pgfsetfillcolor{currentfill}%
\pgfsetlinewidth{1.003750pt}%
\definecolor{currentstroke}{rgb}{0.121569,0.466667,0.705882}%
\pgfsetstrokecolor{currentstroke}%
\pgfsetdash{}{0pt}%
\pgfpathmoveto{\pgfqpoint{2.365342in}{3.824778in}}%
\pgfpathcurveto{\pgfqpoint{2.375429in}{3.824778in}}{\pgfqpoint{2.385105in}{3.828786in}}{\pgfqpoint{2.392238in}{3.835918in}}%
\pgfpathcurveto{\pgfqpoint{2.399371in}{3.843051in}}{\pgfqpoint{2.403378in}{3.852727in}}{\pgfqpoint{2.403378in}{3.862814in}}%
\pgfpathcurveto{\pgfqpoint{2.403378in}{3.872901in}}{\pgfqpoint{2.399371in}{3.882577in}}{\pgfqpoint{2.392238in}{3.889710in}}%
\pgfpathcurveto{\pgfqpoint{2.385105in}{3.896843in}}{\pgfqpoint{2.375429in}{3.900850in}}{\pgfqpoint{2.365342in}{3.900850in}}%
\pgfpathcurveto{\pgfqpoint{2.355255in}{3.900850in}}{\pgfqpoint{2.345579in}{3.896843in}}{\pgfqpoint{2.338446in}{3.889710in}}%
\pgfpathcurveto{\pgfqpoint{2.331313in}{3.882577in}}{\pgfqpoint{2.327306in}{3.872901in}}{\pgfqpoint{2.327306in}{3.862814in}}%
\pgfpathcurveto{\pgfqpoint{2.327306in}{3.852727in}}{\pgfqpoint{2.331313in}{3.843051in}}{\pgfqpoint{2.338446in}{3.835918in}}%
\pgfpathcurveto{\pgfqpoint{2.345579in}{3.828786in}}{\pgfqpoint{2.355255in}{3.824778in}}{\pgfqpoint{2.365342in}{3.824778in}}%
\pgfpathclose%
\pgfusepath{stroke,fill}%
\end{pgfscope}%
\begin{pgfscope}%
\pgfpathrectangle{\pgfqpoint{0.800000in}{0.528000in}}{\pgfqpoint{4.960000in}{3.696000in}} %
\pgfusepath{clip}%
\pgfsetbuttcap%
\pgfsetroundjoin%
\definecolor{currentfill}{rgb}{0.121569,0.466667,0.705882}%
\pgfsetfillcolor{currentfill}%
\pgfsetlinewidth{1.003750pt}%
\definecolor{currentstroke}{rgb}{0.121569,0.466667,0.705882}%
\pgfsetstrokecolor{currentstroke}%
\pgfsetdash{}{0pt}%
\pgfpathmoveto{\pgfqpoint{2.486487in}{1.104607in}}%
\pgfpathcurveto{\pgfqpoint{2.496574in}{1.104607in}}{\pgfqpoint{2.506250in}{1.108615in}}{\pgfqpoint{2.513382in}{1.115747in}}%
\pgfpathcurveto{\pgfqpoint{2.520515in}{1.122880in}}{\pgfqpoint{2.524523in}{1.132556in}}{\pgfqpoint{2.524523in}{1.142643in}}%
\pgfpathcurveto{\pgfqpoint{2.524523in}{1.152731in}}{\pgfqpoint{2.520515in}{1.162406in}}{\pgfqpoint{2.513382in}{1.169539in}}%
\pgfpathcurveto{\pgfqpoint{2.506250in}{1.176672in}}{\pgfqpoint{2.496574in}{1.180679in}}{\pgfqpoint{2.486487in}{1.180679in}}%
\pgfpathcurveto{\pgfqpoint{2.476399in}{1.180679in}}{\pgfqpoint{2.466724in}{1.176672in}}{\pgfqpoint{2.459591in}{1.169539in}}%
\pgfpathcurveto{\pgfqpoint{2.452458in}{1.162406in}}{\pgfqpoint{2.448450in}{1.152731in}}{\pgfqpoint{2.448450in}{1.142643in}}%
\pgfpathcurveto{\pgfqpoint{2.448450in}{1.132556in}}{\pgfqpoint{2.452458in}{1.122880in}}{\pgfqpoint{2.459591in}{1.115747in}}%
\pgfpathcurveto{\pgfqpoint{2.466724in}{1.108615in}}{\pgfqpoint{2.476399in}{1.104607in}}{\pgfqpoint{2.486487in}{1.104607in}}%
\pgfpathclose%
\pgfusepath{stroke,fill}%
\end{pgfscope}%
\begin{pgfscope}%
\pgfpathrectangle{\pgfqpoint{0.800000in}{0.528000in}}{\pgfqpoint{4.960000in}{3.696000in}} %
\pgfusepath{clip}%
\pgfsetbuttcap%
\pgfsetroundjoin%
\definecolor{currentfill}{rgb}{0.121569,0.466667,0.705882}%
\pgfsetfillcolor{currentfill}%
\pgfsetlinewidth{1.003750pt}%
\definecolor{currentstroke}{rgb}{0.121569,0.466667,0.705882}%
\pgfsetstrokecolor{currentstroke}%
\pgfsetdash{}{0pt}%
\pgfpathmoveto{\pgfqpoint{4.516479in}{1.485470in}}%
\pgfpathcurveto{\pgfqpoint{4.526566in}{1.485470in}}{\pgfqpoint{4.536242in}{1.489478in}}{\pgfqpoint{4.543375in}{1.496611in}}%
\pgfpathcurveto{\pgfqpoint{4.550507in}{1.503744in}}{\pgfqpoint{4.554515in}{1.513419in}}{\pgfqpoint{4.554515in}{1.523506in}}%
\pgfpathcurveto{\pgfqpoint{4.554515in}{1.533594in}}{\pgfqpoint{4.550507in}{1.543269in}}{\pgfqpoint{4.543375in}{1.550402in}}%
\pgfpathcurveto{\pgfqpoint{4.536242in}{1.557535in}}{\pgfqpoint{4.526566in}{1.561543in}}{\pgfqpoint{4.516479in}{1.561543in}}%
\pgfpathcurveto{\pgfqpoint{4.506392in}{1.561543in}}{\pgfqpoint{4.496716in}{1.557535in}}{\pgfqpoint{4.489583in}{1.550402in}}%
\pgfpathcurveto{\pgfqpoint{4.482450in}{1.543269in}}{\pgfqpoint{4.478443in}{1.533594in}}{\pgfqpoint{4.478443in}{1.523506in}}%
\pgfpathcurveto{\pgfqpoint{4.478443in}{1.513419in}}{\pgfqpoint{4.482450in}{1.503744in}}{\pgfqpoint{4.489583in}{1.496611in}}%
\pgfpathcurveto{\pgfqpoint{4.496716in}{1.489478in}}{\pgfqpoint{4.506392in}{1.485470in}}{\pgfqpoint{4.516479in}{1.485470in}}%
\pgfpathclose%
\pgfusepath{stroke,fill}%
\end{pgfscope}%
\begin{pgfscope}%
\pgfpathrectangle{\pgfqpoint{0.800000in}{0.528000in}}{\pgfqpoint{4.960000in}{3.696000in}} %
\pgfusepath{clip}%
\pgfsetbuttcap%
\pgfsetroundjoin%
\definecolor{currentfill}{rgb}{0.121569,0.466667,0.705882}%
\pgfsetfillcolor{currentfill}%
\pgfsetlinewidth{1.003750pt}%
\definecolor{currentstroke}{rgb}{0.121569,0.466667,0.705882}%
\pgfsetstrokecolor{currentstroke}%
\pgfsetdash{}{0pt}%
\pgfpathmoveto{\pgfqpoint{4.851717in}{1.634693in}}%
\pgfpathcurveto{\pgfqpoint{4.861805in}{1.634693in}}{\pgfqpoint{4.871480in}{1.638701in}}{\pgfqpoint{4.878613in}{1.645834in}}%
\pgfpathcurveto{\pgfqpoint{4.885746in}{1.652967in}}{\pgfqpoint{4.889754in}{1.662642in}}{\pgfqpoint{4.889754in}{1.672729in}}%
\pgfpathcurveto{\pgfqpoint{4.889754in}{1.682817in}}{\pgfqpoint{4.885746in}{1.692492in}}{\pgfqpoint{4.878613in}{1.699625in}}%
\pgfpathcurveto{\pgfqpoint{4.871480in}{1.706758in}}{\pgfqpoint{4.861805in}{1.710766in}}{\pgfqpoint{4.851717in}{1.710766in}}%
\pgfpathcurveto{\pgfqpoint{4.841630in}{1.710766in}}{\pgfqpoint{4.831955in}{1.706758in}}{\pgfqpoint{4.824822in}{1.699625in}}%
\pgfpathcurveto{\pgfqpoint{4.817689in}{1.692492in}}{\pgfqpoint{4.813681in}{1.682817in}}{\pgfqpoint{4.813681in}{1.672729in}}%
\pgfpathcurveto{\pgfqpoint{4.813681in}{1.662642in}}{\pgfqpoint{4.817689in}{1.652967in}}{\pgfqpoint{4.824822in}{1.645834in}}%
\pgfpathcurveto{\pgfqpoint{4.831955in}{1.638701in}}{\pgfqpoint{4.841630in}{1.634693in}}{\pgfqpoint{4.851717in}{1.634693in}}%
\pgfpathclose%
\pgfusepath{stroke,fill}%
\end{pgfscope}%
\begin{pgfscope}%
\pgfpathrectangle{\pgfqpoint{0.800000in}{0.528000in}}{\pgfqpoint{4.960000in}{3.696000in}} %
\pgfusepath{clip}%
\pgfsetbuttcap%
\pgfsetroundjoin%
\definecolor{currentfill}{rgb}{0.121569,0.466667,0.705882}%
\pgfsetfillcolor{currentfill}%
\pgfsetlinewidth{1.003750pt}%
\definecolor{currentstroke}{rgb}{0.121569,0.466667,0.705882}%
\pgfsetstrokecolor{currentstroke}%
\pgfsetdash{}{0pt}%
\pgfpathmoveto{\pgfqpoint{4.833796in}{3.121590in}}%
\pgfpathcurveto{\pgfqpoint{4.843883in}{3.121590in}}{\pgfqpoint{4.853559in}{3.125598in}}{\pgfqpoint{4.860692in}{3.132731in}}%
\pgfpathcurveto{\pgfqpoint{4.867825in}{3.139864in}}{\pgfqpoint{4.871832in}{3.149539in}}{\pgfqpoint{4.871832in}{3.159627in}}%
\pgfpathcurveto{\pgfqpoint{4.871832in}{3.169714in}}{\pgfqpoint{4.867825in}{3.179390in}}{\pgfqpoint{4.860692in}{3.186522in}}%
\pgfpathcurveto{\pgfqpoint{4.853559in}{3.193655in}}{\pgfqpoint{4.843883in}{3.197663in}}{\pgfqpoint{4.833796in}{3.197663in}}%
\pgfpathcurveto{\pgfqpoint{4.823709in}{3.197663in}}{\pgfqpoint{4.814033in}{3.193655in}}{\pgfqpoint{4.806900in}{3.186522in}}%
\pgfpathcurveto{\pgfqpoint{4.799767in}{3.179390in}}{\pgfqpoint{4.795760in}{3.169714in}}{\pgfqpoint{4.795760in}{3.159627in}}%
\pgfpathcurveto{\pgfqpoint{4.795760in}{3.149539in}}{\pgfqpoint{4.799767in}{3.139864in}}{\pgfqpoint{4.806900in}{3.132731in}}%
\pgfpathcurveto{\pgfqpoint{4.814033in}{3.125598in}}{\pgfqpoint{4.823709in}{3.121590in}}{\pgfqpoint{4.833796in}{3.121590in}}%
\pgfpathclose%
\pgfusepath{stroke,fill}%
\end{pgfscope}%
\begin{pgfscope}%
\pgfpathrectangle{\pgfqpoint{0.800000in}{0.528000in}}{\pgfqpoint{4.960000in}{3.696000in}} %
\pgfusepath{clip}%
\pgfsetbuttcap%
\pgfsetroundjoin%
\definecolor{currentfill}{rgb}{0.121569,0.466667,0.705882}%
\pgfsetfillcolor{currentfill}%
\pgfsetlinewidth{1.003750pt}%
\definecolor{currentstroke}{rgb}{0.121569,0.466667,0.705882}%
\pgfsetstrokecolor{currentstroke}%
\pgfsetdash{}{0pt}%
\pgfpathmoveto{\pgfqpoint{5.286477in}{2.860080in}}%
\pgfpathcurveto{\pgfqpoint{5.296565in}{2.860080in}}{\pgfqpoint{5.306240in}{2.864088in}}{\pgfqpoint{5.313373in}{2.871221in}}%
\pgfpathcurveto{\pgfqpoint{5.320506in}{2.878354in}}{\pgfqpoint{5.324514in}{2.888029in}}{\pgfqpoint{5.324514in}{2.898116in}}%
\pgfpathcurveto{\pgfqpoint{5.324514in}{2.908204in}}{\pgfqpoint{5.320506in}{2.917879in}}{\pgfqpoint{5.313373in}{2.925012in}}%
\pgfpathcurveto{\pgfqpoint{5.306240in}{2.932145in}}{\pgfqpoint{5.296565in}{2.936153in}}{\pgfqpoint{5.286477in}{2.936153in}}%
\pgfpathcurveto{\pgfqpoint{5.276390in}{2.936153in}}{\pgfqpoint{5.266714in}{2.932145in}}{\pgfqpoint{5.259582in}{2.925012in}}%
\pgfpathcurveto{\pgfqpoint{5.252449in}{2.917879in}}{\pgfqpoint{5.248441in}{2.908204in}}{\pgfqpoint{5.248441in}{2.898116in}}%
\pgfpathcurveto{\pgfqpoint{5.248441in}{2.888029in}}{\pgfqpoint{5.252449in}{2.878354in}}{\pgfqpoint{5.259582in}{2.871221in}}%
\pgfpathcurveto{\pgfqpoint{5.266714in}{2.864088in}}{\pgfqpoint{5.276390in}{2.860080in}}{\pgfqpoint{5.286477in}{2.860080in}}%
\pgfpathclose%
\pgfusepath{stroke,fill}%
\end{pgfscope}%
\begin{pgfscope}%
\pgfpathrectangle{\pgfqpoint{0.800000in}{0.528000in}}{\pgfqpoint{4.960000in}{3.696000in}} %
\pgfusepath{clip}%
\pgfsetbuttcap%
\pgfsetroundjoin%
\definecolor{currentfill}{rgb}{0.121569,0.466667,0.705882}%
\pgfsetfillcolor{currentfill}%
\pgfsetlinewidth{1.003750pt}%
\definecolor{currentstroke}{rgb}{0.121569,0.466667,0.705882}%
\pgfsetstrokecolor{currentstroke}%
\pgfsetdash{}{0pt}%
\pgfpathmoveto{\pgfqpoint{4.201906in}{3.638249in}}%
\pgfpathcurveto{\pgfqpoint{4.211994in}{3.638249in}}{\pgfqpoint{4.221669in}{3.642257in}}{\pgfqpoint{4.228802in}{3.649390in}}%
\pgfpathcurveto{\pgfqpoint{4.235935in}{3.656523in}}{\pgfqpoint{4.239943in}{3.666198in}}{\pgfqpoint{4.239943in}{3.676286in}}%
\pgfpathcurveto{\pgfqpoint{4.239943in}{3.686373in}}{\pgfqpoint{4.235935in}{3.696049in}}{\pgfqpoint{4.228802in}{3.703181in}}%
\pgfpathcurveto{\pgfqpoint{4.221669in}{3.710314in}}{\pgfqpoint{4.211994in}{3.714322in}}{\pgfqpoint{4.201906in}{3.714322in}}%
\pgfpathcurveto{\pgfqpoint{4.191819in}{3.714322in}}{\pgfqpoint{4.182144in}{3.710314in}}{\pgfqpoint{4.175011in}{3.703181in}}%
\pgfpathcurveto{\pgfqpoint{4.167878in}{3.696049in}}{\pgfqpoint{4.163870in}{3.686373in}}{\pgfqpoint{4.163870in}{3.676286in}}%
\pgfpathcurveto{\pgfqpoint{4.163870in}{3.666198in}}{\pgfqpoint{4.167878in}{3.656523in}}{\pgfqpoint{4.175011in}{3.649390in}}%
\pgfpathcurveto{\pgfqpoint{4.182144in}{3.642257in}}{\pgfqpoint{4.191819in}{3.638249in}}{\pgfqpoint{4.201906in}{3.638249in}}%
\pgfpathclose%
\pgfusepath{stroke,fill}%
\end{pgfscope}%
\begin{pgfscope}%
\pgfpathrectangle{\pgfqpoint{0.800000in}{0.528000in}}{\pgfqpoint{4.960000in}{3.696000in}} %
\pgfusepath{clip}%
\pgfsetbuttcap%
\pgfsetroundjoin%
\definecolor{currentfill}{rgb}{0.121569,0.466667,0.705882}%
\pgfsetfillcolor{currentfill}%
\pgfsetlinewidth{1.003750pt}%
\definecolor{currentstroke}{rgb}{0.121569,0.466667,0.705882}%
\pgfsetstrokecolor{currentstroke}%
\pgfsetdash{}{0pt}%
\pgfpathmoveto{\pgfqpoint{4.702390in}{1.614654in}}%
\pgfpathcurveto{\pgfqpoint{4.712477in}{1.614654in}}{\pgfqpoint{4.722153in}{1.618662in}}{\pgfqpoint{4.729286in}{1.625795in}}%
\pgfpathcurveto{\pgfqpoint{4.736418in}{1.632928in}}{\pgfqpoint{4.740426in}{1.642603in}}{\pgfqpoint{4.740426in}{1.652691in}}%
\pgfpathcurveto{\pgfqpoint{4.740426in}{1.662778in}}{\pgfqpoint{4.736418in}{1.672454in}}{\pgfqpoint{4.729286in}{1.679586in}}%
\pgfpathcurveto{\pgfqpoint{4.722153in}{1.686719in}}{\pgfqpoint{4.712477in}{1.690727in}}{\pgfqpoint{4.702390in}{1.690727in}}%
\pgfpathcurveto{\pgfqpoint{4.692303in}{1.690727in}}{\pgfqpoint{4.682627in}{1.686719in}}{\pgfqpoint{4.675494in}{1.679586in}}%
\pgfpathcurveto{\pgfqpoint{4.668361in}{1.672454in}}{\pgfqpoint{4.664354in}{1.662778in}}{\pgfqpoint{4.664354in}{1.652691in}}%
\pgfpathcurveto{\pgfqpoint{4.664354in}{1.642603in}}{\pgfqpoint{4.668361in}{1.632928in}}{\pgfqpoint{4.675494in}{1.625795in}}%
\pgfpathcurveto{\pgfqpoint{4.682627in}{1.618662in}}{\pgfqpoint{4.692303in}{1.614654in}}{\pgfqpoint{4.702390in}{1.614654in}}%
\pgfpathclose%
\pgfusepath{stroke,fill}%
\end{pgfscope}%
\begin{pgfscope}%
\pgfpathrectangle{\pgfqpoint{0.800000in}{0.528000in}}{\pgfqpoint{4.960000in}{3.696000in}} %
\pgfusepath{clip}%
\pgfsetbuttcap%
\pgfsetroundjoin%
\definecolor{currentfill}{rgb}{0.121569,0.466667,0.705882}%
\pgfsetfillcolor{currentfill}%
\pgfsetlinewidth{1.003750pt}%
\definecolor{currentstroke}{rgb}{0.121569,0.466667,0.705882}%
\pgfsetstrokecolor{currentstroke}%
\pgfsetdash{}{0pt}%
\pgfpathmoveto{\pgfqpoint{4.652188in}{3.127749in}}%
\pgfpathcurveto{\pgfqpoint{4.662276in}{3.127749in}}{\pgfqpoint{4.671951in}{3.131757in}}{\pgfqpoint{4.679084in}{3.138890in}}%
\pgfpathcurveto{\pgfqpoint{4.686217in}{3.146023in}}{\pgfqpoint{4.690225in}{3.155698in}}{\pgfqpoint{4.690225in}{3.165786in}}%
\pgfpathcurveto{\pgfqpoint{4.690225in}{3.175873in}}{\pgfqpoint{4.686217in}{3.185548in}}{\pgfqpoint{4.679084in}{3.192681in}}%
\pgfpathcurveto{\pgfqpoint{4.671951in}{3.199814in}}{\pgfqpoint{4.662276in}{3.203822in}}{\pgfqpoint{4.652188in}{3.203822in}}%
\pgfpathcurveto{\pgfqpoint{4.642101in}{3.203822in}}{\pgfqpoint{4.632425in}{3.199814in}}{\pgfqpoint{4.625293in}{3.192681in}}%
\pgfpathcurveto{\pgfqpoint{4.618160in}{3.185548in}}{\pgfqpoint{4.614152in}{3.175873in}}{\pgfqpoint{4.614152in}{3.165786in}}%
\pgfpathcurveto{\pgfqpoint{4.614152in}{3.155698in}}{\pgfqpoint{4.618160in}{3.146023in}}{\pgfqpoint{4.625293in}{3.138890in}}%
\pgfpathcurveto{\pgfqpoint{4.632425in}{3.131757in}}{\pgfqpoint{4.642101in}{3.127749in}}{\pgfqpoint{4.652188in}{3.127749in}}%
\pgfpathclose%
\pgfusepath{stroke,fill}%
\end{pgfscope}%
\begin{pgfscope}%
\pgfpathrectangle{\pgfqpoint{0.800000in}{0.528000in}}{\pgfqpoint{4.960000in}{3.696000in}} %
\pgfusepath{clip}%
\pgfsetbuttcap%
\pgfsetroundjoin%
\definecolor{currentfill}{rgb}{0.121569,0.466667,0.705882}%
\pgfsetfillcolor{currentfill}%
\pgfsetlinewidth{1.003750pt}%
\definecolor{currentstroke}{rgb}{0.121569,0.466667,0.705882}%
\pgfsetstrokecolor{currentstroke}%
\pgfsetdash{}{0pt}%
\pgfpathmoveto{\pgfqpoint{1.459949in}{2.932749in}}%
\pgfpathcurveto{\pgfqpoint{1.470037in}{2.932749in}}{\pgfqpoint{1.479712in}{2.936757in}}{\pgfqpoint{1.486845in}{2.943889in}}%
\pgfpathcurveto{\pgfqpoint{1.493978in}{2.951022in}}{\pgfqpoint{1.497986in}{2.960698in}}{\pgfqpoint{1.497986in}{2.970785in}}%
\pgfpathcurveto{\pgfqpoint{1.497986in}{2.980872in}}{\pgfqpoint{1.493978in}{2.990548in}}{\pgfqpoint{1.486845in}{2.997681in}}%
\pgfpathcurveto{\pgfqpoint{1.479712in}{3.004814in}}{\pgfqpoint{1.470037in}{3.008821in}}{\pgfqpoint{1.459949in}{3.008821in}}%
\pgfpathcurveto{\pgfqpoint{1.449862in}{3.008821in}}{\pgfqpoint{1.440186in}{3.004814in}}{\pgfqpoint{1.433054in}{2.997681in}}%
\pgfpathcurveto{\pgfqpoint{1.425921in}{2.990548in}}{\pgfqpoint{1.421913in}{2.980872in}}{\pgfqpoint{1.421913in}{2.970785in}}%
\pgfpathcurveto{\pgfqpoint{1.421913in}{2.960698in}}{\pgfqpoint{1.425921in}{2.951022in}}{\pgfqpoint{1.433054in}{2.943889in}}%
\pgfpathcurveto{\pgfqpoint{1.440186in}{2.936757in}}{\pgfqpoint{1.449862in}{2.932749in}}{\pgfqpoint{1.459949in}{2.932749in}}%
\pgfpathclose%
\pgfusepath{stroke,fill}%
\end{pgfscope}%
\begin{pgfscope}%
\pgfpathrectangle{\pgfqpoint{0.800000in}{0.528000in}}{\pgfqpoint{4.960000in}{3.696000in}} %
\pgfusepath{clip}%
\pgfsetbuttcap%
\pgfsetroundjoin%
\definecolor{currentfill}{rgb}{0.121569,0.466667,0.705882}%
\pgfsetfillcolor{currentfill}%
\pgfsetlinewidth{1.003750pt}%
\definecolor{currentstroke}{rgb}{0.121569,0.466667,0.705882}%
\pgfsetstrokecolor{currentstroke}%
\pgfsetdash{}{0pt}%
\pgfpathmoveto{\pgfqpoint{5.208798in}{1.871519in}}%
\pgfpathcurveto{\pgfqpoint{5.218886in}{1.871519in}}{\pgfqpoint{5.228561in}{1.875527in}}{\pgfqpoint{5.235694in}{1.882660in}}%
\pgfpathcurveto{\pgfqpoint{5.242827in}{1.889793in}}{\pgfqpoint{5.246835in}{1.899468in}}{\pgfqpoint{5.246835in}{1.909555in}}%
\pgfpathcurveto{\pgfqpoint{5.246835in}{1.919643in}}{\pgfqpoint{5.242827in}{1.929318in}}{\pgfqpoint{5.235694in}{1.936451in}}%
\pgfpathcurveto{\pgfqpoint{5.228561in}{1.943584in}}{\pgfqpoint{5.218886in}{1.947592in}}{\pgfqpoint{5.208798in}{1.947592in}}%
\pgfpathcurveto{\pgfqpoint{5.198711in}{1.947592in}}{\pgfqpoint{5.189036in}{1.943584in}}{\pgfqpoint{5.181903in}{1.936451in}}%
\pgfpathcurveto{\pgfqpoint{5.174770in}{1.929318in}}{\pgfqpoint{5.170762in}{1.919643in}}{\pgfqpoint{5.170762in}{1.909555in}}%
\pgfpathcurveto{\pgfqpoint{5.170762in}{1.899468in}}{\pgfqpoint{5.174770in}{1.889793in}}{\pgfqpoint{5.181903in}{1.882660in}}%
\pgfpathcurveto{\pgfqpoint{5.189036in}{1.875527in}}{\pgfqpoint{5.198711in}{1.871519in}}{\pgfqpoint{5.208798in}{1.871519in}}%
\pgfpathclose%
\pgfusepath{stroke,fill}%
\end{pgfscope}%
\begin{pgfscope}%
\pgfpathrectangle{\pgfqpoint{0.800000in}{0.528000in}}{\pgfqpoint{4.960000in}{3.696000in}} %
\pgfusepath{clip}%
\pgfsetbuttcap%
\pgfsetroundjoin%
\definecolor{currentfill}{rgb}{0.121569,0.466667,0.705882}%
\pgfsetfillcolor{currentfill}%
\pgfsetlinewidth{1.003750pt}%
\definecolor{currentstroke}{rgb}{0.121569,0.466667,0.705882}%
\pgfsetstrokecolor{currentstroke}%
\pgfsetdash{}{0pt}%
\pgfpathmoveto{\pgfqpoint{1.105706in}{2.029216in}}%
\pgfpathcurveto{\pgfqpoint{1.115793in}{2.029216in}}{\pgfqpoint{1.125469in}{2.033224in}}{\pgfqpoint{1.132602in}{2.040357in}}%
\pgfpathcurveto{\pgfqpoint{1.139735in}{2.047490in}}{\pgfqpoint{1.143742in}{2.057165in}}{\pgfqpoint{1.143742in}{2.067252in}}%
\pgfpathcurveto{\pgfqpoint{1.143742in}{2.077340in}}{\pgfqpoint{1.139735in}{2.087015in}}{\pgfqpoint{1.132602in}{2.094148in}}%
\pgfpathcurveto{\pgfqpoint{1.125469in}{2.101281in}}{\pgfqpoint{1.115793in}{2.105289in}}{\pgfqpoint{1.105706in}{2.105289in}}%
\pgfpathcurveto{\pgfqpoint{1.095619in}{2.105289in}}{\pgfqpoint{1.085943in}{2.101281in}}{\pgfqpoint{1.078810in}{2.094148in}}%
\pgfpathcurveto{\pgfqpoint{1.071677in}{2.087015in}}{\pgfqpoint{1.067670in}{2.077340in}}{\pgfqpoint{1.067670in}{2.067252in}}%
\pgfpathcurveto{\pgfqpoint{1.067670in}{2.057165in}}{\pgfqpoint{1.071677in}{2.047490in}}{\pgfqpoint{1.078810in}{2.040357in}}%
\pgfpathcurveto{\pgfqpoint{1.085943in}{2.033224in}}{\pgfqpoint{1.095619in}{2.029216in}}{\pgfqpoint{1.105706in}{2.029216in}}%
\pgfpathclose%
\pgfusepath{stroke,fill}%
\end{pgfscope}%
\begin{pgfscope}%
\pgfpathrectangle{\pgfqpoint{0.800000in}{0.528000in}}{\pgfqpoint{4.960000in}{3.696000in}} %
\pgfusepath{clip}%
\pgfsetbuttcap%
\pgfsetroundjoin%
\definecolor{currentfill}{rgb}{0.121569,0.466667,0.705882}%
\pgfsetfillcolor{currentfill}%
\pgfsetlinewidth{1.003750pt}%
\definecolor{currentstroke}{rgb}{0.121569,0.466667,0.705882}%
\pgfsetstrokecolor{currentstroke}%
\pgfsetdash{}{0pt}%
\pgfpathmoveto{\pgfqpoint{2.437721in}{0.947679in}}%
\pgfpathcurveto{\pgfqpoint{2.447809in}{0.947679in}}{\pgfqpoint{2.457484in}{0.951687in}}{\pgfqpoint{2.464617in}{0.958820in}}%
\pgfpathcurveto{\pgfqpoint{2.471750in}{0.965953in}}{\pgfqpoint{2.475758in}{0.975628in}}{\pgfqpoint{2.475758in}{0.985715in}}%
\pgfpathcurveto{\pgfqpoint{2.475758in}{0.995803in}}{\pgfqpoint{2.471750in}{1.005478in}}{\pgfqpoint{2.464617in}{1.012611in}}%
\pgfpathcurveto{\pgfqpoint{2.457484in}{1.019744in}}{\pgfqpoint{2.447809in}{1.023752in}}{\pgfqpoint{2.437721in}{1.023752in}}%
\pgfpathcurveto{\pgfqpoint{2.427634in}{1.023752in}}{\pgfqpoint{2.417958in}{1.019744in}}{\pgfqpoint{2.410826in}{1.012611in}}%
\pgfpathcurveto{\pgfqpoint{2.403693in}{1.005478in}}{\pgfqpoint{2.399685in}{0.995803in}}{\pgfqpoint{2.399685in}{0.985715in}}%
\pgfpathcurveto{\pgfqpoint{2.399685in}{0.975628in}}{\pgfqpoint{2.403693in}{0.965953in}}{\pgfqpoint{2.410826in}{0.958820in}}%
\pgfpathcurveto{\pgfqpoint{2.417958in}{0.951687in}}{\pgfqpoint{2.427634in}{0.947679in}}{\pgfqpoint{2.437721in}{0.947679in}}%
\pgfpathclose%
\pgfusepath{stroke,fill}%
\end{pgfscope}%
\begin{pgfscope}%
\pgfpathrectangle{\pgfqpoint{0.800000in}{0.528000in}}{\pgfqpoint{4.960000in}{3.696000in}} %
\pgfusepath{clip}%
\pgfsetbuttcap%
\pgfsetroundjoin%
\definecolor{currentfill}{rgb}{0.121569,0.466667,0.705882}%
\pgfsetfillcolor{currentfill}%
\pgfsetlinewidth{1.003750pt}%
\definecolor{currentstroke}{rgb}{0.121569,0.466667,0.705882}%
\pgfsetstrokecolor{currentstroke}%
\pgfsetdash{}{0pt}%
\pgfpathmoveto{\pgfqpoint{2.394261in}{3.956521in}}%
\pgfpathcurveto{\pgfqpoint{2.404348in}{3.956521in}}{\pgfqpoint{2.414024in}{3.960529in}}{\pgfqpoint{2.421157in}{3.967662in}}%
\pgfpathcurveto{\pgfqpoint{2.428290in}{3.974794in}}{\pgfqpoint{2.432297in}{3.984470in}}{\pgfqpoint{2.432297in}{3.994557in}}%
\pgfpathcurveto{\pgfqpoint{2.432297in}{4.004645in}}{\pgfqpoint{2.428290in}{4.014320in}}{\pgfqpoint{2.421157in}{4.021453in}}%
\pgfpathcurveto{\pgfqpoint{2.414024in}{4.028586in}}{\pgfqpoint{2.404348in}{4.032594in}}{\pgfqpoint{2.394261in}{4.032594in}}%
\pgfpathcurveto{\pgfqpoint{2.384174in}{4.032594in}}{\pgfqpoint{2.374498in}{4.028586in}}{\pgfqpoint{2.367365in}{4.021453in}}%
\pgfpathcurveto{\pgfqpoint{2.360233in}{4.014320in}}{\pgfqpoint{2.356225in}{4.004645in}}{\pgfqpoint{2.356225in}{3.994557in}}%
\pgfpathcurveto{\pgfqpoint{2.356225in}{3.984470in}}{\pgfqpoint{2.360233in}{3.974794in}}{\pgfqpoint{2.367365in}{3.967662in}}%
\pgfpathcurveto{\pgfqpoint{2.374498in}{3.960529in}}{\pgfqpoint{2.384174in}{3.956521in}}{\pgfqpoint{2.394261in}{3.956521in}}%
\pgfpathclose%
\pgfusepath{stroke,fill}%
\end{pgfscope}%
\begin{pgfscope}%
\pgfpathrectangle{\pgfqpoint{0.800000in}{0.528000in}}{\pgfqpoint{4.960000in}{3.696000in}} %
\pgfusepath{clip}%
\pgfsetbuttcap%
\pgfsetroundjoin%
\definecolor{currentfill}{rgb}{0.121569,0.466667,0.705882}%
\pgfsetfillcolor{currentfill}%
\pgfsetlinewidth{1.003750pt}%
\definecolor{currentstroke}{rgb}{0.121569,0.466667,0.705882}%
\pgfsetstrokecolor{currentstroke}%
\pgfsetdash{}{0pt}%
\pgfpathmoveto{\pgfqpoint{2.219754in}{1.047550in}}%
\pgfpathcurveto{\pgfqpoint{2.229841in}{1.047550in}}{\pgfqpoint{2.239517in}{1.051558in}}{\pgfqpoint{2.246650in}{1.058690in}}%
\pgfpathcurveto{\pgfqpoint{2.253782in}{1.065823in}}{\pgfqpoint{2.257790in}{1.075499in}}{\pgfqpoint{2.257790in}{1.085586in}}%
\pgfpathcurveto{\pgfqpoint{2.257790in}{1.095674in}}{\pgfqpoint{2.253782in}{1.105349in}}{\pgfqpoint{2.246650in}{1.112482in}}%
\pgfpathcurveto{\pgfqpoint{2.239517in}{1.119615in}}{\pgfqpoint{2.229841in}{1.123623in}}{\pgfqpoint{2.219754in}{1.123623in}}%
\pgfpathcurveto{\pgfqpoint{2.209667in}{1.123623in}}{\pgfqpoint{2.199991in}{1.119615in}}{\pgfqpoint{2.192858in}{1.112482in}}%
\pgfpathcurveto{\pgfqpoint{2.185725in}{1.105349in}}{\pgfqpoint{2.181718in}{1.095674in}}{\pgfqpoint{2.181718in}{1.085586in}}%
\pgfpathcurveto{\pgfqpoint{2.181718in}{1.075499in}}{\pgfqpoint{2.185725in}{1.065823in}}{\pgfqpoint{2.192858in}{1.058690in}}%
\pgfpathcurveto{\pgfqpoint{2.199991in}{1.051558in}}{\pgfqpoint{2.209667in}{1.047550in}}{\pgfqpoint{2.219754in}{1.047550in}}%
\pgfpathclose%
\pgfusepath{stroke,fill}%
\end{pgfscope}%
\begin{pgfscope}%
\pgfpathrectangle{\pgfqpoint{0.800000in}{0.528000in}}{\pgfqpoint{4.960000in}{3.696000in}} %
\pgfusepath{clip}%
\pgfsetbuttcap%
\pgfsetroundjoin%
\definecolor{currentfill}{rgb}{0.121569,0.466667,0.705882}%
\pgfsetfillcolor{currentfill}%
\pgfsetlinewidth{1.003750pt}%
\definecolor{currentstroke}{rgb}{0.121569,0.466667,0.705882}%
\pgfsetstrokecolor{currentstroke}%
\pgfsetdash{}{0pt}%
\pgfpathmoveto{\pgfqpoint{4.237864in}{1.029607in}}%
\pgfpathcurveto{\pgfqpoint{4.247951in}{1.029607in}}{\pgfqpoint{4.257627in}{1.033614in}}{\pgfqpoint{4.264760in}{1.040747in}}%
\pgfpathcurveto{\pgfqpoint{4.271893in}{1.047880in}}{\pgfqpoint{4.275900in}{1.057556in}}{\pgfqpoint{4.275900in}{1.067643in}}%
\pgfpathcurveto{\pgfqpoint{4.275900in}{1.077730in}}{\pgfqpoint{4.271893in}{1.087406in}}{\pgfqpoint{4.264760in}{1.094539in}}%
\pgfpathcurveto{\pgfqpoint{4.257627in}{1.101672in}}{\pgfqpoint{4.247951in}{1.105679in}}{\pgfqpoint{4.237864in}{1.105679in}}%
\pgfpathcurveto{\pgfqpoint{4.227777in}{1.105679in}}{\pgfqpoint{4.218101in}{1.101672in}}{\pgfqpoint{4.210968in}{1.094539in}}%
\pgfpathcurveto{\pgfqpoint{4.203835in}{1.087406in}}{\pgfqpoint{4.199828in}{1.077730in}}{\pgfqpoint{4.199828in}{1.067643in}}%
\pgfpathcurveto{\pgfqpoint{4.199828in}{1.057556in}}{\pgfqpoint{4.203835in}{1.047880in}}{\pgfqpoint{4.210968in}{1.040747in}}%
\pgfpathcurveto{\pgfqpoint{4.218101in}{1.033614in}}{\pgfqpoint{4.227777in}{1.029607in}}{\pgfqpoint{4.237864in}{1.029607in}}%
\pgfpathclose%
\pgfusepath{stroke,fill}%
\end{pgfscope}%
\begin{pgfscope}%
\pgfpathrectangle{\pgfqpoint{0.800000in}{0.528000in}}{\pgfqpoint{4.960000in}{3.696000in}} %
\pgfusepath{clip}%
\pgfsetbuttcap%
\pgfsetroundjoin%
\definecolor{currentfill}{rgb}{0.121569,0.466667,0.705882}%
\pgfsetfillcolor{currentfill}%
\pgfsetlinewidth{1.003750pt}%
\definecolor{currentstroke}{rgb}{0.121569,0.466667,0.705882}%
\pgfsetstrokecolor{currentstroke}%
\pgfsetdash{}{0pt}%
\pgfpathmoveto{\pgfqpoint{5.279734in}{2.703315in}}%
\pgfpathcurveto{\pgfqpoint{5.289822in}{2.703315in}}{\pgfqpoint{5.299497in}{2.707322in}}{\pgfqpoint{5.306630in}{2.714455in}}%
\pgfpathcurveto{\pgfqpoint{5.313763in}{2.721588in}}{\pgfqpoint{5.317770in}{2.731263in}}{\pgfqpoint{5.317770in}{2.741351in}}%
\pgfpathcurveto{\pgfqpoint{5.317770in}{2.751438in}}{\pgfqpoint{5.313763in}{2.761114in}}{\pgfqpoint{5.306630in}{2.768247in}}%
\pgfpathcurveto{\pgfqpoint{5.299497in}{2.775379in}}{\pgfqpoint{5.289822in}{2.779387in}}{\pgfqpoint{5.279734in}{2.779387in}}%
\pgfpathcurveto{\pgfqpoint{5.269647in}{2.779387in}}{\pgfqpoint{5.259971in}{2.775379in}}{\pgfqpoint{5.252838in}{2.768247in}}%
\pgfpathcurveto{\pgfqpoint{5.245706in}{2.761114in}}{\pgfqpoint{5.241698in}{2.751438in}}{\pgfqpoint{5.241698in}{2.741351in}}%
\pgfpathcurveto{\pgfqpoint{5.241698in}{2.731263in}}{\pgfqpoint{5.245706in}{2.721588in}}{\pgfqpoint{5.252838in}{2.714455in}}%
\pgfpathcurveto{\pgfqpoint{5.259971in}{2.707322in}}{\pgfqpoint{5.269647in}{2.703315in}}{\pgfqpoint{5.279734in}{2.703315in}}%
\pgfpathclose%
\pgfusepath{stroke,fill}%
\end{pgfscope}%
\begin{pgfscope}%
\pgfpathrectangle{\pgfqpoint{0.800000in}{0.528000in}}{\pgfqpoint{4.960000in}{3.696000in}} %
\pgfusepath{clip}%
\pgfsetbuttcap%
\pgfsetroundjoin%
\definecolor{currentfill}{rgb}{0.121569,0.466667,0.705882}%
\pgfsetfillcolor{currentfill}%
\pgfsetlinewidth{1.003750pt}%
\definecolor{currentstroke}{rgb}{0.121569,0.466667,0.705882}%
\pgfsetstrokecolor{currentstroke}%
\pgfsetdash{}{0pt}%
\pgfpathmoveto{\pgfqpoint{3.066274in}{0.906523in}}%
\pgfpathcurveto{\pgfqpoint{3.076361in}{0.906523in}}{\pgfqpoint{3.086037in}{0.910531in}}{\pgfqpoint{3.093170in}{0.917664in}}%
\pgfpathcurveto{\pgfqpoint{3.100302in}{0.924797in}}{\pgfqpoint{3.104310in}{0.934472in}}{\pgfqpoint{3.104310in}{0.944560in}}%
\pgfpathcurveto{\pgfqpoint{3.104310in}{0.954647in}}{\pgfqpoint{3.100302in}{0.964323in}}{\pgfqpoint{3.093170in}{0.971455in}}%
\pgfpathcurveto{\pgfqpoint{3.086037in}{0.978588in}}{\pgfqpoint{3.076361in}{0.982596in}}{\pgfqpoint{3.066274in}{0.982596in}}%
\pgfpathcurveto{\pgfqpoint{3.056187in}{0.982596in}}{\pgfqpoint{3.046511in}{0.978588in}}{\pgfqpoint{3.039378in}{0.971455in}}%
\pgfpathcurveto{\pgfqpoint{3.032245in}{0.964323in}}{\pgfqpoint{3.028238in}{0.954647in}}{\pgfqpoint{3.028238in}{0.944560in}}%
\pgfpathcurveto{\pgfqpoint{3.028238in}{0.934472in}}{\pgfqpoint{3.032245in}{0.924797in}}{\pgfqpoint{3.039378in}{0.917664in}}%
\pgfpathcurveto{\pgfqpoint{3.046511in}{0.910531in}}{\pgfqpoint{3.056187in}{0.906523in}}{\pgfqpoint{3.066274in}{0.906523in}}%
\pgfpathclose%
\pgfusepath{stroke,fill}%
\end{pgfscope}%
\begin{pgfscope}%
\pgfpathrectangle{\pgfqpoint{0.800000in}{0.528000in}}{\pgfqpoint{4.960000in}{3.696000in}} %
\pgfusepath{clip}%
\pgfsetbuttcap%
\pgfsetroundjoin%
\definecolor{currentfill}{rgb}{0.121569,0.466667,0.705882}%
\pgfsetfillcolor{currentfill}%
\pgfsetlinewidth{1.003750pt}%
\definecolor{currentstroke}{rgb}{0.121569,0.466667,0.705882}%
\pgfsetstrokecolor{currentstroke}%
\pgfsetdash{}{0pt}%
\pgfpathmoveto{\pgfqpoint{5.218291in}{2.192844in}}%
\pgfpathcurveto{\pgfqpoint{5.228378in}{2.192844in}}{\pgfqpoint{5.238054in}{2.196852in}}{\pgfqpoint{5.245187in}{2.203985in}}%
\pgfpathcurveto{\pgfqpoint{5.252320in}{2.211118in}}{\pgfqpoint{5.256327in}{2.220793in}}{\pgfqpoint{5.256327in}{2.230881in}}%
\pgfpathcurveto{\pgfqpoint{5.256327in}{2.240968in}}{\pgfqpoint{5.252320in}{2.250644in}}{\pgfqpoint{5.245187in}{2.257776in}}%
\pgfpathcurveto{\pgfqpoint{5.238054in}{2.264909in}}{\pgfqpoint{5.228378in}{2.268917in}}{\pgfqpoint{5.218291in}{2.268917in}}%
\pgfpathcurveto{\pgfqpoint{5.208204in}{2.268917in}}{\pgfqpoint{5.198528in}{2.264909in}}{\pgfqpoint{5.191395in}{2.257776in}}%
\pgfpathcurveto{\pgfqpoint{5.184262in}{2.250644in}}{\pgfqpoint{5.180255in}{2.240968in}}{\pgfqpoint{5.180255in}{2.230881in}}%
\pgfpathcurveto{\pgfqpoint{5.180255in}{2.220793in}}{\pgfqpoint{5.184262in}{2.211118in}}{\pgfqpoint{5.191395in}{2.203985in}}%
\pgfpathcurveto{\pgfqpoint{5.198528in}{2.196852in}}{\pgfqpoint{5.208204in}{2.192844in}}{\pgfqpoint{5.218291in}{2.192844in}}%
\pgfpathclose%
\pgfusepath{stroke,fill}%
\end{pgfscope}%
\begin{pgfscope}%
\pgfpathrectangle{\pgfqpoint{0.800000in}{0.528000in}}{\pgfqpoint{4.960000in}{3.696000in}} %
\pgfusepath{clip}%
\pgfsetbuttcap%
\pgfsetroundjoin%
\definecolor{currentfill}{rgb}{0.121569,0.466667,0.705882}%
\pgfsetfillcolor{currentfill}%
\pgfsetlinewidth{1.003750pt}%
\definecolor{currentstroke}{rgb}{0.121569,0.466667,0.705882}%
\pgfsetstrokecolor{currentstroke}%
\pgfsetdash{}{0pt}%
\pgfpathmoveto{\pgfqpoint{2.977888in}{1.057071in}}%
\pgfpathcurveto{\pgfqpoint{2.987976in}{1.057071in}}{\pgfqpoint{2.997651in}{1.061079in}}{\pgfqpoint{3.004784in}{1.068211in}}%
\pgfpathcurveto{\pgfqpoint{3.011917in}{1.075344in}}{\pgfqpoint{3.015925in}{1.085020in}}{\pgfqpoint{3.015925in}{1.095107in}}%
\pgfpathcurveto{\pgfqpoint{3.015925in}{1.105194in}}{\pgfqpoint{3.011917in}{1.114870in}}{\pgfqpoint{3.004784in}{1.122003in}}%
\pgfpathcurveto{\pgfqpoint{2.997651in}{1.129136in}}{\pgfqpoint{2.987976in}{1.133143in}}{\pgfqpoint{2.977888in}{1.133143in}}%
\pgfpathcurveto{\pgfqpoint{2.967801in}{1.133143in}}{\pgfqpoint{2.958125in}{1.129136in}}{\pgfqpoint{2.950993in}{1.122003in}}%
\pgfpathcurveto{\pgfqpoint{2.943860in}{1.114870in}}{\pgfqpoint{2.939852in}{1.105194in}}{\pgfqpoint{2.939852in}{1.095107in}}%
\pgfpathcurveto{\pgfqpoint{2.939852in}{1.085020in}}{\pgfqpoint{2.943860in}{1.075344in}}{\pgfqpoint{2.950993in}{1.068211in}}%
\pgfpathcurveto{\pgfqpoint{2.958125in}{1.061079in}}{\pgfqpoint{2.967801in}{1.057071in}}{\pgfqpoint{2.977888in}{1.057071in}}%
\pgfpathclose%
\pgfusepath{stroke,fill}%
\end{pgfscope}%
\begin{pgfscope}%
\pgfpathrectangle{\pgfqpoint{0.800000in}{0.528000in}}{\pgfqpoint{4.960000in}{3.696000in}} %
\pgfusepath{clip}%
\pgfsetbuttcap%
\pgfsetroundjoin%
\definecolor{currentfill}{rgb}{0.121569,0.466667,0.705882}%
\pgfsetfillcolor{currentfill}%
\pgfsetlinewidth{1.003750pt}%
\definecolor{currentstroke}{rgb}{0.121569,0.466667,0.705882}%
\pgfsetstrokecolor{currentstroke}%
\pgfsetdash{}{0pt}%
\pgfpathmoveto{\pgfqpoint{2.801824in}{3.914414in}}%
\pgfpathcurveto{\pgfqpoint{2.811912in}{3.914414in}}{\pgfqpoint{2.821587in}{3.918422in}}{\pgfqpoint{2.828720in}{3.925555in}}%
\pgfpathcurveto{\pgfqpoint{2.835853in}{3.932688in}}{\pgfqpoint{2.839861in}{3.942363in}}{\pgfqpoint{2.839861in}{3.952451in}}%
\pgfpathcurveto{\pgfqpoint{2.839861in}{3.962538in}}{\pgfqpoint{2.835853in}{3.972214in}}{\pgfqpoint{2.828720in}{3.979346in}}%
\pgfpathcurveto{\pgfqpoint{2.821587in}{3.986479in}}{\pgfqpoint{2.811912in}{3.990487in}}{\pgfqpoint{2.801824in}{3.990487in}}%
\pgfpathcurveto{\pgfqpoint{2.791737in}{3.990487in}}{\pgfqpoint{2.782061in}{3.986479in}}{\pgfqpoint{2.774929in}{3.979346in}}%
\pgfpathcurveto{\pgfqpoint{2.767796in}{3.972214in}}{\pgfqpoint{2.763788in}{3.962538in}}{\pgfqpoint{2.763788in}{3.952451in}}%
\pgfpathcurveto{\pgfqpoint{2.763788in}{3.942363in}}{\pgfqpoint{2.767796in}{3.932688in}}{\pgfqpoint{2.774929in}{3.925555in}}%
\pgfpathcurveto{\pgfqpoint{2.782061in}{3.918422in}}{\pgfqpoint{2.791737in}{3.914414in}}{\pgfqpoint{2.801824in}{3.914414in}}%
\pgfpathclose%
\pgfusepath{stroke,fill}%
\end{pgfscope}%
\begin{pgfscope}%
\pgfpathrectangle{\pgfqpoint{0.800000in}{0.528000in}}{\pgfqpoint{4.960000in}{3.696000in}} %
\pgfusepath{clip}%
\pgfsetbuttcap%
\pgfsetroundjoin%
\definecolor{currentfill}{rgb}{0.121569,0.466667,0.705882}%
\pgfsetfillcolor{currentfill}%
\pgfsetlinewidth{1.003750pt}%
\definecolor{currentstroke}{rgb}{0.121569,0.466667,0.705882}%
\pgfsetstrokecolor{currentstroke}%
\pgfsetdash{}{0pt}%
\pgfpathmoveto{\pgfqpoint{3.443936in}{0.966203in}}%
\pgfpathcurveto{\pgfqpoint{3.454024in}{0.966203in}}{\pgfqpoint{3.463699in}{0.970211in}}{\pgfqpoint{3.470832in}{0.977344in}}%
\pgfpathcurveto{\pgfqpoint{3.477965in}{0.984477in}}{\pgfqpoint{3.481972in}{0.994152in}}{\pgfqpoint{3.481972in}{1.004239in}}%
\pgfpathcurveto{\pgfqpoint{3.481972in}{1.014327in}}{\pgfqpoint{3.477965in}{1.024002in}}{\pgfqpoint{3.470832in}{1.031135in}}%
\pgfpathcurveto{\pgfqpoint{3.463699in}{1.038268in}}{\pgfqpoint{3.454024in}{1.042276in}}{\pgfqpoint{3.443936in}{1.042276in}}%
\pgfpathcurveto{\pgfqpoint{3.433849in}{1.042276in}}{\pgfqpoint{3.424173in}{1.038268in}}{\pgfqpoint{3.417040in}{1.031135in}}%
\pgfpathcurveto{\pgfqpoint{3.409908in}{1.024002in}}{\pgfqpoint{3.405900in}{1.014327in}}{\pgfqpoint{3.405900in}{1.004239in}}%
\pgfpathcurveto{\pgfqpoint{3.405900in}{0.994152in}}{\pgfqpoint{3.409908in}{0.984477in}}{\pgfqpoint{3.417040in}{0.977344in}}%
\pgfpathcurveto{\pgfqpoint{3.424173in}{0.970211in}}{\pgfqpoint{3.433849in}{0.966203in}}{\pgfqpoint{3.443936in}{0.966203in}}%
\pgfpathclose%
\pgfusepath{stroke,fill}%
\end{pgfscope}%
\begin{pgfscope}%
\pgfpathrectangle{\pgfqpoint{0.800000in}{0.528000in}}{\pgfqpoint{4.960000in}{3.696000in}} %
\pgfusepath{clip}%
\pgfsetbuttcap%
\pgfsetroundjoin%
\definecolor{currentfill}{rgb}{0.121569,0.466667,0.705882}%
\pgfsetfillcolor{currentfill}%
\pgfsetlinewidth{1.003750pt}%
\definecolor{currentstroke}{rgb}{0.121569,0.466667,0.705882}%
\pgfsetstrokecolor{currentstroke}%
\pgfsetdash{}{0pt}%
\pgfpathmoveto{\pgfqpoint{4.927754in}{3.445549in}}%
\pgfpathcurveto{\pgfqpoint{4.937842in}{3.445549in}}{\pgfqpoint{4.947517in}{3.449557in}}{\pgfqpoint{4.954650in}{3.456690in}}%
\pgfpathcurveto{\pgfqpoint{4.961783in}{3.463823in}}{\pgfqpoint{4.965791in}{3.473498in}}{\pgfqpoint{4.965791in}{3.483586in}}%
\pgfpathcurveto{\pgfqpoint{4.965791in}{3.493673in}}{\pgfqpoint{4.961783in}{3.503349in}}{\pgfqpoint{4.954650in}{3.510481in}}%
\pgfpathcurveto{\pgfqpoint{4.947517in}{3.517614in}}{\pgfqpoint{4.937842in}{3.521622in}}{\pgfqpoint{4.927754in}{3.521622in}}%
\pgfpathcurveto{\pgfqpoint{4.917667in}{3.521622in}}{\pgfqpoint{4.907991in}{3.517614in}}{\pgfqpoint{4.900858in}{3.510481in}}%
\pgfpathcurveto{\pgfqpoint{4.893726in}{3.503349in}}{\pgfqpoint{4.889718in}{3.493673in}}{\pgfqpoint{4.889718in}{3.483586in}}%
\pgfpathcurveto{\pgfqpoint{4.889718in}{3.473498in}}{\pgfqpoint{4.893726in}{3.463823in}}{\pgfqpoint{4.900858in}{3.456690in}}%
\pgfpathcurveto{\pgfqpoint{4.907991in}{3.449557in}}{\pgfqpoint{4.917667in}{3.445549in}}{\pgfqpoint{4.927754in}{3.445549in}}%
\pgfpathclose%
\pgfusepath{stroke,fill}%
\end{pgfscope}%
\begin{pgfscope}%
\pgfpathrectangle{\pgfqpoint{0.800000in}{0.528000in}}{\pgfqpoint{4.960000in}{3.696000in}} %
\pgfusepath{clip}%
\pgfsetbuttcap%
\pgfsetroundjoin%
\definecolor{currentfill}{rgb}{0.121569,0.466667,0.705882}%
\pgfsetfillcolor{currentfill}%
\pgfsetlinewidth{1.003750pt}%
\definecolor{currentstroke}{rgb}{0.121569,0.466667,0.705882}%
\pgfsetstrokecolor{currentstroke}%
\pgfsetdash{}{0pt}%
\pgfpathmoveto{\pgfqpoint{1.359908in}{2.903012in}}%
\pgfpathcurveto{\pgfqpoint{1.369996in}{2.903012in}}{\pgfqpoint{1.379671in}{2.907020in}}{\pgfqpoint{1.386804in}{2.914153in}}%
\pgfpathcurveto{\pgfqpoint{1.393937in}{2.921285in}}{\pgfqpoint{1.397945in}{2.930961in}}{\pgfqpoint{1.397945in}{2.941048in}}%
\pgfpathcurveto{\pgfqpoint{1.397945in}{2.951136in}}{\pgfqpoint{1.393937in}{2.960811in}}{\pgfqpoint{1.386804in}{2.967944in}}%
\pgfpathcurveto{\pgfqpoint{1.379671in}{2.975077in}}{\pgfqpoint{1.369996in}{2.979085in}}{\pgfqpoint{1.359908in}{2.979085in}}%
\pgfpathcurveto{\pgfqpoint{1.349821in}{2.979085in}}{\pgfqpoint{1.340145in}{2.975077in}}{\pgfqpoint{1.333013in}{2.967944in}}%
\pgfpathcurveto{\pgfqpoint{1.325880in}{2.960811in}}{\pgfqpoint{1.321872in}{2.951136in}}{\pgfqpoint{1.321872in}{2.941048in}}%
\pgfpathcurveto{\pgfqpoint{1.321872in}{2.930961in}}{\pgfqpoint{1.325880in}{2.921285in}}{\pgfqpoint{1.333013in}{2.914153in}}%
\pgfpathcurveto{\pgfqpoint{1.340145in}{2.907020in}}{\pgfqpoint{1.349821in}{2.903012in}}{\pgfqpoint{1.359908in}{2.903012in}}%
\pgfpathclose%
\pgfusepath{stroke,fill}%
\end{pgfscope}%
\begin{pgfscope}%
\pgfpathrectangle{\pgfqpoint{0.800000in}{0.528000in}}{\pgfqpoint{4.960000in}{3.696000in}} %
\pgfusepath{clip}%
\pgfsetbuttcap%
\pgfsetroundjoin%
\definecolor{currentfill}{rgb}{0.121569,0.466667,0.705882}%
\pgfsetfillcolor{currentfill}%
\pgfsetlinewidth{1.003750pt}%
\definecolor{currentstroke}{rgb}{0.121569,0.466667,0.705882}%
\pgfsetstrokecolor{currentstroke}%
\pgfsetdash{}{0pt}%
\pgfpathmoveto{\pgfqpoint{4.851379in}{1.797772in}}%
\pgfpathcurveto{\pgfqpoint{4.861466in}{1.797772in}}{\pgfqpoint{4.871142in}{1.801780in}}{\pgfqpoint{4.878275in}{1.808913in}}%
\pgfpathcurveto{\pgfqpoint{4.885407in}{1.816046in}}{\pgfqpoint{4.889415in}{1.825721in}}{\pgfqpoint{4.889415in}{1.835808in}}%
\pgfpathcurveto{\pgfqpoint{4.889415in}{1.845896in}}{\pgfqpoint{4.885407in}{1.855571in}}{\pgfqpoint{4.878275in}{1.862704in}}%
\pgfpathcurveto{\pgfqpoint{4.871142in}{1.869837in}}{\pgfqpoint{4.861466in}{1.873845in}}{\pgfqpoint{4.851379in}{1.873845in}}%
\pgfpathcurveto{\pgfqpoint{4.841292in}{1.873845in}}{\pgfqpoint{4.831616in}{1.869837in}}{\pgfqpoint{4.824483in}{1.862704in}}%
\pgfpathcurveto{\pgfqpoint{4.817350in}{1.855571in}}{\pgfqpoint{4.813343in}{1.845896in}}{\pgfqpoint{4.813343in}{1.835808in}}%
\pgfpathcurveto{\pgfqpoint{4.813343in}{1.825721in}}{\pgfqpoint{4.817350in}{1.816046in}}{\pgfqpoint{4.824483in}{1.808913in}}%
\pgfpathcurveto{\pgfqpoint{4.831616in}{1.801780in}}{\pgfqpoint{4.841292in}{1.797772in}}{\pgfqpoint{4.851379in}{1.797772in}}%
\pgfpathclose%
\pgfusepath{stroke,fill}%
\end{pgfscope}%
\begin{pgfscope}%
\pgfpathrectangle{\pgfqpoint{0.800000in}{0.528000in}}{\pgfqpoint{4.960000in}{3.696000in}} %
\pgfusepath{clip}%
\pgfsetbuttcap%
\pgfsetroundjoin%
\definecolor{currentfill}{rgb}{0.121569,0.466667,0.705882}%
\pgfsetfillcolor{currentfill}%
\pgfsetlinewidth{1.003750pt}%
\definecolor{currentstroke}{rgb}{0.121569,0.466667,0.705882}%
\pgfsetstrokecolor{currentstroke}%
\pgfsetdash{}{0pt}%
\pgfpathmoveto{\pgfqpoint{2.538196in}{3.874252in}}%
\pgfpathcurveto{\pgfqpoint{2.548283in}{3.874252in}}{\pgfqpoint{2.557958in}{3.878260in}}{\pgfqpoint{2.565091in}{3.885393in}}%
\pgfpathcurveto{\pgfqpoint{2.572224in}{3.892526in}}{\pgfqpoint{2.576232in}{3.902201in}}{\pgfqpoint{2.576232in}{3.912288in}}%
\pgfpathcurveto{\pgfqpoint{2.576232in}{3.922376in}}{\pgfqpoint{2.572224in}{3.932051in}}{\pgfqpoint{2.565091in}{3.939184in}}%
\pgfpathcurveto{\pgfqpoint{2.557958in}{3.946317in}}{\pgfqpoint{2.548283in}{3.950325in}}{\pgfqpoint{2.538196in}{3.950325in}}%
\pgfpathcurveto{\pgfqpoint{2.528108in}{3.950325in}}{\pgfqpoint{2.518433in}{3.946317in}}{\pgfqpoint{2.511300in}{3.939184in}}%
\pgfpathcurveto{\pgfqpoint{2.504167in}{3.932051in}}{\pgfqpoint{2.500159in}{3.922376in}}{\pgfqpoint{2.500159in}{3.912288in}}%
\pgfpathcurveto{\pgfqpoint{2.500159in}{3.902201in}}{\pgfqpoint{2.504167in}{3.892526in}}{\pgfqpoint{2.511300in}{3.885393in}}%
\pgfpathcurveto{\pgfqpoint{2.518433in}{3.878260in}}{\pgfqpoint{2.528108in}{3.874252in}}{\pgfqpoint{2.538196in}{3.874252in}}%
\pgfpathclose%
\pgfusepath{stroke,fill}%
\end{pgfscope}%
\begin{pgfscope}%
\pgfpathrectangle{\pgfqpoint{0.800000in}{0.528000in}}{\pgfqpoint{4.960000in}{3.696000in}} %
\pgfusepath{clip}%
\pgfsetbuttcap%
\pgfsetroundjoin%
\definecolor{currentfill}{rgb}{0.121569,0.466667,0.705882}%
\pgfsetfillcolor{currentfill}%
\pgfsetlinewidth{1.003750pt}%
\definecolor{currentstroke}{rgb}{0.121569,0.466667,0.705882}%
\pgfsetstrokecolor{currentstroke}%
\pgfsetdash{}{0pt}%
\pgfpathmoveto{\pgfqpoint{2.896872in}{3.996402in}}%
\pgfpathcurveto{\pgfqpoint{2.906960in}{3.996402in}}{\pgfqpoint{2.916635in}{4.000410in}}{\pgfqpoint{2.923768in}{4.007542in}}%
\pgfpathcurveto{\pgfqpoint{2.930901in}{4.014675in}}{\pgfqpoint{2.934909in}{4.024351in}}{\pgfqpoint{2.934909in}{4.034438in}}%
\pgfpathcurveto{\pgfqpoint{2.934909in}{4.044526in}}{\pgfqpoint{2.930901in}{4.054201in}}{\pgfqpoint{2.923768in}{4.061334in}}%
\pgfpathcurveto{\pgfqpoint{2.916635in}{4.068467in}}{\pgfqpoint{2.906960in}{4.072475in}}{\pgfqpoint{2.896872in}{4.072475in}}%
\pgfpathcurveto{\pgfqpoint{2.886785in}{4.072475in}}{\pgfqpoint{2.877110in}{4.068467in}}{\pgfqpoint{2.869977in}{4.061334in}}%
\pgfpathcurveto{\pgfqpoint{2.862844in}{4.054201in}}{\pgfqpoint{2.858836in}{4.044526in}}{\pgfqpoint{2.858836in}{4.034438in}}%
\pgfpathcurveto{\pgfqpoint{2.858836in}{4.024351in}}{\pgfqpoint{2.862844in}{4.014675in}}{\pgfqpoint{2.869977in}{4.007542in}}%
\pgfpathcurveto{\pgfqpoint{2.877110in}{4.000410in}}{\pgfqpoint{2.886785in}{3.996402in}}{\pgfqpoint{2.896872in}{3.996402in}}%
\pgfpathclose%
\pgfusepath{stroke,fill}%
\end{pgfscope}%
\begin{pgfscope}%
\pgfpathrectangle{\pgfqpoint{0.800000in}{0.528000in}}{\pgfqpoint{4.960000in}{3.696000in}} %
\pgfusepath{clip}%
\pgfsetbuttcap%
\pgfsetroundjoin%
\definecolor{currentfill}{rgb}{0.121569,0.466667,0.705882}%
\pgfsetfillcolor{currentfill}%
\pgfsetlinewidth{1.003750pt}%
\definecolor{currentstroke}{rgb}{0.121569,0.466667,0.705882}%
\pgfsetstrokecolor{currentstroke}%
\pgfsetdash{}{0pt}%
\pgfpathmoveto{\pgfqpoint{4.822914in}{3.493909in}}%
\pgfpathcurveto{\pgfqpoint{4.833001in}{3.493909in}}{\pgfqpoint{4.842677in}{3.497917in}}{\pgfqpoint{4.849809in}{3.505050in}}%
\pgfpathcurveto{\pgfqpoint{4.856942in}{3.512183in}}{\pgfqpoint{4.860950in}{3.521858in}}{\pgfqpoint{4.860950in}{3.531946in}}%
\pgfpathcurveto{\pgfqpoint{4.860950in}{3.542033in}}{\pgfqpoint{4.856942in}{3.551708in}}{\pgfqpoint{4.849809in}{3.558841in}}%
\pgfpathcurveto{\pgfqpoint{4.842677in}{3.565974in}}{\pgfqpoint{4.833001in}{3.569982in}}{\pgfqpoint{4.822914in}{3.569982in}}%
\pgfpathcurveto{\pgfqpoint{4.812826in}{3.569982in}}{\pgfqpoint{4.803151in}{3.565974in}}{\pgfqpoint{4.796018in}{3.558841in}}%
\pgfpathcurveto{\pgfqpoint{4.788885in}{3.551708in}}{\pgfqpoint{4.784877in}{3.542033in}}{\pgfqpoint{4.784877in}{3.531946in}}%
\pgfpathcurveto{\pgfqpoint{4.784877in}{3.521858in}}{\pgfqpoint{4.788885in}{3.512183in}}{\pgfqpoint{4.796018in}{3.505050in}}%
\pgfpathcurveto{\pgfqpoint{4.803151in}{3.497917in}}{\pgfqpoint{4.812826in}{3.493909in}}{\pgfqpoint{4.822914in}{3.493909in}}%
\pgfpathclose%
\pgfusepath{stroke,fill}%
\end{pgfscope}%
\begin{pgfscope}%
\pgfpathrectangle{\pgfqpoint{0.800000in}{0.528000in}}{\pgfqpoint{4.960000in}{3.696000in}} %
\pgfusepath{clip}%
\pgfsetbuttcap%
\pgfsetroundjoin%
\definecolor{currentfill}{rgb}{0.121569,0.466667,0.705882}%
\pgfsetfillcolor{currentfill}%
\pgfsetlinewidth{1.003750pt}%
\definecolor{currentstroke}{rgb}{0.121569,0.466667,0.705882}%
\pgfsetstrokecolor{currentstroke}%
\pgfsetdash{}{0pt}%
\pgfpathmoveto{\pgfqpoint{4.897911in}{2.472966in}}%
\pgfpathcurveto{\pgfqpoint{4.907999in}{2.472966in}}{\pgfqpoint{4.917674in}{2.476974in}}{\pgfqpoint{4.924807in}{2.484107in}}%
\pgfpathcurveto{\pgfqpoint{4.931940in}{2.491240in}}{\pgfqpoint{4.935948in}{2.500915in}}{\pgfqpoint{4.935948in}{2.511002in}}%
\pgfpathcurveto{\pgfqpoint{4.935948in}{2.521090in}}{\pgfqpoint{4.931940in}{2.530765in}}{\pgfqpoint{4.924807in}{2.537898in}}%
\pgfpathcurveto{\pgfqpoint{4.917674in}{2.545031in}}{\pgfqpoint{4.907999in}{2.549039in}}{\pgfqpoint{4.897911in}{2.549039in}}%
\pgfpathcurveto{\pgfqpoint{4.887824in}{2.549039in}}{\pgfqpoint{4.878148in}{2.545031in}}{\pgfqpoint{4.871016in}{2.537898in}}%
\pgfpathcurveto{\pgfqpoint{4.863883in}{2.530765in}}{\pgfqpoint{4.859875in}{2.521090in}}{\pgfqpoint{4.859875in}{2.511002in}}%
\pgfpathcurveto{\pgfqpoint{4.859875in}{2.500915in}}{\pgfqpoint{4.863883in}{2.491240in}}{\pgfqpoint{4.871016in}{2.484107in}}%
\pgfpathcurveto{\pgfqpoint{4.878148in}{2.476974in}}{\pgfqpoint{4.887824in}{2.472966in}}{\pgfqpoint{4.897911in}{2.472966in}}%
\pgfpathclose%
\pgfusepath{stroke,fill}%
\end{pgfscope}%
\begin{pgfscope}%
\pgfpathrectangle{\pgfqpoint{0.800000in}{0.528000in}}{\pgfqpoint{4.960000in}{3.696000in}} %
\pgfusepath{clip}%
\pgfsetbuttcap%
\pgfsetroundjoin%
\definecolor{currentfill}{rgb}{0.121569,0.466667,0.705882}%
\pgfsetfillcolor{currentfill}%
\pgfsetlinewidth{1.003750pt}%
\definecolor{currentstroke}{rgb}{0.121569,0.466667,0.705882}%
\pgfsetstrokecolor{currentstroke}%
\pgfsetdash{}{0pt}%
\pgfpathmoveto{\pgfqpoint{2.499325in}{1.361905in}}%
\pgfpathcurveto{\pgfqpoint{2.509413in}{1.361905in}}{\pgfqpoint{2.519088in}{1.365913in}}{\pgfqpoint{2.526221in}{1.373045in}}%
\pgfpathcurveto{\pgfqpoint{2.533354in}{1.380178in}}{\pgfqpoint{2.537362in}{1.389854in}}{\pgfqpoint{2.537362in}{1.399941in}}%
\pgfpathcurveto{\pgfqpoint{2.537362in}{1.410029in}}{\pgfqpoint{2.533354in}{1.419704in}}{\pgfqpoint{2.526221in}{1.426837in}}%
\pgfpathcurveto{\pgfqpoint{2.519088in}{1.433970in}}{\pgfqpoint{2.509413in}{1.437977in}}{\pgfqpoint{2.499325in}{1.437977in}}%
\pgfpathcurveto{\pgfqpoint{2.489238in}{1.437977in}}{\pgfqpoint{2.479563in}{1.433970in}}{\pgfqpoint{2.472430in}{1.426837in}}%
\pgfpathcurveto{\pgfqpoint{2.465297in}{1.419704in}}{\pgfqpoint{2.461289in}{1.410029in}}{\pgfqpoint{2.461289in}{1.399941in}}%
\pgfpathcurveto{\pgfqpoint{2.461289in}{1.389854in}}{\pgfqpoint{2.465297in}{1.380178in}}{\pgfqpoint{2.472430in}{1.373045in}}%
\pgfpathcurveto{\pgfqpoint{2.479563in}{1.365913in}}{\pgfqpoint{2.489238in}{1.361905in}}{\pgfqpoint{2.499325in}{1.361905in}}%
\pgfpathclose%
\pgfusepath{stroke,fill}%
\end{pgfscope}%
\begin{pgfscope}%
\pgfpathrectangle{\pgfqpoint{0.800000in}{0.528000in}}{\pgfqpoint{4.960000in}{3.696000in}} %
\pgfusepath{clip}%
\pgfsetbuttcap%
\pgfsetroundjoin%
\definecolor{currentfill}{rgb}{0.121569,0.466667,0.705882}%
\pgfsetfillcolor{currentfill}%
\pgfsetlinewidth{1.003750pt}%
\definecolor{currentstroke}{rgb}{0.121569,0.466667,0.705882}%
\pgfsetstrokecolor{currentstroke}%
\pgfsetdash{}{0pt}%
\pgfpathmoveto{\pgfqpoint{3.888995in}{3.778586in}}%
\pgfpathcurveto{\pgfqpoint{3.899082in}{3.778586in}}{\pgfqpoint{3.908758in}{3.782593in}}{\pgfqpoint{3.915891in}{3.789726in}}%
\pgfpathcurveto{\pgfqpoint{3.923024in}{3.796859in}}{\pgfqpoint{3.927031in}{3.806534in}}{\pgfqpoint{3.927031in}{3.816622in}}%
\pgfpathcurveto{\pgfqpoint{3.927031in}{3.826709in}}{\pgfqpoint{3.923024in}{3.836385in}}{\pgfqpoint{3.915891in}{3.843518in}}%
\pgfpathcurveto{\pgfqpoint{3.908758in}{3.850650in}}{\pgfqpoint{3.899082in}{3.854658in}}{\pgfqpoint{3.888995in}{3.854658in}}%
\pgfpathcurveto{\pgfqpoint{3.878908in}{3.854658in}}{\pgfqpoint{3.869232in}{3.850650in}}{\pgfqpoint{3.862099in}{3.843518in}}%
\pgfpathcurveto{\pgfqpoint{3.854967in}{3.836385in}}{\pgfqpoint{3.850959in}{3.826709in}}{\pgfqpoint{3.850959in}{3.816622in}}%
\pgfpathcurveto{\pgfqpoint{3.850959in}{3.806534in}}{\pgfqpoint{3.854967in}{3.796859in}}{\pgfqpoint{3.862099in}{3.789726in}}%
\pgfpathcurveto{\pgfqpoint{3.869232in}{3.782593in}}{\pgfqpoint{3.878908in}{3.778586in}}{\pgfqpoint{3.888995in}{3.778586in}}%
\pgfpathclose%
\pgfusepath{stroke,fill}%
\end{pgfscope}%
\begin{pgfscope}%
\pgfpathrectangle{\pgfqpoint{0.800000in}{0.528000in}}{\pgfqpoint{4.960000in}{3.696000in}} %
\pgfusepath{clip}%
\pgfsetbuttcap%
\pgfsetroundjoin%
\definecolor{currentfill}{rgb}{0.121569,0.466667,0.705882}%
\pgfsetfillcolor{currentfill}%
\pgfsetlinewidth{1.003750pt}%
\definecolor{currentstroke}{rgb}{0.121569,0.466667,0.705882}%
\pgfsetstrokecolor{currentstroke}%
\pgfsetdash{}{0pt}%
\pgfpathmoveto{\pgfqpoint{3.038920in}{3.849426in}}%
\pgfpathcurveto{\pgfqpoint{3.049007in}{3.849426in}}{\pgfqpoint{3.058682in}{3.853434in}}{\pgfqpoint{3.065815in}{3.860567in}}%
\pgfpathcurveto{\pgfqpoint{3.072948in}{3.867700in}}{\pgfqpoint{3.076956in}{3.877375in}}{\pgfqpoint{3.076956in}{3.887462in}}%
\pgfpathcurveto{\pgfqpoint{3.076956in}{3.897550in}}{\pgfqpoint{3.072948in}{3.907225in}}{\pgfqpoint{3.065815in}{3.914358in}}%
\pgfpathcurveto{\pgfqpoint{3.058682in}{3.921491in}}{\pgfqpoint{3.049007in}{3.925499in}}{\pgfqpoint{3.038920in}{3.925499in}}%
\pgfpathcurveto{\pgfqpoint{3.028832in}{3.925499in}}{\pgfqpoint{3.019157in}{3.921491in}}{\pgfqpoint{3.012024in}{3.914358in}}%
\pgfpathcurveto{\pgfqpoint{3.004891in}{3.907225in}}{\pgfqpoint{3.000883in}{3.897550in}}{\pgfqpoint{3.000883in}{3.887462in}}%
\pgfpathcurveto{\pgfqpoint{3.000883in}{3.877375in}}{\pgfqpoint{3.004891in}{3.867700in}}{\pgfqpoint{3.012024in}{3.860567in}}%
\pgfpathcurveto{\pgfqpoint{3.019157in}{3.853434in}}{\pgfqpoint{3.028832in}{3.849426in}}{\pgfqpoint{3.038920in}{3.849426in}}%
\pgfpathclose%
\pgfusepath{stroke,fill}%
\end{pgfscope}%
\begin{pgfscope}%
\pgfpathrectangle{\pgfqpoint{0.800000in}{0.528000in}}{\pgfqpoint{4.960000in}{3.696000in}} %
\pgfusepath{clip}%
\pgfsetbuttcap%
\pgfsetroundjoin%
\definecolor{currentfill}{rgb}{0.121569,0.466667,0.705882}%
\pgfsetfillcolor{currentfill}%
\pgfsetlinewidth{1.003750pt}%
\definecolor{currentstroke}{rgb}{0.121569,0.466667,0.705882}%
\pgfsetstrokecolor{currentstroke}%
\pgfsetdash{}{0pt}%
\pgfpathmoveto{\pgfqpoint{1.030944in}{2.426785in}}%
\pgfpathcurveto{\pgfqpoint{1.041032in}{2.426785in}}{\pgfqpoint{1.050707in}{2.430793in}}{\pgfqpoint{1.057840in}{2.437926in}}%
\pgfpathcurveto{\pgfqpoint{1.064973in}{2.445058in}}{\pgfqpoint{1.068981in}{2.454734in}}{\pgfqpoint{1.068981in}{2.464821in}}%
\pgfpathcurveto{\pgfqpoint{1.068981in}{2.474909in}}{\pgfqpoint{1.064973in}{2.484584in}}{\pgfqpoint{1.057840in}{2.491717in}}%
\pgfpathcurveto{\pgfqpoint{1.050707in}{2.498850in}}{\pgfqpoint{1.041032in}{2.502858in}}{\pgfqpoint{1.030944in}{2.502858in}}%
\pgfpathcurveto{\pgfqpoint{1.020857in}{2.502858in}}{\pgfqpoint{1.011181in}{2.498850in}}{\pgfqpoint{1.004049in}{2.491717in}}%
\pgfpathcurveto{\pgfqpoint{0.996916in}{2.484584in}}{\pgfqpoint{0.992908in}{2.474909in}}{\pgfqpoint{0.992908in}{2.464821in}}%
\pgfpathcurveto{\pgfqpoint{0.992908in}{2.454734in}}{\pgfqpoint{0.996916in}{2.445058in}}{\pgfqpoint{1.004049in}{2.437926in}}%
\pgfpathcurveto{\pgfqpoint{1.011181in}{2.430793in}}{\pgfqpoint{1.020857in}{2.426785in}}{\pgfqpoint{1.030944in}{2.426785in}}%
\pgfpathclose%
\pgfusepath{stroke,fill}%
\end{pgfscope}%
\begin{pgfscope}%
\pgfpathrectangle{\pgfqpoint{0.800000in}{0.528000in}}{\pgfqpoint{4.960000in}{3.696000in}} %
\pgfusepath{clip}%
\pgfsetbuttcap%
\pgfsetroundjoin%
\definecolor{currentfill}{rgb}{0.121569,0.466667,0.705882}%
\pgfsetfillcolor{currentfill}%
\pgfsetlinewidth{1.003750pt}%
\definecolor{currentstroke}{rgb}{0.121569,0.466667,0.705882}%
\pgfsetstrokecolor{currentstroke}%
\pgfsetdash{}{0pt}%
\pgfpathmoveto{\pgfqpoint{5.184535in}{2.879662in}}%
\pgfpathcurveto{\pgfqpoint{5.194622in}{2.879662in}}{\pgfqpoint{5.204297in}{2.883670in}}{\pgfqpoint{5.211430in}{2.890803in}}%
\pgfpathcurveto{\pgfqpoint{5.218563in}{2.897936in}}{\pgfqpoint{5.222571in}{2.907611in}}{\pgfqpoint{5.222571in}{2.917699in}}%
\pgfpathcurveto{\pgfqpoint{5.222571in}{2.927786in}}{\pgfqpoint{5.218563in}{2.937461in}}{\pgfqpoint{5.211430in}{2.944594in}}%
\pgfpathcurveto{\pgfqpoint{5.204297in}{2.951727in}}{\pgfqpoint{5.194622in}{2.955735in}}{\pgfqpoint{5.184535in}{2.955735in}}%
\pgfpathcurveto{\pgfqpoint{5.174447in}{2.955735in}}{\pgfqpoint{5.164772in}{2.951727in}}{\pgfqpoint{5.157639in}{2.944594in}}%
\pgfpathcurveto{\pgfqpoint{5.150506in}{2.937461in}}{\pgfqpoint{5.146498in}{2.927786in}}{\pgfqpoint{5.146498in}{2.917699in}}%
\pgfpathcurveto{\pgfqpoint{5.146498in}{2.907611in}}{\pgfqpoint{5.150506in}{2.897936in}}{\pgfqpoint{5.157639in}{2.890803in}}%
\pgfpathcurveto{\pgfqpoint{5.164772in}{2.883670in}}{\pgfqpoint{5.174447in}{2.879662in}}{\pgfqpoint{5.184535in}{2.879662in}}%
\pgfpathclose%
\pgfusepath{stroke,fill}%
\end{pgfscope}%
\begin{pgfscope}%
\pgfpathrectangle{\pgfqpoint{0.800000in}{0.528000in}}{\pgfqpoint{4.960000in}{3.696000in}} %
\pgfusepath{clip}%
\pgfsetbuttcap%
\pgfsetroundjoin%
\definecolor{currentfill}{rgb}{0.121569,0.466667,0.705882}%
\pgfsetfillcolor{currentfill}%
\pgfsetlinewidth{1.003750pt}%
\definecolor{currentstroke}{rgb}{0.121569,0.466667,0.705882}%
\pgfsetstrokecolor{currentstroke}%
\pgfsetdash{}{0pt}%
\pgfpathmoveto{\pgfqpoint{1.383108in}{1.793461in}}%
\pgfpathcurveto{\pgfqpoint{1.393196in}{1.793461in}}{\pgfqpoint{1.402871in}{1.797469in}}{\pgfqpoint{1.410004in}{1.804602in}}%
\pgfpathcurveto{\pgfqpoint{1.417137in}{1.811734in}}{\pgfqpoint{1.421145in}{1.821410in}}{\pgfqpoint{1.421145in}{1.831497in}}%
\pgfpathcurveto{\pgfqpoint{1.421145in}{1.841585in}}{\pgfqpoint{1.417137in}{1.851260in}}{\pgfqpoint{1.410004in}{1.858393in}}%
\pgfpathcurveto{\pgfqpoint{1.402871in}{1.865526in}}{\pgfqpoint{1.393196in}{1.869534in}}{\pgfqpoint{1.383108in}{1.869534in}}%
\pgfpathcurveto{\pgfqpoint{1.373021in}{1.869534in}}{\pgfqpoint{1.363346in}{1.865526in}}{\pgfqpoint{1.356213in}{1.858393in}}%
\pgfpathcurveto{\pgfqpoint{1.349080in}{1.851260in}}{\pgfqpoint{1.345072in}{1.841585in}}{\pgfqpoint{1.345072in}{1.831497in}}%
\pgfpathcurveto{\pgfqpoint{1.345072in}{1.821410in}}{\pgfqpoint{1.349080in}{1.811734in}}{\pgfqpoint{1.356213in}{1.804602in}}%
\pgfpathcurveto{\pgfqpoint{1.363346in}{1.797469in}}{\pgfqpoint{1.373021in}{1.793461in}}{\pgfqpoint{1.383108in}{1.793461in}}%
\pgfpathclose%
\pgfusepath{stroke,fill}%
\end{pgfscope}%
\begin{pgfscope}%
\pgfpathrectangle{\pgfqpoint{0.800000in}{0.528000in}}{\pgfqpoint{4.960000in}{3.696000in}} %
\pgfusepath{clip}%
\pgfsetbuttcap%
\pgfsetroundjoin%
\definecolor{currentfill}{rgb}{0.121569,0.466667,0.705882}%
\pgfsetfillcolor{currentfill}%
\pgfsetlinewidth{1.003750pt}%
\definecolor{currentstroke}{rgb}{0.121569,0.466667,0.705882}%
\pgfsetstrokecolor{currentstroke}%
\pgfsetdash{}{0pt}%
\pgfpathmoveto{\pgfqpoint{4.302064in}{3.601882in}}%
\pgfpathcurveto{\pgfqpoint{4.312151in}{3.601882in}}{\pgfqpoint{4.321827in}{3.605890in}}{\pgfqpoint{4.328960in}{3.613022in}}%
\pgfpathcurveto{\pgfqpoint{4.336092in}{3.620155in}}{\pgfqpoint{4.340100in}{3.629831in}}{\pgfqpoint{4.340100in}{3.639918in}}%
\pgfpathcurveto{\pgfqpoint{4.340100in}{3.650005in}}{\pgfqpoint{4.336092in}{3.659681in}}{\pgfqpoint{4.328960in}{3.666814in}}%
\pgfpathcurveto{\pgfqpoint{4.321827in}{3.673947in}}{\pgfqpoint{4.312151in}{3.677954in}}{\pgfqpoint{4.302064in}{3.677954in}}%
\pgfpathcurveto{\pgfqpoint{4.291977in}{3.677954in}}{\pgfqpoint{4.282301in}{3.673947in}}{\pgfqpoint{4.275168in}{3.666814in}}%
\pgfpathcurveto{\pgfqpoint{4.268035in}{3.659681in}}{\pgfqpoint{4.264028in}{3.650005in}}{\pgfqpoint{4.264028in}{3.639918in}}%
\pgfpathcurveto{\pgfqpoint{4.264028in}{3.629831in}}{\pgfqpoint{4.268035in}{3.620155in}}{\pgfqpoint{4.275168in}{3.613022in}}%
\pgfpathcurveto{\pgfqpoint{4.282301in}{3.605890in}}{\pgfqpoint{4.291977in}{3.601882in}}{\pgfqpoint{4.302064in}{3.601882in}}%
\pgfpathclose%
\pgfusepath{stroke,fill}%
\end{pgfscope}%
\begin{pgfscope}%
\pgfpathrectangle{\pgfqpoint{0.800000in}{0.528000in}}{\pgfqpoint{4.960000in}{3.696000in}} %
\pgfusepath{clip}%
\pgfsetbuttcap%
\pgfsetroundjoin%
\definecolor{currentfill}{rgb}{0.121569,0.466667,0.705882}%
\pgfsetfillcolor{currentfill}%
\pgfsetlinewidth{1.003750pt}%
\definecolor{currentstroke}{rgb}{0.121569,0.466667,0.705882}%
\pgfsetstrokecolor{currentstroke}%
\pgfsetdash{}{0pt}%
\pgfpathmoveto{\pgfqpoint{1.328755in}{1.715602in}}%
\pgfpathcurveto{\pgfqpoint{1.338842in}{1.715602in}}{\pgfqpoint{1.348518in}{1.719610in}}{\pgfqpoint{1.355651in}{1.726743in}}%
\pgfpathcurveto{\pgfqpoint{1.362784in}{1.733876in}}{\pgfqpoint{1.366791in}{1.743551in}}{\pgfqpoint{1.366791in}{1.753638in}}%
\pgfpathcurveto{\pgfqpoint{1.366791in}{1.763726in}}{\pgfqpoint{1.362784in}{1.773401in}}{\pgfqpoint{1.355651in}{1.780534in}}%
\pgfpathcurveto{\pgfqpoint{1.348518in}{1.787667in}}{\pgfqpoint{1.338842in}{1.791675in}}{\pgfqpoint{1.328755in}{1.791675in}}%
\pgfpathcurveto{\pgfqpoint{1.318668in}{1.791675in}}{\pgfqpoint{1.308992in}{1.787667in}}{\pgfqpoint{1.301859in}{1.780534in}}%
\pgfpathcurveto{\pgfqpoint{1.294727in}{1.773401in}}{\pgfqpoint{1.290719in}{1.763726in}}{\pgfqpoint{1.290719in}{1.753638in}}%
\pgfpathcurveto{\pgfqpoint{1.290719in}{1.743551in}}{\pgfqpoint{1.294727in}{1.733876in}}{\pgfqpoint{1.301859in}{1.726743in}}%
\pgfpathcurveto{\pgfqpoint{1.308992in}{1.719610in}}{\pgfqpoint{1.318668in}{1.715602in}}{\pgfqpoint{1.328755in}{1.715602in}}%
\pgfpathclose%
\pgfusepath{stroke,fill}%
\end{pgfscope}%
\begin{pgfscope}%
\pgfpathrectangle{\pgfqpoint{0.800000in}{0.528000in}}{\pgfqpoint{4.960000in}{3.696000in}} %
\pgfusepath{clip}%
\pgfsetbuttcap%
\pgfsetroundjoin%
\definecolor{currentfill}{rgb}{0.121569,0.466667,0.705882}%
\pgfsetfillcolor{currentfill}%
\pgfsetlinewidth{1.003750pt}%
\definecolor{currentstroke}{rgb}{0.121569,0.466667,0.705882}%
\pgfsetstrokecolor{currentstroke}%
\pgfsetdash{}{0pt}%
\pgfpathmoveto{\pgfqpoint{2.645376in}{1.223143in}}%
\pgfpathcurveto{\pgfqpoint{2.655464in}{1.223143in}}{\pgfqpoint{2.665139in}{1.227151in}}{\pgfqpoint{2.672272in}{1.234284in}}%
\pgfpathcurveto{\pgfqpoint{2.679405in}{1.241417in}}{\pgfqpoint{2.683413in}{1.251092in}}{\pgfqpoint{2.683413in}{1.261179in}}%
\pgfpathcurveto{\pgfqpoint{2.683413in}{1.271267in}}{\pgfqpoint{2.679405in}{1.280942in}}{\pgfqpoint{2.672272in}{1.288075in}}%
\pgfpathcurveto{\pgfqpoint{2.665139in}{1.295208in}}{\pgfqpoint{2.655464in}{1.299216in}}{\pgfqpoint{2.645376in}{1.299216in}}%
\pgfpathcurveto{\pgfqpoint{2.635289in}{1.299216in}}{\pgfqpoint{2.625614in}{1.295208in}}{\pgfqpoint{2.618481in}{1.288075in}}%
\pgfpathcurveto{\pgfqpoint{2.611348in}{1.280942in}}{\pgfqpoint{2.607340in}{1.271267in}}{\pgfqpoint{2.607340in}{1.261179in}}%
\pgfpathcurveto{\pgfqpoint{2.607340in}{1.251092in}}{\pgfqpoint{2.611348in}{1.241417in}}{\pgfqpoint{2.618481in}{1.234284in}}%
\pgfpathcurveto{\pgfqpoint{2.625614in}{1.227151in}}{\pgfqpoint{2.635289in}{1.223143in}}{\pgfqpoint{2.645376in}{1.223143in}}%
\pgfpathclose%
\pgfusepath{stroke,fill}%
\end{pgfscope}%
\begin{pgfscope}%
\pgfpathrectangle{\pgfqpoint{0.800000in}{0.528000in}}{\pgfqpoint{4.960000in}{3.696000in}} %
\pgfusepath{clip}%
\pgfsetbuttcap%
\pgfsetroundjoin%
\definecolor{currentfill}{rgb}{0.121569,0.466667,0.705882}%
\pgfsetfillcolor{currentfill}%
\pgfsetlinewidth{1.003750pt}%
\definecolor{currentstroke}{rgb}{0.121569,0.466667,0.705882}%
\pgfsetstrokecolor{currentstroke}%
\pgfsetdash{}{0pt}%
\pgfpathmoveto{\pgfqpoint{4.888252in}{3.281835in}}%
\pgfpathcurveto{\pgfqpoint{4.898339in}{3.281835in}}{\pgfqpoint{4.908015in}{3.285843in}}{\pgfqpoint{4.915148in}{3.292976in}}%
\pgfpathcurveto{\pgfqpoint{4.922281in}{3.300109in}}{\pgfqpoint{4.926288in}{3.309784in}}{\pgfqpoint{4.926288in}{3.319872in}}%
\pgfpathcurveto{\pgfqpoint{4.926288in}{3.329959in}}{\pgfqpoint{4.922281in}{3.339634in}}{\pgfqpoint{4.915148in}{3.346767in}}%
\pgfpathcurveto{\pgfqpoint{4.908015in}{3.353900in}}{\pgfqpoint{4.898339in}{3.357908in}}{\pgfqpoint{4.888252in}{3.357908in}}%
\pgfpathcurveto{\pgfqpoint{4.878165in}{3.357908in}}{\pgfqpoint{4.868489in}{3.353900in}}{\pgfqpoint{4.861356in}{3.346767in}}%
\pgfpathcurveto{\pgfqpoint{4.854224in}{3.339634in}}{\pgfqpoint{4.850216in}{3.329959in}}{\pgfqpoint{4.850216in}{3.319872in}}%
\pgfpathcurveto{\pgfqpoint{4.850216in}{3.309784in}}{\pgfqpoint{4.854224in}{3.300109in}}{\pgfqpoint{4.861356in}{3.292976in}}%
\pgfpathcurveto{\pgfqpoint{4.868489in}{3.285843in}}{\pgfqpoint{4.878165in}{3.281835in}}{\pgfqpoint{4.888252in}{3.281835in}}%
\pgfpathclose%
\pgfusepath{stroke,fill}%
\end{pgfscope}%
\begin{pgfscope}%
\pgfpathrectangle{\pgfqpoint{0.800000in}{0.528000in}}{\pgfqpoint{4.960000in}{3.696000in}} %
\pgfusepath{clip}%
\pgfsetbuttcap%
\pgfsetroundjoin%
\definecolor{currentfill}{rgb}{0.121569,0.466667,0.705882}%
\pgfsetfillcolor{currentfill}%
\pgfsetlinewidth{1.003750pt}%
\definecolor{currentstroke}{rgb}{0.121569,0.466667,0.705882}%
\pgfsetstrokecolor{currentstroke}%
\pgfsetdash{}{0pt}%
\pgfpathmoveto{\pgfqpoint{1.392067in}{3.043035in}}%
\pgfpathcurveto{\pgfqpoint{1.402154in}{3.043035in}}{\pgfqpoint{1.411830in}{3.047042in}}{\pgfqpoint{1.418963in}{3.054175in}}%
\pgfpathcurveto{\pgfqpoint{1.426095in}{3.061308in}}{\pgfqpoint{1.430103in}{3.070984in}}{\pgfqpoint{1.430103in}{3.081071in}}%
\pgfpathcurveto{\pgfqpoint{1.430103in}{3.091158in}}{\pgfqpoint{1.426095in}{3.100834in}}{\pgfqpoint{1.418963in}{3.107967in}}%
\pgfpathcurveto{\pgfqpoint{1.411830in}{3.115099in}}{\pgfqpoint{1.402154in}{3.119107in}}{\pgfqpoint{1.392067in}{3.119107in}}%
\pgfpathcurveto{\pgfqpoint{1.381979in}{3.119107in}}{\pgfqpoint{1.372304in}{3.115099in}}{\pgfqpoint{1.365171in}{3.107967in}}%
\pgfpathcurveto{\pgfqpoint{1.358038in}{3.100834in}}{\pgfqpoint{1.354031in}{3.091158in}}{\pgfqpoint{1.354031in}{3.081071in}}%
\pgfpathcurveto{\pgfqpoint{1.354031in}{3.070984in}}{\pgfqpoint{1.358038in}{3.061308in}}{\pgfqpoint{1.365171in}{3.054175in}}%
\pgfpathcurveto{\pgfqpoint{1.372304in}{3.047042in}}{\pgfqpoint{1.381979in}{3.043035in}}{\pgfqpoint{1.392067in}{3.043035in}}%
\pgfpathclose%
\pgfusepath{stroke,fill}%
\end{pgfscope}%
\begin{pgfscope}%
\pgfpathrectangle{\pgfqpoint{0.800000in}{0.528000in}}{\pgfqpoint{4.960000in}{3.696000in}} %
\pgfusepath{clip}%
\pgfsetbuttcap%
\pgfsetroundjoin%
\definecolor{currentfill}{rgb}{0.121569,0.466667,0.705882}%
\pgfsetfillcolor{currentfill}%
\pgfsetlinewidth{1.003750pt}%
\definecolor{currentstroke}{rgb}{0.121569,0.466667,0.705882}%
\pgfsetstrokecolor{currentstroke}%
\pgfsetdash{}{0pt}%
\pgfpathmoveto{\pgfqpoint{2.577886in}{0.933110in}}%
\pgfpathcurveto{\pgfqpoint{2.587974in}{0.933110in}}{\pgfqpoint{2.597649in}{0.937118in}}{\pgfqpoint{2.604782in}{0.944251in}}%
\pgfpathcurveto{\pgfqpoint{2.611915in}{0.951384in}}{\pgfqpoint{2.615923in}{0.961059in}}{\pgfqpoint{2.615923in}{0.971147in}}%
\pgfpathcurveto{\pgfqpoint{2.615923in}{0.981234in}}{\pgfqpoint{2.611915in}{0.990910in}}{\pgfqpoint{2.604782in}{0.998042in}}%
\pgfpathcurveto{\pgfqpoint{2.597649in}{1.005175in}}{\pgfqpoint{2.587974in}{1.009183in}}{\pgfqpoint{2.577886in}{1.009183in}}%
\pgfpathcurveto{\pgfqpoint{2.567799in}{1.009183in}}{\pgfqpoint{2.558123in}{1.005175in}}{\pgfqpoint{2.550991in}{0.998042in}}%
\pgfpathcurveto{\pgfqpoint{2.543858in}{0.990910in}}{\pgfqpoint{2.539850in}{0.981234in}}{\pgfqpoint{2.539850in}{0.971147in}}%
\pgfpathcurveto{\pgfqpoint{2.539850in}{0.961059in}}{\pgfqpoint{2.543858in}{0.951384in}}{\pgfqpoint{2.550991in}{0.944251in}}%
\pgfpathcurveto{\pgfqpoint{2.558123in}{0.937118in}}{\pgfqpoint{2.567799in}{0.933110in}}{\pgfqpoint{2.577886in}{0.933110in}}%
\pgfpathclose%
\pgfusepath{stroke,fill}%
\end{pgfscope}%
\begin{pgfscope}%
\pgfpathrectangle{\pgfqpoint{0.800000in}{0.528000in}}{\pgfqpoint{4.960000in}{3.696000in}} %
\pgfusepath{clip}%
\pgfsetbuttcap%
\pgfsetroundjoin%
\definecolor{currentfill}{rgb}{0.121569,0.466667,0.705882}%
\pgfsetfillcolor{currentfill}%
\pgfsetlinewidth{1.003750pt}%
\definecolor{currentstroke}{rgb}{0.121569,0.466667,0.705882}%
\pgfsetstrokecolor{currentstroke}%
\pgfsetdash{}{0pt}%
\pgfpathmoveto{\pgfqpoint{1.318044in}{3.100440in}}%
\pgfpathcurveto{\pgfqpoint{1.328131in}{3.100440in}}{\pgfqpoint{1.337807in}{3.104447in}}{\pgfqpoint{1.344939in}{3.111580in}}%
\pgfpathcurveto{\pgfqpoint{1.352072in}{3.118713in}}{\pgfqpoint{1.356080in}{3.128389in}}{\pgfqpoint{1.356080in}{3.138476in}}%
\pgfpathcurveto{\pgfqpoint{1.356080in}{3.148563in}}{\pgfqpoint{1.352072in}{3.158239in}}{\pgfqpoint{1.344939in}{3.165372in}}%
\pgfpathcurveto{\pgfqpoint{1.337807in}{3.172505in}}{\pgfqpoint{1.328131in}{3.176512in}}{\pgfqpoint{1.318044in}{3.176512in}}%
\pgfpathcurveto{\pgfqpoint{1.307956in}{3.176512in}}{\pgfqpoint{1.298281in}{3.172505in}}{\pgfqpoint{1.291148in}{3.165372in}}%
\pgfpathcurveto{\pgfqpoint{1.284015in}{3.158239in}}{\pgfqpoint{1.280007in}{3.148563in}}{\pgfqpoint{1.280007in}{3.138476in}}%
\pgfpathcurveto{\pgfqpoint{1.280007in}{3.128389in}}{\pgfqpoint{1.284015in}{3.118713in}}{\pgfqpoint{1.291148in}{3.111580in}}%
\pgfpathcurveto{\pgfqpoint{1.298281in}{3.104447in}}{\pgfqpoint{1.307956in}{3.100440in}}{\pgfqpoint{1.318044in}{3.100440in}}%
\pgfpathclose%
\pgfusepath{stroke,fill}%
\end{pgfscope}%
\begin{pgfscope}%
\pgfpathrectangle{\pgfqpoint{0.800000in}{0.528000in}}{\pgfqpoint{4.960000in}{3.696000in}} %
\pgfusepath{clip}%
\pgfsetbuttcap%
\pgfsetroundjoin%
\definecolor{currentfill}{rgb}{0.121569,0.466667,0.705882}%
\pgfsetfillcolor{currentfill}%
\pgfsetlinewidth{1.003750pt}%
\definecolor{currentstroke}{rgb}{0.121569,0.466667,0.705882}%
\pgfsetstrokecolor{currentstroke}%
\pgfsetdash{}{0pt}%
\pgfpathmoveto{\pgfqpoint{5.189473in}{2.698805in}}%
\pgfpathcurveto{\pgfqpoint{5.199560in}{2.698805in}}{\pgfqpoint{5.209236in}{2.702813in}}{\pgfqpoint{5.216368in}{2.709946in}}%
\pgfpathcurveto{\pgfqpoint{5.223501in}{2.717078in}}{\pgfqpoint{5.227509in}{2.726754in}}{\pgfqpoint{5.227509in}{2.736841in}}%
\pgfpathcurveto{\pgfqpoint{5.227509in}{2.746929in}}{\pgfqpoint{5.223501in}{2.756604in}}{\pgfqpoint{5.216368in}{2.763737in}}%
\pgfpathcurveto{\pgfqpoint{5.209236in}{2.770870in}}{\pgfqpoint{5.199560in}{2.774878in}}{\pgfqpoint{5.189473in}{2.774878in}}%
\pgfpathcurveto{\pgfqpoint{5.179385in}{2.774878in}}{\pgfqpoint{5.169710in}{2.770870in}}{\pgfqpoint{5.162577in}{2.763737in}}%
\pgfpathcurveto{\pgfqpoint{5.155444in}{2.756604in}}{\pgfqpoint{5.151436in}{2.746929in}}{\pgfqpoint{5.151436in}{2.736841in}}%
\pgfpathcurveto{\pgfqpoint{5.151436in}{2.726754in}}{\pgfqpoint{5.155444in}{2.717078in}}{\pgfqpoint{5.162577in}{2.709946in}}%
\pgfpathcurveto{\pgfqpoint{5.169710in}{2.702813in}}{\pgfqpoint{5.179385in}{2.698805in}}{\pgfqpoint{5.189473in}{2.698805in}}%
\pgfpathclose%
\pgfusepath{stroke,fill}%
\end{pgfscope}%
\begin{pgfscope}%
\pgfpathrectangle{\pgfqpoint{0.800000in}{0.528000in}}{\pgfqpoint{4.960000in}{3.696000in}} %
\pgfusepath{clip}%
\pgfsetbuttcap%
\pgfsetroundjoin%
\definecolor{currentfill}{rgb}{0.121569,0.466667,0.705882}%
\pgfsetfillcolor{currentfill}%
\pgfsetlinewidth{1.003750pt}%
\definecolor{currentstroke}{rgb}{0.121569,0.466667,0.705882}%
\pgfsetstrokecolor{currentstroke}%
\pgfsetdash{}{0pt}%
\pgfpathmoveto{\pgfqpoint{1.253700in}{2.181338in}}%
\pgfpathcurveto{\pgfqpoint{1.263788in}{2.181338in}}{\pgfqpoint{1.273463in}{2.185346in}}{\pgfqpoint{1.280596in}{2.192478in}}%
\pgfpathcurveto{\pgfqpoint{1.287729in}{2.199611in}}{\pgfqpoint{1.291737in}{2.209287in}}{\pgfqpoint{1.291737in}{2.219374in}}%
\pgfpathcurveto{\pgfqpoint{1.291737in}{2.229461in}}{\pgfqpoint{1.287729in}{2.239137in}}{\pgfqpoint{1.280596in}{2.246270in}}%
\pgfpathcurveto{\pgfqpoint{1.273463in}{2.253403in}}{\pgfqpoint{1.263788in}{2.257410in}}{\pgfqpoint{1.253700in}{2.257410in}}%
\pgfpathcurveto{\pgfqpoint{1.243613in}{2.257410in}}{\pgfqpoint{1.233937in}{2.253403in}}{\pgfqpoint{1.226805in}{2.246270in}}%
\pgfpathcurveto{\pgfqpoint{1.219672in}{2.239137in}}{\pgfqpoint{1.215664in}{2.229461in}}{\pgfqpoint{1.215664in}{2.219374in}}%
\pgfpathcurveto{\pgfqpoint{1.215664in}{2.209287in}}{\pgfqpoint{1.219672in}{2.199611in}}{\pgfqpoint{1.226805in}{2.192478in}}%
\pgfpathcurveto{\pgfqpoint{1.233937in}{2.185346in}}{\pgfqpoint{1.243613in}{2.181338in}}{\pgfqpoint{1.253700in}{2.181338in}}%
\pgfpathclose%
\pgfusepath{stroke,fill}%
\end{pgfscope}%
\begin{pgfscope}%
\pgfpathrectangle{\pgfqpoint{0.800000in}{0.528000in}}{\pgfqpoint{4.960000in}{3.696000in}} %
\pgfusepath{clip}%
\pgfsetbuttcap%
\pgfsetroundjoin%
\definecolor{currentfill}{rgb}{0.121569,0.466667,0.705882}%
\pgfsetfillcolor{currentfill}%
\pgfsetlinewidth{1.003750pt}%
\definecolor{currentstroke}{rgb}{0.121569,0.466667,0.705882}%
\pgfsetstrokecolor{currentstroke}%
\pgfsetdash{}{0pt}%
\pgfpathmoveto{\pgfqpoint{2.133911in}{1.123871in}}%
\pgfpathcurveto{\pgfqpoint{2.143999in}{1.123871in}}{\pgfqpoint{2.153674in}{1.127878in}}{\pgfqpoint{2.160807in}{1.135011in}}%
\pgfpathcurveto{\pgfqpoint{2.167940in}{1.142144in}}{\pgfqpoint{2.171948in}{1.151820in}}{\pgfqpoint{2.171948in}{1.161907in}}%
\pgfpathcurveto{\pgfqpoint{2.171948in}{1.171994in}}{\pgfqpoint{2.167940in}{1.181670in}}{\pgfqpoint{2.160807in}{1.188803in}}%
\pgfpathcurveto{\pgfqpoint{2.153674in}{1.195935in}}{\pgfqpoint{2.143999in}{1.199943in}}{\pgfqpoint{2.133911in}{1.199943in}}%
\pgfpathcurveto{\pgfqpoint{2.123824in}{1.199943in}}{\pgfqpoint{2.114148in}{1.195935in}}{\pgfqpoint{2.107016in}{1.188803in}}%
\pgfpathcurveto{\pgfqpoint{2.099883in}{1.181670in}}{\pgfqpoint{2.095875in}{1.171994in}}{\pgfqpoint{2.095875in}{1.161907in}}%
\pgfpathcurveto{\pgfqpoint{2.095875in}{1.151820in}}{\pgfqpoint{2.099883in}{1.142144in}}{\pgfqpoint{2.107016in}{1.135011in}}%
\pgfpathcurveto{\pgfqpoint{2.114148in}{1.127878in}}{\pgfqpoint{2.123824in}{1.123871in}}{\pgfqpoint{2.133911in}{1.123871in}}%
\pgfpathclose%
\pgfusepath{stroke,fill}%
\end{pgfscope}%
\begin{pgfscope}%
\pgfpathrectangle{\pgfqpoint{0.800000in}{0.528000in}}{\pgfqpoint{4.960000in}{3.696000in}} %
\pgfusepath{clip}%
\pgfsetbuttcap%
\pgfsetroundjoin%
\definecolor{currentfill}{rgb}{0.121569,0.466667,0.705882}%
\pgfsetfillcolor{currentfill}%
\pgfsetlinewidth{1.003750pt}%
\definecolor{currentstroke}{rgb}{0.121569,0.466667,0.705882}%
\pgfsetstrokecolor{currentstroke}%
\pgfsetdash{}{0pt}%
\pgfpathmoveto{\pgfqpoint{1.191177in}{2.253931in}}%
\pgfpathcurveto{\pgfqpoint{1.201264in}{2.253931in}}{\pgfqpoint{1.210940in}{2.257939in}}{\pgfqpoint{1.218072in}{2.265072in}}%
\pgfpathcurveto{\pgfqpoint{1.225205in}{2.272205in}}{\pgfqpoint{1.229213in}{2.281880in}}{\pgfqpoint{1.229213in}{2.291968in}}%
\pgfpathcurveto{\pgfqpoint{1.229213in}{2.302055in}}{\pgfqpoint{1.225205in}{2.311731in}}{\pgfqpoint{1.218072in}{2.318863in}}%
\pgfpathcurveto{\pgfqpoint{1.210940in}{2.325996in}}{\pgfqpoint{1.201264in}{2.330004in}}{\pgfqpoint{1.191177in}{2.330004in}}%
\pgfpathcurveto{\pgfqpoint{1.181089in}{2.330004in}}{\pgfqpoint{1.171414in}{2.325996in}}{\pgfqpoint{1.164281in}{2.318863in}}%
\pgfpathcurveto{\pgfqpoint{1.157148in}{2.311731in}}{\pgfqpoint{1.153140in}{2.302055in}}{\pgfqpoint{1.153140in}{2.291968in}}%
\pgfpathcurveto{\pgfqpoint{1.153140in}{2.281880in}}{\pgfqpoint{1.157148in}{2.272205in}}{\pgfqpoint{1.164281in}{2.265072in}}%
\pgfpathcurveto{\pgfqpoint{1.171414in}{2.257939in}}{\pgfqpoint{1.181089in}{2.253931in}}{\pgfqpoint{1.191177in}{2.253931in}}%
\pgfpathclose%
\pgfusepath{stroke,fill}%
\end{pgfscope}%
\begin{pgfscope}%
\pgfpathrectangle{\pgfqpoint{0.800000in}{0.528000in}}{\pgfqpoint{4.960000in}{3.696000in}} %
\pgfusepath{clip}%
\pgfsetbuttcap%
\pgfsetroundjoin%
\definecolor{currentfill}{rgb}{0.121569,0.466667,0.705882}%
\pgfsetfillcolor{currentfill}%
\pgfsetlinewidth{1.003750pt}%
\definecolor{currentstroke}{rgb}{0.121569,0.466667,0.705882}%
\pgfsetstrokecolor{currentstroke}%
\pgfsetdash{}{0pt}%
\pgfpathmoveto{\pgfqpoint{5.176761in}{1.925308in}}%
\pgfpathcurveto{\pgfqpoint{5.186849in}{1.925308in}}{\pgfqpoint{5.196524in}{1.929316in}}{\pgfqpoint{5.203657in}{1.936449in}}%
\pgfpathcurveto{\pgfqpoint{5.210790in}{1.943581in}}{\pgfqpoint{5.214798in}{1.953257in}}{\pgfqpoint{5.214798in}{1.963344in}}%
\pgfpathcurveto{\pgfqpoint{5.214798in}{1.973432in}}{\pgfqpoint{5.210790in}{1.983107in}}{\pgfqpoint{5.203657in}{1.990240in}}%
\pgfpathcurveto{\pgfqpoint{5.196524in}{1.997373in}}{\pgfqpoint{5.186849in}{2.001381in}}{\pgfqpoint{5.176761in}{2.001381in}}%
\pgfpathcurveto{\pgfqpoint{5.166674in}{2.001381in}}{\pgfqpoint{5.156999in}{1.997373in}}{\pgfqpoint{5.149866in}{1.990240in}}%
\pgfpathcurveto{\pgfqpoint{5.142733in}{1.983107in}}{\pgfqpoint{5.138725in}{1.973432in}}{\pgfqpoint{5.138725in}{1.963344in}}%
\pgfpathcurveto{\pgfqpoint{5.138725in}{1.953257in}}{\pgfqpoint{5.142733in}{1.943581in}}{\pgfqpoint{5.149866in}{1.936449in}}%
\pgfpathcurveto{\pgfqpoint{5.156999in}{1.929316in}}{\pgfqpoint{5.166674in}{1.925308in}}{\pgfqpoint{5.176761in}{1.925308in}}%
\pgfpathclose%
\pgfusepath{stroke,fill}%
\end{pgfscope}%
\begin{pgfscope}%
\pgfpathrectangle{\pgfqpoint{0.800000in}{0.528000in}}{\pgfqpoint{4.960000in}{3.696000in}} %
\pgfusepath{clip}%
\pgfsetbuttcap%
\pgfsetroundjoin%
\definecolor{currentfill}{rgb}{0.121569,0.466667,0.705882}%
\pgfsetfillcolor{currentfill}%
\pgfsetlinewidth{1.003750pt}%
\definecolor{currentstroke}{rgb}{0.121569,0.466667,0.705882}%
\pgfsetstrokecolor{currentstroke}%
\pgfsetdash{}{0pt}%
\pgfpathmoveto{\pgfqpoint{1.589196in}{1.525311in}}%
\pgfpathcurveto{\pgfqpoint{1.599283in}{1.525311in}}{\pgfqpoint{1.608959in}{1.529318in}}{\pgfqpoint{1.616092in}{1.536451in}}%
\pgfpathcurveto{\pgfqpoint{1.623225in}{1.543584in}}{\pgfqpoint{1.627232in}{1.553260in}}{\pgfqpoint{1.627232in}{1.563347in}}%
\pgfpathcurveto{\pgfqpoint{1.627232in}{1.573434in}}{\pgfqpoint{1.623225in}{1.583110in}}{\pgfqpoint{1.616092in}{1.590243in}}%
\pgfpathcurveto{\pgfqpoint{1.608959in}{1.597376in}}{\pgfqpoint{1.599283in}{1.601383in}}{\pgfqpoint{1.589196in}{1.601383in}}%
\pgfpathcurveto{\pgfqpoint{1.579109in}{1.601383in}}{\pgfqpoint{1.569433in}{1.597376in}}{\pgfqpoint{1.562300in}{1.590243in}}%
\pgfpathcurveto{\pgfqpoint{1.555167in}{1.583110in}}{\pgfqpoint{1.551160in}{1.573434in}}{\pgfqpoint{1.551160in}{1.563347in}}%
\pgfpathcurveto{\pgfqpoint{1.551160in}{1.553260in}}{\pgfqpoint{1.555167in}{1.543584in}}{\pgfqpoint{1.562300in}{1.536451in}}%
\pgfpathcurveto{\pgfqpoint{1.569433in}{1.529318in}}{\pgfqpoint{1.579109in}{1.525311in}}{\pgfqpoint{1.589196in}{1.525311in}}%
\pgfpathclose%
\pgfusepath{stroke,fill}%
\end{pgfscope}%
\begin{pgfscope}%
\pgfpathrectangle{\pgfqpoint{0.800000in}{0.528000in}}{\pgfqpoint{4.960000in}{3.696000in}} %
\pgfusepath{clip}%
\pgfsetbuttcap%
\pgfsetroundjoin%
\definecolor{currentfill}{rgb}{0.121569,0.466667,0.705882}%
\pgfsetfillcolor{currentfill}%
\pgfsetlinewidth{1.003750pt}%
\definecolor{currentstroke}{rgb}{0.121569,0.466667,0.705882}%
\pgfsetstrokecolor{currentstroke}%
\pgfsetdash{}{0pt}%
\pgfpathmoveto{\pgfqpoint{4.893154in}{1.503368in}}%
\pgfpathcurveto{\pgfqpoint{4.903241in}{1.503368in}}{\pgfqpoint{4.912917in}{1.507376in}}{\pgfqpoint{4.920050in}{1.514509in}}%
\pgfpathcurveto{\pgfqpoint{4.927183in}{1.521642in}}{\pgfqpoint{4.931190in}{1.531317in}}{\pgfqpoint{4.931190in}{1.541404in}}%
\pgfpathcurveto{\pgfqpoint{4.931190in}{1.551492in}}{\pgfqpoint{4.927183in}{1.561167in}}{\pgfqpoint{4.920050in}{1.568300in}}%
\pgfpathcurveto{\pgfqpoint{4.912917in}{1.575433in}}{\pgfqpoint{4.903241in}{1.579441in}}{\pgfqpoint{4.893154in}{1.579441in}}%
\pgfpathcurveto{\pgfqpoint{4.883067in}{1.579441in}}{\pgfqpoint{4.873391in}{1.575433in}}{\pgfqpoint{4.866258in}{1.568300in}}%
\pgfpathcurveto{\pgfqpoint{4.859126in}{1.561167in}}{\pgfqpoint{4.855118in}{1.551492in}}{\pgfqpoint{4.855118in}{1.541404in}}%
\pgfpathcurveto{\pgfqpoint{4.855118in}{1.531317in}}{\pgfqpoint{4.859126in}{1.521642in}}{\pgfqpoint{4.866258in}{1.514509in}}%
\pgfpathcurveto{\pgfqpoint{4.873391in}{1.507376in}}{\pgfqpoint{4.883067in}{1.503368in}}{\pgfqpoint{4.893154in}{1.503368in}}%
\pgfpathclose%
\pgfusepath{stroke,fill}%
\end{pgfscope}%
\begin{pgfscope}%
\pgfpathrectangle{\pgfqpoint{0.800000in}{0.528000in}}{\pgfqpoint{4.960000in}{3.696000in}} %
\pgfusepath{clip}%
\pgfsetbuttcap%
\pgfsetroundjoin%
\definecolor{currentfill}{rgb}{0.121569,0.466667,0.705882}%
\pgfsetfillcolor{currentfill}%
\pgfsetlinewidth{1.003750pt}%
\definecolor{currentstroke}{rgb}{0.121569,0.466667,0.705882}%
\pgfsetstrokecolor{currentstroke}%
\pgfsetdash{}{0pt}%
\pgfpathmoveto{\pgfqpoint{4.496385in}{3.597481in}}%
\pgfpathcurveto{\pgfqpoint{4.506472in}{3.597481in}}{\pgfqpoint{4.516147in}{3.601489in}}{\pgfqpoint{4.523280in}{3.608622in}}%
\pgfpathcurveto{\pgfqpoint{4.530413in}{3.615755in}}{\pgfqpoint{4.534421in}{3.625430in}}{\pgfqpoint{4.534421in}{3.635517in}}%
\pgfpathcurveto{\pgfqpoint{4.534421in}{3.645605in}}{\pgfqpoint{4.530413in}{3.655280in}}{\pgfqpoint{4.523280in}{3.662413in}}%
\pgfpathcurveto{\pgfqpoint{4.516147in}{3.669546in}}{\pgfqpoint{4.506472in}{3.673554in}}{\pgfqpoint{4.496385in}{3.673554in}}%
\pgfpathcurveto{\pgfqpoint{4.486297in}{3.673554in}}{\pgfqpoint{4.476622in}{3.669546in}}{\pgfqpoint{4.469489in}{3.662413in}}%
\pgfpathcurveto{\pgfqpoint{4.462356in}{3.655280in}}{\pgfqpoint{4.458348in}{3.645605in}}{\pgfqpoint{4.458348in}{3.635517in}}%
\pgfpathcurveto{\pgfqpoint{4.458348in}{3.625430in}}{\pgfqpoint{4.462356in}{3.615755in}}{\pgfqpoint{4.469489in}{3.608622in}}%
\pgfpathcurveto{\pgfqpoint{4.476622in}{3.601489in}}{\pgfqpoint{4.486297in}{3.597481in}}{\pgfqpoint{4.496385in}{3.597481in}}%
\pgfpathclose%
\pgfusepath{stroke,fill}%
\end{pgfscope}%
\begin{pgfscope}%
\pgfpathrectangle{\pgfqpoint{0.800000in}{0.528000in}}{\pgfqpoint{4.960000in}{3.696000in}} %
\pgfusepath{clip}%
\pgfsetbuttcap%
\pgfsetroundjoin%
\definecolor{currentfill}{rgb}{0.121569,0.466667,0.705882}%
\pgfsetfillcolor{currentfill}%
\pgfsetlinewidth{1.003750pt}%
\definecolor{currentstroke}{rgb}{0.121569,0.466667,0.705882}%
\pgfsetstrokecolor{currentstroke}%
\pgfsetdash{}{0pt}%
\pgfpathmoveto{\pgfqpoint{4.470768in}{3.602330in}}%
\pgfpathcurveto{\pgfqpoint{4.480855in}{3.602330in}}{\pgfqpoint{4.490530in}{3.606337in}}{\pgfqpoint{4.497663in}{3.613470in}}%
\pgfpathcurveto{\pgfqpoint{4.504796in}{3.620603in}}{\pgfqpoint{4.508804in}{3.630279in}}{\pgfqpoint{4.508804in}{3.640366in}}%
\pgfpathcurveto{\pgfqpoint{4.508804in}{3.650453in}}{\pgfqpoint{4.504796in}{3.660129in}}{\pgfqpoint{4.497663in}{3.667262in}}%
\pgfpathcurveto{\pgfqpoint{4.490530in}{3.674394in}}{\pgfqpoint{4.480855in}{3.678402in}}{\pgfqpoint{4.470768in}{3.678402in}}%
\pgfpathcurveto{\pgfqpoint{4.460680in}{3.678402in}}{\pgfqpoint{4.451005in}{3.674394in}}{\pgfqpoint{4.443872in}{3.667262in}}%
\pgfpathcurveto{\pgfqpoint{4.436739in}{3.660129in}}{\pgfqpoint{4.432731in}{3.650453in}}{\pgfqpoint{4.432731in}{3.640366in}}%
\pgfpathcurveto{\pgfqpoint{4.432731in}{3.630279in}}{\pgfqpoint{4.436739in}{3.620603in}}{\pgfqpoint{4.443872in}{3.613470in}}%
\pgfpathcurveto{\pgfqpoint{4.451005in}{3.606337in}}{\pgfqpoint{4.460680in}{3.602330in}}{\pgfqpoint{4.470768in}{3.602330in}}%
\pgfpathclose%
\pgfusepath{stroke,fill}%
\end{pgfscope}%
\begin{pgfscope}%
\pgfpathrectangle{\pgfqpoint{0.800000in}{0.528000in}}{\pgfqpoint{4.960000in}{3.696000in}} %
\pgfusepath{clip}%
\pgfsetbuttcap%
\pgfsetroundjoin%
\definecolor{currentfill}{rgb}{0.121569,0.466667,0.705882}%
\pgfsetfillcolor{currentfill}%
\pgfsetlinewidth{1.003750pt}%
\definecolor{currentstroke}{rgb}{0.121569,0.466667,0.705882}%
\pgfsetstrokecolor{currentstroke}%
\pgfsetdash{}{0pt}%
\pgfpathmoveto{\pgfqpoint{3.973908in}{1.014848in}}%
\pgfpathcurveto{\pgfqpoint{3.983995in}{1.014848in}}{\pgfqpoint{3.993670in}{1.018855in}}{\pgfqpoint{4.000803in}{1.025988in}}%
\pgfpathcurveto{\pgfqpoint{4.007936in}{1.033121in}}{\pgfqpoint{4.011944in}{1.042797in}}{\pgfqpoint{4.011944in}{1.052884in}}%
\pgfpathcurveto{\pgfqpoint{4.011944in}{1.062971in}}{\pgfqpoint{4.007936in}{1.072647in}}{\pgfqpoint{4.000803in}{1.079780in}}%
\pgfpathcurveto{\pgfqpoint{3.993670in}{1.086913in}}{\pgfqpoint{3.983995in}{1.090920in}}{\pgfqpoint{3.973908in}{1.090920in}}%
\pgfpathcurveto{\pgfqpoint{3.963820in}{1.090920in}}{\pgfqpoint{3.954145in}{1.086913in}}{\pgfqpoint{3.947012in}{1.079780in}}%
\pgfpathcurveto{\pgfqpoint{3.939879in}{1.072647in}}{\pgfqpoint{3.935871in}{1.062971in}}{\pgfqpoint{3.935871in}{1.052884in}}%
\pgfpathcurveto{\pgfqpoint{3.935871in}{1.042797in}}{\pgfqpoint{3.939879in}{1.033121in}}{\pgfqpoint{3.947012in}{1.025988in}}%
\pgfpathcurveto{\pgfqpoint{3.954145in}{1.018855in}}{\pgfqpoint{3.963820in}{1.014848in}}{\pgfqpoint{3.973908in}{1.014848in}}%
\pgfpathclose%
\pgfusepath{stroke,fill}%
\end{pgfscope}%
\begin{pgfscope}%
\pgfpathrectangle{\pgfqpoint{0.800000in}{0.528000in}}{\pgfqpoint{4.960000in}{3.696000in}} %
\pgfusepath{clip}%
\pgfsetbuttcap%
\pgfsetroundjoin%
\definecolor{currentfill}{rgb}{0.121569,0.466667,0.705882}%
\pgfsetfillcolor{currentfill}%
\pgfsetlinewidth{1.003750pt}%
\definecolor{currentstroke}{rgb}{0.121569,0.466667,0.705882}%
\pgfsetstrokecolor{currentstroke}%
\pgfsetdash{}{0pt}%
\pgfpathmoveto{\pgfqpoint{1.626489in}{3.480419in}}%
\pgfpathcurveto{\pgfqpoint{1.636576in}{3.480419in}}{\pgfqpoint{1.646252in}{3.484426in}}{\pgfqpoint{1.653384in}{3.491559in}}%
\pgfpathcurveto{\pgfqpoint{1.660517in}{3.498692in}}{\pgfqpoint{1.664525in}{3.508368in}}{\pgfqpoint{1.664525in}{3.518455in}}%
\pgfpathcurveto{\pgfqpoint{1.664525in}{3.528542in}}{\pgfqpoint{1.660517in}{3.538218in}}{\pgfqpoint{1.653384in}{3.545351in}}%
\pgfpathcurveto{\pgfqpoint{1.646252in}{3.552484in}}{\pgfqpoint{1.636576in}{3.556491in}}{\pgfqpoint{1.626489in}{3.556491in}}%
\pgfpathcurveto{\pgfqpoint{1.616401in}{3.556491in}}{\pgfqpoint{1.606726in}{3.552484in}}{\pgfqpoint{1.599593in}{3.545351in}}%
\pgfpathcurveto{\pgfqpoint{1.592460in}{3.538218in}}{\pgfqpoint{1.588452in}{3.528542in}}{\pgfqpoint{1.588452in}{3.518455in}}%
\pgfpathcurveto{\pgfqpoint{1.588452in}{3.508368in}}{\pgfqpoint{1.592460in}{3.498692in}}{\pgfqpoint{1.599593in}{3.491559in}}%
\pgfpathcurveto{\pgfqpoint{1.606726in}{3.484426in}}{\pgfqpoint{1.616401in}{3.480419in}}{\pgfqpoint{1.626489in}{3.480419in}}%
\pgfpathclose%
\pgfusepath{stroke,fill}%
\end{pgfscope}%
\begin{pgfscope}%
\pgfpathrectangle{\pgfqpoint{0.800000in}{0.528000in}}{\pgfqpoint{4.960000in}{3.696000in}} %
\pgfusepath{clip}%
\pgfsetbuttcap%
\pgfsetroundjoin%
\definecolor{currentfill}{rgb}{0.121569,0.466667,0.705882}%
\pgfsetfillcolor{currentfill}%
\pgfsetlinewidth{1.003750pt}%
\definecolor{currentstroke}{rgb}{0.121569,0.466667,0.705882}%
\pgfsetstrokecolor{currentstroke}%
\pgfsetdash{}{0pt}%
\pgfpathmoveto{\pgfqpoint{4.289740in}{3.541511in}}%
\pgfpathcurveto{\pgfqpoint{4.299828in}{3.541511in}}{\pgfqpoint{4.309503in}{3.545519in}}{\pgfqpoint{4.316636in}{3.552651in}}%
\pgfpathcurveto{\pgfqpoint{4.323769in}{3.559784in}}{\pgfqpoint{4.327776in}{3.569460in}}{\pgfqpoint{4.327776in}{3.579547in}}%
\pgfpathcurveto{\pgfqpoint{4.327776in}{3.589635in}}{\pgfqpoint{4.323769in}{3.599310in}}{\pgfqpoint{4.316636in}{3.606443in}}%
\pgfpathcurveto{\pgfqpoint{4.309503in}{3.613576in}}{\pgfqpoint{4.299828in}{3.617584in}}{\pgfqpoint{4.289740in}{3.617584in}}%
\pgfpathcurveto{\pgfqpoint{4.279653in}{3.617584in}}{\pgfqpoint{4.269977in}{3.613576in}}{\pgfqpoint{4.262844in}{3.606443in}}%
\pgfpathcurveto{\pgfqpoint{4.255712in}{3.599310in}}{\pgfqpoint{4.251704in}{3.589635in}}{\pgfqpoint{4.251704in}{3.579547in}}%
\pgfpathcurveto{\pgfqpoint{4.251704in}{3.569460in}}{\pgfqpoint{4.255712in}{3.559784in}}{\pgfqpoint{4.262844in}{3.552651in}}%
\pgfpathcurveto{\pgfqpoint{4.269977in}{3.545519in}}{\pgfqpoint{4.279653in}{3.541511in}}{\pgfqpoint{4.289740in}{3.541511in}}%
\pgfpathclose%
\pgfusepath{stroke,fill}%
\end{pgfscope}%
\begin{pgfscope}%
\pgfpathrectangle{\pgfqpoint{0.800000in}{0.528000in}}{\pgfqpoint{4.960000in}{3.696000in}} %
\pgfusepath{clip}%
\pgfsetbuttcap%
\pgfsetroundjoin%
\definecolor{currentfill}{rgb}{0.121569,0.466667,0.705882}%
\pgfsetfillcolor{currentfill}%
\pgfsetlinewidth{1.003750pt}%
\definecolor{currentstroke}{rgb}{0.121569,0.466667,0.705882}%
\pgfsetstrokecolor{currentstroke}%
\pgfsetdash{}{0pt}%
\pgfpathmoveto{\pgfqpoint{4.927534in}{1.746319in}}%
\pgfpathcurveto{\pgfqpoint{4.937621in}{1.746319in}}{\pgfqpoint{4.947297in}{1.750326in}}{\pgfqpoint{4.954429in}{1.757459in}}%
\pgfpathcurveto{\pgfqpoint{4.961562in}{1.764592in}}{\pgfqpoint{4.965570in}{1.774268in}}{\pgfqpoint{4.965570in}{1.784355in}}%
\pgfpathcurveto{\pgfqpoint{4.965570in}{1.794442in}}{\pgfqpoint{4.961562in}{1.804118in}}{\pgfqpoint{4.954429in}{1.811251in}}%
\pgfpathcurveto{\pgfqpoint{4.947297in}{1.818383in}}{\pgfqpoint{4.937621in}{1.822391in}}{\pgfqpoint{4.927534in}{1.822391in}}%
\pgfpathcurveto{\pgfqpoint{4.917446in}{1.822391in}}{\pgfqpoint{4.907771in}{1.818383in}}{\pgfqpoint{4.900638in}{1.811251in}}%
\pgfpathcurveto{\pgfqpoint{4.893505in}{1.804118in}}{\pgfqpoint{4.889497in}{1.794442in}}{\pgfqpoint{4.889497in}{1.784355in}}%
\pgfpathcurveto{\pgfqpoint{4.889497in}{1.774268in}}{\pgfqpoint{4.893505in}{1.764592in}}{\pgfqpoint{4.900638in}{1.757459in}}%
\pgfpathcurveto{\pgfqpoint{4.907771in}{1.750326in}}{\pgfqpoint{4.917446in}{1.746319in}}{\pgfqpoint{4.927534in}{1.746319in}}%
\pgfpathclose%
\pgfusepath{stroke,fill}%
\end{pgfscope}%
\begin{pgfscope}%
\pgfpathrectangle{\pgfqpoint{0.800000in}{0.528000in}}{\pgfqpoint{4.960000in}{3.696000in}} %
\pgfusepath{clip}%
\pgfsetbuttcap%
\pgfsetroundjoin%
\definecolor{currentfill}{rgb}{0.121569,0.466667,0.705882}%
\pgfsetfillcolor{currentfill}%
\pgfsetlinewidth{1.003750pt}%
\definecolor{currentstroke}{rgb}{0.121569,0.466667,0.705882}%
\pgfsetstrokecolor{currentstroke}%
\pgfsetdash{}{0pt}%
\pgfpathmoveto{\pgfqpoint{1.696826in}{3.587083in}}%
\pgfpathcurveto{\pgfqpoint{1.706913in}{3.587083in}}{\pgfqpoint{1.716589in}{3.591090in}}{\pgfqpoint{1.723722in}{3.598223in}}%
\pgfpathcurveto{\pgfqpoint{1.730855in}{3.605356in}}{\pgfqpoint{1.734862in}{3.615032in}}{\pgfqpoint{1.734862in}{3.625119in}}%
\pgfpathcurveto{\pgfqpoint{1.734862in}{3.635206in}}{\pgfqpoint{1.730855in}{3.644882in}}{\pgfqpoint{1.723722in}{3.652015in}}%
\pgfpathcurveto{\pgfqpoint{1.716589in}{3.659148in}}{\pgfqpoint{1.706913in}{3.663155in}}{\pgfqpoint{1.696826in}{3.663155in}}%
\pgfpathcurveto{\pgfqpoint{1.686739in}{3.663155in}}{\pgfqpoint{1.677063in}{3.659148in}}{\pgfqpoint{1.669930in}{3.652015in}}%
\pgfpathcurveto{\pgfqpoint{1.662797in}{3.644882in}}{\pgfqpoint{1.658790in}{3.635206in}}{\pgfqpoint{1.658790in}{3.625119in}}%
\pgfpathcurveto{\pgfqpoint{1.658790in}{3.615032in}}{\pgfqpoint{1.662797in}{3.605356in}}{\pgfqpoint{1.669930in}{3.598223in}}%
\pgfpathcurveto{\pgfqpoint{1.677063in}{3.591090in}}{\pgfqpoint{1.686739in}{3.587083in}}{\pgfqpoint{1.696826in}{3.587083in}}%
\pgfpathclose%
\pgfusepath{stroke,fill}%
\end{pgfscope}%
\begin{pgfscope}%
\pgfpathrectangle{\pgfqpoint{0.800000in}{0.528000in}}{\pgfqpoint{4.960000in}{3.696000in}} %
\pgfusepath{clip}%
\pgfsetbuttcap%
\pgfsetroundjoin%
\definecolor{currentfill}{rgb}{0.121569,0.466667,0.705882}%
\pgfsetfillcolor{currentfill}%
\pgfsetlinewidth{1.003750pt}%
\definecolor{currentstroke}{rgb}{0.121569,0.466667,0.705882}%
\pgfsetstrokecolor{currentstroke}%
\pgfsetdash{}{0pt}%
\pgfpathmoveto{\pgfqpoint{3.341545in}{0.679525in}}%
\pgfpathcurveto{\pgfqpoint{3.351632in}{0.679525in}}{\pgfqpoint{3.361308in}{0.683533in}}{\pgfqpoint{3.368441in}{0.690666in}}%
\pgfpathcurveto{\pgfqpoint{3.375574in}{0.697799in}}{\pgfqpoint{3.379581in}{0.707474in}}{\pgfqpoint{3.379581in}{0.717562in}}%
\pgfpathcurveto{\pgfqpoint{3.379581in}{0.727649in}}{\pgfqpoint{3.375574in}{0.737325in}}{\pgfqpoint{3.368441in}{0.744458in}}%
\pgfpathcurveto{\pgfqpoint{3.361308in}{0.751590in}}{\pgfqpoint{3.351632in}{0.755598in}}{\pgfqpoint{3.341545in}{0.755598in}}%
\pgfpathcurveto{\pgfqpoint{3.331458in}{0.755598in}}{\pgfqpoint{3.321782in}{0.751590in}}{\pgfqpoint{3.314649in}{0.744458in}}%
\pgfpathcurveto{\pgfqpoint{3.307517in}{0.737325in}}{\pgfqpoint{3.303509in}{0.727649in}}{\pgfqpoint{3.303509in}{0.717562in}}%
\pgfpathcurveto{\pgfqpoint{3.303509in}{0.707474in}}{\pgfqpoint{3.307517in}{0.697799in}}{\pgfqpoint{3.314649in}{0.690666in}}%
\pgfpathcurveto{\pgfqpoint{3.321782in}{0.683533in}}{\pgfqpoint{3.331458in}{0.679525in}}{\pgfqpoint{3.341545in}{0.679525in}}%
\pgfpathclose%
\pgfusepath{stroke,fill}%
\end{pgfscope}%
\begin{pgfscope}%
\pgfpathrectangle{\pgfqpoint{0.800000in}{0.528000in}}{\pgfqpoint{4.960000in}{3.696000in}} %
\pgfusepath{clip}%
\pgfsetbuttcap%
\pgfsetroundjoin%
\definecolor{currentfill}{rgb}{0.121569,0.466667,0.705882}%
\pgfsetfillcolor{currentfill}%
\pgfsetlinewidth{1.003750pt}%
\definecolor{currentstroke}{rgb}{0.121569,0.466667,0.705882}%
\pgfsetstrokecolor{currentstroke}%
\pgfsetdash{}{0pt}%
\pgfpathmoveto{\pgfqpoint{3.194087in}{1.182718in}}%
\pgfpathcurveto{\pgfqpoint{3.204175in}{1.182718in}}{\pgfqpoint{3.213850in}{1.186725in}}{\pgfqpoint{3.220983in}{1.193858in}}%
\pgfpathcurveto{\pgfqpoint{3.228116in}{1.200991in}}{\pgfqpoint{3.232123in}{1.210667in}}{\pgfqpoint{3.232123in}{1.220754in}}%
\pgfpathcurveto{\pgfqpoint{3.232123in}{1.230841in}}{\pgfqpoint{3.228116in}{1.240517in}}{\pgfqpoint{3.220983in}{1.247650in}}%
\pgfpathcurveto{\pgfqpoint{3.213850in}{1.254782in}}{\pgfqpoint{3.204175in}{1.258790in}}{\pgfqpoint{3.194087in}{1.258790in}}%
\pgfpathcurveto{\pgfqpoint{3.184000in}{1.258790in}}{\pgfqpoint{3.174324in}{1.254782in}}{\pgfqpoint{3.167191in}{1.247650in}}%
\pgfpathcurveto{\pgfqpoint{3.160059in}{1.240517in}}{\pgfqpoint{3.156051in}{1.230841in}}{\pgfqpoint{3.156051in}{1.220754in}}%
\pgfpathcurveto{\pgfqpoint{3.156051in}{1.210667in}}{\pgfqpoint{3.160059in}{1.200991in}}{\pgfqpoint{3.167191in}{1.193858in}}%
\pgfpathcurveto{\pgfqpoint{3.174324in}{1.186725in}}{\pgfqpoint{3.184000in}{1.182718in}}{\pgfqpoint{3.194087in}{1.182718in}}%
\pgfpathclose%
\pgfusepath{stroke,fill}%
\end{pgfscope}%
\begin{pgfscope}%
\pgfpathrectangle{\pgfqpoint{0.800000in}{0.528000in}}{\pgfqpoint{4.960000in}{3.696000in}} %
\pgfusepath{clip}%
\pgfsetbuttcap%
\pgfsetroundjoin%
\definecolor{currentfill}{rgb}{0.121569,0.466667,0.705882}%
\pgfsetfillcolor{currentfill}%
\pgfsetlinewidth{1.003750pt}%
\definecolor{currentstroke}{rgb}{0.121569,0.466667,0.705882}%
\pgfsetstrokecolor{currentstroke}%
\pgfsetdash{}{0pt}%
\pgfpathmoveto{\pgfqpoint{4.724935in}{1.485777in}}%
\pgfpathcurveto{\pgfqpoint{4.735022in}{1.485777in}}{\pgfqpoint{4.744698in}{1.489785in}}{\pgfqpoint{4.751830in}{1.496918in}}%
\pgfpathcurveto{\pgfqpoint{4.758963in}{1.504051in}}{\pgfqpoint{4.762971in}{1.513726in}}{\pgfqpoint{4.762971in}{1.523813in}}%
\pgfpathcurveto{\pgfqpoint{4.762971in}{1.533901in}}{\pgfqpoint{4.758963in}{1.543576in}}{\pgfqpoint{4.751830in}{1.550709in}}%
\pgfpathcurveto{\pgfqpoint{4.744698in}{1.557842in}}{\pgfqpoint{4.735022in}{1.561850in}}{\pgfqpoint{4.724935in}{1.561850in}}%
\pgfpathcurveto{\pgfqpoint{4.714847in}{1.561850in}}{\pgfqpoint{4.705172in}{1.557842in}}{\pgfqpoint{4.698039in}{1.550709in}}%
\pgfpathcurveto{\pgfqpoint{4.690906in}{1.543576in}}{\pgfqpoint{4.686898in}{1.533901in}}{\pgfqpoint{4.686898in}{1.523813in}}%
\pgfpathcurveto{\pgfqpoint{4.686898in}{1.513726in}}{\pgfqpoint{4.690906in}{1.504051in}}{\pgfqpoint{4.698039in}{1.496918in}}%
\pgfpathcurveto{\pgfqpoint{4.705172in}{1.489785in}}{\pgfqpoint{4.714847in}{1.485777in}}{\pgfqpoint{4.724935in}{1.485777in}}%
\pgfpathclose%
\pgfusepath{stroke,fill}%
\end{pgfscope}%
\begin{pgfscope}%
\pgfpathrectangle{\pgfqpoint{0.800000in}{0.528000in}}{\pgfqpoint{4.960000in}{3.696000in}} %
\pgfusepath{clip}%
\pgfsetbuttcap%
\pgfsetroundjoin%
\definecolor{currentfill}{rgb}{0.121569,0.466667,0.705882}%
\pgfsetfillcolor{currentfill}%
\pgfsetlinewidth{1.003750pt}%
\definecolor{currentstroke}{rgb}{0.121569,0.466667,0.705882}%
\pgfsetstrokecolor{currentstroke}%
\pgfsetdash{}{0pt}%
\pgfpathmoveto{\pgfqpoint{1.753453in}{1.294168in}}%
\pgfpathcurveto{\pgfqpoint{1.763541in}{1.294168in}}{\pgfqpoint{1.773216in}{1.298175in}}{\pgfqpoint{1.780349in}{1.305308in}}%
\pgfpathcurveto{\pgfqpoint{1.787482in}{1.312441in}}{\pgfqpoint{1.791489in}{1.322117in}}{\pgfqpoint{1.791489in}{1.332204in}}%
\pgfpathcurveto{\pgfqpoint{1.791489in}{1.342291in}}{\pgfqpoint{1.787482in}{1.351967in}}{\pgfqpoint{1.780349in}{1.359100in}}%
\pgfpathcurveto{\pgfqpoint{1.773216in}{1.366233in}}{\pgfqpoint{1.763541in}{1.370240in}}{\pgfqpoint{1.753453in}{1.370240in}}%
\pgfpathcurveto{\pgfqpoint{1.743366in}{1.370240in}}{\pgfqpoint{1.733690in}{1.366233in}}{\pgfqpoint{1.726557in}{1.359100in}}%
\pgfpathcurveto{\pgfqpoint{1.719425in}{1.351967in}}{\pgfqpoint{1.715417in}{1.342291in}}{\pgfqpoint{1.715417in}{1.332204in}}%
\pgfpathcurveto{\pgfqpoint{1.715417in}{1.322117in}}{\pgfqpoint{1.719425in}{1.312441in}}{\pgfqpoint{1.726557in}{1.305308in}}%
\pgfpathcurveto{\pgfqpoint{1.733690in}{1.298175in}}{\pgfqpoint{1.743366in}{1.294168in}}{\pgfqpoint{1.753453in}{1.294168in}}%
\pgfpathclose%
\pgfusepath{stroke,fill}%
\end{pgfscope}%
\begin{pgfscope}%
\pgfpathrectangle{\pgfqpoint{0.800000in}{0.528000in}}{\pgfqpoint{4.960000in}{3.696000in}} %
\pgfusepath{clip}%
\pgfsetbuttcap%
\pgfsetroundjoin%
\definecolor{currentfill}{rgb}{0.121569,0.466667,0.705882}%
\pgfsetfillcolor{currentfill}%
\pgfsetlinewidth{1.003750pt}%
\definecolor{currentstroke}{rgb}{0.121569,0.466667,0.705882}%
\pgfsetstrokecolor{currentstroke}%
\pgfsetdash{}{0pt}%
\pgfpathmoveto{\pgfqpoint{3.177918in}{1.177921in}}%
\pgfpathcurveto{\pgfqpoint{3.188005in}{1.177921in}}{\pgfqpoint{3.197681in}{1.181929in}}{\pgfqpoint{3.204813in}{1.189062in}}%
\pgfpathcurveto{\pgfqpoint{3.211946in}{1.196194in}}{\pgfqpoint{3.215954in}{1.205870in}}{\pgfqpoint{3.215954in}{1.215957in}}%
\pgfpathcurveto{\pgfqpoint{3.215954in}{1.226045in}}{\pgfqpoint{3.211946in}{1.235720in}}{\pgfqpoint{3.204813in}{1.242853in}}%
\pgfpathcurveto{\pgfqpoint{3.197681in}{1.249986in}}{\pgfqpoint{3.188005in}{1.253994in}}{\pgfqpoint{3.177918in}{1.253994in}}%
\pgfpathcurveto{\pgfqpoint{3.167830in}{1.253994in}}{\pgfqpoint{3.158155in}{1.249986in}}{\pgfqpoint{3.151022in}{1.242853in}}%
\pgfpathcurveto{\pgfqpoint{3.143889in}{1.235720in}}{\pgfqpoint{3.139881in}{1.226045in}}{\pgfqpoint{3.139881in}{1.215957in}}%
\pgfpathcurveto{\pgfqpoint{3.139881in}{1.205870in}}{\pgfqpoint{3.143889in}{1.196194in}}{\pgfqpoint{3.151022in}{1.189062in}}%
\pgfpathcurveto{\pgfqpoint{3.158155in}{1.181929in}}{\pgfqpoint{3.167830in}{1.177921in}}{\pgfqpoint{3.177918in}{1.177921in}}%
\pgfpathclose%
\pgfusepath{stroke,fill}%
\end{pgfscope}%
\begin{pgfscope}%
\pgfpathrectangle{\pgfqpoint{0.800000in}{0.528000in}}{\pgfqpoint{4.960000in}{3.696000in}} %
\pgfusepath{clip}%
\pgfsetbuttcap%
\pgfsetroundjoin%
\definecolor{currentfill}{rgb}{0.121569,0.466667,0.705882}%
\pgfsetfillcolor{currentfill}%
\pgfsetlinewidth{1.003750pt}%
\definecolor{currentstroke}{rgb}{0.121569,0.466667,0.705882}%
\pgfsetstrokecolor{currentstroke}%
\pgfsetdash{}{0pt}%
\pgfpathmoveto{\pgfqpoint{2.010917in}{3.665089in}}%
\pgfpathcurveto{\pgfqpoint{2.021004in}{3.665089in}}{\pgfqpoint{2.030680in}{3.669097in}}{\pgfqpoint{2.037813in}{3.676230in}}%
\pgfpathcurveto{\pgfqpoint{2.044945in}{3.683363in}}{\pgfqpoint{2.048953in}{3.693038in}}{\pgfqpoint{2.048953in}{3.703125in}}%
\pgfpathcurveto{\pgfqpoint{2.048953in}{3.713213in}}{\pgfqpoint{2.044945in}{3.722888in}}{\pgfqpoint{2.037813in}{3.730021in}}%
\pgfpathcurveto{\pgfqpoint{2.030680in}{3.737154in}}{\pgfqpoint{2.021004in}{3.741162in}}{\pgfqpoint{2.010917in}{3.741162in}}%
\pgfpathcurveto{\pgfqpoint{2.000830in}{3.741162in}}{\pgfqpoint{1.991154in}{3.737154in}}{\pgfqpoint{1.984021in}{3.730021in}}%
\pgfpathcurveto{\pgfqpoint{1.976888in}{3.722888in}}{\pgfqpoint{1.972881in}{3.713213in}}{\pgfqpoint{1.972881in}{3.703125in}}%
\pgfpathcurveto{\pgfqpoint{1.972881in}{3.693038in}}{\pgfqpoint{1.976888in}{3.683363in}}{\pgfqpoint{1.984021in}{3.676230in}}%
\pgfpathcurveto{\pgfqpoint{1.991154in}{3.669097in}}{\pgfqpoint{2.000830in}{3.665089in}}{\pgfqpoint{2.010917in}{3.665089in}}%
\pgfpathclose%
\pgfusepath{stroke,fill}%
\end{pgfscope}%
\begin{pgfscope}%
\pgfpathrectangle{\pgfqpoint{0.800000in}{0.528000in}}{\pgfqpoint{4.960000in}{3.696000in}} %
\pgfusepath{clip}%
\pgfsetbuttcap%
\pgfsetroundjoin%
\definecolor{currentfill}{rgb}{0.121569,0.466667,0.705882}%
\pgfsetfillcolor{currentfill}%
\pgfsetlinewidth{1.003750pt}%
\definecolor{currentstroke}{rgb}{0.121569,0.466667,0.705882}%
\pgfsetstrokecolor{currentstroke}%
\pgfsetdash{}{0pt}%
\pgfpathmoveto{\pgfqpoint{2.939440in}{3.909619in}}%
\pgfpathcurveto{\pgfqpoint{2.949527in}{3.909619in}}{\pgfqpoint{2.959203in}{3.913627in}}{\pgfqpoint{2.966336in}{3.920760in}}%
\pgfpathcurveto{\pgfqpoint{2.973468in}{3.927893in}}{\pgfqpoint{2.977476in}{3.937568in}}{\pgfqpoint{2.977476in}{3.947656in}}%
\pgfpathcurveto{\pgfqpoint{2.977476in}{3.957743in}}{\pgfqpoint{2.973468in}{3.967419in}}{\pgfqpoint{2.966336in}{3.974551in}}%
\pgfpathcurveto{\pgfqpoint{2.959203in}{3.981684in}}{\pgfqpoint{2.949527in}{3.985692in}}{\pgfqpoint{2.939440in}{3.985692in}}%
\pgfpathcurveto{\pgfqpoint{2.929353in}{3.985692in}}{\pgfqpoint{2.919677in}{3.981684in}}{\pgfqpoint{2.912544in}{3.974551in}}%
\pgfpathcurveto{\pgfqpoint{2.905411in}{3.967419in}}{\pgfqpoint{2.901404in}{3.957743in}}{\pgfqpoint{2.901404in}{3.947656in}}%
\pgfpathcurveto{\pgfqpoint{2.901404in}{3.937568in}}{\pgfqpoint{2.905411in}{3.927893in}}{\pgfqpoint{2.912544in}{3.920760in}}%
\pgfpathcurveto{\pgfqpoint{2.919677in}{3.913627in}}{\pgfqpoint{2.929353in}{3.909619in}}{\pgfqpoint{2.939440in}{3.909619in}}%
\pgfpathclose%
\pgfusepath{stroke,fill}%
\end{pgfscope}%
\begin{pgfscope}%
\pgfpathrectangle{\pgfqpoint{0.800000in}{0.528000in}}{\pgfqpoint{4.960000in}{3.696000in}} %
\pgfusepath{clip}%
\pgfsetbuttcap%
\pgfsetroundjoin%
\definecolor{currentfill}{rgb}{0.121569,0.466667,0.705882}%
\pgfsetfillcolor{currentfill}%
\pgfsetlinewidth{1.003750pt}%
\definecolor{currentstroke}{rgb}{0.121569,0.466667,0.705882}%
\pgfsetstrokecolor{currentstroke}%
\pgfsetdash{}{0pt}%
\pgfpathmoveto{\pgfqpoint{4.605577in}{1.474043in}}%
\pgfpathcurveto{\pgfqpoint{4.615665in}{1.474043in}}{\pgfqpoint{4.625340in}{1.478051in}}{\pgfqpoint{4.632473in}{1.485184in}}%
\pgfpathcurveto{\pgfqpoint{4.639606in}{1.492316in}}{\pgfqpoint{4.643614in}{1.501992in}}{\pgfqpoint{4.643614in}{1.512079in}}%
\pgfpathcurveto{\pgfqpoint{4.643614in}{1.522167in}}{\pgfqpoint{4.639606in}{1.531842in}}{\pgfqpoint{4.632473in}{1.538975in}}%
\pgfpathcurveto{\pgfqpoint{4.625340in}{1.546108in}}{\pgfqpoint{4.615665in}{1.550116in}}{\pgfqpoint{4.605577in}{1.550116in}}%
\pgfpathcurveto{\pgfqpoint{4.595490in}{1.550116in}}{\pgfqpoint{4.585814in}{1.546108in}}{\pgfqpoint{4.578682in}{1.538975in}}%
\pgfpathcurveto{\pgfqpoint{4.571549in}{1.531842in}}{\pgfqpoint{4.567541in}{1.522167in}}{\pgfqpoint{4.567541in}{1.512079in}}%
\pgfpathcurveto{\pgfqpoint{4.567541in}{1.501992in}}{\pgfqpoint{4.571549in}{1.492316in}}{\pgfqpoint{4.578682in}{1.485184in}}%
\pgfpathcurveto{\pgfqpoint{4.585814in}{1.478051in}}{\pgfqpoint{4.595490in}{1.474043in}}{\pgfqpoint{4.605577in}{1.474043in}}%
\pgfpathclose%
\pgfusepath{stroke,fill}%
\end{pgfscope}%
\begin{pgfscope}%
\pgfpathrectangle{\pgfqpoint{0.800000in}{0.528000in}}{\pgfqpoint{4.960000in}{3.696000in}} %
\pgfusepath{clip}%
\pgfsetbuttcap%
\pgfsetroundjoin%
\definecolor{currentfill}{rgb}{0.121569,0.466667,0.705882}%
\pgfsetfillcolor{currentfill}%
\pgfsetlinewidth{1.003750pt}%
\definecolor{currentstroke}{rgb}{0.121569,0.466667,0.705882}%
\pgfsetstrokecolor{currentstroke}%
\pgfsetdash{}{0pt}%
\pgfpathmoveto{\pgfqpoint{1.323618in}{3.068014in}}%
\pgfpathcurveto{\pgfqpoint{1.333705in}{3.068014in}}{\pgfqpoint{1.343381in}{3.072022in}}{\pgfqpoint{1.350513in}{3.079155in}}%
\pgfpathcurveto{\pgfqpoint{1.357646in}{3.086288in}}{\pgfqpoint{1.361654in}{3.095963in}}{\pgfqpoint{1.361654in}{3.106051in}}%
\pgfpathcurveto{\pgfqpoint{1.361654in}{3.116138in}}{\pgfqpoint{1.357646in}{3.125813in}}{\pgfqpoint{1.350513in}{3.132946in}}%
\pgfpathcurveto{\pgfqpoint{1.343381in}{3.140079in}}{\pgfqpoint{1.333705in}{3.144087in}}{\pgfqpoint{1.323618in}{3.144087in}}%
\pgfpathcurveto{\pgfqpoint{1.313530in}{3.144087in}}{\pgfqpoint{1.303855in}{3.140079in}}{\pgfqpoint{1.296722in}{3.132946in}}%
\pgfpathcurveto{\pgfqpoint{1.289589in}{3.125813in}}{\pgfqpoint{1.285581in}{3.116138in}}{\pgfqpoint{1.285581in}{3.106051in}}%
\pgfpathcurveto{\pgfqpoint{1.285581in}{3.095963in}}{\pgfqpoint{1.289589in}{3.086288in}}{\pgfqpoint{1.296722in}{3.079155in}}%
\pgfpathcurveto{\pgfqpoint{1.303855in}{3.072022in}}{\pgfqpoint{1.313530in}{3.068014in}}{\pgfqpoint{1.323618in}{3.068014in}}%
\pgfpathclose%
\pgfusepath{stroke,fill}%
\end{pgfscope}%
\begin{pgfscope}%
\pgfpathrectangle{\pgfqpoint{0.800000in}{0.528000in}}{\pgfqpoint{4.960000in}{3.696000in}} %
\pgfusepath{clip}%
\pgfsetbuttcap%
\pgfsetroundjoin%
\definecolor{currentfill}{rgb}{0.121569,0.466667,0.705882}%
\pgfsetfillcolor{currentfill}%
\pgfsetlinewidth{1.003750pt}%
\definecolor{currentstroke}{rgb}{0.121569,0.466667,0.705882}%
\pgfsetstrokecolor{currentstroke}%
\pgfsetdash{}{0pt}%
\pgfpathmoveto{\pgfqpoint{4.985348in}{1.946057in}}%
\pgfpathcurveto{\pgfqpoint{4.995436in}{1.946057in}}{\pgfqpoint{5.005111in}{1.950064in}}{\pgfqpoint{5.012244in}{1.957197in}}%
\pgfpathcurveto{\pgfqpoint{5.019377in}{1.964330in}}{\pgfqpoint{5.023384in}{1.974006in}}{\pgfqpoint{5.023384in}{1.984093in}}%
\pgfpathcurveto{\pgfqpoint{5.023384in}{1.994180in}}{\pgfqpoint{5.019377in}{2.003856in}}{\pgfqpoint{5.012244in}{2.010989in}}%
\pgfpathcurveto{\pgfqpoint{5.005111in}{2.018121in}}{\pgfqpoint{4.995436in}{2.022129in}}{\pgfqpoint{4.985348in}{2.022129in}}%
\pgfpathcurveto{\pgfqpoint{4.975261in}{2.022129in}}{\pgfqpoint{4.965585in}{2.018121in}}{\pgfqpoint{4.958452in}{2.010989in}}%
\pgfpathcurveto{\pgfqpoint{4.951320in}{2.003856in}}{\pgfqpoint{4.947312in}{1.994180in}}{\pgfqpoint{4.947312in}{1.984093in}}%
\pgfpathcurveto{\pgfqpoint{4.947312in}{1.974006in}}{\pgfqpoint{4.951320in}{1.964330in}}{\pgfqpoint{4.958452in}{1.957197in}}%
\pgfpathcurveto{\pgfqpoint{4.965585in}{1.950064in}}{\pgfqpoint{4.975261in}{1.946057in}}{\pgfqpoint{4.985348in}{1.946057in}}%
\pgfpathclose%
\pgfusepath{stroke,fill}%
\end{pgfscope}%
\begin{pgfscope}%
\pgfpathrectangle{\pgfqpoint{0.800000in}{0.528000in}}{\pgfqpoint{4.960000in}{3.696000in}} %
\pgfusepath{clip}%
\pgfsetbuttcap%
\pgfsetroundjoin%
\definecolor{currentfill}{rgb}{0.121569,0.466667,0.705882}%
\pgfsetfillcolor{currentfill}%
\pgfsetlinewidth{1.003750pt}%
\definecolor{currentstroke}{rgb}{0.121569,0.466667,0.705882}%
\pgfsetstrokecolor{currentstroke}%
\pgfsetdash{}{0pt}%
\pgfpathmoveto{\pgfqpoint{2.191351in}{1.258590in}}%
\pgfpathcurveto{\pgfqpoint{2.201438in}{1.258590in}}{\pgfqpoint{2.211114in}{1.262597in}}{\pgfqpoint{2.218246in}{1.269730in}}%
\pgfpathcurveto{\pgfqpoint{2.225379in}{1.276863in}}{\pgfqpoint{2.229387in}{1.286539in}}{\pgfqpoint{2.229387in}{1.296626in}}%
\pgfpathcurveto{\pgfqpoint{2.229387in}{1.306713in}}{\pgfqpoint{2.225379in}{1.316389in}}{\pgfqpoint{2.218246in}{1.323522in}}%
\pgfpathcurveto{\pgfqpoint{2.211114in}{1.330654in}}{\pgfqpoint{2.201438in}{1.334662in}}{\pgfqpoint{2.191351in}{1.334662in}}%
\pgfpathcurveto{\pgfqpoint{2.181263in}{1.334662in}}{\pgfqpoint{2.171588in}{1.330654in}}{\pgfqpoint{2.164455in}{1.323522in}}%
\pgfpathcurveto{\pgfqpoint{2.157322in}{1.316389in}}{\pgfqpoint{2.153314in}{1.306713in}}{\pgfqpoint{2.153314in}{1.296626in}}%
\pgfpathcurveto{\pgfqpoint{2.153314in}{1.286539in}}{\pgfqpoint{2.157322in}{1.276863in}}{\pgfqpoint{2.164455in}{1.269730in}}%
\pgfpathcurveto{\pgfqpoint{2.171588in}{1.262597in}}{\pgfqpoint{2.181263in}{1.258590in}}{\pgfqpoint{2.191351in}{1.258590in}}%
\pgfpathclose%
\pgfusepath{stroke,fill}%
\end{pgfscope}%
\begin{pgfscope}%
\pgfpathrectangle{\pgfqpoint{0.800000in}{0.528000in}}{\pgfqpoint{4.960000in}{3.696000in}} %
\pgfusepath{clip}%
\pgfsetbuttcap%
\pgfsetroundjoin%
\definecolor{currentfill}{rgb}{0.121569,0.466667,0.705882}%
\pgfsetfillcolor{currentfill}%
\pgfsetlinewidth{1.003750pt}%
\definecolor{currentstroke}{rgb}{0.121569,0.466667,0.705882}%
\pgfsetstrokecolor{currentstroke}%
\pgfsetdash{}{0pt}%
\pgfpathmoveto{\pgfqpoint{1.708619in}{1.376076in}}%
\pgfpathcurveto{\pgfqpoint{1.718707in}{1.376076in}}{\pgfqpoint{1.728382in}{1.380083in}}{\pgfqpoint{1.735515in}{1.387216in}}%
\pgfpathcurveto{\pgfqpoint{1.742648in}{1.394349in}}{\pgfqpoint{1.746655in}{1.404024in}}{\pgfqpoint{1.746655in}{1.414112in}}%
\pgfpathcurveto{\pgfqpoint{1.746655in}{1.424199in}}{\pgfqpoint{1.742648in}{1.433875in}}{\pgfqpoint{1.735515in}{1.441008in}}%
\pgfpathcurveto{\pgfqpoint{1.728382in}{1.448140in}}{\pgfqpoint{1.718707in}{1.452148in}}{\pgfqpoint{1.708619in}{1.452148in}}%
\pgfpathcurveto{\pgfqpoint{1.698532in}{1.452148in}}{\pgfqpoint{1.688856in}{1.448140in}}{\pgfqpoint{1.681723in}{1.441008in}}%
\pgfpathcurveto{\pgfqpoint{1.674591in}{1.433875in}}{\pgfqpoint{1.670583in}{1.424199in}}{\pgfqpoint{1.670583in}{1.414112in}}%
\pgfpathcurveto{\pgfqpoint{1.670583in}{1.404024in}}{\pgfqpoint{1.674591in}{1.394349in}}{\pgfqpoint{1.681723in}{1.387216in}}%
\pgfpathcurveto{\pgfqpoint{1.688856in}{1.380083in}}{\pgfqpoint{1.698532in}{1.376076in}}{\pgfqpoint{1.708619in}{1.376076in}}%
\pgfpathclose%
\pgfusepath{stroke,fill}%
\end{pgfscope}%
\begin{pgfscope}%
\pgfpathrectangle{\pgfqpoint{0.800000in}{0.528000in}}{\pgfqpoint{4.960000in}{3.696000in}} %
\pgfusepath{clip}%
\pgfsetbuttcap%
\pgfsetroundjoin%
\definecolor{currentfill}{rgb}{0.121569,0.466667,0.705882}%
\pgfsetfillcolor{currentfill}%
\pgfsetlinewidth{1.003750pt}%
\definecolor{currentstroke}{rgb}{0.121569,0.466667,0.705882}%
\pgfsetstrokecolor{currentstroke}%
\pgfsetdash{}{0pt}%
\pgfpathmoveto{\pgfqpoint{4.748684in}{3.396084in}}%
\pgfpathcurveto{\pgfqpoint{4.758771in}{3.396084in}}{\pgfqpoint{4.768446in}{3.400092in}}{\pgfqpoint{4.775579in}{3.407224in}}%
\pgfpathcurveto{\pgfqpoint{4.782712in}{3.414357in}}{\pgfqpoint{4.786720in}{3.424033in}}{\pgfqpoint{4.786720in}{3.434120in}}%
\pgfpathcurveto{\pgfqpoint{4.786720in}{3.444207in}}{\pgfqpoint{4.782712in}{3.453883in}}{\pgfqpoint{4.775579in}{3.461016in}}%
\pgfpathcurveto{\pgfqpoint{4.768446in}{3.468149in}}{\pgfqpoint{4.758771in}{3.472156in}}{\pgfqpoint{4.748684in}{3.472156in}}%
\pgfpathcurveto{\pgfqpoint{4.738596in}{3.472156in}}{\pgfqpoint{4.728921in}{3.468149in}}{\pgfqpoint{4.721788in}{3.461016in}}%
\pgfpathcurveto{\pgfqpoint{4.714655in}{3.453883in}}{\pgfqpoint{4.710647in}{3.444207in}}{\pgfqpoint{4.710647in}{3.434120in}}%
\pgfpathcurveto{\pgfqpoint{4.710647in}{3.424033in}}{\pgfqpoint{4.714655in}{3.414357in}}{\pgfqpoint{4.721788in}{3.407224in}}%
\pgfpathcurveto{\pgfqpoint{4.728921in}{3.400092in}}{\pgfqpoint{4.738596in}{3.396084in}}{\pgfqpoint{4.748684in}{3.396084in}}%
\pgfpathclose%
\pgfusepath{stroke,fill}%
\end{pgfscope}%
\begin{pgfscope}%
\pgfpathrectangle{\pgfqpoint{0.800000in}{0.528000in}}{\pgfqpoint{4.960000in}{3.696000in}} %
\pgfusepath{clip}%
\pgfsetbuttcap%
\pgfsetroundjoin%
\definecolor{currentfill}{rgb}{0.121569,0.466667,0.705882}%
\pgfsetfillcolor{currentfill}%
\pgfsetlinewidth{1.003750pt}%
\definecolor{currentstroke}{rgb}{0.121569,0.466667,0.705882}%
\pgfsetstrokecolor{currentstroke}%
\pgfsetdash{}{0pt}%
\pgfpathmoveto{\pgfqpoint{5.049996in}{2.091462in}}%
\pgfpathcurveto{\pgfqpoint{5.060083in}{2.091462in}}{\pgfqpoint{5.069758in}{2.095470in}}{\pgfqpoint{5.076891in}{2.102603in}}%
\pgfpathcurveto{\pgfqpoint{5.084024in}{2.109735in}}{\pgfqpoint{5.088032in}{2.119411in}}{\pgfqpoint{5.088032in}{2.129498in}}%
\pgfpathcurveto{\pgfqpoint{5.088032in}{2.139586in}}{\pgfqpoint{5.084024in}{2.149261in}}{\pgfqpoint{5.076891in}{2.156394in}}%
\pgfpathcurveto{\pgfqpoint{5.069758in}{2.163527in}}{\pgfqpoint{5.060083in}{2.167535in}}{\pgfqpoint{5.049996in}{2.167535in}}%
\pgfpathcurveto{\pgfqpoint{5.039908in}{2.167535in}}{\pgfqpoint{5.030233in}{2.163527in}}{\pgfqpoint{5.023100in}{2.156394in}}%
\pgfpathcurveto{\pgfqpoint{5.015967in}{2.149261in}}{\pgfqpoint{5.011959in}{2.139586in}}{\pgfqpoint{5.011959in}{2.129498in}}%
\pgfpathcurveto{\pgfqpoint{5.011959in}{2.119411in}}{\pgfqpoint{5.015967in}{2.109735in}}{\pgfqpoint{5.023100in}{2.102603in}}%
\pgfpathcurveto{\pgfqpoint{5.030233in}{2.095470in}}{\pgfqpoint{5.039908in}{2.091462in}}{\pgfqpoint{5.049996in}{2.091462in}}%
\pgfpathclose%
\pgfusepath{stroke,fill}%
\end{pgfscope}%
\begin{pgfscope}%
\pgfpathrectangle{\pgfqpoint{0.800000in}{0.528000in}}{\pgfqpoint{4.960000in}{3.696000in}} %
\pgfusepath{clip}%
\pgfsetbuttcap%
\pgfsetroundjoin%
\definecolor{currentfill}{rgb}{0.121569,0.466667,0.705882}%
\pgfsetfillcolor{currentfill}%
\pgfsetlinewidth{1.003750pt}%
\definecolor{currentstroke}{rgb}{0.121569,0.466667,0.705882}%
\pgfsetstrokecolor{currentstroke}%
\pgfsetdash{}{0pt}%
\pgfpathmoveto{\pgfqpoint{1.433143in}{2.901514in}}%
\pgfpathcurveto{\pgfqpoint{1.443230in}{2.901514in}}{\pgfqpoint{1.452906in}{2.905522in}}{\pgfqpoint{1.460039in}{2.912655in}}%
\pgfpathcurveto{\pgfqpoint{1.467172in}{2.919788in}}{\pgfqpoint{1.471179in}{2.929463in}}{\pgfqpoint{1.471179in}{2.939551in}}%
\pgfpathcurveto{\pgfqpoint{1.471179in}{2.949638in}}{\pgfqpoint{1.467172in}{2.959314in}}{\pgfqpoint{1.460039in}{2.966446in}}%
\pgfpathcurveto{\pgfqpoint{1.452906in}{2.973579in}}{\pgfqpoint{1.443230in}{2.977587in}}{\pgfqpoint{1.433143in}{2.977587in}}%
\pgfpathcurveto{\pgfqpoint{1.423056in}{2.977587in}}{\pgfqpoint{1.413380in}{2.973579in}}{\pgfqpoint{1.406247in}{2.966446in}}%
\pgfpathcurveto{\pgfqpoint{1.399115in}{2.959314in}}{\pgfqpoint{1.395107in}{2.949638in}}{\pgfqpoint{1.395107in}{2.939551in}}%
\pgfpathcurveto{\pgfqpoint{1.395107in}{2.929463in}}{\pgfqpoint{1.399115in}{2.919788in}}{\pgfqpoint{1.406247in}{2.912655in}}%
\pgfpathcurveto{\pgfqpoint{1.413380in}{2.905522in}}{\pgfqpoint{1.423056in}{2.901514in}}{\pgfqpoint{1.433143in}{2.901514in}}%
\pgfpathclose%
\pgfusepath{stroke,fill}%
\end{pgfscope}%
\begin{pgfscope}%
\pgfpathrectangle{\pgfqpoint{0.800000in}{0.528000in}}{\pgfqpoint{4.960000in}{3.696000in}} %
\pgfusepath{clip}%
\pgfsetbuttcap%
\pgfsetroundjoin%
\definecolor{currentfill}{rgb}{0.121569,0.466667,0.705882}%
\pgfsetfillcolor{currentfill}%
\pgfsetlinewidth{1.003750pt}%
\definecolor{currentstroke}{rgb}{0.121569,0.466667,0.705882}%
\pgfsetstrokecolor{currentstroke}%
\pgfsetdash{}{0pt}%
\pgfpathmoveto{\pgfqpoint{1.463178in}{1.760842in}}%
\pgfpathcurveto{\pgfqpoint{1.473266in}{1.760842in}}{\pgfqpoint{1.482941in}{1.764850in}}{\pgfqpoint{1.490074in}{1.771983in}}%
\pgfpathcurveto{\pgfqpoint{1.497207in}{1.779115in}}{\pgfqpoint{1.501215in}{1.788791in}}{\pgfqpoint{1.501215in}{1.798878in}}%
\pgfpathcurveto{\pgfqpoint{1.501215in}{1.808966in}}{\pgfqpoint{1.497207in}{1.818641in}}{\pgfqpoint{1.490074in}{1.825774in}}%
\pgfpathcurveto{\pgfqpoint{1.482941in}{1.832907in}}{\pgfqpoint{1.473266in}{1.836915in}}{\pgfqpoint{1.463178in}{1.836915in}}%
\pgfpathcurveto{\pgfqpoint{1.453091in}{1.836915in}}{\pgfqpoint{1.443416in}{1.832907in}}{\pgfqpoint{1.436283in}{1.825774in}}%
\pgfpathcurveto{\pgfqpoint{1.429150in}{1.818641in}}{\pgfqpoint{1.425142in}{1.808966in}}{\pgfqpoint{1.425142in}{1.798878in}}%
\pgfpathcurveto{\pgfqpoint{1.425142in}{1.788791in}}{\pgfqpoint{1.429150in}{1.779115in}}{\pgfqpoint{1.436283in}{1.771983in}}%
\pgfpathcurveto{\pgfqpoint{1.443416in}{1.764850in}}{\pgfqpoint{1.453091in}{1.760842in}}{\pgfqpoint{1.463178in}{1.760842in}}%
\pgfpathclose%
\pgfusepath{stroke,fill}%
\end{pgfscope}%
\begin{pgfscope}%
\pgfpathrectangle{\pgfqpoint{0.800000in}{0.528000in}}{\pgfqpoint{4.960000in}{3.696000in}} %
\pgfusepath{clip}%
\pgfsetbuttcap%
\pgfsetroundjoin%
\definecolor{currentfill}{rgb}{0.121569,0.466667,0.705882}%
\pgfsetfillcolor{currentfill}%
\pgfsetlinewidth{1.003750pt}%
\definecolor{currentstroke}{rgb}{0.121569,0.466667,0.705882}%
\pgfsetstrokecolor{currentstroke}%
\pgfsetdash{}{0pt}%
\pgfpathmoveto{\pgfqpoint{1.826973in}{3.244233in}}%
\pgfpathcurveto{\pgfqpoint{1.837061in}{3.244233in}}{\pgfqpoint{1.846736in}{3.248240in}}{\pgfqpoint{1.853869in}{3.255373in}}%
\pgfpathcurveto{\pgfqpoint{1.861002in}{3.262506in}}{\pgfqpoint{1.865009in}{3.272182in}}{\pgfqpoint{1.865009in}{3.282269in}}%
\pgfpathcurveto{\pgfqpoint{1.865009in}{3.292356in}}{\pgfqpoint{1.861002in}{3.302032in}}{\pgfqpoint{1.853869in}{3.309165in}}%
\pgfpathcurveto{\pgfqpoint{1.846736in}{3.316298in}}{\pgfqpoint{1.837061in}{3.320305in}}{\pgfqpoint{1.826973in}{3.320305in}}%
\pgfpathcurveto{\pgfqpoint{1.816886in}{3.320305in}}{\pgfqpoint{1.807210in}{3.316298in}}{\pgfqpoint{1.800077in}{3.309165in}}%
\pgfpathcurveto{\pgfqpoint{1.792945in}{3.302032in}}{\pgfqpoint{1.788937in}{3.292356in}}{\pgfqpoint{1.788937in}{3.282269in}}%
\pgfpathcurveto{\pgfqpoint{1.788937in}{3.272182in}}{\pgfqpoint{1.792945in}{3.262506in}}{\pgfqpoint{1.800077in}{3.255373in}}%
\pgfpathcurveto{\pgfqpoint{1.807210in}{3.248240in}}{\pgfqpoint{1.816886in}{3.244233in}}{\pgfqpoint{1.826973in}{3.244233in}}%
\pgfpathclose%
\pgfusepath{stroke,fill}%
\end{pgfscope}%
\begin{pgfscope}%
\pgfpathrectangle{\pgfqpoint{0.800000in}{0.528000in}}{\pgfqpoint{4.960000in}{3.696000in}} %
\pgfusepath{clip}%
\pgfsetbuttcap%
\pgfsetroundjoin%
\definecolor{currentfill}{rgb}{0.121569,0.466667,0.705882}%
\pgfsetfillcolor{currentfill}%
\pgfsetlinewidth{1.003750pt}%
\definecolor{currentstroke}{rgb}{0.121569,0.466667,0.705882}%
\pgfsetstrokecolor{currentstroke}%
\pgfsetdash{}{0pt}%
\pgfpathmoveto{\pgfqpoint{3.080589in}{3.796548in}}%
\pgfpathcurveto{\pgfqpoint{3.090677in}{3.796548in}}{\pgfqpoint{3.100352in}{3.800555in}}{\pgfqpoint{3.107485in}{3.807688in}}%
\pgfpathcurveto{\pgfqpoint{3.114618in}{3.814821in}}{\pgfqpoint{3.118626in}{3.824496in}}{\pgfqpoint{3.118626in}{3.834584in}}%
\pgfpathcurveto{\pgfqpoint{3.118626in}{3.844671in}}{\pgfqpoint{3.114618in}{3.854347in}}{\pgfqpoint{3.107485in}{3.861480in}}%
\pgfpathcurveto{\pgfqpoint{3.100352in}{3.868612in}}{\pgfqpoint{3.090677in}{3.872620in}}{\pgfqpoint{3.080589in}{3.872620in}}%
\pgfpathcurveto{\pgfqpoint{3.070502in}{3.872620in}}{\pgfqpoint{3.060826in}{3.868612in}}{\pgfqpoint{3.053693in}{3.861480in}}%
\pgfpathcurveto{\pgfqpoint{3.046561in}{3.854347in}}{\pgfqpoint{3.042553in}{3.844671in}}{\pgfqpoint{3.042553in}{3.834584in}}%
\pgfpathcurveto{\pgfqpoint{3.042553in}{3.824496in}}{\pgfqpoint{3.046561in}{3.814821in}}{\pgfqpoint{3.053693in}{3.807688in}}%
\pgfpathcurveto{\pgfqpoint{3.060826in}{3.800555in}}{\pgfqpoint{3.070502in}{3.796548in}}{\pgfqpoint{3.080589in}{3.796548in}}%
\pgfpathclose%
\pgfusepath{stroke,fill}%
\end{pgfscope}%
\begin{pgfscope}%
\pgfpathrectangle{\pgfqpoint{0.800000in}{0.528000in}}{\pgfqpoint{4.960000in}{3.696000in}} %
\pgfusepath{clip}%
\pgfsetbuttcap%
\pgfsetroundjoin%
\definecolor{currentfill}{rgb}{0.121569,0.466667,0.705882}%
\pgfsetfillcolor{currentfill}%
\pgfsetlinewidth{1.003750pt}%
\definecolor{currentstroke}{rgb}{0.121569,0.466667,0.705882}%
\pgfsetstrokecolor{currentstroke}%
\pgfsetdash{}{0pt}%
\pgfpathmoveto{\pgfqpoint{4.727582in}{1.677824in}}%
\pgfpathcurveto{\pgfqpoint{4.737670in}{1.677824in}}{\pgfqpoint{4.747345in}{1.681832in}}{\pgfqpoint{4.754478in}{1.688965in}}%
\pgfpathcurveto{\pgfqpoint{4.761611in}{1.696098in}}{\pgfqpoint{4.765619in}{1.705773in}}{\pgfqpoint{4.765619in}{1.715861in}}%
\pgfpathcurveto{\pgfqpoint{4.765619in}{1.725948in}}{\pgfqpoint{4.761611in}{1.735624in}}{\pgfqpoint{4.754478in}{1.742756in}}%
\pgfpathcurveto{\pgfqpoint{4.747345in}{1.749889in}}{\pgfqpoint{4.737670in}{1.753897in}}{\pgfqpoint{4.727582in}{1.753897in}}%
\pgfpathcurveto{\pgfqpoint{4.717495in}{1.753897in}}{\pgfqpoint{4.707819in}{1.749889in}}{\pgfqpoint{4.700687in}{1.742756in}}%
\pgfpathcurveto{\pgfqpoint{4.693554in}{1.735624in}}{\pgfqpoint{4.689546in}{1.725948in}}{\pgfqpoint{4.689546in}{1.715861in}}%
\pgfpathcurveto{\pgfqpoint{4.689546in}{1.705773in}}{\pgfqpoint{4.693554in}{1.696098in}}{\pgfqpoint{4.700687in}{1.688965in}}%
\pgfpathcurveto{\pgfqpoint{4.707819in}{1.681832in}}{\pgfqpoint{4.717495in}{1.677824in}}{\pgfqpoint{4.727582in}{1.677824in}}%
\pgfpathclose%
\pgfusepath{stroke,fill}%
\end{pgfscope}%
\begin{pgfscope}%
\pgfpathrectangle{\pgfqpoint{0.800000in}{0.528000in}}{\pgfqpoint{4.960000in}{3.696000in}} %
\pgfusepath{clip}%
\pgfsetbuttcap%
\pgfsetroundjoin%
\definecolor{currentfill}{rgb}{0.121569,0.466667,0.705882}%
\pgfsetfillcolor{currentfill}%
\pgfsetlinewidth{1.003750pt}%
\definecolor{currentstroke}{rgb}{0.121569,0.466667,0.705882}%
\pgfsetstrokecolor{currentstroke}%
\pgfsetdash{}{0pt}%
\pgfpathmoveto{\pgfqpoint{1.532856in}{1.930860in}}%
\pgfpathcurveto{\pgfqpoint{1.542943in}{1.930860in}}{\pgfqpoint{1.552618in}{1.934868in}}{\pgfqpoint{1.559751in}{1.942001in}}%
\pgfpathcurveto{\pgfqpoint{1.566884in}{1.949134in}}{\pgfqpoint{1.570892in}{1.958809in}}{\pgfqpoint{1.570892in}{1.968896in}}%
\pgfpathcurveto{\pgfqpoint{1.570892in}{1.978984in}}{\pgfqpoint{1.566884in}{1.988659in}}{\pgfqpoint{1.559751in}{1.995792in}}%
\pgfpathcurveto{\pgfqpoint{1.552618in}{2.002925in}}{\pgfqpoint{1.542943in}{2.006933in}}{\pgfqpoint{1.532856in}{2.006933in}}%
\pgfpathcurveto{\pgfqpoint{1.522768in}{2.006933in}}{\pgfqpoint{1.513093in}{2.002925in}}{\pgfqpoint{1.505960in}{1.995792in}}%
\pgfpathcurveto{\pgfqpoint{1.498827in}{1.988659in}}{\pgfqpoint{1.494819in}{1.978984in}}{\pgfqpoint{1.494819in}{1.968896in}}%
\pgfpathcurveto{\pgfqpoint{1.494819in}{1.958809in}}{\pgfqpoint{1.498827in}{1.949134in}}{\pgfqpoint{1.505960in}{1.942001in}}%
\pgfpathcurveto{\pgfqpoint{1.513093in}{1.934868in}}{\pgfqpoint{1.522768in}{1.930860in}}{\pgfqpoint{1.532856in}{1.930860in}}%
\pgfpathclose%
\pgfusepath{stroke,fill}%
\end{pgfscope}%
\begin{pgfscope}%
\pgfpathrectangle{\pgfqpoint{0.800000in}{0.528000in}}{\pgfqpoint{4.960000in}{3.696000in}} %
\pgfusepath{clip}%
\pgfsetbuttcap%
\pgfsetroundjoin%
\definecolor{currentfill}{rgb}{0.121569,0.466667,0.705882}%
\pgfsetfillcolor{currentfill}%
\pgfsetlinewidth{1.003750pt}%
\definecolor{currentstroke}{rgb}{0.121569,0.466667,0.705882}%
\pgfsetstrokecolor{currentstroke}%
\pgfsetdash{}{0pt}%
\pgfpathmoveto{\pgfqpoint{5.529056in}{2.526149in}}%
\pgfpathcurveto{\pgfqpoint{5.539143in}{2.526149in}}{\pgfqpoint{5.548819in}{2.530157in}}{\pgfqpoint{5.555951in}{2.537290in}}%
\pgfpathcurveto{\pgfqpoint{5.563084in}{2.544422in}}{\pgfqpoint{5.567092in}{2.554098in}}{\pgfqpoint{5.567092in}{2.564185in}}%
\pgfpathcurveto{\pgfqpoint{5.567092in}{2.574273in}}{\pgfqpoint{5.563084in}{2.583948in}}{\pgfqpoint{5.555951in}{2.591081in}}%
\pgfpathcurveto{\pgfqpoint{5.548819in}{2.598214in}}{\pgfqpoint{5.539143in}{2.602222in}}{\pgfqpoint{5.529056in}{2.602222in}}%
\pgfpathcurveto{\pgfqpoint{5.518968in}{2.602222in}}{\pgfqpoint{5.509293in}{2.598214in}}{\pgfqpoint{5.502160in}{2.591081in}}%
\pgfpathcurveto{\pgfqpoint{5.495027in}{2.583948in}}{\pgfqpoint{5.491019in}{2.574273in}}{\pgfqpoint{5.491019in}{2.564185in}}%
\pgfpathcurveto{\pgfqpoint{5.491019in}{2.554098in}}{\pgfqpoint{5.495027in}{2.544422in}}{\pgfqpoint{5.502160in}{2.537290in}}%
\pgfpathcurveto{\pgfqpoint{5.509293in}{2.530157in}}{\pgfqpoint{5.518968in}{2.526149in}}{\pgfqpoint{5.529056in}{2.526149in}}%
\pgfpathclose%
\pgfusepath{stroke,fill}%
\end{pgfscope}%
\begin{pgfscope}%
\pgfpathrectangle{\pgfqpoint{0.800000in}{0.528000in}}{\pgfqpoint{4.960000in}{3.696000in}} %
\pgfusepath{clip}%
\pgfsetbuttcap%
\pgfsetroundjoin%
\definecolor{currentfill}{rgb}{0.121569,0.466667,0.705882}%
\pgfsetfillcolor{currentfill}%
\pgfsetlinewidth{1.003750pt}%
\definecolor{currentstroke}{rgb}{0.121569,0.466667,0.705882}%
\pgfsetstrokecolor{currentstroke}%
\pgfsetdash{}{0pt}%
\pgfpathmoveto{\pgfqpoint{1.676599in}{1.536070in}}%
\pgfpathcurveto{\pgfqpoint{1.686686in}{1.536070in}}{\pgfqpoint{1.696362in}{1.540078in}}{\pgfqpoint{1.703495in}{1.547210in}}%
\pgfpathcurveto{\pgfqpoint{1.710628in}{1.554343in}}{\pgfqpoint{1.714635in}{1.564019in}}{\pgfqpoint{1.714635in}{1.574106in}}%
\pgfpathcurveto{\pgfqpoint{1.714635in}{1.584194in}}{\pgfqpoint{1.710628in}{1.593869in}}{\pgfqpoint{1.703495in}{1.601002in}}%
\pgfpathcurveto{\pgfqpoint{1.696362in}{1.608135in}}{\pgfqpoint{1.686686in}{1.612142in}}{\pgfqpoint{1.676599in}{1.612142in}}%
\pgfpathcurveto{\pgfqpoint{1.666512in}{1.612142in}}{\pgfqpoint{1.656836in}{1.608135in}}{\pgfqpoint{1.649703in}{1.601002in}}%
\pgfpathcurveto{\pgfqpoint{1.642570in}{1.593869in}}{\pgfqpoint{1.638563in}{1.584194in}}{\pgfqpoint{1.638563in}{1.574106in}}%
\pgfpathcurveto{\pgfqpoint{1.638563in}{1.564019in}}{\pgfqpoint{1.642570in}{1.554343in}}{\pgfqpoint{1.649703in}{1.547210in}}%
\pgfpathcurveto{\pgfqpoint{1.656836in}{1.540078in}}{\pgfqpoint{1.666512in}{1.536070in}}{\pgfqpoint{1.676599in}{1.536070in}}%
\pgfpathclose%
\pgfusepath{stroke,fill}%
\end{pgfscope}%
\begin{pgfscope}%
\pgfsetbuttcap%
\pgfsetroundjoin%
\definecolor{currentfill}{rgb}{0.000000,0.000000,0.000000}%
\pgfsetfillcolor{currentfill}%
\pgfsetlinewidth{0.803000pt}%
\definecolor{currentstroke}{rgb}{0.000000,0.000000,0.000000}%
\pgfsetstrokecolor{currentstroke}%
\pgfsetdash{}{0pt}%
\pgfsys@defobject{currentmarker}{\pgfqpoint{0.000000in}{-0.048611in}}{\pgfqpoint{0.000000in}{0.000000in}}{%
\pgfpathmoveto{\pgfqpoint{0.000000in}{0.000000in}}%
\pgfpathlineto{\pgfqpoint{0.000000in}{-0.048611in}}%
\pgfusepath{stroke,fill}%
}%
\begin{pgfscope}%
\pgfsys@transformshift{1.196788in}{0.528000in}%
\pgfsys@useobject{currentmarker}{}%
\end{pgfscope}%
\end{pgfscope}%
\begin{pgfscope}%
\pgftext[x=1.196788in,y=0.430778in,,top]{\rmfamily\fontsize{10.000000}{12.000000}\selectfont \(\displaystyle -2\)}%
\end{pgfscope}%
\begin{pgfscope}%
\pgfsetbuttcap%
\pgfsetroundjoin%
\definecolor{currentfill}{rgb}{0.000000,0.000000,0.000000}%
\pgfsetfillcolor{currentfill}%
\pgfsetlinewidth{0.803000pt}%
\definecolor{currentstroke}{rgb}{0.000000,0.000000,0.000000}%
\pgfsetstrokecolor{currentstroke}%
\pgfsetdash{}{0pt}%
\pgfsys@defobject{currentmarker}{\pgfqpoint{0.000000in}{-0.048611in}}{\pgfqpoint{0.000000in}{0.000000in}}{%
\pgfpathmoveto{\pgfqpoint{0.000000in}{0.000000in}}%
\pgfpathlineto{\pgfqpoint{0.000000in}{-0.048611in}}%
\pgfusepath{stroke,fill}%
}%
\begin{pgfscope}%
\pgfsys@transformshift{2.191058in}{0.528000in}%
\pgfsys@useobject{currentmarker}{}%
\end{pgfscope}%
\end{pgfscope}%
\begin{pgfscope}%
\pgftext[x=2.191058in,y=0.430778in,,top]{\rmfamily\fontsize{10.000000}{12.000000}\selectfont \(\displaystyle -1\)}%
\end{pgfscope}%
\begin{pgfscope}%
\pgfsetbuttcap%
\pgfsetroundjoin%
\definecolor{currentfill}{rgb}{0.000000,0.000000,0.000000}%
\pgfsetfillcolor{currentfill}%
\pgfsetlinewidth{0.803000pt}%
\definecolor{currentstroke}{rgb}{0.000000,0.000000,0.000000}%
\pgfsetstrokecolor{currentstroke}%
\pgfsetdash{}{0pt}%
\pgfsys@defobject{currentmarker}{\pgfqpoint{0.000000in}{-0.048611in}}{\pgfqpoint{0.000000in}{0.000000in}}{%
\pgfpathmoveto{\pgfqpoint{0.000000in}{0.000000in}}%
\pgfpathlineto{\pgfqpoint{0.000000in}{-0.048611in}}%
\pgfusepath{stroke,fill}%
}%
\begin{pgfscope}%
\pgfsys@transformshift{3.185328in}{0.528000in}%
\pgfsys@useobject{currentmarker}{}%
\end{pgfscope}%
\end{pgfscope}%
\begin{pgfscope}%
\pgftext[x=3.185328in,y=0.430778in,,top]{\rmfamily\fontsize{10.000000}{12.000000}\selectfont \(\displaystyle 0\)}%
\end{pgfscope}%
\begin{pgfscope}%
\pgfsetbuttcap%
\pgfsetroundjoin%
\definecolor{currentfill}{rgb}{0.000000,0.000000,0.000000}%
\pgfsetfillcolor{currentfill}%
\pgfsetlinewidth{0.803000pt}%
\definecolor{currentstroke}{rgb}{0.000000,0.000000,0.000000}%
\pgfsetstrokecolor{currentstroke}%
\pgfsetdash{}{0pt}%
\pgfsys@defobject{currentmarker}{\pgfqpoint{0.000000in}{-0.048611in}}{\pgfqpoint{0.000000in}{0.000000in}}{%
\pgfpathmoveto{\pgfqpoint{0.000000in}{0.000000in}}%
\pgfpathlineto{\pgfqpoint{0.000000in}{-0.048611in}}%
\pgfusepath{stroke,fill}%
}%
\begin{pgfscope}%
\pgfsys@transformshift{4.179598in}{0.528000in}%
\pgfsys@useobject{currentmarker}{}%
\end{pgfscope}%
\end{pgfscope}%
\begin{pgfscope}%
\pgftext[x=4.179598in,y=0.430778in,,top]{\rmfamily\fontsize{10.000000}{12.000000}\selectfont \(\displaystyle 1\)}%
\end{pgfscope}%
\begin{pgfscope}%
\pgfsetbuttcap%
\pgfsetroundjoin%
\definecolor{currentfill}{rgb}{0.000000,0.000000,0.000000}%
\pgfsetfillcolor{currentfill}%
\pgfsetlinewidth{0.803000pt}%
\definecolor{currentstroke}{rgb}{0.000000,0.000000,0.000000}%
\pgfsetstrokecolor{currentstroke}%
\pgfsetdash{}{0pt}%
\pgfsys@defobject{currentmarker}{\pgfqpoint{0.000000in}{-0.048611in}}{\pgfqpoint{0.000000in}{0.000000in}}{%
\pgfpathmoveto{\pgfqpoint{0.000000in}{0.000000in}}%
\pgfpathlineto{\pgfqpoint{0.000000in}{-0.048611in}}%
\pgfusepath{stroke,fill}%
}%
\begin{pgfscope}%
\pgfsys@transformshift{5.173869in}{0.528000in}%
\pgfsys@useobject{currentmarker}{}%
\end{pgfscope}%
\end{pgfscope}%
\begin{pgfscope}%
\pgftext[x=5.173869in,y=0.430778in,,top]{\rmfamily\fontsize{10.000000}{12.000000}\selectfont \(\displaystyle 2\)}%
\end{pgfscope}%
\begin{pgfscope}%
\pgfsetbuttcap%
\pgfsetroundjoin%
\definecolor{currentfill}{rgb}{0.000000,0.000000,0.000000}%
\pgfsetfillcolor{currentfill}%
\pgfsetlinewidth{0.803000pt}%
\definecolor{currentstroke}{rgb}{0.000000,0.000000,0.000000}%
\pgfsetstrokecolor{currentstroke}%
\pgfsetdash{}{0pt}%
\pgfsys@defobject{currentmarker}{\pgfqpoint{-0.048611in}{0.000000in}}{\pgfqpoint{0.000000in}{0.000000in}}{%
\pgfpathmoveto{\pgfqpoint{0.000000in}{0.000000in}}%
\pgfpathlineto{\pgfqpoint{-0.048611in}{0.000000in}}%
\pgfusepath{stroke,fill}%
}%
\begin{pgfscope}%
\pgfsys@transformshift{0.800000in}{0.713907in}%
\pgfsys@useobject{currentmarker}{}%
\end{pgfscope}%
\end{pgfscope}%
\begin{pgfscope}%
\pgftext[x=0.417283in,y=0.665712in,left,base]{\rmfamily\fontsize{10.000000}{12.000000}\selectfont \(\displaystyle -0.6\)}%
\end{pgfscope}%
\begin{pgfscope}%
\pgfsetbuttcap%
\pgfsetroundjoin%
\definecolor{currentfill}{rgb}{0.000000,0.000000,0.000000}%
\pgfsetfillcolor{currentfill}%
\pgfsetlinewidth{0.803000pt}%
\definecolor{currentstroke}{rgb}{0.000000,0.000000,0.000000}%
\pgfsetstrokecolor{currentstroke}%
\pgfsetdash{}{0pt}%
\pgfsys@defobject{currentmarker}{\pgfqpoint{-0.048611in}{0.000000in}}{\pgfqpoint{0.000000in}{0.000000in}}{%
\pgfpathmoveto{\pgfqpoint{0.000000in}{0.000000in}}%
\pgfpathlineto{\pgfqpoint{-0.048611in}{0.000000in}}%
\pgfusepath{stroke,fill}%
}%
\begin{pgfscope}%
\pgfsys@transformshift{0.800000in}{1.295897in}%
\pgfsys@useobject{currentmarker}{}%
\end{pgfscope}%
\end{pgfscope}%
\begin{pgfscope}%
\pgftext[x=0.417283in,y=1.247703in,left,base]{\rmfamily\fontsize{10.000000}{12.000000}\selectfont \(\displaystyle -0.4\)}%
\end{pgfscope}%
\begin{pgfscope}%
\pgfsetbuttcap%
\pgfsetroundjoin%
\definecolor{currentfill}{rgb}{0.000000,0.000000,0.000000}%
\pgfsetfillcolor{currentfill}%
\pgfsetlinewidth{0.803000pt}%
\definecolor{currentstroke}{rgb}{0.000000,0.000000,0.000000}%
\pgfsetstrokecolor{currentstroke}%
\pgfsetdash{}{0pt}%
\pgfsys@defobject{currentmarker}{\pgfqpoint{-0.048611in}{0.000000in}}{\pgfqpoint{0.000000in}{0.000000in}}{%
\pgfpathmoveto{\pgfqpoint{0.000000in}{0.000000in}}%
\pgfpathlineto{\pgfqpoint{-0.048611in}{0.000000in}}%
\pgfusepath{stroke,fill}%
}%
\begin{pgfscope}%
\pgfsys@transformshift{0.800000in}{1.877888in}%
\pgfsys@useobject{currentmarker}{}%
\end{pgfscope}%
\end{pgfscope}%
\begin{pgfscope}%
\pgftext[x=0.417283in,y=1.829694in,left,base]{\rmfamily\fontsize{10.000000}{12.000000}\selectfont \(\displaystyle -0.2\)}%
\end{pgfscope}%
\begin{pgfscope}%
\pgfsetbuttcap%
\pgfsetroundjoin%
\definecolor{currentfill}{rgb}{0.000000,0.000000,0.000000}%
\pgfsetfillcolor{currentfill}%
\pgfsetlinewidth{0.803000pt}%
\definecolor{currentstroke}{rgb}{0.000000,0.000000,0.000000}%
\pgfsetstrokecolor{currentstroke}%
\pgfsetdash{}{0pt}%
\pgfsys@defobject{currentmarker}{\pgfqpoint{-0.048611in}{0.000000in}}{\pgfqpoint{0.000000in}{0.000000in}}{%
\pgfpathmoveto{\pgfqpoint{0.000000in}{0.000000in}}%
\pgfpathlineto{\pgfqpoint{-0.048611in}{0.000000in}}%
\pgfusepath{stroke,fill}%
}%
\begin{pgfscope}%
\pgfsys@transformshift{0.800000in}{2.459879in}%
\pgfsys@useobject{currentmarker}{}%
\end{pgfscope}%
\end{pgfscope}%
\begin{pgfscope}%
\pgftext[x=0.525308in,y=2.411684in,left,base]{\rmfamily\fontsize{10.000000}{12.000000}\selectfont \(\displaystyle 0.0\)}%
\end{pgfscope}%
\begin{pgfscope}%
\pgfsetbuttcap%
\pgfsetroundjoin%
\definecolor{currentfill}{rgb}{0.000000,0.000000,0.000000}%
\pgfsetfillcolor{currentfill}%
\pgfsetlinewidth{0.803000pt}%
\definecolor{currentstroke}{rgb}{0.000000,0.000000,0.000000}%
\pgfsetstrokecolor{currentstroke}%
\pgfsetdash{}{0pt}%
\pgfsys@defobject{currentmarker}{\pgfqpoint{-0.048611in}{0.000000in}}{\pgfqpoint{0.000000in}{0.000000in}}{%
\pgfpathmoveto{\pgfqpoint{0.000000in}{0.000000in}}%
\pgfpathlineto{\pgfqpoint{-0.048611in}{0.000000in}}%
\pgfusepath{stroke,fill}%
}%
\begin{pgfscope}%
\pgfsys@transformshift{0.800000in}{3.041869in}%
\pgfsys@useobject{currentmarker}{}%
\end{pgfscope}%
\end{pgfscope}%
\begin{pgfscope}%
\pgftext[x=0.525308in,y=2.993675in,left,base]{\rmfamily\fontsize{10.000000}{12.000000}\selectfont \(\displaystyle 0.2\)}%
\end{pgfscope}%
\begin{pgfscope}%
\pgfsetbuttcap%
\pgfsetroundjoin%
\definecolor{currentfill}{rgb}{0.000000,0.000000,0.000000}%
\pgfsetfillcolor{currentfill}%
\pgfsetlinewidth{0.803000pt}%
\definecolor{currentstroke}{rgb}{0.000000,0.000000,0.000000}%
\pgfsetstrokecolor{currentstroke}%
\pgfsetdash{}{0pt}%
\pgfsys@defobject{currentmarker}{\pgfqpoint{-0.048611in}{0.000000in}}{\pgfqpoint{0.000000in}{0.000000in}}{%
\pgfpathmoveto{\pgfqpoint{0.000000in}{0.000000in}}%
\pgfpathlineto{\pgfqpoint{-0.048611in}{0.000000in}}%
\pgfusepath{stroke,fill}%
}%
\begin{pgfscope}%
\pgfsys@transformshift{0.800000in}{3.623860in}%
\pgfsys@useobject{currentmarker}{}%
\end{pgfscope}%
\end{pgfscope}%
\begin{pgfscope}%
\pgftext[x=0.525308in,y=3.575666in,left,base]{\rmfamily\fontsize{10.000000}{12.000000}\selectfont \(\displaystyle 0.4\)}%
\end{pgfscope}%
\begin{pgfscope}%
\pgfsetbuttcap%
\pgfsetroundjoin%
\definecolor{currentfill}{rgb}{0.000000,0.000000,0.000000}%
\pgfsetfillcolor{currentfill}%
\pgfsetlinewidth{0.803000pt}%
\definecolor{currentstroke}{rgb}{0.000000,0.000000,0.000000}%
\pgfsetstrokecolor{currentstroke}%
\pgfsetdash{}{0pt}%
\pgfsys@defobject{currentmarker}{\pgfqpoint{-0.048611in}{0.000000in}}{\pgfqpoint{0.000000in}{0.000000in}}{%
\pgfpathmoveto{\pgfqpoint{0.000000in}{0.000000in}}%
\pgfpathlineto{\pgfqpoint{-0.048611in}{0.000000in}}%
\pgfusepath{stroke,fill}%
}%
\begin{pgfscope}%
\pgfsys@transformshift{0.800000in}{4.205851in}%
\pgfsys@useobject{currentmarker}{}%
\end{pgfscope}%
\end{pgfscope}%
\begin{pgfscope}%
\pgftext[x=0.525308in,y=4.157656in,left,base]{\rmfamily\fontsize{10.000000}{12.000000}\selectfont \(\displaystyle 0.6\)}%
\end{pgfscope}%
\begin{pgfscope}%
\pgfsetrectcap%
\pgfsetmiterjoin%
\pgfsetlinewidth{0.803000pt}%
\definecolor{currentstroke}{rgb}{0.000000,0.000000,0.000000}%
\pgfsetstrokecolor{currentstroke}%
\pgfsetdash{}{0pt}%
\pgfpathmoveto{\pgfqpoint{0.800000in}{0.528000in}}%
\pgfpathlineto{\pgfqpoint{0.800000in}{4.224000in}}%
\pgfusepath{stroke}%
\end{pgfscope}%
\begin{pgfscope}%
\pgfsetrectcap%
\pgfsetmiterjoin%
\pgfsetlinewidth{0.803000pt}%
\definecolor{currentstroke}{rgb}{0.000000,0.000000,0.000000}%
\pgfsetstrokecolor{currentstroke}%
\pgfsetdash{}{0pt}%
\pgfpathmoveto{\pgfqpoint{5.760000in}{0.528000in}}%
\pgfpathlineto{\pgfqpoint{5.760000in}{4.224000in}}%
\pgfusepath{stroke}%
\end{pgfscope}%
\begin{pgfscope}%
\pgfsetrectcap%
\pgfsetmiterjoin%
\pgfsetlinewidth{0.803000pt}%
\definecolor{currentstroke}{rgb}{0.000000,0.000000,0.000000}%
\pgfsetstrokecolor{currentstroke}%
\pgfsetdash{}{0pt}%
\pgfpathmoveto{\pgfqpoint{0.800000in}{0.528000in}}%
\pgfpathlineto{\pgfqpoint{5.760000in}{0.528000in}}%
\pgfusepath{stroke}%
\end{pgfscope}%
\begin{pgfscope}%
\pgfsetrectcap%
\pgfsetmiterjoin%
\pgfsetlinewidth{0.803000pt}%
\definecolor{currentstroke}{rgb}{0.000000,0.000000,0.000000}%
\pgfsetstrokecolor{currentstroke}%
\pgfsetdash{}{0pt}%
\pgfpathmoveto{\pgfqpoint{0.800000in}{4.224000in}}%
\pgfpathlineto{\pgfqpoint{5.760000in}{4.224000in}}%
\pgfusepath{stroke}%
\end{pgfscope}%
\end{pgfpicture}%
\makeatother%
\endgroup%
}
\scalebox{0.4}{%% Creator: Matplotlib, PGF backend
%%
%% To include the figure in your LaTeX document, write
%%   \input{<filename>.pgf}
%%
%% Make sure the required packages are loaded in your preamble
%%   \usepackage{pgf}
%%
%% Figures using additional raster images can only be included by \input if
%% they are in the same directory as the main LaTeX file. For loading figures
%% from other directories you can use the `import` package
%%   \usepackage{import}
%% and then include the figures with
%%   \import{<path to file>}{<filename>.pgf}
%%
%% Matplotlib used the following preamble
%%   \usepackage{fontspec}
%%
\begingroup%
\makeatletter%
\begin{pgfpicture}%
\pgfpathrectangle{\pgfpointorigin}{\pgfqpoint{6.400000in}{4.800000in}}%
\pgfusepath{use as bounding box, clip}%
\begin{pgfscope}%
\pgfsetbuttcap%
\pgfsetmiterjoin%
\definecolor{currentfill}{rgb}{1.000000,1.000000,1.000000}%
\pgfsetfillcolor{currentfill}%
\pgfsetlinewidth{0.000000pt}%
\definecolor{currentstroke}{rgb}{1.000000,1.000000,1.000000}%
\pgfsetstrokecolor{currentstroke}%
\pgfsetdash{}{0pt}%
\pgfpathmoveto{\pgfqpoint{0.000000in}{0.000000in}}%
\pgfpathlineto{\pgfqpoint{6.400000in}{0.000000in}}%
\pgfpathlineto{\pgfqpoint{6.400000in}{4.800000in}}%
\pgfpathlineto{\pgfqpoint{0.000000in}{4.800000in}}%
\pgfpathclose%
\pgfusepath{fill}%
\end{pgfscope}%
\begin{pgfscope}%
\pgfsetbuttcap%
\pgfsetmiterjoin%
\definecolor{currentfill}{rgb}{1.000000,1.000000,1.000000}%
\pgfsetfillcolor{currentfill}%
\pgfsetlinewidth{0.000000pt}%
\definecolor{currentstroke}{rgb}{0.000000,0.000000,0.000000}%
\pgfsetstrokecolor{currentstroke}%
\pgfsetstrokeopacity{0.000000}%
\pgfsetdash{}{0pt}%
\pgfpathmoveto{\pgfqpoint{0.800000in}{0.528000in}}%
\pgfpathlineto{\pgfqpoint{5.760000in}{0.528000in}}%
\pgfpathlineto{\pgfqpoint{5.760000in}{4.224000in}}%
\pgfpathlineto{\pgfqpoint{0.800000in}{4.224000in}}%
\pgfpathclose%
\pgfusepath{fill}%
\end{pgfscope}%
\begin{pgfscope}%
\pgfpathrectangle{\pgfqpoint{0.800000in}{0.528000in}}{\pgfqpoint{4.960000in}{3.696000in}} %
\pgfusepath{clip}%
\pgfsetbuttcap%
\pgfsetroundjoin%
\definecolor{currentfill}{rgb}{0.121569,0.466667,0.705882}%
\pgfsetfillcolor{currentfill}%
\pgfsetlinewidth{1.003750pt}%
\definecolor{currentstroke}{rgb}{0.121569,0.466667,0.705882}%
\pgfsetstrokecolor{currentstroke}%
\pgfsetdash{}{0pt}%
\pgfpathmoveto{\pgfqpoint{1.673483in}{1.012308in}}%
\pgfpathcurveto{\pgfqpoint{1.683570in}{1.012308in}}{\pgfqpoint{1.693246in}{1.016316in}}{\pgfqpoint{1.700378in}{1.023448in}}%
\pgfpathcurveto{\pgfqpoint{1.707511in}{1.030581in}}{\pgfqpoint{1.711519in}{1.040257in}}{\pgfqpoint{1.711519in}{1.050344in}}%
\pgfpathcurveto{\pgfqpoint{1.711519in}{1.060431in}}{\pgfqpoint{1.707511in}{1.070107in}}{\pgfqpoint{1.700378in}{1.077240in}}%
\pgfpathcurveto{\pgfqpoint{1.693246in}{1.084373in}}{\pgfqpoint{1.683570in}{1.088380in}}{\pgfqpoint{1.673483in}{1.088380in}}%
\pgfpathcurveto{\pgfqpoint{1.663395in}{1.088380in}}{\pgfqpoint{1.653720in}{1.084373in}}{\pgfqpoint{1.646587in}{1.077240in}}%
\pgfpathcurveto{\pgfqpoint{1.639454in}{1.070107in}}{\pgfqpoint{1.635446in}{1.060431in}}{\pgfqpoint{1.635446in}{1.050344in}}%
\pgfpathcurveto{\pgfqpoint{1.635446in}{1.040257in}}{\pgfqpoint{1.639454in}{1.030581in}}{\pgfqpoint{1.646587in}{1.023448in}}%
\pgfpathcurveto{\pgfqpoint{1.653720in}{1.016316in}}{\pgfqpoint{1.663395in}{1.012308in}}{\pgfqpoint{1.673483in}{1.012308in}}%
\pgfpathclose%
\pgfusepath{stroke,fill}%
\end{pgfscope}%
\begin{pgfscope}%
\pgfpathrectangle{\pgfqpoint{0.800000in}{0.528000in}}{\pgfqpoint{4.960000in}{3.696000in}} %
\pgfusepath{clip}%
\pgfsetbuttcap%
\pgfsetroundjoin%
\definecolor{currentfill}{rgb}{0.121569,0.466667,0.705882}%
\pgfsetfillcolor{currentfill}%
\pgfsetlinewidth{1.003750pt}%
\definecolor{currentstroke}{rgb}{0.121569,0.466667,0.705882}%
\pgfsetstrokecolor{currentstroke}%
\pgfsetdash{}{0pt}%
\pgfpathmoveto{\pgfqpoint{1.342173in}{1.561903in}}%
\pgfpathcurveto{\pgfqpoint{1.352261in}{1.561903in}}{\pgfqpoint{1.361936in}{1.565911in}}{\pgfqpoint{1.369069in}{1.573044in}}%
\pgfpathcurveto{\pgfqpoint{1.376202in}{1.580177in}}{\pgfqpoint{1.380210in}{1.589852in}}{\pgfqpoint{1.380210in}{1.599940in}}%
\pgfpathcurveto{\pgfqpoint{1.380210in}{1.610027in}}{\pgfqpoint{1.376202in}{1.619703in}}{\pgfqpoint{1.369069in}{1.626835in}}%
\pgfpathcurveto{\pgfqpoint{1.361936in}{1.633968in}}{\pgfqpoint{1.352261in}{1.637976in}}{\pgfqpoint{1.342173in}{1.637976in}}%
\pgfpathcurveto{\pgfqpoint{1.332086in}{1.637976in}}{\pgfqpoint{1.322411in}{1.633968in}}{\pgfqpoint{1.315278in}{1.626835in}}%
\pgfpathcurveto{\pgfqpoint{1.308145in}{1.619703in}}{\pgfqpoint{1.304137in}{1.610027in}}{\pgfqpoint{1.304137in}{1.599940in}}%
\pgfpathcurveto{\pgfqpoint{1.304137in}{1.589852in}}{\pgfqpoint{1.308145in}{1.580177in}}{\pgfqpoint{1.315278in}{1.573044in}}%
\pgfpathcurveto{\pgfqpoint{1.322411in}{1.565911in}}{\pgfqpoint{1.332086in}{1.561903in}}{\pgfqpoint{1.342173in}{1.561903in}}%
\pgfpathclose%
\pgfusepath{stroke,fill}%
\end{pgfscope}%
\begin{pgfscope}%
\pgfpathrectangle{\pgfqpoint{0.800000in}{0.528000in}}{\pgfqpoint{4.960000in}{3.696000in}} %
\pgfusepath{clip}%
\pgfsetbuttcap%
\pgfsetroundjoin%
\definecolor{currentfill}{rgb}{0.121569,0.466667,0.705882}%
\pgfsetfillcolor{currentfill}%
\pgfsetlinewidth{1.003750pt}%
\definecolor{currentstroke}{rgb}{0.121569,0.466667,0.705882}%
\pgfsetstrokecolor{currentstroke}%
\pgfsetdash{}{0pt}%
\pgfpathmoveto{\pgfqpoint{1.031720in}{2.142608in}}%
\pgfpathcurveto{\pgfqpoint{1.041808in}{2.142608in}}{\pgfqpoint{1.051483in}{2.146615in}}{\pgfqpoint{1.058616in}{2.153748in}}%
\pgfpathcurveto{\pgfqpoint{1.065749in}{2.160881in}}{\pgfqpoint{1.069757in}{2.170557in}}{\pgfqpoint{1.069757in}{2.180644in}}%
\pgfpathcurveto{\pgfqpoint{1.069757in}{2.190731in}}{\pgfqpoint{1.065749in}{2.200407in}}{\pgfqpoint{1.058616in}{2.207540in}}%
\pgfpathcurveto{\pgfqpoint{1.051483in}{2.214673in}}{\pgfqpoint{1.041808in}{2.218680in}}{\pgfqpoint{1.031720in}{2.218680in}}%
\pgfpathcurveto{\pgfqpoint{1.021633in}{2.218680in}}{\pgfqpoint{1.011957in}{2.214673in}}{\pgfqpoint{1.004825in}{2.207540in}}%
\pgfpathcurveto{\pgfqpoint{0.997692in}{2.200407in}}{\pgfqpoint{0.993684in}{2.190731in}}{\pgfqpoint{0.993684in}{2.180644in}}%
\pgfpathcurveto{\pgfqpoint{0.993684in}{2.170557in}}{\pgfqpoint{0.997692in}{2.160881in}}{\pgfqpoint{1.004825in}{2.153748in}}%
\pgfpathcurveto{\pgfqpoint{1.011957in}{2.146615in}}{\pgfqpoint{1.021633in}{2.142608in}}{\pgfqpoint{1.031720in}{2.142608in}}%
\pgfpathclose%
\pgfusepath{stroke,fill}%
\end{pgfscope}%
\begin{pgfscope}%
\pgfpathrectangle{\pgfqpoint{0.800000in}{0.528000in}}{\pgfqpoint{4.960000in}{3.696000in}} %
\pgfusepath{clip}%
\pgfsetbuttcap%
\pgfsetroundjoin%
\definecolor{currentfill}{rgb}{0.121569,0.466667,0.705882}%
\pgfsetfillcolor{currentfill}%
\pgfsetlinewidth{1.003750pt}%
\definecolor{currentstroke}{rgb}{0.121569,0.466667,0.705882}%
\pgfsetstrokecolor{currentstroke}%
\pgfsetdash{}{0pt}%
\pgfpathmoveto{\pgfqpoint{1.356249in}{3.160048in}}%
\pgfpathcurveto{\pgfqpoint{1.366337in}{3.160048in}}{\pgfqpoint{1.376012in}{3.164056in}}{\pgfqpoint{1.383145in}{3.171189in}}%
\pgfpathcurveto{\pgfqpoint{1.390278in}{3.178321in}}{\pgfqpoint{1.394286in}{3.187997in}}{\pgfqpoint{1.394286in}{3.198084in}}%
\pgfpathcurveto{\pgfqpoint{1.394286in}{3.208172in}}{\pgfqpoint{1.390278in}{3.217847in}}{\pgfqpoint{1.383145in}{3.224980in}}%
\pgfpathcurveto{\pgfqpoint{1.376012in}{3.232113in}}{\pgfqpoint{1.366337in}{3.236121in}}{\pgfqpoint{1.356249in}{3.236121in}}%
\pgfpathcurveto{\pgfqpoint{1.346162in}{3.236121in}}{\pgfqpoint{1.336486in}{3.232113in}}{\pgfqpoint{1.329354in}{3.224980in}}%
\pgfpathcurveto{\pgfqpoint{1.322221in}{3.217847in}}{\pgfqpoint{1.318213in}{3.208172in}}{\pgfqpoint{1.318213in}{3.198084in}}%
\pgfpathcurveto{\pgfqpoint{1.318213in}{3.187997in}}{\pgfqpoint{1.322221in}{3.178321in}}{\pgfqpoint{1.329354in}{3.171189in}}%
\pgfpathcurveto{\pgfqpoint{1.336486in}{3.164056in}}{\pgfqpoint{1.346162in}{3.160048in}}{\pgfqpoint{1.356249in}{3.160048in}}%
\pgfpathclose%
\pgfusepath{stroke,fill}%
\end{pgfscope}%
\begin{pgfscope}%
\pgfpathrectangle{\pgfqpoint{0.800000in}{0.528000in}}{\pgfqpoint{4.960000in}{3.696000in}} %
\pgfusepath{clip}%
\pgfsetbuttcap%
\pgfsetroundjoin%
\definecolor{currentfill}{rgb}{0.121569,0.466667,0.705882}%
\pgfsetfillcolor{currentfill}%
\pgfsetlinewidth{1.003750pt}%
\definecolor{currentstroke}{rgb}{0.121569,0.466667,0.705882}%
\pgfsetstrokecolor{currentstroke}%
\pgfsetdash{}{0pt}%
\pgfpathmoveto{\pgfqpoint{1.528570in}{3.587024in}}%
\pgfpathcurveto{\pgfqpoint{1.538657in}{3.587024in}}{\pgfqpoint{1.548333in}{3.591032in}}{\pgfqpoint{1.555466in}{3.598165in}}%
\pgfpathcurveto{\pgfqpoint{1.562599in}{3.605297in}}{\pgfqpoint{1.566606in}{3.614973in}}{\pgfqpoint{1.566606in}{3.625060in}}%
\pgfpathcurveto{\pgfqpoint{1.566606in}{3.635148in}}{\pgfqpoint{1.562599in}{3.644823in}}{\pgfqpoint{1.555466in}{3.651956in}}%
\pgfpathcurveto{\pgfqpoint{1.548333in}{3.659089in}}{\pgfqpoint{1.538657in}{3.663097in}}{\pgfqpoint{1.528570in}{3.663097in}}%
\pgfpathcurveto{\pgfqpoint{1.518483in}{3.663097in}}{\pgfqpoint{1.508807in}{3.659089in}}{\pgfqpoint{1.501674in}{3.651956in}}%
\pgfpathcurveto{\pgfqpoint{1.494541in}{3.644823in}}{\pgfqpoint{1.490534in}{3.635148in}}{\pgfqpoint{1.490534in}{3.625060in}}%
\pgfpathcurveto{\pgfqpoint{1.490534in}{3.614973in}}{\pgfqpoint{1.494541in}{3.605297in}}{\pgfqpoint{1.501674in}{3.598165in}}%
\pgfpathcurveto{\pgfqpoint{1.508807in}{3.591032in}}{\pgfqpoint{1.518483in}{3.587024in}}{\pgfqpoint{1.528570in}{3.587024in}}%
\pgfpathclose%
\pgfusepath{stroke,fill}%
\end{pgfscope}%
\begin{pgfscope}%
\pgfpathrectangle{\pgfqpoint{0.800000in}{0.528000in}}{\pgfqpoint{4.960000in}{3.696000in}} %
\pgfusepath{clip}%
\pgfsetbuttcap%
\pgfsetroundjoin%
\definecolor{currentfill}{rgb}{0.121569,0.466667,0.705882}%
\pgfsetfillcolor{currentfill}%
\pgfsetlinewidth{1.003750pt}%
\definecolor{currentstroke}{rgb}{0.121569,0.466667,0.705882}%
\pgfsetstrokecolor{currentstroke}%
\pgfsetdash{}{0pt}%
\pgfpathmoveto{\pgfqpoint{2.428361in}{0.692038in}}%
\pgfpathcurveto{\pgfqpoint{2.438448in}{0.692038in}}{\pgfqpoint{2.448123in}{0.696046in}}{\pgfqpoint{2.455256in}{0.703179in}}%
\pgfpathcurveto{\pgfqpoint{2.462389in}{0.710312in}}{\pgfqpoint{2.466397in}{0.719987in}}{\pgfqpoint{2.466397in}{0.730075in}}%
\pgfpathcurveto{\pgfqpoint{2.466397in}{0.740162in}}{\pgfqpoint{2.462389in}{0.749838in}}{\pgfqpoint{2.455256in}{0.756970in}}%
\pgfpathcurveto{\pgfqpoint{2.448123in}{0.764103in}}{\pgfqpoint{2.438448in}{0.768111in}}{\pgfqpoint{2.428361in}{0.768111in}}%
\pgfpathcurveto{\pgfqpoint{2.418273in}{0.768111in}}{\pgfqpoint{2.408598in}{0.764103in}}{\pgfqpoint{2.401465in}{0.756970in}}%
\pgfpathcurveto{\pgfqpoint{2.394332in}{0.749838in}}{\pgfqpoint{2.390324in}{0.740162in}}{\pgfqpoint{2.390324in}{0.730075in}}%
\pgfpathcurveto{\pgfqpoint{2.390324in}{0.719987in}}{\pgfqpoint{2.394332in}{0.710312in}}{\pgfqpoint{2.401465in}{0.703179in}}%
\pgfpathcurveto{\pgfqpoint{2.408598in}{0.696046in}}{\pgfqpoint{2.418273in}{0.692038in}}{\pgfqpoint{2.428361in}{0.692038in}}%
\pgfpathclose%
\pgfusepath{stroke,fill}%
\end{pgfscope}%
\begin{pgfscope}%
\pgfpathrectangle{\pgfqpoint{0.800000in}{0.528000in}}{\pgfqpoint{4.960000in}{3.696000in}} %
\pgfusepath{clip}%
\pgfsetbuttcap%
\pgfsetroundjoin%
\definecolor{currentfill}{rgb}{0.121569,0.466667,0.705882}%
\pgfsetfillcolor{currentfill}%
\pgfsetlinewidth{1.003750pt}%
\definecolor{currentstroke}{rgb}{0.121569,0.466667,0.705882}%
\pgfsetstrokecolor{currentstroke}%
\pgfsetdash{}{0pt}%
\pgfpathmoveto{\pgfqpoint{2.524795in}{1.092081in}}%
\pgfpathcurveto{\pgfqpoint{2.534882in}{1.092081in}}{\pgfqpoint{2.544558in}{1.096089in}}{\pgfqpoint{2.551690in}{1.103222in}}%
\pgfpathcurveto{\pgfqpoint{2.558823in}{1.110354in}}{\pgfqpoint{2.562831in}{1.120030in}}{\pgfqpoint{2.562831in}{1.130117in}}%
\pgfpathcurveto{\pgfqpoint{2.562831in}{1.140205in}}{\pgfqpoint{2.558823in}{1.149880in}}{\pgfqpoint{2.551690in}{1.157013in}}%
\pgfpathcurveto{\pgfqpoint{2.544558in}{1.164146in}}{\pgfqpoint{2.534882in}{1.168154in}}{\pgfqpoint{2.524795in}{1.168154in}}%
\pgfpathcurveto{\pgfqpoint{2.514707in}{1.168154in}}{\pgfqpoint{2.505032in}{1.164146in}}{\pgfqpoint{2.497899in}{1.157013in}}%
\pgfpathcurveto{\pgfqpoint{2.490766in}{1.149880in}}{\pgfqpoint{2.486758in}{1.140205in}}{\pgfqpoint{2.486758in}{1.130117in}}%
\pgfpathcurveto{\pgfqpoint{2.486758in}{1.120030in}}{\pgfqpoint{2.490766in}{1.110354in}}{\pgfqpoint{2.497899in}{1.103222in}}%
\pgfpathcurveto{\pgfqpoint{2.505032in}{1.096089in}}{\pgfqpoint{2.514707in}{1.092081in}}{\pgfqpoint{2.524795in}{1.092081in}}%
\pgfpathclose%
\pgfusepath{stroke,fill}%
\end{pgfscope}%
\begin{pgfscope}%
\pgfpathrectangle{\pgfqpoint{0.800000in}{0.528000in}}{\pgfqpoint{4.960000in}{3.696000in}} %
\pgfusepath{clip}%
\pgfsetbuttcap%
\pgfsetroundjoin%
\definecolor{currentfill}{rgb}{0.121569,0.466667,0.705882}%
\pgfsetfillcolor{currentfill}%
\pgfsetlinewidth{1.003750pt}%
\definecolor{currentstroke}{rgb}{0.121569,0.466667,0.705882}%
\pgfsetstrokecolor{currentstroke}%
\pgfsetdash{}{0pt}%
\pgfpathmoveto{\pgfqpoint{2.371871in}{3.992571in}}%
\pgfpathcurveto{\pgfqpoint{2.381958in}{3.992571in}}{\pgfqpoint{2.391634in}{3.996578in}}{\pgfqpoint{2.398766in}{4.003711in}}%
\pgfpathcurveto{\pgfqpoint{2.405899in}{4.010844in}}{\pgfqpoint{2.409907in}{4.020520in}}{\pgfqpoint{2.409907in}{4.030607in}}%
\pgfpathcurveto{\pgfqpoint{2.409907in}{4.040694in}}{\pgfqpoint{2.405899in}{4.050370in}}{\pgfqpoint{2.398766in}{4.057503in}}%
\pgfpathcurveto{\pgfqpoint{2.391634in}{4.064636in}}{\pgfqpoint{2.381958in}{4.068643in}}{\pgfqpoint{2.371871in}{4.068643in}}%
\pgfpathcurveto{\pgfqpoint{2.361783in}{4.068643in}}{\pgfqpoint{2.352108in}{4.064636in}}{\pgfqpoint{2.344975in}{4.057503in}}%
\pgfpathcurveto{\pgfqpoint{2.337842in}{4.050370in}}{\pgfqpoint{2.333834in}{4.040694in}}{\pgfqpoint{2.333834in}{4.030607in}}%
\pgfpathcurveto{\pgfqpoint{2.333834in}{4.020520in}}{\pgfqpoint{2.337842in}{4.010844in}}{\pgfqpoint{2.344975in}{4.003711in}}%
\pgfpathcurveto{\pgfqpoint{2.352108in}{3.996578in}}{\pgfqpoint{2.361783in}{3.992571in}}{\pgfqpoint{2.371871in}{3.992571in}}%
\pgfpathclose%
\pgfusepath{stroke,fill}%
\end{pgfscope}%
\begin{pgfscope}%
\pgfpathrectangle{\pgfqpoint{0.800000in}{0.528000in}}{\pgfqpoint{4.960000in}{3.696000in}} %
\pgfusepath{clip}%
\pgfsetbuttcap%
\pgfsetroundjoin%
\definecolor{currentfill}{rgb}{0.121569,0.466667,0.705882}%
\pgfsetfillcolor{currentfill}%
\pgfsetlinewidth{1.003750pt}%
\definecolor{currentstroke}{rgb}{0.121569,0.466667,0.705882}%
\pgfsetstrokecolor{currentstroke}%
\pgfsetdash{}{0pt}%
\pgfpathmoveto{\pgfqpoint{3.317771in}{0.881055in}}%
\pgfpathcurveto{\pgfqpoint{3.327858in}{0.881055in}}{\pgfqpoint{3.337534in}{0.885062in}}{\pgfqpoint{3.344667in}{0.892195in}}%
\pgfpathcurveto{\pgfqpoint{3.351799in}{0.899328in}}{\pgfqpoint{3.355807in}{0.909004in}}{\pgfqpoint{3.355807in}{0.919091in}}%
\pgfpathcurveto{\pgfqpoint{3.355807in}{0.929178in}}{\pgfqpoint{3.351799in}{0.938854in}}{\pgfqpoint{3.344667in}{0.945987in}}%
\pgfpathcurveto{\pgfqpoint{3.337534in}{0.953120in}}{\pgfqpoint{3.327858in}{0.957127in}}{\pgfqpoint{3.317771in}{0.957127in}}%
\pgfpathcurveto{\pgfqpoint{3.307683in}{0.957127in}}{\pgfqpoint{3.298008in}{0.953120in}}{\pgfqpoint{3.290875in}{0.945987in}}%
\pgfpathcurveto{\pgfqpoint{3.283742in}{0.938854in}}{\pgfqpoint{3.279735in}{0.929178in}}{\pgfqpoint{3.279735in}{0.919091in}}%
\pgfpathcurveto{\pgfqpoint{3.279735in}{0.909004in}}{\pgfqpoint{3.283742in}{0.899328in}}{\pgfqpoint{3.290875in}{0.892195in}}%
\pgfpathcurveto{\pgfqpoint{3.298008in}{0.885062in}}{\pgfqpoint{3.307683in}{0.881055in}}{\pgfqpoint{3.317771in}{0.881055in}}%
\pgfpathclose%
\pgfusepath{stroke,fill}%
\end{pgfscope}%
\begin{pgfscope}%
\pgfpathrectangle{\pgfqpoint{0.800000in}{0.528000in}}{\pgfqpoint{4.960000in}{3.696000in}} %
\pgfusepath{clip}%
\pgfsetbuttcap%
\pgfsetroundjoin%
\definecolor{currentfill}{rgb}{0.121569,0.466667,0.705882}%
\pgfsetfillcolor{currentfill}%
\pgfsetlinewidth{1.003750pt}%
\definecolor{currentstroke}{rgb}{0.121569,0.466667,0.705882}%
\pgfsetstrokecolor{currentstroke}%
\pgfsetdash{}{0pt}%
\pgfpathmoveto{\pgfqpoint{3.188228in}{3.959324in}}%
\pgfpathcurveto{\pgfqpoint{3.198316in}{3.959324in}}{\pgfqpoint{3.207991in}{3.963332in}}{\pgfqpoint{3.215124in}{3.970465in}}%
\pgfpathcurveto{\pgfqpoint{3.222257in}{3.977598in}}{\pgfqpoint{3.226265in}{3.987273in}}{\pgfqpoint{3.226265in}{3.997361in}}%
\pgfpathcurveto{\pgfqpoint{3.226265in}{4.007448in}}{\pgfqpoint{3.222257in}{4.017123in}}{\pgfqpoint{3.215124in}{4.024256in}}%
\pgfpathcurveto{\pgfqpoint{3.207991in}{4.031389in}}{\pgfqpoint{3.198316in}{4.035397in}}{\pgfqpoint{3.188228in}{4.035397in}}%
\pgfpathcurveto{\pgfqpoint{3.178141in}{4.035397in}}{\pgfqpoint{3.168465in}{4.031389in}}{\pgfqpoint{3.161333in}{4.024256in}}%
\pgfpathcurveto{\pgfqpoint{3.154200in}{4.017123in}}{\pgfqpoint{3.150192in}{4.007448in}}{\pgfqpoint{3.150192in}{3.997361in}}%
\pgfpathcurveto{\pgfqpoint{3.150192in}{3.987273in}}{\pgfqpoint{3.154200in}{3.977598in}}{\pgfqpoint{3.161333in}{3.970465in}}%
\pgfpathcurveto{\pgfqpoint{3.168465in}{3.963332in}}{\pgfqpoint{3.178141in}{3.959324in}}{\pgfqpoint{3.188228in}{3.959324in}}%
\pgfpathclose%
\pgfusepath{stroke,fill}%
\end{pgfscope}%
\begin{pgfscope}%
\pgfpathrectangle{\pgfqpoint{0.800000in}{0.528000in}}{\pgfqpoint{4.960000in}{3.696000in}} %
\pgfusepath{clip}%
\pgfsetbuttcap%
\pgfsetroundjoin%
\definecolor{currentfill}{rgb}{0.121569,0.466667,0.705882}%
\pgfsetfillcolor{currentfill}%
\pgfsetlinewidth{1.003750pt}%
\definecolor{currentstroke}{rgb}{0.121569,0.466667,0.705882}%
\pgfsetstrokecolor{currentstroke}%
\pgfsetdash{}{0pt}%
\pgfpathmoveto{\pgfqpoint{4.207830in}{0.683357in}}%
\pgfpathcurveto{\pgfqpoint{4.217917in}{0.683357in}}{\pgfqpoint{4.227593in}{0.687364in}}{\pgfqpoint{4.234726in}{0.694497in}}%
\pgfpathcurveto{\pgfqpoint{4.241858in}{0.701630in}}{\pgfqpoint{4.245866in}{0.711306in}}{\pgfqpoint{4.245866in}{0.721393in}}%
\pgfpathcurveto{\pgfqpoint{4.245866in}{0.731480in}}{\pgfqpoint{4.241858in}{0.741156in}}{\pgfqpoint{4.234726in}{0.748289in}}%
\pgfpathcurveto{\pgfqpoint{4.227593in}{0.755422in}}{\pgfqpoint{4.217917in}{0.759429in}}{\pgfqpoint{4.207830in}{0.759429in}}%
\pgfpathcurveto{\pgfqpoint{4.197743in}{0.759429in}}{\pgfqpoint{4.188067in}{0.755422in}}{\pgfqpoint{4.180934in}{0.748289in}}%
\pgfpathcurveto{\pgfqpoint{4.173801in}{0.741156in}}{\pgfqpoint{4.169794in}{0.731480in}}{\pgfqpoint{4.169794in}{0.721393in}}%
\pgfpathcurveto{\pgfqpoint{4.169794in}{0.711306in}}{\pgfqpoint{4.173801in}{0.701630in}}{\pgfqpoint{4.180934in}{0.694497in}}%
\pgfpathcurveto{\pgfqpoint{4.188067in}{0.687364in}}{\pgfqpoint{4.197743in}{0.683357in}}{\pgfqpoint{4.207830in}{0.683357in}}%
\pgfpathclose%
\pgfusepath{stroke,fill}%
\end{pgfscope}%
\begin{pgfscope}%
\pgfpathrectangle{\pgfqpoint{0.800000in}{0.528000in}}{\pgfqpoint{4.960000in}{3.696000in}} %
\pgfusepath{clip}%
\pgfsetbuttcap%
\pgfsetroundjoin%
\definecolor{currentfill}{rgb}{0.121569,0.466667,0.705882}%
\pgfsetfillcolor{currentfill}%
\pgfsetlinewidth{1.003750pt}%
\definecolor{currentstroke}{rgb}{0.121569,0.466667,0.705882}%
\pgfsetstrokecolor{currentstroke}%
\pgfsetdash{}{0pt}%
\pgfpathmoveto{\pgfqpoint{4.827101in}{1.237602in}}%
\pgfpathcurveto{\pgfqpoint{4.837189in}{1.237602in}}{\pgfqpoint{4.846864in}{1.241610in}}{\pgfqpoint{4.853997in}{1.248743in}}%
\pgfpathcurveto{\pgfqpoint{4.861130in}{1.255875in}}{\pgfqpoint{4.865138in}{1.265551in}}{\pgfqpoint{4.865138in}{1.275638in}}%
\pgfpathcurveto{\pgfqpoint{4.865138in}{1.285726in}}{\pgfqpoint{4.861130in}{1.295401in}}{\pgfqpoint{4.853997in}{1.302534in}}%
\pgfpathcurveto{\pgfqpoint{4.846864in}{1.309667in}}{\pgfqpoint{4.837189in}{1.313675in}}{\pgfqpoint{4.827101in}{1.313675in}}%
\pgfpathcurveto{\pgfqpoint{4.817014in}{1.313675in}}{\pgfqpoint{4.807338in}{1.309667in}}{\pgfqpoint{4.800206in}{1.302534in}}%
\pgfpathcurveto{\pgfqpoint{4.793073in}{1.295401in}}{\pgfqpoint{4.789065in}{1.285726in}}{\pgfqpoint{4.789065in}{1.275638in}}%
\pgfpathcurveto{\pgfqpoint{4.789065in}{1.265551in}}{\pgfqpoint{4.793073in}{1.255875in}}{\pgfqpoint{4.800206in}{1.248743in}}%
\pgfpathcurveto{\pgfqpoint{4.807338in}{1.241610in}}{\pgfqpoint{4.817014in}{1.237602in}}{\pgfqpoint{4.827101in}{1.237602in}}%
\pgfpathclose%
\pgfusepath{stroke,fill}%
\end{pgfscope}%
\begin{pgfscope}%
\pgfpathrectangle{\pgfqpoint{0.800000in}{0.528000in}}{\pgfqpoint{4.960000in}{3.696000in}} %
\pgfusepath{clip}%
\pgfsetbuttcap%
\pgfsetroundjoin%
\definecolor{currentfill}{rgb}{0.121569,0.466667,0.705882}%
\pgfsetfillcolor{currentfill}%
\pgfsetlinewidth{1.003750pt}%
\definecolor{currentstroke}{rgb}{0.121569,0.466667,0.705882}%
\pgfsetstrokecolor{currentstroke}%
\pgfsetdash{}{0pt}%
\pgfpathmoveto{\pgfqpoint{4.468060in}{3.772899in}}%
\pgfpathcurveto{\pgfqpoint{4.478147in}{3.772899in}}{\pgfqpoint{4.487822in}{3.776907in}}{\pgfqpoint{4.494955in}{3.784039in}}%
\pgfpathcurveto{\pgfqpoint{4.502088in}{3.791172in}}{\pgfqpoint{4.506096in}{3.800848in}}{\pgfqpoint{4.506096in}{3.810935in}}%
\pgfpathcurveto{\pgfqpoint{4.506096in}{3.821022in}}{\pgfqpoint{4.502088in}{3.830698in}}{\pgfqpoint{4.494955in}{3.837831in}}%
\pgfpathcurveto{\pgfqpoint{4.487822in}{3.844964in}}{\pgfqpoint{4.478147in}{3.848971in}}{\pgfqpoint{4.468060in}{3.848971in}}%
\pgfpathcurveto{\pgfqpoint{4.457972in}{3.848971in}}{\pgfqpoint{4.448297in}{3.844964in}}{\pgfqpoint{4.441164in}{3.837831in}}%
\pgfpathcurveto{\pgfqpoint{4.434031in}{3.830698in}}{\pgfqpoint{4.430023in}{3.821022in}}{\pgfqpoint{4.430023in}{3.810935in}}%
\pgfpathcurveto{\pgfqpoint{4.430023in}{3.800848in}}{\pgfqpoint{4.434031in}{3.791172in}}{\pgfqpoint{4.441164in}{3.784039in}}%
\pgfpathcurveto{\pgfqpoint{4.448297in}{3.776907in}}{\pgfqpoint{4.457972in}{3.772899in}}{\pgfqpoint{4.468060in}{3.772899in}}%
\pgfpathclose%
\pgfusepath{stroke,fill}%
\end{pgfscope}%
\begin{pgfscope}%
\pgfpathrectangle{\pgfqpoint{0.800000in}{0.528000in}}{\pgfqpoint{4.960000in}{3.696000in}} %
\pgfusepath{clip}%
\pgfsetbuttcap%
\pgfsetroundjoin%
\definecolor{currentfill}{rgb}{0.121569,0.466667,0.705882}%
\pgfsetfillcolor{currentfill}%
\pgfsetlinewidth{1.003750pt}%
\definecolor{currentstroke}{rgb}{0.121569,0.466667,0.705882}%
\pgfsetstrokecolor{currentstroke}%
\pgfsetdash{}{0pt}%
\pgfpathmoveto{\pgfqpoint{5.296265in}{1.544799in}}%
\pgfpathcurveto{\pgfqpoint{5.306352in}{1.544799in}}{\pgfqpoint{5.316027in}{1.548807in}}{\pgfqpoint{5.323160in}{1.555940in}}%
\pgfpathcurveto{\pgfqpoint{5.330293in}{1.563073in}}{\pgfqpoint{5.334301in}{1.572748in}}{\pgfqpoint{5.334301in}{1.582836in}}%
\pgfpathcurveto{\pgfqpoint{5.334301in}{1.592923in}}{\pgfqpoint{5.330293in}{1.602599in}}{\pgfqpoint{5.323160in}{1.609731in}}%
\pgfpathcurveto{\pgfqpoint{5.316027in}{1.616864in}}{\pgfqpoint{5.306352in}{1.620872in}}{\pgfqpoint{5.296265in}{1.620872in}}%
\pgfpathcurveto{\pgfqpoint{5.286177in}{1.620872in}}{\pgfqpoint{5.276502in}{1.616864in}}{\pgfqpoint{5.269369in}{1.609731in}}%
\pgfpathcurveto{\pgfqpoint{5.262236in}{1.602599in}}{\pgfqpoint{5.258228in}{1.592923in}}{\pgfqpoint{5.258228in}{1.582836in}}%
\pgfpathcurveto{\pgfqpoint{5.258228in}{1.572748in}}{\pgfqpoint{5.262236in}{1.563073in}}{\pgfqpoint{5.269369in}{1.555940in}}%
\pgfpathcurveto{\pgfqpoint{5.276502in}{1.548807in}}{\pgfqpoint{5.286177in}{1.544799in}}{\pgfqpoint{5.296265in}{1.544799in}}%
\pgfpathclose%
\pgfusepath{stroke,fill}%
\end{pgfscope}%
\begin{pgfscope}%
\pgfpathrectangle{\pgfqpoint{0.800000in}{0.528000in}}{\pgfqpoint{4.960000in}{3.696000in}} %
\pgfusepath{clip}%
\pgfsetbuttcap%
\pgfsetroundjoin%
\definecolor{currentfill}{rgb}{0.121569,0.466667,0.705882}%
\pgfsetfillcolor{currentfill}%
\pgfsetlinewidth{1.003750pt}%
\definecolor{currentstroke}{rgb}{0.121569,0.466667,0.705882}%
\pgfsetstrokecolor{currentstroke}%
\pgfsetdash{}{0pt}%
\pgfpathmoveto{\pgfqpoint{5.528280in}{2.502909in}}%
\pgfpathcurveto{\pgfqpoint{5.538367in}{2.502909in}}{\pgfqpoint{5.548043in}{2.506917in}}{\pgfqpoint{5.555175in}{2.514049in}}%
\pgfpathcurveto{\pgfqpoint{5.562308in}{2.521182in}}{\pgfqpoint{5.566316in}{2.530858in}}{\pgfqpoint{5.566316in}{2.540945in}}%
\pgfpathcurveto{\pgfqpoint{5.566316in}{2.551033in}}{\pgfqpoint{5.562308in}{2.560708in}}{\pgfqpoint{5.555175in}{2.567841in}}%
\pgfpathcurveto{\pgfqpoint{5.548043in}{2.574974in}}{\pgfqpoint{5.538367in}{2.578982in}}{\pgfqpoint{5.528280in}{2.578982in}}%
\pgfpathcurveto{\pgfqpoint{5.518192in}{2.578982in}}{\pgfqpoint{5.508517in}{2.574974in}}{\pgfqpoint{5.501384in}{2.567841in}}%
\pgfpathcurveto{\pgfqpoint{5.494251in}{2.560708in}}{\pgfqpoint{5.490243in}{2.551033in}}{\pgfqpoint{5.490243in}{2.540945in}}%
\pgfpathcurveto{\pgfqpoint{5.490243in}{2.530858in}}{\pgfqpoint{5.494251in}{2.521182in}}{\pgfqpoint{5.501384in}{2.514049in}}%
\pgfpathcurveto{\pgfqpoint{5.508517in}{2.506917in}}{\pgfqpoint{5.518192in}{2.502909in}}{\pgfqpoint{5.528280in}{2.502909in}}%
\pgfpathclose%
\pgfusepath{stroke,fill}%
\end{pgfscope}%
\begin{pgfscope}%
\pgfpathrectangle{\pgfqpoint{0.800000in}{0.528000in}}{\pgfqpoint{4.960000in}{3.696000in}} %
\pgfusepath{clip}%
\pgfsetbuttcap%
\pgfsetroundjoin%
\definecolor{currentfill}{rgb}{0.121569,0.466667,0.705882}%
\pgfsetfillcolor{currentfill}%
\pgfsetlinewidth{1.003750pt}%
\definecolor{currentstroke}{rgb}{0.121569,0.466667,0.705882}%
\pgfsetstrokecolor{currentstroke}%
\pgfsetdash{}{0pt}%
\pgfpathmoveto{\pgfqpoint{5.328319in}{3.240906in}}%
\pgfpathcurveto{\pgfqpoint{5.338407in}{3.240906in}}{\pgfqpoint{5.348082in}{3.244914in}}{\pgfqpoint{5.355215in}{3.252047in}}%
\pgfpathcurveto{\pgfqpoint{5.362348in}{3.259180in}}{\pgfqpoint{5.366356in}{3.268855in}}{\pgfqpoint{5.366356in}{3.278943in}}%
\pgfpathcurveto{\pgfqpoint{5.366356in}{3.289030in}}{\pgfqpoint{5.362348in}{3.298706in}}{\pgfqpoint{5.355215in}{3.305838in}}%
\pgfpathcurveto{\pgfqpoint{5.348082in}{3.312971in}}{\pgfqpoint{5.338407in}{3.316979in}}{\pgfqpoint{5.328319in}{3.316979in}}%
\pgfpathcurveto{\pgfqpoint{5.318232in}{3.316979in}}{\pgfqpoint{5.308557in}{3.312971in}}{\pgfqpoint{5.301424in}{3.305838in}}%
\pgfpathcurveto{\pgfqpoint{5.294291in}{3.298706in}}{\pgfqpoint{5.290283in}{3.289030in}}{\pgfqpoint{5.290283in}{3.278943in}}%
\pgfpathcurveto{\pgfqpoint{5.290283in}{3.268855in}}{\pgfqpoint{5.294291in}{3.259180in}}{\pgfqpoint{5.301424in}{3.252047in}}%
\pgfpathcurveto{\pgfqpoint{5.308557in}{3.244914in}}{\pgfqpoint{5.318232in}{3.240906in}}{\pgfqpoint{5.328319in}{3.240906in}}%
\pgfpathclose%
\pgfusepath{stroke,fill}%
\end{pgfscope}%
\begin{pgfscope}%
\pgfpathrectangle{\pgfqpoint{0.800000in}{0.528000in}}{\pgfqpoint{4.960000in}{3.696000in}} %
\pgfusepath{clip}%
\pgfsetbuttcap%
\pgfsetroundjoin%
\definecolor{currentfill}{rgb}{0.121569,0.466667,0.705882}%
\pgfsetfillcolor{currentfill}%
\pgfsetlinewidth{1.003750pt}%
\definecolor{currentstroke}{rgb}{0.121569,0.466667,0.705882}%
\pgfsetstrokecolor{currentstroke}%
\pgfsetdash{}{0pt}%
\pgfpathmoveto{\pgfqpoint{5.176855in}{3.602912in}}%
\pgfpathcurveto{\pgfqpoint{5.186942in}{3.602912in}}{\pgfqpoint{5.196618in}{3.606919in}}{\pgfqpoint{5.203751in}{3.614052in}}%
\pgfpathcurveto{\pgfqpoint{5.210884in}{3.621185in}}{\pgfqpoint{5.214891in}{3.630860in}}{\pgfqpoint{5.214891in}{3.640948in}}%
\pgfpathcurveto{\pgfqpoint{5.214891in}{3.651035in}}{\pgfqpoint{5.210884in}{3.660711in}}{\pgfqpoint{5.203751in}{3.667844in}}%
\pgfpathcurveto{\pgfqpoint{5.196618in}{3.674976in}}{\pgfqpoint{5.186942in}{3.678984in}}{\pgfqpoint{5.176855in}{3.678984in}}%
\pgfpathcurveto{\pgfqpoint{5.166768in}{3.678984in}}{\pgfqpoint{5.157092in}{3.674976in}}{\pgfqpoint{5.149959in}{3.667844in}}%
\pgfpathcurveto{\pgfqpoint{5.142827in}{3.660711in}}{\pgfqpoint{5.138819in}{3.651035in}}{\pgfqpoint{5.138819in}{3.640948in}}%
\pgfpathcurveto{\pgfqpoint{5.138819in}{3.630860in}}{\pgfqpoint{5.142827in}{3.621185in}}{\pgfqpoint{5.149959in}{3.614052in}}%
\pgfpathcurveto{\pgfqpoint{5.157092in}{3.606919in}}{\pgfqpoint{5.166768in}{3.602912in}}{\pgfqpoint{5.176855in}{3.602912in}}%
\pgfpathclose%
\pgfusepath{stroke,fill}%
\end{pgfscope}%
\begin{pgfscope}%
\pgfpathrectangle{\pgfqpoint{0.800000in}{0.528000in}}{\pgfqpoint{4.960000in}{3.696000in}} %
\pgfusepath{clip}%
\pgfsetbuttcap%
\pgfsetroundjoin%
\definecolor{currentfill}{rgb}{0.121569,0.466667,0.705882}%
\pgfsetfillcolor{currentfill}%
\pgfsetlinewidth{1.003750pt}%
\definecolor{currentstroke}{rgb}{0.121569,0.466667,0.705882}%
\pgfsetstrokecolor{currentstroke}%
\pgfsetdash{}{0pt}%
\pgfpathmoveto{\pgfqpoint{1.673483in}{1.012308in}}%
\pgfpathcurveto{\pgfqpoint{1.683570in}{1.012308in}}{\pgfqpoint{1.693246in}{1.016316in}}{\pgfqpoint{1.700378in}{1.023448in}}%
\pgfpathcurveto{\pgfqpoint{1.707511in}{1.030581in}}{\pgfqpoint{1.711519in}{1.040257in}}{\pgfqpoint{1.711519in}{1.050344in}}%
\pgfpathcurveto{\pgfqpoint{1.711519in}{1.060431in}}{\pgfqpoint{1.707511in}{1.070107in}}{\pgfqpoint{1.700378in}{1.077240in}}%
\pgfpathcurveto{\pgfqpoint{1.693246in}{1.084373in}}{\pgfqpoint{1.683570in}{1.088380in}}{\pgfqpoint{1.673483in}{1.088380in}}%
\pgfpathcurveto{\pgfqpoint{1.663395in}{1.088380in}}{\pgfqpoint{1.653720in}{1.084373in}}{\pgfqpoint{1.646587in}{1.077240in}}%
\pgfpathcurveto{\pgfqpoint{1.639454in}{1.070107in}}{\pgfqpoint{1.635446in}{1.060431in}}{\pgfqpoint{1.635446in}{1.050344in}}%
\pgfpathcurveto{\pgfqpoint{1.635446in}{1.040257in}}{\pgfqpoint{1.639454in}{1.030581in}}{\pgfqpoint{1.646587in}{1.023448in}}%
\pgfpathcurveto{\pgfqpoint{1.653720in}{1.016316in}}{\pgfqpoint{1.663395in}{1.012308in}}{\pgfqpoint{1.673483in}{1.012308in}}%
\pgfpathclose%
\pgfusepath{stroke,fill}%
\end{pgfscope}%
\begin{pgfscope}%
\pgfpathrectangle{\pgfqpoint{0.800000in}{0.528000in}}{\pgfqpoint{4.960000in}{3.696000in}} %
\pgfusepath{clip}%
\pgfsetbuttcap%
\pgfsetroundjoin%
\definecolor{currentfill}{rgb}{0.121569,0.466667,0.705882}%
\pgfsetfillcolor{currentfill}%
\pgfsetlinewidth{1.003750pt}%
\definecolor{currentstroke}{rgb}{0.121569,0.466667,0.705882}%
\pgfsetstrokecolor{currentstroke}%
\pgfsetdash{}{0pt}%
\pgfpathmoveto{\pgfqpoint{1.342173in}{1.561903in}}%
\pgfpathcurveto{\pgfqpoint{1.352261in}{1.561903in}}{\pgfqpoint{1.361936in}{1.565911in}}{\pgfqpoint{1.369069in}{1.573044in}}%
\pgfpathcurveto{\pgfqpoint{1.376202in}{1.580177in}}{\pgfqpoint{1.380210in}{1.589852in}}{\pgfqpoint{1.380210in}{1.599940in}}%
\pgfpathcurveto{\pgfqpoint{1.380210in}{1.610027in}}{\pgfqpoint{1.376202in}{1.619703in}}{\pgfqpoint{1.369069in}{1.626835in}}%
\pgfpathcurveto{\pgfqpoint{1.361936in}{1.633968in}}{\pgfqpoint{1.352261in}{1.637976in}}{\pgfqpoint{1.342173in}{1.637976in}}%
\pgfpathcurveto{\pgfqpoint{1.332086in}{1.637976in}}{\pgfqpoint{1.322411in}{1.633968in}}{\pgfqpoint{1.315278in}{1.626835in}}%
\pgfpathcurveto{\pgfqpoint{1.308145in}{1.619703in}}{\pgfqpoint{1.304137in}{1.610027in}}{\pgfqpoint{1.304137in}{1.599940in}}%
\pgfpathcurveto{\pgfqpoint{1.304137in}{1.589852in}}{\pgfqpoint{1.308145in}{1.580177in}}{\pgfqpoint{1.315278in}{1.573044in}}%
\pgfpathcurveto{\pgfqpoint{1.322411in}{1.565911in}}{\pgfqpoint{1.332086in}{1.561903in}}{\pgfqpoint{1.342173in}{1.561903in}}%
\pgfpathclose%
\pgfusepath{stroke,fill}%
\end{pgfscope}%
\begin{pgfscope}%
\pgfpathrectangle{\pgfqpoint{0.800000in}{0.528000in}}{\pgfqpoint{4.960000in}{3.696000in}} %
\pgfusepath{clip}%
\pgfsetbuttcap%
\pgfsetroundjoin%
\definecolor{currentfill}{rgb}{0.121569,0.466667,0.705882}%
\pgfsetfillcolor{currentfill}%
\pgfsetlinewidth{1.003750pt}%
\definecolor{currentstroke}{rgb}{0.121569,0.466667,0.705882}%
\pgfsetstrokecolor{currentstroke}%
\pgfsetdash{}{0pt}%
\pgfpathmoveto{\pgfqpoint{1.031720in}{2.142608in}}%
\pgfpathcurveto{\pgfqpoint{1.041808in}{2.142608in}}{\pgfqpoint{1.051483in}{2.146615in}}{\pgfqpoint{1.058616in}{2.153748in}}%
\pgfpathcurveto{\pgfqpoint{1.065749in}{2.160881in}}{\pgfqpoint{1.069757in}{2.170557in}}{\pgfqpoint{1.069757in}{2.180644in}}%
\pgfpathcurveto{\pgfqpoint{1.069757in}{2.190731in}}{\pgfqpoint{1.065749in}{2.200407in}}{\pgfqpoint{1.058616in}{2.207540in}}%
\pgfpathcurveto{\pgfqpoint{1.051483in}{2.214673in}}{\pgfqpoint{1.041808in}{2.218680in}}{\pgfqpoint{1.031720in}{2.218680in}}%
\pgfpathcurveto{\pgfqpoint{1.021633in}{2.218680in}}{\pgfqpoint{1.011957in}{2.214673in}}{\pgfqpoint{1.004825in}{2.207540in}}%
\pgfpathcurveto{\pgfqpoint{0.997692in}{2.200407in}}{\pgfqpoint{0.993684in}{2.190731in}}{\pgfqpoint{0.993684in}{2.180644in}}%
\pgfpathcurveto{\pgfqpoint{0.993684in}{2.170557in}}{\pgfqpoint{0.997692in}{2.160881in}}{\pgfqpoint{1.004825in}{2.153748in}}%
\pgfpathcurveto{\pgfqpoint{1.011957in}{2.146615in}}{\pgfqpoint{1.021633in}{2.142608in}}{\pgfqpoint{1.031720in}{2.142608in}}%
\pgfpathclose%
\pgfusepath{stroke,fill}%
\end{pgfscope}%
\begin{pgfscope}%
\pgfpathrectangle{\pgfqpoint{0.800000in}{0.528000in}}{\pgfqpoint{4.960000in}{3.696000in}} %
\pgfusepath{clip}%
\pgfsetbuttcap%
\pgfsetroundjoin%
\definecolor{currentfill}{rgb}{0.121569,0.466667,0.705882}%
\pgfsetfillcolor{currentfill}%
\pgfsetlinewidth{1.003750pt}%
\definecolor{currentstroke}{rgb}{0.121569,0.466667,0.705882}%
\pgfsetstrokecolor{currentstroke}%
\pgfsetdash{}{0pt}%
\pgfpathmoveto{\pgfqpoint{1.356249in}{3.160048in}}%
\pgfpathcurveto{\pgfqpoint{1.366337in}{3.160048in}}{\pgfqpoint{1.376012in}{3.164056in}}{\pgfqpoint{1.383145in}{3.171189in}}%
\pgfpathcurveto{\pgfqpoint{1.390278in}{3.178321in}}{\pgfqpoint{1.394286in}{3.187997in}}{\pgfqpoint{1.394286in}{3.198084in}}%
\pgfpathcurveto{\pgfqpoint{1.394286in}{3.208172in}}{\pgfqpoint{1.390278in}{3.217847in}}{\pgfqpoint{1.383145in}{3.224980in}}%
\pgfpathcurveto{\pgfqpoint{1.376012in}{3.232113in}}{\pgfqpoint{1.366337in}{3.236121in}}{\pgfqpoint{1.356249in}{3.236121in}}%
\pgfpathcurveto{\pgfqpoint{1.346162in}{3.236121in}}{\pgfqpoint{1.336486in}{3.232113in}}{\pgfqpoint{1.329354in}{3.224980in}}%
\pgfpathcurveto{\pgfqpoint{1.322221in}{3.217847in}}{\pgfqpoint{1.318213in}{3.208172in}}{\pgfqpoint{1.318213in}{3.198084in}}%
\pgfpathcurveto{\pgfqpoint{1.318213in}{3.187997in}}{\pgfqpoint{1.322221in}{3.178321in}}{\pgfqpoint{1.329354in}{3.171189in}}%
\pgfpathcurveto{\pgfqpoint{1.336486in}{3.164056in}}{\pgfqpoint{1.346162in}{3.160048in}}{\pgfqpoint{1.356249in}{3.160048in}}%
\pgfpathclose%
\pgfusepath{stroke,fill}%
\end{pgfscope}%
\begin{pgfscope}%
\pgfpathrectangle{\pgfqpoint{0.800000in}{0.528000in}}{\pgfqpoint{4.960000in}{3.696000in}} %
\pgfusepath{clip}%
\pgfsetbuttcap%
\pgfsetroundjoin%
\definecolor{currentfill}{rgb}{0.121569,0.466667,0.705882}%
\pgfsetfillcolor{currentfill}%
\pgfsetlinewidth{1.003750pt}%
\definecolor{currentstroke}{rgb}{0.121569,0.466667,0.705882}%
\pgfsetstrokecolor{currentstroke}%
\pgfsetdash{}{0pt}%
\pgfpathmoveto{\pgfqpoint{1.528570in}{3.587024in}}%
\pgfpathcurveto{\pgfqpoint{1.538657in}{3.587024in}}{\pgfqpoint{1.548333in}{3.591032in}}{\pgfqpoint{1.555466in}{3.598165in}}%
\pgfpathcurveto{\pgfqpoint{1.562599in}{3.605297in}}{\pgfqpoint{1.566606in}{3.614973in}}{\pgfqpoint{1.566606in}{3.625060in}}%
\pgfpathcurveto{\pgfqpoint{1.566606in}{3.635148in}}{\pgfqpoint{1.562599in}{3.644823in}}{\pgfqpoint{1.555466in}{3.651956in}}%
\pgfpathcurveto{\pgfqpoint{1.548333in}{3.659089in}}{\pgfqpoint{1.538657in}{3.663097in}}{\pgfqpoint{1.528570in}{3.663097in}}%
\pgfpathcurveto{\pgfqpoint{1.518483in}{3.663097in}}{\pgfqpoint{1.508807in}{3.659089in}}{\pgfqpoint{1.501674in}{3.651956in}}%
\pgfpathcurveto{\pgfqpoint{1.494541in}{3.644823in}}{\pgfqpoint{1.490534in}{3.635148in}}{\pgfqpoint{1.490534in}{3.625060in}}%
\pgfpathcurveto{\pgfqpoint{1.490534in}{3.614973in}}{\pgfqpoint{1.494541in}{3.605297in}}{\pgfqpoint{1.501674in}{3.598165in}}%
\pgfpathcurveto{\pgfqpoint{1.508807in}{3.591032in}}{\pgfqpoint{1.518483in}{3.587024in}}{\pgfqpoint{1.528570in}{3.587024in}}%
\pgfpathclose%
\pgfusepath{stroke,fill}%
\end{pgfscope}%
\begin{pgfscope}%
\pgfpathrectangle{\pgfqpoint{0.800000in}{0.528000in}}{\pgfqpoint{4.960000in}{3.696000in}} %
\pgfusepath{clip}%
\pgfsetbuttcap%
\pgfsetroundjoin%
\definecolor{currentfill}{rgb}{0.121569,0.466667,0.705882}%
\pgfsetfillcolor{currentfill}%
\pgfsetlinewidth{1.003750pt}%
\definecolor{currentstroke}{rgb}{0.121569,0.466667,0.705882}%
\pgfsetstrokecolor{currentstroke}%
\pgfsetdash{}{0pt}%
\pgfpathmoveto{\pgfqpoint{2.428361in}{0.692038in}}%
\pgfpathcurveto{\pgfqpoint{2.438448in}{0.692038in}}{\pgfqpoint{2.448123in}{0.696046in}}{\pgfqpoint{2.455256in}{0.703179in}}%
\pgfpathcurveto{\pgfqpoint{2.462389in}{0.710312in}}{\pgfqpoint{2.466397in}{0.719987in}}{\pgfqpoint{2.466397in}{0.730075in}}%
\pgfpathcurveto{\pgfqpoint{2.466397in}{0.740162in}}{\pgfqpoint{2.462389in}{0.749838in}}{\pgfqpoint{2.455256in}{0.756970in}}%
\pgfpathcurveto{\pgfqpoint{2.448123in}{0.764103in}}{\pgfqpoint{2.438448in}{0.768111in}}{\pgfqpoint{2.428361in}{0.768111in}}%
\pgfpathcurveto{\pgfqpoint{2.418273in}{0.768111in}}{\pgfqpoint{2.408598in}{0.764103in}}{\pgfqpoint{2.401465in}{0.756970in}}%
\pgfpathcurveto{\pgfqpoint{2.394332in}{0.749838in}}{\pgfqpoint{2.390324in}{0.740162in}}{\pgfqpoint{2.390324in}{0.730075in}}%
\pgfpathcurveto{\pgfqpoint{2.390324in}{0.719987in}}{\pgfqpoint{2.394332in}{0.710312in}}{\pgfqpoint{2.401465in}{0.703179in}}%
\pgfpathcurveto{\pgfqpoint{2.408598in}{0.696046in}}{\pgfqpoint{2.418273in}{0.692038in}}{\pgfqpoint{2.428361in}{0.692038in}}%
\pgfpathclose%
\pgfusepath{stroke,fill}%
\end{pgfscope}%
\begin{pgfscope}%
\pgfpathrectangle{\pgfqpoint{0.800000in}{0.528000in}}{\pgfqpoint{4.960000in}{3.696000in}} %
\pgfusepath{clip}%
\pgfsetbuttcap%
\pgfsetroundjoin%
\definecolor{currentfill}{rgb}{0.121569,0.466667,0.705882}%
\pgfsetfillcolor{currentfill}%
\pgfsetlinewidth{1.003750pt}%
\definecolor{currentstroke}{rgb}{0.121569,0.466667,0.705882}%
\pgfsetstrokecolor{currentstroke}%
\pgfsetdash{}{0pt}%
\pgfpathmoveto{\pgfqpoint{2.524795in}{1.092081in}}%
\pgfpathcurveto{\pgfqpoint{2.534882in}{1.092081in}}{\pgfqpoint{2.544558in}{1.096089in}}{\pgfqpoint{2.551690in}{1.103222in}}%
\pgfpathcurveto{\pgfqpoint{2.558823in}{1.110354in}}{\pgfqpoint{2.562831in}{1.120030in}}{\pgfqpoint{2.562831in}{1.130117in}}%
\pgfpathcurveto{\pgfqpoint{2.562831in}{1.140205in}}{\pgfqpoint{2.558823in}{1.149880in}}{\pgfqpoint{2.551690in}{1.157013in}}%
\pgfpathcurveto{\pgfqpoint{2.544558in}{1.164146in}}{\pgfqpoint{2.534882in}{1.168154in}}{\pgfqpoint{2.524795in}{1.168154in}}%
\pgfpathcurveto{\pgfqpoint{2.514707in}{1.168154in}}{\pgfqpoint{2.505032in}{1.164146in}}{\pgfqpoint{2.497899in}{1.157013in}}%
\pgfpathcurveto{\pgfqpoint{2.490766in}{1.149880in}}{\pgfqpoint{2.486758in}{1.140205in}}{\pgfqpoint{2.486758in}{1.130117in}}%
\pgfpathcurveto{\pgfqpoint{2.486758in}{1.120030in}}{\pgfqpoint{2.490766in}{1.110354in}}{\pgfqpoint{2.497899in}{1.103222in}}%
\pgfpathcurveto{\pgfqpoint{2.505032in}{1.096089in}}{\pgfqpoint{2.514707in}{1.092081in}}{\pgfqpoint{2.524795in}{1.092081in}}%
\pgfpathclose%
\pgfusepath{stroke,fill}%
\end{pgfscope}%
\begin{pgfscope}%
\pgfpathrectangle{\pgfqpoint{0.800000in}{0.528000in}}{\pgfqpoint{4.960000in}{3.696000in}} %
\pgfusepath{clip}%
\pgfsetbuttcap%
\pgfsetroundjoin%
\definecolor{currentfill}{rgb}{0.121569,0.466667,0.705882}%
\pgfsetfillcolor{currentfill}%
\pgfsetlinewidth{1.003750pt}%
\definecolor{currentstroke}{rgb}{0.121569,0.466667,0.705882}%
\pgfsetstrokecolor{currentstroke}%
\pgfsetdash{}{0pt}%
\pgfpathmoveto{\pgfqpoint{2.371871in}{3.992571in}}%
\pgfpathcurveto{\pgfqpoint{2.381958in}{3.992571in}}{\pgfqpoint{2.391634in}{3.996578in}}{\pgfqpoint{2.398766in}{4.003711in}}%
\pgfpathcurveto{\pgfqpoint{2.405899in}{4.010844in}}{\pgfqpoint{2.409907in}{4.020520in}}{\pgfqpoint{2.409907in}{4.030607in}}%
\pgfpathcurveto{\pgfqpoint{2.409907in}{4.040694in}}{\pgfqpoint{2.405899in}{4.050370in}}{\pgfqpoint{2.398766in}{4.057503in}}%
\pgfpathcurveto{\pgfqpoint{2.391634in}{4.064636in}}{\pgfqpoint{2.381958in}{4.068643in}}{\pgfqpoint{2.371871in}{4.068643in}}%
\pgfpathcurveto{\pgfqpoint{2.361783in}{4.068643in}}{\pgfqpoint{2.352108in}{4.064636in}}{\pgfqpoint{2.344975in}{4.057503in}}%
\pgfpathcurveto{\pgfqpoint{2.337842in}{4.050370in}}{\pgfqpoint{2.333834in}{4.040694in}}{\pgfqpoint{2.333834in}{4.030607in}}%
\pgfpathcurveto{\pgfqpoint{2.333834in}{4.020520in}}{\pgfqpoint{2.337842in}{4.010844in}}{\pgfqpoint{2.344975in}{4.003711in}}%
\pgfpathcurveto{\pgfqpoint{2.352108in}{3.996578in}}{\pgfqpoint{2.361783in}{3.992571in}}{\pgfqpoint{2.371871in}{3.992571in}}%
\pgfpathclose%
\pgfusepath{stroke,fill}%
\end{pgfscope}%
\begin{pgfscope}%
\pgfpathrectangle{\pgfqpoint{0.800000in}{0.528000in}}{\pgfqpoint{4.960000in}{3.696000in}} %
\pgfusepath{clip}%
\pgfsetbuttcap%
\pgfsetroundjoin%
\definecolor{currentfill}{rgb}{0.121569,0.466667,0.705882}%
\pgfsetfillcolor{currentfill}%
\pgfsetlinewidth{1.003750pt}%
\definecolor{currentstroke}{rgb}{0.121569,0.466667,0.705882}%
\pgfsetstrokecolor{currentstroke}%
\pgfsetdash{}{0pt}%
\pgfpathmoveto{\pgfqpoint{3.317771in}{0.881055in}}%
\pgfpathcurveto{\pgfqpoint{3.327858in}{0.881055in}}{\pgfqpoint{3.337534in}{0.885062in}}{\pgfqpoint{3.344667in}{0.892195in}}%
\pgfpathcurveto{\pgfqpoint{3.351799in}{0.899328in}}{\pgfqpoint{3.355807in}{0.909004in}}{\pgfqpoint{3.355807in}{0.919091in}}%
\pgfpathcurveto{\pgfqpoint{3.355807in}{0.929178in}}{\pgfqpoint{3.351799in}{0.938854in}}{\pgfqpoint{3.344667in}{0.945987in}}%
\pgfpathcurveto{\pgfqpoint{3.337534in}{0.953120in}}{\pgfqpoint{3.327858in}{0.957127in}}{\pgfqpoint{3.317771in}{0.957127in}}%
\pgfpathcurveto{\pgfqpoint{3.307683in}{0.957127in}}{\pgfqpoint{3.298008in}{0.953120in}}{\pgfqpoint{3.290875in}{0.945987in}}%
\pgfpathcurveto{\pgfqpoint{3.283742in}{0.938854in}}{\pgfqpoint{3.279735in}{0.929178in}}{\pgfqpoint{3.279735in}{0.919091in}}%
\pgfpathcurveto{\pgfqpoint{3.279735in}{0.909004in}}{\pgfqpoint{3.283742in}{0.899328in}}{\pgfqpoint{3.290875in}{0.892195in}}%
\pgfpathcurveto{\pgfqpoint{3.298008in}{0.885062in}}{\pgfqpoint{3.307683in}{0.881055in}}{\pgfqpoint{3.317771in}{0.881055in}}%
\pgfpathclose%
\pgfusepath{stroke,fill}%
\end{pgfscope}%
\begin{pgfscope}%
\pgfpathrectangle{\pgfqpoint{0.800000in}{0.528000in}}{\pgfqpoint{4.960000in}{3.696000in}} %
\pgfusepath{clip}%
\pgfsetbuttcap%
\pgfsetroundjoin%
\definecolor{currentfill}{rgb}{0.121569,0.466667,0.705882}%
\pgfsetfillcolor{currentfill}%
\pgfsetlinewidth{1.003750pt}%
\definecolor{currentstroke}{rgb}{0.121569,0.466667,0.705882}%
\pgfsetstrokecolor{currentstroke}%
\pgfsetdash{}{0pt}%
\pgfpathmoveto{\pgfqpoint{3.188228in}{3.959324in}}%
\pgfpathcurveto{\pgfqpoint{3.198316in}{3.959324in}}{\pgfqpoint{3.207991in}{3.963332in}}{\pgfqpoint{3.215124in}{3.970465in}}%
\pgfpathcurveto{\pgfqpoint{3.222257in}{3.977598in}}{\pgfqpoint{3.226265in}{3.987273in}}{\pgfqpoint{3.226265in}{3.997361in}}%
\pgfpathcurveto{\pgfqpoint{3.226265in}{4.007448in}}{\pgfqpoint{3.222257in}{4.017123in}}{\pgfqpoint{3.215124in}{4.024256in}}%
\pgfpathcurveto{\pgfqpoint{3.207991in}{4.031389in}}{\pgfqpoint{3.198316in}{4.035397in}}{\pgfqpoint{3.188228in}{4.035397in}}%
\pgfpathcurveto{\pgfqpoint{3.178141in}{4.035397in}}{\pgfqpoint{3.168465in}{4.031389in}}{\pgfqpoint{3.161333in}{4.024256in}}%
\pgfpathcurveto{\pgfqpoint{3.154200in}{4.017123in}}{\pgfqpoint{3.150192in}{4.007448in}}{\pgfqpoint{3.150192in}{3.997361in}}%
\pgfpathcurveto{\pgfqpoint{3.150192in}{3.987273in}}{\pgfqpoint{3.154200in}{3.977598in}}{\pgfqpoint{3.161333in}{3.970465in}}%
\pgfpathcurveto{\pgfqpoint{3.168465in}{3.963332in}}{\pgfqpoint{3.178141in}{3.959324in}}{\pgfqpoint{3.188228in}{3.959324in}}%
\pgfpathclose%
\pgfusepath{stroke,fill}%
\end{pgfscope}%
\begin{pgfscope}%
\pgfpathrectangle{\pgfqpoint{0.800000in}{0.528000in}}{\pgfqpoint{4.960000in}{3.696000in}} %
\pgfusepath{clip}%
\pgfsetbuttcap%
\pgfsetroundjoin%
\definecolor{currentfill}{rgb}{0.121569,0.466667,0.705882}%
\pgfsetfillcolor{currentfill}%
\pgfsetlinewidth{1.003750pt}%
\definecolor{currentstroke}{rgb}{0.121569,0.466667,0.705882}%
\pgfsetstrokecolor{currentstroke}%
\pgfsetdash{}{0pt}%
\pgfpathmoveto{\pgfqpoint{4.207830in}{0.683357in}}%
\pgfpathcurveto{\pgfqpoint{4.217917in}{0.683357in}}{\pgfqpoint{4.227593in}{0.687364in}}{\pgfqpoint{4.234726in}{0.694497in}}%
\pgfpathcurveto{\pgfqpoint{4.241858in}{0.701630in}}{\pgfqpoint{4.245866in}{0.711306in}}{\pgfqpoint{4.245866in}{0.721393in}}%
\pgfpathcurveto{\pgfqpoint{4.245866in}{0.731480in}}{\pgfqpoint{4.241858in}{0.741156in}}{\pgfqpoint{4.234726in}{0.748289in}}%
\pgfpathcurveto{\pgfqpoint{4.227593in}{0.755422in}}{\pgfqpoint{4.217917in}{0.759429in}}{\pgfqpoint{4.207830in}{0.759429in}}%
\pgfpathcurveto{\pgfqpoint{4.197743in}{0.759429in}}{\pgfqpoint{4.188067in}{0.755422in}}{\pgfqpoint{4.180934in}{0.748289in}}%
\pgfpathcurveto{\pgfqpoint{4.173801in}{0.741156in}}{\pgfqpoint{4.169794in}{0.731480in}}{\pgfqpoint{4.169794in}{0.721393in}}%
\pgfpathcurveto{\pgfqpoint{4.169794in}{0.711306in}}{\pgfqpoint{4.173801in}{0.701630in}}{\pgfqpoint{4.180934in}{0.694497in}}%
\pgfpathcurveto{\pgfqpoint{4.188067in}{0.687364in}}{\pgfqpoint{4.197743in}{0.683357in}}{\pgfqpoint{4.207830in}{0.683357in}}%
\pgfpathclose%
\pgfusepath{stroke,fill}%
\end{pgfscope}%
\begin{pgfscope}%
\pgfpathrectangle{\pgfqpoint{0.800000in}{0.528000in}}{\pgfqpoint{4.960000in}{3.696000in}} %
\pgfusepath{clip}%
\pgfsetbuttcap%
\pgfsetroundjoin%
\definecolor{currentfill}{rgb}{0.121569,0.466667,0.705882}%
\pgfsetfillcolor{currentfill}%
\pgfsetlinewidth{1.003750pt}%
\definecolor{currentstroke}{rgb}{0.121569,0.466667,0.705882}%
\pgfsetstrokecolor{currentstroke}%
\pgfsetdash{}{0pt}%
\pgfpathmoveto{\pgfqpoint{4.827101in}{1.237602in}}%
\pgfpathcurveto{\pgfqpoint{4.837189in}{1.237602in}}{\pgfqpoint{4.846864in}{1.241610in}}{\pgfqpoint{4.853997in}{1.248743in}}%
\pgfpathcurveto{\pgfqpoint{4.861130in}{1.255875in}}{\pgfqpoint{4.865138in}{1.265551in}}{\pgfqpoint{4.865138in}{1.275638in}}%
\pgfpathcurveto{\pgfqpoint{4.865138in}{1.285726in}}{\pgfqpoint{4.861130in}{1.295401in}}{\pgfqpoint{4.853997in}{1.302534in}}%
\pgfpathcurveto{\pgfqpoint{4.846864in}{1.309667in}}{\pgfqpoint{4.837189in}{1.313675in}}{\pgfqpoint{4.827101in}{1.313675in}}%
\pgfpathcurveto{\pgfqpoint{4.817014in}{1.313675in}}{\pgfqpoint{4.807338in}{1.309667in}}{\pgfqpoint{4.800206in}{1.302534in}}%
\pgfpathcurveto{\pgfqpoint{4.793073in}{1.295401in}}{\pgfqpoint{4.789065in}{1.285726in}}{\pgfqpoint{4.789065in}{1.275638in}}%
\pgfpathcurveto{\pgfqpoint{4.789065in}{1.265551in}}{\pgfqpoint{4.793073in}{1.255875in}}{\pgfqpoint{4.800206in}{1.248743in}}%
\pgfpathcurveto{\pgfqpoint{4.807338in}{1.241610in}}{\pgfqpoint{4.817014in}{1.237602in}}{\pgfqpoint{4.827101in}{1.237602in}}%
\pgfpathclose%
\pgfusepath{stroke,fill}%
\end{pgfscope}%
\begin{pgfscope}%
\pgfpathrectangle{\pgfqpoint{0.800000in}{0.528000in}}{\pgfqpoint{4.960000in}{3.696000in}} %
\pgfusepath{clip}%
\pgfsetbuttcap%
\pgfsetroundjoin%
\definecolor{currentfill}{rgb}{0.121569,0.466667,0.705882}%
\pgfsetfillcolor{currentfill}%
\pgfsetlinewidth{1.003750pt}%
\definecolor{currentstroke}{rgb}{0.121569,0.466667,0.705882}%
\pgfsetstrokecolor{currentstroke}%
\pgfsetdash{}{0pt}%
\pgfpathmoveto{\pgfqpoint{4.468060in}{3.772899in}}%
\pgfpathcurveto{\pgfqpoint{4.478147in}{3.772899in}}{\pgfqpoint{4.487822in}{3.776907in}}{\pgfqpoint{4.494955in}{3.784039in}}%
\pgfpathcurveto{\pgfqpoint{4.502088in}{3.791172in}}{\pgfqpoint{4.506096in}{3.800848in}}{\pgfqpoint{4.506096in}{3.810935in}}%
\pgfpathcurveto{\pgfqpoint{4.506096in}{3.821022in}}{\pgfqpoint{4.502088in}{3.830698in}}{\pgfqpoint{4.494955in}{3.837831in}}%
\pgfpathcurveto{\pgfqpoint{4.487822in}{3.844964in}}{\pgfqpoint{4.478147in}{3.848971in}}{\pgfqpoint{4.468060in}{3.848971in}}%
\pgfpathcurveto{\pgfqpoint{4.457972in}{3.848971in}}{\pgfqpoint{4.448297in}{3.844964in}}{\pgfqpoint{4.441164in}{3.837831in}}%
\pgfpathcurveto{\pgfqpoint{4.434031in}{3.830698in}}{\pgfqpoint{4.430023in}{3.821022in}}{\pgfqpoint{4.430023in}{3.810935in}}%
\pgfpathcurveto{\pgfqpoint{4.430023in}{3.800848in}}{\pgfqpoint{4.434031in}{3.791172in}}{\pgfqpoint{4.441164in}{3.784039in}}%
\pgfpathcurveto{\pgfqpoint{4.448297in}{3.776907in}}{\pgfqpoint{4.457972in}{3.772899in}}{\pgfqpoint{4.468060in}{3.772899in}}%
\pgfpathclose%
\pgfusepath{stroke,fill}%
\end{pgfscope}%
\begin{pgfscope}%
\pgfpathrectangle{\pgfqpoint{0.800000in}{0.528000in}}{\pgfqpoint{4.960000in}{3.696000in}} %
\pgfusepath{clip}%
\pgfsetbuttcap%
\pgfsetroundjoin%
\definecolor{currentfill}{rgb}{0.121569,0.466667,0.705882}%
\pgfsetfillcolor{currentfill}%
\pgfsetlinewidth{1.003750pt}%
\definecolor{currentstroke}{rgb}{0.121569,0.466667,0.705882}%
\pgfsetstrokecolor{currentstroke}%
\pgfsetdash{}{0pt}%
\pgfpathmoveto{\pgfqpoint{5.296265in}{1.544799in}}%
\pgfpathcurveto{\pgfqpoint{5.306352in}{1.544799in}}{\pgfqpoint{5.316027in}{1.548807in}}{\pgfqpoint{5.323160in}{1.555940in}}%
\pgfpathcurveto{\pgfqpoint{5.330293in}{1.563073in}}{\pgfqpoint{5.334301in}{1.572748in}}{\pgfqpoint{5.334301in}{1.582836in}}%
\pgfpathcurveto{\pgfqpoint{5.334301in}{1.592923in}}{\pgfqpoint{5.330293in}{1.602599in}}{\pgfqpoint{5.323160in}{1.609731in}}%
\pgfpathcurveto{\pgfqpoint{5.316027in}{1.616864in}}{\pgfqpoint{5.306352in}{1.620872in}}{\pgfqpoint{5.296265in}{1.620872in}}%
\pgfpathcurveto{\pgfqpoint{5.286177in}{1.620872in}}{\pgfqpoint{5.276502in}{1.616864in}}{\pgfqpoint{5.269369in}{1.609731in}}%
\pgfpathcurveto{\pgfqpoint{5.262236in}{1.602599in}}{\pgfqpoint{5.258228in}{1.592923in}}{\pgfqpoint{5.258228in}{1.582836in}}%
\pgfpathcurveto{\pgfqpoint{5.258228in}{1.572748in}}{\pgfqpoint{5.262236in}{1.563073in}}{\pgfqpoint{5.269369in}{1.555940in}}%
\pgfpathcurveto{\pgfqpoint{5.276502in}{1.548807in}}{\pgfqpoint{5.286177in}{1.544799in}}{\pgfqpoint{5.296265in}{1.544799in}}%
\pgfpathclose%
\pgfusepath{stroke,fill}%
\end{pgfscope}%
\begin{pgfscope}%
\pgfpathrectangle{\pgfqpoint{0.800000in}{0.528000in}}{\pgfqpoint{4.960000in}{3.696000in}} %
\pgfusepath{clip}%
\pgfsetbuttcap%
\pgfsetroundjoin%
\definecolor{currentfill}{rgb}{0.121569,0.466667,0.705882}%
\pgfsetfillcolor{currentfill}%
\pgfsetlinewidth{1.003750pt}%
\definecolor{currentstroke}{rgb}{0.121569,0.466667,0.705882}%
\pgfsetstrokecolor{currentstroke}%
\pgfsetdash{}{0pt}%
\pgfpathmoveto{\pgfqpoint{5.528280in}{2.502909in}}%
\pgfpathcurveto{\pgfqpoint{5.538367in}{2.502909in}}{\pgfqpoint{5.548043in}{2.506917in}}{\pgfqpoint{5.555175in}{2.514049in}}%
\pgfpathcurveto{\pgfqpoint{5.562308in}{2.521182in}}{\pgfqpoint{5.566316in}{2.530858in}}{\pgfqpoint{5.566316in}{2.540945in}}%
\pgfpathcurveto{\pgfqpoint{5.566316in}{2.551033in}}{\pgfqpoint{5.562308in}{2.560708in}}{\pgfqpoint{5.555175in}{2.567841in}}%
\pgfpathcurveto{\pgfqpoint{5.548043in}{2.574974in}}{\pgfqpoint{5.538367in}{2.578982in}}{\pgfqpoint{5.528280in}{2.578982in}}%
\pgfpathcurveto{\pgfqpoint{5.518192in}{2.578982in}}{\pgfqpoint{5.508517in}{2.574974in}}{\pgfqpoint{5.501384in}{2.567841in}}%
\pgfpathcurveto{\pgfqpoint{5.494251in}{2.560708in}}{\pgfqpoint{5.490243in}{2.551033in}}{\pgfqpoint{5.490243in}{2.540945in}}%
\pgfpathcurveto{\pgfqpoint{5.490243in}{2.530858in}}{\pgfqpoint{5.494251in}{2.521182in}}{\pgfqpoint{5.501384in}{2.514049in}}%
\pgfpathcurveto{\pgfqpoint{5.508517in}{2.506917in}}{\pgfqpoint{5.518192in}{2.502909in}}{\pgfqpoint{5.528280in}{2.502909in}}%
\pgfpathclose%
\pgfusepath{stroke,fill}%
\end{pgfscope}%
\begin{pgfscope}%
\pgfpathrectangle{\pgfqpoint{0.800000in}{0.528000in}}{\pgfqpoint{4.960000in}{3.696000in}} %
\pgfusepath{clip}%
\pgfsetbuttcap%
\pgfsetroundjoin%
\definecolor{currentfill}{rgb}{0.121569,0.466667,0.705882}%
\pgfsetfillcolor{currentfill}%
\pgfsetlinewidth{1.003750pt}%
\definecolor{currentstroke}{rgb}{0.121569,0.466667,0.705882}%
\pgfsetstrokecolor{currentstroke}%
\pgfsetdash{}{0pt}%
\pgfpathmoveto{\pgfqpoint{5.328319in}{3.240906in}}%
\pgfpathcurveto{\pgfqpoint{5.338407in}{3.240906in}}{\pgfqpoint{5.348082in}{3.244914in}}{\pgfqpoint{5.355215in}{3.252047in}}%
\pgfpathcurveto{\pgfqpoint{5.362348in}{3.259180in}}{\pgfqpoint{5.366356in}{3.268855in}}{\pgfqpoint{5.366356in}{3.278943in}}%
\pgfpathcurveto{\pgfqpoint{5.366356in}{3.289030in}}{\pgfqpoint{5.362348in}{3.298706in}}{\pgfqpoint{5.355215in}{3.305838in}}%
\pgfpathcurveto{\pgfqpoint{5.348082in}{3.312971in}}{\pgfqpoint{5.338407in}{3.316979in}}{\pgfqpoint{5.328319in}{3.316979in}}%
\pgfpathcurveto{\pgfqpoint{5.318232in}{3.316979in}}{\pgfqpoint{5.308557in}{3.312971in}}{\pgfqpoint{5.301424in}{3.305838in}}%
\pgfpathcurveto{\pgfqpoint{5.294291in}{3.298706in}}{\pgfqpoint{5.290283in}{3.289030in}}{\pgfqpoint{5.290283in}{3.278943in}}%
\pgfpathcurveto{\pgfqpoint{5.290283in}{3.268855in}}{\pgfqpoint{5.294291in}{3.259180in}}{\pgfqpoint{5.301424in}{3.252047in}}%
\pgfpathcurveto{\pgfqpoint{5.308557in}{3.244914in}}{\pgfqpoint{5.318232in}{3.240906in}}{\pgfqpoint{5.328319in}{3.240906in}}%
\pgfpathclose%
\pgfusepath{stroke,fill}%
\end{pgfscope}%
\begin{pgfscope}%
\pgfpathrectangle{\pgfqpoint{0.800000in}{0.528000in}}{\pgfqpoint{4.960000in}{3.696000in}} %
\pgfusepath{clip}%
\pgfsetbuttcap%
\pgfsetroundjoin%
\definecolor{currentfill}{rgb}{0.121569,0.466667,0.705882}%
\pgfsetfillcolor{currentfill}%
\pgfsetlinewidth{1.003750pt}%
\definecolor{currentstroke}{rgb}{0.121569,0.466667,0.705882}%
\pgfsetstrokecolor{currentstroke}%
\pgfsetdash{}{0pt}%
\pgfpathmoveto{\pgfqpoint{5.176855in}{3.602912in}}%
\pgfpathcurveto{\pgfqpoint{5.186942in}{3.602912in}}{\pgfqpoint{5.196618in}{3.606919in}}{\pgfqpoint{5.203751in}{3.614052in}}%
\pgfpathcurveto{\pgfqpoint{5.210884in}{3.621185in}}{\pgfqpoint{5.214891in}{3.630860in}}{\pgfqpoint{5.214891in}{3.640948in}}%
\pgfpathcurveto{\pgfqpoint{5.214891in}{3.651035in}}{\pgfqpoint{5.210884in}{3.660711in}}{\pgfqpoint{5.203751in}{3.667844in}}%
\pgfpathcurveto{\pgfqpoint{5.196618in}{3.674976in}}{\pgfqpoint{5.186942in}{3.678984in}}{\pgfqpoint{5.176855in}{3.678984in}}%
\pgfpathcurveto{\pgfqpoint{5.166768in}{3.678984in}}{\pgfqpoint{5.157092in}{3.674976in}}{\pgfqpoint{5.149959in}{3.667844in}}%
\pgfpathcurveto{\pgfqpoint{5.142827in}{3.660711in}}{\pgfqpoint{5.138819in}{3.651035in}}{\pgfqpoint{5.138819in}{3.640948in}}%
\pgfpathcurveto{\pgfqpoint{5.138819in}{3.630860in}}{\pgfqpoint{5.142827in}{3.621185in}}{\pgfqpoint{5.149959in}{3.614052in}}%
\pgfpathcurveto{\pgfqpoint{5.157092in}{3.606919in}}{\pgfqpoint{5.166768in}{3.602912in}}{\pgfqpoint{5.176855in}{3.602912in}}%
\pgfpathclose%
\pgfusepath{stroke,fill}%
\end{pgfscope}%
\begin{pgfscope}%
\pgfpathrectangle{\pgfqpoint{0.800000in}{0.528000in}}{\pgfqpoint{4.960000in}{3.696000in}} %
\pgfusepath{clip}%
\pgfsetbuttcap%
\pgfsetroundjoin%
\definecolor{currentfill}{rgb}{0.121569,0.466667,0.705882}%
\pgfsetfillcolor{currentfill}%
\pgfsetlinewidth{1.003750pt}%
\definecolor{currentstroke}{rgb}{0.121569,0.466667,0.705882}%
\pgfsetstrokecolor{currentstroke}%
\pgfsetdash{}{0pt}%
\pgfpathmoveto{\pgfqpoint{1.673483in}{1.012308in}}%
\pgfpathcurveto{\pgfqpoint{1.683570in}{1.012308in}}{\pgfqpoint{1.693246in}{1.016316in}}{\pgfqpoint{1.700378in}{1.023448in}}%
\pgfpathcurveto{\pgfqpoint{1.707511in}{1.030581in}}{\pgfqpoint{1.711519in}{1.040257in}}{\pgfqpoint{1.711519in}{1.050344in}}%
\pgfpathcurveto{\pgfqpoint{1.711519in}{1.060431in}}{\pgfqpoint{1.707511in}{1.070107in}}{\pgfqpoint{1.700378in}{1.077240in}}%
\pgfpathcurveto{\pgfqpoint{1.693246in}{1.084373in}}{\pgfqpoint{1.683570in}{1.088380in}}{\pgfqpoint{1.673483in}{1.088380in}}%
\pgfpathcurveto{\pgfqpoint{1.663395in}{1.088380in}}{\pgfqpoint{1.653720in}{1.084373in}}{\pgfqpoint{1.646587in}{1.077240in}}%
\pgfpathcurveto{\pgfqpoint{1.639454in}{1.070107in}}{\pgfqpoint{1.635446in}{1.060431in}}{\pgfqpoint{1.635446in}{1.050344in}}%
\pgfpathcurveto{\pgfqpoint{1.635446in}{1.040257in}}{\pgfqpoint{1.639454in}{1.030581in}}{\pgfqpoint{1.646587in}{1.023448in}}%
\pgfpathcurveto{\pgfqpoint{1.653720in}{1.016316in}}{\pgfqpoint{1.663395in}{1.012308in}}{\pgfqpoint{1.673483in}{1.012308in}}%
\pgfpathclose%
\pgfusepath{stroke,fill}%
\end{pgfscope}%
\begin{pgfscope}%
\pgfpathrectangle{\pgfqpoint{0.800000in}{0.528000in}}{\pgfqpoint{4.960000in}{3.696000in}} %
\pgfusepath{clip}%
\pgfsetbuttcap%
\pgfsetroundjoin%
\definecolor{currentfill}{rgb}{0.121569,0.466667,0.705882}%
\pgfsetfillcolor{currentfill}%
\pgfsetlinewidth{1.003750pt}%
\definecolor{currentstroke}{rgb}{0.121569,0.466667,0.705882}%
\pgfsetstrokecolor{currentstroke}%
\pgfsetdash{}{0pt}%
\pgfpathmoveto{\pgfqpoint{1.342173in}{1.561903in}}%
\pgfpathcurveto{\pgfqpoint{1.352261in}{1.561903in}}{\pgfqpoint{1.361936in}{1.565911in}}{\pgfqpoint{1.369069in}{1.573044in}}%
\pgfpathcurveto{\pgfqpoint{1.376202in}{1.580177in}}{\pgfqpoint{1.380210in}{1.589852in}}{\pgfqpoint{1.380210in}{1.599940in}}%
\pgfpathcurveto{\pgfqpoint{1.380210in}{1.610027in}}{\pgfqpoint{1.376202in}{1.619703in}}{\pgfqpoint{1.369069in}{1.626835in}}%
\pgfpathcurveto{\pgfqpoint{1.361936in}{1.633968in}}{\pgfqpoint{1.352261in}{1.637976in}}{\pgfqpoint{1.342173in}{1.637976in}}%
\pgfpathcurveto{\pgfqpoint{1.332086in}{1.637976in}}{\pgfqpoint{1.322411in}{1.633968in}}{\pgfqpoint{1.315278in}{1.626835in}}%
\pgfpathcurveto{\pgfqpoint{1.308145in}{1.619703in}}{\pgfqpoint{1.304137in}{1.610027in}}{\pgfqpoint{1.304137in}{1.599940in}}%
\pgfpathcurveto{\pgfqpoint{1.304137in}{1.589852in}}{\pgfqpoint{1.308145in}{1.580177in}}{\pgfqpoint{1.315278in}{1.573044in}}%
\pgfpathcurveto{\pgfqpoint{1.322411in}{1.565911in}}{\pgfqpoint{1.332086in}{1.561903in}}{\pgfqpoint{1.342173in}{1.561903in}}%
\pgfpathclose%
\pgfusepath{stroke,fill}%
\end{pgfscope}%
\begin{pgfscope}%
\pgfpathrectangle{\pgfqpoint{0.800000in}{0.528000in}}{\pgfqpoint{4.960000in}{3.696000in}} %
\pgfusepath{clip}%
\pgfsetbuttcap%
\pgfsetroundjoin%
\definecolor{currentfill}{rgb}{0.121569,0.466667,0.705882}%
\pgfsetfillcolor{currentfill}%
\pgfsetlinewidth{1.003750pt}%
\definecolor{currentstroke}{rgb}{0.121569,0.466667,0.705882}%
\pgfsetstrokecolor{currentstroke}%
\pgfsetdash{}{0pt}%
\pgfpathmoveto{\pgfqpoint{1.031720in}{2.142608in}}%
\pgfpathcurveto{\pgfqpoint{1.041808in}{2.142608in}}{\pgfqpoint{1.051483in}{2.146615in}}{\pgfqpoint{1.058616in}{2.153748in}}%
\pgfpathcurveto{\pgfqpoint{1.065749in}{2.160881in}}{\pgfqpoint{1.069757in}{2.170557in}}{\pgfqpoint{1.069757in}{2.180644in}}%
\pgfpathcurveto{\pgfqpoint{1.069757in}{2.190731in}}{\pgfqpoint{1.065749in}{2.200407in}}{\pgfqpoint{1.058616in}{2.207540in}}%
\pgfpathcurveto{\pgfqpoint{1.051483in}{2.214673in}}{\pgfqpoint{1.041808in}{2.218680in}}{\pgfqpoint{1.031720in}{2.218680in}}%
\pgfpathcurveto{\pgfqpoint{1.021633in}{2.218680in}}{\pgfqpoint{1.011957in}{2.214673in}}{\pgfqpoint{1.004825in}{2.207540in}}%
\pgfpathcurveto{\pgfqpoint{0.997692in}{2.200407in}}{\pgfqpoint{0.993684in}{2.190731in}}{\pgfqpoint{0.993684in}{2.180644in}}%
\pgfpathcurveto{\pgfqpoint{0.993684in}{2.170557in}}{\pgfqpoint{0.997692in}{2.160881in}}{\pgfqpoint{1.004825in}{2.153748in}}%
\pgfpathcurveto{\pgfqpoint{1.011957in}{2.146615in}}{\pgfqpoint{1.021633in}{2.142608in}}{\pgfqpoint{1.031720in}{2.142608in}}%
\pgfpathclose%
\pgfusepath{stroke,fill}%
\end{pgfscope}%
\begin{pgfscope}%
\pgfpathrectangle{\pgfqpoint{0.800000in}{0.528000in}}{\pgfqpoint{4.960000in}{3.696000in}} %
\pgfusepath{clip}%
\pgfsetbuttcap%
\pgfsetroundjoin%
\definecolor{currentfill}{rgb}{0.121569,0.466667,0.705882}%
\pgfsetfillcolor{currentfill}%
\pgfsetlinewidth{1.003750pt}%
\definecolor{currentstroke}{rgb}{0.121569,0.466667,0.705882}%
\pgfsetstrokecolor{currentstroke}%
\pgfsetdash{}{0pt}%
\pgfpathmoveto{\pgfqpoint{1.356249in}{3.160048in}}%
\pgfpathcurveto{\pgfqpoint{1.366337in}{3.160048in}}{\pgfqpoint{1.376012in}{3.164056in}}{\pgfqpoint{1.383145in}{3.171189in}}%
\pgfpathcurveto{\pgfqpoint{1.390278in}{3.178321in}}{\pgfqpoint{1.394286in}{3.187997in}}{\pgfqpoint{1.394286in}{3.198084in}}%
\pgfpathcurveto{\pgfqpoint{1.394286in}{3.208172in}}{\pgfqpoint{1.390278in}{3.217847in}}{\pgfqpoint{1.383145in}{3.224980in}}%
\pgfpathcurveto{\pgfqpoint{1.376012in}{3.232113in}}{\pgfqpoint{1.366337in}{3.236121in}}{\pgfqpoint{1.356249in}{3.236121in}}%
\pgfpathcurveto{\pgfqpoint{1.346162in}{3.236121in}}{\pgfqpoint{1.336486in}{3.232113in}}{\pgfqpoint{1.329354in}{3.224980in}}%
\pgfpathcurveto{\pgfqpoint{1.322221in}{3.217847in}}{\pgfqpoint{1.318213in}{3.208172in}}{\pgfqpoint{1.318213in}{3.198084in}}%
\pgfpathcurveto{\pgfqpoint{1.318213in}{3.187997in}}{\pgfqpoint{1.322221in}{3.178321in}}{\pgfqpoint{1.329354in}{3.171189in}}%
\pgfpathcurveto{\pgfqpoint{1.336486in}{3.164056in}}{\pgfqpoint{1.346162in}{3.160048in}}{\pgfqpoint{1.356249in}{3.160048in}}%
\pgfpathclose%
\pgfusepath{stroke,fill}%
\end{pgfscope}%
\begin{pgfscope}%
\pgfpathrectangle{\pgfqpoint{0.800000in}{0.528000in}}{\pgfqpoint{4.960000in}{3.696000in}} %
\pgfusepath{clip}%
\pgfsetbuttcap%
\pgfsetroundjoin%
\definecolor{currentfill}{rgb}{0.121569,0.466667,0.705882}%
\pgfsetfillcolor{currentfill}%
\pgfsetlinewidth{1.003750pt}%
\definecolor{currentstroke}{rgb}{0.121569,0.466667,0.705882}%
\pgfsetstrokecolor{currentstroke}%
\pgfsetdash{}{0pt}%
\pgfpathmoveto{\pgfqpoint{1.528570in}{3.587024in}}%
\pgfpathcurveto{\pgfqpoint{1.538657in}{3.587024in}}{\pgfqpoint{1.548333in}{3.591032in}}{\pgfqpoint{1.555466in}{3.598165in}}%
\pgfpathcurveto{\pgfqpoint{1.562599in}{3.605297in}}{\pgfqpoint{1.566606in}{3.614973in}}{\pgfqpoint{1.566606in}{3.625060in}}%
\pgfpathcurveto{\pgfqpoint{1.566606in}{3.635148in}}{\pgfqpoint{1.562599in}{3.644823in}}{\pgfqpoint{1.555466in}{3.651956in}}%
\pgfpathcurveto{\pgfqpoint{1.548333in}{3.659089in}}{\pgfqpoint{1.538657in}{3.663097in}}{\pgfqpoint{1.528570in}{3.663097in}}%
\pgfpathcurveto{\pgfqpoint{1.518483in}{3.663097in}}{\pgfqpoint{1.508807in}{3.659089in}}{\pgfqpoint{1.501674in}{3.651956in}}%
\pgfpathcurveto{\pgfqpoint{1.494541in}{3.644823in}}{\pgfqpoint{1.490534in}{3.635148in}}{\pgfqpoint{1.490534in}{3.625060in}}%
\pgfpathcurveto{\pgfqpoint{1.490534in}{3.614973in}}{\pgfqpoint{1.494541in}{3.605297in}}{\pgfqpoint{1.501674in}{3.598165in}}%
\pgfpathcurveto{\pgfqpoint{1.508807in}{3.591032in}}{\pgfqpoint{1.518483in}{3.587024in}}{\pgfqpoint{1.528570in}{3.587024in}}%
\pgfpathclose%
\pgfusepath{stroke,fill}%
\end{pgfscope}%
\begin{pgfscope}%
\pgfpathrectangle{\pgfqpoint{0.800000in}{0.528000in}}{\pgfqpoint{4.960000in}{3.696000in}} %
\pgfusepath{clip}%
\pgfsetbuttcap%
\pgfsetroundjoin%
\definecolor{currentfill}{rgb}{0.121569,0.466667,0.705882}%
\pgfsetfillcolor{currentfill}%
\pgfsetlinewidth{1.003750pt}%
\definecolor{currentstroke}{rgb}{0.121569,0.466667,0.705882}%
\pgfsetstrokecolor{currentstroke}%
\pgfsetdash{}{0pt}%
\pgfpathmoveto{\pgfqpoint{2.428361in}{0.692038in}}%
\pgfpathcurveto{\pgfqpoint{2.438448in}{0.692038in}}{\pgfqpoint{2.448123in}{0.696046in}}{\pgfqpoint{2.455256in}{0.703179in}}%
\pgfpathcurveto{\pgfqpoint{2.462389in}{0.710312in}}{\pgfqpoint{2.466397in}{0.719987in}}{\pgfqpoint{2.466397in}{0.730075in}}%
\pgfpathcurveto{\pgfqpoint{2.466397in}{0.740162in}}{\pgfqpoint{2.462389in}{0.749838in}}{\pgfqpoint{2.455256in}{0.756970in}}%
\pgfpathcurveto{\pgfqpoint{2.448123in}{0.764103in}}{\pgfqpoint{2.438448in}{0.768111in}}{\pgfqpoint{2.428361in}{0.768111in}}%
\pgfpathcurveto{\pgfqpoint{2.418273in}{0.768111in}}{\pgfqpoint{2.408598in}{0.764103in}}{\pgfqpoint{2.401465in}{0.756970in}}%
\pgfpathcurveto{\pgfqpoint{2.394332in}{0.749838in}}{\pgfqpoint{2.390324in}{0.740162in}}{\pgfqpoint{2.390324in}{0.730075in}}%
\pgfpathcurveto{\pgfqpoint{2.390324in}{0.719987in}}{\pgfqpoint{2.394332in}{0.710312in}}{\pgfqpoint{2.401465in}{0.703179in}}%
\pgfpathcurveto{\pgfqpoint{2.408598in}{0.696046in}}{\pgfqpoint{2.418273in}{0.692038in}}{\pgfqpoint{2.428361in}{0.692038in}}%
\pgfpathclose%
\pgfusepath{stroke,fill}%
\end{pgfscope}%
\begin{pgfscope}%
\pgfpathrectangle{\pgfqpoint{0.800000in}{0.528000in}}{\pgfqpoint{4.960000in}{3.696000in}} %
\pgfusepath{clip}%
\pgfsetbuttcap%
\pgfsetroundjoin%
\definecolor{currentfill}{rgb}{0.121569,0.466667,0.705882}%
\pgfsetfillcolor{currentfill}%
\pgfsetlinewidth{1.003750pt}%
\definecolor{currentstroke}{rgb}{0.121569,0.466667,0.705882}%
\pgfsetstrokecolor{currentstroke}%
\pgfsetdash{}{0pt}%
\pgfpathmoveto{\pgfqpoint{2.524795in}{1.092081in}}%
\pgfpathcurveto{\pgfqpoint{2.534882in}{1.092081in}}{\pgfqpoint{2.544558in}{1.096089in}}{\pgfqpoint{2.551690in}{1.103222in}}%
\pgfpathcurveto{\pgfqpoint{2.558823in}{1.110354in}}{\pgfqpoint{2.562831in}{1.120030in}}{\pgfqpoint{2.562831in}{1.130117in}}%
\pgfpathcurveto{\pgfqpoint{2.562831in}{1.140205in}}{\pgfqpoint{2.558823in}{1.149880in}}{\pgfqpoint{2.551690in}{1.157013in}}%
\pgfpathcurveto{\pgfqpoint{2.544558in}{1.164146in}}{\pgfqpoint{2.534882in}{1.168154in}}{\pgfqpoint{2.524795in}{1.168154in}}%
\pgfpathcurveto{\pgfqpoint{2.514707in}{1.168154in}}{\pgfqpoint{2.505032in}{1.164146in}}{\pgfqpoint{2.497899in}{1.157013in}}%
\pgfpathcurveto{\pgfqpoint{2.490766in}{1.149880in}}{\pgfqpoint{2.486758in}{1.140205in}}{\pgfqpoint{2.486758in}{1.130117in}}%
\pgfpathcurveto{\pgfqpoint{2.486758in}{1.120030in}}{\pgfqpoint{2.490766in}{1.110354in}}{\pgfqpoint{2.497899in}{1.103222in}}%
\pgfpathcurveto{\pgfqpoint{2.505032in}{1.096089in}}{\pgfqpoint{2.514707in}{1.092081in}}{\pgfqpoint{2.524795in}{1.092081in}}%
\pgfpathclose%
\pgfusepath{stroke,fill}%
\end{pgfscope}%
\begin{pgfscope}%
\pgfpathrectangle{\pgfqpoint{0.800000in}{0.528000in}}{\pgfqpoint{4.960000in}{3.696000in}} %
\pgfusepath{clip}%
\pgfsetbuttcap%
\pgfsetroundjoin%
\definecolor{currentfill}{rgb}{0.121569,0.466667,0.705882}%
\pgfsetfillcolor{currentfill}%
\pgfsetlinewidth{1.003750pt}%
\definecolor{currentstroke}{rgb}{0.121569,0.466667,0.705882}%
\pgfsetstrokecolor{currentstroke}%
\pgfsetdash{}{0pt}%
\pgfpathmoveto{\pgfqpoint{2.371871in}{3.992571in}}%
\pgfpathcurveto{\pgfqpoint{2.381958in}{3.992571in}}{\pgfqpoint{2.391634in}{3.996578in}}{\pgfqpoint{2.398766in}{4.003711in}}%
\pgfpathcurveto{\pgfqpoint{2.405899in}{4.010844in}}{\pgfqpoint{2.409907in}{4.020520in}}{\pgfqpoint{2.409907in}{4.030607in}}%
\pgfpathcurveto{\pgfqpoint{2.409907in}{4.040694in}}{\pgfqpoint{2.405899in}{4.050370in}}{\pgfqpoint{2.398766in}{4.057503in}}%
\pgfpathcurveto{\pgfqpoint{2.391634in}{4.064636in}}{\pgfqpoint{2.381958in}{4.068643in}}{\pgfqpoint{2.371871in}{4.068643in}}%
\pgfpathcurveto{\pgfqpoint{2.361783in}{4.068643in}}{\pgfqpoint{2.352108in}{4.064636in}}{\pgfqpoint{2.344975in}{4.057503in}}%
\pgfpathcurveto{\pgfqpoint{2.337842in}{4.050370in}}{\pgfqpoint{2.333834in}{4.040694in}}{\pgfqpoint{2.333834in}{4.030607in}}%
\pgfpathcurveto{\pgfqpoint{2.333834in}{4.020520in}}{\pgfqpoint{2.337842in}{4.010844in}}{\pgfqpoint{2.344975in}{4.003711in}}%
\pgfpathcurveto{\pgfqpoint{2.352108in}{3.996578in}}{\pgfqpoint{2.361783in}{3.992571in}}{\pgfqpoint{2.371871in}{3.992571in}}%
\pgfpathclose%
\pgfusepath{stroke,fill}%
\end{pgfscope}%
\begin{pgfscope}%
\pgfpathrectangle{\pgfqpoint{0.800000in}{0.528000in}}{\pgfqpoint{4.960000in}{3.696000in}} %
\pgfusepath{clip}%
\pgfsetbuttcap%
\pgfsetroundjoin%
\definecolor{currentfill}{rgb}{0.121569,0.466667,0.705882}%
\pgfsetfillcolor{currentfill}%
\pgfsetlinewidth{1.003750pt}%
\definecolor{currentstroke}{rgb}{0.121569,0.466667,0.705882}%
\pgfsetstrokecolor{currentstroke}%
\pgfsetdash{}{0pt}%
\pgfpathmoveto{\pgfqpoint{3.317771in}{0.881055in}}%
\pgfpathcurveto{\pgfqpoint{3.327858in}{0.881055in}}{\pgfqpoint{3.337534in}{0.885062in}}{\pgfqpoint{3.344667in}{0.892195in}}%
\pgfpathcurveto{\pgfqpoint{3.351799in}{0.899328in}}{\pgfqpoint{3.355807in}{0.909004in}}{\pgfqpoint{3.355807in}{0.919091in}}%
\pgfpathcurveto{\pgfqpoint{3.355807in}{0.929178in}}{\pgfqpoint{3.351799in}{0.938854in}}{\pgfqpoint{3.344667in}{0.945987in}}%
\pgfpathcurveto{\pgfqpoint{3.337534in}{0.953120in}}{\pgfqpoint{3.327858in}{0.957127in}}{\pgfqpoint{3.317771in}{0.957127in}}%
\pgfpathcurveto{\pgfqpoint{3.307683in}{0.957127in}}{\pgfqpoint{3.298008in}{0.953120in}}{\pgfqpoint{3.290875in}{0.945987in}}%
\pgfpathcurveto{\pgfqpoint{3.283742in}{0.938854in}}{\pgfqpoint{3.279735in}{0.929178in}}{\pgfqpoint{3.279735in}{0.919091in}}%
\pgfpathcurveto{\pgfqpoint{3.279735in}{0.909004in}}{\pgfqpoint{3.283742in}{0.899328in}}{\pgfqpoint{3.290875in}{0.892195in}}%
\pgfpathcurveto{\pgfqpoint{3.298008in}{0.885062in}}{\pgfqpoint{3.307683in}{0.881055in}}{\pgfqpoint{3.317771in}{0.881055in}}%
\pgfpathclose%
\pgfusepath{stroke,fill}%
\end{pgfscope}%
\begin{pgfscope}%
\pgfpathrectangle{\pgfqpoint{0.800000in}{0.528000in}}{\pgfqpoint{4.960000in}{3.696000in}} %
\pgfusepath{clip}%
\pgfsetbuttcap%
\pgfsetroundjoin%
\definecolor{currentfill}{rgb}{0.121569,0.466667,0.705882}%
\pgfsetfillcolor{currentfill}%
\pgfsetlinewidth{1.003750pt}%
\definecolor{currentstroke}{rgb}{0.121569,0.466667,0.705882}%
\pgfsetstrokecolor{currentstroke}%
\pgfsetdash{}{0pt}%
\pgfpathmoveto{\pgfqpoint{3.188228in}{3.959324in}}%
\pgfpathcurveto{\pgfqpoint{3.198316in}{3.959324in}}{\pgfqpoint{3.207991in}{3.963332in}}{\pgfqpoint{3.215124in}{3.970465in}}%
\pgfpathcurveto{\pgfqpoint{3.222257in}{3.977598in}}{\pgfqpoint{3.226265in}{3.987273in}}{\pgfqpoint{3.226265in}{3.997361in}}%
\pgfpathcurveto{\pgfqpoint{3.226265in}{4.007448in}}{\pgfqpoint{3.222257in}{4.017123in}}{\pgfqpoint{3.215124in}{4.024256in}}%
\pgfpathcurveto{\pgfqpoint{3.207991in}{4.031389in}}{\pgfqpoint{3.198316in}{4.035397in}}{\pgfqpoint{3.188228in}{4.035397in}}%
\pgfpathcurveto{\pgfqpoint{3.178141in}{4.035397in}}{\pgfqpoint{3.168465in}{4.031389in}}{\pgfqpoint{3.161333in}{4.024256in}}%
\pgfpathcurveto{\pgfqpoint{3.154200in}{4.017123in}}{\pgfqpoint{3.150192in}{4.007448in}}{\pgfqpoint{3.150192in}{3.997361in}}%
\pgfpathcurveto{\pgfqpoint{3.150192in}{3.987273in}}{\pgfqpoint{3.154200in}{3.977598in}}{\pgfqpoint{3.161333in}{3.970465in}}%
\pgfpathcurveto{\pgfqpoint{3.168465in}{3.963332in}}{\pgfqpoint{3.178141in}{3.959324in}}{\pgfqpoint{3.188228in}{3.959324in}}%
\pgfpathclose%
\pgfusepath{stroke,fill}%
\end{pgfscope}%
\begin{pgfscope}%
\pgfpathrectangle{\pgfqpoint{0.800000in}{0.528000in}}{\pgfqpoint{4.960000in}{3.696000in}} %
\pgfusepath{clip}%
\pgfsetbuttcap%
\pgfsetroundjoin%
\definecolor{currentfill}{rgb}{0.121569,0.466667,0.705882}%
\pgfsetfillcolor{currentfill}%
\pgfsetlinewidth{1.003750pt}%
\definecolor{currentstroke}{rgb}{0.121569,0.466667,0.705882}%
\pgfsetstrokecolor{currentstroke}%
\pgfsetdash{}{0pt}%
\pgfpathmoveto{\pgfqpoint{4.207830in}{0.683357in}}%
\pgfpathcurveto{\pgfqpoint{4.217917in}{0.683357in}}{\pgfqpoint{4.227593in}{0.687364in}}{\pgfqpoint{4.234726in}{0.694497in}}%
\pgfpathcurveto{\pgfqpoint{4.241858in}{0.701630in}}{\pgfqpoint{4.245866in}{0.711306in}}{\pgfqpoint{4.245866in}{0.721393in}}%
\pgfpathcurveto{\pgfqpoint{4.245866in}{0.731480in}}{\pgfqpoint{4.241858in}{0.741156in}}{\pgfqpoint{4.234726in}{0.748289in}}%
\pgfpathcurveto{\pgfqpoint{4.227593in}{0.755422in}}{\pgfqpoint{4.217917in}{0.759429in}}{\pgfqpoint{4.207830in}{0.759429in}}%
\pgfpathcurveto{\pgfqpoint{4.197743in}{0.759429in}}{\pgfqpoint{4.188067in}{0.755422in}}{\pgfqpoint{4.180934in}{0.748289in}}%
\pgfpathcurveto{\pgfqpoint{4.173801in}{0.741156in}}{\pgfqpoint{4.169794in}{0.731480in}}{\pgfqpoint{4.169794in}{0.721393in}}%
\pgfpathcurveto{\pgfqpoint{4.169794in}{0.711306in}}{\pgfqpoint{4.173801in}{0.701630in}}{\pgfqpoint{4.180934in}{0.694497in}}%
\pgfpathcurveto{\pgfqpoint{4.188067in}{0.687364in}}{\pgfqpoint{4.197743in}{0.683357in}}{\pgfqpoint{4.207830in}{0.683357in}}%
\pgfpathclose%
\pgfusepath{stroke,fill}%
\end{pgfscope}%
\begin{pgfscope}%
\pgfpathrectangle{\pgfqpoint{0.800000in}{0.528000in}}{\pgfqpoint{4.960000in}{3.696000in}} %
\pgfusepath{clip}%
\pgfsetbuttcap%
\pgfsetroundjoin%
\definecolor{currentfill}{rgb}{0.121569,0.466667,0.705882}%
\pgfsetfillcolor{currentfill}%
\pgfsetlinewidth{1.003750pt}%
\definecolor{currentstroke}{rgb}{0.121569,0.466667,0.705882}%
\pgfsetstrokecolor{currentstroke}%
\pgfsetdash{}{0pt}%
\pgfpathmoveto{\pgfqpoint{4.827101in}{1.237602in}}%
\pgfpathcurveto{\pgfqpoint{4.837189in}{1.237602in}}{\pgfqpoint{4.846864in}{1.241610in}}{\pgfqpoint{4.853997in}{1.248743in}}%
\pgfpathcurveto{\pgfqpoint{4.861130in}{1.255875in}}{\pgfqpoint{4.865138in}{1.265551in}}{\pgfqpoint{4.865138in}{1.275638in}}%
\pgfpathcurveto{\pgfqpoint{4.865138in}{1.285726in}}{\pgfqpoint{4.861130in}{1.295401in}}{\pgfqpoint{4.853997in}{1.302534in}}%
\pgfpathcurveto{\pgfqpoint{4.846864in}{1.309667in}}{\pgfqpoint{4.837189in}{1.313675in}}{\pgfqpoint{4.827101in}{1.313675in}}%
\pgfpathcurveto{\pgfqpoint{4.817014in}{1.313675in}}{\pgfqpoint{4.807338in}{1.309667in}}{\pgfqpoint{4.800206in}{1.302534in}}%
\pgfpathcurveto{\pgfqpoint{4.793073in}{1.295401in}}{\pgfqpoint{4.789065in}{1.285726in}}{\pgfqpoint{4.789065in}{1.275638in}}%
\pgfpathcurveto{\pgfqpoint{4.789065in}{1.265551in}}{\pgfqpoint{4.793073in}{1.255875in}}{\pgfqpoint{4.800206in}{1.248743in}}%
\pgfpathcurveto{\pgfqpoint{4.807338in}{1.241610in}}{\pgfqpoint{4.817014in}{1.237602in}}{\pgfqpoint{4.827101in}{1.237602in}}%
\pgfpathclose%
\pgfusepath{stroke,fill}%
\end{pgfscope}%
\begin{pgfscope}%
\pgfpathrectangle{\pgfqpoint{0.800000in}{0.528000in}}{\pgfqpoint{4.960000in}{3.696000in}} %
\pgfusepath{clip}%
\pgfsetbuttcap%
\pgfsetroundjoin%
\definecolor{currentfill}{rgb}{0.121569,0.466667,0.705882}%
\pgfsetfillcolor{currentfill}%
\pgfsetlinewidth{1.003750pt}%
\definecolor{currentstroke}{rgb}{0.121569,0.466667,0.705882}%
\pgfsetstrokecolor{currentstroke}%
\pgfsetdash{}{0pt}%
\pgfpathmoveto{\pgfqpoint{4.468060in}{3.772899in}}%
\pgfpathcurveto{\pgfqpoint{4.478147in}{3.772899in}}{\pgfqpoint{4.487822in}{3.776907in}}{\pgfqpoint{4.494955in}{3.784039in}}%
\pgfpathcurveto{\pgfqpoint{4.502088in}{3.791172in}}{\pgfqpoint{4.506096in}{3.800848in}}{\pgfqpoint{4.506096in}{3.810935in}}%
\pgfpathcurveto{\pgfqpoint{4.506096in}{3.821022in}}{\pgfqpoint{4.502088in}{3.830698in}}{\pgfqpoint{4.494955in}{3.837831in}}%
\pgfpathcurveto{\pgfqpoint{4.487822in}{3.844964in}}{\pgfqpoint{4.478147in}{3.848971in}}{\pgfqpoint{4.468060in}{3.848971in}}%
\pgfpathcurveto{\pgfqpoint{4.457972in}{3.848971in}}{\pgfqpoint{4.448297in}{3.844964in}}{\pgfqpoint{4.441164in}{3.837831in}}%
\pgfpathcurveto{\pgfqpoint{4.434031in}{3.830698in}}{\pgfqpoint{4.430023in}{3.821022in}}{\pgfqpoint{4.430023in}{3.810935in}}%
\pgfpathcurveto{\pgfqpoint{4.430023in}{3.800848in}}{\pgfqpoint{4.434031in}{3.791172in}}{\pgfqpoint{4.441164in}{3.784039in}}%
\pgfpathcurveto{\pgfqpoint{4.448297in}{3.776907in}}{\pgfqpoint{4.457972in}{3.772899in}}{\pgfqpoint{4.468060in}{3.772899in}}%
\pgfpathclose%
\pgfusepath{stroke,fill}%
\end{pgfscope}%
\begin{pgfscope}%
\pgfpathrectangle{\pgfqpoint{0.800000in}{0.528000in}}{\pgfqpoint{4.960000in}{3.696000in}} %
\pgfusepath{clip}%
\pgfsetbuttcap%
\pgfsetroundjoin%
\definecolor{currentfill}{rgb}{0.121569,0.466667,0.705882}%
\pgfsetfillcolor{currentfill}%
\pgfsetlinewidth{1.003750pt}%
\definecolor{currentstroke}{rgb}{0.121569,0.466667,0.705882}%
\pgfsetstrokecolor{currentstroke}%
\pgfsetdash{}{0pt}%
\pgfpathmoveto{\pgfqpoint{5.296265in}{1.544799in}}%
\pgfpathcurveto{\pgfqpoint{5.306352in}{1.544799in}}{\pgfqpoint{5.316027in}{1.548807in}}{\pgfqpoint{5.323160in}{1.555940in}}%
\pgfpathcurveto{\pgfqpoint{5.330293in}{1.563073in}}{\pgfqpoint{5.334301in}{1.572748in}}{\pgfqpoint{5.334301in}{1.582836in}}%
\pgfpathcurveto{\pgfqpoint{5.334301in}{1.592923in}}{\pgfqpoint{5.330293in}{1.602599in}}{\pgfqpoint{5.323160in}{1.609731in}}%
\pgfpathcurveto{\pgfqpoint{5.316027in}{1.616864in}}{\pgfqpoint{5.306352in}{1.620872in}}{\pgfqpoint{5.296265in}{1.620872in}}%
\pgfpathcurveto{\pgfqpoint{5.286177in}{1.620872in}}{\pgfqpoint{5.276502in}{1.616864in}}{\pgfqpoint{5.269369in}{1.609731in}}%
\pgfpathcurveto{\pgfqpoint{5.262236in}{1.602599in}}{\pgfqpoint{5.258228in}{1.592923in}}{\pgfqpoint{5.258228in}{1.582836in}}%
\pgfpathcurveto{\pgfqpoint{5.258228in}{1.572748in}}{\pgfqpoint{5.262236in}{1.563073in}}{\pgfqpoint{5.269369in}{1.555940in}}%
\pgfpathcurveto{\pgfqpoint{5.276502in}{1.548807in}}{\pgfqpoint{5.286177in}{1.544799in}}{\pgfqpoint{5.296265in}{1.544799in}}%
\pgfpathclose%
\pgfusepath{stroke,fill}%
\end{pgfscope}%
\begin{pgfscope}%
\pgfpathrectangle{\pgfqpoint{0.800000in}{0.528000in}}{\pgfqpoint{4.960000in}{3.696000in}} %
\pgfusepath{clip}%
\pgfsetbuttcap%
\pgfsetroundjoin%
\definecolor{currentfill}{rgb}{0.121569,0.466667,0.705882}%
\pgfsetfillcolor{currentfill}%
\pgfsetlinewidth{1.003750pt}%
\definecolor{currentstroke}{rgb}{0.121569,0.466667,0.705882}%
\pgfsetstrokecolor{currentstroke}%
\pgfsetdash{}{0pt}%
\pgfpathmoveto{\pgfqpoint{5.528280in}{2.502909in}}%
\pgfpathcurveto{\pgfqpoint{5.538367in}{2.502909in}}{\pgfqpoint{5.548043in}{2.506917in}}{\pgfqpoint{5.555175in}{2.514049in}}%
\pgfpathcurveto{\pgfqpoint{5.562308in}{2.521182in}}{\pgfqpoint{5.566316in}{2.530858in}}{\pgfqpoint{5.566316in}{2.540945in}}%
\pgfpathcurveto{\pgfqpoint{5.566316in}{2.551033in}}{\pgfqpoint{5.562308in}{2.560708in}}{\pgfqpoint{5.555175in}{2.567841in}}%
\pgfpathcurveto{\pgfqpoint{5.548043in}{2.574974in}}{\pgfqpoint{5.538367in}{2.578982in}}{\pgfqpoint{5.528280in}{2.578982in}}%
\pgfpathcurveto{\pgfqpoint{5.518192in}{2.578982in}}{\pgfqpoint{5.508517in}{2.574974in}}{\pgfqpoint{5.501384in}{2.567841in}}%
\pgfpathcurveto{\pgfqpoint{5.494251in}{2.560708in}}{\pgfqpoint{5.490243in}{2.551033in}}{\pgfqpoint{5.490243in}{2.540945in}}%
\pgfpathcurveto{\pgfqpoint{5.490243in}{2.530858in}}{\pgfqpoint{5.494251in}{2.521182in}}{\pgfqpoint{5.501384in}{2.514049in}}%
\pgfpathcurveto{\pgfqpoint{5.508517in}{2.506917in}}{\pgfqpoint{5.518192in}{2.502909in}}{\pgfqpoint{5.528280in}{2.502909in}}%
\pgfpathclose%
\pgfusepath{stroke,fill}%
\end{pgfscope}%
\begin{pgfscope}%
\pgfpathrectangle{\pgfqpoint{0.800000in}{0.528000in}}{\pgfqpoint{4.960000in}{3.696000in}} %
\pgfusepath{clip}%
\pgfsetbuttcap%
\pgfsetroundjoin%
\definecolor{currentfill}{rgb}{0.121569,0.466667,0.705882}%
\pgfsetfillcolor{currentfill}%
\pgfsetlinewidth{1.003750pt}%
\definecolor{currentstroke}{rgb}{0.121569,0.466667,0.705882}%
\pgfsetstrokecolor{currentstroke}%
\pgfsetdash{}{0pt}%
\pgfpathmoveto{\pgfqpoint{5.328319in}{3.240906in}}%
\pgfpathcurveto{\pgfqpoint{5.338407in}{3.240906in}}{\pgfqpoint{5.348082in}{3.244914in}}{\pgfqpoint{5.355215in}{3.252047in}}%
\pgfpathcurveto{\pgfqpoint{5.362348in}{3.259180in}}{\pgfqpoint{5.366356in}{3.268855in}}{\pgfqpoint{5.366356in}{3.278943in}}%
\pgfpathcurveto{\pgfqpoint{5.366356in}{3.289030in}}{\pgfqpoint{5.362348in}{3.298706in}}{\pgfqpoint{5.355215in}{3.305838in}}%
\pgfpathcurveto{\pgfqpoint{5.348082in}{3.312971in}}{\pgfqpoint{5.338407in}{3.316979in}}{\pgfqpoint{5.328319in}{3.316979in}}%
\pgfpathcurveto{\pgfqpoint{5.318232in}{3.316979in}}{\pgfqpoint{5.308557in}{3.312971in}}{\pgfqpoint{5.301424in}{3.305838in}}%
\pgfpathcurveto{\pgfqpoint{5.294291in}{3.298706in}}{\pgfqpoint{5.290283in}{3.289030in}}{\pgfqpoint{5.290283in}{3.278943in}}%
\pgfpathcurveto{\pgfqpoint{5.290283in}{3.268855in}}{\pgfqpoint{5.294291in}{3.259180in}}{\pgfqpoint{5.301424in}{3.252047in}}%
\pgfpathcurveto{\pgfqpoint{5.308557in}{3.244914in}}{\pgfqpoint{5.318232in}{3.240906in}}{\pgfqpoint{5.328319in}{3.240906in}}%
\pgfpathclose%
\pgfusepath{stroke,fill}%
\end{pgfscope}%
\begin{pgfscope}%
\pgfpathrectangle{\pgfqpoint{0.800000in}{0.528000in}}{\pgfqpoint{4.960000in}{3.696000in}} %
\pgfusepath{clip}%
\pgfsetbuttcap%
\pgfsetroundjoin%
\definecolor{currentfill}{rgb}{0.121569,0.466667,0.705882}%
\pgfsetfillcolor{currentfill}%
\pgfsetlinewidth{1.003750pt}%
\definecolor{currentstroke}{rgb}{0.121569,0.466667,0.705882}%
\pgfsetstrokecolor{currentstroke}%
\pgfsetdash{}{0pt}%
\pgfpathmoveto{\pgfqpoint{5.176855in}{3.602912in}}%
\pgfpathcurveto{\pgfqpoint{5.186942in}{3.602912in}}{\pgfqpoint{5.196618in}{3.606919in}}{\pgfqpoint{5.203751in}{3.614052in}}%
\pgfpathcurveto{\pgfqpoint{5.210884in}{3.621185in}}{\pgfqpoint{5.214891in}{3.630860in}}{\pgfqpoint{5.214891in}{3.640948in}}%
\pgfpathcurveto{\pgfqpoint{5.214891in}{3.651035in}}{\pgfqpoint{5.210884in}{3.660711in}}{\pgfqpoint{5.203751in}{3.667844in}}%
\pgfpathcurveto{\pgfqpoint{5.196618in}{3.674976in}}{\pgfqpoint{5.186942in}{3.678984in}}{\pgfqpoint{5.176855in}{3.678984in}}%
\pgfpathcurveto{\pgfqpoint{5.166768in}{3.678984in}}{\pgfqpoint{5.157092in}{3.674976in}}{\pgfqpoint{5.149959in}{3.667844in}}%
\pgfpathcurveto{\pgfqpoint{5.142827in}{3.660711in}}{\pgfqpoint{5.138819in}{3.651035in}}{\pgfqpoint{5.138819in}{3.640948in}}%
\pgfpathcurveto{\pgfqpoint{5.138819in}{3.630860in}}{\pgfqpoint{5.142827in}{3.621185in}}{\pgfqpoint{5.149959in}{3.614052in}}%
\pgfpathcurveto{\pgfqpoint{5.157092in}{3.606919in}}{\pgfqpoint{5.166768in}{3.602912in}}{\pgfqpoint{5.176855in}{3.602912in}}%
\pgfpathclose%
\pgfusepath{stroke,fill}%
\end{pgfscope}%
\begin{pgfscope}%
\pgfpathrectangle{\pgfqpoint{0.800000in}{0.528000in}}{\pgfqpoint{4.960000in}{3.696000in}} %
\pgfusepath{clip}%
\pgfsetbuttcap%
\pgfsetroundjoin%
\definecolor{currentfill}{rgb}{0.121569,0.466667,0.705882}%
\pgfsetfillcolor{currentfill}%
\pgfsetlinewidth{1.003750pt}%
\definecolor{currentstroke}{rgb}{0.121569,0.466667,0.705882}%
\pgfsetstrokecolor{currentstroke}%
\pgfsetdash{}{0pt}%
\pgfpathmoveto{\pgfqpoint{1.673483in}{1.012308in}}%
\pgfpathcurveto{\pgfqpoint{1.683570in}{1.012308in}}{\pgfqpoint{1.693246in}{1.016316in}}{\pgfqpoint{1.700378in}{1.023448in}}%
\pgfpathcurveto{\pgfqpoint{1.707511in}{1.030581in}}{\pgfqpoint{1.711519in}{1.040257in}}{\pgfqpoint{1.711519in}{1.050344in}}%
\pgfpathcurveto{\pgfqpoint{1.711519in}{1.060431in}}{\pgfqpoint{1.707511in}{1.070107in}}{\pgfqpoint{1.700378in}{1.077240in}}%
\pgfpathcurveto{\pgfqpoint{1.693246in}{1.084373in}}{\pgfqpoint{1.683570in}{1.088380in}}{\pgfqpoint{1.673483in}{1.088380in}}%
\pgfpathcurveto{\pgfqpoint{1.663395in}{1.088380in}}{\pgfqpoint{1.653720in}{1.084373in}}{\pgfqpoint{1.646587in}{1.077240in}}%
\pgfpathcurveto{\pgfqpoint{1.639454in}{1.070107in}}{\pgfqpoint{1.635446in}{1.060431in}}{\pgfqpoint{1.635446in}{1.050344in}}%
\pgfpathcurveto{\pgfqpoint{1.635446in}{1.040257in}}{\pgfqpoint{1.639454in}{1.030581in}}{\pgfqpoint{1.646587in}{1.023448in}}%
\pgfpathcurveto{\pgfqpoint{1.653720in}{1.016316in}}{\pgfqpoint{1.663395in}{1.012308in}}{\pgfqpoint{1.673483in}{1.012308in}}%
\pgfpathclose%
\pgfusepath{stroke,fill}%
\end{pgfscope}%
\begin{pgfscope}%
\pgfpathrectangle{\pgfqpoint{0.800000in}{0.528000in}}{\pgfqpoint{4.960000in}{3.696000in}} %
\pgfusepath{clip}%
\pgfsetbuttcap%
\pgfsetroundjoin%
\definecolor{currentfill}{rgb}{0.121569,0.466667,0.705882}%
\pgfsetfillcolor{currentfill}%
\pgfsetlinewidth{1.003750pt}%
\definecolor{currentstroke}{rgb}{0.121569,0.466667,0.705882}%
\pgfsetstrokecolor{currentstroke}%
\pgfsetdash{}{0pt}%
\pgfpathmoveto{\pgfqpoint{1.342173in}{1.561903in}}%
\pgfpathcurveto{\pgfqpoint{1.352261in}{1.561903in}}{\pgfqpoint{1.361936in}{1.565911in}}{\pgfqpoint{1.369069in}{1.573044in}}%
\pgfpathcurveto{\pgfqpoint{1.376202in}{1.580177in}}{\pgfqpoint{1.380210in}{1.589852in}}{\pgfqpoint{1.380210in}{1.599940in}}%
\pgfpathcurveto{\pgfqpoint{1.380210in}{1.610027in}}{\pgfqpoint{1.376202in}{1.619703in}}{\pgfqpoint{1.369069in}{1.626835in}}%
\pgfpathcurveto{\pgfqpoint{1.361936in}{1.633968in}}{\pgfqpoint{1.352261in}{1.637976in}}{\pgfqpoint{1.342173in}{1.637976in}}%
\pgfpathcurveto{\pgfqpoint{1.332086in}{1.637976in}}{\pgfqpoint{1.322411in}{1.633968in}}{\pgfqpoint{1.315278in}{1.626835in}}%
\pgfpathcurveto{\pgfqpoint{1.308145in}{1.619703in}}{\pgfqpoint{1.304137in}{1.610027in}}{\pgfqpoint{1.304137in}{1.599940in}}%
\pgfpathcurveto{\pgfqpoint{1.304137in}{1.589852in}}{\pgfqpoint{1.308145in}{1.580177in}}{\pgfqpoint{1.315278in}{1.573044in}}%
\pgfpathcurveto{\pgfqpoint{1.322411in}{1.565911in}}{\pgfqpoint{1.332086in}{1.561903in}}{\pgfqpoint{1.342173in}{1.561903in}}%
\pgfpathclose%
\pgfusepath{stroke,fill}%
\end{pgfscope}%
\begin{pgfscope}%
\pgfpathrectangle{\pgfqpoint{0.800000in}{0.528000in}}{\pgfqpoint{4.960000in}{3.696000in}} %
\pgfusepath{clip}%
\pgfsetbuttcap%
\pgfsetroundjoin%
\definecolor{currentfill}{rgb}{0.121569,0.466667,0.705882}%
\pgfsetfillcolor{currentfill}%
\pgfsetlinewidth{1.003750pt}%
\definecolor{currentstroke}{rgb}{0.121569,0.466667,0.705882}%
\pgfsetstrokecolor{currentstroke}%
\pgfsetdash{}{0pt}%
\pgfpathmoveto{\pgfqpoint{1.031720in}{2.142608in}}%
\pgfpathcurveto{\pgfqpoint{1.041808in}{2.142608in}}{\pgfqpoint{1.051483in}{2.146615in}}{\pgfqpoint{1.058616in}{2.153748in}}%
\pgfpathcurveto{\pgfqpoint{1.065749in}{2.160881in}}{\pgfqpoint{1.069757in}{2.170557in}}{\pgfqpoint{1.069757in}{2.180644in}}%
\pgfpathcurveto{\pgfqpoint{1.069757in}{2.190731in}}{\pgfqpoint{1.065749in}{2.200407in}}{\pgfqpoint{1.058616in}{2.207540in}}%
\pgfpathcurveto{\pgfqpoint{1.051483in}{2.214673in}}{\pgfqpoint{1.041808in}{2.218680in}}{\pgfqpoint{1.031720in}{2.218680in}}%
\pgfpathcurveto{\pgfqpoint{1.021633in}{2.218680in}}{\pgfqpoint{1.011957in}{2.214673in}}{\pgfqpoint{1.004825in}{2.207540in}}%
\pgfpathcurveto{\pgfqpoint{0.997692in}{2.200407in}}{\pgfqpoint{0.993684in}{2.190731in}}{\pgfqpoint{0.993684in}{2.180644in}}%
\pgfpathcurveto{\pgfqpoint{0.993684in}{2.170557in}}{\pgfqpoint{0.997692in}{2.160881in}}{\pgfqpoint{1.004825in}{2.153748in}}%
\pgfpathcurveto{\pgfqpoint{1.011957in}{2.146615in}}{\pgfqpoint{1.021633in}{2.142608in}}{\pgfqpoint{1.031720in}{2.142608in}}%
\pgfpathclose%
\pgfusepath{stroke,fill}%
\end{pgfscope}%
\begin{pgfscope}%
\pgfpathrectangle{\pgfqpoint{0.800000in}{0.528000in}}{\pgfqpoint{4.960000in}{3.696000in}} %
\pgfusepath{clip}%
\pgfsetbuttcap%
\pgfsetroundjoin%
\definecolor{currentfill}{rgb}{0.121569,0.466667,0.705882}%
\pgfsetfillcolor{currentfill}%
\pgfsetlinewidth{1.003750pt}%
\definecolor{currentstroke}{rgb}{0.121569,0.466667,0.705882}%
\pgfsetstrokecolor{currentstroke}%
\pgfsetdash{}{0pt}%
\pgfpathmoveto{\pgfqpoint{1.356249in}{3.160048in}}%
\pgfpathcurveto{\pgfqpoint{1.366337in}{3.160048in}}{\pgfqpoint{1.376012in}{3.164056in}}{\pgfqpoint{1.383145in}{3.171189in}}%
\pgfpathcurveto{\pgfqpoint{1.390278in}{3.178321in}}{\pgfqpoint{1.394286in}{3.187997in}}{\pgfqpoint{1.394286in}{3.198084in}}%
\pgfpathcurveto{\pgfqpoint{1.394286in}{3.208172in}}{\pgfqpoint{1.390278in}{3.217847in}}{\pgfqpoint{1.383145in}{3.224980in}}%
\pgfpathcurveto{\pgfqpoint{1.376012in}{3.232113in}}{\pgfqpoint{1.366337in}{3.236121in}}{\pgfqpoint{1.356249in}{3.236121in}}%
\pgfpathcurveto{\pgfqpoint{1.346162in}{3.236121in}}{\pgfqpoint{1.336486in}{3.232113in}}{\pgfqpoint{1.329354in}{3.224980in}}%
\pgfpathcurveto{\pgfqpoint{1.322221in}{3.217847in}}{\pgfqpoint{1.318213in}{3.208172in}}{\pgfqpoint{1.318213in}{3.198084in}}%
\pgfpathcurveto{\pgfqpoint{1.318213in}{3.187997in}}{\pgfqpoint{1.322221in}{3.178321in}}{\pgfqpoint{1.329354in}{3.171189in}}%
\pgfpathcurveto{\pgfqpoint{1.336486in}{3.164056in}}{\pgfqpoint{1.346162in}{3.160048in}}{\pgfqpoint{1.356249in}{3.160048in}}%
\pgfpathclose%
\pgfusepath{stroke,fill}%
\end{pgfscope}%
\begin{pgfscope}%
\pgfpathrectangle{\pgfqpoint{0.800000in}{0.528000in}}{\pgfqpoint{4.960000in}{3.696000in}} %
\pgfusepath{clip}%
\pgfsetbuttcap%
\pgfsetroundjoin%
\definecolor{currentfill}{rgb}{0.121569,0.466667,0.705882}%
\pgfsetfillcolor{currentfill}%
\pgfsetlinewidth{1.003750pt}%
\definecolor{currentstroke}{rgb}{0.121569,0.466667,0.705882}%
\pgfsetstrokecolor{currentstroke}%
\pgfsetdash{}{0pt}%
\pgfpathmoveto{\pgfqpoint{1.528570in}{3.587024in}}%
\pgfpathcurveto{\pgfqpoint{1.538657in}{3.587024in}}{\pgfqpoint{1.548333in}{3.591032in}}{\pgfqpoint{1.555466in}{3.598165in}}%
\pgfpathcurveto{\pgfqpoint{1.562599in}{3.605297in}}{\pgfqpoint{1.566606in}{3.614973in}}{\pgfqpoint{1.566606in}{3.625060in}}%
\pgfpathcurveto{\pgfqpoint{1.566606in}{3.635148in}}{\pgfqpoint{1.562599in}{3.644823in}}{\pgfqpoint{1.555466in}{3.651956in}}%
\pgfpathcurveto{\pgfqpoint{1.548333in}{3.659089in}}{\pgfqpoint{1.538657in}{3.663097in}}{\pgfqpoint{1.528570in}{3.663097in}}%
\pgfpathcurveto{\pgfqpoint{1.518483in}{3.663097in}}{\pgfqpoint{1.508807in}{3.659089in}}{\pgfqpoint{1.501674in}{3.651956in}}%
\pgfpathcurveto{\pgfqpoint{1.494541in}{3.644823in}}{\pgfqpoint{1.490534in}{3.635148in}}{\pgfqpoint{1.490534in}{3.625060in}}%
\pgfpathcurveto{\pgfqpoint{1.490534in}{3.614973in}}{\pgfqpoint{1.494541in}{3.605297in}}{\pgfqpoint{1.501674in}{3.598165in}}%
\pgfpathcurveto{\pgfqpoint{1.508807in}{3.591032in}}{\pgfqpoint{1.518483in}{3.587024in}}{\pgfqpoint{1.528570in}{3.587024in}}%
\pgfpathclose%
\pgfusepath{stroke,fill}%
\end{pgfscope}%
\begin{pgfscope}%
\pgfpathrectangle{\pgfqpoint{0.800000in}{0.528000in}}{\pgfqpoint{4.960000in}{3.696000in}} %
\pgfusepath{clip}%
\pgfsetbuttcap%
\pgfsetroundjoin%
\definecolor{currentfill}{rgb}{0.121569,0.466667,0.705882}%
\pgfsetfillcolor{currentfill}%
\pgfsetlinewidth{1.003750pt}%
\definecolor{currentstroke}{rgb}{0.121569,0.466667,0.705882}%
\pgfsetstrokecolor{currentstroke}%
\pgfsetdash{}{0pt}%
\pgfpathmoveto{\pgfqpoint{2.428361in}{0.692038in}}%
\pgfpathcurveto{\pgfqpoint{2.438448in}{0.692038in}}{\pgfqpoint{2.448123in}{0.696046in}}{\pgfqpoint{2.455256in}{0.703179in}}%
\pgfpathcurveto{\pgfqpoint{2.462389in}{0.710312in}}{\pgfqpoint{2.466397in}{0.719987in}}{\pgfqpoint{2.466397in}{0.730075in}}%
\pgfpathcurveto{\pgfqpoint{2.466397in}{0.740162in}}{\pgfqpoint{2.462389in}{0.749838in}}{\pgfqpoint{2.455256in}{0.756970in}}%
\pgfpathcurveto{\pgfqpoint{2.448123in}{0.764103in}}{\pgfqpoint{2.438448in}{0.768111in}}{\pgfqpoint{2.428361in}{0.768111in}}%
\pgfpathcurveto{\pgfqpoint{2.418273in}{0.768111in}}{\pgfqpoint{2.408598in}{0.764103in}}{\pgfqpoint{2.401465in}{0.756970in}}%
\pgfpathcurveto{\pgfqpoint{2.394332in}{0.749838in}}{\pgfqpoint{2.390324in}{0.740162in}}{\pgfqpoint{2.390324in}{0.730075in}}%
\pgfpathcurveto{\pgfqpoint{2.390324in}{0.719987in}}{\pgfqpoint{2.394332in}{0.710312in}}{\pgfqpoint{2.401465in}{0.703179in}}%
\pgfpathcurveto{\pgfqpoint{2.408598in}{0.696046in}}{\pgfqpoint{2.418273in}{0.692038in}}{\pgfqpoint{2.428361in}{0.692038in}}%
\pgfpathclose%
\pgfusepath{stroke,fill}%
\end{pgfscope}%
\begin{pgfscope}%
\pgfpathrectangle{\pgfqpoint{0.800000in}{0.528000in}}{\pgfqpoint{4.960000in}{3.696000in}} %
\pgfusepath{clip}%
\pgfsetbuttcap%
\pgfsetroundjoin%
\definecolor{currentfill}{rgb}{0.121569,0.466667,0.705882}%
\pgfsetfillcolor{currentfill}%
\pgfsetlinewidth{1.003750pt}%
\definecolor{currentstroke}{rgb}{0.121569,0.466667,0.705882}%
\pgfsetstrokecolor{currentstroke}%
\pgfsetdash{}{0pt}%
\pgfpathmoveto{\pgfqpoint{2.524795in}{1.092081in}}%
\pgfpathcurveto{\pgfqpoint{2.534882in}{1.092081in}}{\pgfqpoint{2.544558in}{1.096089in}}{\pgfqpoint{2.551690in}{1.103222in}}%
\pgfpathcurveto{\pgfqpoint{2.558823in}{1.110354in}}{\pgfqpoint{2.562831in}{1.120030in}}{\pgfqpoint{2.562831in}{1.130117in}}%
\pgfpathcurveto{\pgfqpoint{2.562831in}{1.140205in}}{\pgfqpoint{2.558823in}{1.149880in}}{\pgfqpoint{2.551690in}{1.157013in}}%
\pgfpathcurveto{\pgfqpoint{2.544558in}{1.164146in}}{\pgfqpoint{2.534882in}{1.168154in}}{\pgfqpoint{2.524795in}{1.168154in}}%
\pgfpathcurveto{\pgfqpoint{2.514707in}{1.168154in}}{\pgfqpoint{2.505032in}{1.164146in}}{\pgfqpoint{2.497899in}{1.157013in}}%
\pgfpathcurveto{\pgfqpoint{2.490766in}{1.149880in}}{\pgfqpoint{2.486758in}{1.140205in}}{\pgfqpoint{2.486758in}{1.130117in}}%
\pgfpathcurveto{\pgfqpoint{2.486758in}{1.120030in}}{\pgfqpoint{2.490766in}{1.110354in}}{\pgfqpoint{2.497899in}{1.103222in}}%
\pgfpathcurveto{\pgfqpoint{2.505032in}{1.096089in}}{\pgfqpoint{2.514707in}{1.092081in}}{\pgfqpoint{2.524795in}{1.092081in}}%
\pgfpathclose%
\pgfusepath{stroke,fill}%
\end{pgfscope}%
\begin{pgfscope}%
\pgfpathrectangle{\pgfqpoint{0.800000in}{0.528000in}}{\pgfqpoint{4.960000in}{3.696000in}} %
\pgfusepath{clip}%
\pgfsetbuttcap%
\pgfsetroundjoin%
\definecolor{currentfill}{rgb}{0.121569,0.466667,0.705882}%
\pgfsetfillcolor{currentfill}%
\pgfsetlinewidth{1.003750pt}%
\definecolor{currentstroke}{rgb}{0.121569,0.466667,0.705882}%
\pgfsetstrokecolor{currentstroke}%
\pgfsetdash{}{0pt}%
\pgfpathmoveto{\pgfqpoint{2.371871in}{3.992571in}}%
\pgfpathcurveto{\pgfqpoint{2.381958in}{3.992571in}}{\pgfqpoint{2.391634in}{3.996578in}}{\pgfqpoint{2.398766in}{4.003711in}}%
\pgfpathcurveto{\pgfqpoint{2.405899in}{4.010844in}}{\pgfqpoint{2.409907in}{4.020520in}}{\pgfqpoint{2.409907in}{4.030607in}}%
\pgfpathcurveto{\pgfqpoint{2.409907in}{4.040694in}}{\pgfqpoint{2.405899in}{4.050370in}}{\pgfqpoint{2.398766in}{4.057503in}}%
\pgfpathcurveto{\pgfqpoint{2.391634in}{4.064636in}}{\pgfqpoint{2.381958in}{4.068643in}}{\pgfqpoint{2.371871in}{4.068643in}}%
\pgfpathcurveto{\pgfqpoint{2.361783in}{4.068643in}}{\pgfqpoint{2.352108in}{4.064636in}}{\pgfqpoint{2.344975in}{4.057503in}}%
\pgfpathcurveto{\pgfqpoint{2.337842in}{4.050370in}}{\pgfqpoint{2.333834in}{4.040694in}}{\pgfqpoint{2.333834in}{4.030607in}}%
\pgfpathcurveto{\pgfqpoint{2.333834in}{4.020520in}}{\pgfqpoint{2.337842in}{4.010844in}}{\pgfqpoint{2.344975in}{4.003711in}}%
\pgfpathcurveto{\pgfqpoint{2.352108in}{3.996578in}}{\pgfqpoint{2.361783in}{3.992571in}}{\pgfqpoint{2.371871in}{3.992571in}}%
\pgfpathclose%
\pgfusepath{stroke,fill}%
\end{pgfscope}%
\begin{pgfscope}%
\pgfpathrectangle{\pgfqpoint{0.800000in}{0.528000in}}{\pgfqpoint{4.960000in}{3.696000in}} %
\pgfusepath{clip}%
\pgfsetbuttcap%
\pgfsetroundjoin%
\definecolor{currentfill}{rgb}{0.121569,0.466667,0.705882}%
\pgfsetfillcolor{currentfill}%
\pgfsetlinewidth{1.003750pt}%
\definecolor{currentstroke}{rgb}{0.121569,0.466667,0.705882}%
\pgfsetstrokecolor{currentstroke}%
\pgfsetdash{}{0pt}%
\pgfpathmoveto{\pgfqpoint{3.317771in}{0.881055in}}%
\pgfpathcurveto{\pgfqpoint{3.327858in}{0.881055in}}{\pgfqpoint{3.337534in}{0.885062in}}{\pgfqpoint{3.344667in}{0.892195in}}%
\pgfpathcurveto{\pgfqpoint{3.351799in}{0.899328in}}{\pgfqpoint{3.355807in}{0.909004in}}{\pgfqpoint{3.355807in}{0.919091in}}%
\pgfpathcurveto{\pgfqpoint{3.355807in}{0.929178in}}{\pgfqpoint{3.351799in}{0.938854in}}{\pgfqpoint{3.344667in}{0.945987in}}%
\pgfpathcurveto{\pgfqpoint{3.337534in}{0.953120in}}{\pgfqpoint{3.327858in}{0.957127in}}{\pgfqpoint{3.317771in}{0.957127in}}%
\pgfpathcurveto{\pgfqpoint{3.307683in}{0.957127in}}{\pgfqpoint{3.298008in}{0.953120in}}{\pgfqpoint{3.290875in}{0.945987in}}%
\pgfpathcurveto{\pgfqpoint{3.283742in}{0.938854in}}{\pgfqpoint{3.279735in}{0.929178in}}{\pgfqpoint{3.279735in}{0.919091in}}%
\pgfpathcurveto{\pgfqpoint{3.279735in}{0.909004in}}{\pgfqpoint{3.283742in}{0.899328in}}{\pgfqpoint{3.290875in}{0.892195in}}%
\pgfpathcurveto{\pgfqpoint{3.298008in}{0.885062in}}{\pgfqpoint{3.307683in}{0.881055in}}{\pgfqpoint{3.317771in}{0.881055in}}%
\pgfpathclose%
\pgfusepath{stroke,fill}%
\end{pgfscope}%
\begin{pgfscope}%
\pgfpathrectangle{\pgfqpoint{0.800000in}{0.528000in}}{\pgfqpoint{4.960000in}{3.696000in}} %
\pgfusepath{clip}%
\pgfsetbuttcap%
\pgfsetroundjoin%
\definecolor{currentfill}{rgb}{0.121569,0.466667,0.705882}%
\pgfsetfillcolor{currentfill}%
\pgfsetlinewidth{1.003750pt}%
\definecolor{currentstroke}{rgb}{0.121569,0.466667,0.705882}%
\pgfsetstrokecolor{currentstroke}%
\pgfsetdash{}{0pt}%
\pgfpathmoveto{\pgfqpoint{3.188228in}{3.959324in}}%
\pgfpathcurveto{\pgfqpoint{3.198316in}{3.959324in}}{\pgfqpoint{3.207991in}{3.963332in}}{\pgfqpoint{3.215124in}{3.970465in}}%
\pgfpathcurveto{\pgfqpoint{3.222257in}{3.977598in}}{\pgfqpoint{3.226265in}{3.987273in}}{\pgfqpoint{3.226265in}{3.997361in}}%
\pgfpathcurveto{\pgfqpoint{3.226265in}{4.007448in}}{\pgfqpoint{3.222257in}{4.017123in}}{\pgfqpoint{3.215124in}{4.024256in}}%
\pgfpathcurveto{\pgfqpoint{3.207991in}{4.031389in}}{\pgfqpoint{3.198316in}{4.035397in}}{\pgfqpoint{3.188228in}{4.035397in}}%
\pgfpathcurveto{\pgfqpoint{3.178141in}{4.035397in}}{\pgfqpoint{3.168465in}{4.031389in}}{\pgfqpoint{3.161333in}{4.024256in}}%
\pgfpathcurveto{\pgfqpoint{3.154200in}{4.017123in}}{\pgfqpoint{3.150192in}{4.007448in}}{\pgfqpoint{3.150192in}{3.997361in}}%
\pgfpathcurveto{\pgfqpoint{3.150192in}{3.987273in}}{\pgfqpoint{3.154200in}{3.977598in}}{\pgfqpoint{3.161333in}{3.970465in}}%
\pgfpathcurveto{\pgfqpoint{3.168465in}{3.963332in}}{\pgfqpoint{3.178141in}{3.959324in}}{\pgfqpoint{3.188228in}{3.959324in}}%
\pgfpathclose%
\pgfusepath{stroke,fill}%
\end{pgfscope}%
\begin{pgfscope}%
\pgfpathrectangle{\pgfqpoint{0.800000in}{0.528000in}}{\pgfqpoint{4.960000in}{3.696000in}} %
\pgfusepath{clip}%
\pgfsetbuttcap%
\pgfsetroundjoin%
\definecolor{currentfill}{rgb}{0.121569,0.466667,0.705882}%
\pgfsetfillcolor{currentfill}%
\pgfsetlinewidth{1.003750pt}%
\definecolor{currentstroke}{rgb}{0.121569,0.466667,0.705882}%
\pgfsetstrokecolor{currentstroke}%
\pgfsetdash{}{0pt}%
\pgfpathmoveto{\pgfqpoint{4.207830in}{0.683357in}}%
\pgfpathcurveto{\pgfqpoint{4.217917in}{0.683357in}}{\pgfqpoint{4.227593in}{0.687364in}}{\pgfqpoint{4.234726in}{0.694497in}}%
\pgfpathcurveto{\pgfqpoint{4.241858in}{0.701630in}}{\pgfqpoint{4.245866in}{0.711306in}}{\pgfqpoint{4.245866in}{0.721393in}}%
\pgfpathcurveto{\pgfqpoint{4.245866in}{0.731480in}}{\pgfqpoint{4.241858in}{0.741156in}}{\pgfqpoint{4.234726in}{0.748289in}}%
\pgfpathcurveto{\pgfqpoint{4.227593in}{0.755422in}}{\pgfqpoint{4.217917in}{0.759429in}}{\pgfqpoint{4.207830in}{0.759429in}}%
\pgfpathcurveto{\pgfqpoint{4.197743in}{0.759429in}}{\pgfqpoint{4.188067in}{0.755422in}}{\pgfqpoint{4.180934in}{0.748289in}}%
\pgfpathcurveto{\pgfqpoint{4.173801in}{0.741156in}}{\pgfqpoint{4.169794in}{0.731480in}}{\pgfqpoint{4.169794in}{0.721393in}}%
\pgfpathcurveto{\pgfqpoint{4.169794in}{0.711306in}}{\pgfqpoint{4.173801in}{0.701630in}}{\pgfqpoint{4.180934in}{0.694497in}}%
\pgfpathcurveto{\pgfqpoint{4.188067in}{0.687364in}}{\pgfqpoint{4.197743in}{0.683357in}}{\pgfqpoint{4.207830in}{0.683357in}}%
\pgfpathclose%
\pgfusepath{stroke,fill}%
\end{pgfscope}%
\begin{pgfscope}%
\pgfpathrectangle{\pgfqpoint{0.800000in}{0.528000in}}{\pgfqpoint{4.960000in}{3.696000in}} %
\pgfusepath{clip}%
\pgfsetbuttcap%
\pgfsetroundjoin%
\definecolor{currentfill}{rgb}{0.121569,0.466667,0.705882}%
\pgfsetfillcolor{currentfill}%
\pgfsetlinewidth{1.003750pt}%
\definecolor{currentstroke}{rgb}{0.121569,0.466667,0.705882}%
\pgfsetstrokecolor{currentstroke}%
\pgfsetdash{}{0pt}%
\pgfpathmoveto{\pgfqpoint{4.827101in}{1.237602in}}%
\pgfpathcurveto{\pgfqpoint{4.837189in}{1.237602in}}{\pgfqpoint{4.846864in}{1.241610in}}{\pgfqpoint{4.853997in}{1.248743in}}%
\pgfpathcurveto{\pgfqpoint{4.861130in}{1.255875in}}{\pgfqpoint{4.865138in}{1.265551in}}{\pgfqpoint{4.865138in}{1.275638in}}%
\pgfpathcurveto{\pgfqpoint{4.865138in}{1.285726in}}{\pgfqpoint{4.861130in}{1.295401in}}{\pgfqpoint{4.853997in}{1.302534in}}%
\pgfpathcurveto{\pgfqpoint{4.846864in}{1.309667in}}{\pgfqpoint{4.837189in}{1.313675in}}{\pgfqpoint{4.827101in}{1.313675in}}%
\pgfpathcurveto{\pgfqpoint{4.817014in}{1.313675in}}{\pgfqpoint{4.807338in}{1.309667in}}{\pgfqpoint{4.800206in}{1.302534in}}%
\pgfpathcurveto{\pgfqpoint{4.793073in}{1.295401in}}{\pgfqpoint{4.789065in}{1.285726in}}{\pgfqpoint{4.789065in}{1.275638in}}%
\pgfpathcurveto{\pgfqpoint{4.789065in}{1.265551in}}{\pgfqpoint{4.793073in}{1.255875in}}{\pgfqpoint{4.800206in}{1.248743in}}%
\pgfpathcurveto{\pgfqpoint{4.807338in}{1.241610in}}{\pgfqpoint{4.817014in}{1.237602in}}{\pgfqpoint{4.827101in}{1.237602in}}%
\pgfpathclose%
\pgfusepath{stroke,fill}%
\end{pgfscope}%
\begin{pgfscope}%
\pgfpathrectangle{\pgfqpoint{0.800000in}{0.528000in}}{\pgfqpoint{4.960000in}{3.696000in}} %
\pgfusepath{clip}%
\pgfsetbuttcap%
\pgfsetroundjoin%
\definecolor{currentfill}{rgb}{0.121569,0.466667,0.705882}%
\pgfsetfillcolor{currentfill}%
\pgfsetlinewidth{1.003750pt}%
\definecolor{currentstroke}{rgb}{0.121569,0.466667,0.705882}%
\pgfsetstrokecolor{currentstroke}%
\pgfsetdash{}{0pt}%
\pgfpathmoveto{\pgfqpoint{4.468060in}{3.772899in}}%
\pgfpathcurveto{\pgfqpoint{4.478147in}{3.772899in}}{\pgfqpoint{4.487822in}{3.776907in}}{\pgfqpoint{4.494955in}{3.784039in}}%
\pgfpathcurveto{\pgfqpoint{4.502088in}{3.791172in}}{\pgfqpoint{4.506096in}{3.800848in}}{\pgfqpoint{4.506096in}{3.810935in}}%
\pgfpathcurveto{\pgfqpoint{4.506096in}{3.821022in}}{\pgfqpoint{4.502088in}{3.830698in}}{\pgfqpoint{4.494955in}{3.837831in}}%
\pgfpathcurveto{\pgfqpoint{4.487822in}{3.844964in}}{\pgfqpoint{4.478147in}{3.848971in}}{\pgfqpoint{4.468060in}{3.848971in}}%
\pgfpathcurveto{\pgfqpoint{4.457972in}{3.848971in}}{\pgfqpoint{4.448297in}{3.844964in}}{\pgfqpoint{4.441164in}{3.837831in}}%
\pgfpathcurveto{\pgfqpoint{4.434031in}{3.830698in}}{\pgfqpoint{4.430023in}{3.821022in}}{\pgfqpoint{4.430023in}{3.810935in}}%
\pgfpathcurveto{\pgfqpoint{4.430023in}{3.800848in}}{\pgfqpoint{4.434031in}{3.791172in}}{\pgfqpoint{4.441164in}{3.784039in}}%
\pgfpathcurveto{\pgfqpoint{4.448297in}{3.776907in}}{\pgfqpoint{4.457972in}{3.772899in}}{\pgfqpoint{4.468060in}{3.772899in}}%
\pgfpathclose%
\pgfusepath{stroke,fill}%
\end{pgfscope}%
\begin{pgfscope}%
\pgfpathrectangle{\pgfqpoint{0.800000in}{0.528000in}}{\pgfqpoint{4.960000in}{3.696000in}} %
\pgfusepath{clip}%
\pgfsetbuttcap%
\pgfsetroundjoin%
\definecolor{currentfill}{rgb}{0.121569,0.466667,0.705882}%
\pgfsetfillcolor{currentfill}%
\pgfsetlinewidth{1.003750pt}%
\definecolor{currentstroke}{rgb}{0.121569,0.466667,0.705882}%
\pgfsetstrokecolor{currentstroke}%
\pgfsetdash{}{0pt}%
\pgfpathmoveto{\pgfqpoint{5.296265in}{1.544799in}}%
\pgfpathcurveto{\pgfqpoint{5.306352in}{1.544799in}}{\pgfqpoint{5.316027in}{1.548807in}}{\pgfqpoint{5.323160in}{1.555940in}}%
\pgfpathcurveto{\pgfqpoint{5.330293in}{1.563073in}}{\pgfqpoint{5.334301in}{1.572748in}}{\pgfqpoint{5.334301in}{1.582836in}}%
\pgfpathcurveto{\pgfqpoint{5.334301in}{1.592923in}}{\pgfqpoint{5.330293in}{1.602599in}}{\pgfqpoint{5.323160in}{1.609731in}}%
\pgfpathcurveto{\pgfqpoint{5.316027in}{1.616864in}}{\pgfqpoint{5.306352in}{1.620872in}}{\pgfqpoint{5.296265in}{1.620872in}}%
\pgfpathcurveto{\pgfqpoint{5.286177in}{1.620872in}}{\pgfqpoint{5.276502in}{1.616864in}}{\pgfqpoint{5.269369in}{1.609731in}}%
\pgfpathcurveto{\pgfqpoint{5.262236in}{1.602599in}}{\pgfqpoint{5.258228in}{1.592923in}}{\pgfqpoint{5.258228in}{1.582836in}}%
\pgfpathcurveto{\pgfqpoint{5.258228in}{1.572748in}}{\pgfqpoint{5.262236in}{1.563073in}}{\pgfqpoint{5.269369in}{1.555940in}}%
\pgfpathcurveto{\pgfqpoint{5.276502in}{1.548807in}}{\pgfqpoint{5.286177in}{1.544799in}}{\pgfqpoint{5.296265in}{1.544799in}}%
\pgfpathclose%
\pgfusepath{stroke,fill}%
\end{pgfscope}%
\begin{pgfscope}%
\pgfpathrectangle{\pgfqpoint{0.800000in}{0.528000in}}{\pgfqpoint{4.960000in}{3.696000in}} %
\pgfusepath{clip}%
\pgfsetbuttcap%
\pgfsetroundjoin%
\definecolor{currentfill}{rgb}{0.121569,0.466667,0.705882}%
\pgfsetfillcolor{currentfill}%
\pgfsetlinewidth{1.003750pt}%
\definecolor{currentstroke}{rgb}{0.121569,0.466667,0.705882}%
\pgfsetstrokecolor{currentstroke}%
\pgfsetdash{}{0pt}%
\pgfpathmoveto{\pgfqpoint{5.528280in}{2.502909in}}%
\pgfpathcurveto{\pgfqpoint{5.538367in}{2.502909in}}{\pgfqpoint{5.548043in}{2.506917in}}{\pgfqpoint{5.555175in}{2.514049in}}%
\pgfpathcurveto{\pgfqpoint{5.562308in}{2.521182in}}{\pgfqpoint{5.566316in}{2.530858in}}{\pgfqpoint{5.566316in}{2.540945in}}%
\pgfpathcurveto{\pgfqpoint{5.566316in}{2.551033in}}{\pgfqpoint{5.562308in}{2.560708in}}{\pgfqpoint{5.555175in}{2.567841in}}%
\pgfpathcurveto{\pgfqpoint{5.548043in}{2.574974in}}{\pgfqpoint{5.538367in}{2.578982in}}{\pgfqpoint{5.528280in}{2.578982in}}%
\pgfpathcurveto{\pgfqpoint{5.518192in}{2.578982in}}{\pgfqpoint{5.508517in}{2.574974in}}{\pgfqpoint{5.501384in}{2.567841in}}%
\pgfpathcurveto{\pgfqpoint{5.494251in}{2.560708in}}{\pgfqpoint{5.490243in}{2.551033in}}{\pgfqpoint{5.490243in}{2.540945in}}%
\pgfpathcurveto{\pgfqpoint{5.490243in}{2.530858in}}{\pgfqpoint{5.494251in}{2.521182in}}{\pgfqpoint{5.501384in}{2.514049in}}%
\pgfpathcurveto{\pgfqpoint{5.508517in}{2.506917in}}{\pgfqpoint{5.518192in}{2.502909in}}{\pgfqpoint{5.528280in}{2.502909in}}%
\pgfpathclose%
\pgfusepath{stroke,fill}%
\end{pgfscope}%
\begin{pgfscope}%
\pgfpathrectangle{\pgfqpoint{0.800000in}{0.528000in}}{\pgfqpoint{4.960000in}{3.696000in}} %
\pgfusepath{clip}%
\pgfsetbuttcap%
\pgfsetroundjoin%
\definecolor{currentfill}{rgb}{0.121569,0.466667,0.705882}%
\pgfsetfillcolor{currentfill}%
\pgfsetlinewidth{1.003750pt}%
\definecolor{currentstroke}{rgb}{0.121569,0.466667,0.705882}%
\pgfsetstrokecolor{currentstroke}%
\pgfsetdash{}{0pt}%
\pgfpathmoveto{\pgfqpoint{5.328319in}{3.240906in}}%
\pgfpathcurveto{\pgfqpoint{5.338407in}{3.240906in}}{\pgfqpoint{5.348082in}{3.244914in}}{\pgfqpoint{5.355215in}{3.252047in}}%
\pgfpathcurveto{\pgfqpoint{5.362348in}{3.259180in}}{\pgfqpoint{5.366356in}{3.268855in}}{\pgfqpoint{5.366356in}{3.278943in}}%
\pgfpathcurveto{\pgfqpoint{5.366356in}{3.289030in}}{\pgfqpoint{5.362348in}{3.298706in}}{\pgfqpoint{5.355215in}{3.305838in}}%
\pgfpathcurveto{\pgfqpoint{5.348082in}{3.312971in}}{\pgfqpoint{5.338407in}{3.316979in}}{\pgfqpoint{5.328319in}{3.316979in}}%
\pgfpathcurveto{\pgfqpoint{5.318232in}{3.316979in}}{\pgfqpoint{5.308557in}{3.312971in}}{\pgfqpoint{5.301424in}{3.305838in}}%
\pgfpathcurveto{\pgfqpoint{5.294291in}{3.298706in}}{\pgfqpoint{5.290283in}{3.289030in}}{\pgfqpoint{5.290283in}{3.278943in}}%
\pgfpathcurveto{\pgfqpoint{5.290283in}{3.268855in}}{\pgfqpoint{5.294291in}{3.259180in}}{\pgfqpoint{5.301424in}{3.252047in}}%
\pgfpathcurveto{\pgfqpoint{5.308557in}{3.244914in}}{\pgfqpoint{5.318232in}{3.240906in}}{\pgfqpoint{5.328319in}{3.240906in}}%
\pgfpathclose%
\pgfusepath{stroke,fill}%
\end{pgfscope}%
\begin{pgfscope}%
\pgfpathrectangle{\pgfqpoint{0.800000in}{0.528000in}}{\pgfqpoint{4.960000in}{3.696000in}} %
\pgfusepath{clip}%
\pgfsetbuttcap%
\pgfsetroundjoin%
\definecolor{currentfill}{rgb}{0.121569,0.466667,0.705882}%
\pgfsetfillcolor{currentfill}%
\pgfsetlinewidth{1.003750pt}%
\definecolor{currentstroke}{rgb}{0.121569,0.466667,0.705882}%
\pgfsetstrokecolor{currentstroke}%
\pgfsetdash{}{0pt}%
\pgfpathmoveto{\pgfqpoint{5.176855in}{3.602912in}}%
\pgfpathcurveto{\pgfqpoint{5.186942in}{3.602912in}}{\pgfqpoint{5.196618in}{3.606919in}}{\pgfqpoint{5.203751in}{3.614052in}}%
\pgfpathcurveto{\pgfqpoint{5.210884in}{3.621185in}}{\pgfqpoint{5.214891in}{3.630860in}}{\pgfqpoint{5.214891in}{3.640948in}}%
\pgfpathcurveto{\pgfqpoint{5.214891in}{3.651035in}}{\pgfqpoint{5.210884in}{3.660711in}}{\pgfqpoint{5.203751in}{3.667844in}}%
\pgfpathcurveto{\pgfqpoint{5.196618in}{3.674976in}}{\pgfqpoint{5.186942in}{3.678984in}}{\pgfqpoint{5.176855in}{3.678984in}}%
\pgfpathcurveto{\pgfqpoint{5.166768in}{3.678984in}}{\pgfqpoint{5.157092in}{3.674976in}}{\pgfqpoint{5.149959in}{3.667844in}}%
\pgfpathcurveto{\pgfqpoint{5.142827in}{3.660711in}}{\pgfqpoint{5.138819in}{3.651035in}}{\pgfqpoint{5.138819in}{3.640948in}}%
\pgfpathcurveto{\pgfqpoint{5.138819in}{3.630860in}}{\pgfqpoint{5.142827in}{3.621185in}}{\pgfqpoint{5.149959in}{3.614052in}}%
\pgfpathcurveto{\pgfqpoint{5.157092in}{3.606919in}}{\pgfqpoint{5.166768in}{3.602912in}}{\pgfqpoint{5.176855in}{3.602912in}}%
\pgfpathclose%
\pgfusepath{stroke,fill}%
\end{pgfscope}%
\begin{pgfscope}%
\pgfpathrectangle{\pgfqpoint{0.800000in}{0.528000in}}{\pgfqpoint{4.960000in}{3.696000in}} %
\pgfusepath{clip}%
\pgfsetbuttcap%
\pgfsetroundjoin%
\definecolor{currentfill}{rgb}{0.121569,0.466667,0.705882}%
\pgfsetfillcolor{currentfill}%
\pgfsetlinewidth{1.003750pt}%
\definecolor{currentstroke}{rgb}{0.121569,0.466667,0.705882}%
\pgfsetstrokecolor{currentstroke}%
\pgfsetdash{}{0pt}%
\pgfpathmoveto{\pgfqpoint{1.673483in}{1.012308in}}%
\pgfpathcurveto{\pgfqpoint{1.683570in}{1.012308in}}{\pgfqpoint{1.693246in}{1.016316in}}{\pgfqpoint{1.700378in}{1.023448in}}%
\pgfpathcurveto{\pgfqpoint{1.707511in}{1.030581in}}{\pgfqpoint{1.711519in}{1.040257in}}{\pgfqpoint{1.711519in}{1.050344in}}%
\pgfpathcurveto{\pgfqpoint{1.711519in}{1.060431in}}{\pgfqpoint{1.707511in}{1.070107in}}{\pgfqpoint{1.700378in}{1.077240in}}%
\pgfpathcurveto{\pgfqpoint{1.693246in}{1.084373in}}{\pgfqpoint{1.683570in}{1.088380in}}{\pgfqpoint{1.673483in}{1.088380in}}%
\pgfpathcurveto{\pgfqpoint{1.663395in}{1.088380in}}{\pgfqpoint{1.653720in}{1.084373in}}{\pgfqpoint{1.646587in}{1.077240in}}%
\pgfpathcurveto{\pgfqpoint{1.639454in}{1.070107in}}{\pgfqpoint{1.635446in}{1.060431in}}{\pgfqpoint{1.635446in}{1.050344in}}%
\pgfpathcurveto{\pgfqpoint{1.635446in}{1.040257in}}{\pgfqpoint{1.639454in}{1.030581in}}{\pgfqpoint{1.646587in}{1.023448in}}%
\pgfpathcurveto{\pgfqpoint{1.653720in}{1.016316in}}{\pgfqpoint{1.663395in}{1.012308in}}{\pgfqpoint{1.673483in}{1.012308in}}%
\pgfpathclose%
\pgfusepath{stroke,fill}%
\end{pgfscope}%
\begin{pgfscope}%
\pgfpathrectangle{\pgfqpoint{0.800000in}{0.528000in}}{\pgfqpoint{4.960000in}{3.696000in}} %
\pgfusepath{clip}%
\pgfsetbuttcap%
\pgfsetroundjoin%
\definecolor{currentfill}{rgb}{0.121569,0.466667,0.705882}%
\pgfsetfillcolor{currentfill}%
\pgfsetlinewidth{1.003750pt}%
\definecolor{currentstroke}{rgb}{0.121569,0.466667,0.705882}%
\pgfsetstrokecolor{currentstroke}%
\pgfsetdash{}{0pt}%
\pgfpathmoveto{\pgfqpoint{1.342173in}{1.561903in}}%
\pgfpathcurveto{\pgfqpoint{1.352261in}{1.561903in}}{\pgfqpoint{1.361936in}{1.565911in}}{\pgfqpoint{1.369069in}{1.573044in}}%
\pgfpathcurveto{\pgfqpoint{1.376202in}{1.580177in}}{\pgfqpoint{1.380210in}{1.589852in}}{\pgfqpoint{1.380210in}{1.599940in}}%
\pgfpathcurveto{\pgfqpoint{1.380210in}{1.610027in}}{\pgfqpoint{1.376202in}{1.619703in}}{\pgfqpoint{1.369069in}{1.626835in}}%
\pgfpathcurveto{\pgfqpoint{1.361936in}{1.633968in}}{\pgfqpoint{1.352261in}{1.637976in}}{\pgfqpoint{1.342173in}{1.637976in}}%
\pgfpathcurveto{\pgfqpoint{1.332086in}{1.637976in}}{\pgfqpoint{1.322411in}{1.633968in}}{\pgfqpoint{1.315278in}{1.626835in}}%
\pgfpathcurveto{\pgfqpoint{1.308145in}{1.619703in}}{\pgfqpoint{1.304137in}{1.610027in}}{\pgfqpoint{1.304137in}{1.599940in}}%
\pgfpathcurveto{\pgfqpoint{1.304137in}{1.589852in}}{\pgfqpoint{1.308145in}{1.580177in}}{\pgfqpoint{1.315278in}{1.573044in}}%
\pgfpathcurveto{\pgfqpoint{1.322411in}{1.565911in}}{\pgfqpoint{1.332086in}{1.561903in}}{\pgfqpoint{1.342173in}{1.561903in}}%
\pgfpathclose%
\pgfusepath{stroke,fill}%
\end{pgfscope}%
\begin{pgfscope}%
\pgfpathrectangle{\pgfqpoint{0.800000in}{0.528000in}}{\pgfqpoint{4.960000in}{3.696000in}} %
\pgfusepath{clip}%
\pgfsetbuttcap%
\pgfsetroundjoin%
\definecolor{currentfill}{rgb}{0.121569,0.466667,0.705882}%
\pgfsetfillcolor{currentfill}%
\pgfsetlinewidth{1.003750pt}%
\definecolor{currentstroke}{rgb}{0.121569,0.466667,0.705882}%
\pgfsetstrokecolor{currentstroke}%
\pgfsetdash{}{0pt}%
\pgfpathmoveto{\pgfqpoint{1.031720in}{2.142608in}}%
\pgfpathcurveto{\pgfqpoint{1.041808in}{2.142608in}}{\pgfqpoint{1.051483in}{2.146615in}}{\pgfqpoint{1.058616in}{2.153748in}}%
\pgfpathcurveto{\pgfqpoint{1.065749in}{2.160881in}}{\pgfqpoint{1.069757in}{2.170557in}}{\pgfqpoint{1.069757in}{2.180644in}}%
\pgfpathcurveto{\pgfqpoint{1.069757in}{2.190731in}}{\pgfqpoint{1.065749in}{2.200407in}}{\pgfqpoint{1.058616in}{2.207540in}}%
\pgfpathcurveto{\pgfqpoint{1.051483in}{2.214673in}}{\pgfqpoint{1.041808in}{2.218680in}}{\pgfqpoint{1.031720in}{2.218680in}}%
\pgfpathcurveto{\pgfqpoint{1.021633in}{2.218680in}}{\pgfqpoint{1.011957in}{2.214673in}}{\pgfqpoint{1.004825in}{2.207540in}}%
\pgfpathcurveto{\pgfqpoint{0.997692in}{2.200407in}}{\pgfqpoint{0.993684in}{2.190731in}}{\pgfqpoint{0.993684in}{2.180644in}}%
\pgfpathcurveto{\pgfqpoint{0.993684in}{2.170557in}}{\pgfqpoint{0.997692in}{2.160881in}}{\pgfqpoint{1.004825in}{2.153748in}}%
\pgfpathcurveto{\pgfqpoint{1.011957in}{2.146615in}}{\pgfqpoint{1.021633in}{2.142608in}}{\pgfqpoint{1.031720in}{2.142608in}}%
\pgfpathclose%
\pgfusepath{stroke,fill}%
\end{pgfscope}%
\begin{pgfscope}%
\pgfpathrectangle{\pgfqpoint{0.800000in}{0.528000in}}{\pgfqpoint{4.960000in}{3.696000in}} %
\pgfusepath{clip}%
\pgfsetbuttcap%
\pgfsetroundjoin%
\definecolor{currentfill}{rgb}{0.121569,0.466667,0.705882}%
\pgfsetfillcolor{currentfill}%
\pgfsetlinewidth{1.003750pt}%
\definecolor{currentstroke}{rgb}{0.121569,0.466667,0.705882}%
\pgfsetstrokecolor{currentstroke}%
\pgfsetdash{}{0pt}%
\pgfpathmoveto{\pgfqpoint{1.356249in}{3.160048in}}%
\pgfpathcurveto{\pgfqpoint{1.366337in}{3.160048in}}{\pgfqpoint{1.376012in}{3.164056in}}{\pgfqpoint{1.383145in}{3.171189in}}%
\pgfpathcurveto{\pgfqpoint{1.390278in}{3.178321in}}{\pgfqpoint{1.394286in}{3.187997in}}{\pgfqpoint{1.394286in}{3.198084in}}%
\pgfpathcurveto{\pgfqpoint{1.394286in}{3.208172in}}{\pgfqpoint{1.390278in}{3.217847in}}{\pgfqpoint{1.383145in}{3.224980in}}%
\pgfpathcurveto{\pgfqpoint{1.376012in}{3.232113in}}{\pgfqpoint{1.366337in}{3.236121in}}{\pgfqpoint{1.356249in}{3.236121in}}%
\pgfpathcurveto{\pgfqpoint{1.346162in}{3.236121in}}{\pgfqpoint{1.336486in}{3.232113in}}{\pgfqpoint{1.329354in}{3.224980in}}%
\pgfpathcurveto{\pgfqpoint{1.322221in}{3.217847in}}{\pgfqpoint{1.318213in}{3.208172in}}{\pgfqpoint{1.318213in}{3.198084in}}%
\pgfpathcurveto{\pgfqpoint{1.318213in}{3.187997in}}{\pgfqpoint{1.322221in}{3.178321in}}{\pgfqpoint{1.329354in}{3.171189in}}%
\pgfpathcurveto{\pgfqpoint{1.336486in}{3.164056in}}{\pgfqpoint{1.346162in}{3.160048in}}{\pgfqpoint{1.356249in}{3.160048in}}%
\pgfpathclose%
\pgfusepath{stroke,fill}%
\end{pgfscope}%
\begin{pgfscope}%
\pgfpathrectangle{\pgfqpoint{0.800000in}{0.528000in}}{\pgfqpoint{4.960000in}{3.696000in}} %
\pgfusepath{clip}%
\pgfsetbuttcap%
\pgfsetroundjoin%
\definecolor{currentfill}{rgb}{0.121569,0.466667,0.705882}%
\pgfsetfillcolor{currentfill}%
\pgfsetlinewidth{1.003750pt}%
\definecolor{currentstroke}{rgb}{0.121569,0.466667,0.705882}%
\pgfsetstrokecolor{currentstroke}%
\pgfsetdash{}{0pt}%
\pgfpathmoveto{\pgfqpoint{1.528570in}{3.587024in}}%
\pgfpathcurveto{\pgfqpoint{1.538657in}{3.587024in}}{\pgfqpoint{1.548333in}{3.591032in}}{\pgfqpoint{1.555466in}{3.598165in}}%
\pgfpathcurveto{\pgfqpoint{1.562599in}{3.605297in}}{\pgfqpoint{1.566606in}{3.614973in}}{\pgfqpoint{1.566606in}{3.625060in}}%
\pgfpathcurveto{\pgfqpoint{1.566606in}{3.635148in}}{\pgfqpoint{1.562599in}{3.644823in}}{\pgfqpoint{1.555466in}{3.651956in}}%
\pgfpathcurveto{\pgfqpoint{1.548333in}{3.659089in}}{\pgfqpoint{1.538657in}{3.663097in}}{\pgfqpoint{1.528570in}{3.663097in}}%
\pgfpathcurveto{\pgfqpoint{1.518483in}{3.663097in}}{\pgfqpoint{1.508807in}{3.659089in}}{\pgfqpoint{1.501674in}{3.651956in}}%
\pgfpathcurveto{\pgfqpoint{1.494541in}{3.644823in}}{\pgfqpoint{1.490534in}{3.635148in}}{\pgfqpoint{1.490534in}{3.625060in}}%
\pgfpathcurveto{\pgfqpoint{1.490534in}{3.614973in}}{\pgfqpoint{1.494541in}{3.605297in}}{\pgfqpoint{1.501674in}{3.598165in}}%
\pgfpathcurveto{\pgfqpoint{1.508807in}{3.591032in}}{\pgfqpoint{1.518483in}{3.587024in}}{\pgfqpoint{1.528570in}{3.587024in}}%
\pgfpathclose%
\pgfusepath{stroke,fill}%
\end{pgfscope}%
\begin{pgfscope}%
\pgfpathrectangle{\pgfqpoint{0.800000in}{0.528000in}}{\pgfqpoint{4.960000in}{3.696000in}} %
\pgfusepath{clip}%
\pgfsetbuttcap%
\pgfsetroundjoin%
\definecolor{currentfill}{rgb}{0.121569,0.466667,0.705882}%
\pgfsetfillcolor{currentfill}%
\pgfsetlinewidth{1.003750pt}%
\definecolor{currentstroke}{rgb}{0.121569,0.466667,0.705882}%
\pgfsetstrokecolor{currentstroke}%
\pgfsetdash{}{0pt}%
\pgfpathmoveto{\pgfqpoint{2.428361in}{0.692038in}}%
\pgfpathcurveto{\pgfqpoint{2.438448in}{0.692038in}}{\pgfqpoint{2.448123in}{0.696046in}}{\pgfqpoint{2.455256in}{0.703179in}}%
\pgfpathcurveto{\pgfqpoint{2.462389in}{0.710312in}}{\pgfqpoint{2.466397in}{0.719987in}}{\pgfqpoint{2.466397in}{0.730075in}}%
\pgfpathcurveto{\pgfqpoint{2.466397in}{0.740162in}}{\pgfqpoint{2.462389in}{0.749838in}}{\pgfqpoint{2.455256in}{0.756970in}}%
\pgfpathcurveto{\pgfqpoint{2.448123in}{0.764103in}}{\pgfqpoint{2.438448in}{0.768111in}}{\pgfqpoint{2.428361in}{0.768111in}}%
\pgfpathcurveto{\pgfqpoint{2.418273in}{0.768111in}}{\pgfqpoint{2.408598in}{0.764103in}}{\pgfqpoint{2.401465in}{0.756970in}}%
\pgfpathcurveto{\pgfqpoint{2.394332in}{0.749838in}}{\pgfqpoint{2.390324in}{0.740162in}}{\pgfqpoint{2.390324in}{0.730075in}}%
\pgfpathcurveto{\pgfqpoint{2.390324in}{0.719987in}}{\pgfqpoint{2.394332in}{0.710312in}}{\pgfqpoint{2.401465in}{0.703179in}}%
\pgfpathcurveto{\pgfqpoint{2.408598in}{0.696046in}}{\pgfqpoint{2.418273in}{0.692038in}}{\pgfqpoint{2.428361in}{0.692038in}}%
\pgfpathclose%
\pgfusepath{stroke,fill}%
\end{pgfscope}%
\begin{pgfscope}%
\pgfpathrectangle{\pgfqpoint{0.800000in}{0.528000in}}{\pgfqpoint{4.960000in}{3.696000in}} %
\pgfusepath{clip}%
\pgfsetbuttcap%
\pgfsetroundjoin%
\definecolor{currentfill}{rgb}{0.121569,0.466667,0.705882}%
\pgfsetfillcolor{currentfill}%
\pgfsetlinewidth{1.003750pt}%
\definecolor{currentstroke}{rgb}{0.121569,0.466667,0.705882}%
\pgfsetstrokecolor{currentstroke}%
\pgfsetdash{}{0pt}%
\pgfpathmoveto{\pgfqpoint{2.524795in}{1.092081in}}%
\pgfpathcurveto{\pgfqpoint{2.534882in}{1.092081in}}{\pgfqpoint{2.544558in}{1.096089in}}{\pgfqpoint{2.551690in}{1.103222in}}%
\pgfpathcurveto{\pgfqpoint{2.558823in}{1.110354in}}{\pgfqpoint{2.562831in}{1.120030in}}{\pgfqpoint{2.562831in}{1.130117in}}%
\pgfpathcurveto{\pgfqpoint{2.562831in}{1.140205in}}{\pgfqpoint{2.558823in}{1.149880in}}{\pgfqpoint{2.551690in}{1.157013in}}%
\pgfpathcurveto{\pgfqpoint{2.544558in}{1.164146in}}{\pgfqpoint{2.534882in}{1.168154in}}{\pgfqpoint{2.524795in}{1.168154in}}%
\pgfpathcurveto{\pgfqpoint{2.514707in}{1.168154in}}{\pgfqpoint{2.505032in}{1.164146in}}{\pgfqpoint{2.497899in}{1.157013in}}%
\pgfpathcurveto{\pgfqpoint{2.490766in}{1.149880in}}{\pgfqpoint{2.486758in}{1.140205in}}{\pgfqpoint{2.486758in}{1.130117in}}%
\pgfpathcurveto{\pgfqpoint{2.486758in}{1.120030in}}{\pgfqpoint{2.490766in}{1.110354in}}{\pgfqpoint{2.497899in}{1.103222in}}%
\pgfpathcurveto{\pgfqpoint{2.505032in}{1.096089in}}{\pgfqpoint{2.514707in}{1.092081in}}{\pgfqpoint{2.524795in}{1.092081in}}%
\pgfpathclose%
\pgfusepath{stroke,fill}%
\end{pgfscope}%
\begin{pgfscope}%
\pgfpathrectangle{\pgfqpoint{0.800000in}{0.528000in}}{\pgfqpoint{4.960000in}{3.696000in}} %
\pgfusepath{clip}%
\pgfsetbuttcap%
\pgfsetroundjoin%
\definecolor{currentfill}{rgb}{0.121569,0.466667,0.705882}%
\pgfsetfillcolor{currentfill}%
\pgfsetlinewidth{1.003750pt}%
\definecolor{currentstroke}{rgb}{0.121569,0.466667,0.705882}%
\pgfsetstrokecolor{currentstroke}%
\pgfsetdash{}{0pt}%
\pgfpathmoveto{\pgfqpoint{2.371871in}{3.992571in}}%
\pgfpathcurveto{\pgfqpoint{2.381958in}{3.992571in}}{\pgfqpoint{2.391634in}{3.996578in}}{\pgfqpoint{2.398766in}{4.003711in}}%
\pgfpathcurveto{\pgfqpoint{2.405899in}{4.010844in}}{\pgfqpoint{2.409907in}{4.020520in}}{\pgfqpoint{2.409907in}{4.030607in}}%
\pgfpathcurveto{\pgfqpoint{2.409907in}{4.040694in}}{\pgfqpoint{2.405899in}{4.050370in}}{\pgfqpoint{2.398766in}{4.057503in}}%
\pgfpathcurveto{\pgfqpoint{2.391634in}{4.064636in}}{\pgfqpoint{2.381958in}{4.068643in}}{\pgfqpoint{2.371871in}{4.068643in}}%
\pgfpathcurveto{\pgfqpoint{2.361783in}{4.068643in}}{\pgfqpoint{2.352108in}{4.064636in}}{\pgfqpoint{2.344975in}{4.057503in}}%
\pgfpathcurveto{\pgfqpoint{2.337842in}{4.050370in}}{\pgfqpoint{2.333834in}{4.040694in}}{\pgfqpoint{2.333834in}{4.030607in}}%
\pgfpathcurveto{\pgfqpoint{2.333834in}{4.020520in}}{\pgfqpoint{2.337842in}{4.010844in}}{\pgfqpoint{2.344975in}{4.003711in}}%
\pgfpathcurveto{\pgfqpoint{2.352108in}{3.996578in}}{\pgfqpoint{2.361783in}{3.992571in}}{\pgfqpoint{2.371871in}{3.992571in}}%
\pgfpathclose%
\pgfusepath{stroke,fill}%
\end{pgfscope}%
\begin{pgfscope}%
\pgfpathrectangle{\pgfqpoint{0.800000in}{0.528000in}}{\pgfqpoint{4.960000in}{3.696000in}} %
\pgfusepath{clip}%
\pgfsetbuttcap%
\pgfsetroundjoin%
\definecolor{currentfill}{rgb}{0.121569,0.466667,0.705882}%
\pgfsetfillcolor{currentfill}%
\pgfsetlinewidth{1.003750pt}%
\definecolor{currentstroke}{rgb}{0.121569,0.466667,0.705882}%
\pgfsetstrokecolor{currentstroke}%
\pgfsetdash{}{0pt}%
\pgfpathmoveto{\pgfqpoint{3.317771in}{0.881055in}}%
\pgfpathcurveto{\pgfqpoint{3.327858in}{0.881055in}}{\pgfqpoint{3.337534in}{0.885062in}}{\pgfqpoint{3.344667in}{0.892195in}}%
\pgfpathcurveto{\pgfqpoint{3.351799in}{0.899328in}}{\pgfqpoint{3.355807in}{0.909004in}}{\pgfqpoint{3.355807in}{0.919091in}}%
\pgfpathcurveto{\pgfqpoint{3.355807in}{0.929178in}}{\pgfqpoint{3.351799in}{0.938854in}}{\pgfqpoint{3.344667in}{0.945987in}}%
\pgfpathcurveto{\pgfqpoint{3.337534in}{0.953120in}}{\pgfqpoint{3.327858in}{0.957127in}}{\pgfqpoint{3.317771in}{0.957127in}}%
\pgfpathcurveto{\pgfqpoint{3.307683in}{0.957127in}}{\pgfqpoint{3.298008in}{0.953120in}}{\pgfqpoint{3.290875in}{0.945987in}}%
\pgfpathcurveto{\pgfqpoint{3.283742in}{0.938854in}}{\pgfqpoint{3.279735in}{0.929178in}}{\pgfqpoint{3.279735in}{0.919091in}}%
\pgfpathcurveto{\pgfqpoint{3.279735in}{0.909004in}}{\pgfqpoint{3.283742in}{0.899328in}}{\pgfqpoint{3.290875in}{0.892195in}}%
\pgfpathcurveto{\pgfqpoint{3.298008in}{0.885062in}}{\pgfqpoint{3.307683in}{0.881055in}}{\pgfqpoint{3.317771in}{0.881055in}}%
\pgfpathclose%
\pgfusepath{stroke,fill}%
\end{pgfscope}%
\begin{pgfscope}%
\pgfpathrectangle{\pgfqpoint{0.800000in}{0.528000in}}{\pgfqpoint{4.960000in}{3.696000in}} %
\pgfusepath{clip}%
\pgfsetbuttcap%
\pgfsetroundjoin%
\definecolor{currentfill}{rgb}{0.121569,0.466667,0.705882}%
\pgfsetfillcolor{currentfill}%
\pgfsetlinewidth{1.003750pt}%
\definecolor{currentstroke}{rgb}{0.121569,0.466667,0.705882}%
\pgfsetstrokecolor{currentstroke}%
\pgfsetdash{}{0pt}%
\pgfpathmoveto{\pgfqpoint{3.188228in}{3.959324in}}%
\pgfpathcurveto{\pgfqpoint{3.198316in}{3.959324in}}{\pgfqpoint{3.207991in}{3.963332in}}{\pgfqpoint{3.215124in}{3.970465in}}%
\pgfpathcurveto{\pgfqpoint{3.222257in}{3.977598in}}{\pgfqpoint{3.226265in}{3.987273in}}{\pgfqpoint{3.226265in}{3.997361in}}%
\pgfpathcurveto{\pgfqpoint{3.226265in}{4.007448in}}{\pgfqpoint{3.222257in}{4.017123in}}{\pgfqpoint{3.215124in}{4.024256in}}%
\pgfpathcurveto{\pgfqpoint{3.207991in}{4.031389in}}{\pgfqpoint{3.198316in}{4.035397in}}{\pgfqpoint{3.188228in}{4.035397in}}%
\pgfpathcurveto{\pgfqpoint{3.178141in}{4.035397in}}{\pgfqpoint{3.168465in}{4.031389in}}{\pgfqpoint{3.161333in}{4.024256in}}%
\pgfpathcurveto{\pgfqpoint{3.154200in}{4.017123in}}{\pgfqpoint{3.150192in}{4.007448in}}{\pgfqpoint{3.150192in}{3.997361in}}%
\pgfpathcurveto{\pgfqpoint{3.150192in}{3.987273in}}{\pgfqpoint{3.154200in}{3.977598in}}{\pgfqpoint{3.161333in}{3.970465in}}%
\pgfpathcurveto{\pgfqpoint{3.168465in}{3.963332in}}{\pgfqpoint{3.178141in}{3.959324in}}{\pgfqpoint{3.188228in}{3.959324in}}%
\pgfpathclose%
\pgfusepath{stroke,fill}%
\end{pgfscope}%
\begin{pgfscope}%
\pgfpathrectangle{\pgfqpoint{0.800000in}{0.528000in}}{\pgfqpoint{4.960000in}{3.696000in}} %
\pgfusepath{clip}%
\pgfsetbuttcap%
\pgfsetroundjoin%
\definecolor{currentfill}{rgb}{0.121569,0.466667,0.705882}%
\pgfsetfillcolor{currentfill}%
\pgfsetlinewidth{1.003750pt}%
\definecolor{currentstroke}{rgb}{0.121569,0.466667,0.705882}%
\pgfsetstrokecolor{currentstroke}%
\pgfsetdash{}{0pt}%
\pgfpathmoveto{\pgfqpoint{4.207830in}{0.683357in}}%
\pgfpathcurveto{\pgfqpoint{4.217917in}{0.683357in}}{\pgfqpoint{4.227593in}{0.687364in}}{\pgfqpoint{4.234726in}{0.694497in}}%
\pgfpathcurveto{\pgfqpoint{4.241858in}{0.701630in}}{\pgfqpoint{4.245866in}{0.711306in}}{\pgfqpoint{4.245866in}{0.721393in}}%
\pgfpathcurveto{\pgfqpoint{4.245866in}{0.731480in}}{\pgfqpoint{4.241858in}{0.741156in}}{\pgfqpoint{4.234726in}{0.748289in}}%
\pgfpathcurveto{\pgfqpoint{4.227593in}{0.755422in}}{\pgfqpoint{4.217917in}{0.759429in}}{\pgfqpoint{4.207830in}{0.759429in}}%
\pgfpathcurveto{\pgfqpoint{4.197743in}{0.759429in}}{\pgfqpoint{4.188067in}{0.755422in}}{\pgfqpoint{4.180934in}{0.748289in}}%
\pgfpathcurveto{\pgfqpoint{4.173801in}{0.741156in}}{\pgfqpoint{4.169794in}{0.731480in}}{\pgfqpoint{4.169794in}{0.721393in}}%
\pgfpathcurveto{\pgfqpoint{4.169794in}{0.711306in}}{\pgfqpoint{4.173801in}{0.701630in}}{\pgfqpoint{4.180934in}{0.694497in}}%
\pgfpathcurveto{\pgfqpoint{4.188067in}{0.687364in}}{\pgfqpoint{4.197743in}{0.683357in}}{\pgfqpoint{4.207830in}{0.683357in}}%
\pgfpathclose%
\pgfusepath{stroke,fill}%
\end{pgfscope}%
\begin{pgfscope}%
\pgfpathrectangle{\pgfqpoint{0.800000in}{0.528000in}}{\pgfqpoint{4.960000in}{3.696000in}} %
\pgfusepath{clip}%
\pgfsetbuttcap%
\pgfsetroundjoin%
\definecolor{currentfill}{rgb}{0.121569,0.466667,0.705882}%
\pgfsetfillcolor{currentfill}%
\pgfsetlinewidth{1.003750pt}%
\definecolor{currentstroke}{rgb}{0.121569,0.466667,0.705882}%
\pgfsetstrokecolor{currentstroke}%
\pgfsetdash{}{0pt}%
\pgfpathmoveto{\pgfqpoint{4.827101in}{1.237602in}}%
\pgfpathcurveto{\pgfqpoint{4.837189in}{1.237602in}}{\pgfqpoint{4.846864in}{1.241610in}}{\pgfqpoint{4.853997in}{1.248743in}}%
\pgfpathcurveto{\pgfqpoint{4.861130in}{1.255875in}}{\pgfqpoint{4.865138in}{1.265551in}}{\pgfqpoint{4.865138in}{1.275638in}}%
\pgfpathcurveto{\pgfqpoint{4.865138in}{1.285726in}}{\pgfqpoint{4.861130in}{1.295401in}}{\pgfqpoint{4.853997in}{1.302534in}}%
\pgfpathcurveto{\pgfqpoint{4.846864in}{1.309667in}}{\pgfqpoint{4.837189in}{1.313675in}}{\pgfqpoint{4.827101in}{1.313675in}}%
\pgfpathcurveto{\pgfqpoint{4.817014in}{1.313675in}}{\pgfqpoint{4.807338in}{1.309667in}}{\pgfqpoint{4.800206in}{1.302534in}}%
\pgfpathcurveto{\pgfqpoint{4.793073in}{1.295401in}}{\pgfqpoint{4.789065in}{1.285726in}}{\pgfqpoint{4.789065in}{1.275638in}}%
\pgfpathcurveto{\pgfqpoint{4.789065in}{1.265551in}}{\pgfqpoint{4.793073in}{1.255875in}}{\pgfqpoint{4.800206in}{1.248743in}}%
\pgfpathcurveto{\pgfqpoint{4.807338in}{1.241610in}}{\pgfqpoint{4.817014in}{1.237602in}}{\pgfqpoint{4.827101in}{1.237602in}}%
\pgfpathclose%
\pgfusepath{stroke,fill}%
\end{pgfscope}%
\begin{pgfscope}%
\pgfpathrectangle{\pgfqpoint{0.800000in}{0.528000in}}{\pgfqpoint{4.960000in}{3.696000in}} %
\pgfusepath{clip}%
\pgfsetbuttcap%
\pgfsetroundjoin%
\definecolor{currentfill}{rgb}{0.121569,0.466667,0.705882}%
\pgfsetfillcolor{currentfill}%
\pgfsetlinewidth{1.003750pt}%
\definecolor{currentstroke}{rgb}{0.121569,0.466667,0.705882}%
\pgfsetstrokecolor{currentstroke}%
\pgfsetdash{}{0pt}%
\pgfpathmoveto{\pgfqpoint{4.468060in}{3.772899in}}%
\pgfpathcurveto{\pgfqpoint{4.478147in}{3.772899in}}{\pgfqpoint{4.487822in}{3.776907in}}{\pgfqpoint{4.494955in}{3.784039in}}%
\pgfpathcurveto{\pgfqpoint{4.502088in}{3.791172in}}{\pgfqpoint{4.506096in}{3.800848in}}{\pgfqpoint{4.506096in}{3.810935in}}%
\pgfpathcurveto{\pgfqpoint{4.506096in}{3.821022in}}{\pgfqpoint{4.502088in}{3.830698in}}{\pgfqpoint{4.494955in}{3.837831in}}%
\pgfpathcurveto{\pgfqpoint{4.487822in}{3.844964in}}{\pgfqpoint{4.478147in}{3.848971in}}{\pgfqpoint{4.468060in}{3.848971in}}%
\pgfpathcurveto{\pgfqpoint{4.457972in}{3.848971in}}{\pgfqpoint{4.448297in}{3.844964in}}{\pgfqpoint{4.441164in}{3.837831in}}%
\pgfpathcurveto{\pgfqpoint{4.434031in}{3.830698in}}{\pgfqpoint{4.430023in}{3.821022in}}{\pgfqpoint{4.430023in}{3.810935in}}%
\pgfpathcurveto{\pgfqpoint{4.430023in}{3.800848in}}{\pgfqpoint{4.434031in}{3.791172in}}{\pgfqpoint{4.441164in}{3.784039in}}%
\pgfpathcurveto{\pgfqpoint{4.448297in}{3.776907in}}{\pgfqpoint{4.457972in}{3.772899in}}{\pgfqpoint{4.468060in}{3.772899in}}%
\pgfpathclose%
\pgfusepath{stroke,fill}%
\end{pgfscope}%
\begin{pgfscope}%
\pgfpathrectangle{\pgfqpoint{0.800000in}{0.528000in}}{\pgfqpoint{4.960000in}{3.696000in}} %
\pgfusepath{clip}%
\pgfsetbuttcap%
\pgfsetroundjoin%
\definecolor{currentfill}{rgb}{0.121569,0.466667,0.705882}%
\pgfsetfillcolor{currentfill}%
\pgfsetlinewidth{1.003750pt}%
\definecolor{currentstroke}{rgb}{0.121569,0.466667,0.705882}%
\pgfsetstrokecolor{currentstroke}%
\pgfsetdash{}{0pt}%
\pgfpathmoveto{\pgfqpoint{5.296265in}{1.544799in}}%
\pgfpathcurveto{\pgfqpoint{5.306352in}{1.544799in}}{\pgfqpoint{5.316027in}{1.548807in}}{\pgfqpoint{5.323160in}{1.555940in}}%
\pgfpathcurveto{\pgfqpoint{5.330293in}{1.563073in}}{\pgfqpoint{5.334301in}{1.572748in}}{\pgfqpoint{5.334301in}{1.582836in}}%
\pgfpathcurveto{\pgfqpoint{5.334301in}{1.592923in}}{\pgfqpoint{5.330293in}{1.602599in}}{\pgfqpoint{5.323160in}{1.609731in}}%
\pgfpathcurveto{\pgfqpoint{5.316027in}{1.616864in}}{\pgfqpoint{5.306352in}{1.620872in}}{\pgfqpoint{5.296265in}{1.620872in}}%
\pgfpathcurveto{\pgfqpoint{5.286177in}{1.620872in}}{\pgfqpoint{5.276502in}{1.616864in}}{\pgfqpoint{5.269369in}{1.609731in}}%
\pgfpathcurveto{\pgfqpoint{5.262236in}{1.602599in}}{\pgfqpoint{5.258228in}{1.592923in}}{\pgfqpoint{5.258228in}{1.582836in}}%
\pgfpathcurveto{\pgfqpoint{5.258228in}{1.572748in}}{\pgfqpoint{5.262236in}{1.563073in}}{\pgfqpoint{5.269369in}{1.555940in}}%
\pgfpathcurveto{\pgfqpoint{5.276502in}{1.548807in}}{\pgfqpoint{5.286177in}{1.544799in}}{\pgfqpoint{5.296265in}{1.544799in}}%
\pgfpathclose%
\pgfusepath{stroke,fill}%
\end{pgfscope}%
\begin{pgfscope}%
\pgfpathrectangle{\pgfqpoint{0.800000in}{0.528000in}}{\pgfqpoint{4.960000in}{3.696000in}} %
\pgfusepath{clip}%
\pgfsetbuttcap%
\pgfsetroundjoin%
\definecolor{currentfill}{rgb}{0.121569,0.466667,0.705882}%
\pgfsetfillcolor{currentfill}%
\pgfsetlinewidth{1.003750pt}%
\definecolor{currentstroke}{rgb}{0.121569,0.466667,0.705882}%
\pgfsetstrokecolor{currentstroke}%
\pgfsetdash{}{0pt}%
\pgfpathmoveto{\pgfqpoint{5.528280in}{2.502909in}}%
\pgfpathcurveto{\pgfqpoint{5.538367in}{2.502909in}}{\pgfqpoint{5.548043in}{2.506917in}}{\pgfqpoint{5.555175in}{2.514049in}}%
\pgfpathcurveto{\pgfqpoint{5.562308in}{2.521182in}}{\pgfqpoint{5.566316in}{2.530858in}}{\pgfqpoint{5.566316in}{2.540945in}}%
\pgfpathcurveto{\pgfqpoint{5.566316in}{2.551033in}}{\pgfqpoint{5.562308in}{2.560708in}}{\pgfqpoint{5.555175in}{2.567841in}}%
\pgfpathcurveto{\pgfqpoint{5.548043in}{2.574974in}}{\pgfqpoint{5.538367in}{2.578982in}}{\pgfqpoint{5.528280in}{2.578982in}}%
\pgfpathcurveto{\pgfqpoint{5.518192in}{2.578982in}}{\pgfqpoint{5.508517in}{2.574974in}}{\pgfqpoint{5.501384in}{2.567841in}}%
\pgfpathcurveto{\pgfqpoint{5.494251in}{2.560708in}}{\pgfqpoint{5.490243in}{2.551033in}}{\pgfqpoint{5.490243in}{2.540945in}}%
\pgfpathcurveto{\pgfqpoint{5.490243in}{2.530858in}}{\pgfqpoint{5.494251in}{2.521182in}}{\pgfqpoint{5.501384in}{2.514049in}}%
\pgfpathcurveto{\pgfqpoint{5.508517in}{2.506917in}}{\pgfqpoint{5.518192in}{2.502909in}}{\pgfqpoint{5.528280in}{2.502909in}}%
\pgfpathclose%
\pgfusepath{stroke,fill}%
\end{pgfscope}%
\begin{pgfscope}%
\pgfpathrectangle{\pgfqpoint{0.800000in}{0.528000in}}{\pgfqpoint{4.960000in}{3.696000in}} %
\pgfusepath{clip}%
\pgfsetbuttcap%
\pgfsetroundjoin%
\definecolor{currentfill}{rgb}{0.121569,0.466667,0.705882}%
\pgfsetfillcolor{currentfill}%
\pgfsetlinewidth{1.003750pt}%
\definecolor{currentstroke}{rgb}{0.121569,0.466667,0.705882}%
\pgfsetstrokecolor{currentstroke}%
\pgfsetdash{}{0pt}%
\pgfpathmoveto{\pgfqpoint{5.328319in}{3.240906in}}%
\pgfpathcurveto{\pgfqpoint{5.338407in}{3.240906in}}{\pgfqpoint{5.348082in}{3.244914in}}{\pgfqpoint{5.355215in}{3.252047in}}%
\pgfpathcurveto{\pgfqpoint{5.362348in}{3.259180in}}{\pgfqpoint{5.366356in}{3.268855in}}{\pgfqpoint{5.366356in}{3.278943in}}%
\pgfpathcurveto{\pgfqpoint{5.366356in}{3.289030in}}{\pgfqpoint{5.362348in}{3.298706in}}{\pgfqpoint{5.355215in}{3.305838in}}%
\pgfpathcurveto{\pgfqpoint{5.348082in}{3.312971in}}{\pgfqpoint{5.338407in}{3.316979in}}{\pgfqpoint{5.328319in}{3.316979in}}%
\pgfpathcurveto{\pgfqpoint{5.318232in}{3.316979in}}{\pgfqpoint{5.308557in}{3.312971in}}{\pgfqpoint{5.301424in}{3.305838in}}%
\pgfpathcurveto{\pgfqpoint{5.294291in}{3.298706in}}{\pgfqpoint{5.290283in}{3.289030in}}{\pgfqpoint{5.290283in}{3.278943in}}%
\pgfpathcurveto{\pgfqpoint{5.290283in}{3.268855in}}{\pgfqpoint{5.294291in}{3.259180in}}{\pgfqpoint{5.301424in}{3.252047in}}%
\pgfpathcurveto{\pgfqpoint{5.308557in}{3.244914in}}{\pgfqpoint{5.318232in}{3.240906in}}{\pgfqpoint{5.328319in}{3.240906in}}%
\pgfpathclose%
\pgfusepath{stroke,fill}%
\end{pgfscope}%
\begin{pgfscope}%
\pgfpathrectangle{\pgfqpoint{0.800000in}{0.528000in}}{\pgfqpoint{4.960000in}{3.696000in}} %
\pgfusepath{clip}%
\pgfsetbuttcap%
\pgfsetroundjoin%
\definecolor{currentfill}{rgb}{0.121569,0.466667,0.705882}%
\pgfsetfillcolor{currentfill}%
\pgfsetlinewidth{1.003750pt}%
\definecolor{currentstroke}{rgb}{0.121569,0.466667,0.705882}%
\pgfsetstrokecolor{currentstroke}%
\pgfsetdash{}{0pt}%
\pgfpathmoveto{\pgfqpoint{5.176855in}{3.602912in}}%
\pgfpathcurveto{\pgfqpoint{5.186942in}{3.602912in}}{\pgfqpoint{5.196618in}{3.606919in}}{\pgfqpoint{5.203751in}{3.614052in}}%
\pgfpathcurveto{\pgfqpoint{5.210884in}{3.621185in}}{\pgfqpoint{5.214891in}{3.630860in}}{\pgfqpoint{5.214891in}{3.640948in}}%
\pgfpathcurveto{\pgfqpoint{5.214891in}{3.651035in}}{\pgfqpoint{5.210884in}{3.660711in}}{\pgfqpoint{5.203751in}{3.667844in}}%
\pgfpathcurveto{\pgfqpoint{5.196618in}{3.674976in}}{\pgfqpoint{5.186942in}{3.678984in}}{\pgfqpoint{5.176855in}{3.678984in}}%
\pgfpathcurveto{\pgfqpoint{5.166768in}{3.678984in}}{\pgfqpoint{5.157092in}{3.674976in}}{\pgfqpoint{5.149959in}{3.667844in}}%
\pgfpathcurveto{\pgfqpoint{5.142827in}{3.660711in}}{\pgfqpoint{5.138819in}{3.651035in}}{\pgfqpoint{5.138819in}{3.640948in}}%
\pgfpathcurveto{\pgfqpoint{5.138819in}{3.630860in}}{\pgfqpoint{5.142827in}{3.621185in}}{\pgfqpoint{5.149959in}{3.614052in}}%
\pgfpathcurveto{\pgfqpoint{5.157092in}{3.606919in}}{\pgfqpoint{5.166768in}{3.602912in}}{\pgfqpoint{5.176855in}{3.602912in}}%
\pgfpathclose%
\pgfusepath{stroke,fill}%
\end{pgfscope}%
\begin{pgfscope}%
\pgfpathrectangle{\pgfqpoint{0.800000in}{0.528000in}}{\pgfqpoint{4.960000in}{3.696000in}} %
\pgfusepath{clip}%
\pgfsetbuttcap%
\pgfsetroundjoin%
\definecolor{currentfill}{rgb}{0.121569,0.466667,0.705882}%
\pgfsetfillcolor{currentfill}%
\pgfsetlinewidth{1.003750pt}%
\definecolor{currentstroke}{rgb}{0.121569,0.466667,0.705882}%
\pgfsetstrokecolor{currentstroke}%
\pgfsetdash{}{0pt}%
\pgfpathmoveto{\pgfqpoint{1.673483in}{1.012308in}}%
\pgfpathcurveto{\pgfqpoint{1.683570in}{1.012308in}}{\pgfqpoint{1.693246in}{1.016316in}}{\pgfqpoint{1.700378in}{1.023448in}}%
\pgfpathcurveto{\pgfqpoint{1.707511in}{1.030581in}}{\pgfqpoint{1.711519in}{1.040257in}}{\pgfqpoint{1.711519in}{1.050344in}}%
\pgfpathcurveto{\pgfqpoint{1.711519in}{1.060431in}}{\pgfqpoint{1.707511in}{1.070107in}}{\pgfqpoint{1.700378in}{1.077240in}}%
\pgfpathcurveto{\pgfqpoint{1.693246in}{1.084373in}}{\pgfqpoint{1.683570in}{1.088380in}}{\pgfqpoint{1.673483in}{1.088380in}}%
\pgfpathcurveto{\pgfqpoint{1.663395in}{1.088380in}}{\pgfqpoint{1.653720in}{1.084373in}}{\pgfqpoint{1.646587in}{1.077240in}}%
\pgfpathcurveto{\pgfqpoint{1.639454in}{1.070107in}}{\pgfqpoint{1.635446in}{1.060431in}}{\pgfqpoint{1.635446in}{1.050344in}}%
\pgfpathcurveto{\pgfqpoint{1.635446in}{1.040257in}}{\pgfqpoint{1.639454in}{1.030581in}}{\pgfqpoint{1.646587in}{1.023448in}}%
\pgfpathcurveto{\pgfqpoint{1.653720in}{1.016316in}}{\pgfqpoint{1.663395in}{1.012308in}}{\pgfqpoint{1.673483in}{1.012308in}}%
\pgfpathclose%
\pgfusepath{stroke,fill}%
\end{pgfscope}%
\begin{pgfscope}%
\pgfpathrectangle{\pgfqpoint{0.800000in}{0.528000in}}{\pgfqpoint{4.960000in}{3.696000in}} %
\pgfusepath{clip}%
\pgfsetbuttcap%
\pgfsetroundjoin%
\definecolor{currentfill}{rgb}{0.121569,0.466667,0.705882}%
\pgfsetfillcolor{currentfill}%
\pgfsetlinewidth{1.003750pt}%
\definecolor{currentstroke}{rgb}{0.121569,0.466667,0.705882}%
\pgfsetstrokecolor{currentstroke}%
\pgfsetdash{}{0pt}%
\pgfpathmoveto{\pgfqpoint{1.342173in}{1.561903in}}%
\pgfpathcurveto{\pgfqpoint{1.352261in}{1.561903in}}{\pgfqpoint{1.361936in}{1.565911in}}{\pgfqpoint{1.369069in}{1.573044in}}%
\pgfpathcurveto{\pgfqpoint{1.376202in}{1.580177in}}{\pgfqpoint{1.380210in}{1.589852in}}{\pgfqpoint{1.380210in}{1.599940in}}%
\pgfpathcurveto{\pgfqpoint{1.380210in}{1.610027in}}{\pgfqpoint{1.376202in}{1.619703in}}{\pgfqpoint{1.369069in}{1.626835in}}%
\pgfpathcurveto{\pgfqpoint{1.361936in}{1.633968in}}{\pgfqpoint{1.352261in}{1.637976in}}{\pgfqpoint{1.342173in}{1.637976in}}%
\pgfpathcurveto{\pgfqpoint{1.332086in}{1.637976in}}{\pgfqpoint{1.322411in}{1.633968in}}{\pgfqpoint{1.315278in}{1.626835in}}%
\pgfpathcurveto{\pgfqpoint{1.308145in}{1.619703in}}{\pgfqpoint{1.304137in}{1.610027in}}{\pgfqpoint{1.304137in}{1.599940in}}%
\pgfpathcurveto{\pgfqpoint{1.304137in}{1.589852in}}{\pgfqpoint{1.308145in}{1.580177in}}{\pgfqpoint{1.315278in}{1.573044in}}%
\pgfpathcurveto{\pgfqpoint{1.322411in}{1.565911in}}{\pgfqpoint{1.332086in}{1.561903in}}{\pgfqpoint{1.342173in}{1.561903in}}%
\pgfpathclose%
\pgfusepath{stroke,fill}%
\end{pgfscope}%
\begin{pgfscope}%
\pgfpathrectangle{\pgfqpoint{0.800000in}{0.528000in}}{\pgfqpoint{4.960000in}{3.696000in}} %
\pgfusepath{clip}%
\pgfsetbuttcap%
\pgfsetroundjoin%
\definecolor{currentfill}{rgb}{0.121569,0.466667,0.705882}%
\pgfsetfillcolor{currentfill}%
\pgfsetlinewidth{1.003750pt}%
\definecolor{currentstroke}{rgb}{0.121569,0.466667,0.705882}%
\pgfsetstrokecolor{currentstroke}%
\pgfsetdash{}{0pt}%
\pgfpathmoveto{\pgfqpoint{1.031720in}{2.142608in}}%
\pgfpathcurveto{\pgfqpoint{1.041808in}{2.142608in}}{\pgfqpoint{1.051483in}{2.146615in}}{\pgfqpoint{1.058616in}{2.153748in}}%
\pgfpathcurveto{\pgfqpoint{1.065749in}{2.160881in}}{\pgfqpoint{1.069757in}{2.170557in}}{\pgfqpoint{1.069757in}{2.180644in}}%
\pgfpathcurveto{\pgfqpoint{1.069757in}{2.190731in}}{\pgfqpoint{1.065749in}{2.200407in}}{\pgfqpoint{1.058616in}{2.207540in}}%
\pgfpathcurveto{\pgfqpoint{1.051483in}{2.214673in}}{\pgfqpoint{1.041808in}{2.218680in}}{\pgfqpoint{1.031720in}{2.218680in}}%
\pgfpathcurveto{\pgfqpoint{1.021633in}{2.218680in}}{\pgfqpoint{1.011957in}{2.214673in}}{\pgfqpoint{1.004825in}{2.207540in}}%
\pgfpathcurveto{\pgfqpoint{0.997692in}{2.200407in}}{\pgfqpoint{0.993684in}{2.190731in}}{\pgfqpoint{0.993684in}{2.180644in}}%
\pgfpathcurveto{\pgfqpoint{0.993684in}{2.170557in}}{\pgfqpoint{0.997692in}{2.160881in}}{\pgfqpoint{1.004825in}{2.153748in}}%
\pgfpathcurveto{\pgfqpoint{1.011957in}{2.146615in}}{\pgfqpoint{1.021633in}{2.142608in}}{\pgfqpoint{1.031720in}{2.142608in}}%
\pgfpathclose%
\pgfusepath{stroke,fill}%
\end{pgfscope}%
\begin{pgfscope}%
\pgfpathrectangle{\pgfqpoint{0.800000in}{0.528000in}}{\pgfqpoint{4.960000in}{3.696000in}} %
\pgfusepath{clip}%
\pgfsetbuttcap%
\pgfsetroundjoin%
\definecolor{currentfill}{rgb}{0.121569,0.466667,0.705882}%
\pgfsetfillcolor{currentfill}%
\pgfsetlinewidth{1.003750pt}%
\definecolor{currentstroke}{rgb}{0.121569,0.466667,0.705882}%
\pgfsetstrokecolor{currentstroke}%
\pgfsetdash{}{0pt}%
\pgfpathmoveto{\pgfqpoint{1.356249in}{3.160048in}}%
\pgfpathcurveto{\pgfqpoint{1.366337in}{3.160048in}}{\pgfqpoint{1.376012in}{3.164056in}}{\pgfqpoint{1.383145in}{3.171189in}}%
\pgfpathcurveto{\pgfqpoint{1.390278in}{3.178321in}}{\pgfqpoint{1.394286in}{3.187997in}}{\pgfqpoint{1.394286in}{3.198084in}}%
\pgfpathcurveto{\pgfqpoint{1.394286in}{3.208172in}}{\pgfqpoint{1.390278in}{3.217847in}}{\pgfqpoint{1.383145in}{3.224980in}}%
\pgfpathcurveto{\pgfqpoint{1.376012in}{3.232113in}}{\pgfqpoint{1.366337in}{3.236121in}}{\pgfqpoint{1.356249in}{3.236121in}}%
\pgfpathcurveto{\pgfqpoint{1.346162in}{3.236121in}}{\pgfqpoint{1.336486in}{3.232113in}}{\pgfqpoint{1.329354in}{3.224980in}}%
\pgfpathcurveto{\pgfqpoint{1.322221in}{3.217847in}}{\pgfqpoint{1.318213in}{3.208172in}}{\pgfqpoint{1.318213in}{3.198084in}}%
\pgfpathcurveto{\pgfqpoint{1.318213in}{3.187997in}}{\pgfqpoint{1.322221in}{3.178321in}}{\pgfqpoint{1.329354in}{3.171189in}}%
\pgfpathcurveto{\pgfqpoint{1.336486in}{3.164056in}}{\pgfqpoint{1.346162in}{3.160048in}}{\pgfqpoint{1.356249in}{3.160048in}}%
\pgfpathclose%
\pgfusepath{stroke,fill}%
\end{pgfscope}%
\begin{pgfscope}%
\pgfpathrectangle{\pgfqpoint{0.800000in}{0.528000in}}{\pgfqpoint{4.960000in}{3.696000in}} %
\pgfusepath{clip}%
\pgfsetbuttcap%
\pgfsetroundjoin%
\definecolor{currentfill}{rgb}{0.121569,0.466667,0.705882}%
\pgfsetfillcolor{currentfill}%
\pgfsetlinewidth{1.003750pt}%
\definecolor{currentstroke}{rgb}{0.121569,0.466667,0.705882}%
\pgfsetstrokecolor{currentstroke}%
\pgfsetdash{}{0pt}%
\pgfpathmoveto{\pgfqpoint{1.528570in}{3.587024in}}%
\pgfpathcurveto{\pgfqpoint{1.538657in}{3.587024in}}{\pgfqpoint{1.548333in}{3.591032in}}{\pgfqpoint{1.555466in}{3.598165in}}%
\pgfpathcurveto{\pgfqpoint{1.562599in}{3.605297in}}{\pgfqpoint{1.566606in}{3.614973in}}{\pgfqpoint{1.566606in}{3.625060in}}%
\pgfpathcurveto{\pgfqpoint{1.566606in}{3.635148in}}{\pgfqpoint{1.562599in}{3.644823in}}{\pgfqpoint{1.555466in}{3.651956in}}%
\pgfpathcurveto{\pgfqpoint{1.548333in}{3.659089in}}{\pgfqpoint{1.538657in}{3.663097in}}{\pgfqpoint{1.528570in}{3.663097in}}%
\pgfpathcurveto{\pgfqpoint{1.518483in}{3.663097in}}{\pgfqpoint{1.508807in}{3.659089in}}{\pgfqpoint{1.501674in}{3.651956in}}%
\pgfpathcurveto{\pgfqpoint{1.494541in}{3.644823in}}{\pgfqpoint{1.490534in}{3.635148in}}{\pgfqpoint{1.490534in}{3.625060in}}%
\pgfpathcurveto{\pgfqpoint{1.490534in}{3.614973in}}{\pgfqpoint{1.494541in}{3.605297in}}{\pgfqpoint{1.501674in}{3.598165in}}%
\pgfpathcurveto{\pgfqpoint{1.508807in}{3.591032in}}{\pgfqpoint{1.518483in}{3.587024in}}{\pgfqpoint{1.528570in}{3.587024in}}%
\pgfpathclose%
\pgfusepath{stroke,fill}%
\end{pgfscope}%
\begin{pgfscope}%
\pgfpathrectangle{\pgfqpoint{0.800000in}{0.528000in}}{\pgfqpoint{4.960000in}{3.696000in}} %
\pgfusepath{clip}%
\pgfsetbuttcap%
\pgfsetroundjoin%
\definecolor{currentfill}{rgb}{0.121569,0.466667,0.705882}%
\pgfsetfillcolor{currentfill}%
\pgfsetlinewidth{1.003750pt}%
\definecolor{currentstroke}{rgb}{0.121569,0.466667,0.705882}%
\pgfsetstrokecolor{currentstroke}%
\pgfsetdash{}{0pt}%
\pgfpathmoveto{\pgfqpoint{2.428361in}{0.692038in}}%
\pgfpathcurveto{\pgfqpoint{2.438448in}{0.692038in}}{\pgfqpoint{2.448123in}{0.696046in}}{\pgfqpoint{2.455256in}{0.703179in}}%
\pgfpathcurveto{\pgfqpoint{2.462389in}{0.710312in}}{\pgfqpoint{2.466397in}{0.719987in}}{\pgfqpoint{2.466397in}{0.730075in}}%
\pgfpathcurveto{\pgfqpoint{2.466397in}{0.740162in}}{\pgfqpoint{2.462389in}{0.749838in}}{\pgfqpoint{2.455256in}{0.756970in}}%
\pgfpathcurveto{\pgfqpoint{2.448123in}{0.764103in}}{\pgfqpoint{2.438448in}{0.768111in}}{\pgfqpoint{2.428361in}{0.768111in}}%
\pgfpathcurveto{\pgfqpoint{2.418273in}{0.768111in}}{\pgfqpoint{2.408598in}{0.764103in}}{\pgfqpoint{2.401465in}{0.756970in}}%
\pgfpathcurveto{\pgfqpoint{2.394332in}{0.749838in}}{\pgfqpoint{2.390324in}{0.740162in}}{\pgfqpoint{2.390324in}{0.730075in}}%
\pgfpathcurveto{\pgfqpoint{2.390324in}{0.719987in}}{\pgfqpoint{2.394332in}{0.710312in}}{\pgfqpoint{2.401465in}{0.703179in}}%
\pgfpathcurveto{\pgfqpoint{2.408598in}{0.696046in}}{\pgfqpoint{2.418273in}{0.692038in}}{\pgfqpoint{2.428361in}{0.692038in}}%
\pgfpathclose%
\pgfusepath{stroke,fill}%
\end{pgfscope}%
\begin{pgfscope}%
\pgfpathrectangle{\pgfqpoint{0.800000in}{0.528000in}}{\pgfqpoint{4.960000in}{3.696000in}} %
\pgfusepath{clip}%
\pgfsetbuttcap%
\pgfsetroundjoin%
\definecolor{currentfill}{rgb}{0.121569,0.466667,0.705882}%
\pgfsetfillcolor{currentfill}%
\pgfsetlinewidth{1.003750pt}%
\definecolor{currentstroke}{rgb}{0.121569,0.466667,0.705882}%
\pgfsetstrokecolor{currentstroke}%
\pgfsetdash{}{0pt}%
\pgfpathmoveto{\pgfqpoint{2.524795in}{1.092081in}}%
\pgfpathcurveto{\pgfqpoint{2.534882in}{1.092081in}}{\pgfqpoint{2.544558in}{1.096089in}}{\pgfqpoint{2.551690in}{1.103222in}}%
\pgfpathcurveto{\pgfqpoint{2.558823in}{1.110354in}}{\pgfqpoint{2.562831in}{1.120030in}}{\pgfqpoint{2.562831in}{1.130117in}}%
\pgfpathcurveto{\pgfqpoint{2.562831in}{1.140205in}}{\pgfqpoint{2.558823in}{1.149880in}}{\pgfqpoint{2.551690in}{1.157013in}}%
\pgfpathcurveto{\pgfqpoint{2.544558in}{1.164146in}}{\pgfqpoint{2.534882in}{1.168154in}}{\pgfqpoint{2.524795in}{1.168154in}}%
\pgfpathcurveto{\pgfqpoint{2.514707in}{1.168154in}}{\pgfqpoint{2.505032in}{1.164146in}}{\pgfqpoint{2.497899in}{1.157013in}}%
\pgfpathcurveto{\pgfqpoint{2.490766in}{1.149880in}}{\pgfqpoint{2.486758in}{1.140205in}}{\pgfqpoint{2.486758in}{1.130117in}}%
\pgfpathcurveto{\pgfqpoint{2.486758in}{1.120030in}}{\pgfqpoint{2.490766in}{1.110354in}}{\pgfqpoint{2.497899in}{1.103222in}}%
\pgfpathcurveto{\pgfqpoint{2.505032in}{1.096089in}}{\pgfqpoint{2.514707in}{1.092081in}}{\pgfqpoint{2.524795in}{1.092081in}}%
\pgfpathclose%
\pgfusepath{stroke,fill}%
\end{pgfscope}%
\begin{pgfscope}%
\pgfpathrectangle{\pgfqpoint{0.800000in}{0.528000in}}{\pgfqpoint{4.960000in}{3.696000in}} %
\pgfusepath{clip}%
\pgfsetbuttcap%
\pgfsetroundjoin%
\definecolor{currentfill}{rgb}{0.121569,0.466667,0.705882}%
\pgfsetfillcolor{currentfill}%
\pgfsetlinewidth{1.003750pt}%
\definecolor{currentstroke}{rgb}{0.121569,0.466667,0.705882}%
\pgfsetstrokecolor{currentstroke}%
\pgfsetdash{}{0pt}%
\pgfpathmoveto{\pgfqpoint{2.371871in}{3.992571in}}%
\pgfpathcurveto{\pgfqpoint{2.381958in}{3.992571in}}{\pgfqpoint{2.391634in}{3.996578in}}{\pgfqpoint{2.398766in}{4.003711in}}%
\pgfpathcurveto{\pgfqpoint{2.405899in}{4.010844in}}{\pgfqpoint{2.409907in}{4.020520in}}{\pgfqpoint{2.409907in}{4.030607in}}%
\pgfpathcurveto{\pgfqpoint{2.409907in}{4.040694in}}{\pgfqpoint{2.405899in}{4.050370in}}{\pgfqpoint{2.398766in}{4.057503in}}%
\pgfpathcurveto{\pgfqpoint{2.391634in}{4.064636in}}{\pgfqpoint{2.381958in}{4.068643in}}{\pgfqpoint{2.371871in}{4.068643in}}%
\pgfpathcurveto{\pgfqpoint{2.361783in}{4.068643in}}{\pgfqpoint{2.352108in}{4.064636in}}{\pgfqpoint{2.344975in}{4.057503in}}%
\pgfpathcurveto{\pgfqpoint{2.337842in}{4.050370in}}{\pgfqpoint{2.333834in}{4.040694in}}{\pgfqpoint{2.333834in}{4.030607in}}%
\pgfpathcurveto{\pgfqpoint{2.333834in}{4.020520in}}{\pgfqpoint{2.337842in}{4.010844in}}{\pgfqpoint{2.344975in}{4.003711in}}%
\pgfpathcurveto{\pgfqpoint{2.352108in}{3.996578in}}{\pgfqpoint{2.361783in}{3.992571in}}{\pgfqpoint{2.371871in}{3.992571in}}%
\pgfpathclose%
\pgfusepath{stroke,fill}%
\end{pgfscope}%
\begin{pgfscope}%
\pgfpathrectangle{\pgfqpoint{0.800000in}{0.528000in}}{\pgfqpoint{4.960000in}{3.696000in}} %
\pgfusepath{clip}%
\pgfsetbuttcap%
\pgfsetroundjoin%
\definecolor{currentfill}{rgb}{0.121569,0.466667,0.705882}%
\pgfsetfillcolor{currentfill}%
\pgfsetlinewidth{1.003750pt}%
\definecolor{currentstroke}{rgb}{0.121569,0.466667,0.705882}%
\pgfsetstrokecolor{currentstroke}%
\pgfsetdash{}{0pt}%
\pgfpathmoveto{\pgfqpoint{3.317771in}{0.881055in}}%
\pgfpathcurveto{\pgfqpoint{3.327858in}{0.881055in}}{\pgfqpoint{3.337534in}{0.885062in}}{\pgfqpoint{3.344667in}{0.892195in}}%
\pgfpathcurveto{\pgfqpoint{3.351799in}{0.899328in}}{\pgfqpoint{3.355807in}{0.909004in}}{\pgfqpoint{3.355807in}{0.919091in}}%
\pgfpathcurveto{\pgfqpoint{3.355807in}{0.929178in}}{\pgfqpoint{3.351799in}{0.938854in}}{\pgfqpoint{3.344667in}{0.945987in}}%
\pgfpathcurveto{\pgfqpoint{3.337534in}{0.953120in}}{\pgfqpoint{3.327858in}{0.957127in}}{\pgfqpoint{3.317771in}{0.957127in}}%
\pgfpathcurveto{\pgfqpoint{3.307683in}{0.957127in}}{\pgfqpoint{3.298008in}{0.953120in}}{\pgfqpoint{3.290875in}{0.945987in}}%
\pgfpathcurveto{\pgfqpoint{3.283742in}{0.938854in}}{\pgfqpoint{3.279735in}{0.929178in}}{\pgfqpoint{3.279735in}{0.919091in}}%
\pgfpathcurveto{\pgfqpoint{3.279735in}{0.909004in}}{\pgfqpoint{3.283742in}{0.899328in}}{\pgfqpoint{3.290875in}{0.892195in}}%
\pgfpathcurveto{\pgfqpoint{3.298008in}{0.885062in}}{\pgfqpoint{3.307683in}{0.881055in}}{\pgfqpoint{3.317771in}{0.881055in}}%
\pgfpathclose%
\pgfusepath{stroke,fill}%
\end{pgfscope}%
\begin{pgfscope}%
\pgfpathrectangle{\pgfqpoint{0.800000in}{0.528000in}}{\pgfqpoint{4.960000in}{3.696000in}} %
\pgfusepath{clip}%
\pgfsetbuttcap%
\pgfsetroundjoin%
\definecolor{currentfill}{rgb}{0.121569,0.466667,0.705882}%
\pgfsetfillcolor{currentfill}%
\pgfsetlinewidth{1.003750pt}%
\definecolor{currentstroke}{rgb}{0.121569,0.466667,0.705882}%
\pgfsetstrokecolor{currentstroke}%
\pgfsetdash{}{0pt}%
\pgfpathmoveto{\pgfqpoint{3.188228in}{3.959324in}}%
\pgfpathcurveto{\pgfqpoint{3.198316in}{3.959324in}}{\pgfqpoint{3.207991in}{3.963332in}}{\pgfqpoint{3.215124in}{3.970465in}}%
\pgfpathcurveto{\pgfqpoint{3.222257in}{3.977598in}}{\pgfqpoint{3.226265in}{3.987273in}}{\pgfqpoint{3.226265in}{3.997361in}}%
\pgfpathcurveto{\pgfqpoint{3.226265in}{4.007448in}}{\pgfqpoint{3.222257in}{4.017123in}}{\pgfqpoint{3.215124in}{4.024256in}}%
\pgfpathcurveto{\pgfqpoint{3.207991in}{4.031389in}}{\pgfqpoint{3.198316in}{4.035397in}}{\pgfqpoint{3.188228in}{4.035397in}}%
\pgfpathcurveto{\pgfqpoint{3.178141in}{4.035397in}}{\pgfqpoint{3.168465in}{4.031389in}}{\pgfqpoint{3.161333in}{4.024256in}}%
\pgfpathcurveto{\pgfqpoint{3.154200in}{4.017123in}}{\pgfqpoint{3.150192in}{4.007448in}}{\pgfqpoint{3.150192in}{3.997361in}}%
\pgfpathcurveto{\pgfqpoint{3.150192in}{3.987273in}}{\pgfqpoint{3.154200in}{3.977598in}}{\pgfqpoint{3.161333in}{3.970465in}}%
\pgfpathcurveto{\pgfqpoint{3.168465in}{3.963332in}}{\pgfqpoint{3.178141in}{3.959324in}}{\pgfqpoint{3.188228in}{3.959324in}}%
\pgfpathclose%
\pgfusepath{stroke,fill}%
\end{pgfscope}%
\begin{pgfscope}%
\pgfpathrectangle{\pgfqpoint{0.800000in}{0.528000in}}{\pgfqpoint{4.960000in}{3.696000in}} %
\pgfusepath{clip}%
\pgfsetbuttcap%
\pgfsetroundjoin%
\definecolor{currentfill}{rgb}{0.121569,0.466667,0.705882}%
\pgfsetfillcolor{currentfill}%
\pgfsetlinewidth{1.003750pt}%
\definecolor{currentstroke}{rgb}{0.121569,0.466667,0.705882}%
\pgfsetstrokecolor{currentstroke}%
\pgfsetdash{}{0pt}%
\pgfpathmoveto{\pgfqpoint{4.207830in}{0.683357in}}%
\pgfpathcurveto{\pgfqpoint{4.217917in}{0.683357in}}{\pgfqpoint{4.227593in}{0.687364in}}{\pgfqpoint{4.234726in}{0.694497in}}%
\pgfpathcurveto{\pgfqpoint{4.241858in}{0.701630in}}{\pgfqpoint{4.245866in}{0.711306in}}{\pgfqpoint{4.245866in}{0.721393in}}%
\pgfpathcurveto{\pgfqpoint{4.245866in}{0.731480in}}{\pgfqpoint{4.241858in}{0.741156in}}{\pgfqpoint{4.234726in}{0.748289in}}%
\pgfpathcurveto{\pgfqpoint{4.227593in}{0.755422in}}{\pgfqpoint{4.217917in}{0.759429in}}{\pgfqpoint{4.207830in}{0.759429in}}%
\pgfpathcurveto{\pgfqpoint{4.197743in}{0.759429in}}{\pgfqpoint{4.188067in}{0.755422in}}{\pgfqpoint{4.180934in}{0.748289in}}%
\pgfpathcurveto{\pgfqpoint{4.173801in}{0.741156in}}{\pgfqpoint{4.169794in}{0.731480in}}{\pgfqpoint{4.169794in}{0.721393in}}%
\pgfpathcurveto{\pgfqpoint{4.169794in}{0.711306in}}{\pgfqpoint{4.173801in}{0.701630in}}{\pgfqpoint{4.180934in}{0.694497in}}%
\pgfpathcurveto{\pgfqpoint{4.188067in}{0.687364in}}{\pgfqpoint{4.197743in}{0.683357in}}{\pgfqpoint{4.207830in}{0.683357in}}%
\pgfpathclose%
\pgfusepath{stroke,fill}%
\end{pgfscope}%
\begin{pgfscope}%
\pgfpathrectangle{\pgfqpoint{0.800000in}{0.528000in}}{\pgfqpoint{4.960000in}{3.696000in}} %
\pgfusepath{clip}%
\pgfsetbuttcap%
\pgfsetroundjoin%
\definecolor{currentfill}{rgb}{0.121569,0.466667,0.705882}%
\pgfsetfillcolor{currentfill}%
\pgfsetlinewidth{1.003750pt}%
\definecolor{currentstroke}{rgb}{0.121569,0.466667,0.705882}%
\pgfsetstrokecolor{currentstroke}%
\pgfsetdash{}{0pt}%
\pgfpathmoveto{\pgfqpoint{4.827101in}{1.237602in}}%
\pgfpathcurveto{\pgfqpoint{4.837189in}{1.237602in}}{\pgfqpoint{4.846864in}{1.241610in}}{\pgfqpoint{4.853997in}{1.248743in}}%
\pgfpathcurveto{\pgfqpoint{4.861130in}{1.255875in}}{\pgfqpoint{4.865138in}{1.265551in}}{\pgfqpoint{4.865138in}{1.275638in}}%
\pgfpathcurveto{\pgfqpoint{4.865138in}{1.285726in}}{\pgfqpoint{4.861130in}{1.295401in}}{\pgfqpoint{4.853997in}{1.302534in}}%
\pgfpathcurveto{\pgfqpoint{4.846864in}{1.309667in}}{\pgfqpoint{4.837189in}{1.313675in}}{\pgfqpoint{4.827101in}{1.313675in}}%
\pgfpathcurveto{\pgfqpoint{4.817014in}{1.313675in}}{\pgfqpoint{4.807338in}{1.309667in}}{\pgfqpoint{4.800206in}{1.302534in}}%
\pgfpathcurveto{\pgfqpoint{4.793073in}{1.295401in}}{\pgfqpoint{4.789065in}{1.285726in}}{\pgfqpoint{4.789065in}{1.275638in}}%
\pgfpathcurveto{\pgfqpoint{4.789065in}{1.265551in}}{\pgfqpoint{4.793073in}{1.255875in}}{\pgfqpoint{4.800206in}{1.248743in}}%
\pgfpathcurveto{\pgfqpoint{4.807338in}{1.241610in}}{\pgfqpoint{4.817014in}{1.237602in}}{\pgfqpoint{4.827101in}{1.237602in}}%
\pgfpathclose%
\pgfusepath{stroke,fill}%
\end{pgfscope}%
\begin{pgfscope}%
\pgfpathrectangle{\pgfqpoint{0.800000in}{0.528000in}}{\pgfqpoint{4.960000in}{3.696000in}} %
\pgfusepath{clip}%
\pgfsetbuttcap%
\pgfsetroundjoin%
\definecolor{currentfill}{rgb}{0.121569,0.466667,0.705882}%
\pgfsetfillcolor{currentfill}%
\pgfsetlinewidth{1.003750pt}%
\definecolor{currentstroke}{rgb}{0.121569,0.466667,0.705882}%
\pgfsetstrokecolor{currentstroke}%
\pgfsetdash{}{0pt}%
\pgfpathmoveto{\pgfqpoint{4.468060in}{3.772899in}}%
\pgfpathcurveto{\pgfqpoint{4.478147in}{3.772899in}}{\pgfqpoint{4.487822in}{3.776907in}}{\pgfqpoint{4.494955in}{3.784039in}}%
\pgfpathcurveto{\pgfqpoint{4.502088in}{3.791172in}}{\pgfqpoint{4.506096in}{3.800848in}}{\pgfqpoint{4.506096in}{3.810935in}}%
\pgfpathcurveto{\pgfqpoint{4.506096in}{3.821022in}}{\pgfqpoint{4.502088in}{3.830698in}}{\pgfqpoint{4.494955in}{3.837831in}}%
\pgfpathcurveto{\pgfqpoint{4.487822in}{3.844964in}}{\pgfqpoint{4.478147in}{3.848971in}}{\pgfqpoint{4.468060in}{3.848971in}}%
\pgfpathcurveto{\pgfqpoint{4.457972in}{3.848971in}}{\pgfqpoint{4.448297in}{3.844964in}}{\pgfqpoint{4.441164in}{3.837831in}}%
\pgfpathcurveto{\pgfqpoint{4.434031in}{3.830698in}}{\pgfqpoint{4.430023in}{3.821022in}}{\pgfqpoint{4.430023in}{3.810935in}}%
\pgfpathcurveto{\pgfqpoint{4.430023in}{3.800848in}}{\pgfqpoint{4.434031in}{3.791172in}}{\pgfqpoint{4.441164in}{3.784039in}}%
\pgfpathcurveto{\pgfqpoint{4.448297in}{3.776907in}}{\pgfqpoint{4.457972in}{3.772899in}}{\pgfqpoint{4.468060in}{3.772899in}}%
\pgfpathclose%
\pgfusepath{stroke,fill}%
\end{pgfscope}%
\begin{pgfscope}%
\pgfpathrectangle{\pgfqpoint{0.800000in}{0.528000in}}{\pgfqpoint{4.960000in}{3.696000in}} %
\pgfusepath{clip}%
\pgfsetbuttcap%
\pgfsetroundjoin%
\definecolor{currentfill}{rgb}{0.121569,0.466667,0.705882}%
\pgfsetfillcolor{currentfill}%
\pgfsetlinewidth{1.003750pt}%
\definecolor{currentstroke}{rgb}{0.121569,0.466667,0.705882}%
\pgfsetstrokecolor{currentstroke}%
\pgfsetdash{}{0pt}%
\pgfpathmoveto{\pgfqpoint{5.296265in}{1.544799in}}%
\pgfpathcurveto{\pgfqpoint{5.306352in}{1.544799in}}{\pgfqpoint{5.316027in}{1.548807in}}{\pgfqpoint{5.323160in}{1.555940in}}%
\pgfpathcurveto{\pgfqpoint{5.330293in}{1.563073in}}{\pgfqpoint{5.334301in}{1.572748in}}{\pgfqpoint{5.334301in}{1.582836in}}%
\pgfpathcurveto{\pgfqpoint{5.334301in}{1.592923in}}{\pgfqpoint{5.330293in}{1.602599in}}{\pgfqpoint{5.323160in}{1.609731in}}%
\pgfpathcurveto{\pgfqpoint{5.316027in}{1.616864in}}{\pgfqpoint{5.306352in}{1.620872in}}{\pgfqpoint{5.296265in}{1.620872in}}%
\pgfpathcurveto{\pgfqpoint{5.286177in}{1.620872in}}{\pgfqpoint{5.276502in}{1.616864in}}{\pgfqpoint{5.269369in}{1.609731in}}%
\pgfpathcurveto{\pgfqpoint{5.262236in}{1.602599in}}{\pgfqpoint{5.258228in}{1.592923in}}{\pgfqpoint{5.258228in}{1.582836in}}%
\pgfpathcurveto{\pgfqpoint{5.258228in}{1.572748in}}{\pgfqpoint{5.262236in}{1.563073in}}{\pgfqpoint{5.269369in}{1.555940in}}%
\pgfpathcurveto{\pgfqpoint{5.276502in}{1.548807in}}{\pgfqpoint{5.286177in}{1.544799in}}{\pgfqpoint{5.296265in}{1.544799in}}%
\pgfpathclose%
\pgfusepath{stroke,fill}%
\end{pgfscope}%
\begin{pgfscope}%
\pgfpathrectangle{\pgfqpoint{0.800000in}{0.528000in}}{\pgfqpoint{4.960000in}{3.696000in}} %
\pgfusepath{clip}%
\pgfsetbuttcap%
\pgfsetroundjoin%
\definecolor{currentfill}{rgb}{0.121569,0.466667,0.705882}%
\pgfsetfillcolor{currentfill}%
\pgfsetlinewidth{1.003750pt}%
\definecolor{currentstroke}{rgb}{0.121569,0.466667,0.705882}%
\pgfsetstrokecolor{currentstroke}%
\pgfsetdash{}{0pt}%
\pgfpathmoveto{\pgfqpoint{5.528280in}{2.502909in}}%
\pgfpathcurveto{\pgfqpoint{5.538367in}{2.502909in}}{\pgfqpoint{5.548043in}{2.506917in}}{\pgfqpoint{5.555175in}{2.514049in}}%
\pgfpathcurveto{\pgfqpoint{5.562308in}{2.521182in}}{\pgfqpoint{5.566316in}{2.530858in}}{\pgfqpoint{5.566316in}{2.540945in}}%
\pgfpathcurveto{\pgfqpoint{5.566316in}{2.551033in}}{\pgfqpoint{5.562308in}{2.560708in}}{\pgfqpoint{5.555175in}{2.567841in}}%
\pgfpathcurveto{\pgfqpoint{5.548043in}{2.574974in}}{\pgfqpoint{5.538367in}{2.578982in}}{\pgfqpoint{5.528280in}{2.578982in}}%
\pgfpathcurveto{\pgfqpoint{5.518192in}{2.578982in}}{\pgfqpoint{5.508517in}{2.574974in}}{\pgfqpoint{5.501384in}{2.567841in}}%
\pgfpathcurveto{\pgfqpoint{5.494251in}{2.560708in}}{\pgfqpoint{5.490243in}{2.551033in}}{\pgfqpoint{5.490243in}{2.540945in}}%
\pgfpathcurveto{\pgfqpoint{5.490243in}{2.530858in}}{\pgfqpoint{5.494251in}{2.521182in}}{\pgfqpoint{5.501384in}{2.514049in}}%
\pgfpathcurveto{\pgfqpoint{5.508517in}{2.506917in}}{\pgfqpoint{5.518192in}{2.502909in}}{\pgfqpoint{5.528280in}{2.502909in}}%
\pgfpathclose%
\pgfusepath{stroke,fill}%
\end{pgfscope}%
\begin{pgfscope}%
\pgfsetbuttcap%
\pgfsetroundjoin%
\definecolor{currentfill}{rgb}{0.000000,0.000000,0.000000}%
\pgfsetfillcolor{currentfill}%
\pgfsetlinewidth{0.803000pt}%
\definecolor{currentstroke}{rgb}{0.000000,0.000000,0.000000}%
\pgfsetstrokecolor{currentstroke}%
\pgfsetdash{}{0pt}%
\pgfsys@defobject{currentmarker}{\pgfqpoint{0.000000in}{-0.048611in}}{\pgfqpoint{0.000000in}{0.000000in}}{%
\pgfpathmoveto{\pgfqpoint{0.000000in}{0.000000in}}%
\pgfpathlineto{\pgfqpoint{0.000000in}{-0.048611in}}%
\pgfusepath{stroke,fill}%
}%
\begin{pgfscope}%
\pgfsys@transformshift{1.038125in}{0.528000in}%
\pgfsys@useobject{currentmarker}{}%
\end{pgfscope}%
\end{pgfscope}%
\begin{pgfscope}%
\pgftext[x=1.038125in,y=0.430778in,,top]{\rmfamily\fontsize{10.000000}{12.000000}\selectfont \(\displaystyle -2.0\)}%
\end{pgfscope}%
\begin{pgfscope}%
\pgfsetbuttcap%
\pgfsetroundjoin%
\definecolor{currentfill}{rgb}{0.000000,0.000000,0.000000}%
\pgfsetfillcolor{currentfill}%
\pgfsetlinewidth{0.803000pt}%
\definecolor{currentstroke}{rgb}{0.000000,0.000000,0.000000}%
\pgfsetstrokecolor{currentstroke}%
\pgfsetdash{}{0pt}%
\pgfsys@defobject{currentmarker}{\pgfqpoint{0.000000in}{-0.048611in}}{\pgfqpoint{0.000000in}{0.000000in}}{%
\pgfpathmoveto{\pgfqpoint{0.000000in}{0.000000in}}%
\pgfpathlineto{\pgfqpoint{0.000000in}{-0.048611in}}%
\pgfusepath{stroke,fill}%
}%
\begin{pgfscope}%
\pgfsys@transformshift{1.605537in}{0.528000in}%
\pgfsys@useobject{currentmarker}{}%
\end{pgfscope}%
\end{pgfscope}%
\begin{pgfscope}%
\pgftext[x=1.605537in,y=0.430778in,,top]{\rmfamily\fontsize{10.000000}{12.000000}\selectfont \(\displaystyle -1.5\)}%
\end{pgfscope}%
\begin{pgfscope}%
\pgfsetbuttcap%
\pgfsetroundjoin%
\definecolor{currentfill}{rgb}{0.000000,0.000000,0.000000}%
\pgfsetfillcolor{currentfill}%
\pgfsetlinewidth{0.803000pt}%
\definecolor{currentstroke}{rgb}{0.000000,0.000000,0.000000}%
\pgfsetstrokecolor{currentstroke}%
\pgfsetdash{}{0pt}%
\pgfsys@defobject{currentmarker}{\pgfqpoint{0.000000in}{-0.048611in}}{\pgfqpoint{0.000000in}{0.000000in}}{%
\pgfpathmoveto{\pgfqpoint{0.000000in}{0.000000in}}%
\pgfpathlineto{\pgfqpoint{0.000000in}{-0.048611in}}%
\pgfusepath{stroke,fill}%
}%
\begin{pgfscope}%
\pgfsys@transformshift{2.172949in}{0.528000in}%
\pgfsys@useobject{currentmarker}{}%
\end{pgfscope}%
\end{pgfscope}%
\begin{pgfscope}%
\pgftext[x=2.172949in,y=0.430778in,,top]{\rmfamily\fontsize{10.000000}{12.000000}\selectfont \(\displaystyle -1.0\)}%
\end{pgfscope}%
\begin{pgfscope}%
\pgfsetbuttcap%
\pgfsetroundjoin%
\definecolor{currentfill}{rgb}{0.000000,0.000000,0.000000}%
\pgfsetfillcolor{currentfill}%
\pgfsetlinewidth{0.803000pt}%
\definecolor{currentstroke}{rgb}{0.000000,0.000000,0.000000}%
\pgfsetstrokecolor{currentstroke}%
\pgfsetdash{}{0pt}%
\pgfsys@defobject{currentmarker}{\pgfqpoint{0.000000in}{-0.048611in}}{\pgfqpoint{0.000000in}{0.000000in}}{%
\pgfpathmoveto{\pgfqpoint{0.000000in}{0.000000in}}%
\pgfpathlineto{\pgfqpoint{0.000000in}{-0.048611in}}%
\pgfusepath{stroke,fill}%
}%
\begin{pgfscope}%
\pgfsys@transformshift{2.740361in}{0.528000in}%
\pgfsys@useobject{currentmarker}{}%
\end{pgfscope}%
\end{pgfscope}%
\begin{pgfscope}%
\pgftext[x=2.740361in,y=0.430778in,,top]{\rmfamily\fontsize{10.000000}{12.000000}\selectfont \(\displaystyle -0.5\)}%
\end{pgfscope}%
\begin{pgfscope}%
\pgfsetbuttcap%
\pgfsetroundjoin%
\definecolor{currentfill}{rgb}{0.000000,0.000000,0.000000}%
\pgfsetfillcolor{currentfill}%
\pgfsetlinewidth{0.803000pt}%
\definecolor{currentstroke}{rgb}{0.000000,0.000000,0.000000}%
\pgfsetstrokecolor{currentstroke}%
\pgfsetdash{}{0pt}%
\pgfsys@defobject{currentmarker}{\pgfqpoint{0.000000in}{-0.048611in}}{\pgfqpoint{0.000000in}{0.000000in}}{%
\pgfpathmoveto{\pgfqpoint{0.000000in}{0.000000in}}%
\pgfpathlineto{\pgfqpoint{0.000000in}{-0.048611in}}%
\pgfusepath{stroke,fill}%
}%
\begin{pgfscope}%
\pgfsys@transformshift{3.307774in}{0.528000in}%
\pgfsys@useobject{currentmarker}{}%
\end{pgfscope}%
\end{pgfscope}%
\begin{pgfscope}%
\pgftext[x=3.307774in,y=0.430778in,,top]{\rmfamily\fontsize{10.000000}{12.000000}\selectfont \(\displaystyle 0.0\)}%
\end{pgfscope}%
\begin{pgfscope}%
\pgfsetbuttcap%
\pgfsetroundjoin%
\definecolor{currentfill}{rgb}{0.000000,0.000000,0.000000}%
\pgfsetfillcolor{currentfill}%
\pgfsetlinewidth{0.803000pt}%
\definecolor{currentstroke}{rgb}{0.000000,0.000000,0.000000}%
\pgfsetstrokecolor{currentstroke}%
\pgfsetdash{}{0pt}%
\pgfsys@defobject{currentmarker}{\pgfqpoint{0.000000in}{-0.048611in}}{\pgfqpoint{0.000000in}{0.000000in}}{%
\pgfpathmoveto{\pgfqpoint{0.000000in}{0.000000in}}%
\pgfpathlineto{\pgfqpoint{0.000000in}{-0.048611in}}%
\pgfusepath{stroke,fill}%
}%
\begin{pgfscope}%
\pgfsys@transformshift{3.875186in}{0.528000in}%
\pgfsys@useobject{currentmarker}{}%
\end{pgfscope}%
\end{pgfscope}%
\begin{pgfscope}%
\pgftext[x=3.875186in,y=0.430778in,,top]{\rmfamily\fontsize{10.000000}{12.000000}\selectfont \(\displaystyle 0.5\)}%
\end{pgfscope}%
\begin{pgfscope}%
\pgfsetbuttcap%
\pgfsetroundjoin%
\definecolor{currentfill}{rgb}{0.000000,0.000000,0.000000}%
\pgfsetfillcolor{currentfill}%
\pgfsetlinewidth{0.803000pt}%
\definecolor{currentstroke}{rgb}{0.000000,0.000000,0.000000}%
\pgfsetstrokecolor{currentstroke}%
\pgfsetdash{}{0pt}%
\pgfsys@defobject{currentmarker}{\pgfqpoint{0.000000in}{-0.048611in}}{\pgfqpoint{0.000000in}{0.000000in}}{%
\pgfpathmoveto{\pgfqpoint{0.000000in}{0.000000in}}%
\pgfpathlineto{\pgfqpoint{0.000000in}{-0.048611in}}%
\pgfusepath{stroke,fill}%
}%
\begin{pgfscope}%
\pgfsys@transformshift{4.442598in}{0.528000in}%
\pgfsys@useobject{currentmarker}{}%
\end{pgfscope}%
\end{pgfscope}%
\begin{pgfscope}%
\pgftext[x=4.442598in,y=0.430778in,,top]{\rmfamily\fontsize{10.000000}{12.000000}\selectfont \(\displaystyle 1.0\)}%
\end{pgfscope}%
\begin{pgfscope}%
\pgfsetbuttcap%
\pgfsetroundjoin%
\definecolor{currentfill}{rgb}{0.000000,0.000000,0.000000}%
\pgfsetfillcolor{currentfill}%
\pgfsetlinewidth{0.803000pt}%
\definecolor{currentstroke}{rgb}{0.000000,0.000000,0.000000}%
\pgfsetstrokecolor{currentstroke}%
\pgfsetdash{}{0pt}%
\pgfsys@defobject{currentmarker}{\pgfqpoint{0.000000in}{-0.048611in}}{\pgfqpoint{0.000000in}{0.000000in}}{%
\pgfpathmoveto{\pgfqpoint{0.000000in}{0.000000in}}%
\pgfpathlineto{\pgfqpoint{0.000000in}{-0.048611in}}%
\pgfusepath{stroke,fill}%
}%
\begin{pgfscope}%
\pgfsys@transformshift{5.010010in}{0.528000in}%
\pgfsys@useobject{currentmarker}{}%
\end{pgfscope}%
\end{pgfscope}%
\begin{pgfscope}%
\pgftext[x=5.010010in,y=0.430778in,,top]{\rmfamily\fontsize{10.000000}{12.000000}\selectfont \(\displaystyle 1.5\)}%
\end{pgfscope}%
\begin{pgfscope}%
\pgfsetbuttcap%
\pgfsetroundjoin%
\definecolor{currentfill}{rgb}{0.000000,0.000000,0.000000}%
\pgfsetfillcolor{currentfill}%
\pgfsetlinewidth{0.803000pt}%
\definecolor{currentstroke}{rgb}{0.000000,0.000000,0.000000}%
\pgfsetstrokecolor{currentstroke}%
\pgfsetdash{}{0pt}%
\pgfsys@defobject{currentmarker}{\pgfqpoint{0.000000in}{-0.048611in}}{\pgfqpoint{0.000000in}{0.000000in}}{%
\pgfpathmoveto{\pgfqpoint{0.000000in}{0.000000in}}%
\pgfpathlineto{\pgfqpoint{0.000000in}{-0.048611in}}%
\pgfusepath{stroke,fill}%
}%
\begin{pgfscope}%
\pgfsys@transformshift{5.577422in}{0.528000in}%
\pgfsys@useobject{currentmarker}{}%
\end{pgfscope}%
\end{pgfscope}%
\begin{pgfscope}%
\pgftext[x=5.577422in,y=0.430778in,,top]{\rmfamily\fontsize{10.000000}{12.000000}\selectfont \(\displaystyle 2.0\)}%
\end{pgfscope}%
\begin{pgfscope}%
\pgfsetbuttcap%
\pgfsetroundjoin%
\definecolor{currentfill}{rgb}{0.000000,0.000000,0.000000}%
\pgfsetfillcolor{currentfill}%
\pgfsetlinewidth{0.803000pt}%
\definecolor{currentstroke}{rgb}{0.000000,0.000000,0.000000}%
\pgfsetstrokecolor{currentstroke}%
\pgfsetdash{}{0pt}%
\pgfsys@defobject{currentmarker}{\pgfqpoint{-0.048611in}{0.000000in}}{\pgfqpoint{0.000000in}{0.000000in}}{%
\pgfpathmoveto{\pgfqpoint{0.000000in}{0.000000in}}%
\pgfpathlineto{\pgfqpoint{-0.048611in}{0.000000in}}%
\pgfusepath{stroke,fill}%
}%
\begin{pgfscope}%
\pgfsys@transformshift{0.800000in}{1.007586in}%
\pgfsys@useobject{currentmarker}{}%
\end{pgfscope}%
\end{pgfscope}%
\begin{pgfscope}%
\pgftext[x=0.417283in,y=0.959392in,left,base]{\rmfamily\fontsize{10.000000}{12.000000}\selectfont \(\displaystyle -0.4\)}%
\end{pgfscope}%
\begin{pgfscope}%
\pgfsetbuttcap%
\pgfsetroundjoin%
\definecolor{currentfill}{rgb}{0.000000,0.000000,0.000000}%
\pgfsetfillcolor{currentfill}%
\pgfsetlinewidth{0.803000pt}%
\definecolor{currentstroke}{rgb}{0.000000,0.000000,0.000000}%
\pgfsetstrokecolor{currentstroke}%
\pgfsetdash{}{0pt}%
\pgfsys@defobject{currentmarker}{\pgfqpoint{-0.048611in}{0.000000in}}{\pgfqpoint{0.000000in}{0.000000in}}{%
\pgfpathmoveto{\pgfqpoint{0.000000in}{0.000000in}}%
\pgfpathlineto{\pgfqpoint{-0.048611in}{0.000000in}}%
\pgfusepath{stroke,fill}%
}%
\begin{pgfscope}%
\pgfsys@transformshift{0.800000in}{1.692988in}%
\pgfsys@useobject{currentmarker}{}%
\end{pgfscope}%
\end{pgfscope}%
\begin{pgfscope}%
\pgftext[x=0.417283in,y=1.644794in,left,base]{\rmfamily\fontsize{10.000000}{12.000000}\selectfont \(\displaystyle -0.2\)}%
\end{pgfscope}%
\begin{pgfscope}%
\pgfsetbuttcap%
\pgfsetroundjoin%
\definecolor{currentfill}{rgb}{0.000000,0.000000,0.000000}%
\pgfsetfillcolor{currentfill}%
\pgfsetlinewidth{0.803000pt}%
\definecolor{currentstroke}{rgb}{0.000000,0.000000,0.000000}%
\pgfsetstrokecolor{currentstroke}%
\pgfsetdash{}{0pt}%
\pgfsys@defobject{currentmarker}{\pgfqpoint{-0.048611in}{0.000000in}}{\pgfqpoint{0.000000in}{0.000000in}}{%
\pgfpathmoveto{\pgfqpoint{0.000000in}{0.000000in}}%
\pgfpathlineto{\pgfqpoint{-0.048611in}{0.000000in}}%
\pgfusepath{stroke,fill}%
}%
\begin{pgfscope}%
\pgfsys@transformshift{0.800000in}{2.378390in}%
\pgfsys@useobject{currentmarker}{}%
\end{pgfscope}%
\end{pgfscope}%
\begin{pgfscope}%
\pgftext[x=0.525308in,y=2.330196in,left,base]{\rmfamily\fontsize{10.000000}{12.000000}\selectfont \(\displaystyle 0.0\)}%
\end{pgfscope}%
\begin{pgfscope}%
\pgfsetbuttcap%
\pgfsetroundjoin%
\definecolor{currentfill}{rgb}{0.000000,0.000000,0.000000}%
\pgfsetfillcolor{currentfill}%
\pgfsetlinewidth{0.803000pt}%
\definecolor{currentstroke}{rgb}{0.000000,0.000000,0.000000}%
\pgfsetstrokecolor{currentstroke}%
\pgfsetdash{}{0pt}%
\pgfsys@defobject{currentmarker}{\pgfqpoint{-0.048611in}{0.000000in}}{\pgfqpoint{0.000000in}{0.000000in}}{%
\pgfpathmoveto{\pgfqpoint{0.000000in}{0.000000in}}%
\pgfpathlineto{\pgfqpoint{-0.048611in}{0.000000in}}%
\pgfusepath{stroke,fill}%
}%
\begin{pgfscope}%
\pgfsys@transformshift{0.800000in}{3.063792in}%
\pgfsys@useobject{currentmarker}{}%
\end{pgfscope}%
\end{pgfscope}%
\begin{pgfscope}%
\pgftext[x=0.525308in,y=3.015598in,left,base]{\rmfamily\fontsize{10.000000}{12.000000}\selectfont \(\displaystyle 0.2\)}%
\end{pgfscope}%
\begin{pgfscope}%
\pgfsetbuttcap%
\pgfsetroundjoin%
\definecolor{currentfill}{rgb}{0.000000,0.000000,0.000000}%
\pgfsetfillcolor{currentfill}%
\pgfsetlinewidth{0.803000pt}%
\definecolor{currentstroke}{rgb}{0.000000,0.000000,0.000000}%
\pgfsetstrokecolor{currentstroke}%
\pgfsetdash{}{0pt}%
\pgfsys@defobject{currentmarker}{\pgfqpoint{-0.048611in}{0.000000in}}{\pgfqpoint{0.000000in}{0.000000in}}{%
\pgfpathmoveto{\pgfqpoint{0.000000in}{0.000000in}}%
\pgfpathlineto{\pgfqpoint{-0.048611in}{0.000000in}}%
\pgfusepath{stroke,fill}%
}%
\begin{pgfscope}%
\pgfsys@transformshift{0.800000in}{3.749194in}%
\pgfsys@useobject{currentmarker}{}%
\end{pgfscope}%
\end{pgfscope}%
\begin{pgfscope}%
\pgftext[x=0.525308in,y=3.701000in,left,base]{\rmfamily\fontsize{10.000000}{12.000000}\selectfont \(\displaystyle 0.4\)}%
\end{pgfscope}%
\begin{pgfscope}%
\pgfsetrectcap%
\pgfsetmiterjoin%
\pgfsetlinewidth{0.803000pt}%
\definecolor{currentstroke}{rgb}{0.000000,0.000000,0.000000}%
\pgfsetstrokecolor{currentstroke}%
\pgfsetdash{}{0pt}%
\pgfpathmoveto{\pgfqpoint{0.800000in}{0.528000in}}%
\pgfpathlineto{\pgfqpoint{0.800000in}{4.224000in}}%
\pgfusepath{stroke}%
\end{pgfscope}%
\begin{pgfscope}%
\pgfsetrectcap%
\pgfsetmiterjoin%
\pgfsetlinewidth{0.803000pt}%
\definecolor{currentstroke}{rgb}{0.000000,0.000000,0.000000}%
\pgfsetstrokecolor{currentstroke}%
\pgfsetdash{}{0pt}%
\pgfpathmoveto{\pgfqpoint{5.760000in}{0.528000in}}%
\pgfpathlineto{\pgfqpoint{5.760000in}{4.224000in}}%
\pgfusepath{stroke}%
\end{pgfscope}%
\begin{pgfscope}%
\pgfsetrectcap%
\pgfsetmiterjoin%
\pgfsetlinewidth{0.803000pt}%
\definecolor{currentstroke}{rgb}{0.000000,0.000000,0.000000}%
\pgfsetstrokecolor{currentstroke}%
\pgfsetdash{}{0pt}%
\pgfpathmoveto{\pgfqpoint{0.800000in}{0.528000in}}%
\pgfpathlineto{\pgfqpoint{5.760000in}{0.528000in}}%
\pgfusepath{stroke}%
\end{pgfscope}%
\begin{pgfscope}%
\pgfsetrectcap%
\pgfsetmiterjoin%
\pgfsetlinewidth{0.803000pt}%
\definecolor{currentstroke}{rgb}{0.000000,0.000000,0.000000}%
\pgfsetstrokecolor{currentstroke}%
\pgfsetdash{}{0pt}%
\pgfpathmoveto{\pgfqpoint{0.800000in}{4.224000in}}%
\pgfpathlineto{\pgfqpoint{5.760000in}{4.224000in}}%
\pgfusepath{stroke}%
\end{pgfscope}%
\end{pgfpicture}%
\makeatother%
\endgroup%
}  
\caption{Division on points by a grid} \label{Fig:Div}
\end{figure}

Due to the fact that our data set have a good geometry structure, we inherit the metric from original space as the metric at all scale.

\subsubsection{Propagating and refining}

Propagation strategies is used to determine which path at the next scale should be included when the optimal transport problem at current scale have already solved. A naive approach is to include the include only paths at scale $j+1$ whose source and targets are children of source and targets of non-zero paths at scale $j$. However, all paths in optimal transport plan at scale $j+1$ can not be induced by a non-zero path at scale $j$, this naive approach will lead to accumulate loss of accuracy.

Due to the good geometry structure of the data set, we have a intuition that paths at scale $j+1$ should be close to paths whose source and target are derived from source and targets in the scale $j$. The word ``close'' means that the source and targets are all close.

We introduce the capacity constraint propagation to include possible paths at scale $j+1$ as much as possible. As a standard LP problem, optimal transport problem have inequality constraints $s_{ij}\geq 0$, which implies a series of trivial constraints:
\begin{equation}
s_{ij} \leq \min\{\mu_i, \nu_j\}
\end{equation}
We modify those trivial constraints with a parameter $\lambda$ in order to force the source to be transported to more targets, thus solution at scale $j$ will include more possible paths at scale $j+1$:
\begin{equation}
s_{ij}\leq \lambda\min\{\mu_i, \nu_j\}
\end{equation}
With these constraints, the amount of non-zero paths in the solution at scale $j$ will increase and more possible path and scale $j+1$ will be included. The parameter $\lambda$ is chosen from $[0.1, 0.9]$. The lower $\lambda$ is, the higher accuracy is and the longer algorithm takes.

In numerical experiments we have realized that undersize $\lambda$ may lead to a infeasible subproblem in some extreme case. So we modify the constraints into $s_{ij}\leq\lambda\mu_i$ in order to improve robustness a without loss of accuracy.

Additionally, it should be pointed out that capacity constraint is only used in propagation. At the finest scale (the original problem) the capacity are not restricted.

Refinement is another way to include more path at scale $j+1$ according to the solution at scale $j$. Taking advantage of the geometry structure of the data set, we introduce a kind of refinement which is called neighborhood refinement. As we had said before, paths at scale $j+1$ should be close to paths whose source and target are children of source and targets. So we can include all the paths whose sources and targets are with radius r of the sources and targets of a non-zero path at scale $j$ into $s_j$, then propagate all of them to scale $j+1$.

Actually using both propagating and refining are not necessary in all situation. If we choose an enough small $\lambda$ in the propagating, we may have already got enough paths at scale $j+1$ and we can save the time of refining. Refining has a time cost $O(n \log n)$ because we do not save all paths so we need to search paths in refining.

\subsection{Discussion}

The multiscale method is listed in Algorithm \ref{Alg:MS}.

\begin{algorithm}
\caption{Multiscale strategy for optimal transport problems} \label{Alg:MS}
\begin{algorithmic}
\REQUIRE Source $ \rbr{ X, \mu } $ and target $ \rbr{ Y, \nu } $ as images or point clouds
\STATE Coarsen the datium by grid or downsampling and get two chains $\{ X_j, \mu_j \}^J_{j=0}$,$\{Y_j, \nu_j\}^J_{j=0}$
\STATE Construct a path set include all paths from $\rbr{X_0, \mu_0}$ to $\rbr{Y_0, \nu_0}$
\FOR{$j$ from $0$ to $J - 1$}
\STATE Solve the optimal transport problem on the path set with capacity constraint at scale $j$
\STATE Use propagation and refinement strategy to construct the path set at scale $j + 1$
\ENDFOR
\STATE Solve the optimal transport problem without capacity constraints on the path set at scale $j$
\STATE Construct the optimal solution $s$
\end{algorithmic}
\end{algorithm}

Note that the fact that we add additional constraints to the original optimal transport problem, the subproblem we facing now is not a optimal transport problem . Furthermore, we only consider the transport on some paths so the matrix structure have been destroyed. So the solution can't be easily got by the former algorithm for optimal transport problem. In the original paper, the multiscale strategy is implemented using MOSEK and CPLEX. We implement multiscale strategy with MOSEK.

\section{Numerical results and interpretations} \label{Sec:NumRes}

We have tested our algorithms on three types of datasets:
\begin{partlist}
\item Randomly generated dataset;
\item Ellipses and Caffarelli's smoothness counter examples \parencite{Gerber2017};
\item DOTmark dataset \parencite{Schrieber2017}.
\end{partlist}

The randomly generated dataset consisted of two sets of points uniformly sampled from $ \sbr{ 0, 1 } \times \sbr{ 0, 1 } $. The weights $\mu$ and $\nu$ are randomly sampled from $ \sbr{ 0, 1 } $ and scaled to $ \sume{i}{1}{m}{\mu_i} = 1 $ and $ \sume{j}{1}{n}{\nu_j} = 1 $. Figure \ref{Fig:Random} shows an example.

\begin{figure}
\centering \scalebox{0.65}{%% Creator: Matplotlib, PGF backend
%%
%% To include the figure in your LaTeX document, write
%%   \input{<filename>.pgf}
%%
%% Make sure the required packages are loaded in your preamble
%%   \usepackage{pgf}
%%
%% Figures using additional raster images can only be included by \input if
%% they are in the same directory as the main LaTeX file. For loading figures
%% from other directories you can use the `import` package
%%   \usepackage{import}
%% and then include the figures with
%%   \import{<path to file>}{<filename>.pgf}
%%
%% Matplotlib used the following preamble
%%   \usepackage{fontspec}
%%
\begingroup%
\makeatletter%
\begin{pgfpicture}%
\pgfpathrectangle{\pgfpointorigin}{\pgfqpoint{6.400000in}{4.800000in}}%
\pgfusepath{use as bounding box, clip}%
\begin{pgfscope}%
\pgfsetbuttcap%
\pgfsetmiterjoin%
\definecolor{currentfill}{rgb}{1.000000,1.000000,1.000000}%
\pgfsetfillcolor{currentfill}%
\pgfsetlinewidth{0.000000pt}%
\definecolor{currentstroke}{rgb}{1.000000,1.000000,1.000000}%
\pgfsetstrokecolor{currentstroke}%
\pgfsetdash{}{0pt}%
\pgfpathmoveto{\pgfqpoint{0.000000in}{0.000000in}}%
\pgfpathlineto{\pgfqpoint{6.400000in}{0.000000in}}%
\pgfpathlineto{\pgfqpoint{6.400000in}{4.800000in}}%
\pgfpathlineto{\pgfqpoint{0.000000in}{4.800000in}}%
\pgfpathclose%
\pgfusepath{fill}%
\end{pgfscope}%
\begin{pgfscope}%
\pgfsetbuttcap%
\pgfsetmiterjoin%
\definecolor{currentfill}{rgb}{1.000000,1.000000,1.000000}%
\pgfsetfillcolor{currentfill}%
\pgfsetlinewidth{0.000000pt}%
\definecolor{currentstroke}{rgb}{0.000000,0.000000,0.000000}%
\pgfsetstrokecolor{currentstroke}%
\pgfsetstrokeopacity{0.000000}%
\pgfsetdash{}{0pt}%
\pgfpathmoveto{\pgfqpoint{1.065196in}{0.528000in}}%
\pgfpathlineto{\pgfqpoint{4.768000in}{0.528000in}}%
\pgfpathlineto{\pgfqpoint{4.768000in}{4.224000in}}%
\pgfpathlineto{\pgfqpoint{1.065196in}{4.224000in}}%
\pgfpathclose%
\pgfusepath{fill}%
\end{pgfscope}%
\begin{pgfscope}%
\pgfpathrectangle{\pgfqpoint{1.065196in}{0.528000in}}{\pgfqpoint{3.702804in}{3.696000in}} %
\pgfusepath{clip}%
\pgfsetbuttcap%
\pgfsetroundjoin%
\definecolor{currentfill}{rgb}{0.121569,0.466667,0.705882}%
\pgfsetfillcolor{currentfill}%
\pgfsetlinewidth{1.003750pt}%
\definecolor{currentstroke}{rgb}{0.121569,0.466667,0.705882}%
\pgfsetstrokecolor{currentstroke}%
\pgfsetdash{}{0pt}%
\pgfpathmoveto{\pgfqpoint{2.652528in}{3.046872in}}%
\pgfpathcurveto{\pgfqpoint{2.666420in}{3.046872in}}{\pgfqpoint{2.679745in}{3.052392in}}{\pgfqpoint{2.689569in}{3.062215in}}%
\pgfpathcurveto{\pgfqpoint{2.699392in}{3.072038in}}{\pgfqpoint{2.704912in}{3.085363in}}{\pgfqpoint{2.704912in}{3.099256in}}%
\pgfpathcurveto{\pgfqpoint{2.704912in}{3.113148in}}{\pgfqpoint{2.699392in}{3.126473in}}{\pgfqpoint{2.689569in}{3.136297in}}%
\pgfpathcurveto{\pgfqpoint{2.679745in}{3.146120in}}{\pgfqpoint{2.666420in}{3.151640in}}{\pgfqpoint{2.652528in}{3.151640in}}%
\pgfpathcurveto{\pgfqpoint{2.638635in}{3.151640in}}{\pgfqpoint{2.625310in}{3.146120in}}{\pgfqpoint{2.615487in}{3.136297in}}%
\pgfpathcurveto{\pgfqpoint{2.605663in}{3.126473in}}{\pgfqpoint{2.600144in}{3.113148in}}{\pgfqpoint{2.600144in}{3.099256in}}%
\pgfpathcurveto{\pgfqpoint{2.600144in}{3.085363in}}{\pgfqpoint{2.605663in}{3.072038in}}{\pgfqpoint{2.615487in}{3.062215in}}%
\pgfpathcurveto{\pgfqpoint{2.625310in}{3.052392in}}{\pgfqpoint{2.638635in}{3.046872in}}{\pgfqpoint{2.652528in}{3.046872in}}%
\pgfpathclose%
\pgfusepath{stroke,fill}%
\end{pgfscope}%
\begin{pgfscope}%
\pgfpathrectangle{\pgfqpoint{1.065196in}{0.528000in}}{\pgfqpoint{3.702804in}{3.696000in}} %
\pgfusepath{clip}%
\pgfsetbuttcap%
\pgfsetroundjoin%
\definecolor{currentfill}{rgb}{0.121569,0.466667,0.705882}%
\pgfsetfillcolor{currentfill}%
\pgfsetlinewidth{1.003750pt}%
\definecolor{currentstroke}{rgb}{0.121569,0.466667,0.705882}%
\pgfsetstrokecolor{currentstroke}%
\pgfsetdash{}{0pt}%
\pgfpathmoveto{\pgfqpoint{1.256808in}{1.651954in}}%
\pgfpathcurveto{\pgfqpoint{1.269525in}{1.651954in}}{\pgfqpoint{1.281723in}{1.657007in}}{\pgfqpoint{1.290716in}{1.665999in}}%
\pgfpathcurveto{\pgfqpoint{1.299708in}{1.674992in}}{\pgfqpoint{1.304760in}{1.687189in}}{\pgfqpoint{1.304760in}{1.699906in}}%
\pgfpathcurveto{\pgfqpoint{1.304760in}{1.712624in}}{\pgfqpoint{1.299708in}{1.724821in}}{\pgfqpoint{1.290716in}{1.733814in}}%
\pgfpathcurveto{\pgfqpoint{1.281723in}{1.742806in}}{\pgfqpoint{1.269525in}{1.747859in}}{\pgfqpoint{1.256808in}{1.747859in}}%
\pgfpathcurveto{\pgfqpoint{1.244091in}{1.747859in}}{\pgfqpoint{1.231893in}{1.742806in}}{\pgfqpoint{1.222901in}{1.733814in}}%
\pgfpathcurveto{\pgfqpoint{1.213909in}{1.724821in}}{\pgfqpoint{1.208856in}{1.712624in}}{\pgfqpoint{1.208856in}{1.699906in}}%
\pgfpathcurveto{\pgfqpoint{1.208856in}{1.687189in}}{\pgfqpoint{1.213909in}{1.674992in}}{\pgfqpoint{1.222901in}{1.665999in}}%
\pgfpathcurveto{\pgfqpoint{1.231893in}{1.657007in}}{\pgfqpoint{1.244091in}{1.651954in}}{\pgfqpoint{1.256808in}{1.651954in}}%
\pgfpathclose%
\pgfusepath{stroke,fill}%
\end{pgfscope}%
\begin{pgfscope}%
\pgfpathrectangle{\pgfqpoint{1.065196in}{0.528000in}}{\pgfqpoint{3.702804in}{3.696000in}} %
\pgfusepath{clip}%
\pgfsetbuttcap%
\pgfsetroundjoin%
\definecolor{currentfill}{rgb}{0.121569,0.466667,0.705882}%
\pgfsetfillcolor{currentfill}%
\pgfsetlinewidth{1.003750pt}%
\definecolor{currentstroke}{rgb}{0.121569,0.466667,0.705882}%
\pgfsetstrokecolor{currentstroke}%
\pgfsetdash{}{0pt}%
\pgfpathmoveto{\pgfqpoint{1.747733in}{0.987276in}}%
\pgfpathcurveto{\pgfqpoint{1.750283in}{0.987276in}}{\pgfqpoint{1.752729in}{0.988289in}}{\pgfqpoint{1.754532in}{0.990092in}}%
\pgfpathcurveto{\pgfqpoint{1.756335in}{0.991895in}}{\pgfqpoint{1.757348in}{0.994341in}}{\pgfqpoint{1.757348in}{0.996891in}}%
\pgfpathcurveto{\pgfqpoint{1.757348in}{0.999441in}}{\pgfqpoint{1.756335in}{1.001886in}}{\pgfqpoint{1.754532in}{1.003689in}}%
\pgfpathcurveto{\pgfqpoint{1.752729in}{1.005492in}}{\pgfqpoint{1.750283in}{1.006506in}}{\pgfqpoint{1.747733in}{1.006506in}}%
\pgfpathcurveto{\pgfqpoint{1.745184in}{1.006506in}}{\pgfqpoint{1.742738in}{1.005492in}}{\pgfqpoint{1.740935in}{1.003689in}}%
\pgfpathcurveto{\pgfqpoint{1.739132in}{1.001886in}}{\pgfqpoint{1.738118in}{0.999441in}}{\pgfqpoint{1.738118in}{0.996891in}}%
\pgfpathcurveto{\pgfqpoint{1.738118in}{0.994341in}}{\pgfqpoint{1.739132in}{0.991895in}}{\pgfqpoint{1.740935in}{0.990092in}}%
\pgfpathcurveto{\pgfqpoint{1.742738in}{0.988289in}}{\pgfqpoint{1.745184in}{0.987276in}}{\pgfqpoint{1.747733in}{0.987276in}}%
\pgfpathclose%
\pgfusepath{stroke,fill}%
\end{pgfscope}%
\begin{pgfscope}%
\pgfpathrectangle{\pgfqpoint{1.065196in}{0.528000in}}{\pgfqpoint{3.702804in}{3.696000in}} %
\pgfusepath{clip}%
\pgfsetbuttcap%
\pgfsetroundjoin%
\definecolor{currentfill}{rgb}{0.121569,0.466667,0.705882}%
\pgfsetfillcolor{currentfill}%
\pgfsetlinewidth{1.003750pt}%
\definecolor{currentstroke}{rgb}{0.121569,0.466667,0.705882}%
\pgfsetstrokecolor{currentstroke}%
\pgfsetdash{}{0pt}%
\pgfpathmoveto{\pgfqpoint{1.879986in}{1.799582in}}%
\pgfpathcurveto{\pgfqpoint{1.891931in}{1.799582in}}{\pgfqpoint{1.903389in}{1.804328in}}{\pgfqpoint{1.911836in}{1.812775in}}%
\pgfpathcurveto{\pgfqpoint{1.920283in}{1.821222in}}{\pgfqpoint{1.925029in}{1.832680in}}{\pgfqpoint{1.925029in}{1.844625in}}%
\pgfpathcurveto{\pgfqpoint{1.925029in}{1.856571in}}{\pgfqpoint{1.920283in}{1.868029in}}{\pgfqpoint{1.911836in}{1.876476in}}%
\pgfpathcurveto{\pgfqpoint{1.903389in}{1.884923in}}{\pgfqpoint{1.891931in}{1.889669in}}{\pgfqpoint{1.879986in}{1.889669in}}%
\pgfpathcurveto{\pgfqpoint{1.868040in}{1.889669in}}{\pgfqpoint{1.856582in}{1.884923in}}{\pgfqpoint{1.848135in}{1.876476in}}%
\pgfpathcurveto{\pgfqpoint{1.839688in}{1.868029in}}{\pgfqpoint{1.834942in}{1.856571in}}{\pgfqpoint{1.834942in}{1.844625in}}%
\pgfpathcurveto{\pgfqpoint{1.834942in}{1.832680in}}{\pgfqpoint{1.839688in}{1.821222in}}{\pgfqpoint{1.848135in}{1.812775in}}%
\pgfpathcurveto{\pgfqpoint{1.856582in}{1.804328in}}{\pgfqpoint{1.868040in}{1.799582in}}{\pgfqpoint{1.879986in}{1.799582in}}%
\pgfpathclose%
\pgfusepath{stroke,fill}%
\end{pgfscope}%
\begin{pgfscope}%
\pgfpathrectangle{\pgfqpoint{1.065196in}{0.528000in}}{\pgfqpoint{3.702804in}{3.696000in}} %
\pgfusepath{clip}%
\pgfsetbuttcap%
\pgfsetroundjoin%
\definecolor{currentfill}{rgb}{0.121569,0.466667,0.705882}%
\pgfsetfillcolor{currentfill}%
\pgfsetlinewidth{1.003750pt}%
\definecolor{currentstroke}{rgb}{0.121569,0.466667,0.705882}%
\pgfsetstrokecolor{currentstroke}%
\pgfsetdash{}{0pt}%
\pgfpathmoveto{\pgfqpoint{2.584720in}{2.455137in}}%
\pgfpathcurveto{\pgfqpoint{2.594391in}{2.455137in}}{\pgfqpoint{2.603668in}{2.458980in}}{\pgfqpoint{2.610507in}{2.465819in}}%
\pgfpathcurveto{\pgfqpoint{2.617346in}{2.472658in}}{\pgfqpoint{2.621188in}{2.481934in}}{\pgfqpoint{2.621188in}{2.491606in}}%
\pgfpathcurveto{\pgfqpoint{2.621188in}{2.501277in}}{\pgfqpoint{2.617346in}{2.510554in}}{\pgfqpoint{2.610507in}{2.517393in}}%
\pgfpathcurveto{\pgfqpoint{2.603668in}{2.524232in}}{\pgfqpoint{2.594391in}{2.528074in}}{\pgfqpoint{2.584720in}{2.528074in}}%
\pgfpathcurveto{\pgfqpoint{2.575048in}{2.528074in}}{\pgfqpoint{2.565772in}{2.524232in}}{\pgfqpoint{2.558933in}{2.517393in}}%
\pgfpathcurveto{\pgfqpoint{2.552094in}{2.510554in}}{\pgfqpoint{2.548251in}{2.501277in}}{\pgfqpoint{2.548251in}{2.491606in}}%
\pgfpathcurveto{\pgfqpoint{2.548251in}{2.481934in}}{\pgfqpoint{2.552094in}{2.472658in}}{\pgfqpoint{2.558933in}{2.465819in}}%
\pgfpathcurveto{\pgfqpoint{2.565772in}{2.458980in}}{\pgfqpoint{2.575048in}{2.455137in}}{\pgfqpoint{2.584720in}{2.455137in}}%
\pgfpathclose%
\pgfusepath{stroke,fill}%
\end{pgfscope}%
\begin{pgfscope}%
\pgfpathrectangle{\pgfqpoint{1.065196in}{0.528000in}}{\pgfqpoint{3.702804in}{3.696000in}} %
\pgfusepath{clip}%
\pgfsetbuttcap%
\pgfsetroundjoin%
\definecolor{currentfill}{rgb}{0.121569,0.466667,0.705882}%
\pgfsetfillcolor{currentfill}%
\pgfsetlinewidth{1.003750pt}%
\definecolor{currentstroke}{rgb}{0.121569,0.466667,0.705882}%
\pgfsetstrokecolor{currentstroke}%
\pgfsetdash{}{0pt}%
\pgfpathmoveto{\pgfqpoint{2.659801in}{2.929673in}}%
\pgfpathcurveto{\pgfqpoint{2.673607in}{2.929673in}}{\pgfqpoint{2.686849in}{2.935159in}}{\pgfqpoint{2.696612in}{2.944921in}}%
\pgfpathcurveto{\pgfqpoint{2.706374in}{2.954683in}}{\pgfqpoint{2.711859in}{2.967926in}}{\pgfqpoint{2.711859in}{2.981732in}}%
\pgfpathcurveto{\pgfqpoint{2.711859in}{2.995538in}}{\pgfqpoint{2.706374in}{3.008780in}}{\pgfqpoint{2.696612in}{3.018542in}}%
\pgfpathcurveto{\pgfqpoint{2.686849in}{3.028305in}}{\pgfqpoint{2.673607in}{3.033790in}}{\pgfqpoint{2.659801in}{3.033790in}}%
\pgfpathcurveto{\pgfqpoint{2.645995in}{3.033790in}}{\pgfqpoint{2.632752in}{3.028305in}}{\pgfqpoint{2.622990in}{3.018542in}}%
\pgfpathcurveto{\pgfqpoint{2.613228in}{3.008780in}}{\pgfqpoint{2.607742in}{2.995538in}}{\pgfqpoint{2.607742in}{2.981732in}}%
\pgfpathcurveto{\pgfqpoint{2.607742in}{2.967926in}}{\pgfqpoint{2.613228in}{2.954683in}}{\pgfqpoint{2.622990in}{2.944921in}}%
\pgfpathcurveto{\pgfqpoint{2.632752in}{2.935159in}}{\pgfqpoint{2.645995in}{2.929673in}}{\pgfqpoint{2.659801in}{2.929673in}}%
\pgfpathclose%
\pgfusepath{stroke,fill}%
\end{pgfscope}%
\begin{pgfscope}%
\pgfpathrectangle{\pgfqpoint{1.065196in}{0.528000in}}{\pgfqpoint{3.702804in}{3.696000in}} %
\pgfusepath{clip}%
\pgfsetbuttcap%
\pgfsetroundjoin%
\definecolor{currentfill}{rgb}{0.121569,0.466667,0.705882}%
\pgfsetfillcolor{currentfill}%
\pgfsetlinewidth{1.003750pt}%
\definecolor{currentstroke}{rgb}{0.121569,0.466667,0.705882}%
\pgfsetstrokecolor{currentstroke}%
\pgfsetdash{}{0pt}%
\pgfpathmoveto{\pgfqpoint{1.940889in}{3.602348in}}%
\pgfpathcurveto{\pgfqpoint{1.947563in}{3.602348in}}{\pgfqpoint{1.953964in}{3.604999in}}{\pgfqpoint{1.958684in}{3.609719in}}%
\pgfpathcurveto{\pgfqpoint{1.963403in}{3.614438in}}{\pgfqpoint{1.966055in}{3.620839in}}{\pgfqpoint{1.966055in}{3.627514in}}%
\pgfpathcurveto{\pgfqpoint{1.966055in}{3.634188in}}{\pgfqpoint{1.963403in}{3.640589in}}{\pgfqpoint{1.958684in}{3.645308in}}%
\pgfpathcurveto{\pgfqpoint{1.953964in}{3.650028in}}{\pgfqpoint{1.947563in}{3.652679in}}{\pgfqpoint{1.940889in}{3.652679in}}%
\pgfpathcurveto{\pgfqpoint{1.934215in}{3.652679in}}{\pgfqpoint{1.927813in}{3.650028in}}{\pgfqpoint{1.923094in}{3.645308in}}%
\pgfpathcurveto{\pgfqpoint{1.918375in}{3.640589in}}{\pgfqpoint{1.915723in}{3.634188in}}{\pgfqpoint{1.915723in}{3.627514in}}%
\pgfpathcurveto{\pgfqpoint{1.915723in}{3.620839in}}{\pgfqpoint{1.918375in}{3.614438in}}{\pgfqpoint{1.923094in}{3.609719in}}%
\pgfpathcurveto{\pgfqpoint{1.927813in}{3.604999in}}{\pgfqpoint{1.934215in}{3.602348in}}{\pgfqpoint{1.940889in}{3.602348in}}%
\pgfpathclose%
\pgfusepath{stroke,fill}%
\end{pgfscope}%
\begin{pgfscope}%
\pgfpathrectangle{\pgfqpoint{1.065196in}{0.528000in}}{\pgfqpoint{3.702804in}{3.696000in}} %
\pgfusepath{clip}%
\pgfsetbuttcap%
\pgfsetroundjoin%
\definecolor{currentfill}{rgb}{0.121569,0.466667,0.705882}%
\pgfsetfillcolor{currentfill}%
\pgfsetlinewidth{1.003750pt}%
\definecolor{currentstroke}{rgb}{0.121569,0.466667,0.705882}%
\pgfsetstrokecolor{currentstroke}%
\pgfsetdash{}{0pt}%
\pgfpathmoveto{\pgfqpoint{1.348113in}{2.904707in}}%
\pgfpathcurveto{\pgfqpoint{1.355443in}{2.904707in}}{\pgfqpoint{1.362473in}{2.907619in}}{\pgfqpoint{1.367656in}{2.912802in}}%
\pgfpathcurveto{\pgfqpoint{1.372839in}{2.917985in}}{\pgfqpoint{1.375751in}{2.925015in}}{\pgfqpoint{1.375751in}{2.932345in}}%
\pgfpathcurveto{\pgfqpoint{1.375751in}{2.939675in}}{\pgfqpoint{1.372839in}{2.946705in}}{\pgfqpoint{1.367656in}{2.951888in}}%
\pgfpathcurveto{\pgfqpoint{1.362473in}{2.957071in}}{\pgfqpoint{1.355443in}{2.959983in}}{\pgfqpoint{1.348113in}{2.959983in}}%
\pgfpathcurveto{\pgfqpoint{1.340784in}{2.959983in}}{\pgfqpoint{1.333753in}{2.957071in}}{\pgfqpoint{1.328570in}{2.951888in}}%
\pgfpathcurveto{\pgfqpoint{1.323388in}{2.946705in}}{\pgfqpoint{1.320476in}{2.939675in}}{\pgfqpoint{1.320476in}{2.932345in}}%
\pgfpathcurveto{\pgfqpoint{1.320476in}{2.925015in}}{\pgfqpoint{1.323388in}{2.917985in}}{\pgfqpoint{1.328570in}{2.912802in}}%
\pgfpathcurveto{\pgfqpoint{1.333753in}{2.907619in}}{\pgfqpoint{1.340784in}{2.904707in}}{\pgfqpoint{1.348113in}{2.904707in}}%
\pgfpathclose%
\pgfusepath{stroke,fill}%
\end{pgfscope}%
\begin{pgfscope}%
\pgfpathrectangle{\pgfqpoint{1.065196in}{0.528000in}}{\pgfqpoint{3.702804in}{3.696000in}} %
\pgfusepath{clip}%
\pgfsetbuttcap%
\pgfsetroundjoin%
\definecolor{currentfill}{rgb}{0.121569,0.466667,0.705882}%
\pgfsetfillcolor{currentfill}%
\pgfsetlinewidth{1.003750pt}%
\definecolor{currentstroke}{rgb}{0.121569,0.466667,0.705882}%
\pgfsetstrokecolor{currentstroke}%
\pgfsetdash{}{0pt}%
\pgfpathmoveto{\pgfqpoint{2.653474in}{2.542872in}}%
\pgfpathcurveto{\pgfqpoint{2.657523in}{2.542872in}}{\pgfqpoint{2.661406in}{2.544480in}}{\pgfqpoint{2.664269in}{2.547343in}}%
\pgfpathcurveto{\pgfqpoint{2.667131in}{2.550205in}}{\pgfqpoint{2.668740in}{2.554089in}}{\pgfqpoint{2.668740in}{2.558137in}}%
\pgfpathcurveto{\pgfqpoint{2.668740in}{2.562185in}}{\pgfqpoint{2.667131in}{2.566068in}}{\pgfqpoint{2.664269in}{2.568931in}}%
\pgfpathcurveto{\pgfqpoint{2.661406in}{2.571793in}}{\pgfqpoint{2.657523in}{2.573402in}}{\pgfqpoint{2.653474in}{2.573402in}}%
\pgfpathcurveto{\pgfqpoint{2.649426in}{2.573402in}}{\pgfqpoint{2.645543in}{2.571793in}}{\pgfqpoint{2.642680in}{2.568931in}}%
\pgfpathcurveto{\pgfqpoint{2.639818in}{2.566068in}}{\pgfqpoint{2.638209in}{2.562185in}}{\pgfqpoint{2.638209in}{2.558137in}}%
\pgfpathcurveto{\pgfqpoint{2.638209in}{2.554089in}}{\pgfqpoint{2.639818in}{2.550205in}}{\pgfqpoint{2.642680in}{2.547343in}}%
\pgfpathcurveto{\pgfqpoint{2.645543in}{2.544480in}}{\pgfqpoint{2.649426in}{2.542872in}}{\pgfqpoint{2.653474in}{2.542872in}}%
\pgfpathclose%
\pgfusepath{stroke,fill}%
\end{pgfscope}%
\begin{pgfscope}%
\pgfpathrectangle{\pgfqpoint{1.065196in}{0.528000in}}{\pgfqpoint{3.702804in}{3.696000in}} %
\pgfusepath{clip}%
\pgfsetbuttcap%
\pgfsetroundjoin%
\definecolor{currentfill}{rgb}{0.121569,0.466667,0.705882}%
\pgfsetfillcolor{currentfill}%
\pgfsetlinewidth{1.003750pt}%
\definecolor{currentstroke}{rgb}{0.121569,0.466667,0.705882}%
\pgfsetstrokecolor{currentstroke}%
\pgfsetdash{}{0pt}%
\pgfpathmoveto{\pgfqpoint{1.726411in}{1.315950in}}%
\pgfpathcurveto{\pgfqpoint{1.735697in}{1.315950in}}{\pgfqpoint{1.744603in}{1.319639in}}{\pgfqpoint{1.751169in}{1.326205in}}%
\pgfpathcurveto{\pgfqpoint{1.757735in}{1.332771in}}{\pgfqpoint{1.761424in}{1.341677in}}{\pgfqpoint{1.761424in}{1.350963in}}%
\pgfpathcurveto{\pgfqpoint{1.761424in}{1.360248in}}{\pgfqpoint{1.757735in}{1.369155in}}{\pgfqpoint{1.751169in}{1.375720in}}%
\pgfpathcurveto{\pgfqpoint{1.744603in}{1.382286in}}{\pgfqpoint{1.735697in}{1.385975in}}{\pgfqpoint{1.726411in}{1.385975in}}%
\pgfpathcurveto{\pgfqpoint{1.717126in}{1.385975in}}{\pgfqpoint{1.708220in}{1.382286in}}{\pgfqpoint{1.701654in}{1.375720in}}%
\pgfpathcurveto{\pgfqpoint{1.695088in}{1.369155in}}{\pgfqpoint{1.691399in}{1.360248in}}{\pgfqpoint{1.691399in}{1.350963in}}%
\pgfpathcurveto{\pgfqpoint{1.691399in}{1.341677in}}{\pgfqpoint{1.695088in}{1.332771in}}{\pgfqpoint{1.701654in}{1.326205in}}%
\pgfpathcurveto{\pgfqpoint{1.708220in}{1.319639in}}{\pgfqpoint{1.717126in}{1.315950in}}{\pgfqpoint{1.726411in}{1.315950in}}%
\pgfpathclose%
\pgfusepath{stroke,fill}%
\end{pgfscope}%
\begin{pgfscope}%
\pgfpathrectangle{\pgfqpoint{1.065196in}{0.528000in}}{\pgfqpoint{3.702804in}{3.696000in}} %
\pgfusepath{clip}%
\pgfsetbuttcap%
\pgfsetroundjoin%
\definecolor{currentfill}{rgb}{0.121569,0.466667,0.705882}%
\pgfsetfillcolor{currentfill}%
\pgfsetlinewidth{1.003750pt}%
\definecolor{currentstroke}{rgb}{0.121569,0.466667,0.705882}%
\pgfsetstrokecolor{currentstroke}%
\pgfsetdash{}{0pt}%
\pgfpathmoveto{\pgfqpoint{3.937150in}{3.911639in}}%
\pgfpathcurveto{\pgfqpoint{3.941833in}{3.911639in}}{\pgfqpoint{3.946325in}{3.913500in}}{\pgfqpoint{3.949636in}{3.916811in}}%
\pgfpathcurveto{\pgfqpoint{3.952948in}{3.920123in}}{\pgfqpoint{3.954808in}{3.924614in}}{\pgfqpoint{3.954808in}{3.929297in}}%
\pgfpathcurveto{\pgfqpoint{3.954808in}{3.933980in}}{\pgfqpoint{3.952948in}{3.938472in}}{\pgfqpoint{3.949636in}{3.941783in}}%
\pgfpathcurveto{\pgfqpoint{3.946325in}{3.945095in}}{\pgfqpoint{3.941833in}{3.946955in}}{\pgfqpoint{3.937150in}{3.946955in}}%
\pgfpathcurveto{\pgfqpoint{3.932468in}{3.946955in}}{\pgfqpoint{3.927976in}{3.945095in}}{\pgfqpoint{3.924664in}{3.941783in}}%
\pgfpathcurveto{\pgfqpoint{3.921353in}{3.938472in}}{\pgfqpoint{3.919492in}{3.933980in}}{\pgfqpoint{3.919492in}{3.929297in}}%
\pgfpathcurveto{\pgfqpoint{3.919492in}{3.924614in}}{\pgfqpoint{3.921353in}{3.920123in}}{\pgfqpoint{3.924664in}{3.916811in}}%
\pgfpathcurveto{\pgfqpoint{3.927976in}{3.913500in}}{\pgfqpoint{3.932468in}{3.911639in}}{\pgfqpoint{3.937150in}{3.911639in}}%
\pgfpathclose%
\pgfusepath{stroke,fill}%
\end{pgfscope}%
\begin{pgfscope}%
\pgfpathrectangle{\pgfqpoint{1.065196in}{0.528000in}}{\pgfqpoint{3.702804in}{3.696000in}} %
\pgfusepath{clip}%
\pgfsetbuttcap%
\pgfsetroundjoin%
\definecolor{currentfill}{rgb}{0.121569,0.466667,0.705882}%
\pgfsetfillcolor{currentfill}%
\pgfsetlinewidth{1.003750pt}%
\definecolor{currentstroke}{rgb}{0.121569,0.466667,0.705882}%
\pgfsetstrokecolor{currentstroke}%
\pgfsetdash{}{0pt}%
\pgfpathmoveto{\pgfqpoint{2.305704in}{2.962936in}}%
\pgfpathcurveto{\pgfqpoint{2.316995in}{2.962936in}}{\pgfqpoint{2.327825in}{2.967422in}}{\pgfqpoint{2.335809in}{2.975406in}}%
\pgfpathcurveto{\pgfqpoint{2.343794in}{2.983390in}}{\pgfqpoint{2.348280in}{2.994220in}}{\pgfqpoint{2.348280in}{3.005511in}}%
\pgfpathcurveto{\pgfqpoint{2.348280in}{3.016803in}}{\pgfqpoint{2.343794in}{3.027633in}}{\pgfqpoint{2.335809in}{3.035617in}}%
\pgfpathcurveto{\pgfqpoint{2.327825in}{3.043601in}}{\pgfqpoint{2.316995in}{3.048087in}}{\pgfqpoint{2.305704in}{3.048087in}}%
\pgfpathcurveto{\pgfqpoint{2.294413in}{3.048087in}}{\pgfqpoint{2.283582in}{3.043601in}}{\pgfqpoint{2.275598in}{3.035617in}}%
\pgfpathcurveto{\pgfqpoint{2.267614in}{3.027633in}}{\pgfqpoint{2.263128in}{3.016803in}}{\pgfqpoint{2.263128in}{3.005511in}}%
\pgfpathcurveto{\pgfqpoint{2.263128in}{2.994220in}}{\pgfqpoint{2.267614in}{2.983390in}}{\pgfqpoint{2.275598in}{2.975406in}}%
\pgfpathcurveto{\pgfqpoint{2.283582in}{2.967422in}}{\pgfqpoint{2.294413in}{2.962936in}}{\pgfqpoint{2.305704in}{2.962936in}}%
\pgfpathclose%
\pgfusepath{stroke,fill}%
\end{pgfscope}%
\begin{pgfscope}%
\pgfpathrectangle{\pgfqpoint{1.065196in}{0.528000in}}{\pgfqpoint{3.702804in}{3.696000in}} %
\pgfusepath{clip}%
\pgfsetbuttcap%
\pgfsetroundjoin%
\definecolor{currentfill}{rgb}{0.121569,0.466667,0.705882}%
\pgfsetfillcolor{currentfill}%
\pgfsetlinewidth{1.003750pt}%
\definecolor{currentstroke}{rgb}{0.121569,0.466667,0.705882}%
\pgfsetstrokecolor{currentstroke}%
\pgfsetdash{}{0pt}%
\pgfpathmoveto{\pgfqpoint{4.190393in}{3.634794in}}%
\pgfpathcurveto{\pgfqpoint{4.203102in}{3.634794in}}{\pgfqpoint{4.215292in}{3.639843in}}{\pgfqpoint{4.224279in}{3.648830in}}%
\pgfpathcurveto{\pgfqpoint{4.233266in}{3.657817in}}{\pgfqpoint{4.238315in}{3.670007in}}{\pgfqpoint{4.238315in}{3.682716in}}%
\pgfpathcurveto{\pgfqpoint{4.238315in}{3.695425in}}{\pgfqpoint{4.233266in}{3.707615in}}{\pgfqpoint{4.224279in}{3.716602in}}%
\pgfpathcurveto{\pgfqpoint{4.215292in}{3.725589in}}{\pgfqpoint{4.203102in}{3.730638in}}{\pgfqpoint{4.190393in}{3.730638in}}%
\pgfpathcurveto{\pgfqpoint{4.177684in}{3.730638in}}{\pgfqpoint{4.165493in}{3.725589in}}{\pgfqpoint{4.156506in}{3.716602in}}%
\pgfpathcurveto{\pgfqpoint{4.147520in}{3.707615in}}{\pgfqpoint{4.142470in}{3.695425in}}{\pgfqpoint{4.142470in}{3.682716in}}%
\pgfpathcurveto{\pgfqpoint{4.142470in}{3.670007in}}{\pgfqpoint{4.147520in}{3.657817in}}{\pgfqpoint{4.156506in}{3.648830in}}%
\pgfpathcurveto{\pgfqpoint{4.165493in}{3.639843in}}{\pgfqpoint{4.177684in}{3.634794in}}{\pgfqpoint{4.190393in}{3.634794in}}%
\pgfpathclose%
\pgfusepath{stroke,fill}%
\end{pgfscope}%
\begin{pgfscope}%
\pgfpathrectangle{\pgfqpoint{1.065196in}{0.528000in}}{\pgfqpoint{3.702804in}{3.696000in}} %
\pgfusepath{clip}%
\pgfsetbuttcap%
\pgfsetroundjoin%
\definecolor{currentfill}{rgb}{0.121569,0.466667,0.705882}%
\pgfsetfillcolor{currentfill}%
\pgfsetlinewidth{1.003750pt}%
\definecolor{currentstroke}{rgb}{0.121569,0.466667,0.705882}%
\pgfsetstrokecolor{currentstroke}%
\pgfsetdash{}{0pt}%
\pgfpathmoveto{\pgfqpoint{1.541136in}{0.773866in}}%
\pgfpathcurveto{\pgfqpoint{1.552975in}{0.773866in}}{\pgfqpoint{1.564331in}{0.778569in}}{\pgfqpoint{1.572702in}{0.786941in}}%
\pgfpathcurveto{\pgfqpoint{1.581074in}{0.795313in}}{\pgfqpoint{1.585778in}{0.806668in}}{\pgfqpoint{1.585778in}{0.818508in}}%
\pgfpathcurveto{\pgfqpoint{1.585778in}{0.830347in}}{\pgfqpoint{1.581074in}{0.841703in}}{\pgfqpoint{1.572702in}{0.850074in}}%
\pgfpathcurveto{\pgfqpoint{1.564331in}{0.858446in}}{\pgfqpoint{1.552975in}{0.863150in}}{\pgfqpoint{1.541136in}{0.863150in}}%
\pgfpathcurveto{\pgfqpoint{1.529296in}{0.863150in}}{\pgfqpoint{1.517941in}{0.858446in}}{\pgfqpoint{1.509569in}{0.850074in}}%
\pgfpathcurveto{\pgfqpoint{1.501197in}{0.841703in}}{\pgfqpoint{1.496494in}{0.830347in}}{\pgfqpoint{1.496494in}{0.818508in}}%
\pgfpathcurveto{\pgfqpoint{1.496494in}{0.806668in}}{\pgfqpoint{1.501197in}{0.795313in}}{\pgfqpoint{1.509569in}{0.786941in}}%
\pgfpathcurveto{\pgfqpoint{1.517941in}{0.778569in}}{\pgfqpoint{1.529296in}{0.773866in}}{\pgfqpoint{1.541136in}{0.773866in}}%
\pgfpathclose%
\pgfusepath{stroke,fill}%
\end{pgfscope}%
\begin{pgfscope}%
\pgfpathrectangle{\pgfqpoint{1.065196in}{0.528000in}}{\pgfqpoint{3.702804in}{3.696000in}} %
\pgfusepath{clip}%
\pgfsetbuttcap%
\pgfsetroundjoin%
\definecolor{currentfill}{rgb}{0.121569,0.466667,0.705882}%
\pgfsetfillcolor{currentfill}%
\pgfsetlinewidth{1.003750pt}%
\definecolor{currentstroke}{rgb}{0.121569,0.466667,0.705882}%
\pgfsetstrokecolor{currentstroke}%
\pgfsetdash{}{0pt}%
\pgfpathmoveto{\pgfqpoint{1.824982in}{3.580785in}}%
\pgfpathcurveto{\pgfqpoint{1.837397in}{3.580785in}}{\pgfqpoint{1.849305in}{3.585717in}}{\pgfqpoint{1.858084in}{3.594496in}}%
\pgfpathcurveto{\pgfqpoint{1.866862in}{3.603275in}}{\pgfqpoint{1.871795in}{3.615183in}}{\pgfqpoint{1.871795in}{3.627597in}}%
\pgfpathcurveto{\pgfqpoint{1.871795in}{3.640012in}}{\pgfqpoint{1.866862in}{3.651920in}}{\pgfqpoint{1.858084in}{3.660699in}}%
\pgfpathcurveto{\pgfqpoint{1.849305in}{3.669478in}}{\pgfqpoint{1.837397in}{3.674410in}}{\pgfqpoint{1.824982in}{3.674410in}}%
\pgfpathcurveto{\pgfqpoint{1.812567in}{3.674410in}}{\pgfqpoint{1.800659in}{3.669478in}}{\pgfqpoint{1.791881in}{3.660699in}}%
\pgfpathcurveto{\pgfqpoint{1.783102in}{3.651920in}}{\pgfqpoint{1.778169in}{3.640012in}}{\pgfqpoint{1.778169in}{3.627597in}}%
\pgfpathcurveto{\pgfqpoint{1.778169in}{3.615183in}}{\pgfqpoint{1.783102in}{3.603275in}}{\pgfqpoint{1.791881in}{3.594496in}}%
\pgfpathcurveto{\pgfqpoint{1.800659in}{3.585717in}}{\pgfqpoint{1.812567in}{3.580785in}}{\pgfqpoint{1.824982in}{3.580785in}}%
\pgfpathclose%
\pgfusepath{stroke,fill}%
\end{pgfscope}%
\begin{pgfscope}%
\pgfpathrectangle{\pgfqpoint{1.065196in}{0.528000in}}{\pgfqpoint{3.702804in}{3.696000in}} %
\pgfusepath{clip}%
\pgfsetbuttcap%
\pgfsetroundjoin%
\definecolor{currentfill}{rgb}{0.121569,0.466667,0.705882}%
\pgfsetfillcolor{currentfill}%
\pgfsetlinewidth{1.003750pt}%
\definecolor{currentstroke}{rgb}{0.121569,0.466667,0.705882}%
\pgfsetstrokecolor{currentstroke}%
\pgfsetdash{}{0pt}%
\pgfpathmoveto{\pgfqpoint{1.585670in}{2.066244in}}%
\pgfpathcurveto{\pgfqpoint{1.593970in}{2.066244in}}{\pgfqpoint{1.601931in}{2.069542in}}{\pgfqpoint{1.607800in}{2.075411in}}%
\pgfpathcurveto{\pgfqpoint{1.613669in}{2.081280in}}{\pgfqpoint{1.616966in}{2.089241in}}{\pgfqpoint{1.616966in}{2.097540in}}%
\pgfpathcurveto{\pgfqpoint{1.616966in}{2.105840in}}{\pgfqpoint{1.613669in}{2.113801in}}{\pgfqpoint{1.607800in}{2.119670in}}%
\pgfpathcurveto{\pgfqpoint{1.601931in}{2.125539in}}{\pgfqpoint{1.593970in}{2.128837in}}{\pgfqpoint{1.585670in}{2.128837in}}%
\pgfpathcurveto{\pgfqpoint{1.577370in}{2.128837in}}{\pgfqpoint{1.569409in}{2.125539in}}{\pgfqpoint{1.563540in}{2.119670in}}%
\pgfpathcurveto{\pgfqpoint{1.557671in}{2.113801in}}{\pgfqpoint{1.554374in}{2.105840in}}{\pgfqpoint{1.554374in}{2.097540in}}%
\pgfpathcurveto{\pgfqpoint{1.554374in}{2.089241in}}{\pgfqpoint{1.557671in}{2.081280in}}{\pgfqpoint{1.563540in}{2.075411in}}%
\pgfpathcurveto{\pgfqpoint{1.569409in}{2.069542in}}{\pgfqpoint{1.577370in}{2.066244in}}{\pgfqpoint{1.585670in}{2.066244in}}%
\pgfpathclose%
\pgfusepath{stroke,fill}%
\end{pgfscope}%
\begin{pgfscope}%
\pgfpathrectangle{\pgfqpoint{1.065196in}{0.528000in}}{\pgfqpoint{3.702804in}{3.696000in}} %
\pgfusepath{clip}%
\pgfsetbuttcap%
\pgfsetroundjoin%
\definecolor{currentfill}{rgb}{0.121569,0.466667,0.705882}%
\pgfsetfillcolor{currentfill}%
\pgfsetlinewidth{1.003750pt}%
\definecolor{currentstroke}{rgb}{0.121569,0.466667,0.705882}%
\pgfsetstrokecolor{currentstroke}%
\pgfsetdash{}{0pt}%
\pgfpathmoveto{\pgfqpoint{4.463239in}{2.423501in}}%
\pgfpathcurveto{\pgfqpoint{4.476283in}{2.423501in}}{\pgfqpoint{4.488795in}{2.428683in}}{\pgfqpoint{4.498018in}{2.437907in}}%
\pgfpathcurveto{\pgfqpoint{4.507242in}{2.447130in}}{\pgfqpoint{4.512424in}{2.459642in}}{\pgfqpoint{4.512424in}{2.472686in}}%
\pgfpathcurveto{\pgfqpoint{4.512424in}{2.485730in}}{\pgfqpoint{4.507242in}{2.498242in}}{\pgfqpoint{4.498018in}{2.507465in}}%
\pgfpathcurveto{\pgfqpoint{4.488795in}{2.516689in}}{\pgfqpoint{4.476283in}{2.521871in}}{\pgfqpoint{4.463239in}{2.521871in}}%
\pgfpathcurveto{\pgfqpoint{4.450195in}{2.521871in}}{\pgfqpoint{4.437683in}{2.516689in}}{\pgfqpoint{4.428459in}{2.507465in}}%
\pgfpathcurveto{\pgfqpoint{4.419236in}{2.498242in}}{\pgfqpoint{4.414053in}{2.485730in}}{\pgfqpoint{4.414053in}{2.472686in}}%
\pgfpathcurveto{\pgfqpoint{4.414053in}{2.459642in}}{\pgfqpoint{4.419236in}{2.447130in}}{\pgfqpoint{4.428459in}{2.437907in}}%
\pgfpathcurveto{\pgfqpoint{4.437683in}{2.428683in}}{\pgfqpoint{4.450195in}{2.423501in}}{\pgfqpoint{4.463239in}{2.423501in}}%
\pgfpathclose%
\pgfusepath{stroke,fill}%
\end{pgfscope}%
\begin{pgfscope}%
\pgfpathrectangle{\pgfqpoint{1.065196in}{0.528000in}}{\pgfqpoint{3.702804in}{3.696000in}} %
\pgfusepath{clip}%
\pgfsetbuttcap%
\pgfsetroundjoin%
\definecolor{currentfill}{rgb}{0.121569,0.466667,0.705882}%
\pgfsetfillcolor{currentfill}%
\pgfsetlinewidth{1.003750pt}%
\definecolor{currentstroke}{rgb}{0.121569,0.466667,0.705882}%
\pgfsetstrokecolor{currentstroke}%
\pgfsetdash{}{0pt}%
\pgfpathmoveto{\pgfqpoint{3.572685in}{1.709022in}}%
\pgfpathcurveto{\pgfqpoint{3.581972in}{1.709022in}}{\pgfqpoint{3.590880in}{1.712712in}}{\pgfqpoint{3.597447in}{1.719279in}}%
\pgfpathcurveto{\pgfqpoint{3.604014in}{1.725846in}}{\pgfqpoint{3.607704in}{1.734754in}}{\pgfqpoint{3.607704in}{1.744041in}}%
\pgfpathcurveto{\pgfqpoint{3.607704in}{1.753328in}}{\pgfqpoint{3.604014in}{1.762236in}}{\pgfqpoint{3.597447in}{1.768803in}}%
\pgfpathcurveto{\pgfqpoint{3.590880in}{1.775370in}}{\pgfqpoint{3.581972in}{1.779059in}}{\pgfqpoint{3.572685in}{1.779059in}}%
\pgfpathcurveto{\pgfqpoint{3.563398in}{1.779059in}}{\pgfqpoint{3.554490in}{1.775370in}}{\pgfqpoint{3.547923in}{1.768803in}}%
\pgfpathcurveto{\pgfqpoint{3.541356in}{1.762236in}}{\pgfqpoint{3.537666in}{1.753328in}}{\pgfqpoint{3.537666in}{1.744041in}}%
\pgfpathcurveto{\pgfqpoint{3.537666in}{1.734754in}}{\pgfqpoint{3.541356in}{1.725846in}}{\pgfqpoint{3.547923in}{1.719279in}}%
\pgfpathcurveto{\pgfqpoint{3.554490in}{1.712712in}}{\pgfqpoint{3.563398in}{1.709022in}}{\pgfqpoint{3.572685in}{1.709022in}}%
\pgfpathclose%
\pgfusepath{stroke,fill}%
\end{pgfscope}%
\begin{pgfscope}%
\pgfpathrectangle{\pgfqpoint{1.065196in}{0.528000in}}{\pgfqpoint{3.702804in}{3.696000in}} %
\pgfusepath{clip}%
\pgfsetbuttcap%
\pgfsetroundjoin%
\definecolor{currentfill}{rgb}{0.121569,0.466667,0.705882}%
\pgfsetfillcolor{currentfill}%
\pgfsetlinewidth{1.003750pt}%
\definecolor{currentstroke}{rgb}{0.121569,0.466667,0.705882}%
\pgfsetstrokecolor{currentstroke}%
\pgfsetdash{}{0pt}%
\pgfpathmoveto{\pgfqpoint{3.554687in}{3.433366in}}%
\pgfpathcurveto{\pgfqpoint{3.567561in}{3.433366in}}{\pgfqpoint{3.579910in}{3.438481in}}{\pgfqpoint{3.589014in}{3.447585in}}%
\pgfpathcurveto{\pgfqpoint{3.598118in}{3.456689in}}{\pgfqpoint{3.603233in}{3.469038in}}{\pgfqpoint{3.603233in}{3.481912in}}%
\pgfpathcurveto{\pgfqpoint{3.603233in}{3.494787in}}{\pgfqpoint{3.598118in}{3.507136in}}{\pgfqpoint{3.589014in}{3.516240in}}%
\pgfpathcurveto{\pgfqpoint{3.579910in}{3.525343in}}{\pgfqpoint{3.567561in}{3.530458in}}{\pgfqpoint{3.554687in}{3.530458in}}%
\pgfpathcurveto{\pgfqpoint{3.541812in}{3.530458in}}{\pgfqpoint{3.529463in}{3.525343in}}{\pgfqpoint{3.520359in}{3.516240in}}%
\pgfpathcurveto{\pgfqpoint{3.511256in}{3.507136in}}{\pgfqpoint{3.506140in}{3.494787in}}{\pgfqpoint{3.506140in}{3.481912in}}%
\pgfpathcurveto{\pgfqpoint{3.506140in}{3.469038in}}{\pgfqpoint{3.511256in}{3.456689in}}{\pgfqpoint{3.520359in}{3.447585in}}%
\pgfpathcurveto{\pgfqpoint{3.529463in}{3.438481in}}{\pgfqpoint{3.541812in}{3.433366in}}{\pgfqpoint{3.554687in}{3.433366in}}%
\pgfpathclose%
\pgfusepath{stroke,fill}%
\end{pgfscope}%
\begin{pgfscope}%
\pgfpathrectangle{\pgfqpoint{1.065196in}{0.528000in}}{\pgfqpoint{3.702804in}{3.696000in}} %
\pgfusepath{clip}%
\pgfsetbuttcap%
\pgfsetroundjoin%
\definecolor{currentfill}{rgb}{0.121569,0.466667,0.705882}%
\pgfsetfillcolor{currentfill}%
\pgfsetlinewidth{1.003750pt}%
\definecolor{currentstroke}{rgb}{0.121569,0.466667,0.705882}%
\pgfsetstrokecolor{currentstroke}%
\pgfsetdash{}{0pt}%
\pgfpathmoveto{\pgfqpoint{1.317651in}{3.156756in}}%
\pgfpathcurveto{\pgfqpoint{1.328877in}{3.156756in}}{\pgfqpoint{1.339645in}{3.161216in}}{\pgfqpoint{1.347583in}{3.169154in}}%
\pgfpathcurveto{\pgfqpoint{1.355521in}{3.177093in}}{\pgfqpoint{1.359981in}{3.187860in}}{\pgfqpoint{1.359981in}{3.199086in}}%
\pgfpathcurveto{\pgfqpoint{1.359981in}{3.210312in}}{\pgfqpoint{1.355521in}{3.221080in}}{\pgfqpoint{1.347583in}{3.229018in}}%
\pgfpathcurveto{\pgfqpoint{1.339645in}{3.236956in}}{\pgfqpoint{1.328877in}{3.241416in}}{\pgfqpoint{1.317651in}{3.241416in}}%
\pgfpathcurveto{\pgfqpoint{1.306425in}{3.241416in}}{\pgfqpoint{1.295657in}{3.236956in}}{\pgfqpoint{1.287719in}{3.229018in}}%
\pgfpathcurveto{\pgfqpoint{1.279781in}{3.221080in}}{\pgfqpoint{1.275321in}{3.210312in}}{\pgfqpoint{1.275321in}{3.199086in}}%
\pgfpathcurveto{\pgfqpoint{1.275321in}{3.187860in}}{\pgfqpoint{1.279781in}{3.177093in}}{\pgfqpoint{1.287719in}{3.169154in}}%
\pgfpathcurveto{\pgfqpoint{1.295657in}{3.161216in}}{\pgfqpoint{1.306425in}{3.156756in}}{\pgfqpoint{1.317651in}{3.156756in}}%
\pgfpathclose%
\pgfusepath{stroke,fill}%
\end{pgfscope}%
\begin{pgfscope}%
\pgfpathrectangle{\pgfqpoint{1.065196in}{0.528000in}}{\pgfqpoint{3.702804in}{3.696000in}} %
\pgfusepath{clip}%
\pgfsetbuttcap%
\pgfsetroundjoin%
\definecolor{currentfill}{rgb}{0.121569,0.466667,0.705882}%
\pgfsetfillcolor{currentfill}%
\pgfsetlinewidth{1.003750pt}%
\definecolor{currentstroke}{rgb}{0.121569,0.466667,0.705882}%
\pgfsetstrokecolor{currentstroke}%
\pgfsetdash{}{0pt}%
\pgfpathmoveto{\pgfqpoint{4.566925in}{3.172209in}}%
\pgfpathcurveto{\pgfqpoint{4.572296in}{3.172209in}}{\pgfqpoint{4.577448in}{3.174343in}}{\pgfqpoint{4.581246in}{3.178141in}}%
\pgfpathcurveto{\pgfqpoint{4.585045in}{3.181939in}}{\pgfqpoint{4.587179in}{3.187091in}}{\pgfqpoint{4.587179in}{3.192462in}}%
\pgfpathcurveto{\pgfqpoint{4.587179in}{3.197833in}}{\pgfqpoint{4.585045in}{3.202985in}}{\pgfqpoint{4.581246in}{3.206784in}}%
\pgfpathcurveto{\pgfqpoint{4.577448in}{3.210582in}}{\pgfqpoint{4.572296in}{3.212716in}}{\pgfqpoint{4.566925in}{3.212716in}}%
\pgfpathcurveto{\pgfqpoint{4.561554in}{3.212716in}}{\pgfqpoint{4.556402in}{3.210582in}}{\pgfqpoint{4.552604in}{3.206784in}}%
\pgfpathcurveto{\pgfqpoint{4.548806in}{3.202985in}}{\pgfqpoint{4.546672in}{3.197833in}}{\pgfqpoint{4.546672in}{3.192462in}}%
\pgfpathcurveto{\pgfqpoint{4.546672in}{3.187091in}}{\pgfqpoint{4.548806in}{3.181939in}}{\pgfqpoint{4.552604in}{3.178141in}}%
\pgfpathcurveto{\pgfqpoint{4.556402in}{3.174343in}}{\pgfqpoint{4.561554in}{3.172209in}}{\pgfqpoint{4.566925in}{3.172209in}}%
\pgfpathclose%
\pgfusepath{stroke,fill}%
\end{pgfscope}%
\begin{pgfscope}%
\pgfpathrectangle{\pgfqpoint{1.065196in}{0.528000in}}{\pgfqpoint{3.702804in}{3.696000in}} %
\pgfusepath{clip}%
\pgfsetbuttcap%
\pgfsetroundjoin%
\definecolor{currentfill}{rgb}{0.121569,0.466667,0.705882}%
\pgfsetfillcolor{currentfill}%
\pgfsetlinewidth{1.003750pt}%
\definecolor{currentstroke}{rgb}{0.121569,0.466667,0.705882}%
\pgfsetstrokecolor{currentstroke}%
\pgfsetdash{}{0pt}%
\pgfpathmoveto{\pgfqpoint{2.195293in}{3.315129in}}%
\pgfpathcurveto{\pgfqpoint{2.199264in}{3.315129in}}{\pgfqpoint{2.203073in}{3.316707in}}{\pgfqpoint{2.205881in}{3.319515in}}%
\pgfpathcurveto{\pgfqpoint{2.208689in}{3.322322in}}{\pgfqpoint{2.210266in}{3.326131in}}{\pgfqpoint{2.210266in}{3.330102in}}%
\pgfpathcurveto{\pgfqpoint{2.210266in}{3.334073in}}{\pgfqpoint{2.208689in}{3.337882in}}{\pgfqpoint{2.205881in}{3.340690in}}%
\pgfpathcurveto{\pgfqpoint{2.203073in}{3.343498in}}{\pgfqpoint{2.199264in}{3.345075in}}{\pgfqpoint{2.195293in}{3.345075in}}%
\pgfpathcurveto{\pgfqpoint{2.191322in}{3.345075in}}{\pgfqpoint{2.187513in}{3.343498in}}{\pgfqpoint{2.184706in}{3.340690in}}%
\pgfpathcurveto{\pgfqpoint{2.181898in}{3.337882in}}{\pgfqpoint{2.180320in}{3.334073in}}{\pgfqpoint{2.180320in}{3.330102in}}%
\pgfpathcurveto{\pgfqpoint{2.180320in}{3.326131in}}{\pgfqpoint{2.181898in}{3.322322in}}{\pgfqpoint{2.184706in}{3.319515in}}%
\pgfpathcurveto{\pgfqpoint{2.187513in}{3.316707in}}{\pgfqpoint{2.191322in}{3.315129in}}{\pgfqpoint{2.195293in}{3.315129in}}%
\pgfpathclose%
\pgfusepath{stroke,fill}%
\end{pgfscope}%
\begin{pgfscope}%
\pgfpathrectangle{\pgfqpoint{1.065196in}{0.528000in}}{\pgfqpoint{3.702804in}{3.696000in}} %
\pgfusepath{clip}%
\pgfsetbuttcap%
\pgfsetroundjoin%
\definecolor{currentfill}{rgb}{0.121569,0.466667,0.705882}%
\pgfsetfillcolor{currentfill}%
\pgfsetlinewidth{1.003750pt}%
\definecolor{currentstroke}{rgb}{0.121569,0.466667,0.705882}%
\pgfsetstrokecolor{currentstroke}%
\pgfsetdash{}{0pt}%
\pgfpathmoveto{\pgfqpoint{1.602004in}{2.179973in}}%
\pgfpathcurveto{\pgfqpoint{1.603925in}{2.179973in}}{\pgfqpoint{1.605767in}{2.180736in}}{\pgfqpoint{1.607125in}{2.182094in}}%
\pgfpathcurveto{\pgfqpoint{1.608483in}{2.183452in}}{\pgfqpoint{1.609246in}{2.185294in}}{\pgfqpoint{1.609246in}{2.187214in}}%
\pgfpathcurveto{\pgfqpoint{1.609246in}{2.189134in}}{\pgfqpoint{1.608483in}{2.190976in}}{\pgfqpoint{1.607125in}{2.192334in}}%
\pgfpathcurveto{\pgfqpoint{1.605767in}{2.193692in}}{\pgfqpoint{1.603925in}{2.194455in}}{\pgfqpoint{1.602004in}{2.194455in}}%
\pgfpathcurveto{\pgfqpoint{1.600084in}{2.194455in}}{\pgfqpoint{1.598242in}{2.193692in}}{\pgfqpoint{1.596884in}{2.192334in}}%
\pgfpathcurveto{\pgfqpoint{1.595526in}{2.190976in}}{\pgfqpoint{1.594763in}{2.189134in}}{\pgfqpoint{1.594763in}{2.187214in}}%
\pgfpathcurveto{\pgfqpoint{1.594763in}{2.185294in}}{\pgfqpoint{1.595526in}{2.183452in}}{\pgfqpoint{1.596884in}{2.182094in}}%
\pgfpathcurveto{\pgfqpoint{1.598242in}{2.180736in}}{\pgfqpoint{1.600084in}{2.179973in}}{\pgfqpoint{1.602004in}{2.179973in}}%
\pgfpathclose%
\pgfusepath{stroke,fill}%
\end{pgfscope}%
\begin{pgfscope}%
\pgfpathrectangle{\pgfqpoint{1.065196in}{0.528000in}}{\pgfqpoint{3.702804in}{3.696000in}} %
\pgfusepath{clip}%
\pgfsetbuttcap%
\pgfsetroundjoin%
\definecolor{currentfill}{rgb}{0.121569,0.466667,0.705882}%
\pgfsetfillcolor{currentfill}%
\pgfsetlinewidth{1.003750pt}%
\definecolor{currentstroke}{rgb}{0.121569,0.466667,0.705882}%
\pgfsetstrokecolor{currentstroke}%
\pgfsetdash{}{0pt}%
\pgfpathmoveto{\pgfqpoint{4.298213in}{1.656905in}}%
\pgfpathcurveto{\pgfqpoint{4.301876in}{1.656905in}}{\pgfqpoint{4.305390in}{1.658360in}}{\pgfqpoint{4.307981in}{1.660951in}}%
\pgfpathcurveto{\pgfqpoint{4.310572in}{1.663541in}}{\pgfqpoint{4.312027in}{1.667055in}}{\pgfqpoint{4.312027in}{1.670719in}}%
\pgfpathcurveto{\pgfqpoint{4.312027in}{1.674383in}}{\pgfqpoint{4.310572in}{1.677897in}}{\pgfqpoint{4.307981in}{1.680487in}}%
\pgfpathcurveto{\pgfqpoint{4.305390in}{1.683078in}}{\pgfqpoint{4.301876in}{1.684533in}}{\pgfqpoint{4.298213in}{1.684533in}}%
\pgfpathcurveto{\pgfqpoint{4.294549in}{1.684533in}}{\pgfqpoint{4.291035in}{1.683078in}}{\pgfqpoint{4.288444in}{1.680487in}}%
\pgfpathcurveto{\pgfqpoint{4.285854in}{1.677897in}}{\pgfqpoint{4.284398in}{1.674383in}}{\pgfqpoint{4.284398in}{1.670719in}}%
\pgfpathcurveto{\pgfqpoint{4.284398in}{1.667055in}}{\pgfqpoint{4.285854in}{1.663541in}}{\pgfqpoint{4.288444in}{1.660951in}}%
\pgfpathcurveto{\pgfqpoint{4.291035in}{1.658360in}}{\pgfqpoint{4.294549in}{1.656905in}}{\pgfqpoint{4.298213in}{1.656905in}}%
\pgfpathclose%
\pgfusepath{stroke,fill}%
\end{pgfscope}%
\begin{pgfscope}%
\pgfpathrectangle{\pgfqpoint{1.065196in}{0.528000in}}{\pgfqpoint{3.702804in}{3.696000in}} %
\pgfusepath{clip}%
\pgfsetbuttcap%
\pgfsetroundjoin%
\definecolor{currentfill}{rgb}{0.121569,0.466667,0.705882}%
\pgfsetfillcolor{currentfill}%
\pgfsetlinewidth{1.003750pt}%
\definecolor{currentstroke}{rgb}{0.121569,0.466667,0.705882}%
\pgfsetstrokecolor{currentstroke}%
\pgfsetdash{}{0pt}%
\pgfpathmoveto{\pgfqpoint{2.219837in}{1.086853in}}%
\pgfpathcurveto{\pgfqpoint{2.229442in}{1.086853in}}{\pgfqpoint{2.238654in}{1.090669in}}{\pgfqpoint{2.245445in}{1.097460in}}%
\pgfpathcurveto{\pgfqpoint{2.252237in}{1.104252in}}{\pgfqpoint{2.256053in}{1.113464in}}{\pgfqpoint{2.256053in}{1.123069in}}%
\pgfpathcurveto{\pgfqpoint{2.256053in}{1.132673in}}{\pgfqpoint{2.252237in}{1.141886in}}{\pgfqpoint{2.245445in}{1.148677in}}%
\pgfpathcurveto{\pgfqpoint{2.238654in}{1.155469in}}{\pgfqpoint{2.229442in}{1.159285in}}{\pgfqpoint{2.219837in}{1.159285in}}%
\pgfpathcurveto{\pgfqpoint{2.210232in}{1.159285in}}{\pgfqpoint{2.201020in}{1.155469in}}{\pgfqpoint{2.194228in}{1.148677in}}%
\pgfpathcurveto{\pgfqpoint{2.187437in}{1.141886in}}{\pgfqpoint{2.183621in}{1.132673in}}{\pgfqpoint{2.183621in}{1.123069in}}%
\pgfpathcurveto{\pgfqpoint{2.183621in}{1.113464in}}{\pgfqpoint{2.187437in}{1.104252in}}{\pgfqpoint{2.194228in}{1.097460in}}%
\pgfpathcurveto{\pgfqpoint{2.201020in}{1.090669in}}{\pgfqpoint{2.210232in}{1.086853in}}{\pgfqpoint{2.219837in}{1.086853in}}%
\pgfpathclose%
\pgfusepath{stroke,fill}%
\end{pgfscope}%
\begin{pgfscope}%
\pgfpathrectangle{\pgfqpoint{1.065196in}{0.528000in}}{\pgfqpoint{3.702804in}{3.696000in}} %
\pgfusepath{clip}%
\pgfsetbuttcap%
\pgfsetroundjoin%
\definecolor{currentfill}{rgb}{0.121569,0.466667,0.705882}%
\pgfsetfillcolor{currentfill}%
\pgfsetlinewidth{1.003750pt}%
\definecolor{currentstroke}{rgb}{0.121569,0.466667,0.705882}%
\pgfsetstrokecolor{currentstroke}%
\pgfsetdash{}{0pt}%
\pgfpathmoveto{\pgfqpoint{1.321262in}{2.942355in}}%
\pgfpathcurveto{\pgfqpoint{1.326037in}{2.942355in}}{\pgfqpoint{1.330617in}{2.944252in}}{\pgfqpoint{1.333993in}{2.947628in}}%
\pgfpathcurveto{\pgfqpoint{1.337370in}{2.951005in}}{\pgfqpoint{1.339267in}{2.955585in}}{\pgfqpoint{1.339267in}{2.960359in}}%
\pgfpathcurveto{\pgfqpoint{1.339267in}{2.965134in}}{\pgfqpoint{1.337370in}{2.969714in}}{\pgfqpoint{1.333993in}{2.973091in}}%
\pgfpathcurveto{\pgfqpoint{1.330617in}{2.976467in}}{\pgfqpoint{1.326037in}{2.978364in}}{\pgfqpoint{1.321262in}{2.978364in}}%
\pgfpathcurveto{\pgfqpoint{1.316487in}{2.978364in}}{\pgfqpoint{1.311907in}{2.976467in}}{\pgfqpoint{1.308531in}{2.973091in}}%
\pgfpathcurveto{\pgfqpoint{1.305154in}{2.969714in}}{\pgfqpoint{1.303257in}{2.965134in}}{\pgfqpoint{1.303257in}{2.960359in}}%
\pgfpathcurveto{\pgfqpoint{1.303257in}{2.955585in}}{\pgfqpoint{1.305154in}{2.951005in}}{\pgfqpoint{1.308531in}{2.947628in}}%
\pgfpathcurveto{\pgfqpoint{1.311907in}{2.944252in}}{\pgfqpoint{1.316487in}{2.942355in}}{\pgfqpoint{1.321262in}{2.942355in}}%
\pgfpathclose%
\pgfusepath{stroke,fill}%
\end{pgfscope}%
\begin{pgfscope}%
\pgfpathrectangle{\pgfqpoint{1.065196in}{0.528000in}}{\pgfqpoint{3.702804in}{3.696000in}} %
\pgfusepath{clip}%
\pgfsetbuttcap%
\pgfsetroundjoin%
\definecolor{currentfill}{rgb}{0.121569,0.466667,0.705882}%
\pgfsetfillcolor{currentfill}%
\pgfsetlinewidth{1.003750pt}%
\definecolor{currentstroke}{rgb}{0.121569,0.466667,0.705882}%
\pgfsetstrokecolor{currentstroke}%
\pgfsetdash{}{0pt}%
\pgfpathmoveto{\pgfqpoint{1.964912in}{1.567841in}}%
\pgfpathcurveto{\pgfqpoint{1.967276in}{1.567841in}}{\pgfqpoint{1.969544in}{1.568780in}}{\pgfqpoint{1.971215in}{1.570452in}}%
\pgfpathcurveto{\pgfqpoint{1.972887in}{1.572123in}}{\pgfqpoint{1.973826in}{1.574391in}}{\pgfqpoint{1.973826in}{1.576755in}}%
\pgfpathcurveto{\pgfqpoint{1.973826in}{1.579119in}}{\pgfqpoint{1.972887in}{1.581387in}}{\pgfqpoint{1.971215in}{1.583058in}}%
\pgfpathcurveto{\pgfqpoint{1.969544in}{1.584730in}}{\pgfqpoint{1.967276in}{1.585669in}}{\pgfqpoint{1.964912in}{1.585669in}}%
\pgfpathcurveto{\pgfqpoint{1.962548in}{1.585669in}}{\pgfqpoint{1.960280in}{1.584730in}}{\pgfqpoint{1.958609in}{1.583058in}}%
\pgfpathcurveto{\pgfqpoint{1.956937in}{1.581387in}}{\pgfqpoint{1.955998in}{1.579119in}}{\pgfqpoint{1.955998in}{1.576755in}}%
\pgfpathcurveto{\pgfqpoint{1.955998in}{1.574391in}}{\pgfqpoint{1.956937in}{1.572123in}}{\pgfqpoint{1.958609in}{1.570452in}}%
\pgfpathcurveto{\pgfqpoint{1.960280in}{1.568780in}}{\pgfqpoint{1.962548in}{1.567841in}}{\pgfqpoint{1.964912in}{1.567841in}}%
\pgfpathclose%
\pgfusepath{stroke,fill}%
\end{pgfscope}%
\begin{pgfscope}%
\pgfpathrectangle{\pgfqpoint{1.065196in}{0.528000in}}{\pgfqpoint{3.702804in}{3.696000in}} %
\pgfusepath{clip}%
\pgfsetbuttcap%
\pgfsetroundjoin%
\definecolor{currentfill}{rgb}{0.121569,0.466667,0.705882}%
\pgfsetfillcolor{currentfill}%
\pgfsetlinewidth{1.003750pt}%
\definecolor{currentstroke}{rgb}{0.121569,0.466667,0.705882}%
\pgfsetstrokecolor{currentstroke}%
\pgfsetdash{}{0pt}%
\pgfpathmoveto{\pgfqpoint{2.902109in}{0.819942in}}%
\pgfpathcurveto{\pgfqpoint{2.914432in}{0.819942in}}{\pgfqpoint{2.926251in}{0.824838in}}{\pgfqpoint{2.934965in}{0.833552in}}%
\pgfpathcurveto{\pgfqpoint{2.943678in}{0.842265in}}{\pgfqpoint{2.948574in}{0.854084in}}{\pgfqpoint{2.948574in}{0.866407in}}%
\pgfpathcurveto{\pgfqpoint{2.948574in}{0.878730in}}{\pgfqpoint{2.943678in}{0.890549in}}{\pgfqpoint{2.934965in}{0.899263in}}%
\pgfpathcurveto{\pgfqpoint{2.926251in}{0.907976in}}{\pgfqpoint{2.914432in}{0.912872in}}{\pgfqpoint{2.902109in}{0.912872in}}%
\pgfpathcurveto{\pgfqpoint{2.889787in}{0.912872in}}{\pgfqpoint{2.877967in}{0.907976in}}{\pgfqpoint{2.869254in}{0.899263in}}%
\pgfpathcurveto{\pgfqpoint{2.860540in}{0.890549in}}{\pgfqpoint{2.855645in}{0.878730in}}{\pgfqpoint{2.855645in}{0.866407in}}%
\pgfpathcurveto{\pgfqpoint{2.855645in}{0.854084in}}{\pgfqpoint{2.860540in}{0.842265in}}{\pgfqpoint{2.869254in}{0.833552in}}%
\pgfpathcurveto{\pgfqpoint{2.877967in}{0.824838in}}{\pgfqpoint{2.889787in}{0.819942in}}{\pgfqpoint{2.902109in}{0.819942in}}%
\pgfpathclose%
\pgfusepath{stroke,fill}%
\end{pgfscope}%
\begin{pgfscope}%
\pgfpathrectangle{\pgfqpoint{1.065196in}{0.528000in}}{\pgfqpoint{3.702804in}{3.696000in}} %
\pgfusepath{clip}%
\pgfsetbuttcap%
\pgfsetroundjoin%
\definecolor{currentfill}{rgb}{0.121569,0.466667,0.705882}%
\pgfsetfillcolor{currentfill}%
\pgfsetlinewidth{1.003750pt}%
\definecolor{currentstroke}{rgb}{0.121569,0.466667,0.705882}%
\pgfsetstrokecolor{currentstroke}%
\pgfsetdash{}{0pt}%
\pgfpathmoveto{\pgfqpoint{3.178451in}{1.145372in}}%
\pgfpathcurveto{\pgfqpoint{3.187363in}{1.145372in}}{\pgfqpoint{3.195911in}{1.148913in}}{\pgfqpoint{3.202213in}{1.155214in}}%
\pgfpathcurveto{\pgfqpoint{3.208515in}{1.161516in}}{\pgfqpoint{3.212056in}{1.170065in}}{\pgfqpoint{3.212056in}{1.178977in}}%
\pgfpathcurveto{\pgfqpoint{3.212056in}{1.187889in}}{\pgfqpoint{3.208515in}{1.196437in}}{\pgfqpoint{3.202213in}{1.202739in}}%
\pgfpathcurveto{\pgfqpoint{3.195911in}{1.209041in}}{\pgfqpoint{3.187363in}{1.212582in}}{\pgfqpoint{3.178451in}{1.212582in}}%
\pgfpathcurveto{\pgfqpoint{3.169539in}{1.212582in}}{\pgfqpoint{3.160990in}{1.209041in}}{\pgfqpoint{3.154688in}{1.202739in}}%
\pgfpathcurveto{\pgfqpoint{3.148387in}{1.196437in}}{\pgfqpoint{3.144846in}{1.187889in}}{\pgfqpoint{3.144846in}{1.178977in}}%
\pgfpathcurveto{\pgfqpoint{3.144846in}{1.170065in}}{\pgfqpoint{3.148387in}{1.161516in}}{\pgfqpoint{3.154688in}{1.155214in}}%
\pgfpathcurveto{\pgfqpoint{3.160990in}{1.148913in}}{\pgfqpoint{3.169539in}{1.145372in}}{\pgfqpoint{3.178451in}{1.145372in}}%
\pgfpathclose%
\pgfusepath{stroke,fill}%
\end{pgfscope}%
\begin{pgfscope}%
\pgfpathrectangle{\pgfqpoint{1.065196in}{0.528000in}}{\pgfqpoint{3.702804in}{3.696000in}} %
\pgfusepath{clip}%
\pgfsetbuttcap%
\pgfsetroundjoin%
\definecolor{currentfill}{rgb}{0.121569,0.466667,0.705882}%
\pgfsetfillcolor{currentfill}%
\pgfsetlinewidth{1.003750pt}%
\definecolor{currentstroke}{rgb}{0.121569,0.466667,0.705882}%
\pgfsetstrokecolor{currentstroke}%
\pgfsetdash{}{0pt}%
\pgfpathmoveto{\pgfqpoint{3.229297in}{2.984185in}}%
\pgfpathcurveto{\pgfqpoint{3.241555in}{2.984185in}}{\pgfqpoint{3.253312in}{2.989055in}}{\pgfqpoint{3.261980in}{2.997722in}}%
\pgfpathcurveto{\pgfqpoint{3.270647in}{3.006390in}}{\pgfqpoint{3.275517in}{3.018147in}}{\pgfqpoint{3.275517in}{3.030405in}}%
\pgfpathcurveto{\pgfqpoint{3.275517in}{3.042662in}}{\pgfqpoint{3.270647in}{3.054420in}}{\pgfqpoint{3.261980in}{3.063087in}}%
\pgfpathcurveto{\pgfqpoint{3.253312in}{3.071755in}}{\pgfqpoint{3.241555in}{3.076625in}}{\pgfqpoint{3.229297in}{3.076625in}}%
\pgfpathcurveto{\pgfqpoint{3.217039in}{3.076625in}}{\pgfqpoint{3.205282in}{3.071755in}}{\pgfqpoint{3.196614in}{3.063087in}}%
\pgfpathcurveto{\pgfqpoint{3.187947in}{3.054420in}}{\pgfqpoint{3.183077in}{3.042662in}}{\pgfqpoint{3.183077in}{3.030405in}}%
\pgfpathcurveto{\pgfqpoint{3.183077in}{3.018147in}}{\pgfqpoint{3.187947in}{3.006390in}}{\pgfqpoint{3.196614in}{2.997722in}}%
\pgfpathcurveto{\pgfqpoint{3.205282in}{2.989055in}}{\pgfqpoint{3.217039in}{2.984185in}}{\pgfqpoint{3.229297in}{2.984185in}}%
\pgfpathclose%
\pgfusepath{stroke,fill}%
\end{pgfscope}%
\begin{pgfscope}%
\pgfpathrectangle{\pgfqpoint{1.065196in}{0.528000in}}{\pgfqpoint{3.702804in}{3.696000in}} %
\pgfusepath{clip}%
\pgfsetbuttcap%
\pgfsetroundjoin%
\definecolor{currentfill}{rgb}{0.121569,0.466667,0.705882}%
\pgfsetfillcolor{currentfill}%
\pgfsetlinewidth{1.003750pt}%
\definecolor{currentstroke}{rgb}{0.121569,0.466667,0.705882}%
\pgfsetstrokecolor{currentstroke}%
\pgfsetdash{}{0pt}%
\pgfpathmoveto{\pgfqpoint{1.599020in}{2.037962in}}%
\pgfpathcurveto{\pgfqpoint{1.608559in}{2.037962in}}{\pgfqpoint{1.617710in}{2.041752in}}{\pgfqpoint{1.624455in}{2.048498in}}%
\pgfpathcurveto{\pgfqpoint{1.631201in}{2.055243in}}{\pgfqpoint{1.634991in}{2.064393in}}{\pgfqpoint{1.634991in}{2.073933in}}%
\pgfpathcurveto{\pgfqpoint{1.634991in}{2.083473in}}{\pgfqpoint{1.631201in}{2.092623in}}{\pgfqpoint{1.624455in}{2.099368in}}%
\pgfpathcurveto{\pgfqpoint{1.617710in}{2.106114in}}{\pgfqpoint{1.608559in}{2.109904in}}{\pgfqpoint{1.599020in}{2.109904in}}%
\pgfpathcurveto{\pgfqpoint{1.589480in}{2.109904in}}{\pgfqpoint{1.580330in}{2.106114in}}{\pgfqpoint{1.573584in}{2.099368in}}%
\pgfpathcurveto{\pgfqpoint{1.566839in}{2.092623in}}{\pgfqpoint{1.563049in}{2.083473in}}{\pgfqpoint{1.563049in}{2.073933in}}%
\pgfpathcurveto{\pgfqpoint{1.563049in}{2.064393in}}{\pgfqpoint{1.566839in}{2.055243in}}{\pgfqpoint{1.573584in}{2.048498in}}%
\pgfpathcurveto{\pgfqpoint{1.580330in}{2.041752in}}{\pgfqpoint{1.589480in}{2.037962in}}{\pgfqpoint{1.599020in}{2.037962in}}%
\pgfpathclose%
\pgfusepath{stroke,fill}%
\end{pgfscope}%
\begin{pgfscope}%
\pgfpathrectangle{\pgfqpoint{1.065196in}{0.528000in}}{\pgfqpoint{3.702804in}{3.696000in}} %
\pgfusepath{clip}%
\pgfsetbuttcap%
\pgfsetroundjoin%
\definecolor{currentfill}{rgb}{0.121569,0.466667,0.705882}%
\pgfsetfillcolor{currentfill}%
\pgfsetlinewidth{1.003750pt}%
\definecolor{currentstroke}{rgb}{0.121569,0.466667,0.705882}%
\pgfsetstrokecolor{currentstroke}%
\pgfsetdash{}{0pt}%
\pgfpathmoveto{\pgfqpoint{3.581132in}{2.038467in}}%
\pgfpathcurveto{\pgfqpoint{3.590647in}{2.038467in}}{\pgfqpoint{3.599774in}{2.042247in}}{\pgfqpoint{3.606502in}{2.048976in}}%
\pgfpathcurveto{\pgfqpoint{3.613230in}{2.055704in}}{\pgfqpoint{3.617010in}{2.064831in}}{\pgfqpoint{3.617010in}{2.074346in}}%
\pgfpathcurveto{\pgfqpoint{3.617010in}{2.083861in}}{\pgfqpoint{3.613230in}{2.092988in}}{\pgfqpoint{3.606502in}{2.099716in}}%
\pgfpathcurveto{\pgfqpoint{3.599774in}{2.106444in}}{\pgfqpoint{3.590647in}{2.110225in}}{\pgfqpoint{3.581132in}{2.110225in}}%
\pgfpathcurveto{\pgfqpoint{3.571616in}{2.110225in}}{\pgfqpoint{3.562490in}{2.106444in}}{\pgfqpoint{3.555762in}{2.099716in}}%
\pgfpathcurveto{\pgfqpoint{3.549033in}{2.092988in}}{\pgfqpoint{3.545253in}{2.083861in}}{\pgfqpoint{3.545253in}{2.074346in}}%
\pgfpathcurveto{\pgfqpoint{3.545253in}{2.064831in}}{\pgfqpoint{3.549033in}{2.055704in}}{\pgfqpoint{3.555762in}{2.048976in}}%
\pgfpathcurveto{\pgfqpoint{3.562490in}{2.042247in}}{\pgfqpoint{3.571616in}{2.038467in}}{\pgfqpoint{3.581132in}{2.038467in}}%
\pgfpathclose%
\pgfusepath{stroke,fill}%
\end{pgfscope}%
\begin{pgfscope}%
\pgfpathrectangle{\pgfqpoint{1.065196in}{0.528000in}}{\pgfqpoint{3.702804in}{3.696000in}} %
\pgfusepath{clip}%
\pgfsetbuttcap%
\pgfsetroundjoin%
\definecolor{currentfill}{rgb}{0.121569,0.466667,0.705882}%
\pgfsetfillcolor{currentfill}%
\pgfsetlinewidth{1.003750pt}%
\definecolor{currentstroke}{rgb}{0.121569,0.466667,0.705882}%
\pgfsetstrokecolor{currentstroke}%
\pgfsetdash{}{0pt}%
\pgfpathmoveto{\pgfqpoint{1.423659in}{2.444852in}}%
\pgfpathcurveto{\pgfqpoint{1.433466in}{2.444852in}}{\pgfqpoint{1.442872in}{2.448748in}}{\pgfqpoint{1.449806in}{2.455682in}}%
\pgfpathcurveto{\pgfqpoint{1.456740in}{2.462617in}}{\pgfqpoint{1.460637in}{2.472023in}}{\pgfqpoint{1.460637in}{2.481829in}}%
\pgfpathcurveto{\pgfqpoint{1.460637in}{2.491636in}}{\pgfqpoint{1.456740in}{2.501042in}}{\pgfqpoint{1.449806in}{2.507976in}}%
\pgfpathcurveto{\pgfqpoint{1.442872in}{2.514910in}}{\pgfqpoint{1.433466in}{2.518807in}}{\pgfqpoint{1.423659in}{2.518807in}}%
\pgfpathcurveto{\pgfqpoint{1.413853in}{2.518807in}}{\pgfqpoint{1.404446in}{2.514910in}}{\pgfqpoint{1.397512in}{2.507976in}}%
\pgfpathcurveto{\pgfqpoint{1.390578in}{2.501042in}}{\pgfqpoint{1.386682in}{2.491636in}}{\pgfqpoint{1.386682in}{2.481829in}}%
\pgfpathcurveto{\pgfqpoint{1.386682in}{2.472023in}}{\pgfqpoint{1.390578in}{2.462617in}}{\pgfqpoint{1.397512in}{2.455682in}}%
\pgfpathcurveto{\pgfqpoint{1.404446in}{2.448748in}}{\pgfqpoint{1.413853in}{2.444852in}}{\pgfqpoint{1.423659in}{2.444852in}}%
\pgfpathclose%
\pgfusepath{stroke,fill}%
\end{pgfscope}%
\begin{pgfscope}%
\pgfpathrectangle{\pgfqpoint{1.065196in}{0.528000in}}{\pgfqpoint{3.702804in}{3.696000in}} %
\pgfusepath{clip}%
\pgfsetbuttcap%
\pgfsetroundjoin%
\definecolor{currentfill}{rgb}{0.121569,0.466667,0.705882}%
\pgfsetfillcolor{currentfill}%
\pgfsetlinewidth{1.003750pt}%
\definecolor{currentstroke}{rgb}{0.121569,0.466667,0.705882}%
\pgfsetstrokecolor{currentstroke}%
\pgfsetdash{}{0pt}%
\pgfpathmoveto{\pgfqpoint{3.478671in}{2.374682in}}%
\pgfpathcurveto{\pgfqpoint{3.488436in}{2.374682in}}{\pgfqpoint{3.497802in}{2.378561in}}{\pgfqpoint{3.504706in}{2.385466in}}%
\pgfpathcurveto{\pgfqpoint{3.511611in}{2.392370in}}{\pgfqpoint{3.515491in}{2.401737in}}{\pgfqpoint{3.515491in}{2.411501in}}%
\pgfpathcurveto{\pgfqpoint{3.515491in}{2.421266in}}{\pgfqpoint{3.511611in}{2.430632in}}{\pgfqpoint{3.504706in}{2.437537in}}%
\pgfpathcurveto{\pgfqpoint{3.497802in}{2.444441in}}{\pgfqpoint{3.488436in}{2.448321in}}{\pgfqpoint{3.478671in}{2.448321in}}%
\pgfpathcurveto{\pgfqpoint{3.468906in}{2.448321in}}{\pgfqpoint{3.459540in}{2.444441in}}{\pgfqpoint{3.452635in}{2.437537in}}%
\pgfpathcurveto{\pgfqpoint{3.445731in}{2.430632in}}{\pgfqpoint{3.441851in}{2.421266in}}{\pgfqpoint{3.441851in}{2.411501in}}%
\pgfpathcurveto{\pgfqpoint{3.441851in}{2.401737in}}{\pgfqpoint{3.445731in}{2.392370in}}{\pgfqpoint{3.452635in}{2.385466in}}%
\pgfpathcurveto{\pgfqpoint{3.459540in}{2.378561in}}{\pgfqpoint{3.468906in}{2.374682in}}{\pgfqpoint{3.478671in}{2.374682in}}%
\pgfpathclose%
\pgfusepath{stroke,fill}%
\end{pgfscope}%
\begin{pgfscope}%
\pgfpathrectangle{\pgfqpoint{1.065196in}{0.528000in}}{\pgfqpoint{3.702804in}{3.696000in}} %
\pgfusepath{clip}%
\pgfsetbuttcap%
\pgfsetroundjoin%
\definecolor{currentfill}{rgb}{0.121569,0.466667,0.705882}%
\pgfsetfillcolor{currentfill}%
\pgfsetlinewidth{1.003750pt}%
\definecolor{currentstroke}{rgb}{0.121569,0.466667,0.705882}%
\pgfsetstrokecolor{currentstroke}%
\pgfsetdash{}{0pt}%
\pgfpathmoveto{\pgfqpoint{4.418731in}{2.603496in}}%
\pgfpathcurveto{\pgfqpoint{4.431441in}{2.603496in}}{\pgfqpoint{4.443633in}{2.608546in}}{\pgfqpoint{4.452621in}{2.617533in}}%
\pgfpathcurveto{\pgfqpoint{4.461609in}{2.626521in}}{\pgfqpoint{4.466659in}{2.638713in}}{\pgfqpoint{4.466659in}{2.651424in}}%
\pgfpathcurveto{\pgfqpoint{4.466659in}{2.664134in}}{\pgfqpoint{4.461609in}{2.676326in}}{\pgfqpoint{4.452621in}{2.685314in}}%
\pgfpathcurveto{\pgfqpoint{4.443633in}{2.694302in}}{\pgfqpoint{4.431441in}{2.699352in}}{\pgfqpoint{4.418731in}{2.699352in}}%
\pgfpathcurveto{\pgfqpoint{4.406020in}{2.699352in}}{\pgfqpoint{4.393828in}{2.694302in}}{\pgfqpoint{4.384840in}{2.685314in}}%
\pgfpathcurveto{\pgfqpoint{4.375853in}{2.676326in}}{\pgfqpoint{4.370802in}{2.664134in}}{\pgfqpoint{4.370802in}{2.651424in}}%
\pgfpathcurveto{\pgfqpoint{4.370802in}{2.638713in}}{\pgfqpoint{4.375853in}{2.626521in}}{\pgfqpoint{4.384840in}{2.617533in}}%
\pgfpathcurveto{\pgfqpoint{4.393828in}{2.608546in}}{\pgfqpoint{4.406020in}{2.603496in}}{\pgfqpoint{4.418731in}{2.603496in}}%
\pgfpathclose%
\pgfusepath{stroke,fill}%
\end{pgfscope}%
\begin{pgfscope}%
\pgfpathrectangle{\pgfqpoint{1.065196in}{0.528000in}}{\pgfqpoint{3.702804in}{3.696000in}} %
\pgfusepath{clip}%
\pgfsetbuttcap%
\pgfsetroundjoin%
\definecolor{currentfill}{rgb}{0.121569,0.466667,0.705882}%
\pgfsetfillcolor{currentfill}%
\pgfsetlinewidth{1.003750pt}%
\definecolor{currentstroke}{rgb}{0.121569,0.466667,0.705882}%
\pgfsetstrokecolor{currentstroke}%
\pgfsetdash{}{0pt}%
\pgfpathmoveto{\pgfqpoint{4.280826in}{1.114072in}}%
\pgfpathcurveto{\pgfqpoint{4.289823in}{1.114072in}}{\pgfqpoint{4.298452in}{1.117647in}}{\pgfqpoint{4.304814in}{1.124009in}}%
\pgfpathcurveto{\pgfqpoint{4.311176in}{1.130370in}}{\pgfqpoint{4.314750in}{1.139000in}}{\pgfqpoint{4.314750in}{1.147997in}}%
\pgfpathcurveto{\pgfqpoint{4.314750in}{1.156994in}}{\pgfqpoint{4.311176in}{1.165623in}}{\pgfqpoint{4.304814in}{1.171985in}}%
\pgfpathcurveto{\pgfqpoint{4.298452in}{1.178347in}}{\pgfqpoint{4.289823in}{1.181921in}}{\pgfqpoint{4.280826in}{1.181921in}}%
\pgfpathcurveto{\pgfqpoint{4.271829in}{1.181921in}}{\pgfqpoint{4.263199in}{1.178347in}}{\pgfqpoint{4.256837in}{1.171985in}}%
\pgfpathcurveto{\pgfqpoint{4.250476in}{1.165623in}}{\pgfqpoint{4.246901in}{1.156994in}}{\pgfqpoint{4.246901in}{1.147997in}}%
\pgfpathcurveto{\pgfqpoint{4.246901in}{1.139000in}}{\pgfqpoint{4.250476in}{1.130370in}}{\pgfqpoint{4.256837in}{1.124009in}}%
\pgfpathcurveto{\pgfqpoint{4.263199in}{1.117647in}}{\pgfqpoint{4.271829in}{1.114072in}}{\pgfqpoint{4.280826in}{1.114072in}}%
\pgfpathclose%
\pgfusepath{stroke,fill}%
\end{pgfscope}%
\begin{pgfscope}%
\pgfpathrectangle{\pgfqpoint{1.065196in}{0.528000in}}{\pgfqpoint{3.702804in}{3.696000in}} %
\pgfusepath{clip}%
\pgfsetbuttcap%
\pgfsetroundjoin%
\definecolor{currentfill}{rgb}{0.121569,0.466667,0.705882}%
\pgfsetfillcolor{currentfill}%
\pgfsetlinewidth{1.003750pt}%
\definecolor{currentstroke}{rgb}{0.121569,0.466667,0.705882}%
\pgfsetstrokecolor{currentstroke}%
\pgfsetdash{}{0pt}%
\pgfpathmoveto{\pgfqpoint{1.722693in}{3.339870in}}%
\pgfpathcurveto{\pgfqpoint{1.736184in}{3.339870in}}{\pgfqpoint{1.749123in}{3.345230in}}{\pgfqpoint{1.758662in}{3.354769in}}%
\pgfpathcurveto{\pgfqpoint{1.768201in}{3.364308in}}{\pgfqpoint{1.773561in}{3.377247in}}{\pgfqpoint{1.773561in}{3.390737in}}%
\pgfpathcurveto{\pgfqpoint{1.773561in}{3.404227in}}{\pgfqpoint{1.768201in}{3.417167in}}{\pgfqpoint{1.758662in}{3.426706in}}%
\pgfpathcurveto{\pgfqpoint{1.749123in}{3.436245in}}{\pgfqpoint{1.736184in}{3.441604in}}{\pgfqpoint{1.722693in}{3.441604in}}%
\pgfpathcurveto{\pgfqpoint{1.709203in}{3.441604in}}{\pgfqpoint{1.696264in}{3.436245in}}{\pgfqpoint{1.686725in}{3.426706in}}%
\pgfpathcurveto{\pgfqpoint{1.677186in}{3.417167in}}{\pgfqpoint{1.671826in}{3.404227in}}{\pgfqpoint{1.671826in}{3.390737in}}%
\pgfpathcurveto{\pgfqpoint{1.671826in}{3.377247in}}{\pgfqpoint{1.677186in}{3.364308in}}{\pgfqpoint{1.686725in}{3.354769in}}%
\pgfpathcurveto{\pgfqpoint{1.696264in}{3.345230in}}{\pgfqpoint{1.709203in}{3.339870in}}{\pgfqpoint{1.722693in}{3.339870in}}%
\pgfpathclose%
\pgfusepath{stroke,fill}%
\end{pgfscope}%
\begin{pgfscope}%
\pgfpathrectangle{\pgfqpoint{1.065196in}{0.528000in}}{\pgfqpoint{3.702804in}{3.696000in}} %
\pgfusepath{clip}%
\pgfsetbuttcap%
\pgfsetroundjoin%
\definecolor{currentfill}{rgb}{0.121569,0.466667,0.705882}%
\pgfsetfillcolor{currentfill}%
\pgfsetlinewidth{1.003750pt}%
\definecolor{currentstroke}{rgb}{0.121569,0.466667,0.705882}%
\pgfsetstrokecolor{currentstroke}%
\pgfsetdash{}{0pt}%
\pgfpathmoveto{\pgfqpoint{2.587764in}{1.231036in}}%
\pgfpathcurveto{\pgfqpoint{2.590495in}{1.231036in}}{\pgfqpoint{2.593114in}{1.232121in}}{\pgfqpoint{2.595044in}{1.234052in}}%
\pgfpathcurveto{\pgfqpoint{2.596975in}{1.235982in}}{\pgfqpoint{2.598060in}{1.238601in}}{\pgfqpoint{2.598060in}{1.241332in}}%
\pgfpathcurveto{\pgfqpoint{2.598060in}{1.244062in}}{\pgfqpoint{2.596975in}{1.246681in}}{\pgfqpoint{2.595044in}{1.248612in}}%
\pgfpathcurveto{\pgfqpoint{2.593114in}{1.250542in}}{\pgfqpoint{2.590495in}{1.251627in}}{\pgfqpoint{2.587764in}{1.251627in}}%
\pgfpathcurveto{\pgfqpoint{2.585034in}{1.251627in}}{\pgfqpoint{2.582415in}{1.250542in}}{\pgfqpoint{2.580484in}{1.248612in}}%
\pgfpathcurveto{\pgfqpoint{2.578553in}{1.246681in}}{\pgfqpoint{2.577469in}{1.244062in}}{\pgfqpoint{2.577469in}{1.241332in}}%
\pgfpathcurveto{\pgfqpoint{2.577469in}{1.238601in}}{\pgfqpoint{2.578553in}{1.235982in}}{\pgfqpoint{2.580484in}{1.234052in}}%
\pgfpathcurveto{\pgfqpoint{2.582415in}{1.232121in}}{\pgfqpoint{2.585034in}{1.231036in}}{\pgfqpoint{2.587764in}{1.231036in}}%
\pgfpathclose%
\pgfusepath{stroke,fill}%
\end{pgfscope}%
\begin{pgfscope}%
\pgfpathrectangle{\pgfqpoint{1.065196in}{0.528000in}}{\pgfqpoint{3.702804in}{3.696000in}} %
\pgfusepath{clip}%
\pgfsetbuttcap%
\pgfsetroundjoin%
\definecolor{currentfill}{rgb}{0.121569,0.466667,0.705882}%
\pgfsetfillcolor{currentfill}%
\pgfsetlinewidth{1.003750pt}%
\definecolor{currentstroke}{rgb}{0.121569,0.466667,0.705882}%
\pgfsetstrokecolor{currentstroke}%
\pgfsetdash{}{0pt}%
\pgfpathmoveto{\pgfqpoint{4.361530in}{1.804961in}}%
\pgfpathcurveto{\pgfqpoint{4.374007in}{1.804961in}}{\pgfqpoint{4.385974in}{1.809918in}}{\pgfqpoint{4.394796in}{1.818741in}}%
\pgfpathcurveto{\pgfqpoint{4.403619in}{1.827563in}}{\pgfqpoint{4.408576in}{1.839531in}}{\pgfqpoint{4.408576in}{1.852008in}}%
\pgfpathcurveto{\pgfqpoint{4.408576in}{1.864484in}}{\pgfqpoint{4.403619in}{1.876452in}}{\pgfqpoint{4.394796in}{1.885274in}}%
\pgfpathcurveto{\pgfqpoint{4.385974in}{1.894097in}}{\pgfqpoint{4.374007in}{1.899054in}}{\pgfqpoint{4.361530in}{1.899054in}}%
\pgfpathcurveto{\pgfqpoint{4.349053in}{1.899054in}}{\pgfqpoint{4.337086in}{1.894097in}}{\pgfqpoint{4.328263in}{1.885274in}}%
\pgfpathcurveto{\pgfqpoint{4.319441in}{1.876452in}}{\pgfqpoint{4.314484in}{1.864484in}}{\pgfqpoint{4.314484in}{1.852008in}}%
\pgfpathcurveto{\pgfqpoint{4.314484in}{1.839531in}}{\pgfqpoint{4.319441in}{1.827563in}}{\pgfqpoint{4.328263in}{1.818741in}}%
\pgfpathcurveto{\pgfqpoint{4.337086in}{1.809918in}}{\pgfqpoint{4.349053in}{1.804961in}}{\pgfqpoint{4.361530in}{1.804961in}}%
\pgfpathclose%
\pgfusepath{stroke,fill}%
\end{pgfscope}%
\begin{pgfscope}%
\pgfpathrectangle{\pgfqpoint{1.065196in}{0.528000in}}{\pgfqpoint{3.702804in}{3.696000in}} %
\pgfusepath{clip}%
\pgfsetbuttcap%
\pgfsetroundjoin%
\definecolor{currentfill}{rgb}{0.121569,0.466667,0.705882}%
\pgfsetfillcolor{currentfill}%
\pgfsetlinewidth{1.003750pt}%
\definecolor{currentstroke}{rgb}{0.121569,0.466667,0.705882}%
\pgfsetstrokecolor{currentstroke}%
\pgfsetdash{}{0pt}%
\pgfpathmoveto{\pgfqpoint{3.769987in}{3.099293in}}%
\pgfpathcurveto{\pgfqpoint{3.775014in}{3.099293in}}{\pgfqpoint{3.779836in}{3.101290in}}{\pgfqpoint{3.783391in}{3.104845in}}%
\pgfpathcurveto{\pgfqpoint{3.786946in}{3.108400in}}{\pgfqpoint{3.788943in}{3.113222in}}{\pgfqpoint{3.788943in}{3.118249in}}%
\pgfpathcurveto{\pgfqpoint{3.788943in}{3.123277in}}{\pgfqpoint{3.786946in}{3.128099in}}{\pgfqpoint{3.783391in}{3.131654in}}%
\pgfpathcurveto{\pgfqpoint{3.779836in}{3.135209in}}{\pgfqpoint{3.775014in}{3.137206in}}{\pgfqpoint{3.769987in}{3.137206in}}%
\pgfpathcurveto{\pgfqpoint{3.764960in}{3.137206in}}{\pgfqpoint{3.760138in}{3.135209in}}{\pgfqpoint{3.756583in}{3.131654in}}%
\pgfpathcurveto{\pgfqpoint{3.753028in}{3.128099in}}{\pgfqpoint{3.751031in}{3.123277in}}{\pgfqpoint{3.751031in}{3.118249in}}%
\pgfpathcurveto{\pgfqpoint{3.751031in}{3.113222in}}{\pgfqpoint{3.753028in}{3.108400in}}{\pgfqpoint{3.756583in}{3.104845in}}%
\pgfpathcurveto{\pgfqpoint{3.760138in}{3.101290in}}{\pgfqpoint{3.764960in}{3.099293in}}{\pgfqpoint{3.769987in}{3.099293in}}%
\pgfpathclose%
\pgfusepath{stroke,fill}%
\end{pgfscope}%
\begin{pgfscope}%
\pgfpathrectangle{\pgfqpoint{1.065196in}{0.528000in}}{\pgfqpoint{3.702804in}{3.696000in}} %
\pgfusepath{clip}%
\pgfsetbuttcap%
\pgfsetroundjoin%
\definecolor{currentfill}{rgb}{0.121569,0.466667,0.705882}%
\pgfsetfillcolor{currentfill}%
\pgfsetlinewidth{1.003750pt}%
\definecolor{currentstroke}{rgb}{0.121569,0.466667,0.705882}%
\pgfsetstrokecolor{currentstroke}%
\pgfsetdash{}{0pt}%
\pgfpathmoveto{\pgfqpoint{4.213549in}{2.733625in}}%
\pgfpathcurveto{\pgfqpoint{4.224703in}{2.733625in}}{\pgfqpoint{4.235402in}{2.738056in}}{\pgfqpoint{4.243290in}{2.745944in}}%
\pgfpathcurveto{\pgfqpoint{4.251177in}{2.753831in}}{\pgfqpoint{4.255609in}{2.764530in}}{\pgfqpoint{4.255609in}{2.775684in}}%
\pgfpathcurveto{\pgfqpoint{4.255609in}{2.786839in}}{\pgfqpoint{4.251177in}{2.797538in}}{\pgfqpoint{4.243290in}{2.805425in}}%
\pgfpathcurveto{\pgfqpoint{4.235402in}{2.813312in}}{\pgfqpoint{4.224703in}{2.817744in}}{\pgfqpoint{4.213549in}{2.817744in}}%
\pgfpathcurveto{\pgfqpoint{4.202395in}{2.817744in}}{\pgfqpoint{4.191696in}{2.813312in}}{\pgfqpoint{4.183809in}{2.805425in}}%
\pgfpathcurveto{\pgfqpoint{4.175921in}{2.797538in}}{\pgfqpoint{4.171490in}{2.786839in}}{\pgfqpoint{4.171490in}{2.775684in}}%
\pgfpathcurveto{\pgfqpoint{4.171490in}{2.764530in}}{\pgfqpoint{4.175921in}{2.753831in}}{\pgfqpoint{4.183809in}{2.745944in}}%
\pgfpathcurveto{\pgfqpoint{4.191696in}{2.738056in}}{\pgfqpoint{4.202395in}{2.733625in}}{\pgfqpoint{4.213549in}{2.733625in}}%
\pgfpathclose%
\pgfusepath{stroke,fill}%
\end{pgfscope}%
\begin{pgfscope}%
\pgfpathrectangle{\pgfqpoint{1.065196in}{0.528000in}}{\pgfqpoint{3.702804in}{3.696000in}} %
\pgfusepath{clip}%
\pgfsetbuttcap%
\pgfsetroundjoin%
\definecolor{currentfill}{rgb}{0.121569,0.466667,0.705882}%
\pgfsetfillcolor{currentfill}%
\pgfsetlinewidth{1.003750pt}%
\definecolor{currentstroke}{rgb}{0.121569,0.466667,0.705882}%
\pgfsetstrokecolor{currentstroke}%
\pgfsetdash{}{0pt}%
\pgfpathmoveto{\pgfqpoint{3.770423in}{1.850354in}}%
\pgfpathcurveto{\pgfqpoint{3.771867in}{1.850354in}}{\pgfqpoint{3.773252in}{1.850928in}}{\pgfqpoint{3.774273in}{1.851949in}}%
\pgfpathcurveto{\pgfqpoint{3.775294in}{1.852970in}}{\pgfqpoint{3.775868in}{1.854355in}}{\pgfqpoint{3.775868in}{1.855799in}}%
\pgfpathcurveto{\pgfqpoint{3.775868in}{1.857243in}}{\pgfqpoint{3.775294in}{1.858628in}}{\pgfqpoint{3.774273in}{1.859649in}}%
\pgfpathcurveto{\pgfqpoint{3.773252in}{1.860670in}}{\pgfqpoint{3.771867in}{1.861243in}}{\pgfqpoint{3.770423in}{1.861243in}}%
\pgfpathcurveto{\pgfqpoint{3.768979in}{1.861243in}}{\pgfqpoint{3.767594in}{1.860670in}}{\pgfqpoint{3.766573in}{1.859649in}}%
\pgfpathcurveto{\pgfqpoint{3.765552in}{1.858628in}}{\pgfqpoint{3.764979in}{1.857243in}}{\pgfqpoint{3.764979in}{1.855799in}}%
\pgfpathcurveto{\pgfqpoint{3.764979in}{1.854355in}}{\pgfqpoint{3.765552in}{1.852970in}}{\pgfqpoint{3.766573in}{1.851949in}}%
\pgfpathcurveto{\pgfqpoint{3.767594in}{1.850928in}}{\pgfqpoint{3.768979in}{1.850354in}}{\pgfqpoint{3.770423in}{1.850354in}}%
\pgfpathclose%
\pgfusepath{stroke,fill}%
\end{pgfscope}%
\begin{pgfscope}%
\pgfpathrectangle{\pgfqpoint{1.065196in}{0.528000in}}{\pgfqpoint{3.702804in}{3.696000in}} %
\pgfusepath{clip}%
\pgfsetbuttcap%
\pgfsetroundjoin%
\definecolor{currentfill}{rgb}{0.121569,0.466667,0.705882}%
\pgfsetfillcolor{currentfill}%
\pgfsetlinewidth{1.003750pt}%
\definecolor{currentstroke}{rgb}{0.121569,0.466667,0.705882}%
\pgfsetstrokecolor{currentstroke}%
\pgfsetdash{}{0pt}%
\pgfpathmoveto{\pgfqpoint{2.160087in}{3.647750in}}%
\pgfpathcurveto{\pgfqpoint{2.170497in}{3.647750in}}{\pgfqpoint{2.180481in}{3.651886in}}{\pgfqpoint{2.187841in}{3.659246in}}%
\pgfpathcurveto{\pgfqpoint{2.195201in}{3.666606in}}{\pgfqpoint{2.199337in}{3.676591in}}{\pgfqpoint{2.199337in}{3.687000in}}%
\pgfpathcurveto{\pgfqpoint{2.199337in}{3.697409in}}{\pgfqpoint{2.195201in}{3.707393in}}{\pgfqpoint{2.187841in}{3.714753in}}%
\pgfpathcurveto{\pgfqpoint{2.180481in}{3.722114in}}{\pgfqpoint{2.170497in}{3.726249in}}{\pgfqpoint{2.160087in}{3.726249in}}%
\pgfpathcurveto{\pgfqpoint{2.149678in}{3.726249in}}{\pgfqpoint{2.139694in}{3.722114in}}{\pgfqpoint{2.132334in}{3.714753in}}%
\pgfpathcurveto{\pgfqpoint{2.124974in}{3.707393in}}{\pgfqpoint{2.120838in}{3.697409in}}{\pgfqpoint{2.120838in}{3.687000in}}%
\pgfpathcurveto{\pgfqpoint{2.120838in}{3.676591in}}{\pgfqpoint{2.124974in}{3.666606in}}{\pgfqpoint{2.132334in}{3.659246in}}%
\pgfpathcurveto{\pgfqpoint{2.139694in}{3.651886in}}{\pgfqpoint{2.149678in}{3.647750in}}{\pgfqpoint{2.160087in}{3.647750in}}%
\pgfpathclose%
\pgfusepath{stroke,fill}%
\end{pgfscope}%
\begin{pgfscope}%
\pgfpathrectangle{\pgfqpoint{1.065196in}{0.528000in}}{\pgfqpoint{3.702804in}{3.696000in}} %
\pgfusepath{clip}%
\pgfsetbuttcap%
\pgfsetroundjoin%
\definecolor{currentfill}{rgb}{0.121569,0.466667,0.705882}%
\pgfsetfillcolor{currentfill}%
\pgfsetlinewidth{1.003750pt}%
\definecolor{currentstroke}{rgb}{0.121569,0.466667,0.705882}%
\pgfsetstrokecolor{currentstroke}%
\pgfsetdash{}{0pt}%
\pgfpathmoveto{\pgfqpoint{2.689585in}{3.914905in}}%
\pgfpathcurveto{\pgfqpoint{2.690364in}{3.914905in}}{\pgfqpoint{2.691112in}{3.915214in}}{\pgfqpoint{2.691662in}{3.915765in}}%
\pgfpathcurveto{\pgfqpoint{2.692213in}{3.916316in}}{\pgfqpoint{2.692523in}{3.917063in}}{\pgfqpoint{2.692523in}{3.917843in}}%
\pgfpathcurveto{\pgfqpoint{2.692523in}{3.918622in}}{\pgfqpoint{2.692213in}{3.919369in}}{\pgfqpoint{2.691662in}{3.919920in}}%
\pgfpathcurveto{\pgfqpoint{2.691112in}{3.920471in}}{\pgfqpoint{2.690364in}{3.920781in}}{\pgfqpoint{2.689585in}{3.920781in}}%
\pgfpathcurveto{\pgfqpoint{2.688806in}{3.920781in}}{\pgfqpoint{2.688059in}{3.920471in}}{\pgfqpoint{2.687508in}{3.919920in}}%
\pgfpathcurveto{\pgfqpoint{2.686957in}{3.919369in}}{\pgfqpoint{2.686647in}{3.918622in}}{\pgfqpoint{2.686647in}{3.917843in}}%
\pgfpathcurveto{\pgfqpoint{2.686647in}{3.917063in}}{\pgfqpoint{2.686957in}{3.916316in}}{\pgfqpoint{2.687508in}{3.915765in}}%
\pgfpathcurveto{\pgfqpoint{2.688059in}{3.915214in}}{\pgfqpoint{2.688806in}{3.914905in}}{\pgfqpoint{2.689585in}{3.914905in}}%
\pgfpathclose%
\pgfusepath{stroke,fill}%
\end{pgfscope}%
\begin{pgfscope}%
\pgfpathrectangle{\pgfqpoint{1.065196in}{0.528000in}}{\pgfqpoint{3.702804in}{3.696000in}} %
\pgfusepath{clip}%
\pgfsetbuttcap%
\pgfsetroundjoin%
\definecolor{currentfill}{rgb}{0.121569,0.466667,0.705882}%
\pgfsetfillcolor{currentfill}%
\pgfsetlinewidth{1.003750pt}%
\definecolor{currentstroke}{rgb}{0.121569,0.466667,0.705882}%
\pgfsetstrokecolor{currentstroke}%
\pgfsetdash{}{0pt}%
\pgfpathmoveto{\pgfqpoint{3.477489in}{2.716909in}}%
\pgfpathcurveto{\pgfqpoint{3.491321in}{2.716909in}}{\pgfqpoint{3.504589in}{2.722405in}}{\pgfqpoint{3.514370in}{2.732186in}}%
\pgfpathcurveto{\pgfqpoint{3.524151in}{2.741967in}}{\pgfqpoint{3.529647in}{2.755235in}}{\pgfqpoint{3.529647in}{2.769067in}}%
\pgfpathcurveto{\pgfqpoint{3.529647in}{2.782900in}}{\pgfqpoint{3.524151in}{2.796168in}}{\pgfqpoint{3.514370in}{2.805949in}}%
\pgfpathcurveto{\pgfqpoint{3.504589in}{2.815730in}}{\pgfqpoint{3.491321in}{2.821226in}}{\pgfqpoint{3.477489in}{2.821226in}}%
\pgfpathcurveto{\pgfqpoint{3.463656in}{2.821226in}}{\pgfqpoint{3.450388in}{2.815730in}}{\pgfqpoint{3.440607in}{2.805949in}}%
\pgfpathcurveto{\pgfqpoint{3.430826in}{2.796168in}}{\pgfqpoint{3.425330in}{2.782900in}}{\pgfqpoint{3.425330in}{2.769067in}}%
\pgfpathcurveto{\pgfqpoint{3.425330in}{2.755235in}}{\pgfqpoint{3.430826in}{2.741967in}}{\pgfqpoint{3.440607in}{2.732186in}}%
\pgfpathcurveto{\pgfqpoint{3.450388in}{2.722405in}}{\pgfqpoint{3.463656in}{2.716909in}}{\pgfqpoint{3.477489in}{2.716909in}}%
\pgfpathclose%
\pgfusepath{stroke,fill}%
\end{pgfscope}%
\begin{pgfscope}%
\pgfpathrectangle{\pgfqpoint{1.065196in}{0.528000in}}{\pgfqpoint{3.702804in}{3.696000in}} %
\pgfusepath{clip}%
\pgfsetbuttcap%
\pgfsetroundjoin%
\definecolor{currentfill}{rgb}{0.121569,0.466667,0.705882}%
\pgfsetfillcolor{currentfill}%
\pgfsetlinewidth{1.003750pt}%
\definecolor{currentstroke}{rgb}{0.121569,0.466667,0.705882}%
\pgfsetstrokecolor{currentstroke}%
\pgfsetdash{}{0pt}%
\pgfpathmoveto{\pgfqpoint{1.640571in}{3.815564in}}%
\pgfpathcurveto{\pgfqpoint{1.654066in}{3.815564in}}{\pgfqpoint{1.667011in}{3.820926in}}{\pgfqpoint{1.676554in}{3.830469in}}%
\pgfpathcurveto{\pgfqpoint{1.686097in}{3.840011in}}{\pgfqpoint{1.691458in}{3.852956in}}{\pgfqpoint{1.691458in}{3.866451in}}%
\pgfpathcurveto{\pgfqpoint{1.691458in}{3.879947in}}{\pgfqpoint{1.686097in}{3.892892in}}{\pgfqpoint{1.676554in}{3.902434in}}%
\pgfpathcurveto{\pgfqpoint{1.667011in}{3.911977in}}{\pgfqpoint{1.654066in}{3.917339in}}{\pgfqpoint{1.640571in}{3.917339in}}%
\pgfpathcurveto{\pgfqpoint{1.627075in}{3.917339in}}{\pgfqpoint{1.614131in}{3.911977in}}{\pgfqpoint{1.604588in}{3.902434in}}%
\pgfpathcurveto{\pgfqpoint{1.595045in}{3.892892in}}{\pgfqpoint{1.589684in}{3.879947in}}{\pgfqpoint{1.589684in}{3.866451in}}%
\pgfpathcurveto{\pgfqpoint{1.589684in}{3.852956in}}{\pgfqpoint{1.595045in}{3.840011in}}{\pgfqpoint{1.604588in}{3.830469in}}%
\pgfpathcurveto{\pgfqpoint{1.614131in}{3.820926in}}{\pgfqpoint{1.627075in}{3.815564in}}{\pgfqpoint{1.640571in}{3.815564in}}%
\pgfpathclose%
\pgfusepath{stroke,fill}%
\end{pgfscope}%
\begin{pgfscope}%
\pgfpathrectangle{\pgfqpoint{1.065196in}{0.528000in}}{\pgfqpoint{3.702804in}{3.696000in}} %
\pgfusepath{clip}%
\pgfsetbuttcap%
\pgfsetroundjoin%
\definecolor{currentfill}{rgb}{0.121569,0.466667,0.705882}%
\pgfsetfillcolor{currentfill}%
\pgfsetlinewidth{1.003750pt}%
\definecolor{currentstroke}{rgb}{0.121569,0.466667,0.705882}%
\pgfsetstrokecolor{currentstroke}%
\pgfsetdash{}{0pt}%
\pgfpathmoveto{\pgfqpoint{2.762637in}{2.576375in}}%
\pgfpathcurveto{\pgfqpoint{2.775291in}{2.576375in}}{\pgfqpoint{2.787428in}{2.581402in}}{\pgfqpoint{2.796375in}{2.590349in}}%
\pgfpathcurveto{\pgfqpoint{2.805323in}{2.599297in}}{\pgfqpoint{2.810350in}{2.611434in}}{\pgfqpoint{2.810350in}{2.624088in}}%
\pgfpathcurveto{\pgfqpoint{2.810350in}{2.636741in}}{\pgfqpoint{2.805323in}{2.648878in}}{\pgfqpoint{2.796375in}{2.657826in}}%
\pgfpathcurveto{\pgfqpoint{2.787428in}{2.666773in}}{\pgfqpoint{2.775291in}{2.671801in}}{\pgfqpoint{2.762637in}{2.671801in}}%
\pgfpathcurveto{\pgfqpoint{2.749983in}{2.671801in}}{\pgfqpoint{2.737846in}{2.666773in}}{\pgfqpoint{2.728899in}{2.657826in}}%
\pgfpathcurveto{\pgfqpoint{2.719951in}{2.648878in}}{\pgfqpoint{2.714924in}{2.636741in}}{\pgfqpoint{2.714924in}{2.624088in}}%
\pgfpathcurveto{\pgfqpoint{2.714924in}{2.611434in}}{\pgfqpoint{2.719951in}{2.599297in}}{\pgfqpoint{2.728899in}{2.590349in}}%
\pgfpathcurveto{\pgfqpoint{2.737846in}{2.581402in}}{\pgfqpoint{2.749983in}{2.576375in}}{\pgfqpoint{2.762637in}{2.576375in}}%
\pgfpathclose%
\pgfusepath{stroke,fill}%
\end{pgfscope}%
\begin{pgfscope}%
\pgfpathrectangle{\pgfqpoint{1.065196in}{0.528000in}}{\pgfqpoint{3.702804in}{3.696000in}} %
\pgfusepath{clip}%
\pgfsetbuttcap%
\pgfsetroundjoin%
\definecolor{currentfill}{rgb}{0.121569,0.466667,0.705882}%
\pgfsetfillcolor{currentfill}%
\pgfsetlinewidth{1.003750pt}%
\definecolor{currentstroke}{rgb}{0.121569,0.466667,0.705882}%
\pgfsetstrokecolor{currentstroke}%
\pgfsetdash{}{0pt}%
\pgfpathmoveto{\pgfqpoint{2.622782in}{1.430113in}}%
\pgfpathcurveto{\pgfqpoint{2.636351in}{1.430113in}}{\pgfqpoint{2.649366in}{1.435504in}}{\pgfqpoint{2.658960in}{1.445099in}}%
\pgfpathcurveto{\pgfqpoint{2.668555in}{1.454693in}}{\pgfqpoint{2.673946in}{1.467708in}}{\pgfqpoint{2.673946in}{1.481277in}}%
\pgfpathcurveto{\pgfqpoint{2.673946in}{1.494846in}}{\pgfqpoint{2.668555in}{1.507861in}}{\pgfqpoint{2.658960in}{1.517456in}}%
\pgfpathcurveto{\pgfqpoint{2.649366in}{1.527050in}}{\pgfqpoint{2.636351in}{1.532441in}}{\pgfqpoint{2.622782in}{1.532441in}}%
\pgfpathcurveto{\pgfqpoint{2.609213in}{1.532441in}}{\pgfqpoint{2.596198in}{1.527050in}}{\pgfqpoint{2.586603in}{1.517456in}}%
\pgfpathcurveto{\pgfqpoint{2.577009in}{1.507861in}}{\pgfqpoint{2.571618in}{1.494846in}}{\pgfqpoint{2.571618in}{1.481277in}}%
\pgfpathcurveto{\pgfqpoint{2.571618in}{1.467708in}}{\pgfqpoint{2.577009in}{1.454693in}}{\pgfqpoint{2.586603in}{1.445099in}}%
\pgfpathcurveto{\pgfqpoint{2.596198in}{1.435504in}}{\pgfqpoint{2.609213in}{1.430113in}}{\pgfqpoint{2.622782in}{1.430113in}}%
\pgfpathclose%
\pgfusepath{stroke,fill}%
\end{pgfscope}%
\begin{pgfscope}%
\pgfpathrectangle{\pgfqpoint{1.065196in}{0.528000in}}{\pgfqpoint{3.702804in}{3.696000in}} %
\pgfusepath{clip}%
\pgfsetbuttcap%
\pgfsetroundjoin%
\definecolor{currentfill}{rgb}{0.121569,0.466667,0.705882}%
\pgfsetfillcolor{currentfill}%
\pgfsetlinewidth{1.003750pt}%
\definecolor{currentstroke}{rgb}{0.121569,0.466667,0.705882}%
\pgfsetstrokecolor{currentstroke}%
\pgfsetdash{}{0pt}%
\pgfpathmoveto{\pgfqpoint{4.280751in}{2.587915in}}%
\pgfpathcurveto{\pgfqpoint{4.286162in}{2.587915in}}{\pgfqpoint{4.291352in}{2.590065in}}{\pgfqpoint{4.295178in}{2.593892in}}%
\pgfpathcurveto{\pgfqpoint{4.299005in}{2.597718in}}{\pgfqpoint{4.301154in}{2.602908in}}{\pgfqpoint{4.301154in}{2.608319in}}%
\pgfpathcurveto{\pgfqpoint{4.301154in}{2.613730in}}{\pgfqpoint{4.299005in}{2.618920in}}{\pgfqpoint{4.295178in}{2.622747in}}%
\pgfpathcurveto{\pgfqpoint{4.291352in}{2.626573in}}{\pgfqpoint{4.286162in}{2.628723in}}{\pgfqpoint{4.280751in}{2.628723in}}%
\pgfpathcurveto{\pgfqpoint{4.275340in}{2.628723in}}{\pgfqpoint{4.270149in}{2.626573in}}{\pgfqpoint{4.266323in}{2.622747in}}%
\pgfpathcurveto{\pgfqpoint{4.262497in}{2.618920in}}{\pgfqpoint{4.260347in}{2.613730in}}{\pgfqpoint{4.260347in}{2.608319in}}%
\pgfpathcurveto{\pgfqpoint{4.260347in}{2.602908in}}{\pgfqpoint{4.262497in}{2.597718in}}{\pgfqpoint{4.266323in}{2.593892in}}%
\pgfpathcurveto{\pgfqpoint{4.270149in}{2.590065in}}{\pgfqpoint{4.275340in}{2.587915in}}{\pgfqpoint{4.280751in}{2.587915in}}%
\pgfpathclose%
\pgfusepath{stroke,fill}%
\end{pgfscope}%
\begin{pgfscope}%
\pgfpathrectangle{\pgfqpoint{1.065196in}{0.528000in}}{\pgfqpoint{3.702804in}{3.696000in}} %
\pgfusepath{clip}%
\pgfsetbuttcap%
\pgfsetroundjoin%
\definecolor{currentfill}{rgb}{0.121569,0.466667,0.705882}%
\pgfsetfillcolor{currentfill}%
\pgfsetlinewidth{1.003750pt}%
\definecolor{currentstroke}{rgb}{0.121569,0.466667,0.705882}%
\pgfsetstrokecolor{currentstroke}%
\pgfsetdash{}{0pt}%
\pgfpathmoveto{\pgfqpoint{1.266035in}{2.732593in}}%
\pgfpathcurveto{\pgfqpoint{1.271668in}{2.732593in}}{\pgfqpoint{1.277070in}{2.734831in}}{\pgfqpoint{1.281053in}{2.738814in}}%
\pgfpathcurveto{\pgfqpoint{1.285036in}{2.742797in}}{\pgfqpoint{1.287274in}{2.748199in}}{\pgfqpoint{1.287274in}{2.753832in}}%
\pgfpathcurveto{\pgfqpoint{1.287274in}{2.759465in}}{\pgfqpoint{1.285036in}{2.764868in}}{\pgfqpoint{1.281053in}{2.768851in}}%
\pgfpathcurveto{\pgfqpoint{1.277070in}{2.772834in}}{\pgfqpoint{1.271668in}{2.775072in}}{\pgfqpoint{1.266035in}{2.775072in}}%
\pgfpathcurveto{\pgfqpoint{1.260402in}{2.775072in}}{\pgfqpoint{1.254999in}{2.772834in}}{\pgfqpoint{1.251016in}{2.768851in}}%
\pgfpathcurveto{\pgfqpoint{1.247033in}{2.764868in}}{\pgfqpoint{1.244795in}{2.759465in}}{\pgfqpoint{1.244795in}{2.753832in}}%
\pgfpathcurveto{\pgfqpoint{1.244795in}{2.748199in}}{\pgfqpoint{1.247033in}{2.742797in}}{\pgfqpoint{1.251016in}{2.738814in}}%
\pgfpathcurveto{\pgfqpoint{1.254999in}{2.734831in}}{\pgfqpoint{1.260402in}{2.732593in}}{\pgfqpoint{1.266035in}{2.732593in}}%
\pgfpathclose%
\pgfusepath{stroke,fill}%
\end{pgfscope}%
\begin{pgfscope}%
\pgfpathrectangle{\pgfqpoint{1.065196in}{0.528000in}}{\pgfqpoint{3.702804in}{3.696000in}} %
\pgfusepath{clip}%
\pgfsetbuttcap%
\pgfsetroundjoin%
\definecolor{currentfill}{rgb}{0.121569,0.466667,0.705882}%
\pgfsetfillcolor{currentfill}%
\pgfsetlinewidth{1.003750pt}%
\definecolor{currentstroke}{rgb}{0.121569,0.466667,0.705882}%
\pgfsetstrokecolor{currentstroke}%
\pgfsetdash{}{0pt}%
\pgfpathmoveto{\pgfqpoint{2.349964in}{2.429075in}}%
\pgfpathcurveto{\pgfqpoint{2.356108in}{2.429075in}}{\pgfqpoint{2.362000in}{2.431516in}}{\pgfqpoint{2.366345in}{2.435860in}}%
\pgfpathcurveto{\pgfqpoint{2.370689in}{2.440204in}}{\pgfqpoint{2.373130in}{2.446097in}}{\pgfqpoint{2.373130in}{2.452240in}}%
\pgfpathcurveto{\pgfqpoint{2.373130in}{2.458384in}}{\pgfqpoint{2.370689in}{2.464277in}}{\pgfqpoint{2.366345in}{2.468621in}}%
\pgfpathcurveto{\pgfqpoint{2.362000in}{2.472965in}}{\pgfqpoint{2.356108in}{2.475406in}}{\pgfqpoint{2.349964in}{2.475406in}}%
\pgfpathcurveto{\pgfqpoint{2.343821in}{2.475406in}}{\pgfqpoint{2.337928in}{2.472965in}}{\pgfqpoint{2.333584in}{2.468621in}}%
\pgfpathcurveto{\pgfqpoint{2.329239in}{2.464277in}}{\pgfqpoint{2.326799in}{2.458384in}}{\pgfqpoint{2.326799in}{2.452240in}}%
\pgfpathcurveto{\pgfqpoint{2.326799in}{2.446097in}}{\pgfqpoint{2.329239in}{2.440204in}}{\pgfqpoint{2.333584in}{2.435860in}}%
\pgfpathcurveto{\pgfqpoint{2.337928in}{2.431516in}}{\pgfqpoint{2.343821in}{2.429075in}}{\pgfqpoint{2.349964in}{2.429075in}}%
\pgfpathclose%
\pgfusepath{stroke,fill}%
\end{pgfscope}%
\begin{pgfscope}%
\pgfpathrectangle{\pgfqpoint{1.065196in}{0.528000in}}{\pgfqpoint{3.702804in}{3.696000in}} %
\pgfusepath{clip}%
\pgfsetbuttcap%
\pgfsetroundjoin%
\definecolor{currentfill}{rgb}{0.121569,0.466667,0.705882}%
\pgfsetfillcolor{currentfill}%
\pgfsetlinewidth{1.003750pt}%
\definecolor{currentstroke}{rgb}{0.121569,0.466667,0.705882}%
\pgfsetstrokecolor{currentstroke}%
\pgfsetdash{}{0pt}%
\pgfpathmoveto{\pgfqpoint{4.222374in}{1.841630in}}%
\pgfpathcurveto{\pgfqpoint{4.233564in}{1.841630in}}{\pgfqpoint{4.244297in}{1.846076in}}{\pgfqpoint{4.252210in}{1.853989in}}%
\pgfpathcurveto{\pgfqpoint{4.260123in}{1.861901in}}{\pgfqpoint{4.264569in}{1.872635in}}{\pgfqpoint{4.264569in}{1.883825in}}%
\pgfpathcurveto{\pgfqpoint{4.264569in}{1.895015in}}{\pgfqpoint{4.260123in}{1.905748in}}{\pgfqpoint{4.252210in}{1.913661in}}%
\pgfpathcurveto{\pgfqpoint{4.244297in}{1.921573in}}{\pgfqpoint{4.233564in}{1.926019in}}{\pgfqpoint{4.222374in}{1.926019in}}%
\pgfpathcurveto{\pgfqpoint{4.211184in}{1.926019in}}{\pgfqpoint{4.200450in}{1.921573in}}{\pgfqpoint{4.192538in}{1.913661in}}%
\pgfpathcurveto{\pgfqpoint{4.184625in}{1.905748in}}{\pgfqpoint{4.180179in}{1.895015in}}{\pgfqpoint{4.180179in}{1.883825in}}%
\pgfpathcurveto{\pgfqpoint{4.180179in}{1.872635in}}{\pgfqpoint{4.184625in}{1.861901in}}{\pgfqpoint{4.192538in}{1.853989in}}%
\pgfpathcurveto{\pgfqpoint{4.200450in}{1.846076in}}{\pgfqpoint{4.211184in}{1.841630in}}{\pgfqpoint{4.222374in}{1.841630in}}%
\pgfpathclose%
\pgfusepath{stroke,fill}%
\end{pgfscope}%
\begin{pgfscope}%
\pgfpathrectangle{\pgfqpoint{1.065196in}{0.528000in}}{\pgfqpoint{3.702804in}{3.696000in}} %
\pgfusepath{clip}%
\pgfsetbuttcap%
\pgfsetroundjoin%
\definecolor{currentfill}{rgb}{0.121569,0.466667,0.705882}%
\pgfsetfillcolor{currentfill}%
\pgfsetlinewidth{1.003750pt}%
\definecolor{currentstroke}{rgb}{0.121569,0.466667,0.705882}%
\pgfsetstrokecolor{currentstroke}%
\pgfsetdash{}{0pt}%
\pgfpathmoveto{\pgfqpoint{4.298011in}{2.723741in}}%
\pgfpathcurveto{\pgfqpoint{4.311509in}{2.723741in}}{\pgfqpoint{4.324457in}{2.729104in}}{\pgfqpoint{4.334002in}{2.738648in}}%
\pgfpathcurveto{\pgfqpoint{4.343547in}{2.748193in}}{\pgfqpoint{4.348910in}{2.761141in}}{\pgfqpoint{4.348910in}{2.774639in}}%
\pgfpathcurveto{\pgfqpoint{4.348910in}{2.788138in}}{\pgfqpoint{4.343547in}{2.801085in}}{\pgfqpoint{4.334002in}{2.810630in}}%
\pgfpathcurveto{\pgfqpoint{4.324457in}{2.820175in}}{\pgfqpoint{4.311509in}{2.825538in}}{\pgfqpoint{4.298011in}{2.825538in}}%
\pgfpathcurveto{\pgfqpoint{4.284512in}{2.825538in}}{\pgfqpoint{4.271565in}{2.820175in}}{\pgfqpoint{4.262020in}{2.810630in}}%
\pgfpathcurveto{\pgfqpoint{4.252475in}{2.801085in}}{\pgfqpoint{4.247112in}{2.788138in}}{\pgfqpoint{4.247112in}{2.774639in}}%
\pgfpathcurveto{\pgfqpoint{4.247112in}{2.761141in}}{\pgfqpoint{4.252475in}{2.748193in}}{\pgfqpoint{4.262020in}{2.738648in}}%
\pgfpathcurveto{\pgfqpoint{4.271565in}{2.729104in}}{\pgfqpoint{4.284512in}{2.723741in}}{\pgfqpoint{4.298011in}{2.723741in}}%
\pgfpathclose%
\pgfusepath{stroke,fill}%
\end{pgfscope}%
\begin{pgfscope}%
\pgfpathrectangle{\pgfqpoint{1.065196in}{0.528000in}}{\pgfqpoint{3.702804in}{3.696000in}} %
\pgfusepath{clip}%
\pgfsetbuttcap%
\pgfsetroundjoin%
\definecolor{currentfill}{rgb}{0.121569,0.466667,0.705882}%
\pgfsetfillcolor{currentfill}%
\pgfsetlinewidth{1.003750pt}%
\definecolor{currentstroke}{rgb}{0.121569,0.466667,0.705882}%
\pgfsetstrokecolor{currentstroke}%
\pgfsetdash{}{0pt}%
\pgfpathmoveto{\pgfqpoint{1.309392in}{3.746111in}}%
\pgfpathcurveto{\pgfqpoint{1.323503in}{3.746111in}}{\pgfqpoint{1.337039in}{3.751718in}}{\pgfqpoint{1.347017in}{3.761696in}}%
\pgfpathcurveto{\pgfqpoint{1.356996in}{3.771674in}}{\pgfqpoint{1.362602in}{3.785210in}}{\pgfqpoint{1.362602in}{3.799321in}}%
\pgfpathcurveto{\pgfqpoint{1.362602in}{3.813433in}}{\pgfqpoint{1.356996in}{3.826969in}}{\pgfqpoint{1.347017in}{3.836947in}}%
\pgfpathcurveto{\pgfqpoint{1.337039in}{3.846925in}}{\pgfqpoint{1.323503in}{3.852532in}}{\pgfqpoint{1.309392in}{3.852532in}}%
\pgfpathcurveto{\pgfqpoint{1.295280in}{3.852532in}}{\pgfqpoint{1.281745in}{3.846925in}}{\pgfqpoint{1.271766in}{3.836947in}}%
\pgfpathcurveto{\pgfqpoint{1.261788in}{3.826969in}}{\pgfqpoint{1.256181in}{3.813433in}}{\pgfqpoint{1.256181in}{3.799321in}}%
\pgfpathcurveto{\pgfqpoint{1.256181in}{3.785210in}}{\pgfqpoint{1.261788in}{3.771674in}}{\pgfqpoint{1.271766in}{3.761696in}}%
\pgfpathcurveto{\pgfqpoint{1.281745in}{3.751718in}}{\pgfqpoint{1.295280in}{3.746111in}}{\pgfqpoint{1.309392in}{3.746111in}}%
\pgfpathclose%
\pgfusepath{stroke,fill}%
\end{pgfscope}%
\begin{pgfscope}%
\pgfpathrectangle{\pgfqpoint{1.065196in}{0.528000in}}{\pgfqpoint{3.702804in}{3.696000in}} %
\pgfusepath{clip}%
\pgfsetbuttcap%
\pgfsetroundjoin%
\definecolor{currentfill}{rgb}{0.121569,0.466667,0.705882}%
\pgfsetfillcolor{currentfill}%
\pgfsetlinewidth{1.003750pt}%
\definecolor{currentstroke}{rgb}{0.121569,0.466667,0.705882}%
\pgfsetstrokecolor{currentstroke}%
\pgfsetdash{}{0pt}%
\pgfpathmoveto{\pgfqpoint{3.569404in}{3.981490in}}%
\pgfpathcurveto{\pgfqpoint{3.581364in}{3.981490in}}{\pgfqpoint{3.592836in}{3.986242in}}{\pgfqpoint{3.601293in}{3.994699in}}%
\pgfpathcurveto{\pgfqpoint{3.609750in}{4.003156in}}{\pgfqpoint{3.614502in}{4.014628in}}{\pgfqpoint{3.614502in}{4.026588in}}%
\pgfpathcurveto{\pgfqpoint{3.614502in}{4.038549in}}{\pgfqpoint{3.609750in}{4.050021in}}{\pgfqpoint{3.601293in}{4.058478in}}%
\pgfpathcurveto{\pgfqpoint{3.592836in}{4.066935in}}{\pgfqpoint{3.581364in}{4.071687in}}{\pgfqpoint{3.569404in}{4.071687in}}%
\pgfpathcurveto{\pgfqpoint{3.557443in}{4.071687in}}{\pgfqpoint{3.545971in}{4.066935in}}{\pgfqpoint{3.537514in}{4.058478in}}%
\pgfpathcurveto{\pgfqpoint{3.529057in}{4.050021in}}{\pgfqpoint{3.524305in}{4.038549in}}{\pgfqpoint{3.524305in}{4.026588in}}%
\pgfpathcurveto{\pgfqpoint{3.524305in}{4.014628in}}{\pgfqpoint{3.529057in}{4.003156in}}{\pgfqpoint{3.537514in}{3.994699in}}%
\pgfpathcurveto{\pgfqpoint{3.545971in}{3.986242in}}{\pgfqpoint{3.557443in}{3.981490in}}{\pgfqpoint{3.569404in}{3.981490in}}%
\pgfpathclose%
\pgfusepath{stroke,fill}%
\end{pgfscope}%
\begin{pgfscope}%
\pgfpathrectangle{\pgfqpoint{1.065196in}{0.528000in}}{\pgfqpoint{3.702804in}{3.696000in}} %
\pgfusepath{clip}%
\pgfsetbuttcap%
\pgfsetroundjoin%
\definecolor{currentfill}{rgb}{0.121569,0.466667,0.705882}%
\pgfsetfillcolor{currentfill}%
\pgfsetlinewidth{1.003750pt}%
\definecolor{currentstroke}{rgb}{0.121569,0.466667,0.705882}%
\pgfsetstrokecolor{currentstroke}%
\pgfsetdash{}{0pt}%
\pgfpathmoveto{\pgfqpoint{1.833385in}{1.101113in}}%
\pgfpathcurveto{\pgfqpoint{1.845518in}{1.101113in}}{\pgfqpoint{1.857156in}{1.105933in}}{\pgfqpoint{1.865735in}{1.114512in}}%
\pgfpathcurveto{\pgfqpoint{1.874314in}{1.123092in}}{\pgfqpoint{1.879135in}{1.134729in}}{\pgfqpoint{1.879135in}{1.146862in}}%
\pgfpathcurveto{\pgfqpoint{1.879135in}{1.158995in}}{\pgfqpoint{1.874314in}{1.170633in}}{\pgfqpoint{1.865735in}{1.179212in}}%
\pgfpathcurveto{\pgfqpoint{1.857156in}{1.187791in}}{\pgfqpoint{1.845518in}{1.192612in}}{\pgfqpoint{1.833385in}{1.192612in}}%
\pgfpathcurveto{\pgfqpoint{1.821252in}{1.192612in}}{\pgfqpoint{1.809615in}{1.187791in}}{\pgfqpoint{1.801036in}{1.179212in}}%
\pgfpathcurveto{\pgfqpoint{1.792456in}{1.170633in}}{\pgfqpoint{1.787636in}{1.158995in}}{\pgfqpoint{1.787636in}{1.146862in}}%
\pgfpathcurveto{\pgfqpoint{1.787636in}{1.134729in}}{\pgfqpoint{1.792456in}{1.123092in}}{\pgfqpoint{1.801036in}{1.114512in}}%
\pgfpathcurveto{\pgfqpoint{1.809615in}{1.105933in}}{\pgfqpoint{1.821252in}{1.101113in}}{\pgfqpoint{1.833385in}{1.101113in}}%
\pgfpathclose%
\pgfusepath{stroke,fill}%
\end{pgfscope}%
\begin{pgfscope}%
\pgfpathrectangle{\pgfqpoint{1.065196in}{0.528000in}}{\pgfqpoint{3.702804in}{3.696000in}} %
\pgfusepath{clip}%
\pgfsetbuttcap%
\pgfsetroundjoin%
\definecolor{currentfill}{rgb}{0.121569,0.466667,0.705882}%
\pgfsetfillcolor{currentfill}%
\pgfsetlinewidth{1.003750pt}%
\definecolor{currentstroke}{rgb}{0.121569,0.466667,0.705882}%
\pgfsetstrokecolor{currentstroke}%
\pgfsetdash{}{0pt}%
\pgfpathmoveto{\pgfqpoint{4.378560in}{2.969565in}}%
\pgfpathcurveto{\pgfqpoint{4.392084in}{2.969565in}}{\pgfqpoint{4.405056in}{2.974938in}}{\pgfqpoint{4.414620in}{2.984501in}}%
\pgfpathcurveto{\pgfqpoint{4.424183in}{2.994065in}}{\pgfqpoint{4.429556in}{3.007037in}}{\pgfqpoint{4.429556in}{3.020562in}}%
\pgfpathcurveto{\pgfqpoint{4.429556in}{3.034086in}}{\pgfqpoint{4.424183in}{3.047058in}}{\pgfqpoint{4.414620in}{3.056622in}}%
\pgfpathcurveto{\pgfqpoint{4.405056in}{3.066185in}}{\pgfqpoint{4.392084in}{3.071558in}}{\pgfqpoint{4.378560in}{3.071558in}}%
\pgfpathcurveto{\pgfqpoint{4.365035in}{3.071558in}}{\pgfqpoint{4.352063in}{3.066185in}}{\pgfqpoint{4.342500in}{3.056622in}}%
\pgfpathcurveto{\pgfqpoint{4.332936in}{3.047058in}}{\pgfqpoint{4.327563in}{3.034086in}}{\pgfqpoint{4.327563in}{3.020562in}}%
\pgfpathcurveto{\pgfqpoint{4.327563in}{3.007037in}}{\pgfqpoint{4.332936in}{2.994065in}}{\pgfqpoint{4.342500in}{2.984501in}}%
\pgfpathcurveto{\pgfqpoint{4.352063in}{2.974938in}}{\pgfqpoint{4.365035in}{2.969565in}}{\pgfqpoint{4.378560in}{2.969565in}}%
\pgfpathclose%
\pgfusepath{stroke,fill}%
\end{pgfscope}%
\begin{pgfscope}%
\pgfpathrectangle{\pgfqpoint{1.065196in}{0.528000in}}{\pgfqpoint{3.702804in}{3.696000in}} %
\pgfusepath{clip}%
\pgfsetbuttcap%
\pgfsetroundjoin%
\definecolor{currentfill}{rgb}{0.121569,0.466667,0.705882}%
\pgfsetfillcolor{currentfill}%
\pgfsetlinewidth{1.003750pt}%
\definecolor{currentstroke}{rgb}{0.121569,0.466667,0.705882}%
\pgfsetstrokecolor{currentstroke}%
\pgfsetdash{}{0pt}%
\pgfpathmoveto{\pgfqpoint{1.477380in}{3.183032in}}%
\pgfpathcurveto{\pgfqpoint{1.486360in}{3.183032in}}{\pgfqpoint{1.494973in}{3.186600in}}{\pgfqpoint{1.501323in}{3.192949in}}%
\pgfpathcurveto{\pgfqpoint{1.507673in}{3.199299in}}{\pgfqpoint{1.511241in}{3.207912in}}{\pgfqpoint{1.511241in}{3.216892in}}%
\pgfpathcurveto{\pgfqpoint{1.511241in}{3.225872in}}{\pgfqpoint{1.507673in}{3.234486in}}{\pgfqpoint{1.501323in}{3.240835in}}%
\pgfpathcurveto{\pgfqpoint{1.494973in}{3.247185in}}{\pgfqpoint{1.486360in}{3.250753in}}{\pgfqpoint{1.477380in}{3.250753in}}%
\pgfpathcurveto{\pgfqpoint{1.468400in}{3.250753in}}{\pgfqpoint{1.459787in}{3.247185in}}{\pgfqpoint{1.453437in}{3.240835in}}%
\pgfpathcurveto{\pgfqpoint{1.447087in}{3.234486in}}{\pgfqpoint{1.443520in}{3.225872in}}{\pgfqpoint{1.443520in}{3.216892in}}%
\pgfpathcurveto{\pgfqpoint{1.443520in}{3.207912in}}{\pgfqpoint{1.447087in}{3.199299in}}{\pgfqpoint{1.453437in}{3.192949in}}%
\pgfpathcurveto{\pgfqpoint{1.459787in}{3.186600in}}{\pgfqpoint{1.468400in}{3.183032in}}{\pgfqpoint{1.477380in}{3.183032in}}%
\pgfpathclose%
\pgfusepath{stroke,fill}%
\end{pgfscope}%
\begin{pgfscope}%
\pgfpathrectangle{\pgfqpoint{1.065196in}{0.528000in}}{\pgfqpoint{3.702804in}{3.696000in}} %
\pgfusepath{clip}%
\pgfsetbuttcap%
\pgfsetroundjoin%
\definecolor{currentfill}{rgb}{0.121569,0.466667,0.705882}%
\pgfsetfillcolor{currentfill}%
\pgfsetlinewidth{1.003750pt}%
\definecolor{currentstroke}{rgb}{0.121569,0.466667,0.705882}%
\pgfsetstrokecolor{currentstroke}%
\pgfsetdash{}{0pt}%
\pgfpathmoveto{\pgfqpoint{3.780245in}{3.751121in}}%
\pgfpathcurveto{\pgfqpoint{3.787334in}{3.751121in}}{\pgfqpoint{3.794134in}{3.753938in}}{\pgfqpoint{3.799147in}{3.758951in}}%
\pgfpathcurveto{\pgfqpoint{3.804160in}{3.763964in}}{\pgfqpoint{3.806977in}{3.770764in}}{\pgfqpoint{3.806977in}{3.777853in}}%
\pgfpathcurveto{\pgfqpoint{3.806977in}{3.784942in}}{\pgfqpoint{3.804160in}{3.791742in}}{\pgfqpoint{3.799147in}{3.796755in}}%
\pgfpathcurveto{\pgfqpoint{3.794134in}{3.801768in}}{\pgfqpoint{3.787334in}{3.804585in}}{\pgfqpoint{3.780245in}{3.804585in}}%
\pgfpathcurveto{\pgfqpoint{3.773156in}{3.804585in}}{\pgfqpoint{3.766356in}{3.801768in}}{\pgfqpoint{3.761343in}{3.796755in}}%
\pgfpathcurveto{\pgfqpoint{3.756330in}{3.791742in}}{\pgfqpoint{3.753513in}{3.784942in}}{\pgfqpoint{3.753513in}{3.777853in}}%
\pgfpathcurveto{\pgfqpoint{3.753513in}{3.770764in}}{\pgfqpoint{3.756330in}{3.763964in}}{\pgfqpoint{3.761343in}{3.758951in}}%
\pgfpathcurveto{\pgfqpoint{3.766356in}{3.753938in}}{\pgfqpoint{3.773156in}{3.751121in}}{\pgfqpoint{3.780245in}{3.751121in}}%
\pgfpathclose%
\pgfusepath{stroke,fill}%
\end{pgfscope}%
\begin{pgfscope}%
\pgfpathrectangle{\pgfqpoint{1.065196in}{0.528000in}}{\pgfqpoint{3.702804in}{3.696000in}} %
\pgfusepath{clip}%
\pgfsetbuttcap%
\pgfsetroundjoin%
\definecolor{currentfill}{rgb}{0.121569,0.466667,0.705882}%
\pgfsetfillcolor{currentfill}%
\pgfsetlinewidth{1.003750pt}%
\definecolor{currentstroke}{rgb}{0.121569,0.466667,0.705882}%
\pgfsetstrokecolor{currentstroke}%
\pgfsetdash{}{0pt}%
\pgfpathmoveto{\pgfqpoint{3.638461in}{1.081522in}}%
\pgfpathcurveto{\pgfqpoint{3.644368in}{1.081522in}}{\pgfqpoint{3.650033in}{1.083869in}}{\pgfqpoint{3.654210in}{1.088045in}}%
\pgfpathcurveto{\pgfqpoint{3.658386in}{1.092222in}}{\pgfqpoint{3.660733in}{1.097887in}}{\pgfqpoint{3.660733in}{1.103794in}}%
\pgfpathcurveto{\pgfqpoint{3.660733in}{1.109700in}}{\pgfqpoint{3.658386in}{1.115365in}}{\pgfqpoint{3.654210in}{1.119542in}}%
\pgfpathcurveto{\pgfqpoint{3.650033in}{1.123718in}}{\pgfqpoint{3.644368in}{1.126065in}}{\pgfqpoint{3.638461in}{1.126065in}}%
\pgfpathcurveto{\pgfqpoint{3.632555in}{1.126065in}}{\pgfqpoint{3.626889in}{1.123718in}}{\pgfqpoint{3.622713in}{1.119542in}}%
\pgfpathcurveto{\pgfqpoint{3.618536in}{1.115365in}}{\pgfqpoint{3.616190in}{1.109700in}}{\pgfqpoint{3.616190in}{1.103794in}}%
\pgfpathcurveto{\pgfqpoint{3.616190in}{1.097887in}}{\pgfqpoint{3.618536in}{1.092222in}}{\pgfqpoint{3.622713in}{1.088045in}}%
\pgfpathcurveto{\pgfqpoint{3.626889in}{1.083869in}}{\pgfqpoint{3.632555in}{1.081522in}}{\pgfqpoint{3.638461in}{1.081522in}}%
\pgfpathclose%
\pgfusepath{stroke,fill}%
\end{pgfscope}%
\begin{pgfscope}%
\pgfpathrectangle{\pgfqpoint{1.065196in}{0.528000in}}{\pgfqpoint{3.702804in}{3.696000in}} %
\pgfusepath{clip}%
\pgfsetbuttcap%
\pgfsetroundjoin%
\definecolor{currentfill}{rgb}{0.121569,0.466667,0.705882}%
\pgfsetfillcolor{currentfill}%
\pgfsetlinewidth{1.003750pt}%
\definecolor{currentstroke}{rgb}{0.121569,0.466667,0.705882}%
\pgfsetstrokecolor{currentstroke}%
\pgfsetdash{}{0pt}%
\pgfpathmoveto{\pgfqpoint{1.322980in}{0.757025in}}%
\pgfpathcurveto{\pgfqpoint{1.327882in}{0.757025in}}{\pgfqpoint{1.332584in}{0.758973in}}{\pgfqpoint{1.336050in}{0.762439in}}%
\pgfpathcurveto{\pgfqpoint{1.339516in}{0.765905in}}{\pgfqpoint{1.341464in}{0.770607in}}{\pgfqpoint{1.341464in}{0.775509in}}%
\pgfpathcurveto{\pgfqpoint{1.341464in}{0.780411in}}{\pgfqpoint{1.339516in}{0.785113in}}{\pgfqpoint{1.336050in}{0.788579in}}%
\pgfpathcurveto{\pgfqpoint{1.332584in}{0.792046in}}{\pgfqpoint{1.327882in}{0.793993in}}{\pgfqpoint{1.322980in}{0.793993in}}%
\pgfpathcurveto{\pgfqpoint{1.318078in}{0.793993in}}{\pgfqpoint{1.313376in}{0.792046in}}{\pgfqpoint{1.309910in}{0.788579in}}%
\pgfpathcurveto{\pgfqpoint{1.306444in}{0.785113in}}{\pgfqpoint{1.304496in}{0.780411in}}{\pgfqpoint{1.304496in}{0.775509in}}%
\pgfpathcurveto{\pgfqpoint{1.304496in}{0.770607in}}{\pgfqpoint{1.306444in}{0.765905in}}{\pgfqpoint{1.309910in}{0.762439in}}%
\pgfpathcurveto{\pgfqpoint{1.313376in}{0.758973in}}{\pgfqpoint{1.318078in}{0.757025in}}{\pgfqpoint{1.322980in}{0.757025in}}%
\pgfpathclose%
\pgfusepath{stroke,fill}%
\end{pgfscope}%
\begin{pgfscope}%
\pgfpathrectangle{\pgfqpoint{1.065196in}{0.528000in}}{\pgfqpoint{3.702804in}{3.696000in}} %
\pgfusepath{clip}%
\pgfsetbuttcap%
\pgfsetroundjoin%
\definecolor{currentfill}{rgb}{0.121569,0.466667,0.705882}%
\pgfsetfillcolor{currentfill}%
\pgfsetlinewidth{1.003750pt}%
\definecolor{currentstroke}{rgb}{0.121569,0.466667,0.705882}%
\pgfsetstrokecolor{currentstroke}%
\pgfsetdash{}{0pt}%
\pgfpathmoveto{\pgfqpoint{1.351190in}{1.463814in}}%
\pgfpathcurveto{\pgfqpoint{1.363975in}{1.463814in}}{\pgfqpoint{1.376238in}{1.468894in}}{\pgfqpoint{1.385279in}{1.477934in}}%
\pgfpathcurveto{\pgfqpoint{1.394319in}{1.486975in}}{\pgfqpoint{1.399399in}{1.499238in}}{\pgfqpoint{1.399399in}{1.512023in}}%
\pgfpathcurveto{\pgfqpoint{1.399399in}{1.524809in}}{\pgfqpoint{1.394319in}{1.537072in}}{\pgfqpoint{1.385279in}{1.546113in}}%
\pgfpathcurveto{\pgfqpoint{1.376238in}{1.555153in}}{\pgfqpoint{1.363975in}{1.560233in}}{\pgfqpoint{1.351190in}{1.560233in}}%
\pgfpathcurveto{\pgfqpoint{1.338404in}{1.560233in}}{\pgfqpoint{1.326141in}{1.555153in}}{\pgfqpoint{1.317101in}{1.546113in}}%
\pgfpathcurveto{\pgfqpoint{1.308060in}{1.537072in}}{\pgfqpoint{1.302980in}{1.524809in}}{\pgfqpoint{1.302980in}{1.512023in}}%
\pgfpathcurveto{\pgfqpoint{1.302980in}{1.499238in}}{\pgfqpoint{1.308060in}{1.486975in}}{\pgfqpoint{1.317101in}{1.477934in}}%
\pgfpathcurveto{\pgfqpoint{1.326141in}{1.468894in}}{\pgfqpoint{1.338404in}{1.463814in}}{\pgfqpoint{1.351190in}{1.463814in}}%
\pgfpathclose%
\pgfusepath{stroke,fill}%
\end{pgfscope}%
\begin{pgfscope}%
\pgfpathrectangle{\pgfqpoint{1.065196in}{0.528000in}}{\pgfqpoint{3.702804in}{3.696000in}} %
\pgfusepath{clip}%
\pgfsetbuttcap%
\pgfsetroundjoin%
\definecolor{currentfill}{rgb}{0.121569,0.466667,0.705882}%
\pgfsetfillcolor{currentfill}%
\pgfsetlinewidth{1.003750pt}%
\definecolor{currentstroke}{rgb}{0.121569,0.466667,0.705882}%
\pgfsetstrokecolor{currentstroke}%
\pgfsetdash{}{0pt}%
\pgfpathmoveto{\pgfqpoint{4.135619in}{2.471164in}}%
\pgfpathcurveto{\pgfqpoint{4.141053in}{2.471164in}}{\pgfqpoint{4.146265in}{2.473323in}}{\pgfqpoint{4.150107in}{2.477165in}}%
\pgfpathcurveto{\pgfqpoint{4.153950in}{2.481008in}}{\pgfqpoint{4.156109in}{2.486220in}}{\pgfqpoint{4.156109in}{2.491654in}}%
\pgfpathcurveto{\pgfqpoint{4.156109in}{2.497088in}}{\pgfqpoint{4.153950in}{2.502300in}}{\pgfqpoint{4.150107in}{2.506142in}}%
\pgfpathcurveto{\pgfqpoint{4.146265in}{2.509985in}}{\pgfqpoint{4.141053in}{2.512144in}}{\pgfqpoint{4.135619in}{2.512144in}}%
\pgfpathcurveto{\pgfqpoint{4.130185in}{2.512144in}}{\pgfqpoint{4.124973in}{2.509985in}}{\pgfqpoint{4.121130in}{2.506142in}}%
\pgfpathcurveto{\pgfqpoint{4.117288in}{2.502300in}}{\pgfqpoint{4.115129in}{2.497088in}}{\pgfqpoint{4.115129in}{2.491654in}}%
\pgfpathcurveto{\pgfqpoint{4.115129in}{2.486220in}}{\pgfqpoint{4.117288in}{2.481008in}}{\pgfqpoint{4.121130in}{2.477165in}}%
\pgfpathcurveto{\pgfqpoint{4.124973in}{2.473323in}}{\pgfqpoint{4.130185in}{2.471164in}}{\pgfqpoint{4.135619in}{2.471164in}}%
\pgfpathclose%
\pgfusepath{stroke,fill}%
\end{pgfscope}%
\begin{pgfscope}%
\pgfpathrectangle{\pgfqpoint{1.065196in}{0.528000in}}{\pgfqpoint{3.702804in}{3.696000in}} %
\pgfusepath{clip}%
\pgfsetbuttcap%
\pgfsetroundjoin%
\definecolor{currentfill}{rgb}{0.121569,0.466667,0.705882}%
\pgfsetfillcolor{currentfill}%
\pgfsetlinewidth{1.003750pt}%
\definecolor{currentstroke}{rgb}{0.121569,0.466667,0.705882}%
\pgfsetstrokecolor{currentstroke}%
\pgfsetdash{}{0pt}%
\pgfpathmoveto{\pgfqpoint{3.107158in}{3.479209in}}%
\pgfpathcurveto{\pgfqpoint{3.114449in}{3.479209in}}{\pgfqpoint{3.121443in}{3.482106in}}{\pgfqpoint{3.126599in}{3.487262in}}%
\pgfpathcurveto{\pgfqpoint{3.131755in}{3.492418in}}{\pgfqpoint{3.134651in}{3.499412in}}{\pgfqpoint{3.134651in}{3.506703in}}%
\pgfpathcurveto{\pgfqpoint{3.134651in}{3.513995in}}{\pgfqpoint{3.131755in}{3.520989in}}{\pgfqpoint{3.126599in}{3.526144in}}%
\pgfpathcurveto{\pgfqpoint{3.121443in}{3.531300in}}{\pgfqpoint{3.114449in}{3.534197in}}{\pgfqpoint{3.107158in}{3.534197in}}%
\pgfpathcurveto{\pgfqpoint{3.099866in}{3.534197in}}{\pgfqpoint{3.092872in}{3.531300in}}{\pgfqpoint{3.087716in}{3.526144in}}%
\pgfpathcurveto{\pgfqpoint{3.082561in}{3.520989in}}{\pgfqpoint{3.079664in}{3.513995in}}{\pgfqpoint{3.079664in}{3.506703in}}%
\pgfpathcurveto{\pgfqpoint{3.079664in}{3.499412in}}{\pgfqpoint{3.082561in}{3.492418in}}{\pgfqpoint{3.087716in}{3.487262in}}%
\pgfpathcurveto{\pgfqpoint{3.092872in}{3.482106in}}{\pgfqpoint{3.099866in}{3.479209in}}{\pgfqpoint{3.107158in}{3.479209in}}%
\pgfpathclose%
\pgfusepath{stroke,fill}%
\end{pgfscope}%
\begin{pgfscope}%
\pgfpathrectangle{\pgfqpoint{1.065196in}{0.528000in}}{\pgfqpoint{3.702804in}{3.696000in}} %
\pgfusepath{clip}%
\pgfsetbuttcap%
\pgfsetroundjoin%
\definecolor{currentfill}{rgb}{0.121569,0.466667,0.705882}%
\pgfsetfillcolor{currentfill}%
\pgfsetlinewidth{1.003750pt}%
\definecolor{currentstroke}{rgb}{0.121569,0.466667,0.705882}%
\pgfsetstrokecolor{currentstroke}%
\pgfsetdash{}{0pt}%
\pgfpathmoveto{\pgfqpoint{1.672132in}{1.574008in}}%
\pgfpathcurveto{\pgfqpoint{1.684968in}{1.574008in}}{\pgfqpoint{1.697280in}{1.579108in}}{\pgfqpoint{1.706356in}{1.588184in}}%
\pgfpathcurveto{\pgfqpoint{1.715433in}{1.597261in}}{\pgfqpoint{1.720533in}{1.609573in}}{\pgfqpoint{1.720533in}{1.622409in}}%
\pgfpathcurveto{\pgfqpoint{1.720533in}{1.635245in}}{\pgfqpoint{1.715433in}{1.647557in}}{\pgfqpoint{1.706356in}{1.656634in}}%
\pgfpathcurveto{\pgfqpoint{1.697280in}{1.665710in}}{\pgfqpoint{1.684968in}{1.670810in}}{\pgfqpoint{1.672132in}{1.670810in}}%
\pgfpathcurveto{\pgfqpoint{1.659296in}{1.670810in}}{\pgfqpoint{1.646983in}{1.665710in}}{\pgfqpoint{1.637907in}{1.656634in}}%
\pgfpathcurveto{\pgfqpoint{1.628830in}{1.647557in}}{\pgfqpoint{1.623731in}{1.635245in}}{\pgfqpoint{1.623731in}{1.622409in}}%
\pgfpathcurveto{\pgfqpoint{1.623731in}{1.609573in}}{\pgfqpoint{1.628830in}{1.597261in}}{\pgfqpoint{1.637907in}{1.588184in}}%
\pgfpathcurveto{\pgfqpoint{1.646983in}{1.579108in}}{\pgfqpoint{1.659296in}{1.574008in}}{\pgfqpoint{1.672132in}{1.574008in}}%
\pgfpathclose%
\pgfusepath{stroke,fill}%
\end{pgfscope}%
\begin{pgfscope}%
\pgfpathrectangle{\pgfqpoint{1.065196in}{0.528000in}}{\pgfqpoint{3.702804in}{3.696000in}} %
\pgfusepath{clip}%
\pgfsetbuttcap%
\pgfsetroundjoin%
\definecolor{currentfill}{rgb}{0.121569,0.466667,0.705882}%
\pgfsetfillcolor{currentfill}%
\pgfsetlinewidth{1.003750pt}%
\definecolor{currentstroke}{rgb}{0.121569,0.466667,0.705882}%
\pgfsetstrokecolor{currentstroke}%
\pgfsetdash{}{0pt}%
\pgfpathmoveto{\pgfqpoint{3.217425in}{3.903959in}}%
\pgfpathcurveto{\pgfqpoint{3.225329in}{3.903959in}}{\pgfqpoint{3.232911in}{3.907100in}}{\pgfqpoint{3.238500in}{3.912689in}}%
\pgfpathcurveto{\pgfqpoint{3.244089in}{3.918278in}}{\pgfqpoint{3.247229in}{3.925860in}}{\pgfqpoint{3.247229in}{3.933764in}}%
\pgfpathcurveto{\pgfqpoint{3.247229in}{3.941668in}}{\pgfqpoint{3.244089in}{3.949250in}}{\pgfqpoint{3.238500in}{3.954839in}}%
\pgfpathcurveto{\pgfqpoint{3.232911in}{3.960428in}}{\pgfqpoint{3.225329in}{3.963568in}}{\pgfqpoint{3.217425in}{3.963568in}}%
\pgfpathcurveto{\pgfqpoint{3.209520in}{3.963568in}}{\pgfqpoint{3.201939in}{3.960428in}}{\pgfqpoint{3.196350in}{3.954839in}}%
\pgfpathcurveto{\pgfqpoint{3.190761in}{3.949250in}}{\pgfqpoint{3.187620in}{3.941668in}}{\pgfqpoint{3.187620in}{3.933764in}}%
\pgfpathcurveto{\pgfqpoint{3.187620in}{3.925860in}}{\pgfqpoint{3.190761in}{3.918278in}}{\pgfqpoint{3.196350in}{3.912689in}}%
\pgfpathcurveto{\pgfqpoint{3.201939in}{3.907100in}}{\pgfqpoint{3.209520in}{3.903959in}}{\pgfqpoint{3.217425in}{3.903959in}}%
\pgfpathclose%
\pgfusepath{stroke,fill}%
\end{pgfscope}%
\begin{pgfscope}%
\pgfpathrectangle{\pgfqpoint{1.065196in}{0.528000in}}{\pgfqpoint{3.702804in}{3.696000in}} %
\pgfusepath{clip}%
\pgfsetbuttcap%
\pgfsetroundjoin%
\definecolor{currentfill}{rgb}{0.121569,0.466667,0.705882}%
\pgfsetfillcolor{currentfill}%
\pgfsetlinewidth{1.003750pt}%
\definecolor{currentstroke}{rgb}{0.121569,0.466667,0.705882}%
\pgfsetstrokecolor{currentstroke}%
\pgfsetdash{}{0pt}%
\pgfpathmoveto{\pgfqpoint{3.134637in}{0.697180in}}%
\pgfpathcurveto{\pgfqpoint{3.148695in}{0.697180in}}{\pgfqpoint{3.162179in}{0.702765in}}{\pgfqpoint{3.172119in}{0.712706in}}%
\pgfpathcurveto{\pgfqpoint{3.182059in}{0.722646in}}{\pgfqpoint{3.187645in}{0.736130in}}{\pgfqpoint{3.187645in}{0.750188in}}%
\pgfpathcurveto{\pgfqpoint{3.187645in}{0.764245in}}{\pgfqpoint{3.182059in}{0.777729in}}{\pgfqpoint{3.172119in}{0.787670in}}%
\pgfpathcurveto{\pgfqpoint{3.162179in}{0.797610in}}{\pgfqpoint{3.148695in}{0.803195in}}{\pgfqpoint{3.134637in}{0.803195in}}%
\pgfpathcurveto{\pgfqpoint{3.120579in}{0.803195in}}{\pgfqpoint{3.107095in}{0.797610in}}{\pgfqpoint{3.097155in}{0.787670in}}%
\pgfpathcurveto{\pgfqpoint{3.087215in}{0.777729in}}{\pgfqpoint{3.081630in}{0.764245in}}{\pgfqpoint{3.081630in}{0.750188in}}%
\pgfpathcurveto{\pgfqpoint{3.081630in}{0.736130in}}{\pgfqpoint{3.087215in}{0.722646in}}{\pgfqpoint{3.097155in}{0.712706in}}%
\pgfpathcurveto{\pgfqpoint{3.107095in}{0.702765in}}{\pgfqpoint{3.120579in}{0.697180in}}{\pgfqpoint{3.134637in}{0.697180in}}%
\pgfpathclose%
\pgfusepath{stroke,fill}%
\end{pgfscope}%
\begin{pgfscope}%
\pgfpathrectangle{\pgfqpoint{1.065196in}{0.528000in}}{\pgfqpoint{3.702804in}{3.696000in}} %
\pgfusepath{clip}%
\pgfsetbuttcap%
\pgfsetroundjoin%
\definecolor{currentfill}{rgb}{0.121569,0.466667,0.705882}%
\pgfsetfillcolor{currentfill}%
\pgfsetlinewidth{1.003750pt}%
\definecolor{currentstroke}{rgb}{0.121569,0.466667,0.705882}%
\pgfsetstrokecolor{currentstroke}%
\pgfsetdash{}{0pt}%
\pgfpathmoveto{\pgfqpoint{3.936776in}{1.440094in}}%
\pgfpathcurveto{\pgfqpoint{3.944100in}{1.440094in}}{\pgfqpoint{3.951124in}{1.443004in}}{\pgfqpoint{3.956303in}{1.448182in}}%
\pgfpathcurveto{\pgfqpoint{3.961481in}{1.453361in}}{\pgfqpoint{3.964391in}{1.460386in}}{\pgfqpoint{3.964391in}{1.467709in}}%
\pgfpathcurveto{\pgfqpoint{3.964391in}{1.475033in}}{\pgfqpoint{3.961481in}{1.482058in}}{\pgfqpoint{3.956303in}{1.487236in}}%
\pgfpathcurveto{\pgfqpoint{3.951124in}{1.492415in}}{\pgfqpoint{3.944100in}{1.495325in}}{\pgfqpoint{3.936776in}{1.495325in}}%
\pgfpathcurveto{\pgfqpoint{3.929452in}{1.495325in}}{\pgfqpoint{3.922427in}{1.492415in}}{\pgfqpoint{3.917249in}{1.487236in}}%
\pgfpathcurveto{\pgfqpoint{3.912070in}{1.482058in}}{\pgfqpoint{3.909160in}{1.475033in}}{\pgfqpoint{3.909160in}{1.467709in}}%
\pgfpathcurveto{\pgfqpoint{3.909160in}{1.460386in}}{\pgfqpoint{3.912070in}{1.453361in}}{\pgfqpoint{3.917249in}{1.448182in}}%
\pgfpathcurveto{\pgfqpoint{3.922427in}{1.443004in}}{\pgfqpoint{3.929452in}{1.440094in}}{\pgfqpoint{3.936776in}{1.440094in}}%
\pgfpathclose%
\pgfusepath{stroke,fill}%
\end{pgfscope}%
\begin{pgfscope}%
\pgfpathrectangle{\pgfqpoint{1.065196in}{0.528000in}}{\pgfqpoint{3.702804in}{3.696000in}} %
\pgfusepath{clip}%
\pgfsetbuttcap%
\pgfsetroundjoin%
\definecolor{currentfill}{rgb}{0.121569,0.466667,0.705882}%
\pgfsetfillcolor{currentfill}%
\pgfsetlinewidth{1.003750pt}%
\definecolor{currentstroke}{rgb}{0.121569,0.466667,0.705882}%
\pgfsetstrokecolor{currentstroke}%
\pgfsetdash{}{0pt}%
\pgfpathmoveto{\pgfqpoint{3.958444in}{1.947169in}}%
\pgfpathcurveto{\pgfqpoint{3.968805in}{1.947169in}}{\pgfqpoint{3.978743in}{1.951285in}}{\pgfqpoint{3.986070in}{1.958611in}}%
\pgfpathcurveto{\pgfqpoint{3.993396in}{1.965938in}}{\pgfqpoint{3.997512in}{1.975876in}}{\pgfqpoint{3.997512in}{1.986237in}}%
\pgfpathcurveto{\pgfqpoint{3.997512in}{1.996597in}}{\pgfqpoint{3.993396in}{2.006535in}}{\pgfqpoint{3.986070in}{2.013862in}}%
\pgfpathcurveto{\pgfqpoint{3.978743in}{2.021188in}}{\pgfqpoint{3.968805in}{2.025304in}}{\pgfqpoint{3.958444in}{2.025304in}}%
\pgfpathcurveto{\pgfqpoint{3.948084in}{2.025304in}}{\pgfqpoint{3.938146in}{2.021188in}}{\pgfqpoint{3.930819in}{2.013862in}}%
\pgfpathcurveto{\pgfqpoint{3.923493in}{2.006535in}}{\pgfqpoint{3.919377in}{1.996597in}}{\pgfqpoint{3.919377in}{1.986237in}}%
\pgfpathcurveto{\pgfqpoint{3.919377in}{1.975876in}}{\pgfqpoint{3.923493in}{1.965938in}}{\pgfqpoint{3.930819in}{1.958611in}}%
\pgfpathcurveto{\pgfqpoint{3.938146in}{1.951285in}}{\pgfqpoint{3.948084in}{1.947169in}}{\pgfqpoint{3.958444in}{1.947169in}}%
\pgfpathclose%
\pgfusepath{stroke,fill}%
\end{pgfscope}%
\begin{pgfscope}%
\pgfpathrectangle{\pgfqpoint{1.065196in}{0.528000in}}{\pgfqpoint{3.702804in}{3.696000in}} %
\pgfusepath{clip}%
\pgfsetbuttcap%
\pgfsetroundjoin%
\definecolor{currentfill}{rgb}{0.121569,0.466667,0.705882}%
\pgfsetfillcolor{currentfill}%
\pgfsetlinewidth{1.003750pt}%
\definecolor{currentstroke}{rgb}{0.121569,0.466667,0.705882}%
\pgfsetstrokecolor{currentstroke}%
\pgfsetdash{}{0pt}%
\pgfpathmoveto{\pgfqpoint{4.147383in}{3.158977in}}%
\pgfpathcurveto{\pgfqpoint{4.155336in}{3.158977in}}{\pgfqpoint{4.162965in}{3.162137in}}{\pgfqpoint{4.168589in}{3.167761in}}%
\pgfpathcurveto{\pgfqpoint{4.174213in}{3.173385in}}{\pgfqpoint{4.177373in}{3.181014in}}{\pgfqpoint{4.177373in}{3.188967in}}%
\pgfpathcurveto{\pgfqpoint{4.177373in}{3.196921in}}{\pgfqpoint{4.174213in}{3.204549in}}{\pgfqpoint{4.168589in}{3.210173in}}%
\pgfpathcurveto{\pgfqpoint{4.162965in}{3.215797in}}{\pgfqpoint{4.155336in}{3.218957in}}{\pgfqpoint{4.147383in}{3.218957in}}%
\pgfpathcurveto{\pgfqpoint{4.139429in}{3.218957in}}{\pgfqpoint{4.131800in}{3.215797in}}{\pgfqpoint{4.126176in}{3.210173in}}%
\pgfpathcurveto{\pgfqpoint{4.120552in}{3.204549in}}{\pgfqpoint{4.117393in}{3.196921in}}{\pgfqpoint{4.117393in}{3.188967in}}%
\pgfpathcurveto{\pgfqpoint{4.117393in}{3.181014in}}{\pgfqpoint{4.120552in}{3.173385in}}{\pgfqpoint{4.126176in}{3.167761in}}%
\pgfpathcurveto{\pgfqpoint{4.131800in}{3.162137in}}{\pgfqpoint{4.139429in}{3.158977in}}{\pgfqpoint{4.147383in}{3.158977in}}%
\pgfpathclose%
\pgfusepath{stroke,fill}%
\end{pgfscope}%
\begin{pgfscope}%
\pgfpathrectangle{\pgfqpoint{1.065196in}{0.528000in}}{\pgfqpoint{3.702804in}{3.696000in}} %
\pgfusepath{clip}%
\pgfsetbuttcap%
\pgfsetroundjoin%
\definecolor{currentfill}{rgb}{0.121569,0.466667,0.705882}%
\pgfsetfillcolor{currentfill}%
\pgfsetlinewidth{1.003750pt}%
\definecolor{currentstroke}{rgb}{0.121569,0.466667,0.705882}%
\pgfsetstrokecolor{currentstroke}%
\pgfsetdash{}{0pt}%
\pgfpathmoveto{\pgfqpoint{3.118601in}{1.093546in}}%
\pgfpathcurveto{\pgfqpoint{3.132137in}{1.093546in}}{\pgfqpoint{3.145120in}{1.098923in}}{\pgfqpoint{3.154691in}{1.108495in}}%
\pgfpathcurveto{\pgfqpoint{3.164262in}{1.118066in}}{\pgfqpoint{3.169639in}{1.131048in}}{\pgfqpoint{3.169639in}{1.144584in}}%
\pgfpathcurveto{\pgfqpoint{3.169639in}{1.158119in}}{\pgfqpoint{3.164262in}{1.171102in}}{\pgfqpoint{3.154691in}{1.180673in}}%
\pgfpathcurveto{\pgfqpoint{3.145120in}{1.190244in}}{\pgfqpoint{3.132137in}{1.195622in}}{\pgfqpoint{3.118601in}{1.195622in}}%
\pgfpathcurveto{\pgfqpoint{3.105066in}{1.195622in}}{\pgfqpoint{3.092083in}{1.190244in}}{\pgfqpoint{3.082512in}{1.180673in}}%
\pgfpathcurveto{\pgfqpoint{3.072941in}{1.171102in}}{\pgfqpoint{3.067563in}{1.158119in}}{\pgfqpoint{3.067563in}{1.144584in}}%
\pgfpathcurveto{\pgfqpoint{3.067563in}{1.131048in}}{\pgfqpoint{3.072941in}{1.118066in}}{\pgfqpoint{3.082512in}{1.108495in}}%
\pgfpathcurveto{\pgfqpoint{3.092083in}{1.098923in}}{\pgfqpoint{3.105066in}{1.093546in}}{\pgfqpoint{3.118601in}{1.093546in}}%
\pgfpathclose%
\pgfusepath{stroke,fill}%
\end{pgfscope}%
\begin{pgfscope}%
\pgfpathrectangle{\pgfqpoint{1.065196in}{0.528000in}}{\pgfqpoint{3.702804in}{3.696000in}} %
\pgfusepath{clip}%
\pgfsetbuttcap%
\pgfsetroundjoin%
\definecolor{currentfill}{rgb}{0.121569,0.466667,0.705882}%
\pgfsetfillcolor{currentfill}%
\pgfsetlinewidth{1.003750pt}%
\definecolor{currentstroke}{rgb}{0.121569,0.466667,0.705882}%
\pgfsetstrokecolor{currentstroke}%
\pgfsetdash{}{0pt}%
\pgfpathmoveto{\pgfqpoint{1.457017in}{1.061614in}}%
\pgfpathcurveto{\pgfqpoint{1.465604in}{1.061614in}}{\pgfqpoint{1.473841in}{1.065026in}}{\pgfqpoint{1.479913in}{1.071098in}}%
\pgfpathcurveto{\pgfqpoint{1.485984in}{1.077170in}}{\pgfqpoint{1.489396in}{1.085406in}}{\pgfqpoint{1.489396in}{1.093993in}}%
\pgfpathcurveto{\pgfqpoint{1.489396in}{1.102580in}}{\pgfqpoint{1.485984in}{1.110816in}}{\pgfqpoint{1.479913in}{1.116888in}}%
\pgfpathcurveto{\pgfqpoint{1.473841in}{1.122960in}}{\pgfqpoint{1.465604in}{1.126372in}}{\pgfqpoint{1.457017in}{1.126372in}}%
\pgfpathcurveto{\pgfqpoint{1.448430in}{1.126372in}}{\pgfqpoint{1.440194in}{1.122960in}}{\pgfqpoint{1.434122in}{1.116888in}}%
\pgfpathcurveto{\pgfqpoint{1.428050in}{1.110816in}}{\pgfqpoint{1.424639in}{1.102580in}}{\pgfqpoint{1.424639in}{1.093993in}}%
\pgfpathcurveto{\pgfqpoint{1.424639in}{1.085406in}}{\pgfqpoint{1.428050in}{1.077170in}}{\pgfqpoint{1.434122in}{1.071098in}}%
\pgfpathcurveto{\pgfqpoint{1.440194in}{1.065026in}}{\pgfqpoint{1.448430in}{1.061614in}}{\pgfqpoint{1.457017in}{1.061614in}}%
\pgfpathclose%
\pgfusepath{stroke,fill}%
\end{pgfscope}%
\begin{pgfscope}%
\pgfpathrectangle{\pgfqpoint{1.065196in}{0.528000in}}{\pgfqpoint{3.702804in}{3.696000in}} %
\pgfusepath{clip}%
\pgfsetbuttcap%
\pgfsetroundjoin%
\definecolor{currentfill}{rgb}{0.121569,0.466667,0.705882}%
\pgfsetfillcolor{currentfill}%
\pgfsetlinewidth{1.003750pt}%
\definecolor{currentstroke}{rgb}{0.121569,0.466667,0.705882}%
\pgfsetstrokecolor{currentstroke}%
\pgfsetdash{}{0pt}%
\pgfpathmoveto{\pgfqpoint{1.405576in}{1.012413in}}%
\pgfpathcurveto{\pgfqpoint{1.414915in}{1.012413in}}{\pgfqpoint{1.423873in}{1.016123in}}{\pgfqpoint{1.430477in}{1.022727in}}%
\pgfpathcurveto{\pgfqpoint{1.437081in}{1.029331in}}{\pgfqpoint{1.440791in}{1.038289in}}{\pgfqpoint{1.440791in}{1.047628in}}%
\pgfpathcurveto{\pgfqpoint{1.440791in}{1.056967in}}{\pgfqpoint{1.437081in}{1.065925in}}{\pgfqpoint{1.430477in}{1.072529in}}%
\pgfpathcurveto{\pgfqpoint{1.423873in}{1.079133in}}{\pgfqpoint{1.414915in}{1.082843in}}{\pgfqpoint{1.405576in}{1.082843in}}%
\pgfpathcurveto{\pgfqpoint{1.396237in}{1.082843in}}{\pgfqpoint{1.387279in}{1.079133in}}{\pgfqpoint{1.380675in}{1.072529in}}%
\pgfpathcurveto{\pgfqpoint{1.374071in}{1.065925in}}{\pgfqpoint{1.370361in}{1.056967in}}{\pgfqpoint{1.370361in}{1.047628in}}%
\pgfpathcurveto{\pgfqpoint{1.370361in}{1.038289in}}{\pgfqpoint{1.374071in}{1.029331in}}{\pgfqpoint{1.380675in}{1.022727in}}%
\pgfpathcurveto{\pgfqpoint{1.387279in}{1.016123in}}{\pgfqpoint{1.396237in}{1.012413in}}{\pgfqpoint{1.405576in}{1.012413in}}%
\pgfpathclose%
\pgfusepath{stroke,fill}%
\end{pgfscope}%
\begin{pgfscope}%
\pgfpathrectangle{\pgfqpoint{1.065196in}{0.528000in}}{\pgfqpoint{3.702804in}{3.696000in}} %
\pgfusepath{clip}%
\pgfsetbuttcap%
\pgfsetroundjoin%
\definecolor{currentfill}{rgb}{0.121569,0.466667,0.705882}%
\pgfsetfillcolor{currentfill}%
\pgfsetlinewidth{1.003750pt}%
\definecolor{currentstroke}{rgb}{0.121569,0.466667,0.705882}%
\pgfsetstrokecolor{currentstroke}%
\pgfsetdash{}{0pt}%
\pgfpathmoveto{\pgfqpoint{2.012053in}{3.036420in}}%
\pgfpathcurveto{\pgfqpoint{2.022204in}{3.036420in}}{\pgfqpoint{2.031941in}{3.040453in}}{\pgfqpoint{2.039120in}{3.047631in}}%
\pgfpathcurveto{\pgfqpoint{2.046298in}{3.054810in}}{\pgfqpoint{2.050331in}{3.064547in}}{\pgfqpoint{2.050331in}{3.074698in}}%
\pgfpathcurveto{\pgfqpoint{2.050331in}{3.084849in}}{\pgfqpoint{2.046298in}{3.094587in}}{\pgfqpoint{2.039120in}{3.101765in}}%
\pgfpathcurveto{\pgfqpoint{2.031941in}{3.108943in}}{\pgfqpoint{2.022204in}{3.112976in}}{\pgfqpoint{2.012053in}{3.112976in}}%
\pgfpathcurveto{\pgfqpoint{2.001902in}{3.112976in}}{\pgfqpoint{1.992165in}{3.108943in}}{\pgfqpoint{1.984986in}{3.101765in}}%
\pgfpathcurveto{\pgfqpoint{1.977808in}{3.094587in}}{\pgfqpoint{1.973775in}{3.084849in}}{\pgfqpoint{1.973775in}{3.074698in}}%
\pgfpathcurveto{\pgfqpoint{1.973775in}{3.064547in}}{\pgfqpoint{1.977808in}{3.054810in}}{\pgfqpoint{1.984986in}{3.047631in}}%
\pgfpathcurveto{\pgfqpoint{1.992165in}{3.040453in}}{\pgfqpoint{2.001902in}{3.036420in}}{\pgfqpoint{2.012053in}{3.036420in}}%
\pgfpathclose%
\pgfusepath{stroke,fill}%
\end{pgfscope}%
\begin{pgfscope}%
\pgfpathrectangle{\pgfqpoint{1.065196in}{0.528000in}}{\pgfqpoint{3.702804in}{3.696000in}} %
\pgfusepath{clip}%
\pgfsetbuttcap%
\pgfsetroundjoin%
\definecolor{currentfill}{rgb}{0.121569,0.466667,0.705882}%
\pgfsetfillcolor{currentfill}%
\pgfsetlinewidth{1.003750pt}%
\definecolor{currentstroke}{rgb}{0.121569,0.466667,0.705882}%
\pgfsetstrokecolor{currentstroke}%
\pgfsetdash{}{0pt}%
\pgfpathmoveto{\pgfqpoint{3.130241in}{0.677975in}}%
\pgfpathcurveto{\pgfqpoint{3.143983in}{0.677975in}}{\pgfqpoint{3.157165in}{0.683435in}}{\pgfqpoint{3.166883in}{0.693153in}}%
\pgfpathcurveto{\pgfqpoint{3.176600in}{0.702871in}}{\pgfqpoint{3.182060in}{0.716052in}}{\pgfqpoint{3.182060in}{0.729795in}}%
\pgfpathcurveto{\pgfqpoint{3.182060in}{0.743538in}}{\pgfqpoint{3.176600in}{0.756720in}}{\pgfqpoint{3.166883in}{0.766437in}}%
\pgfpathcurveto{\pgfqpoint{3.157165in}{0.776155in}}{\pgfqpoint{3.143983in}{0.781615in}}{\pgfqpoint{3.130241in}{0.781615in}}%
\pgfpathcurveto{\pgfqpoint{3.116498in}{0.781615in}}{\pgfqpoint{3.103316in}{0.776155in}}{\pgfqpoint{3.093598in}{0.766437in}}%
\pgfpathcurveto{\pgfqpoint{3.083881in}{0.756720in}}{\pgfqpoint{3.078421in}{0.743538in}}{\pgfqpoint{3.078421in}{0.729795in}}%
\pgfpathcurveto{\pgfqpoint{3.078421in}{0.716052in}}{\pgfqpoint{3.083881in}{0.702871in}}{\pgfqpoint{3.093598in}{0.693153in}}%
\pgfpathcurveto{\pgfqpoint{3.103316in}{0.683435in}}{\pgfqpoint{3.116498in}{0.677975in}}{\pgfqpoint{3.130241in}{0.677975in}}%
\pgfpathclose%
\pgfusepath{stroke,fill}%
\end{pgfscope}%
\begin{pgfscope}%
\pgfpathrectangle{\pgfqpoint{1.065196in}{0.528000in}}{\pgfqpoint{3.702804in}{3.696000in}} %
\pgfusepath{clip}%
\pgfsetbuttcap%
\pgfsetroundjoin%
\definecolor{currentfill}{rgb}{0.121569,0.466667,0.705882}%
\pgfsetfillcolor{currentfill}%
\pgfsetlinewidth{1.003750pt}%
\definecolor{currentstroke}{rgb}{0.121569,0.466667,0.705882}%
\pgfsetstrokecolor{currentstroke}%
\pgfsetdash{}{0pt}%
\pgfpathmoveto{\pgfqpoint{1.497380in}{3.916590in}}%
\pgfpathcurveto{\pgfqpoint{1.499875in}{3.916590in}}{\pgfqpoint{1.502269in}{3.917582in}}{\pgfqpoint{1.504033in}{3.919346in}}%
\pgfpathcurveto{\pgfqpoint{1.505797in}{3.921110in}}{\pgfqpoint{1.506789in}{3.923504in}}{\pgfqpoint{1.506789in}{3.925999in}}%
\pgfpathcurveto{\pgfqpoint{1.506789in}{3.928494in}}{\pgfqpoint{1.505797in}{3.930887in}}{\pgfqpoint{1.504033in}{3.932652in}}%
\pgfpathcurveto{\pgfqpoint{1.502269in}{3.934416in}}{\pgfqpoint{1.499875in}{3.935407in}}{\pgfqpoint{1.497380in}{3.935407in}}%
\pgfpathcurveto{\pgfqpoint{1.494885in}{3.935407in}}{\pgfqpoint{1.492492in}{3.934416in}}{\pgfqpoint{1.490728in}{3.932652in}}%
\pgfpathcurveto{\pgfqpoint{1.488963in}{3.930887in}}{\pgfqpoint{1.487972in}{3.928494in}}{\pgfqpoint{1.487972in}{3.925999in}}%
\pgfpathcurveto{\pgfqpoint{1.487972in}{3.923504in}}{\pgfqpoint{1.488963in}{3.921110in}}{\pgfqpoint{1.490728in}{3.919346in}}%
\pgfpathcurveto{\pgfqpoint{1.492492in}{3.917582in}}{\pgfqpoint{1.494885in}{3.916590in}}{\pgfqpoint{1.497380in}{3.916590in}}%
\pgfpathclose%
\pgfusepath{stroke,fill}%
\end{pgfscope}%
\begin{pgfscope}%
\pgfpathrectangle{\pgfqpoint{1.065196in}{0.528000in}}{\pgfqpoint{3.702804in}{3.696000in}} %
\pgfusepath{clip}%
\pgfsetbuttcap%
\pgfsetroundjoin%
\definecolor{currentfill}{rgb}{0.121569,0.466667,0.705882}%
\pgfsetfillcolor{currentfill}%
\pgfsetlinewidth{1.003750pt}%
\definecolor{currentstroke}{rgb}{0.121569,0.466667,0.705882}%
\pgfsetstrokecolor{currentstroke}%
\pgfsetdash{}{0pt}%
\pgfpathmoveto{\pgfqpoint{3.158307in}{1.323063in}}%
\pgfpathcurveto{\pgfqpoint{3.170315in}{1.323063in}}{\pgfqpoint{3.181834in}{1.327834in}}{\pgfqpoint{3.190325in}{1.336325in}}%
\pgfpathcurveto{\pgfqpoint{3.198816in}{1.344817in}}{\pgfqpoint{3.203587in}{1.356335in}}{\pgfqpoint{3.203587in}{1.368344in}}%
\pgfpathcurveto{\pgfqpoint{3.203587in}{1.380352in}}{\pgfqpoint{3.198816in}{1.391870in}}{\pgfqpoint{3.190325in}{1.400362in}}%
\pgfpathcurveto{\pgfqpoint{3.181834in}{1.408853in}}{\pgfqpoint{3.170315in}{1.413624in}}{\pgfqpoint{3.158307in}{1.413624in}}%
\pgfpathcurveto{\pgfqpoint{3.146298in}{1.413624in}}{\pgfqpoint{3.134780in}{1.408853in}}{\pgfqpoint{3.126288in}{1.400362in}}%
\pgfpathcurveto{\pgfqpoint{3.117797in}{1.391870in}}{\pgfqpoint{3.113026in}{1.380352in}}{\pgfqpoint{3.113026in}{1.368344in}}%
\pgfpathcurveto{\pgfqpoint{3.113026in}{1.356335in}}{\pgfqpoint{3.117797in}{1.344817in}}{\pgfqpoint{3.126288in}{1.336325in}}%
\pgfpathcurveto{\pgfqpoint{3.134780in}{1.327834in}}{\pgfqpoint{3.146298in}{1.323063in}}{\pgfqpoint{3.158307in}{1.323063in}}%
\pgfpathclose%
\pgfusepath{stroke,fill}%
\end{pgfscope}%
\begin{pgfscope}%
\pgfpathrectangle{\pgfqpoint{1.065196in}{0.528000in}}{\pgfqpoint{3.702804in}{3.696000in}} %
\pgfusepath{clip}%
\pgfsetbuttcap%
\pgfsetroundjoin%
\definecolor{currentfill}{rgb}{0.121569,0.466667,0.705882}%
\pgfsetfillcolor{currentfill}%
\pgfsetlinewidth{1.003750pt}%
\definecolor{currentstroke}{rgb}{0.121569,0.466667,0.705882}%
\pgfsetstrokecolor{currentstroke}%
\pgfsetdash{}{0pt}%
\pgfpathmoveto{\pgfqpoint{2.101159in}{3.127452in}}%
\pgfpathcurveto{\pgfqpoint{2.114547in}{3.127452in}}{\pgfqpoint{2.127388in}{3.132771in}}{\pgfqpoint{2.136855in}{3.142238in}}%
\pgfpathcurveto{\pgfqpoint{2.146322in}{3.151704in}}{\pgfqpoint{2.151641in}{3.164546in}}{\pgfqpoint{2.151641in}{3.177933in}}%
\pgfpathcurveto{\pgfqpoint{2.151641in}{3.191321in}}{\pgfqpoint{2.146322in}{3.204163in}}{\pgfqpoint{2.136855in}{3.213629in}}%
\pgfpathcurveto{\pgfqpoint{2.127388in}{3.223096in}}{\pgfqpoint{2.114547in}{3.228415in}}{\pgfqpoint{2.101159in}{3.228415in}}%
\pgfpathcurveto{\pgfqpoint{2.087771in}{3.228415in}}{\pgfqpoint{2.074930in}{3.223096in}}{\pgfqpoint{2.065463in}{3.213629in}}%
\pgfpathcurveto{\pgfqpoint{2.055997in}{3.204163in}}{\pgfqpoint{2.050678in}{3.191321in}}{\pgfqpoint{2.050678in}{3.177933in}}%
\pgfpathcurveto{\pgfqpoint{2.050678in}{3.164546in}}{\pgfqpoint{2.055997in}{3.151704in}}{\pgfqpoint{2.065463in}{3.142238in}}%
\pgfpathcurveto{\pgfqpoint{2.074930in}{3.132771in}}{\pgfqpoint{2.087771in}{3.127452in}}{\pgfqpoint{2.101159in}{3.127452in}}%
\pgfpathclose%
\pgfusepath{stroke,fill}%
\end{pgfscope}%
\begin{pgfscope}%
\pgfpathrectangle{\pgfqpoint{1.065196in}{0.528000in}}{\pgfqpoint{3.702804in}{3.696000in}} %
\pgfusepath{clip}%
\pgfsetbuttcap%
\pgfsetroundjoin%
\definecolor{currentfill}{rgb}{0.121569,0.466667,0.705882}%
\pgfsetfillcolor{currentfill}%
\pgfsetlinewidth{1.003750pt}%
\definecolor{currentstroke}{rgb}{0.121569,0.466667,0.705882}%
\pgfsetstrokecolor{currentstroke}%
\pgfsetdash{}{0pt}%
\pgfpathmoveto{\pgfqpoint{1.910682in}{2.625194in}}%
\pgfpathcurveto{\pgfqpoint{1.913025in}{2.625194in}}{\pgfqpoint{1.915272in}{2.626125in}}{\pgfqpoint{1.916929in}{2.627781in}}%
\pgfpathcurveto{\pgfqpoint{1.918586in}{2.629438in}}{\pgfqpoint{1.919517in}{2.631685in}}{\pgfqpoint{1.919517in}{2.634028in}}%
\pgfpathcurveto{\pgfqpoint{1.919517in}{2.636371in}}{\pgfqpoint{1.918586in}{2.638618in}}{\pgfqpoint{1.916929in}{2.640275in}}%
\pgfpathcurveto{\pgfqpoint{1.915272in}{2.641932in}}{\pgfqpoint{1.913025in}{2.642862in}}{\pgfqpoint{1.910682in}{2.642862in}}%
\pgfpathcurveto{\pgfqpoint{1.908340in}{2.642862in}}{\pgfqpoint{1.906092in}{2.641932in}}{\pgfqpoint{1.904436in}{2.640275in}}%
\pgfpathcurveto{\pgfqpoint{1.902779in}{2.638618in}}{\pgfqpoint{1.901848in}{2.636371in}}{\pgfqpoint{1.901848in}{2.634028in}}%
\pgfpathcurveto{\pgfqpoint{1.901848in}{2.631685in}}{\pgfqpoint{1.902779in}{2.629438in}}{\pgfqpoint{1.904436in}{2.627781in}}%
\pgfpathcurveto{\pgfqpoint{1.906092in}{2.626125in}}{\pgfqpoint{1.908340in}{2.625194in}}{\pgfqpoint{1.910682in}{2.625194in}}%
\pgfpathclose%
\pgfusepath{stroke,fill}%
\end{pgfscope}%
\begin{pgfscope}%
\pgfpathrectangle{\pgfqpoint{1.065196in}{0.528000in}}{\pgfqpoint{3.702804in}{3.696000in}} %
\pgfusepath{clip}%
\pgfsetbuttcap%
\pgfsetroundjoin%
\definecolor{currentfill}{rgb}{0.121569,0.466667,0.705882}%
\pgfsetfillcolor{currentfill}%
\pgfsetlinewidth{1.003750pt}%
\definecolor{currentstroke}{rgb}{0.121569,0.466667,0.705882}%
\pgfsetstrokecolor{currentstroke}%
\pgfsetdash{}{0pt}%
\pgfpathmoveto{\pgfqpoint{4.503849in}{3.484125in}}%
\pgfpathcurveto{\pgfqpoint{4.514097in}{3.484125in}}{\pgfqpoint{4.523926in}{3.488196in}}{\pgfqpoint{4.531172in}{3.495443in}}%
\pgfpathcurveto{\pgfqpoint{4.538418in}{3.502689in}}{\pgfqpoint{4.542490in}{3.512518in}}{\pgfqpoint{4.542490in}{3.522766in}}%
\pgfpathcurveto{\pgfqpoint{4.542490in}{3.533013in}}{\pgfqpoint{4.538418in}{3.542843in}}{\pgfqpoint{4.531172in}{3.550089in}}%
\pgfpathcurveto{\pgfqpoint{4.523926in}{3.557335in}}{\pgfqpoint{4.514097in}{3.561407in}}{\pgfqpoint{4.503849in}{3.561407in}}%
\pgfpathcurveto{\pgfqpoint{4.493601in}{3.561407in}}{\pgfqpoint{4.483772in}{3.557335in}}{\pgfqpoint{4.476526in}{3.550089in}}%
\pgfpathcurveto{\pgfqpoint{4.469280in}{3.542843in}}{\pgfqpoint{4.465208in}{3.533013in}}{\pgfqpoint{4.465208in}{3.522766in}}%
\pgfpathcurveto{\pgfqpoint{4.465208in}{3.512518in}}{\pgfqpoint{4.469280in}{3.502689in}}{\pgfqpoint{4.476526in}{3.495443in}}%
\pgfpathcurveto{\pgfqpoint{4.483772in}{3.488196in}}{\pgfqpoint{4.493601in}{3.484125in}}{\pgfqpoint{4.503849in}{3.484125in}}%
\pgfpathclose%
\pgfusepath{stroke,fill}%
\end{pgfscope}%
\begin{pgfscope}%
\pgfpathrectangle{\pgfqpoint{1.065196in}{0.528000in}}{\pgfqpoint{3.702804in}{3.696000in}} %
\pgfusepath{clip}%
\pgfsetbuttcap%
\pgfsetroundjoin%
\definecolor{currentfill}{rgb}{0.121569,0.466667,0.705882}%
\pgfsetfillcolor{currentfill}%
\pgfsetlinewidth{1.003750pt}%
\definecolor{currentstroke}{rgb}{0.121569,0.466667,0.705882}%
\pgfsetstrokecolor{currentstroke}%
\pgfsetdash{}{0pt}%
\pgfpathmoveto{\pgfqpoint{2.059385in}{2.310263in}}%
\pgfpathcurveto{\pgfqpoint{2.067483in}{2.310263in}}{\pgfqpoint{2.075251in}{2.313481in}}{\pgfqpoint{2.080977in}{2.319207in}}%
\pgfpathcurveto{\pgfqpoint{2.086703in}{2.324933in}}{\pgfqpoint{2.089920in}{2.332700in}}{\pgfqpoint{2.089920in}{2.340798in}}%
\pgfpathcurveto{\pgfqpoint{2.089920in}{2.348896in}}{\pgfqpoint{2.086703in}{2.356663in}}{\pgfqpoint{2.080977in}{2.362389in}}%
\pgfpathcurveto{\pgfqpoint{2.075251in}{2.368115in}}{\pgfqpoint{2.067483in}{2.371332in}}{\pgfqpoint{2.059385in}{2.371332in}}%
\pgfpathcurveto{\pgfqpoint{2.051288in}{2.371332in}}{\pgfqpoint{2.043520in}{2.368115in}}{\pgfqpoint{2.037794in}{2.362389in}}%
\pgfpathcurveto{\pgfqpoint{2.032068in}{2.356663in}}{\pgfqpoint{2.028851in}{2.348896in}}{\pgfqpoint{2.028851in}{2.340798in}}%
\pgfpathcurveto{\pgfqpoint{2.028851in}{2.332700in}}{\pgfqpoint{2.032068in}{2.324933in}}{\pgfqpoint{2.037794in}{2.319207in}}%
\pgfpathcurveto{\pgfqpoint{2.043520in}{2.313481in}}{\pgfqpoint{2.051288in}{2.310263in}}{\pgfqpoint{2.059385in}{2.310263in}}%
\pgfpathclose%
\pgfusepath{stroke,fill}%
\end{pgfscope}%
\begin{pgfscope}%
\pgfpathrectangle{\pgfqpoint{1.065196in}{0.528000in}}{\pgfqpoint{3.702804in}{3.696000in}} %
\pgfusepath{clip}%
\pgfsetbuttcap%
\pgfsetroundjoin%
\definecolor{currentfill}{rgb}{0.121569,0.466667,0.705882}%
\pgfsetfillcolor{currentfill}%
\pgfsetlinewidth{1.003750pt}%
\definecolor{currentstroke}{rgb}{0.121569,0.466667,0.705882}%
\pgfsetstrokecolor{currentstroke}%
\pgfsetdash{}{0pt}%
\pgfpathmoveto{\pgfqpoint{3.331907in}{3.413434in}}%
\pgfpathcurveto{\pgfqpoint{3.345056in}{3.413434in}}{\pgfqpoint{3.357668in}{3.418658in}}{\pgfqpoint{3.366966in}{3.427956in}}%
\pgfpathcurveto{\pgfqpoint{3.376264in}{3.437254in}}{\pgfqpoint{3.381488in}{3.449866in}}{\pgfqpoint{3.381488in}{3.463015in}}%
\pgfpathcurveto{\pgfqpoint{3.381488in}{3.476164in}}{\pgfqpoint{3.376264in}{3.488776in}}{\pgfqpoint{3.366966in}{3.498073in}}%
\pgfpathcurveto{\pgfqpoint{3.357668in}{3.507371in}}{\pgfqpoint{3.345056in}{3.512595in}}{\pgfqpoint{3.331907in}{3.512595in}}%
\pgfpathcurveto{\pgfqpoint{3.318758in}{3.512595in}}{\pgfqpoint{3.306146in}{3.507371in}}{\pgfqpoint{3.296849in}{3.498073in}}%
\pgfpathcurveto{\pgfqpoint{3.287551in}{3.488776in}}{\pgfqpoint{3.282327in}{3.476164in}}{\pgfqpoint{3.282327in}{3.463015in}}%
\pgfpathcurveto{\pgfqpoint{3.282327in}{3.449866in}}{\pgfqpoint{3.287551in}{3.437254in}}{\pgfqpoint{3.296849in}{3.427956in}}%
\pgfpathcurveto{\pgfqpoint{3.306146in}{3.418658in}}{\pgfqpoint{3.318758in}{3.413434in}}{\pgfqpoint{3.331907in}{3.413434in}}%
\pgfpathclose%
\pgfusepath{stroke,fill}%
\end{pgfscope}%
\begin{pgfscope}%
\pgfpathrectangle{\pgfqpoint{1.065196in}{0.528000in}}{\pgfqpoint{3.702804in}{3.696000in}} %
\pgfusepath{clip}%
\pgfsetbuttcap%
\pgfsetroundjoin%
\definecolor{currentfill}{rgb}{0.121569,0.466667,0.705882}%
\pgfsetfillcolor{currentfill}%
\pgfsetlinewidth{1.003750pt}%
\definecolor{currentstroke}{rgb}{0.121569,0.466667,0.705882}%
\pgfsetstrokecolor{currentstroke}%
\pgfsetdash{}{0pt}%
\pgfpathmoveto{\pgfqpoint{1.781330in}{0.709982in}}%
\pgfpathcurveto{\pgfqpoint{1.791930in}{0.709982in}}{\pgfqpoint{1.802097in}{0.714193in}}{\pgfqpoint{1.809592in}{0.721688in}}%
\pgfpathcurveto{\pgfqpoint{1.817086in}{0.729183in}}{\pgfqpoint{1.821298in}{0.739350in}}{\pgfqpoint{1.821298in}{0.749950in}}%
\pgfpathcurveto{\pgfqpoint{1.821298in}{0.760549in}}{\pgfqpoint{1.817086in}{0.770716in}}{\pgfqpoint{1.809592in}{0.778211in}}%
\pgfpathcurveto{\pgfqpoint{1.802097in}{0.785706in}}{\pgfqpoint{1.791930in}{0.789917in}}{\pgfqpoint{1.781330in}{0.789917in}}%
\pgfpathcurveto{\pgfqpoint{1.770731in}{0.789917in}}{\pgfqpoint{1.760564in}{0.785706in}}{\pgfqpoint{1.753069in}{0.778211in}}%
\pgfpathcurveto{\pgfqpoint{1.745574in}{0.770716in}}{\pgfqpoint{1.741363in}{0.760549in}}{\pgfqpoint{1.741363in}{0.749950in}}%
\pgfpathcurveto{\pgfqpoint{1.741363in}{0.739350in}}{\pgfqpoint{1.745574in}{0.729183in}}{\pgfqpoint{1.753069in}{0.721688in}}%
\pgfpathcurveto{\pgfqpoint{1.760564in}{0.714193in}}{\pgfqpoint{1.770731in}{0.709982in}}{\pgfqpoint{1.781330in}{0.709982in}}%
\pgfpathclose%
\pgfusepath{stroke,fill}%
\end{pgfscope}%
\begin{pgfscope}%
\pgfpathrectangle{\pgfqpoint{1.065196in}{0.528000in}}{\pgfqpoint{3.702804in}{3.696000in}} %
\pgfusepath{clip}%
\pgfsetbuttcap%
\pgfsetroundjoin%
\definecolor{currentfill}{rgb}{0.121569,0.466667,0.705882}%
\pgfsetfillcolor{currentfill}%
\pgfsetlinewidth{1.003750pt}%
\definecolor{currentstroke}{rgb}{0.121569,0.466667,0.705882}%
\pgfsetstrokecolor{currentstroke}%
\pgfsetdash{}{0pt}%
\pgfpathmoveto{\pgfqpoint{1.490845in}{2.271511in}}%
\pgfpathcurveto{\pgfqpoint{1.502628in}{2.271511in}}{\pgfqpoint{1.513930in}{2.276192in}}{\pgfqpoint{1.522262in}{2.284524in}}%
\pgfpathcurveto{\pgfqpoint{1.530594in}{2.292856in}}{\pgfqpoint{1.535276in}{2.304159in}}{\pgfqpoint{1.535276in}{2.315942in}}%
\pgfpathcurveto{\pgfqpoint{1.535276in}{2.327725in}}{\pgfqpoint{1.530594in}{2.339027in}}{\pgfqpoint{1.522262in}{2.347359in}}%
\pgfpathcurveto{\pgfqpoint{1.513930in}{2.355691in}}{\pgfqpoint{1.502628in}{2.360373in}}{\pgfqpoint{1.490845in}{2.360373in}}%
\pgfpathcurveto{\pgfqpoint{1.479062in}{2.360373in}}{\pgfqpoint{1.467759in}{2.355691in}}{\pgfqpoint{1.459427in}{2.347359in}}%
\pgfpathcurveto{\pgfqpoint{1.451095in}{2.339027in}}{\pgfqpoint{1.446414in}{2.327725in}}{\pgfqpoint{1.446414in}{2.315942in}}%
\pgfpathcurveto{\pgfqpoint{1.446414in}{2.304159in}}{\pgfqpoint{1.451095in}{2.292856in}}{\pgfqpoint{1.459427in}{2.284524in}}%
\pgfpathcurveto{\pgfqpoint{1.467759in}{2.276192in}}{\pgfqpoint{1.479062in}{2.271511in}}{\pgfqpoint{1.490845in}{2.271511in}}%
\pgfpathclose%
\pgfusepath{stroke,fill}%
\end{pgfscope}%
\begin{pgfscope}%
\pgfpathrectangle{\pgfqpoint{1.065196in}{0.528000in}}{\pgfqpoint{3.702804in}{3.696000in}} %
\pgfusepath{clip}%
\pgfsetbuttcap%
\pgfsetroundjoin%
\definecolor{currentfill}{rgb}{0.121569,0.466667,0.705882}%
\pgfsetfillcolor{currentfill}%
\pgfsetlinewidth{1.003750pt}%
\definecolor{currentstroke}{rgb}{0.121569,0.466667,0.705882}%
\pgfsetstrokecolor{currentstroke}%
\pgfsetdash{}{0pt}%
\pgfpathmoveto{\pgfqpoint{3.286289in}{2.556167in}}%
\pgfpathcurveto{\pgfqpoint{3.295834in}{2.556167in}}{\pgfqpoint{3.304989in}{2.559959in}}{\pgfqpoint{3.311737in}{2.566708in}}%
\pgfpathcurveto{\pgfqpoint{3.318486in}{2.573457in}}{\pgfqpoint{3.322278in}{2.582611in}}{\pgfqpoint{3.322278in}{2.592156in}}%
\pgfpathcurveto{\pgfqpoint{3.322278in}{2.601700in}}{\pgfqpoint{3.318486in}{2.610855in}}{\pgfqpoint{3.311737in}{2.617604in}}%
\pgfpathcurveto{\pgfqpoint{3.304989in}{2.624353in}}{\pgfqpoint{3.295834in}{2.628145in}}{\pgfqpoint{3.286289in}{2.628145in}}%
\pgfpathcurveto{\pgfqpoint{3.276745in}{2.628145in}}{\pgfqpoint{3.267590in}{2.624353in}}{\pgfqpoint{3.260841in}{2.617604in}}%
\pgfpathcurveto{\pgfqpoint{3.254093in}{2.610855in}}{\pgfqpoint{3.250301in}{2.601700in}}{\pgfqpoint{3.250301in}{2.592156in}}%
\pgfpathcurveto{\pgfqpoint{3.250301in}{2.582611in}}{\pgfqpoint{3.254093in}{2.573457in}}{\pgfqpoint{3.260841in}{2.566708in}}%
\pgfpathcurveto{\pgfqpoint{3.267590in}{2.559959in}}{\pgfqpoint{3.276745in}{2.556167in}}{\pgfqpoint{3.286289in}{2.556167in}}%
\pgfpathclose%
\pgfusepath{stroke,fill}%
\end{pgfscope}%
\begin{pgfscope}%
\pgfpathrectangle{\pgfqpoint{1.065196in}{0.528000in}}{\pgfqpoint{3.702804in}{3.696000in}} %
\pgfusepath{clip}%
\pgfsetbuttcap%
\pgfsetroundjoin%
\definecolor{currentfill}{rgb}{0.121569,0.466667,0.705882}%
\pgfsetfillcolor{currentfill}%
\pgfsetlinewidth{1.003750pt}%
\definecolor{currentstroke}{rgb}{0.121569,0.466667,0.705882}%
\pgfsetstrokecolor{currentstroke}%
\pgfsetdash{}{0pt}%
\pgfpathmoveto{\pgfqpoint{2.318888in}{3.955049in}}%
\pgfpathcurveto{\pgfqpoint{2.330131in}{3.955049in}}{\pgfqpoint{2.340914in}{3.959515in}}{\pgfqpoint{2.348863in}{3.967465in}}%
\pgfpathcurveto{\pgfqpoint{2.356813in}{3.975414in}}{\pgfqpoint{2.361280in}{3.986198in}}{\pgfqpoint{2.361280in}{3.997440in}}%
\pgfpathcurveto{\pgfqpoint{2.361280in}{4.008682in}}{\pgfqpoint{2.356813in}{4.019466in}}{\pgfqpoint{2.348863in}{4.027415in}}%
\pgfpathcurveto{\pgfqpoint{2.340914in}{4.035365in}}{\pgfqpoint{2.330131in}{4.039831in}}{\pgfqpoint{2.318888in}{4.039831in}}%
\pgfpathcurveto{\pgfqpoint{2.307646in}{4.039831in}}{\pgfqpoint{2.296863in}{4.035365in}}{\pgfqpoint{2.288913in}{4.027415in}}%
\pgfpathcurveto{\pgfqpoint{2.280964in}{4.019466in}}{\pgfqpoint{2.276497in}{4.008682in}}{\pgfqpoint{2.276497in}{3.997440in}}%
\pgfpathcurveto{\pgfqpoint{2.276497in}{3.986198in}}{\pgfqpoint{2.280964in}{3.975414in}}{\pgfqpoint{2.288913in}{3.967465in}}%
\pgfpathcurveto{\pgfqpoint{2.296863in}{3.959515in}}{\pgfqpoint{2.307646in}{3.955049in}}{\pgfqpoint{2.318888in}{3.955049in}}%
\pgfpathclose%
\pgfusepath{stroke,fill}%
\end{pgfscope}%
\begin{pgfscope}%
\pgfpathrectangle{\pgfqpoint{1.065196in}{0.528000in}}{\pgfqpoint{3.702804in}{3.696000in}} %
\pgfusepath{clip}%
\pgfsetbuttcap%
\pgfsetroundjoin%
\definecolor{currentfill}{rgb}{0.121569,0.466667,0.705882}%
\pgfsetfillcolor{currentfill}%
\pgfsetlinewidth{1.003750pt}%
\definecolor{currentstroke}{rgb}{0.121569,0.466667,0.705882}%
\pgfsetstrokecolor{currentstroke}%
\pgfsetdash{}{0pt}%
\pgfpathmoveto{\pgfqpoint{3.197291in}{1.931589in}}%
\pgfpathcurveto{\pgfqpoint{3.204930in}{1.931589in}}{\pgfqpoint{3.212257in}{1.934624in}}{\pgfqpoint{3.217659in}{1.940025in}}%
\pgfpathcurveto{\pgfqpoint{3.223060in}{1.945427in}}{\pgfqpoint{3.226095in}{1.952754in}}{\pgfqpoint{3.226095in}{1.960393in}}%
\pgfpathcurveto{\pgfqpoint{3.226095in}{1.968032in}}{\pgfqpoint{3.223060in}{1.975360in}}{\pgfqpoint{3.217659in}{1.980761in}}%
\pgfpathcurveto{\pgfqpoint{3.212257in}{1.986163in}}{\pgfqpoint{3.204930in}{1.989198in}}{\pgfqpoint{3.197291in}{1.989198in}}%
\pgfpathcurveto{\pgfqpoint{3.189652in}{1.989198in}}{\pgfqpoint{3.182325in}{1.986163in}}{\pgfqpoint{3.176923in}{1.980761in}}%
\pgfpathcurveto{\pgfqpoint{3.171521in}{1.975360in}}{\pgfqpoint{3.168486in}{1.968032in}}{\pgfqpoint{3.168486in}{1.960393in}}%
\pgfpathcurveto{\pgfqpoint{3.168486in}{1.952754in}}{\pgfqpoint{3.171521in}{1.945427in}}{\pgfqpoint{3.176923in}{1.940025in}}%
\pgfpathcurveto{\pgfqpoint{3.182325in}{1.934624in}}{\pgfqpoint{3.189652in}{1.931589in}}{\pgfqpoint{3.197291in}{1.931589in}}%
\pgfpathclose%
\pgfusepath{stroke,fill}%
\end{pgfscope}%
\begin{pgfscope}%
\pgfpathrectangle{\pgfqpoint{1.065196in}{0.528000in}}{\pgfqpoint{3.702804in}{3.696000in}} %
\pgfusepath{clip}%
\pgfsetbuttcap%
\pgfsetroundjoin%
\definecolor{currentfill}{rgb}{0.121569,0.466667,0.705882}%
\pgfsetfillcolor{currentfill}%
\pgfsetlinewidth{1.003750pt}%
\definecolor{currentstroke}{rgb}{0.121569,0.466667,0.705882}%
\pgfsetstrokecolor{currentstroke}%
\pgfsetdash{}{0pt}%
\pgfpathmoveto{\pgfqpoint{3.100885in}{3.177813in}}%
\pgfpathcurveto{\pgfqpoint{3.102256in}{3.177813in}}{\pgfqpoint{3.103571in}{3.178358in}}{\pgfqpoint{3.104541in}{3.179328in}}%
\pgfpathcurveto{\pgfqpoint{3.105511in}{3.180297in}}{\pgfqpoint{3.106055in}{3.181613in}}{\pgfqpoint{3.106055in}{3.182984in}}%
\pgfpathcurveto{\pgfqpoint{3.106055in}{3.184355in}}{\pgfqpoint{3.105511in}{3.185671in}}{\pgfqpoint{3.104541in}{3.186640in}}%
\pgfpathcurveto{\pgfqpoint{3.103571in}{3.187610in}}{\pgfqpoint{3.102256in}{3.188155in}}{\pgfqpoint{3.100885in}{3.188155in}}%
\pgfpathcurveto{\pgfqpoint{3.099513in}{3.188155in}}{\pgfqpoint{3.098198in}{3.187610in}}{\pgfqpoint{3.097228in}{3.186640in}}%
\pgfpathcurveto{\pgfqpoint{3.096259in}{3.185671in}}{\pgfqpoint{3.095714in}{3.184355in}}{\pgfqpoint{3.095714in}{3.182984in}}%
\pgfpathcurveto{\pgfqpoint{3.095714in}{3.181613in}}{\pgfqpoint{3.096259in}{3.180297in}}{\pgfqpoint{3.097228in}{3.179328in}}%
\pgfpathcurveto{\pgfqpoint{3.098198in}{3.178358in}}{\pgfqpoint{3.099513in}{3.177813in}}{\pgfqpoint{3.100885in}{3.177813in}}%
\pgfpathclose%
\pgfusepath{stroke,fill}%
\end{pgfscope}%
\begin{pgfscope}%
\pgfpathrectangle{\pgfqpoint{1.065196in}{0.528000in}}{\pgfqpoint{3.702804in}{3.696000in}} %
\pgfusepath{clip}%
\pgfsetbuttcap%
\pgfsetroundjoin%
\definecolor{currentfill}{rgb}{0.121569,0.466667,0.705882}%
\pgfsetfillcolor{currentfill}%
\pgfsetlinewidth{1.003750pt}%
\definecolor{currentstroke}{rgb}{0.121569,0.466667,0.705882}%
\pgfsetstrokecolor{currentstroke}%
\pgfsetdash{}{0pt}%
\pgfpathmoveto{\pgfqpoint{3.496877in}{1.534041in}}%
\pgfpathcurveto{\pgfqpoint{3.507648in}{1.534041in}}{\pgfqpoint{3.517980in}{1.538320in}}{\pgfqpoint{3.525596in}{1.545936in}}%
\pgfpathcurveto{\pgfqpoint{3.533212in}{1.553553in}}{\pgfqpoint{3.537492in}{1.563884in}}{\pgfqpoint{3.537492in}{1.574656in}}%
\pgfpathcurveto{\pgfqpoint{3.537492in}{1.585427in}}{\pgfqpoint{3.533212in}{1.595758in}}{\pgfqpoint{3.525596in}{1.603375in}}%
\pgfpathcurveto{\pgfqpoint{3.517980in}{1.610991in}}{\pgfqpoint{3.507648in}{1.615270in}}{\pgfqpoint{3.496877in}{1.615270in}}%
\pgfpathcurveto{\pgfqpoint{3.486106in}{1.615270in}}{\pgfqpoint{3.475774in}{1.610991in}}{\pgfqpoint{3.468158in}{1.603375in}}%
\pgfpathcurveto{\pgfqpoint{3.460541in}{1.595758in}}{\pgfqpoint{3.456262in}{1.585427in}}{\pgfqpoint{3.456262in}{1.574656in}}%
\pgfpathcurveto{\pgfqpoint{3.456262in}{1.563884in}}{\pgfqpoint{3.460541in}{1.553553in}}{\pgfqpoint{3.468158in}{1.545936in}}%
\pgfpathcurveto{\pgfqpoint{3.475774in}{1.538320in}}{\pgfqpoint{3.486106in}{1.534041in}}{\pgfqpoint{3.496877in}{1.534041in}}%
\pgfpathclose%
\pgfusepath{stroke,fill}%
\end{pgfscope}%
\begin{pgfscope}%
\pgfpathrectangle{\pgfqpoint{1.065196in}{0.528000in}}{\pgfqpoint{3.702804in}{3.696000in}} %
\pgfusepath{clip}%
\pgfsetbuttcap%
\pgfsetroundjoin%
\definecolor{currentfill}{rgb}{0.121569,0.466667,0.705882}%
\pgfsetfillcolor{currentfill}%
\pgfsetlinewidth{1.003750pt}%
\definecolor{currentstroke}{rgb}{0.121569,0.466667,0.705882}%
\pgfsetstrokecolor{currentstroke}%
\pgfsetdash{}{0pt}%
\pgfpathmoveto{\pgfqpoint{1.478501in}{1.896879in}}%
\pgfpathcurveto{\pgfqpoint{1.486416in}{1.896879in}}{\pgfqpoint{1.494008in}{1.900024in}}{\pgfqpoint{1.499605in}{1.905621in}}%
\pgfpathcurveto{\pgfqpoint{1.505201in}{1.911218in}}{\pgfqpoint{1.508346in}{1.918810in}}{\pgfqpoint{1.508346in}{1.926725in}}%
\pgfpathcurveto{\pgfqpoint{1.508346in}{1.934640in}}{\pgfqpoint{1.505201in}{1.942232in}}{\pgfqpoint{1.499605in}{1.947829in}}%
\pgfpathcurveto{\pgfqpoint{1.494008in}{1.953426in}}{\pgfqpoint{1.486416in}{1.956570in}}{\pgfqpoint{1.478501in}{1.956570in}}%
\pgfpathcurveto{\pgfqpoint{1.470585in}{1.956570in}}{\pgfqpoint{1.462993in}{1.953426in}}{\pgfqpoint{1.457397in}{1.947829in}}%
\pgfpathcurveto{\pgfqpoint{1.451800in}{1.942232in}}{\pgfqpoint{1.448655in}{1.934640in}}{\pgfqpoint{1.448655in}{1.926725in}}%
\pgfpathcurveto{\pgfqpoint{1.448655in}{1.918810in}}{\pgfqpoint{1.451800in}{1.911218in}}{\pgfqpoint{1.457397in}{1.905621in}}%
\pgfpathcurveto{\pgfqpoint{1.462993in}{1.900024in}}{\pgfqpoint{1.470585in}{1.896879in}}{\pgfqpoint{1.478501in}{1.896879in}}%
\pgfpathclose%
\pgfusepath{stroke,fill}%
\end{pgfscope}%
\begin{pgfscope}%
\pgfpathrectangle{\pgfqpoint{1.065196in}{0.528000in}}{\pgfqpoint{3.702804in}{3.696000in}} %
\pgfusepath{clip}%
\pgfsetbuttcap%
\pgfsetroundjoin%
\definecolor{currentfill}{rgb}{0.121569,0.466667,0.705882}%
\pgfsetfillcolor{currentfill}%
\pgfsetlinewidth{1.003750pt}%
\definecolor{currentstroke}{rgb}{0.121569,0.466667,0.705882}%
\pgfsetstrokecolor{currentstroke}%
\pgfsetdash{}{0pt}%
\pgfpathmoveto{\pgfqpoint{3.364588in}{1.352916in}}%
\pgfpathcurveto{\pgfqpoint{3.374788in}{1.352916in}}{\pgfqpoint{3.384572in}{1.356968in}}{\pgfqpoint{3.391785in}{1.364181in}}%
\pgfpathcurveto{\pgfqpoint{3.398998in}{1.371394in}}{\pgfqpoint{3.403051in}{1.381178in}}{\pgfqpoint{3.403051in}{1.391379in}}%
\pgfpathcurveto{\pgfqpoint{3.403051in}{1.401580in}}{\pgfqpoint{3.398998in}{1.411364in}}{\pgfqpoint{3.391785in}{1.418577in}}%
\pgfpathcurveto{\pgfqpoint{3.384572in}{1.425790in}}{\pgfqpoint{3.374788in}{1.429842in}}{\pgfqpoint{3.364588in}{1.429842in}}%
\pgfpathcurveto{\pgfqpoint{3.354387in}{1.429842in}}{\pgfqpoint{3.344603in}{1.425790in}}{\pgfqpoint{3.337390in}{1.418577in}}%
\pgfpathcurveto{\pgfqpoint{3.330177in}{1.411364in}}{\pgfqpoint{3.326124in}{1.401580in}}{\pgfqpoint{3.326124in}{1.391379in}}%
\pgfpathcurveto{\pgfqpoint{3.326124in}{1.381178in}}{\pgfqpoint{3.330177in}{1.371394in}}{\pgfqpoint{3.337390in}{1.364181in}}%
\pgfpathcurveto{\pgfqpoint{3.344603in}{1.356968in}}{\pgfqpoint{3.354387in}{1.352916in}}{\pgfqpoint{3.364588in}{1.352916in}}%
\pgfpathclose%
\pgfusepath{stroke,fill}%
\end{pgfscope}%
\begin{pgfscope}%
\pgfpathrectangle{\pgfqpoint{1.065196in}{0.528000in}}{\pgfqpoint{3.702804in}{3.696000in}} %
\pgfusepath{clip}%
\pgfsetbuttcap%
\pgfsetroundjoin%
\definecolor{currentfill}{rgb}{0.121569,0.466667,0.705882}%
\pgfsetfillcolor{currentfill}%
\pgfsetlinewidth{1.003750pt}%
\definecolor{currentstroke}{rgb}{0.121569,0.466667,0.705882}%
\pgfsetstrokecolor{currentstroke}%
\pgfsetdash{}{0pt}%
\pgfpathmoveto{\pgfqpoint{3.776493in}{0.859315in}}%
\pgfpathcurveto{\pgfqpoint{3.790071in}{0.859315in}}{\pgfqpoint{3.803094in}{0.864709in}}{\pgfqpoint{3.812694in}{0.874310in}}%
\pgfpathcurveto{\pgfqpoint{3.822295in}{0.883910in}}{\pgfqpoint{3.827689in}{0.896933in}}{\pgfqpoint{3.827689in}{0.910511in}}%
\pgfpathcurveto{\pgfqpoint{3.827689in}{0.924088in}}{\pgfqpoint{3.822295in}{0.937111in}}{\pgfqpoint{3.812694in}{0.946711in}}%
\pgfpathcurveto{\pgfqpoint{3.803094in}{0.956312in}}{\pgfqpoint{3.790071in}{0.961706in}}{\pgfqpoint{3.776493in}{0.961706in}}%
\pgfpathcurveto{\pgfqpoint{3.762916in}{0.961706in}}{\pgfqpoint{3.749893in}{0.956312in}}{\pgfqpoint{3.740292in}{0.946711in}}%
\pgfpathcurveto{\pgfqpoint{3.730692in}{0.937111in}}{\pgfqpoint{3.725298in}{0.924088in}}{\pgfqpoint{3.725298in}{0.910511in}}%
\pgfpathcurveto{\pgfqpoint{3.725298in}{0.896933in}}{\pgfqpoint{3.730692in}{0.883910in}}{\pgfqpoint{3.740292in}{0.874310in}}%
\pgfpathcurveto{\pgfqpoint{3.749893in}{0.864709in}}{\pgfqpoint{3.762916in}{0.859315in}}{\pgfqpoint{3.776493in}{0.859315in}}%
\pgfpathclose%
\pgfusepath{stroke,fill}%
\end{pgfscope}%
\begin{pgfscope}%
\pgfpathrectangle{\pgfqpoint{1.065196in}{0.528000in}}{\pgfqpoint{3.702804in}{3.696000in}} %
\pgfusepath{clip}%
\pgfsetbuttcap%
\pgfsetroundjoin%
\definecolor{currentfill}{rgb}{0.121569,0.466667,0.705882}%
\pgfsetfillcolor{currentfill}%
\pgfsetlinewidth{1.003750pt}%
\definecolor{currentstroke}{rgb}{0.121569,0.466667,0.705882}%
\pgfsetstrokecolor{currentstroke}%
\pgfsetdash{}{0pt}%
\pgfpathmoveto{\pgfqpoint{2.127906in}{3.346985in}}%
\pgfpathcurveto{\pgfqpoint{2.137168in}{3.346985in}}{\pgfqpoint{2.146052in}{3.350665in}}{\pgfqpoint{2.152602in}{3.357214in}}%
\pgfpathcurveto{\pgfqpoint{2.159151in}{3.363764in}}{\pgfqpoint{2.162831in}{3.372648in}}{\pgfqpoint{2.162831in}{3.381910in}}%
\pgfpathcurveto{\pgfqpoint{2.162831in}{3.391172in}}{\pgfqpoint{2.159151in}{3.400056in}}{\pgfqpoint{2.152602in}{3.406606in}}%
\pgfpathcurveto{\pgfqpoint{2.146052in}{3.413155in}}{\pgfqpoint{2.137168in}{3.416835in}}{\pgfqpoint{2.127906in}{3.416835in}}%
\pgfpathcurveto{\pgfqpoint{2.118644in}{3.416835in}}{\pgfqpoint{2.109760in}{3.413155in}}{\pgfqpoint{2.103210in}{3.406606in}}%
\pgfpathcurveto{\pgfqpoint{2.096661in}{3.400056in}}{\pgfqpoint{2.092981in}{3.391172in}}{\pgfqpoint{2.092981in}{3.381910in}}%
\pgfpathcurveto{\pgfqpoint{2.092981in}{3.372648in}}{\pgfqpoint{2.096661in}{3.363764in}}{\pgfqpoint{2.103210in}{3.357214in}}%
\pgfpathcurveto{\pgfqpoint{2.109760in}{3.350665in}}{\pgfqpoint{2.118644in}{3.346985in}}{\pgfqpoint{2.127906in}{3.346985in}}%
\pgfpathclose%
\pgfusepath{stroke,fill}%
\end{pgfscope}%
\begin{pgfscope}%
\pgfpathrectangle{\pgfqpoint{1.065196in}{0.528000in}}{\pgfqpoint{3.702804in}{3.696000in}} %
\pgfusepath{clip}%
\pgfsetbuttcap%
\pgfsetroundjoin%
\definecolor{currentfill}{rgb}{0.121569,0.466667,0.705882}%
\pgfsetfillcolor{currentfill}%
\pgfsetlinewidth{1.003750pt}%
\definecolor{currentstroke}{rgb}{0.121569,0.466667,0.705882}%
\pgfsetstrokecolor{currentstroke}%
\pgfsetdash{}{0pt}%
\pgfpathmoveto{\pgfqpoint{1.904003in}{2.801941in}}%
\pgfpathcurveto{\pgfqpoint{1.911057in}{2.801941in}}{\pgfqpoint{1.917824in}{2.804744in}}{\pgfqpoint{1.922812in}{2.809732in}}%
\pgfpathcurveto{\pgfqpoint{1.927800in}{2.814720in}}{\pgfqpoint{1.930603in}{2.821487in}}{\pgfqpoint{1.930603in}{2.828541in}}%
\pgfpathcurveto{\pgfqpoint{1.930603in}{2.835596in}}{\pgfqpoint{1.927800in}{2.842362in}}{\pgfqpoint{1.922812in}{2.847351in}}%
\pgfpathcurveto{\pgfqpoint{1.917824in}{2.852339in}}{\pgfqpoint{1.911057in}{2.855142in}}{\pgfqpoint{1.904003in}{2.855142in}}%
\pgfpathcurveto{\pgfqpoint{1.896948in}{2.855142in}}{\pgfqpoint{1.890182in}{2.852339in}}{\pgfqpoint{1.885194in}{2.847351in}}%
\pgfpathcurveto{\pgfqpoint{1.880205in}{2.842362in}}{\pgfqpoint{1.877403in}{2.835596in}}{\pgfqpoint{1.877403in}{2.828541in}}%
\pgfpathcurveto{\pgfqpoint{1.877403in}{2.821487in}}{\pgfqpoint{1.880205in}{2.814720in}}{\pgfqpoint{1.885194in}{2.809732in}}%
\pgfpathcurveto{\pgfqpoint{1.890182in}{2.804744in}}{\pgfqpoint{1.896948in}{2.801941in}}{\pgfqpoint{1.904003in}{2.801941in}}%
\pgfpathclose%
\pgfusepath{stroke,fill}%
\end{pgfscope}%
\begin{pgfscope}%
\pgfpathrectangle{\pgfqpoint{1.065196in}{0.528000in}}{\pgfqpoint{3.702804in}{3.696000in}} %
\pgfusepath{clip}%
\pgfsetbuttcap%
\pgfsetroundjoin%
\definecolor{currentfill}{rgb}{0.121569,0.466667,0.705882}%
\pgfsetfillcolor{currentfill}%
\pgfsetlinewidth{1.003750pt}%
\definecolor{currentstroke}{rgb}{0.121569,0.466667,0.705882}%
\pgfsetstrokecolor{currentstroke}%
\pgfsetdash{}{0pt}%
\pgfpathmoveto{\pgfqpoint{3.012912in}{3.751236in}}%
\pgfpathcurveto{\pgfqpoint{3.021554in}{3.751236in}}{\pgfqpoint{3.029843in}{3.754670in}}{\pgfqpoint{3.035954in}{3.760781in}}%
\pgfpathcurveto{\pgfqpoint{3.042065in}{3.766892in}}{\pgfqpoint{3.045499in}{3.775181in}}{\pgfqpoint{3.045499in}{3.783824in}}%
\pgfpathcurveto{\pgfqpoint{3.045499in}{3.792466in}}{\pgfqpoint{3.042065in}{3.800755in}}{\pgfqpoint{3.035954in}{3.806866in}}%
\pgfpathcurveto{\pgfqpoint{3.029843in}{3.812977in}}{\pgfqpoint{3.021554in}{3.816411in}}{\pgfqpoint{3.012912in}{3.816411in}}%
\pgfpathcurveto{\pgfqpoint{3.004269in}{3.816411in}}{\pgfqpoint{2.995980in}{3.812977in}}{\pgfqpoint{2.989869in}{3.806866in}}%
\pgfpathcurveto{\pgfqpoint{2.983758in}{3.800755in}}{\pgfqpoint{2.980324in}{3.792466in}}{\pgfqpoint{2.980324in}{3.783824in}}%
\pgfpathcurveto{\pgfqpoint{2.980324in}{3.775181in}}{\pgfqpoint{2.983758in}{3.766892in}}{\pgfqpoint{2.989869in}{3.760781in}}%
\pgfpathcurveto{\pgfqpoint{2.995980in}{3.754670in}}{\pgfqpoint{3.004269in}{3.751236in}}{\pgfqpoint{3.012912in}{3.751236in}}%
\pgfpathclose%
\pgfusepath{stroke,fill}%
\end{pgfscope}%
\begin{pgfscope}%
\pgfpathrectangle{\pgfqpoint{1.065196in}{0.528000in}}{\pgfqpoint{3.702804in}{3.696000in}} %
\pgfusepath{clip}%
\pgfsetbuttcap%
\pgfsetroundjoin%
\definecolor{currentfill}{rgb}{0.121569,0.466667,0.705882}%
\pgfsetfillcolor{currentfill}%
\pgfsetlinewidth{1.003750pt}%
\definecolor{currentstroke}{rgb}{0.121569,0.466667,0.705882}%
\pgfsetstrokecolor{currentstroke}%
\pgfsetdash{}{0pt}%
\pgfpathmoveto{\pgfqpoint{2.137888in}{0.856959in}}%
\pgfpathcurveto{\pgfqpoint{2.151579in}{0.856959in}}{\pgfqpoint{2.164711in}{0.862398in}}{\pgfqpoint{2.174393in}{0.872079in}}%
\pgfpathcurveto{\pgfqpoint{2.184074in}{0.881761in}}{\pgfqpoint{2.189513in}{0.894893in}}{\pgfqpoint{2.189513in}{0.908584in}}%
\pgfpathcurveto{\pgfqpoint{2.189513in}{0.922276in}}{\pgfqpoint{2.184074in}{0.935408in}}{\pgfqpoint{2.174393in}{0.945089in}}%
\pgfpathcurveto{\pgfqpoint{2.164711in}{0.954770in}}{\pgfqpoint{2.151579in}{0.960210in}}{\pgfqpoint{2.137888in}{0.960210in}}%
\pgfpathcurveto{\pgfqpoint{2.124197in}{0.960210in}}{\pgfqpoint{2.111064in}{0.954770in}}{\pgfqpoint{2.101383in}{0.945089in}}%
\pgfpathcurveto{\pgfqpoint{2.091702in}{0.935408in}}{\pgfqpoint{2.086262in}{0.922276in}}{\pgfqpoint{2.086262in}{0.908584in}}%
\pgfpathcurveto{\pgfqpoint{2.086262in}{0.894893in}}{\pgfqpoint{2.091702in}{0.881761in}}{\pgfqpoint{2.101383in}{0.872079in}}%
\pgfpathcurveto{\pgfqpoint{2.111064in}{0.862398in}}{\pgfqpoint{2.124197in}{0.856959in}}{\pgfqpoint{2.137888in}{0.856959in}}%
\pgfpathclose%
\pgfusepath{stroke,fill}%
\end{pgfscope}%
\begin{pgfscope}%
\pgfpathrectangle{\pgfqpoint{1.065196in}{0.528000in}}{\pgfqpoint{3.702804in}{3.696000in}} %
\pgfusepath{clip}%
\pgfsetbuttcap%
\pgfsetroundjoin%
\definecolor{currentfill}{rgb}{0.121569,0.466667,0.705882}%
\pgfsetfillcolor{currentfill}%
\pgfsetlinewidth{1.003750pt}%
\definecolor{currentstroke}{rgb}{0.121569,0.466667,0.705882}%
\pgfsetstrokecolor{currentstroke}%
\pgfsetdash{}{0pt}%
\pgfpathmoveto{\pgfqpoint{3.717272in}{3.221086in}}%
\pgfpathcurveto{\pgfqpoint{3.731000in}{3.221086in}}{\pgfqpoint{3.744168in}{3.226541in}}{\pgfqpoint{3.753875in}{3.236248in}}%
\pgfpathcurveto{\pgfqpoint{3.763582in}{3.245955in}}{\pgfqpoint{3.769036in}{3.259123in}}{\pgfqpoint{3.769036in}{3.272851in}}%
\pgfpathcurveto{\pgfqpoint{3.769036in}{3.286579in}}{\pgfqpoint{3.763582in}{3.299746in}}{\pgfqpoint{3.753875in}{3.309453in}}%
\pgfpathcurveto{\pgfqpoint{3.744168in}{3.319161in}}{\pgfqpoint{3.731000in}{3.324615in}}{\pgfqpoint{3.717272in}{3.324615in}}%
\pgfpathcurveto{\pgfqpoint{3.703544in}{3.324615in}}{\pgfqpoint{3.690377in}{3.319161in}}{\pgfqpoint{3.680669in}{3.309453in}}%
\pgfpathcurveto{\pgfqpoint{3.670962in}{3.299746in}}{\pgfqpoint{3.665508in}{3.286579in}}{\pgfqpoint{3.665508in}{3.272851in}}%
\pgfpathcurveto{\pgfqpoint{3.665508in}{3.259123in}}{\pgfqpoint{3.670962in}{3.245955in}}{\pgfqpoint{3.680669in}{3.236248in}}%
\pgfpathcurveto{\pgfqpoint{3.690377in}{3.226541in}}{\pgfqpoint{3.703544in}{3.221086in}}{\pgfqpoint{3.717272in}{3.221086in}}%
\pgfpathclose%
\pgfusepath{stroke,fill}%
\end{pgfscope}%
\begin{pgfscope}%
\pgfpathrectangle{\pgfqpoint{1.065196in}{0.528000in}}{\pgfqpoint{3.702804in}{3.696000in}} %
\pgfusepath{clip}%
\pgfsetbuttcap%
\pgfsetroundjoin%
\definecolor{currentfill}{rgb}{0.121569,0.466667,0.705882}%
\pgfsetfillcolor{currentfill}%
\pgfsetlinewidth{1.003750pt}%
\definecolor{currentstroke}{rgb}{0.121569,0.466667,0.705882}%
\pgfsetstrokecolor{currentstroke}%
\pgfsetdash{}{0pt}%
\pgfpathmoveto{\pgfqpoint{4.295603in}{3.758666in}}%
\pgfpathcurveto{\pgfqpoint{4.308635in}{3.758666in}}{\pgfqpoint{4.321135in}{3.763844in}}{\pgfqpoint{4.330351in}{3.773059in}}%
\pgfpathcurveto{\pgfqpoint{4.339566in}{3.782275in}}{\pgfqpoint{4.344744in}{3.794775in}}{\pgfqpoint{4.344744in}{3.807808in}}%
\pgfpathcurveto{\pgfqpoint{4.344744in}{3.820840in}}{\pgfqpoint{4.339566in}{3.833340in}}{\pgfqpoint{4.330351in}{3.842556in}}%
\pgfpathcurveto{\pgfqpoint{4.321135in}{3.851771in}}{\pgfqpoint{4.308635in}{3.856949in}}{\pgfqpoint{4.295603in}{3.856949in}}%
\pgfpathcurveto{\pgfqpoint{4.282570in}{3.856949in}}{\pgfqpoint{4.270070in}{3.851771in}}{\pgfqpoint{4.260855in}{3.842556in}}%
\pgfpathcurveto{\pgfqpoint{4.251639in}{3.833340in}}{\pgfqpoint{4.246461in}{3.820840in}}{\pgfqpoint{4.246461in}{3.807808in}}%
\pgfpathcurveto{\pgfqpoint{4.246461in}{3.794775in}}{\pgfqpoint{4.251639in}{3.782275in}}{\pgfqpoint{4.260855in}{3.773059in}}%
\pgfpathcurveto{\pgfqpoint{4.270070in}{3.763844in}}{\pgfqpoint{4.282570in}{3.758666in}}{\pgfqpoint{4.295603in}{3.758666in}}%
\pgfpathclose%
\pgfusepath{stroke,fill}%
\end{pgfscope}%
\begin{pgfscope}%
\pgfpathrectangle{\pgfqpoint{1.065196in}{0.528000in}}{\pgfqpoint{3.702804in}{3.696000in}} %
\pgfusepath{clip}%
\pgfsetbuttcap%
\pgfsetroundjoin%
\definecolor{currentfill}{rgb}{0.121569,0.466667,0.705882}%
\pgfsetfillcolor{currentfill}%
\pgfsetlinewidth{1.003750pt}%
\definecolor{currentstroke}{rgb}{0.121569,0.466667,0.705882}%
\pgfsetstrokecolor{currentstroke}%
\pgfsetdash{}{0pt}%
\pgfpathmoveto{\pgfqpoint{1.303132in}{1.421054in}}%
\pgfpathcurveto{\pgfqpoint{1.316738in}{1.421054in}}{\pgfqpoint{1.329788in}{1.426459in}}{\pgfqpoint{1.339408in}{1.436080in}}%
\pgfpathcurveto{\pgfqpoint{1.349029in}{1.445700in}}{\pgfqpoint{1.354434in}{1.458750in}}{\pgfqpoint{1.354434in}{1.472356in}}%
\pgfpathcurveto{\pgfqpoint{1.354434in}{1.485961in}}{\pgfqpoint{1.349029in}{1.499011in}}{\pgfqpoint{1.339408in}{1.508631in}}%
\pgfpathcurveto{\pgfqpoint{1.329788in}{1.518252in}}{\pgfqpoint{1.316738in}{1.523657in}}{\pgfqpoint{1.303132in}{1.523657in}}%
\pgfpathcurveto{\pgfqpoint{1.289527in}{1.523657in}}{\pgfqpoint{1.276477in}{1.518252in}}{\pgfqpoint{1.266857in}{1.508631in}}%
\pgfpathcurveto{\pgfqpoint{1.257236in}{1.499011in}}{\pgfqpoint{1.251831in}{1.485961in}}{\pgfqpoint{1.251831in}{1.472356in}}%
\pgfpathcurveto{\pgfqpoint{1.251831in}{1.458750in}}{\pgfqpoint{1.257236in}{1.445700in}}{\pgfqpoint{1.266857in}{1.436080in}}%
\pgfpathcurveto{\pgfqpoint{1.276477in}{1.426459in}}{\pgfqpoint{1.289527in}{1.421054in}}{\pgfqpoint{1.303132in}{1.421054in}}%
\pgfpathclose%
\pgfusepath{stroke,fill}%
\end{pgfscope}%
\begin{pgfscope}%
\pgfpathrectangle{\pgfqpoint{1.065196in}{0.528000in}}{\pgfqpoint{3.702804in}{3.696000in}} %
\pgfusepath{clip}%
\pgfsetbuttcap%
\pgfsetroundjoin%
\definecolor{currentfill}{rgb}{0.121569,0.466667,0.705882}%
\pgfsetfillcolor{currentfill}%
\pgfsetlinewidth{1.003750pt}%
\definecolor{currentstroke}{rgb}{0.121569,0.466667,0.705882}%
\pgfsetstrokecolor{currentstroke}%
\pgfsetdash{}{0pt}%
\pgfpathmoveto{\pgfqpoint{3.321270in}{3.839337in}}%
\pgfpathcurveto{\pgfqpoint{3.328041in}{3.839337in}}{\pgfqpoint{3.334535in}{3.842028in}}{\pgfqpoint{3.339323in}{3.846815in}}%
\pgfpathcurveto{\pgfqpoint{3.344111in}{3.851603in}}{\pgfqpoint{3.346801in}{3.858097in}}{\pgfqpoint{3.346801in}{3.864868in}}%
\pgfpathcurveto{\pgfqpoint{3.346801in}{3.871639in}}{\pgfqpoint{3.344111in}{3.878133in}}{\pgfqpoint{3.339323in}{3.882921in}}%
\pgfpathcurveto{\pgfqpoint{3.334535in}{3.887709in}}{\pgfqpoint{3.328041in}{3.890399in}}{\pgfqpoint{3.321270in}{3.890399in}}%
\pgfpathcurveto{\pgfqpoint{3.314499in}{3.890399in}}{\pgfqpoint{3.308005in}{3.887709in}}{\pgfqpoint{3.303217in}{3.882921in}}%
\pgfpathcurveto{\pgfqpoint{3.298430in}{3.878133in}}{\pgfqpoint{3.295739in}{3.871639in}}{\pgfqpoint{3.295739in}{3.864868in}}%
\pgfpathcurveto{\pgfqpoint{3.295739in}{3.858097in}}{\pgfqpoint{3.298430in}{3.851603in}}{\pgfqpoint{3.303217in}{3.846815in}}%
\pgfpathcurveto{\pgfqpoint{3.308005in}{3.842028in}}{\pgfqpoint{3.314499in}{3.839337in}}{\pgfqpoint{3.321270in}{3.839337in}}%
\pgfpathclose%
\pgfusepath{stroke,fill}%
\end{pgfscope}%
\begin{pgfscope}%
\pgfpathrectangle{\pgfqpoint{1.065196in}{0.528000in}}{\pgfqpoint{3.702804in}{3.696000in}} %
\pgfusepath{clip}%
\pgfsetbuttcap%
\pgfsetroundjoin%
\definecolor{currentfill}{rgb}{1.000000,0.498039,0.054902}%
\pgfsetfillcolor{currentfill}%
\pgfsetlinewidth{1.003750pt}%
\definecolor{currentstroke}{rgb}{1.000000,0.498039,0.054902}%
\pgfsetstrokecolor{currentstroke}%
\pgfsetdash{}{0pt}%
\pgfpathmoveto{\pgfqpoint{4.437416in}{2.534997in}}%
\pgfpathcurveto{\pgfqpoint{4.441744in}{2.534997in}}{\pgfqpoint{4.445896in}{2.536717in}}{\pgfqpoint{4.448957in}{2.539778in}}%
\pgfpathcurveto{\pgfqpoint{4.452017in}{2.542838in}}{\pgfqpoint{4.453737in}{2.546990in}}{\pgfqpoint{4.453737in}{2.551319in}}%
\pgfpathcurveto{\pgfqpoint{4.453737in}{2.555647in}}{\pgfqpoint{4.452017in}{2.559799in}}{\pgfqpoint{4.448957in}{2.562859in}}%
\pgfpathcurveto{\pgfqpoint{4.445896in}{2.565920in}}{\pgfqpoint{4.441744in}{2.567640in}}{\pgfqpoint{4.437416in}{2.567640in}}%
\pgfpathcurveto{\pgfqpoint{4.433088in}{2.567640in}}{\pgfqpoint{4.428936in}{2.565920in}}{\pgfqpoint{4.425875in}{2.562859in}}%
\pgfpathcurveto{\pgfqpoint{4.422815in}{2.559799in}}{\pgfqpoint{4.421095in}{2.555647in}}{\pgfqpoint{4.421095in}{2.551319in}}%
\pgfpathcurveto{\pgfqpoint{4.421095in}{2.546990in}}{\pgfqpoint{4.422815in}{2.542838in}}{\pgfqpoint{4.425875in}{2.539778in}}%
\pgfpathcurveto{\pgfqpoint{4.428936in}{2.536717in}}{\pgfqpoint{4.433088in}{2.534997in}}{\pgfqpoint{4.437416in}{2.534997in}}%
\pgfpathclose%
\pgfusepath{stroke,fill}%
\end{pgfscope}%
\begin{pgfscope}%
\pgfpathrectangle{\pgfqpoint{1.065196in}{0.528000in}}{\pgfqpoint{3.702804in}{3.696000in}} %
\pgfusepath{clip}%
\pgfsetbuttcap%
\pgfsetroundjoin%
\definecolor{currentfill}{rgb}{1.000000,0.498039,0.054902}%
\pgfsetfillcolor{currentfill}%
\pgfsetlinewidth{1.003750pt}%
\definecolor{currentstroke}{rgb}{1.000000,0.498039,0.054902}%
\pgfsetstrokecolor{currentstroke}%
\pgfsetdash{}{0pt}%
\pgfpathmoveto{\pgfqpoint{4.321684in}{2.809281in}}%
\pgfpathcurveto{\pgfqpoint{4.328661in}{2.809281in}}{\pgfqpoint{4.335353in}{2.812053in}}{\pgfqpoint{4.340287in}{2.816987in}}%
\pgfpathcurveto{\pgfqpoint{4.345220in}{2.821920in}}{\pgfqpoint{4.347992in}{2.828612in}}{\pgfqpoint{4.347992in}{2.835590in}}%
\pgfpathcurveto{\pgfqpoint{4.347992in}{2.842567in}}{\pgfqpoint{4.345220in}{2.849259in}}{\pgfqpoint{4.340287in}{2.854193in}}%
\pgfpathcurveto{\pgfqpoint{4.335353in}{2.859126in}}{\pgfqpoint{4.328661in}{2.861898in}}{\pgfqpoint{4.321684in}{2.861898in}}%
\pgfpathcurveto{\pgfqpoint{4.314706in}{2.861898in}}{\pgfqpoint{4.308014in}{2.859126in}}{\pgfqpoint{4.303081in}{2.854193in}}%
\pgfpathcurveto{\pgfqpoint{4.298147in}{2.849259in}}{\pgfqpoint{4.295375in}{2.842567in}}{\pgfqpoint{4.295375in}{2.835590in}}%
\pgfpathcurveto{\pgfqpoint{4.295375in}{2.828612in}}{\pgfqpoint{4.298147in}{2.821920in}}{\pgfqpoint{4.303081in}{2.816987in}}%
\pgfpathcurveto{\pgfqpoint{4.308014in}{2.812053in}}{\pgfqpoint{4.314706in}{2.809281in}}{\pgfqpoint{4.321684in}{2.809281in}}%
\pgfpathclose%
\pgfusepath{stroke,fill}%
\end{pgfscope}%
\begin{pgfscope}%
\pgfpathrectangle{\pgfqpoint{1.065196in}{0.528000in}}{\pgfqpoint{3.702804in}{3.696000in}} %
\pgfusepath{clip}%
\pgfsetbuttcap%
\pgfsetroundjoin%
\definecolor{currentfill}{rgb}{1.000000,0.498039,0.054902}%
\pgfsetfillcolor{currentfill}%
\pgfsetlinewidth{1.003750pt}%
\definecolor{currentstroke}{rgb}{1.000000,0.498039,0.054902}%
\pgfsetstrokecolor{currentstroke}%
\pgfsetdash{}{0pt}%
\pgfpathmoveto{\pgfqpoint{2.562090in}{2.283816in}}%
\pgfpathcurveto{\pgfqpoint{2.570295in}{2.283816in}}{\pgfqpoint{2.578165in}{2.287076in}}{\pgfqpoint{2.583967in}{2.292878in}}%
\pgfpathcurveto{\pgfqpoint{2.589769in}{2.298680in}}{\pgfqpoint{2.593029in}{2.306550in}}{\pgfqpoint{2.593029in}{2.314755in}}%
\pgfpathcurveto{\pgfqpoint{2.593029in}{2.322961in}}{\pgfqpoint{2.589769in}{2.330831in}}{\pgfqpoint{2.583967in}{2.336633in}}%
\pgfpathcurveto{\pgfqpoint{2.578165in}{2.342435in}}{\pgfqpoint{2.570295in}{2.345695in}}{\pgfqpoint{2.562090in}{2.345695in}}%
\pgfpathcurveto{\pgfqpoint{2.553884in}{2.345695in}}{\pgfqpoint{2.546014in}{2.342435in}}{\pgfqpoint{2.540212in}{2.336633in}}%
\pgfpathcurveto{\pgfqpoint{2.534410in}{2.330831in}}{\pgfqpoint{2.531150in}{2.322961in}}{\pgfqpoint{2.531150in}{2.314755in}}%
\pgfpathcurveto{\pgfqpoint{2.531150in}{2.306550in}}{\pgfqpoint{2.534410in}{2.298680in}}{\pgfqpoint{2.540212in}{2.292878in}}%
\pgfpathcurveto{\pgfqpoint{2.546014in}{2.287076in}}{\pgfqpoint{2.553884in}{2.283816in}}{\pgfqpoint{2.562090in}{2.283816in}}%
\pgfpathclose%
\pgfusepath{stroke,fill}%
\end{pgfscope}%
\begin{pgfscope}%
\pgfpathrectangle{\pgfqpoint{1.065196in}{0.528000in}}{\pgfqpoint{3.702804in}{3.696000in}} %
\pgfusepath{clip}%
\pgfsetbuttcap%
\pgfsetroundjoin%
\definecolor{currentfill}{rgb}{1.000000,0.498039,0.054902}%
\pgfsetfillcolor{currentfill}%
\pgfsetlinewidth{1.003750pt}%
\definecolor{currentstroke}{rgb}{1.000000,0.498039,0.054902}%
\pgfsetstrokecolor{currentstroke}%
\pgfsetdash{}{0pt}%
\pgfpathmoveto{\pgfqpoint{3.279530in}{2.504463in}}%
\pgfpathcurveto{\pgfqpoint{3.285648in}{2.504463in}}{\pgfqpoint{3.291516in}{2.506894in}}{\pgfqpoint{3.295842in}{2.511220in}}%
\pgfpathcurveto{\pgfqpoint{3.300168in}{2.515546in}}{\pgfqpoint{3.302599in}{2.521414in}}{\pgfqpoint{3.302599in}{2.527532in}}%
\pgfpathcurveto{\pgfqpoint{3.302599in}{2.533649in}}{\pgfqpoint{3.300168in}{2.539518in}}{\pgfqpoint{3.295842in}{2.543844in}}%
\pgfpathcurveto{\pgfqpoint{3.291516in}{2.548169in}}{\pgfqpoint{3.285648in}{2.550600in}}{\pgfqpoint{3.279530in}{2.550600in}}%
\pgfpathcurveto{\pgfqpoint{3.273412in}{2.550600in}}{\pgfqpoint{3.267544in}{2.548169in}}{\pgfqpoint{3.263218in}{2.543844in}}%
\pgfpathcurveto{\pgfqpoint{3.258892in}{2.539518in}}{\pgfqpoint{3.256462in}{2.533649in}}{\pgfqpoint{3.256462in}{2.527532in}}%
\pgfpathcurveto{\pgfqpoint{3.256462in}{2.521414in}}{\pgfqpoint{3.258892in}{2.515546in}}{\pgfqpoint{3.263218in}{2.511220in}}%
\pgfpathcurveto{\pgfqpoint{3.267544in}{2.506894in}}{\pgfqpoint{3.273412in}{2.504463in}}{\pgfqpoint{3.279530in}{2.504463in}}%
\pgfpathclose%
\pgfusepath{stroke,fill}%
\end{pgfscope}%
\begin{pgfscope}%
\pgfpathrectangle{\pgfqpoint{1.065196in}{0.528000in}}{\pgfqpoint{3.702804in}{3.696000in}} %
\pgfusepath{clip}%
\pgfsetbuttcap%
\pgfsetroundjoin%
\definecolor{currentfill}{rgb}{1.000000,0.498039,0.054902}%
\pgfsetfillcolor{currentfill}%
\pgfsetlinewidth{1.003750pt}%
\definecolor{currentstroke}{rgb}{1.000000,0.498039,0.054902}%
\pgfsetstrokecolor{currentstroke}%
\pgfsetdash{}{0pt}%
\pgfpathmoveto{\pgfqpoint{4.357087in}{3.720492in}}%
\pgfpathcurveto{\pgfqpoint{4.368489in}{3.720492in}}{\pgfqpoint{4.379426in}{3.725023in}}{\pgfqpoint{4.387489in}{3.733085in}}%
\pgfpathcurveto{\pgfqpoint{4.395552in}{3.741148in}}{\pgfqpoint{4.400082in}{3.752085in}}{\pgfqpoint{4.400082in}{3.763487in}}%
\pgfpathcurveto{\pgfqpoint{4.400082in}{3.774890in}}{\pgfqpoint{4.395552in}{3.785827in}}{\pgfqpoint{4.387489in}{3.793889in}}%
\pgfpathcurveto{\pgfqpoint{4.379426in}{3.801952in}}{\pgfqpoint{4.368489in}{3.806482in}}{\pgfqpoint{4.357087in}{3.806482in}}%
\pgfpathcurveto{\pgfqpoint{4.345684in}{3.806482in}}{\pgfqpoint{4.334747in}{3.801952in}}{\pgfqpoint{4.326685in}{3.793889in}}%
\pgfpathcurveto{\pgfqpoint{4.318622in}{3.785827in}}{\pgfqpoint{4.314092in}{3.774890in}}{\pgfqpoint{4.314092in}{3.763487in}}%
\pgfpathcurveto{\pgfqpoint{4.314092in}{3.752085in}}{\pgfqpoint{4.318622in}{3.741148in}}{\pgfqpoint{4.326685in}{3.733085in}}%
\pgfpathcurveto{\pgfqpoint{4.334747in}{3.725023in}}{\pgfqpoint{4.345684in}{3.720492in}}{\pgfqpoint{4.357087in}{3.720492in}}%
\pgfpathclose%
\pgfusepath{stroke,fill}%
\end{pgfscope}%
\begin{pgfscope}%
\pgfpathrectangle{\pgfqpoint{1.065196in}{0.528000in}}{\pgfqpoint{3.702804in}{3.696000in}} %
\pgfusepath{clip}%
\pgfsetbuttcap%
\pgfsetroundjoin%
\definecolor{currentfill}{rgb}{1.000000,0.498039,0.054902}%
\pgfsetfillcolor{currentfill}%
\pgfsetlinewidth{1.003750pt}%
\definecolor{currentstroke}{rgb}{1.000000,0.498039,0.054902}%
\pgfsetstrokecolor{currentstroke}%
\pgfsetdash{}{0pt}%
\pgfpathmoveto{\pgfqpoint{2.578386in}{3.877074in}}%
\pgfpathcurveto{\pgfqpoint{2.587798in}{3.877074in}}{\pgfqpoint{2.596825in}{3.880813in}}{\pgfqpoint{2.603480in}{3.887468in}}%
\pgfpathcurveto{\pgfqpoint{2.610134in}{3.894123in}}{\pgfqpoint{2.613874in}{3.903150in}}{\pgfqpoint{2.613874in}{3.912561in}}%
\pgfpathcurveto{\pgfqpoint{2.613874in}{3.921973in}}{\pgfqpoint{2.610134in}{3.931000in}}{\pgfqpoint{2.603480in}{3.937655in}}%
\pgfpathcurveto{\pgfqpoint{2.596825in}{3.944310in}}{\pgfqpoint{2.587798in}{3.948049in}}{\pgfqpoint{2.578386in}{3.948049in}}%
\pgfpathcurveto{\pgfqpoint{2.568975in}{3.948049in}}{\pgfqpoint{2.559948in}{3.944310in}}{\pgfqpoint{2.553293in}{3.937655in}}%
\pgfpathcurveto{\pgfqpoint{2.546638in}{3.931000in}}{\pgfqpoint{2.542899in}{3.921973in}}{\pgfqpoint{2.542899in}{3.912561in}}%
\pgfpathcurveto{\pgfqpoint{2.542899in}{3.903150in}}{\pgfqpoint{2.546638in}{3.894123in}}{\pgfqpoint{2.553293in}{3.887468in}}%
\pgfpathcurveto{\pgfqpoint{2.559948in}{3.880813in}}{\pgfqpoint{2.568975in}{3.877074in}}{\pgfqpoint{2.578386in}{3.877074in}}%
\pgfpathclose%
\pgfusepath{stroke,fill}%
\end{pgfscope}%
\begin{pgfscope}%
\pgfpathrectangle{\pgfqpoint{1.065196in}{0.528000in}}{\pgfqpoint{3.702804in}{3.696000in}} %
\pgfusepath{clip}%
\pgfsetbuttcap%
\pgfsetroundjoin%
\definecolor{currentfill}{rgb}{1.000000,0.498039,0.054902}%
\pgfsetfillcolor{currentfill}%
\pgfsetlinewidth{1.003750pt}%
\definecolor{currentstroke}{rgb}{1.000000,0.498039,0.054902}%
\pgfsetstrokecolor{currentstroke}%
\pgfsetdash{}{0pt}%
\pgfpathmoveto{\pgfqpoint{1.838793in}{1.060828in}}%
\pgfpathcurveto{\pgfqpoint{1.852015in}{1.060828in}}{\pgfqpoint{1.864697in}{1.066081in}}{\pgfqpoint{1.874047in}{1.075431in}}%
\pgfpathcurveto{\pgfqpoint{1.883397in}{1.084780in}}{\pgfqpoint{1.888650in}{1.097463in}}{\pgfqpoint{1.888650in}{1.110685in}}%
\pgfpathcurveto{\pgfqpoint{1.888650in}{1.123907in}}{\pgfqpoint{1.883397in}{1.136590in}}{\pgfqpoint{1.874047in}{1.145940in}}%
\pgfpathcurveto{\pgfqpoint{1.864697in}{1.155289in}}{\pgfqpoint{1.852015in}{1.160542in}}{\pgfqpoint{1.838793in}{1.160542in}}%
\pgfpathcurveto{\pgfqpoint{1.825570in}{1.160542in}}{\pgfqpoint{1.812888in}{1.155289in}}{\pgfqpoint{1.803538in}{1.145940in}}%
\pgfpathcurveto{\pgfqpoint{1.794189in}{1.136590in}}{\pgfqpoint{1.788935in}{1.123907in}}{\pgfqpoint{1.788935in}{1.110685in}}%
\pgfpathcurveto{\pgfqpoint{1.788935in}{1.097463in}}{\pgfqpoint{1.794189in}{1.084780in}}{\pgfqpoint{1.803538in}{1.075431in}}%
\pgfpathcurveto{\pgfqpoint{1.812888in}{1.066081in}}{\pgfqpoint{1.825570in}{1.060828in}}{\pgfqpoint{1.838793in}{1.060828in}}%
\pgfpathclose%
\pgfusepath{stroke,fill}%
\end{pgfscope}%
\begin{pgfscope}%
\pgfpathrectangle{\pgfqpoint{1.065196in}{0.528000in}}{\pgfqpoint{3.702804in}{3.696000in}} %
\pgfusepath{clip}%
\pgfsetbuttcap%
\pgfsetroundjoin%
\definecolor{currentfill}{rgb}{1.000000,0.498039,0.054902}%
\pgfsetfillcolor{currentfill}%
\pgfsetlinewidth{1.003750pt}%
\definecolor{currentstroke}{rgb}{1.000000,0.498039,0.054902}%
\pgfsetstrokecolor{currentstroke}%
\pgfsetdash{}{0pt}%
\pgfpathmoveto{\pgfqpoint{1.708642in}{2.356869in}}%
\pgfpathcurveto{\pgfqpoint{1.714939in}{2.356869in}}{\pgfqpoint{1.720978in}{2.359370in}}{\pgfqpoint{1.725431in}{2.363823in}}%
\pgfpathcurveto{\pgfqpoint{1.729883in}{2.368275in}}{\pgfqpoint{1.732385in}{2.374315in}}{\pgfqpoint{1.732385in}{2.380611in}}%
\pgfpathcurveto{\pgfqpoint{1.732385in}{2.386908in}}{\pgfqpoint{1.729883in}{2.392948in}}{\pgfqpoint{1.725431in}{2.397400in}}%
\pgfpathcurveto{\pgfqpoint{1.720978in}{2.401852in}}{\pgfqpoint{1.714939in}{2.404354in}}{\pgfqpoint{1.708642in}{2.404354in}}%
\pgfpathcurveto{\pgfqpoint{1.702346in}{2.404354in}}{\pgfqpoint{1.696306in}{2.401852in}}{\pgfqpoint{1.691854in}{2.397400in}}%
\pgfpathcurveto{\pgfqpoint{1.687401in}{2.392948in}}{\pgfqpoint{1.684900in}{2.386908in}}{\pgfqpoint{1.684900in}{2.380611in}}%
\pgfpathcurveto{\pgfqpoint{1.684900in}{2.374315in}}{\pgfqpoint{1.687401in}{2.368275in}}{\pgfqpoint{1.691854in}{2.363823in}}%
\pgfpathcurveto{\pgfqpoint{1.696306in}{2.359370in}}{\pgfqpoint{1.702346in}{2.356869in}}{\pgfqpoint{1.708642in}{2.356869in}}%
\pgfpathclose%
\pgfusepath{stroke,fill}%
\end{pgfscope}%
\begin{pgfscope}%
\pgfpathrectangle{\pgfqpoint{1.065196in}{0.528000in}}{\pgfqpoint{3.702804in}{3.696000in}} %
\pgfusepath{clip}%
\pgfsetbuttcap%
\pgfsetroundjoin%
\definecolor{currentfill}{rgb}{1.000000,0.498039,0.054902}%
\pgfsetfillcolor{currentfill}%
\pgfsetlinewidth{1.003750pt}%
\definecolor{currentstroke}{rgb}{1.000000,0.498039,0.054902}%
\pgfsetstrokecolor{currentstroke}%
\pgfsetdash{}{0pt}%
\pgfpathmoveto{\pgfqpoint{1.328486in}{3.815136in}}%
\pgfpathcurveto{\pgfqpoint{1.340746in}{3.815136in}}{\pgfqpoint{1.352506in}{3.820007in}}{\pgfqpoint{1.361175in}{3.828677in}}%
\pgfpathcurveto{\pgfqpoint{1.369845in}{3.837346in}}{\pgfqpoint{1.374716in}{3.849106in}}{\pgfqpoint{1.374716in}{3.861366in}}%
\pgfpathcurveto{\pgfqpoint{1.374716in}{3.873626in}}{\pgfqpoint{1.369845in}{3.885386in}}{\pgfqpoint{1.361175in}{3.894055in}}%
\pgfpathcurveto{\pgfqpoint{1.352506in}{3.902725in}}{\pgfqpoint{1.340746in}{3.907596in}}{\pgfqpoint{1.328486in}{3.907596in}}%
\pgfpathcurveto{\pgfqpoint{1.316226in}{3.907596in}}{\pgfqpoint{1.304466in}{3.902725in}}{\pgfqpoint{1.295797in}{3.894055in}}%
\pgfpathcurveto{\pgfqpoint{1.287127in}{3.885386in}}{\pgfqpoint{1.282256in}{3.873626in}}{\pgfqpoint{1.282256in}{3.861366in}}%
\pgfpathcurveto{\pgfqpoint{1.282256in}{3.849106in}}{\pgfqpoint{1.287127in}{3.837346in}}{\pgfqpoint{1.295797in}{3.828677in}}%
\pgfpathcurveto{\pgfqpoint{1.304466in}{3.820007in}}{\pgfqpoint{1.316226in}{3.815136in}}{\pgfqpoint{1.328486in}{3.815136in}}%
\pgfpathclose%
\pgfusepath{stroke,fill}%
\end{pgfscope}%
\begin{pgfscope}%
\pgfpathrectangle{\pgfqpoint{1.065196in}{0.528000in}}{\pgfqpoint{3.702804in}{3.696000in}} %
\pgfusepath{clip}%
\pgfsetbuttcap%
\pgfsetroundjoin%
\definecolor{currentfill}{rgb}{1.000000,0.498039,0.054902}%
\pgfsetfillcolor{currentfill}%
\pgfsetlinewidth{1.003750pt}%
\definecolor{currentstroke}{rgb}{1.000000,0.498039,0.054902}%
\pgfsetstrokecolor{currentstroke}%
\pgfsetdash{}{0pt}%
\pgfpathmoveto{\pgfqpoint{4.025435in}{0.710988in}}%
\pgfpathcurveto{\pgfqpoint{4.032609in}{0.710988in}}{\pgfqpoint{4.039491in}{0.713838in}}{\pgfqpoint{4.044564in}{0.718911in}}%
\pgfpathcurveto{\pgfqpoint{4.049637in}{0.723985in}}{\pgfqpoint{4.052488in}{0.730866in}}{\pgfqpoint{4.052488in}{0.738041in}}%
\pgfpathcurveto{\pgfqpoint{4.052488in}{0.745215in}}{\pgfqpoint{4.049637in}{0.752097in}}{\pgfqpoint{4.044564in}{0.757170in}}%
\pgfpathcurveto{\pgfqpoint{4.039491in}{0.762243in}}{\pgfqpoint{4.032609in}{0.765094in}}{\pgfqpoint{4.025435in}{0.765094in}}%
\pgfpathcurveto{\pgfqpoint{4.018260in}{0.765094in}}{\pgfqpoint{4.011378in}{0.762243in}}{\pgfqpoint{4.006305in}{0.757170in}}%
\pgfpathcurveto{\pgfqpoint{4.001232in}{0.752097in}}{\pgfqpoint{3.998382in}{0.745215in}}{\pgfqpoint{3.998382in}{0.738041in}}%
\pgfpathcurveto{\pgfqpoint{3.998382in}{0.730866in}}{\pgfqpoint{4.001232in}{0.723985in}}{\pgfqpoint{4.006305in}{0.718911in}}%
\pgfpathcurveto{\pgfqpoint{4.011378in}{0.713838in}}{\pgfqpoint{4.018260in}{0.710988in}}{\pgfqpoint{4.025435in}{0.710988in}}%
\pgfpathclose%
\pgfusepath{stroke,fill}%
\end{pgfscope}%
\begin{pgfscope}%
\pgfpathrectangle{\pgfqpoint{1.065196in}{0.528000in}}{\pgfqpoint{3.702804in}{3.696000in}} %
\pgfusepath{clip}%
\pgfsetbuttcap%
\pgfsetroundjoin%
\definecolor{currentfill}{rgb}{1.000000,0.498039,0.054902}%
\pgfsetfillcolor{currentfill}%
\pgfsetlinewidth{1.003750pt}%
\definecolor{currentstroke}{rgb}{1.000000,0.498039,0.054902}%
\pgfsetstrokecolor{currentstroke}%
\pgfsetdash{}{0pt}%
\pgfpathmoveto{\pgfqpoint{1.846294in}{1.757625in}}%
\pgfpathcurveto{\pgfqpoint{1.857383in}{1.757625in}}{\pgfqpoint{1.868020in}{1.762031in}}{\pgfqpoint{1.875861in}{1.769872in}}%
\pgfpathcurveto{\pgfqpoint{1.883702in}{1.777714in}}{\pgfqpoint{1.888108in}{1.788350in}}{\pgfqpoint{1.888108in}{1.799440in}}%
\pgfpathcurveto{\pgfqpoint{1.888108in}{1.810529in}}{\pgfqpoint{1.883702in}{1.821166in}}{\pgfqpoint{1.875861in}{1.829007in}}%
\pgfpathcurveto{\pgfqpoint{1.868020in}{1.836848in}}{\pgfqpoint{1.857383in}{1.841254in}}{\pgfqpoint{1.846294in}{1.841254in}}%
\pgfpathcurveto{\pgfqpoint{1.835204in}{1.841254in}}{\pgfqpoint{1.824568in}{1.836848in}}{\pgfqpoint{1.816726in}{1.829007in}}%
\pgfpathcurveto{\pgfqpoint{1.808885in}{1.821166in}}{\pgfqpoint{1.804479in}{1.810529in}}{\pgfqpoint{1.804479in}{1.799440in}}%
\pgfpathcurveto{\pgfqpoint{1.804479in}{1.788350in}}{\pgfqpoint{1.808885in}{1.777714in}}{\pgfqpoint{1.816726in}{1.769872in}}%
\pgfpathcurveto{\pgfqpoint{1.824568in}{1.762031in}}{\pgfqpoint{1.835204in}{1.757625in}}{\pgfqpoint{1.846294in}{1.757625in}}%
\pgfpathclose%
\pgfusepath{stroke,fill}%
\end{pgfscope}%
\begin{pgfscope}%
\pgfpathrectangle{\pgfqpoint{1.065196in}{0.528000in}}{\pgfqpoint{3.702804in}{3.696000in}} %
\pgfusepath{clip}%
\pgfsetbuttcap%
\pgfsetroundjoin%
\definecolor{currentfill}{rgb}{1.000000,0.498039,0.054902}%
\pgfsetfillcolor{currentfill}%
\pgfsetlinewidth{1.003750pt}%
\definecolor{currentstroke}{rgb}{1.000000,0.498039,0.054902}%
\pgfsetstrokecolor{currentstroke}%
\pgfsetdash{}{0pt}%
\pgfpathmoveto{\pgfqpoint{1.694975in}{3.365167in}}%
\pgfpathcurveto{\pgfqpoint{1.703621in}{3.365167in}}{\pgfqpoint{1.711913in}{3.368602in}}{\pgfqpoint{1.718026in}{3.374715in}}%
\pgfpathcurveto{\pgfqpoint{1.724139in}{3.380828in}}{\pgfqpoint{1.727574in}{3.389120in}}{\pgfqpoint{1.727574in}{3.397766in}}%
\pgfpathcurveto{\pgfqpoint{1.727574in}{3.406411in}}{\pgfqpoint{1.724139in}{3.414703in}}{\pgfqpoint{1.718026in}{3.420816in}}%
\pgfpathcurveto{\pgfqpoint{1.711913in}{3.426929in}}{\pgfqpoint{1.703621in}{3.430364in}}{\pgfqpoint{1.694975in}{3.430364in}}%
\pgfpathcurveto{\pgfqpoint{1.686330in}{3.430364in}}{\pgfqpoint{1.678038in}{3.426929in}}{\pgfqpoint{1.671925in}{3.420816in}}%
\pgfpathcurveto{\pgfqpoint{1.665812in}{3.414703in}}{\pgfqpoint{1.662377in}{3.406411in}}{\pgfqpoint{1.662377in}{3.397766in}}%
\pgfpathcurveto{\pgfqpoint{1.662377in}{3.389120in}}{\pgfqpoint{1.665812in}{3.380828in}}{\pgfqpoint{1.671925in}{3.374715in}}%
\pgfpathcurveto{\pgfqpoint{1.678038in}{3.368602in}}{\pgfqpoint{1.686330in}{3.365167in}}{\pgfqpoint{1.694975in}{3.365167in}}%
\pgfpathclose%
\pgfusepath{stroke,fill}%
\end{pgfscope}%
\begin{pgfscope}%
\pgfpathrectangle{\pgfqpoint{1.065196in}{0.528000in}}{\pgfqpoint{3.702804in}{3.696000in}} %
\pgfusepath{clip}%
\pgfsetbuttcap%
\pgfsetroundjoin%
\definecolor{currentfill}{rgb}{1.000000,0.498039,0.054902}%
\pgfsetfillcolor{currentfill}%
\pgfsetlinewidth{1.003750pt}%
\definecolor{currentstroke}{rgb}{1.000000,0.498039,0.054902}%
\pgfsetstrokecolor{currentstroke}%
\pgfsetdash{}{0pt}%
\pgfpathmoveto{\pgfqpoint{2.410532in}{3.821873in}}%
\pgfpathcurveto{\pgfqpoint{2.414024in}{3.821873in}}{\pgfqpoint{2.417374in}{3.823261in}}{\pgfqpoint{2.419844in}{3.825730in}}%
\pgfpathcurveto{\pgfqpoint{2.422314in}{3.828200in}}{\pgfqpoint{2.423702in}{3.831550in}}{\pgfqpoint{2.423702in}{3.835043in}}%
\pgfpathcurveto{\pgfqpoint{2.423702in}{3.838536in}}{\pgfqpoint{2.422314in}{3.841886in}}{\pgfqpoint{2.419844in}{3.844356in}}%
\pgfpathcurveto{\pgfqpoint{2.417374in}{3.846826in}}{\pgfqpoint{2.414024in}{3.848213in}}{\pgfqpoint{2.410532in}{3.848213in}}%
\pgfpathcurveto{\pgfqpoint{2.407039in}{3.848213in}}{\pgfqpoint{2.403689in}{3.846826in}}{\pgfqpoint{2.401219in}{3.844356in}}%
\pgfpathcurveto{\pgfqpoint{2.398749in}{3.841886in}}{\pgfqpoint{2.397361in}{3.838536in}}{\pgfqpoint{2.397361in}{3.835043in}}%
\pgfpathcurveto{\pgfqpoint{2.397361in}{3.831550in}}{\pgfqpoint{2.398749in}{3.828200in}}{\pgfqpoint{2.401219in}{3.825730in}}%
\pgfpathcurveto{\pgfqpoint{2.403689in}{3.823261in}}{\pgfqpoint{2.407039in}{3.821873in}}{\pgfqpoint{2.410532in}{3.821873in}}%
\pgfpathclose%
\pgfusepath{stroke,fill}%
\end{pgfscope}%
\begin{pgfscope}%
\pgfpathrectangle{\pgfqpoint{1.065196in}{0.528000in}}{\pgfqpoint{3.702804in}{3.696000in}} %
\pgfusepath{clip}%
\pgfsetbuttcap%
\pgfsetroundjoin%
\definecolor{currentfill}{rgb}{1.000000,0.498039,0.054902}%
\pgfsetfillcolor{currentfill}%
\pgfsetlinewidth{1.003750pt}%
\definecolor{currentstroke}{rgb}{1.000000,0.498039,0.054902}%
\pgfsetstrokecolor{currentstroke}%
\pgfsetdash{}{0pt}%
\pgfpathmoveto{\pgfqpoint{3.204887in}{3.603510in}}%
\pgfpathcurveto{\pgfqpoint{3.211887in}{3.603510in}}{\pgfqpoint{3.218602in}{3.606291in}}{\pgfqpoint{3.223551in}{3.611241in}}%
\pgfpathcurveto{\pgfqpoint{3.228501in}{3.616191in}}{\pgfqpoint{3.231283in}{3.622906in}}{\pgfqpoint{3.231283in}{3.629906in}}%
\pgfpathcurveto{\pgfqpoint{3.231283in}{3.636906in}}{\pgfqpoint{3.228501in}{3.643620in}}{\pgfqpoint{3.223551in}{3.648570in}}%
\pgfpathcurveto{\pgfqpoint{3.218602in}{3.653520in}}{\pgfqpoint{3.211887in}{3.656301in}}{\pgfqpoint{3.204887in}{3.656301in}}%
\pgfpathcurveto{\pgfqpoint{3.197887in}{3.656301in}}{\pgfqpoint{3.191172in}{3.653520in}}{\pgfqpoint{3.186222in}{3.648570in}}%
\pgfpathcurveto{\pgfqpoint{3.181273in}{3.643620in}}{\pgfqpoint{3.178491in}{3.636906in}}{\pgfqpoint{3.178491in}{3.629906in}}%
\pgfpathcurveto{\pgfqpoint{3.178491in}{3.622906in}}{\pgfqpoint{3.181273in}{3.616191in}}{\pgfqpoint{3.186222in}{3.611241in}}%
\pgfpathcurveto{\pgfqpoint{3.191172in}{3.606291in}}{\pgfqpoint{3.197887in}{3.603510in}}{\pgfqpoint{3.204887in}{3.603510in}}%
\pgfpathclose%
\pgfusepath{stroke,fill}%
\end{pgfscope}%
\begin{pgfscope}%
\pgfpathrectangle{\pgfqpoint{1.065196in}{0.528000in}}{\pgfqpoint{3.702804in}{3.696000in}} %
\pgfusepath{clip}%
\pgfsetbuttcap%
\pgfsetroundjoin%
\definecolor{currentfill}{rgb}{1.000000,0.498039,0.054902}%
\pgfsetfillcolor{currentfill}%
\pgfsetlinewidth{1.003750pt}%
\definecolor{currentstroke}{rgb}{1.000000,0.498039,0.054902}%
\pgfsetstrokecolor{currentstroke}%
\pgfsetdash{}{0pt}%
\pgfpathmoveto{\pgfqpoint{4.084419in}{3.673856in}}%
\pgfpathcurveto{\pgfqpoint{4.096345in}{3.673856in}}{\pgfqpoint{4.107784in}{3.678594in}}{\pgfqpoint{4.116217in}{3.687027in}}%
\pgfpathcurveto{\pgfqpoint{4.124650in}{3.695459in}}{\pgfqpoint{4.129388in}{3.706898in}}{\pgfqpoint{4.129388in}{3.718824in}}%
\pgfpathcurveto{\pgfqpoint{4.129388in}{3.730750in}}{\pgfqpoint{4.124650in}{3.742189in}}{\pgfqpoint{4.116217in}{3.750622in}}%
\pgfpathcurveto{\pgfqpoint{4.107784in}{3.759054in}}{\pgfqpoint{4.096345in}{3.763793in}}{\pgfqpoint{4.084419in}{3.763793in}}%
\pgfpathcurveto{\pgfqpoint{4.072494in}{3.763793in}}{\pgfqpoint{4.061055in}{3.759054in}}{\pgfqpoint{4.052622in}{3.750622in}}%
\pgfpathcurveto{\pgfqpoint{4.044189in}{3.742189in}}{\pgfqpoint{4.039451in}{3.730750in}}{\pgfqpoint{4.039451in}{3.718824in}}%
\pgfpathcurveto{\pgfqpoint{4.039451in}{3.706898in}}{\pgfqpoint{4.044189in}{3.695459in}}{\pgfqpoint{4.052622in}{3.687027in}}%
\pgfpathcurveto{\pgfqpoint{4.061055in}{3.678594in}}{\pgfqpoint{4.072494in}{3.673856in}}{\pgfqpoint{4.084419in}{3.673856in}}%
\pgfpathclose%
\pgfusepath{stroke,fill}%
\end{pgfscope}%
\begin{pgfscope}%
\pgfpathrectangle{\pgfqpoint{1.065196in}{0.528000in}}{\pgfqpoint{3.702804in}{3.696000in}} %
\pgfusepath{clip}%
\pgfsetbuttcap%
\pgfsetroundjoin%
\definecolor{currentfill}{rgb}{1.000000,0.498039,0.054902}%
\pgfsetfillcolor{currentfill}%
\pgfsetlinewidth{1.003750pt}%
\definecolor{currentstroke}{rgb}{1.000000,0.498039,0.054902}%
\pgfsetstrokecolor{currentstroke}%
\pgfsetdash{}{0pt}%
\pgfpathmoveto{\pgfqpoint{2.796008in}{2.477904in}}%
\pgfpathcurveto{\pgfqpoint{2.806328in}{2.477904in}}{\pgfqpoint{2.816226in}{2.482004in}}{\pgfqpoint{2.823523in}{2.489300in}}%
\pgfpathcurveto{\pgfqpoint{2.830820in}{2.496597in}}{\pgfqpoint{2.834920in}{2.506496in}}{\pgfqpoint{2.834920in}{2.516815in}}%
\pgfpathcurveto{\pgfqpoint{2.834920in}{2.527134in}}{\pgfqpoint{2.830820in}{2.537033in}}{\pgfqpoint{2.823523in}{2.544330in}}%
\pgfpathcurveto{\pgfqpoint{2.816226in}{2.551627in}}{\pgfqpoint{2.806328in}{2.555726in}}{\pgfqpoint{2.796008in}{2.555726in}}%
\pgfpathcurveto{\pgfqpoint{2.785689in}{2.555726in}}{\pgfqpoint{2.775791in}{2.551627in}}{\pgfqpoint{2.768494in}{2.544330in}}%
\pgfpathcurveto{\pgfqpoint{2.761197in}{2.537033in}}{\pgfqpoint{2.757097in}{2.527134in}}{\pgfqpoint{2.757097in}{2.516815in}}%
\pgfpathcurveto{\pgfqpoint{2.757097in}{2.506496in}}{\pgfqpoint{2.761197in}{2.496597in}}{\pgfqpoint{2.768494in}{2.489300in}}%
\pgfpathcurveto{\pgfqpoint{2.775791in}{2.482004in}}{\pgfqpoint{2.785689in}{2.477904in}}{\pgfqpoint{2.796008in}{2.477904in}}%
\pgfpathclose%
\pgfusepath{stroke,fill}%
\end{pgfscope}%
\begin{pgfscope}%
\pgfpathrectangle{\pgfqpoint{1.065196in}{0.528000in}}{\pgfqpoint{3.702804in}{3.696000in}} %
\pgfusepath{clip}%
\pgfsetbuttcap%
\pgfsetroundjoin%
\definecolor{currentfill}{rgb}{1.000000,0.498039,0.054902}%
\pgfsetfillcolor{currentfill}%
\pgfsetlinewidth{1.003750pt}%
\definecolor{currentstroke}{rgb}{1.000000,0.498039,0.054902}%
\pgfsetstrokecolor{currentstroke}%
\pgfsetdash{}{0pt}%
\pgfpathmoveto{\pgfqpoint{3.929983in}{1.604533in}}%
\pgfpathcurveto{\pgfqpoint{3.940526in}{1.604533in}}{\pgfqpoint{3.950638in}{1.608722in}}{\pgfqpoint{3.958093in}{1.616177in}}%
\pgfpathcurveto{\pgfqpoint{3.965548in}{1.623632in}}{\pgfqpoint{3.969737in}{1.633744in}}{\pgfqpoint{3.969737in}{1.644287in}}%
\pgfpathcurveto{\pgfqpoint{3.969737in}{1.654830in}}{\pgfqpoint{3.965548in}{1.664943in}}{\pgfqpoint{3.958093in}{1.672398in}}%
\pgfpathcurveto{\pgfqpoint{3.950638in}{1.679852in}}{\pgfqpoint{3.940526in}{1.684041in}}{\pgfqpoint{3.929983in}{1.684041in}}%
\pgfpathcurveto{\pgfqpoint{3.919440in}{1.684041in}}{\pgfqpoint{3.909328in}{1.679852in}}{\pgfqpoint{3.901873in}{1.672398in}}%
\pgfpathcurveto{\pgfqpoint{3.894418in}{1.664943in}}{\pgfqpoint{3.890229in}{1.654830in}}{\pgfqpoint{3.890229in}{1.644287in}}%
\pgfpathcurveto{\pgfqpoint{3.890229in}{1.633744in}}{\pgfqpoint{3.894418in}{1.623632in}}{\pgfqpoint{3.901873in}{1.616177in}}%
\pgfpathcurveto{\pgfqpoint{3.909328in}{1.608722in}}{\pgfqpoint{3.919440in}{1.604533in}}{\pgfqpoint{3.929983in}{1.604533in}}%
\pgfpathclose%
\pgfusepath{stroke,fill}%
\end{pgfscope}%
\begin{pgfscope}%
\pgfpathrectangle{\pgfqpoint{1.065196in}{0.528000in}}{\pgfqpoint{3.702804in}{3.696000in}} %
\pgfusepath{clip}%
\pgfsetbuttcap%
\pgfsetroundjoin%
\definecolor{currentfill}{rgb}{1.000000,0.498039,0.054902}%
\pgfsetfillcolor{currentfill}%
\pgfsetlinewidth{1.003750pt}%
\definecolor{currentstroke}{rgb}{1.000000,0.498039,0.054902}%
\pgfsetstrokecolor{currentstroke}%
\pgfsetdash{}{0pt}%
\pgfpathmoveto{\pgfqpoint{2.897691in}{2.670391in}}%
\pgfpathcurveto{\pgfqpoint{2.903809in}{2.670391in}}{\pgfqpoint{2.909676in}{2.672821in}}{\pgfqpoint{2.914001in}{2.677146in}}%
\pgfpathcurveto{\pgfqpoint{2.918327in}{2.681472in}}{\pgfqpoint{2.920757in}{2.687339in}}{\pgfqpoint{2.920757in}{2.693456in}}%
\pgfpathcurveto{\pgfqpoint{2.920757in}{2.699573in}}{\pgfqpoint{2.918327in}{2.705440in}}{\pgfqpoint{2.914001in}{2.709766in}}%
\pgfpathcurveto{\pgfqpoint{2.909676in}{2.714091in}}{\pgfqpoint{2.903809in}{2.716522in}}{\pgfqpoint{2.897691in}{2.716522in}}%
\pgfpathcurveto{\pgfqpoint{2.891574in}{2.716522in}}{\pgfqpoint{2.885707in}{2.714091in}}{\pgfqpoint{2.881382in}{2.709766in}}%
\pgfpathcurveto{\pgfqpoint{2.877056in}{2.705440in}}{\pgfqpoint{2.874626in}{2.699573in}}{\pgfqpoint{2.874626in}{2.693456in}}%
\pgfpathcurveto{\pgfqpoint{2.874626in}{2.687339in}}{\pgfqpoint{2.877056in}{2.681472in}}{\pgfqpoint{2.881382in}{2.677146in}}%
\pgfpathcurveto{\pgfqpoint{2.885707in}{2.672821in}}{\pgfqpoint{2.891574in}{2.670391in}}{\pgfqpoint{2.897691in}{2.670391in}}%
\pgfpathclose%
\pgfusepath{stroke,fill}%
\end{pgfscope}%
\begin{pgfscope}%
\pgfpathrectangle{\pgfqpoint{1.065196in}{0.528000in}}{\pgfqpoint{3.702804in}{3.696000in}} %
\pgfusepath{clip}%
\pgfsetbuttcap%
\pgfsetroundjoin%
\definecolor{currentfill}{rgb}{1.000000,0.498039,0.054902}%
\pgfsetfillcolor{currentfill}%
\pgfsetlinewidth{1.003750pt}%
\definecolor{currentstroke}{rgb}{1.000000,0.498039,0.054902}%
\pgfsetstrokecolor{currentstroke}%
\pgfsetdash{}{0pt}%
\pgfpathmoveto{\pgfqpoint{1.308428in}{2.632925in}}%
\pgfpathcurveto{\pgfqpoint{1.319483in}{2.632925in}}{\pgfqpoint{1.330087in}{2.637317in}}{\pgfqpoint{1.337905in}{2.645135in}}%
\pgfpathcurveto{\pgfqpoint{1.345722in}{2.652952in}}{\pgfqpoint{1.350115in}{2.663556in}}{\pgfqpoint{1.350115in}{2.674612in}}%
\pgfpathcurveto{\pgfqpoint{1.350115in}{2.685667in}}{\pgfqpoint{1.345722in}{2.696271in}}{\pgfqpoint{1.337905in}{2.704089in}}%
\pgfpathcurveto{\pgfqpoint{1.330087in}{2.711906in}}{\pgfqpoint{1.319483in}{2.716299in}}{\pgfqpoint{1.308428in}{2.716299in}}%
\pgfpathcurveto{\pgfqpoint{1.297372in}{2.716299in}}{\pgfqpoint{1.286768in}{2.711906in}}{\pgfqpoint{1.278950in}{2.704089in}}%
\pgfpathcurveto{\pgfqpoint{1.271133in}{2.696271in}}{\pgfqpoint{1.266741in}{2.685667in}}{\pgfqpoint{1.266741in}{2.674612in}}%
\pgfpathcurveto{\pgfqpoint{1.266741in}{2.663556in}}{\pgfqpoint{1.271133in}{2.652952in}}{\pgfqpoint{1.278950in}{2.645135in}}%
\pgfpathcurveto{\pgfqpoint{1.286768in}{2.637317in}}{\pgfqpoint{1.297372in}{2.632925in}}{\pgfqpoint{1.308428in}{2.632925in}}%
\pgfpathclose%
\pgfusepath{stroke,fill}%
\end{pgfscope}%
\begin{pgfscope}%
\pgfpathrectangle{\pgfqpoint{1.065196in}{0.528000in}}{\pgfqpoint{3.702804in}{3.696000in}} %
\pgfusepath{clip}%
\pgfsetbuttcap%
\pgfsetroundjoin%
\definecolor{currentfill}{rgb}{1.000000,0.498039,0.054902}%
\pgfsetfillcolor{currentfill}%
\pgfsetlinewidth{1.003750pt}%
\definecolor{currentstroke}{rgb}{1.000000,0.498039,0.054902}%
\pgfsetstrokecolor{currentstroke}%
\pgfsetdash{}{0pt}%
\pgfpathmoveto{\pgfqpoint{2.708283in}{3.335543in}}%
\pgfpathcurveto{\pgfqpoint{2.722893in}{3.335543in}}{\pgfqpoint{2.736907in}{3.341348in}}{\pgfqpoint{2.747238in}{3.351679in}}%
\pgfpathcurveto{\pgfqpoint{2.757570in}{3.362010in}}{\pgfqpoint{2.763374in}{3.376024in}}{\pgfqpoint{2.763374in}{3.390634in}}%
\pgfpathcurveto{\pgfqpoint{2.763374in}{3.405245in}}{\pgfqpoint{2.757570in}{3.419259in}}{\pgfqpoint{2.747238in}{3.429590in}}%
\pgfpathcurveto{\pgfqpoint{2.736907in}{3.439921in}}{\pgfqpoint{2.722893in}{3.445726in}}{\pgfqpoint{2.708283in}{3.445726in}}%
\pgfpathcurveto{\pgfqpoint{2.693673in}{3.445726in}}{\pgfqpoint{2.679659in}{3.439921in}}{\pgfqpoint{2.669327in}{3.429590in}}%
\pgfpathcurveto{\pgfqpoint{2.658996in}{3.419259in}}{\pgfqpoint{2.653192in}{3.405245in}}{\pgfqpoint{2.653192in}{3.390634in}}%
\pgfpathcurveto{\pgfqpoint{2.653192in}{3.376024in}}{\pgfqpoint{2.658996in}{3.362010in}}{\pgfqpoint{2.669327in}{3.351679in}}%
\pgfpathcurveto{\pgfqpoint{2.679659in}{3.341348in}}{\pgfqpoint{2.693673in}{3.335543in}}{\pgfqpoint{2.708283in}{3.335543in}}%
\pgfpathclose%
\pgfusepath{stroke,fill}%
\end{pgfscope}%
\begin{pgfscope}%
\pgfpathrectangle{\pgfqpoint{1.065196in}{0.528000in}}{\pgfqpoint{3.702804in}{3.696000in}} %
\pgfusepath{clip}%
\pgfsetbuttcap%
\pgfsetroundjoin%
\definecolor{currentfill}{rgb}{1.000000,0.498039,0.054902}%
\pgfsetfillcolor{currentfill}%
\pgfsetlinewidth{1.003750pt}%
\definecolor{currentstroke}{rgb}{1.000000,0.498039,0.054902}%
\pgfsetstrokecolor{currentstroke}%
\pgfsetdash{}{0pt}%
\pgfpathmoveto{\pgfqpoint{2.311799in}{3.627093in}}%
\pgfpathcurveto{\pgfqpoint{2.325025in}{3.627093in}}{\pgfqpoint{2.337711in}{3.632347in}}{\pgfqpoint{2.347064in}{3.641700in}}%
\pgfpathcurveto{\pgfqpoint{2.356416in}{3.651052in}}{\pgfqpoint{2.361671in}{3.663738in}}{\pgfqpoint{2.361671in}{3.676965in}}%
\pgfpathcurveto{\pgfqpoint{2.361671in}{3.690191in}}{\pgfqpoint{2.356416in}{3.702877in}}{\pgfqpoint{2.347064in}{3.712230in}}%
\pgfpathcurveto{\pgfqpoint{2.337711in}{3.721582in}}{\pgfqpoint{2.325025in}{3.726837in}}{\pgfqpoint{2.311799in}{3.726837in}}%
\pgfpathcurveto{\pgfqpoint{2.298573in}{3.726837in}}{\pgfqpoint{2.285886in}{3.721582in}}{\pgfqpoint{2.276534in}{3.712230in}}%
\pgfpathcurveto{\pgfqpoint{2.267182in}{3.702877in}}{\pgfqpoint{2.261927in}{3.690191in}}{\pgfqpoint{2.261927in}{3.676965in}}%
\pgfpathcurveto{\pgfqpoint{2.261927in}{3.663738in}}{\pgfqpoint{2.267182in}{3.651052in}}{\pgfqpoint{2.276534in}{3.641700in}}%
\pgfpathcurveto{\pgfqpoint{2.285886in}{3.632347in}}{\pgfqpoint{2.298573in}{3.627093in}}{\pgfqpoint{2.311799in}{3.627093in}}%
\pgfpathclose%
\pgfusepath{stroke,fill}%
\end{pgfscope}%
\begin{pgfscope}%
\pgfpathrectangle{\pgfqpoint{1.065196in}{0.528000in}}{\pgfqpoint{3.702804in}{3.696000in}} %
\pgfusepath{clip}%
\pgfsetbuttcap%
\pgfsetroundjoin%
\definecolor{currentfill}{rgb}{1.000000,0.498039,0.054902}%
\pgfsetfillcolor{currentfill}%
\pgfsetlinewidth{1.003750pt}%
\definecolor{currentstroke}{rgb}{1.000000,0.498039,0.054902}%
\pgfsetstrokecolor{currentstroke}%
\pgfsetdash{}{0pt}%
\pgfpathmoveto{\pgfqpoint{3.190970in}{1.261246in}}%
\pgfpathcurveto{\pgfqpoint{3.202253in}{1.261246in}}{\pgfqpoint{3.213074in}{1.265728in}}{\pgfqpoint{3.221052in}{1.273706in}}%
\pgfpathcurveto{\pgfqpoint{3.229030in}{1.281684in}}{\pgfqpoint{3.233512in}{1.292506in}}{\pgfqpoint{3.233512in}{1.303788in}}%
\pgfpathcurveto{\pgfqpoint{3.233512in}{1.315070in}}{\pgfqpoint{3.229030in}{1.325892in}}{\pgfqpoint{3.221052in}{1.333870in}}%
\pgfpathcurveto{\pgfqpoint{3.213074in}{1.341848in}}{\pgfqpoint{3.202253in}{1.346330in}}{\pgfqpoint{3.190970in}{1.346330in}}%
\pgfpathcurveto{\pgfqpoint{3.179688in}{1.346330in}}{\pgfqpoint{3.168866in}{1.341848in}}{\pgfqpoint{3.160888in}{1.333870in}}%
\pgfpathcurveto{\pgfqpoint{3.152911in}{1.325892in}}{\pgfqpoint{3.148428in}{1.315070in}}{\pgfqpoint{3.148428in}{1.303788in}}%
\pgfpathcurveto{\pgfqpoint{3.148428in}{1.292506in}}{\pgfqpoint{3.152911in}{1.281684in}}{\pgfqpoint{3.160888in}{1.273706in}}%
\pgfpathcurveto{\pgfqpoint{3.168866in}{1.265728in}}{\pgfqpoint{3.179688in}{1.261246in}}{\pgfqpoint{3.190970in}{1.261246in}}%
\pgfpathclose%
\pgfusepath{stroke,fill}%
\end{pgfscope}%
\begin{pgfscope}%
\pgfpathrectangle{\pgfqpoint{1.065196in}{0.528000in}}{\pgfqpoint{3.702804in}{3.696000in}} %
\pgfusepath{clip}%
\pgfsetbuttcap%
\pgfsetroundjoin%
\definecolor{currentfill}{rgb}{1.000000,0.498039,0.054902}%
\pgfsetfillcolor{currentfill}%
\pgfsetlinewidth{1.003750pt}%
\definecolor{currentstroke}{rgb}{1.000000,0.498039,0.054902}%
\pgfsetstrokecolor{currentstroke}%
\pgfsetdash{}{0pt}%
\pgfpathmoveto{\pgfqpoint{3.894247in}{2.682198in}}%
\pgfpathcurveto{\pgfqpoint{3.908705in}{2.682198in}}{\pgfqpoint{3.922572in}{2.687942in}}{\pgfqpoint{3.932795in}{2.698165in}}%
\pgfpathcurveto{\pgfqpoint{3.943018in}{2.708388in}}{\pgfqpoint{3.948762in}{2.722255in}}{\pgfqpoint{3.948762in}{2.736713in}}%
\pgfpathcurveto{\pgfqpoint{3.948762in}{2.751170in}}{\pgfqpoint{3.943018in}{2.765037in}}{\pgfqpoint{3.932795in}{2.775260in}}%
\pgfpathcurveto{\pgfqpoint{3.922572in}{2.785483in}}{\pgfqpoint{3.908705in}{2.791227in}}{\pgfqpoint{3.894247in}{2.791227in}}%
\pgfpathcurveto{\pgfqpoint{3.879790in}{2.791227in}}{\pgfqpoint{3.865923in}{2.785483in}}{\pgfqpoint{3.855700in}{2.775260in}}%
\pgfpathcurveto{\pgfqpoint{3.845477in}{2.765037in}}{\pgfqpoint{3.839733in}{2.751170in}}{\pgfqpoint{3.839733in}{2.736713in}}%
\pgfpathcurveto{\pgfqpoint{3.839733in}{2.722255in}}{\pgfqpoint{3.845477in}{2.708388in}}{\pgfqpoint{3.855700in}{2.698165in}}%
\pgfpathcurveto{\pgfqpoint{3.865923in}{2.687942in}}{\pgfqpoint{3.879790in}{2.682198in}}{\pgfqpoint{3.894247in}{2.682198in}}%
\pgfpathclose%
\pgfusepath{stroke,fill}%
\end{pgfscope}%
\begin{pgfscope}%
\pgfpathrectangle{\pgfqpoint{1.065196in}{0.528000in}}{\pgfqpoint{3.702804in}{3.696000in}} %
\pgfusepath{clip}%
\pgfsetbuttcap%
\pgfsetroundjoin%
\definecolor{currentfill}{rgb}{1.000000,0.498039,0.054902}%
\pgfsetfillcolor{currentfill}%
\pgfsetlinewidth{1.003750pt}%
\definecolor{currentstroke}{rgb}{1.000000,0.498039,0.054902}%
\pgfsetstrokecolor{currentstroke}%
\pgfsetdash{}{0pt}%
\pgfpathmoveto{\pgfqpoint{1.436902in}{2.042086in}}%
\pgfpathcurveto{\pgfqpoint{1.450798in}{2.042086in}}{\pgfqpoint{1.464126in}{2.047607in}}{\pgfqpoint{1.473951in}{2.057432in}}%
\pgfpathcurveto{\pgfqpoint{1.483777in}{2.067258in}}{\pgfqpoint{1.489297in}{2.080586in}}{\pgfqpoint{1.489297in}{2.094481in}}%
\pgfpathcurveto{\pgfqpoint{1.489297in}{2.108376in}}{\pgfqpoint{1.483777in}{2.121704in}}{\pgfqpoint{1.473951in}{2.131529in}}%
\pgfpathcurveto{\pgfqpoint{1.464126in}{2.141355in}}{\pgfqpoint{1.450798in}{2.146876in}}{\pgfqpoint{1.436902in}{2.146876in}}%
\pgfpathcurveto{\pgfqpoint{1.423007in}{2.146876in}}{\pgfqpoint{1.409679in}{2.141355in}}{\pgfqpoint{1.399854in}{2.131529in}}%
\pgfpathcurveto{\pgfqpoint{1.390028in}{2.121704in}}{\pgfqpoint{1.384508in}{2.108376in}}{\pgfqpoint{1.384508in}{2.094481in}}%
\pgfpathcurveto{\pgfqpoint{1.384508in}{2.080586in}}{\pgfqpoint{1.390028in}{2.067258in}}{\pgfqpoint{1.399854in}{2.057432in}}%
\pgfpathcurveto{\pgfqpoint{1.409679in}{2.047607in}}{\pgfqpoint{1.423007in}{2.042086in}}{\pgfqpoint{1.436902in}{2.042086in}}%
\pgfpathclose%
\pgfusepath{stroke,fill}%
\end{pgfscope}%
\begin{pgfscope}%
\pgfpathrectangle{\pgfqpoint{1.065196in}{0.528000in}}{\pgfqpoint{3.702804in}{3.696000in}} %
\pgfusepath{clip}%
\pgfsetbuttcap%
\pgfsetroundjoin%
\definecolor{currentfill}{rgb}{1.000000,0.498039,0.054902}%
\pgfsetfillcolor{currentfill}%
\pgfsetlinewidth{1.003750pt}%
\definecolor{currentstroke}{rgb}{1.000000,0.498039,0.054902}%
\pgfsetstrokecolor{currentstroke}%
\pgfsetdash{}{0pt}%
\pgfpathmoveto{\pgfqpoint{3.529806in}{3.720460in}}%
\pgfpathcurveto{\pgfqpoint{3.541100in}{3.720460in}}{\pgfqpoint{3.551933in}{3.724947in}}{\pgfqpoint{3.559919in}{3.732934in}}%
\pgfpathcurveto{\pgfqpoint{3.567905in}{3.740920in}}{\pgfqpoint{3.572392in}{3.751753in}}{\pgfqpoint{3.572392in}{3.763047in}}%
\pgfpathcurveto{\pgfqpoint{3.572392in}{3.774341in}}{\pgfqpoint{3.567905in}{3.785174in}}{\pgfqpoint{3.559919in}{3.793160in}}%
\pgfpathcurveto{\pgfqpoint{3.551933in}{3.801146in}}{\pgfqpoint{3.541100in}{3.805633in}}{\pgfqpoint{3.529806in}{3.805633in}}%
\pgfpathcurveto{\pgfqpoint{3.518512in}{3.805633in}}{\pgfqpoint{3.507679in}{3.801146in}}{\pgfqpoint{3.499693in}{3.793160in}}%
\pgfpathcurveto{\pgfqpoint{3.491706in}{3.785174in}}{\pgfqpoint{3.487219in}{3.774341in}}{\pgfqpoint{3.487219in}{3.763047in}}%
\pgfpathcurveto{\pgfqpoint{3.487219in}{3.751753in}}{\pgfqpoint{3.491706in}{3.740920in}}{\pgfqpoint{3.499693in}{3.732934in}}%
\pgfpathcurveto{\pgfqpoint{3.507679in}{3.724947in}}{\pgfqpoint{3.518512in}{3.720460in}}{\pgfqpoint{3.529806in}{3.720460in}}%
\pgfpathclose%
\pgfusepath{stroke,fill}%
\end{pgfscope}%
\begin{pgfscope}%
\pgfpathrectangle{\pgfqpoint{1.065196in}{0.528000in}}{\pgfqpoint{3.702804in}{3.696000in}} %
\pgfusepath{clip}%
\pgfsetbuttcap%
\pgfsetroundjoin%
\definecolor{currentfill}{rgb}{1.000000,0.498039,0.054902}%
\pgfsetfillcolor{currentfill}%
\pgfsetlinewidth{1.003750pt}%
\definecolor{currentstroke}{rgb}{1.000000,0.498039,0.054902}%
\pgfsetstrokecolor{currentstroke}%
\pgfsetdash{}{0pt}%
\pgfpathmoveto{\pgfqpoint{1.257771in}{3.947809in}}%
\pgfpathcurveto{\pgfqpoint{1.260407in}{3.947809in}}{\pgfqpoint{1.262934in}{3.948856in}}{\pgfqpoint{1.264798in}{3.950719in}}%
\pgfpathcurveto{\pgfqpoint{1.266661in}{3.952582in}}{\pgfqpoint{1.267708in}{3.955110in}}{\pgfqpoint{1.267708in}{3.957745in}}%
\pgfpathcurveto{\pgfqpoint{1.267708in}{3.960381in}}{\pgfqpoint{1.266661in}{3.962908in}}{\pgfqpoint{1.264798in}{3.964772in}}%
\pgfpathcurveto{\pgfqpoint{1.262934in}{3.966635in}}{\pgfqpoint{1.260407in}{3.967682in}}{\pgfqpoint{1.257771in}{3.967682in}}%
\pgfpathcurveto{\pgfqpoint{1.255136in}{3.967682in}}{\pgfqpoint{1.252608in}{3.966635in}}{\pgfqpoint{1.250745in}{3.964772in}}%
\pgfpathcurveto{\pgfqpoint{1.248882in}{3.962908in}}{\pgfqpoint{1.247835in}{3.960381in}}{\pgfqpoint{1.247835in}{3.957745in}}%
\pgfpathcurveto{\pgfqpoint{1.247835in}{3.955110in}}{\pgfqpoint{1.248882in}{3.952582in}}{\pgfqpoint{1.250745in}{3.950719in}}%
\pgfpathcurveto{\pgfqpoint{1.252608in}{3.948856in}}{\pgfqpoint{1.255136in}{3.947809in}}{\pgfqpoint{1.257771in}{3.947809in}}%
\pgfpathclose%
\pgfusepath{stroke,fill}%
\end{pgfscope}%
\begin{pgfscope}%
\pgfpathrectangle{\pgfqpoint{1.065196in}{0.528000in}}{\pgfqpoint{3.702804in}{3.696000in}} %
\pgfusepath{clip}%
\pgfsetbuttcap%
\pgfsetroundjoin%
\definecolor{currentfill}{rgb}{1.000000,0.498039,0.054902}%
\pgfsetfillcolor{currentfill}%
\pgfsetlinewidth{1.003750pt}%
\definecolor{currentstroke}{rgb}{1.000000,0.498039,0.054902}%
\pgfsetstrokecolor{currentstroke}%
\pgfsetdash{}{0pt}%
\pgfpathmoveto{\pgfqpoint{2.517137in}{3.930904in}}%
\pgfpathcurveto{\pgfqpoint{2.521614in}{3.930904in}}{\pgfqpoint{2.525908in}{3.932682in}}{\pgfqpoint{2.529073in}{3.935848in}}%
\pgfpathcurveto{\pgfqpoint{2.532239in}{3.939013in}}{\pgfqpoint{2.534018in}{3.943307in}}{\pgfqpoint{2.534018in}{3.947784in}}%
\pgfpathcurveto{\pgfqpoint{2.534018in}{3.952260in}}{\pgfqpoint{2.532239in}{3.956554in}}{\pgfqpoint{2.529073in}{3.959720in}}%
\pgfpathcurveto{\pgfqpoint{2.525908in}{3.962885in}}{\pgfqpoint{2.521614in}{3.964664in}}{\pgfqpoint{2.517137in}{3.964664in}}%
\pgfpathcurveto{\pgfqpoint{2.512661in}{3.964664in}}{\pgfqpoint{2.508367in}{3.962885in}}{\pgfqpoint{2.505201in}{3.959720in}}%
\pgfpathcurveto{\pgfqpoint{2.502036in}{3.956554in}}{\pgfqpoint{2.500257in}{3.952260in}}{\pgfqpoint{2.500257in}{3.947784in}}%
\pgfpathcurveto{\pgfqpoint{2.500257in}{3.943307in}}{\pgfqpoint{2.502036in}{3.939013in}}{\pgfqpoint{2.505201in}{3.935848in}}%
\pgfpathcurveto{\pgfqpoint{2.508367in}{3.932682in}}{\pgfqpoint{2.512661in}{3.930904in}}{\pgfqpoint{2.517137in}{3.930904in}}%
\pgfpathclose%
\pgfusepath{stroke,fill}%
\end{pgfscope}%
\begin{pgfscope}%
\pgfpathrectangle{\pgfqpoint{1.065196in}{0.528000in}}{\pgfqpoint{3.702804in}{3.696000in}} %
\pgfusepath{clip}%
\pgfsetbuttcap%
\pgfsetroundjoin%
\definecolor{currentfill}{rgb}{1.000000,0.498039,0.054902}%
\pgfsetfillcolor{currentfill}%
\pgfsetlinewidth{1.003750pt}%
\definecolor{currentstroke}{rgb}{1.000000,0.498039,0.054902}%
\pgfsetstrokecolor{currentstroke}%
\pgfsetdash{}{0pt}%
\pgfpathmoveto{\pgfqpoint{3.280888in}{3.448454in}}%
\pgfpathcurveto{\pgfqpoint{3.284630in}{3.448454in}}{\pgfqpoint{3.288219in}{3.449940in}}{\pgfqpoint{3.290865in}{3.452586in}}%
\pgfpathcurveto{\pgfqpoint{3.293510in}{3.455232in}}{\pgfqpoint{3.294997in}{3.458821in}}{\pgfqpoint{3.294997in}{3.462562in}}%
\pgfpathcurveto{\pgfqpoint{3.294997in}{3.466304in}}{\pgfqpoint{3.293510in}{3.469893in}}{\pgfqpoint{3.290865in}{3.472539in}}%
\pgfpathcurveto{\pgfqpoint{3.288219in}{3.475185in}}{\pgfqpoint{3.284630in}{3.476671in}}{\pgfqpoint{3.280888in}{3.476671in}}%
\pgfpathcurveto{\pgfqpoint{3.277147in}{3.476671in}}{\pgfqpoint{3.273558in}{3.475185in}}{\pgfqpoint{3.270912in}{3.472539in}}%
\pgfpathcurveto{\pgfqpoint{3.268266in}{3.469893in}}{\pgfqpoint{3.266780in}{3.466304in}}{\pgfqpoint{3.266780in}{3.462562in}}%
\pgfpathcurveto{\pgfqpoint{3.266780in}{3.458821in}}{\pgfqpoint{3.268266in}{3.455232in}}{\pgfqpoint{3.270912in}{3.452586in}}%
\pgfpathcurveto{\pgfqpoint{3.273558in}{3.449940in}}{\pgfqpoint{3.277147in}{3.448454in}}{\pgfqpoint{3.280888in}{3.448454in}}%
\pgfpathclose%
\pgfusepath{stroke,fill}%
\end{pgfscope}%
\begin{pgfscope}%
\pgfpathrectangle{\pgfqpoint{1.065196in}{0.528000in}}{\pgfqpoint{3.702804in}{3.696000in}} %
\pgfusepath{clip}%
\pgfsetbuttcap%
\pgfsetroundjoin%
\definecolor{currentfill}{rgb}{1.000000,0.498039,0.054902}%
\pgfsetfillcolor{currentfill}%
\pgfsetlinewidth{1.003750pt}%
\definecolor{currentstroke}{rgb}{1.000000,0.498039,0.054902}%
\pgfsetstrokecolor{currentstroke}%
\pgfsetdash{}{0pt}%
\pgfpathmoveto{\pgfqpoint{3.180439in}{2.753340in}}%
\pgfpathcurveto{\pgfqpoint{3.190275in}{2.753340in}}{\pgfqpoint{3.199709in}{2.757248in}}{\pgfqpoint{3.206664in}{2.764203in}}%
\pgfpathcurveto{\pgfqpoint{3.213619in}{2.771158in}}{\pgfqpoint{3.217527in}{2.780592in}}{\pgfqpoint{3.217527in}{2.790428in}}%
\pgfpathcurveto{\pgfqpoint{3.217527in}{2.800264in}}{\pgfqpoint{3.213619in}{2.809698in}}{\pgfqpoint{3.206664in}{2.816653in}}%
\pgfpathcurveto{\pgfqpoint{3.199709in}{2.823608in}}{\pgfqpoint{3.190275in}{2.827516in}}{\pgfqpoint{3.180439in}{2.827516in}}%
\pgfpathcurveto{\pgfqpoint{3.170603in}{2.827516in}}{\pgfqpoint{3.161169in}{2.823608in}}{\pgfqpoint{3.154214in}{2.816653in}}%
\pgfpathcurveto{\pgfqpoint{3.147259in}{2.809698in}}{\pgfqpoint{3.143351in}{2.800264in}}{\pgfqpoint{3.143351in}{2.790428in}}%
\pgfpathcurveto{\pgfqpoint{3.143351in}{2.780592in}}{\pgfqpoint{3.147259in}{2.771158in}}{\pgfqpoint{3.154214in}{2.764203in}}%
\pgfpathcurveto{\pgfqpoint{3.161169in}{2.757248in}}{\pgfqpoint{3.170603in}{2.753340in}}{\pgfqpoint{3.180439in}{2.753340in}}%
\pgfpathclose%
\pgfusepath{stroke,fill}%
\end{pgfscope}%
\begin{pgfscope}%
\pgfpathrectangle{\pgfqpoint{1.065196in}{0.528000in}}{\pgfqpoint{3.702804in}{3.696000in}} %
\pgfusepath{clip}%
\pgfsetbuttcap%
\pgfsetroundjoin%
\definecolor{currentfill}{rgb}{1.000000,0.498039,0.054902}%
\pgfsetfillcolor{currentfill}%
\pgfsetlinewidth{1.003750pt}%
\definecolor{currentstroke}{rgb}{1.000000,0.498039,0.054902}%
\pgfsetstrokecolor{currentstroke}%
\pgfsetdash{}{0pt}%
\pgfpathmoveto{\pgfqpoint{2.212475in}{2.618544in}}%
\pgfpathcurveto{\pgfqpoint{2.221442in}{2.618544in}}{\pgfqpoint{2.230042in}{2.622107in}}{\pgfqpoint{2.236383in}{2.628447in}}%
\pgfpathcurveto{\pgfqpoint{2.242723in}{2.634788in}}{\pgfqpoint{2.246286in}{2.643389in}}{\pgfqpoint{2.246286in}{2.652355in}}%
\pgfpathcurveto{\pgfqpoint{2.246286in}{2.661322in}}{\pgfqpoint{2.242723in}{2.669923in}}{\pgfqpoint{2.236383in}{2.676263in}}%
\pgfpathcurveto{\pgfqpoint{2.230042in}{2.682604in}}{\pgfqpoint{2.221442in}{2.686166in}}{\pgfqpoint{2.212475in}{2.686166in}}%
\pgfpathcurveto{\pgfqpoint{2.203508in}{2.686166in}}{\pgfqpoint{2.194908in}{2.682604in}}{\pgfqpoint{2.188567in}{2.676263in}}%
\pgfpathcurveto{\pgfqpoint{2.182227in}{2.669923in}}{\pgfqpoint{2.178664in}{2.661322in}}{\pgfqpoint{2.178664in}{2.652355in}}%
\pgfpathcurveto{\pgfqpoint{2.178664in}{2.643389in}}{\pgfqpoint{2.182227in}{2.634788in}}{\pgfqpoint{2.188567in}{2.628447in}}%
\pgfpathcurveto{\pgfqpoint{2.194908in}{2.622107in}}{\pgfqpoint{2.203508in}{2.618544in}}{\pgfqpoint{2.212475in}{2.618544in}}%
\pgfpathclose%
\pgfusepath{stroke,fill}%
\end{pgfscope}%
\begin{pgfscope}%
\pgfpathrectangle{\pgfqpoint{1.065196in}{0.528000in}}{\pgfqpoint{3.702804in}{3.696000in}} %
\pgfusepath{clip}%
\pgfsetbuttcap%
\pgfsetroundjoin%
\definecolor{currentfill}{rgb}{1.000000,0.498039,0.054902}%
\pgfsetfillcolor{currentfill}%
\pgfsetlinewidth{1.003750pt}%
\definecolor{currentstroke}{rgb}{1.000000,0.498039,0.054902}%
\pgfsetstrokecolor{currentstroke}%
\pgfsetdash{}{0pt}%
\pgfpathmoveto{\pgfqpoint{3.767341in}{3.506719in}}%
\pgfpathcurveto{\pgfqpoint{3.781794in}{3.506719in}}{\pgfqpoint{3.795657in}{3.512461in}}{\pgfqpoint{3.805876in}{3.522680in}}%
\pgfpathcurveto{\pgfqpoint{3.816096in}{3.532900in}}{\pgfqpoint{3.821838in}{3.546763in}}{\pgfqpoint{3.821838in}{3.561215in}}%
\pgfpathcurveto{\pgfqpoint{3.821838in}{3.575668in}}{\pgfqpoint{3.816096in}{3.589531in}}{\pgfqpoint{3.805876in}{3.599750in}}%
\pgfpathcurveto{\pgfqpoint{3.795657in}{3.609970in}}{\pgfqpoint{3.781794in}{3.615712in}}{\pgfqpoint{3.767341in}{3.615712in}}%
\pgfpathcurveto{\pgfqpoint{3.752888in}{3.615712in}}{\pgfqpoint{3.739026in}{3.609970in}}{\pgfqpoint{3.728806in}{3.599750in}}%
\pgfpathcurveto{\pgfqpoint{3.718587in}{3.589531in}}{\pgfqpoint{3.712844in}{3.575668in}}{\pgfqpoint{3.712844in}{3.561215in}}%
\pgfpathcurveto{\pgfqpoint{3.712844in}{3.546763in}}{\pgfqpoint{3.718587in}{3.532900in}}{\pgfqpoint{3.728806in}{3.522680in}}%
\pgfpathcurveto{\pgfqpoint{3.739026in}{3.512461in}}{\pgfqpoint{3.752888in}{3.506719in}}{\pgfqpoint{3.767341in}{3.506719in}}%
\pgfpathclose%
\pgfusepath{stroke,fill}%
\end{pgfscope}%
\begin{pgfscope}%
\pgfpathrectangle{\pgfqpoint{1.065196in}{0.528000in}}{\pgfqpoint{3.702804in}{3.696000in}} %
\pgfusepath{clip}%
\pgfsetbuttcap%
\pgfsetroundjoin%
\definecolor{currentfill}{rgb}{1.000000,0.498039,0.054902}%
\pgfsetfillcolor{currentfill}%
\pgfsetlinewidth{1.003750pt}%
\definecolor{currentstroke}{rgb}{1.000000,0.498039,0.054902}%
\pgfsetstrokecolor{currentstroke}%
\pgfsetdash{}{0pt}%
\pgfpathmoveto{\pgfqpoint{3.784282in}{3.002093in}}%
\pgfpathcurveto{\pgfqpoint{3.790280in}{3.002093in}}{\pgfqpoint{3.796033in}{3.004476in}}{\pgfqpoint{3.800275in}{3.008718in}}%
\pgfpathcurveto{\pgfqpoint{3.804516in}{3.012959in}}{\pgfqpoint{3.806899in}{3.018712in}}{\pgfqpoint{3.806899in}{3.024710in}}%
\pgfpathcurveto{\pgfqpoint{3.806899in}{3.030708in}}{\pgfqpoint{3.804516in}{3.036461in}}{\pgfqpoint{3.800275in}{3.040702in}}%
\pgfpathcurveto{\pgfqpoint{3.796033in}{3.044943in}}{\pgfqpoint{3.790280in}{3.047326in}}{\pgfqpoint{3.784282in}{3.047326in}}%
\pgfpathcurveto{\pgfqpoint{3.778284in}{3.047326in}}{\pgfqpoint{3.772531in}{3.044943in}}{\pgfqpoint{3.768290in}{3.040702in}}%
\pgfpathcurveto{\pgfqpoint{3.764049in}{3.036461in}}{\pgfqpoint{3.761666in}{3.030708in}}{\pgfqpoint{3.761666in}{3.024710in}}%
\pgfpathcurveto{\pgfqpoint{3.761666in}{3.018712in}}{\pgfqpoint{3.764049in}{3.012959in}}{\pgfqpoint{3.768290in}{3.008718in}}%
\pgfpathcurveto{\pgfqpoint{3.772531in}{3.004476in}}{\pgfqpoint{3.778284in}{3.002093in}}{\pgfqpoint{3.784282in}{3.002093in}}%
\pgfpathclose%
\pgfusepath{stroke,fill}%
\end{pgfscope}%
\begin{pgfscope}%
\pgfpathrectangle{\pgfqpoint{1.065196in}{0.528000in}}{\pgfqpoint{3.702804in}{3.696000in}} %
\pgfusepath{clip}%
\pgfsetbuttcap%
\pgfsetroundjoin%
\definecolor{currentfill}{rgb}{1.000000,0.498039,0.054902}%
\pgfsetfillcolor{currentfill}%
\pgfsetlinewidth{1.003750pt}%
\definecolor{currentstroke}{rgb}{1.000000,0.498039,0.054902}%
\pgfsetstrokecolor{currentstroke}%
\pgfsetdash{}{0pt}%
\pgfpathmoveto{\pgfqpoint{4.150521in}{1.713604in}}%
\pgfpathcurveto{\pgfqpoint{4.164955in}{1.713604in}}{\pgfqpoint{4.178800in}{1.719338in}}{\pgfqpoint{4.189006in}{1.729544in}}%
\pgfpathcurveto{\pgfqpoint{4.199212in}{1.739751in}}{\pgfqpoint{4.204946in}{1.753595in}}{\pgfqpoint{4.204946in}{1.768029in}}%
\pgfpathcurveto{\pgfqpoint{4.204946in}{1.782462in}}{\pgfqpoint{4.199212in}{1.796307in}}{\pgfqpoint{4.189006in}{1.806513in}}%
\pgfpathcurveto{\pgfqpoint{4.178800in}{1.816719in}}{\pgfqpoint{4.164955in}{1.822454in}}{\pgfqpoint{4.150521in}{1.822454in}}%
\pgfpathcurveto{\pgfqpoint{4.136088in}{1.822454in}}{\pgfqpoint{4.122243in}{1.816719in}}{\pgfqpoint{4.112037in}{1.806513in}}%
\pgfpathcurveto{\pgfqpoint{4.101831in}{1.796307in}}{\pgfqpoint{4.096096in}{1.782462in}}{\pgfqpoint{4.096096in}{1.768029in}}%
\pgfpathcurveto{\pgfqpoint{4.096096in}{1.753595in}}{\pgfqpoint{4.101831in}{1.739751in}}{\pgfqpoint{4.112037in}{1.729544in}}%
\pgfpathcurveto{\pgfqpoint{4.122243in}{1.719338in}}{\pgfqpoint{4.136088in}{1.713604in}}{\pgfqpoint{4.150521in}{1.713604in}}%
\pgfpathclose%
\pgfusepath{stroke,fill}%
\end{pgfscope}%
\begin{pgfscope}%
\pgfpathrectangle{\pgfqpoint{1.065196in}{0.528000in}}{\pgfqpoint{3.702804in}{3.696000in}} %
\pgfusepath{clip}%
\pgfsetbuttcap%
\pgfsetroundjoin%
\definecolor{currentfill}{rgb}{1.000000,0.498039,0.054902}%
\pgfsetfillcolor{currentfill}%
\pgfsetlinewidth{1.003750pt}%
\definecolor{currentstroke}{rgb}{1.000000,0.498039,0.054902}%
\pgfsetstrokecolor{currentstroke}%
\pgfsetdash{}{0pt}%
\pgfpathmoveto{\pgfqpoint{3.502086in}{2.148850in}}%
\pgfpathcurveto{\pgfqpoint{3.514906in}{2.148850in}}{\pgfqpoint{3.527203in}{2.153944in}}{\pgfqpoint{3.536268in}{2.163009in}}%
\pgfpathcurveto{\pgfqpoint{3.545334in}{2.172074in}}{\pgfqpoint{3.550427in}{2.184371in}}{\pgfqpoint{3.550427in}{2.197192in}}%
\pgfpathcurveto{\pgfqpoint{3.550427in}{2.210012in}}{\pgfqpoint{3.545334in}{2.222309in}}{\pgfqpoint{3.536268in}{2.231375in}}%
\pgfpathcurveto{\pgfqpoint{3.527203in}{2.240440in}}{\pgfqpoint{3.514906in}{2.245534in}}{\pgfqpoint{3.502086in}{2.245534in}}%
\pgfpathcurveto{\pgfqpoint{3.489265in}{2.245534in}}{\pgfqpoint{3.476968in}{2.240440in}}{\pgfqpoint{3.467903in}{2.231375in}}%
\pgfpathcurveto{\pgfqpoint{3.458838in}{2.222309in}}{\pgfqpoint{3.453744in}{2.210012in}}{\pgfqpoint{3.453744in}{2.197192in}}%
\pgfpathcurveto{\pgfqpoint{3.453744in}{2.184371in}}{\pgfqpoint{3.458838in}{2.172074in}}{\pgfqpoint{3.467903in}{2.163009in}}%
\pgfpathcurveto{\pgfqpoint{3.476968in}{2.153944in}}{\pgfqpoint{3.489265in}{2.148850in}}{\pgfqpoint{3.502086in}{2.148850in}}%
\pgfpathclose%
\pgfusepath{stroke,fill}%
\end{pgfscope}%
\begin{pgfscope}%
\pgfpathrectangle{\pgfqpoint{1.065196in}{0.528000in}}{\pgfqpoint{3.702804in}{3.696000in}} %
\pgfusepath{clip}%
\pgfsetbuttcap%
\pgfsetroundjoin%
\definecolor{currentfill}{rgb}{1.000000,0.498039,0.054902}%
\pgfsetfillcolor{currentfill}%
\pgfsetlinewidth{1.003750pt}%
\definecolor{currentstroke}{rgb}{1.000000,0.498039,0.054902}%
\pgfsetstrokecolor{currentstroke}%
\pgfsetdash{}{0pt}%
\pgfpathmoveto{\pgfqpoint{2.535625in}{2.012973in}}%
\pgfpathcurveto{\pgfqpoint{2.548911in}{2.012973in}}{\pgfqpoint{2.561655in}{2.018252in}}{\pgfqpoint{2.571050in}{2.027646in}}%
\pgfpathcurveto{\pgfqpoint{2.580444in}{2.037041in}}{\pgfqpoint{2.585723in}{2.049785in}}{\pgfqpoint{2.585723in}{2.063071in}}%
\pgfpathcurveto{\pgfqpoint{2.585723in}{2.076357in}}{\pgfqpoint{2.580444in}{2.089100in}}{\pgfqpoint{2.571050in}{2.098495in}}%
\pgfpathcurveto{\pgfqpoint{2.561655in}{2.107890in}}{\pgfqpoint{2.548911in}{2.113168in}}{\pgfqpoint{2.535625in}{2.113168in}}%
\pgfpathcurveto{\pgfqpoint{2.522339in}{2.113168in}}{\pgfqpoint{2.509596in}{2.107890in}}{\pgfqpoint{2.500201in}{2.098495in}}%
\pgfpathcurveto{\pgfqpoint{2.490806in}{2.089100in}}{\pgfqpoint{2.485528in}{2.076357in}}{\pgfqpoint{2.485528in}{2.063071in}}%
\pgfpathcurveto{\pgfqpoint{2.485528in}{2.049785in}}{\pgfqpoint{2.490806in}{2.037041in}}{\pgfqpoint{2.500201in}{2.027646in}}%
\pgfpathcurveto{\pgfqpoint{2.509596in}{2.018252in}}{\pgfqpoint{2.522339in}{2.012973in}}{\pgfqpoint{2.535625in}{2.012973in}}%
\pgfpathclose%
\pgfusepath{stroke,fill}%
\end{pgfscope}%
\begin{pgfscope}%
\pgfpathrectangle{\pgfqpoint{1.065196in}{0.528000in}}{\pgfqpoint{3.702804in}{3.696000in}} %
\pgfusepath{clip}%
\pgfsetbuttcap%
\pgfsetroundjoin%
\definecolor{currentfill}{rgb}{1.000000,0.498039,0.054902}%
\pgfsetfillcolor{currentfill}%
\pgfsetlinewidth{1.003750pt}%
\definecolor{currentstroke}{rgb}{1.000000,0.498039,0.054902}%
\pgfsetstrokecolor{currentstroke}%
\pgfsetdash{}{0pt}%
\pgfpathmoveto{\pgfqpoint{2.600495in}{1.706406in}}%
\pgfpathcurveto{\pgfqpoint{2.612135in}{1.706406in}}{\pgfqpoint{2.623299in}{1.711030in}}{\pgfqpoint{2.631529in}{1.719261in}}%
\pgfpathcurveto{\pgfqpoint{2.639760in}{1.727491in}}{\pgfqpoint{2.644384in}{1.738656in}}{\pgfqpoint{2.644384in}{1.750295in}}%
\pgfpathcurveto{\pgfqpoint{2.644384in}{1.761935in}}{\pgfqpoint{2.639760in}{1.773099in}}{\pgfqpoint{2.631529in}{1.781330in}}%
\pgfpathcurveto{\pgfqpoint{2.623299in}{1.789560in}}{\pgfqpoint{2.612135in}{1.794185in}}{\pgfqpoint{2.600495in}{1.794185in}}%
\pgfpathcurveto{\pgfqpoint{2.588855in}{1.794185in}}{\pgfqpoint{2.577691in}{1.789560in}}{\pgfqpoint{2.569460in}{1.781330in}}%
\pgfpathcurveto{\pgfqpoint{2.561230in}{1.773099in}}{\pgfqpoint{2.556606in}{1.761935in}}{\pgfqpoint{2.556606in}{1.750295in}}%
\pgfpathcurveto{\pgfqpoint{2.556606in}{1.738656in}}{\pgfqpoint{2.561230in}{1.727491in}}{\pgfqpoint{2.569460in}{1.719261in}}%
\pgfpathcurveto{\pgfqpoint{2.577691in}{1.711030in}}{\pgfqpoint{2.588855in}{1.706406in}}{\pgfqpoint{2.600495in}{1.706406in}}%
\pgfpathclose%
\pgfusepath{stroke,fill}%
\end{pgfscope}%
\begin{pgfscope}%
\pgfpathrectangle{\pgfqpoint{1.065196in}{0.528000in}}{\pgfqpoint{3.702804in}{3.696000in}} %
\pgfusepath{clip}%
\pgfsetbuttcap%
\pgfsetroundjoin%
\definecolor{currentfill}{rgb}{1.000000,0.498039,0.054902}%
\pgfsetfillcolor{currentfill}%
\pgfsetlinewidth{1.003750pt}%
\definecolor{currentstroke}{rgb}{1.000000,0.498039,0.054902}%
\pgfsetstrokecolor{currentstroke}%
\pgfsetdash{}{0pt}%
\pgfpathmoveto{\pgfqpoint{3.338481in}{2.083013in}}%
\pgfpathcurveto{\pgfqpoint{3.350448in}{2.083013in}}{\pgfqpoint{3.361927in}{2.087768in}}{\pgfqpoint{3.370389in}{2.096230in}}%
\pgfpathcurveto{\pgfqpoint{3.378851in}{2.104692in}}{\pgfqpoint{3.383606in}{2.116171in}}{\pgfqpoint{3.383606in}{2.128138in}}%
\pgfpathcurveto{\pgfqpoint{3.383606in}{2.140105in}}{\pgfqpoint{3.378851in}{2.151584in}}{\pgfqpoint{3.370389in}{2.160046in}}%
\pgfpathcurveto{\pgfqpoint{3.361927in}{2.168508in}}{\pgfqpoint{3.350448in}{2.173263in}}{\pgfqpoint{3.338481in}{2.173263in}}%
\pgfpathcurveto{\pgfqpoint{3.326514in}{2.173263in}}{\pgfqpoint{3.315035in}{2.168508in}}{\pgfqpoint{3.306573in}{2.160046in}}%
\pgfpathcurveto{\pgfqpoint{3.298111in}{2.151584in}}{\pgfqpoint{3.293356in}{2.140105in}}{\pgfqpoint{3.293356in}{2.128138in}}%
\pgfpathcurveto{\pgfqpoint{3.293356in}{2.116171in}}{\pgfqpoint{3.298111in}{2.104692in}}{\pgfqpoint{3.306573in}{2.096230in}}%
\pgfpathcurveto{\pgfqpoint{3.315035in}{2.087768in}}{\pgfqpoint{3.326514in}{2.083013in}}{\pgfqpoint{3.338481in}{2.083013in}}%
\pgfpathclose%
\pgfusepath{stroke,fill}%
\end{pgfscope}%
\begin{pgfscope}%
\pgfpathrectangle{\pgfqpoint{1.065196in}{0.528000in}}{\pgfqpoint{3.702804in}{3.696000in}} %
\pgfusepath{clip}%
\pgfsetbuttcap%
\pgfsetroundjoin%
\definecolor{currentfill}{rgb}{1.000000,0.498039,0.054902}%
\pgfsetfillcolor{currentfill}%
\pgfsetlinewidth{1.003750pt}%
\definecolor{currentstroke}{rgb}{1.000000,0.498039,0.054902}%
\pgfsetstrokecolor{currentstroke}%
\pgfsetdash{}{0pt}%
\pgfpathmoveto{\pgfqpoint{4.516511in}{2.918790in}}%
\pgfpathcurveto{\pgfqpoint{4.526617in}{2.918790in}}{\pgfqpoint{4.536310in}{2.922805in}}{\pgfqpoint{4.543456in}{2.929951in}}%
\pgfpathcurveto{\pgfqpoint{4.550602in}{2.937097in}}{\pgfqpoint{4.554617in}{2.946790in}}{\pgfqpoint{4.554617in}{2.956896in}}%
\pgfpathcurveto{\pgfqpoint{4.554617in}{2.967002in}}{\pgfqpoint{4.550602in}{2.976695in}}{\pgfqpoint{4.543456in}{2.983841in}}%
\pgfpathcurveto{\pgfqpoint{4.536310in}{2.990987in}}{\pgfqpoint{4.526617in}{2.995002in}}{\pgfqpoint{4.516511in}{2.995002in}}%
\pgfpathcurveto{\pgfqpoint{4.506405in}{2.995002in}}{\pgfqpoint{4.496712in}{2.990987in}}{\pgfqpoint{4.489566in}{2.983841in}}%
\pgfpathcurveto{\pgfqpoint{4.482420in}{2.976695in}}{\pgfqpoint{4.478405in}{2.967002in}}{\pgfqpoint{4.478405in}{2.956896in}}%
\pgfpathcurveto{\pgfqpoint{4.478405in}{2.946790in}}{\pgfqpoint{4.482420in}{2.937097in}}{\pgfqpoint{4.489566in}{2.929951in}}%
\pgfpathcurveto{\pgfqpoint{4.496712in}{2.922805in}}{\pgfqpoint{4.506405in}{2.918790in}}{\pgfqpoint{4.516511in}{2.918790in}}%
\pgfpathclose%
\pgfusepath{stroke,fill}%
\end{pgfscope}%
\begin{pgfscope}%
\pgfpathrectangle{\pgfqpoint{1.065196in}{0.528000in}}{\pgfqpoint{3.702804in}{3.696000in}} %
\pgfusepath{clip}%
\pgfsetbuttcap%
\pgfsetroundjoin%
\definecolor{currentfill}{rgb}{1.000000,0.498039,0.054902}%
\pgfsetfillcolor{currentfill}%
\pgfsetlinewidth{1.003750pt}%
\definecolor{currentstroke}{rgb}{1.000000,0.498039,0.054902}%
\pgfsetstrokecolor{currentstroke}%
\pgfsetdash{}{0pt}%
\pgfpathmoveto{\pgfqpoint{1.921196in}{2.109941in}}%
\pgfpathcurveto{\pgfqpoint{1.922873in}{2.109941in}}{\pgfqpoint{1.924482in}{2.110608in}}{\pgfqpoint{1.925668in}{2.111794in}}%
\pgfpathcurveto{\pgfqpoint{1.926854in}{2.112980in}}{\pgfqpoint{1.927520in}{2.114589in}}{\pgfqpoint{1.927520in}{2.116266in}}%
\pgfpathcurveto{\pgfqpoint{1.927520in}{2.117943in}}{\pgfqpoint{1.926854in}{2.119552in}}{\pgfqpoint{1.925668in}{2.120738in}}%
\pgfpathcurveto{\pgfqpoint{1.924482in}{2.121924in}}{\pgfqpoint{1.922873in}{2.122590in}}{\pgfqpoint{1.921196in}{2.122590in}}%
\pgfpathcurveto{\pgfqpoint{1.919519in}{2.122590in}}{\pgfqpoint{1.917910in}{2.121924in}}{\pgfqpoint{1.916724in}{2.120738in}}%
\pgfpathcurveto{\pgfqpoint{1.915538in}{2.119552in}}{\pgfqpoint{1.914871in}{2.117943in}}{\pgfqpoint{1.914871in}{2.116266in}}%
\pgfpathcurveto{\pgfqpoint{1.914871in}{2.114589in}}{\pgfqpoint{1.915538in}{2.112980in}}{\pgfqpoint{1.916724in}{2.111794in}}%
\pgfpathcurveto{\pgfqpoint{1.917910in}{2.110608in}}{\pgfqpoint{1.919519in}{2.109941in}}{\pgfqpoint{1.921196in}{2.109941in}}%
\pgfpathclose%
\pgfusepath{stroke,fill}%
\end{pgfscope}%
\begin{pgfscope}%
\pgfpathrectangle{\pgfqpoint{1.065196in}{0.528000in}}{\pgfqpoint{3.702804in}{3.696000in}} %
\pgfusepath{clip}%
\pgfsetbuttcap%
\pgfsetroundjoin%
\definecolor{currentfill}{rgb}{1.000000,0.498039,0.054902}%
\pgfsetfillcolor{currentfill}%
\pgfsetlinewidth{1.003750pt}%
\definecolor{currentstroke}{rgb}{1.000000,0.498039,0.054902}%
\pgfsetstrokecolor{currentstroke}%
\pgfsetdash{}{0pt}%
\pgfpathmoveto{\pgfqpoint{2.405877in}{3.325302in}}%
\pgfpathcurveto{\pgfqpoint{2.414572in}{3.325302in}}{\pgfqpoint{2.422911in}{3.328757in}}{\pgfqpoint{2.429060in}{3.334905in}}%
\pgfpathcurveto{\pgfqpoint{2.435208in}{3.341053in}}{\pgfqpoint{2.438662in}{3.349393in}}{\pgfqpoint{2.438662in}{3.358088in}}%
\pgfpathcurveto{\pgfqpoint{2.438662in}{3.366783in}}{\pgfqpoint{2.435208in}{3.375123in}}{\pgfqpoint{2.429060in}{3.381271in}}%
\pgfpathcurveto{\pgfqpoint{2.422911in}{3.387419in}}{\pgfqpoint{2.414572in}{3.390873in}}{\pgfqpoint{2.405877in}{3.390873in}}%
\pgfpathcurveto{\pgfqpoint{2.397182in}{3.390873in}}{\pgfqpoint{2.388842in}{3.387419in}}{\pgfqpoint{2.382694in}{3.381271in}}%
\pgfpathcurveto{\pgfqpoint{2.376546in}{3.375123in}}{\pgfqpoint{2.373091in}{3.366783in}}{\pgfqpoint{2.373091in}{3.358088in}}%
\pgfpathcurveto{\pgfqpoint{2.373091in}{3.349393in}}{\pgfqpoint{2.376546in}{3.341053in}}{\pgfqpoint{2.382694in}{3.334905in}}%
\pgfpathcurveto{\pgfqpoint{2.388842in}{3.328757in}}{\pgfqpoint{2.397182in}{3.325302in}}{\pgfqpoint{2.405877in}{3.325302in}}%
\pgfpathclose%
\pgfusepath{stroke,fill}%
\end{pgfscope}%
\begin{pgfscope}%
\pgfpathrectangle{\pgfqpoint{1.065196in}{0.528000in}}{\pgfqpoint{3.702804in}{3.696000in}} %
\pgfusepath{clip}%
\pgfsetbuttcap%
\pgfsetroundjoin%
\definecolor{currentfill}{rgb}{1.000000,0.498039,0.054902}%
\pgfsetfillcolor{currentfill}%
\pgfsetlinewidth{1.003750pt}%
\definecolor{currentstroke}{rgb}{1.000000,0.498039,0.054902}%
\pgfsetstrokecolor{currentstroke}%
\pgfsetdash{}{0pt}%
\pgfpathmoveto{\pgfqpoint{4.202475in}{3.674926in}}%
\pgfpathcurveto{\pgfqpoint{4.212741in}{3.674926in}}{\pgfqpoint{4.222587in}{3.679004in}}{\pgfqpoint{4.229846in}{3.686263in}}%
\pgfpathcurveto{\pgfqpoint{4.237105in}{3.693522in}}{\pgfqpoint{4.241184in}{3.703368in}}{\pgfqpoint{4.241184in}{3.713634in}}%
\pgfpathcurveto{\pgfqpoint{4.241184in}{3.723899in}}{\pgfqpoint{4.237105in}{3.733746in}}{\pgfqpoint{4.229846in}{3.741005in}}%
\pgfpathcurveto{\pgfqpoint{4.222587in}{3.748264in}}{\pgfqpoint{4.212741in}{3.752342in}}{\pgfqpoint{4.202475in}{3.752342in}}%
\pgfpathcurveto{\pgfqpoint{4.192210in}{3.752342in}}{\pgfqpoint{4.182363in}{3.748264in}}{\pgfqpoint{4.175104in}{3.741005in}}%
\pgfpathcurveto{\pgfqpoint{4.167846in}{3.733746in}}{\pgfqpoint{4.163767in}{3.723899in}}{\pgfqpoint{4.163767in}{3.713634in}}%
\pgfpathcurveto{\pgfqpoint{4.163767in}{3.703368in}}{\pgfqpoint{4.167846in}{3.693522in}}{\pgfqpoint{4.175104in}{3.686263in}}%
\pgfpathcurveto{\pgfqpoint{4.182363in}{3.679004in}}{\pgfqpoint{4.192210in}{3.674926in}}{\pgfqpoint{4.202475in}{3.674926in}}%
\pgfpathclose%
\pgfusepath{stroke,fill}%
\end{pgfscope}%
\begin{pgfscope}%
\pgfpathrectangle{\pgfqpoint{1.065196in}{0.528000in}}{\pgfqpoint{3.702804in}{3.696000in}} %
\pgfusepath{clip}%
\pgfsetbuttcap%
\pgfsetroundjoin%
\definecolor{currentfill}{rgb}{1.000000,0.498039,0.054902}%
\pgfsetfillcolor{currentfill}%
\pgfsetlinewidth{1.003750pt}%
\definecolor{currentstroke}{rgb}{1.000000,0.498039,0.054902}%
\pgfsetstrokecolor{currentstroke}%
\pgfsetdash{}{0pt}%
\pgfpathmoveto{\pgfqpoint{3.475072in}{1.545211in}}%
\pgfpathcurveto{\pgfqpoint{3.487577in}{1.545211in}}{\pgfqpoint{3.499570in}{1.550179in}}{\pgfqpoint{3.508412in}{1.559021in}}%
\pgfpathcurveto{\pgfqpoint{3.517254in}{1.567863in}}{\pgfqpoint{3.522222in}{1.579857in}}{\pgfqpoint{3.522222in}{1.592361in}}%
\pgfpathcurveto{\pgfqpoint{3.522222in}{1.604865in}}{\pgfqpoint{3.517254in}{1.616859in}}{\pgfqpoint{3.508412in}{1.625701in}}%
\pgfpathcurveto{\pgfqpoint{3.499570in}{1.634543in}}{\pgfqpoint{3.487577in}{1.639511in}}{\pgfqpoint{3.475072in}{1.639511in}}%
\pgfpathcurveto{\pgfqpoint{3.462568in}{1.639511in}}{\pgfqpoint{3.450574in}{1.634543in}}{\pgfqpoint{3.441733in}{1.625701in}}%
\pgfpathcurveto{\pgfqpoint{3.432891in}{1.616859in}}{\pgfqpoint{3.427923in}{1.604865in}}{\pgfqpoint{3.427923in}{1.592361in}}%
\pgfpathcurveto{\pgfqpoint{3.427923in}{1.579857in}}{\pgfqpoint{3.432891in}{1.567863in}}{\pgfqpoint{3.441733in}{1.559021in}}%
\pgfpathcurveto{\pgfqpoint{3.450574in}{1.550179in}}{\pgfqpoint{3.462568in}{1.545211in}}{\pgfqpoint{3.475072in}{1.545211in}}%
\pgfpathclose%
\pgfusepath{stroke,fill}%
\end{pgfscope}%
\begin{pgfscope}%
\pgfpathrectangle{\pgfqpoint{1.065196in}{0.528000in}}{\pgfqpoint{3.702804in}{3.696000in}} %
\pgfusepath{clip}%
\pgfsetbuttcap%
\pgfsetroundjoin%
\definecolor{currentfill}{rgb}{1.000000,0.498039,0.054902}%
\pgfsetfillcolor{currentfill}%
\pgfsetlinewidth{1.003750pt}%
\definecolor{currentstroke}{rgb}{1.000000,0.498039,0.054902}%
\pgfsetstrokecolor{currentstroke}%
\pgfsetdash{}{0pt}%
\pgfpathmoveto{\pgfqpoint{2.101296in}{3.512005in}}%
\pgfpathcurveto{\pgfqpoint{2.111314in}{3.512005in}}{\pgfqpoint{2.120923in}{3.515985in}}{\pgfqpoint{2.128007in}{3.523069in}}%
\pgfpathcurveto{\pgfqpoint{2.135091in}{3.530152in}}{\pgfqpoint{2.139071in}{3.539761in}}{\pgfqpoint{2.139071in}{3.549780in}}%
\pgfpathcurveto{\pgfqpoint{2.139071in}{3.559798in}}{\pgfqpoint{2.135091in}{3.569407in}}{\pgfqpoint{2.128007in}{3.576490in}}%
\pgfpathcurveto{\pgfqpoint{2.120923in}{3.583574in}}{\pgfqpoint{2.111314in}{3.587555in}}{\pgfqpoint{2.101296in}{3.587555in}}%
\pgfpathcurveto{\pgfqpoint{2.091278in}{3.587555in}}{\pgfqpoint{2.081669in}{3.583574in}}{\pgfqpoint{2.074585in}{3.576490in}}%
\pgfpathcurveto{\pgfqpoint{2.067502in}{3.569407in}}{\pgfqpoint{2.063521in}{3.559798in}}{\pgfqpoint{2.063521in}{3.549780in}}%
\pgfpathcurveto{\pgfqpoint{2.063521in}{3.539761in}}{\pgfqpoint{2.067502in}{3.530152in}}{\pgfqpoint{2.074585in}{3.523069in}}%
\pgfpathcurveto{\pgfqpoint{2.081669in}{3.515985in}}{\pgfqpoint{2.091278in}{3.512005in}}{\pgfqpoint{2.101296in}{3.512005in}}%
\pgfpathclose%
\pgfusepath{stroke,fill}%
\end{pgfscope}%
\begin{pgfscope}%
\pgfpathrectangle{\pgfqpoint{1.065196in}{0.528000in}}{\pgfqpoint{3.702804in}{3.696000in}} %
\pgfusepath{clip}%
\pgfsetbuttcap%
\pgfsetroundjoin%
\definecolor{currentfill}{rgb}{1.000000,0.498039,0.054902}%
\pgfsetfillcolor{currentfill}%
\pgfsetlinewidth{1.003750pt}%
\definecolor{currentstroke}{rgb}{1.000000,0.498039,0.054902}%
\pgfsetstrokecolor{currentstroke}%
\pgfsetdash{}{0pt}%
\pgfpathmoveto{\pgfqpoint{3.023103in}{3.335908in}}%
\pgfpathcurveto{\pgfqpoint{3.033001in}{3.335908in}}{\pgfqpoint{3.042494in}{3.339840in}}{\pgfqpoint{3.049493in}{3.346838in}}%
\pgfpathcurveto{\pgfqpoint{3.056491in}{3.353837in}}{\pgfqpoint{3.060423in}{3.363330in}}{\pgfqpoint{3.060423in}{3.373228in}}%
\pgfpathcurveto{\pgfqpoint{3.060423in}{3.383125in}}{\pgfqpoint{3.056491in}{3.392618in}}{\pgfqpoint{3.049493in}{3.399617in}}%
\pgfpathcurveto{\pgfqpoint{3.042494in}{3.406615in}}{\pgfqpoint{3.033001in}{3.410547in}}{\pgfqpoint{3.023103in}{3.410547in}}%
\pgfpathcurveto{\pgfqpoint{3.013206in}{3.410547in}}{\pgfqpoint{3.003713in}{3.406615in}}{\pgfqpoint{2.996714in}{3.399617in}}%
\pgfpathcurveto{\pgfqpoint{2.989716in}{3.392618in}}{\pgfqpoint{2.985784in}{3.383125in}}{\pgfqpoint{2.985784in}{3.373228in}}%
\pgfpathcurveto{\pgfqpoint{2.985784in}{3.363330in}}{\pgfqpoint{2.989716in}{3.353837in}}{\pgfqpoint{2.996714in}{3.346838in}}%
\pgfpathcurveto{\pgfqpoint{3.003713in}{3.339840in}}{\pgfqpoint{3.013206in}{3.335908in}}{\pgfqpoint{3.023103in}{3.335908in}}%
\pgfpathclose%
\pgfusepath{stroke,fill}%
\end{pgfscope}%
\begin{pgfscope}%
\pgfpathrectangle{\pgfqpoint{1.065196in}{0.528000in}}{\pgfqpoint{3.702804in}{3.696000in}} %
\pgfusepath{clip}%
\pgfsetbuttcap%
\pgfsetroundjoin%
\definecolor{currentfill}{rgb}{1.000000,0.498039,0.054902}%
\pgfsetfillcolor{currentfill}%
\pgfsetlinewidth{1.003750pt}%
\definecolor{currentstroke}{rgb}{1.000000,0.498039,0.054902}%
\pgfsetstrokecolor{currentstroke}%
\pgfsetdash{}{0pt}%
\pgfpathmoveto{\pgfqpoint{3.172997in}{3.121694in}}%
\pgfpathcurveto{\pgfqpoint{3.178427in}{3.121694in}}{\pgfqpoint{3.183635in}{3.123851in}}{\pgfqpoint{3.187474in}{3.127691in}}%
\pgfpathcurveto{\pgfqpoint{3.191314in}{3.131530in}}{\pgfqpoint{3.193471in}{3.136738in}}{\pgfqpoint{3.193471in}{3.142168in}}%
\pgfpathcurveto{\pgfqpoint{3.193471in}{3.147598in}}{\pgfqpoint{3.191314in}{3.152806in}}{\pgfqpoint{3.187474in}{3.156645in}}%
\pgfpathcurveto{\pgfqpoint{3.183635in}{3.160484in}}{\pgfqpoint{3.178427in}{3.162642in}}{\pgfqpoint{3.172997in}{3.162642in}}%
\pgfpathcurveto{\pgfqpoint{3.167567in}{3.162642in}}{\pgfqpoint{3.162359in}{3.160484in}}{\pgfqpoint{3.158520in}{3.156645in}}%
\pgfpathcurveto{\pgfqpoint{3.154680in}{3.152806in}}{\pgfqpoint{3.152523in}{3.147598in}}{\pgfqpoint{3.152523in}{3.142168in}}%
\pgfpathcurveto{\pgfqpoint{3.152523in}{3.136738in}}{\pgfqpoint{3.154680in}{3.131530in}}{\pgfqpoint{3.158520in}{3.127691in}}%
\pgfpathcurveto{\pgfqpoint{3.162359in}{3.123851in}}{\pgfqpoint{3.167567in}{3.121694in}}{\pgfqpoint{3.172997in}{3.121694in}}%
\pgfpathclose%
\pgfusepath{stroke,fill}%
\end{pgfscope}%
\begin{pgfscope}%
\pgfpathrectangle{\pgfqpoint{1.065196in}{0.528000in}}{\pgfqpoint{3.702804in}{3.696000in}} %
\pgfusepath{clip}%
\pgfsetbuttcap%
\pgfsetroundjoin%
\definecolor{currentfill}{rgb}{1.000000,0.498039,0.054902}%
\pgfsetfillcolor{currentfill}%
\pgfsetlinewidth{1.003750pt}%
\definecolor{currentstroke}{rgb}{1.000000,0.498039,0.054902}%
\pgfsetstrokecolor{currentstroke}%
\pgfsetdash{}{0pt}%
\pgfpathmoveto{\pgfqpoint{2.993968in}{3.262760in}}%
\pgfpathcurveto{\pgfqpoint{2.995495in}{3.262760in}}{\pgfqpoint{2.996959in}{3.263367in}}{\pgfqpoint{2.998039in}{3.264447in}}%
\pgfpathcurveto{\pgfqpoint{2.999119in}{3.265526in}}{\pgfqpoint{2.999726in}{3.266991in}}{\pgfqpoint{2.999726in}{3.268518in}}%
\pgfpathcurveto{\pgfqpoint{2.999726in}{3.270045in}}{\pgfqpoint{2.999119in}{3.271510in}}{\pgfqpoint{2.998039in}{3.272590in}}%
\pgfpathcurveto{\pgfqpoint{2.996959in}{3.273670in}}{\pgfqpoint{2.995495in}{3.274276in}}{\pgfqpoint{2.993968in}{3.274276in}}%
\pgfpathcurveto{\pgfqpoint{2.992441in}{3.274276in}}{\pgfqpoint{2.990976in}{3.273670in}}{\pgfqpoint{2.989896in}{3.272590in}}%
\pgfpathcurveto{\pgfqpoint{2.988816in}{3.271510in}}{\pgfqpoint{2.988210in}{3.270045in}}{\pgfqpoint{2.988210in}{3.268518in}}%
\pgfpathcurveto{\pgfqpoint{2.988210in}{3.266991in}}{\pgfqpoint{2.988816in}{3.265526in}}{\pgfqpoint{2.989896in}{3.264447in}}%
\pgfpathcurveto{\pgfqpoint{2.990976in}{3.263367in}}{\pgfqpoint{2.992441in}{3.262760in}}{\pgfqpoint{2.993968in}{3.262760in}}%
\pgfpathclose%
\pgfusepath{stroke,fill}%
\end{pgfscope}%
\begin{pgfscope}%
\pgfpathrectangle{\pgfqpoint{1.065196in}{0.528000in}}{\pgfqpoint{3.702804in}{3.696000in}} %
\pgfusepath{clip}%
\pgfsetbuttcap%
\pgfsetroundjoin%
\definecolor{currentfill}{rgb}{1.000000,0.498039,0.054902}%
\pgfsetfillcolor{currentfill}%
\pgfsetlinewidth{1.003750pt}%
\definecolor{currentstroke}{rgb}{1.000000,0.498039,0.054902}%
\pgfsetstrokecolor{currentstroke}%
\pgfsetdash{}{0pt}%
\pgfpathmoveto{\pgfqpoint{3.160843in}{2.198808in}}%
\pgfpathcurveto{\pgfqpoint{3.173586in}{2.198808in}}{\pgfqpoint{3.185809in}{2.203871in}}{\pgfqpoint{3.194820in}{2.212882in}}%
\pgfpathcurveto{\pgfqpoint{3.203831in}{2.221893in}}{\pgfqpoint{3.208894in}{2.234116in}}{\pgfqpoint{3.208894in}{2.246859in}}%
\pgfpathcurveto{\pgfqpoint{3.208894in}{2.259603in}}{\pgfqpoint{3.203831in}{2.271826in}}{\pgfqpoint{3.194820in}{2.280837in}}%
\pgfpathcurveto{\pgfqpoint{3.185809in}{2.289848in}}{\pgfqpoint{3.173586in}{2.294911in}}{\pgfqpoint{3.160843in}{2.294911in}}%
\pgfpathcurveto{\pgfqpoint{3.148099in}{2.294911in}}{\pgfqpoint{3.135876in}{2.289848in}}{\pgfqpoint{3.126865in}{2.280837in}}%
\pgfpathcurveto{\pgfqpoint{3.117855in}{2.271826in}}{\pgfqpoint{3.112792in}{2.259603in}}{\pgfqpoint{3.112792in}{2.246859in}}%
\pgfpathcurveto{\pgfqpoint{3.112792in}{2.234116in}}{\pgfqpoint{3.117855in}{2.221893in}}{\pgfqpoint{3.126865in}{2.212882in}}%
\pgfpathcurveto{\pgfqpoint{3.135876in}{2.203871in}}{\pgfqpoint{3.148099in}{2.198808in}}{\pgfqpoint{3.160843in}{2.198808in}}%
\pgfpathclose%
\pgfusepath{stroke,fill}%
\end{pgfscope}%
\begin{pgfscope}%
\pgfpathrectangle{\pgfqpoint{1.065196in}{0.528000in}}{\pgfqpoint{3.702804in}{3.696000in}} %
\pgfusepath{clip}%
\pgfsetbuttcap%
\pgfsetroundjoin%
\definecolor{currentfill}{rgb}{1.000000,0.498039,0.054902}%
\pgfsetfillcolor{currentfill}%
\pgfsetlinewidth{1.003750pt}%
\definecolor{currentstroke}{rgb}{1.000000,0.498039,0.054902}%
\pgfsetstrokecolor{currentstroke}%
\pgfsetdash{}{0pt}%
\pgfpathmoveto{\pgfqpoint{2.403676in}{0.884898in}}%
\pgfpathcurveto{\pgfqpoint{2.411954in}{0.884898in}}{\pgfqpoint{2.419894in}{0.888187in}}{\pgfqpoint{2.425747in}{0.894040in}}%
\pgfpathcurveto{\pgfqpoint{2.431600in}{0.899893in}}{\pgfqpoint{2.434889in}{0.907833in}}{\pgfqpoint{2.434889in}{0.916111in}}%
\pgfpathcurveto{\pgfqpoint{2.434889in}{0.924389in}}{\pgfqpoint{2.431600in}{0.932329in}}{\pgfqpoint{2.425747in}{0.938182in}}%
\pgfpathcurveto{\pgfqpoint{2.419894in}{0.944035in}}{\pgfqpoint{2.411954in}{0.947324in}}{\pgfqpoint{2.403676in}{0.947324in}}%
\pgfpathcurveto{\pgfqpoint{2.395398in}{0.947324in}}{\pgfqpoint{2.387459in}{0.944035in}}{\pgfqpoint{2.381605in}{0.938182in}}%
\pgfpathcurveto{\pgfqpoint{2.375752in}{0.932329in}}{\pgfqpoint{2.372463in}{0.924389in}}{\pgfqpoint{2.372463in}{0.916111in}}%
\pgfpathcurveto{\pgfqpoint{2.372463in}{0.907833in}}{\pgfqpoint{2.375752in}{0.899893in}}{\pgfqpoint{2.381605in}{0.894040in}}%
\pgfpathcurveto{\pgfqpoint{2.387459in}{0.888187in}}{\pgfqpoint{2.395398in}{0.884898in}}{\pgfqpoint{2.403676in}{0.884898in}}%
\pgfpathclose%
\pgfusepath{stroke,fill}%
\end{pgfscope}%
\begin{pgfscope}%
\pgfpathrectangle{\pgfqpoint{1.065196in}{0.528000in}}{\pgfqpoint{3.702804in}{3.696000in}} %
\pgfusepath{clip}%
\pgfsetbuttcap%
\pgfsetroundjoin%
\definecolor{currentfill}{rgb}{1.000000,0.498039,0.054902}%
\pgfsetfillcolor{currentfill}%
\pgfsetlinewidth{1.003750pt}%
\definecolor{currentstroke}{rgb}{1.000000,0.498039,0.054902}%
\pgfsetstrokecolor{currentstroke}%
\pgfsetdash{}{0pt}%
\pgfpathmoveto{\pgfqpoint{2.521636in}{0.899583in}}%
\pgfpathcurveto{\pgfqpoint{2.536156in}{0.899583in}}{\pgfqpoint{2.550083in}{0.905352in}}{\pgfqpoint{2.560349in}{0.915619in}}%
\pgfpathcurveto{\pgfqpoint{2.570616in}{0.925886in}}{\pgfqpoint{2.576385in}{0.939812in}}{\pgfqpoint{2.576385in}{0.954332in}}%
\pgfpathcurveto{\pgfqpoint{2.576385in}{0.968851in}}{\pgfqpoint{2.570616in}{0.982778in}}{\pgfqpoint{2.560349in}{0.993045in}}%
\pgfpathcurveto{\pgfqpoint{2.550083in}{1.003312in}}{\pgfqpoint{2.536156in}{1.009080in}}{\pgfqpoint{2.521636in}{1.009080in}}%
\pgfpathcurveto{\pgfqpoint{2.507117in}{1.009080in}}{\pgfqpoint{2.493190in}{1.003312in}}{\pgfqpoint{2.482923in}{0.993045in}}%
\pgfpathcurveto{\pgfqpoint{2.472657in}{0.982778in}}{\pgfqpoint{2.466888in}{0.968851in}}{\pgfqpoint{2.466888in}{0.954332in}}%
\pgfpathcurveto{\pgfqpoint{2.466888in}{0.939812in}}{\pgfqpoint{2.472657in}{0.925886in}}{\pgfqpoint{2.482923in}{0.915619in}}%
\pgfpathcurveto{\pgfqpoint{2.493190in}{0.905352in}}{\pgfqpoint{2.507117in}{0.899583in}}{\pgfqpoint{2.521636in}{0.899583in}}%
\pgfpathclose%
\pgfusepath{stroke,fill}%
\end{pgfscope}%
\begin{pgfscope}%
\pgfpathrectangle{\pgfqpoint{1.065196in}{0.528000in}}{\pgfqpoint{3.702804in}{3.696000in}} %
\pgfusepath{clip}%
\pgfsetbuttcap%
\pgfsetroundjoin%
\definecolor{currentfill}{rgb}{1.000000,0.498039,0.054902}%
\pgfsetfillcolor{currentfill}%
\pgfsetlinewidth{1.003750pt}%
\definecolor{currentstroke}{rgb}{1.000000,0.498039,0.054902}%
\pgfsetstrokecolor{currentstroke}%
\pgfsetdash{}{0pt}%
\pgfpathmoveto{\pgfqpoint{4.546691in}{1.269866in}}%
\pgfpathcurveto{\pgfqpoint{4.553559in}{1.269866in}}{\pgfqpoint{4.560146in}{1.272595in}}{\pgfqpoint{4.565002in}{1.277451in}}%
\pgfpathcurveto{\pgfqpoint{4.569859in}{1.282307in}}{\pgfqpoint{4.572587in}{1.288894in}}{\pgfqpoint{4.572587in}{1.295762in}}%
\pgfpathcurveto{\pgfqpoint{4.572587in}{1.302630in}}{\pgfqpoint{4.569859in}{1.309217in}}{\pgfqpoint{4.565002in}{1.314073in}}%
\pgfpathcurveto{\pgfqpoint{4.560146in}{1.318930in}}{\pgfqpoint{4.553559in}{1.321658in}}{\pgfqpoint{4.546691in}{1.321658in}}%
\pgfpathcurveto{\pgfqpoint{4.539823in}{1.321658in}}{\pgfqpoint{4.533236in}{1.318930in}}{\pgfqpoint{4.528380in}{1.314073in}}%
\pgfpathcurveto{\pgfqpoint{4.523524in}{1.309217in}}{\pgfqpoint{4.520795in}{1.302630in}}{\pgfqpoint{4.520795in}{1.295762in}}%
\pgfpathcurveto{\pgfqpoint{4.520795in}{1.288894in}}{\pgfqpoint{4.523524in}{1.282307in}}{\pgfqpoint{4.528380in}{1.277451in}}%
\pgfpathcurveto{\pgfqpoint{4.533236in}{1.272595in}}{\pgfqpoint{4.539823in}{1.269866in}}{\pgfqpoint{4.546691in}{1.269866in}}%
\pgfpathclose%
\pgfusepath{stroke,fill}%
\end{pgfscope}%
\begin{pgfscope}%
\pgfpathrectangle{\pgfqpoint{1.065196in}{0.528000in}}{\pgfqpoint{3.702804in}{3.696000in}} %
\pgfusepath{clip}%
\pgfsetbuttcap%
\pgfsetroundjoin%
\definecolor{currentfill}{rgb}{1.000000,0.498039,0.054902}%
\pgfsetfillcolor{currentfill}%
\pgfsetlinewidth{1.003750pt}%
\definecolor{currentstroke}{rgb}{1.000000,0.498039,0.054902}%
\pgfsetstrokecolor{currentstroke}%
\pgfsetdash{}{0pt}%
\pgfpathmoveto{\pgfqpoint{3.974358in}{3.584834in}}%
\pgfpathcurveto{\pgfqpoint{3.982875in}{3.584834in}}{\pgfqpoint{3.991044in}{3.588218in}}{\pgfqpoint{3.997067in}{3.594240in}}%
\pgfpathcurveto{\pgfqpoint{4.003089in}{3.600262in}}{\pgfqpoint{4.006473in}{3.608432in}}{\pgfqpoint{4.006473in}{3.616949in}}%
\pgfpathcurveto{\pgfqpoint{4.006473in}{3.625465in}}{\pgfqpoint{4.003089in}{3.633635in}}{\pgfqpoint{3.997067in}{3.639657in}}%
\pgfpathcurveto{\pgfqpoint{3.991044in}{3.645679in}}{\pgfqpoint{3.982875in}{3.649063in}}{\pgfqpoint{3.974358in}{3.649063in}}%
\pgfpathcurveto{\pgfqpoint{3.965841in}{3.649063in}}{\pgfqpoint{3.957672in}{3.645679in}}{\pgfqpoint{3.951650in}{3.639657in}}%
\pgfpathcurveto{\pgfqpoint{3.945627in}{3.633635in}}{\pgfqpoint{3.942244in}{3.625465in}}{\pgfqpoint{3.942244in}{3.616949in}}%
\pgfpathcurveto{\pgfqpoint{3.942244in}{3.608432in}}{\pgfqpoint{3.945627in}{3.600262in}}{\pgfqpoint{3.951650in}{3.594240in}}%
\pgfpathcurveto{\pgfqpoint{3.957672in}{3.588218in}}{\pgfqpoint{3.965841in}{3.584834in}}{\pgfqpoint{3.974358in}{3.584834in}}%
\pgfpathclose%
\pgfusepath{stroke,fill}%
\end{pgfscope}%
\begin{pgfscope}%
\pgfpathrectangle{\pgfqpoint{1.065196in}{0.528000in}}{\pgfqpoint{3.702804in}{3.696000in}} %
\pgfusepath{clip}%
\pgfsetbuttcap%
\pgfsetroundjoin%
\definecolor{currentfill}{rgb}{1.000000,0.498039,0.054902}%
\pgfsetfillcolor{currentfill}%
\pgfsetlinewidth{1.003750pt}%
\definecolor{currentstroke}{rgb}{1.000000,0.498039,0.054902}%
\pgfsetstrokecolor{currentstroke}%
\pgfsetdash{}{0pt}%
\pgfpathmoveto{\pgfqpoint{3.561089in}{2.554368in}}%
\pgfpathcurveto{\pgfqpoint{3.571681in}{2.554368in}}{\pgfqpoint{3.581841in}{2.558576in}}{\pgfqpoint{3.589331in}{2.566066in}}%
\pgfpathcurveto{\pgfqpoint{3.596821in}{2.573556in}}{\pgfqpoint{3.601029in}{2.583716in}}{\pgfqpoint{3.601029in}{2.594308in}}%
\pgfpathcurveto{\pgfqpoint{3.601029in}{2.604901in}}{\pgfqpoint{3.596821in}{2.615061in}}{\pgfqpoint{3.589331in}{2.622550in}}%
\pgfpathcurveto{\pgfqpoint{3.581841in}{2.630040in}}{\pgfqpoint{3.571681in}{2.634249in}}{\pgfqpoint{3.561089in}{2.634249in}}%
\pgfpathcurveto{\pgfqpoint{3.550496in}{2.634249in}}{\pgfqpoint{3.540337in}{2.630040in}}{\pgfqpoint{3.532847in}{2.622550in}}%
\pgfpathcurveto{\pgfqpoint{3.525357in}{2.615061in}}{\pgfqpoint{3.521148in}{2.604901in}}{\pgfqpoint{3.521148in}{2.594308in}}%
\pgfpathcurveto{\pgfqpoint{3.521148in}{2.583716in}}{\pgfqpoint{3.525357in}{2.573556in}}{\pgfqpoint{3.532847in}{2.566066in}}%
\pgfpathcurveto{\pgfqpoint{3.540337in}{2.558576in}}{\pgfqpoint{3.550496in}{2.554368in}}{\pgfqpoint{3.561089in}{2.554368in}}%
\pgfpathclose%
\pgfusepath{stroke,fill}%
\end{pgfscope}%
\begin{pgfscope}%
\pgfpathrectangle{\pgfqpoint{1.065196in}{0.528000in}}{\pgfqpoint{3.702804in}{3.696000in}} %
\pgfusepath{clip}%
\pgfsetbuttcap%
\pgfsetroundjoin%
\definecolor{currentfill}{rgb}{1.000000,0.498039,0.054902}%
\pgfsetfillcolor{currentfill}%
\pgfsetlinewidth{1.003750pt}%
\definecolor{currentstroke}{rgb}{1.000000,0.498039,0.054902}%
\pgfsetstrokecolor{currentstroke}%
\pgfsetdash{}{0pt}%
\pgfpathmoveto{\pgfqpoint{1.795324in}{2.202833in}}%
\pgfpathcurveto{\pgfqpoint{1.808039in}{2.202833in}}{\pgfqpoint{1.820235in}{2.207885in}}{\pgfqpoint{1.829225in}{2.216875in}}%
\pgfpathcurveto{\pgfqpoint{1.838216in}{2.225866in}}{\pgfqpoint{1.843268in}{2.238062in}}{\pgfqpoint{1.843268in}{2.250777in}}%
\pgfpathcurveto{\pgfqpoint{1.843268in}{2.263492in}}{\pgfqpoint{1.838216in}{2.275687in}}{\pgfqpoint{1.829225in}{2.284678in}}%
\pgfpathcurveto{\pgfqpoint{1.820235in}{2.293669in}}{\pgfqpoint{1.808039in}{2.298721in}}{\pgfqpoint{1.795324in}{2.298721in}}%
\pgfpathcurveto{\pgfqpoint{1.782609in}{2.298721in}}{\pgfqpoint{1.770413in}{2.293669in}}{\pgfqpoint{1.761423in}{2.284678in}}%
\pgfpathcurveto{\pgfqpoint{1.752432in}{2.275687in}}{\pgfqpoint{1.747380in}{2.263492in}}{\pgfqpoint{1.747380in}{2.250777in}}%
\pgfpathcurveto{\pgfqpoint{1.747380in}{2.238062in}}{\pgfqpoint{1.752432in}{2.225866in}}{\pgfqpoint{1.761423in}{2.216875in}}%
\pgfpathcurveto{\pgfqpoint{1.770413in}{2.207885in}}{\pgfqpoint{1.782609in}{2.202833in}}{\pgfqpoint{1.795324in}{2.202833in}}%
\pgfpathclose%
\pgfusepath{stroke,fill}%
\end{pgfscope}%
\begin{pgfscope}%
\pgfpathrectangle{\pgfqpoint{1.065196in}{0.528000in}}{\pgfqpoint{3.702804in}{3.696000in}} %
\pgfusepath{clip}%
\pgfsetbuttcap%
\pgfsetroundjoin%
\definecolor{currentfill}{rgb}{1.000000,0.498039,0.054902}%
\pgfsetfillcolor{currentfill}%
\pgfsetlinewidth{1.003750pt}%
\definecolor{currentstroke}{rgb}{1.000000,0.498039,0.054902}%
\pgfsetstrokecolor{currentstroke}%
\pgfsetdash{}{0pt}%
\pgfpathmoveto{\pgfqpoint{2.411989in}{1.403565in}}%
\pgfpathcurveto{\pgfqpoint{2.421956in}{1.403565in}}{\pgfqpoint{2.431516in}{1.407525in}}{\pgfqpoint{2.438564in}{1.414572in}}%
\pgfpathcurveto{\pgfqpoint{2.445611in}{1.421620in}}{\pgfqpoint{2.449571in}{1.431180in}}{\pgfqpoint{2.449571in}{1.441147in}}%
\pgfpathcurveto{\pgfqpoint{2.449571in}{1.451114in}}{\pgfqpoint{2.445611in}{1.460674in}}{\pgfqpoint{2.438564in}{1.467722in}}%
\pgfpathcurveto{\pgfqpoint{2.431516in}{1.474769in}}{\pgfqpoint{2.421956in}{1.478729in}}{\pgfqpoint{2.411989in}{1.478729in}}%
\pgfpathcurveto{\pgfqpoint{2.402022in}{1.478729in}}{\pgfqpoint{2.392462in}{1.474769in}}{\pgfqpoint{2.385414in}{1.467722in}}%
\pgfpathcurveto{\pgfqpoint{2.378367in}{1.460674in}}{\pgfqpoint{2.374407in}{1.451114in}}{\pgfqpoint{2.374407in}{1.441147in}}%
\pgfpathcurveto{\pgfqpoint{2.374407in}{1.431180in}}{\pgfqpoint{2.378367in}{1.421620in}}{\pgfqpoint{2.385414in}{1.414572in}}%
\pgfpathcurveto{\pgfqpoint{2.392462in}{1.407525in}}{\pgfqpoint{2.402022in}{1.403565in}}{\pgfqpoint{2.411989in}{1.403565in}}%
\pgfpathclose%
\pgfusepath{stroke,fill}%
\end{pgfscope}%
\begin{pgfscope}%
\pgfpathrectangle{\pgfqpoint{1.065196in}{0.528000in}}{\pgfqpoint{3.702804in}{3.696000in}} %
\pgfusepath{clip}%
\pgfsetbuttcap%
\pgfsetroundjoin%
\definecolor{currentfill}{rgb}{1.000000,0.498039,0.054902}%
\pgfsetfillcolor{currentfill}%
\pgfsetlinewidth{1.003750pt}%
\definecolor{currentstroke}{rgb}{1.000000,0.498039,0.054902}%
\pgfsetstrokecolor{currentstroke}%
\pgfsetdash{}{0pt}%
\pgfpathmoveto{\pgfqpoint{3.240031in}{1.713679in}}%
\pgfpathcurveto{\pgfqpoint{3.245201in}{1.713679in}}{\pgfqpoint{3.250161in}{1.715733in}}{\pgfqpoint{3.253816in}{1.719389in}}%
\pgfpathcurveto{\pgfqpoint{3.257472in}{1.723045in}}{\pgfqpoint{3.259527in}{1.728004in}}{\pgfqpoint{3.259527in}{1.733174in}}%
\pgfpathcurveto{\pgfqpoint{3.259527in}{1.738345in}}{\pgfqpoint{3.257472in}{1.743304in}}{\pgfqpoint{3.253816in}{1.746960in}}%
\pgfpathcurveto{\pgfqpoint{3.250161in}{1.750616in}}{\pgfqpoint{3.245201in}{1.752670in}}{\pgfqpoint{3.240031in}{1.752670in}}%
\pgfpathcurveto{\pgfqpoint{3.234861in}{1.752670in}}{\pgfqpoint{3.229901in}{1.750616in}}{\pgfqpoint{3.226246in}{1.746960in}}%
\pgfpathcurveto{\pgfqpoint{3.222590in}{1.743304in}}{\pgfqpoint{3.220535in}{1.738345in}}{\pgfqpoint{3.220535in}{1.733174in}}%
\pgfpathcurveto{\pgfqpoint{3.220535in}{1.728004in}}{\pgfqpoint{3.222590in}{1.723045in}}{\pgfqpoint{3.226246in}{1.719389in}}%
\pgfpathcurveto{\pgfqpoint{3.229901in}{1.715733in}}{\pgfqpoint{3.234861in}{1.713679in}}{\pgfqpoint{3.240031in}{1.713679in}}%
\pgfpathclose%
\pgfusepath{stroke,fill}%
\end{pgfscope}%
\begin{pgfscope}%
\pgfpathrectangle{\pgfqpoint{1.065196in}{0.528000in}}{\pgfqpoint{3.702804in}{3.696000in}} %
\pgfusepath{clip}%
\pgfsetbuttcap%
\pgfsetroundjoin%
\definecolor{currentfill}{rgb}{1.000000,0.498039,0.054902}%
\pgfsetfillcolor{currentfill}%
\pgfsetlinewidth{1.003750pt}%
\definecolor{currentstroke}{rgb}{1.000000,0.498039,0.054902}%
\pgfsetstrokecolor{currentstroke}%
\pgfsetdash{}{0pt}%
\pgfpathmoveto{\pgfqpoint{4.324024in}{3.702182in}}%
\pgfpathcurveto{\pgfqpoint{4.332205in}{3.702182in}}{\pgfqpoint{4.340052in}{3.705433in}}{\pgfqpoint{4.345837in}{3.711217in}}%
\pgfpathcurveto{\pgfqpoint{4.351621in}{3.717002in}}{\pgfqpoint{4.354872in}{3.724849in}}{\pgfqpoint{4.354872in}{3.733029in}}%
\pgfpathcurveto{\pgfqpoint{4.354872in}{3.741210in}}{\pgfqpoint{4.351621in}{3.749057in}}{\pgfqpoint{4.345837in}{3.754842in}}%
\pgfpathcurveto{\pgfqpoint{4.340052in}{3.760626in}}{\pgfqpoint{4.332205in}{3.763877in}}{\pgfqpoint{4.324024in}{3.763877in}}%
\pgfpathcurveto{\pgfqpoint{4.315844in}{3.763877in}}{\pgfqpoint{4.307997in}{3.760626in}}{\pgfqpoint{4.302212in}{3.754842in}}%
\pgfpathcurveto{\pgfqpoint{4.296427in}{3.749057in}}{\pgfqpoint{4.293177in}{3.741210in}}{\pgfqpoint{4.293177in}{3.733029in}}%
\pgfpathcurveto{\pgfqpoint{4.293177in}{3.724849in}}{\pgfqpoint{4.296427in}{3.717002in}}{\pgfqpoint{4.302212in}{3.711217in}}%
\pgfpathcurveto{\pgfqpoint{4.307997in}{3.705433in}}{\pgfqpoint{4.315844in}{3.702182in}}{\pgfqpoint{4.324024in}{3.702182in}}%
\pgfpathclose%
\pgfusepath{stroke,fill}%
\end{pgfscope}%
\begin{pgfscope}%
\pgfpathrectangle{\pgfqpoint{1.065196in}{0.528000in}}{\pgfqpoint{3.702804in}{3.696000in}} %
\pgfusepath{clip}%
\pgfsetbuttcap%
\pgfsetroundjoin%
\definecolor{currentfill}{rgb}{1.000000,0.498039,0.054902}%
\pgfsetfillcolor{currentfill}%
\pgfsetlinewidth{1.003750pt}%
\definecolor{currentstroke}{rgb}{1.000000,0.498039,0.054902}%
\pgfsetstrokecolor{currentstroke}%
\pgfsetdash{}{0pt}%
\pgfpathmoveto{\pgfqpoint{2.117204in}{1.019806in}}%
\pgfpathcurveto{\pgfqpoint{2.127598in}{1.019806in}}{\pgfqpoint{2.137568in}{1.023936in}}{\pgfqpoint{2.144919in}{1.031286in}}%
\pgfpathcurveto{\pgfqpoint{2.152269in}{1.038636in}}{\pgfqpoint{2.156398in}{1.048607in}}{\pgfqpoint{2.156398in}{1.059001in}}%
\pgfpathcurveto{\pgfqpoint{2.156398in}{1.069396in}}{\pgfqpoint{2.152269in}{1.079366in}}{\pgfqpoint{2.144919in}{1.086716in}}%
\pgfpathcurveto{\pgfqpoint{2.137568in}{1.094066in}}{\pgfqpoint{2.127598in}{1.098196in}}{\pgfqpoint{2.117204in}{1.098196in}}%
\pgfpathcurveto{\pgfqpoint{2.106809in}{1.098196in}}{\pgfqpoint{2.096839in}{1.094066in}}{\pgfqpoint{2.089489in}{1.086716in}}%
\pgfpathcurveto{\pgfqpoint{2.082139in}{1.079366in}}{\pgfqpoint{2.078009in}{1.069396in}}{\pgfqpoint{2.078009in}{1.059001in}}%
\pgfpathcurveto{\pgfqpoint{2.078009in}{1.048607in}}{\pgfqpoint{2.082139in}{1.038636in}}{\pgfqpoint{2.089489in}{1.031286in}}%
\pgfpathcurveto{\pgfqpoint{2.096839in}{1.023936in}}{\pgfqpoint{2.106809in}{1.019806in}}{\pgfqpoint{2.117204in}{1.019806in}}%
\pgfpathclose%
\pgfusepath{stroke,fill}%
\end{pgfscope}%
\begin{pgfscope}%
\pgfpathrectangle{\pgfqpoint{1.065196in}{0.528000in}}{\pgfqpoint{3.702804in}{3.696000in}} %
\pgfusepath{clip}%
\pgfsetbuttcap%
\pgfsetroundjoin%
\definecolor{currentfill}{rgb}{1.000000,0.498039,0.054902}%
\pgfsetfillcolor{currentfill}%
\pgfsetlinewidth{1.003750pt}%
\definecolor{currentstroke}{rgb}{1.000000,0.498039,0.054902}%
\pgfsetstrokecolor{currentstroke}%
\pgfsetdash{}{0pt}%
\pgfpathmoveto{\pgfqpoint{1.902424in}{2.314966in}}%
\pgfpathcurveto{\pgfqpoint{1.914437in}{2.314966in}}{\pgfqpoint{1.925960in}{2.319739in}}{\pgfqpoint{1.934454in}{2.328234in}}%
\pgfpathcurveto{\pgfqpoint{1.942949in}{2.336728in}}{\pgfqpoint{1.947721in}{2.348251in}}{\pgfqpoint{1.947721in}{2.360263in}}%
\pgfpathcurveto{\pgfqpoint{1.947721in}{2.372276in}}{\pgfqpoint{1.942949in}{2.383799in}}{\pgfqpoint{1.934454in}{2.392293in}}%
\pgfpathcurveto{\pgfqpoint{1.925960in}{2.400788in}}{\pgfqpoint{1.914437in}{2.405561in}}{\pgfqpoint{1.902424in}{2.405561in}}%
\pgfpathcurveto{\pgfqpoint{1.890411in}{2.405561in}}{\pgfqpoint{1.878889in}{2.400788in}}{\pgfqpoint{1.870394in}{2.392293in}}%
\pgfpathcurveto{\pgfqpoint{1.861900in}{2.383799in}}{\pgfqpoint{1.857127in}{2.372276in}}{\pgfqpoint{1.857127in}{2.360263in}}%
\pgfpathcurveto{\pgfqpoint{1.857127in}{2.348251in}}{\pgfqpoint{1.861900in}{2.336728in}}{\pgfqpoint{1.870394in}{2.328234in}}%
\pgfpathcurveto{\pgfqpoint{1.878889in}{2.319739in}}{\pgfqpoint{1.890411in}{2.314966in}}{\pgfqpoint{1.902424in}{2.314966in}}%
\pgfpathclose%
\pgfusepath{stroke,fill}%
\end{pgfscope}%
\begin{pgfscope}%
\pgfpathrectangle{\pgfqpoint{1.065196in}{0.528000in}}{\pgfqpoint{3.702804in}{3.696000in}} %
\pgfusepath{clip}%
\pgfsetbuttcap%
\pgfsetroundjoin%
\definecolor{currentfill}{rgb}{1.000000,0.498039,0.054902}%
\pgfsetfillcolor{currentfill}%
\pgfsetlinewidth{1.003750pt}%
\definecolor{currentstroke}{rgb}{1.000000,0.498039,0.054902}%
\pgfsetstrokecolor{currentstroke}%
\pgfsetdash{}{0pt}%
\pgfpathmoveto{\pgfqpoint{3.695578in}{1.336326in}}%
\pgfpathcurveto{\pgfqpoint{3.708420in}{1.336326in}}{\pgfqpoint{3.720739in}{1.341428in}}{\pgfqpoint{3.729820in}{1.350509in}}%
\pgfpathcurveto{\pgfqpoint{3.738901in}{1.359591in}}{\pgfqpoint{3.744003in}{1.371909in}}{\pgfqpoint{3.744003in}{1.384752in}}%
\pgfpathcurveto{\pgfqpoint{3.744003in}{1.397594in}}{\pgfqpoint{3.738901in}{1.409913in}}{\pgfqpoint{3.729820in}{1.418994in}}%
\pgfpathcurveto{\pgfqpoint{3.720739in}{1.428075in}}{\pgfqpoint{3.708420in}{1.433178in}}{\pgfqpoint{3.695578in}{1.433178in}}%
\pgfpathcurveto{\pgfqpoint{3.682735in}{1.433178in}}{\pgfqpoint{3.670416in}{1.428075in}}{\pgfqpoint{3.661335in}{1.418994in}}%
\pgfpathcurveto{\pgfqpoint{3.652254in}{1.409913in}}{\pgfqpoint{3.647152in}{1.397594in}}{\pgfqpoint{3.647152in}{1.384752in}}%
\pgfpathcurveto{\pgfqpoint{3.647152in}{1.371909in}}{\pgfqpoint{3.652254in}{1.359591in}}{\pgfqpoint{3.661335in}{1.350509in}}%
\pgfpathcurveto{\pgfqpoint{3.670416in}{1.341428in}}{\pgfqpoint{3.682735in}{1.336326in}}{\pgfqpoint{3.695578in}{1.336326in}}%
\pgfpathclose%
\pgfusepath{stroke,fill}%
\end{pgfscope}%
\begin{pgfscope}%
\pgfpathrectangle{\pgfqpoint{1.065196in}{0.528000in}}{\pgfqpoint{3.702804in}{3.696000in}} %
\pgfusepath{clip}%
\pgfsetbuttcap%
\pgfsetroundjoin%
\definecolor{currentfill}{rgb}{1.000000,0.498039,0.054902}%
\pgfsetfillcolor{currentfill}%
\pgfsetlinewidth{1.003750pt}%
\definecolor{currentstroke}{rgb}{1.000000,0.498039,0.054902}%
\pgfsetstrokecolor{currentstroke}%
\pgfsetdash{}{0pt}%
\pgfpathmoveto{\pgfqpoint{2.086790in}{3.519058in}}%
\pgfpathcurveto{\pgfqpoint{2.092073in}{3.519058in}}{\pgfqpoint{2.097141in}{3.521157in}}{\pgfqpoint{2.100876in}{3.524893in}}%
\pgfpathcurveto{\pgfqpoint{2.104612in}{3.528629in}}{\pgfqpoint{2.106711in}{3.533696in}}{\pgfqpoint{2.106711in}{3.538979in}}%
\pgfpathcurveto{\pgfqpoint{2.106711in}{3.544263in}}{\pgfqpoint{2.104612in}{3.549330in}}{\pgfqpoint{2.100876in}{3.553066in}}%
\pgfpathcurveto{\pgfqpoint{2.097141in}{3.556802in}}{\pgfqpoint{2.092073in}{3.558901in}}{\pgfqpoint{2.086790in}{3.558901in}}%
\pgfpathcurveto{\pgfqpoint{2.081507in}{3.558901in}}{\pgfqpoint{2.076439in}{3.556802in}}{\pgfqpoint{2.072703in}{3.553066in}}%
\pgfpathcurveto{\pgfqpoint{2.068967in}{3.549330in}}{\pgfqpoint{2.066868in}{3.544263in}}{\pgfqpoint{2.066868in}{3.538979in}}%
\pgfpathcurveto{\pgfqpoint{2.066868in}{3.533696in}}{\pgfqpoint{2.068967in}{3.528629in}}{\pgfqpoint{2.072703in}{3.524893in}}%
\pgfpathcurveto{\pgfqpoint{2.076439in}{3.521157in}}{\pgfqpoint{2.081507in}{3.519058in}}{\pgfqpoint{2.086790in}{3.519058in}}%
\pgfpathclose%
\pgfusepath{stroke,fill}%
\end{pgfscope}%
\begin{pgfscope}%
\pgfpathrectangle{\pgfqpoint{1.065196in}{0.528000in}}{\pgfqpoint{3.702804in}{3.696000in}} %
\pgfusepath{clip}%
\pgfsetbuttcap%
\pgfsetroundjoin%
\definecolor{currentfill}{rgb}{1.000000,0.498039,0.054902}%
\pgfsetfillcolor{currentfill}%
\pgfsetlinewidth{1.003750pt}%
\definecolor{currentstroke}{rgb}{1.000000,0.498039,0.054902}%
\pgfsetstrokecolor{currentstroke}%
\pgfsetdash{}{0pt}%
\pgfpathmoveto{\pgfqpoint{2.648600in}{2.743940in}}%
\pgfpathcurveto{\pgfqpoint{2.650815in}{2.743940in}}{\pgfqpoint{2.652940in}{2.744820in}}{\pgfqpoint{2.654506in}{2.746386in}}%
\pgfpathcurveto{\pgfqpoint{2.656073in}{2.747953in}}{\pgfqpoint{2.656953in}{2.750077in}}{\pgfqpoint{2.656953in}{2.752293in}}%
\pgfpathcurveto{\pgfqpoint{2.656953in}{2.754508in}}{\pgfqpoint{2.656073in}{2.756633in}}{\pgfqpoint{2.654506in}{2.758199in}}%
\pgfpathcurveto{\pgfqpoint{2.652940in}{2.759766in}}{\pgfqpoint{2.650815in}{2.760646in}}{\pgfqpoint{2.648600in}{2.760646in}}%
\pgfpathcurveto{\pgfqpoint{2.646385in}{2.760646in}}{\pgfqpoint{2.644260in}{2.759766in}}{\pgfqpoint{2.642693in}{2.758199in}}%
\pgfpathcurveto{\pgfqpoint{2.641127in}{2.756633in}}{\pgfqpoint{2.640247in}{2.754508in}}{\pgfqpoint{2.640247in}{2.752293in}}%
\pgfpathcurveto{\pgfqpoint{2.640247in}{2.750077in}}{\pgfqpoint{2.641127in}{2.747953in}}{\pgfqpoint{2.642693in}{2.746386in}}%
\pgfpathcurveto{\pgfqpoint{2.644260in}{2.744820in}}{\pgfqpoint{2.646385in}{2.743940in}}{\pgfqpoint{2.648600in}{2.743940in}}%
\pgfpathclose%
\pgfusepath{stroke,fill}%
\end{pgfscope}%
\begin{pgfscope}%
\pgfpathrectangle{\pgfqpoint{1.065196in}{0.528000in}}{\pgfqpoint{3.702804in}{3.696000in}} %
\pgfusepath{clip}%
\pgfsetbuttcap%
\pgfsetroundjoin%
\definecolor{currentfill}{rgb}{1.000000,0.498039,0.054902}%
\pgfsetfillcolor{currentfill}%
\pgfsetlinewidth{1.003750pt}%
\definecolor{currentstroke}{rgb}{1.000000,0.498039,0.054902}%
\pgfsetstrokecolor{currentstroke}%
\pgfsetdash{}{0pt}%
\pgfpathmoveto{\pgfqpoint{2.038691in}{0.989369in}}%
\pgfpathcurveto{\pgfqpoint{2.049234in}{0.989369in}}{\pgfqpoint{2.059347in}{0.993558in}}{\pgfqpoint{2.066803in}{1.001013in}}%
\pgfpathcurveto{\pgfqpoint{2.074258in}{1.008469in}}{\pgfqpoint{2.078447in}{1.018582in}}{\pgfqpoint{2.078447in}{1.029125in}}%
\pgfpathcurveto{\pgfqpoint{2.078447in}{1.039669in}}{\pgfqpoint{2.074258in}{1.049782in}}{\pgfqpoint{2.066803in}{1.057237in}}%
\pgfpathcurveto{\pgfqpoint{2.059347in}{1.064693in}}{\pgfqpoint{2.049234in}{1.068882in}}{\pgfqpoint{2.038691in}{1.068882in}}%
\pgfpathcurveto{\pgfqpoint{2.028147in}{1.068882in}}{\pgfqpoint{2.018034in}{1.064693in}}{\pgfqpoint{2.010579in}{1.057237in}}%
\pgfpathcurveto{\pgfqpoint{2.003123in}{1.049782in}}{\pgfqpoint{1.998934in}{1.039669in}}{\pgfqpoint{1.998934in}{1.029125in}}%
\pgfpathcurveto{\pgfqpoint{1.998934in}{1.018582in}}{\pgfqpoint{2.003123in}{1.008469in}}{\pgfqpoint{2.010579in}{1.001013in}}%
\pgfpathcurveto{\pgfqpoint{2.018034in}{0.993558in}}{\pgfqpoint{2.028147in}{0.989369in}}{\pgfqpoint{2.038691in}{0.989369in}}%
\pgfpathclose%
\pgfusepath{stroke,fill}%
\end{pgfscope}%
\begin{pgfscope}%
\pgfpathrectangle{\pgfqpoint{1.065196in}{0.528000in}}{\pgfqpoint{3.702804in}{3.696000in}} %
\pgfusepath{clip}%
\pgfsetbuttcap%
\pgfsetroundjoin%
\definecolor{currentfill}{rgb}{1.000000,0.498039,0.054902}%
\pgfsetfillcolor{currentfill}%
\pgfsetlinewidth{1.003750pt}%
\definecolor{currentstroke}{rgb}{1.000000,0.498039,0.054902}%
\pgfsetstrokecolor{currentstroke}%
\pgfsetdash{}{0pt}%
\pgfpathmoveto{\pgfqpoint{2.983407in}{2.235466in}}%
\pgfpathcurveto{\pgfqpoint{2.996577in}{2.235466in}}{\pgfqpoint{3.009210in}{2.240699in}}{\pgfqpoint{3.018523in}{2.250012in}}%
\pgfpathcurveto{\pgfqpoint{3.027836in}{2.259325in}}{\pgfqpoint{3.033069in}{2.271958in}}{\pgfqpoint{3.033069in}{2.285128in}}%
\pgfpathcurveto{\pgfqpoint{3.033069in}{2.298299in}}{\pgfqpoint{3.027836in}{2.310932in}}{\pgfqpoint{3.018523in}{2.320245in}}%
\pgfpathcurveto{\pgfqpoint{3.009210in}{2.329558in}}{\pgfqpoint{2.996577in}{2.334791in}}{\pgfqpoint{2.983407in}{2.334791in}}%
\pgfpathcurveto{\pgfqpoint{2.970236in}{2.334791in}}{\pgfqpoint{2.957603in}{2.329558in}}{\pgfqpoint{2.948290in}{2.320245in}}%
\pgfpathcurveto{\pgfqpoint{2.938977in}{2.310932in}}{\pgfqpoint{2.933744in}{2.298299in}}{\pgfqpoint{2.933744in}{2.285128in}}%
\pgfpathcurveto{\pgfqpoint{2.933744in}{2.271958in}}{\pgfqpoint{2.938977in}{2.259325in}}{\pgfqpoint{2.948290in}{2.250012in}}%
\pgfpathcurveto{\pgfqpoint{2.957603in}{2.240699in}}{\pgfqpoint{2.970236in}{2.235466in}}{\pgfqpoint{2.983407in}{2.235466in}}%
\pgfpathclose%
\pgfusepath{stroke,fill}%
\end{pgfscope}%
\begin{pgfscope}%
\pgfpathrectangle{\pgfqpoint{1.065196in}{0.528000in}}{\pgfqpoint{3.702804in}{3.696000in}} %
\pgfusepath{clip}%
\pgfsetbuttcap%
\pgfsetroundjoin%
\definecolor{currentfill}{rgb}{1.000000,0.498039,0.054902}%
\pgfsetfillcolor{currentfill}%
\pgfsetlinewidth{1.003750pt}%
\definecolor{currentstroke}{rgb}{1.000000,0.498039,0.054902}%
\pgfsetstrokecolor{currentstroke}%
\pgfsetdash{}{0pt}%
\pgfpathmoveto{\pgfqpoint{1.767538in}{2.763242in}}%
\pgfpathcurveto{\pgfqpoint{1.769181in}{2.763242in}}{\pgfqpoint{1.770757in}{2.763895in}}{\pgfqpoint{1.771919in}{2.765057in}}%
\pgfpathcurveto{\pgfqpoint{1.773080in}{2.766219in}}{\pgfqpoint{1.773733in}{2.767794in}}{\pgfqpoint{1.773733in}{2.769437in}}%
\pgfpathcurveto{\pgfqpoint{1.773733in}{2.771080in}}{\pgfqpoint{1.773080in}{2.772656in}}{\pgfqpoint{1.771919in}{2.773818in}}%
\pgfpathcurveto{\pgfqpoint{1.770757in}{2.774980in}}{\pgfqpoint{1.769181in}{2.775632in}}{\pgfqpoint{1.767538in}{2.775632in}}%
\pgfpathcurveto{\pgfqpoint{1.765895in}{2.775632in}}{\pgfqpoint{1.764319in}{2.774980in}}{\pgfqpoint{1.763158in}{2.773818in}}%
\pgfpathcurveto{\pgfqpoint{1.761996in}{2.772656in}}{\pgfqpoint{1.761343in}{2.771080in}}{\pgfqpoint{1.761343in}{2.769437in}}%
\pgfpathcurveto{\pgfqpoint{1.761343in}{2.767794in}}{\pgfqpoint{1.761996in}{2.766219in}}{\pgfqpoint{1.763158in}{2.765057in}}%
\pgfpathcurveto{\pgfqpoint{1.764319in}{2.763895in}}{\pgfqpoint{1.765895in}{2.763242in}}{\pgfqpoint{1.767538in}{2.763242in}}%
\pgfpathclose%
\pgfusepath{stroke,fill}%
\end{pgfscope}%
\begin{pgfscope}%
\pgfpathrectangle{\pgfqpoint{1.065196in}{0.528000in}}{\pgfqpoint{3.702804in}{3.696000in}} %
\pgfusepath{clip}%
\pgfsetbuttcap%
\pgfsetroundjoin%
\definecolor{currentfill}{rgb}{1.000000,0.498039,0.054902}%
\pgfsetfillcolor{currentfill}%
\pgfsetlinewidth{1.003750pt}%
\definecolor{currentstroke}{rgb}{1.000000,0.498039,0.054902}%
\pgfsetstrokecolor{currentstroke}%
\pgfsetdash{}{0pt}%
\pgfpathmoveto{\pgfqpoint{3.077657in}{2.832424in}}%
\pgfpathcurveto{\pgfqpoint{3.089658in}{2.832424in}}{\pgfqpoint{3.101169in}{2.837192in}}{\pgfqpoint{3.109654in}{2.845678in}}%
\pgfpathcurveto{\pgfqpoint{3.118140in}{2.854164in}}{\pgfqpoint{3.122908in}{2.865674in}}{\pgfqpoint{3.122908in}{2.877675in}}%
\pgfpathcurveto{\pgfqpoint{3.122908in}{2.889676in}}{\pgfqpoint{3.118140in}{2.901187in}}{\pgfqpoint{3.109654in}{2.909672in}}%
\pgfpathcurveto{\pgfqpoint{3.101169in}{2.918158in}}{\pgfqpoint{3.089658in}{2.922926in}}{\pgfqpoint{3.077657in}{2.922926in}}%
\pgfpathcurveto{\pgfqpoint{3.065657in}{2.922926in}}{\pgfqpoint{3.054146in}{2.918158in}}{\pgfqpoint{3.045660in}{2.909672in}}%
\pgfpathcurveto{\pgfqpoint{3.037174in}{2.901187in}}{\pgfqpoint{3.032407in}{2.889676in}}{\pgfqpoint{3.032407in}{2.877675in}}%
\pgfpathcurveto{\pgfqpoint{3.032407in}{2.865674in}}{\pgfqpoint{3.037174in}{2.854164in}}{\pgfqpoint{3.045660in}{2.845678in}}%
\pgfpathcurveto{\pgfqpoint{3.054146in}{2.837192in}}{\pgfqpoint{3.065657in}{2.832424in}}{\pgfqpoint{3.077657in}{2.832424in}}%
\pgfpathclose%
\pgfusepath{stroke,fill}%
\end{pgfscope}%
\begin{pgfscope}%
\pgfpathrectangle{\pgfqpoint{1.065196in}{0.528000in}}{\pgfqpoint{3.702804in}{3.696000in}} %
\pgfusepath{clip}%
\pgfsetbuttcap%
\pgfsetroundjoin%
\definecolor{currentfill}{rgb}{1.000000,0.498039,0.054902}%
\pgfsetfillcolor{currentfill}%
\pgfsetlinewidth{1.003750pt}%
\definecolor{currentstroke}{rgb}{1.000000,0.498039,0.054902}%
\pgfsetstrokecolor{currentstroke}%
\pgfsetdash{}{0pt}%
\pgfpathmoveto{\pgfqpoint{1.740334in}{3.157987in}}%
\pgfpathcurveto{\pgfqpoint{1.752462in}{3.157987in}}{\pgfqpoint{1.764094in}{3.162806in}}{\pgfqpoint{1.772670in}{3.171382in}}%
\pgfpathcurveto{\pgfqpoint{1.781246in}{3.179957in}}{\pgfqpoint{1.786064in}{3.191590in}}{\pgfqpoint{1.786064in}{3.203718in}}%
\pgfpathcurveto{\pgfqpoint{1.786064in}{3.215846in}}{\pgfqpoint{1.781246in}{3.227479in}}{\pgfqpoint{1.772670in}{3.236054in}}%
\pgfpathcurveto{\pgfqpoint{1.764094in}{3.244630in}}{\pgfqpoint{1.752462in}{3.249449in}}{\pgfqpoint{1.740334in}{3.249449in}}%
\pgfpathcurveto{\pgfqpoint{1.728206in}{3.249449in}}{\pgfqpoint{1.716573in}{3.244630in}}{\pgfqpoint{1.707997in}{3.236054in}}%
\pgfpathcurveto{\pgfqpoint{1.699421in}{3.227479in}}{\pgfqpoint{1.694603in}{3.215846in}}{\pgfqpoint{1.694603in}{3.203718in}}%
\pgfpathcurveto{\pgfqpoint{1.694603in}{3.191590in}}{\pgfqpoint{1.699421in}{3.179957in}}{\pgfqpoint{1.707997in}{3.171382in}}%
\pgfpathcurveto{\pgfqpoint{1.716573in}{3.162806in}}{\pgfqpoint{1.728206in}{3.157987in}}{\pgfqpoint{1.740334in}{3.157987in}}%
\pgfpathclose%
\pgfusepath{stroke,fill}%
\end{pgfscope}%
\begin{pgfscope}%
\pgfpathrectangle{\pgfqpoint{1.065196in}{0.528000in}}{\pgfqpoint{3.702804in}{3.696000in}} %
\pgfusepath{clip}%
\pgfsetbuttcap%
\pgfsetroundjoin%
\definecolor{currentfill}{rgb}{1.000000,0.498039,0.054902}%
\pgfsetfillcolor{currentfill}%
\pgfsetlinewidth{1.003750pt}%
\definecolor{currentstroke}{rgb}{1.000000,0.498039,0.054902}%
\pgfsetstrokecolor{currentstroke}%
\pgfsetdash{}{0pt}%
\pgfpathmoveto{\pgfqpoint{1.999799in}{2.389456in}}%
\pgfpathcurveto{\pgfqpoint{2.009608in}{2.389456in}}{\pgfqpoint{2.019016in}{2.393353in}}{\pgfqpoint{2.025952in}{2.400289in}}%
\pgfpathcurveto{\pgfqpoint{2.032888in}{2.407224in}}{\pgfqpoint{2.036785in}{2.416633in}}{\pgfqpoint{2.036785in}{2.426441in}}%
\pgfpathcurveto{\pgfqpoint{2.036785in}{2.436250in}}{\pgfqpoint{2.032888in}{2.445658in}}{\pgfqpoint{2.025952in}{2.452594in}}%
\pgfpathcurveto{\pgfqpoint{2.019016in}{2.459530in}}{\pgfqpoint{2.009608in}{2.463427in}}{\pgfqpoint{1.999799in}{2.463427in}}%
\pgfpathcurveto{\pgfqpoint{1.989991in}{2.463427in}}{\pgfqpoint{1.980582in}{2.459530in}}{\pgfqpoint{1.973647in}{2.452594in}}%
\pgfpathcurveto{\pgfqpoint{1.966711in}{2.445658in}}{\pgfqpoint{1.962814in}{2.436250in}}{\pgfqpoint{1.962814in}{2.426441in}}%
\pgfpathcurveto{\pgfqpoint{1.962814in}{2.416633in}}{\pgfqpoint{1.966711in}{2.407224in}}{\pgfqpoint{1.973647in}{2.400289in}}%
\pgfpathcurveto{\pgfqpoint{1.980582in}{2.393353in}}{\pgfqpoint{1.989991in}{2.389456in}}{\pgfqpoint{1.999799in}{2.389456in}}%
\pgfpathclose%
\pgfusepath{stroke,fill}%
\end{pgfscope}%
\begin{pgfscope}%
\pgfpathrectangle{\pgfqpoint{1.065196in}{0.528000in}}{\pgfqpoint{3.702804in}{3.696000in}} %
\pgfusepath{clip}%
\pgfsetbuttcap%
\pgfsetroundjoin%
\definecolor{currentfill}{rgb}{1.000000,0.498039,0.054902}%
\pgfsetfillcolor{currentfill}%
\pgfsetlinewidth{1.003750pt}%
\definecolor{currentstroke}{rgb}{1.000000,0.498039,0.054902}%
\pgfsetstrokecolor{currentstroke}%
\pgfsetdash{}{0pt}%
\pgfpathmoveto{\pgfqpoint{3.885432in}{0.709740in}}%
\pgfpathcurveto{\pgfqpoint{3.899429in}{0.709740in}}{\pgfqpoint{3.912854in}{0.715301in}}{\pgfqpoint{3.922751in}{0.725199in}}%
\pgfpathcurveto{\pgfqpoint{3.932649in}{0.735096in}}{\pgfqpoint{3.938210in}{0.748521in}}{\pgfqpoint{3.938210in}{0.762518in}}%
\pgfpathcurveto{\pgfqpoint{3.938210in}{0.776515in}}{\pgfqpoint{3.932649in}{0.789940in}}{\pgfqpoint{3.922751in}{0.799837in}}%
\pgfpathcurveto{\pgfqpoint{3.912854in}{0.809735in}}{\pgfqpoint{3.899429in}{0.815296in}}{\pgfqpoint{3.885432in}{0.815296in}}%
\pgfpathcurveto{\pgfqpoint{3.871435in}{0.815296in}}{\pgfqpoint{3.858010in}{0.809735in}}{\pgfqpoint{3.848113in}{0.799837in}}%
\pgfpathcurveto{\pgfqpoint{3.838215in}{0.789940in}}{\pgfqpoint{3.832654in}{0.776515in}}{\pgfqpoint{3.832654in}{0.762518in}}%
\pgfpathcurveto{\pgfqpoint{3.832654in}{0.748521in}}{\pgfqpoint{3.838215in}{0.735096in}}{\pgfqpoint{3.848113in}{0.725199in}}%
\pgfpathcurveto{\pgfqpoint{3.858010in}{0.715301in}}{\pgfqpoint{3.871435in}{0.709740in}}{\pgfqpoint{3.885432in}{0.709740in}}%
\pgfpathclose%
\pgfusepath{stroke,fill}%
\end{pgfscope}%
\begin{pgfscope}%
\pgfpathrectangle{\pgfqpoint{1.065196in}{0.528000in}}{\pgfqpoint{3.702804in}{3.696000in}} %
\pgfusepath{clip}%
\pgfsetbuttcap%
\pgfsetroundjoin%
\definecolor{currentfill}{rgb}{1.000000,0.498039,0.054902}%
\pgfsetfillcolor{currentfill}%
\pgfsetlinewidth{1.003750pt}%
\definecolor{currentstroke}{rgb}{1.000000,0.498039,0.054902}%
\pgfsetstrokecolor{currentstroke}%
\pgfsetdash{}{0pt}%
\pgfpathmoveto{\pgfqpoint{2.342323in}{3.565827in}}%
\pgfpathcurveto{\pgfqpoint{2.354070in}{3.565827in}}{\pgfqpoint{2.365338in}{3.570494in}}{\pgfqpoint{2.373644in}{3.578800in}}%
\pgfpathcurveto{\pgfqpoint{2.381951in}{3.587107in}}{\pgfqpoint{2.386618in}{3.598374in}}{\pgfqpoint{2.386618in}{3.610122in}}%
\pgfpathcurveto{\pgfqpoint{2.386618in}{3.621869in}}{\pgfqpoint{2.381951in}{3.633136in}}{\pgfqpoint{2.373644in}{3.641443in}}%
\pgfpathcurveto{\pgfqpoint{2.365338in}{3.649749in}}{\pgfqpoint{2.354070in}{3.654417in}}{\pgfqpoint{2.342323in}{3.654417in}}%
\pgfpathcurveto{\pgfqpoint{2.330576in}{3.654417in}}{\pgfqpoint{2.319308in}{3.649749in}}{\pgfqpoint{2.311002in}{3.641443in}}%
\pgfpathcurveto{\pgfqpoint{2.302695in}{3.633136in}}{\pgfqpoint{2.298028in}{3.621869in}}{\pgfqpoint{2.298028in}{3.610122in}}%
\pgfpathcurveto{\pgfqpoint{2.298028in}{3.598374in}}{\pgfqpoint{2.302695in}{3.587107in}}{\pgfqpoint{2.311002in}{3.578800in}}%
\pgfpathcurveto{\pgfqpoint{2.319308in}{3.570494in}}{\pgfqpoint{2.330576in}{3.565827in}}{\pgfqpoint{2.342323in}{3.565827in}}%
\pgfpathclose%
\pgfusepath{stroke,fill}%
\end{pgfscope}%
\begin{pgfscope}%
\pgfpathrectangle{\pgfqpoint{1.065196in}{0.528000in}}{\pgfqpoint{3.702804in}{3.696000in}} %
\pgfusepath{clip}%
\pgfsetbuttcap%
\pgfsetroundjoin%
\definecolor{currentfill}{rgb}{1.000000,0.498039,0.054902}%
\pgfsetfillcolor{currentfill}%
\pgfsetlinewidth{1.003750pt}%
\definecolor{currentstroke}{rgb}{1.000000,0.498039,0.054902}%
\pgfsetstrokecolor{currentstroke}%
\pgfsetdash{}{0pt}%
\pgfpathmoveto{\pgfqpoint{4.084336in}{2.486352in}}%
\pgfpathcurveto{\pgfqpoint{4.085395in}{2.486352in}}{\pgfqpoint{4.086412in}{2.486773in}}{\pgfqpoint{4.087161in}{2.487522in}}%
\pgfpathcurveto{\pgfqpoint{4.087910in}{2.488271in}}{\pgfqpoint{4.088331in}{2.489287in}}{\pgfqpoint{4.088331in}{2.490347in}}%
\pgfpathcurveto{\pgfqpoint{4.088331in}{2.491406in}}{\pgfqpoint{4.087910in}{2.492422in}}{\pgfqpoint{4.087161in}{2.493171in}}%
\pgfpathcurveto{\pgfqpoint{4.086412in}{2.493920in}}{\pgfqpoint{4.085395in}{2.494341in}}{\pgfqpoint{4.084336in}{2.494341in}}%
\pgfpathcurveto{\pgfqpoint{4.083277in}{2.494341in}}{\pgfqpoint{4.082261in}{2.493920in}}{\pgfqpoint{4.081512in}{2.493171in}}%
\pgfpathcurveto{\pgfqpoint{4.080763in}{2.492422in}}{\pgfqpoint{4.080342in}{2.491406in}}{\pgfqpoint{4.080342in}{2.490347in}}%
\pgfpathcurveto{\pgfqpoint{4.080342in}{2.489287in}}{\pgfqpoint{4.080763in}{2.488271in}}{\pgfqpoint{4.081512in}{2.487522in}}%
\pgfpathcurveto{\pgfqpoint{4.082261in}{2.486773in}}{\pgfqpoint{4.083277in}{2.486352in}}{\pgfqpoint{4.084336in}{2.486352in}}%
\pgfpathclose%
\pgfusepath{stroke,fill}%
\end{pgfscope}%
\begin{pgfscope}%
\pgfpathrectangle{\pgfqpoint{1.065196in}{0.528000in}}{\pgfqpoint{3.702804in}{3.696000in}} %
\pgfusepath{clip}%
\pgfsetbuttcap%
\pgfsetroundjoin%
\definecolor{currentfill}{rgb}{1.000000,0.498039,0.054902}%
\pgfsetfillcolor{currentfill}%
\pgfsetlinewidth{1.003750pt}%
\definecolor{currentstroke}{rgb}{1.000000,0.498039,0.054902}%
\pgfsetstrokecolor{currentstroke}%
\pgfsetdash{}{0pt}%
\pgfpathmoveto{\pgfqpoint{4.157648in}{3.829105in}}%
\pgfpathcurveto{\pgfqpoint{4.167834in}{3.829105in}}{\pgfqpoint{4.177604in}{3.833152in}}{\pgfqpoint{4.184806in}{3.840354in}}%
\pgfpathcurveto{\pgfqpoint{4.192008in}{3.847557in}}{\pgfqpoint{4.196055in}{3.857326in}}{\pgfqpoint{4.196055in}{3.867512in}}%
\pgfpathcurveto{\pgfqpoint{4.196055in}{3.877697in}}{\pgfqpoint{4.192008in}{3.887467in}}{\pgfqpoint{4.184806in}{3.894669in}}%
\pgfpathcurveto{\pgfqpoint{4.177604in}{3.901872in}}{\pgfqpoint{4.167834in}{3.905918in}}{\pgfqpoint{4.157648in}{3.905918in}}%
\pgfpathcurveto{\pgfqpoint{4.147463in}{3.905918in}}{\pgfqpoint{4.137693in}{3.901872in}}{\pgfqpoint{4.130491in}{3.894669in}}%
\pgfpathcurveto{\pgfqpoint{4.123289in}{3.887467in}}{\pgfqpoint{4.119242in}{3.877697in}}{\pgfqpoint{4.119242in}{3.867512in}}%
\pgfpathcurveto{\pgfqpoint{4.119242in}{3.857326in}}{\pgfqpoint{4.123289in}{3.847557in}}{\pgfqpoint{4.130491in}{3.840354in}}%
\pgfpathcurveto{\pgfqpoint{4.137693in}{3.833152in}}{\pgfqpoint{4.147463in}{3.829105in}}{\pgfqpoint{4.157648in}{3.829105in}}%
\pgfpathclose%
\pgfusepath{stroke,fill}%
\end{pgfscope}%
\begin{pgfscope}%
\pgfpathrectangle{\pgfqpoint{1.065196in}{0.528000in}}{\pgfqpoint{3.702804in}{3.696000in}} %
\pgfusepath{clip}%
\pgfsetbuttcap%
\pgfsetroundjoin%
\definecolor{currentfill}{rgb}{1.000000,0.498039,0.054902}%
\pgfsetfillcolor{currentfill}%
\pgfsetlinewidth{1.003750pt}%
\definecolor{currentstroke}{rgb}{1.000000,0.498039,0.054902}%
\pgfsetstrokecolor{currentstroke}%
\pgfsetdash{}{0pt}%
\pgfpathmoveto{\pgfqpoint{4.023063in}{3.496007in}}%
\pgfpathcurveto{\pgfqpoint{4.036629in}{3.496007in}}{\pgfqpoint{4.049641in}{3.501397in}}{\pgfqpoint{4.059233in}{3.510990in}}%
\pgfpathcurveto{\pgfqpoint{4.068826in}{3.520582in}}{\pgfqpoint{4.074215in}{3.533594in}}{\pgfqpoint{4.074215in}{3.547160in}}%
\pgfpathcurveto{\pgfqpoint{4.074215in}{3.560726in}}{\pgfqpoint{4.068826in}{3.573738in}}{\pgfqpoint{4.059233in}{3.583330in}}%
\pgfpathcurveto{\pgfqpoint{4.049641in}{3.592923in}}{\pgfqpoint{4.036629in}{3.598312in}}{\pgfqpoint{4.023063in}{3.598312in}}%
\pgfpathcurveto{\pgfqpoint{4.009497in}{3.598312in}}{\pgfqpoint{3.996485in}{3.592923in}}{\pgfqpoint{3.986893in}{3.583330in}}%
\pgfpathcurveto{\pgfqpoint{3.977300in}{3.573738in}}{\pgfqpoint{3.971910in}{3.560726in}}{\pgfqpoint{3.971910in}{3.547160in}}%
\pgfpathcurveto{\pgfqpoint{3.971910in}{3.533594in}}{\pgfqpoint{3.977300in}{3.520582in}}{\pgfqpoint{3.986893in}{3.510990in}}%
\pgfpathcurveto{\pgfqpoint{3.996485in}{3.501397in}}{\pgfqpoint{4.009497in}{3.496007in}}{\pgfqpoint{4.023063in}{3.496007in}}%
\pgfpathclose%
\pgfusepath{stroke,fill}%
\end{pgfscope}%
\begin{pgfscope}%
\pgfpathrectangle{\pgfqpoint{1.065196in}{0.528000in}}{\pgfqpoint{3.702804in}{3.696000in}} %
\pgfusepath{clip}%
\pgfsetbuttcap%
\pgfsetroundjoin%
\definecolor{currentfill}{rgb}{1.000000,0.498039,0.054902}%
\pgfsetfillcolor{currentfill}%
\pgfsetlinewidth{1.003750pt}%
\definecolor{currentstroke}{rgb}{1.000000,0.498039,0.054902}%
\pgfsetstrokecolor{currentstroke}%
\pgfsetdash{}{0pt}%
\pgfpathmoveto{\pgfqpoint{1.586998in}{2.817906in}}%
\pgfpathcurveto{\pgfqpoint{1.600333in}{2.817906in}}{\pgfqpoint{1.613124in}{2.823205in}}{\pgfqpoint{1.622554in}{2.832634in}}%
\pgfpathcurveto{\pgfqpoint{1.631984in}{2.842064in}}{\pgfqpoint{1.637282in}{2.854855in}}{\pgfqpoint{1.637282in}{2.868191in}}%
\pgfpathcurveto{\pgfqpoint{1.637282in}{2.881526in}}{\pgfqpoint{1.631984in}{2.894318in}}{\pgfqpoint{1.622554in}{2.903747in}}%
\pgfpathcurveto{\pgfqpoint{1.613124in}{2.913177in}}{\pgfqpoint{1.600333in}{2.918475in}}{\pgfqpoint{1.586998in}{2.918475in}}%
\pgfpathcurveto{\pgfqpoint{1.573662in}{2.918475in}}{\pgfqpoint{1.560871in}{2.913177in}}{\pgfqpoint{1.551441in}{2.903747in}}%
\pgfpathcurveto{\pgfqpoint{1.542012in}{2.894318in}}{\pgfqpoint{1.536713in}{2.881526in}}{\pgfqpoint{1.536713in}{2.868191in}}%
\pgfpathcurveto{\pgfqpoint{1.536713in}{2.854855in}}{\pgfqpoint{1.542012in}{2.842064in}}{\pgfqpoint{1.551441in}{2.832634in}}%
\pgfpathcurveto{\pgfqpoint{1.560871in}{2.823205in}}{\pgfqpoint{1.573662in}{2.817906in}}{\pgfqpoint{1.586998in}{2.817906in}}%
\pgfpathclose%
\pgfusepath{stroke,fill}%
\end{pgfscope}%
\begin{pgfscope}%
\pgfpathrectangle{\pgfqpoint{1.065196in}{0.528000in}}{\pgfqpoint{3.702804in}{3.696000in}} %
\pgfusepath{clip}%
\pgfsetbuttcap%
\pgfsetroundjoin%
\definecolor{currentfill}{rgb}{1.000000,0.498039,0.054902}%
\pgfsetfillcolor{currentfill}%
\pgfsetlinewidth{1.003750pt}%
\definecolor{currentstroke}{rgb}{1.000000,0.498039,0.054902}%
\pgfsetstrokecolor{currentstroke}%
\pgfsetdash{}{0pt}%
\pgfpathmoveto{\pgfqpoint{3.611653in}{2.686259in}}%
\pgfpathcurveto{\pgfqpoint{3.623444in}{2.686259in}}{\pgfqpoint{3.634753in}{2.690944in}}{\pgfqpoint{3.643090in}{2.699281in}}%
\pgfpathcurveto{\pgfqpoint{3.651428in}{2.707619in}}{\pgfqpoint{3.656112in}{2.718928in}}{\pgfqpoint{3.656112in}{2.730719in}}%
\pgfpathcurveto{\pgfqpoint{3.656112in}{2.742510in}}{\pgfqpoint{3.651428in}{2.753819in}}{\pgfqpoint{3.643090in}{2.762156in}}%
\pgfpathcurveto{\pgfqpoint{3.634753in}{2.770494in}}{\pgfqpoint{3.623444in}{2.775178in}}{\pgfqpoint{3.611653in}{2.775178in}}%
\pgfpathcurveto{\pgfqpoint{3.599862in}{2.775178in}}{\pgfqpoint{3.588553in}{2.770494in}}{\pgfqpoint{3.580215in}{2.762156in}}%
\pgfpathcurveto{\pgfqpoint{3.571878in}{2.753819in}}{\pgfqpoint{3.567193in}{2.742510in}}{\pgfqpoint{3.567193in}{2.730719in}}%
\pgfpathcurveto{\pgfqpoint{3.567193in}{2.718928in}}{\pgfqpoint{3.571878in}{2.707619in}}{\pgfqpoint{3.580215in}{2.699281in}}%
\pgfpathcurveto{\pgfqpoint{3.588553in}{2.690944in}}{\pgfqpoint{3.599862in}{2.686259in}}{\pgfqpoint{3.611653in}{2.686259in}}%
\pgfpathclose%
\pgfusepath{stroke,fill}%
\end{pgfscope}%
\begin{pgfscope}%
\pgfpathrectangle{\pgfqpoint{1.065196in}{0.528000in}}{\pgfqpoint{3.702804in}{3.696000in}} %
\pgfusepath{clip}%
\pgfsetbuttcap%
\pgfsetroundjoin%
\definecolor{currentfill}{rgb}{1.000000,0.498039,0.054902}%
\pgfsetfillcolor{currentfill}%
\pgfsetlinewidth{1.003750pt}%
\definecolor{currentstroke}{rgb}{1.000000,0.498039,0.054902}%
\pgfsetstrokecolor{currentstroke}%
\pgfsetdash{}{0pt}%
\pgfpathmoveto{\pgfqpoint{3.933370in}{0.758205in}}%
\pgfpathcurveto{\pgfqpoint{3.945381in}{0.758205in}}{\pgfqpoint{3.956903in}{0.762978in}}{\pgfqpoint{3.965396in}{0.771471in}}%
\pgfpathcurveto{\pgfqpoint{3.973890in}{0.779965in}}{\pgfqpoint{3.978662in}{0.791486in}}{\pgfqpoint{3.978662in}{0.803498in}}%
\pgfpathcurveto{\pgfqpoint{3.978662in}{0.815509in}}{\pgfqpoint{3.973890in}{0.827030in}}{\pgfqpoint{3.965396in}{0.835524in}}%
\pgfpathcurveto{\pgfqpoint{3.956903in}{0.844017in}}{\pgfqpoint{3.945381in}{0.848790in}}{\pgfqpoint{3.933370in}{0.848790in}}%
\pgfpathcurveto{\pgfqpoint{3.921358in}{0.848790in}}{\pgfqpoint{3.909837in}{0.844017in}}{\pgfqpoint{3.901343in}{0.835524in}}%
\pgfpathcurveto{\pgfqpoint{3.892850in}{0.827030in}}{\pgfqpoint{3.888078in}{0.815509in}}{\pgfqpoint{3.888078in}{0.803498in}}%
\pgfpathcurveto{\pgfqpoint{3.888078in}{0.791486in}}{\pgfqpoint{3.892850in}{0.779965in}}{\pgfqpoint{3.901343in}{0.771471in}}%
\pgfpathcurveto{\pgfqpoint{3.909837in}{0.762978in}}{\pgfqpoint{3.921358in}{0.758205in}}{\pgfqpoint{3.933370in}{0.758205in}}%
\pgfpathclose%
\pgfusepath{stroke,fill}%
\end{pgfscope}%
\begin{pgfscope}%
\pgfpathrectangle{\pgfqpoint{1.065196in}{0.528000in}}{\pgfqpoint{3.702804in}{3.696000in}} %
\pgfusepath{clip}%
\pgfsetbuttcap%
\pgfsetroundjoin%
\definecolor{currentfill}{rgb}{1.000000,0.498039,0.054902}%
\pgfsetfillcolor{currentfill}%
\pgfsetlinewidth{1.003750pt}%
\definecolor{currentstroke}{rgb}{1.000000,0.498039,0.054902}%
\pgfsetstrokecolor{currentstroke}%
\pgfsetdash{}{0pt}%
\pgfpathmoveto{\pgfqpoint{3.835023in}{3.095464in}}%
\pgfpathcurveto{\pgfqpoint{3.846154in}{3.095464in}}{\pgfqpoint{3.856830in}{3.099886in}}{\pgfqpoint{3.864701in}{3.107757in}}%
\pgfpathcurveto{\pgfqpoint{3.872571in}{3.115627in}}{\pgfqpoint{3.876994in}{3.126304in}}{\pgfqpoint{3.876994in}{3.137434in}}%
\pgfpathcurveto{\pgfqpoint{3.876994in}{3.148565in}}{\pgfqpoint{3.872571in}{3.159241in}}{\pgfqpoint{3.864701in}{3.167112in}}%
\pgfpathcurveto{\pgfqpoint{3.856830in}{3.174982in}}{\pgfqpoint{3.846154in}{3.179405in}}{\pgfqpoint{3.835023in}{3.179405in}}%
\pgfpathcurveto{\pgfqpoint{3.823893in}{3.179405in}}{\pgfqpoint{3.813216in}{3.174982in}}{\pgfqpoint{3.805346in}{3.167112in}}%
\pgfpathcurveto{\pgfqpoint{3.797475in}{3.159241in}}{\pgfqpoint{3.793053in}{3.148565in}}{\pgfqpoint{3.793053in}{3.137434in}}%
\pgfpathcurveto{\pgfqpoint{3.793053in}{3.126304in}}{\pgfqpoint{3.797475in}{3.115627in}}{\pgfqpoint{3.805346in}{3.107757in}}%
\pgfpathcurveto{\pgfqpoint{3.813216in}{3.099886in}}{\pgfqpoint{3.823893in}{3.095464in}}{\pgfqpoint{3.835023in}{3.095464in}}%
\pgfpathclose%
\pgfusepath{stroke,fill}%
\end{pgfscope}%
\begin{pgfscope}%
\pgfpathrectangle{\pgfqpoint{1.065196in}{0.528000in}}{\pgfqpoint{3.702804in}{3.696000in}} %
\pgfusepath{clip}%
\pgfsetbuttcap%
\pgfsetroundjoin%
\definecolor{currentfill}{rgb}{1.000000,0.498039,0.054902}%
\pgfsetfillcolor{currentfill}%
\pgfsetlinewidth{1.003750pt}%
\definecolor{currentstroke}{rgb}{1.000000,0.498039,0.054902}%
\pgfsetstrokecolor{currentstroke}%
\pgfsetdash{}{0pt}%
\pgfpathmoveto{\pgfqpoint{2.125841in}{1.519484in}}%
\pgfpathcurveto{\pgfqpoint{2.133503in}{1.519484in}}{\pgfqpoint{2.140852in}{1.522528in}}{\pgfqpoint{2.146270in}{1.527946in}}%
\pgfpathcurveto{\pgfqpoint{2.151688in}{1.533363in}}{\pgfqpoint{2.154732in}{1.540713in}}{\pgfqpoint{2.154732in}{1.548375in}}%
\pgfpathcurveto{\pgfqpoint{2.154732in}{1.556036in}}{\pgfqpoint{2.151688in}{1.563386in}}{\pgfqpoint{2.146270in}{1.568803in}}%
\pgfpathcurveto{\pgfqpoint{2.140852in}{1.574221in}}{\pgfqpoint{2.133503in}{1.577265in}}{\pgfqpoint{2.125841in}{1.577265in}}%
\pgfpathcurveto{\pgfqpoint{2.118179in}{1.577265in}}{\pgfqpoint{2.110830in}{1.574221in}}{\pgfqpoint{2.105412in}{1.568803in}}%
\pgfpathcurveto{\pgfqpoint{2.099995in}{1.563386in}}{\pgfqpoint{2.096950in}{1.556036in}}{\pgfqpoint{2.096950in}{1.548375in}}%
\pgfpathcurveto{\pgfqpoint{2.096950in}{1.540713in}}{\pgfqpoint{2.099995in}{1.533363in}}{\pgfqpoint{2.105412in}{1.527946in}}%
\pgfpathcurveto{\pgfqpoint{2.110830in}{1.522528in}}{\pgfqpoint{2.118179in}{1.519484in}}{\pgfqpoint{2.125841in}{1.519484in}}%
\pgfpathclose%
\pgfusepath{stroke,fill}%
\end{pgfscope}%
\begin{pgfscope}%
\pgfpathrectangle{\pgfqpoint{1.065196in}{0.528000in}}{\pgfqpoint{3.702804in}{3.696000in}} %
\pgfusepath{clip}%
\pgfsetbuttcap%
\pgfsetroundjoin%
\definecolor{currentfill}{rgb}{1.000000,0.498039,0.054902}%
\pgfsetfillcolor{currentfill}%
\pgfsetlinewidth{1.003750pt}%
\definecolor{currentstroke}{rgb}{1.000000,0.498039,0.054902}%
\pgfsetstrokecolor{currentstroke}%
\pgfsetdash{}{0pt}%
\pgfpathmoveto{\pgfqpoint{3.373244in}{1.802431in}}%
\pgfpathcurveto{\pgfqpoint{3.384201in}{1.802431in}}{\pgfqpoint{3.394710in}{1.806784in}}{\pgfqpoint{3.402457in}{1.814531in}}%
\pgfpathcurveto{\pgfqpoint{3.410205in}{1.822278in}}{\pgfqpoint{3.414558in}{1.832788in}}{\pgfqpoint{3.414558in}{1.843744in}}%
\pgfpathcurveto{\pgfqpoint{3.414558in}{1.854700in}}{\pgfqpoint{3.410205in}{1.865209in}}{\pgfqpoint{3.402457in}{1.872957in}}%
\pgfpathcurveto{\pgfqpoint{3.394710in}{1.880704in}}{\pgfqpoint{3.384201in}{1.885057in}}{\pgfqpoint{3.373244in}{1.885057in}}%
\pgfpathcurveto{\pgfqpoint{3.362288in}{1.885057in}}{\pgfqpoint{3.351779in}{1.880704in}}{\pgfqpoint{3.344032in}{1.872957in}}%
\pgfpathcurveto{\pgfqpoint{3.336284in}{1.865209in}}{\pgfqpoint{3.331931in}{1.854700in}}{\pgfqpoint{3.331931in}{1.843744in}}%
\pgfpathcurveto{\pgfqpoint{3.331931in}{1.832788in}}{\pgfqpoint{3.336284in}{1.822278in}}{\pgfqpoint{3.344032in}{1.814531in}}%
\pgfpathcurveto{\pgfqpoint{3.351779in}{1.806784in}}{\pgfqpoint{3.362288in}{1.802431in}}{\pgfqpoint{3.373244in}{1.802431in}}%
\pgfpathclose%
\pgfusepath{stroke,fill}%
\end{pgfscope}%
\begin{pgfscope}%
\pgfpathrectangle{\pgfqpoint{1.065196in}{0.528000in}}{\pgfqpoint{3.702804in}{3.696000in}} %
\pgfusepath{clip}%
\pgfsetbuttcap%
\pgfsetroundjoin%
\definecolor{currentfill}{rgb}{1.000000,0.498039,0.054902}%
\pgfsetfillcolor{currentfill}%
\pgfsetlinewidth{1.003750pt}%
\definecolor{currentstroke}{rgb}{1.000000,0.498039,0.054902}%
\pgfsetstrokecolor{currentstroke}%
\pgfsetdash{}{0pt}%
\pgfpathmoveto{\pgfqpoint{3.923237in}{2.136139in}}%
\pgfpathcurveto{\pgfqpoint{3.935231in}{2.136139in}}{\pgfqpoint{3.946736in}{2.140904in}}{\pgfqpoint{3.955217in}{2.149385in}}%
\pgfpathcurveto{\pgfqpoint{3.963698in}{2.157866in}}{\pgfqpoint{3.968463in}{2.169370in}}{\pgfqpoint{3.968463in}{2.181364in}}%
\pgfpathcurveto{\pgfqpoint{3.968463in}{2.193359in}}{\pgfqpoint{3.963698in}{2.204863in}}{\pgfqpoint{3.955217in}{2.213344in}}%
\pgfpathcurveto{\pgfqpoint{3.946736in}{2.221825in}}{\pgfqpoint{3.935231in}{2.226590in}}{\pgfqpoint{3.923237in}{2.226590in}}%
\pgfpathcurveto{\pgfqpoint{3.911243in}{2.226590in}}{\pgfqpoint{3.899739in}{2.221825in}}{\pgfqpoint{3.891258in}{2.213344in}}%
\pgfpathcurveto{\pgfqpoint{3.882777in}{2.204863in}}{\pgfqpoint{3.878011in}{2.193359in}}{\pgfqpoint{3.878011in}{2.181364in}}%
\pgfpathcurveto{\pgfqpoint{3.878011in}{2.169370in}}{\pgfqpoint{3.882777in}{2.157866in}}{\pgfqpoint{3.891258in}{2.149385in}}%
\pgfpathcurveto{\pgfqpoint{3.899739in}{2.140904in}}{\pgfqpoint{3.911243in}{2.136139in}}{\pgfqpoint{3.923237in}{2.136139in}}%
\pgfpathclose%
\pgfusepath{stroke,fill}%
\end{pgfscope}%
\begin{pgfscope}%
\pgfpathrectangle{\pgfqpoint{1.065196in}{0.528000in}}{\pgfqpoint{3.702804in}{3.696000in}} %
\pgfusepath{clip}%
\pgfsetbuttcap%
\pgfsetroundjoin%
\definecolor{currentfill}{rgb}{1.000000,0.498039,0.054902}%
\pgfsetfillcolor{currentfill}%
\pgfsetlinewidth{1.003750pt}%
\definecolor{currentstroke}{rgb}{1.000000,0.498039,0.054902}%
\pgfsetstrokecolor{currentstroke}%
\pgfsetdash{}{0pt}%
\pgfpathmoveto{\pgfqpoint{3.876906in}{3.970894in}}%
\pgfpathcurveto{\pgfqpoint{3.885594in}{3.970894in}}{\pgfqpoint{3.893927in}{3.974345in}}{\pgfqpoint{3.900070in}{3.980488in}}%
\pgfpathcurveto{\pgfqpoint{3.906213in}{3.986632in}}{\pgfqpoint{3.909665in}{3.994965in}}{\pgfqpoint{3.909665in}{4.003652in}}%
\pgfpathcurveto{\pgfqpoint{3.909665in}{4.012340in}}{\pgfqpoint{3.906213in}{4.020673in}}{\pgfqpoint{3.900070in}{4.026816in}}%
\pgfpathcurveto{\pgfqpoint{3.893927in}{4.032959in}}{\pgfqpoint{3.885594in}{4.036411in}}{\pgfqpoint{3.876906in}{4.036411in}}%
\pgfpathcurveto{\pgfqpoint{3.868219in}{4.036411in}}{\pgfqpoint{3.859886in}{4.032959in}}{\pgfqpoint{3.853743in}{4.026816in}}%
\pgfpathcurveto{\pgfqpoint{3.847599in}{4.020673in}}{\pgfqpoint{3.844148in}{4.012340in}}{\pgfqpoint{3.844148in}{4.003652in}}%
\pgfpathcurveto{\pgfqpoint{3.844148in}{3.994965in}}{\pgfqpoint{3.847599in}{3.986632in}}{\pgfqpoint{3.853743in}{3.980488in}}%
\pgfpathcurveto{\pgfqpoint{3.859886in}{3.974345in}}{\pgfqpoint{3.868219in}{3.970894in}}{\pgfqpoint{3.876906in}{3.970894in}}%
\pgfpathclose%
\pgfusepath{stroke,fill}%
\end{pgfscope}%
\begin{pgfscope}%
\pgfpathrectangle{\pgfqpoint{1.065196in}{0.528000in}}{\pgfqpoint{3.702804in}{3.696000in}} %
\pgfusepath{clip}%
\pgfsetbuttcap%
\pgfsetroundjoin%
\definecolor{currentfill}{rgb}{1.000000,0.498039,0.054902}%
\pgfsetfillcolor{currentfill}%
\pgfsetlinewidth{1.003750pt}%
\definecolor{currentstroke}{rgb}{1.000000,0.498039,0.054902}%
\pgfsetstrokecolor{currentstroke}%
\pgfsetdash{}{0pt}%
\pgfpathmoveto{\pgfqpoint{2.261594in}{1.115465in}}%
\pgfpathcurveto{\pgfqpoint{2.275132in}{1.115465in}}{\pgfqpoint{2.288118in}{1.120844in}}{\pgfqpoint{2.297691in}{1.130417in}}%
\pgfpathcurveto{\pgfqpoint{2.307264in}{1.139990in}}{\pgfqpoint{2.312642in}{1.152976in}}{\pgfqpoint{2.312642in}{1.166514in}}%
\pgfpathcurveto{\pgfqpoint{2.312642in}{1.180052in}}{\pgfqpoint{2.307264in}{1.193038in}}{\pgfqpoint{2.297691in}{1.202611in}}%
\pgfpathcurveto{\pgfqpoint{2.288118in}{1.212184in}}{\pgfqpoint{2.275132in}{1.217562in}}{\pgfqpoint{2.261594in}{1.217562in}}%
\pgfpathcurveto{\pgfqpoint{2.248056in}{1.217562in}}{\pgfqpoint{2.235070in}{1.212184in}}{\pgfqpoint{2.225497in}{1.202611in}}%
\pgfpathcurveto{\pgfqpoint{2.215924in}{1.193038in}}{\pgfqpoint{2.210545in}{1.180052in}}{\pgfqpoint{2.210545in}{1.166514in}}%
\pgfpathcurveto{\pgfqpoint{2.210545in}{1.152976in}}{\pgfqpoint{2.215924in}{1.139990in}}{\pgfqpoint{2.225497in}{1.130417in}}%
\pgfpathcurveto{\pgfqpoint{2.235070in}{1.120844in}}{\pgfqpoint{2.248056in}{1.115465in}}{\pgfqpoint{2.261594in}{1.115465in}}%
\pgfpathclose%
\pgfusepath{stroke,fill}%
\end{pgfscope}%
\begin{pgfscope}%
\pgfpathrectangle{\pgfqpoint{1.065196in}{0.528000in}}{\pgfqpoint{3.702804in}{3.696000in}} %
\pgfusepath{clip}%
\pgfsetbuttcap%
\pgfsetroundjoin%
\definecolor{currentfill}{rgb}{1.000000,0.498039,0.054902}%
\pgfsetfillcolor{currentfill}%
\pgfsetlinewidth{1.003750pt}%
\definecolor{currentstroke}{rgb}{1.000000,0.498039,0.054902}%
\pgfsetstrokecolor{currentstroke}%
\pgfsetdash{}{0pt}%
\pgfpathmoveto{\pgfqpoint{4.273817in}{2.476418in}}%
\pgfpathcurveto{\pgfqpoint{4.280280in}{2.476418in}}{\pgfqpoint{4.286479in}{2.478986in}}{\pgfqpoint{4.291049in}{2.483556in}}%
\pgfpathcurveto{\pgfqpoint{4.295619in}{2.488126in}}{\pgfqpoint{4.298187in}{2.494325in}}{\pgfqpoint{4.298187in}{2.500788in}}%
\pgfpathcurveto{\pgfqpoint{4.298187in}{2.507251in}}{\pgfqpoint{4.295619in}{2.513450in}}{\pgfqpoint{4.291049in}{2.518020in}}%
\pgfpathcurveto{\pgfqpoint{4.286479in}{2.522590in}}{\pgfqpoint{4.280280in}{2.525157in}}{\pgfqpoint{4.273817in}{2.525157in}}%
\pgfpathcurveto{\pgfqpoint{4.267354in}{2.525157in}}{\pgfqpoint{4.261155in}{2.522590in}}{\pgfqpoint{4.256585in}{2.518020in}}%
\pgfpathcurveto{\pgfqpoint{4.252015in}{2.513450in}}{\pgfqpoint{4.249448in}{2.507251in}}{\pgfqpoint{4.249448in}{2.500788in}}%
\pgfpathcurveto{\pgfqpoint{4.249448in}{2.494325in}}{\pgfqpoint{4.252015in}{2.488126in}}{\pgfqpoint{4.256585in}{2.483556in}}%
\pgfpathcurveto{\pgfqpoint{4.261155in}{2.478986in}}{\pgfqpoint{4.267354in}{2.476418in}}{\pgfqpoint{4.273817in}{2.476418in}}%
\pgfpathclose%
\pgfusepath{stroke,fill}%
\end{pgfscope}%
\begin{pgfscope}%
\pgfpathrectangle{\pgfqpoint{1.065196in}{0.528000in}}{\pgfqpoint{3.702804in}{3.696000in}} %
\pgfusepath{clip}%
\pgfsetbuttcap%
\pgfsetroundjoin%
\definecolor{currentfill}{rgb}{1.000000,0.498039,0.054902}%
\pgfsetfillcolor{currentfill}%
\pgfsetlinewidth{1.003750pt}%
\definecolor{currentstroke}{rgb}{1.000000,0.498039,0.054902}%
\pgfsetstrokecolor{currentstroke}%
\pgfsetdash{}{0pt}%
\pgfpathmoveto{\pgfqpoint{4.519652in}{2.771276in}}%
\pgfpathcurveto{\pgfqpoint{4.532303in}{2.771276in}}{\pgfqpoint{4.544437in}{2.776302in}}{\pgfqpoint{4.553383in}{2.785248in}}%
\pgfpathcurveto{\pgfqpoint{4.562328in}{2.794193in}}{\pgfqpoint{4.567355in}{2.806328in}}{\pgfqpoint{4.567355in}{2.818979in}}%
\pgfpathcurveto{\pgfqpoint{4.567355in}{2.831629in}}{\pgfqpoint{4.562328in}{2.843764in}}{\pgfqpoint{4.553383in}{2.852709in}}%
\pgfpathcurveto{\pgfqpoint{4.544437in}{2.861655in}}{\pgfqpoint{4.532303in}{2.866681in}}{\pgfqpoint{4.519652in}{2.866681in}}%
\pgfpathcurveto{\pgfqpoint{4.507001in}{2.866681in}}{\pgfqpoint{4.494867in}{2.861655in}}{\pgfqpoint{4.485921in}{2.852709in}}%
\pgfpathcurveto{\pgfqpoint{4.476975in}{2.843764in}}{\pgfqpoint{4.471949in}{2.831629in}}{\pgfqpoint{4.471949in}{2.818979in}}%
\pgfpathcurveto{\pgfqpoint{4.471949in}{2.806328in}}{\pgfqpoint{4.476975in}{2.794193in}}{\pgfqpoint{4.485921in}{2.785248in}}%
\pgfpathcurveto{\pgfqpoint{4.494867in}{2.776302in}}{\pgfqpoint{4.507001in}{2.771276in}}{\pgfqpoint{4.519652in}{2.771276in}}%
\pgfpathclose%
\pgfusepath{stroke,fill}%
\end{pgfscope}%
\begin{pgfscope}%
\pgfpathrectangle{\pgfqpoint{1.065196in}{0.528000in}}{\pgfqpoint{3.702804in}{3.696000in}} %
\pgfusepath{clip}%
\pgfsetbuttcap%
\pgfsetroundjoin%
\definecolor{currentfill}{rgb}{1.000000,0.498039,0.054902}%
\pgfsetfillcolor{currentfill}%
\pgfsetlinewidth{1.003750pt}%
\definecolor{currentstroke}{rgb}{1.000000,0.498039,0.054902}%
\pgfsetstrokecolor{currentstroke}%
\pgfsetdash{}{0pt}%
\pgfpathmoveto{\pgfqpoint{4.583838in}{2.486195in}}%
\pgfpathcurveto{\pgfqpoint{4.591713in}{2.486195in}}{\pgfqpoint{4.599267in}{2.489324in}}{\pgfqpoint{4.604836in}{2.494893in}}%
\pgfpathcurveto{\pgfqpoint{4.610405in}{2.500462in}}{\pgfqpoint{4.613533in}{2.508016in}}{\pgfqpoint{4.613533in}{2.515891in}}%
\pgfpathcurveto{\pgfqpoint{4.613533in}{2.523766in}}{\pgfqpoint{4.610405in}{2.531320in}}{\pgfqpoint{4.604836in}{2.536889in}}%
\pgfpathcurveto{\pgfqpoint{4.599267in}{2.542458in}}{\pgfqpoint{4.591713in}{2.545587in}}{\pgfqpoint{4.583838in}{2.545587in}}%
\pgfpathcurveto{\pgfqpoint{4.575963in}{2.545587in}}{\pgfqpoint{4.568409in}{2.542458in}}{\pgfqpoint{4.562840in}{2.536889in}}%
\pgfpathcurveto{\pgfqpoint{4.557271in}{2.531320in}}{\pgfqpoint{4.554142in}{2.523766in}}{\pgfqpoint{4.554142in}{2.515891in}}%
\pgfpathcurveto{\pgfqpoint{4.554142in}{2.508016in}}{\pgfqpoint{4.557271in}{2.500462in}}{\pgfqpoint{4.562840in}{2.494893in}}%
\pgfpathcurveto{\pgfqpoint{4.568409in}{2.489324in}}{\pgfqpoint{4.575963in}{2.486195in}}{\pgfqpoint{4.583838in}{2.486195in}}%
\pgfpathclose%
\pgfusepath{stroke,fill}%
\end{pgfscope}%
\begin{pgfscope}%
\pgfpathrectangle{\pgfqpoint{1.065196in}{0.528000in}}{\pgfqpoint{3.702804in}{3.696000in}} %
\pgfusepath{clip}%
\pgfsetbuttcap%
\pgfsetroundjoin%
\definecolor{currentfill}{rgb}{1.000000,0.498039,0.054902}%
\pgfsetfillcolor{currentfill}%
\pgfsetlinewidth{1.003750pt}%
\definecolor{currentstroke}{rgb}{1.000000,0.498039,0.054902}%
\pgfsetstrokecolor{currentstroke}%
\pgfsetdash{}{0pt}%
\pgfpathmoveto{\pgfqpoint{3.018789in}{1.092604in}}%
\pgfpathcurveto{\pgfqpoint{3.031662in}{1.092604in}}{\pgfqpoint{3.044010in}{1.097719in}}{\pgfqpoint{3.053112in}{1.106821in}}%
\pgfpathcurveto{\pgfqpoint{3.062215in}{1.115924in}}{\pgfqpoint{3.067330in}{1.128272in}}{\pgfqpoint{3.067330in}{1.141145in}}%
\pgfpathcurveto{\pgfqpoint{3.067330in}{1.154018in}}{\pgfqpoint{3.062215in}{1.166365in}}{\pgfqpoint{3.053112in}{1.175468in}}%
\pgfpathcurveto{\pgfqpoint{3.044010in}{1.184571in}}{\pgfqpoint{3.031662in}{1.189685in}}{\pgfqpoint{3.018789in}{1.189685in}}%
\pgfpathcurveto{\pgfqpoint{3.005916in}{1.189685in}}{\pgfqpoint{2.993568in}{1.184571in}}{\pgfqpoint{2.984466in}{1.175468in}}%
\pgfpathcurveto{\pgfqpoint{2.975363in}{1.166365in}}{\pgfqpoint{2.970249in}{1.154018in}}{\pgfqpoint{2.970249in}{1.141145in}}%
\pgfpathcurveto{\pgfqpoint{2.970249in}{1.128272in}}{\pgfqpoint{2.975363in}{1.115924in}}{\pgfqpoint{2.984466in}{1.106821in}}%
\pgfpathcurveto{\pgfqpoint{2.993568in}{1.097719in}}{\pgfqpoint{3.005916in}{1.092604in}}{\pgfqpoint{3.018789in}{1.092604in}}%
\pgfpathclose%
\pgfusepath{stroke,fill}%
\end{pgfscope}%
\begin{pgfscope}%
\pgfpathrectangle{\pgfqpoint{1.065196in}{0.528000in}}{\pgfqpoint{3.702804in}{3.696000in}} %
\pgfusepath{clip}%
\pgfsetbuttcap%
\pgfsetroundjoin%
\definecolor{currentfill}{rgb}{1.000000,0.498039,0.054902}%
\pgfsetfillcolor{currentfill}%
\pgfsetlinewidth{1.003750pt}%
\definecolor{currentstroke}{rgb}{1.000000,0.498039,0.054902}%
\pgfsetstrokecolor{currentstroke}%
\pgfsetdash{}{0pt}%
\pgfpathmoveto{\pgfqpoint{2.447252in}{0.739445in}}%
\pgfpathcurveto{\pgfqpoint{2.456823in}{0.739445in}}{\pgfqpoint{2.466003in}{0.743248in}}{\pgfqpoint{2.472771in}{0.750016in}}%
\pgfpathcurveto{\pgfqpoint{2.479539in}{0.756783in}}{\pgfqpoint{2.483341in}{0.765964in}}{\pgfqpoint{2.483341in}{0.775535in}}%
\pgfpathcurveto{\pgfqpoint{2.483341in}{0.785106in}}{\pgfqpoint{2.479539in}{0.794286in}}{\pgfqpoint{2.472771in}{0.801054in}}%
\pgfpathcurveto{\pgfqpoint{2.466003in}{0.807821in}}{\pgfqpoint{2.456823in}{0.811624in}}{\pgfqpoint{2.447252in}{0.811624in}}%
\pgfpathcurveto{\pgfqpoint{2.437681in}{0.811624in}}{\pgfqpoint{2.428501in}{0.807821in}}{\pgfqpoint{2.421733in}{0.801054in}}%
\pgfpathcurveto{\pgfqpoint{2.414965in}{0.794286in}}{\pgfqpoint{2.411163in}{0.785106in}}{\pgfqpoint{2.411163in}{0.775535in}}%
\pgfpathcurveto{\pgfqpoint{2.411163in}{0.765964in}}{\pgfqpoint{2.414965in}{0.756783in}}{\pgfqpoint{2.421733in}{0.750016in}}%
\pgfpathcurveto{\pgfqpoint{2.428501in}{0.743248in}}{\pgfqpoint{2.437681in}{0.739445in}}{\pgfqpoint{2.447252in}{0.739445in}}%
\pgfpathclose%
\pgfusepath{stroke,fill}%
\end{pgfscope}%
\begin{pgfscope}%
\pgfpathrectangle{\pgfqpoint{1.065196in}{0.528000in}}{\pgfqpoint{3.702804in}{3.696000in}} %
\pgfusepath{clip}%
\pgfsetbuttcap%
\pgfsetroundjoin%
\definecolor{currentfill}{rgb}{1.000000,0.498039,0.054902}%
\pgfsetfillcolor{currentfill}%
\pgfsetlinewidth{1.003750pt}%
\definecolor{currentstroke}{rgb}{1.000000,0.498039,0.054902}%
\pgfsetstrokecolor{currentstroke}%
\pgfsetdash{}{0pt}%
\pgfpathmoveto{\pgfqpoint{1.793395in}{3.134405in}}%
\pgfpathcurveto{\pgfqpoint{1.806547in}{3.134405in}}{\pgfqpoint{1.819162in}{3.139630in}}{\pgfqpoint{1.828462in}{3.148930in}}%
\pgfpathcurveto{\pgfqpoint{1.837762in}{3.158230in}}{\pgfqpoint{1.842987in}{3.170845in}}{\pgfqpoint{1.842987in}{3.183997in}}%
\pgfpathcurveto{\pgfqpoint{1.842987in}{3.197149in}}{\pgfqpoint{1.837762in}{3.209765in}}{\pgfqpoint{1.828462in}{3.219065in}}%
\pgfpathcurveto{\pgfqpoint{1.819162in}{3.228364in}}{\pgfqpoint{1.806547in}{3.233590in}}{\pgfqpoint{1.793395in}{3.233590in}}%
\pgfpathcurveto{\pgfqpoint{1.780243in}{3.233590in}}{\pgfqpoint{1.767628in}{3.228364in}}{\pgfqpoint{1.758328in}{3.219065in}}%
\pgfpathcurveto{\pgfqpoint{1.749028in}{3.209765in}}{\pgfqpoint{1.743803in}{3.197149in}}{\pgfqpoint{1.743803in}{3.183997in}}%
\pgfpathcurveto{\pgfqpoint{1.743803in}{3.170845in}}{\pgfqpoint{1.749028in}{3.158230in}}{\pgfqpoint{1.758328in}{3.148930in}}%
\pgfpathcurveto{\pgfqpoint{1.767628in}{3.139630in}}{\pgfqpoint{1.780243in}{3.134405in}}{\pgfqpoint{1.793395in}{3.134405in}}%
\pgfpathclose%
\pgfusepath{stroke,fill}%
\end{pgfscope}%
\begin{pgfscope}%
\pgfpathrectangle{\pgfqpoint{1.065196in}{0.528000in}}{\pgfqpoint{3.702804in}{3.696000in}} %
\pgfusepath{clip}%
\pgfsetbuttcap%
\pgfsetroundjoin%
\definecolor{currentfill}{rgb}{1.000000,0.498039,0.054902}%
\pgfsetfillcolor{currentfill}%
\pgfsetlinewidth{1.003750pt}%
\definecolor{currentstroke}{rgb}{1.000000,0.498039,0.054902}%
\pgfsetstrokecolor{currentstroke}%
\pgfsetdash{}{0pt}%
\pgfpathmoveto{\pgfqpoint{1.358197in}{1.882060in}}%
\pgfpathcurveto{\pgfqpoint{1.366899in}{1.882060in}}{\pgfqpoint{1.375245in}{1.885517in}}{\pgfqpoint{1.381397in}{1.891670in}}%
\pgfpathcurveto{\pgfqpoint{1.387550in}{1.897822in}}{\pgfqpoint{1.391007in}{1.906169in}}{\pgfqpoint{1.391007in}{1.914870in}}%
\pgfpathcurveto{\pgfqpoint{1.391007in}{1.923571in}}{\pgfqpoint{1.387550in}{1.931917in}}{\pgfqpoint{1.381397in}{1.938070in}}%
\pgfpathcurveto{\pgfqpoint{1.375245in}{1.944223in}}{\pgfqpoint{1.366899in}{1.947680in}}{\pgfqpoint{1.358197in}{1.947680in}}%
\pgfpathcurveto{\pgfqpoint{1.349496in}{1.947680in}}{\pgfqpoint{1.341150in}{1.944223in}}{\pgfqpoint{1.334997in}{1.938070in}}%
\pgfpathcurveto{\pgfqpoint{1.328844in}{1.931917in}}{\pgfqpoint{1.325387in}{1.923571in}}{\pgfqpoint{1.325387in}{1.914870in}}%
\pgfpathcurveto{\pgfqpoint{1.325387in}{1.906169in}}{\pgfqpoint{1.328844in}{1.897822in}}{\pgfqpoint{1.334997in}{1.891670in}}%
\pgfpathcurveto{\pgfqpoint{1.341150in}{1.885517in}}{\pgfqpoint{1.349496in}{1.882060in}}{\pgfqpoint{1.358197in}{1.882060in}}%
\pgfpathclose%
\pgfusepath{stroke,fill}%
\end{pgfscope}%
\begin{pgfscope}%
\pgfpathrectangle{\pgfqpoint{1.065196in}{0.528000in}}{\pgfqpoint{3.702804in}{3.696000in}} %
\pgfusepath{clip}%
\pgfsetbuttcap%
\pgfsetroundjoin%
\definecolor{currentfill}{rgb}{1.000000,0.498039,0.054902}%
\pgfsetfillcolor{currentfill}%
\pgfsetlinewidth{1.003750pt}%
\definecolor{currentstroke}{rgb}{1.000000,0.498039,0.054902}%
\pgfsetstrokecolor{currentstroke}%
\pgfsetdash{}{0pt}%
\pgfpathmoveto{\pgfqpoint{4.143380in}{2.981192in}}%
\pgfpathcurveto{\pgfqpoint{4.150145in}{2.981192in}}{\pgfqpoint{4.156634in}{2.983879in}}{\pgfqpoint{4.161417in}{2.988663in}}%
\pgfpathcurveto{\pgfqpoint{4.166201in}{2.993447in}}{\pgfqpoint{4.168888in}{2.999935in}}{\pgfqpoint{4.168888in}{3.006700in}}%
\pgfpathcurveto{\pgfqpoint{4.168888in}{3.013465in}}{\pgfqpoint{4.166201in}{3.019954in}}{\pgfqpoint{4.161417in}{3.024737in}}%
\pgfpathcurveto{\pgfqpoint{4.156634in}{3.029521in}}{\pgfqpoint{4.150145in}{3.032209in}}{\pgfqpoint{4.143380in}{3.032209in}}%
\pgfpathcurveto{\pgfqpoint{4.136615in}{3.032209in}}{\pgfqpoint{4.130126in}{3.029521in}}{\pgfqpoint{4.125343in}{3.024737in}}%
\pgfpathcurveto{\pgfqpoint{4.120559in}{3.019954in}}{\pgfqpoint{4.117871in}{3.013465in}}{\pgfqpoint{4.117871in}{3.006700in}}%
\pgfpathcurveto{\pgfqpoint{4.117871in}{2.999935in}}{\pgfqpoint{4.120559in}{2.993447in}}{\pgfqpoint{4.125343in}{2.988663in}}%
\pgfpathcurveto{\pgfqpoint{4.130126in}{2.983879in}}{\pgfqpoint{4.136615in}{2.981192in}}{\pgfqpoint{4.143380in}{2.981192in}}%
\pgfpathclose%
\pgfusepath{stroke,fill}%
\end{pgfscope}%
\begin{pgfscope}%
\pgfpathrectangle{\pgfqpoint{1.065196in}{0.528000in}}{\pgfqpoint{3.702804in}{3.696000in}} %
\pgfusepath{clip}%
\pgfsetbuttcap%
\pgfsetroundjoin%
\definecolor{currentfill}{rgb}{1.000000,0.498039,0.054902}%
\pgfsetfillcolor{currentfill}%
\pgfsetlinewidth{1.003750pt}%
\definecolor{currentstroke}{rgb}{1.000000,0.498039,0.054902}%
\pgfsetstrokecolor{currentstroke}%
\pgfsetdash{}{0pt}%
\pgfpathmoveto{\pgfqpoint{3.569555in}{1.270941in}}%
\pgfpathcurveto{\pgfqpoint{3.582374in}{1.270941in}}{\pgfqpoint{3.594669in}{1.276034in}}{\pgfqpoint{3.603733in}{1.285098in}}%
\pgfpathcurveto{\pgfqpoint{3.612798in}{1.294163in}}{\pgfqpoint{3.617891in}{1.306458in}}{\pgfqpoint{3.617891in}{1.319277in}}%
\pgfpathcurveto{\pgfqpoint{3.617891in}{1.332096in}}{\pgfqpoint{3.612798in}{1.344391in}}{\pgfqpoint{3.603733in}{1.353455in}}%
\pgfpathcurveto{\pgfqpoint{3.594669in}{1.362520in}}{\pgfqpoint{3.582374in}{1.367613in}}{\pgfqpoint{3.569555in}{1.367613in}}%
\pgfpathcurveto{\pgfqpoint{3.556736in}{1.367613in}}{\pgfqpoint{3.544441in}{1.362520in}}{\pgfqpoint{3.535376in}{1.353455in}}%
\pgfpathcurveto{\pgfqpoint{3.526312in}{1.344391in}}{\pgfqpoint{3.521219in}{1.332096in}}{\pgfqpoint{3.521219in}{1.319277in}}%
\pgfpathcurveto{\pgfqpoint{3.521219in}{1.306458in}}{\pgfqpoint{3.526312in}{1.294163in}}{\pgfqpoint{3.535376in}{1.285098in}}%
\pgfpathcurveto{\pgfqpoint{3.544441in}{1.276034in}}{\pgfqpoint{3.556736in}{1.270941in}}{\pgfqpoint{3.569555in}{1.270941in}}%
\pgfpathclose%
\pgfusepath{stroke,fill}%
\end{pgfscope}%
\begin{pgfscope}%
\pgfpathrectangle{\pgfqpoint{1.065196in}{0.528000in}}{\pgfqpoint{3.702804in}{3.696000in}} %
\pgfusepath{clip}%
\pgfsetbuttcap%
\pgfsetroundjoin%
\definecolor{currentfill}{rgb}{1.000000,0.498039,0.054902}%
\pgfsetfillcolor{currentfill}%
\pgfsetlinewidth{1.003750pt}%
\definecolor{currentstroke}{rgb}{1.000000,0.498039,0.054902}%
\pgfsetstrokecolor{currentstroke}%
\pgfsetdash{}{0pt}%
\pgfpathmoveto{\pgfqpoint{2.735828in}{2.604104in}}%
\pgfpathcurveto{\pgfqpoint{2.743958in}{2.604104in}}{\pgfqpoint{2.751757in}{2.607334in}}{\pgfqpoint{2.757506in}{2.613083in}}%
\pgfpathcurveto{\pgfqpoint{2.763254in}{2.618831in}}{\pgfqpoint{2.766485in}{2.626630in}}{\pgfqpoint{2.766485in}{2.634760in}}%
\pgfpathcurveto{\pgfqpoint{2.766485in}{2.642890in}}{\pgfqpoint{2.763254in}{2.650688in}}{\pgfqpoint{2.757506in}{2.656437in}}%
\pgfpathcurveto{\pgfqpoint{2.751757in}{2.662186in}}{\pgfqpoint{2.743958in}{2.665416in}}{\pgfqpoint{2.735828in}{2.665416in}}%
\pgfpathcurveto{\pgfqpoint{2.727698in}{2.665416in}}{\pgfqpoint{2.719900in}{2.662186in}}{\pgfqpoint{2.714151in}{2.656437in}}%
\pgfpathcurveto{\pgfqpoint{2.708402in}{2.650688in}}{\pgfqpoint{2.705172in}{2.642890in}}{\pgfqpoint{2.705172in}{2.634760in}}%
\pgfpathcurveto{\pgfqpoint{2.705172in}{2.626630in}}{\pgfqpoint{2.708402in}{2.618831in}}{\pgfqpoint{2.714151in}{2.613083in}}%
\pgfpathcurveto{\pgfqpoint{2.719900in}{2.607334in}}{\pgfqpoint{2.727698in}{2.604104in}}{\pgfqpoint{2.735828in}{2.604104in}}%
\pgfpathclose%
\pgfusepath{stroke,fill}%
\end{pgfscope}%
\begin{pgfscope}%
\pgfpathrectangle{\pgfqpoint{1.065196in}{0.528000in}}{\pgfqpoint{3.702804in}{3.696000in}} %
\pgfusepath{clip}%
\pgfsetbuttcap%
\pgfsetroundjoin%
\definecolor{currentfill}{rgb}{1.000000,0.498039,0.054902}%
\pgfsetfillcolor{currentfill}%
\pgfsetlinewidth{1.003750pt}%
\definecolor{currentstroke}{rgb}{1.000000,0.498039,0.054902}%
\pgfsetstrokecolor{currentstroke}%
\pgfsetdash{}{0pt}%
\pgfpathmoveto{\pgfqpoint{4.569907in}{1.323144in}}%
\pgfpathcurveto{\pgfqpoint{4.582438in}{1.323144in}}{\pgfqpoint{4.594458in}{1.328123in}}{\pgfqpoint{4.603319in}{1.336984in}}%
\pgfpathcurveto{\pgfqpoint{4.612179in}{1.345845in}}{\pgfqpoint{4.617158in}{1.357864in}}{\pgfqpoint{4.617158in}{1.370396in}}%
\pgfpathcurveto{\pgfqpoint{4.617158in}{1.382927in}}{\pgfqpoint{4.612179in}{1.394947in}}{\pgfqpoint{4.603319in}{1.403807in}}%
\pgfpathcurveto{\pgfqpoint{4.594458in}{1.412668in}}{\pgfqpoint{4.582438in}{1.417647in}}{\pgfqpoint{4.569907in}{1.417647in}}%
\pgfpathcurveto{\pgfqpoint{4.557376in}{1.417647in}}{\pgfqpoint{4.545356in}{1.412668in}}{\pgfqpoint{4.536495in}{1.403807in}}%
\pgfpathcurveto{\pgfqpoint{4.527634in}{1.394947in}}{\pgfqpoint{4.522655in}{1.382927in}}{\pgfqpoint{4.522655in}{1.370396in}}%
\pgfpathcurveto{\pgfqpoint{4.522655in}{1.357864in}}{\pgfqpoint{4.527634in}{1.345845in}}{\pgfqpoint{4.536495in}{1.336984in}}%
\pgfpathcurveto{\pgfqpoint{4.545356in}{1.328123in}}{\pgfqpoint{4.557376in}{1.323144in}}{\pgfqpoint{4.569907in}{1.323144in}}%
\pgfpathclose%
\pgfusepath{stroke,fill}%
\end{pgfscope}%
\begin{pgfscope}%
\pgfpathrectangle{\pgfqpoint{1.065196in}{0.528000in}}{\pgfqpoint{3.702804in}{3.696000in}} %
\pgfusepath{clip}%
\pgfsetbuttcap%
\pgfsetroundjoin%
\definecolor{currentfill}{rgb}{1.000000,0.498039,0.054902}%
\pgfsetfillcolor{currentfill}%
\pgfsetlinewidth{1.003750pt}%
\definecolor{currentstroke}{rgb}{1.000000,0.498039,0.054902}%
\pgfsetstrokecolor{currentstroke}%
\pgfsetdash{}{0pt}%
\pgfpathmoveto{\pgfqpoint{2.085783in}{1.517849in}}%
\pgfpathcurveto{\pgfqpoint{2.098410in}{1.517849in}}{\pgfqpoint{2.110521in}{1.522866in}}{\pgfqpoint{2.119450in}{1.531794in}}%
\pgfpathcurveto{\pgfqpoint{2.128378in}{1.540723in}}{\pgfqpoint{2.133395in}{1.552834in}}{\pgfqpoint{2.133395in}{1.565461in}}%
\pgfpathcurveto{\pgfqpoint{2.133395in}{1.578088in}}{\pgfqpoint{2.128378in}{1.590199in}}{\pgfqpoint{2.119450in}{1.599128in}}%
\pgfpathcurveto{\pgfqpoint{2.110521in}{1.608056in}}{\pgfqpoint{2.098410in}{1.613073in}}{\pgfqpoint{2.085783in}{1.613073in}}%
\pgfpathcurveto{\pgfqpoint{2.073156in}{1.613073in}}{\pgfqpoint{2.061045in}{1.608056in}}{\pgfqpoint{2.052117in}{1.599128in}}%
\pgfpathcurveto{\pgfqpoint{2.043188in}{1.590199in}}{\pgfqpoint{2.038172in}{1.578088in}}{\pgfqpoint{2.038172in}{1.565461in}}%
\pgfpathcurveto{\pgfqpoint{2.038172in}{1.552834in}}{\pgfqpoint{2.043188in}{1.540723in}}{\pgfqpoint{2.052117in}{1.531794in}}%
\pgfpathcurveto{\pgfqpoint{2.061045in}{1.522866in}}{\pgfqpoint{2.073156in}{1.517849in}}{\pgfqpoint{2.085783in}{1.517849in}}%
\pgfpathclose%
\pgfusepath{stroke,fill}%
\end{pgfscope}%
\begin{pgfscope}%
\pgfpathrectangle{\pgfqpoint{1.065196in}{0.528000in}}{\pgfqpoint{3.702804in}{3.696000in}} %
\pgfusepath{clip}%
\pgfsetbuttcap%
\pgfsetroundjoin%
\definecolor{currentfill}{rgb}{1.000000,0.498039,0.054902}%
\pgfsetfillcolor{currentfill}%
\pgfsetlinewidth{1.003750pt}%
\definecolor{currentstroke}{rgb}{1.000000,0.498039,0.054902}%
\pgfsetstrokecolor{currentstroke}%
\pgfsetdash{}{0pt}%
\pgfpathmoveto{\pgfqpoint{3.767845in}{2.191654in}}%
\pgfpathcurveto{\pgfqpoint{3.774731in}{2.191654in}}{\pgfqpoint{3.781336in}{2.194390in}}{\pgfqpoint{3.786205in}{2.199259in}}%
\pgfpathcurveto{\pgfqpoint{3.791074in}{2.204128in}}{\pgfqpoint{3.793810in}{2.210732in}}{\pgfqpoint{3.793810in}{2.217618in}}%
\pgfpathcurveto{\pgfqpoint{3.793810in}{2.224504in}}{\pgfqpoint{3.791074in}{2.231109in}}{\pgfqpoint{3.786205in}{2.235978in}}%
\pgfpathcurveto{\pgfqpoint{3.781336in}{2.240846in}}{\pgfqpoint{3.774731in}{2.243582in}}{\pgfqpoint{3.767845in}{2.243582in}}%
\pgfpathcurveto{\pgfqpoint{3.760960in}{2.243582in}}{\pgfqpoint{3.754355in}{2.240846in}}{\pgfqpoint{3.749486in}{2.235978in}}%
\pgfpathcurveto{\pgfqpoint{3.744617in}{2.231109in}}{\pgfqpoint{3.741881in}{2.224504in}}{\pgfqpoint{3.741881in}{2.217618in}}%
\pgfpathcurveto{\pgfqpoint{3.741881in}{2.210732in}}{\pgfqpoint{3.744617in}{2.204128in}}{\pgfqpoint{3.749486in}{2.199259in}}%
\pgfpathcurveto{\pgfqpoint{3.754355in}{2.194390in}}{\pgfqpoint{3.760960in}{2.191654in}}{\pgfqpoint{3.767845in}{2.191654in}}%
\pgfpathclose%
\pgfusepath{stroke,fill}%
\end{pgfscope}%
\begin{pgfscope}%
\pgfpathrectangle{\pgfqpoint{1.065196in}{0.528000in}}{\pgfqpoint{3.702804in}{3.696000in}} %
\pgfusepath{clip}%
\pgfsetbuttcap%
\pgfsetroundjoin%
\definecolor{currentfill}{rgb}{1.000000,0.498039,0.054902}%
\pgfsetfillcolor{currentfill}%
\pgfsetlinewidth{1.003750pt}%
\definecolor{currentstroke}{rgb}{1.000000,0.498039,0.054902}%
\pgfsetstrokecolor{currentstroke}%
\pgfsetdash{}{0pt}%
\pgfpathmoveto{\pgfqpoint{1.447013in}{2.364633in}}%
\pgfpathcurveto{\pgfqpoint{1.453785in}{2.364633in}}{\pgfqpoint{1.460280in}{2.367323in}}{\pgfqpoint{1.465068in}{2.372111in}}%
\pgfpathcurveto{\pgfqpoint{1.469856in}{2.376899in}}{\pgfqpoint{1.472547in}{2.383395in}}{\pgfqpoint{1.472547in}{2.390166in}}%
\pgfpathcurveto{\pgfqpoint{1.472547in}{2.396938in}}{\pgfqpoint{1.469856in}{2.403433in}}{\pgfqpoint{1.465068in}{2.408221in}}%
\pgfpathcurveto{\pgfqpoint{1.460280in}{2.413009in}}{\pgfqpoint{1.453785in}{2.415700in}}{\pgfqpoint{1.447013in}{2.415700in}}%
\pgfpathcurveto{\pgfqpoint{1.440242in}{2.415700in}}{\pgfqpoint{1.433747in}{2.413009in}}{\pgfqpoint{1.428958in}{2.408221in}}%
\pgfpathcurveto{\pgfqpoint{1.424170in}{2.403433in}}{\pgfqpoint{1.421480in}{2.396938in}}{\pgfqpoint{1.421480in}{2.390166in}}%
\pgfpathcurveto{\pgfqpoint{1.421480in}{2.383395in}}{\pgfqpoint{1.424170in}{2.376899in}}{\pgfqpoint{1.428958in}{2.372111in}}%
\pgfpathcurveto{\pgfqpoint{1.433747in}{2.367323in}}{\pgfqpoint{1.440242in}{2.364633in}}{\pgfqpoint{1.447013in}{2.364633in}}%
\pgfpathclose%
\pgfusepath{stroke,fill}%
\end{pgfscope}%
\begin{pgfscope}%
\pgfpathrectangle{\pgfqpoint{1.065196in}{0.528000in}}{\pgfqpoint{3.702804in}{3.696000in}} %
\pgfusepath{clip}%
\pgfsetbuttcap%
\pgfsetroundjoin%
\definecolor{currentfill}{rgb}{1.000000,0.498039,0.054902}%
\pgfsetfillcolor{currentfill}%
\pgfsetlinewidth{1.003750pt}%
\definecolor{currentstroke}{rgb}{1.000000,0.498039,0.054902}%
\pgfsetstrokecolor{currentstroke}%
\pgfsetdash{}{0pt}%
\pgfpathmoveto{\pgfqpoint{1.966024in}{3.336707in}}%
\pgfpathcurveto{\pgfqpoint{1.972551in}{3.336707in}}{\pgfqpoint{1.978812in}{3.339301in}}{\pgfqpoint{1.983427in}{3.343916in}}%
\pgfpathcurveto{\pgfqpoint{1.988043in}{3.348532in}}{\pgfqpoint{1.990636in}{3.354793in}}{\pgfqpoint{1.990636in}{3.361320in}}%
\pgfpathcurveto{\pgfqpoint{1.990636in}{3.367847in}}{\pgfqpoint{1.988043in}{3.374108in}}{\pgfqpoint{1.983427in}{3.378724in}}%
\pgfpathcurveto{\pgfqpoint{1.978812in}{3.383339in}}{\pgfqpoint{1.972551in}{3.385933in}}{\pgfqpoint{1.966024in}{3.385933in}}%
\pgfpathcurveto{\pgfqpoint{1.959496in}{3.385933in}}{\pgfqpoint{1.953235in}{3.383339in}}{\pgfqpoint{1.948620in}{3.378724in}}%
\pgfpathcurveto{\pgfqpoint{1.944004in}{3.374108in}}{\pgfqpoint{1.941411in}{3.367847in}}{\pgfqpoint{1.941411in}{3.361320in}}%
\pgfpathcurveto{\pgfqpoint{1.941411in}{3.354793in}}{\pgfqpoint{1.944004in}{3.348532in}}{\pgfqpoint{1.948620in}{3.343916in}}%
\pgfpathcurveto{\pgfqpoint{1.953235in}{3.339301in}}{\pgfqpoint{1.959496in}{3.336707in}}{\pgfqpoint{1.966024in}{3.336707in}}%
\pgfpathclose%
\pgfusepath{stroke,fill}%
\end{pgfscope}%
\begin{pgfscope}%
\pgfpathrectangle{\pgfqpoint{1.065196in}{0.528000in}}{\pgfqpoint{3.702804in}{3.696000in}} %
\pgfusepath{clip}%
\pgfsetbuttcap%
\pgfsetroundjoin%
\definecolor{currentfill}{rgb}{1.000000,0.498039,0.054902}%
\pgfsetfillcolor{currentfill}%
\pgfsetlinewidth{1.003750pt}%
\definecolor{currentstroke}{rgb}{1.000000,0.498039,0.054902}%
\pgfsetstrokecolor{currentstroke}%
\pgfsetdash{}{0pt}%
\pgfpathmoveto{\pgfqpoint{2.251829in}{0.759348in}}%
\pgfpathcurveto{\pgfqpoint{2.257353in}{0.759348in}}{\pgfqpoint{2.262652in}{0.761543in}}{\pgfqpoint{2.266559in}{0.765449in}}%
\pgfpathcurveto{\pgfqpoint{2.270465in}{0.769356in}}{\pgfqpoint{2.272660in}{0.774655in}}{\pgfqpoint{2.272660in}{0.780180in}}%
\pgfpathcurveto{\pgfqpoint{2.272660in}{0.785704in}}{\pgfqpoint{2.270465in}{0.791003in}}{\pgfqpoint{2.266559in}{0.794910in}}%
\pgfpathcurveto{\pgfqpoint{2.262652in}{0.798816in}}{\pgfqpoint{2.257353in}{0.801011in}}{\pgfqpoint{2.251829in}{0.801011in}}%
\pgfpathcurveto{\pgfqpoint{2.246304in}{0.801011in}}{\pgfqpoint{2.241005in}{0.798816in}}{\pgfqpoint{2.237098in}{0.794910in}}%
\pgfpathcurveto{\pgfqpoint{2.233192in}{0.791003in}}{\pgfqpoint{2.230997in}{0.785704in}}{\pgfqpoint{2.230997in}{0.780180in}}%
\pgfpathcurveto{\pgfqpoint{2.230997in}{0.774655in}}{\pgfqpoint{2.233192in}{0.769356in}}{\pgfqpoint{2.237098in}{0.765449in}}%
\pgfpathcurveto{\pgfqpoint{2.241005in}{0.761543in}}{\pgfqpoint{2.246304in}{0.759348in}}{\pgfqpoint{2.251829in}{0.759348in}}%
\pgfpathclose%
\pgfusepath{stroke,fill}%
\end{pgfscope}%
\begin{pgfscope}%
\pgfpathrectangle{\pgfqpoint{1.065196in}{0.528000in}}{\pgfqpoint{3.702804in}{3.696000in}} %
\pgfusepath{clip}%
\pgfsetbuttcap%
\pgfsetroundjoin%
\definecolor{currentfill}{rgb}{1.000000,0.498039,0.054902}%
\pgfsetfillcolor{currentfill}%
\pgfsetlinewidth{1.003750pt}%
\definecolor{currentstroke}{rgb}{1.000000,0.498039,0.054902}%
\pgfsetstrokecolor{currentstroke}%
\pgfsetdash{}{0pt}%
\pgfpathmoveto{\pgfqpoint{3.243113in}{3.478876in}}%
\pgfpathcurveto{\pgfqpoint{3.252099in}{3.478876in}}{\pgfqpoint{3.260719in}{3.482447in}}{\pgfqpoint{3.267073in}{3.488801in}}%
\pgfpathcurveto{\pgfqpoint{3.273428in}{3.495155in}}{\pgfqpoint{3.276998in}{3.503775in}}{\pgfqpoint{3.276998in}{3.512761in}}%
\pgfpathcurveto{\pgfqpoint{3.276998in}{3.521748in}}{\pgfqpoint{3.273428in}{3.530367in}}{\pgfqpoint{3.267073in}{3.536722in}}%
\pgfpathcurveto{\pgfqpoint{3.260719in}{3.543076in}}{\pgfqpoint{3.252099in}{3.546646in}}{\pgfqpoint{3.243113in}{3.546646in}}%
\pgfpathcurveto{\pgfqpoint{3.234126in}{3.546646in}}{\pgfqpoint{3.225507in}{3.543076in}}{\pgfqpoint{3.219153in}{3.536722in}}%
\pgfpathcurveto{\pgfqpoint{3.212798in}{3.530367in}}{\pgfqpoint{3.209228in}{3.521748in}}{\pgfqpoint{3.209228in}{3.512761in}}%
\pgfpathcurveto{\pgfqpoint{3.209228in}{3.503775in}}{\pgfqpoint{3.212798in}{3.495155in}}{\pgfqpoint{3.219153in}{3.488801in}}%
\pgfpathcurveto{\pgfqpoint{3.225507in}{3.482447in}}{\pgfqpoint{3.234126in}{3.478876in}}{\pgfqpoint{3.243113in}{3.478876in}}%
\pgfpathclose%
\pgfusepath{stroke,fill}%
\end{pgfscope}%
\begin{pgfscope}%
\pgfpathrectangle{\pgfqpoint{1.065196in}{0.528000in}}{\pgfqpoint{3.702804in}{3.696000in}} %
\pgfusepath{clip}%
\pgfsetbuttcap%
\pgfsetroundjoin%
\definecolor{currentfill}{rgb}{1.000000,0.498039,0.054902}%
\pgfsetfillcolor{currentfill}%
\pgfsetlinewidth{1.003750pt}%
\definecolor{currentstroke}{rgb}{1.000000,0.498039,0.054902}%
\pgfsetstrokecolor{currentstroke}%
\pgfsetdash{}{0pt}%
\pgfpathmoveto{\pgfqpoint{2.531988in}{3.189124in}}%
\pgfpathcurveto{\pgfqpoint{2.534376in}{3.189124in}}{\pgfqpoint{2.536666in}{3.190073in}}{\pgfqpoint{2.538355in}{3.191762in}}%
\pgfpathcurveto{\pgfqpoint{2.540043in}{3.193450in}}{\pgfqpoint{2.540992in}{3.195741in}}{\pgfqpoint{2.540992in}{3.198129in}}%
\pgfpathcurveto{\pgfqpoint{2.540992in}{3.200517in}}{\pgfqpoint{2.540043in}{3.202807in}}{\pgfqpoint{2.538355in}{3.204496in}}%
\pgfpathcurveto{\pgfqpoint{2.536666in}{3.206185in}}{\pgfqpoint{2.534376in}{3.207133in}}{\pgfqpoint{2.531988in}{3.207133in}}%
\pgfpathcurveto{\pgfqpoint{2.529600in}{3.207133in}}{\pgfqpoint{2.527309in}{3.206185in}}{\pgfqpoint{2.525620in}{3.204496in}}%
\pgfpathcurveto{\pgfqpoint{2.523932in}{3.202807in}}{\pgfqpoint{2.522983in}{3.200517in}}{\pgfqpoint{2.522983in}{3.198129in}}%
\pgfpathcurveto{\pgfqpoint{2.522983in}{3.195741in}}{\pgfqpoint{2.523932in}{3.193450in}}{\pgfqpoint{2.525620in}{3.191762in}}%
\pgfpathcurveto{\pgfqpoint{2.527309in}{3.190073in}}{\pgfqpoint{2.529600in}{3.189124in}}{\pgfqpoint{2.531988in}{3.189124in}}%
\pgfpathclose%
\pgfusepath{stroke,fill}%
\end{pgfscope}%
\begin{pgfscope}%
\pgfpathrectangle{\pgfqpoint{1.065196in}{0.528000in}}{\pgfqpoint{3.702804in}{3.696000in}} %
\pgfusepath{clip}%
\pgfsetbuttcap%
\pgfsetroundjoin%
\definecolor{currentfill}{rgb}{1.000000,0.498039,0.054902}%
\pgfsetfillcolor{currentfill}%
\pgfsetlinewidth{1.003750pt}%
\definecolor{currentstroke}{rgb}{1.000000,0.498039,0.054902}%
\pgfsetstrokecolor{currentstroke}%
\pgfsetdash{}{0pt}%
\pgfpathmoveto{\pgfqpoint{2.967620in}{2.480376in}}%
\pgfpathcurveto{\pgfqpoint{2.972494in}{2.480376in}}{\pgfqpoint{2.977169in}{2.482312in}}{\pgfqpoint{2.980615in}{2.485759in}}%
\pgfpathcurveto{\pgfqpoint{2.984061in}{2.489205in}}{\pgfqpoint{2.985998in}{2.493880in}}{\pgfqpoint{2.985998in}{2.498754in}}%
\pgfpathcurveto{\pgfqpoint{2.985998in}{2.503628in}}{\pgfqpoint{2.984061in}{2.508302in}}{\pgfqpoint{2.980615in}{2.511749in}}%
\pgfpathcurveto{\pgfqpoint{2.977169in}{2.515195in}}{\pgfqpoint{2.972494in}{2.517132in}}{\pgfqpoint{2.967620in}{2.517132in}}%
\pgfpathcurveto{\pgfqpoint{2.962746in}{2.517132in}}{\pgfqpoint{2.958071in}{2.515195in}}{\pgfqpoint{2.954625in}{2.511749in}}%
\pgfpathcurveto{\pgfqpoint{2.951179in}{2.508302in}}{\pgfqpoint{2.949242in}{2.503628in}}{\pgfqpoint{2.949242in}{2.498754in}}%
\pgfpathcurveto{\pgfqpoint{2.949242in}{2.493880in}}{\pgfqpoint{2.951179in}{2.489205in}}{\pgfqpoint{2.954625in}{2.485759in}}%
\pgfpathcurveto{\pgfqpoint{2.958071in}{2.482312in}}{\pgfqpoint{2.962746in}{2.480376in}}{\pgfqpoint{2.967620in}{2.480376in}}%
\pgfpathclose%
\pgfusepath{stroke,fill}%
\end{pgfscope}%
\begin{pgfscope}%
\pgfpathrectangle{\pgfqpoint{1.065196in}{0.528000in}}{\pgfqpoint{3.702804in}{3.696000in}} %
\pgfusepath{clip}%
\pgfsetbuttcap%
\pgfsetroundjoin%
\definecolor{currentfill}{rgb}{0.231674,0.318106,0.544834}%
\pgfsetfillcolor{currentfill}%
\pgfsetlinewidth{0.000000pt}%
\definecolor{currentstroke}{rgb}{0.000000,0.000000,0.000000}%
\pgfsetstrokecolor{currentstroke}%
\pgfsetdash{}{0pt}%
\pgfpathmoveto{\pgfqpoint{2.643590in}{3.100966in}}%
\pgfpathlineto{\pgfqpoint{2.683952in}{3.311901in}}%
\pgfpathlineto{\pgfqpoint{2.664365in}{3.306383in}}%
\pgfpathlineto{\pgfqpoint{2.708283in}{3.390634in}}%
\pgfpathlineto{\pgfqpoint{2.717994in}{3.296121in}}%
\pgfpathlineto{\pgfqpoint{2.701828in}{3.308480in}}%
\pgfpathlineto{\pgfqpoint{2.661466in}{3.097546in}}%
\pgfpathlineto{\pgfqpoint{2.643590in}{3.100966in}}%
\pgfusepath{fill}%
\end{pgfscope}%
\begin{pgfscope}%
\pgfpathrectangle{\pgfqpoint{1.065196in}{0.528000in}}{\pgfqpoint{3.702804in}{3.696000in}} %
\pgfusepath{clip}%
\pgfsetbuttcap%
\pgfsetroundjoin%
\definecolor{currentfill}{rgb}{0.267004,0.004874,0.329415}%
\pgfsetfillcolor{currentfill}%
\pgfsetlinewidth{0.000000pt}%
\definecolor{currentstroke}{rgb}{0.000000,0.000000,0.000000}%
\pgfsetstrokecolor{currentstroke}%
\pgfsetdash{}{0pt}%
\pgfpathmoveto{\pgfqpoint{2.647118in}{3.106573in}}%
\pgfpathlineto{\pgfqpoint{2.951835in}{3.331855in}}%
\pgfpathlineto{\pgfqpoint{2.933697in}{3.341080in}}%
\pgfpathlineto{\pgfqpoint{3.023103in}{3.373228in}}%
\pgfpathlineto{\pgfqpoint{2.966157in}{3.297174in}}%
\pgfpathlineto{\pgfqpoint{2.962655in}{3.317220in}}%
\pgfpathlineto{\pgfqpoint{2.657938in}{3.091938in}}%
\pgfpathlineto{\pgfqpoint{2.647118in}{3.106573in}}%
\pgfusepath{fill}%
\end{pgfscope}%
\begin{pgfscope}%
\pgfpathrectangle{\pgfqpoint{1.065196in}{0.528000in}}{\pgfqpoint{3.702804in}{3.696000in}} %
\pgfusepath{clip}%
\pgfsetbuttcap%
\pgfsetroundjoin%
\definecolor{currentfill}{rgb}{0.278791,0.062145,0.386592}%
\pgfsetfillcolor{currentfill}%
\pgfsetlinewidth{0.000000pt}%
\definecolor{currentstroke}{rgb}{0.000000,0.000000,0.000000}%
\pgfsetstrokecolor{currentstroke}%
\pgfsetdash{}{0pt}%
\pgfpathmoveto{\pgfqpoint{2.651780in}{3.108325in}}%
\pgfpathlineto{\pgfqpoint{3.090623in}{3.144508in}}%
\pgfpathlineto{\pgfqpoint{3.080058in}{3.161899in}}%
\pgfpathlineto{\pgfqpoint{3.172997in}{3.142168in}}%
\pgfpathlineto{\pgfqpoint{3.084544in}{3.107481in}}%
\pgfpathlineto{\pgfqpoint{3.092118in}{3.126368in}}%
\pgfpathlineto{\pgfqpoint{2.653275in}{3.090186in}}%
\pgfpathlineto{\pgfqpoint{2.651780in}{3.108325in}}%
\pgfusepath{fill}%
\end{pgfscope}%
\begin{pgfscope}%
\pgfpathrectangle{\pgfqpoint{1.065196in}{0.528000in}}{\pgfqpoint{3.702804in}{3.696000in}} %
\pgfusepath{clip}%
\pgfsetbuttcap%
\pgfsetroundjoin%
\definecolor{currentfill}{rgb}{0.180629,0.429975,0.557282}%
\pgfsetfillcolor{currentfill}%
\pgfsetlinewidth{0.000000pt}%
\definecolor{currentstroke}{rgb}{0.000000,0.000000,0.000000}%
\pgfsetstrokecolor{currentstroke}%
\pgfsetdash{}{0pt}%
\pgfpathmoveto{\pgfqpoint{2.656734in}{3.107326in}}%
\pgfpathlineto{\pgfqpoint{3.009233in}{2.923600in}}%
\pgfpathlineto{\pgfqpoint{3.009576in}{2.943947in}}%
\pgfpathlineto{\pgfqpoint{3.077657in}{2.877675in}}%
\pgfpathlineto{\pgfqpoint{2.984339in}{2.895527in}}%
\pgfpathlineto{\pgfqpoint{3.000821in}{2.907460in}}%
\pgfpathlineto{\pgfqpoint{2.648322in}{3.091186in}}%
\pgfpathlineto{\pgfqpoint{2.656734in}{3.107326in}}%
\pgfusepath{fill}%
\end{pgfscope}%
\begin{pgfscope}%
\pgfpathrectangle{\pgfqpoint{1.065196in}{0.528000in}}{\pgfqpoint{3.702804in}{3.696000in}} %
\pgfusepath{clip}%
\pgfsetbuttcap%
\pgfsetroundjoin%
\definecolor{currentfill}{rgb}{0.335885,0.777018,0.402049}%
\pgfsetfillcolor{currentfill}%
\pgfsetlinewidth{0.000000pt}%
\definecolor{currentstroke}{rgb}{0.000000,0.000000,0.000000}%
\pgfsetstrokecolor{currentstroke}%
\pgfsetdash{}{0pt}%
\pgfpathmoveto{\pgfqpoint{1.248530in}{1.703685in}}%
\pgfpathlineto{\pgfqpoint{1.394616in}{2.023750in}}%
\pgfpathlineto{\pgfqpoint{1.374279in}{2.023029in}}%
\pgfpathlineto{\pgfqpoint{1.436902in}{2.094481in}}%
\pgfpathlineto{\pgfqpoint{1.423952in}{2.000357in}}%
\pgfpathlineto{\pgfqpoint{1.411173in}{2.016193in}}%
\pgfpathlineto{\pgfqpoint{1.265087in}{1.696128in}}%
\pgfpathlineto{\pgfqpoint{1.248530in}{1.703685in}}%
\pgfusepath{fill}%
\end{pgfscope}%
\begin{pgfscope}%
\pgfpathrectangle{\pgfqpoint{1.065196in}{0.528000in}}{\pgfqpoint{3.702804in}{3.696000in}} %
\pgfusepath{clip}%
\pgfsetbuttcap%
\pgfsetroundjoin%
\definecolor{currentfill}{rgb}{0.311925,0.767822,0.415586}%
\pgfsetfillcolor{currentfill}%
\pgfsetlinewidth{0.000000pt}%
\definecolor{currentstroke}{rgb}{0.000000,0.000000,0.000000}%
\pgfsetstrokecolor{currentstroke}%
\pgfsetdash{}{0pt}%
\pgfpathmoveto{\pgfqpoint{1.877109in}{1.853259in}}%
\pgfpathlineto{\pgfqpoint{2.455045in}{2.045815in}}%
\pgfpathlineto{\pgfqpoint{2.440658in}{2.060206in}}%
\pgfpathlineto{\pgfqpoint{2.535625in}{2.063071in}}%
\pgfpathlineto{\pgfqpoint{2.457918in}{2.008404in}}%
\pgfpathlineto{\pgfqpoint{2.460798in}{2.028548in}}%
\pgfpathlineto{\pgfqpoint{1.882862in}{1.835992in}}%
\pgfpathlineto{\pgfqpoint{1.877109in}{1.853259in}}%
\pgfusepath{fill}%
\end{pgfscope}%
\begin{pgfscope}%
\pgfpathrectangle{\pgfqpoint{1.065196in}{0.528000in}}{\pgfqpoint{3.702804in}{3.696000in}} %
\pgfusepath{clip}%
\pgfsetbuttcap%
\pgfsetroundjoin%
\definecolor{currentfill}{rgb}{0.150476,0.504369,0.557430}%
\pgfsetfillcolor{currentfill}%
\pgfsetlinewidth{0.000000pt}%
\definecolor{currentstroke}{rgb}{0.000000,0.000000,0.000000}%
\pgfsetstrokecolor{currentstroke}%
\pgfsetdash{}{0pt}%
\pgfpathmoveto{\pgfqpoint{2.588905in}{2.499687in}}%
\pgfpathlineto{\pgfqpoint{2.914863in}{2.330875in}}%
\pgfpathlineto{\pgfqpoint{2.915152in}{2.351222in}}%
\pgfpathlineto{\pgfqpoint{2.983407in}{2.285128in}}%
\pgfpathlineto{\pgfqpoint{2.890042in}{2.302736in}}%
\pgfpathlineto{\pgfqpoint{2.906493in}{2.314713in}}%
\pgfpathlineto{\pgfqpoint{2.580535in}{2.483525in}}%
\pgfpathlineto{\pgfqpoint{2.588905in}{2.499687in}}%
\pgfusepath{fill}%
\end{pgfscope}%
\begin{pgfscope}%
\pgfpathrectangle{\pgfqpoint{1.065196in}{0.528000in}}{\pgfqpoint{3.702804in}{3.696000in}} %
\pgfusepath{clip}%
\pgfsetbuttcap%
\pgfsetroundjoin%
\definecolor{currentfill}{rgb}{0.265145,0.232956,0.516599}%
\pgfsetfillcolor{currentfill}%
\pgfsetlinewidth{0.000000pt}%
\definecolor{currentstroke}{rgb}{0.000000,0.000000,0.000000}%
\pgfsetstrokecolor{currentstroke}%
\pgfsetdash{}{0pt}%
\pgfpathmoveto{\pgfqpoint{2.668534in}{2.984290in}}%
\pgfpathlineto{\pgfqpoint{2.781714in}{2.597973in}}%
\pgfpathlineto{\pgfqpoint{2.796622in}{2.611824in}}%
\pgfpathlineto{\pgfqpoint{2.796008in}{2.516815in}}%
\pgfpathlineto{\pgfqpoint{2.744222in}{2.596472in}}%
\pgfpathlineto{\pgfqpoint{2.764247in}{2.592856in}}%
\pgfpathlineto{\pgfqpoint{2.651068in}{2.979173in}}%
\pgfpathlineto{\pgfqpoint{2.668534in}{2.984290in}}%
\pgfusepath{fill}%
\end{pgfscope}%
\begin{pgfscope}%
\pgfpathrectangle{\pgfqpoint{1.065196in}{0.528000in}}{\pgfqpoint{3.702804in}{3.696000in}} %
\pgfusepath{clip}%
\pgfsetbuttcap%
\pgfsetroundjoin%
\definecolor{currentfill}{rgb}{0.282623,0.140926,0.457517}%
\pgfsetfillcolor{currentfill}%
\pgfsetlinewidth{0.000000pt}%
\definecolor{currentstroke}{rgb}{0.000000,0.000000,0.000000}%
\pgfsetstrokecolor{currentstroke}%
\pgfsetdash{}{0pt}%
\pgfpathmoveto{\pgfqpoint{2.666820in}{2.987524in}}%
\pgfpathlineto{\pgfqpoint{2.852580in}{2.762420in}}%
\pgfpathlineto{\pgfqpoint{2.860826in}{2.781023in}}%
\pgfpathlineto{\pgfqpoint{2.897691in}{2.693456in}}%
\pgfpathlineto{\pgfqpoint{2.818712in}{2.746270in}}%
\pgfpathlineto{\pgfqpoint{2.838542in}{2.750835in}}%
\pgfpathlineto{\pgfqpoint{2.652782in}{2.975939in}}%
\pgfpathlineto{\pgfqpoint{2.666820in}{2.987524in}}%
\pgfusepath{fill}%
\end{pgfscope}%
\begin{pgfscope}%
\pgfpathrectangle{\pgfqpoint{1.065196in}{0.528000in}}{\pgfqpoint{3.702804in}{3.696000in}} %
\pgfusepath{clip}%
\pgfsetbuttcap%
\pgfsetroundjoin%
\definecolor{currentfill}{rgb}{0.227802,0.326594,0.546532}%
\pgfsetfillcolor{currentfill}%
\pgfsetlinewidth{0.000000pt}%
\definecolor{currentstroke}{rgb}{0.000000,0.000000,0.000000}%
\pgfsetstrokecolor{currentstroke}%
\pgfsetdash{}{0pt}%
\pgfpathmoveto{\pgfqpoint{2.662000in}{2.990562in}}%
\pgfpathlineto{\pgfqpoint{3.000380in}{2.906297in}}%
\pgfpathlineto{\pgfqpoint{2.995948in}{2.926158in}}%
\pgfpathlineto{\pgfqpoint{3.077657in}{2.877675in}}%
\pgfpathlineto{\pgfqpoint{2.982753in}{2.873174in}}%
\pgfpathlineto{\pgfqpoint{2.995982in}{2.888636in}}%
\pgfpathlineto{\pgfqpoint{2.657602in}{2.972901in}}%
\pgfpathlineto{\pgfqpoint{2.662000in}{2.990562in}}%
\pgfusepath{fill}%
\end{pgfscope}%
\begin{pgfscope}%
\pgfpathrectangle{\pgfqpoint{1.065196in}{0.528000in}}{\pgfqpoint{3.702804in}{3.696000in}} %
\pgfusepath{clip}%
\pgfsetbuttcap%
\pgfsetroundjoin%
\definecolor{currentfill}{rgb}{0.273809,0.031497,0.358853}%
\pgfsetfillcolor{currentfill}%
\pgfsetlinewidth{0.000000pt}%
\definecolor{currentstroke}{rgb}{0.000000,0.000000,0.000000}%
\pgfsetstrokecolor{currentstroke}%
\pgfsetdash{}{0pt}%
\pgfpathmoveto{\pgfqpoint{2.667475in}{2.986623in}}%
\pgfpathlineto{\pgfqpoint{2.931275in}{2.572713in}}%
\pgfpathlineto{\pgfqpoint{2.941732in}{2.590169in}}%
\pgfpathlineto{\pgfqpoint{2.967620in}{2.498754in}}%
\pgfpathlineto{\pgfqpoint{2.895687in}{2.560823in}}%
\pgfpathlineto{\pgfqpoint{2.915926in}{2.562931in}}%
\pgfpathlineto{\pgfqpoint{2.652127in}{2.976841in}}%
\pgfpathlineto{\pgfqpoint{2.667475in}{2.986623in}}%
\pgfusepath{fill}%
\end{pgfscope}%
\begin{pgfscope}%
\pgfpathrectangle{\pgfqpoint{1.065196in}{0.528000in}}{\pgfqpoint{3.702804in}{3.696000in}} %
\pgfusepath{clip}%
\pgfsetbuttcap%
\pgfsetroundjoin%
\definecolor{currentfill}{rgb}{0.281924,0.089666,0.412415}%
\pgfsetfillcolor{currentfill}%
\pgfsetlinewidth{0.000000pt}%
\definecolor{currentstroke}{rgb}{0.000000,0.000000,0.000000}%
\pgfsetstrokecolor{currentstroke}%
\pgfsetdash{}{0pt}%
\pgfpathmoveto{\pgfqpoint{1.939686in}{3.636534in}}%
\pgfpathlineto{\pgfqpoint{2.229411in}{3.675161in}}%
\pgfpathlineto{\pgfqpoint{2.217986in}{3.692000in}}%
\pgfpathlineto{\pgfqpoint{2.311799in}{3.676965in}}%
\pgfpathlineto{\pgfqpoint{2.225201in}{3.637876in}}%
\pgfpathlineto{\pgfqpoint{2.231817in}{3.657120in}}%
\pgfpathlineto{\pgfqpoint{1.942091in}{3.618493in}}%
\pgfpathlineto{\pgfqpoint{1.939686in}{3.636534in}}%
\pgfusepath{fill}%
\end{pgfscope}%
\begin{pgfscope}%
\pgfpathrectangle{\pgfqpoint{1.065196in}{0.528000in}}{\pgfqpoint{3.702804in}{3.696000in}} %
\pgfusepath{clip}%
\pgfsetbuttcap%
\pgfsetroundjoin%
\definecolor{currentfill}{rgb}{0.255645,0.260703,0.528312}%
\pgfsetfillcolor{currentfill}%
\pgfsetlinewidth{0.000000pt}%
\definecolor{currentstroke}{rgb}{0.000000,0.000000,0.000000}%
\pgfsetstrokecolor{currentstroke}%
\pgfsetdash{}{0pt}%
\pgfpathmoveto{\pgfqpoint{1.350474in}{2.941134in}}%
\pgfpathlineto{\pgfqpoint{1.510258in}{2.898223in}}%
\pgfpathlineto{\pgfqpoint{1.506189in}{2.918161in}}%
\pgfpathlineto{\pgfqpoint{1.586998in}{2.868191in}}%
\pgfpathlineto{\pgfqpoint{1.492027in}{2.865427in}}%
\pgfpathlineto{\pgfqpoint{1.505537in}{2.880645in}}%
\pgfpathlineto{\pgfqpoint{1.345753in}{2.923556in}}%
\pgfpathlineto{\pgfqpoint{1.350474in}{2.941134in}}%
\pgfusepath{fill}%
\end{pgfscope}%
\begin{pgfscope}%
\pgfpathrectangle{\pgfqpoint{1.065196in}{0.528000in}}{\pgfqpoint{3.702804in}{3.696000in}} %
\pgfusepath{clip}%
\pgfsetbuttcap%
\pgfsetroundjoin%
\definecolor{currentfill}{rgb}{0.263663,0.237631,0.518762}%
\pgfsetfillcolor{currentfill}%
\pgfsetlinewidth{0.000000pt}%
\definecolor{currentstroke}{rgb}{0.000000,0.000000,0.000000}%
\pgfsetstrokecolor{currentstroke}%
\pgfsetdash{}{0pt}%
\pgfpathmoveto{\pgfqpoint{1.725225in}{1.359985in}}%
\pgfpathlineto{\pgfqpoint{2.329599in}{1.439488in}}%
\pgfpathlineto{\pgfqpoint{2.318202in}{1.456346in}}%
\pgfpathlineto{\pgfqpoint{2.411989in}{1.441147in}}%
\pgfpathlineto{\pgfqpoint{2.325323in}{1.402210in}}%
\pgfpathlineto{\pgfqpoint{2.331972in}{1.421442in}}%
\pgfpathlineto{\pgfqpoint{1.727598in}{1.341940in}}%
\pgfpathlineto{\pgfqpoint{1.725225in}{1.359985in}}%
\pgfusepath{fill}%
\end{pgfscope}%
\begin{pgfscope}%
\pgfpathrectangle{\pgfqpoint{1.065196in}{0.528000in}}{\pgfqpoint{3.702804in}{3.696000in}} %
\pgfusepath{clip}%
\pgfsetbuttcap%
\pgfsetroundjoin%
\definecolor{currentfill}{rgb}{0.283072,0.130895,0.449241}%
\pgfsetfillcolor{currentfill}%
\pgfsetlinewidth{0.000000pt}%
\definecolor{currentstroke}{rgb}{0.000000,0.000000,0.000000}%
\pgfsetstrokecolor{currentstroke}%
\pgfsetdash{}{0pt}%
\pgfpathmoveto{\pgfqpoint{1.729377in}{1.359566in}}%
\pgfpathlineto{\pgfqpoint{2.187126in}{1.201805in}}%
\pgfpathlineto{\pgfqpoint{2.184452in}{1.221978in}}%
\pgfpathlineto{\pgfqpoint{2.261594in}{1.166514in}}%
\pgfpathlineto{\pgfqpoint{2.166661in}{1.170355in}}%
\pgfpathlineto{\pgfqpoint{2.181195in}{1.184597in}}%
\pgfpathlineto{\pgfqpoint{1.723446in}{1.342359in}}%
\pgfpathlineto{\pgfqpoint{1.729377in}{1.359566in}}%
\pgfusepath{fill}%
\end{pgfscope}%
\begin{pgfscope}%
\pgfpathrectangle{\pgfqpoint{1.065196in}{0.528000in}}{\pgfqpoint{3.702804in}{3.696000in}} %
\pgfusepath{clip}%
\pgfsetbuttcap%
\pgfsetroundjoin%
\definecolor{currentfill}{rgb}{0.269944,0.014625,0.341379}%
\pgfsetfillcolor{currentfill}%
\pgfsetlinewidth{0.000000pt}%
\definecolor{currentstroke}{rgb}{0.000000,0.000000,0.000000}%
\pgfsetstrokecolor{currentstroke}%
\pgfsetdash{}{0pt}%
\pgfpathmoveto{\pgfqpoint{3.939606in}{3.938060in}}%
\pgfpathlineto{\pgfqpoint{4.081238in}{3.898373in}}%
\pgfpathlineto{\pgfqpoint{4.077386in}{3.918355in}}%
\pgfpathlineto{\pgfqpoint{4.157648in}{3.867512in}}%
\pgfpathlineto{\pgfqpoint{4.062654in}{3.865778in}}%
\pgfpathlineto{\pgfqpoint{4.076327in}{3.880848in}}%
\pgfpathlineto{\pgfqpoint{3.934695in}{3.920534in}}%
\pgfpathlineto{\pgfqpoint{3.939606in}{3.938060in}}%
\pgfusepath{fill}%
\end{pgfscope}%
\begin{pgfscope}%
\pgfpathrectangle{\pgfqpoint{1.065196in}{0.528000in}}{\pgfqpoint{3.702804in}{3.696000in}} %
\pgfusepath{clip}%
\pgfsetbuttcap%
\pgfsetroundjoin%
\definecolor{currentfill}{rgb}{0.248629,0.278775,0.534556}%
\pgfsetfillcolor{currentfill}%
\pgfsetlinewidth{0.000000pt}%
\definecolor{currentstroke}{rgb}{0.000000,0.000000,0.000000}%
\pgfsetstrokecolor{currentstroke}%
\pgfsetdash{}{0pt}%
\pgfpathmoveto{\pgfqpoint{2.312128in}{3.011957in}}%
\pgfpathlineto{\pgfqpoint{2.744423in}{2.581080in}}%
\pgfpathlineto{\pgfqpoint{2.750826in}{2.600395in}}%
\pgfpathlineto{\pgfqpoint{2.796008in}{2.516815in}}%
\pgfpathlineto{\pgfqpoint{2.712280in}{2.561722in}}%
\pgfpathlineto{\pgfqpoint{2.731574in}{2.568189in}}%
\pgfpathlineto{\pgfqpoint{2.299280in}{2.999066in}}%
\pgfpathlineto{\pgfqpoint{2.312128in}{3.011957in}}%
\pgfusepath{fill}%
\end{pgfscope}%
\begin{pgfscope}%
\pgfpathrectangle{\pgfqpoint{1.065196in}{0.528000in}}{\pgfqpoint{3.702804in}{3.696000in}} %
\pgfusepath{clip}%
\pgfsetbuttcap%
\pgfsetroundjoin%
\definecolor{currentfill}{rgb}{0.220057,0.343307,0.549413}%
\pgfsetfillcolor{currentfill}%
\pgfsetlinewidth{0.000000pt}%
\definecolor{currentstroke}{rgb}{0.000000,0.000000,0.000000}%
\pgfsetstrokecolor{currentstroke}%
\pgfsetdash{}{0pt}%
\pgfpathmoveto{\pgfqpoint{2.311645in}{3.012404in}}%
\pgfpathlineto{\pgfqpoint{2.679732in}{2.695127in}}%
\pgfpathlineto{\pgfqpoint{2.684722in}{2.714855in}}%
\pgfpathlineto{\pgfqpoint{2.735828in}{2.634760in}}%
\pgfpathlineto{\pgfqpoint{2.649073in}{2.673496in}}%
\pgfpathlineto{\pgfqpoint{2.667849in}{2.681341in}}%
\pgfpathlineto{\pgfqpoint{2.299762in}{2.998618in}}%
\pgfpathlineto{\pgfqpoint{2.311645in}{3.012404in}}%
\pgfusepath{fill}%
\end{pgfscope}%
\begin{pgfscope}%
\pgfpathrectangle{\pgfqpoint{1.065196in}{0.528000in}}{\pgfqpoint{3.702804in}{3.696000in}} %
\pgfusepath{clip}%
\pgfsetbuttcap%
\pgfsetroundjoin%
\definecolor{currentfill}{rgb}{0.179019,0.433756,0.557430}%
\pgfsetfillcolor{currentfill}%
\pgfsetlinewidth{0.000000pt}%
\definecolor{currentstroke}{rgb}{0.000000,0.000000,0.000000}%
\pgfsetstrokecolor{currentstroke}%
\pgfsetdash{}{0pt}%
\pgfpathmoveto{\pgfqpoint{4.187458in}{3.674102in}}%
\pgfpathlineto{\pgfqpoint{4.159011in}{3.683795in}}%
\pgfpathlineto{\pgfqpoint{4.161755in}{3.663631in}}%
\pgfpathlineto{\pgfqpoint{4.084419in}{3.718824in}}%
\pgfpathlineto{\pgfqpoint{4.179365in}{3.715316in}}%
\pgfpathlineto{\pgfqpoint{4.164881in}{3.701023in}}%
\pgfpathlineto{\pgfqpoint{4.193328in}{3.691330in}}%
\pgfpathlineto{\pgfqpoint{4.187458in}{3.674102in}}%
\pgfusepath{fill}%
\end{pgfscope}%
\begin{pgfscope}%
\pgfpathrectangle{\pgfqpoint{1.065196in}{0.528000in}}{\pgfqpoint{3.702804in}{3.696000in}} %
\pgfusepath{clip}%
\pgfsetbuttcap%
\pgfsetroundjoin%
\definecolor{currentfill}{rgb}{0.180629,0.429975,0.557282}%
\pgfsetfillcolor{currentfill}%
\pgfsetlinewidth{0.000000pt}%
\definecolor{currentstroke}{rgb}{0.000000,0.000000,0.000000}%
\pgfsetstrokecolor{currentstroke}%
\pgfsetdash{}{0pt}%
\pgfpathmoveto{\pgfqpoint{4.187301in}{3.683924in}}%
\pgfpathlineto{\pgfqpoint{4.188509in}{3.687016in}}%
\pgfpathlineto{\pgfqpoint{4.181117in}{3.686341in}}%
\pgfpathlineto{\pgfqpoint{4.202475in}{3.713634in}}%
\pgfpathlineto{\pgfqpoint{4.199668in}{3.679091in}}%
\pgfpathlineto{\pgfqpoint{4.194693in}{3.684600in}}%
\pgfpathlineto{\pgfqpoint{4.193484in}{3.681508in}}%
\pgfpathlineto{\pgfqpoint{4.187301in}{3.683924in}}%
\pgfusepath{fill}%
\end{pgfscope}%
\begin{pgfscope}%
\pgfpathrectangle{\pgfqpoint{1.065196in}{0.528000in}}{\pgfqpoint{3.702804in}{3.696000in}} %
\pgfusepath{clip}%
\pgfsetbuttcap%
\pgfsetroundjoin%
\definecolor{currentfill}{rgb}{0.280894,0.078907,0.402329}%
\pgfsetfillcolor{currentfill}%
\pgfsetlinewidth{0.000000pt}%
\definecolor{currentstroke}{rgb}{0.000000,0.000000,0.000000}%
\pgfsetstrokecolor{currentstroke}%
\pgfsetdash{}{0pt}%
\pgfpathmoveto{\pgfqpoint{1.537630in}{0.826906in}}%
\pgfpathlineto{\pgfqpoint{2.038116in}{1.035846in}}%
\pgfpathlineto{\pgfqpoint{2.022707in}{1.049136in}}%
\pgfpathlineto{\pgfqpoint{2.117204in}{1.059001in}}%
\pgfpathlineto{\pgfqpoint{2.043742in}{0.998748in}}%
\pgfpathlineto{\pgfqpoint{2.045128in}{1.019050in}}%
\pgfpathlineto{\pgfqpoint{1.544642in}{0.810110in}}%
\pgfpathlineto{\pgfqpoint{1.537630in}{0.826906in}}%
\pgfusepath{fill}%
\end{pgfscope}%
\begin{pgfscope}%
\pgfpathrectangle{\pgfqpoint{1.065196in}{0.528000in}}{\pgfqpoint{3.702804in}{3.696000in}} %
\pgfusepath{clip}%
\pgfsetbuttcap%
\pgfsetroundjoin%
\definecolor{currentfill}{rgb}{0.120092,0.600104,0.542530}%
\pgfsetfillcolor{currentfill}%
\pgfsetlinewidth{0.000000pt}%
\definecolor{currentstroke}{rgb}{0.000000,0.000000,0.000000}%
\pgfsetstrokecolor{currentstroke}%
\pgfsetdash{}{0pt}%
\pgfpathmoveto{\pgfqpoint{1.537588in}{0.826888in}}%
\pgfpathlineto{\pgfqpoint{1.959719in}{1.005578in}}%
\pgfpathlineto{\pgfqpoint{1.944244in}{1.018792in}}%
\pgfpathlineto{\pgfqpoint{2.038691in}{1.029125in}}%
\pgfpathlineto{\pgfqpoint{1.965529in}{0.968509in}}%
\pgfpathlineto{\pgfqpoint{1.966814in}{0.988818in}}%
\pgfpathlineto{\pgfqpoint{1.544683in}{0.810127in}}%
\pgfpathlineto{\pgfqpoint{1.537588in}{0.826888in}}%
\pgfusepath{fill}%
\end{pgfscope}%
\begin{pgfscope}%
\pgfpathrectangle{\pgfqpoint{1.065196in}{0.528000in}}{\pgfqpoint{3.702804in}{3.696000in}} %
\pgfusepath{clip}%
\pgfsetbuttcap%
\pgfsetroundjoin%
\definecolor{currentfill}{rgb}{0.119738,0.603785,0.541400}%
\pgfsetfillcolor{currentfill}%
\pgfsetlinewidth{0.000000pt}%
\definecolor{currentstroke}{rgb}{0.000000,0.000000,0.000000}%
\pgfsetstrokecolor{currentstroke}%
\pgfsetdash{}{0pt}%
\pgfpathmoveto{\pgfqpoint{1.824064in}{3.636651in}}%
\pgfpathlineto{\pgfqpoint{2.229395in}{3.677755in}}%
\pgfpathlineto{\pgfqpoint{2.218505in}{3.694945in}}%
\pgfpathlineto{\pgfqpoint{2.311799in}{3.676965in}}%
\pgfpathlineto{\pgfqpoint{2.224014in}{3.640621in}}%
\pgfpathlineto{\pgfqpoint{2.231232in}{3.659647in}}%
\pgfpathlineto{\pgfqpoint{1.825900in}{3.618544in}}%
\pgfpathlineto{\pgfqpoint{1.824064in}{3.636651in}}%
\pgfusepath{fill}%
\end{pgfscope}%
\begin{pgfscope}%
\pgfpathrectangle{\pgfqpoint{1.065196in}{0.528000in}}{\pgfqpoint{3.702804in}{3.696000in}} %
\pgfusepath{clip}%
\pgfsetbuttcap%
\pgfsetroundjoin%
\definecolor{currentfill}{rgb}{0.278012,0.180367,0.486697}%
\pgfsetfillcolor{currentfill}%
\pgfsetlinewidth{0.000000pt}%
\definecolor{currentstroke}{rgb}{0.000000,0.000000,0.000000}%
\pgfsetstrokecolor{currentstroke}%
\pgfsetdash{}{0pt}%
\pgfpathmoveto{\pgfqpoint{1.833024in}{3.631857in}}%
\pgfpathlineto{\pgfqpoint{1.935729in}{3.437957in}}%
\pgfpathlineto{\pgfqpoint{1.947553in}{3.454518in}}%
\pgfpathlineto{\pgfqpoint{1.966024in}{3.361320in}}%
\pgfpathlineto{\pgfqpoint{1.899302in}{3.428960in}}%
\pgfpathlineto{\pgfqpoint{1.919645in}{3.429438in}}%
\pgfpathlineto{\pgfqpoint{1.816940in}{3.623338in}}%
\pgfpathlineto{\pgfqpoint{1.833024in}{3.631857in}}%
\pgfusepath{fill}%
\end{pgfscope}%
\begin{pgfscope}%
\pgfpathrectangle{\pgfqpoint{1.065196in}{0.528000in}}{\pgfqpoint{3.702804in}{3.696000in}} %
\pgfusepath{clip}%
\pgfsetbuttcap%
\pgfsetroundjoin%
\definecolor{currentfill}{rgb}{0.235526,0.309527,0.542944}%
\pgfsetfillcolor{currentfill}%
\pgfsetlinewidth{0.000000pt}%
\definecolor{currentstroke}{rgb}{0.000000,0.000000,0.000000}%
\pgfsetstrokecolor{currentstroke}%
\pgfsetdash{}{0pt}%
\pgfpathmoveto{\pgfqpoint{1.579860in}{2.104545in}}%
\pgfpathlineto{\pgfqpoint{1.833574in}{2.314980in}}%
\pgfpathlineto{\pgfqpoint{1.814950in}{2.323180in}}%
\pgfpathlineto{\pgfqpoint{1.902424in}{2.360263in}}%
\pgfpathlineto{\pgfqpoint{1.849808in}{2.281153in}}%
\pgfpathlineto{\pgfqpoint{1.845193in}{2.300971in}}%
\pgfpathlineto{\pgfqpoint{1.591480in}{2.090536in}}%
\pgfpathlineto{\pgfqpoint{1.579860in}{2.104545in}}%
\pgfusepath{fill}%
\end{pgfscope}%
\begin{pgfscope}%
\pgfpathrectangle{\pgfqpoint{1.065196in}{0.528000in}}{\pgfqpoint{3.702804in}{3.696000in}} %
\pgfusepath{clip}%
\pgfsetbuttcap%
\pgfsetroundjoin%
\definecolor{currentfill}{rgb}{0.220057,0.343307,0.549413}%
\pgfsetfillcolor{currentfill}%
\pgfsetlinewidth{0.000000pt}%
\definecolor{currentstroke}{rgb}{0.000000,0.000000,0.000000}%
\pgfsetstrokecolor{currentstroke}%
\pgfsetdash{}{0pt}%
\pgfpathmoveto{\pgfqpoint{4.467560in}{2.464677in}}%
\pgfpathlineto{\pgfqpoint{3.999641in}{2.212243in}}%
\pgfpathlineto{\pgfqpoint{4.016292in}{2.200545in}}%
\pgfpathlineto{\pgfqpoint{3.923237in}{2.181364in}}%
\pgfpathlineto{\pgfqpoint{3.990367in}{2.248600in}}%
\pgfpathlineto{\pgfqpoint{3.990999in}{2.228261in}}%
\pgfpathlineto{\pgfqpoint{4.458918in}{2.480695in}}%
\pgfpathlineto{\pgfqpoint{4.467560in}{2.464677in}}%
\pgfusepath{fill}%
\end{pgfscope}%
\begin{pgfscope}%
\pgfpathrectangle{\pgfqpoint{1.065196in}{0.528000in}}{\pgfqpoint{3.702804in}{3.696000in}} %
\pgfusepath{clip}%
\pgfsetbuttcap%
\pgfsetroundjoin%
\definecolor{currentfill}{rgb}{0.278826,0.175490,0.483397}%
\pgfsetfillcolor{currentfill}%
\pgfsetlinewidth{0.000000pt}%
\definecolor{currentstroke}{rgb}{0.000000,0.000000,0.000000}%
\pgfsetstrokecolor{currentstroke}%
\pgfsetdash{}{0pt}%
\pgfpathmoveto{\pgfqpoint{4.461903in}{2.463684in}}%
\pgfpathlineto{\pgfqpoint{4.353498in}{2.479767in}}%
\pgfpathlineto{\pgfqpoint{4.359829in}{2.460427in}}%
\pgfpathlineto{\pgfqpoint{4.273817in}{2.500788in}}%
\pgfpathlineto{\pgfqpoint{4.367842in}{2.514439in}}%
\pgfpathlineto{\pgfqpoint{4.356169in}{2.497770in}}%
\pgfpathlineto{\pgfqpoint{4.464574in}{2.481688in}}%
\pgfpathlineto{\pgfqpoint{4.461903in}{2.463684in}}%
\pgfusepath{fill}%
\end{pgfscope}%
\begin{pgfscope}%
\pgfpathrectangle{\pgfqpoint{1.065196in}{0.528000in}}{\pgfqpoint{3.702804in}{3.696000in}} %
\pgfusepath{clip}%
\pgfsetbuttcap%
\pgfsetroundjoin%
\definecolor{currentfill}{rgb}{0.231674,0.318106,0.544834}%
\pgfsetfillcolor{currentfill}%
\pgfsetlinewidth{0.000000pt}%
\definecolor{currentstroke}{rgb}{0.000000,0.000000,0.000000}%
\pgfsetstrokecolor{currentstroke}%
\pgfsetdash{}{0pt}%
\pgfpathmoveto{\pgfqpoint{4.460170in}{2.481253in}}%
\pgfpathlineto{\pgfqpoint{4.503664in}{2.496835in}}%
\pgfpathlineto{\pgfqpoint{4.488958in}{2.510900in}}%
\pgfpathlineto{\pgfqpoint{4.583838in}{2.515891in}}%
\pgfpathlineto{\pgfqpoint{4.507374in}{2.459497in}}%
\pgfpathlineto{\pgfqpoint{4.509803in}{2.479701in}}%
\pgfpathlineto{\pgfqpoint{4.466308in}{2.464119in}}%
\pgfpathlineto{\pgfqpoint{4.460170in}{2.481253in}}%
\pgfusepath{fill}%
\end{pgfscope}%
\begin{pgfscope}%
\pgfpathrectangle{\pgfqpoint{1.065196in}{0.528000in}}{\pgfqpoint{3.702804in}{3.696000in}} %
\pgfusepath{clip}%
\pgfsetbuttcap%
\pgfsetroundjoin%
\definecolor{currentfill}{rgb}{0.267004,0.004874,0.329415}%
\pgfsetfillcolor{currentfill}%
\pgfsetlinewidth{0.000000pt}%
\definecolor{currentstroke}{rgb}{0.000000,0.000000,0.000000}%
\pgfsetstrokecolor{currentstroke}%
\pgfsetdash{}{0pt}%
\pgfpathmoveto{\pgfqpoint{3.575132in}{1.752806in}}%
\pgfpathlineto{\pgfqpoint{3.853544in}{1.675077in}}%
\pgfpathlineto{\pgfqpoint{3.849673in}{1.695054in}}%
\pgfpathlineto{\pgfqpoint{3.929983in}{1.644287in}}%
\pgfpathlineto{\pgfqpoint{3.834990in}{1.642463in}}%
\pgfpathlineto{\pgfqpoint{3.848649in}{1.657546in}}%
\pgfpathlineto{\pgfqpoint{3.570238in}{1.735275in}}%
\pgfpathlineto{\pgfqpoint{3.575132in}{1.752806in}}%
\pgfusepath{fill}%
\end{pgfscope}%
\begin{pgfscope}%
\pgfpathrectangle{\pgfqpoint{1.065196in}{0.528000in}}{\pgfqpoint{3.702804in}{3.696000in}} %
\pgfusepath{clip}%
\pgfsetbuttcap%
\pgfsetroundjoin%
\definecolor{currentfill}{rgb}{0.221989,0.339161,0.548752}%
\pgfsetfillcolor{currentfill}%
\pgfsetlinewidth{0.000000pt}%
\definecolor{currentstroke}{rgb}{0.000000,0.000000,0.000000}%
\pgfsetstrokecolor{currentstroke}%
\pgfsetdash{}{0pt}%
\pgfpathmoveto{\pgfqpoint{3.564915in}{1.739303in}}%
\pgfpathlineto{\pgfqpoint{3.373350in}{2.053472in}}%
\pgfpathlineto{\pgfqpoint{3.362548in}{2.036226in}}%
\pgfpathlineto{\pgfqpoint{3.338481in}{2.128138in}}%
\pgfpathlineto{\pgfqpoint{3.409168in}{2.064652in}}%
\pgfpathlineto{\pgfqpoint{3.388890in}{2.062947in}}%
\pgfpathlineto{\pgfqpoint{3.580455in}{1.748778in}}%
\pgfpathlineto{\pgfqpoint{3.564915in}{1.739303in}}%
\pgfusepath{fill}%
\end{pgfscope}%
\begin{pgfscope}%
\pgfpathrectangle{\pgfqpoint{1.065196in}{0.528000in}}{\pgfqpoint{3.702804in}{3.696000in}} %
\pgfusepath{clip}%
\pgfsetbuttcap%
\pgfsetroundjoin%
\definecolor{currentfill}{rgb}{0.180653,0.701402,0.488189}%
\pgfsetfillcolor{currentfill}%
\pgfsetlinewidth{0.000000pt}%
\definecolor{currentstroke}{rgb}{0.000000,0.000000,0.000000}%
\pgfsetstrokecolor{currentstroke}%
\pgfsetdash{}{0pt}%
\pgfpathmoveto{\pgfqpoint{3.551507in}{3.490439in}}%
\pgfpathlineto{\pgfqpoint{3.687421in}{3.541124in}}%
\pgfpathlineto{\pgfqpoint{3.672534in}{3.554998in}}%
\pgfpathlineto{\pgfqpoint{3.767341in}{3.561215in}}%
\pgfpathlineto{\pgfqpoint{3.691613in}{3.503837in}}%
\pgfpathlineto{\pgfqpoint{3.693780in}{3.524070in}}%
\pgfpathlineto{\pgfqpoint{3.557866in}{3.473385in}}%
\pgfpathlineto{\pgfqpoint{3.551507in}{3.490439in}}%
\pgfusepath{fill}%
\end{pgfscope}%
\begin{pgfscope}%
\pgfpathrectangle{\pgfqpoint{1.065196in}{0.528000in}}{\pgfqpoint{3.702804in}{3.696000in}} %
\pgfusepath{clip}%
\pgfsetbuttcap%
\pgfsetroundjoin%
\definecolor{currentfill}{rgb}{0.283187,0.125848,0.444960}%
\pgfsetfillcolor{currentfill}%
\pgfsetlinewidth{0.000000pt}%
\definecolor{currentstroke}{rgb}{0.000000,0.000000,0.000000}%
\pgfsetstrokecolor{currentstroke}%
\pgfsetdash{}{0pt}%
\pgfpathmoveto{\pgfqpoint{3.551899in}{3.490575in}}%
\pgfpathlineto{\pgfqpoint{3.893604in}{3.600524in}}%
\pgfpathlineto{\pgfqpoint{3.879366in}{3.615063in}}%
\pgfpathlineto{\pgfqpoint{3.974358in}{3.616949in}}%
\pgfpathlineto{\pgfqpoint{3.896091in}{3.563085in}}%
\pgfpathlineto{\pgfqpoint{3.899179in}{3.583199in}}%
\pgfpathlineto{\pgfqpoint{3.557474in}{3.473249in}}%
\pgfpathlineto{\pgfqpoint{3.551899in}{3.490575in}}%
\pgfusepath{fill}%
\end{pgfscope}%
\begin{pgfscope}%
\pgfpathrectangle{\pgfqpoint{1.065196in}{0.528000in}}{\pgfqpoint{3.702804in}{3.696000in}} %
\pgfusepath{clip}%
\pgfsetbuttcap%
\pgfsetroundjoin%
\definecolor{currentfill}{rgb}{0.146616,0.673050,0.508936}%
\pgfsetfillcolor{currentfill}%
\pgfsetlinewidth{0.000000pt}%
\definecolor{currentstroke}{rgb}{0.000000,0.000000,0.000000}%
\pgfsetstrokecolor{currentstroke}%
\pgfsetdash{}{0pt}%
\pgfpathmoveto{\pgfqpoint{1.324709in}{3.204831in}}%
\pgfpathlineto{\pgfqpoint{1.542351in}{2.937456in}}%
\pgfpathlineto{\pgfqpoint{1.550721in}{2.956003in}}%
\pgfpathlineto{\pgfqpoint{1.586998in}{2.868191in}}%
\pgfpathlineto{\pgfqpoint{1.508375in}{2.921533in}}%
\pgfpathlineto{\pgfqpoint{1.528235in}{2.925966in}}%
\pgfpathlineto{\pgfqpoint{1.310593in}{3.193341in}}%
\pgfpathlineto{\pgfqpoint{1.324709in}{3.204831in}}%
\pgfusepath{fill}%
\end{pgfscope}%
\begin{pgfscope}%
\pgfpathrectangle{\pgfqpoint{1.065196in}{0.528000in}}{\pgfqpoint{3.702804in}{3.696000in}} %
\pgfusepath{clip}%
\pgfsetbuttcap%
\pgfsetroundjoin%
\definecolor{currentfill}{rgb}{0.280267,0.073417,0.397163}%
\pgfsetfillcolor{currentfill}%
\pgfsetlinewidth{0.000000pt}%
\definecolor{currentstroke}{rgb}{0.000000,0.000000,0.000000}%
\pgfsetstrokecolor{currentstroke}%
\pgfsetdash{}{0pt}%
\pgfpathmoveto{\pgfqpoint{4.558624in}{3.188732in}}%
\pgfpathlineto{\pgfqpoint{4.349293in}{3.654592in}}%
\pgfpathlineto{\pgfqpoint{4.336421in}{3.638831in}}%
\pgfpathlineto{\pgfqpoint{4.324024in}{3.733029in}}%
\pgfpathlineto{\pgfqpoint{4.386226in}{3.661211in}}%
\pgfpathlineto{\pgfqpoint{4.365895in}{3.662052in}}%
\pgfpathlineto{\pgfqpoint{4.575226in}{3.196192in}}%
\pgfpathlineto{\pgfqpoint{4.558624in}{3.188732in}}%
\pgfusepath{fill}%
\end{pgfscope}%
\begin{pgfscope}%
\pgfpathrectangle{\pgfqpoint{1.065196in}{0.528000in}}{\pgfqpoint{3.702804in}{3.696000in}} %
\pgfusepath{clip}%
\pgfsetbuttcap%
\pgfsetroundjoin%
\definecolor{currentfill}{rgb}{0.175841,0.441290,0.557685}%
\pgfsetfillcolor{currentfill}%
\pgfsetlinewidth{0.000000pt}%
\definecolor{currentstroke}{rgb}{0.000000,0.000000,0.000000}%
\pgfsetstrokecolor{currentstroke}%
\pgfsetdash{}{0pt}%
\pgfpathmoveto{\pgfqpoint{2.224278in}{1.131012in}}%
\pgfpathlineto{\pgfqpoint{2.454589in}{1.002244in}}%
\pgfpathlineto{\pgfqpoint{2.455528in}{1.022572in}}%
\pgfpathlineto{\pgfqpoint{2.521636in}{0.954332in}}%
\pgfpathlineto{\pgfqpoint{2.428882in}{0.974913in}}%
\pgfpathlineto{\pgfqpoint{2.445707in}{0.986358in}}%
\pgfpathlineto{\pgfqpoint{2.215396in}{1.115126in}}%
\pgfpathlineto{\pgfqpoint{2.224278in}{1.131012in}}%
\pgfusepath{fill}%
\end{pgfscope}%
\begin{pgfscope}%
\pgfpathrectangle{\pgfqpoint{1.065196in}{0.528000in}}{\pgfqpoint{3.702804in}{3.696000in}} %
\pgfusepath{clip}%
\pgfsetbuttcap%
\pgfsetroundjoin%
\definecolor{currentfill}{rgb}{0.271305,0.019942,0.347269}%
\pgfsetfillcolor{currentfill}%
\pgfsetlinewidth{0.000000pt}%
\definecolor{currentstroke}{rgb}{0.000000,0.000000,0.000000}%
\pgfsetstrokecolor{currentstroke}%
\pgfsetdash{}{0pt}%
\pgfpathmoveto{\pgfqpoint{1.330353in}{2.959951in}}%
\pgfpathlineto{\pgfqpoint{1.321194in}{2.756024in}}%
\pgfpathlineto{\pgfqpoint{1.339785in}{2.764299in}}%
\pgfpathlineto{\pgfqpoint{1.308428in}{2.674612in}}%
\pgfpathlineto{\pgfqpoint{1.285237in}{2.766749in}}%
\pgfpathlineto{\pgfqpoint{1.303011in}{2.756841in}}%
\pgfpathlineto{\pgfqpoint{1.312171in}{2.960768in}}%
\pgfpathlineto{\pgfqpoint{1.330353in}{2.959951in}}%
\pgfusepath{fill}%
\end{pgfscope}%
\begin{pgfscope}%
\pgfpathrectangle{\pgfqpoint{1.065196in}{0.528000in}}{\pgfqpoint{3.702804in}{3.696000in}} %
\pgfusepath{clip}%
\pgfsetbuttcap%
\pgfsetroundjoin%
\definecolor{currentfill}{rgb}{0.282884,0.135920,0.453427}%
\pgfsetfillcolor{currentfill}%
\pgfsetlinewidth{0.000000pt}%
\definecolor{currentstroke}{rgb}{0.000000,0.000000,0.000000}%
\pgfsetstrokecolor{currentstroke}%
\pgfsetdash{}{0pt}%
\pgfpathmoveto{\pgfqpoint{2.900060in}{0.857540in}}%
\pgfpathlineto{\pgfqpoint{2.599388in}{0.927024in}}%
\pgfpathlineto{\pgfqpoint{2.604156in}{0.907241in}}%
\pgfpathlineto{\pgfqpoint{2.521636in}{0.954332in}}%
\pgfpathlineto{\pgfqpoint{2.616450in}{0.960442in}}%
\pgfpathlineto{\pgfqpoint{2.603486in}{0.944757in}}%
\pgfpathlineto{\pgfqpoint{2.904158in}{0.875274in}}%
\pgfpathlineto{\pgfqpoint{2.900060in}{0.857540in}}%
\pgfusepath{fill}%
\end{pgfscope}%
\begin{pgfscope}%
\pgfpathrectangle{\pgfqpoint{1.065196in}{0.528000in}}{\pgfqpoint{3.702804in}{3.696000in}} %
\pgfusepath{clip}%
\pgfsetbuttcap%
\pgfsetroundjoin%
\definecolor{currentfill}{rgb}{0.120638,0.625828,0.533488}%
\pgfsetfillcolor{currentfill}%
\pgfsetlinewidth{0.000000pt}%
\definecolor{currentstroke}{rgb}{0.000000,0.000000,0.000000}%
\pgfsetstrokecolor{currentstroke}%
\pgfsetdash{}{0pt}%
\pgfpathmoveto{\pgfqpoint{2.893733in}{0.869964in}}%
\pgfpathlineto{\pgfqpoint{2.978397in}{1.069316in}}%
\pgfpathlineto{\pgfqpoint{2.958087in}{1.068054in}}%
\pgfpathlineto{\pgfqpoint{3.018789in}{1.141145in}}%
\pgfpathlineto{\pgfqpoint{3.008344in}{1.046710in}}%
\pgfpathlineto{\pgfqpoint{2.995149in}{1.062201in}}%
\pgfpathlineto{\pgfqpoint{2.910486in}{0.862850in}}%
\pgfpathlineto{\pgfqpoint{2.893733in}{0.869964in}}%
\pgfusepath{fill}%
\end{pgfscope}%
\begin{pgfscope}%
\pgfpathrectangle{\pgfqpoint{1.065196in}{0.528000in}}{\pgfqpoint{3.702804in}{3.696000in}} %
\pgfusepath{clip}%
\pgfsetbuttcap%
\pgfsetroundjoin%
\definecolor{currentfill}{rgb}{0.183898,0.422383,0.556944}%
\pgfsetfillcolor{currentfill}%
\pgfsetlinewidth{0.000000pt}%
\definecolor{currentstroke}{rgb}{0.000000,0.000000,0.000000}%
\pgfsetstrokecolor{currentstroke}%
\pgfsetdash{}{0pt}%
\pgfpathmoveto{\pgfqpoint{3.175086in}{1.187432in}}%
\pgfpathlineto{\pgfqpoint{3.616113in}{1.362926in}}%
\pgfpathlineto{\pgfqpoint{3.600928in}{1.376472in}}%
\pgfpathlineto{\pgfqpoint{3.695578in}{1.384752in}}%
\pgfpathlineto{\pgfqpoint{3.621116in}{1.325739in}}%
\pgfpathlineto{\pgfqpoint{3.622842in}{1.346015in}}%
\pgfpathlineto{\pgfqpoint{3.181815in}{1.170521in}}%
\pgfpathlineto{\pgfqpoint{3.175086in}{1.187432in}}%
\pgfusepath{fill}%
\end{pgfscope}%
\begin{pgfscope}%
\pgfpathrectangle{\pgfqpoint{1.065196in}{0.528000in}}{\pgfqpoint{3.702804in}{3.696000in}} %
\pgfusepath{clip}%
\pgfsetbuttcap%
\pgfsetroundjoin%
\definecolor{currentfill}{rgb}{0.180629,0.429975,0.557282}%
\pgfsetfillcolor{currentfill}%
\pgfsetlinewidth{0.000000pt}%
\definecolor{currentstroke}{rgb}{0.000000,0.000000,0.000000}%
\pgfsetstrokecolor{currentstroke}%
\pgfsetdash{}{0pt}%
\pgfpathmoveto{\pgfqpoint{3.238214in}{3.028589in}}%
\pgfpathlineto{\pgfqpoint{3.205696in}{2.868869in}}%
\pgfpathlineto{\pgfqpoint{3.225347in}{2.874156in}}%
\pgfpathlineto{\pgfqpoint{3.180439in}{2.790428in}}%
\pgfpathlineto{\pgfqpoint{3.171842in}{2.885049in}}%
\pgfpathlineto{\pgfqpoint{3.187861in}{2.872500in}}%
\pgfpathlineto{\pgfqpoint{3.220379in}{3.032220in}}%
\pgfpathlineto{\pgfqpoint{3.238214in}{3.028589in}}%
\pgfusepath{fill}%
\end{pgfscope}%
\begin{pgfscope}%
\pgfpathrectangle{\pgfqpoint{1.065196in}{0.528000in}}{\pgfqpoint{3.702804in}{3.696000in}} %
\pgfusepath{clip}%
\pgfsetbuttcap%
\pgfsetroundjoin%
\definecolor{currentfill}{rgb}{0.239346,0.300855,0.540844}%
\pgfsetfillcolor{currentfill}%
\pgfsetlinewidth{0.000000pt}%
\definecolor{currentstroke}{rgb}{0.000000,0.000000,0.000000}%
\pgfsetstrokecolor{currentstroke}%
\pgfsetdash{}{0pt}%
\pgfpathmoveto{\pgfqpoint{3.234911in}{3.037567in}}%
\pgfpathlineto{\pgfqpoint{3.552804in}{2.788406in}}%
\pgfpathlineto{\pgfqpoint{3.556870in}{2.808345in}}%
\pgfpathlineto{\pgfqpoint{3.611653in}{2.730719in}}%
\pgfpathlineto{\pgfqpoint{3.523186in}{2.765370in}}%
\pgfpathlineto{\pgfqpoint{3.541577in}{2.774081in}}%
\pgfpathlineto{\pgfqpoint{3.223683in}{3.023242in}}%
\pgfpathlineto{\pgfqpoint{3.234911in}{3.037567in}}%
\pgfusepath{fill}%
\end{pgfscope}%
\begin{pgfscope}%
\pgfpathrectangle{\pgfqpoint{1.065196in}{0.528000in}}{\pgfqpoint{3.702804in}{3.696000in}} %
\pgfusepath{clip}%
\pgfsetbuttcap%
\pgfsetroundjoin%
\definecolor{currentfill}{rgb}{0.156270,0.489624,0.557936}%
\pgfsetfillcolor{currentfill}%
\pgfsetlinewidth{0.000000pt}%
\definecolor{currentstroke}{rgb}{0.000000,0.000000,0.000000}%
\pgfsetstrokecolor{currentstroke}%
\pgfsetdash{}{0pt}%
\pgfpathmoveto{\pgfqpoint{1.592929in}{2.080694in}}%
\pgfpathlineto{\pgfqpoint{1.728381in}{2.202719in}}%
\pgfpathlineto{\pgfqpoint{1.709438in}{2.210150in}}%
\pgfpathlineto{\pgfqpoint{1.795324in}{2.250777in}}%
\pgfpathlineto{\pgfqpoint{1.745984in}{2.169582in}}%
\pgfpathlineto{\pgfqpoint{1.740563in}{2.189196in}}%
\pgfpathlineto{\pgfqpoint{1.605111in}{2.067172in}}%
\pgfpathlineto{\pgfqpoint{1.592929in}{2.080694in}}%
\pgfusepath{fill}%
\end{pgfscope}%
\begin{pgfscope}%
\pgfpathrectangle{\pgfqpoint{1.065196in}{0.528000in}}{\pgfqpoint{3.702804in}{3.696000in}} %
\pgfusepath{clip}%
\pgfsetbuttcap%
\pgfsetroundjoin%
\definecolor{currentfill}{rgb}{0.212395,0.359683,0.551710}%
\pgfsetfillcolor{currentfill}%
\pgfsetlinewidth{0.000000pt}%
\definecolor{currentstroke}{rgb}{0.000000,0.000000,0.000000}%
\pgfsetstrokecolor{currentstroke}%
\pgfsetdash{}{0pt}%
\pgfpathmoveto{\pgfqpoint{3.573479in}{2.069421in}}%
\pgfpathlineto{\pgfqpoint{3.538752in}{2.123391in}}%
\pgfpathlineto{\pgfqpoint{3.528370in}{2.105889in}}%
\pgfpathlineto{\pgfqpoint{3.502086in}{2.197192in}}%
\pgfpathlineto{\pgfqpoint{3.574288in}{2.135435in}}%
\pgfpathlineto{\pgfqpoint{3.554058in}{2.133240in}}%
\pgfpathlineto{\pgfqpoint{3.588785in}{2.079270in}}%
\pgfpathlineto{\pgfqpoint{3.573479in}{2.069421in}}%
\pgfusepath{fill}%
\end{pgfscope}%
\begin{pgfscope}%
\pgfpathrectangle{\pgfqpoint{1.065196in}{0.528000in}}{\pgfqpoint{3.702804in}{3.696000in}} %
\pgfusepath{clip}%
\pgfsetbuttcap%
\pgfsetroundjoin%
\definecolor{currentfill}{rgb}{0.268510,0.009605,0.335427}%
\pgfsetfillcolor{currentfill}%
\pgfsetlinewidth{0.000000pt}%
\definecolor{currentstroke}{rgb}{0.000000,0.000000,0.000000}%
\pgfsetstrokecolor{currentstroke}%
\pgfsetdash{}{0pt}%
\pgfpathmoveto{\pgfqpoint{3.579162in}{2.065461in}}%
\pgfpathlineto{\pgfqpoint{3.416474in}{2.101527in}}%
\pgfpathlineto{\pgfqpoint{3.421419in}{2.081788in}}%
\pgfpathlineto{\pgfqpoint{3.338481in}{2.128138in}}%
\pgfpathlineto{\pgfqpoint{3.433237in}{2.135096in}}%
\pgfpathlineto{\pgfqpoint{3.420413in}{2.119296in}}%
\pgfpathlineto{\pgfqpoint{3.583101in}{2.083230in}}%
\pgfpathlineto{\pgfqpoint{3.579162in}{2.065461in}}%
\pgfusepath{fill}%
\end{pgfscope}%
\begin{pgfscope}%
\pgfpathrectangle{\pgfqpoint{1.065196in}{0.528000in}}{\pgfqpoint{3.702804in}{3.696000in}} %
\pgfusepath{clip}%
\pgfsetbuttcap%
\pgfsetroundjoin%
\definecolor{currentfill}{rgb}{0.216210,0.351535,0.550627}%
\pgfsetfillcolor{currentfill}%
\pgfsetlinewidth{0.000000pt}%
\definecolor{currentstroke}{rgb}{0.000000,0.000000,0.000000}%
\pgfsetstrokecolor{currentstroke}%
\pgfsetdash{}{0pt}%
\pgfpathmoveto{\pgfqpoint{1.415848in}{2.477160in}}%
\pgfpathlineto{\pgfqpoint{1.342638in}{2.599641in}}%
\pgfpathlineto{\pgfqpoint{1.331684in}{2.582491in}}%
\pgfpathlineto{\pgfqpoint{1.308428in}{2.674612in}}%
\pgfpathlineto{\pgfqpoint{1.378552in}{2.610506in}}%
\pgfpathlineto{\pgfqpoint{1.358260in}{2.608979in}}%
\pgfpathlineto{\pgfqpoint{1.431470in}{2.486498in}}%
\pgfpathlineto{\pgfqpoint{1.415848in}{2.477160in}}%
\pgfusepath{fill}%
\end{pgfscope}%
\begin{pgfscope}%
\pgfpathrectangle{\pgfqpoint{1.065196in}{0.528000in}}{\pgfqpoint{3.702804in}{3.696000in}} %
\pgfusepath{clip}%
\pgfsetbuttcap%
\pgfsetroundjoin%
\definecolor{currentfill}{rgb}{0.280267,0.073417,0.397163}%
\pgfsetfillcolor{currentfill}%
\pgfsetlinewidth{0.000000pt}%
\definecolor{currentstroke}{rgb}{0.000000,0.000000,0.000000}%
\pgfsetstrokecolor{currentstroke}%
\pgfsetdash{}{0pt}%
\pgfpathmoveto{\pgfqpoint{1.432478in}{2.484076in}}%
\pgfpathlineto{\pgfqpoint{1.435610in}{2.471781in}}%
\pgfpathlineto{\pgfqpoint{1.451001in}{2.485093in}}%
\pgfpathlineto{\pgfqpoint{1.447013in}{2.390166in}}%
\pgfpathlineto{\pgfqpoint{1.398089in}{2.471612in}}%
\pgfpathlineto{\pgfqpoint{1.417973in}{2.467287in}}%
\pgfpathlineto{\pgfqpoint{1.414841in}{2.479582in}}%
\pgfpathlineto{\pgfqpoint{1.432478in}{2.484076in}}%
\pgfusepath{fill}%
\end{pgfscope}%
\begin{pgfscope}%
\pgfpathrectangle{\pgfqpoint{1.065196in}{0.528000in}}{\pgfqpoint{3.702804in}{3.696000in}} %
\pgfusepath{clip}%
\pgfsetbuttcap%
\pgfsetroundjoin%
\definecolor{currentfill}{rgb}{0.147607,0.511733,0.557049}%
\pgfsetfillcolor{currentfill}%
\pgfsetlinewidth{0.000000pt}%
\definecolor{currentstroke}{rgb}{0.000000,0.000000,0.000000}%
\pgfsetstrokecolor{currentstroke}%
\pgfsetdash{}{0pt}%
\pgfpathmoveto{\pgfqpoint{3.487717in}{2.412490in}}%
\pgfpathlineto{\pgfqpoint{3.502237in}{2.279599in}}%
\pgfpathlineto{\pgfqpoint{3.519341in}{2.290622in}}%
\pgfpathlineto{\pgfqpoint{3.502086in}{2.197192in}}%
\pgfpathlineto{\pgfqpoint{3.465062in}{2.284692in}}%
\pgfpathlineto{\pgfqpoint{3.484144in}{2.277622in}}%
\pgfpathlineto{\pgfqpoint{3.469624in}{2.410513in}}%
\pgfpathlineto{\pgfqpoint{3.487717in}{2.412490in}}%
\pgfusepath{fill}%
\end{pgfscope}%
\begin{pgfscope}%
\pgfpathrectangle{\pgfqpoint{1.065196in}{0.528000in}}{\pgfqpoint{3.702804in}{3.696000in}} %
\pgfusepath{clip}%
\pgfsetbuttcap%
\pgfsetroundjoin%
\definecolor{currentfill}{rgb}{0.319809,0.770914,0.411152}%
\pgfsetfillcolor{currentfill}%
\pgfsetlinewidth{0.000000pt}%
\definecolor{currentstroke}{rgb}{0.000000,0.000000,0.000000}%
\pgfsetstrokecolor{currentstroke}%
\pgfsetdash{}{0pt}%
\pgfpathmoveto{\pgfqpoint{4.410935in}{2.656119in}}%
\pgfpathlineto{\pgfqpoint{4.469598in}{2.753514in}}%
\pgfpathlineto{\pgfqpoint{4.449312in}{2.755109in}}%
\pgfpathlineto{\pgfqpoint{4.519652in}{2.818979in}}%
\pgfpathlineto{\pgfqpoint{4.496085in}{2.726937in}}%
\pgfpathlineto{\pgfqpoint{4.485189in}{2.744123in}}%
\pgfpathlineto{\pgfqpoint{4.426526in}{2.646728in}}%
\pgfpathlineto{\pgfqpoint{4.410935in}{2.656119in}}%
\pgfusepath{fill}%
\end{pgfscope}%
\begin{pgfscope}%
\pgfpathrectangle{\pgfqpoint{1.065196in}{0.528000in}}{\pgfqpoint{3.702804in}{3.696000in}} %
\pgfusepath{clip}%
\pgfsetbuttcap%
\pgfsetroundjoin%
\definecolor{currentfill}{rgb}{0.270595,0.214069,0.507052}%
\pgfsetfillcolor{currentfill}%
\pgfsetlinewidth{0.000000pt}%
\definecolor{currentstroke}{rgb}{0.000000,0.000000,0.000000}%
\pgfsetstrokecolor{currentstroke}%
\pgfsetdash{}{0pt}%
\pgfpathmoveto{\pgfqpoint{4.276405in}{1.155951in}}%
\pgfpathlineto{\pgfqpoint{4.470681in}{1.263928in}}%
\pgfpathlineto{\pgfqpoint{4.453885in}{1.275416in}}%
\pgfpathlineto{\pgfqpoint{4.546691in}{1.295762in}}%
\pgfpathlineto{\pgfqpoint{4.480410in}{1.227690in}}%
\pgfpathlineto{\pgfqpoint{4.479523in}{1.248019in}}%
\pgfpathlineto{\pgfqpoint{4.285247in}{1.140043in}}%
\pgfpathlineto{\pgfqpoint{4.276405in}{1.155951in}}%
\pgfusepath{fill}%
\end{pgfscope}%
\begin{pgfscope}%
\pgfpathrectangle{\pgfqpoint{1.065196in}{0.528000in}}{\pgfqpoint{3.702804in}{3.696000in}} %
\pgfusepath{clip}%
\pgfsetbuttcap%
\pgfsetroundjoin%
\definecolor{currentfill}{rgb}{0.283197,0.115680,0.436115}%
\pgfsetfillcolor{currentfill}%
\pgfsetlinewidth{0.000000pt}%
\definecolor{currentstroke}{rgb}{0.000000,0.000000,0.000000}%
\pgfsetstrokecolor{currentstroke}%
\pgfsetdash{}{0pt}%
\pgfpathmoveto{\pgfqpoint{4.275277in}{1.155210in}}%
\pgfpathlineto{\pgfqpoint{4.499442in}{1.327667in}}%
\pgfpathlineto{\pgfqpoint{4.481131in}{1.336544in}}%
\pgfpathlineto{\pgfqpoint{4.569907in}{1.370396in}}%
\pgfpathlineto{\pgfqpoint{4.514426in}{1.293267in}}%
\pgfpathlineto{\pgfqpoint{4.510540in}{1.313241in}}%
\pgfpathlineto{\pgfqpoint{4.286375in}{1.140784in}}%
\pgfpathlineto{\pgfqpoint{4.275277in}{1.155210in}}%
\pgfusepath{fill}%
\end{pgfscope}%
\begin{pgfscope}%
\pgfpathrectangle{\pgfqpoint{1.065196in}{0.528000in}}{\pgfqpoint{3.702804in}{3.696000in}} %
\pgfusepath{clip}%
\pgfsetbuttcap%
\pgfsetroundjoin%
\definecolor{currentfill}{rgb}{0.223925,0.334994,0.548053}%
\pgfsetfillcolor{currentfill}%
\pgfsetlinewidth{0.000000pt}%
\definecolor{currentstroke}{rgb}{0.000000,0.000000,0.000000}%
\pgfsetstrokecolor{currentstroke}%
\pgfsetdash{}{0pt}%
\pgfpathmoveto{\pgfqpoint{1.731754in}{3.391592in}}%
\pgfpathlineto{\pgfqpoint{1.741703in}{3.286114in}}%
\pgfpathlineto{\pgfqpoint{1.758968in}{3.296883in}}%
\pgfpathlineto{\pgfqpoint{1.740334in}{3.203718in}}%
\pgfpathlineto{\pgfqpoint{1.704607in}{3.291756in}}%
\pgfpathlineto{\pgfqpoint{1.723582in}{3.284405in}}%
\pgfpathlineto{\pgfqpoint{1.713633in}{3.389883in}}%
\pgfpathlineto{\pgfqpoint{1.731754in}{3.391592in}}%
\pgfusepath{fill}%
\end{pgfscope}%
\begin{pgfscope}%
\pgfpathrectangle{\pgfqpoint{1.065196in}{0.528000in}}{\pgfqpoint{3.702804in}{3.696000in}} %
\pgfusepath{clip}%
\pgfsetbuttcap%
\pgfsetroundjoin%
\definecolor{currentfill}{rgb}{0.123444,0.636809,0.528763}%
\pgfsetfillcolor{currentfill}%
\pgfsetlinewidth{0.000000pt}%
\definecolor{currentstroke}{rgb}{0.000000,0.000000,0.000000}%
\pgfsetstrokecolor{currentstroke}%
\pgfsetdash{}{0pt}%
\pgfpathmoveto{\pgfqpoint{1.731304in}{3.393682in}}%
\pgfpathlineto{\pgfqpoint{1.775503in}{3.264439in}}%
\pgfpathlineto{\pgfqpoint{1.789780in}{3.278939in}}%
\pgfpathlineto{\pgfqpoint{1.793395in}{3.183997in}}%
\pgfpathlineto{\pgfqpoint{1.738115in}{3.261271in}}%
\pgfpathlineto{\pgfqpoint{1.758282in}{3.258550in}}%
\pgfpathlineto{\pgfqpoint{1.714083in}{3.387792in}}%
\pgfpathlineto{\pgfqpoint{1.731304in}{3.393682in}}%
\pgfusepath{fill}%
\end{pgfscope}%
\begin{pgfscope}%
\pgfpathrectangle{\pgfqpoint{1.065196in}{0.528000in}}{\pgfqpoint{3.702804in}{3.696000in}} %
\pgfusepath{clip}%
\pgfsetbuttcap%
\pgfsetroundjoin%
\definecolor{currentfill}{rgb}{0.273006,0.204520,0.501721}%
\pgfsetfillcolor{currentfill}%
\pgfsetlinewidth{0.000000pt}%
\definecolor{currentstroke}{rgb}{0.000000,0.000000,0.000000}%
\pgfsetstrokecolor{currentstroke}%
\pgfsetdash{}{0pt}%
\pgfpathmoveto{\pgfqpoint{4.364895in}{1.843552in}}%
\pgfpathlineto{\pgfqpoint{4.229984in}{1.789860in}}%
\pgfpathlineto{\pgfqpoint{4.245170in}{1.776314in}}%
\pgfpathlineto{\pgfqpoint{4.150521in}{1.768029in}}%
\pgfpathlineto{\pgfqpoint{4.224979in}{1.827046in}}%
\pgfpathlineto{\pgfqpoint{4.223254in}{1.806770in}}%
\pgfpathlineto{\pgfqpoint{4.358165in}{1.860463in}}%
\pgfpathlineto{\pgfqpoint{4.364895in}{1.843552in}}%
\pgfusepath{fill}%
\end{pgfscope}%
\begin{pgfscope}%
\pgfpathrectangle{\pgfqpoint{1.065196in}{0.528000in}}{\pgfqpoint{3.702804in}{3.696000in}} %
\pgfusepath{clip}%
\pgfsetbuttcap%
\pgfsetroundjoin%
\definecolor{currentfill}{rgb}{0.121148,0.592739,0.544641}%
\pgfsetfillcolor{currentfill}%
\pgfsetlinewidth{0.000000pt}%
\definecolor{currentstroke}{rgb}{0.000000,0.000000,0.000000}%
\pgfsetstrokecolor{currentstroke}%
\pgfsetdash{}{0pt}%
\pgfpathmoveto{\pgfqpoint{4.369882in}{1.855621in}}%
\pgfpathlineto{\pgfqpoint{4.545736in}{1.449179in}}%
\pgfpathlineto{\pgfqpoint{4.558826in}{1.464758in}}%
\pgfpathlineto{\pgfqpoint{4.569907in}{1.370396in}}%
\pgfpathlineto{\pgfqpoint{4.508714in}{1.443076in}}%
\pgfpathlineto{\pgfqpoint{4.529032in}{1.441951in}}%
\pgfpathlineto{\pgfqpoint{4.353178in}{1.848394in}}%
\pgfpathlineto{\pgfqpoint{4.369882in}{1.855621in}}%
\pgfusepath{fill}%
\end{pgfscope}%
\begin{pgfscope}%
\pgfpathrectangle{\pgfqpoint{1.065196in}{0.528000in}}{\pgfqpoint{3.702804in}{3.696000in}} %
\pgfusepath{clip}%
\pgfsetbuttcap%
\pgfsetroundjoin%
\definecolor{currentfill}{rgb}{0.276022,0.044167,0.370164}%
\pgfsetfillcolor{currentfill}%
\pgfsetlinewidth{0.000000pt}%
\definecolor{currentstroke}{rgb}{0.000000,0.000000,0.000000}%
\pgfsetstrokecolor{currentstroke}%
\pgfsetdash{}{0pt}%
\pgfpathmoveto{\pgfqpoint{3.778983in}{3.119624in}}%
\pgfpathlineto{\pgfqpoint{3.780905in}{3.107048in}}%
\pgfpathlineto{\pgfqpoint{3.797522in}{3.118793in}}%
\pgfpathlineto{\pgfqpoint{3.784282in}{3.024710in}}%
\pgfpathlineto{\pgfqpoint{3.743546in}{3.110544in}}%
\pgfpathlineto{\pgfqpoint{3.762913in}{3.104298in}}%
\pgfpathlineto{\pgfqpoint{3.760991in}{3.116875in}}%
\pgfpathlineto{\pgfqpoint{3.778983in}{3.119624in}}%
\pgfusepath{fill}%
\end{pgfscope}%
\begin{pgfscope}%
\pgfpathrectangle{\pgfqpoint{1.065196in}{0.528000in}}{\pgfqpoint{3.702804in}{3.696000in}} %
\pgfusepath{clip}%
\pgfsetbuttcap%
\pgfsetroundjoin%
\definecolor{currentfill}{rgb}{0.119483,0.614817,0.537692}%
\pgfsetfillcolor{currentfill}%
\pgfsetlinewidth{0.000000pt}%
\definecolor{currentstroke}{rgb}{0.000000,0.000000,0.000000}%
\pgfsetstrokecolor{currentstroke}%
\pgfsetdash{}{0pt}%
\pgfpathmoveto{\pgfqpoint{4.214652in}{2.766651in}}%
\pgfpathlineto{\pgfqpoint{3.976650in}{2.737602in}}%
\pgfpathlineto{\pgfqpoint{3.987888in}{2.720638in}}%
\pgfpathlineto{\pgfqpoint{3.894247in}{2.736713in}}%
\pgfpathlineto{\pgfqpoint{3.981273in}{2.774838in}}%
\pgfpathlineto{\pgfqpoint{3.974445in}{2.755669in}}%
\pgfpathlineto{\pgfqpoint{4.212447in}{2.784718in}}%
\pgfpathlineto{\pgfqpoint{4.214652in}{2.766651in}}%
\pgfusepath{fill}%
\end{pgfscope}%
\begin{pgfscope}%
\pgfpathrectangle{\pgfqpoint{1.065196in}{0.528000in}}{\pgfqpoint{3.702804in}{3.696000in}} %
\pgfusepath{clip}%
\pgfsetbuttcap%
\pgfsetroundjoin%
\definecolor{currentfill}{rgb}{0.122606,0.585371,0.546557}%
\pgfsetfillcolor{currentfill}%
\pgfsetlinewidth{0.000000pt}%
\definecolor{currentstroke}{rgb}{0.000000,0.000000,0.000000}%
\pgfsetstrokecolor{currentstroke}%
\pgfsetdash{}{0pt}%
\pgfpathmoveto{\pgfqpoint{2.163625in}{3.695384in}}%
\pgfpathlineto{\pgfqpoint{2.270397in}{3.650341in}}%
\pgfpathlineto{\pgfqpoint{2.269087in}{3.670648in}}%
\pgfpathlineto{\pgfqpoint{2.342323in}{3.610122in}}%
\pgfpathlineto{\pgfqpoint{2.247863in}{3.620339in}}%
\pgfpathlineto{\pgfqpoint{2.263323in}{3.633572in}}%
\pgfpathlineto{\pgfqpoint{2.156550in}{3.678615in}}%
\pgfpathlineto{\pgfqpoint{2.163625in}{3.695384in}}%
\pgfusepath{fill}%
\end{pgfscope}%
\begin{pgfscope}%
\pgfpathrectangle{\pgfqpoint{1.065196in}{0.528000in}}{\pgfqpoint{3.702804in}{3.696000in}} %
\pgfusepath{clip}%
\pgfsetbuttcap%
\pgfsetroundjoin%
\definecolor{currentfill}{rgb}{0.119512,0.607464,0.540218}%
\pgfsetfillcolor{currentfill}%
\pgfsetlinewidth{0.000000pt}%
\definecolor{currentstroke}{rgb}{0.000000,0.000000,0.000000}%
\pgfsetstrokecolor{currentstroke}%
\pgfsetdash{}{0pt}%
\pgfpathmoveto{\pgfqpoint{3.485698in}{2.772995in}}%
\pgfpathlineto{\pgfqpoint{3.533954in}{2.672120in}}%
\pgfpathlineto{\pgfqpoint{3.546445in}{2.688184in}}%
\pgfpathlineto{\pgfqpoint{3.561089in}{2.594308in}}%
\pgfpathlineto{\pgfqpoint{3.497189in}{2.664621in}}%
\pgfpathlineto{\pgfqpoint{3.517535in}{2.664266in}}%
\pgfpathlineto{\pgfqpoint{3.469279in}{2.765140in}}%
\pgfpathlineto{\pgfqpoint{3.485698in}{2.772995in}}%
\pgfusepath{fill}%
\end{pgfscope}%
\begin{pgfscope}%
\pgfpathrectangle{\pgfqpoint{1.065196in}{0.528000in}}{\pgfqpoint{3.702804in}{3.696000in}} %
\pgfusepath{clip}%
\pgfsetbuttcap%
\pgfsetroundjoin%
\definecolor{currentfill}{rgb}{0.183898,0.422383,0.556944}%
\pgfsetfillcolor{currentfill}%
\pgfsetlinewidth{0.000000pt}%
\definecolor{currentstroke}{rgb}{0.000000,0.000000,0.000000}%
\pgfsetstrokecolor{currentstroke}%
\pgfsetdash{}{0pt}%
\pgfpathmoveto{\pgfqpoint{3.479990in}{2.777817in}}%
\pgfpathlineto{\pgfqpoint{3.535404in}{2.761978in}}%
\pgfpathlineto{\pgfqpoint{3.531657in}{2.781979in}}%
\pgfpathlineto{\pgfqpoint{3.611653in}{2.730719in}}%
\pgfpathlineto{\pgfqpoint{3.516650in}{2.729479in}}%
\pgfpathlineto{\pgfqpoint{3.530402in}{2.744478in}}%
\pgfpathlineto{\pgfqpoint{3.474988in}{2.760317in}}%
\pgfpathlineto{\pgfqpoint{3.479990in}{2.777817in}}%
\pgfusepath{fill}%
\end{pgfscope}%
\begin{pgfscope}%
\pgfpathrectangle{\pgfqpoint{1.065196in}{0.528000in}}{\pgfqpoint{3.702804in}{3.696000in}} %
\pgfusepath{clip}%
\pgfsetbuttcap%
\pgfsetroundjoin%
\definecolor{currentfill}{rgb}{0.277941,0.056324,0.381191}%
\pgfsetfillcolor{currentfill}%
\pgfsetlinewidth{0.000000pt}%
\definecolor{currentstroke}{rgb}{0.000000,0.000000,0.000000}%
\pgfsetstrokecolor{currentstroke}%
\pgfsetdash{}{0pt}%
\pgfpathmoveto{\pgfqpoint{1.649611in}{3.867501in}}%
\pgfpathlineto{\pgfqpoint{1.694571in}{3.480172in}}%
\pgfpathlineto{\pgfqpoint{1.711601in}{3.491310in}}%
\pgfpathlineto{\pgfqpoint{1.694975in}{3.397766in}}%
\pgfpathlineto{\pgfqpoint{1.657363in}{3.485014in}}%
\pgfpathlineto{\pgfqpoint{1.676492in}{3.478073in}}%
\pgfpathlineto{\pgfqpoint{1.631531in}{3.865402in}}%
\pgfpathlineto{\pgfqpoint{1.649611in}{3.867501in}}%
\pgfusepath{fill}%
\end{pgfscope}%
\begin{pgfscope}%
\pgfpathrectangle{\pgfqpoint{1.065196in}{0.528000in}}{\pgfqpoint{3.702804in}{3.696000in}} %
\pgfusepath{clip}%
\pgfsetbuttcap%
\pgfsetroundjoin%
\definecolor{currentfill}{rgb}{0.278791,0.062145,0.386592}%
\pgfsetfillcolor{currentfill}%
\pgfsetlinewidth{0.000000pt}%
\definecolor{currentstroke}{rgb}{0.000000,0.000000,0.000000}%
\pgfsetstrokecolor{currentstroke}%
\pgfsetdash{}{0pt}%
\pgfpathmoveto{\pgfqpoint{1.643043in}{3.875210in}}%
\pgfpathlineto{\pgfqpoint{2.235449in}{3.707974in}}%
\pgfpathlineto{\pgfqpoint{2.231635in}{3.727963in}}%
\pgfpathlineto{\pgfqpoint{2.311799in}{3.676965in}}%
\pgfpathlineto{\pgfqpoint{2.216801in}{3.675414in}}%
\pgfpathlineto{\pgfqpoint{2.230504in}{3.690458in}}%
\pgfpathlineto{\pgfqpoint{1.638099in}{3.857693in}}%
\pgfpathlineto{\pgfqpoint{1.643043in}{3.875210in}}%
\pgfusepath{fill}%
\end{pgfscope}%
\begin{pgfscope}%
\pgfpathrectangle{\pgfqpoint{1.065196in}{0.528000in}}{\pgfqpoint{3.702804in}{3.696000in}} %
\pgfusepath{clip}%
\pgfsetbuttcap%
\pgfsetroundjoin%
\definecolor{currentfill}{rgb}{0.136408,0.541173,0.554483}%
\pgfsetfillcolor{currentfill}%
\pgfsetlinewidth{0.000000pt}%
\definecolor{currentstroke}{rgb}{0.000000,0.000000,0.000000}%
\pgfsetstrokecolor{currentstroke}%
\pgfsetdash{}{0pt}%
\pgfpathmoveto{\pgfqpoint{1.645726in}{3.873951in}}%
\pgfpathlineto{\pgfqpoint{2.038954in}{3.603672in}}%
\pgfpathlineto{\pgfqpoint{2.041764in}{3.623826in}}%
\pgfpathlineto{\pgfqpoint{2.101296in}{3.549780in}}%
\pgfpathlineto{\pgfqpoint{2.010835in}{3.578828in}}%
\pgfpathlineto{\pgfqpoint{2.028645in}{3.588673in}}%
\pgfpathlineto{\pgfqpoint{1.635416in}{3.858952in}}%
\pgfpathlineto{\pgfqpoint{1.645726in}{3.873951in}}%
\pgfusepath{fill}%
\end{pgfscope}%
\begin{pgfscope}%
\pgfpathrectangle{\pgfqpoint{1.065196in}{0.528000in}}{\pgfqpoint{3.702804in}{3.696000in}} %
\pgfusepath{clip}%
\pgfsetbuttcap%
\pgfsetroundjoin%
\definecolor{currentfill}{rgb}{0.279566,0.067836,0.391917}%
\pgfsetfillcolor{currentfill}%
\pgfsetlinewidth{0.000000pt}%
\definecolor{currentstroke}{rgb}{0.000000,0.000000,0.000000}%
\pgfsetstrokecolor{currentstroke}%
\pgfsetdash{}{0pt}%
\pgfpathmoveto{\pgfqpoint{1.645955in}{3.873788in}}%
\pgfpathlineto{\pgfqpoint{2.026144in}{3.594774in}}%
\pgfpathlineto{\pgfqpoint{2.029576in}{3.614832in}}%
\pgfpathlineto{\pgfqpoint{2.086790in}{3.538979in}}%
\pgfpathlineto{\pgfqpoint{1.997270in}{3.570812in}}%
\pgfpathlineto{\pgfqpoint{2.015376in}{3.580101in}}%
\pgfpathlineto{\pgfqpoint{1.635187in}{3.859115in}}%
\pgfpathlineto{\pgfqpoint{1.645955in}{3.873788in}}%
\pgfusepath{fill}%
\end{pgfscope}%
\begin{pgfscope}%
\pgfpathrectangle{\pgfqpoint{1.065196in}{0.528000in}}{\pgfqpoint{3.702804in}{3.696000in}} %
\pgfusepath{clip}%
\pgfsetbuttcap%
\pgfsetroundjoin%
\definecolor{currentfill}{rgb}{0.144759,0.519093,0.556572}%
\pgfsetfillcolor{currentfill}%
\pgfsetlinewidth{0.000000pt}%
\definecolor{currentstroke}{rgb}{0.000000,0.000000,0.000000}%
\pgfsetstrokecolor{currentstroke}%
\pgfsetdash{}{0pt}%
\pgfpathmoveto{\pgfqpoint{2.768896in}{2.630694in}}%
\pgfpathlineto{\pgfqpoint{3.107642in}{2.309793in}}%
\pgfpathlineto{\pgfqpoint{3.113552in}{2.329265in}}%
\pgfpathlineto{\pgfqpoint{3.160843in}{2.246859in}}%
\pgfpathlineto{\pgfqpoint{3.076001in}{2.289625in}}%
\pgfpathlineto{\pgfqpoint{3.095125in}{2.296580in}}%
\pgfpathlineto{\pgfqpoint{2.756378in}{2.617481in}}%
\pgfpathlineto{\pgfqpoint{2.768896in}{2.630694in}}%
\pgfusepath{fill}%
\end{pgfscope}%
\begin{pgfscope}%
\pgfpathrectangle{\pgfqpoint{1.065196in}{0.528000in}}{\pgfqpoint{3.702804in}{3.696000in}} %
\pgfusepath{clip}%
\pgfsetbuttcap%
\pgfsetroundjoin%
\definecolor{currentfill}{rgb}{0.227802,0.326594,0.546532}%
\pgfsetfillcolor{currentfill}%
\pgfsetlinewidth{0.000000pt}%
\definecolor{currentstroke}{rgb}{0.000000,0.000000,0.000000}%
\pgfsetstrokecolor{currentstroke}%
\pgfsetdash{}{0pt}%
\pgfpathmoveto{\pgfqpoint{2.770263in}{2.629054in}}%
\pgfpathlineto{\pgfqpoint{2.946332in}{2.358725in}}%
\pgfpathlineto{\pgfqpoint{2.956617in}{2.376284in}}%
\pgfpathlineto{\pgfqpoint{2.983407in}{2.285128in}}%
\pgfpathlineto{\pgfqpoint{2.910863in}{2.346484in}}%
\pgfpathlineto{\pgfqpoint{2.931081in}{2.348792in}}%
\pgfpathlineto{\pgfqpoint{2.755011in}{2.619121in}}%
\pgfpathlineto{\pgfqpoint{2.770263in}{2.629054in}}%
\pgfusepath{fill}%
\end{pgfscope}%
\begin{pgfscope}%
\pgfpathrectangle{\pgfqpoint{1.065196in}{0.528000in}}{\pgfqpoint{3.702804in}{3.696000in}} %
\pgfusepath{clip}%
\pgfsetbuttcap%
\pgfsetroundjoin%
\definecolor{currentfill}{rgb}{0.269944,0.014625,0.341379}%
\pgfsetfillcolor{currentfill}%
\pgfsetlinewidth{0.000000pt}%
\definecolor{currentstroke}{rgb}{0.000000,0.000000,0.000000}%
\pgfsetstrokecolor{currentstroke}%
\pgfsetdash{}{0pt}%
\pgfpathmoveto{\pgfqpoint{2.613782in}{1.479929in}}%
\pgfpathlineto{\pgfqpoint{2.538760in}{1.980723in}}%
\pgfpathlineto{\pgfqpoint{2.522108in}{1.969027in}}%
\pgfpathlineto{\pgfqpoint{2.535625in}{2.063071in}}%
\pgfpathlineto{\pgfqpoint{2.576108in}{1.977116in}}%
\pgfpathlineto{\pgfqpoint{2.556760in}{1.983420in}}%
\pgfpathlineto{\pgfqpoint{2.631782in}{1.482625in}}%
\pgfpathlineto{\pgfqpoint{2.613782in}{1.479929in}}%
\pgfusepath{fill}%
\end{pgfscope}%
\begin{pgfscope}%
\pgfpathrectangle{\pgfqpoint{1.065196in}{0.528000in}}{\pgfqpoint{3.702804in}{3.696000in}} %
\pgfusepath{clip}%
\pgfsetbuttcap%
\pgfsetroundjoin%
\definecolor{currentfill}{rgb}{0.232815,0.732247,0.459277}%
\pgfsetfillcolor{currentfill}%
\pgfsetlinewidth{0.000000pt}%
\definecolor{currentstroke}{rgb}{0.000000,0.000000,0.000000}%
\pgfsetstrokecolor{currentstroke}%
\pgfsetdash{}{0pt}%
\pgfpathmoveto{\pgfqpoint{2.613713in}{1.480526in}}%
\pgfpathlineto{\pgfqpoint{2.598188in}{1.667920in}}%
\pgfpathlineto{\pgfqpoint{2.580801in}{1.657348in}}%
\pgfpathlineto{\pgfqpoint{2.600495in}{1.750295in}}%
\pgfpathlineto{\pgfqpoint{2.635216in}{1.661856in}}%
\pgfpathlineto{\pgfqpoint{2.616326in}{1.669423in}}%
\pgfpathlineto{\pgfqpoint{2.631851in}{1.482028in}}%
\pgfpathlineto{\pgfqpoint{2.613713in}{1.480526in}}%
\pgfusepath{fill}%
\end{pgfscope}%
\begin{pgfscope}%
\pgfpathrectangle{\pgfqpoint{1.065196in}{0.528000in}}{\pgfqpoint{3.702804in}{3.696000in}} %
\pgfusepath{clip}%
\pgfsetbuttcap%
\pgfsetroundjoin%
\definecolor{currentfill}{rgb}{0.277941,0.056324,0.381191}%
\pgfsetfillcolor{currentfill}%
\pgfsetlinewidth{0.000000pt}%
\definecolor{currentstroke}{rgb}{0.000000,0.000000,0.000000}%
\pgfsetstrokecolor{currentstroke}%
\pgfsetdash{}{0pt}%
\pgfpathmoveto{\pgfqpoint{2.624484in}{1.472337in}}%
\pgfpathlineto{\pgfqpoint{2.494149in}{1.447525in}}%
\pgfpathlineto{\pgfqpoint{2.506493in}{1.431347in}}%
\pgfpathlineto{\pgfqpoint{2.411989in}{1.441147in}}%
\pgfpathlineto{\pgfqpoint{2.496281in}{1.484986in}}%
\pgfpathlineto{\pgfqpoint{2.490745in}{1.465404in}}%
\pgfpathlineto{\pgfqpoint{2.621080in}{1.490217in}}%
\pgfpathlineto{\pgfqpoint{2.624484in}{1.472337in}}%
\pgfusepath{fill}%
\end{pgfscope}%
\begin{pgfscope}%
\pgfpathrectangle{\pgfqpoint{1.065196in}{0.528000in}}{\pgfqpoint{3.702804in}{3.696000in}} %
\pgfusepath{clip}%
\pgfsetbuttcap%
\pgfsetroundjoin%
\definecolor{currentfill}{rgb}{0.280894,0.078907,0.402329}%
\pgfsetfillcolor{currentfill}%
\pgfsetlinewidth{0.000000pt}%
\definecolor{currentstroke}{rgb}{0.000000,0.000000,0.000000}%
\pgfsetstrokecolor{currentstroke}%
\pgfsetdash{}{0pt}%
\pgfpathmoveto{\pgfqpoint{4.287728in}{2.602477in}}%
\pgfpathlineto{\pgfqpoint{3.982797in}{2.238317in}}%
\pgfpathlineto{\pgfqpoint{4.002594in}{2.233610in}}%
\pgfpathlineto{\pgfqpoint{3.923237in}{2.181364in}}%
\pgfpathlineto{\pgfqpoint{3.960730in}{2.268665in}}%
\pgfpathlineto{\pgfqpoint{3.968842in}{2.250002in}}%
\pgfpathlineto{\pgfqpoint{4.273773in}{2.614162in}}%
\pgfpathlineto{\pgfqpoint{4.287728in}{2.602477in}}%
\pgfusepath{fill}%
\end{pgfscope}%
\begin{pgfscope}%
\pgfpathrectangle{\pgfqpoint{1.065196in}{0.528000in}}{\pgfqpoint{3.702804in}{3.696000in}} %
\pgfusepath{clip}%
\pgfsetbuttcap%
\pgfsetroundjoin%
\definecolor{currentfill}{rgb}{0.282656,0.100196,0.422160}%
\pgfsetfillcolor{currentfill}%
\pgfsetlinewidth{0.000000pt}%
\definecolor{currentstroke}{rgb}{0.000000,0.000000,0.000000}%
\pgfsetstrokecolor{currentstroke}%
\pgfsetdash{}{0pt}%
\pgfpathmoveto{\pgfqpoint{1.273957in}{2.758072in}}%
\pgfpathlineto{\pgfqpoint{1.278196in}{2.750149in}}%
\pgfpathlineto{\pgfqpoint{1.289801in}{2.766550in}}%
\pgfpathlineto{\pgfqpoint{1.308428in}{2.674612in}}%
\pgfpathlineto{\pgfqpoint{1.242269in}{2.741114in}}%
\pgfpathlineto{\pgfqpoint{1.262352in}{2.741671in}}%
\pgfpathlineto{\pgfqpoint{1.258113in}{2.749593in}}%
\pgfpathlineto{\pgfqpoint{1.273957in}{2.758072in}}%
\pgfusepath{fill}%
\end{pgfscope}%
\begin{pgfscope}%
\pgfpathrectangle{\pgfqpoint{1.065196in}{0.528000in}}{\pgfqpoint{3.702804in}{3.696000in}} %
\pgfusepath{clip}%
\pgfsetbuttcap%
\pgfsetroundjoin%
\definecolor{currentfill}{rgb}{0.282290,0.145912,0.461510}%
\pgfsetfillcolor{currentfill}%
\pgfsetlinewidth{0.000000pt}%
\definecolor{currentstroke}{rgb}{0.000000,0.000000,0.000000}%
\pgfsetstrokecolor{currentstroke}%
\pgfsetdash{}{0pt}%
\pgfpathmoveto{\pgfqpoint{2.354914in}{2.459877in}}%
\pgfpathlineto{\pgfqpoint{2.498309in}{2.366938in}}%
\pgfpathlineto{\pgfqpoint{2.500572in}{2.387161in}}%
\pgfpathlineto{\pgfqpoint{2.562090in}{2.314755in}}%
\pgfpathlineto{\pgfqpoint{2.470874in}{2.341341in}}%
\pgfpathlineto{\pgfqpoint{2.488410in}{2.351665in}}%
\pgfpathlineto{\pgfqpoint{2.345015in}{2.444604in}}%
\pgfpathlineto{\pgfqpoint{2.354914in}{2.459877in}}%
\pgfusepath{fill}%
\end{pgfscope}%
\begin{pgfscope}%
\pgfpathrectangle{\pgfqpoint{1.065196in}{0.528000in}}{\pgfqpoint{3.702804in}{3.696000in}} %
\pgfusepath{clip}%
\pgfsetbuttcap%
\pgfsetroundjoin%
\definecolor{currentfill}{rgb}{0.153894,0.680203,0.504172}%
\pgfsetfillcolor{currentfill}%
\pgfsetlinewidth{0.000000pt}%
\definecolor{currentstroke}{rgb}{0.000000,0.000000,0.000000}%
\pgfsetstrokecolor{currentstroke}%
\pgfsetdash{}{0pt}%
\pgfpathmoveto{\pgfqpoint{4.230107in}{1.879026in}}%
\pgfpathlineto{\pgfqpoint{4.201438in}{1.832824in}}%
\pgfpathlineto{\pgfqpoint{4.221701in}{1.830961in}}%
\pgfpathlineto{\pgfqpoint{4.150521in}{1.768029in}}%
\pgfpathlineto{\pgfqpoint{4.175305in}{1.859750in}}%
\pgfpathlineto{\pgfqpoint{4.185972in}{1.842421in}}%
\pgfpathlineto{\pgfqpoint{4.214641in}{1.888623in}}%
\pgfpathlineto{\pgfqpoint{4.230107in}{1.879026in}}%
\pgfusepath{fill}%
\end{pgfscope}%
\begin{pgfscope}%
\pgfpathrectangle{\pgfqpoint{1.065196in}{0.528000in}}{\pgfqpoint{3.702804in}{3.696000in}} %
\pgfusepath{clip}%
\pgfsetbuttcap%
\pgfsetroundjoin%
\definecolor{currentfill}{rgb}{0.267968,0.223549,0.512008}%
\pgfsetfillcolor{currentfill}%
\pgfsetlinewidth{0.000000pt}%
\definecolor{currentstroke}{rgb}{0.000000,0.000000,0.000000}%
\pgfsetstrokecolor{currentstroke}%
\pgfsetdash{}{0pt}%
\pgfpathmoveto{\pgfqpoint{4.291916in}{2.777007in}}%
\pgfpathlineto{\pgfqpoint{4.294283in}{2.783102in}}%
\pgfpathlineto{\pgfqpoint{4.279726in}{2.781741in}}%
\pgfpathlineto{\pgfqpoint{4.321684in}{2.835590in}}%
\pgfpathlineto{\pgfqpoint{4.316296in}{2.767538in}}%
\pgfpathlineto{\pgfqpoint{4.306473in}{2.778367in}}%
\pgfpathlineto{\pgfqpoint{4.304106in}{2.772272in}}%
\pgfpathlineto{\pgfqpoint{4.291916in}{2.777007in}}%
\pgfusepath{fill}%
\end{pgfscope}%
\begin{pgfscope}%
\pgfpathrectangle{\pgfqpoint{1.065196in}{0.528000in}}{\pgfqpoint{3.702804in}{3.696000in}} %
\pgfusepath{clip}%
\pgfsetbuttcap%
\pgfsetroundjoin%
\definecolor{currentfill}{rgb}{0.147607,0.511733,0.557049}%
\pgfsetfillcolor{currentfill}%
\pgfsetlinewidth{0.000000pt}%
\definecolor{currentstroke}{rgb}{0.000000,0.000000,0.000000}%
\pgfsetstrokecolor{currentstroke}%
\pgfsetdash{}{0pt}%
\pgfpathmoveto{\pgfqpoint{4.298862in}{2.765579in}}%
\pgfpathlineto{\pgfqpoint{3.976643in}{2.735312in}}%
\pgfpathlineto{\pgfqpoint{3.987405in}{2.718042in}}%
\pgfpathlineto{\pgfqpoint{3.894247in}{2.736713in}}%
\pgfpathlineto{\pgfqpoint{3.982299in}{2.772405in}}%
\pgfpathlineto{\pgfqpoint{3.974941in}{2.753433in}}%
\pgfpathlineto{\pgfqpoint{4.297160in}{2.783700in}}%
\pgfpathlineto{\pgfqpoint{4.298862in}{2.765579in}}%
\pgfusepath{fill}%
\end{pgfscope}%
\begin{pgfscope}%
\pgfpathrectangle{\pgfqpoint{1.065196in}{0.528000in}}{\pgfqpoint{3.702804in}{3.696000in}} %
\pgfusepath{clip}%
\pgfsetbuttcap%
\pgfsetroundjoin%
\definecolor{currentfill}{rgb}{0.412913,0.803041,0.357269}%
\pgfsetfillcolor{currentfill}%
\pgfsetlinewidth{0.000000pt}%
\definecolor{currentstroke}{rgb}{0.000000,0.000000,0.000000}%
\pgfsetstrokecolor{currentstroke}%
\pgfsetdash{}{0pt}%
\pgfpathmoveto{\pgfqpoint{1.303187in}{3.801231in}}%
\pgfpathlineto{\pgfqpoint{1.305097in}{3.807435in}}%
\pgfpathlineto{\pgfqpoint{1.290778in}{3.805050in}}%
\pgfpathlineto{\pgfqpoint{1.328486in}{3.861366in}}%
\pgfpathlineto{\pgfqpoint{1.328005in}{3.793593in}}%
\pgfpathlineto{\pgfqpoint{1.317506in}{3.803616in}}%
\pgfpathlineto{\pgfqpoint{1.315596in}{3.797412in}}%
\pgfpathlineto{\pgfqpoint{1.303187in}{3.801231in}}%
\pgfusepath{fill}%
\end{pgfscope}%
\begin{pgfscope}%
\pgfpathrectangle{\pgfqpoint{1.065196in}{0.528000in}}{\pgfqpoint{3.702804in}{3.696000in}} %
\pgfusepath{clip}%
\pgfsetbuttcap%
\pgfsetroundjoin%
\definecolor{currentfill}{rgb}{0.278826,0.175490,0.483397}%
\pgfsetfillcolor{currentfill}%
\pgfsetlinewidth{0.000000pt}%
\definecolor{currentstroke}{rgb}{0.000000,0.000000,0.000000}%
\pgfsetstrokecolor{currentstroke}%
\pgfsetdash{}{0pt}%
\pgfpathmoveto{\pgfqpoint{1.315956in}{3.805625in}}%
\pgfpathlineto{\pgfqpoint{1.644812in}{3.463146in}}%
\pgfpathlineto{\pgfqpoint{1.651637in}{3.482316in}}%
\pgfpathlineto{\pgfqpoint{1.694975in}{3.397766in}}%
\pgfpathlineto{\pgfqpoint{1.612252in}{3.444498in}}%
\pgfpathlineto{\pgfqpoint{1.631684in}{3.450540in}}%
\pgfpathlineto{\pgfqpoint{1.302828in}{3.793018in}}%
\pgfpathlineto{\pgfqpoint{1.315956in}{3.805625in}}%
\pgfusepath{fill}%
\end{pgfscope}%
\begin{pgfscope}%
\pgfpathrectangle{\pgfqpoint{1.065196in}{0.528000in}}{\pgfqpoint{3.702804in}{3.696000in}} %
\pgfusepath{clip}%
\pgfsetbuttcap%
\pgfsetroundjoin%
\definecolor{currentfill}{rgb}{0.267968,0.223549,0.512008}%
\pgfsetfillcolor{currentfill}%
\pgfsetlinewidth{0.000000pt}%
\definecolor{currentstroke}{rgb}{0.000000,0.000000,0.000000}%
\pgfsetstrokecolor{currentstroke}%
\pgfsetdash{}{0pt}%
\pgfpathmoveto{\pgfqpoint{3.574072in}{4.034400in}}%
\pgfpathlineto{\pgfqpoint{4.018781in}{3.768650in}}%
\pgfpathlineto{\pgfqpoint{4.020306in}{3.788942in}}%
\pgfpathlineto{\pgfqpoint{4.084419in}{3.718824in}}%
\pgfpathlineto{\pgfqpoint{3.992297in}{3.742071in}}%
\pgfpathlineto{\pgfqpoint{4.009445in}{3.753026in}}%
\pgfpathlineto{\pgfqpoint{3.564735in}{4.018776in}}%
\pgfpathlineto{\pgfqpoint{3.574072in}{4.034400in}}%
\pgfusepath{fill}%
\end{pgfscope}%
\begin{pgfscope}%
\pgfpathrectangle{\pgfqpoint{1.065196in}{0.528000in}}{\pgfqpoint{3.702804in}{3.696000in}} %
\pgfusepath{clip}%
\pgfsetbuttcap%
\pgfsetroundjoin%
\definecolor{currentfill}{rgb}{0.194100,0.399323,0.555565}%
\pgfsetfillcolor{currentfill}%
\pgfsetlinewidth{0.000000pt}%
\definecolor{currentstroke}{rgb}{0.000000,0.000000,0.000000}%
\pgfsetstrokecolor{currentstroke}%
\pgfsetdash{}{0pt}%
\pgfpathmoveto{\pgfqpoint{3.570080in}{4.035663in}}%
\pgfpathlineto{\pgfqpoint{3.795907in}{4.018820in}}%
\pgfpathlineto{\pgfqpoint{3.788186in}{4.037647in}}%
\pgfpathlineto{\pgfqpoint{3.876906in}{4.003652in}}%
\pgfpathlineto{\pgfqpoint{3.784124in}{3.983196in}}%
\pgfpathlineto{\pgfqpoint{3.794553in}{4.000669in}}%
\pgfpathlineto{\pgfqpoint{3.568727in}{4.017513in}}%
\pgfpathlineto{\pgfqpoint{3.570080in}{4.035663in}}%
\pgfusepath{fill}%
\end{pgfscope}%
\begin{pgfscope}%
\pgfpathrectangle{\pgfqpoint{1.065196in}{0.528000in}}{\pgfqpoint{3.702804in}{3.696000in}} %
\pgfusepath{clip}%
\pgfsetbuttcap%
\pgfsetroundjoin%
\definecolor{currentfill}{rgb}{0.369214,0.788888,0.382914}%
\pgfsetfillcolor{currentfill}%
\pgfsetlinewidth{0.000000pt}%
\definecolor{currentstroke}{rgb}{0.000000,0.000000,0.000000}%
\pgfsetstrokecolor{currentstroke}%
\pgfsetdash{}{0pt}%
\pgfpathmoveto{\pgfqpoint{1.832968in}{1.155953in}}%
\pgfpathlineto{\pgfqpoint{2.179360in}{1.171850in}}%
\pgfpathlineto{\pgfqpoint{2.169434in}{1.189614in}}%
\pgfpathlineto{\pgfqpoint{2.261594in}{1.166514in}}%
\pgfpathlineto{\pgfqpoint{2.171938in}{1.135070in}}%
\pgfpathlineto{\pgfqpoint{2.180194in}{1.153668in}}%
\pgfpathlineto{\pgfqpoint{1.833803in}{1.137771in}}%
\pgfpathlineto{\pgfqpoint{1.832968in}{1.155953in}}%
\pgfusepath{fill}%
\end{pgfscope}%
\begin{pgfscope}%
\pgfpathrectangle{\pgfqpoint{1.065196in}{0.528000in}}{\pgfqpoint{3.702804in}{3.696000in}} %
\pgfusepath{clip}%
\pgfsetbuttcap%
\pgfsetroundjoin%
\definecolor{currentfill}{rgb}{0.132444,0.552216,0.553018}%
\pgfsetfillcolor{currentfill}%
\pgfsetlinewidth{0.000000pt}%
\definecolor{currentstroke}{rgb}{0.000000,0.000000,0.000000}%
\pgfsetstrokecolor{currentstroke}%
\pgfsetdash{}{0pt}%
\pgfpathmoveto{\pgfqpoint{4.382373in}{3.028824in}}%
\pgfpathlineto{\pgfqpoint{4.445958in}{2.999479in}}%
\pgfpathlineto{\pgfqpoint{4.445322in}{3.019818in}}%
\pgfpathlineto{\pgfqpoint{4.516511in}{2.956896in}}%
\pgfpathlineto{\pgfqpoint{4.422442in}{2.970241in}}%
\pgfpathlineto{\pgfqpoint{4.438332in}{2.982953in}}%
\pgfpathlineto{\pgfqpoint{4.374746in}{3.012299in}}%
\pgfpathlineto{\pgfqpoint{4.382373in}{3.028824in}}%
\pgfusepath{fill}%
\end{pgfscope}%
\begin{pgfscope}%
\pgfpathrectangle{\pgfqpoint{1.065196in}{0.528000in}}{\pgfqpoint{3.702804in}{3.696000in}} %
\pgfusepath{clip}%
\pgfsetbuttcap%
\pgfsetroundjoin%
\definecolor{currentfill}{rgb}{0.252194,0.269783,0.531579}%
\pgfsetfillcolor{currentfill}%
\pgfsetlinewidth{0.000000pt}%
\definecolor{currentstroke}{rgb}{0.000000,0.000000,0.000000}%
\pgfsetstrokecolor{currentstroke}%
\pgfsetdash{}{0pt}%
\pgfpathmoveto{\pgfqpoint{4.371017in}{3.015470in}}%
\pgfpathlineto{\pgfqpoint{4.061347in}{3.474185in}}%
\pgfpathlineto{\pgfqpoint{4.051354in}{3.456459in}}%
\pgfpathlineto{\pgfqpoint{4.023063in}{3.547160in}}%
\pgfpathlineto{\pgfqpoint{4.096609in}{3.487010in}}%
\pgfpathlineto{\pgfqpoint{4.076432in}{3.484369in}}%
\pgfpathlineto{\pgfqpoint{4.386102in}{3.025653in}}%
\pgfpathlineto{\pgfqpoint{4.371017in}{3.015470in}}%
\pgfusepath{fill}%
\end{pgfscope}%
\begin{pgfscope}%
\pgfpathrectangle{\pgfqpoint{1.065196in}{0.528000in}}{\pgfqpoint{3.702804in}{3.696000in}} %
\pgfusepath{clip}%
\pgfsetbuttcap%
\pgfsetroundjoin%
\definecolor{currentfill}{rgb}{0.277018,0.050344,0.375715}%
\pgfsetfillcolor{currentfill}%
\pgfsetlinewidth{0.000000pt}%
\definecolor{currentstroke}{rgb}{0.000000,0.000000,0.000000}%
\pgfsetstrokecolor{currentstroke}%
\pgfsetdash{}{0pt}%
\pgfpathmoveto{\pgfqpoint{4.379095in}{3.011477in}}%
\pgfpathlineto{\pgfqpoint{4.225677in}{3.002435in}}%
\pgfpathlineto{\pgfqpoint{4.235832in}{2.984801in}}%
\pgfpathlineto{\pgfqpoint{4.143380in}{3.006700in}}%
\pgfpathlineto{\pgfqpoint{4.232620in}{3.039308in}}%
\pgfpathlineto{\pgfqpoint{4.224606in}{3.020604in}}%
\pgfpathlineto{\pgfqpoint{4.378024in}{3.029646in}}%
\pgfpathlineto{\pgfqpoint{4.379095in}{3.011477in}}%
\pgfusepath{fill}%
\end{pgfscope}%
\begin{pgfscope}%
\pgfpathrectangle{\pgfqpoint{1.065196in}{0.528000in}}{\pgfqpoint{3.702804in}{3.696000in}} %
\pgfusepath{clip}%
\pgfsetbuttcap%
\pgfsetroundjoin%
\definecolor{currentfill}{rgb}{0.180629,0.429975,0.557282}%
\pgfsetfillcolor{currentfill}%
\pgfsetlinewidth{0.000000pt}%
\definecolor{currentstroke}{rgb}{0.000000,0.000000,0.000000}%
\pgfsetstrokecolor{currentstroke}%
\pgfsetdash{}{0pt}%
\pgfpathmoveto{\pgfqpoint{1.477836in}{3.225981in}}%
\pgfpathlineto{\pgfqpoint{1.658988in}{3.216905in}}%
\pgfpathlineto{\pgfqpoint{1.650810in}{3.235539in}}%
\pgfpathlineto{\pgfqpoint{1.740334in}{3.203718in}}%
\pgfpathlineto{\pgfqpoint{1.648078in}{3.181005in}}%
\pgfpathlineto{\pgfqpoint{1.658078in}{3.198727in}}%
\pgfpathlineto{\pgfqpoint{1.476925in}{3.207803in}}%
\pgfpathlineto{\pgfqpoint{1.477836in}{3.225981in}}%
\pgfusepath{fill}%
\end{pgfscope}%
\begin{pgfscope}%
\pgfpathrectangle{\pgfqpoint{1.065196in}{0.528000in}}{\pgfqpoint{3.702804in}{3.696000in}} %
\pgfusepath{clip}%
\pgfsetbuttcap%
\pgfsetroundjoin%
\definecolor{currentfill}{rgb}{0.282623,0.140926,0.457517}%
\pgfsetfillcolor{currentfill}%
\pgfsetlinewidth{0.000000pt}%
\definecolor{currentstroke}{rgb}{0.000000,0.000000,0.000000}%
\pgfsetstrokecolor{currentstroke}%
\pgfsetdash{}{0pt}%
\pgfpathmoveto{\pgfqpoint{3.786053in}{3.784859in}}%
\pgfpathlineto{\pgfqpoint{3.917109in}{3.676224in}}%
\pgfpathlineto{\pgfqpoint{3.921718in}{3.696044in}}%
\pgfpathlineto{\pgfqpoint{3.974358in}{3.616949in}}%
\pgfpathlineto{\pgfqpoint{3.886873in}{3.654006in}}%
\pgfpathlineto{\pgfqpoint{3.905494in}{3.662211in}}%
\pgfpathlineto{\pgfqpoint{3.774437in}{3.770847in}}%
\pgfpathlineto{\pgfqpoint{3.786053in}{3.784859in}}%
\pgfusepath{fill}%
\end{pgfscope}%
\begin{pgfscope}%
\pgfpathrectangle{\pgfqpoint{1.065196in}{0.528000in}}{\pgfqpoint{3.702804in}{3.696000in}} %
\pgfusepath{clip}%
\pgfsetbuttcap%
\pgfsetroundjoin%
\definecolor{currentfill}{rgb}{0.283229,0.120777,0.440584}%
\pgfsetfillcolor{currentfill}%
\pgfsetlinewidth{0.000000pt}%
\definecolor{currentstroke}{rgb}{0.000000,0.000000,0.000000}%
\pgfsetstrokecolor{currentstroke}%
\pgfsetdash{}{0pt}%
\pgfpathmoveto{\pgfqpoint{3.629543in}{1.105606in}}%
\pgfpathlineto{\pgfqpoint{3.670343in}{1.306303in}}%
\pgfpathlineto{\pgfqpoint{3.650694in}{1.301011in}}%
\pgfpathlineto{\pgfqpoint{3.695578in}{1.384752in}}%
\pgfpathlineto{\pgfqpoint{3.704202in}{1.290133in}}%
\pgfpathlineto{\pgfqpoint{3.688179in}{1.302677in}}%
\pgfpathlineto{\pgfqpoint{3.647379in}{1.101981in}}%
\pgfpathlineto{\pgfqpoint{3.629543in}{1.105606in}}%
\pgfusepath{fill}%
\end{pgfscope}%
\begin{pgfscope}%
\pgfpathrectangle{\pgfqpoint{1.065196in}{0.528000in}}{\pgfqpoint{3.702804in}{3.696000in}} %
\pgfusepath{clip}%
\pgfsetbuttcap%
\pgfsetroundjoin%
\definecolor{currentfill}{rgb}{0.273809,0.031497,0.358853}%
\pgfsetfillcolor{currentfill}%
\pgfsetlinewidth{0.000000pt}%
\definecolor{currentstroke}{rgb}{0.000000,0.000000,0.000000}%
\pgfsetstrokecolor{currentstroke}%
\pgfsetdash{}{0pt}%
\pgfpathmoveto{\pgfqpoint{1.318021in}{0.783140in}}%
\pgfpathlineto{\pgfqpoint{1.765157in}{1.073689in}}%
\pgfpathlineto{\pgfqpoint{1.747609in}{1.083992in}}%
\pgfpathlineto{\pgfqpoint{1.838793in}{1.110685in}}%
\pgfpathlineto{\pgfqpoint{1.777360in}{1.038207in}}%
\pgfpathlineto{\pgfqpoint{1.775074in}{1.058428in}}%
\pgfpathlineto{\pgfqpoint{1.327939in}{0.767878in}}%
\pgfpathlineto{\pgfqpoint{1.318021in}{0.783140in}}%
\pgfusepath{fill}%
\end{pgfscope}%
\begin{pgfscope}%
\pgfpathrectangle{\pgfqpoint{1.065196in}{0.528000in}}{\pgfqpoint{3.702804in}{3.696000in}} %
\pgfusepath{clip}%
\pgfsetbuttcap%
\pgfsetroundjoin%
\definecolor{currentfill}{rgb}{0.124395,0.578002,0.548287}%
\pgfsetfillcolor{currentfill}%
\pgfsetlinewidth{0.000000pt}%
\definecolor{currentstroke}{rgb}{0.000000,0.000000,0.000000}%
\pgfsetstrokecolor{currentstroke}%
\pgfsetdash{}{0pt}%
\pgfpathmoveto{\pgfqpoint{1.346621in}{1.519894in}}%
\pgfpathlineto{\pgfqpoint{1.770892in}{1.766190in}}%
\pgfpathlineto{\pgfqpoint{1.753884in}{1.777362in}}%
\pgfpathlineto{\pgfqpoint{1.846294in}{1.799440in}}%
\pgfpathlineto{\pgfqpoint{1.781297in}{1.730140in}}%
\pgfpathlineto{\pgfqpoint{1.780029in}{1.750450in}}%
\pgfpathlineto{\pgfqpoint{1.355759in}{1.504153in}}%
\pgfpathlineto{\pgfqpoint{1.346621in}{1.519894in}}%
\pgfusepath{fill}%
\end{pgfscope}%
\begin{pgfscope}%
\pgfpathrectangle{\pgfqpoint{1.065196in}{0.528000in}}{\pgfqpoint{3.702804in}{3.696000in}} %
\pgfusepath{clip}%
\pgfsetbuttcap%
\pgfsetroundjoin%
\definecolor{currentfill}{rgb}{0.248629,0.278775,0.534556}%
\pgfsetfillcolor{currentfill}%
\pgfsetlinewidth{0.000000pt}%
\definecolor{currentstroke}{rgb}{0.000000,0.000000,0.000000}%
\pgfsetstrokecolor{currentstroke}%
\pgfsetdash{}{0pt}%
\pgfpathmoveto{\pgfqpoint{1.350529in}{1.521100in}}%
\pgfpathlineto{\pgfqpoint{2.003436in}{1.568595in}}%
\pgfpathlineto{\pgfqpoint{1.993039in}{1.586088in}}%
\pgfpathlineto{\pgfqpoint{2.085783in}{1.565461in}}%
\pgfpathlineto{\pgfqpoint{1.997000in}{1.531629in}}%
\pgfpathlineto{\pgfqpoint{2.004756in}{1.550442in}}%
\pgfpathlineto{\pgfqpoint{1.351850in}{1.502947in}}%
\pgfpathlineto{\pgfqpoint{1.350529in}{1.521100in}}%
\pgfusepath{fill}%
\end{pgfscope}%
\begin{pgfscope}%
\pgfpathrectangle{\pgfqpoint{1.065196in}{0.528000in}}{\pgfqpoint{3.702804in}{3.696000in}} %
\pgfusepath{clip}%
\pgfsetbuttcap%
\pgfsetroundjoin%
\definecolor{currentfill}{rgb}{0.280894,0.078907,0.402329}%
\pgfsetfillcolor{currentfill}%
\pgfsetlinewidth{0.000000pt}%
\definecolor{currentstroke}{rgb}{0.000000,0.000000,0.000000}%
\pgfsetstrokecolor{currentstroke}%
\pgfsetdash{}{0pt}%
\pgfpathmoveto{\pgfqpoint{4.143128in}{2.486514in}}%
\pgfpathlineto{\pgfqpoint{3.977008in}{2.243812in}}%
\pgfpathlineto{\pgfqpoint{3.997168in}{2.241041in}}%
\pgfpathlineto{\pgfqpoint{3.923237in}{2.181364in}}%
\pgfpathlineto{\pgfqpoint{3.952109in}{2.271882in}}%
\pgfpathlineto{\pgfqpoint{3.961989in}{2.254092in}}%
\pgfpathlineto{\pgfqpoint{4.128109in}{2.496794in}}%
\pgfpathlineto{\pgfqpoint{4.143128in}{2.486514in}}%
\pgfusepath{fill}%
\end{pgfscope}%
\begin{pgfscope}%
\pgfpathrectangle{\pgfqpoint{1.065196in}{0.528000in}}{\pgfqpoint{3.702804in}{3.696000in}} %
\pgfusepath{clip}%
\pgfsetbuttcap%
\pgfsetroundjoin%
\definecolor{currentfill}{rgb}{0.257322,0.256130,0.526563}%
\pgfsetfillcolor{currentfill}%
\pgfsetlinewidth{0.000000pt}%
\definecolor{currentstroke}{rgb}{0.000000,0.000000,0.000000}%
\pgfsetstrokecolor{currentstroke}%
\pgfsetdash{}{0pt}%
\pgfpathmoveto{\pgfqpoint{3.114858in}{3.501854in}}%
\pgfpathlineto{\pgfqpoint{3.074448in}{3.437684in}}%
\pgfpathlineto{\pgfqpoint{3.094699in}{3.435686in}}%
\pgfpathlineto{\pgfqpoint{3.023103in}{3.373228in}}%
\pgfpathlineto{\pgfqpoint{3.048495in}{3.464782in}}%
\pgfpathlineto{\pgfqpoint{3.059047in}{3.447383in}}%
\pgfpathlineto{\pgfqpoint{3.099457in}{3.511553in}}%
\pgfpathlineto{\pgfqpoint{3.114858in}{3.501854in}}%
\pgfusepath{fill}%
\end{pgfscope}%
\begin{pgfscope}%
\pgfpathrectangle{\pgfqpoint{1.065196in}{0.528000in}}{\pgfqpoint{3.702804in}{3.696000in}} %
\pgfusepath{clip}%
\pgfsetbuttcap%
\pgfsetroundjoin%
\definecolor{currentfill}{rgb}{0.241237,0.296485,0.539709}%
\pgfsetfillcolor{currentfill}%
\pgfsetlinewidth{0.000000pt}%
\definecolor{currentstroke}{rgb}{0.000000,0.000000,0.000000}%
\pgfsetstrokecolor{currentstroke}%
\pgfsetdash{}{0pt}%
\pgfpathmoveto{\pgfqpoint{1.673597in}{1.631390in}}%
\pgfpathlineto{\pgfqpoint{2.046473in}{1.570546in}}%
\pgfpathlineto{\pgfqpoint{2.040422in}{1.589975in}}%
\pgfpathlineto{\pgfqpoint{2.125841in}{1.548375in}}%
\pgfpathlineto{\pgfqpoint{2.031629in}{1.536086in}}%
\pgfpathlineto{\pgfqpoint{2.043541in}{1.552583in}}%
\pgfpathlineto{\pgfqpoint{1.670666in}{1.613427in}}%
\pgfpathlineto{\pgfqpoint{1.673597in}{1.631390in}}%
\pgfusepath{fill}%
\end{pgfscope}%
\begin{pgfscope}%
\pgfpathrectangle{\pgfqpoint{1.065196in}{0.528000in}}{\pgfqpoint{3.702804in}{3.696000in}} %
\pgfusepath{clip}%
\pgfsetbuttcap%
\pgfsetroundjoin%
\definecolor{currentfill}{rgb}{0.131172,0.555899,0.552459}%
\pgfsetfillcolor{currentfill}%
\pgfsetlinewidth{0.000000pt}%
\definecolor{currentstroke}{rgb}{0.000000,0.000000,0.000000}%
\pgfsetstrokecolor{currentstroke}%
\pgfsetdash{}{0pt}%
\pgfpathmoveto{\pgfqpoint{1.673373in}{1.631424in}}%
\pgfpathlineto{\pgfqpoint{2.005886in}{1.585647in}}%
\pgfpathlineto{\pgfqpoint{1.999353in}{1.604918in}}%
\pgfpathlineto{\pgfqpoint{2.085783in}{1.565461in}}%
\pgfpathlineto{\pgfqpoint{1.991907in}{1.550826in}}%
\pgfpathlineto{\pgfqpoint{2.003404in}{1.567616in}}%
\pgfpathlineto{\pgfqpoint{1.670891in}{1.613394in}}%
\pgfpathlineto{\pgfqpoint{1.673373in}{1.631424in}}%
\pgfusepath{fill}%
\end{pgfscope}%
\begin{pgfscope}%
\pgfpathrectangle{\pgfqpoint{1.065196in}{0.528000in}}{\pgfqpoint{3.702804in}{3.696000in}} %
\pgfusepath{clip}%
\pgfsetbuttcap%
\pgfsetroundjoin%
\definecolor{currentfill}{rgb}{0.231674,0.318106,0.544834}%
\pgfsetfillcolor{currentfill}%
\pgfsetlinewidth{0.000000pt}%
\definecolor{currentstroke}{rgb}{0.000000,0.000000,0.000000}%
\pgfsetstrokecolor{currentstroke}%
\pgfsetdash{}{0pt}%
\pgfpathmoveto{\pgfqpoint{3.221789in}{3.941749in}}%
\pgfpathlineto{\pgfqpoint{3.462299in}{3.810310in}}%
\pgfpathlineto{\pgfqpoint{3.463042in}{3.830645in}}%
\pgfpathlineto{\pgfqpoint{3.529806in}{3.763047in}}%
\pgfpathlineto{\pgfqpoint{3.436857in}{3.782731in}}%
\pgfpathlineto{\pgfqpoint{3.453571in}{3.794339in}}%
\pgfpathlineto{\pgfqpoint{3.213061in}{3.925778in}}%
\pgfpathlineto{\pgfqpoint{3.221789in}{3.941749in}}%
\pgfusepath{fill}%
\end{pgfscope}%
\begin{pgfscope}%
\pgfpathrectangle{\pgfqpoint{1.065196in}{0.528000in}}{\pgfqpoint{3.702804in}{3.696000in}} %
\pgfusepath{clip}%
\pgfsetbuttcap%
\pgfsetroundjoin%
\definecolor{currentfill}{rgb}{0.166383,0.690856,0.496502}%
\pgfsetfillcolor{currentfill}%
\pgfsetlinewidth{0.000000pt}%
\definecolor{currentstroke}{rgb}{0.000000,0.000000,0.000000}%
\pgfsetstrokecolor{currentstroke}%
\pgfsetdash{}{0pt}%
\pgfpathmoveto{\pgfqpoint{3.125583in}{0.751109in}}%
\pgfpathlineto{\pgfqpoint{3.173625in}{1.223227in}}%
\pgfpathlineto{\pgfqpoint{3.154597in}{1.216016in}}%
\pgfpathlineto{\pgfqpoint{3.190970in}{1.303788in}}%
\pgfpathlineto{\pgfqpoint{3.208918in}{1.210488in}}%
\pgfpathlineto{\pgfqpoint{3.191732in}{1.221384in}}%
\pgfpathlineto{\pgfqpoint{3.143691in}{0.749266in}}%
\pgfpathlineto{\pgfqpoint{3.125583in}{0.751109in}}%
\pgfusepath{fill}%
\end{pgfscope}%
\begin{pgfscope}%
\pgfpathrectangle{\pgfqpoint{1.065196in}{0.528000in}}{\pgfqpoint{3.702804in}{3.696000in}} %
\pgfusepath{clip}%
\pgfsetbuttcap%
\pgfsetroundjoin%
\definecolor{currentfill}{rgb}{0.263663,0.237631,0.518762}%
\pgfsetfillcolor{currentfill}%
\pgfsetlinewidth{0.000000pt}%
\definecolor{currentstroke}{rgb}{0.000000,0.000000,0.000000}%
\pgfsetstrokecolor{currentstroke}%
\pgfsetdash{}{0pt}%
\pgfpathmoveto{\pgfqpoint{3.125912in}{0.747602in}}%
\pgfpathlineto{\pgfqpoint{3.033333in}{1.060031in}}%
\pgfpathlineto{\pgfqpoint{3.018468in}{1.046135in}}%
\pgfpathlineto{\pgfqpoint{3.018789in}{1.141145in}}%
\pgfpathlineto{\pgfqpoint{3.070820in}{1.061647in}}%
\pgfpathlineto{\pgfqpoint{3.050784in}{1.065202in}}%
\pgfpathlineto{\pgfqpoint{3.143362in}{0.752773in}}%
\pgfpathlineto{\pgfqpoint{3.125912in}{0.747602in}}%
\pgfusepath{fill}%
\end{pgfscope}%
\begin{pgfscope}%
\pgfpathrectangle{\pgfqpoint{1.065196in}{0.528000in}}{\pgfqpoint{3.702804in}{3.696000in}} %
\pgfusepath{clip}%
\pgfsetbuttcap%
\pgfsetroundjoin%
\definecolor{currentfill}{rgb}{0.255645,0.260703,0.528312}%
\pgfsetfillcolor{currentfill}%
\pgfsetlinewidth{0.000000pt}%
\definecolor{currentstroke}{rgb}{0.000000,0.000000,0.000000}%
\pgfsetstrokecolor{currentstroke}%
\pgfsetdash{}{0pt}%
\pgfpathmoveto{\pgfqpoint{3.927682in}{1.467360in}}%
\pgfpathlineto{\pgfqpoint{3.924038in}{1.562095in}}%
\pgfpathlineto{\pgfqpoint{3.906200in}{1.552301in}}%
\pgfpathlineto{\pgfqpoint{3.929983in}{1.644287in}}%
\pgfpathlineto{\pgfqpoint{3.960762in}{1.554400in}}%
\pgfpathlineto{\pgfqpoint{3.942225in}{1.562794in}}%
\pgfpathlineto{\pgfqpoint{3.945869in}{1.468059in}}%
\pgfpathlineto{\pgfqpoint{3.927682in}{1.467360in}}%
\pgfusepath{fill}%
\end{pgfscope}%
\begin{pgfscope}%
\pgfpathrectangle{\pgfqpoint{1.065196in}{0.528000in}}{\pgfqpoint{3.702804in}{3.696000in}} %
\pgfusepath{clip}%
\pgfsetbuttcap%
\pgfsetroundjoin%
\definecolor{currentfill}{rgb}{0.282623,0.140926,0.457517}%
\pgfsetfillcolor{currentfill}%
\pgfsetlinewidth{0.000000pt}%
\definecolor{currentstroke}{rgb}{0.000000,0.000000,0.000000}%
\pgfsetstrokecolor{currentstroke}%
\pgfsetdash{}{0pt}%
\pgfpathmoveto{\pgfqpoint{3.965275in}{1.992249in}}%
\pgfpathlineto{\pgfqpoint{4.103236in}{1.835520in}}%
\pgfpathlineto{\pgfqpoint{4.110885in}{1.854377in}}%
\pgfpathlineto{\pgfqpoint{4.150521in}{1.768029in}}%
\pgfpathlineto{\pgfqpoint{4.069900in}{1.818299in}}%
\pgfpathlineto{\pgfqpoint{4.089574in}{1.823494in}}%
\pgfpathlineto{\pgfqpoint{3.951614in}{1.980224in}}%
\pgfpathlineto{\pgfqpoint{3.965275in}{1.992249in}}%
\pgfusepath{fill}%
\end{pgfscope}%
\begin{pgfscope}%
\pgfpathrectangle{\pgfqpoint{1.065196in}{0.528000in}}{\pgfqpoint{3.702804in}{3.696000in}} %
\pgfusepath{clip}%
\pgfsetbuttcap%
\pgfsetroundjoin%
\definecolor{currentfill}{rgb}{0.270595,0.214069,0.507052}%
\pgfsetfillcolor{currentfill}%
\pgfsetlinewidth{0.000000pt}%
\definecolor{currentstroke}{rgb}{0.000000,0.000000,0.000000}%
\pgfsetstrokecolor{currentstroke}%
\pgfsetdash{}{0pt}%
\pgfpathmoveto{\pgfqpoint{3.951420in}{1.980450in}}%
\pgfpathlineto{\pgfqpoint{3.812896in}{2.148615in}}%
\pgfpathlineto{\pgfqpoint{3.804634in}{2.130019in}}%
\pgfpathlineto{\pgfqpoint{3.767845in}{2.217618in}}%
\pgfpathlineto{\pgfqpoint{3.846778in}{2.164735in}}%
\pgfpathlineto{\pgfqpoint{3.826944in}{2.160187in}}%
\pgfpathlineto{\pgfqpoint{3.965469in}{1.992023in}}%
\pgfpathlineto{\pgfqpoint{3.951420in}{1.980450in}}%
\pgfusepath{fill}%
\end{pgfscope}%
\begin{pgfscope}%
\pgfpathrectangle{\pgfqpoint{1.065196in}{0.528000in}}{\pgfqpoint{3.702804in}{3.696000in}} %
\pgfusepath{clip}%
\pgfsetbuttcap%
\pgfsetroundjoin%
\definecolor{currentfill}{rgb}{0.227802,0.326594,0.546532}%
\pgfsetfillcolor{currentfill}%
\pgfsetlinewidth{0.000000pt}%
\definecolor{currentstroke}{rgb}{0.000000,0.000000,0.000000}%
\pgfsetstrokecolor{currentstroke}%
\pgfsetdash{}{0pt}%
\pgfpathmoveto{\pgfqpoint{4.138785in}{3.185983in}}%
\pgfpathlineto{\pgfqpoint{4.041321in}{3.466801in}}%
\pgfpathlineto{\pgfqpoint{4.027110in}{3.452236in}}%
\pgfpathlineto{\pgfqpoint{4.023063in}{3.547160in}}%
\pgfpathlineto{\pgfqpoint{4.078694in}{3.470139in}}%
\pgfpathlineto{\pgfqpoint{4.058515in}{3.472768in}}%
\pgfpathlineto{\pgfqpoint{4.155980in}{3.191951in}}%
\pgfpathlineto{\pgfqpoint{4.138785in}{3.185983in}}%
\pgfusepath{fill}%
\end{pgfscope}%
\begin{pgfscope}%
\pgfpathrectangle{\pgfqpoint{1.065196in}{0.528000in}}{\pgfqpoint{3.702804in}{3.696000in}} %
\pgfusepath{clip}%
\pgfsetbuttcap%
\pgfsetroundjoin%
\definecolor{currentfill}{rgb}{0.267004,0.004874,0.329415}%
\pgfsetfillcolor{currentfill}%
\pgfsetlinewidth{0.000000pt}%
\definecolor{currentstroke}{rgb}{0.000000,0.000000,0.000000}%
\pgfsetstrokecolor{currentstroke}%
\pgfsetdash{}{0pt}%
\pgfpathmoveto{\pgfqpoint{3.111481in}{1.150252in}}%
\pgfpathlineto{\pgfqpoint{3.416941in}{1.533951in}}%
\pgfpathlineto{\pgfqpoint{3.397034in}{1.538167in}}%
\pgfpathlineto{\pgfqpoint{3.475072in}{1.592361in}}%
\pgfpathlineto{\pgfqpoint{3.439752in}{1.504160in}}%
\pgfpathlineto{\pgfqpoint{3.431181in}{1.522615in}}%
\pgfpathlineto{\pgfqpoint{3.125721in}{1.138916in}}%
\pgfpathlineto{\pgfqpoint{3.111481in}{1.150252in}}%
\pgfusepath{fill}%
\end{pgfscope}%
\begin{pgfscope}%
\pgfpathrectangle{\pgfqpoint{1.065196in}{0.528000in}}{\pgfqpoint{3.702804in}{3.696000in}} %
\pgfusepath{clip}%
\pgfsetbuttcap%
\pgfsetroundjoin%
\definecolor{currentfill}{rgb}{0.595839,0.848717,0.243329}%
\pgfsetfillcolor{currentfill}%
\pgfsetlinewidth{0.000000pt}%
\definecolor{currentstroke}{rgb}{0.000000,0.000000,0.000000}%
\pgfsetstrokecolor{currentstroke}%
\pgfsetdash{}{0pt}%
\pgfpathmoveto{\pgfqpoint{3.115314in}{1.153070in}}%
\pgfpathlineto{\pgfqpoint{3.489895in}{1.298177in}}%
\pgfpathlineto{\pgfqpoint{3.474834in}{1.311861in}}%
\pgfpathlineto{\pgfqpoint{3.569555in}{1.319277in}}%
\pgfpathlineto{\pgfqpoint{3.494558in}{1.260946in}}%
\pgfpathlineto{\pgfqpoint{3.496469in}{1.281205in}}%
\pgfpathlineto{\pgfqpoint{3.121888in}{1.136098in}}%
\pgfpathlineto{\pgfqpoint{3.115314in}{1.153070in}}%
\pgfusepath{fill}%
\end{pgfscope}%
\begin{pgfscope}%
\pgfpathrectangle{\pgfqpoint{1.065196in}{0.528000in}}{\pgfqpoint{3.702804in}{3.696000in}} %
\pgfusepath{clip}%
\pgfsetbuttcap%
\pgfsetroundjoin%
\definecolor{currentfill}{rgb}{0.252194,0.269783,0.531579}%
\pgfsetfillcolor{currentfill}%
\pgfsetlinewidth{0.000000pt}%
\definecolor{currentstroke}{rgb}{0.000000,0.000000,0.000000}%
\pgfsetstrokecolor{currentstroke}%
\pgfsetdash{}{0pt}%
\pgfpathmoveto{\pgfqpoint{1.457499in}{1.103080in}}%
\pgfpathlineto{\pgfqpoint{2.035897in}{1.072424in}}%
\pgfpathlineto{\pgfqpoint{2.027773in}{1.091081in}}%
\pgfpathlineto{\pgfqpoint{2.117204in}{1.059001in}}%
\pgfpathlineto{\pgfqpoint{2.024883in}{1.036555in}}%
\pgfpathlineto{\pgfqpoint{2.034933in}{1.054249in}}%
\pgfpathlineto{\pgfqpoint{1.456536in}{1.084905in}}%
\pgfpathlineto{\pgfqpoint{1.457499in}{1.103080in}}%
\pgfusepath{fill}%
\end{pgfscope}%
\begin{pgfscope}%
\pgfpathrectangle{\pgfqpoint{1.065196in}{0.528000in}}{\pgfqpoint{3.702804in}{3.696000in}} %
\pgfusepath{clip}%
\pgfsetbuttcap%
\pgfsetroundjoin%
\definecolor{currentfill}{rgb}{0.165117,0.467423,0.558141}%
\pgfsetfillcolor{currentfill}%
\pgfsetlinewidth{0.000000pt}%
\definecolor{currentstroke}{rgb}{0.000000,0.000000,0.000000}%
\pgfsetstrokecolor{currentstroke}%
\pgfsetdash{}{0pt}%
\pgfpathmoveto{\pgfqpoint{1.404265in}{1.056634in}}%
\pgfpathlineto{\pgfqpoint{1.756433in}{1.107893in}}%
\pgfpathlineto{\pgfqpoint{1.744805in}{1.124594in}}%
\pgfpathlineto{\pgfqpoint{1.838793in}{1.110685in}}%
\pgfpathlineto{\pgfqpoint{1.752670in}{1.070561in}}%
\pgfpathlineto{\pgfqpoint{1.759054in}{1.089883in}}%
\pgfpathlineto{\pgfqpoint{1.406887in}{1.038623in}}%
\pgfpathlineto{\pgfqpoint{1.404265in}{1.056634in}}%
\pgfusepath{fill}%
\end{pgfscope}%
\begin{pgfscope}%
\pgfpathrectangle{\pgfqpoint{1.065196in}{0.528000in}}{\pgfqpoint{3.702804in}{3.696000in}} %
\pgfusepath{clip}%
\pgfsetbuttcap%
\pgfsetroundjoin%
\definecolor{currentfill}{rgb}{0.180629,0.429975,0.557282}%
\pgfsetfillcolor{currentfill}%
\pgfsetlinewidth{0.000000pt}%
\definecolor{currentstroke}{rgb}{0.000000,0.000000,0.000000}%
\pgfsetstrokecolor{currentstroke}%
\pgfsetdash{}{0pt}%
\pgfpathmoveto{\pgfqpoint{2.020275in}{3.078600in}}%
\pgfpathlineto{\pgfqpoint{2.185583in}{2.730251in}}%
\pgfpathlineto{\pgfqpoint{2.198124in}{2.746276in}}%
\pgfpathlineto{\pgfqpoint{2.212475in}{2.652355in}}%
\pgfpathlineto{\pgfqpoint{2.148795in}{2.722867in}}%
\pgfpathlineto{\pgfqpoint{2.169140in}{2.722448in}}%
\pgfpathlineto{\pgfqpoint{2.003831in}{3.070797in}}%
\pgfpathlineto{\pgfqpoint{2.020275in}{3.078600in}}%
\pgfusepath{fill}%
\end{pgfscope}%
\begin{pgfscope}%
\pgfpathrectangle{\pgfqpoint{1.065196in}{0.528000in}}{\pgfqpoint{3.702804in}{3.696000in}} %
\pgfusepath{clip}%
\pgfsetbuttcap%
\pgfsetroundjoin%
\definecolor{currentfill}{rgb}{0.271305,0.019942,0.347269}%
\pgfsetfillcolor{currentfill}%
\pgfsetlinewidth{0.000000pt}%
\definecolor{currentstroke}{rgb}{0.000000,0.000000,0.000000}%
\pgfsetstrokecolor{currentstroke}%
\pgfsetdash{}{0pt}%
\pgfpathmoveto{\pgfqpoint{2.007984in}{3.066558in}}%
\pgfpathlineto{\pgfqpoint{1.862586in}{3.139237in}}%
\pgfpathlineto{\pgfqpoint{1.862589in}{3.118888in}}%
\pgfpathlineto{\pgfqpoint{1.793395in}{3.183997in}}%
\pgfpathlineto{\pgfqpoint{1.887002in}{3.167728in}}%
\pgfpathlineto{\pgfqpoint{1.870724in}{3.155517in}}%
\pgfpathlineto{\pgfqpoint{2.016122in}{3.082838in}}%
\pgfpathlineto{\pgfqpoint{2.007984in}{3.066558in}}%
\pgfusepath{fill}%
\end{pgfscope}%
\begin{pgfscope}%
\pgfpathrectangle{\pgfqpoint{1.065196in}{0.528000in}}{\pgfqpoint{3.702804in}{3.696000in}} %
\pgfusepath{clip}%
\pgfsetbuttcap%
\pgfsetroundjoin%
\definecolor{currentfill}{rgb}{0.993248,0.906157,0.143936}%
\pgfsetfillcolor{currentfill}%
\pgfsetlinewidth{0.000000pt}%
\definecolor{currentstroke}{rgb}{0.000000,0.000000,0.000000}%
\pgfsetstrokecolor{currentstroke}%
\pgfsetdash{}{0pt}%
\pgfpathmoveto{\pgfqpoint{3.129847in}{0.738887in}}%
\pgfpathlineto{\pgfqpoint{3.803211in}{0.768064in}}%
\pgfpathlineto{\pgfqpoint{3.793332in}{0.785854in}}%
\pgfpathlineto{\pgfqpoint{3.885432in}{0.762518in}}%
\pgfpathlineto{\pgfqpoint{3.795695in}{0.731303in}}%
\pgfpathlineto{\pgfqpoint{3.803999in}{0.749881in}}%
\pgfpathlineto{\pgfqpoint{3.130635in}{0.720703in}}%
\pgfpathlineto{\pgfqpoint{3.129847in}{0.738887in}}%
\pgfusepath{fill}%
\end{pgfscope}%
\begin{pgfscope}%
\pgfpathrectangle{\pgfqpoint{1.065196in}{0.528000in}}{\pgfqpoint{3.702804in}{3.696000in}} %
\pgfusepath{clip}%
\pgfsetbuttcap%
\pgfsetroundjoin%
\definecolor{currentfill}{rgb}{0.175841,0.441290,0.557685}%
\pgfsetfillcolor{currentfill}%
\pgfsetlinewidth{0.000000pt}%
\definecolor{currentstroke}{rgb}{0.000000,0.000000,0.000000}%
\pgfsetstrokecolor{currentstroke}%
\pgfsetdash{}{0pt}%
\pgfpathmoveto{\pgfqpoint{3.153052in}{1.375774in}}%
\pgfpathlineto{\pgfqpoint{3.402947in}{1.552500in}}%
\pgfpathlineto{\pgfqpoint{3.385008in}{1.562106in}}%
\pgfpathlineto{\pgfqpoint{3.475072in}{1.592361in}}%
\pgfpathlineto{\pgfqpoint{3.416535in}{1.517525in}}%
\pgfpathlineto{\pgfqpoint{3.413456in}{1.537640in}}%
\pgfpathlineto{\pgfqpoint{3.163561in}{1.360913in}}%
\pgfpathlineto{\pgfqpoint{3.153052in}{1.375774in}}%
\pgfusepath{fill}%
\end{pgfscope}%
\begin{pgfscope}%
\pgfpathrectangle{\pgfqpoint{1.065196in}{0.528000in}}{\pgfqpoint{3.702804in}{3.696000in}} %
\pgfusepath{clip}%
\pgfsetbuttcap%
\pgfsetroundjoin%
\definecolor{currentfill}{rgb}{0.277941,0.056324,0.381191}%
\pgfsetfillcolor{currentfill}%
\pgfsetlinewidth{0.000000pt}%
\definecolor{currentstroke}{rgb}{0.000000,0.000000,0.000000}%
\pgfsetstrokecolor{currentstroke}%
\pgfsetdash{}{0pt}%
\pgfpathmoveto{\pgfqpoint{3.149426in}{1.370333in}}%
\pgfpathlineto{\pgfqpoint{3.213248in}{1.655241in}}%
\pgfpathlineto{\pgfqpoint{3.193498in}{1.650339in}}%
\pgfpathlineto{\pgfqpoint{3.240031in}{1.733174in}}%
\pgfpathlineto{\pgfqpoint{3.246780in}{1.638404in}}%
\pgfpathlineto{\pgfqpoint{3.231008in}{1.651262in}}%
\pgfpathlineto{\pgfqpoint{3.167187in}{1.366354in}}%
\pgfpathlineto{\pgfqpoint{3.149426in}{1.370333in}}%
\pgfusepath{fill}%
\end{pgfscope}%
\begin{pgfscope}%
\pgfpathrectangle{\pgfqpoint{1.065196in}{0.528000in}}{\pgfqpoint{3.702804in}{3.696000in}} %
\pgfusepath{clip}%
\pgfsetbuttcap%
\pgfsetroundjoin%
\definecolor{currentfill}{rgb}{0.283187,0.125848,0.444960}%
\pgfsetfillcolor{currentfill}%
\pgfsetlinewidth{0.000000pt}%
\definecolor{currentstroke}{rgb}{0.000000,0.000000,0.000000}%
\pgfsetstrokecolor{currentstroke}%
\pgfsetdash{}{0pt}%
\pgfpathmoveto{\pgfqpoint{3.150015in}{1.372093in}}%
\pgfpathlineto{\pgfqpoint{3.331211in}{1.772863in}}%
\pgfpathlineto{\pgfqpoint{3.310877in}{1.772069in}}%
\pgfpathlineto{\pgfqpoint{3.373244in}{1.843744in}}%
\pgfpathlineto{\pgfqpoint{3.360630in}{1.749574in}}%
\pgfpathlineto{\pgfqpoint{3.347795in}{1.765365in}}%
\pgfpathlineto{\pgfqpoint{3.166599in}{1.364594in}}%
\pgfpathlineto{\pgfqpoint{3.150015in}{1.372093in}}%
\pgfusepath{fill}%
\end{pgfscope}%
\begin{pgfscope}%
\pgfpathrectangle{\pgfqpoint{1.065196in}{0.528000in}}{\pgfqpoint{3.702804in}{3.696000in}} %
\pgfusepath{clip}%
\pgfsetbuttcap%
\pgfsetroundjoin%
\definecolor{currentfill}{rgb}{0.223925,0.334994,0.548053}%
\pgfsetfillcolor{currentfill}%
\pgfsetlinewidth{0.000000pt}%
\definecolor{currentstroke}{rgb}{0.000000,0.000000,0.000000}%
\pgfsetstrokecolor{currentstroke}%
\pgfsetdash{}{0pt}%
\pgfpathmoveto{\pgfqpoint{2.098150in}{3.186522in}}%
\pgfpathlineto{\pgfqpoint{2.627977in}{3.372142in}}%
\pgfpathlineto{\pgfqpoint{2.613371in}{3.386311in}}%
\pgfpathlineto{\pgfqpoint{2.708283in}{3.390634in}}%
\pgfpathlineto{\pgfqpoint{2.631424in}{3.334779in}}%
\pgfpathlineto{\pgfqpoint{2.633995in}{3.354965in}}%
\pgfpathlineto{\pgfqpoint{2.104168in}{3.169345in}}%
\pgfpathlineto{\pgfqpoint{2.098150in}{3.186522in}}%
\pgfusepath{fill}%
\end{pgfscope}%
\begin{pgfscope}%
\pgfpathrectangle{\pgfqpoint{1.065196in}{0.528000in}}{\pgfqpoint{3.702804in}{3.696000in}} %
\pgfusepath{clip}%
\pgfsetbuttcap%
\pgfsetroundjoin%
\definecolor{currentfill}{rgb}{0.194100,0.399323,0.555565}%
\pgfsetfillcolor{currentfill}%
\pgfsetlinewidth{0.000000pt}%
\definecolor{currentstroke}{rgb}{0.000000,0.000000,0.000000}%
\pgfsetstrokecolor{currentstroke}%
\pgfsetdash{}{0pt}%
\pgfpathmoveto{\pgfqpoint{2.096528in}{3.185767in}}%
\pgfpathlineto{\pgfqpoint{2.330742in}{3.324239in}}%
\pgfpathlineto{\pgfqpoint{2.313646in}{3.335275in}}%
\pgfpathlineto{\pgfqpoint{2.405877in}{3.358088in}}%
\pgfpathlineto{\pgfqpoint{2.341434in}{3.288273in}}%
\pgfpathlineto{\pgfqpoint{2.340005in}{3.308572in}}%
\pgfpathlineto{\pgfqpoint{2.105791in}{3.170100in}}%
\pgfpathlineto{\pgfqpoint{2.096528in}{3.185767in}}%
\pgfusepath{fill}%
\end{pgfscope}%
\begin{pgfscope}%
\pgfpathrectangle{\pgfqpoint{1.065196in}{0.528000in}}{\pgfqpoint{3.702804in}{3.696000in}} %
\pgfusepath{clip}%
\pgfsetbuttcap%
\pgfsetroundjoin%
\definecolor{currentfill}{rgb}{0.282910,0.105393,0.426902}%
\pgfsetfillcolor{currentfill}%
\pgfsetlinewidth{0.000000pt}%
\definecolor{currentstroke}{rgb}{0.000000,0.000000,0.000000}%
\pgfsetstrokecolor{currentstroke}%
\pgfsetdash{}{0pt}%
\pgfpathmoveto{\pgfqpoint{2.100980in}{3.168835in}}%
\pgfpathlineto{\pgfqpoint{1.875103in}{3.173285in}}%
\pgfpathlineto{\pgfqpoint{1.883843in}{3.154909in}}%
\pgfpathlineto{\pgfqpoint{1.793395in}{3.183997in}}%
\pgfpathlineto{\pgfqpoint{1.884919in}{3.209501in}}%
\pgfpathlineto{\pgfqpoint{1.875462in}{3.191483in}}%
\pgfpathlineto{\pgfqpoint{2.101338in}{3.187032in}}%
\pgfpathlineto{\pgfqpoint{2.100980in}{3.168835in}}%
\pgfusepath{fill}%
\end{pgfscope}%
\begin{pgfscope}%
\pgfpathrectangle{\pgfqpoint{1.065196in}{0.528000in}}{\pgfqpoint{3.702804in}{3.696000in}} %
\pgfusepath{clip}%
\pgfsetbuttcap%
\pgfsetroundjoin%
\definecolor{currentfill}{rgb}{0.218130,0.347432,0.550038}%
\pgfsetfillcolor{currentfill}%
\pgfsetlinewidth{0.000000pt}%
\definecolor{currentstroke}{rgb}{0.000000,0.000000,0.000000}%
\pgfsetstrokecolor{currentstroke}%
\pgfsetdash{}{0pt}%
\pgfpathmoveto{\pgfqpoint{4.496079in}{3.518028in}}%
\pgfpathlineto{\pgfqpoint{4.391952in}{3.688819in}}%
\pgfpathlineto{\pgfqpoint{4.381149in}{3.671574in}}%
\pgfpathlineto{\pgfqpoint{4.357087in}{3.763487in}}%
\pgfpathlineto{\pgfqpoint{4.427770in}{3.699998in}}%
\pgfpathlineto{\pgfqpoint{4.407492in}{3.698293in}}%
\pgfpathlineto{\pgfqpoint{4.511619in}{3.527503in}}%
\pgfpathlineto{\pgfqpoint{4.496079in}{3.518028in}}%
\pgfusepath{fill}%
\end{pgfscope}%
\begin{pgfscope}%
\pgfpathrectangle{\pgfqpoint{1.065196in}{0.528000in}}{\pgfqpoint{3.702804in}{3.696000in}} %
\pgfusepath{clip}%
\pgfsetbuttcap%
\pgfsetroundjoin%
\definecolor{currentfill}{rgb}{0.282290,0.145912,0.461510}%
\pgfsetfillcolor{currentfill}%
\pgfsetlinewidth{0.000000pt}%
\definecolor{currentstroke}{rgb}{0.000000,0.000000,0.000000}%
\pgfsetstrokecolor{currentstroke}%
\pgfsetdash{}{0pt}%
\pgfpathmoveto{\pgfqpoint{4.496933in}{3.516851in}}%
\pgfpathlineto{\pgfqpoint{4.370342in}{3.664870in}}%
\pgfpathlineto{\pgfqpoint{4.362425in}{3.646125in}}%
\pgfpathlineto{\pgfqpoint{4.324024in}{3.733029in}}%
\pgfpathlineto{\pgfqpoint{4.403921in}{3.681614in}}%
\pgfpathlineto{\pgfqpoint{4.384174in}{3.676700in}}%
\pgfpathlineto{\pgfqpoint{4.510765in}{3.528681in}}%
\pgfpathlineto{\pgfqpoint{4.496933in}{3.516851in}}%
\pgfusepath{fill}%
\end{pgfscope}%
\begin{pgfscope}%
\pgfpathrectangle{\pgfqpoint{1.065196in}{0.528000in}}{\pgfqpoint{3.702804in}{3.696000in}} %
\pgfusepath{clip}%
\pgfsetbuttcap%
\pgfsetroundjoin%
\definecolor{currentfill}{rgb}{0.280894,0.078907,0.402329}%
\pgfsetfillcolor{currentfill}%
\pgfsetlinewidth{0.000000pt}%
\definecolor{currentstroke}{rgb}{0.000000,0.000000,0.000000}%
\pgfsetstrokecolor{currentstroke}%
\pgfsetdash{}{0pt}%
\pgfpathmoveto{\pgfqpoint{2.059856in}{2.349886in}}%
\pgfpathlineto{\pgfqpoint{2.480767in}{2.328081in}}%
\pgfpathlineto{\pgfqpoint{2.472620in}{2.346728in}}%
\pgfpathlineto{\pgfqpoint{2.562090in}{2.314755in}}%
\pgfpathlineto{\pgfqpoint{2.469795in}{2.292199in}}%
\pgfpathlineto{\pgfqpoint{2.479825in}{2.309904in}}%
\pgfpathlineto{\pgfqpoint{2.058915in}{2.331710in}}%
\pgfpathlineto{\pgfqpoint{2.059856in}{2.349886in}}%
\pgfusepath{fill}%
\end{pgfscope}%
\begin{pgfscope}%
\pgfpathrectangle{\pgfqpoint{1.065196in}{0.528000in}}{\pgfqpoint{3.702804in}{3.696000in}} %
\pgfusepath{clip}%
\pgfsetbuttcap%
\pgfsetroundjoin%
\definecolor{currentfill}{rgb}{0.273809,0.031497,0.358853}%
\pgfsetfillcolor{currentfill}%
\pgfsetlinewidth{0.000000pt}%
\definecolor{currentstroke}{rgb}{0.000000,0.000000,0.000000}%
\pgfsetstrokecolor{currentstroke}%
\pgfsetdash{}{0pt}%
\pgfpathmoveto{\pgfqpoint{2.051915in}{2.335601in}}%
\pgfpathlineto{\pgfqpoint{2.039105in}{2.354012in}}%
\pgfpathlineto{\pgfqpoint{2.029362in}{2.336147in}}%
\pgfpathlineto{\pgfqpoint{1.999799in}{2.426441in}}%
\pgfpathlineto{\pgfqpoint{2.074184in}{2.367331in}}%
\pgfpathlineto{\pgfqpoint{2.054046in}{2.364407in}}%
\pgfpathlineto{\pgfqpoint{2.066856in}{2.345995in}}%
\pgfpathlineto{\pgfqpoint{2.051915in}{2.335601in}}%
\pgfusepath{fill}%
\end{pgfscope}%
\begin{pgfscope}%
\pgfpathrectangle{\pgfqpoint{1.065196in}{0.528000in}}{\pgfqpoint{3.702804in}{3.696000in}} %
\pgfusepath{clip}%
\pgfsetbuttcap%
\pgfsetroundjoin%
\definecolor{currentfill}{rgb}{0.183898,0.422383,0.556944}%
\pgfsetfillcolor{currentfill}%
\pgfsetlinewidth{0.000000pt}%
\definecolor{currentstroke}{rgb}{0.000000,0.000000,0.000000}%
\pgfsetstrokecolor{currentstroke}%
\pgfsetdash{}{0pt}%
\pgfpathmoveto{\pgfqpoint{3.329905in}{3.471892in}}%
\pgfpathlineto{\pgfqpoint{3.685442in}{3.552074in}}%
\pgfpathlineto{\pgfqpoint{3.672561in}{3.567827in}}%
\pgfpathlineto{\pgfqpoint{3.767341in}{3.561215in}}%
\pgfpathlineto{\pgfqpoint{3.684573in}{3.514562in}}%
\pgfpathlineto{\pgfqpoint{3.689446in}{3.534319in}}%
\pgfpathlineto{\pgfqpoint{3.333909in}{3.454137in}}%
\pgfpathlineto{\pgfqpoint{3.329905in}{3.471892in}}%
\pgfusepath{fill}%
\end{pgfscope}%
\begin{pgfscope}%
\pgfpathrectangle{\pgfqpoint{1.065196in}{0.528000in}}{\pgfqpoint{3.702804in}{3.696000in}} %
\pgfusepath{clip}%
\pgfsetbuttcap%
\pgfsetroundjoin%
\definecolor{currentfill}{rgb}{0.183898,0.422383,0.556944}%
\pgfsetfillcolor{currentfill}%
\pgfsetlinewidth{0.000000pt}%
\definecolor{currentstroke}{rgb}{0.000000,0.000000,0.000000}%
\pgfsetstrokecolor{currentstroke}%
\pgfsetdash{}{0pt}%
\pgfpathmoveto{\pgfqpoint{3.327459in}{3.455075in}}%
\pgfpathlineto{\pgfqpoint{3.310119in}{3.464790in}}%
\pgfpathlineto{\pgfqpoint{3.309162in}{3.444464in}}%
\pgfpathlineto{\pgfqpoint{3.243113in}{3.512761in}}%
\pgfpathlineto{\pgfqpoint{3.335850in}{3.492100in}}%
\pgfpathlineto{\pgfqpoint{3.319015in}{3.480669in}}%
\pgfpathlineto{\pgfqpoint{3.336355in}{3.470954in}}%
\pgfpathlineto{\pgfqpoint{3.327459in}{3.455075in}}%
\pgfusepath{fill}%
\end{pgfscope}%
\begin{pgfscope}%
\pgfpathrectangle{\pgfqpoint{1.065196in}{0.528000in}}{\pgfqpoint{3.702804in}{3.696000in}} %
\pgfusepath{clip}%
\pgfsetbuttcap%
\pgfsetroundjoin%
\definecolor{currentfill}{rgb}{0.225863,0.330805,0.547314}%
\pgfsetfillcolor{currentfill}%
\pgfsetlinewidth{0.000000pt}%
\definecolor{currentstroke}{rgb}{0.000000,0.000000,0.000000}%
\pgfsetstrokecolor{currentstroke}%
\pgfsetdash{}{0pt}%
\pgfpathmoveto{\pgfqpoint{1.780981in}{0.759043in}}%
\pgfpathlineto{\pgfqpoint{2.365060in}{0.781484in}}%
\pgfpathlineto{\pgfqpoint{2.355267in}{0.799322in}}%
\pgfpathlineto{\pgfqpoint{2.447252in}{0.775535in}}%
\pgfpathlineto{\pgfqpoint{2.357363in}{0.744760in}}%
\pgfpathlineto{\pgfqpoint{2.365758in}{0.763297in}}%
\pgfpathlineto{\pgfqpoint{1.781680in}{0.740856in}}%
\pgfpathlineto{\pgfqpoint{1.780981in}{0.759043in}}%
\pgfusepath{fill}%
\end{pgfscope}%
\begin{pgfscope}%
\pgfpathrectangle{\pgfqpoint{1.065196in}{0.528000in}}{\pgfqpoint{3.702804in}{3.696000in}} %
\pgfusepath{clip}%
\pgfsetbuttcap%
\pgfsetroundjoin%
\definecolor{currentfill}{rgb}{0.281924,0.089666,0.412415}%
\pgfsetfillcolor{currentfill}%
\pgfsetlinewidth{0.000000pt}%
\definecolor{currentstroke}{rgb}{0.000000,0.000000,0.000000}%
\pgfsetstrokecolor{currentstroke}%
\pgfsetdash{}{0pt}%
\pgfpathmoveto{\pgfqpoint{1.780747in}{0.759031in}}%
\pgfpathlineto{\pgfqpoint{2.169510in}{0.784010in}}%
\pgfpathlineto{\pgfqpoint{2.159262in}{0.801589in}}%
\pgfpathlineto{\pgfqpoint{2.251829in}{0.780180in}}%
\pgfpathlineto{\pgfqpoint{2.162763in}{0.747100in}}%
\pgfpathlineto{\pgfqpoint{2.170677in}{0.765846in}}%
\pgfpathlineto{\pgfqpoint{1.781914in}{0.740868in}}%
\pgfpathlineto{\pgfqpoint{1.780747in}{0.759031in}}%
\pgfusepath{fill}%
\end{pgfscope}%
\begin{pgfscope}%
\pgfpathrectangle{\pgfqpoint{1.065196in}{0.528000in}}{\pgfqpoint{3.702804in}{3.696000in}} %
\pgfusepath{clip}%
\pgfsetbuttcap%
\pgfsetroundjoin%
\definecolor{currentfill}{rgb}{0.280868,0.160771,0.472899}%
\pgfsetfillcolor{currentfill}%
\pgfsetlinewidth{0.000000pt}%
\definecolor{currentstroke}{rgb}{0.000000,0.000000,0.000000}%
\pgfsetstrokecolor{currentstroke}%
\pgfsetdash{}{0pt}%
\pgfpathmoveto{\pgfqpoint{1.488255in}{2.324666in}}%
\pgfpathlineto{\pgfqpoint{1.627537in}{2.366022in}}%
\pgfpathlineto{\pgfqpoint{1.613632in}{2.380880in}}%
\pgfpathlineto{\pgfqpoint{1.708642in}{2.380611in}}%
\pgfpathlineto{\pgfqpoint{1.629174in}{2.328536in}}%
\pgfpathlineto{\pgfqpoint{1.632717in}{2.348574in}}%
\pgfpathlineto{\pgfqpoint{1.493435in}{2.307218in}}%
\pgfpathlineto{\pgfqpoint{1.488255in}{2.324666in}}%
\pgfusepath{fill}%
\end{pgfscope}%
\begin{pgfscope}%
\pgfpathrectangle{\pgfqpoint{1.065196in}{0.528000in}}{\pgfqpoint{3.702804in}{3.696000in}} %
\pgfusepath{clip}%
\pgfsetbuttcap%
\pgfsetroundjoin%
\definecolor{currentfill}{rgb}{0.174274,0.445044,0.557792}%
\pgfsetfillcolor{currentfill}%
\pgfsetlinewidth{0.000000pt}%
\definecolor{currentstroke}{rgb}{0.000000,0.000000,0.000000}%
\pgfsetstrokecolor{currentstroke}%
\pgfsetdash{}{0pt}%
\pgfpathmoveto{\pgfqpoint{1.489871in}{2.324990in}}%
\pgfpathlineto{\pgfqpoint{1.820017in}{2.360542in}}%
\pgfpathlineto{\pgfqpoint{1.809021in}{2.377664in}}%
\pgfpathlineto{\pgfqpoint{1.902424in}{2.360263in}}%
\pgfpathlineto{\pgfqpoint{1.814867in}{2.323376in}}%
\pgfpathlineto{\pgfqpoint{1.821966in}{2.342446in}}%
\pgfpathlineto{\pgfqpoint{1.491819in}{2.306894in}}%
\pgfpathlineto{\pgfqpoint{1.489871in}{2.324990in}}%
\pgfusepath{fill}%
\end{pgfscope}%
\begin{pgfscope}%
\pgfpathrectangle{\pgfqpoint{1.065196in}{0.528000in}}{\pgfqpoint{3.702804in}{3.696000in}} %
\pgfusepath{clip}%
\pgfsetbuttcap%
\pgfsetroundjoin%
\definecolor{currentfill}{rgb}{0.279566,0.067836,0.391917}%
\pgfsetfillcolor{currentfill}%
\pgfsetlinewidth{0.000000pt}%
\definecolor{currentstroke}{rgb}{0.000000,0.000000,0.000000}%
\pgfsetstrokecolor{currentstroke}%
\pgfsetdash{}{0pt}%
\pgfpathmoveto{\pgfqpoint{3.292752in}{2.591480in}}%
\pgfpathlineto{\pgfqpoint{3.292076in}{2.585017in}}%
\pgfpathlineto{\pgfqpoint{3.305677in}{2.590128in}}%
\pgfpathlineto{\pgfqpoint{3.279530in}{2.527532in}}%
\pgfpathlineto{\pgfqpoint{3.266902in}{2.594184in}}%
\pgfpathlineto{\pgfqpoint{3.279151in}{2.586369in}}%
\pgfpathlineto{\pgfqpoint{3.279827in}{2.592832in}}%
\pgfpathlineto{\pgfqpoint{3.292752in}{2.591480in}}%
\pgfusepath{fill}%
\end{pgfscope}%
\begin{pgfscope}%
\pgfpathrectangle{\pgfqpoint{1.065196in}{0.528000in}}{\pgfqpoint{3.702804in}{3.696000in}} %
\pgfusepath{clip}%
\pgfsetbuttcap%
\pgfsetroundjoin%
\definecolor{currentfill}{rgb}{0.227802,0.326594,0.546532}%
\pgfsetfillcolor{currentfill}%
\pgfsetlinewidth{0.000000pt}%
\definecolor{currentstroke}{rgb}{0.000000,0.000000,0.000000}%
\pgfsetstrokecolor{currentstroke}%
\pgfsetdash{}{0pt}%
\pgfpathmoveto{\pgfqpoint{3.294843in}{2.589048in}}%
\pgfpathlineto{\pgfqpoint{3.197363in}{2.320732in}}%
\pgfpathlineto{\pgfqpoint{3.217577in}{2.323071in}}%
\pgfpathlineto{\pgfqpoint{3.160843in}{2.246859in}}%
\pgfpathlineto{\pgfqpoint{3.166257in}{2.341716in}}%
\pgfpathlineto{\pgfqpoint{3.180257in}{2.326947in}}%
\pgfpathlineto{\pgfqpoint{3.277736in}{2.595263in}}%
\pgfpathlineto{\pgfqpoint{3.294843in}{2.589048in}}%
\pgfusepath{fill}%
\end{pgfscope}%
\begin{pgfscope}%
\pgfpathrectangle{\pgfqpoint{1.065196in}{0.528000in}}{\pgfqpoint{3.702804in}{3.696000in}} %
\pgfusepath{clip}%
\pgfsetbuttcap%
\pgfsetroundjoin%
\definecolor{currentfill}{rgb}{0.162142,0.474838,0.558140}%
\pgfsetfillcolor{currentfill}%
\pgfsetlinewidth{0.000000pt}%
\definecolor{currentstroke}{rgb}{0.000000,0.000000,0.000000}%
\pgfsetstrokecolor{currentstroke}%
\pgfsetdash{}{0pt}%
\pgfpathmoveto{\pgfqpoint{2.321717in}{4.006090in}}%
\pgfpathlineto{\pgfqpoint{2.503370in}{3.946673in}}%
\pgfpathlineto{\pgfqpoint{2.500379in}{3.966801in}}%
\pgfpathlineto{\pgfqpoint{2.578386in}{3.912561in}}%
\pgfpathlineto{\pgfqpoint{2.483405in}{3.914904in}}%
\pgfpathlineto{\pgfqpoint{2.497712in}{3.929374in}}%
\pgfpathlineto{\pgfqpoint{2.316059in}{3.988791in}}%
\pgfpathlineto{\pgfqpoint{2.321717in}{4.006090in}}%
\pgfusepath{fill}%
\end{pgfscope}%
\begin{pgfscope}%
\pgfpathrectangle{\pgfqpoint{1.065196in}{0.528000in}}{\pgfqpoint{3.702804in}{3.696000in}} %
\pgfusepath{clip}%
\pgfsetbuttcap%
\pgfsetroundjoin%
\definecolor{currentfill}{rgb}{0.250425,0.274290,0.533103}%
\pgfsetfillcolor{currentfill}%
\pgfsetlinewidth{0.000000pt}%
\definecolor{currentstroke}{rgb}{0.000000,0.000000,0.000000}%
\pgfsetstrokecolor{currentstroke}%
\pgfsetdash{}{0pt}%
\pgfpathmoveto{\pgfqpoint{3.190329in}{1.966254in}}%
\pgfpathlineto{\pgfqpoint{3.278777in}{2.071337in}}%
\pgfpathlineto{\pgfqpoint{3.258992in}{2.076095in}}%
\pgfpathlineto{\pgfqpoint{3.338481in}{2.128138in}}%
\pgfpathlineto{\pgfqpoint{3.300766in}{2.040934in}}%
\pgfpathlineto{\pgfqpoint{3.292702in}{2.059617in}}%
\pgfpathlineto{\pgfqpoint{3.204253in}{1.954533in}}%
\pgfpathlineto{\pgfqpoint{3.190329in}{1.966254in}}%
\pgfusepath{fill}%
\end{pgfscope}%
\begin{pgfscope}%
\pgfpathrectangle{\pgfqpoint{1.065196in}{0.528000in}}{\pgfqpoint{3.702804in}{3.696000in}} %
\pgfusepath{clip}%
\pgfsetbuttcap%
\pgfsetroundjoin%
\definecolor{currentfill}{rgb}{0.282327,0.094955,0.417331}%
\pgfsetfillcolor{currentfill}%
\pgfsetlinewidth{0.000000pt}%
\definecolor{currentstroke}{rgb}{0.000000,0.000000,0.000000}%
\pgfsetstrokecolor{currentstroke}%
\pgfsetdash{}{0pt}%
\pgfpathmoveto{\pgfqpoint{3.495432in}{1.583641in}}%
\pgfpathlineto{\pgfqpoint{3.847673in}{1.640271in}}%
\pgfpathlineto{\pgfqpoint{3.835799in}{1.656797in}}%
\pgfpathlineto{\pgfqpoint{3.929983in}{1.644287in}}%
\pgfpathlineto{\pgfqpoint{3.844467in}{1.602887in}}%
\pgfpathlineto{\pgfqpoint{3.850563in}{1.622301in}}%
\pgfpathlineto{\pgfqpoint{3.498321in}{1.565671in}}%
\pgfpathlineto{\pgfqpoint{3.495432in}{1.583641in}}%
\pgfusepath{fill}%
\end{pgfscope}%
\begin{pgfscope}%
\pgfpathrectangle{\pgfqpoint{1.065196in}{0.528000in}}{\pgfqpoint{3.702804in}{3.696000in}} %
\pgfusepath{clip}%
\pgfsetbuttcap%
\pgfsetroundjoin%
\definecolor{currentfill}{rgb}{0.169646,0.456262,0.558030}%
\pgfsetfillcolor{currentfill}%
\pgfsetlinewidth{0.000000pt}%
\definecolor{currentstroke}{rgb}{0.000000,0.000000,0.000000}%
\pgfsetstrokecolor{currentstroke}%
\pgfsetdash{}{0pt}%
\pgfpathmoveto{\pgfqpoint{3.488608in}{1.570856in}}%
\pgfpathlineto{\pgfqpoint{3.399169in}{1.765521in}}%
\pgfpathlineto{\pgfqpoint{3.386430in}{1.749653in}}%
\pgfpathlineto{\pgfqpoint{3.373244in}{1.843744in}}%
\pgfpathlineto{\pgfqpoint{3.436046in}{1.772449in}}%
\pgfpathlineto{\pgfqpoint{3.415708in}{1.773119in}}%
\pgfpathlineto{\pgfqpoint{3.505146in}{1.578455in}}%
\pgfpathlineto{\pgfqpoint{3.488608in}{1.570856in}}%
\pgfusepath{fill}%
\end{pgfscope}%
\begin{pgfscope}%
\pgfpathrectangle{\pgfqpoint{1.065196in}{0.528000in}}{\pgfqpoint{3.702804in}{3.696000in}} %
\pgfusepath{clip}%
\pgfsetbuttcap%
\pgfsetroundjoin%
\definecolor{currentfill}{rgb}{0.229739,0.322361,0.545706}%
\pgfsetfillcolor{currentfill}%
\pgfsetlinewidth{0.000000pt}%
\definecolor{currentstroke}{rgb}{0.000000,0.000000,0.000000}%
\pgfsetstrokecolor{currentstroke}%
\pgfsetdash{}{0pt}%
\pgfpathmoveto{\pgfqpoint{1.471993in}{1.933087in}}%
\pgfpathlineto{\pgfqpoint{1.731559in}{2.198575in}}%
\pgfpathlineto{\pgfqpoint{1.712183in}{2.204792in}}%
\pgfpathlineto{\pgfqpoint{1.795324in}{2.250777in}}%
\pgfpathlineto{\pgfqpoint{1.751226in}{2.166620in}}%
\pgfpathlineto{\pgfqpoint{1.744574in}{2.185851in}}%
\pgfpathlineto{\pgfqpoint{1.485008in}{1.920363in}}%
\pgfpathlineto{\pgfqpoint{1.471993in}{1.933087in}}%
\pgfusepath{fill}%
\end{pgfscope}%
\begin{pgfscope}%
\pgfpathrectangle{\pgfqpoint{1.065196in}{0.528000in}}{\pgfqpoint{3.702804in}{3.696000in}} %
\pgfusepath{clip}%
\pgfsetbuttcap%
\pgfsetroundjoin%
\definecolor{currentfill}{rgb}{0.255645,0.260703,0.528312}%
\pgfsetfillcolor{currentfill}%
\pgfsetlinewidth{0.000000pt}%
\definecolor{currentstroke}{rgb}{0.000000,0.000000,0.000000}%
\pgfsetstrokecolor{currentstroke}%
\pgfsetdash{}{0pt}%
\pgfpathmoveto{\pgfqpoint{3.356613in}{1.395763in}}%
\pgfpathlineto{\pgfqpoint{3.427642in}{1.524972in}}%
\pgfpathlineto{\pgfqpoint{3.407309in}{1.525765in}}%
\pgfpathlineto{\pgfqpoint{3.475072in}{1.592361in}}%
\pgfpathlineto{\pgfqpoint{3.455157in}{1.499461in}}%
\pgfpathlineto{\pgfqpoint{3.443592in}{1.516204in}}%
\pgfpathlineto{\pgfqpoint{3.372562in}{1.386995in}}%
\pgfpathlineto{\pgfqpoint{3.356613in}{1.395763in}}%
\pgfusepath{fill}%
\end{pgfscope}%
\begin{pgfscope}%
\pgfpathrectangle{\pgfqpoint{1.065196in}{0.528000in}}{\pgfqpoint{3.702804in}{3.696000in}} %
\pgfusepath{clip}%
\pgfsetbuttcap%
\pgfsetroundjoin%
\definecolor{currentfill}{rgb}{0.263663,0.237631,0.518762}%
\pgfsetfillcolor{currentfill}%
\pgfsetlinewidth{0.000000pt}%
\definecolor{currentstroke}{rgb}{0.000000,0.000000,0.000000}%
\pgfsetstrokecolor{currentstroke}%
\pgfsetdash{}{0pt}%
\pgfpathmoveto{\pgfqpoint{3.364770in}{1.400477in}}%
\pgfpathlineto{\pgfqpoint{3.613873in}{1.395490in}}%
\pgfpathlineto{\pgfqpoint{3.605139in}{1.413869in}}%
\pgfpathlineto{\pgfqpoint{3.695578in}{1.384752in}}%
\pgfpathlineto{\pgfqpoint{3.604046in}{1.359278in}}%
\pgfpathlineto{\pgfqpoint{3.613509in}{1.377293in}}%
\pgfpathlineto{\pgfqpoint{3.364406in}{1.382280in}}%
\pgfpathlineto{\pgfqpoint{3.364770in}{1.400477in}}%
\pgfusepath{fill}%
\end{pgfscope}%
\begin{pgfscope}%
\pgfpathrectangle{\pgfqpoint{1.065196in}{0.528000in}}{\pgfqpoint{3.702804in}{3.696000in}} %
\pgfusepath{clip}%
\pgfsetbuttcap%
\pgfsetroundjoin%
\definecolor{currentfill}{rgb}{0.260571,0.246922,0.522828}%
\pgfsetfillcolor{currentfill}%
\pgfsetlinewidth{0.000000pt}%
\definecolor{currentstroke}{rgb}{0.000000,0.000000,0.000000}%
\pgfsetstrokecolor{currentstroke}%
\pgfsetdash{}{0pt}%
\pgfpathmoveto{\pgfqpoint{3.781676in}{0.917991in}}%
\pgfpathlineto{\pgfqpoint{3.963293in}{0.792164in}}%
\pgfpathlineto{\pgfqpoint{3.966178in}{0.812308in}}%
\pgfpathlineto{\pgfqpoint{4.025435in}{0.738041in}}%
\pgfpathlineto{\pgfqpoint{3.935082in}{0.767425in}}%
\pgfpathlineto{\pgfqpoint{3.952928in}{0.777203in}}%
\pgfpathlineto{\pgfqpoint{3.771311in}{0.903030in}}%
\pgfpathlineto{\pgfqpoint{3.781676in}{0.917991in}}%
\pgfusepath{fill}%
\end{pgfscope}%
\begin{pgfscope}%
\pgfpathrectangle{\pgfqpoint{1.065196in}{0.528000in}}{\pgfqpoint{3.702804in}{3.696000in}} %
\pgfusepath{clip}%
\pgfsetbuttcap%
\pgfsetroundjoin%
\definecolor{currentfill}{rgb}{0.208030,0.718701,0.472873}%
\pgfsetfillcolor{currentfill}%
\pgfsetlinewidth{0.000000pt}%
\definecolor{currentstroke}{rgb}{0.000000,0.000000,0.000000}%
\pgfsetstrokecolor{currentstroke}%
\pgfsetdash{}{0pt}%
\pgfpathmoveto{\pgfqpoint{3.781622in}{0.918028in}}%
\pgfpathlineto{\pgfqpoint{3.870838in}{0.857170in}}%
\pgfpathlineto{\pgfqpoint{3.873576in}{0.877334in}}%
\pgfpathlineto{\pgfqpoint{3.933370in}{0.803498in}}%
\pgfpathlineto{\pgfqpoint{3.842807in}{0.832227in}}%
\pgfpathlineto{\pgfqpoint{3.860581in}{0.842134in}}%
\pgfpathlineto{\pgfqpoint{3.771365in}{0.902993in}}%
\pgfpathlineto{\pgfqpoint{3.781622in}{0.918028in}}%
\pgfusepath{fill}%
\end{pgfscope}%
\begin{pgfscope}%
\pgfpathrectangle{\pgfqpoint{1.065196in}{0.528000in}}{\pgfqpoint{3.702804in}{3.696000in}} %
\pgfusepath{clip}%
\pgfsetbuttcap%
\pgfsetroundjoin%
\definecolor{currentfill}{rgb}{0.208623,0.367752,0.552675}%
\pgfsetfillcolor{currentfill}%
\pgfsetlinewidth{0.000000pt}%
\definecolor{currentstroke}{rgb}{0.000000,0.000000,0.000000}%
\pgfsetstrokecolor{currentstroke}%
\pgfsetdash{}{0pt}%
\pgfpathmoveto{\pgfqpoint{2.127769in}{3.391009in}}%
\pgfpathlineto{\pgfqpoint{2.626252in}{3.398503in}}%
\pgfpathlineto{\pgfqpoint{2.616879in}{3.416564in}}%
\pgfpathlineto{\pgfqpoint{2.708283in}{3.390634in}}%
\pgfpathlineto{\pgfqpoint{2.617700in}{3.361968in}}%
\pgfpathlineto{\pgfqpoint{2.626526in}{3.380304in}}%
\pgfpathlineto{\pgfqpoint{2.128043in}{3.372811in}}%
\pgfpathlineto{\pgfqpoint{2.127769in}{3.391009in}}%
\pgfusepath{fill}%
\end{pgfscope}%
\begin{pgfscope}%
\pgfpathrectangle{\pgfqpoint{1.065196in}{0.528000in}}{\pgfqpoint{3.702804in}{3.696000in}} %
\pgfusepath{clip}%
\pgfsetbuttcap%
\pgfsetroundjoin%
\definecolor{currentfill}{rgb}{0.270595,0.214069,0.507052}%
\pgfsetfillcolor{currentfill}%
\pgfsetlinewidth{0.000000pt}%
\definecolor{currentstroke}{rgb}{0.000000,0.000000,0.000000}%
\pgfsetstrokecolor{currentstroke}%
\pgfsetdash{}{0pt}%
\pgfpathmoveto{\pgfqpoint{1.912855in}{2.830650in}}%
\pgfpathlineto{\pgfqpoint{1.989671in}{2.508224in}}%
\pgfpathlineto{\pgfqpoint{2.005267in}{2.521295in}}%
\pgfpathlineto{\pgfqpoint{1.999799in}{2.426441in}}%
\pgfpathlineto{\pgfqpoint{1.952151in}{2.508640in}}%
\pgfpathlineto{\pgfqpoint{1.971965in}{2.504006in}}%
\pgfpathlineto{\pgfqpoint{1.895150in}{2.826432in}}%
\pgfpathlineto{\pgfqpoint{1.912855in}{2.830650in}}%
\pgfusepath{fill}%
\end{pgfscope}%
\begin{pgfscope}%
\pgfpathrectangle{\pgfqpoint{1.065196in}{0.528000in}}{\pgfqpoint{3.702804in}{3.696000in}} %
\pgfusepath{clip}%
\pgfsetbuttcap%
\pgfsetroundjoin%
\definecolor{currentfill}{rgb}{0.266580,0.228262,0.514349}%
\pgfsetfillcolor{currentfill}%
\pgfsetlinewidth{0.000000pt}%
\definecolor{currentstroke}{rgb}{0.000000,0.000000,0.000000}%
\pgfsetstrokecolor{currentstroke}%
\pgfsetdash{}{0pt}%
\pgfpathmoveto{\pgfqpoint{3.018604in}{3.790924in}}%
\pgfpathlineto{\pgfqpoint{3.146679in}{3.688239in}}%
\pgfpathlineto{\pgfqpoint{3.150964in}{3.708132in}}%
\pgfpathlineto{\pgfqpoint{3.204887in}{3.629906in}}%
\pgfpathlineto{\pgfqpoint{3.116808in}{3.665531in}}%
\pgfpathlineto{\pgfqpoint{3.135294in}{3.674039in}}%
\pgfpathlineto{\pgfqpoint{3.007219in}{3.776724in}}%
\pgfpathlineto{\pgfqpoint{3.018604in}{3.790924in}}%
\pgfusepath{fill}%
\end{pgfscope}%
\begin{pgfscope}%
\pgfpathrectangle{\pgfqpoint{1.065196in}{0.528000in}}{\pgfqpoint{3.702804in}{3.696000in}} %
\pgfusepath{clip}%
\pgfsetbuttcap%
\pgfsetroundjoin%
\definecolor{currentfill}{rgb}{0.273809,0.031497,0.358853}%
\pgfsetfillcolor{currentfill}%
\pgfsetlinewidth{0.000000pt}%
\definecolor{currentstroke}{rgb}{0.000000,0.000000,0.000000}%
\pgfsetstrokecolor{currentstroke}%
\pgfsetdash{}{0pt}%
\pgfpathmoveto{\pgfqpoint{3.022009in}{3.784049in}}%
\pgfpathlineto{\pgfqpoint{3.030169in}{3.455331in}}%
\pgfpathlineto{\pgfqpoint{3.048138in}{3.464881in}}%
\pgfpathlineto{\pgfqpoint{3.023103in}{3.373228in}}%
\pgfpathlineto{\pgfqpoint{2.993553in}{3.463526in}}%
\pgfpathlineto{\pgfqpoint{3.011974in}{3.454880in}}%
\pgfpathlineto{\pgfqpoint{3.003814in}{3.783598in}}%
\pgfpathlineto{\pgfqpoint{3.022009in}{3.784049in}}%
\pgfusepath{fill}%
\end{pgfscope}%
\begin{pgfscope}%
\pgfpathrectangle{\pgfqpoint{1.065196in}{0.528000in}}{\pgfqpoint{3.702804in}{3.696000in}} %
\pgfusepath{clip}%
\pgfsetbuttcap%
\pgfsetroundjoin%
\definecolor{currentfill}{rgb}{0.212395,0.359683,0.551710}%
\pgfsetfillcolor{currentfill}%
\pgfsetlinewidth{0.000000pt}%
\definecolor{currentstroke}{rgb}{0.000000,0.000000,0.000000}%
\pgfsetstrokecolor{currentstroke}%
\pgfsetdash{}{0pt}%
\pgfpathmoveto{\pgfqpoint{2.137630in}{0.917681in}}%
\pgfpathlineto{\pgfqpoint{2.321548in}{0.922889in}}%
\pgfpathlineto{\pgfqpoint{2.311936in}{0.940825in}}%
\pgfpathlineto{\pgfqpoint{2.403676in}{0.916111in}}%
\pgfpathlineto{\pgfqpoint{2.313482in}{0.886245in}}%
\pgfpathlineto{\pgfqpoint{2.322063in}{0.904696in}}%
\pgfpathlineto{\pgfqpoint{2.138145in}{0.899488in}}%
\pgfpathlineto{\pgfqpoint{2.137630in}{0.917681in}}%
\pgfusepath{fill}%
\end{pgfscope}%
\begin{pgfscope}%
\pgfpathrectangle{\pgfqpoint{1.065196in}{0.528000in}}{\pgfqpoint{3.702804in}{3.696000in}} %
\pgfusepath{clip}%
\pgfsetbuttcap%
\pgfsetroundjoin%
\definecolor{currentfill}{rgb}{0.154815,0.493313,0.557840}%
\pgfsetfillcolor{currentfill}%
\pgfsetlinewidth{0.000000pt}%
\definecolor{currentstroke}{rgb}{0.000000,0.000000,0.000000}%
\pgfsetstrokecolor{currentstroke}%
\pgfsetdash{}{0pt}%
\pgfpathmoveto{\pgfqpoint{2.136811in}{0.917621in}}%
\pgfpathlineto{\pgfqpoint{2.439232in}{0.953673in}}%
\pgfpathlineto{\pgfqpoint{2.428041in}{0.970668in}}%
\pgfpathlineto{\pgfqpoint{2.521636in}{0.954332in}}%
\pgfpathlineto{\pgfqpoint{2.434504in}{0.916450in}}%
\pgfpathlineto{\pgfqpoint{2.441386in}{0.935600in}}%
\pgfpathlineto{\pgfqpoint{2.138965in}{0.899548in}}%
\pgfpathlineto{\pgfqpoint{2.136811in}{0.917621in}}%
\pgfusepath{fill}%
\end{pgfscope}%
\begin{pgfscope}%
\pgfpathrectangle{\pgfqpoint{1.065196in}{0.528000in}}{\pgfqpoint{3.702804in}{3.696000in}} %
\pgfusepath{clip}%
\pgfsetbuttcap%
\pgfsetroundjoin%
\definecolor{currentfill}{rgb}{0.278791,0.062145,0.386592}%
\pgfsetfillcolor{currentfill}%
\pgfsetlinewidth{0.000000pt}%
\definecolor{currentstroke}{rgb}{0.000000,0.000000,0.000000}%
\pgfsetstrokecolor{currentstroke}%
\pgfsetdash{}{0pt}%
\pgfpathmoveto{\pgfqpoint{2.141483in}{0.916944in}}%
\pgfpathlineto{\pgfqpoint{2.375607in}{0.816253in}}%
\pgfpathlineto{\pgfqpoint{2.374438in}{0.836569in}}%
\pgfpathlineto{\pgfqpoint{2.447252in}{0.775535in}}%
\pgfpathlineto{\pgfqpoint{2.352866in}{0.786409in}}%
\pgfpathlineto{\pgfqpoint{2.368416in}{0.799533in}}%
\pgfpathlineto{\pgfqpoint{2.134292in}{0.900224in}}%
\pgfpathlineto{\pgfqpoint{2.141483in}{0.916944in}}%
\pgfusepath{fill}%
\end{pgfscope}%
\begin{pgfscope}%
\pgfpathrectangle{\pgfqpoint{1.065196in}{0.528000in}}{\pgfqpoint{3.702804in}{3.696000in}} %
\pgfusepath{clip}%
\pgfsetbuttcap%
\pgfsetroundjoin%
\definecolor{currentfill}{rgb}{0.223925,0.334994,0.548053}%
\pgfsetfillcolor{currentfill}%
\pgfsetlinewidth{0.000000pt}%
\definecolor{currentstroke}{rgb}{0.000000,0.000000,0.000000}%
\pgfsetstrokecolor{currentstroke}%
\pgfsetdash{}{0pt}%
\pgfpathmoveto{\pgfqpoint{3.711195in}{3.279625in}}%
\pgfpathlineto{\pgfqpoint{3.956018in}{3.499243in}}%
\pgfpathlineto{\pgfqpoint{3.937091in}{3.506715in}}%
\pgfpathlineto{\pgfqpoint{4.023063in}{3.547160in}}%
\pgfpathlineto{\pgfqpoint{3.973551in}{3.466070in}}%
\pgfpathlineto{\pgfqpoint{3.968172in}{3.485695in}}%
\pgfpathlineto{\pgfqpoint{3.723349in}{3.266076in}}%
\pgfpathlineto{\pgfqpoint{3.711195in}{3.279625in}}%
\pgfusepath{fill}%
\end{pgfscope}%
\begin{pgfscope}%
\pgfpathrectangle{\pgfqpoint{1.065196in}{0.528000in}}{\pgfqpoint{3.702804in}{3.696000in}} %
\pgfusepath{clip}%
\pgfsetbuttcap%
\pgfsetroundjoin%
\definecolor{currentfill}{rgb}{0.146616,0.673050,0.508936}%
\pgfsetfillcolor{currentfill}%
\pgfsetlinewidth{0.000000pt}%
\definecolor{currentstroke}{rgb}{0.000000,0.000000,0.000000}%
\pgfsetstrokecolor{currentstroke}%
\pgfsetdash{}{0pt}%
\pgfpathmoveto{\pgfqpoint{3.724139in}{3.278822in}}%
\pgfpathlineto{\pgfqpoint{3.788148in}{3.205211in}}%
\pgfpathlineto{\pgfqpoint{3.795911in}{3.224021in}}%
\pgfpathlineto{\pgfqpoint{3.835023in}{3.137434in}}%
\pgfpathlineto{\pgfqpoint{3.754708in}{3.188193in}}%
\pgfpathlineto{\pgfqpoint{3.774413in}{3.193268in}}%
\pgfpathlineto{\pgfqpoint{3.710405in}{3.266879in}}%
\pgfpathlineto{\pgfqpoint{3.724139in}{3.278822in}}%
\pgfusepath{fill}%
\end{pgfscope}%
\begin{pgfscope}%
\pgfpathrectangle{\pgfqpoint{1.065196in}{0.528000in}}{\pgfqpoint{3.702804in}{3.696000in}} %
\pgfusepath{clip}%
\pgfsetbuttcap%
\pgfsetroundjoin%
\definecolor{currentfill}{rgb}{0.229739,0.322361,0.545706}%
\pgfsetfillcolor{currentfill}%
\pgfsetlinewidth{0.000000pt}%
\definecolor{currentstroke}{rgb}{0.000000,0.000000,0.000000}%
\pgfsetstrokecolor{currentstroke}%
\pgfsetdash{}{0pt}%
\pgfpathmoveto{\pgfqpoint{4.300035in}{3.813956in}}%
\pgfpathlineto{\pgfqpoint{4.306183in}{3.809524in}}%
\pgfpathlineto{\pgfqpoint{4.308899in}{3.826253in}}%
\pgfpathlineto{\pgfqpoint{4.357087in}{3.763487in}}%
\pgfpathlineto{\pgfqpoint{4.282307in}{3.789362in}}%
\pgfpathlineto{\pgfqpoint{4.297319in}{3.797227in}}%
\pgfpathlineto{\pgfqpoint{4.291171in}{3.801659in}}%
\pgfpathlineto{\pgfqpoint{4.300035in}{3.813956in}}%
\pgfusepath{fill}%
\end{pgfscope}%
\begin{pgfscope}%
\pgfpathrectangle{\pgfqpoint{1.065196in}{0.528000in}}{\pgfqpoint{3.702804in}{3.696000in}} %
\pgfusepath{clip}%
\pgfsetbuttcap%
\pgfsetroundjoin%
\definecolor{currentfill}{rgb}{0.274952,0.037752,0.364543}%
\pgfsetfillcolor{currentfill}%
\pgfsetlinewidth{0.000000pt}%
\definecolor{currentstroke}{rgb}{0.000000,0.000000,0.000000}%
\pgfsetstrokecolor{currentstroke}%
\pgfsetdash{}{0pt}%
\pgfpathmoveto{\pgfqpoint{4.302073in}{3.801409in}}%
\pgfpathlineto{\pgfqpoint{4.266536in}{3.765472in}}%
\pgfpathlineto{\pgfqpoint{4.285876in}{3.759145in}}%
\pgfpathlineto{\pgfqpoint{4.202475in}{3.713634in}}%
\pgfpathlineto{\pgfqpoint{4.247052in}{3.797538in}}%
\pgfpathlineto{\pgfqpoint{4.253594in}{3.778270in}}%
\pgfpathlineto{\pgfqpoint{4.289132in}{3.814206in}}%
\pgfpathlineto{\pgfqpoint{4.302073in}{3.801409in}}%
\pgfusepath{fill}%
\end{pgfscope}%
\begin{pgfscope}%
\pgfpathrectangle{\pgfqpoint{1.065196in}{0.528000in}}{\pgfqpoint{3.702804in}{3.696000in}} %
\pgfusepath{clip}%
\pgfsetbuttcap%
\pgfsetroundjoin%
\definecolor{currentfill}{rgb}{0.177423,0.437527,0.557565}%
\pgfsetfillcolor{currentfill}%
\pgfsetlinewidth{0.000000pt}%
\definecolor{currentstroke}{rgb}{0.000000,0.000000,0.000000}%
\pgfsetstrokecolor{currentstroke}%
\pgfsetdash{}{0pt}%
\pgfpathmoveto{\pgfqpoint{4.291988in}{3.799456in}}%
\pgfpathlineto{\pgfqpoint{4.229200in}{3.826630in}}%
\pgfpathlineto{\pgfqpoint{4.230323in}{3.806311in}}%
\pgfpathlineto{\pgfqpoint{4.157648in}{3.867512in}}%
\pgfpathlineto{\pgfqpoint{4.252010in}{3.856422in}}%
\pgfpathlineto{\pgfqpoint{4.236429in}{3.843333in}}%
\pgfpathlineto{\pgfqpoint{4.299217in}{3.816159in}}%
\pgfpathlineto{\pgfqpoint{4.291988in}{3.799456in}}%
\pgfusepath{fill}%
\end{pgfscope}%
\begin{pgfscope}%
\pgfpathrectangle{\pgfqpoint{1.065196in}{0.528000in}}{\pgfqpoint{3.702804in}{3.696000in}} %
\pgfusepath{clip}%
\pgfsetbuttcap%
\pgfsetroundjoin%
\definecolor{currentfill}{rgb}{0.274128,0.199721,0.498911}%
\pgfsetfillcolor{currentfill}%
\pgfsetlinewidth{0.000000pt}%
\definecolor{currentstroke}{rgb}{0.000000,0.000000,0.000000}%
\pgfsetstrokecolor{currentstroke}%
\pgfsetdash{}{0pt}%
\pgfpathmoveto{\pgfqpoint{1.308225in}{1.479898in}}%
\pgfpathlineto{\pgfqpoint{1.776005in}{1.164059in}}%
\pgfpathlineto{\pgfqpoint{1.778648in}{1.184235in}}%
\pgfpathlineto{\pgfqpoint{1.838793in}{1.110685in}}%
\pgfpathlineto{\pgfqpoint{1.748094in}{1.138982in}}%
\pgfpathlineto{\pgfqpoint{1.765821in}{1.148974in}}%
\pgfpathlineto{\pgfqpoint{1.298040in}{1.464813in}}%
\pgfpathlineto{\pgfqpoint{1.308225in}{1.479898in}}%
\pgfusepath{fill}%
\end{pgfscope}%
\begin{pgfscope}%
\pgfpathrectangle{\pgfqpoint{1.065196in}{0.528000in}}{\pgfqpoint{3.702804in}{3.696000in}} %
\pgfusepath{clip}%
\pgfsetbuttcap%
\pgfsetroundjoin%
\definecolor{currentfill}{rgb}{0.266580,0.228262,0.514349}%
\pgfsetfillcolor{currentfill}%
\pgfsetlinewidth{0.000000pt}%
\definecolor{currentstroke}{rgb}{0.000000,0.000000,0.000000}%
\pgfsetstrokecolor{currentstroke}%
\pgfsetdash{}{0pt}%
\pgfpathmoveto{\pgfqpoint{1.294235in}{1.474269in}}%
\pgfpathlineto{\pgfqpoint{1.410788in}{2.016321in}}%
\pgfpathlineto{\pgfqpoint{1.391081in}{2.011250in}}%
\pgfpathlineto{\pgfqpoint{1.436902in}{2.094481in}}%
\pgfpathlineto{\pgfqpoint{1.444463in}{1.999771in}}%
\pgfpathlineto{\pgfqpoint{1.428582in}{2.012495in}}%
\pgfpathlineto{\pgfqpoint{1.312029in}{1.470442in}}%
\pgfpathlineto{\pgfqpoint{1.294235in}{1.474269in}}%
\pgfusepath{fill}%
\end{pgfscope}%
\begin{pgfscope}%
\pgfpathrectangle{\pgfqpoint{1.065196in}{0.528000in}}{\pgfqpoint{3.702804in}{3.696000in}} %
\pgfusepath{clip}%
\pgfsetbuttcap%
\pgfsetroundjoin%
\definecolor{currentfill}{rgb}{0.192357,0.403199,0.555836}%
\pgfsetfillcolor{currentfill}%
\pgfsetlinewidth{0.000000pt}%
\definecolor{currentstroke}{rgb}{0.000000,0.000000,0.000000}%
\pgfsetstrokecolor{currentstroke}%
\pgfsetdash{}{0pt}%
\pgfpathmoveto{\pgfqpoint{1.294102in}{1.473479in}}%
\pgfpathlineto{\pgfqpoint{1.339053in}{1.834717in}}%
\pgfpathlineto{\pgfqpoint{1.319868in}{1.827934in}}%
\pgfpathlineto{\pgfqpoint{1.358197in}{1.914870in}}%
\pgfpathlineto{\pgfqpoint{1.374052in}{1.821191in}}%
\pgfpathlineto{\pgfqpoint{1.357114in}{1.832470in}}%
\pgfpathlineto{\pgfqpoint{1.312163in}{1.471232in}}%
\pgfpathlineto{\pgfqpoint{1.294102in}{1.473479in}}%
\pgfusepath{fill}%
\end{pgfscope}%
\begin{pgfscope}%
\pgfpathrectangle{\pgfqpoint{1.065196in}{0.528000in}}{\pgfqpoint{3.702804in}{3.696000in}} %
\pgfusepath{clip}%
\pgfsetbuttcap%
\pgfsetroundjoin%
\definecolor{currentfill}{rgb}{0.273006,0.204520,0.501721}%
\pgfsetfillcolor{currentfill}%
\pgfsetlinewidth{0.000000pt}%
\definecolor{currentstroke}{rgb}{0.000000,0.000000,0.000000}%
\pgfsetstrokecolor{currentstroke}%
\pgfsetdash{}{0pt}%
\pgfpathmoveto{\pgfqpoint{3.325263in}{3.873046in}}%
\pgfpathlineto{\pgfqpoint{3.460200in}{3.807160in}}%
\pgfpathlineto{\pgfqpoint{3.460008in}{3.827508in}}%
\pgfpathlineto{\pgfqpoint{3.529806in}{3.763047in}}%
\pgfpathlineto{\pgfqpoint{3.436051in}{3.778443in}}%
\pgfpathlineto{\pgfqpoint{3.452214in}{3.790805in}}%
\pgfpathlineto{\pgfqpoint{3.317277in}{3.856691in}}%
\pgfpathlineto{\pgfqpoint{3.325263in}{3.873046in}}%
\pgfusepath{fill}%
\end{pgfscope}%
\begin{pgfscope}%
\pgfsetbuttcap%
\pgfsetroundjoin%
\definecolor{currentfill}{rgb}{0.000000,0.000000,0.000000}%
\pgfsetfillcolor{currentfill}%
\pgfsetlinewidth{0.803000pt}%
\definecolor{currentstroke}{rgb}{0.000000,0.000000,0.000000}%
\pgfsetstrokecolor{currentstroke}%
\pgfsetdash{}{0pt}%
\pgfsys@defobject{currentmarker}{\pgfqpoint{0.000000in}{-0.048611in}}{\pgfqpoint{0.000000in}{0.000000in}}{%
\pgfpathmoveto{\pgfqpoint{0.000000in}{0.000000in}}%
\pgfpathlineto{\pgfqpoint{0.000000in}{-0.048611in}}%
\pgfusepath{stroke,fill}%
}%
\begin{pgfscope}%
\pgfsys@transformshift{1.256425in}{0.528000in}%
\pgfsys@useobject{currentmarker}{}%
\end{pgfscope}%
\end{pgfscope}%
\begin{pgfscope}%
\pgftext[x=1.256425in,y=0.430778in,,top]{\rmfamily\fontsize{10.000000}{12.000000}\selectfont \(\displaystyle 0.0\)}%
\end{pgfscope}%
\begin{pgfscope}%
\pgfsetbuttcap%
\pgfsetroundjoin%
\definecolor{currentfill}{rgb}{0.000000,0.000000,0.000000}%
\pgfsetfillcolor{currentfill}%
\pgfsetlinewidth{0.803000pt}%
\definecolor{currentstroke}{rgb}{0.000000,0.000000,0.000000}%
\pgfsetstrokecolor{currentstroke}%
\pgfsetdash{}{0pt}%
\pgfsys@defobject{currentmarker}{\pgfqpoint{0.000000in}{-0.048611in}}{\pgfqpoint{0.000000in}{0.000000in}}{%
\pgfpathmoveto{\pgfqpoint{0.000000in}{0.000000in}}%
\pgfpathlineto{\pgfqpoint{0.000000in}{-0.048611in}}%
\pgfusepath{stroke,fill}%
}%
\begin{pgfscope}%
\pgfsys@transformshift{1.925984in}{0.528000in}%
\pgfsys@useobject{currentmarker}{}%
\end{pgfscope}%
\end{pgfscope}%
\begin{pgfscope}%
\pgftext[x=1.925984in,y=0.430778in,,top]{\rmfamily\fontsize{10.000000}{12.000000}\selectfont \(\displaystyle 0.2\)}%
\end{pgfscope}%
\begin{pgfscope}%
\pgfsetbuttcap%
\pgfsetroundjoin%
\definecolor{currentfill}{rgb}{0.000000,0.000000,0.000000}%
\pgfsetfillcolor{currentfill}%
\pgfsetlinewidth{0.803000pt}%
\definecolor{currentstroke}{rgb}{0.000000,0.000000,0.000000}%
\pgfsetstrokecolor{currentstroke}%
\pgfsetdash{}{0pt}%
\pgfsys@defobject{currentmarker}{\pgfqpoint{0.000000in}{-0.048611in}}{\pgfqpoint{0.000000in}{0.000000in}}{%
\pgfpathmoveto{\pgfqpoint{0.000000in}{0.000000in}}%
\pgfpathlineto{\pgfqpoint{0.000000in}{-0.048611in}}%
\pgfusepath{stroke,fill}%
}%
\begin{pgfscope}%
\pgfsys@transformshift{2.595542in}{0.528000in}%
\pgfsys@useobject{currentmarker}{}%
\end{pgfscope}%
\end{pgfscope}%
\begin{pgfscope}%
\pgftext[x=2.595542in,y=0.430778in,,top]{\rmfamily\fontsize{10.000000}{12.000000}\selectfont \(\displaystyle 0.4\)}%
\end{pgfscope}%
\begin{pgfscope}%
\pgfsetbuttcap%
\pgfsetroundjoin%
\definecolor{currentfill}{rgb}{0.000000,0.000000,0.000000}%
\pgfsetfillcolor{currentfill}%
\pgfsetlinewidth{0.803000pt}%
\definecolor{currentstroke}{rgb}{0.000000,0.000000,0.000000}%
\pgfsetstrokecolor{currentstroke}%
\pgfsetdash{}{0pt}%
\pgfsys@defobject{currentmarker}{\pgfqpoint{0.000000in}{-0.048611in}}{\pgfqpoint{0.000000in}{0.000000in}}{%
\pgfpathmoveto{\pgfqpoint{0.000000in}{0.000000in}}%
\pgfpathlineto{\pgfqpoint{0.000000in}{-0.048611in}}%
\pgfusepath{stroke,fill}%
}%
\begin{pgfscope}%
\pgfsys@transformshift{3.265100in}{0.528000in}%
\pgfsys@useobject{currentmarker}{}%
\end{pgfscope}%
\end{pgfscope}%
\begin{pgfscope}%
\pgftext[x=3.265100in,y=0.430778in,,top]{\rmfamily\fontsize{10.000000}{12.000000}\selectfont \(\displaystyle 0.6\)}%
\end{pgfscope}%
\begin{pgfscope}%
\pgfsetbuttcap%
\pgfsetroundjoin%
\definecolor{currentfill}{rgb}{0.000000,0.000000,0.000000}%
\pgfsetfillcolor{currentfill}%
\pgfsetlinewidth{0.803000pt}%
\definecolor{currentstroke}{rgb}{0.000000,0.000000,0.000000}%
\pgfsetstrokecolor{currentstroke}%
\pgfsetdash{}{0pt}%
\pgfsys@defobject{currentmarker}{\pgfqpoint{0.000000in}{-0.048611in}}{\pgfqpoint{0.000000in}{0.000000in}}{%
\pgfpathmoveto{\pgfqpoint{0.000000in}{0.000000in}}%
\pgfpathlineto{\pgfqpoint{0.000000in}{-0.048611in}}%
\pgfusepath{stroke,fill}%
}%
\begin{pgfscope}%
\pgfsys@transformshift{3.934658in}{0.528000in}%
\pgfsys@useobject{currentmarker}{}%
\end{pgfscope}%
\end{pgfscope}%
\begin{pgfscope}%
\pgftext[x=3.934658in,y=0.430778in,,top]{\rmfamily\fontsize{10.000000}{12.000000}\selectfont \(\displaystyle 0.8\)}%
\end{pgfscope}%
\begin{pgfscope}%
\pgfsetbuttcap%
\pgfsetroundjoin%
\definecolor{currentfill}{rgb}{0.000000,0.000000,0.000000}%
\pgfsetfillcolor{currentfill}%
\pgfsetlinewidth{0.803000pt}%
\definecolor{currentstroke}{rgb}{0.000000,0.000000,0.000000}%
\pgfsetstrokecolor{currentstroke}%
\pgfsetdash{}{0pt}%
\pgfsys@defobject{currentmarker}{\pgfqpoint{0.000000in}{-0.048611in}}{\pgfqpoint{0.000000in}{0.000000in}}{%
\pgfpathmoveto{\pgfqpoint{0.000000in}{0.000000in}}%
\pgfpathlineto{\pgfqpoint{0.000000in}{-0.048611in}}%
\pgfusepath{stroke,fill}%
}%
\begin{pgfscope}%
\pgfsys@transformshift{4.604216in}{0.528000in}%
\pgfsys@useobject{currentmarker}{}%
\end{pgfscope}%
\end{pgfscope}%
\begin{pgfscope}%
\pgftext[x=4.604216in,y=0.430778in,,top]{\rmfamily\fontsize{10.000000}{12.000000}\selectfont \(\displaystyle 1.0\)}%
\end{pgfscope}%
\begin{pgfscope}%
\pgfsetbuttcap%
\pgfsetroundjoin%
\definecolor{currentfill}{rgb}{0.000000,0.000000,0.000000}%
\pgfsetfillcolor{currentfill}%
\pgfsetlinewidth{0.803000pt}%
\definecolor{currentstroke}{rgb}{0.000000,0.000000,0.000000}%
\pgfsetstrokecolor{currentstroke}%
\pgfsetdash{}{0pt}%
\pgfsys@defobject{currentmarker}{\pgfqpoint{-0.048611in}{0.000000in}}{\pgfqpoint{0.000000in}{0.000000in}}{%
\pgfpathmoveto{\pgfqpoint{0.000000in}{0.000000in}}%
\pgfpathlineto{\pgfqpoint{-0.048611in}{0.000000in}}%
\pgfusepath{stroke,fill}%
}%
\begin{pgfscope}%
\pgfsys@transformshift{1.065196in}{0.687760in}%
\pgfsys@useobject{currentmarker}{}%
\end{pgfscope}%
\end{pgfscope}%
\begin{pgfscope}%
\pgftext[x=0.790504in,y=0.639566in,left,base]{\rmfamily\fontsize{10.000000}{12.000000}\selectfont \(\displaystyle 0.0\)}%
\end{pgfscope}%
\begin{pgfscope}%
\pgfsetbuttcap%
\pgfsetroundjoin%
\definecolor{currentfill}{rgb}{0.000000,0.000000,0.000000}%
\pgfsetfillcolor{currentfill}%
\pgfsetlinewidth{0.803000pt}%
\definecolor{currentstroke}{rgb}{0.000000,0.000000,0.000000}%
\pgfsetstrokecolor{currentstroke}%
\pgfsetdash{}{0pt}%
\pgfsys@defobject{currentmarker}{\pgfqpoint{-0.048611in}{0.000000in}}{\pgfqpoint{0.000000in}{0.000000in}}{%
\pgfpathmoveto{\pgfqpoint{0.000000in}{0.000000in}}%
\pgfpathlineto{\pgfqpoint{-0.048611in}{0.000000in}}%
\pgfusepath{stroke,fill}%
}%
\begin{pgfscope}%
\pgfsys@transformshift{1.065196in}{1.357318in}%
\pgfsys@useobject{currentmarker}{}%
\end{pgfscope}%
\end{pgfscope}%
\begin{pgfscope}%
\pgftext[x=0.790504in,y=1.309124in,left,base]{\rmfamily\fontsize{10.000000}{12.000000}\selectfont \(\displaystyle 0.2\)}%
\end{pgfscope}%
\begin{pgfscope}%
\pgfsetbuttcap%
\pgfsetroundjoin%
\definecolor{currentfill}{rgb}{0.000000,0.000000,0.000000}%
\pgfsetfillcolor{currentfill}%
\pgfsetlinewidth{0.803000pt}%
\definecolor{currentstroke}{rgb}{0.000000,0.000000,0.000000}%
\pgfsetstrokecolor{currentstroke}%
\pgfsetdash{}{0pt}%
\pgfsys@defobject{currentmarker}{\pgfqpoint{-0.048611in}{0.000000in}}{\pgfqpoint{0.000000in}{0.000000in}}{%
\pgfpathmoveto{\pgfqpoint{0.000000in}{0.000000in}}%
\pgfpathlineto{\pgfqpoint{-0.048611in}{0.000000in}}%
\pgfusepath{stroke,fill}%
}%
\begin{pgfscope}%
\pgfsys@transformshift{1.065196in}{2.026877in}%
\pgfsys@useobject{currentmarker}{}%
\end{pgfscope}%
\end{pgfscope}%
\begin{pgfscope}%
\pgftext[x=0.790504in,y=1.978682in,left,base]{\rmfamily\fontsize{10.000000}{12.000000}\selectfont \(\displaystyle 0.4\)}%
\end{pgfscope}%
\begin{pgfscope}%
\pgfsetbuttcap%
\pgfsetroundjoin%
\definecolor{currentfill}{rgb}{0.000000,0.000000,0.000000}%
\pgfsetfillcolor{currentfill}%
\pgfsetlinewidth{0.803000pt}%
\definecolor{currentstroke}{rgb}{0.000000,0.000000,0.000000}%
\pgfsetstrokecolor{currentstroke}%
\pgfsetdash{}{0pt}%
\pgfsys@defobject{currentmarker}{\pgfqpoint{-0.048611in}{0.000000in}}{\pgfqpoint{0.000000in}{0.000000in}}{%
\pgfpathmoveto{\pgfqpoint{0.000000in}{0.000000in}}%
\pgfpathlineto{\pgfqpoint{-0.048611in}{0.000000in}}%
\pgfusepath{stroke,fill}%
}%
\begin{pgfscope}%
\pgfsys@transformshift{1.065196in}{2.696435in}%
\pgfsys@useobject{currentmarker}{}%
\end{pgfscope}%
\end{pgfscope}%
\begin{pgfscope}%
\pgftext[x=0.790504in,y=2.648240in,left,base]{\rmfamily\fontsize{10.000000}{12.000000}\selectfont \(\displaystyle 0.6\)}%
\end{pgfscope}%
\begin{pgfscope}%
\pgfsetbuttcap%
\pgfsetroundjoin%
\definecolor{currentfill}{rgb}{0.000000,0.000000,0.000000}%
\pgfsetfillcolor{currentfill}%
\pgfsetlinewidth{0.803000pt}%
\definecolor{currentstroke}{rgb}{0.000000,0.000000,0.000000}%
\pgfsetstrokecolor{currentstroke}%
\pgfsetdash{}{0pt}%
\pgfsys@defobject{currentmarker}{\pgfqpoint{-0.048611in}{0.000000in}}{\pgfqpoint{0.000000in}{0.000000in}}{%
\pgfpathmoveto{\pgfqpoint{0.000000in}{0.000000in}}%
\pgfpathlineto{\pgfqpoint{-0.048611in}{0.000000in}}%
\pgfusepath{stroke,fill}%
}%
\begin{pgfscope}%
\pgfsys@transformshift{1.065196in}{3.365993in}%
\pgfsys@useobject{currentmarker}{}%
\end{pgfscope}%
\end{pgfscope}%
\begin{pgfscope}%
\pgftext[x=0.790504in,y=3.317798in,left,base]{\rmfamily\fontsize{10.000000}{12.000000}\selectfont \(\displaystyle 0.8\)}%
\end{pgfscope}%
\begin{pgfscope}%
\pgfsetbuttcap%
\pgfsetroundjoin%
\definecolor{currentfill}{rgb}{0.000000,0.000000,0.000000}%
\pgfsetfillcolor{currentfill}%
\pgfsetlinewidth{0.803000pt}%
\definecolor{currentstroke}{rgb}{0.000000,0.000000,0.000000}%
\pgfsetstrokecolor{currentstroke}%
\pgfsetdash{}{0pt}%
\pgfsys@defobject{currentmarker}{\pgfqpoint{-0.048611in}{0.000000in}}{\pgfqpoint{0.000000in}{0.000000in}}{%
\pgfpathmoveto{\pgfqpoint{0.000000in}{0.000000in}}%
\pgfpathlineto{\pgfqpoint{-0.048611in}{0.000000in}}%
\pgfusepath{stroke,fill}%
}%
\begin{pgfscope}%
\pgfsys@transformshift{1.065196in}{4.035551in}%
\pgfsys@useobject{currentmarker}{}%
\end{pgfscope}%
\end{pgfscope}%
\begin{pgfscope}%
\pgftext[x=0.790504in,y=3.987356in,left,base]{\rmfamily\fontsize{10.000000}{12.000000}\selectfont \(\displaystyle 1.0\)}%
\end{pgfscope}%
\begin{pgfscope}%
\pgfsetrectcap%
\pgfsetmiterjoin%
\pgfsetlinewidth{0.803000pt}%
\definecolor{currentstroke}{rgb}{0.000000,0.000000,0.000000}%
\pgfsetstrokecolor{currentstroke}%
\pgfsetdash{}{0pt}%
\pgfpathmoveto{\pgfqpoint{1.065196in}{0.528000in}}%
\pgfpathlineto{\pgfqpoint{1.065196in}{4.224000in}}%
\pgfusepath{stroke}%
\end{pgfscope}%
\begin{pgfscope}%
\pgfsetrectcap%
\pgfsetmiterjoin%
\pgfsetlinewidth{0.803000pt}%
\definecolor{currentstroke}{rgb}{0.000000,0.000000,0.000000}%
\pgfsetstrokecolor{currentstroke}%
\pgfsetdash{}{0pt}%
\pgfpathmoveto{\pgfqpoint{4.768000in}{0.528000in}}%
\pgfpathlineto{\pgfqpoint{4.768000in}{4.224000in}}%
\pgfusepath{stroke}%
\end{pgfscope}%
\begin{pgfscope}%
\pgfsetrectcap%
\pgfsetmiterjoin%
\pgfsetlinewidth{0.803000pt}%
\definecolor{currentstroke}{rgb}{0.000000,0.000000,0.000000}%
\pgfsetstrokecolor{currentstroke}%
\pgfsetdash{}{0pt}%
\pgfpathmoveto{\pgfqpoint{1.065196in}{0.528000in}}%
\pgfpathlineto{\pgfqpoint{4.768000in}{0.528000in}}%
\pgfusepath{stroke}%
\end{pgfscope}%
\begin{pgfscope}%
\pgfsetrectcap%
\pgfsetmiterjoin%
\pgfsetlinewidth{0.803000pt}%
\definecolor{currentstroke}{rgb}{0.000000,0.000000,0.000000}%
\pgfsetstrokecolor{currentstroke}%
\pgfsetdash{}{0pt}%
\pgfpathmoveto{\pgfqpoint{1.065196in}{4.224000in}}%
\pgfpathlineto{\pgfqpoint{4.768000in}{4.224000in}}%
\pgfusepath{stroke}%
\end{pgfscope}%
\begin{pgfscope}%
\pgfpathrectangle{\pgfqpoint{5.016000in}{0.528000in}}{\pgfqpoint{0.184800in}{3.696000in}} %
\pgfusepath{clip}%
\pgfsetbuttcap%
\pgfsetmiterjoin%
\definecolor{currentfill}{rgb}{1.000000,1.000000,1.000000}%
\pgfsetfillcolor{currentfill}%
\pgfsetlinewidth{0.010037pt}%
\definecolor{currentstroke}{rgb}{1.000000,1.000000,1.000000}%
\pgfsetstrokecolor{currentstroke}%
\pgfsetdash{}{0pt}%
\pgfpathmoveto{\pgfqpoint{5.016000in}{0.528000in}}%
\pgfpathlineto{\pgfqpoint{5.016000in}{0.542438in}}%
\pgfpathlineto{\pgfqpoint{5.016000in}{4.209562in}}%
\pgfpathlineto{\pgfqpoint{5.016000in}{4.224000in}}%
\pgfpathlineto{\pgfqpoint{5.200800in}{4.224000in}}%
\pgfpathlineto{\pgfqpoint{5.200800in}{4.209562in}}%
\pgfpathlineto{\pgfqpoint{5.200800in}{0.542438in}}%
\pgfpathlineto{\pgfqpoint{5.200800in}{0.528000in}}%
\pgfpathclose%
\pgfusepath{stroke,fill}%
\end{pgfscope}%
\begin{pgfscope}%
\pgfsys@transformshift{5.020000in}{0.530000in}%
\pgftext[left,bottom]{\pgfimage[interpolate=true,width=0.180000in,height=3.690000in]{Figure-0002-20180109-004133-417755-img0.png}}%
\end{pgfscope}%
\begin{pgfscope}%
\pgfsetbuttcap%
\pgfsetroundjoin%
\definecolor{currentfill}{rgb}{0.000000,0.000000,0.000000}%
\pgfsetfillcolor{currentfill}%
\pgfsetlinewidth{0.803000pt}%
\definecolor{currentstroke}{rgb}{0.000000,0.000000,0.000000}%
\pgfsetstrokecolor{currentstroke}%
\pgfsetdash{}{0pt}%
\pgfsys@defobject{currentmarker}{\pgfqpoint{0.000000in}{0.000000in}}{\pgfqpoint{0.048611in}{0.000000in}}{%
\pgfpathmoveto{\pgfqpoint{0.000000in}{0.000000in}}%
\pgfpathlineto{\pgfqpoint{0.048611in}{0.000000in}}%
\pgfusepath{stroke,fill}%
}%
\begin{pgfscope}%
\pgfsys@transformshift{5.200800in}{0.972600in}%
\pgfsys@useobject{currentmarker}{}%
\end{pgfscope}%
\end{pgfscope}%
\begin{pgfscope}%
\pgftext[x=5.298022in,y=0.924406in,left,base]{\rmfamily\fontsize{10.000000}{12.000000}\selectfont \(\displaystyle 0.004\)}%
\end{pgfscope}%
\begin{pgfscope}%
\pgfsetbuttcap%
\pgfsetroundjoin%
\definecolor{currentfill}{rgb}{0.000000,0.000000,0.000000}%
\pgfsetfillcolor{currentfill}%
\pgfsetlinewidth{0.803000pt}%
\definecolor{currentstroke}{rgb}{0.000000,0.000000,0.000000}%
\pgfsetstrokecolor{currentstroke}%
\pgfsetdash{}{0pt}%
\pgfsys@defobject{currentmarker}{\pgfqpoint{0.000000in}{0.000000in}}{\pgfqpoint{0.048611in}{0.000000in}}{%
\pgfpathmoveto{\pgfqpoint{0.000000in}{0.000000in}}%
\pgfpathlineto{\pgfqpoint{0.048611in}{0.000000in}}%
\pgfusepath{stroke,fill}%
}%
\begin{pgfscope}%
\pgfsys@transformshift{5.200800in}{1.419197in}%
\pgfsys@useobject{currentmarker}{}%
\end{pgfscope}%
\end{pgfscope}%
\begin{pgfscope}%
\pgftext[x=5.298022in,y=1.371003in,left,base]{\rmfamily\fontsize{10.000000}{12.000000}\selectfont \(\displaystyle 0.006\)}%
\end{pgfscope}%
\begin{pgfscope}%
\pgfsetbuttcap%
\pgfsetroundjoin%
\definecolor{currentfill}{rgb}{0.000000,0.000000,0.000000}%
\pgfsetfillcolor{currentfill}%
\pgfsetlinewidth{0.803000pt}%
\definecolor{currentstroke}{rgb}{0.000000,0.000000,0.000000}%
\pgfsetstrokecolor{currentstroke}%
\pgfsetdash{}{0pt}%
\pgfsys@defobject{currentmarker}{\pgfqpoint{0.000000in}{0.000000in}}{\pgfqpoint{0.048611in}{0.000000in}}{%
\pgfpathmoveto{\pgfqpoint{0.000000in}{0.000000in}}%
\pgfpathlineto{\pgfqpoint{0.048611in}{0.000000in}}%
\pgfusepath{stroke,fill}%
}%
\begin{pgfscope}%
\pgfsys@transformshift{5.200800in}{1.865795in}%
\pgfsys@useobject{currentmarker}{}%
\end{pgfscope}%
\end{pgfscope}%
\begin{pgfscope}%
\pgftext[x=5.298022in,y=1.817600in,left,base]{\rmfamily\fontsize{10.000000}{12.000000}\selectfont \(\displaystyle 0.008\)}%
\end{pgfscope}%
\begin{pgfscope}%
\pgfsetbuttcap%
\pgfsetroundjoin%
\definecolor{currentfill}{rgb}{0.000000,0.000000,0.000000}%
\pgfsetfillcolor{currentfill}%
\pgfsetlinewidth{0.803000pt}%
\definecolor{currentstroke}{rgb}{0.000000,0.000000,0.000000}%
\pgfsetstrokecolor{currentstroke}%
\pgfsetdash{}{0pt}%
\pgfsys@defobject{currentmarker}{\pgfqpoint{0.000000in}{0.000000in}}{\pgfqpoint{0.048611in}{0.000000in}}{%
\pgfpathmoveto{\pgfqpoint{0.000000in}{0.000000in}}%
\pgfpathlineto{\pgfqpoint{0.048611in}{0.000000in}}%
\pgfusepath{stroke,fill}%
}%
\begin{pgfscope}%
\pgfsys@transformshift{5.200800in}{2.312392in}%
\pgfsys@useobject{currentmarker}{}%
\end{pgfscope}%
\end{pgfscope}%
\begin{pgfscope}%
\pgftext[x=5.298022in,y=2.264198in,left,base]{\rmfamily\fontsize{10.000000}{12.000000}\selectfont \(\displaystyle 0.010\)}%
\end{pgfscope}%
\begin{pgfscope}%
\pgfsetbuttcap%
\pgfsetroundjoin%
\definecolor{currentfill}{rgb}{0.000000,0.000000,0.000000}%
\pgfsetfillcolor{currentfill}%
\pgfsetlinewidth{0.803000pt}%
\definecolor{currentstroke}{rgb}{0.000000,0.000000,0.000000}%
\pgfsetstrokecolor{currentstroke}%
\pgfsetdash{}{0pt}%
\pgfsys@defobject{currentmarker}{\pgfqpoint{0.000000in}{0.000000in}}{\pgfqpoint{0.048611in}{0.000000in}}{%
\pgfpathmoveto{\pgfqpoint{0.000000in}{0.000000in}}%
\pgfpathlineto{\pgfqpoint{0.048611in}{0.000000in}}%
\pgfusepath{stroke,fill}%
}%
\begin{pgfscope}%
\pgfsys@transformshift{5.200800in}{2.758990in}%
\pgfsys@useobject{currentmarker}{}%
\end{pgfscope}%
\end{pgfscope}%
\begin{pgfscope}%
\pgftext[x=5.298022in,y=2.710795in,left,base]{\rmfamily\fontsize{10.000000}{12.000000}\selectfont \(\displaystyle 0.012\)}%
\end{pgfscope}%
\begin{pgfscope}%
\pgfsetbuttcap%
\pgfsetroundjoin%
\definecolor{currentfill}{rgb}{0.000000,0.000000,0.000000}%
\pgfsetfillcolor{currentfill}%
\pgfsetlinewidth{0.803000pt}%
\definecolor{currentstroke}{rgb}{0.000000,0.000000,0.000000}%
\pgfsetstrokecolor{currentstroke}%
\pgfsetdash{}{0pt}%
\pgfsys@defobject{currentmarker}{\pgfqpoint{0.000000in}{0.000000in}}{\pgfqpoint{0.048611in}{0.000000in}}{%
\pgfpathmoveto{\pgfqpoint{0.000000in}{0.000000in}}%
\pgfpathlineto{\pgfqpoint{0.048611in}{0.000000in}}%
\pgfusepath{stroke,fill}%
}%
\begin{pgfscope}%
\pgfsys@transformshift{5.200800in}{3.205587in}%
\pgfsys@useobject{currentmarker}{}%
\end{pgfscope}%
\end{pgfscope}%
\begin{pgfscope}%
\pgftext[x=5.298022in,y=3.157393in,left,base]{\rmfamily\fontsize{10.000000}{12.000000}\selectfont \(\displaystyle 0.014\)}%
\end{pgfscope}%
\begin{pgfscope}%
\pgfsetbuttcap%
\pgfsetroundjoin%
\definecolor{currentfill}{rgb}{0.000000,0.000000,0.000000}%
\pgfsetfillcolor{currentfill}%
\pgfsetlinewidth{0.803000pt}%
\definecolor{currentstroke}{rgb}{0.000000,0.000000,0.000000}%
\pgfsetstrokecolor{currentstroke}%
\pgfsetdash{}{0pt}%
\pgfsys@defobject{currentmarker}{\pgfqpoint{0.000000in}{0.000000in}}{\pgfqpoint{0.048611in}{0.000000in}}{%
\pgfpathmoveto{\pgfqpoint{0.000000in}{0.000000in}}%
\pgfpathlineto{\pgfqpoint{0.048611in}{0.000000in}}%
\pgfusepath{stroke,fill}%
}%
\begin{pgfscope}%
\pgfsys@transformshift{5.200800in}{3.652185in}%
\pgfsys@useobject{currentmarker}{}%
\end{pgfscope}%
\end{pgfscope}%
\begin{pgfscope}%
\pgftext[x=5.298022in,y=3.603990in,left,base]{\rmfamily\fontsize{10.000000}{12.000000}\selectfont \(\displaystyle 0.016\)}%
\end{pgfscope}%
\begin{pgfscope}%
\pgfsetbuttcap%
\pgfsetroundjoin%
\definecolor{currentfill}{rgb}{0.000000,0.000000,0.000000}%
\pgfsetfillcolor{currentfill}%
\pgfsetlinewidth{0.803000pt}%
\definecolor{currentstroke}{rgb}{0.000000,0.000000,0.000000}%
\pgfsetstrokecolor{currentstroke}%
\pgfsetdash{}{0pt}%
\pgfsys@defobject{currentmarker}{\pgfqpoint{0.000000in}{0.000000in}}{\pgfqpoint{0.048611in}{0.000000in}}{%
\pgfpathmoveto{\pgfqpoint{0.000000in}{0.000000in}}%
\pgfpathlineto{\pgfqpoint{0.048611in}{0.000000in}}%
\pgfusepath{stroke,fill}%
}%
\begin{pgfscope}%
\pgfsys@transformshift{5.200800in}{4.098782in}%
\pgfsys@useobject{currentmarker}{}%
\end{pgfscope}%
\end{pgfscope}%
\begin{pgfscope}%
\pgftext[x=5.298022in,y=4.050588in,left,base]{\rmfamily\fontsize{10.000000}{12.000000}\selectfont \(\displaystyle 0.018\)}%
\end{pgfscope}%
\begin{pgfscope}%
\pgfsetbuttcap%
\pgfsetmiterjoin%
\pgfsetlinewidth{0.803000pt}%
\definecolor{currentstroke}{rgb}{0.000000,0.000000,0.000000}%
\pgfsetstrokecolor{currentstroke}%
\pgfsetdash{}{0pt}%
\pgfpathmoveto{\pgfqpoint{5.016000in}{0.528000in}}%
\pgfpathlineto{\pgfqpoint{5.016000in}{0.542438in}}%
\pgfpathlineto{\pgfqpoint{5.016000in}{4.209562in}}%
\pgfpathlineto{\pgfqpoint{5.016000in}{4.224000in}}%
\pgfpathlineto{\pgfqpoint{5.200800in}{4.224000in}}%
\pgfpathlineto{\pgfqpoint{5.200800in}{4.209562in}}%
\pgfpathlineto{\pgfqpoint{5.200800in}{0.542438in}}%
\pgfpathlineto{\pgfqpoint{5.200800in}{0.528000in}}%
\pgfpathclose%
\pgfusepath{stroke}%
\end{pgfscope}%
\end{pgfpicture}%
\makeatother%
\endgroup%
}
\caption{An transportation example on randomly generated dataset} \label{Fig:Random}
\end{figure}

The ellipses and Caffarelli's smoothness counter examples (Caffarelli) are generated in the nearly indentical protocol described in \parencite{Gerber2017}. A slight modification is made that the first ellipses is scaled by $ 2 \times 0.5 $ and the second $ 0.5 \times 2 $, and the two half-disks are shifted by $1$ instead of $2$. Figure shows an example. Note that the size of Caffarelli datasets is the number of originally sampled points, and therefore a dataset of size $m$ actually contains $ \spi m / 4 $ on average. Figure \ref{Fig:Ellipse} and \ref{Fig:Caff} provides examples for these two datasets.

\begin{figure}
\centering \scalebox{0.65}{%% Creator: Matplotlib, PGF backend
%%
%% To include the figure in your LaTeX document, write
%%   \input{<filename>.pgf}
%%
%% Make sure the required packages are loaded in your preamble
%%   \usepackage{pgf}
%%
%% Figures using additional raster images can only be included by \input if
%% they are in the same directory as the main LaTeX file. For loading figures
%% from other directories you can use the `import` package
%%   \usepackage{import}
%% and then include the figures with
%%   \import{<path to file>}{<filename>.pgf}
%%
%% Matplotlib used the following preamble
%%   \usepackage{fontspec}
%%
\begingroup%
\makeatletter%
\begin{pgfpicture}%
\pgfpathrectangle{\pgfpointorigin}{\pgfqpoint{6.400000in}{4.800000in}}%
\pgfusepath{use as bounding box, clip}%
\begin{pgfscope}%
\pgfsetbuttcap%
\pgfsetmiterjoin%
\definecolor{currentfill}{rgb}{1.000000,1.000000,1.000000}%
\pgfsetfillcolor{currentfill}%
\pgfsetlinewidth{0.000000pt}%
\definecolor{currentstroke}{rgb}{1.000000,1.000000,1.000000}%
\pgfsetstrokecolor{currentstroke}%
\pgfsetdash{}{0pt}%
\pgfpathmoveto{\pgfqpoint{0.000000in}{0.000000in}}%
\pgfpathlineto{\pgfqpoint{6.400000in}{0.000000in}}%
\pgfpathlineto{\pgfqpoint{6.400000in}{4.800000in}}%
\pgfpathlineto{\pgfqpoint{0.000000in}{4.800000in}}%
\pgfpathclose%
\pgfusepath{fill}%
\end{pgfscope}%
\begin{pgfscope}%
\pgfsetbuttcap%
\pgfsetmiterjoin%
\definecolor{currentfill}{rgb}{1.000000,1.000000,1.000000}%
\pgfsetfillcolor{currentfill}%
\pgfsetlinewidth{0.000000pt}%
\definecolor{currentstroke}{rgb}{0.000000,0.000000,0.000000}%
\pgfsetstrokecolor{currentstroke}%
\pgfsetstrokeopacity{0.000000}%
\pgfsetdash{}{0pt}%
\pgfpathmoveto{\pgfqpoint{0.800000in}{1.363959in}}%
\pgfpathlineto{\pgfqpoint{4.768000in}{1.363959in}}%
\pgfpathlineto{\pgfqpoint{4.768000in}{3.388041in}}%
\pgfpathlineto{\pgfqpoint{0.800000in}{3.388041in}}%
\pgfpathclose%
\pgfusepath{fill}%
\end{pgfscope}%
\begin{pgfscope}%
\pgfpathrectangle{\pgfqpoint{0.800000in}{1.363959in}}{\pgfqpoint{3.968000in}{2.024082in}} %
\pgfusepath{clip}%
\pgfsetbuttcap%
\pgfsetroundjoin%
\definecolor{currentfill}{rgb}{0.121569,0.466667,0.705882}%
\pgfsetfillcolor{currentfill}%
\pgfsetlinewidth{1.003750pt}%
\definecolor{currentstroke}{rgb}{0.121569,0.466667,0.705882}%
\pgfsetstrokecolor{currentstroke}%
\pgfsetdash{}{0pt}%
\pgfpathmoveto{\pgfqpoint{2.574489in}{1.433905in}}%
\pgfpathcurveto{\pgfqpoint{2.584577in}{1.433905in}}{\pgfqpoint{2.594252in}{1.437913in}}{\pgfqpoint{2.601385in}{1.445046in}}%
\pgfpathcurveto{\pgfqpoint{2.608518in}{1.452178in}}{\pgfqpoint{2.612526in}{1.461854in}}{\pgfqpoint{2.612526in}{1.471941in}}%
\pgfpathcurveto{\pgfqpoint{2.612526in}{1.482029in}}{\pgfqpoint{2.608518in}{1.491704in}}{\pgfqpoint{2.601385in}{1.498837in}}%
\pgfpathcurveto{\pgfqpoint{2.594252in}{1.505970in}}{\pgfqpoint{2.584577in}{1.509978in}}{\pgfqpoint{2.574489in}{1.509978in}}%
\pgfpathcurveto{\pgfqpoint{2.564402in}{1.509978in}}{\pgfqpoint{2.554726in}{1.505970in}}{\pgfqpoint{2.547594in}{1.498837in}}%
\pgfpathcurveto{\pgfqpoint{2.540461in}{1.491704in}}{\pgfqpoint{2.536453in}{1.482029in}}{\pgfqpoint{2.536453in}{1.471941in}}%
\pgfpathcurveto{\pgfqpoint{2.536453in}{1.461854in}}{\pgfqpoint{2.540461in}{1.452178in}}{\pgfqpoint{2.547594in}{1.445046in}}%
\pgfpathcurveto{\pgfqpoint{2.554726in}{1.437913in}}{\pgfqpoint{2.564402in}{1.433905in}}{\pgfqpoint{2.574489in}{1.433905in}}%
\pgfpathclose%
\pgfusepath{stroke,fill}%
\end{pgfscope}%
\begin{pgfscope}%
\pgfpathrectangle{\pgfqpoint{0.800000in}{1.363959in}}{\pgfqpoint{3.968000in}{2.024082in}} %
\pgfusepath{clip}%
\pgfsetbuttcap%
\pgfsetroundjoin%
\definecolor{currentfill}{rgb}{0.121569,0.466667,0.705882}%
\pgfsetfillcolor{currentfill}%
\pgfsetlinewidth{1.003750pt}%
\definecolor{currentstroke}{rgb}{0.121569,0.466667,0.705882}%
\pgfsetstrokecolor{currentstroke}%
\pgfsetdash{}{0pt}%
\pgfpathmoveto{\pgfqpoint{2.360556in}{1.856086in}}%
\pgfpathcurveto{\pgfqpoint{2.370644in}{1.856086in}}{\pgfqpoint{2.380319in}{1.860094in}}{\pgfqpoint{2.387452in}{1.867226in}}%
\pgfpathcurveto{\pgfqpoint{2.394585in}{1.874359in}}{\pgfqpoint{2.398593in}{1.884035in}}{\pgfqpoint{2.398593in}{1.894122in}}%
\pgfpathcurveto{\pgfqpoint{2.398593in}{1.904209in}}{\pgfqpoint{2.394585in}{1.913885in}}{\pgfqpoint{2.387452in}{1.921018in}}%
\pgfpathcurveto{\pgfqpoint{2.380319in}{1.928151in}}{\pgfqpoint{2.370644in}{1.932158in}}{\pgfqpoint{2.360556in}{1.932158in}}%
\pgfpathcurveto{\pgfqpoint{2.350469in}{1.932158in}}{\pgfqpoint{2.340793in}{1.928151in}}{\pgfqpoint{2.333661in}{1.921018in}}%
\pgfpathcurveto{\pgfqpoint{2.326528in}{1.913885in}}{\pgfqpoint{2.322520in}{1.904209in}}{\pgfqpoint{2.322520in}{1.894122in}}%
\pgfpathcurveto{\pgfqpoint{2.322520in}{1.884035in}}{\pgfqpoint{2.326528in}{1.874359in}}{\pgfqpoint{2.333661in}{1.867226in}}%
\pgfpathcurveto{\pgfqpoint{2.340793in}{1.860094in}}{\pgfqpoint{2.350469in}{1.856086in}}{\pgfqpoint{2.360556in}{1.856086in}}%
\pgfpathclose%
\pgfusepath{stroke,fill}%
\end{pgfscope}%
\begin{pgfscope}%
\pgfpathrectangle{\pgfqpoint{0.800000in}{1.363959in}}{\pgfqpoint{3.968000in}{2.024082in}} %
\pgfusepath{clip}%
\pgfsetbuttcap%
\pgfsetroundjoin%
\definecolor{currentfill}{rgb}{0.121569,0.466667,0.705882}%
\pgfsetfillcolor{currentfill}%
\pgfsetlinewidth{1.003750pt}%
\definecolor{currentstroke}{rgb}{0.121569,0.466667,0.705882}%
\pgfsetstrokecolor{currentstroke}%
\pgfsetdash{}{0pt}%
\pgfpathmoveto{\pgfqpoint{1.968849in}{2.401917in}}%
\pgfpathcurveto{\pgfqpoint{1.978937in}{2.401917in}}{\pgfqpoint{1.988612in}{2.405925in}}{\pgfqpoint{1.995745in}{2.413058in}}%
\pgfpathcurveto{\pgfqpoint{2.002878in}{2.420191in}}{\pgfqpoint{2.006886in}{2.429866in}}{\pgfqpoint{2.006886in}{2.439954in}}%
\pgfpathcurveto{\pgfqpoint{2.006886in}{2.450041in}}{\pgfqpoint{2.002878in}{2.459716in}}{\pgfqpoint{1.995745in}{2.466849in}}%
\pgfpathcurveto{\pgfqpoint{1.988612in}{2.473982in}}{\pgfqpoint{1.978937in}{2.477990in}}{\pgfqpoint{1.968849in}{2.477990in}}%
\pgfpathcurveto{\pgfqpoint{1.958762in}{2.477990in}}{\pgfqpoint{1.949086in}{2.473982in}}{\pgfqpoint{1.941954in}{2.466849in}}%
\pgfpathcurveto{\pgfqpoint{1.934821in}{2.459716in}}{\pgfqpoint{1.930813in}{2.450041in}}{\pgfqpoint{1.930813in}{2.439954in}}%
\pgfpathcurveto{\pgfqpoint{1.930813in}{2.429866in}}{\pgfqpoint{1.934821in}{2.420191in}}{\pgfqpoint{1.941954in}{2.413058in}}%
\pgfpathcurveto{\pgfqpoint{1.949086in}{2.405925in}}{\pgfqpoint{1.958762in}{2.401917in}}{\pgfqpoint{1.968849in}{2.401917in}}%
\pgfpathclose%
\pgfusepath{stroke,fill}%
\end{pgfscope}%
\begin{pgfscope}%
\pgfpathrectangle{\pgfqpoint{0.800000in}{1.363959in}}{\pgfqpoint{3.968000in}{2.024082in}} %
\pgfusepath{clip}%
\pgfsetbuttcap%
\pgfsetroundjoin%
\definecolor{currentfill}{rgb}{0.121569,0.466667,0.705882}%
\pgfsetfillcolor{currentfill}%
\pgfsetlinewidth{1.003750pt}%
\definecolor{currentstroke}{rgb}{0.121569,0.466667,0.705882}%
\pgfsetstrokecolor{currentstroke}%
\pgfsetdash{}{0pt}%
\pgfpathmoveto{\pgfqpoint{2.144044in}{2.428015in}}%
\pgfpathcurveto{\pgfqpoint{2.154131in}{2.428015in}}{\pgfqpoint{2.163806in}{2.432023in}}{\pgfqpoint{2.170939in}{2.439155in}}%
\pgfpathcurveto{\pgfqpoint{2.178072in}{2.446288in}}{\pgfqpoint{2.182080in}{2.455964in}}{\pgfqpoint{2.182080in}{2.466051in}}%
\pgfpathcurveto{\pgfqpoint{2.182080in}{2.476138in}}{\pgfqpoint{2.178072in}{2.485814in}}{\pgfqpoint{2.170939in}{2.492947in}}%
\pgfpathcurveto{\pgfqpoint{2.163806in}{2.500080in}}{\pgfqpoint{2.154131in}{2.504087in}}{\pgfqpoint{2.144044in}{2.504087in}}%
\pgfpathcurveto{\pgfqpoint{2.133956in}{2.504087in}}{\pgfqpoint{2.124281in}{2.500080in}}{\pgfqpoint{2.117148in}{2.492947in}}%
\pgfpathcurveto{\pgfqpoint{2.110015in}{2.485814in}}{\pgfqpoint{2.106007in}{2.476138in}}{\pgfqpoint{2.106007in}{2.466051in}}%
\pgfpathcurveto{\pgfqpoint{2.106007in}{2.455964in}}{\pgfqpoint{2.110015in}{2.446288in}}{\pgfqpoint{2.117148in}{2.439155in}}%
\pgfpathcurveto{\pgfqpoint{2.124281in}{2.432023in}}{\pgfqpoint{2.133956in}{2.428015in}}{\pgfqpoint{2.144044in}{2.428015in}}%
\pgfpathclose%
\pgfusepath{stroke,fill}%
\end{pgfscope}%
\begin{pgfscope}%
\pgfpathrectangle{\pgfqpoint{0.800000in}{1.363959in}}{\pgfqpoint{3.968000in}{2.024082in}} %
\pgfusepath{clip}%
\pgfsetbuttcap%
\pgfsetroundjoin%
\definecolor{currentfill}{rgb}{0.121569,0.466667,0.705882}%
\pgfsetfillcolor{currentfill}%
\pgfsetlinewidth{1.003750pt}%
\definecolor{currentstroke}{rgb}{0.121569,0.466667,0.705882}%
\pgfsetstrokecolor{currentstroke}%
\pgfsetdash{}{0pt}%
\pgfpathmoveto{\pgfqpoint{2.441191in}{2.967484in}}%
\pgfpathcurveto{\pgfqpoint{2.451278in}{2.967484in}}{\pgfqpoint{2.460954in}{2.971491in}}{\pgfqpoint{2.468087in}{2.978624in}}%
\pgfpathcurveto{\pgfqpoint{2.475219in}{2.985757in}}{\pgfqpoint{2.479227in}{2.995432in}}{\pgfqpoint{2.479227in}{3.005520in}}%
\pgfpathcurveto{\pgfqpoint{2.479227in}{3.015607in}}{\pgfqpoint{2.475219in}{3.025283in}}{\pgfqpoint{2.468087in}{3.032416in}}%
\pgfpathcurveto{\pgfqpoint{2.460954in}{3.039548in}}{\pgfqpoint{2.451278in}{3.043556in}}{\pgfqpoint{2.441191in}{3.043556in}}%
\pgfpathcurveto{\pgfqpoint{2.431103in}{3.043556in}}{\pgfqpoint{2.421428in}{3.039548in}}{\pgfqpoint{2.414295in}{3.032416in}}%
\pgfpathcurveto{\pgfqpoint{2.407162in}{3.025283in}}{\pgfqpoint{2.403155in}{3.015607in}}{\pgfqpoint{2.403155in}{3.005520in}}%
\pgfpathcurveto{\pgfqpoint{2.403155in}{2.995432in}}{\pgfqpoint{2.407162in}{2.985757in}}{\pgfqpoint{2.414295in}{2.978624in}}%
\pgfpathcurveto{\pgfqpoint{2.421428in}{2.971491in}}{\pgfqpoint{2.431103in}{2.967484in}}{\pgfqpoint{2.441191in}{2.967484in}}%
\pgfpathclose%
\pgfusepath{stroke,fill}%
\end{pgfscope}%
\begin{pgfscope}%
\pgfpathrectangle{\pgfqpoint{0.800000in}{1.363959in}}{\pgfqpoint{3.968000in}{2.024082in}} %
\pgfusepath{clip}%
\pgfsetbuttcap%
\pgfsetroundjoin%
\definecolor{currentfill}{rgb}{0.121569,0.466667,0.705882}%
\pgfsetfillcolor{currentfill}%
\pgfsetlinewidth{1.003750pt}%
\definecolor{currentstroke}{rgb}{0.121569,0.466667,0.705882}%
\pgfsetstrokecolor{currentstroke}%
\pgfsetdash{}{0pt}%
\pgfpathmoveto{\pgfqpoint{2.536708in}{1.628446in}}%
\pgfpathcurveto{\pgfqpoint{2.546795in}{1.628446in}}{\pgfqpoint{2.556471in}{1.632453in}}{\pgfqpoint{2.563604in}{1.639586in}}%
\pgfpathcurveto{\pgfqpoint{2.570737in}{1.646719in}}{\pgfqpoint{2.574744in}{1.656395in}}{\pgfqpoint{2.574744in}{1.666482in}}%
\pgfpathcurveto{\pgfqpoint{2.574744in}{1.676569in}}{\pgfqpoint{2.570737in}{1.686245in}}{\pgfqpoint{2.563604in}{1.693378in}}%
\pgfpathcurveto{\pgfqpoint{2.556471in}{1.700511in}}{\pgfqpoint{2.546795in}{1.704518in}}{\pgfqpoint{2.536708in}{1.704518in}}%
\pgfpathcurveto{\pgfqpoint{2.526621in}{1.704518in}}{\pgfqpoint{2.516945in}{1.700511in}}{\pgfqpoint{2.509812in}{1.693378in}}%
\pgfpathcurveto{\pgfqpoint{2.502680in}{1.686245in}}{\pgfqpoint{2.498672in}{1.676569in}}{\pgfqpoint{2.498672in}{1.666482in}}%
\pgfpathcurveto{\pgfqpoint{2.498672in}{1.656395in}}{\pgfqpoint{2.502680in}{1.646719in}}{\pgfqpoint{2.509812in}{1.639586in}}%
\pgfpathcurveto{\pgfqpoint{2.516945in}{1.632453in}}{\pgfqpoint{2.526621in}{1.628446in}}{\pgfqpoint{2.536708in}{1.628446in}}%
\pgfpathclose%
\pgfusepath{stroke,fill}%
\end{pgfscope}%
\begin{pgfscope}%
\pgfpathrectangle{\pgfqpoint{0.800000in}{1.363959in}}{\pgfqpoint{3.968000in}{2.024082in}} %
\pgfusepath{clip}%
\pgfsetbuttcap%
\pgfsetroundjoin%
\definecolor{currentfill}{rgb}{0.121569,0.466667,0.705882}%
\pgfsetfillcolor{currentfill}%
\pgfsetlinewidth{1.003750pt}%
\definecolor{currentstroke}{rgb}{0.121569,0.466667,0.705882}%
\pgfsetstrokecolor{currentstroke}%
\pgfsetdash{}{0pt}%
\pgfpathmoveto{\pgfqpoint{2.801676in}{1.917590in}}%
\pgfpathcurveto{\pgfqpoint{2.811763in}{1.917590in}}{\pgfqpoint{2.821439in}{1.921598in}}{\pgfqpoint{2.828572in}{1.928731in}}%
\pgfpathcurveto{\pgfqpoint{2.835705in}{1.935864in}}{\pgfqpoint{2.839712in}{1.945539in}}{\pgfqpoint{2.839712in}{1.955627in}}%
\pgfpathcurveto{\pgfqpoint{2.839712in}{1.965714in}}{\pgfqpoint{2.835705in}{1.975390in}}{\pgfqpoint{2.828572in}{1.982522in}}%
\pgfpathcurveto{\pgfqpoint{2.821439in}{1.989655in}}{\pgfqpoint{2.811763in}{1.993663in}}{\pgfqpoint{2.801676in}{1.993663in}}%
\pgfpathcurveto{\pgfqpoint{2.791589in}{1.993663in}}{\pgfqpoint{2.781913in}{1.989655in}}{\pgfqpoint{2.774780in}{1.982522in}}%
\pgfpathcurveto{\pgfqpoint{2.767648in}{1.975390in}}{\pgfqpoint{2.763640in}{1.965714in}}{\pgfqpoint{2.763640in}{1.955627in}}%
\pgfpathcurveto{\pgfqpoint{2.763640in}{1.945539in}}{\pgfqpoint{2.767648in}{1.935864in}}{\pgfqpoint{2.774780in}{1.928731in}}%
\pgfpathcurveto{\pgfqpoint{2.781913in}{1.921598in}}{\pgfqpoint{2.791589in}{1.917590in}}{\pgfqpoint{2.801676in}{1.917590in}}%
\pgfpathclose%
\pgfusepath{stroke,fill}%
\end{pgfscope}%
\begin{pgfscope}%
\pgfpathrectangle{\pgfqpoint{0.800000in}{1.363959in}}{\pgfqpoint{3.968000in}{2.024082in}} %
\pgfusepath{clip}%
\pgfsetbuttcap%
\pgfsetroundjoin%
\definecolor{currentfill}{rgb}{0.121569,0.466667,0.705882}%
\pgfsetfillcolor{currentfill}%
\pgfsetlinewidth{1.003750pt}%
\definecolor{currentstroke}{rgb}{0.121569,0.466667,0.705882}%
\pgfsetstrokecolor{currentstroke}%
\pgfsetdash{}{0pt}%
\pgfpathmoveto{\pgfqpoint{2.578542in}{2.489454in}}%
\pgfpathcurveto{\pgfqpoint{2.588629in}{2.489454in}}{\pgfqpoint{2.598305in}{2.493461in}}{\pgfqpoint{2.605438in}{2.500594in}}%
\pgfpathcurveto{\pgfqpoint{2.612570in}{2.507727in}}{\pgfqpoint{2.616578in}{2.517403in}}{\pgfqpoint{2.616578in}{2.527490in}}%
\pgfpathcurveto{\pgfqpoint{2.616578in}{2.537577in}}{\pgfqpoint{2.612570in}{2.547253in}}{\pgfqpoint{2.605438in}{2.554386in}}%
\pgfpathcurveto{\pgfqpoint{2.598305in}{2.561518in}}{\pgfqpoint{2.588629in}{2.565526in}}{\pgfqpoint{2.578542in}{2.565526in}}%
\pgfpathcurveto{\pgfqpoint{2.568454in}{2.565526in}}{\pgfqpoint{2.558779in}{2.561518in}}{\pgfqpoint{2.551646in}{2.554386in}}%
\pgfpathcurveto{\pgfqpoint{2.544513in}{2.547253in}}{\pgfqpoint{2.540506in}{2.537577in}}{\pgfqpoint{2.540506in}{2.527490in}}%
\pgfpathcurveto{\pgfqpoint{2.540506in}{2.517403in}}{\pgfqpoint{2.544513in}{2.507727in}}{\pgfqpoint{2.551646in}{2.500594in}}%
\pgfpathcurveto{\pgfqpoint{2.558779in}{2.493461in}}{\pgfqpoint{2.568454in}{2.489454in}}{\pgfqpoint{2.578542in}{2.489454in}}%
\pgfpathclose%
\pgfusepath{stroke,fill}%
\end{pgfscope}%
\begin{pgfscope}%
\pgfpathrectangle{\pgfqpoint{0.800000in}{1.363959in}}{\pgfqpoint{3.968000in}{2.024082in}} %
\pgfusepath{clip}%
\pgfsetbuttcap%
\pgfsetroundjoin%
\definecolor{currentfill}{rgb}{0.121569,0.466667,0.705882}%
\pgfsetfillcolor{currentfill}%
\pgfsetlinewidth{1.003750pt}%
\definecolor{currentstroke}{rgb}{0.121569,0.466667,0.705882}%
\pgfsetstrokecolor{currentstroke}%
\pgfsetdash{}{0pt}%
\pgfpathmoveto{\pgfqpoint{2.177978in}{2.443326in}}%
\pgfpathcurveto{\pgfqpoint{2.188065in}{2.443326in}}{\pgfqpoint{2.197741in}{2.447334in}}{\pgfqpoint{2.204873in}{2.454466in}}%
\pgfpathcurveto{\pgfqpoint{2.212006in}{2.461599in}}{\pgfqpoint{2.216014in}{2.471275in}}{\pgfqpoint{2.216014in}{2.481362in}}%
\pgfpathcurveto{\pgfqpoint{2.216014in}{2.491450in}}{\pgfqpoint{2.212006in}{2.501125in}}{\pgfqpoint{2.204873in}{2.508258in}}%
\pgfpathcurveto{\pgfqpoint{2.197741in}{2.515391in}}{\pgfqpoint{2.188065in}{2.519398in}}{\pgfqpoint{2.177978in}{2.519398in}}%
\pgfpathcurveto{\pgfqpoint{2.167890in}{2.519398in}}{\pgfqpoint{2.158215in}{2.515391in}}{\pgfqpoint{2.151082in}{2.508258in}}%
\pgfpathcurveto{\pgfqpoint{2.143949in}{2.501125in}}{\pgfqpoint{2.139941in}{2.491450in}}{\pgfqpoint{2.139941in}{2.481362in}}%
\pgfpathcurveto{\pgfqpoint{2.139941in}{2.471275in}}{\pgfqpoint{2.143949in}{2.461599in}}{\pgfqpoint{2.151082in}{2.454466in}}%
\pgfpathcurveto{\pgfqpoint{2.158215in}{2.447334in}}{\pgfqpoint{2.167890in}{2.443326in}}{\pgfqpoint{2.177978in}{2.443326in}}%
\pgfpathclose%
\pgfusepath{stroke,fill}%
\end{pgfscope}%
\begin{pgfscope}%
\pgfpathrectangle{\pgfqpoint{0.800000in}{1.363959in}}{\pgfqpoint{3.968000in}{2.024082in}} %
\pgfusepath{clip}%
\pgfsetbuttcap%
\pgfsetroundjoin%
\definecolor{currentfill}{rgb}{0.121569,0.466667,0.705882}%
\pgfsetfillcolor{currentfill}%
\pgfsetlinewidth{1.003750pt}%
\definecolor{currentstroke}{rgb}{0.121569,0.466667,0.705882}%
\pgfsetstrokecolor{currentstroke}%
\pgfsetdash{}{0pt}%
\pgfpathmoveto{\pgfqpoint{3.047248in}{1.831395in}}%
\pgfpathcurveto{\pgfqpoint{3.057335in}{1.831395in}}{\pgfqpoint{3.067011in}{1.835403in}}{\pgfqpoint{3.074143in}{1.842535in}}%
\pgfpathcurveto{\pgfqpoint{3.081276in}{1.849668in}}{\pgfqpoint{3.085284in}{1.859344in}}{\pgfqpoint{3.085284in}{1.869431in}}%
\pgfpathcurveto{\pgfqpoint{3.085284in}{1.879518in}}{\pgfqpoint{3.081276in}{1.889194in}}{\pgfqpoint{3.074143in}{1.896327in}}%
\pgfpathcurveto{\pgfqpoint{3.067011in}{1.903460in}}{\pgfqpoint{3.057335in}{1.907467in}}{\pgfqpoint{3.047248in}{1.907467in}}%
\pgfpathcurveto{\pgfqpoint{3.037160in}{1.907467in}}{\pgfqpoint{3.027485in}{1.903460in}}{\pgfqpoint{3.020352in}{1.896327in}}%
\pgfpathcurveto{\pgfqpoint{3.013219in}{1.889194in}}{\pgfqpoint{3.009211in}{1.879518in}}{\pgfqpoint{3.009211in}{1.869431in}}%
\pgfpathcurveto{\pgfqpoint{3.009211in}{1.859344in}}{\pgfqpoint{3.013219in}{1.849668in}}{\pgfqpoint{3.020352in}{1.842535in}}%
\pgfpathcurveto{\pgfqpoint{3.027485in}{1.835403in}}{\pgfqpoint{3.037160in}{1.831395in}}{\pgfqpoint{3.047248in}{1.831395in}}%
\pgfpathclose%
\pgfusepath{stroke,fill}%
\end{pgfscope}%
\begin{pgfscope}%
\pgfpathrectangle{\pgfqpoint{0.800000in}{1.363959in}}{\pgfqpoint{3.968000in}{2.024082in}} %
\pgfusepath{clip}%
\pgfsetbuttcap%
\pgfsetroundjoin%
\definecolor{currentfill}{rgb}{0.121569,0.466667,0.705882}%
\pgfsetfillcolor{currentfill}%
\pgfsetlinewidth{1.003750pt}%
\definecolor{currentstroke}{rgb}{0.121569,0.466667,0.705882}%
\pgfsetstrokecolor{currentstroke}%
\pgfsetdash{}{0pt}%
\pgfpathmoveto{\pgfqpoint{2.575017in}{2.902336in}}%
\pgfpathcurveto{\pgfqpoint{2.585104in}{2.902336in}}{\pgfqpoint{2.594780in}{2.906343in}}{\pgfqpoint{2.601913in}{2.913476in}}%
\pgfpathcurveto{\pgfqpoint{2.609045in}{2.920609in}}{\pgfqpoint{2.613053in}{2.930285in}}{\pgfqpoint{2.613053in}{2.940372in}}%
\pgfpathcurveto{\pgfqpoint{2.613053in}{2.950459in}}{\pgfqpoint{2.609045in}{2.960135in}}{\pgfqpoint{2.601913in}{2.967268in}}%
\pgfpathcurveto{\pgfqpoint{2.594780in}{2.974401in}}{\pgfqpoint{2.585104in}{2.978408in}}{\pgfqpoint{2.575017in}{2.978408in}}%
\pgfpathcurveto{\pgfqpoint{2.564930in}{2.978408in}}{\pgfqpoint{2.555254in}{2.974401in}}{\pgfqpoint{2.548121in}{2.967268in}}%
\pgfpathcurveto{\pgfqpoint{2.540988in}{2.960135in}}{\pgfqpoint{2.536981in}{2.950459in}}{\pgfqpoint{2.536981in}{2.940372in}}%
\pgfpathcurveto{\pgfqpoint{2.536981in}{2.930285in}}{\pgfqpoint{2.540988in}{2.920609in}}{\pgfqpoint{2.548121in}{2.913476in}}%
\pgfpathcurveto{\pgfqpoint{2.555254in}{2.906343in}}{\pgfqpoint{2.564930in}{2.902336in}}{\pgfqpoint{2.575017in}{2.902336in}}%
\pgfpathclose%
\pgfusepath{stroke,fill}%
\end{pgfscope}%
\begin{pgfscope}%
\pgfpathrectangle{\pgfqpoint{0.800000in}{1.363959in}}{\pgfqpoint{3.968000in}{2.024082in}} %
\pgfusepath{clip}%
\pgfsetbuttcap%
\pgfsetroundjoin%
\definecolor{currentfill}{rgb}{0.121569,0.466667,0.705882}%
\pgfsetfillcolor{currentfill}%
\pgfsetlinewidth{1.003750pt}%
\definecolor{currentstroke}{rgb}{0.121569,0.466667,0.705882}%
\pgfsetstrokecolor{currentstroke}%
\pgfsetdash{}{0pt}%
\pgfpathmoveto{\pgfqpoint{2.838746in}{2.120308in}}%
\pgfpathcurveto{\pgfqpoint{2.848833in}{2.120308in}}{\pgfqpoint{2.858509in}{2.124316in}}{\pgfqpoint{2.865642in}{2.131449in}}%
\pgfpathcurveto{\pgfqpoint{2.872774in}{2.138582in}}{\pgfqpoint{2.876782in}{2.148257in}}{\pgfqpoint{2.876782in}{2.158345in}}%
\pgfpathcurveto{\pgfqpoint{2.876782in}{2.168432in}}{\pgfqpoint{2.872774in}{2.178107in}}{\pgfqpoint{2.865642in}{2.185240in}}%
\pgfpathcurveto{\pgfqpoint{2.858509in}{2.192373in}}{\pgfqpoint{2.848833in}{2.196381in}}{\pgfqpoint{2.838746in}{2.196381in}}%
\pgfpathcurveto{\pgfqpoint{2.828659in}{2.196381in}}{\pgfqpoint{2.818983in}{2.192373in}}{\pgfqpoint{2.811850in}{2.185240in}}%
\pgfpathcurveto{\pgfqpoint{2.804717in}{2.178107in}}{\pgfqpoint{2.800710in}{2.168432in}}{\pgfqpoint{2.800710in}{2.158345in}}%
\pgfpathcurveto{\pgfqpoint{2.800710in}{2.148257in}}{\pgfqpoint{2.804717in}{2.138582in}}{\pgfqpoint{2.811850in}{2.131449in}}%
\pgfpathcurveto{\pgfqpoint{2.818983in}{2.124316in}}{\pgfqpoint{2.828659in}{2.120308in}}{\pgfqpoint{2.838746in}{2.120308in}}%
\pgfpathclose%
\pgfusepath{stroke,fill}%
\end{pgfscope}%
\begin{pgfscope}%
\pgfpathrectangle{\pgfqpoint{0.800000in}{1.363959in}}{\pgfqpoint{3.968000in}{2.024082in}} %
\pgfusepath{clip}%
\pgfsetbuttcap%
\pgfsetroundjoin%
\definecolor{currentfill}{rgb}{0.121569,0.466667,0.705882}%
\pgfsetfillcolor{currentfill}%
\pgfsetlinewidth{1.003750pt}%
\definecolor{currentstroke}{rgb}{0.121569,0.466667,0.705882}%
\pgfsetstrokecolor{currentstroke}%
\pgfsetdash{}{0pt}%
\pgfpathmoveto{\pgfqpoint{2.166131in}{2.790447in}}%
\pgfpathcurveto{\pgfqpoint{2.176219in}{2.790447in}}{\pgfqpoint{2.185894in}{2.794455in}}{\pgfqpoint{2.193027in}{2.801587in}}%
\pgfpathcurveto{\pgfqpoint{2.200160in}{2.808720in}}{\pgfqpoint{2.204168in}{2.818396in}}{\pgfqpoint{2.204168in}{2.828483in}}%
\pgfpathcurveto{\pgfqpoint{2.204168in}{2.838571in}}{\pgfqpoint{2.200160in}{2.848246in}}{\pgfqpoint{2.193027in}{2.855379in}}%
\pgfpathcurveto{\pgfqpoint{2.185894in}{2.862512in}}{\pgfqpoint{2.176219in}{2.866519in}}{\pgfqpoint{2.166131in}{2.866519in}}%
\pgfpathcurveto{\pgfqpoint{2.156044in}{2.866519in}}{\pgfqpoint{2.146369in}{2.862512in}}{\pgfqpoint{2.139236in}{2.855379in}}%
\pgfpathcurveto{\pgfqpoint{2.132103in}{2.848246in}}{\pgfqpoint{2.128095in}{2.838571in}}{\pgfqpoint{2.128095in}{2.828483in}}%
\pgfpathcurveto{\pgfqpoint{2.128095in}{2.818396in}}{\pgfqpoint{2.132103in}{2.808720in}}{\pgfqpoint{2.139236in}{2.801587in}}%
\pgfpathcurveto{\pgfqpoint{2.146369in}{2.794455in}}{\pgfqpoint{2.156044in}{2.790447in}}{\pgfqpoint{2.166131in}{2.790447in}}%
\pgfpathclose%
\pgfusepath{stroke,fill}%
\end{pgfscope}%
\begin{pgfscope}%
\pgfpathrectangle{\pgfqpoint{0.800000in}{1.363959in}}{\pgfqpoint{3.968000in}{2.024082in}} %
\pgfusepath{clip}%
\pgfsetbuttcap%
\pgfsetroundjoin%
\definecolor{currentfill}{rgb}{0.121569,0.466667,0.705882}%
\pgfsetfillcolor{currentfill}%
\pgfsetlinewidth{1.003750pt}%
\definecolor{currentstroke}{rgb}{0.121569,0.466667,0.705882}%
\pgfsetstrokecolor{currentstroke}%
\pgfsetdash{}{0pt}%
\pgfpathmoveto{\pgfqpoint{3.290257in}{2.434391in}}%
\pgfpathcurveto{\pgfqpoint{3.300344in}{2.434391in}}{\pgfqpoint{3.310020in}{2.438399in}}{\pgfqpoint{3.317152in}{2.445532in}}%
\pgfpathcurveto{\pgfqpoint{3.324285in}{2.452664in}}{\pgfqpoint{3.328293in}{2.462340in}}{\pgfqpoint{3.328293in}{2.472427in}}%
\pgfpathcurveto{\pgfqpoint{3.328293in}{2.482515in}}{\pgfqpoint{3.324285in}{2.492190in}}{\pgfqpoint{3.317152in}{2.499323in}}%
\pgfpathcurveto{\pgfqpoint{3.310020in}{2.506456in}}{\pgfqpoint{3.300344in}{2.510464in}}{\pgfqpoint{3.290257in}{2.510464in}}%
\pgfpathcurveto{\pgfqpoint{3.280169in}{2.510464in}}{\pgfqpoint{3.270494in}{2.506456in}}{\pgfqpoint{3.263361in}{2.499323in}}%
\pgfpathcurveto{\pgfqpoint{3.256228in}{2.492190in}}{\pgfqpoint{3.252220in}{2.482515in}}{\pgfqpoint{3.252220in}{2.472427in}}%
\pgfpathcurveto{\pgfqpoint{3.252220in}{2.462340in}}{\pgfqpoint{3.256228in}{2.452664in}}{\pgfqpoint{3.263361in}{2.445532in}}%
\pgfpathcurveto{\pgfqpoint{3.270494in}{2.438399in}}{\pgfqpoint{3.280169in}{2.434391in}}{\pgfqpoint{3.290257in}{2.434391in}}%
\pgfpathclose%
\pgfusepath{stroke,fill}%
\end{pgfscope}%
\begin{pgfscope}%
\pgfpathrectangle{\pgfqpoint{0.800000in}{1.363959in}}{\pgfqpoint{3.968000in}{2.024082in}} %
\pgfusepath{clip}%
\pgfsetbuttcap%
\pgfsetroundjoin%
\definecolor{currentfill}{rgb}{0.121569,0.466667,0.705882}%
\pgfsetfillcolor{currentfill}%
\pgfsetlinewidth{1.003750pt}%
\definecolor{currentstroke}{rgb}{0.121569,0.466667,0.705882}%
\pgfsetstrokecolor{currentstroke}%
\pgfsetdash{}{0pt}%
\pgfpathmoveto{\pgfqpoint{2.381246in}{1.508588in}}%
\pgfpathcurveto{\pgfqpoint{2.391333in}{1.508588in}}{\pgfqpoint{2.401009in}{1.512596in}}{\pgfqpoint{2.408141in}{1.519729in}}%
\pgfpathcurveto{\pgfqpoint{2.415274in}{1.526861in}}{\pgfqpoint{2.419282in}{1.536537in}}{\pgfqpoint{2.419282in}{1.546624in}}%
\pgfpathcurveto{\pgfqpoint{2.419282in}{1.556712in}}{\pgfqpoint{2.415274in}{1.566387in}}{\pgfqpoint{2.408141in}{1.573520in}}%
\pgfpathcurveto{\pgfqpoint{2.401009in}{1.580653in}}{\pgfqpoint{2.391333in}{1.584661in}}{\pgfqpoint{2.381246in}{1.584661in}}%
\pgfpathcurveto{\pgfqpoint{2.371158in}{1.584661in}}{\pgfqpoint{2.361483in}{1.580653in}}{\pgfqpoint{2.354350in}{1.573520in}}%
\pgfpathcurveto{\pgfqpoint{2.347217in}{1.566387in}}{\pgfqpoint{2.343209in}{1.556712in}}{\pgfqpoint{2.343209in}{1.546624in}}%
\pgfpathcurveto{\pgfqpoint{2.343209in}{1.536537in}}{\pgfqpoint{2.347217in}{1.526861in}}{\pgfqpoint{2.354350in}{1.519729in}}%
\pgfpathcurveto{\pgfqpoint{2.361483in}{1.512596in}}{\pgfqpoint{2.371158in}{1.508588in}}{\pgfqpoint{2.381246in}{1.508588in}}%
\pgfpathclose%
\pgfusepath{stroke,fill}%
\end{pgfscope}%
\begin{pgfscope}%
\pgfpathrectangle{\pgfqpoint{0.800000in}{1.363959in}}{\pgfqpoint{3.968000in}{2.024082in}} %
\pgfusepath{clip}%
\pgfsetbuttcap%
\pgfsetroundjoin%
\definecolor{currentfill}{rgb}{0.121569,0.466667,0.705882}%
\pgfsetfillcolor{currentfill}%
\pgfsetlinewidth{1.003750pt}%
\definecolor{currentstroke}{rgb}{0.121569,0.466667,0.705882}%
\pgfsetstrokecolor{currentstroke}%
\pgfsetdash{}{0pt}%
\pgfpathmoveto{\pgfqpoint{3.088015in}{1.623167in}}%
\pgfpathcurveto{\pgfqpoint{3.098102in}{1.623167in}}{\pgfqpoint{3.107777in}{1.627175in}}{\pgfqpoint{3.114910in}{1.634308in}}%
\pgfpathcurveto{\pgfqpoint{3.122043in}{1.641440in}}{\pgfqpoint{3.126051in}{1.651116in}}{\pgfqpoint{3.126051in}{1.661203in}}%
\pgfpathcurveto{\pgfqpoint{3.126051in}{1.671291in}}{\pgfqpoint{3.122043in}{1.680966in}}{\pgfqpoint{3.114910in}{1.688099in}}%
\pgfpathcurveto{\pgfqpoint{3.107777in}{1.695232in}}{\pgfqpoint{3.098102in}{1.699240in}}{\pgfqpoint{3.088015in}{1.699240in}}%
\pgfpathcurveto{\pgfqpoint{3.077927in}{1.699240in}}{\pgfqpoint{3.068252in}{1.695232in}}{\pgfqpoint{3.061119in}{1.688099in}}%
\pgfpathcurveto{\pgfqpoint{3.053986in}{1.680966in}}{\pgfqpoint{3.049978in}{1.671291in}}{\pgfqpoint{3.049978in}{1.661203in}}%
\pgfpathcurveto{\pgfqpoint{3.049978in}{1.651116in}}{\pgfqpoint{3.053986in}{1.641440in}}{\pgfqpoint{3.061119in}{1.634308in}}%
\pgfpathcurveto{\pgfqpoint{3.068252in}{1.627175in}}{\pgfqpoint{3.077927in}{1.623167in}}{\pgfqpoint{3.088015in}{1.623167in}}%
\pgfpathclose%
\pgfusepath{stroke,fill}%
\end{pgfscope}%
\begin{pgfscope}%
\pgfpathrectangle{\pgfqpoint{0.800000in}{1.363959in}}{\pgfqpoint{3.968000in}{2.024082in}} %
\pgfusepath{clip}%
\pgfsetbuttcap%
\pgfsetroundjoin%
\definecolor{currentfill}{rgb}{0.121569,0.466667,0.705882}%
\pgfsetfillcolor{currentfill}%
\pgfsetlinewidth{1.003750pt}%
\definecolor{currentstroke}{rgb}{0.121569,0.466667,0.705882}%
\pgfsetstrokecolor{currentstroke}%
\pgfsetdash{}{0pt}%
\pgfpathmoveto{\pgfqpoint{1.955243in}{1.817843in}}%
\pgfpathcurveto{\pgfqpoint{1.965330in}{1.817843in}}{\pgfqpoint{1.975006in}{1.821851in}}{\pgfqpoint{1.982139in}{1.828984in}}%
\pgfpathcurveto{\pgfqpoint{1.989271in}{1.836117in}}{\pgfqpoint{1.993279in}{1.845792in}}{\pgfqpoint{1.993279in}{1.855880in}}%
\pgfpathcurveto{\pgfqpoint{1.993279in}{1.865967in}}{\pgfqpoint{1.989271in}{1.875643in}}{\pgfqpoint{1.982139in}{1.882775in}}%
\pgfpathcurveto{\pgfqpoint{1.975006in}{1.889908in}}{\pgfqpoint{1.965330in}{1.893916in}}{\pgfqpoint{1.955243in}{1.893916in}}%
\pgfpathcurveto{\pgfqpoint{1.945155in}{1.893916in}}{\pgfqpoint{1.935480in}{1.889908in}}{\pgfqpoint{1.928347in}{1.882775in}}%
\pgfpathcurveto{\pgfqpoint{1.921214in}{1.875643in}}{\pgfqpoint{1.917207in}{1.865967in}}{\pgfqpoint{1.917207in}{1.855880in}}%
\pgfpathcurveto{\pgfqpoint{1.917207in}{1.845792in}}{\pgfqpoint{1.921214in}{1.836117in}}{\pgfqpoint{1.928347in}{1.828984in}}%
\pgfpathcurveto{\pgfqpoint{1.935480in}{1.821851in}}{\pgfqpoint{1.945155in}{1.817843in}}{\pgfqpoint{1.955243in}{1.817843in}}%
\pgfpathclose%
\pgfusepath{stroke,fill}%
\end{pgfscope}%
\begin{pgfscope}%
\pgfpathrectangle{\pgfqpoint{0.800000in}{1.363959in}}{\pgfqpoint{3.968000in}{2.024082in}} %
\pgfusepath{clip}%
\pgfsetbuttcap%
\pgfsetroundjoin%
\definecolor{currentfill}{rgb}{0.121569,0.466667,0.705882}%
\pgfsetfillcolor{currentfill}%
\pgfsetlinewidth{1.003750pt}%
\definecolor{currentstroke}{rgb}{0.121569,0.466667,0.705882}%
\pgfsetstrokecolor{currentstroke}%
\pgfsetdash{}{0pt}%
\pgfpathmoveto{\pgfqpoint{2.113397in}{2.440876in}}%
\pgfpathcurveto{\pgfqpoint{2.123484in}{2.440876in}}{\pgfqpoint{2.133160in}{2.444884in}}{\pgfqpoint{2.140292in}{2.452017in}}%
\pgfpathcurveto{\pgfqpoint{2.147425in}{2.459150in}}{\pgfqpoint{2.151433in}{2.468825in}}{\pgfqpoint{2.151433in}{2.478913in}}%
\pgfpathcurveto{\pgfqpoint{2.151433in}{2.489000in}}{\pgfqpoint{2.147425in}{2.498675in}}{\pgfqpoint{2.140292in}{2.505808in}}%
\pgfpathcurveto{\pgfqpoint{2.133160in}{2.512941in}}{\pgfqpoint{2.123484in}{2.516949in}}{\pgfqpoint{2.113397in}{2.516949in}}%
\pgfpathcurveto{\pgfqpoint{2.103309in}{2.516949in}}{\pgfqpoint{2.093634in}{2.512941in}}{\pgfqpoint{2.086501in}{2.505808in}}%
\pgfpathcurveto{\pgfqpoint{2.079368in}{2.498675in}}{\pgfqpoint{2.075360in}{2.489000in}}{\pgfqpoint{2.075360in}{2.478913in}}%
\pgfpathcurveto{\pgfqpoint{2.075360in}{2.468825in}}{\pgfqpoint{2.079368in}{2.459150in}}{\pgfqpoint{2.086501in}{2.452017in}}%
\pgfpathcurveto{\pgfqpoint{2.093634in}{2.444884in}}{\pgfqpoint{2.103309in}{2.440876in}}{\pgfqpoint{2.113397in}{2.440876in}}%
\pgfpathclose%
\pgfusepath{stroke,fill}%
\end{pgfscope}%
\begin{pgfscope}%
\pgfpathrectangle{\pgfqpoint{0.800000in}{1.363959in}}{\pgfqpoint{3.968000in}{2.024082in}} %
\pgfusepath{clip}%
\pgfsetbuttcap%
\pgfsetroundjoin%
\definecolor{currentfill}{rgb}{0.121569,0.466667,0.705882}%
\pgfsetfillcolor{currentfill}%
\pgfsetlinewidth{1.003750pt}%
\definecolor{currentstroke}{rgb}{0.121569,0.466667,0.705882}%
\pgfsetstrokecolor{currentstroke}%
\pgfsetdash{}{0pt}%
\pgfpathmoveto{\pgfqpoint{2.582110in}{3.201107in}}%
\pgfpathcurveto{\pgfqpoint{2.592198in}{3.201107in}}{\pgfqpoint{2.601873in}{3.205115in}}{\pgfqpoint{2.609006in}{3.212248in}}%
\pgfpathcurveto{\pgfqpoint{2.616139in}{3.219380in}}{\pgfqpoint{2.620147in}{3.229056in}}{\pgfqpoint{2.620147in}{3.239143in}}%
\pgfpathcurveto{\pgfqpoint{2.620147in}{3.249231in}}{\pgfqpoint{2.616139in}{3.258906in}}{\pgfqpoint{2.609006in}{3.266039in}}%
\pgfpathcurveto{\pgfqpoint{2.601873in}{3.273172in}}{\pgfqpoint{2.592198in}{3.277180in}}{\pgfqpoint{2.582110in}{3.277180in}}%
\pgfpathcurveto{\pgfqpoint{2.572023in}{3.277180in}}{\pgfqpoint{2.562347in}{3.273172in}}{\pgfqpoint{2.555215in}{3.266039in}}%
\pgfpathcurveto{\pgfqpoint{2.548082in}{3.258906in}}{\pgfqpoint{2.544074in}{3.249231in}}{\pgfqpoint{2.544074in}{3.239143in}}%
\pgfpathcurveto{\pgfqpoint{2.544074in}{3.229056in}}{\pgfqpoint{2.548082in}{3.219380in}}{\pgfqpoint{2.555215in}{3.212248in}}%
\pgfpathcurveto{\pgfqpoint{2.562347in}{3.205115in}}{\pgfqpoint{2.572023in}{3.201107in}}{\pgfqpoint{2.582110in}{3.201107in}}%
\pgfpathclose%
\pgfusepath{stroke,fill}%
\end{pgfscope}%
\begin{pgfscope}%
\pgfpathrectangle{\pgfqpoint{0.800000in}{1.363959in}}{\pgfqpoint{3.968000in}{2.024082in}} %
\pgfusepath{clip}%
\pgfsetbuttcap%
\pgfsetroundjoin%
\definecolor{currentfill}{rgb}{0.121569,0.466667,0.705882}%
\pgfsetfillcolor{currentfill}%
\pgfsetlinewidth{1.003750pt}%
\definecolor{currentstroke}{rgb}{0.121569,0.466667,0.705882}%
\pgfsetstrokecolor{currentstroke}%
\pgfsetdash{}{0pt}%
\pgfpathmoveto{\pgfqpoint{3.583383in}{2.456514in}}%
\pgfpathcurveto{\pgfqpoint{3.593471in}{2.456514in}}{\pgfqpoint{3.603146in}{2.460522in}}{\pgfqpoint{3.610279in}{2.467655in}}%
\pgfpathcurveto{\pgfqpoint{3.617412in}{2.474788in}}{\pgfqpoint{3.621419in}{2.484463in}}{\pgfqpoint{3.621419in}{2.494550in}}%
\pgfpathcurveto{\pgfqpoint{3.621419in}{2.504638in}}{\pgfqpoint{3.617412in}{2.514313in}}{\pgfqpoint{3.610279in}{2.521446in}}%
\pgfpathcurveto{\pgfqpoint{3.603146in}{2.528579in}}{\pgfqpoint{3.593471in}{2.532587in}}{\pgfqpoint{3.583383in}{2.532587in}}%
\pgfpathcurveto{\pgfqpoint{3.573296in}{2.532587in}}{\pgfqpoint{3.563620in}{2.528579in}}{\pgfqpoint{3.556487in}{2.521446in}}%
\pgfpathcurveto{\pgfqpoint{3.549355in}{2.514313in}}{\pgfqpoint{3.545347in}{2.504638in}}{\pgfqpoint{3.545347in}{2.494550in}}%
\pgfpathcurveto{\pgfqpoint{3.545347in}{2.484463in}}{\pgfqpoint{3.549355in}{2.474788in}}{\pgfqpoint{3.556487in}{2.467655in}}%
\pgfpathcurveto{\pgfqpoint{3.563620in}{2.460522in}}{\pgfqpoint{3.573296in}{2.456514in}}{\pgfqpoint{3.583383in}{2.456514in}}%
\pgfpathclose%
\pgfusepath{stroke,fill}%
\end{pgfscope}%
\begin{pgfscope}%
\pgfpathrectangle{\pgfqpoint{0.800000in}{1.363959in}}{\pgfqpoint{3.968000in}{2.024082in}} %
\pgfusepath{clip}%
\pgfsetbuttcap%
\pgfsetroundjoin%
\definecolor{currentfill}{rgb}{0.121569,0.466667,0.705882}%
\pgfsetfillcolor{currentfill}%
\pgfsetlinewidth{1.003750pt}%
\definecolor{currentstroke}{rgb}{0.121569,0.466667,0.705882}%
\pgfsetstrokecolor{currentstroke}%
\pgfsetdash{}{0pt}%
\pgfpathmoveto{\pgfqpoint{2.791134in}{1.776030in}}%
\pgfpathcurveto{\pgfqpoint{2.801222in}{1.776030in}}{\pgfqpoint{2.810897in}{1.780038in}}{\pgfqpoint{2.818030in}{1.787171in}}%
\pgfpathcurveto{\pgfqpoint{2.825163in}{1.794303in}}{\pgfqpoint{2.829171in}{1.803979in}}{\pgfqpoint{2.829171in}{1.814066in}}%
\pgfpathcurveto{\pgfqpoint{2.829171in}{1.824154in}}{\pgfqpoint{2.825163in}{1.833829in}}{\pgfqpoint{2.818030in}{1.840962in}}%
\pgfpathcurveto{\pgfqpoint{2.810897in}{1.848095in}}{\pgfqpoint{2.801222in}{1.852103in}}{\pgfqpoint{2.791134in}{1.852103in}}%
\pgfpathcurveto{\pgfqpoint{2.781047in}{1.852103in}}{\pgfqpoint{2.771371in}{1.848095in}}{\pgfqpoint{2.764239in}{1.840962in}}%
\pgfpathcurveto{\pgfqpoint{2.757106in}{1.833829in}}{\pgfqpoint{2.753098in}{1.824154in}}{\pgfqpoint{2.753098in}{1.814066in}}%
\pgfpathcurveto{\pgfqpoint{2.753098in}{1.803979in}}{\pgfqpoint{2.757106in}{1.794303in}}{\pgfqpoint{2.764239in}{1.787171in}}%
\pgfpathcurveto{\pgfqpoint{2.771371in}{1.780038in}}{\pgfqpoint{2.781047in}{1.776030in}}{\pgfqpoint{2.791134in}{1.776030in}}%
\pgfpathclose%
\pgfusepath{stroke,fill}%
\end{pgfscope}%
\begin{pgfscope}%
\pgfpathrectangle{\pgfqpoint{0.800000in}{1.363959in}}{\pgfqpoint{3.968000in}{2.024082in}} %
\pgfusepath{clip}%
\pgfsetbuttcap%
\pgfsetroundjoin%
\definecolor{currentfill}{rgb}{0.121569,0.466667,0.705882}%
\pgfsetfillcolor{currentfill}%
\pgfsetlinewidth{1.003750pt}%
\definecolor{currentstroke}{rgb}{0.121569,0.466667,0.705882}%
\pgfsetstrokecolor{currentstroke}%
\pgfsetdash{}{0pt}%
\pgfpathmoveto{\pgfqpoint{3.087184in}{1.867492in}}%
\pgfpathcurveto{\pgfqpoint{3.097271in}{1.867492in}}{\pgfqpoint{3.106946in}{1.871499in}}{\pgfqpoint{3.114079in}{1.878632in}}%
\pgfpathcurveto{\pgfqpoint{3.121212in}{1.885765in}}{\pgfqpoint{3.125220in}{1.895441in}}{\pgfqpoint{3.125220in}{1.905528in}}%
\pgfpathcurveto{\pgfqpoint{3.125220in}{1.915615in}}{\pgfqpoint{3.121212in}{1.925291in}}{\pgfqpoint{3.114079in}{1.932424in}}%
\pgfpathcurveto{\pgfqpoint{3.106946in}{1.939556in}}{\pgfqpoint{3.097271in}{1.943564in}}{\pgfqpoint{3.087184in}{1.943564in}}%
\pgfpathcurveto{\pgfqpoint{3.077096in}{1.943564in}}{\pgfqpoint{3.067421in}{1.939556in}}{\pgfqpoint{3.060288in}{1.932424in}}%
\pgfpathcurveto{\pgfqpoint{3.053155in}{1.925291in}}{\pgfqpoint{3.049147in}{1.915615in}}{\pgfqpoint{3.049147in}{1.905528in}}%
\pgfpathcurveto{\pgfqpoint{3.049147in}{1.895441in}}{\pgfqpoint{3.053155in}{1.885765in}}{\pgfqpoint{3.060288in}{1.878632in}}%
\pgfpathcurveto{\pgfqpoint{3.067421in}{1.871499in}}{\pgfqpoint{3.077096in}{1.867492in}}{\pgfqpoint{3.087184in}{1.867492in}}%
\pgfpathclose%
\pgfusepath{stroke,fill}%
\end{pgfscope}%
\begin{pgfscope}%
\pgfpathrectangle{\pgfqpoint{0.800000in}{1.363959in}}{\pgfqpoint{3.968000in}{2.024082in}} %
\pgfusepath{clip}%
\pgfsetbuttcap%
\pgfsetroundjoin%
\definecolor{currentfill}{rgb}{0.121569,0.466667,0.705882}%
\pgfsetfillcolor{currentfill}%
\pgfsetlinewidth{1.003750pt}%
\definecolor{currentstroke}{rgb}{0.121569,0.466667,0.705882}%
\pgfsetstrokecolor{currentstroke}%
\pgfsetdash{}{0pt}%
\pgfpathmoveto{\pgfqpoint{2.385147in}{2.784299in}}%
\pgfpathcurveto{\pgfqpoint{2.395234in}{2.784299in}}{\pgfqpoint{2.404910in}{2.788307in}}{\pgfqpoint{2.412043in}{2.795440in}}%
\pgfpathcurveto{\pgfqpoint{2.419176in}{2.802573in}}{\pgfqpoint{2.423183in}{2.812248in}}{\pgfqpoint{2.423183in}{2.822335in}}%
\pgfpathcurveto{\pgfqpoint{2.423183in}{2.832423in}}{\pgfqpoint{2.419176in}{2.842098in}}{\pgfqpoint{2.412043in}{2.849231in}}%
\pgfpathcurveto{\pgfqpoint{2.404910in}{2.856364in}}{\pgfqpoint{2.395234in}{2.860372in}}{\pgfqpoint{2.385147in}{2.860372in}}%
\pgfpathcurveto{\pgfqpoint{2.375060in}{2.860372in}}{\pgfqpoint{2.365384in}{2.856364in}}{\pgfqpoint{2.358251in}{2.849231in}}%
\pgfpathcurveto{\pgfqpoint{2.351118in}{2.842098in}}{\pgfqpoint{2.347111in}{2.832423in}}{\pgfqpoint{2.347111in}{2.822335in}}%
\pgfpathcurveto{\pgfqpoint{2.347111in}{2.812248in}}{\pgfqpoint{2.351118in}{2.802573in}}{\pgfqpoint{2.358251in}{2.795440in}}%
\pgfpathcurveto{\pgfqpoint{2.365384in}{2.788307in}}{\pgfqpoint{2.375060in}{2.784299in}}{\pgfqpoint{2.385147in}{2.784299in}}%
\pgfpathclose%
\pgfusepath{stroke,fill}%
\end{pgfscope}%
\begin{pgfscope}%
\pgfpathrectangle{\pgfqpoint{0.800000in}{1.363959in}}{\pgfqpoint{3.968000in}{2.024082in}} %
\pgfusepath{clip}%
\pgfsetbuttcap%
\pgfsetroundjoin%
\definecolor{currentfill}{rgb}{0.121569,0.466667,0.705882}%
\pgfsetfillcolor{currentfill}%
\pgfsetlinewidth{1.003750pt}%
\definecolor{currentstroke}{rgb}{0.121569,0.466667,0.705882}%
\pgfsetstrokecolor{currentstroke}%
\pgfsetdash{}{0pt}%
\pgfpathmoveto{\pgfqpoint{3.077155in}{1.761362in}}%
\pgfpathcurveto{\pgfqpoint{3.087243in}{1.761362in}}{\pgfqpoint{3.096918in}{1.765369in}}{\pgfqpoint{3.104051in}{1.772502in}}%
\pgfpathcurveto{\pgfqpoint{3.111184in}{1.779635in}}{\pgfqpoint{3.115191in}{1.789311in}}{\pgfqpoint{3.115191in}{1.799398in}}%
\pgfpathcurveto{\pgfqpoint{3.115191in}{1.809485in}}{\pgfqpoint{3.111184in}{1.819161in}}{\pgfqpoint{3.104051in}{1.826294in}}%
\pgfpathcurveto{\pgfqpoint{3.096918in}{1.833426in}}{\pgfqpoint{3.087243in}{1.837434in}}{\pgfqpoint{3.077155in}{1.837434in}}%
\pgfpathcurveto{\pgfqpoint{3.067068in}{1.837434in}}{\pgfqpoint{3.057392in}{1.833426in}}{\pgfqpoint{3.050259in}{1.826294in}}%
\pgfpathcurveto{\pgfqpoint{3.043127in}{1.819161in}}{\pgfqpoint{3.039119in}{1.809485in}}{\pgfqpoint{3.039119in}{1.799398in}}%
\pgfpathcurveto{\pgfqpoint{3.039119in}{1.789311in}}{\pgfqpoint{3.043127in}{1.779635in}}{\pgfqpoint{3.050259in}{1.772502in}}%
\pgfpathcurveto{\pgfqpoint{3.057392in}{1.765369in}}{\pgfqpoint{3.067068in}{1.761362in}}{\pgfqpoint{3.077155in}{1.761362in}}%
\pgfpathclose%
\pgfusepath{stroke,fill}%
\end{pgfscope}%
\begin{pgfscope}%
\pgfpathrectangle{\pgfqpoint{0.800000in}{1.363959in}}{\pgfqpoint{3.968000in}{2.024082in}} %
\pgfusepath{clip}%
\pgfsetbuttcap%
\pgfsetroundjoin%
\definecolor{currentfill}{rgb}{0.121569,0.466667,0.705882}%
\pgfsetfillcolor{currentfill}%
\pgfsetlinewidth{1.003750pt}%
\definecolor{currentstroke}{rgb}{0.121569,0.466667,0.705882}%
\pgfsetstrokecolor{currentstroke}%
\pgfsetdash{}{0pt}%
\pgfpathmoveto{\pgfqpoint{3.353456in}{2.481246in}}%
\pgfpathcurveto{\pgfqpoint{3.363543in}{2.481246in}}{\pgfqpoint{3.373219in}{2.485253in}}{\pgfqpoint{3.380352in}{2.492386in}}%
\pgfpathcurveto{\pgfqpoint{3.387485in}{2.499519in}}{\pgfqpoint{3.391492in}{2.509194in}}{\pgfqpoint{3.391492in}{2.519282in}}%
\pgfpathcurveto{\pgfqpoint{3.391492in}{2.529369in}}{\pgfqpoint{3.387485in}{2.539045in}}{\pgfqpoint{3.380352in}{2.546178in}}%
\pgfpathcurveto{\pgfqpoint{3.373219in}{2.553310in}}{\pgfqpoint{3.363543in}{2.557318in}}{\pgfqpoint{3.353456in}{2.557318in}}%
\pgfpathcurveto{\pgfqpoint{3.343369in}{2.557318in}}{\pgfqpoint{3.333693in}{2.553310in}}{\pgfqpoint{3.326560in}{2.546178in}}%
\pgfpathcurveto{\pgfqpoint{3.319427in}{2.539045in}}{\pgfqpoint{3.315420in}{2.529369in}}{\pgfqpoint{3.315420in}{2.519282in}}%
\pgfpathcurveto{\pgfqpoint{3.315420in}{2.509194in}}{\pgfqpoint{3.319427in}{2.499519in}}{\pgfqpoint{3.326560in}{2.492386in}}%
\pgfpathcurveto{\pgfqpoint{3.333693in}{2.485253in}}{\pgfqpoint{3.343369in}{2.481246in}}{\pgfqpoint{3.353456in}{2.481246in}}%
\pgfpathclose%
\pgfusepath{stroke,fill}%
\end{pgfscope}%
\begin{pgfscope}%
\pgfpathrectangle{\pgfqpoint{0.800000in}{1.363959in}}{\pgfqpoint{3.968000in}{2.024082in}} %
\pgfusepath{clip}%
\pgfsetbuttcap%
\pgfsetroundjoin%
\definecolor{currentfill}{rgb}{0.121569,0.466667,0.705882}%
\pgfsetfillcolor{currentfill}%
\pgfsetlinewidth{1.003750pt}%
\definecolor{currentstroke}{rgb}{0.121569,0.466667,0.705882}%
\pgfsetstrokecolor{currentstroke}%
\pgfsetdash{}{0pt}%
\pgfpathmoveto{\pgfqpoint{3.195871in}{2.976433in}}%
\pgfpathcurveto{\pgfqpoint{3.205958in}{2.976433in}}{\pgfqpoint{3.215634in}{2.980441in}}{\pgfqpoint{3.222767in}{2.987574in}}%
\pgfpathcurveto{\pgfqpoint{3.229899in}{2.994707in}}{\pgfqpoint{3.233907in}{3.004382in}}{\pgfqpoint{3.233907in}{3.014469in}}%
\pgfpathcurveto{\pgfqpoint{3.233907in}{3.024557in}}{\pgfqpoint{3.229899in}{3.034232in}}{\pgfqpoint{3.222767in}{3.041365in}}%
\pgfpathcurveto{\pgfqpoint{3.215634in}{3.048498in}}{\pgfqpoint{3.205958in}{3.052506in}}{\pgfqpoint{3.195871in}{3.052506in}}%
\pgfpathcurveto{\pgfqpoint{3.185783in}{3.052506in}}{\pgfqpoint{3.176108in}{3.048498in}}{\pgfqpoint{3.168975in}{3.041365in}}%
\pgfpathcurveto{\pgfqpoint{3.161842in}{3.034232in}}{\pgfqpoint{3.157835in}{3.024557in}}{\pgfqpoint{3.157835in}{3.014469in}}%
\pgfpathcurveto{\pgfqpoint{3.157835in}{3.004382in}}{\pgfqpoint{3.161842in}{2.994707in}}{\pgfqpoint{3.168975in}{2.987574in}}%
\pgfpathcurveto{\pgfqpoint{3.176108in}{2.980441in}}{\pgfqpoint{3.185783in}{2.976433in}}{\pgfqpoint{3.195871in}{2.976433in}}%
\pgfpathclose%
\pgfusepath{stroke,fill}%
\end{pgfscope}%
\begin{pgfscope}%
\pgfpathrectangle{\pgfqpoint{0.800000in}{1.363959in}}{\pgfqpoint{3.968000in}{2.024082in}} %
\pgfusepath{clip}%
\pgfsetbuttcap%
\pgfsetroundjoin%
\definecolor{currentfill}{rgb}{0.121569,0.466667,0.705882}%
\pgfsetfillcolor{currentfill}%
\pgfsetlinewidth{1.003750pt}%
\definecolor{currentstroke}{rgb}{0.121569,0.466667,0.705882}%
\pgfsetstrokecolor{currentstroke}%
\pgfsetdash{}{0pt}%
\pgfpathmoveto{\pgfqpoint{3.192180in}{2.317863in}}%
\pgfpathcurveto{\pgfqpoint{3.202267in}{2.317863in}}{\pgfqpoint{3.211943in}{2.321871in}}{\pgfqpoint{3.219076in}{2.329004in}}%
\pgfpathcurveto{\pgfqpoint{3.226208in}{2.336137in}}{\pgfqpoint{3.230216in}{2.345812in}}{\pgfqpoint{3.230216in}{2.355899in}}%
\pgfpathcurveto{\pgfqpoint{3.230216in}{2.365987in}}{\pgfqpoint{3.226208in}{2.375662in}}{\pgfqpoint{3.219076in}{2.382795in}}%
\pgfpathcurveto{\pgfqpoint{3.211943in}{2.389928in}}{\pgfqpoint{3.202267in}{2.393936in}}{\pgfqpoint{3.192180in}{2.393936in}}%
\pgfpathcurveto{\pgfqpoint{3.182093in}{2.393936in}}{\pgfqpoint{3.172417in}{2.389928in}}{\pgfqpoint{3.165284in}{2.382795in}}%
\pgfpathcurveto{\pgfqpoint{3.158151in}{2.375662in}}{\pgfqpoint{3.154144in}{2.365987in}}{\pgfqpoint{3.154144in}{2.355899in}}%
\pgfpathcurveto{\pgfqpoint{3.154144in}{2.345812in}}{\pgfqpoint{3.158151in}{2.336137in}}{\pgfqpoint{3.165284in}{2.329004in}}%
\pgfpathcurveto{\pgfqpoint{3.172417in}{2.321871in}}{\pgfqpoint{3.182093in}{2.317863in}}{\pgfqpoint{3.192180in}{2.317863in}}%
\pgfpathclose%
\pgfusepath{stroke,fill}%
\end{pgfscope}%
\begin{pgfscope}%
\pgfpathrectangle{\pgfqpoint{0.800000in}{1.363959in}}{\pgfqpoint{3.968000in}{2.024082in}} %
\pgfusepath{clip}%
\pgfsetbuttcap%
\pgfsetroundjoin%
\definecolor{currentfill}{rgb}{0.121569,0.466667,0.705882}%
\pgfsetfillcolor{currentfill}%
\pgfsetlinewidth{1.003750pt}%
\definecolor{currentstroke}{rgb}{0.121569,0.466667,0.705882}%
\pgfsetstrokecolor{currentstroke}%
\pgfsetdash{}{0pt}%
\pgfpathmoveto{\pgfqpoint{2.319727in}{2.553241in}}%
\pgfpathcurveto{\pgfqpoint{2.329814in}{2.553241in}}{\pgfqpoint{2.339490in}{2.557249in}}{\pgfqpoint{2.346623in}{2.564382in}}%
\pgfpathcurveto{\pgfqpoint{2.353755in}{2.571514in}}{\pgfqpoint{2.357763in}{2.581190in}}{\pgfqpoint{2.357763in}{2.591277in}}%
\pgfpathcurveto{\pgfqpoint{2.357763in}{2.601365in}}{\pgfqpoint{2.353755in}{2.611040in}}{\pgfqpoint{2.346623in}{2.618173in}}%
\pgfpathcurveto{\pgfqpoint{2.339490in}{2.625306in}}{\pgfqpoint{2.329814in}{2.629314in}}{\pgfqpoint{2.319727in}{2.629314in}}%
\pgfpathcurveto{\pgfqpoint{2.309640in}{2.629314in}}{\pgfqpoint{2.299964in}{2.625306in}}{\pgfqpoint{2.292831in}{2.618173in}}%
\pgfpathcurveto{\pgfqpoint{2.285698in}{2.611040in}}{\pgfqpoint{2.281691in}{2.601365in}}{\pgfqpoint{2.281691in}{2.591277in}}%
\pgfpathcurveto{\pgfqpoint{2.281691in}{2.581190in}}{\pgfqpoint{2.285698in}{2.571514in}}{\pgfqpoint{2.292831in}{2.564382in}}%
\pgfpathcurveto{\pgfqpoint{2.299964in}{2.557249in}}{\pgfqpoint{2.309640in}{2.553241in}}{\pgfqpoint{2.319727in}{2.553241in}}%
\pgfpathclose%
\pgfusepath{stroke,fill}%
\end{pgfscope}%
\begin{pgfscope}%
\pgfpathrectangle{\pgfqpoint{0.800000in}{1.363959in}}{\pgfqpoint{3.968000in}{2.024082in}} %
\pgfusepath{clip}%
\pgfsetbuttcap%
\pgfsetroundjoin%
\definecolor{currentfill}{rgb}{0.121569,0.466667,0.705882}%
\pgfsetfillcolor{currentfill}%
\pgfsetlinewidth{1.003750pt}%
\definecolor{currentstroke}{rgb}{0.121569,0.466667,0.705882}%
\pgfsetstrokecolor{currentstroke}%
\pgfsetdash{}{0pt}%
\pgfpathmoveto{\pgfqpoint{3.268870in}{2.943141in}}%
\pgfpathcurveto{\pgfqpoint{3.278958in}{2.943141in}}{\pgfqpoint{3.288633in}{2.947149in}}{\pgfqpoint{3.295766in}{2.954282in}}%
\pgfpathcurveto{\pgfqpoint{3.302899in}{2.961414in}}{\pgfqpoint{3.306907in}{2.971090in}}{\pgfqpoint{3.306907in}{2.981177in}}%
\pgfpathcurveto{\pgfqpoint{3.306907in}{2.991265in}}{\pgfqpoint{3.302899in}{3.000940in}}{\pgfqpoint{3.295766in}{3.008073in}}%
\pgfpathcurveto{\pgfqpoint{3.288633in}{3.015206in}}{\pgfqpoint{3.278958in}{3.019214in}}{\pgfqpoint{3.268870in}{3.019214in}}%
\pgfpathcurveto{\pgfqpoint{3.258783in}{3.019214in}}{\pgfqpoint{3.249107in}{3.015206in}}{\pgfqpoint{3.241975in}{3.008073in}}%
\pgfpathcurveto{\pgfqpoint{3.234842in}{3.000940in}}{\pgfqpoint{3.230834in}{2.991265in}}{\pgfqpoint{3.230834in}{2.981177in}}%
\pgfpathcurveto{\pgfqpoint{3.230834in}{2.971090in}}{\pgfqpoint{3.234842in}{2.961414in}}{\pgfqpoint{3.241975in}{2.954282in}}%
\pgfpathcurveto{\pgfqpoint{3.249107in}{2.947149in}}{\pgfqpoint{3.258783in}{2.943141in}}{\pgfqpoint{3.268870in}{2.943141in}}%
\pgfpathclose%
\pgfusepath{stroke,fill}%
\end{pgfscope}%
\begin{pgfscope}%
\pgfpathrectangle{\pgfqpoint{0.800000in}{1.363959in}}{\pgfqpoint{3.968000in}{2.024082in}} %
\pgfusepath{clip}%
\pgfsetbuttcap%
\pgfsetroundjoin%
\definecolor{currentfill}{rgb}{0.121569,0.466667,0.705882}%
\pgfsetfillcolor{currentfill}%
\pgfsetlinewidth{1.003750pt}%
\definecolor{currentstroke}{rgb}{0.121569,0.466667,0.705882}%
\pgfsetstrokecolor{currentstroke}%
\pgfsetdash{}{0pt}%
\pgfpathmoveto{\pgfqpoint{2.632075in}{1.431473in}}%
\pgfpathcurveto{\pgfqpoint{2.642162in}{1.431473in}}{\pgfqpoint{2.651838in}{1.435480in}}{\pgfqpoint{2.658971in}{1.442613in}}%
\pgfpathcurveto{\pgfqpoint{2.666103in}{1.449746in}}{\pgfqpoint{2.670111in}{1.459422in}}{\pgfqpoint{2.670111in}{1.469509in}}%
\pgfpathcurveto{\pgfqpoint{2.670111in}{1.479596in}}{\pgfqpoint{2.666103in}{1.489272in}}{\pgfqpoint{2.658971in}{1.496405in}}%
\pgfpathcurveto{\pgfqpoint{2.651838in}{1.503538in}}{\pgfqpoint{2.642162in}{1.507545in}}{\pgfqpoint{2.632075in}{1.507545in}}%
\pgfpathcurveto{\pgfqpoint{2.621987in}{1.507545in}}{\pgfqpoint{2.612312in}{1.503538in}}{\pgfqpoint{2.605179in}{1.496405in}}%
\pgfpathcurveto{\pgfqpoint{2.598046in}{1.489272in}}{\pgfqpoint{2.594038in}{1.479596in}}{\pgfqpoint{2.594038in}{1.469509in}}%
\pgfpathcurveto{\pgfqpoint{2.594038in}{1.459422in}}{\pgfqpoint{2.598046in}{1.449746in}}{\pgfqpoint{2.605179in}{1.442613in}}%
\pgfpathcurveto{\pgfqpoint{2.612312in}{1.435480in}}{\pgfqpoint{2.621987in}{1.431473in}}{\pgfqpoint{2.632075in}{1.431473in}}%
\pgfpathclose%
\pgfusepath{stroke,fill}%
\end{pgfscope}%
\begin{pgfscope}%
\pgfpathrectangle{\pgfqpoint{0.800000in}{1.363959in}}{\pgfqpoint{3.968000in}{2.024082in}} %
\pgfusepath{clip}%
\pgfsetbuttcap%
\pgfsetroundjoin%
\definecolor{currentfill}{rgb}{0.121569,0.466667,0.705882}%
\pgfsetfillcolor{currentfill}%
\pgfsetlinewidth{1.003750pt}%
\definecolor{currentstroke}{rgb}{0.121569,0.466667,0.705882}%
\pgfsetstrokecolor{currentstroke}%
\pgfsetdash{}{0pt}%
\pgfpathmoveto{\pgfqpoint{2.344294in}{2.304014in}}%
\pgfpathcurveto{\pgfqpoint{2.354381in}{2.304014in}}{\pgfqpoint{2.364056in}{2.308022in}}{\pgfqpoint{2.371189in}{2.315154in}}%
\pgfpathcurveto{\pgfqpoint{2.378322in}{2.322287in}}{\pgfqpoint{2.382330in}{2.331963in}}{\pgfqpoint{2.382330in}{2.342050in}}%
\pgfpathcurveto{\pgfqpoint{2.382330in}{2.352137in}}{\pgfqpoint{2.378322in}{2.361813in}}{\pgfqpoint{2.371189in}{2.368946in}}%
\pgfpathcurveto{\pgfqpoint{2.364056in}{2.376079in}}{\pgfqpoint{2.354381in}{2.380086in}}{\pgfqpoint{2.344294in}{2.380086in}}%
\pgfpathcurveto{\pgfqpoint{2.334206in}{2.380086in}}{\pgfqpoint{2.324531in}{2.376079in}}{\pgfqpoint{2.317398in}{2.368946in}}%
\pgfpathcurveto{\pgfqpoint{2.310265in}{2.361813in}}{\pgfqpoint{2.306257in}{2.352137in}}{\pgfqpoint{2.306257in}{2.342050in}}%
\pgfpathcurveto{\pgfqpoint{2.306257in}{2.331963in}}{\pgfqpoint{2.310265in}{2.322287in}}{\pgfqpoint{2.317398in}{2.315154in}}%
\pgfpathcurveto{\pgfqpoint{2.324531in}{2.308022in}}{\pgfqpoint{2.334206in}{2.304014in}}{\pgfqpoint{2.344294in}{2.304014in}}%
\pgfpathclose%
\pgfusepath{stroke,fill}%
\end{pgfscope}%
\begin{pgfscope}%
\pgfpathrectangle{\pgfqpoint{0.800000in}{1.363959in}}{\pgfqpoint{3.968000in}{2.024082in}} %
\pgfusepath{clip}%
\pgfsetbuttcap%
\pgfsetroundjoin%
\definecolor{currentfill}{rgb}{0.121569,0.466667,0.705882}%
\pgfsetfillcolor{currentfill}%
\pgfsetlinewidth{1.003750pt}%
\definecolor{currentstroke}{rgb}{0.121569,0.466667,0.705882}%
\pgfsetstrokecolor{currentstroke}%
\pgfsetdash{}{0pt}%
\pgfpathmoveto{\pgfqpoint{2.333402in}{2.527824in}}%
\pgfpathcurveto{\pgfqpoint{2.343490in}{2.527824in}}{\pgfqpoint{2.353165in}{2.531831in}}{\pgfqpoint{2.360298in}{2.538964in}}%
\pgfpathcurveto{\pgfqpoint{2.367431in}{2.546097in}}{\pgfqpoint{2.371439in}{2.555773in}}{\pgfqpoint{2.371439in}{2.565860in}}%
\pgfpathcurveto{\pgfqpoint{2.371439in}{2.575947in}}{\pgfqpoint{2.367431in}{2.585623in}}{\pgfqpoint{2.360298in}{2.592756in}}%
\pgfpathcurveto{\pgfqpoint{2.353165in}{2.599889in}}{\pgfqpoint{2.343490in}{2.603896in}}{\pgfqpoint{2.333402in}{2.603896in}}%
\pgfpathcurveto{\pgfqpoint{2.323315in}{2.603896in}}{\pgfqpoint{2.313639in}{2.599889in}}{\pgfqpoint{2.306507in}{2.592756in}}%
\pgfpathcurveto{\pgfqpoint{2.299374in}{2.585623in}}{\pgfqpoint{2.295366in}{2.575947in}}{\pgfqpoint{2.295366in}{2.565860in}}%
\pgfpathcurveto{\pgfqpoint{2.295366in}{2.555773in}}{\pgfqpoint{2.299374in}{2.546097in}}{\pgfqpoint{2.306507in}{2.538964in}}%
\pgfpathcurveto{\pgfqpoint{2.313639in}{2.531831in}}{\pgfqpoint{2.323315in}{2.527824in}}{\pgfqpoint{2.333402in}{2.527824in}}%
\pgfpathclose%
\pgfusepath{stroke,fill}%
\end{pgfscope}%
\begin{pgfscope}%
\pgfpathrectangle{\pgfqpoint{0.800000in}{1.363959in}}{\pgfqpoint{3.968000in}{2.024082in}} %
\pgfusepath{clip}%
\pgfsetbuttcap%
\pgfsetroundjoin%
\definecolor{currentfill}{rgb}{0.121569,0.466667,0.705882}%
\pgfsetfillcolor{currentfill}%
\pgfsetlinewidth{1.003750pt}%
\definecolor{currentstroke}{rgb}{0.121569,0.466667,0.705882}%
\pgfsetstrokecolor{currentstroke}%
\pgfsetdash{}{0pt}%
\pgfpathmoveto{\pgfqpoint{2.039153in}{2.457915in}}%
\pgfpathcurveto{\pgfqpoint{2.049241in}{2.457915in}}{\pgfqpoint{2.058916in}{2.461923in}}{\pgfqpoint{2.066049in}{2.469056in}}%
\pgfpathcurveto{\pgfqpoint{2.073182in}{2.476188in}}{\pgfqpoint{2.077190in}{2.485864in}}{\pgfqpoint{2.077190in}{2.495951in}}%
\pgfpathcurveto{\pgfqpoint{2.077190in}{2.506039in}}{\pgfqpoint{2.073182in}{2.515714in}}{\pgfqpoint{2.066049in}{2.522847in}}%
\pgfpathcurveto{\pgfqpoint{2.058916in}{2.529980in}}{\pgfqpoint{2.049241in}{2.533988in}}{\pgfqpoint{2.039153in}{2.533988in}}%
\pgfpathcurveto{\pgfqpoint{2.029066in}{2.533988in}}{\pgfqpoint{2.019390in}{2.529980in}}{\pgfqpoint{2.012258in}{2.522847in}}%
\pgfpathcurveto{\pgfqpoint{2.005125in}{2.515714in}}{\pgfqpoint{2.001117in}{2.506039in}}{\pgfqpoint{2.001117in}{2.495951in}}%
\pgfpathcurveto{\pgfqpoint{2.001117in}{2.485864in}}{\pgfqpoint{2.005125in}{2.476188in}}{\pgfqpoint{2.012258in}{2.469056in}}%
\pgfpathcurveto{\pgfqpoint{2.019390in}{2.461923in}}{\pgfqpoint{2.029066in}{2.457915in}}{\pgfqpoint{2.039153in}{2.457915in}}%
\pgfpathclose%
\pgfusepath{stroke,fill}%
\end{pgfscope}%
\begin{pgfscope}%
\pgfpathrectangle{\pgfqpoint{0.800000in}{1.363959in}}{\pgfqpoint{3.968000in}{2.024082in}} %
\pgfusepath{clip}%
\pgfsetbuttcap%
\pgfsetroundjoin%
\definecolor{currentfill}{rgb}{0.121569,0.466667,0.705882}%
\pgfsetfillcolor{currentfill}%
\pgfsetlinewidth{1.003750pt}%
\definecolor{currentstroke}{rgb}{0.121569,0.466667,0.705882}%
\pgfsetstrokecolor{currentstroke}%
\pgfsetdash{}{0pt}%
\pgfpathmoveto{\pgfqpoint{2.191363in}{2.478235in}}%
\pgfpathcurveto{\pgfqpoint{2.201450in}{2.478235in}}{\pgfqpoint{2.211126in}{2.482243in}}{\pgfqpoint{2.218259in}{2.489376in}}%
\pgfpathcurveto{\pgfqpoint{2.225392in}{2.496509in}}{\pgfqpoint{2.229399in}{2.506184in}}{\pgfqpoint{2.229399in}{2.516272in}}%
\pgfpathcurveto{\pgfqpoint{2.229399in}{2.526359in}}{\pgfqpoint{2.225392in}{2.536035in}}{\pgfqpoint{2.218259in}{2.543167in}}%
\pgfpathcurveto{\pgfqpoint{2.211126in}{2.550300in}}{\pgfqpoint{2.201450in}{2.554308in}}{\pgfqpoint{2.191363in}{2.554308in}}%
\pgfpathcurveto{\pgfqpoint{2.181276in}{2.554308in}}{\pgfqpoint{2.171600in}{2.550300in}}{\pgfqpoint{2.164467in}{2.543167in}}%
\pgfpathcurveto{\pgfqpoint{2.157334in}{2.536035in}}{\pgfqpoint{2.153327in}{2.526359in}}{\pgfqpoint{2.153327in}{2.516272in}}%
\pgfpathcurveto{\pgfqpoint{2.153327in}{2.506184in}}{\pgfqpoint{2.157334in}{2.496509in}}{\pgfqpoint{2.164467in}{2.489376in}}%
\pgfpathcurveto{\pgfqpoint{2.171600in}{2.482243in}}{\pgfqpoint{2.181276in}{2.478235in}}{\pgfqpoint{2.191363in}{2.478235in}}%
\pgfpathclose%
\pgfusepath{stroke,fill}%
\end{pgfscope}%
\begin{pgfscope}%
\pgfpathrectangle{\pgfqpoint{0.800000in}{1.363959in}}{\pgfqpoint{3.968000in}{2.024082in}} %
\pgfusepath{clip}%
\pgfsetbuttcap%
\pgfsetroundjoin%
\definecolor{currentfill}{rgb}{0.121569,0.466667,0.705882}%
\pgfsetfillcolor{currentfill}%
\pgfsetlinewidth{1.003750pt}%
\definecolor{currentstroke}{rgb}{0.121569,0.466667,0.705882}%
\pgfsetstrokecolor{currentstroke}%
\pgfsetdash{}{0pt}%
\pgfpathmoveto{\pgfqpoint{2.291939in}{2.105909in}}%
\pgfpathcurveto{\pgfqpoint{2.302026in}{2.105909in}}{\pgfqpoint{2.311701in}{2.109917in}}{\pgfqpoint{2.318834in}{2.117050in}}%
\pgfpathcurveto{\pgfqpoint{2.325967in}{2.124182in}}{\pgfqpoint{2.329975in}{2.133858in}}{\pgfqpoint{2.329975in}{2.143945in}}%
\pgfpathcurveto{\pgfqpoint{2.329975in}{2.154033in}}{\pgfqpoint{2.325967in}{2.163708in}}{\pgfqpoint{2.318834in}{2.170841in}}%
\pgfpathcurveto{\pgfqpoint{2.311701in}{2.177974in}}{\pgfqpoint{2.302026in}{2.181982in}}{\pgfqpoint{2.291939in}{2.181982in}}%
\pgfpathcurveto{\pgfqpoint{2.281851in}{2.181982in}}{\pgfqpoint{2.272176in}{2.177974in}}{\pgfqpoint{2.265043in}{2.170841in}}%
\pgfpathcurveto{\pgfqpoint{2.257910in}{2.163708in}}{\pgfqpoint{2.253902in}{2.154033in}}{\pgfqpoint{2.253902in}{2.143945in}}%
\pgfpathcurveto{\pgfqpoint{2.253902in}{2.133858in}}{\pgfqpoint{2.257910in}{2.124182in}}{\pgfqpoint{2.265043in}{2.117050in}}%
\pgfpathcurveto{\pgfqpoint{2.272176in}{2.109917in}}{\pgfqpoint{2.281851in}{2.105909in}}{\pgfqpoint{2.291939in}{2.105909in}}%
\pgfpathclose%
\pgfusepath{stroke,fill}%
\end{pgfscope}%
\begin{pgfscope}%
\pgfpathrectangle{\pgfqpoint{0.800000in}{1.363959in}}{\pgfqpoint{3.968000in}{2.024082in}} %
\pgfusepath{clip}%
\pgfsetbuttcap%
\pgfsetroundjoin%
\definecolor{currentfill}{rgb}{0.121569,0.466667,0.705882}%
\pgfsetfillcolor{currentfill}%
\pgfsetlinewidth{1.003750pt}%
\definecolor{currentstroke}{rgb}{0.121569,0.466667,0.705882}%
\pgfsetstrokecolor{currentstroke}%
\pgfsetdash{}{0pt}%
\pgfpathmoveto{\pgfqpoint{2.713552in}{2.424520in}}%
\pgfpathcurveto{\pgfqpoint{2.723639in}{2.424520in}}{\pgfqpoint{2.733314in}{2.428527in}}{\pgfqpoint{2.740447in}{2.435660in}}%
\pgfpathcurveto{\pgfqpoint{2.747580in}{2.442793in}}{\pgfqpoint{2.751588in}{2.452469in}}{\pgfqpoint{2.751588in}{2.462556in}}%
\pgfpathcurveto{\pgfqpoint{2.751588in}{2.472643in}}{\pgfqpoint{2.747580in}{2.482319in}}{\pgfqpoint{2.740447in}{2.489452in}}%
\pgfpathcurveto{\pgfqpoint{2.733314in}{2.496585in}}{\pgfqpoint{2.723639in}{2.500592in}}{\pgfqpoint{2.713552in}{2.500592in}}%
\pgfpathcurveto{\pgfqpoint{2.703464in}{2.500592in}}{\pgfqpoint{2.693789in}{2.496585in}}{\pgfqpoint{2.686656in}{2.489452in}}%
\pgfpathcurveto{\pgfqpoint{2.679523in}{2.482319in}}{\pgfqpoint{2.675515in}{2.472643in}}{\pgfqpoint{2.675515in}{2.462556in}}%
\pgfpathcurveto{\pgfqpoint{2.675515in}{2.452469in}}{\pgfqpoint{2.679523in}{2.442793in}}{\pgfqpoint{2.686656in}{2.435660in}}%
\pgfpathcurveto{\pgfqpoint{2.693789in}{2.428527in}}{\pgfqpoint{2.703464in}{2.424520in}}{\pgfqpoint{2.713552in}{2.424520in}}%
\pgfpathclose%
\pgfusepath{stroke,fill}%
\end{pgfscope}%
\begin{pgfscope}%
\pgfpathrectangle{\pgfqpoint{0.800000in}{1.363959in}}{\pgfqpoint{3.968000in}{2.024082in}} %
\pgfusepath{clip}%
\pgfsetbuttcap%
\pgfsetroundjoin%
\definecolor{currentfill}{rgb}{0.121569,0.466667,0.705882}%
\pgfsetfillcolor{currentfill}%
\pgfsetlinewidth{1.003750pt}%
\definecolor{currentstroke}{rgb}{0.121569,0.466667,0.705882}%
\pgfsetstrokecolor{currentstroke}%
\pgfsetdash{}{0pt}%
\pgfpathmoveto{\pgfqpoint{2.867524in}{2.645159in}}%
\pgfpathcurveto{\pgfqpoint{2.877611in}{2.645159in}}{\pgfqpoint{2.887287in}{2.649167in}}{\pgfqpoint{2.894419in}{2.656300in}}%
\pgfpathcurveto{\pgfqpoint{2.901552in}{2.663432in}}{\pgfqpoint{2.905560in}{2.673108in}}{\pgfqpoint{2.905560in}{2.683195in}}%
\pgfpathcurveto{\pgfqpoint{2.905560in}{2.693283in}}{\pgfqpoint{2.901552in}{2.702958in}}{\pgfqpoint{2.894419in}{2.710091in}}%
\pgfpathcurveto{\pgfqpoint{2.887287in}{2.717224in}}{\pgfqpoint{2.877611in}{2.721232in}}{\pgfqpoint{2.867524in}{2.721232in}}%
\pgfpathcurveto{\pgfqpoint{2.857436in}{2.721232in}}{\pgfqpoint{2.847761in}{2.717224in}}{\pgfqpoint{2.840628in}{2.710091in}}%
\pgfpathcurveto{\pgfqpoint{2.833495in}{2.702958in}}{\pgfqpoint{2.829487in}{2.693283in}}{\pgfqpoint{2.829487in}{2.683195in}}%
\pgfpathcurveto{\pgfqpoint{2.829487in}{2.673108in}}{\pgfqpoint{2.833495in}{2.663432in}}{\pgfqpoint{2.840628in}{2.656300in}}%
\pgfpathcurveto{\pgfqpoint{2.847761in}{2.649167in}}{\pgfqpoint{2.857436in}{2.645159in}}{\pgfqpoint{2.867524in}{2.645159in}}%
\pgfpathclose%
\pgfusepath{stroke,fill}%
\end{pgfscope}%
\begin{pgfscope}%
\pgfpathrectangle{\pgfqpoint{0.800000in}{1.363959in}}{\pgfqpoint{3.968000in}{2.024082in}} %
\pgfusepath{clip}%
\pgfsetbuttcap%
\pgfsetroundjoin%
\definecolor{currentfill}{rgb}{0.121569,0.466667,0.705882}%
\pgfsetfillcolor{currentfill}%
\pgfsetlinewidth{1.003750pt}%
\definecolor{currentstroke}{rgb}{0.121569,0.466667,0.705882}%
\pgfsetstrokecolor{currentstroke}%
\pgfsetdash{}{0pt}%
\pgfpathmoveto{\pgfqpoint{2.070304in}{1.890983in}}%
\pgfpathcurveto{\pgfqpoint{2.080392in}{1.890983in}}{\pgfqpoint{2.090067in}{1.894991in}}{\pgfqpoint{2.097200in}{1.902124in}}%
\pgfpathcurveto{\pgfqpoint{2.104333in}{1.909257in}}{\pgfqpoint{2.108341in}{1.918932in}}{\pgfqpoint{2.108341in}{1.929019in}}%
\pgfpathcurveto{\pgfqpoint{2.108341in}{1.939107in}}{\pgfqpoint{2.104333in}{1.948782in}}{\pgfqpoint{2.097200in}{1.955915in}}%
\pgfpathcurveto{\pgfqpoint{2.090067in}{1.963048in}}{\pgfqpoint{2.080392in}{1.967056in}}{\pgfqpoint{2.070304in}{1.967056in}}%
\pgfpathcurveto{\pgfqpoint{2.060217in}{1.967056in}}{\pgfqpoint{2.050541in}{1.963048in}}{\pgfqpoint{2.043409in}{1.955915in}}%
\pgfpathcurveto{\pgfqpoint{2.036276in}{1.948782in}}{\pgfqpoint{2.032268in}{1.939107in}}{\pgfqpoint{2.032268in}{1.929019in}}%
\pgfpathcurveto{\pgfqpoint{2.032268in}{1.918932in}}{\pgfqpoint{2.036276in}{1.909257in}}{\pgfqpoint{2.043409in}{1.902124in}}%
\pgfpathcurveto{\pgfqpoint{2.050541in}{1.894991in}}{\pgfqpoint{2.060217in}{1.890983in}}{\pgfqpoint{2.070304in}{1.890983in}}%
\pgfpathclose%
\pgfusepath{stroke,fill}%
\end{pgfscope}%
\begin{pgfscope}%
\pgfpathrectangle{\pgfqpoint{0.800000in}{1.363959in}}{\pgfqpoint{3.968000in}{2.024082in}} %
\pgfusepath{clip}%
\pgfsetbuttcap%
\pgfsetroundjoin%
\definecolor{currentfill}{rgb}{0.121569,0.466667,0.705882}%
\pgfsetfillcolor{currentfill}%
\pgfsetlinewidth{1.003750pt}%
\definecolor{currentstroke}{rgb}{0.121569,0.466667,0.705882}%
\pgfsetstrokecolor{currentstroke}%
\pgfsetdash{}{0pt}%
\pgfpathmoveto{\pgfqpoint{2.895854in}{1.520558in}}%
\pgfpathcurveto{\pgfqpoint{2.905942in}{1.520558in}}{\pgfqpoint{2.915617in}{1.524566in}}{\pgfqpoint{2.922750in}{1.531699in}}%
\pgfpathcurveto{\pgfqpoint{2.929883in}{1.538832in}}{\pgfqpoint{2.933890in}{1.548507in}}{\pgfqpoint{2.933890in}{1.558594in}}%
\pgfpathcurveto{\pgfqpoint{2.933890in}{1.568682in}}{\pgfqpoint{2.929883in}{1.578357in}}{\pgfqpoint{2.922750in}{1.585490in}}%
\pgfpathcurveto{\pgfqpoint{2.915617in}{1.592623in}}{\pgfqpoint{2.905942in}{1.596631in}}{\pgfqpoint{2.895854in}{1.596631in}}%
\pgfpathcurveto{\pgfqpoint{2.885767in}{1.596631in}}{\pgfqpoint{2.876091in}{1.592623in}}{\pgfqpoint{2.868958in}{1.585490in}}%
\pgfpathcurveto{\pgfqpoint{2.861826in}{1.578357in}}{\pgfqpoint{2.857818in}{1.568682in}}{\pgfqpoint{2.857818in}{1.558594in}}%
\pgfpathcurveto{\pgfqpoint{2.857818in}{1.548507in}}{\pgfqpoint{2.861826in}{1.538832in}}{\pgfqpoint{2.868958in}{1.531699in}}%
\pgfpathcurveto{\pgfqpoint{2.876091in}{1.524566in}}{\pgfqpoint{2.885767in}{1.520558in}}{\pgfqpoint{2.895854in}{1.520558in}}%
\pgfpathclose%
\pgfusepath{stroke,fill}%
\end{pgfscope}%
\begin{pgfscope}%
\pgfpathrectangle{\pgfqpoint{0.800000in}{1.363959in}}{\pgfqpoint{3.968000in}{2.024082in}} %
\pgfusepath{clip}%
\pgfsetbuttcap%
\pgfsetroundjoin%
\definecolor{currentfill}{rgb}{0.121569,0.466667,0.705882}%
\pgfsetfillcolor{currentfill}%
\pgfsetlinewidth{1.003750pt}%
\definecolor{currentstroke}{rgb}{0.121569,0.466667,0.705882}%
\pgfsetstrokecolor{currentstroke}%
\pgfsetdash{}{0pt}%
\pgfpathmoveto{\pgfqpoint{3.101885in}{2.087149in}}%
\pgfpathcurveto{\pgfqpoint{3.111972in}{2.087149in}}{\pgfqpoint{3.121647in}{2.091157in}}{\pgfqpoint{3.128780in}{2.098290in}}%
\pgfpathcurveto{\pgfqpoint{3.135913in}{2.105423in}}{\pgfqpoint{3.139921in}{2.115098in}}{\pgfqpoint{3.139921in}{2.125186in}}%
\pgfpathcurveto{\pgfqpoint{3.139921in}{2.135273in}}{\pgfqpoint{3.135913in}{2.144949in}}{\pgfqpoint{3.128780in}{2.152081in}}%
\pgfpathcurveto{\pgfqpoint{3.121647in}{2.159214in}}{\pgfqpoint{3.111972in}{2.163222in}}{\pgfqpoint{3.101885in}{2.163222in}}%
\pgfpathcurveto{\pgfqpoint{3.091797in}{2.163222in}}{\pgfqpoint{3.082122in}{2.159214in}}{\pgfqpoint{3.074989in}{2.152081in}}%
\pgfpathcurveto{\pgfqpoint{3.067856in}{2.144949in}}{\pgfqpoint{3.063848in}{2.135273in}}{\pgfqpoint{3.063848in}{2.125186in}}%
\pgfpathcurveto{\pgfqpoint{3.063848in}{2.115098in}}{\pgfqpoint{3.067856in}{2.105423in}}{\pgfqpoint{3.074989in}{2.098290in}}%
\pgfpathcurveto{\pgfqpoint{3.082122in}{2.091157in}}{\pgfqpoint{3.091797in}{2.087149in}}{\pgfqpoint{3.101885in}{2.087149in}}%
\pgfpathclose%
\pgfusepath{stroke,fill}%
\end{pgfscope}%
\begin{pgfscope}%
\pgfpathrectangle{\pgfqpoint{0.800000in}{1.363959in}}{\pgfqpoint{3.968000in}{2.024082in}} %
\pgfusepath{clip}%
\pgfsetbuttcap%
\pgfsetroundjoin%
\definecolor{currentfill}{rgb}{0.121569,0.466667,0.705882}%
\pgfsetfillcolor{currentfill}%
\pgfsetlinewidth{1.003750pt}%
\definecolor{currentstroke}{rgb}{0.121569,0.466667,0.705882}%
\pgfsetstrokecolor{currentstroke}%
\pgfsetdash{}{0pt}%
\pgfpathmoveto{\pgfqpoint{1.987495in}{2.571450in}}%
\pgfpathcurveto{\pgfqpoint{1.997582in}{2.571450in}}{\pgfqpoint{2.007258in}{2.575458in}}{\pgfqpoint{2.014390in}{2.582591in}}%
\pgfpathcurveto{\pgfqpoint{2.021523in}{2.589723in}}{\pgfqpoint{2.025531in}{2.599399in}}{\pgfqpoint{2.025531in}{2.609486in}}%
\pgfpathcurveto{\pgfqpoint{2.025531in}{2.619574in}}{\pgfqpoint{2.021523in}{2.629249in}}{\pgfqpoint{2.014390in}{2.636382in}}%
\pgfpathcurveto{\pgfqpoint{2.007258in}{2.643515in}}{\pgfqpoint{1.997582in}{2.647523in}}{\pgfqpoint{1.987495in}{2.647523in}}%
\pgfpathcurveto{\pgfqpoint{1.977407in}{2.647523in}}{\pgfqpoint{1.967732in}{2.643515in}}{\pgfqpoint{1.960599in}{2.636382in}}%
\pgfpathcurveto{\pgfqpoint{1.953466in}{2.629249in}}{\pgfqpoint{1.949458in}{2.619574in}}{\pgfqpoint{1.949458in}{2.609486in}}%
\pgfpathcurveto{\pgfqpoint{1.949458in}{2.599399in}}{\pgfqpoint{1.953466in}{2.589723in}}{\pgfqpoint{1.960599in}{2.582591in}}%
\pgfpathcurveto{\pgfqpoint{1.967732in}{2.575458in}}{\pgfqpoint{1.977407in}{2.571450in}}{\pgfqpoint{1.987495in}{2.571450in}}%
\pgfpathclose%
\pgfusepath{stroke,fill}%
\end{pgfscope}%
\begin{pgfscope}%
\pgfpathrectangle{\pgfqpoint{0.800000in}{1.363959in}}{\pgfqpoint{3.968000in}{2.024082in}} %
\pgfusepath{clip}%
\pgfsetbuttcap%
\pgfsetroundjoin%
\definecolor{currentfill}{rgb}{0.121569,0.466667,0.705882}%
\pgfsetfillcolor{currentfill}%
\pgfsetlinewidth{1.003750pt}%
\definecolor{currentstroke}{rgb}{0.121569,0.466667,0.705882}%
\pgfsetstrokecolor{currentstroke}%
\pgfsetdash{}{0pt}%
\pgfpathmoveto{\pgfqpoint{2.568957in}{1.788865in}}%
\pgfpathcurveto{\pgfqpoint{2.579044in}{1.788865in}}{\pgfqpoint{2.588720in}{1.792873in}}{\pgfqpoint{2.595852in}{1.800006in}}%
\pgfpathcurveto{\pgfqpoint{2.602985in}{1.807138in}}{\pgfqpoint{2.606993in}{1.816814in}}{\pgfqpoint{2.606993in}{1.826901in}}%
\pgfpathcurveto{\pgfqpoint{2.606993in}{1.836989in}}{\pgfqpoint{2.602985in}{1.846664in}}{\pgfqpoint{2.595852in}{1.853797in}}%
\pgfpathcurveto{\pgfqpoint{2.588720in}{1.860930in}}{\pgfqpoint{2.579044in}{1.864938in}}{\pgfqpoint{2.568957in}{1.864938in}}%
\pgfpathcurveto{\pgfqpoint{2.558869in}{1.864938in}}{\pgfqpoint{2.549194in}{1.860930in}}{\pgfqpoint{2.542061in}{1.853797in}}%
\pgfpathcurveto{\pgfqpoint{2.534928in}{1.846664in}}{\pgfqpoint{2.530920in}{1.836989in}}{\pgfqpoint{2.530920in}{1.826901in}}%
\pgfpathcurveto{\pgfqpoint{2.530920in}{1.816814in}}{\pgfqpoint{2.534928in}{1.807138in}}{\pgfqpoint{2.542061in}{1.800006in}}%
\pgfpathcurveto{\pgfqpoint{2.549194in}{1.792873in}}{\pgfqpoint{2.558869in}{1.788865in}}{\pgfqpoint{2.568957in}{1.788865in}}%
\pgfpathclose%
\pgfusepath{stroke,fill}%
\end{pgfscope}%
\begin{pgfscope}%
\pgfpathrectangle{\pgfqpoint{0.800000in}{1.363959in}}{\pgfqpoint{3.968000in}{2.024082in}} %
\pgfusepath{clip}%
\pgfsetbuttcap%
\pgfsetroundjoin%
\definecolor{currentfill}{rgb}{0.121569,0.466667,0.705882}%
\pgfsetfillcolor{currentfill}%
\pgfsetlinewidth{1.003750pt}%
\definecolor{currentstroke}{rgb}{0.121569,0.466667,0.705882}%
\pgfsetstrokecolor{currentstroke}%
\pgfsetdash{}{0pt}%
\pgfpathmoveto{\pgfqpoint{3.091890in}{2.800956in}}%
\pgfpathcurveto{\pgfqpoint{3.101977in}{2.800956in}}{\pgfqpoint{3.111653in}{2.804964in}}{\pgfqpoint{3.118786in}{2.812097in}}%
\pgfpathcurveto{\pgfqpoint{3.125918in}{2.819229in}}{\pgfqpoint{3.129926in}{2.828905in}}{\pgfqpoint{3.129926in}{2.838992in}}%
\pgfpathcurveto{\pgfqpoint{3.129926in}{2.849080in}}{\pgfqpoint{3.125918in}{2.858755in}}{\pgfqpoint{3.118786in}{2.865888in}}%
\pgfpathcurveto{\pgfqpoint{3.111653in}{2.873021in}}{\pgfqpoint{3.101977in}{2.877029in}}{\pgfqpoint{3.091890in}{2.877029in}}%
\pgfpathcurveto{\pgfqpoint{3.081802in}{2.877029in}}{\pgfqpoint{3.072127in}{2.873021in}}{\pgfqpoint{3.064994in}{2.865888in}}%
\pgfpathcurveto{\pgfqpoint{3.057861in}{2.858755in}}{\pgfqpoint{3.053854in}{2.849080in}}{\pgfqpoint{3.053854in}{2.838992in}}%
\pgfpathcurveto{\pgfqpoint{3.053854in}{2.828905in}}{\pgfqpoint{3.057861in}{2.819229in}}{\pgfqpoint{3.064994in}{2.812097in}}%
\pgfpathcurveto{\pgfqpoint{3.072127in}{2.804964in}}{\pgfqpoint{3.081802in}{2.800956in}}{\pgfqpoint{3.091890in}{2.800956in}}%
\pgfpathclose%
\pgfusepath{stroke,fill}%
\end{pgfscope}%
\begin{pgfscope}%
\pgfpathrectangle{\pgfqpoint{0.800000in}{1.363959in}}{\pgfqpoint{3.968000in}{2.024082in}} %
\pgfusepath{clip}%
\pgfsetbuttcap%
\pgfsetroundjoin%
\definecolor{currentfill}{rgb}{0.121569,0.466667,0.705882}%
\pgfsetfillcolor{currentfill}%
\pgfsetlinewidth{1.003750pt}%
\definecolor{currentstroke}{rgb}{0.121569,0.466667,0.705882}%
\pgfsetstrokecolor{currentstroke}%
\pgfsetdash{}{0pt}%
\pgfpathmoveto{\pgfqpoint{2.569187in}{1.520934in}}%
\pgfpathcurveto{\pgfqpoint{2.579274in}{1.520934in}}{\pgfqpoint{2.588950in}{1.524942in}}{\pgfqpoint{2.596082in}{1.532075in}}%
\pgfpathcurveto{\pgfqpoint{2.603215in}{1.539208in}}{\pgfqpoint{2.607223in}{1.548883in}}{\pgfqpoint{2.607223in}{1.558971in}}%
\pgfpathcurveto{\pgfqpoint{2.607223in}{1.569058in}}{\pgfqpoint{2.603215in}{1.578733in}}{\pgfqpoint{2.596082in}{1.585866in}}%
\pgfpathcurveto{\pgfqpoint{2.588950in}{1.592999in}}{\pgfqpoint{2.579274in}{1.597007in}}{\pgfqpoint{2.569187in}{1.597007in}}%
\pgfpathcurveto{\pgfqpoint{2.559099in}{1.597007in}}{\pgfqpoint{2.549424in}{1.592999in}}{\pgfqpoint{2.542291in}{1.585866in}}%
\pgfpathcurveto{\pgfqpoint{2.535158in}{1.578733in}}{\pgfqpoint{2.531150in}{1.569058in}}{\pgfqpoint{2.531150in}{1.558971in}}%
\pgfpathcurveto{\pgfqpoint{2.531150in}{1.548883in}}{\pgfqpoint{2.535158in}{1.539208in}}{\pgfqpoint{2.542291in}{1.532075in}}%
\pgfpathcurveto{\pgfqpoint{2.549424in}{1.524942in}}{\pgfqpoint{2.559099in}{1.520934in}}{\pgfqpoint{2.569187in}{1.520934in}}%
\pgfpathclose%
\pgfusepath{stroke,fill}%
\end{pgfscope}%
\begin{pgfscope}%
\pgfpathrectangle{\pgfqpoint{0.800000in}{1.363959in}}{\pgfqpoint{3.968000in}{2.024082in}} %
\pgfusepath{clip}%
\pgfsetbuttcap%
\pgfsetroundjoin%
\definecolor{currentfill}{rgb}{0.121569,0.466667,0.705882}%
\pgfsetfillcolor{currentfill}%
\pgfsetlinewidth{1.003750pt}%
\definecolor{currentstroke}{rgb}{0.121569,0.466667,0.705882}%
\pgfsetstrokecolor{currentstroke}%
\pgfsetdash{}{0pt}%
\pgfpathmoveto{\pgfqpoint{2.796229in}{2.897951in}}%
\pgfpathcurveto{\pgfqpoint{2.806316in}{2.897951in}}{\pgfqpoint{2.815992in}{2.901959in}}{\pgfqpoint{2.823124in}{2.909092in}}%
\pgfpathcurveto{\pgfqpoint{2.830257in}{2.916224in}}{\pgfqpoint{2.834265in}{2.925900in}}{\pgfqpoint{2.834265in}{2.935987in}}%
\pgfpathcurveto{\pgfqpoint{2.834265in}{2.946075in}}{\pgfqpoint{2.830257in}{2.955750in}}{\pgfqpoint{2.823124in}{2.962883in}}%
\pgfpathcurveto{\pgfqpoint{2.815992in}{2.970016in}}{\pgfqpoint{2.806316in}{2.974024in}}{\pgfqpoint{2.796229in}{2.974024in}}%
\pgfpathcurveto{\pgfqpoint{2.786141in}{2.974024in}}{\pgfqpoint{2.776466in}{2.970016in}}{\pgfqpoint{2.769333in}{2.962883in}}%
\pgfpathcurveto{\pgfqpoint{2.762200in}{2.955750in}}{\pgfqpoint{2.758192in}{2.946075in}}{\pgfqpoint{2.758192in}{2.935987in}}%
\pgfpathcurveto{\pgfqpoint{2.758192in}{2.925900in}}{\pgfqpoint{2.762200in}{2.916224in}}{\pgfqpoint{2.769333in}{2.909092in}}%
\pgfpathcurveto{\pgfqpoint{2.776466in}{2.901959in}}{\pgfqpoint{2.786141in}{2.897951in}}{\pgfqpoint{2.796229in}{2.897951in}}%
\pgfpathclose%
\pgfusepath{stroke,fill}%
\end{pgfscope}%
\begin{pgfscope}%
\pgfpathrectangle{\pgfqpoint{0.800000in}{1.363959in}}{\pgfqpoint{3.968000in}{2.024082in}} %
\pgfusepath{clip}%
\pgfsetbuttcap%
\pgfsetroundjoin%
\definecolor{currentfill}{rgb}{0.121569,0.466667,0.705882}%
\pgfsetfillcolor{currentfill}%
\pgfsetlinewidth{1.003750pt}%
\definecolor{currentstroke}{rgb}{0.121569,0.466667,0.705882}%
\pgfsetstrokecolor{currentstroke}%
\pgfsetdash{}{0pt}%
\pgfpathmoveto{\pgfqpoint{3.034801in}{1.757640in}}%
\pgfpathcurveto{\pgfqpoint{3.044888in}{1.757640in}}{\pgfqpoint{3.054563in}{1.761648in}}{\pgfqpoint{3.061696in}{1.768780in}}%
\pgfpathcurveto{\pgfqpoint{3.068829in}{1.775913in}}{\pgfqpoint{3.072837in}{1.785589in}}{\pgfqpoint{3.072837in}{1.795676in}}%
\pgfpathcurveto{\pgfqpoint{3.072837in}{1.805764in}}{\pgfqpoint{3.068829in}{1.815439in}}{\pgfqpoint{3.061696in}{1.822572in}}%
\pgfpathcurveto{\pgfqpoint{3.054563in}{1.829705in}}{\pgfqpoint{3.044888in}{1.833712in}}{\pgfqpoint{3.034801in}{1.833712in}}%
\pgfpathcurveto{\pgfqpoint{3.024713in}{1.833712in}}{\pgfqpoint{3.015038in}{1.829705in}}{\pgfqpoint{3.007905in}{1.822572in}}%
\pgfpathcurveto{\pgfqpoint{3.000772in}{1.815439in}}{\pgfqpoint{2.996764in}{1.805764in}}{\pgfqpoint{2.996764in}{1.795676in}}%
\pgfpathcurveto{\pgfqpoint{2.996764in}{1.785589in}}{\pgfqpoint{3.000772in}{1.775913in}}{\pgfqpoint{3.007905in}{1.768780in}}%
\pgfpathcurveto{\pgfqpoint{3.015038in}{1.761648in}}{\pgfqpoint{3.024713in}{1.757640in}}{\pgfqpoint{3.034801in}{1.757640in}}%
\pgfpathclose%
\pgfusepath{stroke,fill}%
\end{pgfscope}%
\begin{pgfscope}%
\pgfpathrectangle{\pgfqpoint{0.800000in}{1.363959in}}{\pgfqpoint{3.968000in}{2.024082in}} %
\pgfusepath{clip}%
\pgfsetbuttcap%
\pgfsetroundjoin%
\definecolor{currentfill}{rgb}{0.121569,0.466667,0.705882}%
\pgfsetfillcolor{currentfill}%
\pgfsetlinewidth{1.003750pt}%
\definecolor{currentstroke}{rgb}{0.121569,0.466667,0.705882}%
\pgfsetstrokecolor{currentstroke}%
\pgfsetdash{}{0pt}%
\pgfpathmoveto{\pgfqpoint{2.757043in}{2.589625in}}%
\pgfpathcurveto{\pgfqpoint{2.767131in}{2.589625in}}{\pgfqpoint{2.776806in}{2.593632in}}{\pgfqpoint{2.783939in}{2.600765in}}%
\pgfpathcurveto{\pgfqpoint{2.791072in}{2.607898in}}{\pgfqpoint{2.795080in}{2.617574in}}{\pgfqpoint{2.795080in}{2.627661in}}%
\pgfpathcurveto{\pgfqpoint{2.795080in}{2.637748in}}{\pgfqpoint{2.791072in}{2.647424in}}{\pgfqpoint{2.783939in}{2.654557in}}%
\pgfpathcurveto{\pgfqpoint{2.776806in}{2.661689in}}{\pgfqpoint{2.767131in}{2.665697in}}{\pgfqpoint{2.757043in}{2.665697in}}%
\pgfpathcurveto{\pgfqpoint{2.746956in}{2.665697in}}{\pgfqpoint{2.737280in}{2.661689in}}{\pgfqpoint{2.730148in}{2.654557in}}%
\pgfpathcurveto{\pgfqpoint{2.723015in}{2.647424in}}{\pgfqpoint{2.719007in}{2.637748in}}{\pgfqpoint{2.719007in}{2.627661in}}%
\pgfpathcurveto{\pgfqpoint{2.719007in}{2.617574in}}{\pgfqpoint{2.723015in}{2.607898in}}{\pgfqpoint{2.730148in}{2.600765in}}%
\pgfpathcurveto{\pgfqpoint{2.737280in}{2.593632in}}{\pgfqpoint{2.746956in}{2.589625in}}{\pgfqpoint{2.757043in}{2.589625in}}%
\pgfpathclose%
\pgfusepath{stroke,fill}%
\end{pgfscope}%
\begin{pgfscope}%
\pgfpathrectangle{\pgfqpoint{0.800000in}{1.363959in}}{\pgfqpoint{3.968000in}{2.024082in}} %
\pgfusepath{clip}%
\pgfsetbuttcap%
\pgfsetroundjoin%
\definecolor{currentfill}{rgb}{0.121569,0.466667,0.705882}%
\pgfsetfillcolor{currentfill}%
\pgfsetlinewidth{1.003750pt}%
\definecolor{currentstroke}{rgb}{0.121569,0.466667,0.705882}%
\pgfsetstrokecolor{currentstroke}%
\pgfsetdash{}{0pt}%
\pgfpathmoveto{\pgfqpoint{3.558584in}{2.375503in}}%
\pgfpathcurveto{\pgfqpoint{3.568671in}{2.375503in}}{\pgfqpoint{3.578347in}{2.379511in}}{\pgfqpoint{3.585480in}{2.386643in}}%
\pgfpathcurveto{\pgfqpoint{3.592613in}{2.393776in}}{\pgfqpoint{3.596620in}{2.403452in}}{\pgfqpoint{3.596620in}{2.413539in}}%
\pgfpathcurveto{\pgfqpoint{3.596620in}{2.423626in}}{\pgfqpoint{3.592613in}{2.433302in}}{\pgfqpoint{3.585480in}{2.440435in}}%
\pgfpathcurveto{\pgfqpoint{3.578347in}{2.447568in}}{\pgfqpoint{3.568671in}{2.451575in}}{\pgfqpoint{3.558584in}{2.451575in}}%
\pgfpathcurveto{\pgfqpoint{3.548497in}{2.451575in}}{\pgfqpoint{3.538821in}{2.447568in}}{\pgfqpoint{3.531688in}{2.440435in}}%
\pgfpathcurveto{\pgfqpoint{3.524556in}{2.433302in}}{\pgfqpoint{3.520548in}{2.423626in}}{\pgfqpoint{3.520548in}{2.413539in}}%
\pgfpathcurveto{\pgfqpoint{3.520548in}{2.403452in}}{\pgfqpoint{3.524556in}{2.393776in}}{\pgfqpoint{3.531688in}{2.386643in}}%
\pgfpathcurveto{\pgfqpoint{3.538821in}{2.379511in}}{\pgfqpoint{3.548497in}{2.375503in}}{\pgfqpoint{3.558584in}{2.375503in}}%
\pgfpathclose%
\pgfusepath{stroke,fill}%
\end{pgfscope}%
\begin{pgfscope}%
\pgfpathrectangle{\pgfqpoint{0.800000in}{1.363959in}}{\pgfqpoint{3.968000in}{2.024082in}} %
\pgfusepath{clip}%
\pgfsetbuttcap%
\pgfsetroundjoin%
\definecolor{currentfill}{rgb}{0.121569,0.466667,0.705882}%
\pgfsetfillcolor{currentfill}%
\pgfsetlinewidth{1.003750pt}%
\definecolor{currentstroke}{rgb}{0.121569,0.466667,0.705882}%
\pgfsetstrokecolor{currentstroke}%
\pgfsetdash{}{0pt}%
\pgfpathmoveto{\pgfqpoint{2.890724in}{3.121890in}}%
\pgfpathcurveto{\pgfqpoint{2.900811in}{3.121890in}}{\pgfqpoint{2.910486in}{3.125897in}}{\pgfqpoint{2.917619in}{3.133030in}}%
\pgfpathcurveto{\pgfqpoint{2.924752in}{3.140163in}}{\pgfqpoint{2.928760in}{3.149839in}}{\pgfqpoint{2.928760in}{3.159926in}}%
\pgfpathcurveto{\pgfqpoint{2.928760in}{3.170013in}}{\pgfqpoint{2.924752in}{3.179689in}}{\pgfqpoint{2.917619in}{3.186822in}}%
\pgfpathcurveto{\pgfqpoint{2.910486in}{3.193955in}}{\pgfqpoint{2.900811in}{3.197962in}}{\pgfqpoint{2.890724in}{3.197962in}}%
\pgfpathcurveto{\pgfqpoint{2.880636in}{3.197962in}}{\pgfqpoint{2.870961in}{3.193955in}}{\pgfqpoint{2.863828in}{3.186822in}}%
\pgfpathcurveto{\pgfqpoint{2.856695in}{3.179689in}}{\pgfqpoint{2.852687in}{3.170013in}}{\pgfqpoint{2.852687in}{3.159926in}}%
\pgfpathcurveto{\pgfqpoint{2.852687in}{3.149839in}}{\pgfqpoint{2.856695in}{3.140163in}}{\pgfqpoint{2.863828in}{3.133030in}}%
\pgfpathcurveto{\pgfqpoint{2.870961in}{3.125897in}}{\pgfqpoint{2.880636in}{3.121890in}}{\pgfqpoint{2.890724in}{3.121890in}}%
\pgfpathclose%
\pgfusepath{stroke,fill}%
\end{pgfscope}%
\begin{pgfscope}%
\pgfpathrectangle{\pgfqpoint{0.800000in}{1.363959in}}{\pgfqpoint{3.968000in}{2.024082in}} %
\pgfusepath{clip}%
\pgfsetbuttcap%
\pgfsetroundjoin%
\definecolor{currentfill}{rgb}{0.121569,0.466667,0.705882}%
\pgfsetfillcolor{currentfill}%
\pgfsetlinewidth{1.003750pt}%
\definecolor{currentstroke}{rgb}{0.121569,0.466667,0.705882}%
\pgfsetstrokecolor{currentstroke}%
\pgfsetdash{}{0pt}%
\pgfpathmoveto{\pgfqpoint{3.481746in}{1.887956in}}%
\pgfpathcurveto{\pgfqpoint{3.491833in}{1.887956in}}{\pgfqpoint{3.501509in}{1.891964in}}{\pgfqpoint{3.508642in}{1.899097in}}%
\pgfpathcurveto{\pgfqpoint{3.515775in}{1.906230in}}{\pgfqpoint{3.519782in}{1.915905in}}{\pgfqpoint{3.519782in}{1.925992in}}%
\pgfpathcurveto{\pgfqpoint{3.519782in}{1.936080in}}{\pgfqpoint{3.515775in}{1.945755in}}{\pgfqpoint{3.508642in}{1.952888in}}%
\pgfpathcurveto{\pgfqpoint{3.501509in}{1.960021in}}{\pgfqpoint{3.491833in}{1.964029in}}{\pgfqpoint{3.481746in}{1.964029in}}%
\pgfpathcurveto{\pgfqpoint{3.471659in}{1.964029in}}{\pgfqpoint{3.461983in}{1.960021in}}{\pgfqpoint{3.454850in}{1.952888in}}%
\pgfpathcurveto{\pgfqpoint{3.447718in}{1.945755in}}{\pgfqpoint{3.443710in}{1.936080in}}{\pgfqpoint{3.443710in}{1.925992in}}%
\pgfpathcurveto{\pgfqpoint{3.443710in}{1.915905in}}{\pgfqpoint{3.447718in}{1.906230in}}{\pgfqpoint{3.454850in}{1.899097in}}%
\pgfpathcurveto{\pgfqpoint{3.461983in}{1.891964in}}{\pgfqpoint{3.471659in}{1.887956in}}{\pgfqpoint{3.481746in}{1.887956in}}%
\pgfpathclose%
\pgfusepath{stroke,fill}%
\end{pgfscope}%
\begin{pgfscope}%
\pgfpathrectangle{\pgfqpoint{0.800000in}{1.363959in}}{\pgfqpoint{3.968000in}{2.024082in}} %
\pgfusepath{clip}%
\pgfsetbuttcap%
\pgfsetroundjoin%
\definecolor{currentfill}{rgb}{0.121569,0.466667,0.705882}%
\pgfsetfillcolor{currentfill}%
\pgfsetlinewidth{1.003750pt}%
\definecolor{currentstroke}{rgb}{0.121569,0.466667,0.705882}%
\pgfsetstrokecolor{currentstroke}%
\pgfsetdash{}{0pt}%
\pgfpathmoveto{\pgfqpoint{2.056403in}{2.767959in}}%
\pgfpathcurveto{\pgfqpoint{2.066491in}{2.767959in}}{\pgfqpoint{2.076166in}{2.771967in}}{\pgfqpoint{2.083299in}{2.779100in}}%
\pgfpathcurveto{\pgfqpoint{2.090432in}{2.786233in}}{\pgfqpoint{2.094440in}{2.795908in}}{\pgfqpoint{2.094440in}{2.805995in}}%
\pgfpathcurveto{\pgfqpoint{2.094440in}{2.816083in}}{\pgfqpoint{2.090432in}{2.825758in}}{\pgfqpoint{2.083299in}{2.832891in}}%
\pgfpathcurveto{\pgfqpoint{2.076166in}{2.840024in}}{\pgfqpoint{2.066491in}{2.844032in}}{\pgfqpoint{2.056403in}{2.844032in}}%
\pgfpathcurveto{\pgfqpoint{2.046316in}{2.844032in}}{\pgfqpoint{2.036641in}{2.840024in}}{\pgfqpoint{2.029508in}{2.832891in}}%
\pgfpathcurveto{\pgfqpoint{2.022375in}{2.825758in}}{\pgfqpoint{2.018367in}{2.816083in}}{\pgfqpoint{2.018367in}{2.805995in}}%
\pgfpathcurveto{\pgfqpoint{2.018367in}{2.795908in}}{\pgfqpoint{2.022375in}{2.786233in}}{\pgfqpoint{2.029508in}{2.779100in}}%
\pgfpathcurveto{\pgfqpoint{2.036641in}{2.771967in}}{\pgfqpoint{2.046316in}{2.767959in}}{\pgfqpoint{2.056403in}{2.767959in}}%
\pgfpathclose%
\pgfusepath{stroke,fill}%
\end{pgfscope}%
\begin{pgfscope}%
\pgfpathrectangle{\pgfqpoint{0.800000in}{1.363959in}}{\pgfqpoint{3.968000in}{2.024082in}} %
\pgfusepath{clip}%
\pgfsetbuttcap%
\pgfsetroundjoin%
\definecolor{currentfill}{rgb}{0.121569,0.466667,0.705882}%
\pgfsetfillcolor{currentfill}%
\pgfsetlinewidth{1.003750pt}%
\definecolor{currentstroke}{rgb}{0.121569,0.466667,0.705882}%
\pgfsetstrokecolor{currentstroke}%
\pgfsetdash{}{0pt}%
\pgfpathmoveto{\pgfqpoint{3.302655in}{2.837185in}}%
\pgfpathcurveto{\pgfqpoint{3.312742in}{2.837185in}}{\pgfqpoint{3.322418in}{2.841193in}}{\pgfqpoint{3.329551in}{2.848326in}}%
\pgfpathcurveto{\pgfqpoint{3.336684in}{2.855459in}}{\pgfqpoint{3.340691in}{2.865134in}}{\pgfqpoint{3.340691in}{2.875221in}}%
\pgfpathcurveto{\pgfqpoint{3.340691in}{2.885309in}}{\pgfqpoint{3.336684in}{2.894984in}}{\pgfqpoint{3.329551in}{2.902117in}}%
\pgfpathcurveto{\pgfqpoint{3.322418in}{2.909250in}}{\pgfqpoint{3.312742in}{2.913258in}}{\pgfqpoint{3.302655in}{2.913258in}}%
\pgfpathcurveto{\pgfqpoint{3.292568in}{2.913258in}}{\pgfqpoint{3.282892in}{2.909250in}}{\pgfqpoint{3.275759in}{2.902117in}}%
\pgfpathcurveto{\pgfqpoint{3.268626in}{2.894984in}}{\pgfqpoint{3.264619in}{2.885309in}}{\pgfqpoint{3.264619in}{2.875221in}}%
\pgfpathcurveto{\pgfqpoint{3.264619in}{2.865134in}}{\pgfqpoint{3.268626in}{2.855459in}}{\pgfqpoint{3.275759in}{2.848326in}}%
\pgfpathcurveto{\pgfqpoint{3.282892in}{2.841193in}}{\pgfqpoint{3.292568in}{2.837185in}}{\pgfqpoint{3.302655in}{2.837185in}}%
\pgfpathclose%
\pgfusepath{stroke,fill}%
\end{pgfscope}%
\begin{pgfscope}%
\pgfpathrectangle{\pgfqpoint{0.800000in}{1.363959in}}{\pgfqpoint{3.968000in}{2.024082in}} %
\pgfusepath{clip}%
\pgfsetbuttcap%
\pgfsetroundjoin%
\definecolor{currentfill}{rgb}{0.121569,0.466667,0.705882}%
\pgfsetfillcolor{currentfill}%
\pgfsetlinewidth{1.003750pt}%
\definecolor{currentstroke}{rgb}{0.121569,0.466667,0.705882}%
\pgfsetstrokecolor{currentstroke}%
\pgfsetdash{}{0pt}%
\pgfpathmoveto{\pgfqpoint{2.538404in}{3.090194in}}%
\pgfpathcurveto{\pgfqpoint{2.548492in}{3.090194in}}{\pgfqpoint{2.558167in}{3.094202in}}{\pgfqpoint{2.565300in}{3.101334in}}%
\pgfpathcurveto{\pgfqpoint{2.572433in}{3.108467in}}{\pgfqpoint{2.576441in}{3.118143in}}{\pgfqpoint{2.576441in}{3.128230in}}%
\pgfpathcurveto{\pgfqpoint{2.576441in}{3.138317in}}{\pgfqpoint{2.572433in}{3.147993in}}{\pgfqpoint{2.565300in}{3.155126in}}%
\pgfpathcurveto{\pgfqpoint{2.558167in}{3.162259in}}{\pgfqpoint{2.548492in}{3.166266in}}{\pgfqpoint{2.538404in}{3.166266in}}%
\pgfpathcurveto{\pgfqpoint{2.528317in}{3.166266in}}{\pgfqpoint{2.518641in}{3.162259in}}{\pgfqpoint{2.511509in}{3.155126in}}%
\pgfpathcurveto{\pgfqpoint{2.504376in}{3.147993in}}{\pgfqpoint{2.500368in}{3.138317in}}{\pgfqpoint{2.500368in}{3.128230in}}%
\pgfpathcurveto{\pgfqpoint{2.500368in}{3.118143in}}{\pgfqpoint{2.504376in}{3.108467in}}{\pgfqpoint{2.511509in}{3.101334in}}%
\pgfpathcurveto{\pgfqpoint{2.518641in}{3.094202in}}{\pgfqpoint{2.528317in}{3.090194in}}{\pgfqpoint{2.538404in}{3.090194in}}%
\pgfpathclose%
\pgfusepath{stroke,fill}%
\end{pgfscope}%
\begin{pgfscope}%
\pgfpathrectangle{\pgfqpoint{0.800000in}{1.363959in}}{\pgfqpoint{3.968000in}{2.024082in}} %
\pgfusepath{clip}%
\pgfsetbuttcap%
\pgfsetroundjoin%
\definecolor{currentfill}{rgb}{0.121569,0.466667,0.705882}%
\pgfsetfillcolor{currentfill}%
\pgfsetlinewidth{1.003750pt}%
\definecolor{currentstroke}{rgb}{0.121569,0.466667,0.705882}%
\pgfsetstrokecolor{currentstroke}%
\pgfsetdash{}{0pt}%
\pgfpathmoveto{\pgfqpoint{2.445304in}{1.833984in}}%
\pgfpathcurveto{\pgfqpoint{2.455391in}{1.833984in}}{\pgfqpoint{2.465067in}{1.837991in}}{\pgfqpoint{2.472200in}{1.845124in}}%
\pgfpathcurveto{\pgfqpoint{2.479333in}{1.852257in}}{\pgfqpoint{2.483340in}{1.861933in}}{\pgfqpoint{2.483340in}{1.872020in}}%
\pgfpathcurveto{\pgfqpoint{2.483340in}{1.882107in}}{\pgfqpoint{2.479333in}{1.891783in}}{\pgfqpoint{2.472200in}{1.898916in}}%
\pgfpathcurveto{\pgfqpoint{2.465067in}{1.906048in}}{\pgfqpoint{2.455391in}{1.910056in}}{\pgfqpoint{2.445304in}{1.910056in}}%
\pgfpathcurveto{\pgfqpoint{2.435217in}{1.910056in}}{\pgfqpoint{2.425541in}{1.906048in}}{\pgfqpoint{2.418408in}{1.898916in}}%
\pgfpathcurveto{\pgfqpoint{2.411276in}{1.891783in}}{\pgfqpoint{2.407268in}{1.882107in}}{\pgfqpoint{2.407268in}{1.872020in}}%
\pgfpathcurveto{\pgfqpoint{2.407268in}{1.861933in}}{\pgfqpoint{2.411276in}{1.852257in}}{\pgfqpoint{2.418408in}{1.845124in}}%
\pgfpathcurveto{\pgfqpoint{2.425541in}{1.837991in}}{\pgfqpoint{2.435217in}{1.833984in}}{\pgfqpoint{2.445304in}{1.833984in}}%
\pgfpathclose%
\pgfusepath{stroke,fill}%
\end{pgfscope}%
\begin{pgfscope}%
\pgfpathrectangle{\pgfqpoint{0.800000in}{1.363959in}}{\pgfqpoint{3.968000in}{2.024082in}} %
\pgfusepath{clip}%
\pgfsetbuttcap%
\pgfsetroundjoin%
\definecolor{currentfill}{rgb}{0.121569,0.466667,0.705882}%
\pgfsetfillcolor{currentfill}%
\pgfsetlinewidth{1.003750pt}%
\definecolor{currentstroke}{rgb}{0.121569,0.466667,0.705882}%
\pgfsetstrokecolor{currentstroke}%
\pgfsetdash{}{0pt}%
\pgfpathmoveto{\pgfqpoint{3.197116in}{2.547314in}}%
\pgfpathcurveto{\pgfqpoint{3.207204in}{2.547314in}}{\pgfqpoint{3.216879in}{2.551322in}}{\pgfqpoint{3.224012in}{2.558455in}}%
\pgfpathcurveto{\pgfqpoint{3.231145in}{2.565588in}}{\pgfqpoint{3.235153in}{2.575263in}}{\pgfqpoint{3.235153in}{2.585351in}}%
\pgfpathcurveto{\pgfqpoint{3.235153in}{2.595438in}}{\pgfqpoint{3.231145in}{2.605113in}}{\pgfqpoint{3.224012in}{2.612246in}}%
\pgfpathcurveto{\pgfqpoint{3.216879in}{2.619379in}}{\pgfqpoint{3.207204in}{2.623387in}}{\pgfqpoint{3.197116in}{2.623387in}}%
\pgfpathcurveto{\pgfqpoint{3.187029in}{2.623387in}}{\pgfqpoint{3.177354in}{2.619379in}}{\pgfqpoint{3.170221in}{2.612246in}}%
\pgfpathcurveto{\pgfqpoint{3.163088in}{2.605113in}}{\pgfqpoint{3.159080in}{2.595438in}}{\pgfqpoint{3.159080in}{2.585351in}}%
\pgfpathcurveto{\pgfqpoint{3.159080in}{2.575263in}}{\pgfqpoint{3.163088in}{2.565588in}}{\pgfqpoint{3.170221in}{2.558455in}}%
\pgfpathcurveto{\pgfqpoint{3.177354in}{2.551322in}}{\pgfqpoint{3.187029in}{2.547314in}}{\pgfqpoint{3.197116in}{2.547314in}}%
\pgfpathclose%
\pgfusepath{stroke,fill}%
\end{pgfscope}%
\begin{pgfscope}%
\pgfpathrectangle{\pgfqpoint{0.800000in}{1.363959in}}{\pgfqpoint{3.968000in}{2.024082in}} %
\pgfusepath{clip}%
\pgfsetbuttcap%
\pgfsetroundjoin%
\definecolor{currentfill}{rgb}{0.121569,0.466667,0.705882}%
\pgfsetfillcolor{currentfill}%
\pgfsetlinewidth{1.003750pt}%
\definecolor{currentstroke}{rgb}{0.121569,0.466667,0.705882}%
\pgfsetstrokecolor{currentstroke}%
\pgfsetdash{}{0pt}%
\pgfpathmoveto{\pgfqpoint{2.959959in}{2.435161in}}%
\pgfpathcurveto{\pgfqpoint{2.970047in}{2.435161in}}{\pgfqpoint{2.979722in}{2.439169in}}{\pgfqpoint{2.986855in}{2.446302in}}%
\pgfpathcurveto{\pgfqpoint{2.993988in}{2.453435in}}{\pgfqpoint{2.997995in}{2.463110in}}{\pgfqpoint{2.997995in}{2.473198in}}%
\pgfpathcurveto{\pgfqpoint{2.997995in}{2.483285in}}{\pgfqpoint{2.993988in}{2.492960in}}{\pgfqpoint{2.986855in}{2.500093in}}%
\pgfpathcurveto{\pgfqpoint{2.979722in}{2.507226in}}{\pgfqpoint{2.970047in}{2.511234in}}{\pgfqpoint{2.959959in}{2.511234in}}%
\pgfpathcurveto{\pgfqpoint{2.949872in}{2.511234in}}{\pgfqpoint{2.940196in}{2.507226in}}{\pgfqpoint{2.933063in}{2.500093in}}%
\pgfpathcurveto{\pgfqpoint{2.925931in}{2.492960in}}{\pgfqpoint{2.921923in}{2.483285in}}{\pgfqpoint{2.921923in}{2.473198in}}%
\pgfpathcurveto{\pgfqpoint{2.921923in}{2.463110in}}{\pgfqpoint{2.925931in}{2.453435in}}{\pgfqpoint{2.933063in}{2.446302in}}%
\pgfpathcurveto{\pgfqpoint{2.940196in}{2.439169in}}{\pgfqpoint{2.949872in}{2.435161in}}{\pgfqpoint{2.959959in}{2.435161in}}%
\pgfpathclose%
\pgfusepath{stroke,fill}%
\end{pgfscope}%
\begin{pgfscope}%
\pgfpathrectangle{\pgfqpoint{0.800000in}{1.363959in}}{\pgfqpoint{3.968000in}{2.024082in}} %
\pgfusepath{clip}%
\pgfsetbuttcap%
\pgfsetroundjoin%
\definecolor{currentfill}{rgb}{0.121569,0.466667,0.705882}%
\pgfsetfillcolor{currentfill}%
\pgfsetlinewidth{1.003750pt}%
\definecolor{currentstroke}{rgb}{0.121569,0.466667,0.705882}%
\pgfsetstrokecolor{currentstroke}%
\pgfsetdash{}{0pt}%
\pgfpathmoveto{\pgfqpoint{3.197360in}{3.104726in}}%
\pgfpathcurveto{\pgfqpoint{3.207447in}{3.104726in}}{\pgfqpoint{3.217122in}{3.108733in}}{\pgfqpoint{3.224255in}{3.115866in}}%
\pgfpathcurveto{\pgfqpoint{3.231388in}{3.122999in}}{\pgfqpoint{3.235396in}{3.132675in}}{\pgfqpoint{3.235396in}{3.142762in}}%
\pgfpathcurveto{\pgfqpoint{3.235396in}{3.152849in}}{\pgfqpoint{3.231388in}{3.162525in}}{\pgfqpoint{3.224255in}{3.169658in}}%
\pgfpathcurveto{\pgfqpoint{3.217122in}{3.176790in}}{\pgfqpoint{3.207447in}{3.180798in}}{\pgfqpoint{3.197360in}{3.180798in}}%
\pgfpathcurveto{\pgfqpoint{3.187272in}{3.180798in}}{\pgfqpoint{3.177597in}{3.176790in}}{\pgfqpoint{3.170464in}{3.169658in}}%
\pgfpathcurveto{\pgfqpoint{3.163331in}{3.162525in}}{\pgfqpoint{3.159323in}{3.152849in}}{\pgfqpoint{3.159323in}{3.142762in}}%
\pgfpathcurveto{\pgfqpoint{3.159323in}{3.132675in}}{\pgfqpoint{3.163331in}{3.122999in}}{\pgfqpoint{3.170464in}{3.115866in}}%
\pgfpathcurveto{\pgfqpoint{3.177597in}{3.108733in}}{\pgfqpoint{3.187272in}{3.104726in}}{\pgfqpoint{3.197360in}{3.104726in}}%
\pgfpathclose%
\pgfusepath{stroke,fill}%
\end{pgfscope}%
\begin{pgfscope}%
\pgfpathrectangle{\pgfqpoint{0.800000in}{1.363959in}}{\pgfqpoint{3.968000in}{2.024082in}} %
\pgfusepath{clip}%
\pgfsetbuttcap%
\pgfsetroundjoin%
\definecolor{currentfill}{rgb}{0.121569,0.466667,0.705882}%
\pgfsetfillcolor{currentfill}%
\pgfsetlinewidth{1.003750pt}%
\definecolor{currentstroke}{rgb}{0.121569,0.466667,0.705882}%
\pgfsetstrokecolor{currentstroke}%
\pgfsetdash{}{0pt}%
\pgfpathmoveto{\pgfqpoint{2.447417in}{2.593552in}}%
\pgfpathcurveto{\pgfqpoint{2.457504in}{2.593552in}}{\pgfqpoint{2.467179in}{2.597559in}}{\pgfqpoint{2.474312in}{2.604692in}}%
\pgfpathcurveto{\pgfqpoint{2.481445in}{2.611825in}}{\pgfqpoint{2.485453in}{2.621501in}}{\pgfqpoint{2.485453in}{2.631588in}}%
\pgfpathcurveto{\pgfqpoint{2.485453in}{2.641675in}}{\pgfqpoint{2.481445in}{2.651351in}}{\pgfqpoint{2.474312in}{2.658484in}}%
\pgfpathcurveto{\pgfqpoint{2.467179in}{2.665617in}}{\pgfqpoint{2.457504in}{2.669624in}}{\pgfqpoint{2.447417in}{2.669624in}}%
\pgfpathcurveto{\pgfqpoint{2.437329in}{2.669624in}}{\pgfqpoint{2.427654in}{2.665617in}}{\pgfqpoint{2.420521in}{2.658484in}}%
\pgfpathcurveto{\pgfqpoint{2.413388in}{2.651351in}}{\pgfqpoint{2.409380in}{2.641675in}}{\pgfqpoint{2.409380in}{2.631588in}}%
\pgfpathcurveto{\pgfqpoint{2.409380in}{2.621501in}}{\pgfqpoint{2.413388in}{2.611825in}}{\pgfqpoint{2.420521in}{2.604692in}}%
\pgfpathcurveto{\pgfqpoint{2.427654in}{2.597559in}}{\pgfqpoint{2.437329in}{2.593552in}}{\pgfqpoint{2.447417in}{2.593552in}}%
\pgfpathclose%
\pgfusepath{stroke,fill}%
\end{pgfscope}%
\begin{pgfscope}%
\pgfpathrectangle{\pgfqpoint{0.800000in}{1.363959in}}{\pgfqpoint{3.968000in}{2.024082in}} %
\pgfusepath{clip}%
\pgfsetbuttcap%
\pgfsetroundjoin%
\definecolor{currentfill}{rgb}{0.121569,0.466667,0.705882}%
\pgfsetfillcolor{currentfill}%
\pgfsetlinewidth{1.003750pt}%
\definecolor{currentstroke}{rgb}{0.121569,0.466667,0.705882}%
\pgfsetstrokecolor{currentstroke}%
\pgfsetdash{}{0pt}%
\pgfpathmoveto{\pgfqpoint{2.300111in}{2.124313in}}%
\pgfpathcurveto{\pgfqpoint{2.310198in}{2.124313in}}{\pgfqpoint{2.319874in}{2.128321in}}{\pgfqpoint{2.327007in}{2.135454in}}%
\pgfpathcurveto{\pgfqpoint{2.334140in}{2.142587in}}{\pgfqpoint{2.338147in}{2.152262in}}{\pgfqpoint{2.338147in}{2.162350in}}%
\pgfpathcurveto{\pgfqpoint{2.338147in}{2.172437in}}{\pgfqpoint{2.334140in}{2.182112in}}{\pgfqpoint{2.327007in}{2.189245in}}%
\pgfpathcurveto{\pgfqpoint{2.319874in}{2.196378in}}{\pgfqpoint{2.310198in}{2.200386in}}{\pgfqpoint{2.300111in}{2.200386in}}%
\pgfpathcurveto{\pgfqpoint{2.290024in}{2.200386in}}{\pgfqpoint{2.280348in}{2.196378in}}{\pgfqpoint{2.273215in}{2.189245in}}%
\pgfpathcurveto{\pgfqpoint{2.266082in}{2.182112in}}{\pgfqpoint{2.262075in}{2.172437in}}{\pgfqpoint{2.262075in}{2.162350in}}%
\pgfpathcurveto{\pgfqpoint{2.262075in}{2.152262in}}{\pgfqpoint{2.266082in}{2.142587in}}{\pgfqpoint{2.273215in}{2.135454in}}%
\pgfpathcurveto{\pgfqpoint{2.280348in}{2.128321in}}{\pgfqpoint{2.290024in}{2.124313in}}{\pgfqpoint{2.300111in}{2.124313in}}%
\pgfpathclose%
\pgfusepath{stroke,fill}%
\end{pgfscope}%
\begin{pgfscope}%
\pgfpathrectangle{\pgfqpoint{0.800000in}{1.363959in}}{\pgfqpoint{3.968000in}{2.024082in}} %
\pgfusepath{clip}%
\pgfsetbuttcap%
\pgfsetroundjoin%
\definecolor{currentfill}{rgb}{0.121569,0.466667,0.705882}%
\pgfsetfillcolor{currentfill}%
\pgfsetlinewidth{1.003750pt}%
\definecolor{currentstroke}{rgb}{0.121569,0.466667,0.705882}%
\pgfsetstrokecolor{currentstroke}%
\pgfsetdash{}{0pt}%
\pgfpathmoveto{\pgfqpoint{3.467727in}{2.303353in}}%
\pgfpathcurveto{\pgfqpoint{3.477814in}{2.303353in}}{\pgfqpoint{3.487490in}{2.307360in}}{\pgfqpoint{3.494623in}{2.314493in}}%
\pgfpathcurveto{\pgfqpoint{3.501755in}{2.321626in}}{\pgfqpoint{3.505763in}{2.331302in}}{\pgfqpoint{3.505763in}{2.341389in}}%
\pgfpathcurveto{\pgfqpoint{3.505763in}{2.351476in}}{\pgfqpoint{3.501755in}{2.361152in}}{\pgfqpoint{3.494623in}{2.368285in}}%
\pgfpathcurveto{\pgfqpoint{3.487490in}{2.375418in}}{\pgfqpoint{3.477814in}{2.379425in}}{\pgfqpoint{3.467727in}{2.379425in}}%
\pgfpathcurveto{\pgfqpoint{3.457639in}{2.379425in}}{\pgfqpoint{3.447964in}{2.375418in}}{\pgfqpoint{3.440831in}{2.368285in}}%
\pgfpathcurveto{\pgfqpoint{3.433698in}{2.361152in}}{\pgfqpoint{3.429691in}{2.351476in}}{\pgfqpoint{3.429691in}{2.341389in}}%
\pgfpathcurveto{\pgfqpoint{3.429691in}{2.331302in}}{\pgfqpoint{3.433698in}{2.321626in}}{\pgfqpoint{3.440831in}{2.314493in}}%
\pgfpathcurveto{\pgfqpoint{3.447964in}{2.307360in}}{\pgfqpoint{3.457639in}{2.303353in}}{\pgfqpoint{3.467727in}{2.303353in}}%
\pgfpathclose%
\pgfusepath{stroke,fill}%
\end{pgfscope}%
\begin{pgfscope}%
\pgfpathrectangle{\pgfqpoint{0.800000in}{1.363959in}}{\pgfqpoint{3.968000in}{2.024082in}} %
\pgfusepath{clip}%
\pgfsetbuttcap%
\pgfsetroundjoin%
\definecolor{currentfill}{rgb}{0.121569,0.466667,0.705882}%
\pgfsetfillcolor{currentfill}%
\pgfsetlinewidth{1.003750pt}%
\definecolor{currentstroke}{rgb}{0.121569,0.466667,0.705882}%
\pgfsetstrokecolor{currentstroke}%
\pgfsetdash{}{0pt}%
\pgfpathmoveto{\pgfqpoint{2.595137in}{2.524058in}}%
\pgfpathcurveto{\pgfqpoint{2.605224in}{2.524058in}}{\pgfqpoint{2.614900in}{2.528065in}}{\pgfqpoint{2.622033in}{2.535198in}}%
\pgfpathcurveto{\pgfqpoint{2.629166in}{2.542331in}}{\pgfqpoint{2.633173in}{2.552007in}}{\pgfqpoint{2.633173in}{2.562094in}}%
\pgfpathcurveto{\pgfqpoint{2.633173in}{2.572181in}}{\pgfqpoint{2.629166in}{2.581857in}}{\pgfqpoint{2.622033in}{2.588990in}}%
\pgfpathcurveto{\pgfqpoint{2.614900in}{2.596122in}}{\pgfqpoint{2.605224in}{2.600130in}}{\pgfqpoint{2.595137in}{2.600130in}}%
\pgfpathcurveto{\pgfqpoint{2.585050in}{2.600130in}}{\pgfqpoint{2.575374in}{2.596122in}}{\pgfqpoint{2.568241in}{2.588990in}}%
\pgfpathcurveto{\pgfqpoint{2.561108in}{2.581857in}}{\pgfqpoint{2.557101in}{2.572181in}}{\pgfqpoint{2.557101in}{2.562094in}}%
\pgfpathcurveto{\pgfqpoint{2.557101in}{2.552007in}}{\pgfqpoint{2.561108in}{2.542331in}}{\pgfqpoint{2.568241in}{2.535198in}}%
\pgfpathcurveto{\pgfqpoint{2.575374in}{2.528065in}}{\pgfqpoint{2.585050in}{2.524058in}}{\pgfqpoint{2.595137in}{2.524058in}}%
\pgfpathclose%
\pgfusepath{stroke,fill}%
\end{pgfscope}%
\begin{pgfscope}%
\pgfpathrectangle{\pgfqpoint{0.800000in}{1.363959in}}{\pgfqpoint{3.968000in}{2.024082in}} %
\pgfusepath{clip}%
\pgfsetbuttcap%
\pgfsetroundjoin%
\definecolor{currentfill}{rgb}{0.121569,0.466667,0.705882}%
\pgfsetfillcolor{currentfill}%
\pgfsetlinewidth{1.003750pt}%
\definecolor{currentstroke}{rgb}{0.121569,0.466667,0.705882}%
\pgfsetstrokecolor{currentstroke}%
\pgfsetdash{}{0pt}%
\pgfpathmoveto{\pgfqpoint{3.596348in}{2.421908in}}%
\pgfpathcurveto{\pgfqpoint{3.606435in}{2.421908in}}{\pgfqpoint{3.616111in}{2.425915in}}{\pgfqpoint{3.623244in}{2.433048in}}%
\pgfpathcurveto{\pgfqpoint{3.630377in}{2.440181in}}{\pgfqpoint{3.634384in}{2.449857in}}{\pgfqpoint{3.634384in}{2.459944in}}%
\pgfpathcurveto{\pgfqpoint{3.634384in}{2.470031in}}{\pgfqpoint{3.630377in}{2.479707in}}{\pgfqpoint{3.623244in}{2.486840in}}%
\pgfpathcurveto{\pgfqpoint{3.616111in}{2.493972in}}{\pgfqpoint{3.606435in}{2.497980in}}{\pgfqpoint{3.596348in}{2.497980in}}%
\pgfpathcurveto{\pgfqpoint{3.586261in}{2.497980in}}{\pgfqpoint{3.576585in}{2.493972in}}{\pgfqpoint{3.569452in}{2.486840in}}%
\pgfpathcurveto{\pgfqpoint{3.562320in}{2.479707in}}{\pgfqpoint{3.558312in}{2.470031in}}{\pgfqpoint{3.558312in}{2.459944in}}%
\pgfpathcurveto{\pgfqpoint{3.558312in}{2.449857in}}{\pgfqpoint{3.562320in}{2.440181in}}{\pgfqpoint{3.569452in}{2.433048in}}%
\pgfpathcurveto{\pgfqpoint{3.576585in}{2.425915in}}{\pgfqpoint{3.586261in}{2.421908in}}{\pgfqpoint{3.596348in}{2.421908in}}%
\pgfpathclose%
\pgfusepath{stroke,fill}%
\end{pgfscope}%
\begin{pgfscope}%
\pgfpathrectangle{\pgfqpoint{0.800000in}{1.363959in}}{\pgfqpoint{3.968000in}{2.024082in}} %
\pgfusepath{clip}%
\pgfsetbuttcap%
\pgfsetroundjoin%
\definecolor{currentfill}{rgb}{0.121569,0.466667,0.705882}%
\pgfsetfillcolor{currentfill}%
\pgfsetlinewidth{1.003750pt}%
\definecolor{currentstroke}{rgb}{0.121569,0.466667,0.705882}%
\pgfsetstrokecolor{currentstroke}%
\pgfsetdash{}{0pt}%
\pgfpathmoveto{\pgfqpoint{3.034142in}{3.124452in}}%
\pgfpathcurveto{\pgfqpoint{3.044229in}{3.124452in}}{\pgfqpoint{3.053905in}{3.128459in}}{\pgfqpoint{3.061038in}{3.135592in}}%
\pgfpathcurveto{\pgfqpoint{3.068170in}{3.142725in}}{\pgfqpoint{3.072178in}{3.152401in}}{\pgfqpoint{3.072178in}{3.162488in}}%
\pgfpathcurveto{\pgfqpoint{3.072178in}{3.172575in}}{\pgfqpoint{3.068170in}{3.182251in}}{\pgfqpoint{3.061038in}{3.189384in}}%
\pgfpathcurveto{\pgfqpoint{3.053905in}{3.196516in}}{\pgfqpoint{3.044229in}{3.200524in}}{\pgfqpoint{3.034142in}{3.200524in}}%
\pgfpathcurveto{\pgfqpoint{3.024055in}{3.200524in}}{\pgfqpoint{3.014379in}{3.196516in}}{\pgfqpoint{3.007246in}{3.189384in}}%
\pgfpathcurveto{\pgfqpoint{3.000113in}{3.182251in}}{\pgfqpoint{2.996106in}{3.172575in}}{\pgfqpoint{2.996106in}{3.162488in}}%
\pgfpathcurveto{\pgfqpoint{2.996106in}{3.152401in}}{\pgfqpoint{3.000113in}{3.142725in}}{\pgfqpoint{3.007246in}{3.135592in}}%
\pgfpathcurveto{\pgfqpoint{3.014379in}{3.128459in}}{\pgfqpoint{3.024055in}{3.124452in}}{\pgfqpoint{3.034142in}{3.124452in}}%
\pgfpathclose%
\pgfusepath{stroke,fill}%
\end{pgfscope}%
\begin{pgfscope}%
\pgfpathrectangle{\pgfqpoint{0.800000in}{1.363959in}}{\pgfqpoint{3.968000in}{2.024082in}} %
\pgfusepath{clip}%
\pgfsetbuttcap%
\pgfsetroundjoin%
\definecolor{currentfill}{rgb}{0.121569,0.466667,0.705882}%
\pgfsetfillcolor{currentfill}%
\pgfsetlinewidth{1.003750pt}%
\definecolor{currentstroke}{rgb}{0.121569,0.466667,0.705882}%
\pgfsetstrokecolor{currentstroke}%
\pgfsetdash{}{0pt}%
\pgfpathmoveto{\pgfqpoint{2.956272in}{3.110559in}}%
\pgfpathcurveto{\pgfqpoint{2.966360in}{3.110559in}}{\pgfqpoint{2.976035in}{3.114566in}}{\pgfqpoint{2.983168in}{3.121699in}}%
\pgfpathcurveto{\pgfqpoint{2.990301in}{3.128832in}}{\pgfqpoint{2.994309in}{3.138508in}}{\pgfqpoint{2.994309in}{3.148595in}}%
\pgfpathcurveto{\pgfqpoint{2.994309in}{3.158682in}}{\pgfqpoint{2.990301in}{3.168358in}}{\pgfqpoint{2.983168in}{3.175491in}}%
\pgfpathcurveto{\pgfqpoint{2.976035in}{3.182624in}}{\pgfqpoint{2.966360in}{3.186631in}}{\pgfqpoint{2.956272in}{3.186631in}}%
\pgfpathcurveto{\pgfqpoint{2.946185in}{3.186631in}}{\pgfqpoint{2.936510in}{3.182624in}}{\pgfqpoint{2.929377in}{3.175491in}}%
\pgfpathcurveto{\pgfqpoint{2.922244in}{3.168358in}}{\pgfqpoint{2.918236in}{3.158682in}}{\pgfqpoint{2.918236in}{3.148595in}}%
\pgfpathcurveto{\pgfqpoint{2.918236in}{3.138508in}}{\pgfqpoint{2.922244in}{3.128832in}}{\pgfqpoint{2.929377in}{3.121699in}}%
\pgfpathcurveto{\pgfqpoint{2.936510in}{3.114566in}}{\pgfqpoint{2.946185in}{3.110559in}}{\pgfqpoint{2.956272in}{3.110559in}}%
\pgfpathclose%
\pgfusepath{stroke,fill}%
\end{pgfscope}%
\begin{pgfscope}%
\pgfpathrectangle{\pgfqpoint{0.800000in}{1.363959in}}{\pgfqpoint{3.968000in}{2.024082in}} %
\pgfusepath{clip}%
\pgfsetbuttcap%
\pgfsetroundjoin%
\definecolor{currentfill}{rgb}{0.121569,0.466667,0.705882}%
\pgfsetfillcolor{currentfill}%
\pgfsetlinewidth{1.003750pt}%
\definecolor{currentstroke}{rgb}{0.121569,0.466667,0.705882}%
\pgfsetstrokecolor{currentstroke}%
\pgfsetdash{}{0pt}%
\pgfpathmoveto{\pgfqpoint{2.010646in}{2.133394in}}%
\pgfpathcurveto{\pgfqpoint{2.020734in}{2.133394in}}{\pgfqpoint{2.030409in}{2.137401in}}{\pgfqpoint{2.037542in}{2.144534in}}%
\pgfpathcurveto{\pgfqpoint{2.044675in}{2.151667in}}{\pgfqpoint{2.048683in}{2.161342in}}{\pgfqpoint{2.048683in}{2.171430in}}%
\pgfpathcurveto{\pgfqpoint{2.048683in}{2.181517in}}{\pgfqpoint{2.044675in}{2.191193in}}{\pgfqpoint{2.037542in}{2.198326in}}%
\pgfpathcurveto{\pgfqpoint{2.030409in}{2.205458in}}{\pgfqpoint{2.020734in}{2.209466in}}{\pgfqpoint{2.010646in}{2.209466in}}%
\pgfpathcurveto{\pgfqpoint{2.000559in}{2.209466in}}{\pgfqpoint{1.990883in}{2.205458in}}{\pgfqpoint{1.983751in}{2.198326in}}%
\pgfpathcurveto{\pgfqpoint{1.976618in}{2.191193in}}{\pgfqpoint{1.972610in}{2.181517in}}{\pgfqpoint{1.972610in}{2.171430in}}%
\pgfpathcurveto{\pgfqpoint{1.972610in}{2.161342in}}{\pgfqpoint{1.976618in}{2.151667in}}{\pgfqpoint{1.983751in}{2.144534in}}%
\pgfpathcurveto{\pgfqpoint{1.990883in}{2.137401in}}{\pgfqpoint{2.000559in}{2.133394in}}{\pgfqpoint{2.010646in}{2.133394in}}%
\pgfpathclose%
\pgfusepath{stroke,fill}%
\end{pgfscope}%
\begin{pgfscope}%
\pgfpathrectangle{\pgfqpoint{0.800000in}{1.363959in}}{\pgfqpoint{3.968000in}{2.024082in}} %
\pgfusepath{clip}%
\pgfsetbuttcap%
\pgfsetroundjoin%
\definecolor{currentfill}{rgb}{0.121569,0.466667,0.705882}%
\pgfsetfillcolor{currentfill}%
\pgfsetlinewidth{1.003750pt}%
\definecolor{currentstroke}{rgb}{0.121569,0.466667,0.705882}%
\pgfsetstrokecolor{currentstroke}%
\pgfsetdash{}{0pt}%
\pgfpathmoveto{\pgfqpoint{2.635840in}{1.721306in}}%
\pgfpathcurveto{\pgfqpoint{2.645927in}{1.721306in}}{\pgfqpoint{2.655603in}{1.725314in}}{\pgfqpoint{2.662736in}{1.732447in}}%
\pgfpathcurveto{\pgfqpoint{2.669869in}{1.739579in}}{\pgfqpoint{2.673876in}{1.749255in}}{\pgfqpoint{2.673876in}{1.759342in}}%
\pgfpathcurveto{\pgfqpoint{2.673876in}{1.769430in}}{\pgfqpoint{2.669869in}{1.779105in}}{\pgfqpoint{2.662736in}{1.786238in}}%
\pgfpathcurveto{\pgfqpoint{2.655603in}{1.793371in}}{\pgfqpoint{2.645927in}{1.797379in}}{\pgfqpoint{2.635840in}{1.797379in}}%
\pgfpathcurveto{\pgfqpoint{2.625753in}{1.797379in}}{\pgfqpoint{2.616077in}{1.793371in}}{\pgfqpoint{2.608944in}{1.786238in}}%
\pgfpathcurveto{\pgfqpoint{2.601812in}{1.779105in}}{\pgfqpoint{2.597804in}{1.769430in}}{\pgfqpoint{2.597804in}{1.759342in}}%
\pgfpathcurveto{\pgfqpoint{2.597804in}{1.749255in}}{\pgfqpoint{2.601812in}{1.739579in}}{\pgfqpoint{2.608944in}{1.732447in}}%
\pgfpathcurveto{\pgfqpoint{2.616077in}{1.725314in}}{\pgfqpoint{2.625753in}{1.721306in}}{\pgfqpoint{2.635840in}{1.721306in}}%
\pgfpathclose%
\pgfusepath{stroke,fill}%
\end{pgfscope}%
\begin{pgfscope}%
\pgfpathrectangle{\pgfqpoint{0.800000in}{1.363959in}}{\pgfqpoint{3.968000in}{2.024082in}} %
\pgfusepath{clip}%
\pgfsetbuttcap%
\pgfsetroundjoin%
\definecolor{currentfill}{rgb}{0.121569,0.466667,0.705882}%
\pgfsetfillcolor{currentfill}%
\pgfsetlinewidth{1.003750pt}%
\definecolor{currentstroke}{rgb}{0.121569,0.466667,0.705882}%
\pgfsetstrokecolor{currentstroke}%
\pgfsetdash{}{0pt}%
\pgfpathmoveto{\pgfqpoint{2.875492in}{1.632468in}}%
\pgfpathcurveto{\pgfqpoint{2.885580in}{1.632468in}}{\pgfqpoint{2.895255in}{1.636475in}}{\pgfqpoint{2.902388in}{1.643608in}}%
\pgfpathcurveto{\pgfqpoint{2.909521in}{1.650741in}}{\pgfqpoint{2.913529in}{1.660417in}}{\pgfqpoint{2.913529in}{1.670504in}}%
\pgfpathcurveto{\pgfqpoint{2.913529in}{1.680591in}}{\pgfqpoint{2.909521in}{1.690267in}}{\pgfqpoint{2.902388in}{1.697400in}}%
\pgfpathcurveto{\pgfqpoint{2.895255in}{1.704533in}}{\pgfqpoint{2.885580in}{1.708540in}}{\pgfqpoint{2.875492in}{1.708540in}}%
\pgfpathcurveto{\pgfqpoint{2.865405in}{1.708540in}}{\pgfqpoint{2.855730in}{1.704533in}}{\pgfqpoint{2.848597in}{1.697400in}}%
\pgfpathcurveto{\pgfqpoint{2.841464in}{1.690267in}}{\pgfqpoint{2.837456in}{1.680591in}}{\pgfqpoint{2.837456in}{1.670504in}}%
\pgfpathcurveto{\pgfqpoint{2.837456in}{1.660417in}}{\pgfqpoint{2.841464in}{1.650741in}}{\pgfqpoint{2.848597in}{1.643608in}}%
\pgfpathcurveto{\pgfqpoint{2.855730in}{1.636475in}}{\pgfqpoint{2.865405in}{1.632468in}}{\pgfqpoint{2.875492in}{1.632468in}}%
\pgfpathclose%
\pgfusepath{stroke,fill}%
\end{pgfscope}%
\begin{pgfscope}%
\pgfpathrectangle{\pgfqpoint{0.800000in}{1.363959in}}{\pgfqpoint{3.968000in}{2.024082in}} %
\pgfusepath{clip}%
\pgfsetbuttcap%
\pgfsetroundjoin%
\definecolor{currentfill}{rgb}{0.121569,0.466667,0.705882}%
\pgfsetfillcolor{currentfill}%
\pgfsetlinewidth{1.003750pt}%
\definecolor{currentstroke}{rgb}{0.121569,0.466667,0.705882}%
\pgfsetstrokecolor{currentstroke}%
\pgfsetdash{}{0pt}%
\pgfpathmoveto{\pgfqpoint{2.557916in}{1.648789in}}%
\pgfpathcurveto{\pgfqpoint{2.568003in}{1.648789in}}{\pgfqpoint{2.577678in}{1.652796in}}{\pgfqpoint{2.584811in}{1.659929in}}%
\pgfpathcurveto{\pgfqpoint{2.591944in}{1.667062in}}{\pgfqpoint{2.595952in}{1.676738in}}{\pgfqpoint{2.595952in}{1.686825in}}%
\pgfpathcurveto{\pgfqpoint{2.595952in}{1.696912in}}{\pgfqpoint{2.591944in}{1.706588in}}{\pgfqpoint{2.584811in}{1.713721in}}%
\pgfpathcurveto{\pgfqpoint{2.577678in}{1.720853in}}{\pgfqpoint{2.568003in}{1.724861in}}{\pgfqpoint{2.557916in}{1.724861in}}%
\pgfpathcurveto{\pgfqpoint{2.547828in}{1.724861in}}{\pgfqpoint{2.538153in}{1.720853in}}{\pgfqpoint{2.531020in}{1.713721in}}%
\pgfpathcurveto{\pgfqpoint{2.523887in}{1.706588in}}{\pgfqpoint{2.519879in}{1.696912in}}{\pgfqpoint{2.519879in}{1.686825in}}%
\pgfpathcurveto{\pgfqpoint{2.519879in}{1.676738in}}{\pgfqpoint{2.523887in}{1.667062in}}{\pgfqpoint{2.531020in}{1.659929in}}%
\pgfpathcurveto{\pgfqpoint{2.538153in}{1.652796in}}{\pgfqpoint{2.547828in}{1.648789in}}{\pgfqpoint{2.557916in}{1.648789in}}%
\pgfpathclose%
\pgfusepath{stroke,fill}%
\end{pgfscope}%
\begin{pgfscope}%
\pgfpathrectangle{\pgfqpoint{0.800000in}{1.363959in}}{\pgfqpoint{3.968000in}{2.024082in}} %
\pgfusepath{clip}%
\pgfsetbuttcap%
\pgfsetroundjoin%
\definecolor{currentfill}{rgb}{0.121569,0.466667,0.705882}%
\pgfsetfillcolor{currentfill}%
\pgfsetlinewidth{1.003750pt}%
\definecolor{currentstroke}{rgb}{0.121569,0.466667,0.705882}%
\pgfsetstrokecolor{currentstroke}%
\pgfsetdash{}{0pt}%
\pgfpathmoveto{\pgfqpoint{2.238740in}{2.340046in}}%
\pgfpathcurveto{\pgfqpoint{2.248827in}{2.340046in}}{\pgfqpoint{2.258503in}{2.344054in}}{\pgfqpoint{2.265636in}{2.351187in}}%
\pgfpathcurveto{\pgfqpoint{2.272769in}{2.358320in}}{\pgfqpoint{2.276776in}{2.367995in}}{\pgfqpoint{2.276776in}{2.378083in}}%
\pgfpathcurveto{\pgfqpoint{2.276776in}{2.388170in}}{\pgfqpoint{2.272769in}{2.397846in}}{\pgfqpoint{2.265636in}{2.404978in}}%
\pgfpathcurveto{\pgfqpoint{2.258503in}{2.412111in}}{\pgfqpoint{2.248827in}{2.416119in}}{\pgfqpoint{2.238740in}{2.416119in}}%
\pgfpathcurveto{\pgfqpoint{2.228653in}{2.416119in}}{\pgfqpoint{2.218977in}{2.412111in}}{\pgfqpoint{2.211844in}{2.404978in}}%
\pgfpathcurveto{\pgfqpoint{2.204712in}{2.397846in}}{\pgfqpoint{2.200704in}{2.388170in}}{\pgfqpoint{2.200704in}{2.378083in}}%
\pgfpathcurveto{\pgfqpoint{2.200704in}{2.367995in}}{\pgfqpoint{2.204712in}{2.358320in}}{\pgfqpoint{2.211844in}{2.351187in}}%
\pgfpathcurveto{\pgfqpoint{2.218977in}{2.344054in}}{\pgfqpoint{2.228653in}{2.340046in}}{\pgfqpoint{2.238740in}{2.340046in}}%
\pgfpathclose%
\pgfusepath{stroke,fill}%
\end{pgfscope}%
\begin{pgfscope}%
\pgfpathrectangle{\pgfqpoint{0.800000in}{1.363959in}}{\pgfqpoint{3.968000in}{2.024082in}} %
\pgfusepath{clip}%
\pgfsetbuttcap%
\pgfsetroundjoin%
\definecolor{currentfill}{rgb}{0.121569,0.466667,0.705882}%
\pgfsetfillcolor{currentfill}%
\pgfsetlinewidth{1.003750pt}%
\definecolor{currentstroke}{rgb}{0.121569,0.466667,0.705882}%
\pgfsetstrokecolor{currentstroke}%
\pgfsetdash{}{0pt}%
\pgfpathmoveto{\pgfqpoint{2.866707in}{3.165095in}}%
\pgfpathcurveto{\pgfqpoint{2.876794in}{3.165095in}}{\pgfqpoint{2.886469in}{3.169103in}}{\pgfqpoint{2.893602in}{3.176235in}}%
\pgfpathcurveto{\pgfqpoint{2.900735in}{3.183368in}}{\pgfqpoint{2.904743in}{3.193044in}}{\pgfqpoint{2.904743in}{3.203131in}}%
\pgfpathcurveto{\pgfqpoint{2.904743in}{3.213218in}}{\pgfqpoint{2.900735in}{3.222894in}}{\pgfqpoint{2.893602in}{3.230027in}}%
\pgfpathcurveto{\pgfqpoint{2.886469in}{3.237160in}}{\pgfqpoint{2.876794in}{3.241167in}}{\pgfqpoint{2.866707in}{3.241167in}}%
\pgfpathcurveto{\pgfqpoint{2.856619in}{3.241167in}}{\pgfqpoint{2.846944in}{3.237160in}}{\pgfqpoint{2.839811in}{3.230027in}}%
\pgfpathcurveto{\pgfqpoint{2.832678in}{3.222894in}}{\pgfqpoint{2.828670in}{3.213218in}}{\pgfqpoint{2.828670in}{3.203131in}}%
\pgfpathcurveto{\pgfqpoint{2.828670in}{3.193044in}}{\pgfqpoint{2.832678in}{3.183368in}}{\pgfqpoint{2.839811in}{3.176235in}}%
\pgfpathcurveto{\pgfqpoint{2.846944in}{3.169103in}}{\pgfqpoint{2.856619in}{3.165095in}}{\pgfqpoint{2.866707in}{3.165095in}}%
\pgfpathclose%
\pgfusepath{stroke,fill}%
\end{pgfscope}%
\begin{pgfscope}%
\pgfpathrectangle{\pgfqpoint{0.800000in}{1.363959in}}{\pgfqpoint{3.968000in}{2.024082in}} %
\pgfusepath{clip}%
\pgfsetbuttcap%
\pgfsetroundjoin%
\definecolor{currentfill}{rgb}{0.121569,0.466667,0.705882}%
\pgfsetfillcolor{currentfill}%
\pgfsetlinewidth{1.003750pt}%
\definecolor{currentstroke}{rgb}{0.121569,0.466667,0.705882}%
\pgfsetstrokecolor{currentstroke}%
\pgfsetdash{}{0pt}%
\pgfpathmoveto{\pgfqpoint{2.947784in}{1.424837in}}%
\pgfpathcurveto{\pgfqpoint{2.957871in}{1.424837in}}{\pgfqpoint{2.967547in}{1.428845in}}{\pgfqpoint{2.974679in}{1.435978in}}%
\pgfpathcurveto{\pgfqpoint{2.981812in}{1.443111in}}{\pgfqpoint{2.985820in}{1.452786in}}{\pgfqpoint{2.985820in}{1.462874in}}%
\pgfpathcurveto{\pgfqpoint{2.985820in}{1.472961in}}{\pgfqpoint{2.981812in}{1.482637in}}{\pgfqpoint{2.974679in}{1.489769in}}%
\pgfpathcurveto{\pgfqpoint{2.967547in}{1.496902in}}{\pgfqpoint{2.957871in}{1.500910in}}{\pgfqpoint{2.947784in}{1.500910in}}%
\pgfpathcurveto{\pgfqpoint{2.937696in}{1.500910in}}{\pgfqpoint{2.928021in}{1.496902in}}{\pgfqpoint{2.920888in}{1.489769in}}%
\pgfpathcurveto{\pgfqpoint{2.913755in}{1.482637in}}{\pgfqpoint{2.909747in}{1.472961in}}{\pgfqpoint{2.909747in}{1.462874in}}%
\pgfpathcurveto{\pgfqpoint{2.909747in}{1.452786in}}{\pgfqpoint{2.913755in}{1.443111in}}{\pgfqpoint{2.920888in}{1.435978in}}%
\pgfpathcurveto{\pgfqpoint{2.928021in}{1.428845in}}{\pgfqpoint{2.937696in}{1.424837in}}{\pgfqpoint{2.947784in}{1.424837in}}%
\pgfpathclose%
\pgfusepath{stroke,fill}%
\end{pgfscope}%
\begin{pgfscope}%
\pgfpathrectangle{\pgfqpoint{0.800000in}{1.363959in}}{\pgfqpoint{3.968000in}{2.024082in}} %
\pgfusepath{clip}%
\pgfsetbuttcap%
\pgfsetroundjoin%
\definecolor{currentfill}{rgb}{0.121569,0.466667,0.705882}%
\pgfsetfillcolor{currentfill}%
\pgfsetlinewidth{1.003750pt}%
\definecolor{currentstroke}{rgb}{0.121569,0.466667,0.705882}%
\pgfsetstrokecolor{currentstroke}%
\pgfsetdash{}{0pt}%
\pgfpathmoveto{\pgfqpoint{2.405907in}{1.725485in}}%
\pgfpathcurveto{\pgfqpoint{2.415994in}{1.725485in}}{\pgfqpoint{2.425670in}{1.729493in}}{\pgfqpoint{2.432802in}{1.736626in}}%
\pgfpathcurveto{\pgfqpoint{2.439935in}{1.743759in}}{\pgfqpoint{2.443943in}{1.753434in}}{\pgfqpoint{2.443943in}{1.763522in}}%
\pgfpathcurveto{\pgfqpoint{2.443943in}{1.773609in}}{\pgfqpoint{2.439935in}{1.783285in}}{\pgfqpoint{2.432802in}{1.790417in}}%
\pgfpathcurveto{\pgfqpoint{2.425670in}{1.797550in}}{\pgfqpoint{2.415994in}{1.801558in}}{\pgfqpoint{2.405907in}{1.801558in}}%
\pgfpathcurveto{\pgfqpoint{2.395819in}{1.801558in}}{\pgfqpoint{2.386144in}{1.797550in}}{\pgfqpoint{2.379011in}{1.790417in}}%
\pgfpathcurveto{\pgfqpoint{2.371878in}{1.783285in}}{\pgfqpoint{2.367870in}{1.773609in}}{\pgfqpoint{2.367870in}{1.763522in}}%
\pgfpathcurveto{\pgfqpoint{2.367870in}{1.753434in}}{\pgfqpoint{2.371878in}{1.743759in}}{\pgfqpoint{2.379011in}{1.736626in}}%
\pgfpathcurveto{\pgfqpoint{2.386144in}{1.729493in}}{\pgfqpoint{2.395819in}{1.725485in}}{\pgfqpoint{2.405907in}{1.725485in}}%
\pgfpathclose%
\pgfusepath{stroke,fill}%
\end{pgfscope}%
\begin{pgfscope}%
\pgfpathrectangle{\pgfqpoint{0.800000in}{1.363959in}}{\pgfqpoint{3.968000in}{2.024082in}} %
\pgfusepath{clip}%
\pgfsetbuttcap%
\pgfsetroundjoin%
\definecolor{currentfill}{rgb}{0.121569,0.466667,0.705882}%
\pgfsetfillcolor{currentfill}%
\pgfsetlinewidth{1.003750pt}%
\definecolor{currentstroke}{rgb}{0.121569,0.466667,0.705882}%
\pgfsetstrokecolor{currentstroke}%
\pgfsetdash{}{0pt}%
\pgfpathmoveto{\pgfqpoint{2.779742in}{2.016229in}}%
\pgfpathcurveto{\pgfqpoint{2.789830in}{2.016229in}}{\pgfqpoint{2.799505in}{2.020236in}}{\pgfqpoint{2.806638in}{2.027369in}}%
\pgfpathcurveto{\pgfqpoint{2.813771in}{2.034502in}}{\pgfqpoint{2.817779in}{2.044178in}}{\pgfqpoint{2.817779in}{2.054265in}}%
\pgfpathcurveto{\pgfqpoint{2.817779in}{2.064352in}}{\pgfqpoint{2.813771in}{2.074028in}}{\pgfqpoint{2.806638in}{2.081161in}}%
\pgfpathcurveto{\pgfqpoint{2.799505in}{2.088293in}}{\pgfqpoint{2.789830in}{2.092301in}}{\pgfqpoint{2.779742in}{2.092301in}}%
\pgfpathcurveto{\pgfqpoint{2.769655in}{2.092301in}}{\pgfqpoint{2.759980in}{2.088293in}}{\pgfqpoint{2.752847in}{2.081161in}}%
\pgfpathcurveto{\pgfqpoint{2.745714in}{2.074028in}}{\pgfqpoint{2.741706in}{2.064352in}}{\pgfqpoint{2.741706in}{2.054265in}}%
\pgfpathcurveto{\pgfqpoint{2.741706in}{2.044178in}}{\pgfqpoint{2.745714in}{2.034502in}}{\pgfqpoint{2.752847in}{2.027369in}}%
\pgfpathcurveto{\pgfqpoint{2.759980in}{2.020236in}}{\pgfqpoint{2.769655in}{2.016229in}}{\pgfqpoint{2.779742in}{2.016229in}}%
\pgfpathclose%
\pgfusepath{stroke,fill}%
\end{pgfscope}%
\begin{pgfscope}%
\pgfpathrectangle{\pgfqpoint{0.800000in}{1.363959in}}{\pgfqpoint{3.968000in}{2.024082in}} %
\pgfusepath{clip}%
\pgfsetbuttcap%
\pgfsetroundjoin%
\definecolor{currentfill}{rgb}{0.121569,0.466667,0.705882}%
\pgfsetfillcolor{currentfill}%
\pgfsetlinewidth{1.003750pt}%
\definecolor{currentstroke}{rgb}{0.121569,0.466667,0.705882}%
\pgfsetstrokecolor{currentstroke}%
\pgfsetdash{}{0pt}%
\pgfpathmoveto{\pgfqpoint{2.463032in}{2.906785in}}%
\pgfpathcurveto{\pgfqpoint{2.473119in}{2.906785in}}{\pgfqpoint{2.482795in}{2.910793in}}{\pgfqpoint{2.489928in}{2.917926in}}%
\pgfpathcurveto{\pgfqpoint{2.497061in}{2.925059in}}{\pgfqpoint{2.501068in}{2.934734in}}{\pgfqpoint{2.501068in}{2.944822in}}%
\pgfpathcurveto{\pgfqpoint{2.501068in}{2.954909in}}{\pgfqpoint{2.497061in}{2.964585in}}{\pgfqpoint{2.489928in}{2.971717in}}%
\pgfpathcurveto{\pgfqpoint{2.482795in}{2.978850in}}{\pgfqpoint{2.473119in}{2.982858in}}{\pgfqpoint{2.463032in}{2.982858in}}%
\pgfpathcurveto{\pgfqpoint{2.452945in}{2.982858in}}{\pgfqpoint{2.443269in}{2.978850in}}{\pgfqpoint{2.436136in}{2.971717in}}%
\pgfpathcurveto{\pgfqpoint{2.429003in}{2.964585in}}{\pgfqpoint{2.424996in}{2.954909in}}{\pgfqpoint{2.424996in}{2.944822in}}%
\pgfpathcurveto{\pgfqpoint{2.424996in}{2.934734in}}{\pgfqpoint{2.429003in}{2.925059in}}{\pgfqpoint{2.436136in}{2.917926in}}%
\pgfpathcurveto{\pgfqpoint{2.443269in}{2.910793in}}{\pgfqpoint{2.452945in}{2.906785in}}{\pgfqpoint{2.463032in}{2.906785in}}%
\pgfpathclose%
\pgfusepath{stroke,fill}%
\end{pgfscope}%
\begin{pgfscope}%
\pgfpathrectangle{\pgfqpoint{0.800000in}{1.363959in}}{\pgfqpoint{3.968000in}{2.024082in}} %
\pgfusepath{clip}%
\pgfsetbuttcap%
\pgfsetroundjoin%
\definecolor{currentfill}{rgb}{0.121569,0.466667,0.705882}%
\pgfsetfillcolor{currentfill}%
\pgfsetlinewidth{1.003750pt}%
\definecolor{currentstroke}{rgb}{0.121569,0.466667,0.705882}%
\pgfsetstrokecolor{currentstroke}%
\pgfsetdash{}{0pt}%
\pgfpathmoveto{\pgfqpoint{3.491321in}{2.039868in}}%
\pgfpathcurveto{\pgfqpoint{3.501409in}{2.039868in}}{\pgfqpoint{3.511084in}{2.043876in}}{\pgfqpoint{3.518217in}{2.051009in}}%
\pgfpathcurveto{\pgfqpoint{3.525350in}{2.058141in}}{\pgfqpoint{3.529357in}{2.067817in}}{\pgfqpoint{3.529357in}{2.077904in}}%
\pgfpathcurveto{\pgfqpoint{3.529357in}{2.087992in}}{\pgfqpoint{3.525350in}{2.097667in}}{\pgfqpoint{3.518217in}{2.104800in}}%
\pgfpathcurveto{\pgfqpoint{3.511084in}{2.111933in}}{\pgfqpoint{3.501409in}{2.115941in}}{\pgfqpoint{3.491321in}{2.115941in}}%
\pgfpathcurveto{\pgfqpoint{3.481234in}{2.115941in}}{\pgfqpoint{3.471558in}{2.111933in}}{\pgfqpoint{3.464425in}{2.104800in}}%
\pgfpathcurveto{\pgfqpoint{3.457293in}{2.097667in}}{\pgfqpoint{3.453285in}{2.087992in}}{\pgfqpoint{3.453285in}{2.077904in}}%
\pgfpathcurveto{\pgfqpoint{3.453285in}{2.067817in}}{\pgfqpoint{3.457293in}{2.058141in}}{\pgfqpoint{3.464425in}{2.051009in}}%
\pgfpathcurveto{\pgfqpoint{3.471558in}{2.043876in}}{\pgfqpoint{3.481234in}{2.039868in}}{\pgfqpoint{3.491321in}{2.039868in}}%
\pgfpathclose%
\pgfusepath{stroke,fill}%
\end{pgfscope}%
\begin{pgfscope}%
\pgfpathrectangle{\pgfqpoint{0.800000in}{1.363959in}}{\pgfqpoint{3.968000in}{2.024082in}} %
\pgfusepath{clip}%
\pgfsetbuttcap%
\pgfsetroundjoin%
\definecolor{currentfill}{rgb}{0.121569,0.466667,0.705882}%
\pgfsetfillcolor{currentfill}%
\pgfsetlinewidth{1.003750pt}%
\definecolor{currentstroke}{rgb}{0.121569,0.466667,0.705882}%
\pgfsetstrokecolor{currentstroke}%
\pgfsetdash{}{0pt}%
\pgfpathmoveto{\pgfqpoint{2.959377in}{3.150428in}}%
\pgfpathcurveto{\pgfqpoint{2.969464in}{3.150428in}}{\pgfqpoint{2.979140in}{3.154436in}}{\pgfqpoint{2.986273in}{3.161569in}}%
\pgfpathcurveto{\pgfqpoint{2.993406in}{3.168702in}}{\pgfqpoint{2.997413in}{3.178377in}}{\pgfqpoint{2.997413in}{3.188465in}}%
\pgfpathcurveto{\pgfqpoint{2.997413in}{3.198552in}}{\pgfqpoint{2.993406in}{3.208227in}}{\pgfqpoint{2.986273in}{3.215360in}}%
\pgfpathcurveto{\pgfqpoint{2.979140in}{3.222493in}}{\pgfqpoint{2.969464in}{3.226501in}}{\pgfqpoint{2.959377in}{3.226501in}}%
\pgfpathcurveto{\pgfqpoint{2.949290in}{3.226501in}}{\pgfqpoint{2.939614in}{3.222493in}}{\pgfqpoint{2.932481in}{3.215360in}}%
\pgfpathcurveto{\pgfqpoint{2.925348in}{3.208227in}}{\pgfqpoint{2.921341in}{3.198552in}}{\pgfqpoint{2.921341in}{3.188465in}}%
\pgfpathcurveto{\pgfqpoint{2.921341in}{3.178377in}}{\pgfqpoint{2.925348in}{3.168702in}}{\pgfqpoint{2.932481in}{3.161569in}}%
\pgfpathcurveto{\pgfqpoint{2.939614in}{3.154436in}}{\pgfqpoint{2.949290in}{3.150428in}}{\pgfqpoint{2.959377in}{3.150428in}}%
\pgfpathclose%
\pgfusepath{stroke,fill}%
\end{pgfscope}%
\begin{pgfscope}%
\pgfpathrectangle{\pgfqpoint{0.800000in}{1.363959in}}{\pgfqpoint{3.968000in}{2.024082in}} %
\pgfusepath{clip}%
\pgfsetbuttcap%
\pgfsetroundjoin%
\definecolor{currentfill}{rgb}{0.121569,0.466667,0.705882}%
\pgfsetfillcolor{currentfill}%
\pgfsetlinewidth{1.003750pt}%
\definecolor{currentstroke}{rgb}{0.121569,0.466667,0.705882}%
\pgfsetstrokecolor{currentstroke}%
\pgfsetdash{}{0pt}%
\pgfpathmoveto{\pgfqpoint{1.826119in}{2.482468in}}%
\pgfpathcurveto{\pgfqpoint{1.836207in}{2.482468in}}{\pgfqpoint{1.845882in}{2.486476in}}{\pgfqpoint{1.853015in}{2.493608in}}%
\pgfpathcurveto{\pgfqpoint{1.860148in}{2.500741in}}{\pgfqpoint{1.864156in}{2.510417in}}{\pgfqpoint{1.864156in}{2.520504in}}%
\pgfpathcurveto{\pgfqpoint{1.864156in}{2.530591in}}{\pgfqpoint{1.860148in}{2.540267in}}{\pgfqpoint{1.853015in}{2.547400in}}%
\pgfpathcurveto{\pgfqpoint{1.845882in}{2.554533in}}{\pgfqpoint{1.836207in}{2.558540in}}{\pgfqpoint{1.826119in}{2.558540in}}%
\pgfpathcurveto{\pgfqpoint{1.816032in}{2.558540in}}{\pgfqpoint{1.806357in}{2.554533in}}{\pgfqpoint{1.799224in}{2.547400in}}%
\pgfpathcurveto{\pgfqpoint{1.792091in}{2.540267in}}{\pgfqpoint{1.788083in}{2.530591in}}{\pgfqpoint{1.788083in}{2.520504in}}%
\pgfpathcurveto{\pgfqpoint{1.788083in}{2.510417in}}{\pgfqpoint{1.792091in}{2.500741in}}{\pgfqpoint{1.799224in}{2.493608in}}%
\pgfpathcurveto{\pgfqpoint{1.806357in}{2.486476in}}{\pgfqpoint{1.816032in}{2.482468in}}{\pgfqpoint{1.826119in}{2.482468in}}%
\pgfpathclose%
\pgfusepath{stroke,fill}%
\end{pgfscope}%
\begin{pgfscope}%
\pgfpathrectangle{\pgfqpoint{0.800000in}{1.363959in}}{\pgfqpoint{3.968000in}{2.024082in}} %
\pgfusepath{clip}%
\pgfsetbuttcap%
\pgfsetroundjoin%
\definecolor{currentfill}{rgb}{0.121569,0.466667,0.705882}%
\pgfsetfillcolor{currentfill}%
\pgfsetlinewidth{1.003750pt}%
\definecolor{currentstroke}{rgb}{0.121569,0.466667,0.705882}%
\pgfsetstrokecolor{currentstroke}%
\pgfsetdash{}{0pt}%
\pgfpathmoveto{\pgfqpoint{3.085355in}{2.972527in}}%
\pgfpathcurveto{\pgfqpoint{3.095442in}{2.972527in}}{\pgfqpoint{3.105118in}{2.976534in}}{\pgfqpoint{3.112251in}{2.983667in}}%
\pgfpathcurveto{\pgfqpoint{3.119384in}{2.990800in}}{\pgfqpoint{3.123391in}{3.000475in}}{\pgfqpoint{3.123391in}{3.010563in}}%
\pgfpathcurveto{\pgfqpoint{3.123391in}{3.020650in}}{\pgfqpoint{3.119384in}{3.030326in}}{\pgfqpoint{3.112251in}{3.037459in}}%
\pgfpathcurveto{\pgfqpoint{3.105118in}{3.044591in}}{\pgfqpoint{3.095442in}{3.048599in}}{\pgfqpoint{3.085355in}{3.048599in}}%
\pgfpathcurveto{\pgfqpoint{3.075268in}{3.048599in}}{\pgfqpoint{3.065592in}{3.044591in}}{\pgfqpoint{3.058459in}{3.037459in}}%
\pgfpathcurveto{\pgfqpoint{3.051327in}{3.030326in}}{\pgfqpoint{3.047319in}{3.020650in}}{\pgfqpoint{3.047319in}{3.010563in}}%
\pgfpathcurveto{\pgfqpoint{3.047319in}{3.000475in}}{\pgfqpoint{3.051327in}{2.990800in}}{\pgfqpoint{3.058459in}{2.983667in}}%
\pgfpathcurveto{\pgfqpoint{3.065592in}{2.976534in}}{\pgfqpoint{3.075268in}{2.972527in}}{\pgfqpoint{3.085355in}{2.972527in}}%
\pgfpathclose%
\pgfusepath{stroke,fill}%
\end{pgfscope}%
\begin{pgfscope}%
\pgfpathrectangle{\pgfqpoint{0.800000in}{1.363959in}}{\pgfqpoint{3.968000in}{2.024082in}} %
\pgfusepath{clip}%
\pgfsetbuttcap%
\pgfsetroundjoin%
\definecolor{currentfill}{rgb}{0.121569,0.466667,0.705882}%
\pgfsetfillcolor{currentfill}%
\pgfsetlinewidth{1.003750pt}%
\definecolor{currentstroke}{rgb}{0.121569,0.466667,0.705882}%
\pgfsetstrokecolor{currentstroke}%
\pgfsetdash{}{0pt}%
\pgfpathmoveto{\pgfqpoint{2.118079in}{2.254648in}}%
\pgfpathcurveto{\pgfqpoint{2.128166in}{2.254648in}}{\pgfqpoint{2.137842in}{2.258656in}}{\pgfqpoint{2.144975in}{2.265789in}}%
\pgfpathcurveto{\pgfqpoint{2.152107in}{2.272922in}}{\pgfqpoint{2.156115in}{2.282597in}}{\pgfqpoint{2.156115in}{2.292685in}}%
\pgfpathcurveto{\pgfqpoint{2.156115in}{2.302772in}}{\pgfqpoint{2.152107in}{2.312447in}}{\pgfqpoint{2.144975in}{2.319580in}}%
\pgfpathcurveto{\pgfqpoint{2.137842in}{2.326713in}}{\pgfqpoint{2.128166in}{2.330721in}}{\pgfqpoint{2.118079in}{2.330721in}}%
\pgfpathcurveto{\pgfqpoint{2.107991in}{2.330721in}}{\pgfqpoint{2.098316in}{2.326713in}}{\pgfqpoint{2.091183in}{2.319580in}}%
\pgfpathcurveto{\pgfqpoint{2.084050in}{2.312447in}}{\pgfqpoint{2.080043in}{2.302772in}}{\pgfqpoint{2.080043in}{2.292685in}}%
\pgfpathcurveto{\pgfqpoint{2.080043in}{2.282597in}}{\pgfqpoint{2.084050in}{2.272922in}}{\pgfqpoint{2.091183in}{2.265789in}}%
\pgfpathcurveto{\pgfqpoint{2.098316in}{2.258656in}}{\pgfqpoint{2.107991in}{2.254648in}}{\pgfqpoint{2.118079in}{2.254648in}}%
\pgfpathclose%
\pgfusepath{stroke,fill}%
\end{pgfscope}%
\begin{pgfscope}%
\pgfpathrectangle{\pgfqpoint{0.800000in}{1.363959in}}{\pgfqpoint{3.968000in}{2.024082in}} %
\pgfusepath{clip}%
\pgfsetbuttcap%
\pgfsetroundjoin%
\definecolor{currentfill}{rgb}{0.121569,0.466667,0.705882}%
\pgfsetfillcolor{currentfill}%
\pgfsetlinewidth{1.003750pt}%
\definecolor{currentstroke}{rgb}{0.121569,0.466667,0.705882}%
\pgfsetstrokecolor{currentstroke}%
\pgfsetdash{}{0pt}%
\pgfpathmoveto{\pgfqpoint{2.052410in}{2.415937in}}%
\pgfpathcurveto{\pgfqpoint{2.062498in}{2.415937in}}{\pgfqpoint{2.072173in}{2.419944in}}{\pgfqpoint{2.079306in}{2.427077in}}%
\pgfpathcurveto{\pgfqpoint{2.086439in}{2.434210in}}{\pgfqpoint{2.090447in}{2.443886in}}{\pgfqpoint{2.090447in}{2.453973in}}%
\pgfpathcurveto{\pgfqpoint{2.090447in}{2.464060in}}{\pgfqpoint{2.086439in}{2.473736in}}{\pgfqpoint{2.079306in}{2.480869in}}%
\pgfpathcurveto{\pgfqpoint{2.072173in}{2.488001in}}{\pgfqpoint{2.062498in}{2.492009in}}{\pgfqpoint{2.052410in}{2.492009in}}%
\pgfpathcurveto{\pgfqpoint{2.042323in}{2.492009in}}{\pgfqpoint{2.032648in}{2.488001in}}{\pgfqpoint{2.025515in}{2.480869in}}%
\pgfpathcurveto{\pgfqpoint{2.018382in}{2.473736in}}{\pgfqpoint{2.014374in}{2.464060in}}{\pgfqpoint{2.014374in}{2.453973in}}%
\pgfpathcurveto{\pgfqpoint{2.014374in}{2.443886in}}{\pgfqpoint{2.018382in}{2.434210in}}{\pgfqpoint{2.025515in}{2.427077in}}%
\pgfpathcurveto{\pgfqpoint{2.032648in}{2.419944in}}{\pgfqpoint{2.042323in}{2.415937in}}{\pgfqpoint{2.052410in}{2.415937in}}%
\pgfpathclose%
\pgfusepath{stroke,fill}%
\end{pgfscope}%
\begin{pgfscope}%
\pgfpathrectangle{\pgfqpoint{0.800000in}{1.363959in}}{\pgfqpoint{3.968000in}{2.024082in}} %
\pgfusepath{clip}%
\pgfsetbuttcap%
\pgfsetroundjoin%
\definecolor{currentfill}{rgb}{0.121569,0.466667,0.705882}%
\pgfsetfillcolor{currentfill}%
\pgfsetlinewidth{1.003750pt}%
\definecolor{currentstroke}{rgb}{0.121569,0.466667,0.705882}%
\pgfsetstrokecolor{currentstroke}%
\pgfsetdash{}{0pt}%
\pgfpathmoveto{\pgfqpoint{3.096400in}{1.929781in}}%
\pgfpathcurveto{\pgfqpoint{3.106488in}{1.929781in}}{\pgfqpoint{3.116163in}{1.933788in}}{\pgfqpoint{3.123296in}{1.940921in}}%
\pgfpathcurveto{\pgfqpoint{3.130429in}{1.948054in}}{\pgfqpoint{3.134436in}{1.957730in}}{\pgfqpoint{3.134436in}{1.967817in}}%
\pgfpathcurveto{\pgfqpoint{3.134436in}{1.977904in}}{\pgfqpoint{3.130429in}{1.987580in}}{\pgfqpoint{3.123296in}{1.994713in}}%
\pgfpathcurveto{\pgfqpoint{3.116163in}{2.001845in}}{\pgfqpoint{3.106488in}{2.005853in}}{\pgfqpoint{3.096400in}{2.005853in}}%
\pgfpathcurveto{\pgfqpoint{3.086313in}{2.005853in}}{\pgfqpoint{3.076637in}{2.001845in}}{\pgfqpoint{3.069504in}{1.994713in}}%
\pgfpathcurveto{\pgfqpoint{3.062372in}{1.987580in}}{\pgfqpoint{3.058364in}{1.977904in}}{\pgfqpoint{3.058364in}{1.967817in}}%
\pgfpathcurveto{\pgfqpoint{3.058364in}{1.957730in}}{\pgfqpoint{3.062372in}{1.948054in}}{\pgfqpoint{3.069504in}{1.940921in}}%
\pgfpathcurveto{\pgfqpoint{3.076637in}{1.933788in}}{\pgfqpoint{3.086313in}{1.929781in}}{\pgfqpoint{3.096400in}{1.929781in}}%
\pgfpathclose%
\pgfusepath{stroke,fill}%
\end{pgfscope}%
\begin{pgfscope}%
\pgfpathrectangle{\pgfqpoint{0.800000in}{1.363959in}}{\pgfqpoint{3.968000in}{2.024082in}} %
\pgfusepath{clip}%
\pgfsetbuttcap%
\pgfsetroundjoin%
\definecolor{currentfill}{rgb}{0.121569,0.466667,0.705882}%
\pgfsetfillcolor{currentfill}%
\pgfsetlinewidth{1.003750pt}%
\definecolor{currentstroke}{rgb}{0.121569,0.466667,0.705882}%
\pgfsetstrokecolor{currentstroke}%
\pgfsetdash{}{0pt}%
\pgfpathmoveto{\pgfqpoint{1.919719in}{2.311304in}}%
\pgfpathcurveto{\pgfqpoint{1.929807in}{2.311304in}}{\pgfqpoint{1.939482in}{2.315312in}}{\pgfqpoint{1.946615in}{2.322445in}}%
\pgfpathcurveto{\pgfqpoint{1.953748in}{2.329578in}}{\pgfqpoint{1.957756in}{2.339253in}}{\pgfqpoint{1.957756in}{2.349341in}}%
\pgfpathcurveto{\pgfqpoint{1.957756in}{2.359428in}}{\pgfqpoint{1.953748in}{2.369103in}}{\pgfqpoint{1.946615in}{2.376236in}}%
\pgfpathcurveto{\pgfqpoint{1.939482in}{2.383369in}}{\pgfqpoint{1.929807in}{2.387377in}}{\pgfqpoint{1.919719in}{2.387377in}}%
\pgfpathcurveto{\pgfqpoint{1.909632in}{2.387377in}}{\pgfqpoint{1.899957in}{2.383369in}}{\pgfqpoint{1.892824in}{2.376236in}}%
\pgfpathcurveto{\pgfqpoint{1.885691in}{2.369103in}}{\pgfqpoint{1.881683in}{2.359428in}}{\pgfqpoint{1.881683in}{2.349341in}}%
\pgfpathcurveto{\pgfqpoint{1.881683in}{2.339253in}}{\pgfqpoint{1.885691in}{2.329578in}}{\pgfqpoint{1.892824in}{2.322445in}}%
\pgfpathcurveto{\pgfqpoint{1.899957in}{2.315312in}}{\pgfqpoint{1.909632in}{2.311304in}}{\pgfqpoint{1.919719in}{2.311304in}}%
\pgfpathclose%
\pgfusepath{stroke,fill}%
\end{pgfscope}%
\begin{pgfscope}%
\pgfpathrectangle{\pgfqpoint{0.800000in}{1.363959in}}{\pgfqpoint{3.968000in}{2.024082in}} %
\pgfusepath{clip}%
\pgfsetbuttcap%
\pgfsetroundjoin%
\definecolor{currentfill}{rgb}{0.121569,0.466667,0.705882}%
\pgfsetfillcolor{currentfill}%
\pgfsetlinewidth{1.003750pt}%
\definecolor{currentstroke}{rgb}{0.121569,0.466667,0.705882}%
\pgfsetstrokecolor{currentstroke}%
\pgfsetdash{}{0pt}%
\pgfpathmoveto{\pgfqpoint{3.205792in}{2.514358in}}%
\pgfpathcurveto{\pgfqpoint{3.215879in}{2.514358in}}{\pgfqpoint{3.225555in}{2.518365in}}{\pgfqpoint{3.232688in}{2.525498in}}%
\pgfpathcurveto{\pgfqpoint{3.239821in}{2.532631in}}{\pgfqpoint{3.243828in}{2.542307in}}{\pgfqpoint{3.243828in}{2.552394in}}%
\pgfpathcurveto{\pgfqpoint{3.243828in}{2.562481in}}{\pgfqpoint{3.239821in}{2.572157in}}{\pgfqpoint{3.232688in}{2.579290in}}%
\pgfpathcurveto{\pgfqpoint{3.225555in}{2.586422in}}{\pgfqpoint{3.215879in}{2.590430in}}{\pgfqpoint{3.205792in}{2.590430in}}%
\pgfpathcurveto{\pgfqpoint{3.195705in}{2.590430in}}{\pgfqpoint{3.186029in}{2.586422in}}{\pgfqpoint{3.178896in}{2.579290in}}%
\pgfpathcurveto{\pgfqpoint{3.171763in}{2.572157in}}{\pgfqpoint{3.167756in}{2.562481in}}{\pgfqpoint{3.167756in}{2.552394in}}%
\pgfpathcurveto{\pgfqpoint{3.167756in}{2.542307in}}{\pgfqpoint{3.171763in}{2.532631in}}{\pgfqpoint{3.178896in}{2.525498in}}%
\pgfpathcurveto{\pgfqpoint{3.186029in}{2.518365in}}{\pgfqpoint{3.195705in}{2.514358in}}{\pgfqpoint{3.205792in}{2.514358in}}%
\pgfpathclose%
\pgfusepath{stroke,fill}%
\end{pgfscope}%
\begin{pgfscope}%
\pgfpathrectangle{\pgfqpoint{0.800000in}{1.363959in}}{\pgfqpoint{3.968000in}{2.024082in}} %
\pgfusepath{clip}%
\pgfsetbuttcap%
\pgfsetroundjoin%
\definecolor{currentfill}{rgb}{0.121569,0.466667,0.705882}%
\pgfsetfillcolor{currentfill}%
\pgfsetlinewidth{1.003750pt}%
\definecolor{currentstroke}{rgb}{0.121569,0.466667,0.705882}%
\pgfsetstrokecolor{currentstroke}%
\pgfsetdash{}{0pt}%
\pgfpathmoveto{\pgfqpoint{3.518349in}{2.503858in}}%
\pgfpathcurveto{\pgfqpoint{3.528436in}{2.503858in}}{\pgfqpoint{3.538112in}{2.507866in}}{\pgfqpoint{3.545244in}{2.514998in}}%
\pgfpathcurveto{\pgfqpoint{3.552377in}{2.522131in}}{\pgfqpoint{3.556385in}{2.531807in}}{\pgfqpoint{3.556385in}{2.541894in}}%
\pgfpathcurveto{\pgfqpoint{3.556385in}{2.551982in}}{\pgfqpoint{3.552377in}{2.561657in}}{\pgfqpoint{3.545244in}{2.568790in}}%
\pgfpathcurveto{\pgfqpoint{3.538112in}{2.575923in}}{\pgfqpoint{3.528436in}{2.579930in}}{\pgfqpoint{3.518349in}{2.579930in}}%
\pgfpathcurveto{\pgfqpoint{3.508261in}{2.579930in}}{\pgfqpoint{3.498586in}{2.575923in}}{\pgfqpoint{3.491453in}{2.568790in}}%
\pgfpathcurveto{\pgfqpoint{3.484320in}{2.561657in}}{\pgfqpoint{3.480312in}{2.551982in}}{\pgfqpoint{3.480312in}{2.541894in}}%
\pgfpathcurveto{\pgfqpoint{3.480312in}{2.531807in}}{\pgfqpoint{3.484320in}{2.522131in}}{\pgfqpoint{3.491453in}{2.514998in}}%
\pgfpathcurveto{\pgfqpoint{3.498586in}{2.507866in}}{\pgfqpoint{3.508261in}{2.503858in}}{\pgfqpoint{3.518349in}{2.503858in}}%
\pgfpathclose%
\pgfusepath{stroke,fill}%
\end{pgfscope}%
\begin{pgfscope}%
\pgfpathrectangle{\pgfqpoint{0.800000in}{1.363959in}}{\pgfqpoint{3.968000in}{2.024082in}} %
\pgfusepath{clip}%
\pgfsetbuttcap%
\pgfsetroundjoin%
\definecolor{currentfill}{rgb}{0.121569,0.466667,0.705882}%
\pgfsetfillcolor{currentfill}%
\pgfsetlinewidth{1.003750pt}%
\definecolor{currentstroke}{rgb}{0.121569,0.466667,0.705882}%
\pgfsetstrokecolor{currentstroke}%
\pgfsetdash{}{0pt}%
\pgfpathmoveto{\pgfqpoint{3.123833in}{2.205769in}}%
\pgfpathcurveto{\pgfqpoint{3.133920in}{2.205769in}}{\pgfqpoint{3.143596in}{2.209777in}}{\pgfqpoint{3.150728in}{2.216910in}}%
\pgfpathcurveto{\pgfqpoint{3.157861in}{2.224043in}}{\pgfqpoint{3.161869in}{2.233718in}}{\pgfqpoint{3.161869in}{2.243806in}}%
\pgfpathcurveto{\pgfqpoint{3.161869in}{2.253893in}}{\pgfqpoint{3.157861in}{2.263569in}}{\pgfqpoint{3.150728in}{2.270702in}}%
\pgfpathcurveto{\pgfqpoint{3.143596in}{2.277834in}}{\pgfqpoint{3.133920in}{2.281842in}}{\pgfqpoint{3.123833in}{2.281842in}}%
\pgfpathcurveto{\pgfqpoint{3.113745in}{2.281842in}}{\pgfqpoint{3.104070in}{2.277834in}}{\pgfqpoint{3.096937in}{2.270702in}}%
\pgfpathcurveto{\pgfqpoint{3.089804in}{2.263569in}}{\pgfqpoint{3.085796in}{2.253893in}}{\pgfqpoint{3.085796in}{2.243806in}}%
\pgfpathcurveto{\pgfqpoint{3.085796in}{2.233718in}}{\pgfqpoint{3.089804in}{2.224043in}}{\pgfqpoint{3.096937in}{2.216910in}}%
\pgfpathcurveto{\pgfqpoint{3.104070in}{2.209777in}}{\pgfqpoint{3.113745in}{2.205769in}}{\pgfqpoint{3.123833in}{2.205769in}}%
\pgfpathclose%
\pgfusepath{stroke,fill}%
\end{pgfscope}%
\begin{pgfscope}%
\pgfpathrectangle{\pgfqpoint{0.800000in}{1.363959in}}{\pgfqpoint{3.968000in}{2.024082in}} %
\pgfusepath{clip}%
\pgfsetbuttcap%
\pgfsetroundjoin%
\definecolor{currentfill}{rgb}{0.121569,0.466667,0.705882}%
\pgfsetfillcolor{currentfill}%
\pgfsetlinewidth{1.003750pt}%
\definecolor{currentstroke}{rgb}{0.121569,0.466667,0.705882}%
\pgfsetstrokecolor{currentstroke}%
\pgfsetdash{}{0pt}%
\pgfpathmoveto{\pgfqpoint{2.028413in}{2.902812in}}%
\pgfpathcurveto{\pgfqpoint{2.038501in}{2.902812in}}{\pgfqpoint{2.048176in}{2.906820in}}{\pgfqpoint{2.055309in}{2.913953in}}%
\pgfpathcurveto{\pgfqpoint{2.062442in}{2.921085in}}{\pgfqpoint{2.066450in}{2.930761in}}{\pgfqpoint{2.066450in}{2.940848in}}%
\pgfpathcurveto{\pgfqpoint{2.066450in}{2.950936in}}{\pgfqpoint{2.062442in}{2.960611in}}{\pgfqpoint{2.055309in}{2.967744in}}%
\pgfpathcurveto{\pgfqpoint{2.048176in}{2.974877in}}{\pgfqpoint{2.038501in}{2.978885in}}{\pgfqpoint{2.028413in}{2.978885in}}%
\pgfpathcurveto{\pgfqpoint{2.018326in}{2.978885in}}{\pgfqpoint{2.008651in}{2.974877in}}{\pgfqpoint{2.001518in}{2.967744in}}%
\pgfpathcurveto{\pgfqpoint{1.994385in}{2.960611in}}{\pgfqpoint{1.990377in}{2.950936in}}{\pgfqpoint{1.990377in}{2.940848in}}%
\pgfpathcurveto{\pgfqpoint{1.990377in}{2.930761in}}{\pgfqpoint{1.994385in}{2.921085in}}{\pgfqpoint{2.001518in}{2.913953in}}%
\pgfpathcurveto{\pgfqpoint{2.008651in}{2.906820in}}{\pgfqpoint{2.018326in}{2.902812in}}{\pgfqpoint{2.028413in}{2.902812in}}%
\pgfpathclose%
\pgfusepath{stroke,fill}%
\end{pgfscope}%
\begin{pgfscope}%
\pgfpathrectangle{\pgfqpoint{0.800000in}{1.363959in}}{\pgfqpoint{3.968000in}{2.024082in}} %
\pgfusepath{clip}%
\pgfsetbuttcap%
\pgfsetroundjoin%
\definecolor{currentfill}{rgb}{1.000000,0.498039,0.054902}%
\pgfsetfillcolor{currentfill}%
\pgfsetlinewidth{1.003750pt}%
\definecolor{currentstroke}{rgb}{1.000000,0.498039,0.054902}%
\pgfsetstrokecolor{currentstroke}%
\pgfsetdash{}{0pt}%
\pgfpathmoveto{\pgfqpoint{1.451979in}{1.956883in}}%
\pgfpathcurveto{\pgfqpoint{1.462067in}{1.956883in}}{\pgfqpoint{1.471742in}{1.960891in}}{\pgfqpoint{1.478875in}{1.968023in}}%
\pgfpathcurveto{\pgfqpoint{1.486008in}{1.975156in}}{\pgfqpoint{1.490015in}{1.984832in}}{\pgfqpoint{1.490015in}{1.994919in}}%
\pgfpathcurveto{\pgfqpoint{1.490015in}{2.005006in}}{\pgfqpoint{1.486008in}{2.014682in}}{\pgfqpoint{1.478875in}{2.021815in}}%
\pgfpathcurveto{\pgfqpoint{1.471742in}{2.028948in}}{\pgfqpoint{1.462067in}{2.032955in}}{\pgfqpoint{1.451979in}{2.032955in}}%
\pgfpathcurveto{\pgfqpoint{1.441892in}{2.032955in}}{\pgfqpoint{1.432216in}{2.028948in}}{\pgfqpoint{1.425083in}{2.021815in}}%
\pgfpathcurveto{\pgfqpoint{1.417951in}{2.014682in}}{\pgfqpoint{1.413943in}{2.005006in}}{\pgfqpoint{1.413943in}{1.994919in}}%
\pgfpathcurveto{\pgfqpoint{1.413943in}{1.984832in}}{\pgfqpoint{1.417951in}{1.975156in}}{\pgfqpoint{1.425083in}{1.968023in}}%
\pgfpathcurveto{\pgfqpoint{1.432216in}{1.960891in}}{\pgfqpoint{1.441892in}{1.956883in}}{\pgfqpoint{1.451979in}{1.956883in}}%
\pgfpathclose%
\pgfusepath{stroke,fill}%
\end{pgfscope}%
\begin{pgfscope}%
\pgfpathrectangle{\pgfqpoint{0.800000in}{1.363959in}}{\pgfqpoint{3.968000in}{2.024082in}} %
\pgfusepath{clip}%
\pgfsetbuttcap%
\pgfsetroundjoin%
\definecolor{currentfill}{rgb}{1.000000,0.498039,0.054902}%
\pgfsetfillcolor{currentfill}%
\pgfsetlinewidth{1.003750pt}%
\definecolor{currentstroke}{rgb}{1.000000,0.498039,0.054902}%
\pgfsetstrokecolor{currentstroke}%
\pgfsetdash{}{0pt}%
\pgfpathmoveto{\pgfqpoint{3.807162in}{3.078055in}}%
\pgfpathcurveto{\pgfqpoint{3.817249in}{3.078055in}}{\pgfqpoint{3.826925in}{3.082063in}}{\pgfqpoint{3.834058in}{3.089196in}}%
\pgfpathcurveto{\pgfqpoint{3.841190in}{3.096329in}}{\pgfqpoint{3.845198in}{3.106004in}}{\pgfqpoint{3.845198in}{3.116092in}}%
\pgfpathcurveto{\pgfqpoint{3.845198in}{3.126179in}}{\pgfqpoint{3.841190in}{3.135855in}}{\pgfqpoint{3.834058in}{3.142987in}}%
\pgfpathcurveto{\pgfqpoint{3.826925in}{3.150120in}}{\pgfqpoint{3.817249in}{3.154128in}}{\pgfqpoint{3.807162in}{3.154128in}}%
\pgfpathcurveto{\pgfqpoint{3.797075in}{3.154128in}}{\pgfqpoint{3.787399in}{3.150120in}}{\pgfqpoint{3.780266in}{3.142987in}}%
\pgfpathcurveto{\pgfqpoint{3.773133in}{3.135855in}}{\pgfqpoint{3.769126in}{3.126179in}}{\pgfqpoint{3.769126in}{3.116092in}}%
\pgfpathcurveto{\pgfqpoint{3.769126in}{3.106004in}}{\pgfqpoint{3.773133in}{3.096329in}}{\pgfqpoint{3.780266in}{3.089196in}}%
\pgfpathcurveto{\pgfqpoint{3.787399in}{3.082063in}}{\pgfqpoint{3.797075in}{3.078055in}}{\pgfqpoint{3.807162in}{3.078055in}}%
\pgfpathclose%
\pgfusepath{stroke,fill}%
\end{pgfscope}%
\begin{pgfscope}%
\pgfpathrectangle{\pgfqpoint{0.800000in}{1.363959in}}{\pgfqpoint{3.968000in}{2.024082in}} %
\pgfusepath{clip}%
\pgfsetbuttcap%
\pgfsetroundjoin%
\definecolor{currentfill}{rgb}{1.000000,0.498039,0.054902}%
\pgfsetfillcolor{currentfill}%
\pgfsetlinewidth{1.003750pt}%
\definecolor{currentstroke}{rgb}{1.000000,0.498039,0.054902}%
\pgfsetstrokecolor{currentstroke}%
\pgfsetdash{}{0pt}%
\pgfpathmoveto{\pgfqpoint{1.207184in}{2.407006in}}%
\pgfpathcurveto{\pgfqpoint{1.217271in}{2.407006in}}{\pgfqpoint{1.226947in}{2.411014in}}{\pgfqpoint{1.234080in}{2.418147in}}%
\pgfpathcurveto{\pgfqpoint{1.241213in}{2.425280in}}{\pgfqpoint{1.245220in}{2.434955in}}{\pgfqpoint{1.245220in}{2.445043in}}%
\pgfpathcurveto{\pgfqpoint{1.245220in}{2.455130in}}{\pgfqpoint{1.241213in}{2.464806in}}{\pgfqpoint{1.234080in}{2.471938in}}%
\pgfpathcurveto{\pgfqpoint{1.226947in}{2.479071in}}{\pgfqpoint{1.217271in}{2.483079in}}{\pgfqpoint{1.207184in}{2.483079in}}%
\pgfpathcurveto{\pgfqpoint{1.197097in}{2.483079in}}{\pgfqpoint{1.187421in}{2.479071in}}{\pgfqpoint{1.180288in}{2.471938in}}%
\pgfpathcurveto{\pgfqpoint{1.173155in}{2.464806in}}{\pgfqpoint{1.169148in}{2.455130in}}{\pgfqpoint{1.169148in}{2.445043in}}%
\pgfpathcurveto{\pgfqpoint{1.169148in}{2.434955in}}{\pgfqpoint{1.173155in}{2.425280in}}{\pgfqpoint{1.180288in}{2.418147in}}%
\pgfpathcurveto{\pgfqpoint{1.187421in}{2.411014in}}{\pgfqpoint{1.197097in}{2.407006in}}{\pgfqpoint{1.207184in}{2.407006in}}%
\pgfpathclose%
\pgfusepath{stroke,fill}%
\end{pgfscope}%
\begin{pgfscope}%
\pgfpathrectangle{\pgfqpoint{0.800000in}{1.363959in}}{\pgfqpoint{3.968000in}{2.024082in}} %
\pgfusepath{clip}%
\pgfsetbuttcap%
\pgfsetroundjoin%
\definecolor{currentfill}{rgb}{1.000000,0.498039,0.054902}%
\pgfsetfillcolor{currentfill}%
\pgfsetlinewidth{1.003750pt}%
\definecolor{currentstroke}{rgb}{1.000000,0.498039,0.054902}%
\pgfsetstrokecolor{currentstroke}%
\pgfsetdash{}{0pt}%
\pgfpathmoveto{\pgfqpoint{3.870907in}{2.584296in}}%
\pgfpathcurveto{\pgfqpoint{3.880995in}{2.584296in}}{\pgfqpoint{3.890670in}{2.588304in}}{\pgfqpoint{3.897803in}{2.595437in}}%
\pgfpathcurveto{\pgfqpoint{3.904936in}{2.602570in}}{\pgfqpoint{3.908944in}{2.612245in}}{\pgfqpoint{3.908944in}{2.622333in}}%
\pgfpathcurveto{\pgfqpoint{3.908944in}{2.632420in}}{\pgfqpoint{3.904936in}{2.642095in}}{\pgfqpoint{3.897803in}{2.649228in}}%
\pgfpathcurveto{\pgfqpoint{3.890670in}{2.656361in}}{\pgfqpoint{3.880995in}{2.660369in}}{\pgfqpoint{3.870907in}{2.660369in}}%
\pgfpathcurveto{\pgfqpoint{3.860820in}{2.660369in}}{\pgfqpoint{3.851145in}{2.656361in}}{\pgfqpoint{3.844012in}{2.649228in}}%
\pgfpathcurveto{\pgfqpoint{3.836879in}{2.642095in}}{\pgfqpoint{3.832871in}{2.632420in}}{\pgfqpoint{3.832871in}{2.622333in}}%
\pgfpathcurveto{\pgfqpoint{3.832871in}{2.612245in}}{\pgfqpoint{3.836879in}{2.602570in}}{\pgfqpoint{3.844012in}{2.595437in}}%
\pgfpathcurveto{\pgfqpoint{3.851145in}{2.588304in}}{\pgfqpoint{3.860820in}{2.584296in}}{\pgfqpoint{3.870907in}{2.584296in}}%
\pgfpathclose%
\pgfusepath{stroke,fill}%
\end{pgfscope}%
\begin{pgfscope}%
\pgfpathrectangle{\pgfqpoint{0.800000in}{1.363959in}}{\pgfqpoint{3.968000in}{2.024082in}} %
\pgfusepath{clip}%
\pgfsetbuttcap%
\pgfsetroundjoin%
\definecolor{currentfill}{rgb}{1.000000,0.498039,0.054902}%
\pgfsetfillcolor{currentfill}%
\pgfsetlinewidth{1.003750pt}%
\definecolor{currentstroke}{rgb}{1.000000,0.498039,0.054902}%
\pgfsetstrokecolor{currentstroke}%
\pgfsetdash{}{0pt}%
\pgfpathmoveto{\pgfqpoint{1.647743in}{2.415422in}}%
\pgfpathcurveto{\pgfqpoint{1.657830in}{2.415422in}}{\pgfqpoint{1.667506in}{2.419429in}}{\pgfqpoint{1.674639in}{2.426562in}}%
\pgfpathcurveto{\pgfqpoint{1.681772in}{2.433695in}}{\pgfqpoint{1.685779in}{2.443371in}}{\pgfqpoint{1.685779in}{2.453458in}}%
\pgfpathcurveto{\pgfqpoint{1.685779in}{2.463545in}}{\pgfqpoint{1.681772in}{2.473221in}}{\pgfqpoint{1.674639in}{2.480354in}}%
\pgfpathcurveto{\pgfqpoint{1.667506in}{2.487487in}}{\pgfqpoint{1.657830in}{2.491494in}}{\pgfqpoint{1.647743in}{2.491494in}}%
\pgfpathcurveto{\pgfqpoint{1.637656in}{2.491494in}}{\pgfqpoint{1.627980in}{2.487487in}}{\pgfqpoint{1.620847in}{2.480354in}}%
\pgfpathcurveto{\pgfqpoint{1.613714in}{2.473221in}}{\pgfqpoint{1.609707in}{2.463545in}}{\pgfqpoint{1.609707in}{2.453458in}}%
\pgfpathcurveto{\pgfqpoint{1.609707in}{2.443371in}}{\pgfqpoint{1.613714in}{2.433695in}}{\pgfqpoint{1.620847in}{2.426562in}}%
\pgfpathcurveto{\pgfqpoint{1.627980in}{2.419429in}}{\pgfqpoint{1.637656in}{2.415422in}}{\pgfqpoint{1.647743in}{2.415422in}}%
\pgfpathclose%
\pgfusepath{stroke,fill}%
\end{pgfscope}%
\begin{pgfscope}%
\pgfpathrectangle{\pgfqpoint{0.800000in}{1.363959in}}{\pgfqpoint{3.968000in}{2.024082in}} %
\pgfusepath{clip}%
\pgfsetbuttcap%
\pgfsetroundjoin%
\definecolor{currentfill}{rgb}{1.000000,0.498039,0.054902}%
\pgfsetfillcolor{currentfill}%
\pgfsetlinewidth{1.003750pt}%
\definecolor{currentstroke}{rgb}{1.000000,0.498039,0.054902}%
\pgfsetstrokecolor{currentstroke}%
\pgfsetdash{}{0pt}%
\pgfpathmoveto{\pgfqpoint{3.995955in}{2.378778in}}%
\pgfpathcurveto{\pgfqpoint{4.006042in}{2.378778in}}{\pgfqpoint{4.015717in}{2.382785in}}{\pgfqpoint{4.022850in}{2.389918in}}%
\pgfpathcurveto{\pgfqpoint{4.029983in}{2.397051in}}{\pgfqpoint{4.033991in}{2.406727in}}{\pgfqpoint{4.033991in}{2.416814in}}%
\pgfpathcurveto{\pgfqpoint{4.033991in}{2.426901in}}{\pgfqpoint{4.029983in}{2.436577in}}{\pgfqpoint{4.022850in}{2.443710in}}%
\pgfpathcurveto{\pgfqpoint{4.015717in}{2.450842in}}{\pgfqpoint{4.006042in}{2.454850in}}{\pgfqpoint{3.995955in}{2.454850in}}%
\pgfpathcurveto{\pgfqpoint{3.985867in}{2.454850in}}{\pgfqpoint{3.976192in}{2.450842in}}{\pgfqpoint{3.969059in}{2.443710in}}%
\pgfpathcurveto{\pgfqpoint{3.961926in}{2.436577in}}{\pgfqpoint{3.957918in}{2.426901in}}{\pgfqpoint{3.957918in}{2.416814in}}%
\pgfpathcurveto{\pgfqpoint{3.957918in}{2.406727in}}{\pgfqpoint{3.961926in}{2.397051in}}{\pgfqpoint{3.969059in}{2.389918in}}%
\pgfpathcurveto{\pgfqpoint{3.976192in}{2.382785in}}{\pgfqpoint{3.985867in}{2.378778in}}{\pgfqpoint{3.995955in}{2.378778in}}%
\pgfpathclose%
\pgfusepath{stroke,fill}%
\end{pgfscope}%
\begin{pgfscope}%
\pgfpathrectangle{\pgfqpoint{0.800000in}{1.363959in}}{\pgfqpoint{3.968000in}{2.024082in}} %
\pgfusepath{clip}%
\pgfsetbuttcap%
\pgfsetroundjoin%
\definecolor{currentfill}{rgb}{1.000000,0.498039,0.054902}%
\pgfsetfillcolor{currentfill}%
\pgfsetlinewidth{1.003750pt}%
\definecolor{currentstroke}{rgb}{1.000000,0.498039,0.054902}%
\pgfsetstrokecolor{currentstroke}%
\pgfsetdash{}{0pt}%
\pgfpathmoveto{\pgfqpoint{1.566390in}{1.696011in}}%
\pgfpathcurveto{\pgfqpoint{1.576477in}{1.696011in}}{\pgfqpoint{1.586153in}{1.700019in}}{\pgfqpoint{1.593286in}{1.707152in}}%
\pgfpathcurveto{\pgfqpoint{1.600418in}{1.714285in}}{\pgfqpoint{1.604426in}{1.723960in}}{\pgfqpoint{1.604426in}{1.734048in}}%
\pgfpathcurveto{\pgfqpoint{1.604426in}{1.744135in}}{\pgfqpoint{1.600418in}{1.753810in}}{\pgfqpoint{1.593286in}{1.760943in}}%
\pgfpathcurveto{\pgfqpoint{1.586153in}{1.768076in}}{\pgfqpoint{1.576477in}{1.772084in}}{\pgfqpoint{1.566390in}{1.772084in}}%
\pgfpathcurveto{\pgfqpoint{1.556303in}{1.772084in}}{\pgfqpoint{1.546627in}{1.768076in}}{\pgfqpoint{1.539494in}{1.760943in}}%
\pgfpathcurveto{\pgfqpoint{1.532361in}{1.753810in}}{\pgfqpoint{1.528354in}{1.744135in}}{\pgfqpoint{1.528354in}{1.734048in}}%
\pgfpathcurveto{\pgfqpoint{1.528354in}{1.723960in}}{\pgfqpoint{1.532361in}{1.714285in}}{\pgfqpoint{1.539494in}{1.707152in}}%
\pgfpathcurveto{\pgfqpoint{1.546627in}{1.700019in}}{\pgfqpoint{1.556303in}{1.696011in}}{\pgfqpoint{1.566390in}{1.696011in}}%
\pgfpathclose%
\pgfusepath{stroke,fill}%
\end{pgfscope}%
\begin{pgfscope}%
\pgfpathrectangle{\pgfqpoint{0.800000in}{1.363959in}}{\pgfqpoint{3.968000in}{2.024082in}} %
\pgfusepath{clip}%
\pgfsetbuttcap%
\pgfsetroundjoin%
\definecolor{currentfill}{rgb}{1.000000,0.498039,0.054902}%
\pgfsetfillcolor{currentfill}%
\pgfsetlinewidth{1.003750pt}%
\definecolor{currentstroke}{rgb}{1.000000,0.498039,0.054902}%
\pgfsetstrokecolor{currentstroke}%
\pgfsetdash{}{0pt}%
\pgfpathmoveto{\pgfqpoint{3.857262in}{1.453527in}}%
\pgfpathcurveto{\pgfqpoint{3.867350in}{1.453527in}}{\pgfqpoint{3.877025in}{1.457535in}}{\pgfqpoint{3.884158in}{1.464668in}}%
\pgfpathcurveto{\pgfqpoint{3.891291in}{1.471801in}}{\pgfqpoint{3.895299in}{1.481476in}}{\pgfqpoint{3.895299in}{1.491564in}}%
\pgfpathcurveto{\pgfqpoint{3.895299in}{1.501651in}}{\pgfqpoint{3.891291in}{1.511327in}}{\pgfqpoint{3.884158in}{1.518459in}}%
\pgfpathcurveto{\pgfqpoint{3.877025in}{1.525592in}}{\pgfqpoint{3.867350in}{1.529600in}}{\pgfqpoint{3.857262in}{1.529600in}}%
\pgfpathcurveto{\pgfqpoint{3.847175in}{1.529600in}}{\pgfqpoint{3.837500in}{1.525592in}}{\pgfqpoint{3.830367in}{1.518459in}}%
\pgfpathcurveto{\pgfqpoint{3.823234in}{1.511327in}}{\pgfqpoint{3.819226in}{1.501651in}}{\pgfqpoint{3.819226in}{1.491564in}}%
\pgfpathcurveto{\pgfqpoint{3.819226in}{1.481476in}}{\pgfqpoint{3.823234in}{1.471801in}}{\pgfqpoint{3.830367in}{1.464668in}}%
\pgfpathcurveto{\pgfqpoint{3.837500in}{1.457535in}}{\pgfqpoint{3.847175in}{1.453527in}}{\pgfqpoint{3.857262in}{1.453527in}}%
\pgfpathclose%
\pgfusepath{stroke,fill}%
\end{pgfscope}%
\begin{pgfscope}%
\pgfpathrectangle{\pgfqpoint{0.800000in}{1.363959in}}{\pgfqpoint{3.968000in}{2.024082in}} %
\pgfusepath{clip}%
\pgfsetbuttcap%
\pgfsetroundjoin%
\definecolor{currentfill}{rgb}{1.000000,0.498039,0.054902}%
\pgfsetfillcolor{currentfill}%
\pgfsetlinewidth{1.003750pt}%
\definecolor{currentstroke}{rgb}{1.000000,0.498039,0.054902}%
\pgfsetstrokecolor{currentstroke}%
\pgfsetdash{}{0pt}%
\pgfpathmoveto{\pgfqpoint{4.275337in}{2.080544in}}%
\pgfpathcurveto{\pgfqpoint{4.285425in}{2.080544in}}{\pgfqpoint{4.295100in}{2.084552in}}{\pgfqpoint{4.302233in}{2.091685in}}%
\pgfpathcurveto{\pgfqpoint{4.309366in}{2.098818in}}{\pgfqpoint{4.313374in}{2.108493in}}{\pgfqpoint{4.313374in}{2.118580in}}%
\pgfpathcurveto{\pgfqpoint{4.313374in}{2.128668in}}{\pgfqpoint{4.309366in}{2.138343in}}{\pgfqpoint{4.302233in}{2.145476in}}%
\pgfpathcurveto{\pgfqpoint{4.295100in}{2.152609in}}{\pgfqpoint{4.285425in}{2.156617in}}{\pgfqpoint{4.275337in}{2.156617in}}%
\pgfpathcurveto{\pgfqpoint{4.265250in}{2.156617in}}{\pgfqpoint{4.255574in}{2.152609in}}{\pgfqpoint{4.248442in}{2.145476in}}%
\pgfpathcurveto{\pgfqpoint{4.241309in}{2.138343in}}{\pgfqpoint{4.237301in}{2.128668in}}{\pgfqpoint{4.237301in}{2.118580in}}%
\pgfpathcurveto{\pgfqpoint{4.237301in}{2.108493in}}{\pgfqpoint{4.241309in}{2.098818in}}{\pgfqpoint{4.248442in}{2.091685in}}%
\pgfpathcurveto{\pgfqpoint{4.255574in}{2.084552in}}{\pgfqpoint{4.265250in}{2.080544in}}{\pgfqpoint{4.275337in}{2.080544in}}%
\pgfpathclose%
\pgfusepath{stroke,fill}%
\end{pgfscope}%
\begin{pgfscope}%
\pgfpathrectangle{\pgfqpoint{0.800000in}{1.363959in}}{\pgfqpoint{3.968000in}{2.024082in}} %
\pgfusepath{clip}%
\pgfsetbuttcap%
\pgfsetroundjoin%
\definecolor{currentfill}{rgb}{1.000000,0.498039,0.054902}%
\pgfsetfillcolor{currentfill}%
\pgfsetlinewidth{1.003750pt}%
\definecolor{currentstroke}{rgb}{1.000000,0.498039,0.054902}%
\pgfsetstrokecolor{currentstroke}%
\pgfsetdash{}{0pt}%
\pgfpathmoveto{\pgfqpoint{3.801294in}{3.005378in}}%
\pgfpathcurveto{\pgfqpoint{3.811381in}{3.005378in}}{\pgfqpoint{3.821057in}{3.009386in}}{\pgfqpoint{3.828190in}{3.016519in}}%
\pgfpathcurveto{\pgfqpoint{3.835323in}{3.023652in}}{\pgfqpoint{3.839330in}{3.033327in}}{\pgfqpoint{3.839330in}{3.043415in}}%
\pgfpathcurveto{\pgfqpoint{3.839330in}{3.053502in}}{\pgfqpoint{3.835323in}{3.063177in}}{\pgfqpoint{3.828190in}{3.070310in}}%
\pgfpathcurveto{\pgfqpoint{3.821057in}{3.077443in}}{\pgfqpoint{3.811381in}{3.081451in}}{\pgfqpoint{3.801294in}{3.081451in}}%
\pgfpathcurveto{\pgfqpoint{3.791207in}{3.081451in}}{\pgfqpoint{3.781531in}{3.077443in}}{\pgfqpoint{3.774398in}{3.070310in}}%
\pgfpathcurveto{\pgfqpoint{3.767266in}{3.063177in}}{\pgfqpoint{3.763258in}{3.053502in}}{\pgfqpoint{3.763258in}{3.043415in}}%
\pgfpathcurveto{\pgfqpoint{3.763258in}{3.033327in}}{\pgfqpoint{3.767266in}{3.023652in}}{\pgfqpoint{3.774398in}{3.016519in}}%
\pgfpathcurveto{\pgfqpoint{3.781531in}{3.009386in}}{\pgfqpoint{3.791207in}{3.005378in}}{\pgfqpoint{3.801294in}{3.005378in}}%
\pgfpathclose%
\pgfusepath{stroke,fill}%
\end{pgfscope}%
\begin{pgfscope}%
\pgfpathrectangle{\pgfqpoint{0.800000in}{1.363959in}}{\pgfqpoint{3.968000in}{2.024082in}} %
\pgfusepath{clip}%
\pgfsetbuttcap%
\pgfsetroundjoin%
\definecolor{currentfill}{rgb}{1.000000,0.498039,0.054902}%
\pgfsetfillcolor{currentfill}%
\pgfsetlinewidth{1.003750pt}%
\definecolor{currentstroke}{rgb}{1.000000,0.498039,0.054902}%
\pgfsetstrokecolor{currentstroke}%
\pgfsetdash{}{0pt}%
\pgfpathmoveto{\pgfqpoint{3.900837in}{2.688891in}}%
\pgfpathcurveto{\pgfqpoint{3.910924in}{2.688891in}}{\pgfqpoint{3.920600in}{2.692899in}}{\pgfqpoint{3.927732in}{2.700032in}}%
\pgfpathcurveto{\pgfqpoint{3.934865in}{2.707165in}}{\pgfqpoint{3.938873in}{2.716840in}}{\pgfqpoint{3.938873in}{2.726928in}}%
\pgfpathcurveto{\pgfqpoint{3.938873in}{2.737015in}}{\pgfqpoint{3.934865in}{2.746690in}}{\pgfqpoint{3.927732in}{2.753823in}}%
\pgfpathcurveto{\pgfqpoint{3.920600in}{2.760956in}}{\pgfqpoint{3.910924in}{2.764964in}}{\pgfqpoint{3.900837in}{2.764964in}}%
\pgfpathcurveto{\pgfqpoint{3.890749in}{2.764964in}}{\pgfqpoint{3.881074in}{2.760956in}}{\pgfqpoint{3.873941in}{2.753823in}}%
\pgfpathcurveto{\pgfqpoint{3.866808in}{2.746690in}}{\pgfqpoint{3.862800in}{2.737015in}}{\pgfqpoint{3.862800in}{2.726928in}}%
\pgfpathcurveto{\pgfqpoint{3.862800in}{2.716840in}}{\pgfqpoint{3.866808in}{2.707165in}}{\pgfqpoint{3.873941in}{2.700032in}}%
\pgfpathcurveto{\pgfqpoint{3.881074in}{2.692899in}}{\pgfqpoint{3.890749in}{2.688891in}}{\pgfqpoint{3.900837in}{2.688891in}}%
\pgfpathclose%
\pgfusepath{stroke,fill}%
\end{pgfscope}%
\begin{pgfscope}%
\pgfpathrectangle{\pgfqpoint{0.800000in}{1.363959in}}{\pgfqpoint{3.968000in}{2.024082in}} %
\pgfusepath{clip}%
\pgfsetbuttcap%
\pgfsetroundjoin%
\definecolor{currentfill}{rgb}{1.000000,0.498039,0.054902}%
\pgfsetfillcolor{currentfill}%
\pgfsetlinewidth{1.003750pt}%
\definecolor{currentstroke}{rgb}{1.000000,0.498039,0.054902}%
\pgfsetstrokecolor{currentstroke}%
\pgfsetdash{}{0pt}%
\pgfpathmoveto{\pgfqpoint{1.396638in}{2.685654in}}%
\pgfpathcurveto{\pgfqpoint{1.406725in}{2.685654in}}{\pgfqpoint{1.416401in}{2.689662in}}{\pgfqpoint{1.423533in}{2.696794in}}%
\pgfpathcurveto{\pgfqpoint{1.430666in}{2.703927in}}{\pgfqpoint{1.434674in}{2.713603in}}{\pgfqpoint{1.434674in}{2.723690in}}%
\pgfpathcurveto{\pgfqpoint{1.434674in}{2.733777in}}{\pgfqpoint{1.430666in}{2.743453in}}{\pgfqpoint{1.423533in}{2.750586in}}%
\pgfpathcurveto{\pgfqpoint{1.416401in}{2.757719in}}{\pgfqpoint{1.406725in}{2.761726in}}{\pgfqpoint{1.396638in}{2.761726in}}%
\pgfpathcurveto{\pgfqpoint{1.386550in}{2.761726in}}{\pgfqpoint{1.376875in}{2.757719in}}{\pgfqpoint{1.369742in}{2.750586in}}%
\pgfpathcurveto{\pgfqpoint{1.362609in}{2.743453in}}{\pgfqpoint{1.358601in}{2.733777in}}{\pgfqpoint{1.358601in}{2.723690in}}%
\pgfpathcurveto{\pgfqpoint{1.358601in}{2.713603in}}{\pgfqpoint{1.362609in}{2.703927in}}{\pgfqpoint{1.369742in}{2.696794in}}%
\pgfpathcurveto{\pgfqpoint{1.376875in}{2.689662in}}{\pgfqpoint{1.386550in}{2.685654in}}{\pgfqpoint{1.396638in}{2.685654in}}%
\pgfpathclose%
\pgfusepath{stroke,fill}%
\end{pgfscope}%
\begin{pgfscope}%
\pgfpathrectangle{\pgfqpoint{0.800000in}{1.363959in}}{\pgfqpoint{3.968000in}{2.024082in}} %
\pgfusepath{clip}%
\pgfsetbuttcap%
\pgfsetroundjoin%
\definecolor{currentfill}{rgb}{1.000000,0.498039,0.054902}%
\pgfsetfillcolor{currentfill}%
\pgfsetlinewidth{1.003750pt}%
\definecolor{currentstroke}{rgb}{1.000000,0.498039,0.054902}%
\pgfsetstrokecolor{currentstroke}%
\pgfsetdash{}{0pt}%
\pgfpathmoveto{\pgfqpoint{3.823905in}{1.748691in}}%
\pgfpathcurveto{\pgfqpoint{3.833993in}{1.748691in}}{\pgfqpoint{3.843668in}{1.752699in}}{\pgfqpoint{3.850801in}{1.759832in}}%
\pgfpathcurveto{\pgfqpoint{3.857934in}{1.766964in}}{\pgfqpoint{3.861942in}{1.776640in}}{\pgfqpoint{3.861942in}{1.786727in}}%
\pgfpathcurveto{\pgfqpoint{3.861942in}{1.796815in}}{\pgfqpoint{3.857934in}{1.806490in}}{\pgfqpoint{3.850801in}{1.813623in}}%
\pgfpathcurveto{\pgfqpoint{3.843668in}{1.820756in}}{\pgfqpoint{3.833993in}{1.824764in}}{\pgfqpoint{3.823905in}{1.824764in}}%
\pgfpathcurveto{\pgfqpoint{3.813818in}{1.824764in}}{\pgfqpoint{3.804142in}{1.820756in}}{\pgfqpoint{3.797010in}{1.813623in}}%
\pgfpathcurveto{\pgfqpoint{3.789877in}{1.806490in}}{\pgfqpoint{3.785869in}{1.796815in}}{\pgfqpoint{3.785869in}{1.786727in}}%
\pgfpathcurveto{\pgfqpoint{3.785869in}{1.776640in}}{\pgfqpoint{3.789877in}{1.766964in}}{\pgfqpoint{3.797010in}{1.759832in}}%
\pgfpathcurveto{\pgfqpoint{3.804142in}{1.752699in}}{\pgfqpoint{3.813818in}{1.748691in}}{\pgfqpoint{3.823905in}{1.748691in}}%
\pgfpathclose%
\pgfusepath{stroke,fill}%
\end{pgfscope}%
\begin{pgfscope}%
\pgfpathrectangle{\pgfqpoint{0.800000in}{1.363959in}}{\pgfqpoint{3.968000in}{2.024082in}} %
\pgfusepath{clip}%
\pgfsetbuttcap%
\pgfsetroundjoin%
\definecolor{currentfill}{rgb}{1.000000,0.498039,0.054902}%
\pgfsetfillcolor{currentfill}%
\pgfsetlinewidth{1.003750pt}%
\definecolor{currentstroke}{rgb}{1.000000,0.498039,0.054902}%
\pgfsetstrokecolor{currentstroke}%
\pgfsetdash{}{0pt}%
\pgfpathmoveto{\pgfqpoint{4.128305in}{2.221117in}}%
\pgfpathcurveto{\pgfqpoint{4.138392in}{2.221117in}}{\pgfqpoint{4.148068in}{2.225125in}}{\pgfqpoint{4.155201in}{2.232258in}}%
\pgfpathcurveto{\pgfqpoint{4.162333in}{2.239391in}}{\pgfqpoint{4.166341in}{2.249066in}}{\pgfqpoint{4.166341in}{2.259154in}}%
\pgfpathcurveto{\pgfqpoint{4.166341in}{2.269241in}}{\pgfqpoint{4.162333in}{2.278916in}}{\pgfqpoint{4.155201in}{2.286049in}}%
\pgfpathcurveto{\pgfqpoint{4.148068in}{2.293182in}}{\pgfqpoint{4.138392in}{2.297190in}}{\pgfqpoint{4.128305in}{2.297190in}}%
\pgfpathcurveto{\pgfqpoint{4.118217in}{2.297190in}}{\pgfqpoint{4.108542in}{2.293182in}}{\pgfqpoint{4.101409in}{2.286049in}}%
\pgfpathcurveto{\pgfqpoint{4.094276in}{2.278916in}}{\pgfqpoint{4.090268in}{2.269241in}}{\pgfqpoint{4.090268in}{2.259154in}}%
\pgfpathcurveto{\pgfqpoint{4.090268in}{2.249066in}}{\pgfqpoint{4.094276in}{2.239391in}}{\pgfqpoint{4.101409in}{2.232258in}}%
\pgfpathcurveto{\pgfqpoint{4.108542in}{2.225125in}}{\pgfqpoint{4.118217in}{2.221117in}}{\pgfqpoint{4.128305in}{2.221117in}}%
\pgfpathclose%
\pgfusepath{stroke,fill}%
\end{pgfscope}%
\begin{pgfscope}%
\pgfpathrectangle{\pgfqpoint{0.800000in}{1.363959in}}{\pgfqpoint{3.968000in}{2.024082in}} %
\pgfusepath{clip}%
\pgfsetbuttcap%
\pgfsetroundjoin%
\definecolor{currentfill}{rgb}{1.000000,0.498039,0.054902}%
\pgfsetfillcolor{currentfill}%
\pgfsetlinewidth{1.003750pt}%
\definecolor{currentstroke}{rgb}{1.000000,0.498039,0.054902}%
\pgfsetstrokecolor{currentstroke}%
\pgfsetdash{}{0pt}%
\pgfpathmoveto{\pgfqpoint{4.330305in}{2.481653in}}%
\pgfpathcurveto{\pgfqpoint{4.340392in}{2.481653in}}{\pgfqpoint{4.350068in}{2.485661in}}{\pgfqpoint{4.357200in}{2.492794in}}%
\pgfpathcurveto{\pgfqpoint{4.364333in}{2.499926in}}{\pgfqpoint{4.368341in}{2.509602in}}{\pgfqpoint{4.368341in}{2.519689in}}%
\pgfpathcurveto{\pgfqpoint{4.368341in}{2.529777in}}{\pgfqpoint{4.364333in}{2.539452in}}{\pgfqpoint{4.357200in}{2.546585in}}%
\pgfpathcurveto{\pgfqpoint{4.350068in}{2.553718in}}{\pgfqpoint{4.340392in}{2.557726in}}{\pgfqpoint{4.330305in}{2.557726in}}%
\pgfpathcurveto{\pgfqpoint{4.320217in}{2.557726in}}{\pgfqpoint{4.310542in}{2.553718in}}{\pgfqpoint{4.303409in}{2.546585in}}%
\pgfpathcurveto{\pgfqpoint{4.296276in}{2.539452in}}{\pgfqpoint{4.292268in}{2.529777in}}{\pgfqpoint{4.292268in}{2.519689in}}%
\pgfpathcurveto{\pgfqpoint{4.292268in}{2.509602in}}{\pgfqpoint{4.296276in}{2.499926in}}{\pgfqpoint{4.303409in}{2.492794in}}%
\pgfpathcurveto{\pgfqpoint{4.310542in}{2.485661in}}{\pgfqpoint{4.320217in}{2.481653in}}{\pgfqpoint{4.330305in}{2.481653in}}%
\pgfpathclose%
\pgfusepath{stroke,fill}%
\end{pgfscope}%
\begin{pgfscope}%
\pgfpathrectangle{\pgfqpoint{0.800000in}{1.363959in}}{\pgfqpoint{3.968000in}{2.024082in}} %
\pgfusepath{clip}%
\pgfsetbuttcap%
\pgfsetroundjoin%
\definecolor{currentfill}{rgb}{1.000000,0.498039,0.054902}%
\pgfsetfillcolor{currentfill}%
\pgfsetlinewidth{1.003750pt}%
\definecolor{currentstroke}{rgb}{1.000000,0.498039,0.054902}%
\pgfsetstrokecolor{currentstroke}%
\pgfsetdash{}{0pt}%
\pgfpathmoveto{\pgfqpoint{4.031374in}{1.777173in}}%
\pgfpathcurveto{\pgfqpoint{4.041461in}{1.777173in}}{\pgfqpoint{4.051137in}{1.781181in}}{\pgfqpoint{4.058270in}{1.788314in}}%
\pgfpathcurveto{\pgfqpoint{4.065403in}{1.795447in}}{\pgfqpoint{4.069410in}{1.805122in}}{\pgfqpoint{4.069410in}{1.815210in}}%
\pgfpathcurveto{\pgfqpoint{4.069410in}{1.825297in}}{\pgfqpoint{4.065403in}{1.834973in}}{\pgfqpoint{4.058270in}{1.842106in}}%
\pgfpathcurveto{\pgfqpoint{4.051137in}{1.849238in}}{\pgfqpoint{4.041461in}{1.853246in}}{\pgfqpoint{4.031374in}{1.853246in}}%
\pgfpathcurveto{\pgfqpoint{4.021287in}{1.853246in}}{\pgfqpoint{4.011611in}{1.849238in}}{\pgfqpoint{4.004478in}{1.842106in}}%
\pgfpathcurveto{\pgfqpoint{3.997346in}{1.834973in}}{\pgfqpoint{3.993338in}{1.825297in}}{\pgfqpoint{3.993338in}{1.815210in}}%
\pgfpathcurveto{\pgfqpoint{3.993338in}{1.805122in}}{\pgfqpoint{3.997346in}{1.795447in}}{\pgfqpoint{4.004478in}{1.788314in}}%
\pgfpathcurveto{\pgfqpoint{4.011611in}{1.781181in}}{\pgfqpoint{4.021287in}{1.777173in}}{\pgfqpoint{4.031374in}{1.777173in}}%
\pgfpathclose%
\pgfusepath{stroke,fill}%
\end{pgfscope}%
\begin{pgfscope}%
\pgfpathrectangle{\pgfqpoint{0.800000in}{1.363959in}}{\pgfqpoint{3.968000in}{2.024082in}} %
\pgfusepath{clip}%
\pgfsetbuttcap%
\pgfsetroundjoin%
\definecolor{currentfill}{rgb}{1.000000,0.498039,0.054902}%
\pgfsetfillcolor{currentfill}%
\pgfsetlinewidth{1.003750pt}%
\definecolor{currentstroke}{rgb}{1.000000,0.498039,0.054902}%
\pgfsetstrokecolor{currentstroke}%
\pgfsetdash{}{0pt}%
\pgfpathmoveto{\pgfqpoint{4.341806in}{1.858924in}}%
\pgfpathcurveto{\pgfqpoint{4.351893in}{1.858924in}}{\pgfqpoint{4.361568in}{1.862932in}}{\pgfqpoint{4.368701in}{1.870065in}}%
\pgfpathcurveto{\pgfqpoint{4.375834in}{1.877198in}}{\pgfqpoint{4.379842in}{1.886873in}}{\pgfqpoint{4.379842in}{1.896961in}}%
\pgfpathcurveto{\pgfqpoint{4.379842in}{1.907048in}}{\pgfqpoint{4.375834in}{1.916724in}}{\pgfqpoint{4.368701in}{1.923857in}}%
\pgfpathcurveto{\pgfqpoint{4.361568in}{1.930989in}}{\pgfqpoint{4.351893in}{1.934997in}}{\pgfqpoint{4.341806in}{1.934997in}}%
\pgfpathcurveto{\pgfqpoint{4.331718in}{1.934997in}}{\pgfqpoint{4.322043in}{1.930989in}}{\pgfqpoint{4.314910in}{1.923857in}}%
\pgfpathcurveto{\pgfqpoint{4.307777in}{1.916724in}}{\pgfqpoint{4.303769in}{1.907048in}}{\pgfqpoint{4.303769in}{1.896961in}}%
\pgfpathcurveto{\pgfqpoint{4.303769in}{1.886873in}}{\pgfqpoint{4.307777in}{1.877198in}}{\pgfqpoint{4.314910in}{1.870065in}}%
\pgfpathcurveto{\pgfqpoint{4.322043in}{1.862932in}}{\pgfqpoint{4.331718in}{1.858924in}}{\pgfqpoint{4.341806in}{1.858924in}}%
\pgfpathclose%
\pgfusepath{stroke,fill}%
\end{pgfscope}%
\begin{pgfscope}%
\pgfpathrectangle{\pgfqpoint{0.800000in}{1.363959in}}{\pgfqpoint{3.968000in}{2.024082in}} %
\pgfusepath{clip}%
\pgfsetbuttcap%
\pgfsetroundjoin%
\definecolor{currentfill}{rgb}{1.000000,0.498039,0.054902}%
\pgfsetfillcolor{currentfill}%
\pgfsetlinewidth{1.003750pt}%
\definecolor{currentstroke}{rgb}{1.000000,0.498039,0.054902}%
\pgfsetstrokecolor{currentstroke}%
\pgfsetdash{}{0pt}%
\pgfpathmoveto{\pgfqpoint{1.465850in}{1.885860in}}%
\pgfpathcurveto{\pgfqpoint{1.475937in}{1.885860in}}{\pgfqpoint{1.485613in}{1.889868in}}{\pgfqpoint{1.492746in}{1.897001in}}%
\pgfpathcurveto{\pgfqpoint{1.499879in}{1.904134in}}{\pgfqpoint{1.503886in}{1.913809in}}{\pgfqpoint{1.503886in}{1.923896in}}%
\pgfpathcurveto{\pgfqpoint{1.503886in}{1.933984in}}{\pgfqpoint{1.499879in}{1.943659in}}{\pgfqpoint{1.492746in}{1.950792in}}%
\pgfpathcurveto{\pgfqpoint{1.485613in}{1.957925in}}{\pgfqpoint{1.475937in}{1.961933in}}{\pgfqpoint{1.465850in}{1.961933in}}%
\pgfpathcurveto{\pgfqpoint{1.455763in}{1.961933in}}{\pgfqpoint{1.446087in}{1.957925in}}{\pgfqpoint{1.438954in}{1.950792in}}%
\pgfpathcurveto{\pgfqpoint{1.431822in}{1.943659in}}{\pgfqpoint{1.427814in}{1.933984in}}{\pgfqpoint{1.427814in}{1.923896in}}%
\pgfpathcurveto{\pgfqpoint{1.427814in}{1.913809in}}{\pgfqpoint{1.431822in}{1.904134in}}{\pgfqpoint{1.438954in}{1.897001in}}%
\pgfpathcurveto{\pgfqpoint{1.446087in}{1.889868in}}{\pgfqpoint{1.455763in}{1.885860in}}{\pgfqpoint{1.465850in}{1.885860in}}%
\pgfpathclose%
\pgfusepath{stroke,fill}%
\end{pgfscope}%
\begin{pgfscope}%
\pgfpathrectangle{\pgfqpoint{0.800000in}{1.363959in}}{\pgfqpoint{3.968000in}{2.024082in}} %
\pgfusepath{clip}%
\pgfsetbuttcap%
\pgfsetroundjoin%
\definecolor{currentfill}{rgb}{1.000000,0.498039,0.054902}%
\pgfsetfillcolor{currentfill}%
\pgfsetlinewidth{1.003750pt}%
\definecolor{currentstroke}{rgb}{1.000000,0.498039,0.054902}%
\pgfsetstrokecolor{currentstroke}%
\pgfsetdash{}{0pt}%
\pgfpathmoveto{\pgfqpoint{3.980510in}{2.796138in}}%
\pgfpathcurveto{\pgfqpoint{3.990597in}{2.796138in}}{\pgfqpoint{4.000272in}{2.800145in}}{\pgfqpoint{4.007405in}{2.807278in}}%
\pgfpathcurveto{\pgfqpoint{4.014538in}{2.814411in}}{\pgfqpoint{4.018546in}{2.824087in}}{\pgfqpoint{4.018546in}{2.834174in}}%
\pgfpathcurveto{\pgfqpoint{4.018546in}{2.844261in}}{\pgfqpoint{4.014538in}{2.853937in}}{\pgfqpoint{4.007405in}{2.861070in}}%
\pgfpathcurveto{\pgfqpoint{4.000272in}{2.868202in}}{\pgfqpoint{3.990597in}{2.872210in}}{\pgfqpoint{3.980510in}{2.872210in}}%
\pgfpathcurveto{\pgfqpoint{3.970422in}{2.872210in}}{\pgfqpoint{3.960747in}{2.868202in}}{\pgfqpoint{3.953614in}{2.861070in}}%
\pgfpathcurveto{\pgfqpoint{3.946481in}{2.853937in}}{\pgfqpoint{3.942473in}{2.844261in}}{\pgfqpoint{3.942473in}{2.834174in}}%
\pgfpathcurveto{\pgfqpoint{3.942473in}{2.824087in}}{\pgfqpoint{3.946481in}{2.814411in}}{\pgfqpoint{3.953614in}{2.807278in}}%
\pgfpathcurveto{\pgfqpoint{3.960747in}{2.800145in}}{\pgfqpoint{3.970422in}{2.796138in}}{\pgfqpoint{3.980510in}{2.796138in}}%
\pgfpathclose%
\pgfusepath{stroke,fill}%
\end{pgfscope}%
\begin{pgfscope}%
\pgfpathrectangle{\pgfqpoint{0.800000in}{1.363959in}}{\pgfqpoint{3.968000in}{2.024082in}} %
\pgfusepath{clip}%
\pgfsetbuttcap%
\pgfsetroundjoin%
\definecolor{currentfill}{rgb}{1.000000,0.498039,0.054902}%
\pgfsetfillcolor{currentfill}%
\pgfsetlinewidth{1.003750pt}%
\definecolor{currentstroke}{rgb}{1.000000,0.498039,0.054902}%
\pgfsetstrokecolor{currentstroke}%
\pgfsetdash{}{0pt}%
\pgfpathmoveto{\pgfqpoint{1.704972in}{2.249230in}}%
\pgfpathcurveto{\pgfqpoint{1.715059in}{2.249230in}}{\pgfqpoint{1.724735in}{2.253237in}}{\pgfqpoint{1.731867in}{2.260370in}}%
\pgfpathcurveto{\pgfqpoint{1.739000in}{2.267503in}}{\pgfqpoint{1.743008in}{2.277179in}}{\pgfqpoint{1.743008in}{2.287266in}}%
\pgfpathcurveto{\pgfqpoint{1.743008in}{2.297353in}}{\pgfqpoint{1.739000in}{2.307029in}}{\pgfqpoint{1.731867in}{2.314162in}}%
\pgfpathcurveto{\pgfqpoint{1.724735in}{2.321294in}}{\pgfqpoint{1.715059in}{2.325302in}}{\pgfqpoint{1.704972in}{2.325302in}}%
\pgfpathcurveto{\pgfqpoint{1.694884in}{2.325302in}}{\pgfqpoint{1.685209in}{2.321294in}}{\pgfqpoint{1.678076in}{2.314162in}}%
\pgfpathcurveto{\pgfqpoint{1.670943in}{2.307029in}}{\pgfqpoint{1.666935in}{2.297353in}}{\pgfqpoint{1.666935in}{2.287266in}}%
\pgfpathcurveto{\pgfqpoint{1.666935in}{2.277179in}}{\pgfqpoint{1.670943in}{2.267503in}}{\pgfqpoint{1.678076in}{2.260370in}}%
\pgfpathcurveto{\pgfqpoint{1.685209in}{2.253237in}}{\pgfqpoint{1.694884in}{2.249230in}}{\pgfqpoint{1.704972in}{2.249230in}}%
\pgfpathclose%
\pgfusepath{stroke,fill}%
\end{pgfscope}%
\begin{pgfscope}%
\pgfpathrectangle{\pgfqpoint{0.800000in}{1.363959in}}{\pgfqpoint{3.968000in}{2.024082in}} %
\pgfusepath{clip}%
\pgfsetbuttcap%
\pgfsetroundjoin%
\definecolor{currentfill}{rgb}{1.000000,0.498039,0.054902}%
\pgfsetfillcolor{currentfill}%
\pgfsetlinewidth{1.003750pt}%
\definecolor{currentstroke}{rgb}{1.000000,0.498039,0.054902}%
\pgfsetstrokecolor{currentstroke}%
\pgfsetdash{}{0pt}%
\pgfpathmoveto{\pgfqpoint{1.576691in}{1.503014in}}%
\pgfpathcurveto{\pgfqpoint{1.586778in}{1.503014in}}{\pgfqpoint{1.596454in}{1.507022in}}{\pgfqpoint{1.603587in}{1.514155in}}%
\pgfpathcurveto{\pgfqpoint{1.610720in}{1.521287in}}{\pgfqpoint{1.614727in}{1.530963in}}{\pgfqpoint{1.614727in}{1.541050in}}%
\pgfpathcurveto{\pgfqpoint{1.614727in}{1.551138in}}{\pgfqpoint{1.610720in}{1.560813in}}{\pgfqpoint{1.603587in}{1.567946in}}%
\pgfpathcurveto{\pgfqpoint{1.596454in}{1.575079in}}{\pgfqpoint{1.586778in}{1.579087in}}{\pgfqpoint{1.576691in}{1.579087in}}%
\pgfpathcurveto{\pgfqpoint{1.566604in}{1.579087in}}{\pgfqpoint{1.556928in}{1.575079in}}{\pgfqpoint{1.549795in}{1.567946in}}%
\pgfpathcurveto{\pgfqpoint{1.542662in}{1.560813in}}{\pgfqpoint{1.538655in}{1.551138in}}{\pgfqpoint{1.538655in}{1.541050in}}%
\pgfpathcurveto{\pgfqpoint{1.538655in}{1.530963in}}{\pgfqpoint{1.542662in}{1.521287in}}{\pgfqpoint{1.549795in}{1.514155in}}%
\pgfpathcurveto{\pgfqpoint{1.556928in}{1.507022in}}{\pgfqpoint{1.566604in}{1.503014in}}{\pgfqpoint{1.576691in}{1.503014in}}%
\pgfpathclose%
\pgfusepath{stroke,fill}%
\end{pgfscope}%
\begin{pgfscope}%
\pgfpathrectangle{\pgfqpoint{0.800000in}{1.363959in}}{\pgfqpoint{3.968000in}{2.024082in}} %
\pgfusepath{clip}%
\pgfsetbuttcap%
\pgfsetroundjoin%
\definecolor{currentfill}{rgb}{1.000000,0.498039,0.054902}%
\pgfsetfillcolor{currentfill}%
\pgfsetlinewidth{1.003750pt}%
\definecolor{currentstroke}{rgb}{1.000000,0.498039,0.054902}%
\pgfsetstrokecolor{currentstroke}%
\pgfsetdash{}{0pt}%
\pgfpathmoveto{\pgfqpoint{1.630242in}{2.345370in}}%
\pgfpathcurveto{\pgfqpoint{1.640329in}{2.345370in}}{\pgfqpoint{1.650005in}{2.349378in}}{\pgfqpoint{1.657138in}{2.356511in}}%
\pgfpathcurveto{\pgfqpoint{1.664270in}{2.363644in}}{\pgfqpoint{1.668278in}{2.373319in}}{\pgfqpoint{1.668278in}{2.383407in}}%
\pgfpathcurveto{\pgfqpoint{1.668278in}{2.393494in}}{\pgfqpoint{1.664270in}{2.403169in}}{\pgfqpoint{1.657138in}{2.410302in}}%
\pgfpathcurveto{\pgfqpoint{1.650005in}{2.417435in}}{\pgfqpoint{1.640329in}{2.421443in}}{\pgfqpoint{1.630242in}{2.421443in}}%
\pgfpathcurveto{\pgfqpoint{1.620155in}{2.421443in}}{\pgfqpoint{1.610479in}{2.417435in}}{\pgfqpoint{1.603346in}{2.410302in}}%
\pgfpathcurveto{\pgfqpoint{1.596213in}{2.403169in}}{\pgfqpoint{1.592206in}{2.393494in}}{\pgfqpoint{1.592206in}{2.383407in}}%
\pgfpathcurveto{\pgfqpoint{1.592206in}{2.373319in}}{\pgfqpoint{1.596213in}{2.363644in}}{\pgfqpoint{1.603346in}{2.356511in}}%
\pgfpathcurveto{\pgfqpoint{1.610479in}{2.349378in}}{\pgfqpoint{1.620155in}{2.345370in}}{\pgfqpoint{1.630242in}{2.345370in}}%
\pgfpathclose%
\pgfusepath{stroke,fill}%
\end{pgfscope}%
\begin{pgfscope}%
\pgfpathrectangle{\pgfqpoint{0.800000in}{1.363959in}}{\pgfqpoint{3.968000in}{2.024082in}} %
\pgfusepath{clip}%
\pgfsetbuttcap%
\pgfsetroundjoin%
\definecolor{currentfill}{rgb}{1.000000,0.498039,0.054902}%
\pgfsetfillcolor{currentfill}%
\pgfsetlinewidth{1.003750pt}%
\definecolor{currentstroke}{rgb}{1.000000,0.498039,0.054902}%
\pgfsetstrokecolor{currentstroke}%
\pgfsetdash{}{0pt}%
\pgfpathmoveto{\pgfqpoint{1.612835in}{1.792197in}}%
\pgfpathcurveto{\pgfqpoint{1.622922in}{1.792197in}}{\pgfqpoint{1.632598in}{1.796204in}}{\pgfqpoint{1.639731in}{1.803337in}}%
\pgfpathcurveto{\pgfqpoint{1.646864in}{1.810470in}}{\pgfqpoint{1.650871in}{1.820146in}}{\pgfqpoint{1.650871in}{1.830233in}}%
\pgfpathcurveto{\pgfqpoint{1.650871in}{1.840320in}}{\pgfqpoint{1.646864in}{1.849996in}}{\pgfqpoint{1.639731in}{1.857129in}}%
\pgfpathcurveto{\pgfqpoint{1.632598in}{1.864262in}}{\pgfqpoint{1.622922in}{1.868269in}}{\pgfqpoint{1.612835in}{1.868269in}}%
\pgfpathcurveto{\pgfqpoint{1.602748in}{1.868269in}}{\pgfqpoint{1.593072in}{1.864262in}}{\pgfqpoint{1.585939in}{1.857129in}}%
\pgfpathcurveto{\pgfqpoint{1.578807in}{1.849996in}}{\pgfqpoint{1.574799in}{1.840320in}}{\pgfqpoint{1.574799in}{1.830233in}}%
\pgfpathcurveto{\pgfqpoint{1.574799in}{1.820146in}}{\pgfqpoint{1.578807in}{1.810470in}}{\pgfqpoint{1.585939in}{1.803337in}}%
\pgfpathcurveto{\pgfqpoint{1.593072in}{1.796204in}}{\pgfqpoint{1.602748in}{1.792197in}}{\pgfqpoint{1.612835in}{1.792197in}}%
\pgfpathclose%
\pgfusepath{stroke,fill}%
\end{pgfscope}%
\begin{pgfscope}%
\pgfpathrectangle{\pgfqpoint{0.800000in}{1.363959in}}{\pgfqpoint{3.968000in}{2.024082in}} %
\pgfusepath{clip}%
\pgfsetbuttcap%
\pgfsetroundjoin%
\definecolor{currentfill}{rgb}{1.000000,0.498039,0.054902}%
\pgfsetfillcolor{currentfill}%
\pgfsetlinewidth{1.003750pt}%
\definecolor{currentstroke}{rgb}{1.000000,0.498039,0.054902}%
\pgfsetstrokecolor{currentstroke}%
\pgfsetdash{}{0pt}%
\pgfpathmoveto{\pgfqpoint{1.455969in}{2.886479in}}%
\pgfpathcurveto{\pgfqpoint{1.466057in}{2.886479in}}{\pgfqpoint{1.475732in}{2.890486in}}{\pgfqpoint{1.482865in}{2.897619in}}%
\pgfpathcurveto{\pgfqpoint{1.489998in}{2.904752in}}{\pgfqpoint{1.494006in}{2.914428in}}{\pgfqpoint{1.494006in}{2.924515in}}%
\pgfpathcurveto{\pgfqpoint{1.494006in}{2.934602in}}{\pgfqpoint{1.489998in}{2.944278in}}{\pgfqpoint{1.482865in}{2.951411in}}%
\pgfpathcurveto{\pgfqpoint{1.475732in}{2.958544in}}{\pgfqpoint{1.466057in}{2.962551in}}{\pgfqpoint{1.455969in}{2.962551in}}%
\pgfpathcurveto{\pgfqpoint{1.445882in}{2.962551in}}{\pgfqpoint{1.436206in}{2.958544in}}{\pgfqpoint{1.429074in}{2.951411in}}%
\pgfpathcurveto{\pgfqpoint{1.421941in}{2.944278in}}{\pgfqpoint{1.417933in}{2.934602in}}{\pgfqpoint{1.417933in}{2.924515in}}%
\pgfpathcurveto{\pgfqpoint{1.417933in}{2.914428in}}{\pgfqpoint{1.421941in}{2.904752in}}{\pgfqpoint{1.429074in}{2.897619in}}%
\pgfpathcurveto{\pgfqpoint{1.436206in}{2.890486in}}{\pgfqpoint{1.445882in}{2.886479in}}{\pgfqpoint{1.455969in}{2.886479in}}%
\pgfpathclose%
\pgfusepath{stroke,fill}%
\end{pgfscope}%
\begin{pgfscope}%
\pgfpathrectangle{\pgfqpoint{0.800000in}{1.363959in}}{\pgfqpoint{3.968000in}{2.024082in}} %
\pgfusepath{clip}%
\pgfsetbuttcap%
\pgfsetroundjoin%
\definecolor{currentfill}{rgb}{1.000000,0.498039,0.054902}%
\pgfsetfillcolor{currentfill}%
\pgfsetlinewidth{1.003750pt}%
\definecolor{currentstroke}{rgb}{1.000000,0.498039,0.054902}%
\pgfsetstrokecolor{currentstroke}%
\pgfsetdash{}{0pt}%
\pgfpathmoveto{\pgfqpoint{3.889352in}{1.951442in}}%
\pgfpathcurveto{\pgfqpoint{3.899440in}{1.951442in}}{\pgfqpoint{3.909115in}{1.955450in}}{\pgfqpoint{3.916248in}{1.962582in}}%
\pgfpathcurveto{\pgfqpoint{3.923381in}{1.969715in}}{\pgfqpoint{3.927388in}{1.979391in}}{\pgfqpoint{3.927388in}{1.989478in}}%
\pgfpathcurveto{\pgfqpoint{3.927388in}{1.999565in}}{\pgfqpoint{3.923381in}{2.009241in}}{\pgfqpoint{3.916248in}{2.016374in}}%
\pgfpathcurveto{\pgfqpoint{3.909115in}{2.023507in}}{\pgfqpoint{3.899440in}{2.027514in}}{\pgfqpoint{3.889352in}{2.027514in}}%
\pgfpathcurveto{\pgfqpoint{3.879265in}{2.027514in}}{\pgfqpoint{3.869589in}{2.023507in}}{\pgfqpoint{3.862456in}{2.016374in}}%
\pgfpathcurveto{\pgfqpoint{3.855324in}{2.009241in}}{\pgfqpoint{3.851316in}{1.999565in}}{\pgfqpoint{3.851316in}{1.989478in}}%
\pgfpathcurveto{\pgfqpoint{3.851316in}{1.979391in}}{\pgfqpoint{3.855324in}{1.969715in}}{\pgfqpoint{3.862456in}{1.962582in}}%
\pgfpathcurveto{\pgfqpoint{3.869589in}{1.955450in}}{\pgfqpoint{3.879265in}{1.951442in}}{\pgfqpoint{3.889352in}{1.951442in}}%
\pgfpathclose%
\pgfusepath{stroke,fill}%
\end{pgfscope}%
\begin{pgfscope}%
\pgfpathrectangle{\pgfqpoint{0.800000in}{1.363959in}}{\pgfqpoint{3.968000in}{2.024082in}} %
\pgfusepath{clip}%
\pgfsetbuttcap%
\pgfsetroundjoin%
\definecolor{currentfill}{rgb}{1.000000,0.498039,0.054902}%
\pgfsetfillcolor{currentfill}%
\pgfsetlinewidth{1.003750pt}%
\definecolor{currentstroke}{rgb}{1.000000,0.498039,0.054902}%
\pgfsetstrokecolor{currentstroke}%
\pgfsetdash{}{0pt}%
\pgfpathmoveto{\pgfqpoint{1.666496in}{1.448316in}}%
\pgfpathcurveto{\pgfqpoint{1.676584in}{1.448316in}}{\pgfqpoint{1.686259in}{1.452324in}}{\pgfqpoint{1.693392in}{1.459457in}}%
\pgfpathcurveto{\pgfqpoint{1.700525in}{1.466590in}}{\pgfqpoint{1.704532in}{1.476265in}}{\pgfqpoint{1.704532in}{1.486353in}}%
\pgfpathcurveto{\pgfqpoint{1.704532in}{1.496440in}}{\pgfqpoint{1.700525in}{1.506115in}}{\pgfqpoint{1.693392in}{1.513248in}}%
\pgfpathcurveto{\pgfqpoint{1.686259in}{1.520381in}}{\pgfqpoint{1.676584in}{1.524389in}}{\pgfqpoint{1.666496in}{1.524389in}}%
\pgfpathcurveto{\pgfqpoint{1.656409in}{1.524389in}}{\pgfqpoint{1.646733in}{1.520381in}}{\pgfqpoint{1.639600in}{1.513248in}}%
\pgfpathcurveto{\pgfqpoint{1.632468in}{1.506115in}}{\pgfqpoint{1.628460in}{1.496440in}}{\pgfqpoint{1.628460in}{1.486353in}}%
\pgfpathcurveto{\pgfqpoint{1.628460in}{1.476265in}}{\pgfqpoint{1.632468in}{1.466590in}}{\pgfqpoint{1.639600in}{1.459457in}}%
\pgfpathcurveto{\pgfqpoint{1.646733in}{1.452324in}}{\pgfqpoint{1.656409in}{1.448316in}}{\pgfqpoint{1.666496in}{1.448316in}}%
\pgfpathclose%
\pgfusepath{stroke,fill}%
\end{pgfscope}%
\begin{pgfscope}%
\pgfpathrectangle{\pgfqpoint{0.800000in}{1.363959in}}{\pgfqpoint{3.968000in}{2.024082in}} %
\pgfusepath{clip}%
\pgfsetbuttcap%
\pgfsetroundjoin%
\definecolor{currentfill}{rgb}{1.000000,0.498039,0.054902}%
\pgfsetfillcolor{currentfill}%
\pgfsetlinewidth{1.003750pt}%
\definecolor{currentstroke}{rgb}{1.000000,0.498039,0.054902}%
\pgfsetstrokecolor{currentstroke}%
\pgfsetdash{}{0pt}%
\pgfpathmoveto{\pgfqpoint{4.545728in}{2.503767in}}%
\pgfpathcurveto{\pgfqpoint{4.555815in}{2.503767in}}{\pgfqpoint{4.565491in}{2.507774in}}{\pgfqpoint{4.572624in}{2.514907in}}%
\pgfpathcurveto{\pgfqpoint{4.579756in}{2.522040in}}{\pgfqpoint{4.583764in}{2.531716in}}{\pgfqpoint{4.583764in}{2.541803in}}%
\pgfpathcurveto{\pgfqpoint{4.583764in}{2.551890in}}{\pgfqpoint{4.579756in}{2.561566in}}{\pgfqpoint{4.572624in}{2.568699in}}%
\pgfpathcurveto{\pgfqpoint{4.565491in}{2.575831in}}{\pgfqpoint{4.555815in}{2.579839in}}{\pgfqpoint{4.545728in}{2.579839in}}%
\pgfpathcurveto{\pgfqpoint{4.535640in}{2.579839in}}{\pgfqpoint{4.525965in}{2.575831in}}{\pgfqpoint{4.518832in}{2.568699in}}%
\pgfpathcurveto{\pgfqpoint{4.511699in}{2.561566in}}{\pgfqpoint{4.507692in}{2.551890in}}{\pgfqpoint{4.507692in}{2.541803in}}%
\pgfpathcurveto{\pgfqpoint{4.507692in}{2.531716in}}{\pgfqpoint{4.511699in}{2.522040in}}{\pgfqpoint{4.518832in}{2.514907in}}%
\pgfpathcurveto{\pgfqpoint{4.525965in}{2.507774in}}{\pgfqpoint{4.535640in}{2.503767in}}{\pgfqpoint{4.545728in}{2.503767in}}%
\pgfpathclose%
\pgfusepath{stroke,fill}%
\end{pgfscope}%
\begin{pgfscope}%
\pgfpathrectangle{\pgfqpoint{0.800000in}{1.363959in}}{\pgfqpoint{3.968000in}{2.024082in}} %
\pgfusepath{clip}%
\pgfsetbuttcap%
\pgfsetroundjoin%
\definecolor{currentfill}{rgb}{1.000000,0.498039,0.054902}%
\pgfsetfillcolor{currentfill}%
\pgfsetlinewidth{1.003750pt}%
\definecolor{currentstroke}{rgb}{1.000000,0.498039,0.054902}%
\pgfsetstrokecolor{currentstroke}%
\pgfsetdash{}{0pt}%
\pgfpathmoveto{\pgfqpoint{3.993589in}{2.970859in}}%
\pgfpathcurveto{\pgfqpoint{4.003677in}{2.970859in}}{\pgfqpoint{4.013352in}{2.974867in}}{\pgfqpoint{4.020485in}{2.981999in}}%
\pgfpathcurveto{\pgfqpoint{4.027618in}{2.989132in}}{\pgfqpoint{4.031626in}{2.998808in}}{\pgfqpoint{4.031626in}{3.008895in}}%
\pgfpathcurveto{\pgfqpoint{4.031626in}{3.018983in}}{\pgfqpoint{4.027618in}{3.028658in}}{\pgfqpoint{4.020485in}{3.035791in}}%
\pgfpathcurveto{\pgfqpoint{4.013352in}{3.042924in}}{\pgfqpoint{4.003677in}{3.046931in}}{\pgfqpoint{3.993589in}{3.046931in}}%
\pgfpathcurveto{\pgfqpoint{3.983502in}{3.046931in}}{\pgfqpoint{3.973826in}{3.042924in}}{\pgfqpoint{3.966694in}{3.035791in}}%
\pgfpathcurveto{\pgfqpoint{3.959561in}{3.028658in}}{\pgfqpoint{3.955553in}{3.018983in}}{\pgfqpoint{3.955553in}{3.008895in}}%
\pgfpathcurveto{\pgfqpoint{3.955553in}{2.998808in}}{\pgfqpoint{3.959561in}{2.989132in}}{\pgfqpoint{3.966694in}{2.981999in}}%
\pgfpathcurveto{\pgfqpoint{3.973826in}{2.974867in}}{\pgfqpoint{3.983502in}{2.970859in}}{\pgfqpoint{3.993589in}{2.970859in}}%
\pgfpathclose%
\pgfusepath{stroke,fill}%
\end{pgfscope}%
\begin{pgfscope}%
\pgfpathrectangle{\pgfqpoint{0.800000in}{1.363959in}}{\pgfqpoint{3.968000in}{2.024082in}} %
\pgfusepath{clip}%
\pgfsetbuttcap%
\pgfsetroundjoin%
\definecolor{currentfill}{rgb}{1.000000,0.498039,0.054902}%
\pgfsetfillcolor{currentfill}%
\pgfsetlinewidth{1.003750pt}%
\definecolor{currentstroke}{rgb}{1.000000,0.498039,0.054902}%
\pgfsetstrokecolor{currentstroke}%
\pgfsetdash{}{0pt}%
\pgfpathmoveto{\pgfqpoint{1.234343in}{2.107541in}}%
\pgfpathcurveto{\pgfqpoint{1.244430in}{2.107541in}}{\pgfqpoint{1.254105in}{2.111549in}}{\pgfqpoint{1.261238in}{2.118682in}}%
\pgfpathcurveto{\pgfqpoint{1.268371in}{2.125814in}}{\pgfqpoint{1.272379in}{2.135490in}}{\pgfqpoint{1.272379in}{2.145577in}}%
\pgfpathcurveto{\pgfqpoint{1.272379in}{2.155665in}}{\pgfqpoint{1.268371in}{2.165340in}}{\pgfqpoint{1.261238in}{2.172473in}}%
\pgfpathcurveto{\pgfqpoint{1.254105in}{2.179606in}}{\pgfqpoint{1.244430in}{2.183614in}}{\pgfqpoint{1.234343in}{2.183614in}}%
\pgfpathcurveto{\pgfqpoint{1.224255in}{2.183614in}}{\pgfqpoint{1.214580in}{2.179606in}}{\pgfqpoint{1.207447in}{2.172473in}}%
\pgfpathcurveto{\pgfqpoint{1.200314in}{2.165340in}}{\pgfqpoint{1.196306in}{2.155665in}}{\pgfqpoint{1.196306in}{2.145577in}}%
\pgfpathcurveto{\pgfqpoint{1.196306in}{2.135490in}}{\pgfqpoint{1.200314in}{2.125814in}}{\pgfqpoint{1.207447in}{2.118682in}}%
\pgfpathcurveto{\pgfqpoint{1.214580in}{2.111549in}}{\pgfqpoint{1.224255in}{2.107541in}}{\pgfqpoint{1.234343in}{2.107541in}}%
\pgfpathclose%
\pgfusepath{stroke,fill}%
\end{pgfscope}%
\begin{pgfscope}%
\pgfpathrectangle{\pgfqpoint{0.800000in}{1.363959in}}{\pgfqpoint{3.968000in}{2.024082in}} %
\pgfusepath{clip}%
\pgfsetbuttcap%
\pgfsetroundjoin%
\definecolor{currentfill}{rgb}{1.000000,0.498039,0.054902}%
\pgfsetfillcolor{currentfill}%
\pgfsetlinewidth{1.003750pt}%
\definecolor{currentstroke}{rgb}{1.000000,0.498039,0.054902}%
\pgfsetstrokecolor{currentstroke}%
\pgfsetdash{}{0pt}%
\pgfpathmoveto{\pgfqpoint{1.659881in}{2.795552in}}%
\pgfpathcurveto{\pgfqpoint{1.669969in}{2.795552in}}{\pgfqpoint{1.679644in}{2.799559in}}{\pgfqpoint{1.686777in}{2.806692in}}%
\pgfpathcurveto{\pgfqpoint{1.693910in}{2.813825in}}{\pgfqpoint{1.697918in}{2.823501in}}{\pgfqpoint{1.697918in}{2.833588in}}%
\pgfpathcurveto{\pgfqpoint{1.697918in}{2.843675in}}{\pgfqpoint{1.693910in}{2.853351in}}{\pgfqpoint{1.686777in}{2.860484in}}%
\pgfpathcurveto{\pgfqpoint{1.679644in}{2.867617in}}{\pgfqpoint{1.669969in}{2.871624in}}{\pgfqpoint{1.659881in}{2.871624in}}%
\pgfpathcurveto{\pgfqpoint{1.649794in}{2.871624in}}{\pgfqpoint{1.640118in}{2.867617in}}{\pgfqpoint{1.632986in}{2.860484in}}%
\pgfpathcurveto{\pgfqpoint{1.625853in}{2.853351in}}{\pgfqpoint{1.621845in}{2.843675in}}{\pgfqpoint{1.621845in}{2.833588in}}%
\pgfpathcurveto{\pgfqpoint{1.621845in}{2.823501in}}{\pgfqpoint{1.625853in}{2.813825in}}{\pgfqpoint{1.632986in}{2.806692in}}%
\pgfpathcurveto{\pgfqpoint{1.640118in}{2.799559in}}{\pgfqpoint{1.649794in}{2.795552in}}{\pgfqpoint{1.659881in}{2.795552in}}%
\pgfpathclose%
\pgfusepath{stroke,fill}%
\end{pgfscope}%
\begin{pgfscope}%
\pgfpathrectangle{\pgfqpoint{0.800000in}{1.363959in}}{\pgfqpoint{3.968000in}{2.024082in}} %
\pgfusepath{clip}%
\pgfsetbuttcap%
\pgfsetroundjoin%
\definecolor{currentfill}{rgb}{1.000000,0.498039,0.054902}%
\pgfsetfillcolor{currentfill}%
\pgfsetlinewidth{1.003750pt}%
\definecolor{currentstroke}{rgb}{1.000000,0.498039,0.054902}%
\pgfsetstrokecolor{currentstroke}%
\pgfsetdash{}{0pt}%
\pgfpathmoveto{\pgfqpoint{1.504398in}{2.350267in}}%
\pgfpathcurveto{\pgfqpoint{1.514485in}{2.350267in}}{\pgfqpoint{1.524160in}{2.354275in}}{\pgfqpoint{1.531293in}{2.361408in}}%
\pgfpathcurveto{\pgfqpoint{1.538426in}{2.368541in}}{\pgfqpoint{1.542434in}{2.378216in}}{\pgfqpoint{1.542434in}{2.388303in}}%
\pgfpathcurveto{\pgfqpoint{1.542434in}{2.398391in}}{\pgfqpoint{1.538426in}{2.408066in}}{\pgfqpoint{1.531293in}{2.415199in}}%
\pgfpathcurveto{\pgfqpoint{1.524160in}{2.422332in}}{\pgfqpoint{1.514485in}{2.426340in}}{\pgfqpoint{1.504398in}{2.426340in}}%
\pgfpathcurveto{\pgfqpoint{1.494310in}{2.426340in}}{\pgfqpoint{1.484635in}{2.422332in}}{\pgfqpoint{1.477502in}{2.415199in}}%
\pgfpathcurveto{\pgfqpoint{1.470369in}{2.408066in}}{\pgfqpoint{1.466361in}{2.398391in}}{\pgfqpoint{1.466361in}{2.388303in}}%
\pgfpathcurveto{\pgfqpoint{1.466361in}{2.378216in}}{\pgfqpoint{1.470369in}{2.368541in}}{\pgfqpoint{1.477502in}{2.361408in}}%
\pgfpathcurveto{\pgfqpoint{1.484635in}{2.354275in}}{\pgfqpoint{1.494310in}{2.350267in}}{\pgfqpoint{1.504398in}{2.350267in}}%
\pgfpathclose%
\pgfusepath{stroke,fill}%
\end{pgfscope}%
\begin{pgfscope}%
\pgfpathrectangle{\pgfqpoint{0.800000in}{1.363959in}}{\pgfqpoint{3.968000in}{2.024082in}} %
\pgfusepath{clip}%
\pgfsetbuttcap%
\pgfsetroundjoin%
\definecolor{currentfill}{rgb}{1.000000,0.498039,0.054902}%
\pgfsetfillcolor{currentfill}%
\pgfsetlinewidth{1.003750pt}%
\definecolor{currentstroke}{rgb}{1.000000,0.498039,0.054902}%
\pgfsetstrokecolor{currentstroke}%
\pgfsetdash{}{0pt}%
\pgfpathmoveto{\pgfqpoint{4.217126in}{2.405873in}}%
\pgfpathcurveto{\pgfqpoint{4.227213in}{2.405873in}}{\pgfqpoint{4.236889in}{2.409881in}}{\pgfqpoint{4.244022in}{2.417014in}}%
\pgfpathcurveto{\pgfqpoint{4.251155in}{2.424147in}}{\pgfqpoint{4.255162in}{2.433822in}}{\pgfqpoint{4.255162in}{2.443909in}}%
\pgfpathcurveto{\pgfqpoint{4.255162in}{2.453997in}}{\pgfqpoint{4.251155in}{2.463672in}}{\pgfqpoint{4.244022in}{2.470805in}}%
\pgfpathcurveto{\pgfqpoint{4.236889in}{2.477938in}}{\pgfqpoint{4.227213in}{2.481946in}}{\pgfqpoint{4.217126in}{2.481946in}}%
\pgfpathcurveto{\pgfqpoint{4.207039in}{2.481946in}}{\pgfqpoint{4.197363in}{2.477938in}}{\pgfqpoint{4.190230in}{2.470805in}}%
\pgfpathcurveto{\pgfqpoint{4.183098in}{2.463672in}}{\pgfqpoint{4.179090in}{2.453997in}}{\pgfqpoint{4.179090in}{2.443909in}}%
\pgfpathcurveto{\pgfqpoint{4.179090in}{2.433822in}}{\pgfqpoint{4.183098in}{2.424147in}}{\pgfqpoint{4.190230in}{2.417014in}}%
\pgfpathcurveto{\pgfqpoint{4.197363in}{2.409881in}}{\pgfqpoint{4.207039in}{2.405873in}}{\pgfqpoint{4.217126in}{2.405873in}}%
\pgfpathclose%
\pgfusepath{stroke,fill}%
\end{pgfscope}%
\begin{pgfscope}%
\pgfpathrectangle{\pgfqpoint{0.800000in}{1.363959in}}{\pgfqpoint{3.968000in}{2.024082in}} %
\pgfusepath{clip}%
\pgfsetbuttcap%
\pgfsetroundjoin%
\definecolor{currentfill}{rgb}{1.000000,0.498039,0.054902}%
\pgfsetfillcolor{currentfill}%
\pgfsetlinewidth{1.003750pt}%
\definecolor{currentstroke}{rgb}{1.000000,0.498039,0.054902}%
\pgfsetstrokecolor{currentstroke}%
\pgfsetdash{}{0pt}%
\pgfpathmoveto{\pgfqpoint{3.965458in}{1.457115in}}%
\pgfpathcurveto{\pgfqpoint{3.975546in}{1.457115in}}{\pgfqpoint{3.985221in}{1.461123in}}{\pgfqpoint{3.992354in}{1.468256in}}%
\pgfpathcurveto{\pgfqpoint{3.999487in}{1.475389in}}{\pgfqpoint{4.003495in}{1.485064in}}{\pgfqpoint{4.003495in}{1.495151in}}%
\pgfpathcurveto{\pgfqpoint{4.003495in}{1.505239in}}{\pgfqpoint{3.999487in}{1.514914in}}{\pgfqpoint{3.992354in}{1.522047in}}%
\pgfpathcurveto{\pgfqpoint{3.985221in}{1.529180in}}{\pgfqpoint{3.975546in}{1.533188in}}{\pgfqpoint{3.965458in}{1.533188in}}%
\pgfpathcurveto{\pgfqpoint{3.955371in}{1.533188in}}{\pgfqpoint{3.945695in}{1.529180in}}{\pgfqpoint{3.938563in}{1.522047in}}%
\pgfpathcurveto{\pgfqpoint{3.931430in}{1.514914in}}{\pgfqpoint{3.927422in}{1.505239in}}{\pgfqpoint{3.927422in}{1.495151in}}%
\pgfpathcurveto{\pgfqpoint{3.927422in}{1.485064in}}{\pgfqpoint{3.931430in}{1.475389in}}{\pgfqpoint{3.938563in}{1.468256in}}%
\pgfpathcurveto{\pgfqpoint{3.945695in}{1.461123in}}{\pgfqpoint{3.955371in}{1.457115in}}{\pgfqpoint{3.965458in}{1.457115in}}%
\pgfpathclose%
\pgfusepath{stroke,fill}%
\end{pgfscope}%
\begin{pgfscope}%
\pgfpathrectangle{\pgfqpoint{0.800000in}{1.363959in}}{\pgfqpoint{3.968000in}{2.024082in}} %
\pgfusepath{clip}%
\pgfsetbuttcap%
\pgfsetroundjoin%
\definecolor{currentfill}{rgb}{1.000000,0.498039,0.054902}%
\pgfsetfillcolor{currentfill}%
\pgfsetlinewidth{1.003750pt}%
\definecolor{currentstroke}{rgb}{1.000000,0.498039,0.054902}%
\pgfsetstrokecolor{currentstroke}%
\pgfsetdash{}{0pt}%
\pgfpathmoveto{\pgfqpoint{1.367971in}{2.720060in}}%
\pgfpathcurveto{\pgfqpoint{1.378059in}{2.720060in}}{\pgfqpoint{1.387734in}{2.724068in}}{\pgfqpoint{1.394867in}{2.731201in}}%
\pgfpathcurveto{\pgfqpoint{1.402000in}{2.738333in}}{\pgfqpoint{1.406008in}{2.748009in}}{\pgfqpoint{1.406008in}{2.758096in}}%
\pgfpathcurveto{\pgfqpoint{1.406008in}{2.768184in}}{\pgfqpoint{1.402000in}{2.777859in}}{\pgfqpoint{1.394867in}{2.784992in}}%
\pgfpathcurveto{\pgfqpoint{1.387734in}{2.792125in}}{\pgfqpoint{1.378059in}{2.796133in}}{\pgfqpoint{1.367971in}{2.796133in}}%
\pgfpathcurveto{\pgfqpoint{1.357884in}{2.796133in}}{\pgfqpoint{1.348209in}{2.792125in}}{\pgfqpoint{1.341076in}{2.784992in}}%
\pgfpathcurveto{\pgfqpoint{1.333943in}{2.777859in}}{\pgfqpoint{1.329935in}{2.768184in}}{\pgfqpoint{1.329935in}{2.758096in}}%
\pgfpathcurveto{\pgfqpoint{1.329935in}{2.748009in}}{\pgfqpoint{1.333943in}{2.738333in}}{\pgfqpoint{1.341076in}{2.731201in}}%
\pgfpathcurveto{\pgfqpoint{1.348209in}{2.724068in}}{\pgfqpoint{1.357884in}{2.720060in}}{\pgfqpoint{1.367971in}{2.720060in}}%
\pgfpathclose%
\pgfusepath{stroke,fill}%
\end{pgfscope}%
\begin{pgfscope}%
\pgfpathrectangle{\pgfqpoint{0.800000in}{1.363959in}}{\pgfqpoint{3.968000in}{2.024082in}} %
\pgfusepath{clip}%
\pgfsetbuttcap%
\pgfsetroundjoin%
\definecolor{currentfill}{rgb}{1.000000,0.498039,0.054902}%
\pgfsetfillcolor{currentfill}%
\pgfsetlinewidth{1.003750pt}%
\definecolor{currentstroke}{rgb}{1.000000,0.498039,0.054902}%
\pgfsetstrokecolor{currentstroke}%
\pgfsetdash{}{0pt}%
\pgfpathmoveto{\pgfqpoint{1.334691in}{2.264201in}}%
\pgfpathcurveto{\pgfqpoint{1.344778in}{2.264201in}}{\pgfqpoint{1.354454in}{2.268209in}}{\pgfqpoint{1.361587in}{2.275342in}}%
\pgfpathcurveto{\pgfqpoint{1.368720in}{2.282474in}}{\pgfqpoint{1.372727in}{2.292150in}}{\pgfqpoint{1.372727in}{2.302237in}}%
\pgfpathcurveto{\pgfqpoint{1.372727in}{2.312325in}}{\pgfqpoint{1.368720in}{2.322000in}}{\pgfqpoint{1.361587in}{2.329133in}}%
\pgfpathcurveto{\pgfqpoint{1.354454in}{2.336266in}}{\pgfqpoint{1.344778in}{2.340274in}}{\pgfqpoint{1.334691in}{2.340274in}}%
\pgfpathcurveto{\pgfqpoint{1.324604in}{2.340274in}}{\pgfqpoint{1.314928in}{2.336266in}}{\pgfqpoint{1.307795in}{2.329133in}}%
\pgfpathcurveto{\pgfqpoint{1.300663in}{2.322000in}}{\pgfqpoint{1.296655in}{2.312325in}}{\pgfqpoint{1.296655in}{2.302237in}}%
\pgfpathcurveto{\pgfqpoint{1.296655in}{2.292150in}}{\pgfqpoint{1.300663in}{2.282474in}}{\pgfqpoint{1.307795in}{2.275342in}}%
\pgfpathcurveto{\pgfqpoint{1.314928in}{2.268209in}}{\pgfqpoint{1.324604in}{2.264201in}}{\pgfqpoint{1.334691in}{2.264201in}}%
\pgfpathclose%
\pgfusepath{stroke,fill}%
\end{pgfscope}%
\begin{pgfscope}%
\pgfpathrectangle{\pgfqpoint{0.800000in}{1.363959in}}{\pgfqpoint{3.968000in}{2.024082in}} %
\pgfusepath{clip}%
\pgfsetbuttcap%
\pgfsetroundjoin%
\definecolor{currentfill}{rgb}{1.000000,0.498039,0.054902}%
\pgfsetfillcolor{currentfill}%
\pgfsetlinewidth{1.003750pt}%
\definecolor{currentstroke}{rgb}{1.000000,0.498039,0.054902}%
\pgfsetstrokecolor{currentstroke}%
\pgfsetdash{}{0pt}%
\pgfpathmoveto{\pgfqpoint{3.713630in}{1.809866in}}%
\pgfpathcurveto{\pgfqpoint{3.723717in}{1.809866in}}{\pgfqpoint{3.733393in}{1.813874in}}{\pgfqpoint{3.740525in}{1.821007in}}%
\pgfpathcurveto{\pgfqpoint{3.747658in}{1.828140in}}{\pgfqpoint{3.751666in}{1.837815in}}{\pgfqpoint{3.751666in}{1.847902in}}%
\pgfpathcurveto{\pgfqpoint{3.751666in}{1.857990in}}{\pgfqpoint{3.747658in}{1.867665in}}{\pgfqpoint{3.740525in}{1.874798in}}%
\pgfpathcurveto{\pgfqpoint{3.733393in}{1.881931in}}{\pgfqpoint{3.723717in}{1.885939in}}{\pgfqpoint{3.713630in}{1.885939in}}%
\pgfpathcurveto{\pgfqpoint{3.703542in}{1.885939in}}{\pgfqpoint{3.693867in}{1.881931in}}{\pgfqpoint{3.686734in}{1.874798in}}%
\pgfpathcurveto{\pgfqpoint{3.679601in}{1.867665in}}{\pgfqpoint{3.675593in}{1.857990in}}{\pgfqpoint{3.675593in}{1.847902in}}%
\pgfpathcurveto{\pgfqpoint{3.675593in}{1.837815in}}{\pgfqpoint{3.679601in}{1.828140in}}{\pgfqpoint{3.686734in}{1.821007in}}%
\pgfpathcurveto{\pgfqpoint{3.693867in}{1.813874in}}{\pgfqpoint{3.703542in}{1.809866in}}{\pgfqpoint{3.713630in}{1.809866in}}%
\pgfpathclose%
\pgfusepath{stroke,fill}%
\end{pgfscope}%
\begin{pgfscope}%
\pgfpathrectangle{\pgfqpoint{0.800000in}{1.363959in}}{\pgfqpoint{3.968000in}{2.024082in}} %
\pgfusepath{clip}%
\pgfsetbuttcap%
\pgfsetroundjoin%
\definecolor{currentfill}{rgb}{1.000000,0.498039,0.054902}%
\pgfsetfillcolor{currentfill}%
\pgfsetlinewidth{1.003750pt}%
\definecolor{currentstroke}{rgb}{1.000000,0.498039,0.054902}%
\pgfsetstrokecolor{currentstroke}%
\pgfsetdash{}{0pt}%
\pgfpathmoveto{\pgfqpoint{4.225562in}{1.894998in}}%
\pgfpathcurveto{\pgfqpoint{4.235649in}{1.894998in}}{\pgfqpoint{4.245324in}{1.899006in}}{\pgfqpoint{4.252457in}{1.906139in}}%
\pgfpathcurveto{\pgfqpoint{4.259590in}{1.913272in}}{\pgfqpoint{4.263598in}{1.922947in}}{\pgfqpoint{4.263598in}{1.933035in}}%
\pgfpathcurveto{\pgfqpoint{4.263598in}{1.943122in}}{\pgfqpoint{4.259590in}{1.952797in}}{\pgfqpoint{4.252457in}{1.959930in}}%
\pgfpathcurveto{\pgfqpoint{4.245324in}{1.967063in}}{\pgfqpoint{4.235649in}{1.971071in}}{\pgfqpoint{4.225562in}{1.971071in}}%
\pgfpathcurveto{\pgfqpoint{4.215474in}{1.971071in}}{\pgfqpoint{4.205799in}{1.967063in}}{\pgfqpoint{4.198666in}{1.959930in}}%
\pgfpathcurveto{\pgfqpoint{4.191533in}{1.952797in}}{\pgfqpoint{4.187525in}{1.943122in}}{\pgfqpoint{4.187525in}{1.933035in}}%
\pgfpathcurveto{\pgfqpoint{4.187525in}{1.922947in}}{\pgfqpoint{4.191533in}{1.913272in}}{\pgfqpoint{4.198666in}{1.906139in}}%
\pgfpathcurveto{\pgfqpoint{4.205799in}{1.899006in}}{\pgfqpoint{4.215474in}{1.894998in}}{\pgfqpoint{4.225562in}{1.894998in}}%
\pgfpathclose%
\pgfusepath{stroke,fill}%
\end{pgfscope}%
\begin{pgfscope}%
\pgfpathrectangle{\pgfqpoint{0.800000in}{1.363959in}}{\pgfqpoint{3.968000in}{2.024082in}} %
\pgfusepath{clip}%
\pgfsetbuttcap%
\pgfsetroundjoin%
\definecolor{currentfill}{rgb}{1.000000,0.498039,0.054902}%
\pgfsetfillcolor{currentfill}%
\pgfsetlinewidth{1.003750pt}%
\definecolor{currentstroke}{rgb}{1.000000,0.498039,0.054902}%
\pgfsetstrokecolor{currentstroke}%
\pgfsetdash{}{0pt}%
\pgfpathmoveto{\pgfqpoint{3.797148in}{1.548798in}}%
\pgfpathcurveto{\pgfqpoint{3.807235in}{1.548798in}}{\pgfqpoint{3.816910in}{1.552805in}}{\pgfqpoint{3.824043in}{1.559938in}}%
\pgfpathcurveto{\pgfqpoint{3.831176in}{1.567071in}}{\pgfqpoint{3.835184in}{1.576746in}}{\pgfqpoint{3.835184in}{1.586834in}}%
\pgfpathcurveto{\pgfqpoint{3.835184in}{1.596921in}}{\pgfqpoint{3.831176in}{1.606597in}}{\pgfqpoint{3.824043in}{1.613730in}}%
\pgfpathcurveto{\pgfqpoint{3.816910in}{1.620862in}}{\pgfqpoint{3.807235in}{1.624870in}}{\pgfqpoint{3.797148in}{1.624870in}}%
\pgfpathcurveto{\pgfqpoint{3.787060in}{1.624870in}}{\pgfqpoint{3.777385in}{1.620862in}}{\pgfqpoint{3.770252in}{1.613730in}}%
\pgfpathcurveto{\pgfqpoint{3.763119in}{1.606597in}}{\pgfqpoint{3.759111in}{1.596921in}}{\pgfqpoint{3.759111in}{1.586834in}}%
\pgfpathcurveto{\pgfqpoint{3.759111in}{1.576746in}}{\pgfqpoint{3.763119in}{1.567071in}}{\pgfqpoint{3.770252in}{1.559938in}}%
\pgfpathcurveto{\pgfqpoint{3.777385in}{1.552805in}}{\pgfqpoint{3.787060in}{1.548798in}}{\pgfqpoint{3.797148in}{1.548798in}}%
\pgfpathclose%
\pgfusepath{stroke,fill}%
\end{pgfscope}%
\begin{pgfscope}%
\pgfpathrectangle{\pgfqpoint{0.800000in}{1.363959in}}{\pgfqpoint{3.968000in}{2.024082in}} %
\pgfusepath{clip}%
\pgfsetbuttcap%
\pgfsetroundjoin%
\definecolor{currentfill}{rgb}{1.000000,0.498039,0.054902}%
\pgfsetfillcolor{currentfill}%
\pgfsetlinewidth{1.003750pt}%
\definecolor{currentstroke}{rgb}{1.000000,0.498039,0.054902}%
\pgfsetstrokecolor{currentstroke}%
\pgfsetdash{}{0pt}%
\pgfpathmoveto{\pgfqpoint{4.096820in}{2.196336in}}%
\pgfpathcurveto{\pgfqpoint{4.106907in}{2.196336in}}{\pgfqpoint{4.116582in}{2.200343in}}{\pgfqpoint{4.123715in}{2.207476in}}%
\pgfpathcurveto{\pgfqpoint{4.130848in}{2.214609in}}{\pgfqpoint{4.134856in}{2.224285in}}{\pgfqpoint{4.134856in}{2.234372in}}%
\pgfpathcurveto{\pgfqpoint{4.134856in}{2.244459in}}{\pgfqpoint{4.130848in}{2.254135in}}{\pgfqpoint{4.123715in}{2.261268in}}%
\pgfpathcurveto{\pgfqpoint{4.116582in}{2.268400in}}{\pgfqpoint{4.106907in}{2.272408in}}{\pgfqpoint{4.096820in}{2.272408in}}%
\pgfpathcurveto{\pgfqpoint{4.086732in}{2.272408in}}{\pgfqpoint{4.077057in}{2.268400in}}{\pgfqpoint{4.069924in}{2.261268in}}%
\pgfpathcurveto{\pgfqpoint{4.062791in}{2.254135in}}{\pgfqpoint{4.058783in}{2.244459in}}{\pgfqpoint{4.058783in}{2.234372in}}%
\pgfpathcurveto{\pgfqpoint{4.058783in}{2.224285in}}{\pgfqpoint{4.062791in}{2.214609in}}{\pgfqpoint{4.069924in}{2.207476in}}%
\pgfpathcurveto{\pgfqpoint{4.077057in}{2.200343in}}{\pgfqpoint{4.086732in}{2.196336in}}{\pgfqpoint{4.096820in}{2.196336in}}%
\pgfpathclose%
\pgfusepath{stroke,fill}%
\end{pgfscope}%
\begin{pgfscope}%
\pgfpathrectangle{\pgfqpoint{0.800000in}{1.363959in}}{\pgfqpoint{3.968000in}{2.024082in}} %
\pgfusepath{clip}%
\pgfsetbuttcap%
\pgfsetroundjoin%
\definecolor{currentfill}{rgb}{1.000000,0.498039,0.054902}%
\pgfsetfillcolor{currentfill}%
\pgfsetlinewidth{1.003750pt}%
\definecolor{currentstroke}{rgb}{1.000000,0.498039,0.054902}%
\pgfsetstrokecolor{currentstroke}%
\pgfsetdash{}{0pt}%
\pgfpathmoveto{\pgfqpoint{3.697396in}{1.600178in}}%
\pgfpathcurveto{\pgfqpoint{3.707483in}{1.600178in}}{\pgfqpoint{3.717159in}{1.604185in}}{\pgfqpoint{3.724291in}{1.611318in}}%
\pgfpathcurveto{\pgfqpoint{3.731424in}{1.618451in}}{\pgfqpoint{3.735432in}{1.628126in}}{\pgfqpoint{3.735432in}{1.638214in}}%
\pgfpathcurveto{\pgfqpoint{3.735432in}{1.648301in}}{\pgfqpoint{3.731424in}{1.657977in}}{\pgfqpoint{3.724291in}{1.665110in}}%
\pgfpathcurveto{\pgfqpoint{3.717159in}{1.672242in}}{\pgfqpoint{3.707483in}{1.676250in}}{\pgfqpoint{3.697396in}{1.676250in}}%
\pgfpathcurveto{\pgfqpoint{3.687308in}{1.676250in}}{\pgfqpoint{3.677633in}{1.672242in}}{\pgfqpoint{3.670500in}{1.665110in}}%
\pgfpathcurveto{\pgfqpoint{3.663367in}{1.657977in}}{\pgfqpoint{3.659359in}{1.648301in}}{\pgfqpoint{3.659359in}{1.638214in}}%
\pgfpathcurveto{\pgfqpoint{3.659359in}{1.628126in}}{\pgfqpoint{3.663367in}{1.618451in}}{\pgfqpoint{3.670500in}{1.611318in}}%
\pgfpathcurveto{\pgfqpoint{3.677633in}{1.604185in}}{\pgfqpoint{3.687308in}{1.600178in}}{\pgfqpoint{3.697396in}{1.600178in}}%
\pgfpathclose%
\pgfusepath{stroke,fill}%
\end{pgfscope}%
\begin{pgfscope}%
\pgfpathrectangle{\pgfqpoint{0.800000in}{1.363959in}}{\pgfqpoint{3.968000in}{2.024082in}} %
\pgfusepath{clip}%
\pgfsetbuttcap%
\pgfsetroundjoin%
\definecolor{currentfill}{rgb}{1.000000,0.498039,0.054902}%
\pgfsetfillcolor{currentfill}%
\pgfsetlinewidth{1.003750pt}%
\definecolor{currentstroke}{rgb}{1.000000,0.498039,0.054902}%
\pgfsetstrokecolor{currentstroke}%
\pgfsetdash{}{0pt}%
\pgfpathmoveto{\pgfqpoint{4.167219in}{2.579040in}}%
\pgfpathcurveto{\pgfqpoint{4.177307in}{2.579040in}}{\pgfqpoint{4.186982in}{2.583048in}}{\pgfqpoint{4.194115in}{2.590181in}}%
\pgfpathcurveto{\pgfqpoint{4.201248in}{2.597314in}}{\pgfqpoint{4.205256in}{2.606989in}}{\pgfqpoint{4.205256in}{2.617077in}}%
\pgfpathcurveto{\pgfqpoint{4.205256in}{2.627164in}}{\pgfqpoint{4.201248in}{2.636840in}}{\pgfqpoint{4.194115in}{2.643972in}}%
\pgfpathcurveto{\pgfqpoint{4.186982in}{2.651105in}}{\pgfqpoint{4.177307in}{2.655113in}}{\pgfqpoint{4.167219in}{2.655113in}}%
\pgfpathcurveto{\pgfqpoint{4.157132in}{2.655113in}}{\pgfqpoint{4.147457in}{2.651105in}}{\pgfqpoint{4.140324in}{2.643972in}}%
\pgfpathcurveto{\pgfqpoint{4.133191in}{2.636840in}}{\pgfqpoint{4.129183in}{2.627164in}}{\pgfqpoint{4.129183in}{2.617077in}}%
\pgfpathcurveto{\pgfqpoint{4.129183in}{2.606989in}}{\pgfqpoint{4.133191in}{2.597314in}}{\pgfqpoint{4.140324in}{2.590181in}}%
\pgfpathcurveto{\pgfqpoint{4.147457in}{2.583048in}}{\pgfqpoint{4.157132in}{2.579040in}}{\pgfqpoint{4.167219in}{2.579040in}}%
\pgfpathclose%
\pgfusepath{stroke,fill}%
\end{pgfscope}%
\begin{pgfscope}%
\pgfpathrectangle{\pgfqpoint{0.800000in}{1.363959in}}{\pgfqpoint{3.968000in}{2.024082in}} %
\pgfusepath{clip}%
\pgfsetbuttcap%
\pgfsetroundjoin%
\definecolor{currentfill}{rgb}{1.000000,0.498039,0.054902}%
\pgfsetfillcolor{currentfill}%
\pgfsetlinewidth{1.003750pt}%
\definecolor{currentstroke}{rgb}{1.000000,0.498039,0.054902}%
\pgfsetstrokecolor{currentstroke}%
\pgfsetdash{}{0pt}%
\pgfpathmoveto{\pgfqpoint{3.790375in}{2.894610in}}%
\pgfpathcurveto{\pgfqpoint{3.800463in}{2.894610in}}{\pgfqpoint{3.810138in}{2.898617in}}{\pgfqpoint{3.817271in}{2.905750in}}%
\pgfpathcurveto{\pgfqpoint{3.824404in}{2.912883in}}{\pgfqpoint{3.828412in}{2.922559in}}{\pgfqpoint{3.828412in}{2.932646in}}%
\pgfpathcurveto{\pgfqpoint{3.828412in}{2.942733in}}{\pgfqpoint{3.824404in}{2.952409in}}{\pgfqpoint{3.817271in}{2.959542in}}%
\pgfpathcurveto{\pgfqpoint{3.810138in}{2.966674in}}{\pgfqpoint{3.800463in}{2.970682in}}{\pgfqpoint{3.790375in}{2.970682in}}%
\pgfpathcurveto{\pgfqpoint{3.780288in}{2.970682in}}{\pgfqpoint{3.770613in}{2.966674in}}{\pgfqpoint{3.763480in}{2.959542in}}%
\pgfpathcurveto{\pgfqpoint{3.756347in}{2.952409in}}{\pgfqpoint{3.752339in}{2.942733in}}{\pgfqpoint{3.752339in}{2.932646in}}%
\pgfpathcurveto{\pgfqpoint{3.752339in}{2.922559in}}{\pgfqpoint{3.756347in}{2.912883in}}{\pgfqpoint{3.763480in}{2.905750in}}%
\pgfpathcurveto{\pgfqpoint{3.770613in}{2.898617in}}{\pgfqpoint{3.780288in}{2.894610in}}{\pgfqpoint{3.790375in}{2.894610in}}%
\pgfpathclose%
\pgfusepath{stroke,fill}%
\end{pgfscope}%
\begin{pgfscope}%
\pgfpathrectangle{\pgfqpoint{0.800000in}{1.363959in}}{\pgfqpoint{3.968000in}{2.024082in}} %
\pgfusepath{clip}%
\pgfsetbuttcap%
\pgfsetroundjoin%
\definecolor{currentfill}{rgb}{1.000000,0.498039,0.054902}%
\pgfsetfillcolor{currentfill}%
\pgfsetlinewidth{1.003750pt}%
\definecolor{currentstroke}{rgb}{1.000000,0.498039,0.054902}%
\pgfsetstrokecolor{currentstroke}%
\pgfsetdash{}{0pt}%
\pgfpathmoveto{\pgfqpoint{1.732645in}{2.696582in}}%
\pgfpathcurveto{\pgfqpoint{1.742733in}{2.696582in}}{\pgfqpoint{1.752408in}{2.700589in}}{\pgfqpoint{1.759541in}{2.707722in}}%
\pgfpathcurveto{\pgfqpoint{1.766674in}{2.714855in}}{\pgfqpoint{1.770682in}{2.724531in}}{\pgfqpoint{1.770682in}{2.734618in}}%
\pgfpathcurveto{\pgfqpoint{1.770682in}{2.744705in}}{\pgfqpoint{1.766674in}{2.754381in}}{\pgfqpoint{1.759541in}{2.761514in}}%
\pgfpathcurveto{\pgfqpoint{1.752408in}{2.768646in}}{\pgfqpoint{1.742733in}{2.772654in}}{\pgfqpoint{1.732645in}{2.772654in}}%
\pgfpathcurveto{\pgfqpoint{1.722558in}{2.772654in}}{\pgfqpoint{1.712883in}{2.768646in}}{\pgfqpoint{1.705750in}{2.761514in}}%
\pgfpathcurveto{\pgfqpoint{1.698617in}{2.754381in}}{\pgfqpoint{1.694609in}{2.744705in}}{\pgfqpoint{1.694609in}{2.734618in}}%
\pgfpathcurveto{\pgfqpoint{1.694609in}{2.724531in}}{\pgfqpoint{1.698617in}{2.714855in}}{\pgfqpoint{1.705750in}{2.707722in}}%
\pgfpathcurveto{\pgfqpoint{1.712883in}{2.700589in}}{\pgfqpoint{1.722558in}{2.696582in}}{\pgfqpoint{1.732645in}{2.696582in}}%
\pgfpathclose%
\pgfusepath{stroke,fill}%
\end{pgfscope}%
\begin{pgfscope}%
\pgfpathrectangle{\pgfqpoint{0.800000in}{1.363959in}}{\pgfqpoint{3.968000in}{2.024082in}} %
\pgfusepath{clip}%
\pgfsetbuttcap%
\pgfsetroundjoin%
\definecolor{currentfill}{rgb}{1.000000,0.498039,0.054902}%
\pgfsetfillcolor{currentfill}%
\pgfsetlinewidth{1.003750pt}%
\definecolor{currentstroke}{rgb}{1.000000,0.498039,0.054902}%
\pgfsetstrokecolor{currentstroke}%
\pgfsetdash{}{0pt}%
\pgfpathmoveto{\pgfqpoint{1.503171in}{2.826055in}}%
\pgfpathcurveto{\pgfqpoint{1.513259in}{2.826055in}}{\pgfqpoint{1.522934in}{2.830063in}}{\pgfqpoint{1.530067in}{2.837196in}}%
\pgfpathcurveto{\pgfqpoint{1.537200in}{2.844329in}}{\pgfqpoint{1.541208in}{2.854004in}}{\pgfqpoint{1.541208in}{2.864092in}}%
\pgfpathcurveto{\pgfqpoint{1.541208in}{2.874179in}}{\pgfqpoint{1.537200in}{2.883855in}}{\pgfqpoint{1.530067in}{2.890987in}}%
\pgfpathcurveto{\pgfqpoint{1.522934in}{2.898120in}}{\pgfqpoint{1.513259in}{2.902128in}}{\pgfqpoint{1.503171in}{2.902128in}}%
\pgfpathcurveto{\pgfqpoint{1.493084in}{2.902128in}}{\pgfqpoint{1.483408in}{2.898120in}}{\pgfqpoint{1.476276in}{2.890987in}}%
\pgfpathcurveto{\pgfqpoint{1.469143in}{2.883855in}}{\pgfqpoint{1.465135in}{2.874179in}}{\pgfqpoint{1.465135in}{2.864092in}}%
\pgfpathcurveto{\pgfqpoint{1.465135in}{2.854004in}}{\pgfqpoint{1.469143in}{2.844329in}}{\pgfqpoint{1.476276in}{2.837196in}}%
\pgfpathcurveto{\pgfqpoint{1.483408in}{2.830063in}}{\pgfqpoint{1.493084in}{2.826055in}}{\pgfqpoint{1.503171in}{2.826055in}}%
\pgfpathclose%
\pgfusepath{stroke,fill}%
\end{pgfscope}%
\begin{pgfscope}%
\pgfpathrectangle{\pgfqpoint{0.800000in}{1.363959in}}{\pgfqpoint{3.968000in}{2.024082in}} %
\pgfusepath{clip}%
\pgfsetbuttcap%
\pgfsetroundjoin%
\definecolor{currentfill}{rgb}{1.000000,0.498039,0.054902}%
\pgfsetfillcolor{currentfill}%
\pgfsetlinewidth{1.003750pt}%
\definecolor{currentstroke}{rgb}{1.000000,0.498039,0.054902}%
\pgfsetstrokecolor{currentstroke}%
\pgfsetdash{}{0pt}%
\pgfpathmoveto{\pgfqpoint{0.991178in}{2.035610in}}%
\pgfpathcurveto{\pgfqpoint{1.001265in}{2.035610in}}{\pgfqpoint{1.010941in}{2.039618in}}{\pgfqpoint{1.018073in}{2.046751in}}%
\pgfpathcurveto{\pgfqpoint{1.025206in}{2.053884in}}{\pgfqpoint{1.029214in}{2.063559in}}{\pgfqpoint{1.029214in}{2.073647in}}%
\pgfpathcurveto{\pgfqpoint{1.029214in}{2.083734in}}{\pgfqpoint{1.025206in}{2.093410in}}{\pgfqpoint{1.018073in}{2.100542in}}%
\pgfpathcurveto{\pgfqpoint{1.010941in}{2.107675in}}{\pgfqpoint{1.001265in}{2.111683in}}{\pgfqpoint{0.991178in}{2.111683in}}%
\pgfpathcurveto{\pgfqpoint{0.981090in}{2.111683in}}{\pgfqpoint{0.971415in}{2.107675in}}{\pgfqpoint{0.964282in}{2.100542in}}%
\pgfpathcurveto{\pgfqpoint{0.957149in}{2.093410in}}{\pgfqpoint{0.953141in}{2.083734in}}{\pgfqpoint{0.953141in}{2.073647in}}%
\pgfpathcurveto{\pgfqpoint{0.953141in}{2.063559in}}{\pgfqpoint{0.957149in}{2.053884in}}{\pgfqpoint{0.964282in}{2.046751in}}%
\pgfpathcurveto{\pgfqpoint{0.971415in}{2.039618in}}{\pgfqpoint{0.981090in}{2.035610in}}{\pgfqpoint{0.991178in}{2.035610in}}%
\pgfpathclose%
\pgfusepath{stroke,fill}%
\end{pgfscope}%
\begin{pgfscope}%
\pgfpathrectangle{\pgfqpoint{0.800000in}{1.363959in}}{\pgfqpoint{3.968000in}{2.024082in}} %
\pgfusepath{clip}%
\pgfsetbuttcap%
\pgfsetroundjoin%
\definecolor{currentfill}{rgb}{1.000000,0.498039,0.054902}%
\pgfsetfillcolor{currentfill}%
\pgfsetlinewidth{1.003750pt}%
\definecolor{currentstroke}{rgb}{1.000000,0.498039,0.054902}%
\pgfsetstrokecolor{currentstroke}%
\pgfsetdash{}{0pt}%
\pgfpathmoveto{\pgfqpoint{1.568897in}{2.974610in}}%
\pgfpathcurveto{\pgfqpoint{1.578984in}{2.974610in}}{\pgfqpoint{1.588660in}{2.978618in}}{\pgfqpoint{1.595792in}{2.985751in}}%
\pgfpathcurveto{\pgfqpoint{1.602925in}{2.992884in}}{\pgfqpoint{1.606933in}{3.002559in}}{\pgfqpoint{1.606933in}{3.012646in}}%
\pgfpathcurveto{\pgfqpoint{1.606933in}{3.022734in}}{\pgfqpoint{1.602925in}{3.032409in}}{\pgfqpoint{1.595792in}{3.039542in}}%
\pgfpathcurveto{\pgfqpoint{1.588660in}{3.046675in}}{\pgfqpoint{1.578984in}{3.050683in}}{\pgfqpoint{1.568897in}{3.050683in}}%
\pgfpathcurveto{\pgfqpoint{1.558809in}{3.050683in}}{\pgfqpoint{1.549134in}{3.046675in}}{\pgfqpoint{1.542001in}{3.039542in}}%
\pgfpathcurveto{\pgfqpoint{1.534868in}{3.032409in}}{\pgfqpoint{1.530860in}{3.022734in}}{\pgfqpoint{1.530860in}{3.012646in}}%
\pgfpathcurveto{\pgfqpoint{1.530860in}{3.002559in}}{\pgfqpoint{1.534868in}{2.992884in}}{\pgfqpoint{1.542001in}{2.985751in}}%
\pgfpathcurveto{\pgfqpoint{1.549134in}{2.978618in}}{\pgfqpoint{1.558809in}{2.974610in}}{\pgfqpoint{1.568897in}{2.974610in}}%
\pgfpathclose%
\pgfusepath{stroke,fill}%
\end{pgfscope}%
\begin{pgfscope}%
\pgfpathrectangle{\pgfqpoint{0.800000in}{1.363959in}}{\pgfqpoint{3.968000in}{2.024082in}} %
\pgfusepath{clip}%
\pgfsetbuttcap%
\pgfsetroundjoin%
\definecolor{currentfill}{rgb}{1.000000,0.498039,0.054902}%
\pgfsetfillcolor{currentfill}%
\pgfsetlinewidth{1.003750pt}%
\definecolor{currentstroke}{rgb}{1.000000,0.498039,0.054902}%
\pgfsetstrokecolor{currentstroke}%
\pgfsetdash{}{0pt}%
\pgfpathmoveto{\pgfqpoint{1.012474in}{2.196615in}}%
\pgfpathcurveto{\pgfqpoint{1.022561in}{2.196615in}}{\pgfqpoint{1.032237in}{2.200623in}}{\pgfqpoint{1.039369in}{2.207756in}}%
\pgfpathcurveto{\pgfqpoint{1.046502in}{2.214889in}}{\pgfqpoint{1.050510in}{2.224564in}}{\pgfqpoint{1.050510in}{2.234652in}}%
\pgfpathcurveto{\pgfqpoint{1.050510in}{2.244739in}}{\pgfqpoint{1.046502in}{2.254414in}}{\pgfqpoint{1.039369in}{2.261547in}}%
\pgfpathcurveto{\pgfqpoint{1.032237in}{2.268680in}}{\pgfqpoint{1.022561in}{2.272688in}}{\pgfqpoint{1.012474in}{2.272688in}}%
\pgfpathcurveto{\pgfqpoint{1.002386in}{2.272688in}}{\pgfqpoint{0.992711in}{2.268680in}}{\pgfqpoint{0.985578in}{2.261547in}}%
\pgfpathcurveto{\pgfqpoint{0.978445in}{2.254414in}}{\pgfqpoint{0.974437in}{2.244739in}}{\pgfqpoint{0.974437in}{2.234652in}}%
\pgfpathcurveto{\pgfqpoint{0.974437in}{2.224564in}}{\pgfqpoint{0.978445in}{2.214889in}}{\pgfqpoint{0.985578in}{2.207756in}}%
\pgfpathcurveto{\pgfqpoint{0.992711in}{2.200623in}}{\pgfqpoint{1.002386in}{2.196615in}}{\pgfqpoint{1.012474in}{2.196615in}}%
\pgfpathclose%
\pgfusepath{stroke,fill}%
\end{pgfscope}%
\begin{pgfscope}%
\pgfpathrectangle{\pgfqpoint{0.800000in}{1.363959in}}{\pgfqpoint{3.968000in}{2.024082in}} %
\pgfusepath{clip}%
\pgfsetbuttcap%
\pgfsetroundjoin%
\definecolor{currentfill}{rgb}{1.000000,0.498039,0.054902}%
\pgfsetfillcolor{currentfill}%
\pgfsetlinewidth{1.003750pt}%
\definecolor{currentstroke}{rgb}{1.000000,0.498039,0.054902}%
\pgfsetstrokecolor{currentstroke}%
\pgfsetdash{}{0pt}%
\pgfpathmoveto{\pgfqpoint{1.202712in}{2.565441in}}%
\pgfpathcurveto{\pgfqpoint{1.212799in}{2.565441in}}{\pgfqpoint{1.222475in}{2.569449in}}{\pgfqpoint{1.229608in}{2.576582in}}%
\pgfpathcurveto{\pgfqpoint{1.236741in}{2.583715in}}{\pgfqpoint{1.240748in}{2.593390in}}{\pgfqpoint{1.240748in}{2.603477in}}%
\pgfpathcurveto{\pgfqpoint{1.240748in}{2.613565in}}{\pgfqpoint{1.236741in}{2.623240in}}{\pgfqpoint{1.229608in}{2.630373in}}%
\pgfpathcurveto{\pgfqpoint{1.222475in}{2.637506in}}{\pgfqpoint{1.212799in}{2.641514in}}{\pgfqpoint{1.202712in}{2.641514in}}%
\pgfpathcurveto{\pgfqpoint{1.192625in}{2.641514in}}{\pgfqpoint{1.182949in}{2.637506in}}{\pgfqpoint{1.175816in}{2.630373in}}%
\pgfpathcurveto{\pgfqpoint{1.168684in}{2.623240in}}{\pgfqpoint{1.164676in}{2.613565in}}{\pgfqpoint{1.164676in}{2.603477in}}%
\pgfpathcurveto{\pgfqpoint{1.164676in}{2.593390in}}{\pgfqpoint{1.168684in}{2.583715in}}{\pgfqpoint{1.175816in}{2.576582in}}%
\pgfpathcurveto{\pgfqpoint{1.182949in}{2.569449in}}{\pgfqpoint{1.192625in}{2.565441in}}{\pgfqpoint{1.202712in}{2.565441in}}%
\pgfpathclose%
\pgfusepath{stroke,fill}%
\end{pgfscope}%
\begin{pgfscope}%
\pgfpathrectangle{\pgfqpoint{0.800000in}{1.363959in}}{\pgfqpoint{3.968000in}{2.024082in}} %
\pgfusepath{clip}%
\pgfsetbuttcap%
\pgfsetroundjoin%
\definecolor{currentfill}{rgb}{1.000000,0.498039,0.054902}%
\pgfsetfillcolor{currentfill}%
\pgfsetlinewidth{1.003750pt}%
\definecolor{currentstroke}{rgb}{1.000000,0.498039,0.054902}%
\pgfsetstrokecolor{currentstroke}%
\pgfsetdash{}{0pt}%
\pgfpathmoveto{\pgfqpoint{4.243651in}{1.664353in}}%
\pgfpathcurveto{\pgfqpoint{4.253738in}{1.664353in}}{\pgfqpoint{4.263414in}{1.668360in}}{\pgfqpoint{4.270546in}{1.675493in}}%
\pgfpathcurveto{\pgfqpoint{4.277679in}{1.682626in}}{\pgfqpoint{4.281687in}{1.692302in}}{\pgfqpoint{4.281687in}{1.702389in}}%
\pgfpathcurveto{\pgfqpoint{4.281687in}{1.712476in}}{\pgfqpoint{4.277679in}{1.722152in}}{\pgfqpoint{4.270546in}{1.729285in}}%
\pgfpathcurveto{\pgfqpoint{4.263414in}{1.736418in}}{\pgfqpoint{4.253738in}{1.740425in}}{\pgfqpoint{4.243651in}{1.740425in}}%
\pgfpathcurveto{\pgfqpoint{4.233563in}{1.740425in}}{\pgfqpoint{4.223888in}{1.736418in}}{\pgfqpoint{4.216755in}{1.729285in}}%
\pgfpathcurveto{\pgfqpoint{4.209622in}{1.722152in}}{\pgfqpoint{4.205614in}{1.712476in}}{\pgfqpoint{4.205614in}{1.702389in}}%
\pgfpathcurveto{\pgfqpoint{4.205614in}{1.692302in}}{\pgfqpoint{4.209622in}{1.682626in}}{\pgfqpoint{4.216755in}{1.675493in}}%
\pgfpathcurveto{\pgfqpoint{4.223888in}{1.668360in}}{\pgfqpoint{4.233563in}{1.664353in}}{\pgfqpoint{4.243651in}{1.664353in}}%
\pgfpathclose%
\pgfusepath{stroke,fill}%
\end{pgfscope}%
\begin{pgfscope}%
\pgfpathrectangle{\pgfqpoint{0.800000in}{1.363959in}}{\pgfqpoint{3.968000in}{2.024082in}} %
\pgfusepath{clip}%
\pgfsetbuttcap%
\pgfsetroundjoin%
\definecolor{currentfill}{rgb}{1.000000,0.498039,0.054902}%
\pgfsetfillcolor{currentfill}%
\pgfsetlinewidth{1.003750pt}%
\definecolor{currentstroke}{rgb}{1.000000,0.498039,0.054902}%
\pgfsetstrokecolor{currentstroke}%
\pgfsetdash{}{0pt}%
\pgfpathmoveto{\pgfqpoint{3.791563in}{1.521286in}}%
\pgfpathcurveto{\pgfqpoint{3.801650in}{1.521286in}}{\pgfqpoint{3.811325in}{1.525294in}}{\pgfqpoint{3.818458in}{1.532426in}}%
\pgfpathcurveto{\pgfqpoint{3.825591in}{1.539559in}}{\pgfqpoint{3.829599in}{1.549235in}}{\pgfqpoint{3.829599in}{1.559322in}}%
\pgfpathcurveto{\pgfqpoint{3.829599in}{1.569410in}}{\pgfqpoint{3.825591in}{1.579085in}}{\pgfqpoint{3.818458in}{1.586218in}}%
\pgfpathcurveto{\pgfqpoint{3.811325in}{1.593351in}}{\pgfqpoint{3.801650in}{1.597358in}}{\pgfqpoint{3.791563in}{1.597358in}}%
\pgfpathcurveto{\pgfqpoint{3.781475in}{1.597358in}}{\pgfqpoint{3.771800in}{1.593351in}}{\pgfqpoint{3.764667in}{1.586218in}}%
\pgfpathcurveto{\pgfqpoint{3.757534in}{1.579085in}}{\pgfqpoint{3.753526in}{1.569410in}}{\pgfqpoint{3.753526in}{1.559322in}}%
\pgfpathcurveto{\pgfqpoint{3.753526in}{1.549235in}}{\pgfqpoint{3.757534in}{1.539559in}}{\pgfqpoint{3.764667in}{1.532426in}}%
\pgfpathcurveto{\pgfqpoint{3.771800in}{1.525294in}}{\pgfqpoint{3.781475in}{1.521286in}}{\pgfqpoint{3.791563in}{1.521286in}}%
\pgfpathclose%
\pgfusepath{stroke,fill}%
\end{pgfscope}%
\begin{pgfscope}%
\pgfpathrectangle{\pgfqpoint{0.800000in}{1.363959in}}{\pgfqpoint{3.968000in}{2.024082in}} %
\pgfusepath{clip}%
\pgfsetbuttcap%
\pgfsetroundjoin%
\definecolor{currentfill}{rgb}{1.000000,0.498039,0.054902}%
\pgfsetfillcolor{currentfill}%
\pgfsetlinewidth{1.003750pt}%
\definecolor{currentstroke}{rgb}{1.000000,0.498039,0.054902}%
\pgfsetstrokecolor{currentstroke}%
\pgfsetdash{}{0pt}%
\pgfpathmoveto{\pgfqpoint{1.164209in}{2.252230in}}%
\pgfpathcurveto{\pgfqpoint{1.174297in}{2.252230in}}{\pgfqpoint{1.183972in}{2.256238in}}{\pgfqpoint{1.191105in}{2.263371in}}%
\pgfpathcurveto{\pgfqpoint{1.198238in}{2.270503in}}{\pgfqpoint{1.202245in}{2.280179in}}{\pgfqpoint{1.202245in}{2.290266in}}%
\pgfpathcurveto{\pgfqpoint{1.202245in}{2.300354in}}{\pgfqpoint{1.198238in}{2.310029in}}{\pgfqpoint{1.191105in}{2.317162in}}%
\pgfpathcurveto{\pgfqpoint{1.183972in}{2.324295in}}{\pgfqpoint{1.174297in}{2.328303in}}{\pgfqpoint{1.164209in}{2.328303in}}%
\pgfpathcurveto{\pgfqpoint{1.154122in}{2.328303in}}{\pgfqpoint{1.144446in}{2.324295in}}{\pgfqpoint{1.137313in}{2.317162in}}%
\pgfpathcurveto{\pgfqpoint{1.130181in}{2.310029in}}{\pgfqpoint{1.126173in}{2.300354in}}{\pgfqpoint{1.126173in}{2.290266in}}%
\pgfpathcurveto{\pgfqpoint{1.126173in}{2.280179in}}{\pgfqpoint{1.130181in}{2.270503in}}{\pgfqpoint{1.137313in}{2.263371in}}%
\pgfpathcurveto{\pgfqpoint{1.144446in}{2.256238in}}{\pgfqpoint{1.154122in}{2.252230in}}{\pgfqpoint{1.164209in}{2.252230in}}%
\pgfpathclose%
\pgfusepath{stroke,fill}%
\end{pgfscope}%
\begin{pgfscope}%
\pgfpathrectangle{\pgfqpoint{0.800000in}{1.363959in}}{\pgfqpoint{3.968000in}{2.024082in}} %
\pgfusepath{clip}%
\pgfsetbuttcap%
\pgfsetroundjoin%
\definecolor{currentfill}{rgb}{1.000000,0.498039,0.054902}%
\pgfsetfillcolor{currentfill}%
\pgfsetlinewidth{1.003750pt}%
\definecolor{currentstroke}{rgb}{1.000000,0.498039,0.054902}%
\pgfsetstrokecolor{currentstroke}%
\pgfsetdash{}{0pt}%
\pgfpathmoveto{\pgfqpoint{1.734828in}{1.608242in}}%
\pgfpathcurveto{\pgfqpoint{1.744916in}{1.608242in}}{\pgfqpoint{1.754591in}{1.612249in}}{\pgfqpoint{1.761724in}{1.619382in}}%
\pgfpathcurveto{\pgfqpoint{1.768857in}{1.626515in}}{\pgfqpoint{1.772864in}{1.636191in}}{\pgfqpoint{1.772864in}{1.646278in}}%
\pgfpathcurveto{\pgfqpoint{1.772864in}{1.656365in}}{\pgfqpoint{1.768857in}{1.666041in}}{\pgfqpoint{1.761724in}{1.673174in}}%
\pgfpathcurveto{\pgfqpoint{1.754591in}{1.680306in}}{\pgfqpoint{1.744916in}{1.684314in}}{\pgfqpoint{1.734828in}{1.684314in}}%
\pgfpathcurveto{\pgfqpoint{1.724741in}{1.684314in}}{\pgfqpoint{1.715065in}{1.680306in}}{\pgfqpoint{1.707932in}{1.673174in}}%
\pgfpathcurveto{\pgfqpoint{1.700800in}{1.666041in}}{\pgfqpoint{1.696792in}{1.656365in}}{\pgfqpoint{1.696792in}{1.646278in}}%
\pgfpathcurveto{\pgfqpoint{1.696792in}{1.636191in}}{\pgfqpoint{1.700800in}{1.626515in}}{\pgfqpoint{1.707932in}{1.619382in}}%
\pgfpathcurveto{\pgfqpoint{1.715065in}{1.612249in}}{\pgfqpoint{1.724741in}{1.608242in}}{\pgfqpoint{1.734828in}{1.608242in}}%
\pgfpathclose%
\pgfusepath{stroke,fill}%
\end{pgfscope}%
\begin{pgfscope}%
\pgfpathrectangle{\pgfqpoint{0.800000in}{1.363959in}}{\pgfqpoint{3.968000in}{2.024082in}} %
\pgfusepath{clip}%
\pgfsetbuttcap%
\pgfsetroundjoin%
\definecolor{currentfill}{rgb}{1.000000,0.498039,0.054902}%
\pgfsetfillcolor{currentfill}%
\pgfsetlinewidth{1.003750pt}%
\definecolor{currentstroke}{rgb}{1.000000,0.498039,0.054902}%
\pgfsetstrokecolor{currentstroke}%
\pgfsetdash{}{0pt}%
\pgfpathmoveto{\pgfqpoint{1.507803in}{1.448647in}}%
\pgfpathcurveto{\pgfqpoint{1.517891in}{1.448647in}}{\pgfqpoint{1.527566in}{1.452655in}}{\pgfqpoint{1.534699in}{1.459788in}}%
\pgfpathcurveto{\pgfqpoint{1.541832in}{1.466920in}}{\pgfqpoint{1.545840in}{1.476596in}}{\pgfqpoint{1.545840in}{1.486683in}}%
\pgfpathcurveto{\pgfqpoint{1.545840in}{1.496771in}}{\pgfqpoint{1.541832in}{1.506446in}}{\pgfqpoint{1.534699in}{1.513579in}}%
\pgfpathcurveto{\pgfqpoint{1.527566in}{1.520712in}}{\pgfqpoint{1.517891in}{1.524720in}}{\pgfqpoint{1.507803in}{1.524720in}}%
\pgfpathcurveto{\pgfqpoint{1.497716in}{1.524720in}}{\pgfqpoint{1.488040in}{1.520712in}}{\pgfqpoint{1.480908in}{1.513579in}}%
\pgfpathcurveto{\pgfqpoint{1.473775in}{1.506446in}}{\pgfqpoint{1.469767in}{1.496771in}}{\pgfqpoint{1.469767in}{1.486683in}}%
\pgfpathcurveto{\pgfqpoint{1.469767in}{1.476596in}}{\pgfqpoint{1.473775in}{1.466920in}}{\pgfqpoint{1.480908in}{1.459788in}}%
\pgfpathcurveto{\pgfqpoint{1.488040in}{1.452655in}}{\pgfqpoint{1.497716in}{1.448647in}}{\pgfqpoint{1.507803in}{1.448647in}}%
\pgfpathclose%
\pgfusepath{stroke,fill}%
\end{pgfscope}%
\begin{pgfscope}%
\pgfpathrectangle{\pgfqpoint{0.800000in}{1.363959in}}{\pgfqpoint{3.968000in}{2.024082in}} %
\pgfusepath{clip}%
\pgfsetbuttcap%
\pgfsetroundjoin%
\definecolor{currentfill}{rgb}{1.000000,0.498039,0.054902}%
\pgfsetfillcolor{currentfill}%
\pgfsetlinewidth{1.003750pt}%
\definecolor{currentstroke}{rgb}{1.000000,0.498039,0.054902}%
\pgfsetstrokecolor{currentstroke}%
\pgfsetdash{}{0pt}%
\pgfpathmoveto{\pgfqpoint{1.283718in}{2.804884in}}%
\pgfpathcurveto{\pgfqpoint{1.293805in}{2.804884in}}{\pgfqpoint{1.303481in}{2.808892in}}{\pgfqpoint{1.310614in}{2.816025in}}%
\pgfpathcurveto{\pgfqpoint{1.317746in}{2.823158in}}{\pgfqpoint{1.321754in}{2.832833in}}{\pgfqpoint{1.321754in}{2.842920in}}%
\pgfpathcurveto{\pgfqpoint{1.321754in}{2.853008in}}{\pgfqpoint{1.317746in}{2.862683in}}{\pgfqpoint{1.310614in}{2.869816in}}%
\pgfpathcurveto{\pgfqpoint{1.303481in}{2.876949in}}{\pgfqpoint{1.293805in}{2.880957in}}{\pgfqpoint{1.283718in}{2.880957in}}%
\pgfpathcurveto{\pgfqpoint{1.273630in}{2.880957in}}{\pgfqpoint{1.263955in}{2.876949in}}{\pgfqpoint{1.256822in}{2.869816in}}%
\pgfpathcurveto{\pgfqpoint{1.249689in}{2.862683in}}{\pgfqpoint{1.245682in}{2.853008in}}{\pgfqpoint{1.245682in}{2.842920in}}%
\pgfpathcurveto{\pgfqpoint{1.245682in}{2.832833in}}{\pgfqpoint{1.249689in}{2.823158in}}{\pgfqpoint{1.256822in}{2.816025in}}%
\pgfpathcurveto{\pgfqpoint{1.263955in}{2.808892in}}{\pgfqpoint{1.273630in}{2.804884in}}{\pgfqpoint{1.283718in}{2.804884in}}%
\pgfpathclose%
\pgfusepath{stroke,fill}%
\end{pgfscope}%
\begin{pgfscope}%
\pgfpathrectangle{\pgfqpoint{0.800000in}{1.363959in}}{\pgfqpoint{3.968000in}{2.024082in}} %
\pgfusepath{clip}%
\pgfsetbuttcap%
\pgfsetroundjoin%
\definecolor{currentfill}{rgb}{1.000000,0.498039,0.054902}%
\pgfsetfillcolor{currentfill}%
\pgfsetlinewidth{1.003750pt}%
\definecolor{currentstroke}{rgb}{1.000000,0.498039,0.054902}%
\pgfsetstrokecolor{currentstroke}%
\pgfsetdash{}{0pt}%
\pgfpathmoveto{\pgfqpoint{3.834498in}{2.133347in}}%
\pgfpathcurveto{\pgfqpoint{3.844585in}{2.133347in}}{\pgfqpoint{3.854261in}{2.137354in}}{\pgfqpoint{3.861393in}{2.144487in}}%
\pgfpathcurveto{\pgfqpoint{3.868526in}{2.151620in}}{\pgfqpoint{3.872534in}{2.161296in}}{\pgfqpoint{3.872534in}{2.171383in}}%
\pgfpathcurveto{\pgfqpoint{3.872534in}{2.181470in}}{\pgfqpoint{3.868526in}{2.191146in}}{\pgfqpoint{3.861393in}{2.198279in}}%
\pgfpathcurveto{\pgfqpoint{3.854261in}{2.205411in}}{\pgfqpoint{3.844585in}{2.209419in}}{\pgfqpoint{3.834498in}{2.209419in}}%
\pgfpathcurveto{\pgfqpoint{3.824410in}{2.209419in}}{\pgfqpoint{3.814735in}{2.205411in}}{\pgfqpoint{3.807602in}{2.198279in}}%
\pgfpathcurveto{\pgfqpoint{3.800469in}{2.191146in}}{\pgfqpoint{3.796461in}{2.181470in}}{\pgfqpoint{3.796461in}{2.171383in}}%
\pgfpathcurveto{\pgfqpoint{3.796461in}{2.161296in}}{\pgfqpoint{3.800469in}{2.151620in}}{\pgfqpoint{3.807602in}{2.144487in}}%
\pgfpathcurveto{\pgfqpoint{3.814735in}{2.137354in}}{\pgfqpoint{3.824410in}{2.133347in}}{\pgfqpoint{3.834498in}{2.133347in}}%
\pgfpathclose%
\pgfusepath{stroke,fill}%
\end{pgfscope}%
\begin{pgfscope}%
\pgfpathrectangle{\pgfqpoint{0.800000in}{1.363959in}}{\pgfqpoint{3.968000in}{2.024082in}} %
\pgfusepath{clip}%
\pgfsetbuttcap%
\pgfsetroundjoin%
\definecolor{currentfill}{rgb}{1.000000,0.498039,0.054902}%
\pgfsetfillcolor{currentfill}%
\pgfsetlinewidth{1.003750pt}%
\definecolor{currentstroke}{rgb}{1.000000,0.498039,0.054902}%
\pgfsetstrokecolor{currentstroke}%
\pgfsetdash{}{0pt}%
\pgfpathmoveto{\pgfqpoint{1.446430in}{2.790105in}}%
\pgfpathcurveto{\pgfqpoint{1.456517in}{2.790105in}}{\pgfqpoint{1.466193in}{2.794113in}}{\pgfqpoint{1.473326in}{2.801246in}}%
\pgfpathcurveto{\pgfqpoint{1.480458in}{2.808379in}}{\pgfqpoint{1.484466in}{2.818054in}}{\pgfqpoint{1.484466in}{2.828142in}}%
\pgfpathcurveto{\pgfqpoint{1.484466in}{2.838229in}}{\pgfqpoint{1.480458in}{2.847904in}}{\pgfqpoint{1.473326in}{2.855037in}}%
\pgfpathcurveto{\pgfqpoint{1.466193in}{2.862170in}}{\pgfqpoint{1.456517in}{2.866178in}}{\pgfqpoint{1.446430in}{2.866178in}}%
\pgfpathcurveto{\pgfqpoint{1.436343in}{2.866178in}}{\pgfqpoint{1.426667in}{2.862170in}}{\pgfqpoint{1.419534in}{2.855037in}}%
\pgfpathcurveto{\pgfqpoint{1.412401in}{2.847904in}}{\pgfqpoint{1.408394in}{2.838229in}}{\pgfqpoint{1.408394in}{2.828142in}}%
\pgfpathcurveto{\pgfqpoint{1.408394in}{2.818054in}}{\pgfqpoint{1.412401in}{2.808379in}}{\pgfqpoint{1.419534in}{2.801246in}}%
\pgfpathcurveto{\pgfqpoint{1.426667in}{2.794113in}}{\pgfqpoint{1.436343in}{2.790105in}}{\pgfqpoint{1.446430in}{2.790105in}}%
\pgfpathclose%
\pgfusepath{stroke,fill}%
\end{pgfscope}%
\begin{pgfscope}%
\pgfpathrectangle{\pgfqpoint{0.800000in}{1.363959in}}{\pgfqpoint{3.968000in}{2.024082in}} %
\pgfusepath{clip}%
\pgfsetbuttcap%
\pgfsetroundjoin%
\definecolor{currentfill}{rgb}{1.000000,0.498039,0.054902}%
\pgfsetfillcolor{currentfill}%
\pgfsetlinewidth{1.003750pt}%
\definecolor{currentstroke}{rgb}{1.000000,0.498039,0.054902}%
\pgfsetstrokecolor{currentstroke}%
\pgfsetdash{}{0pt}%
\pgfpathmoveto{\pgfqpoint{4.438478in}{2.240704in}}%
\pgfpathcurveto{\pgfqpoint{4.448565in}{2.240704in}}{\pgfqpoint{4.458241in}{2.244712in}}{\pgfqpoint{4.465374in}{2.251845in}}%
\pgfpathcurveto{\pgfqpoint{4.472507in}{2.258978in}}{\pgfqpoint{4.476514in}{2.268653in}}{\pgfqpoint{4.476514in}{2.278740in}}%
\pgfpathcurveto{\pgfqpoint{4.476514in}{2.288828in}}{\pgfqpoint{4.472507in}{2.298503in}}{\pgfqpoint{4.465374in}{2.305636in}}%
\pgfpathcurveto{\pgfqpoint{4.458241in}{2.312769in}}{\pgfqpoint{4.448565in}{2.316777in}}{\pgfqpoint{4.438478in}{2.316777in}}%
\pgfpathcurveto{\pgfqpoint{4.428391in}{2.316777in}}{\pgfqpoint{4.418715in}{2.312769in}}{\pgfqpoint{4.411582in}{2.305636in}}%
\pgfpathcurveto{\pgfqpoint{4.404450in}{2.298503in}}{\pgfqpoint{4.400442in}{2.288828in}}{\pgfqpoint{4.400442in}{2.278740in}}%
\pgfpathcurveto{\pgfqpoint{4.400442in}{2.268653in}}{\pgfqpoint{4.404450in}{2.258978in}}{\pgfqpoint{4.411582in}{2.251845in}}%
\pgfpathcurveto{\pgfqpoint{4.418715in}{2.244712in}}{\pgfqpoint{4.428391in}{2.240704in}}{\pgfqpoint{4.438478in}{2.240704in}}%
\pgfpathclose%
\pgfusepath{stroke,fill}%
\end{pgfscope}%
\begin{pgfscope}%
\pgfpathrectangle{\pgfqpoint{0.800000in}{1.363959in}}{\pgfqpoint{3.968000in}{2.024082in}} %
\pgfusepath{clip}%
\pgfsetbuttcap%
\pgfsetroundjoin%
\definecolor{currentfill}{rgb}{1.000000,0.498039,0.054902}%
\pgfsetfillcolor{currentfill}%
\pgfsetlinewidth{1.003750pt}%
\definecolor{currentstroke}{rgb}{1.000000,0.498039,0.054902}%
\pgfsetstrokecolor{currentstroke}%
\pgfsetdash{}{0pt}%
\pgfpathmoveto{\pgfqpoint{4.426036in}{2.236380in}}%
\pgfpathcurveto{\pgfqpoint{4.436124in}{2.236380in}}{\pgfqpoint{4.445799in}{2.240388in}}{\pgfqpoint{4.452932in}{2.247521in}}%
\pgfpathcurveto{\pgfqpoint{4.460065in}{2.254654in}}{\pgfqpoint{4.464073in}{2.264329in}}{\pgfqpoint{4.464073in}{2.274417in}}%
\pgfpathcurveto{\pgfqpoint{4.464073in}{2.284504in}}{\pgfqpoint{4.460065in}{2.294179in}}{\pgfqpoint{4.452932in}{2.301312in}}%
\pgfpathcurveto{\pgfqpoint{4.445799in}{2.308445in}}{\pgfqpoint{4.436124in}{2.312453in}}{\pgfqpoint{4.426036in}{2.312453in}}%
\pgfpathcurveto{\pgfqpoint{4.415949in}{2.312453in}}{\pgfqpoint{4.406273in}{2.308445in}}{\pgfqpoint{4.399141in}{2.301312in}}%
\pgfpathcurveto{\pgfqpoint{4.392008in}{2.294179in}}{\pgfqpoint{4.388000in}{2.284504in}}{\pgfqpoint{4.388000in}{2.274417in}}%
\pgfpathcurveto{\pgfqpoint{4.388000in}{2.264329in}}{\pgfqpoint{4.392008in}{2.254654in}}{\pgfqpoint{4.399141in}{2.247521in}}%
\pgfpathcurveto{\pgfqpoint{4.406273in}{2.240388in}}{\pgfqpoint{4.415949in}{2.236380in}}{\pgfqpoint{4.426036in}{2.236380in}}%
\pgfpathclose%
\pgfusepath{stroke,fill}%
\end{pgfscope}%
\begin{pgfscope}%
\pgfpathrectangle{\pgfqpoint{0.800000in}{1.363959in}}{\pgfqpoint{3.968000in}{2.024082in}} %
\pgfusepath{clip}%
\pgfsetbuttcap%
\pgfsetroundjoin%
\definecolor{currentfill}{rgb}{1.000000,0.498039,0.054902}%
\pgfsetfillcolor{currentfill}%
\pgfsetlinewidth{1.003750pt}%
\definecolor{currentstroke}{rgb}{1.000000,0.498039,0.054902}%
\pgfsetstrokecolor{currentstroke}%
\pgfsetdash{}{0pt}%
\pgfpathmoveto{\pgfqpoint{1.343554in}{2.288583in}}%
\pgfpathcurveto{\pgfqpoint{1.353642in}{2.288583in}}{\pgfqpoint{1.363317in}{2.292591in}}{\pgfqpoint{1.370450in}{2.299723in}}%
\pgfpathcurveto{\pgfqpoint{1.377583in}{2.306856in}}{\pgfqpoint{1.381591in}{2.316532in}}{\pgfqpoint{1.381591in}{2.326619in}}%
\pgfpathcurveto{\pgfqpoint{1.381591in}{2.336706in}}{\pgfqpoint{1.377583in}{2.346382in}}{\pgfqpoint{1.370450in}{2.353515in}}%
\pgfpathcurveto{\pgfqpoint{1.363317in}{2.360648in}}{\pgfqpoint{1.353642in}{2.364655in}}{\pgfqpoint{1.343554in}{2.364655in}}%
\pgfpathcurveto{\pgfqpoint{1.333467in}{2.364655in}}{\pgfqpoint{1.323791in}{2.360648in}}{\pgfqpoint{1.316659in}{2.353515in}}%
\pgfpathcurveto{\pgfqpoint{1.309526in}{2.346382in}}{\pgfqpoint{1.305518in}{2.336706in}}{\pgfqpoint{1.305518in}{2.326619in}}%
\pgfpathcurveto{\pgfqpoint{1.305518in}{2.316532in}}{\pgfqpoint{1.309526in}{2.306856in}}{\pgfqpoint{1.316659in}{2.299723in}}%
\pgfpathcurveto{\pgfqpoint{1.323791in}{2.292591in}}{\pgfqpoint{1.333467in}{2.288583in}}{\pgfqpoint{1.343554in}{2.288583in}}%
\pgfpathclose%
\pgfusepath{stroke,fill}%
\end{pgfscope}%
\begin{pgfscope}%
\pgfpathrectangle{\pgfqpoint{0.800000in}{1.363959in}}{\pgfqpoint{3.968000in}{2.024082in}} %
\pgfusepath{clip}%
\pgfsetbuttcap%
\pgfsetroundjoin%
\definecolor{currentfill}{rgb}{1.000000,0.498039,0.054902}%
\pgfsetfillcolor{currentfill}%
\pgfsetlinewidth{1.003750pt}%
\definecolor{currentstroke}{rgb}{1.000000,0.498039,0.054902}%
\pgfsetstrokecolor{currentstroke}%
\pgfsetdash{}{0pt}%
\pgfpathmoveto{\pgfqpoint{1.070793in}{2.280994in}}%
\pgfpathcurveto{\pgfqpoint{1.080881in}{2.280994in}}{\pgfqpoint{1.090556in}{2.285001in}}{\pgfqpoint{1.097689in}{2.292134in}}%
\pgfpathcurveto{\pgfqpoint{1.104822in}{2.299267in}}{\pgfqpoint{1.108830in}{2.308942in}}{\pgfqpoint{1.108830in}{2.319030in}}%
\pgfpathcurveto{\pgfqpoint{1.108830in}{2.329117in}}{\pgfqpoint{1.104822in}{2.338793in}}{\pgfqpoint{1.097689in}{2.345926in}}%
\pgfpathcurveto{\pgfqpoint{1.090556in}{2.353058in}}{\pgfqpoint{1.080881in}{2.357066in}}{\pgfqpoint{1.070793in}{2.357066in}}%
\pgfpathcurveto{\pgfqpoint{1.060706in}{2.357066in}}{\pgfqpoint{1.051031in}{2.353058in}}{\pgfqpoint{1.043898in}{2.345926in}}%
\pgfpathcurveto{\pgfqpoint{1.036765in}{2.338793in}}{\pgfqpoint{1.032757in}{2.329117in}}{\pgfqpoint{1.032757in}{2.319030in}}%
\pgfpathcurveto{\pgfqpoint{1.032757in}{2.308942in}}{\pgfqpoint{1.036765in}{2.299267in}}{\pgfqpoint{1.043898in}{2.292134in}}%
\pgfpathcurveto{\pgfqpoint{1.051031in}{2.285001in}}{\pgfqpoint{1.060706in}{2.280994in}}{\pgfqpoint{1.070793in}{2.280994in}}%
\pgfpathclose%
\pgfusepath{stroke,fill}%
\end{pgfscope}%
\begin{pgfscope}%
\pgfpathrectangle{\pgfqpoint{0.800000in}{1.363959in}}{\pgfqpoint{3.968000in}{2.024082in}} %
\pgfusepath{clip}%
\pgfsetbuttcap%
\pgfsetroundjoin%
\definecolor{currentfill}{rgb}{1.000000,0.498039,0.054902}%
\pgfsetfillcolor{currentfill}%
\pgfsetlinewidth{1.003750pt}%
\definecolor{currentstroke}{rgb}{1.000000,0.498039,0.054902}%
\pgfsetstrokecolor{currentstroke}%
\pgfsetdash{}{0pt}%
\pgfpathmoveto{\pgfqpoint{1.223883in}{2.894983in}}%
\pgfpathcurveto{\pgfqpoint{1.233971in}{2.894983in}}{\pgfqpoint{1.243646in}{2.898991in}}{\pgfqpoint{1.250779in}{2.906123in}}%
\pgfpathcurveto{\pgfqpoint{1.257912in}{2.913256in}}{\pgfqpoint{1.261920in}{2.922932in}}{\pgfqpoint{1.261920in}{2.933019in}}%
\pgfpathcurveto{\pgfqpoint{1.261920in}{2.943107in}}{\pgfqpoint{1.257912in}{2.952782in}}{\pgfqpoint{1.250779in}{2.959915in}}%
\pgfpathcurveto{\pgfqpoint{1.243646in}{2.967048in}}{\pgfqpoint{1.233971in}{2.971055in}}{\pgfqpoint{1.223883in}{2.971055in}}%
\pgfpathcurveto{\pgfqpoint{1.213796in}{2.971055in}}{\pgfqpoint{1.204120in}{2.967048in}}{\pgfqpoint{1.196988in}{2.959915in}}%
\pgfpathcurveto{\pgfqpoint{1.189855in}{2.952782in}}{\pgfqpoint{1.185847in}{2.943107in}}{\pgfqpoint{1.185847in}{2.933019in}}%
\pgfpathcurveto{\pgfqpoint{1.185847in}{2.922932in}}{\pgfqpoint{1.189855in}{2.913256in}}{\pgfqpoint{1.196988in}{2.906123in}}%
\pgfpathcurveto{\pgfqpoint{1.204120in}{2.898991in}}{\pgfqpoint{1.213796in}{2.894983in}}{\pgfqpoint{1.223883in}{2.894983in}}%
\pgfpathclose%
\pgfusepath{stroke,fill}%
\end{pgfscope}%
\begin{pgfscope}%
\pgfpathrectangle{\pgfqpoint{0.800000in}{1.363959in}}{\pgfqpoint{3.968000in}{2.024082in}} %
\pgfusepath{clip}%
\pgfsetbuttcap%
\pgfsetroundjoin%
\definecolor{currentfill}{rgb}{1.000000,0.498039,0.054902}%
\pgfsetfillcolor{currentfill}%
\pgfsetlinewidth{1.003750pt}%
\definecolor{currentstroke}{rgb}{1.000000,0.498039,0.054902}%
\pgfsetstrokecolor{currentstroke}%
\pgfsetdash{}{0pt}%
\pgfpathmoveto{\pgfqpoint{1.795832in}{2.147415in}}%
\pgfpathcurveto{\pgfqpoint{1.805919in}{2.147415in}}{\pgfqpoint{1.815595in}{2.151423in}}{\pgfqpoint{1.822728in}{2.158556in}}%
\pgfpathcurveto{\pgfqpoint{1.829861in}{2.165688in}}{\pgfqpoint{1.833868in}{2.175364in}}{\pgfqpoint{1.833868in}{2.185451in}}%
\pgfpathcurveto{\pgfqpoint{1.833868in}{2.195539in}}{\pgfqpoint{1.829861in}{2.205214in}}{\pgfqpoint{1.822728in}{2.212347in}}%
\pgfpathcurveto{\pgfqpoint{1.815595in}{2.219480in}}{\pgfqpoint{1.805919in}{2.223488in}}{\pgfqpoint{1.795832in}{2.223488in}}%
\pgfpathcurveto{\pgfqpoint{1.785745in}{2.223488in}}{\pgfqpoint{1.776069in}{2.219480in}}{\pgfqpoint{1.768936in}{2.212347in}}%
\pgfpathcurveto{\pgfqpoint{1.761804in}{2.205214in}}{\pgfqpoint{1.757796in}{2.195539in}}{\pgfqpoint{1.757796in}{2.185451in}}%
\pgfpathcurveto{\pgfqpoint{1.757796in}{2.175364in}}{\pgfqpoint{1.761804in}{2.165688in}}{\pgfqpoint{1.768936in}{2.158556in}}%
\pgfpathcurveto{\pgfqpoint{1.776069in}{2.151423in}}{\pgfqpoint{1.785745in}{2.147415in}}{\pgfqpoint{1.795832in}{2.147415in}}%
\pgfpathclose%
\pgfusepath{stroke,fill}%
\end{pgfscope}%
\begin{pgfscope}%
\pgfpathrectangle{\pgfqpoint{0.800000in}{1.363959in}}{\pgfqpoint{3.968000in}{2.024082in}} %
\pgfusepath{clip}%
\pgfsetbuttcap%
\pgfsetroundjoin%
\definecolor{currentfill}{rgb}{1.000000,0.498039,0.054902}%
\pgfsetfillcolor{currentfill}%
\pgfsetlinewidth{1.003750pt}%
\definecolor{currentstroke}{rgb}{1.000000,0.498039,0.054902}%
\pgfsetstrokecolor{currentstroke}%
\pgfsetdash{}{0pt}%
\pgfpathmoveto{\pgfqpoint{4.088319in}{3.084356in}}%
\pgfpathcurveto{\pgfqpoint{4.098407in}{3.084356in}}{\pgfqpoint{4.108082in}{3.088364in}}{\pgfqpoint{4.115215in}{3.095497in}}%
\pgfpathcurveto{\pgfqpoint{4.122348in}{3.102629in}}{\pgfqpoint{4.126356in}{3.112305in}}{\pgfqpoint{4.126356in}{3.122392in}}%
\pgfpathcurveto{\pgfqpoint{4.126356in}{3.132480in}}{\pgfqpoint{4.122348in}{3.142155in}}{\pgfqpoint{4.115215in}{3.149288in}}%
\pgfpathcurveto{\pgfqpoint{4.108082in}{3.156421in}}{\pgfqpoint{4.098407in}{3.160429in}}{\pgfqpoint{4.088319in}{3.160429in}}%
\pgfpathcurveto{\pgfqpoint{4.078232in}{3.160429in}}{\pgfqpoint{4.068557in}{3.156421in}}{\pgfqpoint{4.061424in}{3.149288in}}%
\pgfpathcurveto{\pgfqpoint{4.054291in}{3.142155in}}{\pgfqpoint{4.050283in}{3.132480in}}{\pgfqpoint{4.050283in}{3.122392in}}%
\pgfpathcurveto{\pgfqpoint{4.050283in}{3.112305in}}{\pgfqpoint{4.054291in}{3.102629in}}{\pgfqpoint{4.061424in}{3.095497in}}%
\pgfpathcurveto{\pgfqpoint{4.068557in}{3.088364in}}{\pgfqpoint{4.078232in}{3.084356in}}{\pgfqpoint{4.088319in}{3.084356in}}%
\pgfpathclose%
\pgfusepath{stroke,fill}%
\end{pgfscope}%
\begin{pgfscope}%
\pgfpathrectangle{\pgfqpoint{0.800000in}{1.363959in}}{\pgfqpoint{3.968000in}{2.024082in}} %
\pgfusepath{clip}%
\pgfsetbuttcap%
\pgfsetroundjoin%
\definecolor{currentfill}{rgb}{1.000000,0.498039,0.054902}%
\pgfsetfillcolor{currentfill}%
\pgfsetlinewidth{1.003750pt}%
\definecolor{currentstroke}{rgb}{1.000000,0.498039,0.054902}%
\pgfsetstrokecolor{currentstroke}%
\pgfsetdash{}{0pt}%
\pgfpathmoveto{\pgfqpoint{1.326608in}{2.840349in}}%
\pgfpathcurveto{\pgfqpoint{1.336696in}{2.840349in}}{\pgfqpoint{1.346371in}{2.844357in}}{\pgfqpoint{1.353504in}{2.851490in}}%
\pgfpathcurveto{\pgfqpoint{1.360637in}{2.858623in}}{\pgfqpoint{1.364645in}{2.868298in}}{\pgfqpoint{1.364645in}{2.878386in}}%
\pgfpathcurveto{\pgfqpoint{1.364645in}{2.888473in}}{\pgfqpoint{1.360637in}{2.898149in}}{\pgfqpoint{1.353504in}{2.905281in}}%
\pgfpathcurveto{\pgfqpoint{1.346371in}{2.912414in}}{\pgfqpoint{1.336696in}{2.916422in}}{\pgfqpoint{1.326608in}{2.916422in}}%
\pgfpathcurveto{\pgfqpoint{1.316521in}{2.916422in}}{\pgfqpoint{1.306845in}{2.912414in}}{\pgfqpoint{1.299713in}{2.905281in}}%
\pgfpathcurveto{\pgfqpoint{1.292580in}{2.898149in}}{\pgfqpoint{1.288572in}{2.888473in}}{\pgfqpoint{1.288572in}{2.878386in}}%
\pgfpathcurveto{\pgfqpoint{1.288572in}{2.868298in}}{\pgfqpoint{1.292580in}{2.858623in}}{\pgfqpoint{1.299713in}{2.851490in}}%
\pgfpathcurveto{\pgfqpoint{1.306845in}{2.844357in}}{\pgfqpoint{1.316521in}{2.840349in}}{\pgfqpoint{1.326608in}{2.840349in}}%
\pgfpathclose%
\pgfusepath{stroke,fill}%
\end{pgfscope}%
\begin{pgfscope}%
\pgfpathrectangle{\pgfqpoint{0.800000in}{1.363959in}}{\pgfqpoint{3.968000in}{2.024082in}} %
\pgfusepath{clip}%
\pgfsetbuttcap%
\pgfsetroundjoin%
\definecolor{currentfill}{rgb}{1.000000,0.498039,0.054902}%
\pgfsetfillcolor{currentfill}%
\pgfsetlinewidth{1.003750pt}%
\definecolor{currentstroke}{rgb}{1.000000,0.498039,0.054902}%
\pgfsetstrokecolor{currentstroke}%
\pgfsetdash{}{0pt}%
\pgfpathmoveto{\pgfqpoint{1.639638in}{2.550551in}}%
\pgfpathcurveto{\pgfqpoint{1.649726in}{2.550551in}}{\pgfqpoint{1.659401in}{2.554559in}}{\pgfqpoint{1.666534in}{2.561692in}}%
\pgfpathcurveto{\pgfqpoint{1.673667in}{2.568824in}}{\pgfqpoint{1.677674in}{2.578500in}}{\pgfqpoint{1.677674in}{2.588587in}}%
\pgfpathcurveto{\pgfqpoint{1.677674in}{2.598675in}}{\pgfqpoint{1.673667in}{2.608350in}}{\pgfqpoint{1.666534in}{2.615483in}}%
\pgfpathcurveto{\pgfqpoint{1.659401in}{2.622616in}}{\pgfqpoint{1.649726in}{2.626624in}}{\pgfqpoint{1.639638in}{2.626624in}}%
\pgfpathcurveto{\pgfqpoint{1.629551in}{2.626624in}}{\pgfqpoint{1.619875in}{2.622616in}}{\pgfqpoint{1.612742in}{2.615483in}}%
\pgfpathcurveto{\pgfqpoint{1.605610in}{2.608350in}}{\pgfqpoint{1.601602in}{2.598675in}}{\pgfqpoint{1.601602in}{2.588587in}}%
\pgfpathcurveto{\pgfqpoint{1.601602in}{2.578500in}}{\pgfqpoint{1.605610in}{2.568824in}}{\pgfqpoint{1.612742in}{2.561692in}}%
\pgfpathcurveto{\pgfqpoint{1.619875in}{2.554559in}}{\pgfqpoint{1.629551in}{2.550551in}}{\pgfqpoint{1.639638in}{2.550551in}}%
\pgfpathclose%
\pgfusepath{stroke,fill}%
\end{pgfscope}%
\begin{pgfscope}%
\pgfpathrectangle{\pgfqpoint{0.800000in}{1.363959in}}{\pgfqpoint{3.968000in}{2.024082in}} %
\pgfusepath{clip}%
\pgfsetbuttcap%
\pgfsetroundjoin%
\definecolor{currentfill}{rgb}{1.000000,0.498039,0.054902}%
\pgfsetfillcolor{currentfill}%
\pgfsetlinewidth{1.003750pt}%
\definecolor{currentstroke}{rgb}{1.000000,0.498039,0.054902}%
\pgfsetstrokecolor{currentstroke}%
\pgfsetdash{}{0pt}%
\pgfpathmoveto{\pgfqpoint{1.299808in}{2.401537in}}%
\pgfpathcurveto{\pgfqpoint{1.309896in}{2.401537in}}{\pgfqpoint{1.319571in}{2.405545in}}{\pgfqpoint{1.326704in}{2.412678in}}%
\pgfpathcurveto{\pgfqpoint{1.333837in}{2.419811in}}{\pgfqpoint{1.337845in}{2.429486in}}{\pgfqpoint{1.337845in}{2.439573in}}%
\pgfpathcurveto{\pgfqpoint{1.337845in}{2.449661in}}{\pgfqpoint{1.333837in}{2.459336in}}{\pgfqpoint{1.326704in}{2.466469in}}%
\pgfpathcurveto{\pgfqpoint{1.319571in}{2.473602in}}{\pgfqpoint{1.309896in}{2.477610in}}{\pgfqpoint{1.299808in}{2.477610in}}%
\pgfpathcurveto{\pgfqpoint{1.289721in}{2.477610in}}{\pgfqpoint{1.280046in}{2.473602in}}{\pgfqpoint{1.272913in}{2.466469in}}%
\pgfpathcurveto{\pgfqpoint{1.265780in}{2.459336in}}{\pgfqpoint{1.261772in}{2.449661in}}{\pgfqpoint{1.261772in}{2.439573in}}%
\pgfpathcurveto{\pgfqpoint{1.261772in}{2.429486in}}{\pgfqpoint{1.265780in}{2.419811in}}{\pgfqpoint{1.272913in}{2.412678in}}%
\pgfpathcurveto{\pgfqpoint{1.280046in}{2.405545in}}{\pgfqpoint{1.289721in}{2.401537in}}{\pgfqpoint{1.299808in}{2.401537in}}%
\pgfpathclose%
\pgfusepath{stroke,fill}%
\end{pgfscope}%
\begin{pgfscope}%
\pgfpathrectangle{\pgfqpoint{0.800000in}{1.363959in}}{\pgfqpoint{3.968000in}{2.024082in}} %
\pgfusepath{clip}%
\pgfsetbuttcap%
\pgfsetroundjoin%
\definecolor{currentfill}{rgb}{1.000000,0.498039,0.054902}%
\pgfsetfillcolor{currentfill}%
\pgfsetlinewidth{1.003750pt}%
\definecolor{currentstroke}{rgb}{1.000000,0.498039,0.054902}%
\pgfsetstrokecolor{currentstroke}%
\pgfsetdash{}{0pt}%
\pgfpathmoveto{\pgfqpoint{3.691511in}{3.171108in}}%
\pgfpathcurveto{\pgfqpoint{3.701599in}{3.171108in}}{\pgfqpoint{3.711274in}{3.175116in}}{\pgfqpoint{3.718407in}{3.182248in}}%
\pgfpathcurveto{\pgfqpoint{3.725540in}{3.189381in}}{\pgfqpoint{3.729548in}{3.199057in}}{\pgfqpoint{3.729548in}{3.209144in}}%
\pgfpathcurveto{\pgfqpoint{3.729548in}{3.219231in}}{\pgfqpoint{3.725540in}{3.228907in}}{\pgfqpoint{3.718407in}{3.236040in}}%
\pgfpathcurveto{\pgfqpoint{3.711274in}{3.243173in}}{\pgfqpoint{3.701599in}{3.247180in}}{\pgfqpoint{3.691511in}{3.247180in}}%
\pgfpathcurveto{\pgfqpoint{3.681424in}{3.247180in}}{\pgfqpoint{3.671748in}{3.243173in}}{\pgfqpoint{3.664616in}{3.236040in}}%
\pgfpathcurveto{\pgfqpoint{3.657483in}{3.228907in}}{\pgfqpoint{3.653475in}{3.219231in}}{\pgfqpoint{3.653475in}{3.209144in}}%
\pgfpathcurveto{\pgfqpoint{3.653475in}{3.199057in}}{\pgfqpoint{3.657483in}{3.189381in}}{\pgfqpoint{3.664616in}{3.182248in}}%
\pgfpathcurveto{\pgfqpoint{3.671748in}{3.175116in}}{\pgfqpoint{3.681424in}{3.171108in}}{\pgfqpoint{3.691511in}{3.171108in}}%
\pgfpathclose%
\pgfusepath{stroke,fill}%
\end{pgfscope}%
\begin{pgfscope}%
\pgfpathrectangle{\pgfqpoint{0.800000in}{1.363959in}}{\pgfqpoint{3.968000in}{2.024082in}} %
\pgfusepath{clip}%
\pgfsetbuttcap%
\pgfsetroundjoin%
\definecolor{currentfill}{rgb}{1.000000,0.498039,0.054902}%
\pgfsetfillcolor{currentfill}%
\pgfsetlinewidth{1.003750pt}%
\definecolor{currentstroke}{rgb}{1.000000,0.498039,0.054902}%
\pgfsetstrokecolor{currentstroke}%
\pgfsetdash{}{0pt}%
\pgfpathmoveto{\pgfqpoint{1.753968in}{3.085691in}}%
\pgfpathcurveto{\pgfqpoint{1.764056in}{3.085691in}}{\pgfqpoint{1.773731in}{3.089699in}}{\pgfqpoint{1.780864in}{3.096832in}}%
\pgfpathcurveto{\pgfqpoint{1.787997in}{3.103965in}}{\pgfqpoint{1.792005in}{3.113640in}}{\pgfqpoint{1.792005in}{3.123728in}}%
\pgfpathcurveto{\pgfqpoint{1.792005in}{3.133815in}}{\pgfqpoint{1.787997in}{3.143490in}}{\pgfqpoint{1.780864in}{3.150623in}}%
\pgfpathcurveto{\pgfqpoint{1.773731in}{3.157756in}}{\pgfqpoint{1.764056in}{3.161764in}}{\pgfqpoint{1.753968in}{3.161764in}}%
\pgfpathcurveto{\pgfqpoint{1.743881in}{3.161764in}}{\pgfqpoint{1.734205in}{3.157756in}}{\pgfqpoint{1.727073in}{3.150623in}}%
\pgfpathcurveto{\pgfqpoint{1.719940in}{3.143490in}}{\pgfqpoint{1.715932in}{3.133815in}}{\pgfqpoint{1.715932in}{3.123728in}}%
\pgfpathcurveto{\pgfqpoint{1.715932in}{3.113640in}}{\pgfqpoint{1.719940in}{3.103965in}}{\pgfqpoint{1.727073in}{3.096832in}}%
\pgfpathcurveto{\pgfqpoint{1.734205in}{3.089699in}}{\pgfqpoint{1.743881in}{3.085691in}}{\pgfqpoint{1.753968in}{3.085691in}}%
\pgfpathclose%
\pgfusepath{stroke,fill}%
\end{pgfscope}%
\begin{pgfscope}%
\pgfpathrectangle{\pgfqpoint{0.800000in}{1.363959in}}{\pgfqpoint{3.968000in}{2.024082in}} %
\pgfusepath{clip}%
\pgfsetbuttcap%
\pgfsetroundjoin%
\definecolor{currentfill}{rgb}{1.000000,0.498039,0.054902}%
\pgfsetfillcolor{currentfill}%
\pgfsetlinewidth{1.003750pt}%
\definecolor{currentstroke}{rgb}{1.000000,0.498039,0.054902}%
\pgfsetstrokecolor{currentstroke}%
\pgfsetdash{}{0pt}%
\pgfpathmoveto{\pgfqpoint{1.148727in}{2.881559in}}%
\pgfpathcurveto{\pgfqpoint{1.158815in}{2.881559in}}{\pgfqpoint{1.168490in}{2.885567in}}{\pgfqpoint{1.175623in}{2.892700in}}%
\pgfpathcurveto{\pgfqpoint{1.182756in}{2.899833in}}{\pgfqpoint{1.186764in}{2.909508in}}{\pgfqpoint{1.186764in}{2.919596in}}%
\pgfpathcurveto{\pgfqpoint{1.186764in}{2.929683in}}{\pgfqpoint{1.182756in}{2.939358in}}{\pgfqpoint{1.175623in}{2.946491in}}%
\pgfpathcurveto{\pgfqpoint{1.168490in}{2.953624in}}{\pgfqpoint{1.158815in}{2.957632in}}{\pgfqpoint{1.148727in}{2.957632in}}%
\pgfpathcurveto{\pgfqpoint{1.138640in}{2.957632in}}{\pgfqpoint{1.128964in}{2.953624in}}{\pgfqpoint{1.121832in}{2.946491in}}%
\pgfpathcurveto{\pgfqpoint{1.114699in}{2.939358in}}{\pgfqpoint{1.110691in}{2.929683in}}{\pgfqpoint{1.110691in}{2.919596in}}%
\pgfpathcurveto{\pgfqpoint{1.110691in}{2.909508in}}{\pgfqpoint{1.114699in}{2.899833in}}{\pgfqpoint{1.121832in}{2.892700in}}%
\pgfpathcurveto{\pgfqpoint{1.128964in}{2.885567in}}{\pgfqpoint{1.138640in}{2.881559in}}{\pgfqpoint{1.148727in}{2.881559in}}%
\pgfpathclose%
\pgfusepath{stroke,fill}%
\end{pgfscope}%
\begin{pgfscope}%
\pgfpathrectangle{\pgfqpoint{0.800000in}{1.363959in}}{\pgfqpoint{3.968000in}{2.024082in}} %
\pgfusepath{clip}%
\pgfsetbuttcap%
\pgfsetroundjoin%
\definecolor{currentfill}{rgb}{1.000000,0.498039,0.054902}%
\pgfsetfillcolor{currentfill}%
\pgfsetlinewidth{1.003750pt}%
\definecolor{currentstroke}{rgb}{1.000000,0.498039,0.054902}%
\pgfsetstrokecolor{currentstroke}%
\pgfsetdash{}{0pt}%
\pgfpathmoveto{\pgfqpoint{3.889141in}{3.104106in}}%
\pgfpathcurveto{\pgfqpoint{3.899228in}{3.104106in}}{\pgfqpoint{3.908904in}{3.108114in}}{\pgfqpoint{3.916037in}{3.115247in}}%
\pgfpathcurveto{\pgfqpoint{3.923170in}{3.122380in}}{\pgfqpoint{3.927177in}{3.132055in}}{\pgfqpoint{3.927177in}{3.142143in}}%
\pgfpathcurveto{\pgfqpoint{3.927177in}{3.152230in}}{\pgfqpoint{3.923170in}{3.161905in}}{\pgfqpoint{3.916037in}{3.169038in}}%
\pgfpathcurveto{\pgfqpoint{3.908904in}{3.176171in}}{\pgfqpoint{3.899228in}{3.180179in}}{\pgfqpoint{3.889141in}{3.180179in}}%
\pgfpathcurveto{\pgfqpoint{3.879054in}{3.180179in}}{\pgfqpoint{3.869378in}{3.176171in}}{\pgfqpoint{3.862245in}{3.169038in}}%
\pgfpathcurveto{\pgfqpoint{3.855113in}{3.161905in}}{\pgfqpoint{3.851105in}{3.152230in}}{\pgfqpoint{3.851105in}{3.142143in}}%
\pgfpathcurveto{\pgfqpoint{3.851105in}{3.132055in}}{\pgfqpoint{3.855113in}{3.122380in}}{\pgfqpoint{3.862245in}{3.115247in}}%
\pgfpathcurveto{\pgfqpoint{3.869378in}{3.108114in}}{\pgfqpoint{3.879054in}{3.104106in}}{\pgfqpoint{3.889141in}{3.104106in}}%
\pgfpathclose%
\pgfusepath{stroke,fill}%
\end{pgfscope}%
\begin{pgfscope}%
\pgfpathrectangle{\pgfqpoint{0.800000in}{1.363959in}}{\pgfqpoint{3.968000in}{2.024082in}} %
\pgfusepath{clip}%
\pgfsetbuttcap%
\pgfsetroundjoin%
\definecolor{currentfill}{rgb}{1.000000,0.498039,0.054902}%
\pgfsetfillcolor{currentfill}%
\pgfsetlinewidth{1.003750pt}%
\definecolor{currentstroke}{rgb}{1.000000,0.498039,0.054902}%
\pgfsetstrokecolor{currentstroke}%
\pgfsetdash{}{0pt}%
\pgfpathmoveto{\pgfqpoint{3.744026in}{1.668336in}}%
\pgfpathcurveto{\pgfqpoint{3.754113in}{1.668336in}}{\pgfqpoint{3.763789in}{1.672343in}}{\pgfqpoint{3.770922in}{1.679476in}}%
\pgfpathcurveto{\pgfqpoint{3.778055in}{1.686609in}}{\pgfqpoint{3.782062in}{1.696284in}}{\pgfqpoint{3.782062in}{1.706372in}}%
\pgfpathcurveto{\pgfqpoint{3.782062in}{1.716459in}}{\pgfqpoint{3.778055in}{1.726135in}}{\pgfqpoint{3.770922in}{1.733268in}}%
\pgfpathcurveto{\pgfqpoint{3.763789in}{1.740400in}}{\pgfqpoint{3.754113in}{1.744408in}}{\pgfqpoint{3.744026in}{1.744408in}}%
\pgfpathcurveto{\pgfqpoint{3.733939in}{1.744408in}}{\pgfqpoint{3.724263in}{1.740400in}}{\pgfqpoint{3.717130in}{1.733268in}}%
\pgfpathcurveto{\pgfqpoint{3.709998in}{1.726135in}}{\pgfqpoint{3.705990in}{1.716459in}}{\pgfqpoint{3.705990in}{1.706372in}}%
\pgfpathcurveto{\pgfqpoint{3.705990in}{1.696284in}}{\pgfqpoint{3.709998in}{1.686609in}}{\pgfqpoint{3.717130in}{1.679476in}}%
\pgfpathcurveto{\pgfqpoint{3.724263in}{1.672343in}}{\pgfqpoint{3.733939in}{1.668336in}}{\pgfqpoint{3.744026in}{1.668336in}}%
\pgfpathclose%
\pgfusepath{stroke,fill}%
\end{pgfscope}%
\begin{pgfscope}%
\pgfpathrectangle{\pgfqpoint{0.800000in}{1.363959in}}{\pgfqpoint{3.968000in}{2.024082in}} %
\pgfusepath{clip}%
\pgfsetbuttcap%
\pgfsetroundjoin%
\definecolor{currentfill}{rgb}{1.000000,0.498039,0.054902}%
\pgfsetfillcolor{currentfill}%
\pgfsetlinewidth{1.003750pt}%
\definecolor{currentstroke}{rgb}{1.000000,0.498039,0.054902}%
\pgfsetstrokecolor{currentstroke}%
\pgfsetdash{}{0pt}%
\pgfpathmoveto{\pgfqpoint{3.949449in}{1.691040in}}%
\pgfpathcurveto{\pgfqpoint{3.959537in}{1.691040in}}{\pgfqpoint{3.969212in}{1.695048in}}{\pgfqpoint{3.976345in}{1.702181in}}%
\pgfpathcurveto{\pgfqpoint{3.983478in}{1.709313in}}{\pgfqpoint{3.987485in}{1.718989in}}{\pgfqpoint{3.987485in}{1.729076in}}%
\pgfpathcurveto{\pgfqpoint{3.987485in}{1.739164in}}{\pgfqpoint{3.983478in}{1.748839in}}{\pgfqpoint{3.976345in}{1.755972in}}%
\pgfpathcurveto{\pgfqpoint{3.969212in}{1.763105in}}{\pgfqpoint{3.959537in}{1.767113in}}{\pgfqpoint{3.949449in}{1.767113in}}%
\pgfpathcurveto{\pgfqpoint{3.939362in}{1.767113in}}{\pgfqpoint{3.929686in}{1.763105in}}{\pgfqpoint{3.922553in}{1.755972in}}%
\pgfpathcurveto{\pgfqpoint{3.915421in}{1.748839in}}{\pgfqpoint{3.911413in}{1.739164in}}{\pgfqpoint{3.911413in}{1.729076in}}%
\pgfpathcurveto{\pgfqpoint{3.911413in}{1.718989in}}{\pgfqpoint{3.915421in}{1.709313in}}{\pgfqpoint{3.922553in}{1.702181in}}%
\pgfpathcurveto{\pgfqpoint{3.929686in}{1.695048in}}{\pgfqpoint{3.939362in}{1.691040in}}{\pgfqpoint{3.949449in}{1.691040in}}%
\pgfpathclose%
\pgfusepath{stroke,fill}%
\end{pgfscope}%
\begin{pgfscope}%
\pgfpathrectangle{\pgfqpoint{0.800000in}{1.363959in}}{\pgfqpoint{3.968000in}{2.024082in}} %
\pgfusepath{clip}%
\pgfsetbuttcap%
\pgfsetroundjoin%
\definecolor{currentfill}{rgb}{1.000000,0.498039,0.054902}%
\pgfsetfillcolor{currentfill}%
\pgfsetlinewidth{1.003750pt}%
\definecolor{currentstroke}{rgb}{1.000000,0.498039,0.054902}%
\pgfsetstrokecolor{currentstroke}%
\pgfsetdash{}{0pt}%
\pgfpathmoveto{\pgfqpoint{1.133570in}{1.746816in}}%
\pgfpathcurveto{\pgfqpoint{1.143657in}{1.746816in}}{\pgfqpoint{1.153332in}{1.750824in}}{\pgfqpoint{1.160465in}{1.757957in}}%
\pgfpathcurveto{\pgfqpoint{1.167598in}{1.765090in}}{\pgfqpoint{1.171606in}{1.774765in}}{\pgfqpoint{1.171606in}{1.784852in}}%
\pgfpathcurveto{\pgfqpoint{1.171606in}{1.794940in}}{\pgfqpoint{1.167598in}{1.804615in}}{\pgfqpoint{1.160465in}{1.811748in}}%
\pgfpathcurveto{\pgfqpoint{1.153332in}{1.818881in}}{\pgfqpoint{1.143657in}{1.822889in}}{\pgfqpoint{1.133570in}{1.822889in}}%
\pgfpathcurveto{\pgfqpoint{1.123482in}{1.822889in}}{\pgfqpoint{1.113807in}{1.818881in}}{\pgfqpoint{1.106674in}{1.811748in}}%
\pgfpathcurveto{\pgfqpoint{1.099541in}{1.804615in}}{\pgfqpoint{1.095533in}{1.794940in}}{\pgfqpoint{1.095533in}{1.784852in}}%
\pgfpathcurveto{\pgfqpoint{1.095533in}{1.774765in}}{\pgfqpoint{1.099541in}{1.765090in}}{\pgfqpoint{1.106674in}{1.757957in}}%
\pgfpathcurveto{\pgfqpoint{1.113807in}{1.750824in}}{\pgfqpoint{1.123482in}{1.746816in}}{\pgfqpoint{1.133570in}{1.746816in}}%
\pgfpathclose%
\pgfusepath{stroke,fill}%
\end{pgfscope}%
\begin{pgfscope}%
\pgfpathrectangle{\pgfqpoint{0.800000in}{1.363959in}}{\pgfqpoint{3.968000in}{2.024082in}} %
\pgfusepath{clip}%
\pgfsetbuttcap%
\pgfsetroundjoin%
\definecolor{currentfill}{rgb}{1.000000,0.498039,0.054902}%
\pgfsetfillcolor{currentfill}%
\pgfsetlinewidth{1.003750pt}%
\definecolor{currentstroke}{rgb}{1.000000,0.498039,0.054902}%
\pgfsetstrokecolor{currentstroke}%
\pgfsetdash{}{0pt}%
\pgfpathmoveto{\pgfqpoint{4.131114in}{2.557979in}}%
\pgfpathcurveto{\pgfqpoint{4.141201in}{2.557979in}}{\pgfqpoint{4.150877in}{2.561987in}}{\pgfqpoint{4.158010in}{2.569120in}}%
\pgfpathcurveto{\pgfqpoint{4.165143in}{2.576253in}}{\pgfqpoint{4.169150in}{2.585928in}}{\pgfqpoint{4.169150in}{2.596016in}}%
\pgfpathcurveto{\pgfqpoint{4.169150in}{2.606103in}}{\pgfqpoint{4.165143in}{2.615779in}}{\pgfqpoint{4.158010in}{2.622911in}}%
\pgfpathcurveto{\pgfqpoint{4.150877in}{2.630044in}}{\pgfqpoint{4.141201in}{2.634052in}}{\pgfqpoint{4.131114in}{2.634052in}}%
\pgfpathcurveto{\pgfqpoint{4.121027in}{2.634052in}}{\pgfqpoint{4.111351in}{2.630044in}}{\pgfqpoint{4.104218in}{2.622911in}}%
\pgfpathcurveto{\pgfqpoint{4.097086in}{2.615779in}}{\pgfqpoint{4.093078in}{2.606103in}}{\pgfqpoint{4.093078in}{2.596016in}}%
\pgfpathcurveto{\pgfqpoint{4.093078in}{2.585928in}}{\pgfqpoint{4.097086in}{2.576253in}}{\pgfqpoint{4.104218in}{2.569120in}}%
\pgfpathcurveto{\pgfqpoint{4.111351in}{2.561987in}}{\pgfqpoint{4.121027in}{2.557979in}}{\pgfqpoint{4.131114in}{2.557979in}}%
\pgfpathclose%
\pgfusepath{stroke,fill}%
\end{pgfscope}%
\begin{pgfscope}%
\pgfpathrectangle{\pgfqpoint{0.800000in}{1.363959in}}{\pgfqpoint{3.968000in}{2.024082in}} %
\pgfusepath{clip}%
\pgfsetbuttcap%
\pgfsetroundjoin%
\definecolor{currentfill}{rgb}{1.000000,0.498039,0.054902}%
\pgfsetfillcolor{currentfill}%
\pgfsetlinewidth{1.003750pt}%
\definecolor{currentstroke}{rgb}{1.000000,0.498039,0.054902}%
\pgfsetstrokecolor{currentstroke}%
\pgfsetdash{}{0pt}%
\pgfpathmoveto{\pgfqpoint{1.278139in}{3.086451in}}%
\pgfpathcurveto{\pgfqpoint{1.288226in}{3.086451in}}{\pgfqpoint{1.297902in}{3.090459in}}{\pgfqpoint{1.305035in}{3.097592in}}%
\pgfpathcurveto{\pgfqpoint{1.312168in}{3.104725in}}{\pgfqpoint{1.316175in}{3.114400in}}{\pgfqpoint{1.316175in}{3.124488in}}%
\pgfpathcurveto{\pgfqpoint{1.316175in}{3.134575in}}{\pgfqpoint{1.312168in}{3.144251in}}{\pgfqpoint{1.305035in}{3.151383in}}%
\pgfpathcurveto{\pgfqpoint{1.297902in}{3.158516in}}{\pgfqpoint{1.288226in}{3.162524in}}{\pgfqpoint{1.278139in}{3.162524in}}%
\pgfpathcurveto{\pgfqpoint{1.268052in}{3.162524in}}{\pgfqpoint{1.258376in}{3.158516in}}{\pgfqpoint{1.251243in}{3.151383in}}%
\pgfpathcurveto{\pgfqpoint{1.244110in}{3.144251in}}{\pgfqpoint{1.240103in}{3.134575in}}{\pgfqpoint{1.240103in}{3.124488in}}%
\pgfpathcurveto{\pgfqpoint{1.240103in}{3.114400in}}{\pgfqpoint{1.244110in}{3.104725in}}{\pgfqpoint{1.251243in}{3.097592in}}%
\pgfpathcurveto{\pgfqpoint{1.258376in}{3.090459in}}{\pgfqpoint{1.268052in}{3.086451in}}{\pgfqpoint{1.278139in}{3.086451in}}%
\pgfpathclose%
\pgfusepath{stroke,fill}%
\end{pgfscope}%
\begin{pgfscope}%
\pgfpathrectangle{\pgfqpoint{0.800000in}{1.363959in}}{\pgfqpoint{3.968000in}{2.024082in}} %
\pgfusepath{clip}%
\pgfsetbuttcap%
\pgfsetroundjoin%
\definecolor{currentfill}{rgb}{1.000000,0.498039,0.054902}%
\pgfsetfillcolor{currentfill}%
\pgfsetlinewidth{1.003750pt}%
\definecolor{currentstroke}{rgb}{1.000000,0.498039,0.054902}%
\pgfsetstrokecolor{currentstroke}%
\pgfsetdash{}{0pt}%
\pgfpathmoveto{\pgfqpoint{3.698030in}{3.243410in}}%
\pgfpathcurveto{\pgfqpoint{3.708118in}{3.243410in}}{\pgfqpoint{3.717793in}{3.247418in}}{\pgfqpoint{3.724926in}{3.254551in}}%
\pgfpathcurveto{\pgfqpoint{3.732059in}{3.261684in}}{\pgfqpoint{3.736067in}{3.271359in}}{\pgfqpoint{3.736067in}{3.281447in}}%
\pgfpathcurveto{\pgfqpoint{3.736067in}{3.291534in}}{\pgfqpoint{3.732059in}{3.301210in}}{\pgfqpoint{3.724926in}{3.308342in}}%
\pgfpathcurveto{\pgfqpoint{3.717793in}{3.315475in}}{\pgfqpoint{3.708118in}{3.319483in}}{\pgfqpoint{3.698030in}{3.319483in}}%
\pgfpathcurveto{\pgfqpoint{3.687943in}{3.319483in}}{\pgfqpoint{3.678267in}{3.315475in}}{\pgfqpoint{3.671135in}{3.308342in}}%
\pgfpathcurveto{\pgfqpoint{3.664002in}{3.301210in}}{\pgfqpoint{3.659994in}{3.291534in}}{\pgfqpoint{3.659994in}{3.281447in}}%
\pgfpathcurveto{\pgfqpoint{3.659994in}{3.271359in}}{\pgfqpoint{3.664002in}{3.261684in}}{\pgfqpoint{3.671135in}{3.254551in}}%
\pgfpathcurveto{\pgfqpoint{3.678267in}{3.247418in}}{\pgfqpoint{3.687943in}{3.243410in}}{\pgfqpoint{3.698030in}{3.243410in}}%
\pgfpathclose%
\pgfusepath{stroke,fill}%
\end{pgfscope}%
\begin{pgfscope}%
\pgfpathrectangle{\pgfqpoint{0.800000in}{1.363959in}}{\pgfqpoint{3.968000in}{2.024082in}} %
\pgfusepath{clip}%
\pgfsetbuttcap%
\pgfsetroundjoin%
\definecolor{currentfill}{rgb}{1.000000,0.498039,0.054902}%
\pgfsetfillcolor{currentfill}%
\pgfsetlinewidth{1.003750pt}%
\definecolor{currentstroke}{rgb}{1.000000,0.498039,0.054902}%
\pgfsetstrokecolor{currentstroke}%
\pgfsetdash{}{0pt}%
\pgfpathmoveto{\pgfqpoint{4.194103in}{2.723297in}}%
\pgfpathcurveto{\pgfqpoint{4.204190in}{2.723297in}}{\pgfqpoint{4.213866in}{2.727304in}}{\pgfqpoint{4.220998in}{2.734437in}}%
\pgfpathcurveto{\pgfqpoint{4.228131in}{2.741570in}}{\pgfqpoint{4.232139in}{2.751246in}}{\pgfqpoint{4.232139in}{2.761333in}}%
\pgfpathcurveto{\pgfqpoint{4.232139in}{2.771420in}}{\pgfqpoint{4.228131in}{2.781096in}}{\pgfqpoint{4.220998in}{2.788229in}}%
\pgfpathcurveto{\pgfqpoint{4.213866in}{2.795362in}}{\pgfqpoint{4.204190in}{2.799369in}}{\pgfqpoint{4.194103in}{2.799369in}}%
\pgfpathcurveto{\pgfqpoint{4.184015in}{2.799369in}}{\pgfqpoint{4.174340in}{2.795362in}}{\pgfqpoint{4.167207in}{2.788229in}}%
\pgfpathcurveto{\pgfqpoint{4.160074in}{2.781096in}}{\pgfqpoint{4.156066in}{2.771420in}}{\pgfqpoint{4.156066in}{2.761333in}}%
\pgfpathcurveto{\pgfqpoint{4.156066in}{2.751246in}}{\pgfqpoint{4.160074in}{2.741570in}}{\pgfqpoint{4.167207in}{2.734437in}}%
\pgfpathcurveto{\pgfqpoint{4.174340in}{2.727304in}}{\pgfqpoint{4.184015in}{2.723297in}}{\pgfqpoint{4.194103in}{2.723297in}}%
\pgfpathclose%
\pgfusepath{stroke,fill}%
\end{pgfscope}%
\begin{pgfscope}%
\pgfpathrectangle{\pgfqpoint{0.800000in}{1.363959in}}{\pgfqpoint{3.968000in}{2.024082in}} %
\pgfusepath{clip}%
\pgfsetbuttcap%
\pgfsetroundjoin%
\definecolor{currentfill}{rgb}{1.000000,0.498039,0.054902}%
\pgfsetfillcolor{currentfill}%
\pgfsetlinewidth{1.003750pt}%
\definecolor{currentstroke}{rgb}{1.000000,0.498039,0.054902}%
\pgfsetstrokecolor{currentstroke}%
\pgfsetdash{}{0pt}%
\pgfpathmoveto{\pgfqpoint{1.468987in}{3.092950in}}%
\pgfpathcurveto{\pgfqpoint{1.479074in}{3.092950in}}{\pgfqpoint{1.488749in}{3.096957in}}{\pgfqpoint{1.495882in}{3.104090in}}%
\pgfpathcurveto{\pgfqpoint{1.503015in}{3.111223in}}{\pgfqpoint{1.507023in}{3.120899in}}{\pgfqpoint{1.507023in}{3.130986in}}%
\pgfpathcurveto{\pgfqpoint{1.507023in}{3.141073in}}{\pgfqpoint{1.503015in}{3.150749in}}{\pgfqpoint{1.495882in}{3.157882in}}%
\pgfpathcurveto{\pgfqpoint{1.488749in}{3.165014in}}{\pgfqpoint{1.479074in}{3.169022in}}{\pgfqpoint{1.468987in}{3.169022in}}%
\pgfpathcurveto{\pgfqpoint{1.458899in}{3.169022in}}{\pgfqpoint{1.449224in}{3.165014in}}{\pgfqpoint{1.442091in}{3.157882in}}%
\pgfpathcurveto{\pgfqpoint{1.434958in}{3.150749in}}{\pgfqpoint{1.430950in}{3.141073in}}{\pgfqpoint{1.430950in}{3.130986in}}%
\pgfpathcurveto{\pgfqpoint{1.430950in}{3.120899in}}{\pgfqpoint{1.434958in}{3.111223in}}{\pgfqpoint{1.442091in}{3.104090in}}%
\pgfpathcurveto{\pgfqpoint{1.449224in}{3.096957in}}{\pgfqpoint{1.458899in}{3.092950in}}{\pgfqpoint{1.468987in}{3.092950in}}%
\pgfpathclose%
\pgfusepath{stroke,fill}%
\end{pgfscope}%
\begin{pgfscope}%
\pgfpathrectangle{\pgfqpoint{0.800000in}{1.363959in}}{\pgfqpoint{3.968000in}{2.024082in}} %
\pgfusepath{clip}%
\pgfsetbuttcap%
\pgfsetroundjoin%
\definecolor{currentfill}{rgb}{1.000000,0.498039,0.054902}%
\pgfsetfillcolor{currentfill}%
\pgfsetlinewidth{1.003750pt}%
\definecolor{currentstroke}{rgb}{1.000000,0.498039,0.054902}%
\pgfsetstrokecolor{currentstroke}%
\pgfsetdash{}{0pt}%
\pgfpathmoveto{\pgfqpoint{4.357554in}{2.144582in}}%
\pgfpathcurveto{\pgfqpoint{4.367642in}{2.144582in}}{\pgfqpoint{4.377317in}{2.148590in}}{\pgfqpoint{4.384450in}{2.155723in}}%
\pgfpathcurveto{\pgfqpoint{4.391583in}{2.162855in}}{\pgfqpoint{4.395591in}{2.172531in}}{\pgfqpoint{4.395591in}{2.182618in}}%
\pgfpathcurveto{\pgfqpoint{4.395591in}{2.192706in}}{\pgfqpoint{4.391583in}{2.202381in}}{\pgfqpoint{4.384450in}{2.209514in}}%
\pgfpathcurveto{\pgfqpoint{4.377317in}{2.216647in}}{\pgfqpoint{4.367642in}{2.220655in}}{\pgfqpoint{4.357554in}{2.220655in}}%
\pgfpathcurveto{\pgfqpoint{4.347467in}{2.220655in}}{\pgfqpoint{4.337792in}{2.216647in}}{\pgfqpoint{4.330659in}{2.209514in}}%
\pgfpathcurveto{\pgfqpoint{4.323526in}{2.202381in}}{\pgfqpoint{4.319518in}{2.192706in}}{\pgfqpoint{4.319518in}{2.182618in}}%
\pgfpathcurveto{\pgfqpoint{4.319518in}{2.172531in}}{\pgfqpoint{4.323526in}{2.162855in}}{\pgfqpoint{4.330659in}{2.155723in}}%
\pgfpathcurveto{\pgfqpoint{4.337792in}{2.148590in}}{\pgfqpoint{4.347467in}{2.144582in}}{\pgfqpoint{4.357554in}{2.144582in}}%
\pgfpathclose%
\pgfusepath{stroke,fill}%
\end{pgfscope}%
\begin{pgfscope}%
\pgfpathrectangle{\pgfqpoint{0.800000in}{1.363959in}}{\pgfqpoint{3.968000in}{2.024082in}} %
\pgfusepath{clip}%
\pgfsetbuttcap%
\pgfsetroundjoin%
\definecolor{currentfill}{rgb}{1.000000,0.498039,0.054902}%
\pgfsetfillcolor{currentfill}%
\pgfsetlinewidth{1.003750pt}%
\definecolor{currentstroke}{rgb}{1.000000,0.498039,0.054902}%
\pgfsetstrokecolor{currentstroke}%
\pgfsetdash{}{0pt}%
\pgfpathmoveto{\pgfqpoint{4.304928in}{1.862875in}}%
\pgfpathcurveto{\pgfqpoint{4.315016in}{1.862875in}}{\pgfqpoint{4.324691in}{1.866883in}}{\pgfqpoint{4.331824in}{1.874015in}}%
\pgfpathcurveto{\pgfqpoint{4.338957in}{1.881148in}}{\pgfqpoint{4.342965in}{1.890824in}}{\pgfqpoint{4.342965in}{1.900911in}}%
\pgfpathcurveto{\pgfqpoint{4.342965in}{1.910998in}}{\pgfqpoint{4.338957in}{1.920674in}}{\pgfqpoint{4.331824in}{1.927807in}}%
\pgfpathcurveto{\pgfqpoint{4.324691in}{1.934940in}}{\pgfqpoint{4.315016in}{1.938947in}}{\pgfqpoint{4.304928in}{1.938947in}}%
\pgfpathcurveto{\pgfqpoint{4.294841in}{1.938947in}}{\pgfqpoint{4.285166in}{1.934940in}}{\pgfqpoint{4.278033in}{1.927807in}}%
\pgfpathcurveto{\pgfqpoint{4.270900in}{1.920674in}}{\pgfqpoint{4.266892in}{1.910998in}}{\pgfqpoint{4.266892in}{1.900911in}}%
\pgfpathcurveto{\pgfqpoint{4.266892in}{1.890824in}}{\pgfqpoint{4.270900in}{1.881148in}}{\pgfqpoint{4.278033in}{1.874015in}}%
\pgfpathcurveto{\pgfqpoint{4.285166in}{1.866883in}}{\pgfqpoint{4.294841in}{1.862875in}}{\pgfqpoint{4.304928in}{1.862875in}}%
\pgfpathclose%
\pgfusepath{stroke,fill}%
\end{pgfscope}%
\begin{pgfscope}%
\pgfpathrectangle{\pgfqpoint{0.800000in}{1.363959in}}{\pgfqpoint{3.968000in}{2.024082in}} %
\pgfusepath{clip}%
\pgfsetbuttcap%
\pgfsetroundjoin%
\definecolor{currentfill}{rgb}{1.000000,0.498039,0.054902}%
\pgfsetfillcolor{currentfill}%
\pgfsetlinewidth{1.003750pt}%
\definecolor{currentstroke}{rgb}{1.000000,0.498039,0.054902}%
\pgfsetstrokecolor{currentstroke}%
\pgfsetdash{}{0pt}%
\pgfpathmoveto{\pgfqpoint{3.733637in}{1.720326in}}%
\pgfpathcurveto{\pgfqpoint{3.743724in}{1.720326in}}{\pgfqpoint{3.753400in}{1.724333in}}{\pgfqpoint{3.760533in}{1.731466in}}%
\pgfpathcurveto{\pgfqpoint{3.767666in}{1.738599in}}{\pgfqpoint{3.771673in}{1.748275in}}{\pgfqpoint{3.771673in}{1.758362in}}%
\pgfpathcurveto{\pgfqpoint{3.771673in}{1.768449in}}{\pgfqpoint{3.767666in}{1.778125in}}{\pgfqpoint{3.760533in}{1.785258in}}%
\pgfpathcurveto{\pgfqpoint{3.753400in}{1.792391in}}{\pgfqpoint{3.743724in}{1.796398in}}{\pgfqpoint{3.733637in}{1.796398in}}%
\pgfpathcurveto{\pgfqpoint{3.723550in}{1.796398in}}{\pgfqpoint{3.713874in}{1.792391in}}{\pgfqpoint{3.706741in}{1.785258in}}%
\pgfpathcurveto{\pgfqpoint{3.699609in}{1.778125in}}{\pgfqpoint{3.695601in}{1.768449in}}{\pgfqpoint{3.695601in}{1.758362in}}%
\pgfpathcurveto{\pgfqpoint{3.695601in}{1.748275in}}{\pgfqpoint{3.699609in}{1.738599in}}{\pgfqpoint{3.706741in}{1.731466in}}%
\pgfpathcurveto{\pgfqpoint{3.713874in}{1.724333in}}{\pgfqpoint{3.723550in}{1.720326in}}{\pgfqpoint{3.733637in}{1.720326in}}%
\pgfpathclose%
\pgfusepath{stroke,fill}%
\end{pgfscope}%
\begin{pgfscope}%
\pgfpathrectangle{\pgfqpoint{0.800000in}{1.363959in}}{\pgfqpoint{3.968000in}{2.024082in}} %
\pgfusepath{clip}%
\pgfsetbuttcap%
\pgfsetroundjoin%
\definecolor{currentfill}{rgb}{1.000000,0.498039,0.054902}%
\pgfsetfillcolor{currentfill}%
\pgfsetlinewidth{1.003750pt}%
\definecolor{currentstroke}{rgb}{1.000000,0.498039,0.054902}%
\pgfsetstrokecolor{currentstroke}%
\pgfsetdash{}{0pt}%
\pgfpathmoveto{\pgfqpoint{4.270788in}{1.670638in}}%
\pgfpathcurveto{\pgfqpoint{4.280875in}{1.670638in}}{\pgfqpoint{4.290551in}{1.674645in}}{\pgfqpoint{4.297684in}{1.681778in}}%
\pgfpathcurveto{\pgfqpoint{4.304817in}{1.688911in}}{\pgfqpoint{4.308824in}{1.698587in}}{\pgfqpoint{4.308824in}{1.708674in}}%
\pgfpathcurveto{\pgfqpoint{4.308824in}{1.718761in}}{\pgfqpoint{4.304817in}{1.728437in}}{\pgfqpoint{4.297684in}{1.735570in}}%
\pgfpathcurveto{\pgfqpoint{4.290551in}{1.742702in}}{\pgfqpoint{4.280875in}{1.746710in}}{\pgfqpoint{4.270788in}{1.746710in}}%
\pgfpathcurveto{\pgfqpoint{4.260701in}{1.746710in}}{\pgfqpoint{4.251025in}{1.742702in}}{\pgfqpoint{4.243892in}{1.735570in}}%
\pgfpathcurveto{\pgfqpoint{4.236760in}{1.728437in}}{\pgfqpoint{4.232752in}{1.718761in}}{\pgfqpoint{4.232752in}{1.708674in}}%
\pgfpathcurveto{\pgfqpoint{4.232752in}{1.698587in}}{\pgfqpoint{4.236760in}{1.688911in}}{\pgfqpoint{4.243892in}{1.681778in}}%
\pgfpathcurveto{\pgfqpoint{4.251025in}{1.674645in}}{\pgfqpoint{4.260701in}{1.670638in}}{\pgfqpoint{4.270788in}{1.670638in}}%
\pgfpathclose%
\pgfusepath{stroke,fill}%
\end{pgfscope}%
\begin{pgfscope}%
\pgfpathrectangle{\pgfqpoint{0.800000in}{1.363959in}}{\pgfqpoint{3.968000in}{2.024082in}} %
\pgfusepath{clip}%
\pgfsetbuttcap%
\pgfsetroundjoin%
\definecolor{currentfill}{rgb}{1.000000,0.498039,0.054902}%
\pgfsetfillcolor{currentfill}%
\pgfsetlinewidth{1.003750pt}%
\definecolor{currentstroke}{rgb}{1.000000,0.498039,0.054902}%
\pgfsetstrokecolor{currentstroke}%
\pgfsetdash{}{0pt}%
\pgfpathmoveto{\pgfqpoint{4.322473in}{1.889823in}}%
\pgfpathcurveto{\pgfqpoint{4.332561in}{1.889823in}}{\pgfqpoint{4.342236in}{1.893831in}}{\pgfqpoint{4.349369in}{1.900963in}}%
\pgfpathcurveto{\pgfqpoint{4.356502in}{1.908096in}}{\pgfqpoint{4.360510in}{1.917772in}}{\pgfqpoint{4.360510in}{1.927859in}}%
\pgfpathcurveto{\pgfqpoint{4.360510in}{1.937946in}}{\pgfqpoint{4.356502in}{1.947622in}}{\pgfqpoint{4.349369in}{1.954755in}}%
\pgfpathcurveto{\pgfqpoint{4.342236in}{1.961888in}}{\pgfqpoint{4.332561in}{1.965895in}}{\pgfqpoint{4.322473in}{1.965895in}}%
\pgfpathcurveto{\pgfqpoint{4.312386in}{1.965895in}}{\pgfqpoint{4.302710in}{1.961888in}}{\pgfqpoint{4.295578in}{1.954755in}}%
\pgfpathcurveto{\pgfqpoint{4.288445in}{1.947622in}}{\pgfqpoint{4.284437in}{1.937946in}}{\pgfqpoint{4.284437in}{1.927859in}}%
\pgfpathcurveto{\pgfqpoint{4.284437in}{1.917772in}}{\pgfqpoint{4.288445in}{1.908096in}}{\pgfqpoint{4.295578in}{1.900963in}}%
\pgfpathcurveto{\pgfqpoint{4.302710in}{1.893831in}}{\pgfqpoint{4.312386in}{1.889823in}}{\pgfqpoint{4.322473in}{1.889823in}}%
\pgfpathclose%
\pgfusepath{stroke,fill}%
\end{pgfscope}%
\begin{pgfscope}%
\pgfpathrectangle{\pgfqpoint{0.800000in}{1.363959in}}{\pgfqpoint{3.968000in}{2.024082in}} %
\pgfusepath{clip}%
\pgfsetbuttcap%
\pgfsetroundjoin%
\definecolor{currentfill}{rgb}{1.000000,0.498039,0.054902}%
\pgfsetfillcolor{currentfill}%
\pgfsetlinewidth{1.003750pt}%
\definecolor{currentstroke}{rgb}{1.000000,0.498039,0.054902}%
\pgfsetstrokecolor{currentstroke}%
\pgfsetdash{}{0pt}%
\pgfpathmoveto{\pgfqpoint{3.944165in}{1.976168in}}%
\pgfpathcurveto{\pgfqpoint{3.954252in}{1.976168in}}{\pgfqpoint{3.963928in}{1.980176in}}{\pgfqpoint{3.971060in}{1.987309in}}%
\pgfpathcurveto{\pgfqpoint{3.978193in}{1.994442in}}{\pgfqpoint{3.982201in}{2.004117in}}{\pgfqpoint{3.982201in}{2.014205in}}%
\pgfpathcurveto{\pgfqpoint{3.982201in}{2.024292in}}{\pgfqpoint{3.978193in}{2.033967in}}{\pgfqpoint{3.971060in}{2.041100in}}%
\pgfpathcurveto{\pgfqpoint{3.963928in}{2.048233in}}{\pgfqpoint{3.954252in}{2.052241in}}{\pgfqpoint{3.944165in}{2.052241in}}%
\pgfpathcurveto{\pgfqpoint{3.934077in}{2.052241in}}{\pgfqpoint{3.924402in}{2.048233in}}{\pgfqpoint{3.917269in}{2.041100in}}%
\pgfpathcurveto{\pgfqpoint{3.910136in}{2.033967in}}{\pgfqpoint{3.906128in}{2.024292in}}{\pgfqpoint{3.906128in}{2.014205in}}%
\pgfpathcurveto{\pgfqpoint{3.906128in}{2.004117in}}{\pgfqpoint{3.910136in}{1.994442in}}{\pgfqpoint{3.917269in}{1.987309in}}%
\pgfpathcurveto{\pgfqpoint{3.924402in}{1.980176in}}{\pgfqpoint{3.934077in}{1.976168in}}{\pgfqpoint{3.944165in}{1.976168in}}%
\pgfpathclose%
\pgfusepath{stroke,fill}%
\end{pgfscope}%
\begin{pgfscope}%
\pgfpathrectangle{\pgfqpoint{0.800000in}{1.363959in}}{\pgfqpoint{3.968000in}{2.024082in}} %
\pgfusepath{clip}%
\pgfsetbuttcap%
\pgfsetroundjoin%
\definecolor{currentfill}{rgb}{1.000000,0.498039,0.054902}%
\pgfsetfillcolor{currentfill}%
\pgfsetlinewidth{1.003750pt}%
\definecolor{currentstroke}{rgb}{1.000000,0.498039,0.054902}%
\pgfsetstrokecolor{currentstroke}%
\pgfsetdash{}{0pt}%
\pgfpathmoveto{\pgfqpoint{3.867568in}{1.894190in}}%
\pgfpathcurveto{\pgfqpoint{3.877655in}{1.894190in}}{\pgfqpoint{3.887331in}{1.898198in}}{\pgfqpoint{3.894464in}{1.905331in}}%
\pgfpathcurveto{\pgfqpoint{3.901596in}{1.912463in}}{\pgfqpoint{3.905604in}{1.922139in}}{\pgfqpoint{3.905604in}{1.932226in}}%
\pgfpathcurveto{\pgfqpoint{3.905604in}{1.942314in}}{\pgfqpoint{3.901596in}{1.951989in}}{\pgfqpoint{3.894464in}{1.959122in}}%
\pgfpathcurveto{\pgfqpoint{3.887331in}{1.966255in}}{\pgfqpoint{3.877655in}{1.970263in}}{\pgfqpoint{3.867568in}{1.970263in}}%
\pgfpathcurveto{\pgfqpoint{3.857481in}{1.970263in}}{\pgfqpoint{3.847805in}{1.966255in}}{\pgfqpoint{3.840672in}{1.959122in}}%
\pgfpathcurveto{\pgfqpoint{3.833539in}{1.951989in}}{\pgfqpoint{3.829532in}{1.942314in}}{\pgfqpoint{3.829532in}{1.932226in}}%
\pgfpathcurveto{\pgfqpoint{3.829532in}{1.922139in}}{\pgfqpoint{3.833539in}{1.912463in}}{\pgfqpoint{3.840672in}{1.905331in}}%
\pgfpathcurveto{\pgfqpoint{3.847805in}{1.898198in}}{\pgfqpoint{3.857481in}{1.894190in}}{\pgfqpoint{3.867568in}{1.894190in}}%
\pgfpathclose%
\pgfusepath{stroke,fill}%
\end{pgfscope}%
\begin{pgfscope}%
\pgfpathrectangle{\pgfqpoint{0.800000in}{1.363959in}}{\pgfqpoint{3.968000in}{2.024082in}} %
\pgfusepath{clip}%
\pgfsetbuttcap%
\pgfsetroundjoin%
\definecolor{currentfill}{rgb}{1.000000,0.498039,0.054902}%
\pgfsetfillcolor{currentfill}%
\pgfsetlinewidth{1.003750pt}%
\definecolor{currentstroke}{rgb}{1.000000,0.498039,0.054902}%
\pgfsetstrokecolor{currentstroke}%
\pgfsetdash{}{0pt}%
\pgfpathmoveto{\pgfqpoint{4.220813in}{2.392259in}}%
\pgfpathcurveto{\pgfqpoint{4.230900in}{2.392259in}}{\pgfqpoint{4.240576in}{2.396267in}}{\pgfqpoint{4.247709in}{2.403400in}}%
\pgfpathcurveto{\pgfqpoint{4.254841in}{2.410533in}}{\pgfqpoint{4.258849in}{2.420208in}}{\pgfqpoint{4.258849in}{2.430295in}}%
\pgfpathcurveto{\pgfqpoint{4.258849in}{2.440383in}}{\pgfqpoint{4.254841in}{2.450058in}}{\pgfqpoint{4.247709in}{2.457191in}}%
\pgfpathcurveto{\pgfqpoint{4.240576in}{2.464324in}}{\pgfqpoint{4.230900in}{2.468332in}}{\pgfqpoint{4.220813in}{2.468332in}}%
\pgfpathcurveto{\pgfqpoint{4.210725in}{2.468332in}}{\pgfqpoint{4.201050in}{2.464324in}}{\pgfqpoint{4.193917in}{2.457191in}}%
\pgfpathcurveto{\pgfqpoint{4.186784in}{2.450058in}}{\pgfqpoint{4.182776in}{2.440383in}}{\pgfqpoint{4.182776in}{2.430295in}}%
\pgfpathcurveto{\pgfqpoint{4.182776in}{2.420208in}}{\pgfqpoint{4.186784in}{2.410533in}}{\pgfqpoint{4.193917in}{2.403400in}}%
\pgfpathcurveto{\pgfqpoint{4.201050in}{2.396267in}}{\pgfqpoint{4.210725in}{2.392259in}}{\pgfqpoint{4.220813in}{2.392259in}}%
\pgfpathclose%
\pgfusepath{stroke,fill}%
\end{pgfscope}%
\begin{pgfscope}%
\pgfpathrectangle{\pgfqpoint{0.800000in}{1.363959in}}{\pgfqpoint{3.968000in}{2.024082in}} %
\pgfusepath{clip}%
\pgfsetbuttcap%
\pgfsetroundjoin%
\definecolor{currentfill}{rgb}{1.000000,0.498039,0.054902}%
\pgfsetfillcolor{currentfill}%
\pgfsetlinewidth{1.003750pt}%
\definecolor{currentstroke}{rgb}{1.000000,0.498039,0.054902}%
\pgfsetstrokecolor{currentstroke}%
\pgfsetdash{}{0pt}%
\pgfpathmoveto{\pgfqpoint{4.166016in}{3.095710in}}%
\pgfpathcurveto{\pgfqpoint{4.176103in}{3.095710in}}{\pgfqpoint{4.185779in}{3.099717in}}{\pgfqpoint{4.192912in}{3.106850in}}%
\pgfpathcurveto{\pgfqpoint{4.200045in}{3.113983in}}{\pgfqpoint{4.204052in}{3.123659in}}{\pgfqpoint{4.204052in}{3.133746in}}%
\pgfpathcurveto{\pgfqpoint{4.204052in}{3.143833in}}{\pgfqpoint{4.200045in}{3.153509in}}{\pgfqpoint{4.192912in}{3.160642in}}%
\pgfpathcurveto{\pgfqpoint{4.185779in}{3.167774in}}{\pgfqpoint{4.176103in}{3.171782in}}{\pgfqpoint{4.166016in}{3.171782in}}%
\pgfpathcurveto{\pgfqpoint{4.155929in}{3.171782in}}{\pgfqpoint{4.146253in}{3.167774in}}{\pgfqpoint{4.139120in}{3.160642in}}%
\pgfpathcurveto{\pgfqpoint{4.131987in}{3.153509in}}{\pgfqpoint{4.127980in}{3.143833in}}{\pgfqpoint{4.127980in}{3.133746in}}%
\pgfpathcurveto{\pgfqpoint{4.127980in}{3.123659in}}{\pgfqpoint{4.131987in}{3.113983in}}{\pgfqpoint{4.139120in}{3.106850in}}%
\pgfpathcurveto{\pgfqpoint{4.146253in}{3.099717in}}{\pgfqpoint{4.155929in}{3.095710in}}{\pgfqpoint{4.166016in}{3.095710in}}%
\pgfpathclose%
\pgfusepath{stroke,fill}%
\end{pgfscope}%
\begin{pgfscope}%
\pgfpathrectangle{\pgfqpoint{0.800000in}{1.363959in}}{\pgfqpoint{3.968000in}{2.024082in}} %
\pgfusepath{clip}%
\pgfsetbuttcap%
\pgfsetroundjoin%
\definecolor{currentfill}{rgb}{1.000000,0.498039,0.054902}%
\pgfsetfillcolor{currentfill}%
\pgfsetlinewidth{1.003750pt}%
\definecolor{currentstroke}{rgb}{1.000000,0.498039,0.054902}%
\pgfsetstrokecolor{currentstroke}%
\pgfsetdash{}{0pt}%
\pgfpathmoveto{\pgfqpoint{4.094182in}{2.080569in}}%
\pgfpathcurveto{\pgfqpoint{4.104269in}{2.080569in}}{\pgfqpoint{4.113945in}{2.084577in}}{\pgfqpoint{4.121078in}{2.091710in}}%
\pgfpathcurveto{\pgfqpoint{4.128211in}{2.098843in}}{\pgfqpoint{4.132218in}{2.108518in}}{\pgfqpoint{4.132218in}{2.118606in}}%
\pgfpathcurveto{\pgfqpoint{4.132218in}{2.128693in}}{\pgfqpoint{4.128211in}{2.138369in}}{\pgfqpoint{4.121078in}{2.145501in}}%
\pgfpathcurveto{\pgfqpoint{4.113945in}{2.152634in}}{\pgfqpoint{4.104269in}{2.156642in}}{\pgfqpoint{4.094182in}{2.156642in}}%
\pgfpathcurveto{\pgfqpoint{4.084095in}{2.156642in}}{\pgfqpoint{4.074419in}{2.152634in}}{\pgfqpoint{4.067286in}{2.145501in}}%
\pgfpathcurveto{\pgfqpoint{4.060154in}{2.138369in}}{\pgfqpoint{4.056146in}{2.128693in}}{\pgfqpoint{4.056146in}{2.118606in}}%
\pgfpathcurveto{\pgfqpoint{4.056146in}{2.108518in}}{\pgfqpoint{4.060154in}{2.098843in}}{\pgfqpoint{4.067286in}{2.091710in}}%
\pgfpathcurveto{\pgfqpoint{4.074419in}{2.084577in}}{\pgfqpoint{4.084095in}{2.080569in}}{\pgfqpoint{4.094182in}{2.080569in}}%
\pgfpathclose%
\pgfusepath{stroke,fill}%
\end{pgfscope}%
\begin{pgfscope}%
\pgfpathrectangle{\pgfqpoint{0.800000in}{1.363959in}}{\pgfqpoint{3.968000in}{2.024082in}} %
\pgfusepath{clip}%
\pgfsetbuttcap%
\pgfsetroundjoin%
\definecolor{currentfill}{rgb}{1.000000,0.498039,0.054902}%
\pgfsetfillcolor{currentfill}%
\pgfsetlinewidth{1.003750pt}%
\definecolor{currentstroke}{rgb}{1.000000,0.498039,0.054902}%
\pgfsetstrokecolor{currentstroke}%
\pgfsetdash{}{0pt}%
\pgfpathmoveto{\pgfqpoint{1.348367in}{2.205613in}}%
\pgfpathcurveto{\pgfqpoint{1.358454in}{2.205613in}}{\pgfqpoint{1.368130in}{2.209621in}}{\pgfqpoint{1.375263in}{2.216753in}}%
\pgfpathcurveto{\pgfqpoint{1.382396in}{2.223886in}}{\pgfqpoint{1.386403in}{2.233562in}}{\pgfqpoint{1.386403in}{2.243649in}}%
\pgfpathcurveto{\pgfqpoint{1.386403in}{2.253736in}}{\pgfqpoint{1.382396in}{2.263412in}}{\pgfqpoint{1.375263in}{2.270545in}}%
\pgfpathcurveto{\pgfqpoint{1.368130in}{2.277678in}}{\pgfqpoint{1.358454in}{2.281685in}}{\pgfqpoint{1.348367in}{2.281685in}}%
\pgfpathcurveto{\pgfqpoint{1.338280in}{2.281685in}}{\pgfqpoint{1.328604in}{2.277678in}}{\pgfqpoint{1.321471in}{2.270545in}}%
\pgfpathcurveto{\pgfqpoint{1.314339in}{2.263412in}}{\pgfqpoint{1.310331in}{2.253736in}}{\pgfqpoint{1.310331in}{2.243649in}}%
\pgfpathcurveto{\pgfqpoint{1.310331in}{2.233562in}}{\pgfqpoint{1.314339in}{2.223886in}}{\pgfqpoint{1.321471in}{2.216753in}}%
\pgfpathcurveto{\pgfqpoint{1.328604in}{2.209621in}}{\pgfqpoint{1.338280in}{2.205613in}}{\pgfqpoint{1.348367in}{2.205613in}}%
\pgfpathclose%
\pgfusepath{stroke,fill}%
\end{pgfscope}%
\begin{pgfscope}%
\pgfpathrectangle{\pgfqpoint{0.800000in}{1.363959in}}{\pgfqpoint{3.968000in}{2.024082in}} %
\pgfusepath{clip}%
\pgfsetbuttcap%
\pgfsetroundjoin%
\definecolor{currentfill}{rgb}{1.000000,0.498039,0.054902}%
\pgfsetfillcolor{currentfill}%
\pgfsetlinewidth{1.003750pt}%
\definecolor{currentstroke}{rgb}{1.000000,0.498039,0.054902}%
\pgfsetstrokecolor{currentstroke}%
\pgfsetdash{}{0pt}%
\pgfpathmoveto{\pgfqpoint{1.343463in}{2.352415in}}%
\pgfpathcurveto{\pgfqpoint{1.353550in}{2.352415in}}{\pgfqpoint{1.363226in}{2.356422in}}{\pgfqpoint{1.370359in}{2.363555in}}%
\pgfpathcurveto{\pgfqpoint{1.377492in}{2.370688in}}{\pgfqpoint{1.381499in}{2.380363in}}{\pgfqpoint{1.381499in}{2.390451in}}%
\pgfpathcurveto{\pgfqpoint{1.381499in}{2.400538in}}{\pgfqpoint{1.377492in}{2.410214in}}{\pgfqpoint{1.370359in}{2.417347in}}%
\pgfpathcurveto{\pgfqpoint{1.363226in}{2.424479in}}{\pgfqpoint{1.353550in}{2.428487in}}{\pgfqpoint{1.343463in}{2.428487in}}%
\pgfpathcurveto{\pgfqpoint{1.333376in}{2.428487in}}{\pgfqpoint{1.323700in}{2.424479in}}{\pgfqpoint{1.316567in}{2.417347in}}%
\pgfpathcurveto{\pgfqpoint{1.309434in}{2.410214in}}{\pgfqpoint{1.305427in}{2.400538in}}{\pgfqpoint{1.305427in}{2.390451in}}%
\pgfpathcurveto{\pgfqpoint{1.305427in}{2.380363in}}{\pgfqpoint{1.309434in}{2.370688in}}{\pgfqpoint{1.316567in}{2.363555in}}%
\pgfpathcurveto{\pgfqpoint{1.323700in}{2.356422in}}{\pgfqpoint{1.333376in}{2.352415in}}{\pgfqpoint{1.343463in}{2.352415in}}%
\pgfpathclose%
\pgfusepath{stroke,fill}%
\end{pgfscope}%
\begin{pgfscope}%
\pgfpathrectangle{\pgfqpoint{0.800000in}{1.363959in}}{\pgfqpoint{3.968000in}{2.024082in}} %
\pgfusepath{clip}%
\pgfsetbuttcap%
\pgfsetroundjoin%
\definecolor{currentfill}{rgb}{1.000000,0.498039,0.054902}%
\pgfsetfillcolor{currentfill}%
\pgfsetlinewidth{1.003750pt}%
\definecolor{currentstroke}{rgb}{1.000000,0.498039,0.054902}%
\pgfsetstrokecolor{currentstroke}%
\pgfsetdash{}{0pt}%
\pgfpathmoveto{\pgfqpoint{3.908722in}{3.148151in}}%
\pgfpathcurveto{\pgfqpoint{3.918809in}{3.148151in}}{\pgfqpoint{3.928484in}{3.152158in}}{\pgfqpoint{3.935617in}{3.159291in}}%
\pgfpathcurveto{\pgfqpoint{3.942750in}{3.166424in}}{\pgfqpoint{3.946758in}{3.176100in}}{\pgfqpoint{3.946758in}{3.186187in}}%
\pgfpathcurveto{\pgfqpoint{3.946758in}{3.196274in}}{\pgfqpoint{3.942750in}{3.205950in}}{\pgfqpoint{3.935617in}{3.213083in}}%
\pgfpathcurveto{\pgfqpoint{3.928484in}{3.220216in}}{\pgfqpoint{3.918809in}{3.224223in}}{\pgfqpoint{3.908722in}{3.224223in}}%
\pgfpathcurveto{\pgfqpoint{3.898634in}{3.224223in}}{\pgfqpoint{3.888959in}{3.220216in}}{\pgfqpoint{3.881826in}{3.213083in}}%
\pgfpathcurveto{\pgfqpoint{3.874693in}{3.205950in}}{\pgfqpoint{3.870685in}{3.196274in}}{\pgfqpoint{3.870685in}{3.186187in}}%
\pgfpathcurveto{\pgfqpoint{3.870685in}{3.176100in}}{\pgfqpoint{3.874693in}{3.166424in}}{\pgfqpoint{3.881826in}{3.159291in}}%
\pgfpathcurveto{\pgfqpoint{3.888959in}{3.152158in}}{\pgfqpoint{3.898634in}{3.148151in}}{\pgfqpoint{3.908722in}{3.148151in}}%
\pgfpathclose%
\pgfusepath{stroke,fill}%
\end{pgfscope}%
\begin{pgfscope}%
\pgfpathrectangle{\pgfqpoint{0.800000in}{1.363959in}}{\pgfqpoint{3.968000in}{2.024082in}} %
\pgfusepath{clip}%
\pgfsetbuttcap%
\pgfsetroundjoin%
\definecolor{currentfill}{rgb}{1.000000,0.498039,0.054902}%
\pgfsetfillcolor{currentfill}%
\pgfsetlinewidth{1.003750pt}%
\definecolor{currentstroke}{rgb}{1.000000,0.498039,0.054902}%
\pgfsetstrokecolor{currentstroke}%
\pgfsetdash{}{0pt}%
\pgfpathmoveto{\pgfqpoint{1.508037in}{1.454552in}}%
\pgfpathcurveto{\pgfqpoint{1.518125in}{1.454552in}}{\pgfqpoint{1.527800in}{1.458560in}}{\pgfqpoint{1.534933in}{1.465693in}}%
\pgfpathcurveto{\pgfqpoint{1.542066in}{1.472825in}}{\pgfqpoint{1.546073in}{1.482501in}}{\pgfqpoint{1.546073in}{1.492588in}}%
\pgfpathcurveto{\pgfqpoint{1.546073in}{1.502676in}}{\pgfqpoint{1.542066in}{1.512351in}}{\pgfqpoint{1.534933in}{1.519484in}}%
\pgfpathcurveto{\pgfqpoint{1.527800in}{1.526617in}}{\pgfqpoint{1.518125in}{1.530625in}}{\pgfqpoint{1.508037in}{1.530625in}}%
\pgfpathcurveto{\pgfqpoint{1.497950in}{1.530625in}}{\pgfqpoint{1.488274in}{1.526617in}}{\pgfqpoint{1.481141in}{1.519484in}}%
\pgfpathcurveto{\pgfqpoint{1.474009in}{1.512351in}}{\pgfqpoint{1.470001in}{1.502676in}}{\pgfqpoint{1.470001in}{1.492588in}}%
\pgfpathcurveto{\pgfqpoint{1.470001in}{1.482501in}}{\pgfqpoint{1.474009in}{1.472825in}}{\pgfqpoint{1.481141in}{1.465693in}}%
\pgfpathcurveto{\pgfqpoint{1.488274in}{1.458560in}}{\pgfqpoint{1.497950in}{1.454552in}}{\pgfqpoint{1.508037in}{1.454552in}}%
\pgfpathclose%
\pgfusepath{stroke,fill}%
\end{pgfscope}%
\begin{pgfscope}%
\pgfpathrectangle{\pgfqpoint{0.800000in}{1.363959in}}{\pgfqpoint{3.968000in}{2.024082in}} %
\pgfusepath{clip}%
\pgfsetbuttcap%
\pgfsetroundjoin%
\definecolor{currentfill}{rgb}{1.000000,0.498039,0.054902}%
\pgfsetfillcolor{currentfill}%
\pgfsetlinewidth{1.003750pt}%
\definecolor{currentstroke}{rgb}{1.000000,0.498039,0.054902}%
\pgfsetstrokecolor{currentstroke}%
\pgfsetdash{}{0pt}%
\pgfpathmoveto{\pgfqpoint{4.215167in}{2.734034in}}%
\pgfpathcurveto{\pgfqpoint{4.225255in}{2.734034in}}{\pgfqpoint{4.234930in}{2.738042in}}{\pgfqpoint{4.242063in}{2.745174in}}%
\pgfpathcurveto{\pgfqpoint{4.249196in}{2.752307in}}{\pgfqpoint{4.253204in}{2.761983in}}{\pgfqpoint{4.253204in}{2.772070in}}%
\pgfpathcurveto{\pgfqpoint{4.253204in}{2.782158in}}{\pgfqpoint{4.249196in}{2.791833in}}{\pgfqpoint{4.242063in}{2.798966in}}%
\pgfpathcurveto{\pgfqpoint{4.234930in}{2.806099in}}{\pgfqpoint{4.225255in}{2.810106in}}{\pgfqpoint{4.215167in}{2.810106in}}%
\pgfpathcurveto{\pgfqpoint{4.205080in}{2.810106in}}{\pgfqpoint{4.195404in}{2.806099in}}{\pgfqpoint{4.188272in}{2.798966in}}%
\pgfpathcurveto{\pgfqpoint{4.181139in}{2.791833in}}{\pgfqpoint{4.177131in}{2.782158in}}{\pgfqpoint{4.177131in}{2.772070in}}%
\pgfpathcurveto{\pgfqpoint{4.177131in}{2.761983in}}{\pgfqpoint{4.181139in}{2.752307in}}{\pgfqpoint{4.188272in}{2.745174in}}%
\pgfpathcurveto{\pgfqpoint{4.195404in}{2.738042in}}{\pgfqpoint{4.205080in}{2.734034in}}{\pgfqpoint{4.215167in}{2.734034in}}%
\pgfpathclose%
\pgfusepath{stroke,fill}%
\end{pgfscope}%
\begin{pgfscope}%
\pgfpathrectangle{\pgfqpoint{0.800000in}{1.363959in}}{\pgfqpoint{3.968000in}{2.024082in}} %
\pgfusepath{clip}%
\pgfsetbuttcap%
\pgfsetroundjoin%
\definecolor{currentfill}{rgb}{1.000000,0.498039,0.054902}%
\pgfsetfillcolor{currentfill}%
\pgfsetlinewidth{1.003750pt}%
\definecolor{currentstroke}{rgb}{1.000000,0.498039,0.054902}%
\pgfsetstrokecolor{currentstroke}%
\pgfsetdash{}{0pt}%
\pgfpathmoveto{\pgfqpoint{1.696153in}{3.058862in}}%
\pgfpathcurveto{\pgfqpoint{1.706240in}{3.058862in}}{\pgfqpoint{1.715916in}{3.062870in}}{\pgfqpoint{1.723049in}{3.070003in}}%
\pgfpathcurveto{\pgfqpoint{1.730181in}{3.077136in}}{\pgfqpoint{1.734189in}{3.086811in}}{\pgfqpoint{1.734189in}{3.096898in}}%
\pgfpathcurveto{\pgfqpoint{1.734189in}{3.106986in}}{\pgfqpoint{1.730181in}{3.116661in}}{\pgfqpoint{1.723049in}{3.123794in}}%
\pgfpathcurveto{\pgfqpoint{1.715916in}{3.130927in}}{\pgfqpoint{1.706240in}{3.134935in}}{\pgfqpoint{1.696153in}{3.134935in}}%
\pgfpathcurveto{\pgfqpoint{1.686066in}{3.134935in}}{\pgfqpoint{1.676390in}{3.130927in}}{\pgfqpoint{1.669257in}{3.123794in}}%
\pgfpathcurveto{\pgfqpoint{1.662124in}{3.116661in}}{\pgfqpoint{1.658117in}{3.106986in}}{\pgfqpoint{1.658117in}{3.096898in}}%
\pgfpathcurveto{\pgfqpoint{1.658117in}{3.086811in}}{\pgfqpoint{1.662124in}{3.077136in}}{\pgfqpoint{1.669257in}{3.070003in}}%
\pgfpathcurveto{\pgfqpoint{1.676390in}{3.062870in}}{\pgfqpoint{1.686066in}{3.058862in}}{\pgfqpoint{1.696153in}{3.058862in}}%
\pgfpathclose%
\pgfusepath{stroke,fill}%
\end{pgfscope}%
\begin{pgfscope}%
\pgfpathrectangle{\pgfqpoint{0.800000in}{1.363959in}}{\pgfqpoint{3.968000in}{2.024082in}} %
\pgfusepath{clip}%
\pgfsetbuttcap%
\pgfsetroundjoin%
\definecolor{currentfill}{rgb}{1.000000,0.498039,0.054902}%
\pgfsetfillcolor{currentfill}%
\pgfsetlinewidth{1.003750pt}%
\definecolor{currentstroke}{rgb}{1.000000,0.498039,0.054902}%
\pgfsetstrokecolor{currentstroke}%
\pgfsetdash{}{0pt}%
\pgfpathmoveto{\pgfqpoint{4.576822in}{2.370617in}}%
\pgfpathcurveto{\pgfqpoint{4.586910in}{2.370617in}}{\pgfqpoint{4.596585in}{2.374625in}}{\pgfqpoint{4.603718in}{2.381758in}}%
\pgfpathcurveto{\pgfqpoint{4.610851in}{2.388891in}}{\pgfqpoint{4.614859in}{2.398566in}}{\pgfqpoint{4.614859in}{2.408654in}}%
\pgfpathcurveto{\pgfqpoint{4.614859in}{2.418741in}}{\pgfqpoint{4.610851in}{2.428417in}}{\pgfqpoint{4.603718in}{2.435549in}}%
\pgfpathcurveto{\pgfqpoint{4.596585in}{2.442682in}}{\pgfqpoint{4.586910in}{2.446690in}}{\pgfqpoint{4.576822in}{2.446690in}}%
\pgfpathcurveto{\pgfqpoint{4.566735in}{2.446690in}}{\pgfqpoint{4.557059in}{2.442682in}}{\pgfqpoint{4.549927in}{2.435549in}}%
\pgfpathcurveto{\pgfqpoint{4.542794in}{2.428417in}}{\pgfqpoint{4.538786in}{2.418741in}}{\pgfqpoint{4.538786in}{2.408654in}}%
\pgfpathcurveto{\pgfqpoint{4.538786in}{2.398566in}}{\pgfqpoint{4.542794in}{2.388891in}}{\pgfqpoint{4.549927in}{2.381758in}}%
\pgfpathcurveto{\pgfqpoint{4.557059in}{2.374625in}}{\pgfqpoint{4.566735in}{2.370617in}}{\pgfqpoint{4.576822in}{2.370617in}}%
\pgfpathclose%
\pgfusepath{stroke,fill}%
\end{pgfscope}%
\begin{pgfscope}%
\pgfpathrectangle{\pgfqpoint{0.800000in}{1.363959in}}{\pgfqpoint{3.968000in}{2.024082in}} %
\pgfusepath{clip}%
\pgfsetbuttcap%
\pgfsetroundjoin%
\definecolor{currentfill}{rgb}{0.993248,0.906157,0.143936}%
\pgfsetfillcolor{currentfill}%
\pgfsetlinewidth{0.000000pt}%
\definecolor{currentstroke}{rgb}{0.000000,0.000000,0.000000}%
\pgfsetstrokecolor{currentstroke}%
\pgfsetdash{}{0pt}%
\pgfpathmoveto{\pgfqpoint{2.574317in}{1.461076in}}%
\pgfpathlineto{\pgfqpoint{1.764113in}{1.473935in}}%
\pgfpathlineto{\pgfqpoint{1.774633in}{1.452032in}}%
\pgfpathlineto{\pgfqpoint{1.666496in}{1.486353in}}%
\pgfpathlineto{\pgfqpoint{1.775668in}{1.517224in}}%
\pgfpathlineto{\pgfqpoint{1.764458in}{1.495666in}}%
\pgfpathlineto{\pgfqpoint{2.574662in}{1.482807in}}%
\pgfpathlineto{\pgfqpoint{2.574317in}{1.461076in}}%
\pgfusepath{fill}%
\end{pgfscope}%
\begin{pgfscope}%
\pgfpathrectangle{\pgfqpoint{0.800000in}{1.363959in}}{\pgfqpoint{3.968000in}{2.024082in}} %
\pgfusepath{clip}%
\pgfsetbuttcap%
\pgfsetroundjoin%
\definecolor{currentfill}{rgb}{0.280255,0.165693,0.476498}%
\pgfsetfillcolor{currentfill}%
\pgfsetlinewidth{0.000000pt}%
\definecolor{currentstroke}{rgb}{0.000000,0.000000,0.000000}%
\pgfsetstrokecolor{currentstroke}%
\pgfsetdash{}{0pt}%
\pgfpathmoveto{\pgfqpoint{2.359358in}{1.883322in}}%
\pgfpathlineto{\pgfqpoint{1.547986in}{1.973335in}}%
\pgfpathlineto{\pgfqpoint{1.556390in}{1.950535in}}%
\pgfpathlineto{\pgfqpoint{1.451979in}{1.994919in}}%
\pgfpathlineto{\pgfqpoint{1.563579in}{2.015339in}}%
\pgfpathlineto{\pgfqpoint{1.550382in}{1.994936in}}%
\pgfpathlineto{\pgfqpoint{2.361754in}{1.904923in}}%
\pgfpathlineto{\pgfqpoint{2.359358in}{1.883322in}}%
\pgfusepath{fill}%
\end{pgfscope}%
\begin{pgfscope}%
\pgfpathrectangle{\pgfqpoint{0.800000in}{1.363959in}}{\pgfqpoint{3.968000in}{2.024082in}} %
\pgfusepath{clip}%
\pgfsetbuttcap%
\pgfsetroundjoin%
\definecolor{currentfill}{rgb}{0.327796,0.773980,0.406640}%
\pgfsetfillcolor{currentfill}%
\pgfsetlinewidth{0.000000pt}%
\definecolor{currentstroke}{rgb}{0.000000,0.000000,0.000000}%
\pgfsetstrokecolor{currentstroke}%
\pgfsetdash{}{0pt}%
\pgfpathmoveto{\pgfqpoint{2.360195in}{1.883261in}}%
\pgfpathlineto{\pgfqpoint{1.563236in}{1.909783in}}%
\pgfpathlineto{\pgfqpoint{1.573374in}{1.887700in}}%
\pgfpathlineto{\pgfqpoint{1.465850in}{1.923896in}}%
\pgfpathlineto{\pgfqpoint{1.575542in}{1.952865in}}%
\pgfpathlineto{\pgfqpoint{1.563959in}{1.931504in}}%
\pgfpathlineto{\pgfqpoint{2.360918in}{1.904983in}}%
\pgfpathlineto{\pgfqpoint{2.360195in}{1.883261in}}%
\pgfusepath{fill}%
\end{pgfscope}%
\begin{pgfscope}%
\pgfpathrectangle{\pgfqpoint{0.800000in}{1.363959in}}{\pgfqpoint{3.968000in}{2.024082in}} %
\pgfusepath{clip}%
\pgfsetbuttcap%
\pgfsetroundjoin%
\definecolor{currentfill}{rgb}{0.993248,0.906157,0.143936}%
\pgfsetfillcolor{currentfill}%
\pgfsetlinewidth{0.000000pt}%
\definecolor{currentstroke}{rgb}{0.000000,0.000000,0.000000}%
\pgfsetstrokecolor{currentstroke}%
\pgfsetdash{}{0pt}%
\pgfpathmoveto{\pgfqpoint{1.970837in}{2.429270in}}%
\pgfpathlineto{\pgfqpoint{1.262348in}{2.297470in}}%
\pgfpathlineto{\pgfqpoint{1.277007in}{2.278090in}}%
\pgfpathlineto{\pgfqpoint{1.164209in}{2.290266in}}%
\pgfpathlineto{\pgfqpoint{1.265082in}{2.342191in}}%
\pgfpathlineto{\pgfqpoint{1.258373in}{2.318837in}}%
\pgfpathlineto{\pgfqpoint{1.966862in}{2.450637in}}%
\pgfpathlineto{\pgfqpoint{1.970837in}{2.429270in}}%
\pgfusepath{fill}%
\end{pgfscope}%
\begin{pgfscope}%
\pgfpathrectangle{\pgfqpoint{0.800000in}{1.363959in}}{\pgfqpoint{3.968000in}{2.024082in}} %
\pgfusepath{clip}%
\pgfsetbuttcap%
\pgfsetroundjoin%
\definecolor{currentfill}{rgb}{0.282290,0.145912,0.461510}%
\pgfsetfillcolor{currentfill}%
\pgfsetlinewidth{0.000000pt}%
\definecolor{currentstroke}{rgb}{0.000000,0.000000,0.000000}%
\pgfsetstrokecolor{currentstroke}%
\pgfsetdash{}{0pt}%
\pgfpathmoveto{\pgfqpoint{2.146199in}{2.455400in}}%
\pgfpathlineto{\pgfqpoint{1.432704in}{2.310988in}}%
\pgfpathlineto{\pgfqpoint{1.447667in}{2.291842in}}%
\pgfpathlineto{\pgfqpoint{1.334691in}{2.302237in}}%
\pgfpathlineto{\pgfqpoint{1.434732in}{2.355747in}}%
\pgfpathlineto{\pgfqpoint{1.428393in}{2.332290in}}%
\pgfpathlineto{\pgfqpoint{2.141888in}{2.476702in}}%
\pgfpathlineto{\pgfqpoint{2.146199in}{2.455400in}}%
\pgfusepath{fill}%
\end{pgfscope}%
\begin{pgfscope}%
\pgfpathrectangle{\pgfqpoint{0.800000in}{1.363959in}}{\pgfqpoint{3.968000in}{2.024082in}} %
\pgfusepath{clip}%
\pgfsetbuttcap%
\pgfsetroundjoin%
\definecolor{currentfill}{rgb}{0.360741,0.785964,0.387814}%
\pgfsetfillcolor{currentfill}%
\pgfsetlinewidth{0.000000pt}%
\definecolor{currentstroke}{rgb}{0.000000,0.000000,0.000000}%
\pgfsetstrokecolor{currentstroke}%
\pgfsetdash{}{0pt}%
\pgfpathmoveto{\pgfqpoint{2.145908in}{2.455346in}}%
\pgfpathlineto{\pgfqpoint{1.441770in}{2.332696in}}%
\pgfpathlineto{\pgfqpoint{1.456205in}{2.313150in}}%
\pgfpathlineto{\pgfqpoint{1.343554in}{2.326619in}}%
\pgfpathlineto{\pgfqpoint{1.445016in}{2.377383in}}%
\pgfpathlineto{\pgfqpoint{1.438040in}{2.354107in}}%
\pgfpathlineto{\pgfqpoint{2.142179in}{2.476757in}}%
\pgfpathlineto{\pgfqpoint{2.145908in}{2.455346in}}%
\pgfusepath{fill}%
\end{pgfscope}%
\begin{pgfscope}%
\pgfpathrectangle{\pgfqpoint{0.800000in}{1.363959in}}{\pgfqpoint{3.968000in}{2.024082in}} %
\pgfusepath{clip}%
\pgfsetbuttcap%
\pgfsetroundjoin%
\definecolor{currentfill}{rgb}{0.280255,0.165693,0.476498}%
\pgfsetfillcolor{currentfill}%
\pgfsetlinewidth{0.000000pt}%
\definecolor{currentstroke}{rgb}{0.000000,0.000000,0.000000}%
\pgfsetstrokecolor{currentstroke}%
\pgfsetdash{}{0pt}%
\pgfpathmoveto{\pgfqpoint{2.441102in}{2.994653in}}%
\pgfpathlineto{\pgfqpoint{1.666606in}{3.000981in}}%
\pgfpathlineto{\pgfqpoint{1.677295in}{2.979159in}}%
\pgfpathlineto{\pgfqpoint{1.568897in}{3.012646in}}%
\pgfpathlineto{\pgfqpoint{1.677827in}{3.044358in}}%
\pgfpathlineto{\pgfqpoint{1.666783in}{3.022714in}}%
\pgfpathlineto{\pgfqpoint{2.441280in}{3.016386in}}%
\pgfpathlineto{\pgfqpoint{2.441102in}{2.994653in}}%
\pgfusepath{fill}%
\end{pgfscope}%
\begin{pgfscope}%
\pgfpathrectangle{\pgfqpoint{0.800000in}{1.363959in}}{\pgfqpoint{3.968000in}{2.024082in}} %
\pgfusepath{clip}%
\pgfsetbuttcap%
\pgfsetroundjoin%
\definecolor{currentfill}{rgb}{0.327796,0.773980,0.406640}%
\pgfsetfillcolor{currentfill}%
\pgfsetlinewidth{0.000000pt}%
\definecolor{currentstroke}{rgb}{0.000000,0.000000,0.000000}%
\pgfsetstrokecolor{currentstroke}%
\pgfsetdash{}{0pt}%
\pgfpathmoveto{\pgfqpoint{2.439800in}{2.994742in}}%
\pgfpathlineto{\pgfqpoint{1.564593in}{3.107691in}}%
\pgfpathlineto{\pgfqpoint{1.572588in}{3.084745in}}%
\pgfpathlineto{\pgfqpoint{1.468987in}{3.130986in}}%
\pgfpathlineto{\pgfqpoint{1.580934in}{3.149410in}}%
\pgfpathlineto{\pgfqpoint{1.567374in}{3.129246in}}%
\pgfpathlineto{\pgfqpoint{2.442582in}{3.016297in}}%
\pgfpathlineto{\pgfqpoint{2.439800in}{2.994742in}}%
\pgfusepath{fill}%
\end{pgfscope}%
\begin{pgfscope}%
\pgfpathrectangle{\pgfqpoint{0.800000in}{1.363959in}}{\pgfqpoint{3.968000in}{2.024082in}} %
\pgfusepath{clip}%
\pgfsetbuttcap%
\pgfsetroundjoin%
\definecolor{currentfill}{rgb}{0.647257,0.858400,0.209861}%
\pgfsetfillcolor{currentfill}%
\pgfsetlinewidth{0.000000pt}%
\definecolor{currentstroke}{rgb}{0.000000,0.000000,0.000000}%
\pgfsetstrokecolor{currentstroke}%
\pgfsetdash{}{0pt}%
\pgfpathmoveto{\pgfqpoint{2.538116in}{1.655707in}}%
\pgfpathlineto{\pgfqpoint{1.675076in}{1.542946in}}%
\pgfpathlineto{\pgfqpoint{1.688667in}{1.522803in}}%
\pgfpathlineto{\pgfqpoint{1.576691in}{1.541050in}}%
\pgfpathlineto{\pgfqpoint{1.680220in}{1.587454in}}%
\pgfpathlineto{\pgfqpoint{1.672260in}{1.564496in}}%
\pgfpathlineto{\pgfqpoint{2.535300in}{1.677257in}}%
\pgfpathlineto{\pgfqpoint{2.538116in}{1.655707in}}%
\pgfusepath{fill}%
\end{pgfscope}%
\begin{pgfscope}%
\pgfpathrectangle{\pgfqpoint{0.800000in}{1.363959in}}{\pgfqpoint{3.968000in}{2.024082in}} %
\pgfusepath{clip}%
\pgfsetbuttcap%
\pgfsetroundjoin%
\definecolor{currentfill}{rgb}{0.266941,0.748751,0.440573}%
\pgfsetfillcolor{currentfill}%
\pgfsetlinewidth{0.000000pt}%
\definecolor{currentstroke}{rgb}{0.000000,0.000000,0.000000}%
\pgfsetstrokecolor{currentstroke}%
\pgfsetdash{}{0pt}%
\pgfpathmoveto{\pgfqpoint{2.802951in}{1.966419in}}%
\pgfpathlineto{\pgfqpoint{3.617778in}{1.870167in}}%
\pgfpathlineto{\pgfqpoint{3.609536in}{1.893026in}}%
\pgfpathlineto{\pgfqpoint{3.713630in}{1.847902in}}%
\pgfpathlineto{\pgfqpoint{3.601888in}{1.828275in}}%
\pgfpathlineto{\pgfqpoint{3.615229in}{1.848584in}}%
\pgfpathlineto{\pgfqpoint{2.800401in}{1.944835in}}%
\pgfpathlineto{\pgfqpoint{2.802951in}{1.966419in}}%
\pgfusepath{fill}%
\end{pgfscope}%
\begin{pgfscope}%
\pgfpathrectangle{\pgfqpoint{0.800000in}{1.363959in}}{\pgfqpoint{3.968000in}{2.024082in}} %
\pgfusepath{clip}%
\pgfsetbuttcap%
\pgfsetroundjoin%
\definecolor{currentfill}{rgb}{0.273006,0.204520,0.501721}%
\pgfsetfillcolor{currentfill}%
\pgfsetlinewidth{0.000000pt}%
\definecolor{currentstroke}{rgb}{0.000000,0.000000,0.000000}%
\pgfsetstrokecolor{currentstroke}%
\pgfsetdash{}{0pt}%
\pgfpathmoveto{\pgfqpoint{2.801915in}{1.966491in}}%
\pgfpathlineto{\pgfqpoint{3.770029in}{1.945237in}}%
\pgfpathlineto{\pgfqpoint{3.759641in}{1.967204in}}%
\pgfpathlineto{\pgfqpoint{3.867568in}{1.932226in}}%
\pgfpathlineto{\pgfqpoint{3.758210in}{1.902019in}}%
\pgfpathlineto{\pgfqpoint{3.769552in}{1.923509in}}%
\pgfpathlineto{\pgfqpoint{2.801438in}{1.944763in}}%
\pgfpathlineto{\pgfqpoint{2.801915in}{1.966491in}}%
\pgfusepath{fill}%
\end{pgfscope}%
\begin{pgfscope}%
\pgfpathrectangle{\pgfqpoint{0.800000in}{1.363959in}}{\pgfqpoint{3.968000in}{2.024082in}} %
\pgfusepath{clip}%
\pgfsetbuttcap%
\pgfsetroundjoin%
\definecolor{currentfill}{rgb}{0.993248,0.906157,0.143936}%
\pgfsetfillcolor{currentfill}%
\pgfsetlinewidth{0.000000pt}%
\definecolor{currentstroke}{rgb}{0.000000,0.000000,0.000000}%
\pgfsetstrokecolor{currentstroke}%
\pgfsetdash{}{0pt}%
\pgfpathmoveto{\pgfqpoint{2.579403in}{2.516657in}}%
\pgfpathlineto{\pgfqpoint{1.746098in}{2.450380in}}%
\pgfpathlineto{\pgfqpoint{1.758654in}{2.429576in}}%
\pgfpathlineto{\pgfqpoint{1.647743in}{2.453458in}}%
\pgfpathlineto{\pgfqpoint{1.753484in}{2.494572in}}%
\pgfpathlineto{\pgfqpoint{1.744375in}{2.472045in}}%
\pgfpathlineto{\pgfqpoint{2.577680in}{2.538322in}}%
\pgfpathlineto{\pgfqpoint{2.579403in}{2.516657in}}%
\pgfusepath{fill}%
\end{pgfscope}%
\begin{pgfscope}%
\pgfpathrectangle{\pgfqpoint{0.800000in}{1.363959in}}{\pgfqpoint{3.968000in}{2.024082in}} %
\pgfusepath{clip}%
\pgfsetbuttcap%
\pgfsetroundjoin%
\definecolor{currentfill}{rgb}{0.993248,0.906157,0.143936}%
\pgfsetfillcolor{currentfill}%
\pgfsetlinewidth{0.000000pt}%
\definecolor{currentstroke}{rgb}{0.000000,0.000000,0.000000}%
\pgfsetstrokecolor{currentstroke}%
\pgfsetdash{}{0pt}%
\pgfpathmoveto{\pgfqpoint{2.179155in}{2.470559in}}%
\pgfpathlineto{\pgfqpoint{1.441866in}{2.390240in}}%
\pgfpathlineto{\pgfqpoint{1.455023in}{2.369811in}}%
\pgfpathlineto{\pgfqpoint{1.343463in}{2.390451in}}%
\pgfpathlineto{\pgfqpoint{1.447961in}{2.434628in}}%
\pgfpathlineto{\pgfqpoint{1.439512in}{2.411845in}}%
\pgfpathlineto{\pgfqpoint{2.176801in}{2.492165in}}%
\pgfpathlineto{\pgfqpoint{2.179155in}{2.470559in}}%
\pgfusepath{fill}%
\end{pgfscope}%
\begin{pgfscope}%
\pgfpathrectangle{\pgfqpoint{0.800000in}{1.363959in}}{\pgfqpoint{3.968000in}{2.024082in}} %
\pgfusepath{clip}%
\pgfsetbuttcap%
\pgfsetroundjoin%
\definecolor{currentfill}{rgb}{0.993248,0.906157,0.143936}%
\pgfsetfillcolor{currentfill}%
\pgfsetlinewidth{0.000000pt}%
\definecolor{currentstroke}{rgb}{0.000000,0.000000,0.000000}%
\pgfsetstrokecolor{currentstroke}%
\pgfsetdash{}{0pt}%
\pgfpathmoveto{\pgfqpoint{3.047845in}{1.880282in}}%
\pgfpathlineto{\pgfqpoint{3.934319in}{1.831440in}}%
\pgfpathlineto{\pgfqpoint{3.924664in}{1.853739in}}%
\pgfpathlineto{\pgfqpoint{4.031374in}{1.815210in}}%
\pgfpathlineto{\pgfqpoint{3.921077in}{1.788637in}}%
\pgfpathlineto{\pgfqpoint{3.933123in}{1.809740in}}%
\pgfpathlineto{\pgfqpoint{3.046650in}{1.858581in}}%
\pgfpathlineto{\pgfqpoint{3.047845in}{1.880282in}}%
\pgfusepath{fill}%
\end{pgfscope}%
\begin{pgfscope}%
\pgfpathrectangle{\pgfqpoint{0.800000in}{1.363959in}}{\pgfqpoint{3.968000in}{2.024082in}} %
\pgfusepath{clip}%
\pgfsetbuttcap%
\pgfsetroundjoin%
\definecolor{currentfill}{rgb}{0.993248,0.906157,0.143936}%
\pgfsetfillcolor{currentfill}%
\pgfsetlinewidth{0.000000pt}%
\definecolor{currentstroke}{rgb}{0.000000,0.000000,0.000000}%
\pgfsetstrokecolor{currentstroke}%
\pgfsetdash{}{0pt}%
\pgfpathmoveto{\pgfqpoint{2.576276in}{2.929578in}}%
\pgfpathlineto{\pgfqpoint{1.758283in}{2.834130in}}%
\pgfpathlineto{\pgfqpoint{1.771595in}{2.813802in}}%
\pgfpathlineto{\pgfqpoint{1.659881in}{2.833588in}}%
\pgfpathlineto{\pgfqpoint{1.764039in}{2.878563in}}%
\pgfpathlineto{\pgfqpoint{1.755764in}{2.855717in}}%
\pgfpathlineto{\pgfqpoint{2.573757in}{2.951166in}}%
\pgfpathlineto{\pgfqpoint{2.576276in}{2.929578in}}%
\pgfusepath{fill}%
\end{pgfscope}%
\begin{pgfscope}%
\pgfpathrectangle{\pgfqpoint{0.800000in}{1.363959in}}{\pgfqpoint{3.968000in}{2.024082in}} %
\pgfusepath{clip}%
\pgfsetbuttcap%
\pgfsetroundjoin%
\definecolor{currentfill}{rgb}{0.206756,0.371758,0.553117}%
\pgfsetfillcolor{currentfill}%
\pgfsetlinewidth{0.000000pt}%
\definecolor{currentstroke}{rgb}{0.000000,0.000000,0.000000}%
\pgfsetstrokecolor{currentstroke}%
\pgfsetdash{}{0pt}%
\pgfpathmoveto{\pgfqpoint{2.840470in}{2.169074in}}%
\pgfpathlineto{\pgfqpoint{3.794515in}{2.015728in}}%
\pgfpathlineto{\pgfqpoint{3.787235in}{2.038911in}}%
\pgfpathlineto{\pgfqpoint{3.889352in}{1.989478in}}%
\pgfpathlineto{\pgfqpoint{3.776888in}{1.974536in}}%
\pgfpathlineto{\pgfqpoint{3.791066in}{1.994270in}}%
\pgfpathlineto{\pgfqpoint{2.837021in}{2.147615in}}%
\pgfpathlineto{\pgfqpoint{2.840470in}{2.169074in}}%
\pgfusepath{fill}%
\end{pgfscope}%
\begin{pgfscope}%
\pgfpathrectangle{\pgfqpoint{0.800000in}{1.363959in}}{\pgfqpoint{3.968000in}{2.024082in}} %
\pgfusepath{clip}%
\pgfsetbuttcap%
\pgfsetroundjoin%
\definecolor{currentfill}{rgb}{0.119483,0.614817,0.537692}%
\pgfsetfillcolor{currentfill}%
\pgfsetlinewidth{0.000000pt}%
\definecolor{currentstroke}{rgb}{0.000000,0.000000,0.000000}%
\pgfsetstrokecolor{currentstroke}%
\pgfsetdash{}{0pt}%
\pgfpathmoveto{\pgfqpoint{2.840151in}{2.169120in}}%
\pgfpathlineto{\pgfqpoint{3.848589in}{2.037626in}}%
\pgfpathlineto{\pgfqpoint{3.840624in}{2.060582in}}%
\pgfpathlineto{\pgfqpoint{3.944165in}{2.014205in}}%
\pgfpathlineto{\pgfqpoint{3.832194in}{1.995929in}}%
\pgfpathlineto{\pgfqpoint{3.845779in}{2.016075in}}%
\pgfpathlineto{\pgfqpoint{2.837341in}{2.147569in}}%
\pgfpathlineto{\pgfqpoint{2.840151in}{2.169120in}}%
\pgfusepath{fill}%
\end{pgfscope}%
\begin{pgfscope}%
\pgfpathrectangle{\pgfqpoint{0.800000in}{1.363959in}}{\pgfqpoint{3.968000in}{2.024082in}} %
\pgfusepath{clip}%
\pgfsetbuttcap%
\pgfsetroundjoin%
\definecolor{currentfill}{rgb}{0.993248,0.906157,0.143936}%
\pgfsetfillcolor{currentfill}%
\pgfsetlinewidth{0.000000pt}%
\definecolor{currentstroke}{rgb}{0.000000,0.000000,0.000000}%
\pgfsetstrokecolor{currentstroke}%
\pgfsetdash{}{0pt}%
\pgfpathmoveto{\pgfqpoint{2.165487in}{2.817636in}}%
\pgfpathlineto{\pgfqpoint{1.423593in}{2.861735in}}%
\pgfpathlineto{\pgfqpoint{1.433151in}{2.839395in}}%
\pgfpathlineto{\pgfqpoint{1.326608in}{2.878386in}}%
\pgfpathlineto{\pgfqpoint{1.437019in}{2.904481in}}%
\pgfpathlineto{\pgfqpoint{1.424882in}{2.883430in}}%
\pgfpathlineto{\pgfqpoint{2.166776in}{2.839331in}}%
\pgfpathlineto{\pgfqpoint{2.165487in}{2.817636in}}%
\pgfusepath{fill}%
\end{pgfscope}%
\begin{pgfscope}%
\pgfpathrectangle{\pgfqpoint{0.800000in}{1.363959in}}{\pgfqpoint{3.968000in}{2.024082in}} %
\pgfusepath{clip}%
\pgfsetbuttcap%
\pgfsetroundjoin%
\definecolor{currentfill}{rgb}{0.993248,0.906157,0.143936}%
\pgfsetfillcolor{currentfill}%
\pgfsetlinewidth{0.000000pt}%
\definecolor{currentstroke}{rgb}{0.000000,0.000000,0.000000}%
\pgfsetstrokecolor{currentstroke}%
\pgfsetdash{}{0pt}%
\pgfpathmoveto{\pgfqpoint{3.290748in}{2.483283in}}%
\pgfpathlineto{\pgfqpoint{4.123603in}{2.445575in}}%
\pgfpathlineto{\pgfqpoint{4.113730in}{2.467778in}}%
\pgfpathlineto{\pgfqpoint{4.220813in}{2.430295in}}%
\pgfpathlineto{\pgfqpoint{4.110781in}{2.402643in}}%
\pgfpathlineto{\pgfqpoint{4.122620in}{2.423863in}}%
\pgfpathlineto{\pgfqpoint{3.289765in}{2.461572in}}%
\pgfpathlineto{\pgfqpoint{3.290748in}{2.483283in}}%
\pgfusepath{fill}%
\end{pgfscope}%
\begin{pgfscope}%
\pgfpathrectangle{\pgfqpoint{0.800000in}{1.363959in}}{\pgfqpoint{3.968000in}{2.024082in}} %
\pgfusepath{clip}%
\pgfsetbuttcap%
\pgfsetroundjoin%
\definecolor{currentfill}{rgb}{0.647257,0.858400,0.209861}%
\pgfsetfillcolor{currentfill}%
\pgfsetlinewidth{0.000000pt}%
\definecolor{currentstroke}{rgb}{0.000000,0.000000,0.000000}%
\pgfsetstrokecolor{currentstroke}%
\pgfsetdash{}{0pt}%
\pgfpathmoveto{\pgfqpoint{2.381990in}{1.535783in}}%
\pgfpathlineto{\pgfqpoint{1.606119in}{1.482538in}}%
\pgfpathlineto{\pgfqpoint{1.618448in}{1.461599in}}%
\pgfpathlineto{\pgfqpoint{1.507803in}{1.486683in}}%
\pgfpathlineto{\pgfqpoint{1.613984in}{1.526647in}}%
\pgfpathlineto{\pgfqpoint{1.604631in}{1.504221in}}%
\pgfpathlineto{\pgfqpoint{2.380502in}{1.557466in}}%
\pgfpathlineto{\pgfqpoint{2.381990in}{1.535783in}}%
\pgfusepath{fill}%
\end{pgfscope}%
\begin{pgfscope}%
\pgfpathrectangle{\pgfqpoint{0.800000in}{1.363959in}}{\pgfqpoint{3.968000in}{2.024082in}} %
\pgfusepath{clip}%
\pgfsetbuttcap%
\pgfsetroundjoin%
\definecolor{currentfill}{rgb}{0.993248,0.906157,0.143936}%
\pgfsetfillcolor{currentfill}%
\pgfsetlinewidth{0.000000pt}%
\definecolor{currentstroke}{rgb}{0.000000,0.000000,0.000000}%
\pgfsetstrokecolor{currentstroke}%
\pgfsetdash{}{0pt}%
\pgfpathmoveto{\pgfqpoint{3.087627in}{1.672063in}}%
\pgfpathlineto{\pgfqpoint{4.145524in}{1.709766in}}%
\pgfpathlineto{\pgfqpoint{4.133890in}{1.731098in}}%
\pgfpathlineto{\pgfqpoint{4.243651in}{1.702389in}}%
\pgfpathlineto{\pgfqpoint{4.136213in}{1.665939in}}%
\pgfpathlineto{\pgfqpoint{4.146299in}{1.688046in}}%
\pgfpathlineto{\pgfqpoint{3.088402in}{1.650343in}}%
\pgfpathlineto{\pgfqpoint{3.087627in}{1.672063in}}%
\pgfusepath{fill}%
\end{pgfscope}%
\begin{pgfscope}%
\pgfpathrectangle{\pgfqpoint{0.800000in}{1.363959in}}{\pgfqpoint{3.968000in}{2.024082in}} %
\pgfusepath{clip}%
\pgfsetbuttcap%
\pgfsetroundjoin%
\definecolor{currentfill}{rgb}{0.993248,0.906157,0.143936}%
\pgfsetfillcolor{currentfill}%
\pgfsetlinewidth{0.000000pt}%
\definecolor{currentstroke}{rgb}{0.000000,0.000000,0.000000}%
\pgfsetstrokecolor{currentstroke}%
\pgfsetdash{}{0pt}%
\pgfpathmoveto{\pgfqpoint{1.956179in}{1.845053in}}%
\pgfpathlineto{\pgfqpoint{1.231943in}{1.782449in}}%
\pgfpathlineto{\pgfqpoint{1.244642in}{1.761732in}}%
\pgfpathlineto{\pgfqpoint{1.133570in}{1.784852in}}%
\pgfpathlineto{\pgfqpoint{1.239026in}{1.826690in}}%
\pgfpathlineto{\pgfqpoint{1.230072in}{1.804102in}}%
\pgfpathlineto{\pgfqpoint{1.954307in}{1.866706in}}%
\pgfpathlineto{\pgfqpoint{1.956179in}{1.845053in}}%
\pgfusepath{fill}%
\end{pgfscope}%
\begin{pgfscope}%
\pgfpathrectangle{\pgfqpoint{0.800000in}{1.363959in}}{\pgfqpoint{3.968000in}{2.024082in}} %
\pgfusepath{clip}%
\pgfsetbuttcap%
\pgfsetroundjoin%
\definecolor{currentfill}{rgb}{0.993248,0.906157,0.143936}%
\pgfsetfillcolor{currentfill}%
\pgfsetlinewidth{0.000000pt}%
\definecolor{currentstroke}{rgb}{0.000000,0.000000,0.000000}%
\pgfsetstrokecolor{currentstroke}%
\pgfsetdash{}{0pt}%
\pgfpathmoveto{\pgfqpoint{2.113921in}{2.468058in}}%
\pgfpathlineto{\pgfqpoint{1.398020in}{2.433443in}}%
\pgfpathlineto{\pgfqpoint{1.409924in}{2.412259in}}%
\pgfpathlineto{\pgfqpoint{1.299808in}{2.439573in}}%
\pgfpathlineto{\pgfqpoint{1.406775in}{2.477384in}}%
\pgfpathlineto{\pgfqpoint{1.396971in}{2.455151in}}%
\pgfpathlineto{\pgfqpoint{2.112872in}{2.489767in}}%
\pgfpathlineto{\pgfqpoint{2.113921in}{2.468058in}}%
\pgfusepath{fill}%
\end{pgfscope}%
\begin{pgfscope}%
\pgfpathrectangle{\pgfqpoint{0.800000in}{1.363959in}}{\pgfqpoint{3.968000in}{2.024082in}} %
\pgfusepath{clip}%
\pgfsetbuttcap%
\pgfsetroundjoin%
\definecolor{currentfill}{rgb}{0.993248,0.906157,0.143936}%
\pgfsetfillcolor{currentfill}%
\pgfsetlinewidth{0.000000pt}%
\definecolor{currentstroke}{rgb}{0.000000,0.000000,0.000000}%
\pgfsetstrokecolor{currentstroke}%
\pgfsetdash{}{0pt}%
\pgfpathmoveto{\pgfqpoint{2.583610in}{3.228380in}}%
\pgfpathlineto{\pgfqpoint{1.852333in}{3.126465in}}%
\pgfpathlineto{\pgfqpoint{1.866096in}{3.106439in}}%
\pgfpathlineto{\pgfqpoint{1.753968in}{3.123728in}}%
\pgfpathlineto{\pgfqpoint{1.857096in}{3.171016in}}%
\pgfpathlineto{\pgfqpoint{1.849333in}{3.147990in}}%
\pgfpathlineto{\pgfqpoint{2.580610in}{3.249906in}}%
\pgfpathlineto{\pgfqpoint{2.583610in}{3.228380in}}%
\pgfusepath{fill}%
\end{pgfscope}%
\begin{pgfscope}%
\pgfpathrectangle{\pgfqpoint{0.800000in}{1.363959in}}{\pgfqpoint{3.968000in}{2.024082in}} %
\pgfusepath{clip}%
\pgfsetbuttcap%
\pgfsetroundjoin%
\definecolor{currentfill}{rgb}{0.964894,0.902323,0.123941}%
\pgfsetfillcolor{currentfill}%
\pgfsetlinewidth{0.000000pt}%
\definecolor{currentstroke}{rgb}{0.000000,0.000000,0.000000}%
\pgfsetstrokecolor{currentstroke}%
\pgfsetdash{}{0pt}%
\pgfpathmoveto{\pgfqpoint{3.584319in}{2.505377in}}%
\pgfpathlineto{\pgfqpoint{4.480321in}{2.427905in}}%
\pgfpathlineto{\pgfqpoint{4.471366in}{2.450494in}}%
\pgfpathlineto{\pgfqpoint{4.576822in}{2.408654in}}%
\pgfpathlineto{\pgfqpoint{4.465750in}{2.385535in}}%
\pgfpathlineto{\pgfqpoint{4.478448in}{2.406252in}}%
\pgfpathlineto{\pgfqpoint{3.582447in}{2.483724in}}%
\pgfpathlineto{\pgfqpoint{3.584319in}{2.505377in}}%
\pgfusepath{fill}%
\end{pgfscope}%
\begin{pgfscope}%
\pgfpathrectangle{\pgfqpoint{0.800000in}{1.363959in}}{\pgfqpoint{3.968000in}{2.024082in}} %
\pgfusepath{clip}%
\pgfsetbuttcap%
\pgfsetroundjoin%
\definecolor{currentfill}{rgb}{0.214000,0.722114,0.469588}%
\pgfsetfillcolor{currentfill}%
\pgfsetlinewidth{0.000000pt}%
\definecolor{currentstroke}{rgb}{0.000000,0.000000,0.000000}%
\pgfsetstrokecolor{currentstroke}%
\pgfsetdash{}{0pt}%
\pgfpathmoveto{\pgfqpoint{2.791422in}{1.824929in}}%
\pgfpathlineto{\pgfqpoint{3.726426in}{1.800178in}}%
\pgfpathlineto{\pgfqpoint{3.716138in}{1.822192in}}%
\pgfpathlineto{\pgfqpoint{3.823905in}{1.786727in}}%
\pgfpathlineto{\pgfqpoint{3.714413in}{1.757014in}}%
\pgfpathlineto{\pgfqpoint{3.725851in}{1.778452in}}%
\pgfpathlineto{\pgfqpoint{2.790847in}{1.803203in}}%
\pgfpathlineto{\pgfqpoint{2.791422in}{1.824929in}}%
\pgfusepath{fill}%
\end{pgfscope}%
\begin{pgfscope}%
\pgfpathrectangle{\pgfqpoint{0.800000in}{1.363959in}}{\pgfqpoint{3.968000in}{2.024082in}} %
\pgfusepath{clip}%
\pgfsetbuttcap%
\pgfsetroundjoin%
\definecolor{currentfill}{rgb}{0.263663,0.237631,0.518762}%
\pgfsetfillcolor{currentfill}%
\pgfsetlinewidth{0.000000pt}%
\definecolor{currentstroke}{rgb}{0.000000,0.000000,0.000000}%
\pgfsetstrokecolor{currentstroke}%
\pgfsetdash{}{0pt}%
\pgfpathmoveto{\pgfqpoint{2.791775in}{1.824914in}}%
\pgfpathlineto{\pgfqpoint{3.636647in}{1.774980in}}%
\pgfpathlineto{\pgfqpoint{3.627082in}{1.797317in}}%
\pgfpathlineto{\pgfqpoint{3.733637in}{1.758362in}}%
\pgfpathlineto{\pgfqpoint{3.623235in}{1.732230in}}%
\pgfpathlineto{\pgfqpoint{3.635365in}{1.753284in}}%
\pgfpathlineto{\pgfqpoint{2.790493in}{1.803218in}}%
\pgfpathlineto{\pgfqpoint{2.791775in}{1.824914in}}%
\pgfusepath{fill}%
\end{pgfscope}%
\begin{pgfscope}%
\pgfpathrectangle{\pgfqpoint{0.800000in}{1.363959in}}{\pgfqpoint{3.968000in}{2.024082in}} %
\pgfusepath{clip}%
\pgfsetbuttcap%
\pgfsetroundjoin%
\definecolor{currentfill}{rgb}{0.276022,0.044167,0.370164}%
\pgfsetfillcolor{currentfill}%
\pgfsetlinewidth{0.000000pt}%
\definecolor{currentstroke}{rgb}{0.000000,0.000000,0.000000}%
\pgfsetstrokecolor{currentstroke}%
\pgfsetdash{}{0pt}%
\pgfpathmoveto{\pgfqpoint{3.086921in}{1.916392in}}%
\pgfpathlineto{\pgfqpoint{4.127526in}{1.941536in}}%
\pgfpathlineto{\pgfqpoint{4.116138in}{1.963000in}}%
\pgfpathlineto{\pgfqpoint{4.225562in}{1.933035in}}%
\pgfpathlineto{\pgfqpoint{4.117713in}{1.897819in}}%
\pgfpathlineto{\pgfqpoint{4.128051in}{1.919808in}}%
\pgfpathlineto{\pgfqpoint{3.087446in}{1.894664in}}%
\pgfpathlineto{\pgfqpoint{3.086921in}{1.916392in}}%
\pgfusepath{fill}%
\end{pgfscope}%
\begin{pgfscope}%
\pgfpathrectangle{\pgfqpoint{0.800000in}{1.363959in}}{\pgfqpoint{3.968000in}{2.024082in}} %
\pgfusepath{clip}%
\pgfsetbuttcap%
\pgfsetroundjoin%
\definecolor{currentfill}{rgb}{0.535621,0.835785,0.281908}%
\pgfsetfillcolor{currentfill}%
\pgfsetlinewidth{0.000000pt}%
\definecolor{currentstroke}{rgb}{0.000000,0.000000,0.000000}%
\pgfsetstrokecolor{currentstroke}%
\pgfsetdash{}{0pt}%
\pgfpathmoveto{\pgfqpoint{3.087225in}{1.916395in}}%
\pgfpathlineto{\pgfqpoint{4.207169in}{1.912149in}}%
\pgfpathlineto{\pgfqpoint{4.196385in}{1.933923in}}%
\pgfpathlineto{\pgfqpoint{4.304928in}{1.900911in}}%
\pgfpathlineto{\pgfqpoint{4.196137in}{1.868723in}}%
\pgfpathlineto{\pgfqpoint{4.207087in}{1.890415in}}%
\pgfpathlineto{\pgfqpoint{3.087142in}{1.894661in}}%
\pgfpathlineto{\pgfqpoint{3.087225in}{1.916395in}}%
\pgfusepath{fill}%
\end{pgfscope}%
\begin{pgfscope}%
\pgfpathrectangle{\pgfqpoint{0.800000in}{1.363959in}}{\pgfqpoint{3.968000in}{2.024082in}} %
\pgfusepath{clip}%
\pgfsetbuttcap%
\pgfsetroundjoin%
\definecolor{currentfill}{rgb}{0.192357,0.403199,0.555836}%
\pgfsetfillcolor{currentfill}%
\pgfsetlinewidth{0.000000pt}%
\definecolor{currentstroke}{rgb}{0.000000,0.000000,0.000000}%
\pgfsetstrokecolor{currentstroke}%
\pgfsetdash{}{0pt}%
\pgfpathmoveto{\pgfqpoint{2.383959in}{2.811534in}}%
\pgfpathlineto{\pgfqpoint{1.551997in}{2.903023in}}%
\pgfpathlineto{\pgfqpoint{1.560423in}{2.880231in}}%
\pgfpathlineto{\pgfqpoint{1.455969in}{2.924515in}}%
\pgfpathlineto{\pgfqpoint{1.567550in}{2.945042in}}%
\pgfpathlineto{\pgfqpoint{1.554373in}{2.924626in}}%
\pgfpathlineto{\pgfqpoint{2.386335in}{2.833137in}}%
\pgfpathlineto{\pgfqpoint{2.383959in}{2.811534in}}%
\pgfusepath{fill}%
\end{pgfscope}%
\begin{pgfscope}%
\pgfpathrectangle{\pgfqpoint{0.800000in}{1.363959in}}{\pgfqpoint{3.968000in}{2.024082in}} %
\pgfusepath{clip}%
\pgfsetbuttcap%
\pgfsetroundjoin%
\definecolor{currentfill}{rgb}{0.122606,0.585371,0.546557}%
\pgfsetfillcolor{currentfill}%
\pgfsetlinewidth{0.000000pt}%
\definecolor{currentstroke}{rgb}{0.000000,0.000000,0.000000}%
\pgfsetstrokecolor{currentstroke}%
\pgfsetdash{}{0pt}%
\pgfpathmoveto{\pgfqpoint{2.384633in}{2.811481in}}%
\pgfpathlineto{\pgfqpoint{1.600349in}{2.848612in}}%
\pgfpathlineto{\pgfqpoint{1.610176in}{2.826389in}}%
\pgfpathlineto{\pgfqpoint{1.503171in}{2.864092in}}%
\pgfpathlineto{\pgfqpoint{1.613260in}{2.891517in}}%
\pgfpathlineto{\pgfqpoint{1.601377in}{2.870321in}}%
\pgfpathlineto{\pgfqpoint{2.385661in}{2.833190in}}%
\pgfpathlineto{\pgfqpoint{2.384633in}{2.811481in}}%
\pgfusepath{fill}%
\end{pgfscope}%
\begin{pgfscope}%
\pgfpathrectangle{\pgfqpoint{0.800000in}{1.363959in}}{\pgfqpoint{3.968000in}{2.024082in}} %
\pgfusepath{clip}%
\pgfsetbuttcap%
\pgfsetroundjoin%
\definecolor{currentfill}{rgb}{0.845561,0.887322,0.099702}%
\pgfsetfillcolor{currentfill}%
\pgfsetlinewidth{0.000000pt}%
\definecolor{currentstroke}{rgb}{0.000000,0.000000,0.000000}%
\pgfsetstrokecolor{currentstroke}%
\pgfsetdash{}{0pt}%
\pgfpathmoveto{\pgfqpoint{3.077979in}{1.810233in}}%
\pgfpathlineto{\pgfqpoint{4.174092in}{1.726922in}}%
\pgfpathlineto{\pgfqpoint{4.164903in}{1.749416in}}%
\pgfpathlineto{\pgfqpoint{4.270788in}{1.708674in}}%
\pgfpathlineto{\pgfqpoint{4.159962in}{1.684403in}}%
\pgfpathlineto{\pgfqpoint{4.172444in}{1.705250in}}%
\pgfpathlineto{\pgfqpoint{3.076332in}{1.788562in}}%
\pgfpathlineto{\pgfqpoint{3.077979in}{1.810233in}}%
\pgfusepath{fill}%
\end{pgfscope}%
\begin{pgfscope}%
\pgfpathrectangle{\pgfqpoint{0.800000in}{1.363959in}}{\pgfqpoint{3.968000in}{2.024082in}} %
\pgfusepath{clip}%
\pgfsetbuttcap%
\pgfsetroundjoin%
\definecolor{currentfill}{rgb}{0.296479,0.761561,0.424223}%
\pgfsetfillcolor{currentfill}%
\pgfsetlinewidth{0.000000pt}%
\definecolor{currentstroke}{rgb}{0.000000,0.000000,0.000000}%
\pgfsetstrokecolor{currentstroke}%
\pgfsetdash{}{0pt}%
\pgfpathmoveto{\pgfqpoint{3.353451in}{2.530149in}}%
\pgfpathlineto{\pgfqpoint{4.232499in}{2.530515in}}%
\pgfpathlineto{\pgfqpoint{4.221623in}{2.552244in}}%
\pgfpathlineto{\pgfqpoint{4.330305in}{2.519689in}}%
\pgfpathlineto{\pgfqpoint{4.221650in}{2.487044in}}%
\pgfpathlineto{\pgfqpoint{4.232508in}{2.508782in}}%
\pgfpathlineto{\pgfqpoint{3.353461in}{2.508415in}}%
\pgfpathlineto{\pgfqpoint{3.353451in}{2.530149in}}%
\pgfusepath{fill}%
\end{pgfscope}%
\begin{pgfscope}%
\pgfpathrectangle{\pgfqpoint{0.800000in}{1.363959in}}{\pgfqpoint{3.968000in}{2.024082in}} %
\pgfusepath{clip}%
\pgfsetbuttcap%
\pgfsetroundjoin%
\definecolor{currentfill}{rgb}{0.277134,0.185228,0.489898}%
\pgfsetfillcolor{currentfill}%
\pgfsetlinewidth{0.000000pt}%
\definecolor{currentstroke}{rgb}{0.000000,0.000000,0.000000}%
\pgfsetstrokecolor{currentstroke}%
\pgfsetdash{}{0pt}%
\pgfpathmoveto{\pgfqpoint{3.354401in}{2.530107in}}%
\pgfpathlineto{\pgfqpoint{4.120640in}{2.463238in}}%
\pgfpathlineto{\pgfqpoint{4.111704in}{2.485834in}}%
\pgfpathlineto{\pgfqpoint{4.217126in}{2.443909in}}%
\pgfpathlineto{\pgfqpoint{4.106035in}{2.420880in}}%
\pgfpathlineto{\pgfqpoint{4.118750in}{2.441587in}}%
\pgfpathlineto{\pgfqpoint{3.352511in}{2.508456in}}%
\pgfpathlineto{\pgfqpoint{3.354401in}{2.530107in}}%
\pgfusepath{fill}%
\end{pgfscope}%
\begin{pgfscope}%
\pgfpathrectangle{\pgfqpoint{0.800000in}{1.363959in}}{\pgfqpoint{3.968000in}{2.024082in}} %
\pgfusepath{clip}%
\pgfsetbuttcap%
\pgfsetroundjoin%
\definecolor{currentfill}{rgb}{0.647257,0.858400,0.209861}%
\pgfsetfillcolor{currentfill}%
\pgfsetlinewidth{0.000000pt}%
\definecolor{currentstroke}{rgb}{0.000000,0.000000,0.000000}%
\pgfsetstrokecolor{currentstroke}%
\pgfsetdash{}{0pt}%
\pgfpathmoveto{\pgfqpoint{3.194566in}{3.025258in}}%
\pgfpathlineto{\pgfqpoint{3.989921in}{3.121439in}}%
\pgfpathlineto{\pgfqpoint{3.976523in}{3.141711in}}%
\pgfpathlineto{\pgfqpoint{4.088319in}{3.122392in}}%
\pgfpathlineto{\pgfqpoint{3.984351in}{3.076982in}}%
\pgfpathlineto{\pgfqpoint{3.992530in}{3.099863in}}%
\pgfpathlineto{\pgfqpoint{3.197175in}{3.003681in}}%
\pgfpathlineto{\pgfqpoint{3.194566in}{3.025258in}}%
\pgfusepath{fill}%
\end{pgfscope}%
\begin{pgfscope}%
\pgfpathrectangle{\pgfqpoint{0.800000in}{1.363959in}}{\pgfqpoint{3.968000in}{2.024082in}} %
\pgfusepath{clip}%
\pgfsetbuttcap%
\pgfsetroundjoin%
\definecolor{currentfill}{rgb}{0.993248,0.906157,0.143936}%
\pgfsetfillcolor{currentfill}%
\pgfsetlinewidth{0.000000pt}%
\definecolor{currentstroke}{rgb}{0.000000,0.000000,0.000000}%
\pgfsetstrokecolor{currentstroke}%
\pgfsetdash{}{0pt}%
\pgfpathmoveto{\pgfqpoint{3.193297in}{2.366709in}}%
\pgfpathlineto{\pgfqpoint{4.032139in}{2.280017in}}%
\pgfpathlineto{\pgfqpoint{4.023564in}{2.302752in}}%
\pgfpathlineto{\pgfqpoint{4.128305in}{2.259154in}}%
\pgfpathlineto{\pgfqpoint{4.016861in}{2.237897in}}%
\pgfpathlineto{\pgfqpoint{4.029904in}{2.258398in}}%
\pgfpathlineto{\pgfqpoint{3.191063in}{2.345090in}}%
\pgfpathlineto{\pgfqpoint{3.193297in}{2.366709in}}%
\pgfusepath{fill}%
\end{pgfscope}%
\begin{pgfscope}%
\pgfpathrectangle{\pgfqpoint{0.800000in}{1.363959in}}{\pgfqpoint{3.968000in}{2.024082in}} %
\pgfusepath{clip}%
\pgfsetbuttcap%
\pgfsetroundjoin%
\definecolor{currentfill}{rgb}{0.993248,0.906157,0.143936}%
\pgfsetfillcolor{currentfill}%
\pgfsetlinewidth{0.000000pt}%
\definecolor{currentstroke}{rgb}{0.000000,0.000000,0.000000}%
\pgfsetstrokecolor{currentstroke}%
\pgfsetdash{}{0pt}%
\pgfpathmoveto{\pgfqpoint{2.316882in}{2.580789in}}%
\pgfpathlineto{\pgfqpoint{1.537976in}{2.792052in}}%
\pgfpathlineto{\pgfqpoint{1.542775in}{2.768232in}}%
\pgfpathlineto{\pgfqpoint{1.446430in}{2.828142in}}%
\pgfpathlineto{\pgfqpoint{1.559843in}{2.831159in}}%
\pgfpathlineto{\pgfqpoint{1.543666in}{2.813028in}}%
\pgfpathlineto{\pgfqpoint{2.322572in}{2.601765in}}%
\pgfpathlineto{\pgfqpoint{2.316882in}{2.580789in}}%
\pgfusepath{fill}%
\end{pgfscope}%
\begin{pgfscope}%
\pgfpathrectangle{\pgfqpoint{0.800000in}{1.363959in}}{\pgfqpoint{3.968000in}{2.024082in}} %
\pgfusepath{clip}%
\pgfsetbuttcap%
\pgfsetroundjoin%
\definecolor{currentfill}{rgb}{0.993248,0.906157,0.143936}%
\pgfsetfillcolor{currentfill}%
\pgfsetlinewidth{0.000000pt}%
\definecolor{currentstroke}{rgb}{0.000000,0.000000,0.000000}%
\pgfsetstrokecolor{currentstroke}%
\pgfsetdash{}{0pt}%
\pgfpathmoveto{\pgfqpoint{3.271215in}{2.991788in}}%
\pgfpathlineto{\pgfqpoint{4.122014in}{2.803784in}}%
\pgfpathlineto{\pgfqpoint{4.116093in}{2.827350in}}%
\pgfpathlineto{\pgfqpoint{4.215167in}{2.772070in}}%
\pgfpathlineto{\pgfqpoint{4.102025in}{2.763685in}}%
\pgfpathlineto{\pgfqpoint{4.117325in}{2.782562in}}%
\pgfpathlineto{\pgfqpoint{3.266526in}{2.970566in}}%
\pgfpathlineto{\pgfqpoint{3.271215in}{2.991788in}}%
\pgfusepath{fill}%
\end{pgfscope}%
\begin{pgfscope}%
\pgfpathrectangle{\pgfqpoint{0.800000in}{1.363959in}}{\pgfqpoint{3.968000in}{2.024082in}} %
\pgfusepath{clip}%
\pgfsetbuttcap%
\pgfsetroundjoin%
\definecolor{currentfill}{rgb}{0.192357,0.403199,0.555836}%
\pgfsetfillcolor{currentfill}%
\pgfsetlinewidth{0.000000pt}%
\definecolor{currentstroke}{rgb}{0.000000,0.000000,0.000000}%
\pgfsetstrokecolor{currentstroke}%
\pgfsetdash{}{0pt}%
\pgfpathmoveto{\pgfqpoint{2.630375in}{1.480242in}}%
\pgfpathlineto{\pgfqpoint{3.599098in}{1.633650in}}%
\pgfpathlineto{\pgfqpoint{3.584966in}{1.653416in}}%
\pgfpathlineto{\pgfqpoint{3.697396in}{1.638214in}}%
\pgfpathlineto{\pgfqpoint{3.595164in}{1.589018in}}%
\pgfpathlineto{\pgfqpoint{3.602498in}{1.612184in}}%
\pgfpathlineto{\pgfqpoint{2.633774in}{1.458776in}}%
\pgfpathlineto{\pgfqpoint{2.630375in}{1.480242in}}%
\pgfusepath{fill}%
\end{pgfscope}%
\begin{pgfscope}%
\pgfpathrectangle{\pgfqpoint{0.800000in}{1.363959in}}{\pgfqpoint{3.968000in}{2.024082in}} %
\pgfusepath{clip}%
\pgfsetbuttcap%
\pgfsetroundjoin%
\definecolor{currentfill}{rgb}{0.122606,0.585371,0.546557}%
\pgfsetfillcolor{currentfill}%
\pgfsetlinewidth{0.000000pt}%
\definecolor{currentstroke}{rgb}{0.000000,0.000000,0.000000}%
\pgfsetstrokecolor{currentstroke}%
\pgfsetdash{}{0pt}%
\pgfpathmoveto{\pgfqpoint{2.631236in}{1.480343in}}%
\pgfpathlineto{\pgfqpoint{3.693214in}{1.562604in}}%
\pgfpathlineto{\pgfqpoint{3.680701in}{1.583433in}}%
\pgfpathlineto{\pgfqpoint{3.791563in}{1.559322in}}%
\pgfpathlineto{\pgfqpoint{3.685737in}{1.518427in}}%
\pgfpathlineto{\pgfqpoint{3.694893in}{1.540935in}}%
\pgfpathlineto{\pgfqpoint{2.632914in}{1.458675in}}%
\pgfpathlineto{\pgfqpoint{2.631236in}{1.480343in}}%
\pgfusepath{fill}%
\end{pgfscope}%
\begin{pgfscope}%
\pgfpathrectangle{\pgfqpoint{0.800000in}{1.363959in}}{\pgfqpoint{3.968000in}{2.024082in}} %
\pgfusepath{clip}%
\pgfsetbuttcap%
\pgfsetroundjoin%
\definecolor{currentfill}{rgb}{0.647257,0.858400,0.209861}%
\pgfsetfillcolor{currentfill}%
\pgfsetlinewidth{0.000000pt}%
\definecolor{currentstroke}{rgb}{0.000000,0.000000,0.000000}%
\pgfsetstrokecolor{currentstroke}%
\pgfsetdash{}{0pt}%
\pgfpathmoveto{\pgfqpoint{2.343665in}{2.331202in}}%
\pgfpathlineto{\pgfqpoint{1.727251in}{2.366903in}}%
\pgfpathlineto{\pgfqpoint{1.736843in}{2.344577in}}%
\pgfpathlineto{\pgfqpoint{1.630242in}{2.383407in}}%
\pgfpathlineto{\pgfqpoint{1.740613in}{2.409669in}}%
\pgfpathlineto{\pgfqpoint{1.728508in}{2.388600in}}%
\pgfpathlineto{\pgfqpoint{2.344922in}{2.352899in}}%
\pgfpathlineto{\pgfqpoint{2.343665in}{2.331202in}}%
\pgfusepath{fill}%
\end{pgfscope}%
\begin{pgfscope}%
\pgfpathrectangle{\pgfqpoint{0.800000in}{1.363959in}}{\pgfqpoint{3.968000in}{2.024082in}} %
\pgfusepath{clip}%
\pgfsetbuttcap%
\pgfsetroundjoin%
\definecolor{currentfill}{rgb}{0.804182,0.882046,0.114965}%
\pgfsetfillcolor{currentfill}%
\pgfsetlinewidth{0.000000pt}%
\definecolor{currentstroke}{rgb}{0.000000,0.000000,0.000000}%
\pgfsetstrokecolor{currentstroke}%
\pgfsetdash{}{0pt}%
\pgfpathmoveto{\pgfqpoint{2.331597in}{2.555144in}}%
\pgfpathlineto{\pgfqpoint{1.491274in}{2.696725in}}%
\pgfpathlineto{\pgfqpoint{1.498379in}{2.673488in}}%
\pgfpathlineto{\pgfqpoint{1.396638in}{2.723690in}}%
\pgfpathlineto{\pgfqpoint{1.509212in}{2.737783in}}%
\pgfpathlineto{\pgfqpoint{1.494885in}{2.718157in}}%
\pgfpathlineto{\pgfqpoint{2.335208in}{2.576576in}}%
\pgfpathlineto{\pgfqpoint{2.331597in}{2.555144in}}%
\pgfusepath{fill}%
\end{pgfscope}%
\begin{pgfscope}%
\pgfpathrectangle{\pgfqpoint{0.800000in}{1.363959in}}{\pgfqpoint{3.968000in}{2.024082in}} %
\pgfusepath{clip}%
\pgfsetbuttcap%
\pgfsetroundjoin%
\definecolor{currentfill}{rgb}{0.993248,0.906157,0.143936}%
\pgfsetfillcolor{currentfill}%
\pgfsetlinewidth{0.000000pt}%
\definecolor{currentstroke}{rgb}{0.000000,0.000000,0.000000}%
\pgfsetstrokecolor{currentstroke}%
\pgfsetdash{}{0pt}%
\pgfpathmoveto{\pgfqpoint{2.037768in}{2.485173in}}%
\pgfpathlineto{\pgfqpoint{1.298330in}{2.580229in}}%
\pgfpathlineto{\pgfqpoint{1.306337in}{2.557288in}}%
\pgfpathlineto{\pgfqpoint{1.202712in}{2.603477in}}%
\pgfpathlineto{\pgfqpoint{1.314650in}{2.621956in}}%
\pgfpathlineto{\pgfqpoint{1.301101in}{2.601786in}}%
\pgfpathlineto{\pgfqpoint{2.040539in}{2.506729in}}%
\pgfpathlineto{\pgfqpoint{2.037768in}{2.485173in}}%
\pgfusepath{fill}%
\end{pgfscope}%
\begin{pgfscope}%
\pgfpathrectangle{\pgfqpoint{0.800000in}{1.363959in}}{\pgfqpoint{3.968000in}{2.024082in}} %
\pgfusepath{clip}%
\pgfsetbuttcap%
\pgfsetroundjoin%
\definecolor{currentfill}{rgb}{0.845561,0.887322,0.099702}%
\pgfsetfillcolor{currentfill}%
\pgfsetlinewidth{0.000000pt}%
\definecolor{currentstroke}{rgb}{0.000000,0.000000,0.000000}%
\pgfsetstrokecolor{currentstroke}%
\pgfsetdash{}{0pt}%
\pgfpathmoveto{\pgfqpoint{2.188301in}{2.505845in}}%
\pgfpathlineto{\pgfqpoint{1.458747in}{2.720110in}}%
\pgfpathlineto{\pgfqpoint{1.463049in}{2.696195in}}%
\pgfpathlineto{\pgfqpoint{1.367971in}{2.758096in}}%
\pgfpathlineto{\pgfqpoint{1.481422in}{2.758754in}}%
\pgfpathlineto{\pgfqpoint{1.464872in}{2.740963in}}%
\pgfpathlineto{\pgfqpoint{2.194425in}{2.526698in}}%
\pgfpathlineto{\pgfqpoint{2.188301in}{2.505845in}}%
\pgfusepath{fill}%
\end{pgfscope}%
\begin{pgfscope}%
\pgfpathrectangle{\pgfqpoint{0.800000in}{1.363959in}}{\pgfqpoint{3.968000in}{2.024082in}} %
\pgfusepath{clip}%
\pgfsetbuttcap%
\pgfsetroundjoin%
\definecolor{currentfill}{rgb}{0.257322,0.256130,0.526563}%
\pgfsetfillcolor{currentfill}%
\pgfsetlinewidth{0.000000pt}%
\definecolor{currentstroke}{rgb}{0.000000,0.000000,0.000000}%
\pgfsetstrokecolor{currentstroke}%
\pgfsetdash{}{0pt}%
\pgfpathmoveto{\pgfqpoint{2.293837in}{2.133246in}}%
\pgfpathlineto{\pgfqpoint{1.550175in}{2.001305in}}%
\pgfpathlineto{\pgfqpoint{1.564671in}{1.981804in}}%
\pgfpathlineto{\pgfqpoint{1.451979in}{1.994919in}}%
\pgfpathlineto{\pgfqpoint{1.553281in}{2.046002in}}%
\pgfpathlineto{\pgfqpoint{1.546378in}{2.022704in}}%
\pgfpathlineto{\pgfqpoint{2.290040in}{2.154645in}}%
\pgfpathlineto{\pgfqpoint{2.293837in}{2.133246in}}%
\pgfusepath{fill}%
\end{pgfscope}%
\begin{pgfscope}%
\pgfpathrectangle{\pgfqpoint{0.800000in}{1.363959in}}{\pgfqpoint{3.968000in}{2.024082in}} %
\pgfusepath{clip}%
\pgfsetbuttcap%
\pgfsetroundjoin%
\definecolor{currentfill}{rgb}{0.267968,0.223549,0.512008}%
\pgfsetfillcolor{currentfill}%
\pgfsetlinewidth{0.000000pt}%
\definecolor{currentstroke}{rgb}{0.000000,0.000000,0.000000}%
\pgfsetstrokecolor{currentstroke}%
\pgfsetdash{}{0pt}%
\pgfpathmoveto{\pgfqpoint{2.291033in}{2.133116in}}%
\pgfpathlineto{\pgfqpoint{1.892387in}{2.166468in}}%
\pgfpathlineto{\pgfqpoint{1.901404in}{2.143904in}}%
\pgfpathlineto{\pgfqpoint{1.795832in}{2.185451in}}%
\pgfpathlineto{\pgfqpoint{1.906840in}{2.208878in}}%
\pgfpathlineto{\pgfqpoint{1.894199in}{2.188126in}}%
\pgfpathlineto{\pgfqpoint{2.292845in}{2.154774in}}%
\pgfpathlineto{\pgfqpoint{2.291033in}{2.133116in}}%
\pgfusepath{fill}%
\end{pgfscope}%
\begin{pgfscope}%
\pgfpathrectangle{\pgfqpoint{0.800000in}{1.363959in}}{\pgfqpoint{3.968000in}{2.024082in}} %
\pgfusepath{clip}%
\pgfsetbuttcap%
\pgfsetroundjoin%
\definecolor{currentfill}{rgb}{0.250425,0.274290,0.533103}%
\pgfsetfillcolor{currentfill}%
\pgfsetlinewidth{0.000000pt}%
\definecolor{currentstroke}{rgb}{0.000000,0.000000,0.000000}%
\pgfsetstrokecolor{currentstroke}%
\pgfsetdash{}{0pt}%
\pgfpathmoveto{\pgfqpoint{2.290797in}{2.133139in}}%
\pgfpathlineto{\pgfqpoint{1.444485in}{2.222565in}}%
\pgfpathlineto{\pgfqpoint{1.453008in}{2.199810in}}%
\pgfpathlineto{\pgfqpoint{1.348367in}{2.243649in}}%
\pgfpathlineto{\pgfqpoint{1.459859in}{2.264650in}}%
\pgfpathlineto{\pgfqpoint{1.446769in}{2.244179in}}%
\pgfpathlineto{\pgfqpoint{2.293080in}{2.154752in}}%
\pgfpathlineto{\pgfqpoint{2.290797in}{2.133139in}}%
\pgfusepath{fill}%
\end{pgfscope}%
\begin{pgfscope}%
\pgfpathrectangle{\pgfqpoint{0.800000in}{1.363959in}}{\pgfqpoint{3.968000in}{2.024082in}} %
\pgfusepath{clip}%
\pgfsetbuttcap%
\pgfsetroundjoin%
\definecolor{currentfill}{rgb}{0.993248,0.906157,0.143936}%
\pgfsetfillcolor{currentfill}%
\pgfsetlinewidth{0.000000pt}%
\definecolor{currentstroke}{rgb}{0.000000,0.000000,0.000000}%
\pgfsetstrokecolor{currentstroke}%
\pgfsetdash{}{0pt}%
\pgfpathmoveto{\pgfqpoint{2.716284in}{2.473074in}}%
\pgfpathlineto{\pgfqpoint{3.742570in}{2.206489in}}%
\pgfpathlineto{\pgfqpoint{3.737516in}{2.230257in}}%
\pgfpathlineto{\pgfqpoint{3.834498in}{2.171383in}}%
\pgfpathlineto{\pgfqpoint{3.721124in}{2.167150in}}%
\pgfpathlineto{\pgfqpoint{3.737106in}{2.185454in}}%
\pgfpathlineto{\pgfqpoint{2.710819in}{2.452038in}}%
\pgfpathlineto{\pgfqpoint{2.716284in}{2.473074in}}%
\pgfusepath{fill}%
\end{pgfscope}%
\begin{pgfscope}%
\pgfpathrectangle{\pgfqpoint{0.800000in}{1.363959in}}{\pgfqpoint{3.968000in}{2.024082in}} %
\pgfusepath{clip}%
\pgfsetbuttcap%
\pgfsetroundjoin%
\definecolor{currentfill}{rgb}{0.993248,0.906157,0.143936}%
\pgfsetfillcolor{currentfill}%
\pgfsetlinewidth{0.000000pt}%
\definecolor{currentstroke}{rgb}{0.000000,0.000000,0.000000}%
\pgfsetstrokecolor{currentstroke}%
\pgfsetdash{}{0pt}%
\pgfpathmoveto{\pgfqpoint{2.867064in}{2.694052in}}%
\pgfpathlineto{\pgfqpoint{3.802663in}{2.733649in}}%
\pgfpathlineto{\pgfqpoint{3.790887in}{2.754904in}}%
\pgfpathlineto{\pgfqpoint{3.900837in}{2.726928in}}%
\pgfpathlineto{\pgfqpoint{3.793644in}{2.689761in}}%
\pgfpathlineto{\pgfqpoint{3.803582in}{2.711935in}}%
\pgfpathlineto{\pgfqpoint{2.867983in}{2.672338in}}%
\pgfpathlineto{\pgfqpoint{2.867064in}{2.694052in}}%
\pgfusepath{fill}%
\end{pgfscope}%
\begin{pgfscope}%
\pgfpathrectangle{\pgfqpoint{0.800000in}{1.363959in}}{\pgfqpoint{3.968000in}{2.024082in}} %
\pgfusepath{clip}%
\pgfsetbuttcap%
\pgfsetroundjoin%
\definecolor{currentfill}{rgb}{0.280255,0.165693,0.476498}%
\pgfsetfillcolor{currentfill}%
\pgfsetlinewidth{0.000000pt}%
\definecolor{currentstroke}{rgb}{0.000000,0.000000,0.000000}%
\pgfsetstrokecolor{currentstroke}%
\pgfsetdash{}{0pt}%
\pgfpathmoveto{\pgfqpoint{2.069153in}{1.918214in}}%
\pgfpathlineto{\pgfqpoint{1.548078in}{1.973749in}}%
\pgfpathlineto{\pgfqpoint{1.556580in}{1.950986in}}%
\pgfpathlineto{\pgfqpoint{1.451979in}{1.994919in}}%
\pgfpathlineto{\pgfqpoint{1.563490in}{2.015820in}}%
\pgfpathlineto{\pgfqpoint{1.550381in}{1.995360in}}%
\pgfpathlineto{\pgfqpoint{2.071456in}{1.939825in}}%
\pgfpathlineto{\pgfqpoint{2.069153in}{1.918214in}}%
\pgfusepath{fill}%
\end{pgfscope}%
\begin{pgfscope}%
\pgfpathrectangle{\pgfqpoint{0.800000in}{1.363959in}}{\pgfqpoint{3.968000in}{2.024082in}} %
\pgfusepath{clip}%
\pgfsetbuttcap%
\pgfsetroundjoin%
\definecolor{currentfill}{rgb}{0.327796,0.773980,0.406640}%
\pgfsetfillcolor{currentfill}%
\pgfsetlinewidth{0.000000pt}%
\definecolor{currentstroke}{rgb}{0.000000,0.000000,0.000000}%
\pgfsetstrokecolor{currentstroke}%
\pgfsetdash{}{0pt}%
\pgfpathmoveto{\pgfqpoint{2.068861in}{1.918249in}}%
\pgfpathlineto{\pgfqpoint{1.086669in}{2.049885in}}%
\pgfpathlineto{\pgfqpoint{1.094552in}{2.026900in}}%
\pgfpathlineto{\pgfqpoint{0.991178in}{2.073647in}}%
\pgfpathlineto{\pgfqpoint{1.103213in}{2.091523in}}%
\pgfpathlineto{\pgfqpoint{1.089556in}{2.071426in}}%
\pgfpathlineto{\pgfqpoint{2.071748in}{1.939790in}}%
\pgfpathlineto{\pgfqpoint{2.068861in}{1.918249in}}%
\pgfusepath{fill}%
\end{pgfscope}%
\begin{pgfscope}%
\pgfpathrectangle{\pgfqpoint{0.800000in}{1.363959in}}{\pgfqpoint{3.968000in}{2.024082in}} %
\pgfusepath{clip}%
\pgfsetbuttcap%
\pgfsetroundjoin%
\definecolor{currentfill}{rgb}{0.458674,0.816363,0.329727}%
\pgfsetfillcolor{currentfill}%
\pgfsetlinewidth{0.000000pt}%
\definecolor{currentstroke}{rgb}{0.000000,0.000000,0.000000}%
\pgfsetstrokecolor{currentstroke}%
\pgfsetdash{}{0pt}%
\pgfpathmoveto{\pgfqpoint{2.896610in}{1.569435in}}%
\pgfpathlineto{\pgfqpoint{3.760454in}{1.509207in}}%
\pgfpathlineto{\pgfqpoint{3.751125in}{1.531643in}}%
\pgfpathlineto{\pgfqpoint{3.857262in}{1.491564in}}%
\pgfpathlineto{\pgfqpoint{3.746590in}{1.466600in}}%
\pgfpathlineto{\pgfqpoint{3.758942in}{1.487526in}}%
\pgfpathlineto{\pgfqpoint{2.895098in}{1.547754in}}%
\pgfpathlineto{\pgfqpoint{2.896610in}{1.569435in}}%
\pgfusepath{fill}%
\end{pgfscope}%
\begin{pgfscope}%
\pgfpathrectangle{\pgfqpoint{0.800000in}{1.363959in}}{\pgfqpoint{3.968000in}{2.024082in}} %
\pgfusepath{clip}%
\pgfsetbuttcap%
\pgfsetroundjoin%
\definecolor{currentfill}{rgb}{0.281924,0.089666,0.412415}%
\pgfsetfillcolor{currentfill}%
\pgfsetlinewidth{0.000000pt}%
\definecolor{currentstroke}{rgb}{0.000000,0.000000,0.000000}%
\pgfsetstrokecolor{currentstroke}%
\pgfsetdash{}{0pt}%
\pgfpathmoveto{\pgfqpoint{2.895845in}{1.569461in}}%
\pgfpathlineto{\pgfqpoint{3.693752in}{1.570110in}}%
\pgfpathlineto{\pgfqpoint{3.682868in}{1.591834in}}%
\pgfpathlineto{\pgfqpoint{3.791563in}{1.559322in}}%
\pgfpathlineto{\pgfqpoint{3.682921in}{1.526633in}}%
\pgfpathlineto{\pgfqpoint{3.693770in}{1.548376in}}%
\pgfpathlineto{\pgfqpoint{2.895863in}{1.547728in}}%
\pgfpathlineto{\pgfqpoint{2.895845in}{1.569461in}}%
\pgfusepath{fill}%
\end{pgfscope}%
\begin{pgfscope}%
\pgfpathrectangle{\pgfqpoint{0.800000in}{1.363959in}}{\pgfqpoint{3.968000in}{2.024082in}} %
\pgfusepath{clip}%
\pgfsetbuttcap%
\pgfsetroundjoin%
\definecolor{currentfill}{rgb}{0.993248,0.906157,0.143936}%
\pgfsetfillcolor{currentfill}%
\pgfsetlinewidth{0.000000pt}%
\definecolor{currentstroke}{rgb}{0.000000,0.000000,0.000000}%
\pgfsetstrokecolor{currentstroke}%
\pgfsetdash{}{0pt}%
\pgfpathmoveto{\pgfqpoint{3.101957in}{2.136052in}}%
\pgfpathlineto{\pgfqpoint{3.996455in}{2.130121in}}%
\pgfpathlineto{\pgfqpoint{3.985732in}{2.151926in}}%
\pgfpathlineto{\pgfqpoint{4.094182in}{2.118606in}}%
\pgfpathlineto{\pgfqpoint{3.985300in}{2.086727in}}%
\pgfpathlineto{\pgfqpoint{3.996311in}{2.108388in}}%
\pgfpathlineto{\pgfqpoint{3.101813in}{2.114319in}}%
\pgfpathlineto{\pgfqpoint{3.101957in}{2.136052in}}%
\pgfusepath{fill}%
\end{pgfscope}%
\begin{pgfscope}%
\pgfpathrectangle{\pgfqpoint{0.800000in}{1.363959in}}{\pgfqpoint{3.968000in}{2.024082in}} %
\pgfusepath{clip}%
\pgfsetbuttcap%
\pgfsetroundjoin%
\definecolor{currentfill}{rgb}{0.196571,0.711827,0.479221}%
\pgfsetfillcolor{currentfill}%
\pgfsetlinewidth{0.000000pt}%
\definecolor{currentstroke}{rgb}{0.000000,0.000000,0.000000}%
\pgfsetstrokecolor{currentstroke}%
\pgfsetdash{}{0pt}%
\pgfpathmoveto{\pgfqpoint{1.984074in}{2.599172in}}%
\pgfpathlineto{\pgfqpoint{1.373125in}{2.801816in}}%
\pgfpathlineto{\pgfqpoint{1.376597in}{2.777767in}}%
\pgfpathlineto{\pgfqpoint{1.283718in}{2.842920in}}%
\pgfpathlineto{\pgfqpoint{1.397124in}{2.839652in}}%
\pgfpathlineto{\pgfqpoint{1.379967in}{2.822445in}}%
\pgfpathlineto{\pgfqpoint{1.990916in}{2.619801in}}%
\pgfpathlineto{\pgfqpoint{1.984074in}{2.599172in}}%
\pgfusepath{fill}%
\end{pgfscope}%
\begin{pgfscope}%
\pgfpathrectangle{\pgfqpoint{0.800000in}{1.363959in}}{\pgfqpoint{3.968000in}{2.024082in}} %
\pgfusepath{clip}%
\pgfsetbuttcap%
\pgfsetroundjoin%
\definecolor{currentfill}{rgb}{0.257322,0.256130,0.526563}%
\pgfsetfillcolor{currentfill}%
\pgfsetlinewidth{0.000000pt}%
\definecolor{currentstroke}{rgb}{0.000000,0.000000,0.000000}%
\pgfsetstrokecolor{currentstroke}%
\pgfsetdash{}{0pt}%
\pgfpathmoveto{\pgfqpoint{1.983726in}{2.599294in}}%
\pgfpathlineto{\pgfqpoint{1.236692in}{2.875488in}}%
\pgfpathlineto{\pgfqpoint{1.239347in}{2.851334in}}%
\pgfpathlineto{\pgfqpoint{1.148727in}{2.919596in}}%
\pgfpathlineto{\pgfqpoint{1.261957in}{2.912489in}}%
\pgfpathlineto{\pgfqpoint{1.244228in}{2.895873in}}%
\pgfpathlineto{\pgfqpoint{1.991263in}{2.619679in}}%
\pgfpathlineto{\pgfqpoint{1.983726in}{2.599294in}}%
\pgfusepath{fill}%
\end{pgfscope}%
\begin{pgfscope}%
\pgfpathrectangle{\pgfqpoint{0.800000in}{1.363959in}}{\pgfqpoint{3.968000in}{2.024082in}} %
\pgfusepath{clip}%
\pgfsetbuttcap%
\pgfsetroundjoin%
\definecolor{currentfill}{rgb}{0.271305,0.019942,0.347269}%
\pgfsetfillcolor{currentfill}%
\pgfsetlinewidth{0.000000pt}%
\definecolor{currentstroke}{rgb}{0.000000,0.000000,0.000000}%
\pgfsetstrokecolor{currentstroke}%
\pgfsetdash{}{0pt}%
\pgfpathmoveto{\pgfqpoint{2.568757in}{1.837766in}}%
\pgfpathlineto{\pgfqpoint{3.615645in}{1.856973in}}%
\pgfpathlineto{\pgfqpoint{3.604382in}{1.878504in}}%
\pgfpathlineto{\pgfqpoint{3.713630in}{1.847902in}}%
\pgfpathlineto{\pgfqpoint{3.605578in}{1.813314in}}%
\pgfpathlineto{\pgfqpoint{3.616044in}{1.835243in}}%
\pgfpathlineto{\pgfqpoint{2.569156in}{1.816036in}}%
\pgfpathlineto{\pgfqpoint{2.568757in}{1.837766in}}%
\pgfusepath{fill}%
\end{pgfscope}%
\begin{pgfscope}%
\pgfpathrectangle{\pgfqpoint{0.800000in}{1.363959in}}{\pgfqpoint{3.968000in}{2.024082in}} %
\pgfusepath{clip}%
\pgfsetbuttcap%
\pgfsetroundjoin%
\definecolor{currentfill}{rgb}{0.273006,0.204520,0.501721}%
\pgfsetfillcolor{currentfill}%
\pgfsetlinewidth{0.000000pt}%
\definecolor{currentstroke}{rgb}{0.000000,0.000000,0.000000}%
\pgfsetstrokecolor{currentstroke}%
\pgfsetdash{}{0pt}%
\pgfpathmoveto{\pgfqpoint{2.564385in}{1.817043in}}%
\pgfpathlineto{\pgfqpoint{1.879984in}{2.134446in}}%
\pgfpathlineto{\pgfqpoint{1.880699in}{2.110157in}}%
\pgfpathlineto{\pgfqpoint{1.795832in}{2.185451in}}%
\pgfpathlineto{\pgfqpoint{1.908130in}{2.169307in}}%
\pgfpathlineto{\pgfqpoint{1.889128in}{2.154162in}}%
\pgfpathlineto{\pgfqpoint{2.573529in}{1.836759in}}%
\pgfpathlineto{\pgfqpoint{2.564385in}{1.817043in}}%
\pgfusepath{fill}%
\end{pgfscope}%
\begin{pgfscope}%
\pgfpathrectangle{\pgfqpoint{0.800000in}{1.363959in}}{\pgfqpoint{3.968000in}{2.024082in}} %
\pgfusepath{clip}%
\pgfsetbuttcap%
\pgfsetroundjoin%
\definecolor{currentfill}{rgb}{0.162142,0.474838,0.558140}%
\pgfsetfillcolor{currentfill}%
\pgfsetlinewidth{0.000000pt}%
\definecolor{currentstroke}{rgb}{0.000000,0.000000,0.000000}%
\pgfsetstrokecolor{currentstroke}%
\pgfsetdash{}{0pt}%
\pgfpathmoveto{\pgfqpoint{2.569595in}{1.837749in}}%
\pgfpathlineto{\pgfqpoint{3.636643in}{1.774956in}}%
\pgfpathlineto{\pgfqpoint{3.627072in}{1.797290in}}%
\pgfpathlineto{\pgfqpoint{3.733637in}{1.758362in}}%
\pgfpathlineto{\pgfqpoint{3.623241in}{1.732202in}}%
\pgfpathlineto{\pgfqpoint{3.635366in}{1.753259in}}%
\pgfpathlineto{\pgfqpoint{2.568318in}{1.816053in}}%
\pgfpathlineto{\pgfqpoint{2.569595in}{1.837749in}}%
\pgfusepath{fill}%
\end{pgfscope}%
\begin{pgfscope}%
\pgfpathrectangle{\pgfqpoint{0.800000in}{1.363959in}}{\pgfqpoint{3.968000in}{2.024082in}} %
\pgfusepath{clip}%
\pgfsetbuttcap%
\pgfsetroundjoin%
\definecolor{currentfill}{rgb}{0.993248,0.906157,0.143936}%
\pgfsetfillcolor{currentfill}%
\pgfsetlinewidth{0.000000pt}%
\definecolor{currentstroke}{rgb}{0.000000,0.000000,0.000000}%
\pgfsetstrokecolor{currentstroke}%
\pgfsetdash{}{0pt}%
\pgfpathmoveto{\pgfqpoint{3.091949in}{2.849859in}}%
\pgfpathlineto{\pgfqpoint{3.882769in}{2.845571in}}%
\pgfpathlineto{\pgfqpoint{3.872020in}{2.867363in}}%
\pgfpathlineto{\pgfqpoint{3.980510in}{2.834174in}}%
\pgfpathlineto{\pgfqpoint{3.871666in}{2.802163in}}%
\pgfpathlineto{\pgfqpoint{3.882651in}{2.823838in}}%
\pgfpathlineto{\pgfqpoint{3.091831in}{2.828126in}}%
\pgfpathlineto{\pgfqpoint{3.091949in}{2.849859in}}%
\pgfusepath{fill}%
\end{pgfscope}%
\begin{pgfscope}%
\pgfpathrectangle{\pgfqpoint{0.800000in}{1.363959in}}{\pgfqpoint{3.968000in}{2.024082in}} %
\pgfusepath{clip}%
\pgfsetbuttcap%
\pgfsetroundjoin%
\definecolor{currentfill}{rgb}{0.168126,0.459988,0.558082}%
\pgfsetfillcolor{currentfill}%
\pgfsetlinewidth{0.000000pt}%
\definecolor{currentstroke}{rgb}{0.000000,0.000000,0.000000}%
\pgfsetstrokecolor{currentstroke}%
\pgfsetdash{}{0pt}%
\pgfpathmoveto{\pgfqpoint{2.568425in}{1.569811in}}%
\pgfpathlineto{\pgfqpoint{3.599073in}{1.642201in}}%
\pgfpathlineto{\pgfqpoint{3.586711in}{1.663120in}}%
\pgfpathlineto{\pgfqpoint{3.697396in}{1.638214in}}%
\pgfpathlineto{\pgfqpoint{3.591279in}{1.598080in}}%
\pgfpathlineto{\pgfqpoint{3.600596in}{1.620521in}}%
\pgfpathlineto{\pgfqpoint{2.569948in}{1.548130in}}%
\pgfpathlineto{\pgfqpoint{2.568425in}{1.569811in}}%
\pgfusepath{fill}%
\end{pgfscope}%
\begin{pgfscope}%
\pgfpathrectangle{\pgfqpoint{0.800000in}{1.363959in}}{\pgfqpoint{3.968000in}{2.024082in}} %
\pgfusepath{clip}%
\pgfsetbuttcap%
\pgfsetroundjoin%
\definecolor{currentfill}{rgb}{0.140536,0.530132,0.555659}%
\pgfsetfillcolor{currentfill}%
\pgfsetlinewidth{0.000000pt}%
\definecolor{currentstroke}{rgb}{0.000000,0.000000,0.000000}%
\pgfsetstrokecolor{currentstroke}%
\pgfsetdash{}{0pt}%
\pgfpathmoveto{\pgfqpoint{2.569865in}{1.548125in}}%
\pgfpathlineto{\pgfqpoint{1.606326in}{1.487849in}}%
\pgfpathlineto{\pgfqpoint{1.618529in}{1.466836in}}%
\pgfpathlineto{\pgfqpoint{1.508037in}{1.492588in}}%
\pgfpathlineto{\pgfqpoint{1.614458in}{1.531910in}}%
\pgfpathlineto{\pgfqpoint{1.604969in}{1.509540in}}%
\pgfpathlineto{\pgfqpoint{2.568508in}{1.569816in}}%
\pgfpathlineto{\pgfqpoint{2.569865in}{1.548125in}}%
\pgfusepath{fill}%
\end{pgfscope}%
\begin{pgfscope}%
\pgfpathrectangle{\pgfqpoint{0.800000in}{1.363959in}}{\pgfqpoint{3.968000in}{2.024082in}} %
\pgfusepath{clip}%
\pgfsetbuttcap%
\pgfsetroundjoin%
\definecolor{currentfill}{rgb}{0.804182,0.882046,0.114965}%
\pgfsetfillcolor{currentfill}%
\pgfsetlinewidth{0.000000pt}%
\definecolor{currentstroke}{rgb}{0.000000,0.000000,0.000000}%
\pgfsetstrokecolor{currentstroke}%
\pgfsetdash{}{0pt}%
\pgfpathmoveto{\pgfqpoint{2.796265in}{2.946854in}}%
\pgfpathlineto{\pgfqpoint{3.692611in}{2.943841in}}%
\pgfpathlineto{\pgfqpoint{3.681817in}{2.965611in}}%
\pgfpathlineto{\pgfqpoint{3.790375in}{2.932646in}}%
\pgfpathlineto{\pgfqpoint{3.681598in}{2.900411in}}%
\pgfpathlineto{\pgfqpoint{3.692538in}{2.922108in}}%
\pgfpathlineto{\pgfqpoint{2.796192in}{2.925121in}}%
\pgfpathlineto{\pgfqpoint{2.796265in}{2.946854in}}%
\pgfusepath{fill}%
\end{pgfscope}%
\begin{pgfscope}%
\pgfpathrectangle{\pgfqpoint{0.800000in}{1.363959in}}{\pgfqpoint{3.968000in}{2.024082in}} %
\pgfusepath{clip}%
\pgfsetbuttcap%
\pgfsetroundjoin%
\definecolor{currentfill}{rgb}{0.279566,0.067836,0.391917}%
\pgfsetfillcolor{currentfill}%
\pgfsetlinewidth{0.000000pt}%
\definecolor{currentstroke}{rgb}{0.000000,0.000000,0.000000}%
\pgfsetstrokecolor{currentstroke}%
\pgfsetdash{}{0pt}%
\pgfpathmoveto{\pgfqpoint{3.034924in}{1.806542in}}%
\pgfpathlineto{\pgfqpoint{3.726233in}{1.798703in}}%
\pgfpathlineto{\pgfqpoint{3.715614in}{1.820558in}}%
\pgfpathlineto{\pgfqpoint{3.823905in}{1.786727in}}%
\pgfpathlineto{\pgfqpoint{3.714874in}{1.755361in}}%
\pgfpathlineto{\pgfqpoint{3.725987in}{1.776970in}}%
\pgfpathlineto{\pgfqpoint{3.034677in}{1.784810in}}%
\pgfpathlineto{\pgfqpoint{3.034924in}{1.806542in}}%
\pgfusepath{fill}%
\end{pgfscope}%
\begin{pgfscope}%
\pgfpathrectangle{\pgfqpoint{0.800000in}{1.363959in}}{\pgfqpoint{3.968000in}{2.024082in}} %
\pgfusepath{clip}%
\pgfsetbuttcap%
\pgfsetroundjoin%
\definecolor{currentfill}{rgb}{0.496615,0.826376,0.306377}%
\pgfsetfillcolor{currentfill}%
\pgfsetlinewidth{0.000000pt}%
\definecolor{currentstroke}{rgb}{0.000000,0.000000,0.000000}%
\pgfsetstrokecolor{currentstroke}%
\pgfsetdash{}{0pt}%
\pgfpathmoveto{\pgfqpoint{3.035590in}{1.806514in}}%
\pgfpathlineto{\pgfqpoint{3.852695in}{1.747017in}}%
\pgfpathlineto{\pgfqpoint{3.843435in}{1.769482in}}%
\pgfpathlineto{\pgfqpoint{3.949449in}{1.729076in}}%
\pgfpathlineto{\pgfqpoint{3.838700in}{1.704454in}}%
\pgfpathlineto{\pgfqpoint{3.851117in}{1.725341in}}%
\pgfpathlineto{\pgfqpoint{3.034011in}{1.784838in}}%
\pgfpathlineto{\pgfqpoint{3.035590in}{1.806514in}}%
\pgfusepath{fill}%
\end{pgfscope}%
\begin{pgfscope}%
\pgfpathrectangle{\pgfqpoint{0.800000in}{1.363959in}}{\pgfqpoint{3.968000in}{2.024082in}} %
\pgfusepath{clip}%
\pgfsetbuttcap%
\pgfsetroundjoin%
\definecolor{currentfill}{rgb}{0.993248,0.906157,0.143936}%
\pgfsetfillcolor{currentfill}%
\pgfsetlinewidth{0.000000pt}%
\definecolor{currentstroke}{rgb}{0.000000,0.000000,0.000000}%
\pgfsetstrokecolor{currentstroke}%
\pgfsetdash{}{0pt}%
\pgfpathmoveto{\pgfqpoint{2.757095in}{2.638528in}}%
\pgfpathlineto{\pgfqpoint{3.773159in}{2.633667in}}%
\pgfpathlineto{\pgfqpoint{3.762397in}{2.655452in}}%
\pgfpathlineto{\pgfqpoint{3.870907in}{2.622333in}}%
\pgfpathlineto{\pgfqpoint{3.762085in}{2.590252in}}%
\pgfpathlineto{\pgfqpoint{3.773055in}{2.611934in}}%
\pgfpathlineto{\pgfqpoint{2.756991in}{2.616794in}}%
\pgfpathlineto{\pgfqpoint{2.757095in}{2.638528in}}%
\pgfusepath{fill}%
\end{pgfscope}%
\begin{pgfscope}%
\pgfpathrectangle{\pgfqpoint{0.800000in}{1.363959in}}{\pgfqpoint{3.968000in}{2.024082in}} %
\pgfusepath{clip}%
\pgfsetbuttcap%
\pgfsetroundjoin%
\definecolor{currentfill}{rgb}{0.688944,0.865448,0.182725}%
\pgfsetfillcolor{currentfill}%
\pgfsetlinewidth{0.000000pt}%
\definecolor{currentstroke}{rgb}{0.000000,0.000000,0.000000}%
\pgfsetstrokecolor{currentstroke}%
\pgfsetdash{}{0pt}%
\pgfpathmoveto{\pgfqpoint{3.560305in}{2.424269in}}%
\pgfpathlineto{\pgfqpoint{4.331190in}{2.300634in}}%
\pgfpathlineto{\pgfqpoint{4.323902in}{2.323814in}}%
\pgfpathlineto{\pgfqpoint{4.426036in}{2.274417in}}%
\pgfpathlineto{\pgfqpoint{4.313577in}{2.259436in}}%
\pgfpathlineto{\pgfqpoint{4.327748in}{2.279174in}}%
\pgfpathlineto{\pgfqpoint{3.556863in}{2.402809in}}%
\pgfpathlineto{\pgfqpoint{3.560305in}{2.424269in}}%
\pgfusepath{fill}%
\end{pgfscope}%
\begin{pgfscope}%
\pgfpathrectangle{\pgfqpoint{0.800000in}{1.363959in}}{\pgfqpoint{3.968000in}{2.024082in}} %
\pgfusepath{clip}%
\pgfsetbuttcap%
\pgfsetroundjoin%
\definecolor{currentfill}{rgb}{0.647257,0.858400,0.209861}%
\pgfsetfillcolor{currentfill}%
\pgfsetlinewidth{0.000000pt}%
\definecolor{currentstroke}{rgb}{0.000000,0.000000,0.000000}%
\pgfsetstrokecolor{currentstroke}%
\pgfsetdash{}{0pt}%
\pgfpathmoveto{\pgfqpoint{2.890057in}{3.170772in}}%
\pgfpathlineto{\pgfqpoint{3.593228in}{3.213991in}}%
\pgfpathlineto{\pgfqpoint{3.581048in}{3.235017in}}%
\pgfpathlineto{\pgfqpoint{3.691511in}{3.209144in}}%
\pgfpathlineto{\pgfqpoint{3.585048in}{3.169939in}}%
\pgfpathlineto{\pgfqpoint{3.594561in}{3.192298in}}%
\pgfpathlineto{\pgfqpoint{2.891390in}{3.149080in}}%
\pgfpathlineto{\pgfqpoint{2.890057in}{3.170772in}}%
\pgfusepath{fill}%
\end{pgfscope}%
\begin{pgfscope}%
\pgfpathrectangle{\pgfqpoint{0.800000in}{1.363959in}}{\pgfqpoint{3.968000in}{2.024082in}} %
\pgfusepath{clip}%
\pgfsetbuttcap%
\pgfsetroundjoin%
\definecolor{currentfill}{rgb}{0.993248,0.906157,0.143936}%
\pgfsetfillcolor{currentfill}%
\pgfsetlinewidth{0.000000pt}%
\definecolor{currentstroke}{rgb}{0.000000,0.000000,0.000000}%
\pgfsetstrokecolor{currentstroke}%
\pgfsetdash{}{0pt}%
\pgfpathmoveto{\pgfqpoint{3.482113in}{1.936853in}}%
\pgfpathlineto{\pgfqpoint{4.244426in}{1.911121in}}%
\pgfpathlineto{\pgfqpoint{4.234299in}{1.933209in}}%
\pgfpathlineto{\pgfqpoint{4.341806in}{1.896961in}}%
\pgfpathlineto{\pgfqpoint{4.232099in}{1.868045in}}%
\pgfpathlineto{\pgfqpoint{4.243693in}{1.889400in}}%
\pgfpathlineto{\pgfqpoint{3.481379in}{1.915132in}}%
\pgfpathlineto{\pgfqpoint{3.482113in}{1.936853in}}%
\pgfusepath{fill}%
\end{pgfscope}%
\begin{pgfscope}%
\pgfpathrectangle{\pgfqpoint{0.800000in}{1.363959in}}{\pgfqpoint{3.968000in}{2.024082in}} %
\pgfusepath{clip}%
\pgfsetbuttcap%
\pgfsetroundjoin%
\definecolor{currentfill}{rgb}{0.647257,0.858400,0.209861}%
\pgfsetfillcolor{currentfill}%
\pgfsetlinewidth{0.000000pt}%
\definecolor{currentstroke}{rgb}{0.000000,0.000000,0.000000}%
\pgfsetstrokecolor{currentstroke}%
\pgfsetdash{}{0pt}%
\pgfpathmoveto{\pgfqpoint{2.054764in}{2.795253in}}%
\pgfpathlineto{\pgfqpoint{1.318927in}{2.907525in}}%
\pgfpathlineto{\pgfqpoint{1.326391in}{2.884401in}}%
\pgfpathlineto{\pgfqpoint{1.223883in}{2.933019in}}%
\pgfpathlineto{\pgfqpoint{1.336225in}{2.948856in}}%
\pgfpathlineto{\pgfqpoint{1.322205in}{2.929010in}}%
\pgfpathlineto{\pgfqpoint{2.058042in}{2.816738in}}%
\pgfpathlineto{\pgfqpoint{2.054764in}{2.795253in}}%
\pgfusepath{fill}%
\end{pgfscope}%
\begin{pgfscope}%
\pgfpathrectangle{\pgfqpoint{0.800000in}{1.363959in}}{\pgfqpoint{3.968000in}{2.024082in}} %
\pgfusepath{clip}%
\pgfsetbuttcap%
\pgfsetroundjoin%
\definecolor{currentfill}{rgb}{0.647257,0.858400,0.209861}%
\pgfsetfillcolor{currentfill}%
\pgfsetlinewidth{0.000000pt}%
\definecolor{currentstroke}{rgb}{0.000000,0.000000,0.000000}%
\pgfsetstrokecolor{currentstroke}%
\pgfsetdash{}{0pt}%
\pgfpathmoveto{\pgfqpoint{3.304032in}{2.886001in}}%
\pgfpathlineto{\pgfqpoint{4.098467in}{2.784506in}}%
\pgfpathlineto{\pgfqpoint{4.090442in}{2.807442in}}%
\pgfpathlineto{\pgfqpoint{4.194103in}{2.761333in}}%
\pgfpathlineto{\pgfqpoint{4.082179in}{2.742767in}}%
\pgfpathlineto{\pgfqpoint{4.095713in}{2.762948in}}%
\pgfpathlineto{\pgfqpoint{3.301278in}{2.864442in}}%
\pgfpathlineto{\pgfqpoint{3.304032in}{2.886001in}}%
\pgfusepath{fill}%
\end{pgfscope}%
\begin{pgfscope}%
\pgfpathrectangle{\pgfqpoint{0.800000in}{1.363959in}}{\pgfqpoint{3.968000in}{2.024082in}} %
\pgfusepath{clip}%
\pgfsetbuttcap%
\pgfsetroundjoin%
\definecolor{currentfill}{rgb}{0.647257,0.858400,0.209861}%
\pgfsetfillcolor{currentfill}%
\pgfsetlinewidth{0.000000pt}%
\definecolor{currentstroke}{rgb}{0.000000,0.000000,0.000000}%
\pgfsetstrokecolor{currentstroke}%
\pgfsetdash{}{0pt}%
\pgfpathmoveto{\pgfqpoint{2.538808in}{3.117371in}}%
\pgfpathlineto{\pgfqpoint{1.794291in}{3.089675in}}%
\pgfpathlineto{\pgfqpoint{1.805958in}{3.068360in}}%
\pgfpathlineto{\pgfqpoint{1.696153in}{3.096898in}}%
\pgfpathlineto{\pgfqpoint{1.803534in}{3.133516in}}%
\pgfpathlineto{\pgfqpoint{1.793483in}{3.111393in}}%
\pgfpathlineto{\pgfqpoint{2.538000in}{3.139089in}}%
\pgfpathlineto{\pgfqpoint{2.538808in}{3.117371in}}%
\pgfusepath{fill}%
\end{pgfscope}%
\begin{pgfscope}%
\pgfpathrectangle{\pgfqpoint{0.800000in}{1.363959in}}{\pgfqpoint{3.968000in}{2.024082in}} %
\pgfusepath{clip}%
\pgfsetbuttcap%
\pgfsetroundjoin%
\definecolor{currentfill}{rgb}{0.647257,0.858400,0.209861}%
\pgfsetfillcolor{currentfill}%
\pgfsetlinewidth{0.000000pt}%
\definecolor{currentstroke}{rgb}{0.000000,0.000000,0.000000}%
\pgfsetstrokecolor{currentstroke}%
\pgfsetdash{}{0pt}%
\pgfpathmoveto{\pgfqpoint{2.445849in}{1.861167in}}%
\pgfpathlineto{\pgfqpoint{1.711058in}{1.824283in}}%
\pgfpathlineto{\pgfqpoint{1.723001in}{1.803121in}}%
\pgfpathlineto{\pgfqpoint{1.612835in}{1.830233in}}%
\pgfpathlineto{\pgfqpoint{1.719732in}{1.868240in}}%
\pgfpathlineto{\pgfqpoint{1.709969in}{1.845989in}}%
\pgfpathlineto{\pgfqpoint{2.444759in}{1.882873in}}%
\pgfpathlineto{\pgfqpoint{2.445849in}{1.861167in}}%
\pgfusepath{fill}%
\end{pgfscope}%
\begin{pgfscope}%
\pgfpathrectangle{\pgfqpoint{0.800000in}{1.363959in}}{\pgfqpoint{3.968000in}{2.024082in}} %
\pgfusepath{clip}%
\pgfsetbuttcap%
\pgfsetroundjoin%
\definecolor{currentfill}{rgb}{0.132268,0.655014,0.519661}%
\pgfsetfillcolor{currentfill}%
\pgfsetlinewidth{0.000000pt}%
\definecolor{currentstroke}{rgb}{0.000000,0.000000,0.000000}%
\pgfsetstrokecolor{currentstroke}%
\pgfsetdash{}{0pt}%
\pgfpathmoveto{\pgfqpoint{3.196761in}{2.596212in}}%
\pgfpathlineto{\pgfqpoint{4.069115in}{2.624741in}}%
\pgfpathlineto{\pgfqpoint{4.057544in}{2.646108in}}%
\pgfpathlineto{\pgfqpoint{4.167219in}{2.617077in}}%
\pgfpathlineto{\pgfqpoint{4.059675in}{2.580942in}}%
\pgfpathlineto{\pgfqpoint{4.069826in}{2.603019in}}%
\pgfpathlineto{\pgfqpoint{3.197472in}{2.574490in}}%
\pgfpathlineto{\pgfqpoint{3.196761in}{2.596212in}}%
\pgfusepath{fill}%
\end{pgfscope}%
\begin{pgfscope}%
\pgfpathrectangle{\pgfqpoint{0.800000in}{1.363959in}}{\pgfqpoint{3.968000in}{2.024082in}} %
\pgfusepath{clip}%
\pgfsetbuttcap%
\pgfsetroundjoin%
\definecolor{currentfill}{rgb}{0.229739,0.322361,0.545706}%
\pgfsetfillcolor{currentfill}%
\pgfsetlinewidth{0.000000pt}%
\definecolor{currentstroke}{rgb}{0.000000,0.000000,0.000000}%
\pgfsetstrokecolor{currentstroke}%
\pgfsetdash{}{0pt}%
\pgfpathmoveto{\pgfqpoint{3.196992in}{2.596217in}}%
\pgfpathlineto{\pgfqpoint{4.033195in}{2.605765in}}%
\pgfpathlineto{\pgfqpoint{4.022081in}{2.627373in}}%
\pgfpathlineto{\pgfqpoint{4.131114in}{2.596016in}}%
\pgfpathlineto{\pgfqpoint{4.022825in}{2.562177in}}%
\pgfpathlineto{\pgfqpoint{4.033443in}{2.584033in}}%
\pgfpathlineto{\pgfqpoint{3.197241in}{2.574484in}}%
\pgfpathlineto{\pgfqpoint{3.196992in}{2.596217in}}%
\pgfusepath{fill}%
\end{pgfscope}%
\begin{pgfscope}%
\pgfpathrectangle{\pgfqpoint{0.800000in}{1.363959in}}{\pgfqpoint{3.968000in}{2.024082in}} %
\pgfusepath{clip}%
\pgfsetbuttcap%
\pgfsetroundjoin%
\definecolor{currentfill}{rgb}{0.327796,0.773980,0.406640}%
\pgfsetfillcolor{currentfill}%
\pgfsetlinewidth{0.000000pt}%
\definecolor{currentstroke}{rgb}{0.000000,0.000000,0.000000}%
\pgfsetstrokecolor{currentstroke}%
\pgfsetdash{}{0pt}%
\pgfpathmoveto{\pgfqpoint{2.960550in}{2.484048in}}%
\pgfpathlineto{\pgfqpoint{3.898888in}{2.432980in}}%
\pgfpathlineto{\pgfqpoint{3.889219in}{2.455272in}}%
\pgfpathlineto{\pgfqpoint{3.995955in}{2.416814in}}%
\pgfpathlineto{\pgfqpoint{3.885675in}{2.390167in}}%
\pgfpathlineto{\pgfqpoint{3.897707in}{2.411278in}}%
\pgfpathlineto{\pgfqpoint{2.959369in}{2.462347in}}%
\pgfpathlineto{\pgfqpoint{2.960550in}{2.484048in}}%
\pgfusepath{fill}%
\end{pgfscope}%
\begin{pgfscope}%
\pgfpathrectangle{\pgfqpoint{0.800000in}{1.363959in}}{\pgfqpoint{3.968000in}{2.024082in}} %
\pgfusepath{clip}%
\pgfsetbuttcap%
\pgfsetroundjoin%
\definecolor{currentfill}{rgb}{0.280255,0.165693,0.476498}%
\pgfsetfillcolor{currentfill}%
\pgfsetlinewidth{0.000000pt}%
\definecolor{currentstroke}{rgb}{0.000000,0.000000,0.000000}%
\pgfsetstrokecolor{currentstroke}%
\pgfsetdash{}{0pt}%
\pgfpathmoveto{\pgfqpoint{2.962193in}{2.483832in}}%
\pgfpathlineto{\pgfqpoint{4.003341in}{2.265113in}}%
\pgfpathlineto{\pgfqpoint{3.997175in}{2.288617in}}%
\pgfpathlineto{\pgfqpoint{4.096820in}{2.234372in}}%
\pgfpathlineto{\pgfqpoint{3.983770in}{2.224809in}}%
\pgfpathlineto{\pgfqpoint{3.998873in}{2.243844in}}%
\pgfpathlineto{\pgfqpoint{2.957725in}{2.462563in}}%
\pgfpathlineto{\pgfqpoint{2.962193in}{2.483832in}}%
\pgfusepath{fill}%
\end{pgfscope}%
\begin{pgfscope}%
\pgfpathrectangle{\pgfqpoint{0.800000in}{1.363959in}}{\pgfqpoint{3.968000in}{2.024082in}} %
\pgfusepath{clip}%
\pgfsetbuttcap%
\pgfsetroundjoin%
\definecolor{currentfill}{rgb}{0.993248,0.906157,0.143936}%
\pgfsetfillcolor{currentfill}%
\pgfsetlinewidth{0.000000pt}%
\definecolor{currentstroke}{rgb}{0.000000,0.000000,0.000000}%
\pgfsetstrokecolor{currentstroke}%
\pgfsetdash{}{0pt}%
\pgfpathmoveto{\pgfqpoint{3.197461in}{3.153628in}}%
\pgfpathlineto{\pgfqpoint{4.068320in}{3.145522in}}%
\pgfpathlineto{\pgfqpoint{4.057656in}{3.167356in}}%
\pgfpathlineto{\pgfqpoint{4.166016in}{3.133746in}}%
\pgfpathlineto{\pgfqpoint{4.057049in}{3.102158in}}%
\pgfpathlineto{\pgfqpoint{4.068118in}{3.123790in}}%
\pgfpathlineto{\pgfqpoint{3.197258in}{3.131896in}}%
\pgfpathlineto{\pgfqpoint{3.197461in}{3.153628in}}%
\pgfusepath{fill}%
\end{pgfscope}%
\begin{pgfscope}%
\pgfpathrectangle{\pgfqpoint{0.800000in}{1.363959in}}{\pgfqpoint{3.968000in}{2.024082in}} %
\pgfusepath{clip}%
\pgfsetbuttcap%
\pgfsetroundjoin%
\definecolor{currentfill}{rgb}{0.257322,0.256130,0.526563}%
\pgfsetfillcolor{currentfill}%
\pgfsetlinewidth{0.000000pt}%
\definecolor{currentstroke}{rgb}{0.000000,0.000000,0.000000}%
\pgfsetstrokecolor{currentstroke}%
\pgfsetdash{}{0pt}%
\pgfpathmoveto{\pgfqpoint{2.444818in}{2.621036in}}%
\pgfpathlineto{\pgfqpoint{1.595538in}{2.830157in}}%
\pgfpathlineto{\pgfqpoint{1.600893in}{2.806455in}}%
\pgfpathlineto{\pgfqpoint{1.503171in}{2.864092in}}%
\pgfpathlineto{\pgfqpoint{1.616482in}{2.869765in}}%
\pgfpathlineto{\pgfqpoint{1.600734in}{2.851260in}}%
\pgfpathlineto{\pgfqpoint{2.450015in}{2.642140in}}%
\pgfpathlineto{\pgfqpoint{2.444818in}{2.621036in}}%
\pgfusepath{fill}%
\end{pgfscope}%
\begin{pgfscope}%
\pgfpathrectangle{\pgfqpoint{0.800000in}{1.363959in}}{\pgfqpoint{3.968000in}{2.024082in}} %
\pgfusepath{clip}%
\pgfsetbuttcap%
\pgfsetroundjoin%
\definecolor{currentfill}{rgb}{0.196571,0.711827,0.479221}%
\pgfsetfillcolor{currentfill}%
\pgfsetlinewidth{0.000000pt}%
\definecolor{currentstroke}{rgb}{0.000000,0.000000,0.000000}%
\pgfsetstrokecolor{currentstroke}%
\pgfsetdash{}{0pt}%
\pgfpathmoveto{\pgfqpoint{2.447994in}{2.620737in}}%
\pgfpathlineto{\pgfqpoint{1.737879in}{2.582935in}}%
\pgfpathlineto{\pgfqpoint{1.749886in}{2.561810in}}%
\pgfpathlineto{\pgfqpoint{1.639638in}{2.588587in}}%
\pgfpathlineto{\pgfqpoint{1.746420in}{2.626918in}}%
\pgfpathlineto{\pgfqpoint{1.736724in}{2.604638in}}%
\pgfpathlineto{\pgfqpoint{2.446839in}{2.642439in}}%
\pgfpathlineto{\pgfqpoint{2.447994in}{2.620737in}}%
\pgfusepath{fill}%
\end{pgfscope}%
\begin{pgfscope}%
\pgfpathrectangle{\pgfqpoint{0.800000in}{1.363959in}}{\pgfqpoint{3.968000in}{2.024082in}} %
\pgfusepath{clip}%
\pgfsetbuttcap%
\pgfsetroundjoin%
\definecolor{currentfill}{rgb}{0.327796,0.773980,0.406640}%
\pgfsetfillcolor{currentfill}%
\pgfsetlinewidth{0.000000pt}%
\definecolor{currentstroke}{rgb}{0.000000,0.000000,0.000000}%
\pgfsetstrokecolor{currentstroke}%
\pgfsetdash{}{0pt}%
\pgfpathmoveto{\pgfqpoint{2.297879in}{2.151715in}}%
\pgfpathlineto{\pgfqpoint{1.798455in}{2.256541in}}%
\pgfpathlineto{\pgfqpoint{1.804626in}{2.233038in}}%
\pgfpathlineto{\pgfqpoint{1.704972in}{2.287266in}}%
\pgfpathlineto{\pgfqpoint{1.818019in}{2.296849in}}%
\pgfpathlineto{\pgfqpoint{1.802920in}{2.277811in}}%
\pgfpathlineto{\pgfqpoint{2.302343in}{2.172985in}}%
\pgfpathlineto{\pgfqpoint{2.297879in}{2.151715in}}%
\pgfusepath{fill}%
\end{pgfscope}%
\begin{pgfscope}%
\pgfpathrectangle{\pgfqpoint{0.800000in}{1.363959in}}{\pgfqpoint{3.968000in}{2.024082in}} %
\pgfusepath{clip}%
\pgfsetbuttcap%
\pgfsetroundjoin%
\definecolor{currentfill}{rgb}{0.280255,0.165693,0.476498}%
\pgfsetfillcolor{currentfill}%
\pgfsetlinewidth{0.000000pt}%
\definecolor{currentstroke}{rgb}{0.000000,0.000000,0.000000}%
\pgfsetstrokecolor{currentstroke}%
\pgfsetdash{}{0pt}%
\pgfpathmoveto{\pgfqpoint{2.299614in}{2.151494in}}%
\pgfpathlineto{\pgfqpoint{1.893034in}{2.170120in}}%
\pgfpathlineto{\pgfqpoint{1.902894in}{2.147912in}}%
\pgfpathlineto{\pgfqpoint{1.795832in}{2.185451in}}%
\pgfpathlineto{\pgfqpoint{1.905878in}{2.213045in}}%
\pgfpathlineto{\pgfqpoint{1.894028in}{2.191831in}}%
\pgfpathlineto{\pgfqpoint{2.300608in}{2.173205in}}%
\pgfpathlineto{\pgfqpoint{2.299614in}{2.151494in}}%
\pgfusepath{fill}%
\end{pgfscope}%
\begin{pgfscope}%
\pgfpathrectangle{\pgfqpoint{0.800000in}{1.363959in}}{\pgfqpoint{3.968000in}{2.024082in}} %
\pgfusepath{clip}%
\pgfsetbuttcap%
\pgfsetroundjoin%
\definecolor{currentfill}{rgb}{0.282290,0.145912,0.461510}%
\pgfsetfillcolor{currentfill}%
\pgfsetlinewidth{0.000000pt}%
\definecolor{currentstroke}{rgb}{0.000000,0.000000,0.000000}%
\pgfsetstrokecolor{currentstroke}%
\pgfsetdash{}{0pt}%
\pgfpathmoveto{\pgfqpoint{3.470617in}{2.351864in}}%
\pgfpathlineto{\pgfqpoint{4.183948in}{2.155066in}}%
\pgfpathlineto{\pgfqpoint{4.179253in}{2.178907in}}%
\pgfpathlineto{\pgfqpoint{4.275337in}{2.118580in}}%
\pgfpathlineto{\pgfqpoint{4.161913in}{2.116054in}}%
\pgfpathlineto{\pgfqpoint{4.178168in}{2.134115in}}%
\pgfpathlineto{\pgfqpoint{3.464837in}{2.330914in}}%
\pgfpathlineto{\pgfqpoint{3.470617in}{2.351864in}}%
\pgfusepath{fill}%
\end{pgfscope}%
\begin{pgfscope}%
\pgfpathrectangle{\pgfqpoint{0.800000in}{1.363959in}}{\pgfqpoint{3.968000in}{2.024082in}} %
\pgfusepath{clip}%
\pgfsetbuttcap%
\pgfsetroundjoin%
\definecolor{currentfill}{rgb}{0.360741,0.785964,0.387814}%
\pgfsetfillcolor{currentfill}%
\pgfsetlinewidth{0.000000pt}%
\definecolor{currentstroke}{rgb}{0.000000,0.000000,0.000000}%
\pgfsetstrokecolor{currentstroke}%
\pgfsetdash{}{0pt}%
\pgfpathmoveto{\pgfqpoint{3.469636in}{2.352087in}}%
\pgfpathlineto{\pgfqpoint{4.263182in}{2.210495in}}%
\pgfpathlineto{\pgfqpoint{4.256302in}{2.233800in}}%
\pgfpathlineto{\pgfqpoint{4.357554in}{2.182618in}}%
\pgfpathlineto{\pgfqpoint{4.244849in}{2.169613in}}%
\pgfpathlineto{\pgfqpoint{4.259365in}{2.189100in}}%
\pgfpathlineto{\pgfqpoint{3.465818in}{2.330691in}}%
\pgfpathlineto{\pgfqpoint{3.469636in}{2.352087in}}%
\pgfusepath{fill}%
\end{pgfscope}%
\begin{pgfscope}%
\pgfpathrectangle{\pgfqpoint{0.800000in}{1.363959in}}{\pgfqpoint{3.968000in}{2.024082in}} %
\pgfusepath{clip}%
\pgfsetbuttcap%
\pgfsetroundjoin%
\definecolor{currentfill}{rgb}{0.993248,0.906157,0.143936}%
\pgfsetfillcolor{currentfill}%
\pgfsetlinewidth{0.000000pt}%
\definecolor{currentstroke}{rgb}{0.000000,0.000000,0.000000}%
\pgfsetstrokecolor{currentstroke}%
\pgfsetdash{}{0pt}%
\pgfpathmoveto{\pgfqpoint{2.593006in}{2.551438in}}%
\pgfpathlineto{\pgfqpoint{1.826416in}{2.704779in}}%
\pgfpathlineto{\pgfqpoint{1.832808in}{2.681336in}}%
\pgfpathlineto{\pgfqpoint{1.732645in}{2.734618in}}%
\pgfpathlineto{\pgfqpoint{1.845597in}{2.745270in}}%
\pgfpathlineto{\pgfqpoint{1.830679in}{2.726090in}}%
\pgfpathlineto{\pgfqpoint{2.597268in}{2.572750in}}%
\pgfpathlineto{\pgfqpoint{2.593006in}{2.551438in}}%
\pgfusepath{fill}%
\end{pgfscope}%
\begin{pgfscope}%
\pgfpathrectangle{\pgfqpoint{0.800000in}{1.363959in}}{\pgfqpoint{3.968000in}{2.024082in}} %
\pgfusepath{clip}%
\pgfsetbuttcap%
\pgfsetroundjoin%
\definecolor{currentfill}{rgb}{0.993248,0.906157,0.143936}%
\pgfsetfillcolor{currentfill}%
\pgfsetlinewidth{0.000000pt}%
\definecolor{currentstroke}{rgb}{0.000000,0.000000,0.000000}%
\pgfsetstrokecolor{currentstroke}%
\pgfsetdash{}{0pt}%
\pgfpathmoveto{\pgfqpoint{3.598634in}{2.470568in}}%
\pgfpathlineto{\pgfqpoint{4.345151in}{2.309937in}}%
\pgfpathlineto{\pgfqpoint{4.339099in}{2.333471in}}%
\pgfpathlineto{\pgfqpoint{4.438478in}{2.278740in}}%
\pgfpathlineto{\pgfqpoint{4.325384in}{2.269729in}}%
\pgfpathlineto{\pgfqpoint{4.340579in}{2.288690in}}%
\pgfpathlineto{\pgfqpoint{3.594062in}{2.449320in}}%
\pgfpathlineto{\pgfqpoint{3.598634in}{2.470568in}}%
\pgfusepath{fill}%
\end{pgfscope}%
\begin{pgfscope}%
\pgfpathrectangle{\pgfqpoint{0.800000in}{1.363959in}}{\pgfqpoint{3.968000in}{2.024082in}} %
\pgfusepath{clip}%
\pgfsetbuttcap%
\pgfsetroundjoin%
\definecolor{currentfill}{rgb}{0.993248,0.906157,0.143936}%
\pgfsetfillcolor{currentfill}%
\pgfsetlinewidth{0.000000pt}%
\definecolor{currentstroke}{rgb}{0.000000,0.000000,0.000000}%
\pgfsetstrokecolor{currentstroke}%
\pgfsetdash{}{0pt}%
\pgfpathmoveto{\pgfqpoint{3.033847in}{3.173351in}}%
\pgfpathlineto{\pgfqpoint{3.810662in}{3.194401in}}%
\pgfpathlineto{\pgfqpoint{3.799210in}{3.215832in}}%
\pgfpathlineto{\pgfqpoint{3.908722in}{3.186187in}}%
\pgfpathlineto{\pgfqpoint{3.800976in}{3.150655in}}%
\pgfpathlineto{\pgfqpoint{3.811251in}{3.172675in}}%
\pgfpathlineto{\pgfqpoint{3.034436in}{3.151625in}}%
\pgfpathlineto{\pgfqpoint{3.033847in}{3.173351in}}%
\pgfusepath{fill}%
\end{pgfscope}%
\begin{pgfscope}%
\pgfpathrectangle{\pgfqpoint{0.800000in}{1.363959in}}{\pgfqpoint{3.968000in}{2.024082in}} %
\pgfusepath{clip}%
\pgfsetbuttcap%
\pgfsetroundjoin%
\definecolor{currentfill}{rgb}{0.140536,0.530132,0.555659}%
\pgfsetfillcolor{currentfill}%
\pgfsetlinewidth{0.000000pt}%
\definecolor{currentstroke}{rgb}{0.000000,0.000000,0.000000}%
\pgfsetstrokecolor{currentstroke}%
\pgfsetdash{}{0pt}%
\pgfpathmoveto{\pgfqpoint{2.956687in}{3.159454in}}%
\pgfpathlineto{\pgfqpoint{3.709847in}{3.130684in}}%
\pgfpathlineto{\pgfqpoint{3.699817in}{3.152816in}}%
\pgfpathlineto{\pgfqpoint{3.807162in}{3.116092in}}%
\pgfpathlineto{\pgfqpoint{3.697329in}{3.087663in}}%
\pgfpathlineto{\pgfqpoint{3.709017in}{3.108966in}}%
\pgfpathlineto{\pgfqpoint{2.955858in}{3.137736in}}%
\pgfpathlineto{\pgfqpoint{2.956687in}{3.159454in}}%
\pgfusepath{fill}%
\end{pgfscope}%
\begin{pgfscope}%
\pgfpathrectangle{\pgfqpoint{0.800000in}{1.363959in}}{\pgfqpoint{3.968000in}{2.024082in}} %
\pgfusepath{clip}%
\pgfsetbuttcap%
\pgfsetroundjoin%
\definecolor{currentfill}{rgb}{0.168126,0.459988,0.558082}%
\pgfsetfillcolor{currentfill}%
\pgfsetlinewidth{0.000000pt}%
\definecolor{currentstroke}{rgb}{0.000000,0.000000,0.000000}%
\pgfsetstrokecolor{currentstroke}%
\pgfsetdash{}{0pt}%
\pgfpathmoveto{\pgfqpoint{2.957615in}{3.159379in}}%
\pgfpathlineto{\pgfqpoint{3.705584in}{3.066278in}}%
\pgfpathlineto{\pgfqpoint{3.697485in}{3.089188in}}%
\pgfpathlineto{\pgfqpoint{3.801294in}{3.043415in}}%
\pgfpathlineto{\pgfqpoint{3.689431in}{3.024486in}}%
\pgfpathlineto{\pgfqpoint{3.702899in}{3.044711in}}%
\pgfpathlineto{\pgfqpoint{2.954930in}{3.137811in}}%
\pgfpathlineto{\pgfqpoint{2.957615in}{3.159379in}}%
\pgfusepath{fill}%
\end{pgfscope}%
\begin{pgfscope}%
\pgfpathrectangle{\pgfqpoint{0.800000in}{1.363959in}}{\pgfqpoint{3.968000in}{2.024082in}} %
\pgfusepath{clip}%
\pgfsetbuttcap%
\pgfsetroundjoin%
\definecolor{currentfill}{rgb}{0.993248,0.906157,0.143936}%
\pgfsetfillcolor{currentfill}%
\pgfsetlinewidth{0.000000pt}%
\definecolor{currentstroke}{rgb}{0.000000,0.000000,0.000000}%
\pgfsetstrokecolor{currentstroke}%
\pgfsetdash{}{0pt}%
\pgfpathmoveto{\pgfqpoint{2.011008in}{2.160569in}}%
\pgfpathlineto{\pgfqpoint{1.332451in}{2.137972in}}%
\pgfpathlineto{\pgfqpoint{1.344036in}{2.116612in}}%
\pgfpathlineto{\pgfqpoint{1.234343in}{2.145577in}}%
\pgfpathlineto{\pgfqpoint{1.341865in}{2.181777in}}%
\pgfpathlineto{\pgfqpoint{1.331728in}{2.159693in}}%
\pgfpathlineto{\pgfqpoint{2.010285in}{2.182291in}}%
\pgfpathlineto{\pgfqpoint{2.011008in}{2.160569in}}%
\pgfusepath{fill}%
\end{pgfscope}%
\begin{pgfscope}%
\pgfpathrectangle{\pgfqpoint{0.800000in}{1.363959in}}{\pgfqpoint{3.968000in}{2.024082in}} %
\pgfusepath{clip}%
\pgfsetbuttcap%
\pgfsetroundjoin%
\definecolor{currentfill}{rgb}{0.993248,0.906157,0.143936}%
\pgfsetfillcolor{currentfill}%
\pgfsetlinewidth{0.000000pt}%
\definecolor{currentstroke}{rgb}{0.000000,0.000000,0.000000}%
\pgfsetstrokecolor{currentstroke}%
\pgfsetdash{}{0pt}%
\pgfpathmoveto{\pgfqpoint{2.636359in}{1.770197in}}%
\pgfpathlineto{\pgfqpoint{3.646855in}{1.721896in}}%
\pgfpathlineto{\pgfqpoint{3.637038in}{1.744123in}}%
\pgfpathlineto{\pgfqpoint{3.744026in}{1.706372in}}%
\pgfpathlineto{\pgfqpoint{3.633925in}{1.678997in}}%
\pgfpathlineto{\pgfqpoint{3.645817in}{1.700187in}}%
\pgfpathlineto{\pgfqpoint{2.635321in}{1.748488in}}%
\pgfpathlineto{\pgfqpoint{2.636359in}{1.770197in}}%
\pgfusepath{fill}%
\end{pgfscope}%
\begin{pgfscope}%
\pgfpathrectangle{\pgfqpoint{0.800000in}{1.363959in}}{\pgfqpoint{3.968000in}{2.024082in}} %
\pgfusepath{clip}%
\pgfsetbuttcap%
\pgfsetroundjoin%
\definecolor{currentfill}{rgb}{0.993248,0.906157,0.143936}%
\pgfsetfillcolor{currentfill}%
\pgfsetlinewidth{0.000000pt}%
\definecolor{currentstroke}{rgb}{0.000000,0.000000,0.000000}%
\pgfsetstrokecolor{currentstroke}%
\pgfsetdash{}{0pt}%
\pgfpathmoveto{\pgfqpoint{2.876475in}{1.681326in}}%
\pgfpathlineto{\pgfqpoint{3.700729in}{1.606498in}}%
\pgfpathlineto{\pgfqpoint{3.691872in}{1.629126in}}%
\pgfpathlineto{\pgfqpoint{3.797148in}{1.586834in}}%
\pgfpathlineto{\pgfqpoint{3.685977in}{1.564192in}}%
\pgfpathlineto{\pgfqpoint{3.698764in}{1.584854in}}%
\pgfpathlineto{\pgfqpoint{2.874510in}{1.659682in}}%
\pgfpathlineto{\pgfqpoint{2.876475in}{1.681326in}}%
\pgfusepath{fill}%
\end{pgfscope}%
\begin{pgfscope}%
\pgfpathrectangle{\pgfqpoint{0.800000in}{1.363959in}}{\pgfqpoint{3.968000in}{2.024082in}} %
\pgfusepath{clip}%
\pgfsetbuttcap%
\pgfsetroundjoin%
\definecolor{currentfill}{rgb}{0.993248,0.906157,0.143936}%
\pgfsetfillcolor{currentfill}%
\pgfsetlinewidth{0.000000pt}%
\definecolor{currentstroke}{rgb}{0.000000,0.000000,0.000000}%
\pgfsetstrokecolor{currentstroke}%
\pgfsetdash{}{0pt}%
\pgfpathmoveto{\pgfqpoint{2.558450in}{1.675971in}}%
\pgfpathlineto{\pgfqpoint{1.833046in}{1.640236in}}%
\pgfpathlineto{\pgfqpoint{1.844969in}{1.619064in}}%
\pgfpathlineto{\pgfqpoint{1.734828in}{1.646278in}}%
\pgfpathlineto{\pgfqpoint{1.841761in}{1.684186in}}%
\pgfpathlineto{\pgfqpoint{1.831976in}{1.661944in}}%
\pgfpathlineto{\pgfqpoint{2.557381in}{1.697679in}}%
\pgfpathlineto{\pgfqpoint{2.558450in}{1.675971in}}%
\pgfusepath{fill}%
\end{pgfscope}%
\begin{pgfscope}%
\pgfpathrectangle{\pgfqpoint{0.800000in}{1.363959in}}{\pgfqpoint{3.968000in}{2.024082in}} %
\pgfusepath{clip}%
\pgfsetbuttcap%
\pgfsetroundjoin%
\definecolor{currentfill}{rgb}{0.993248,0.906157,0.143936}%
\pgfsetfillcolor{currentfill}%
\pgfsetlinewidth{0.000000pt}%
\definecolor{currentstroke}{rgb}{0.000000,0.000000,0.000000}%
\pgfsetstrokecolor{currentstroke}%
\pgfsetdash{}{0pt}%
\pgfpathmoveto{\pgfqpoint{2.238589in}{2.367217in}}%
\pgfpathlineto{\pgfqpoint{1.602038in}{2.376077in}}%
\pgfpathlineto{\pgfqpoint{1.612601in}{2.354194in}}%
\pgfpathlineto{\pgfqpoint{1.504398in}{2.388303in}}%
\pgfpathlineto{\pgfqpoint{1.613509in}{2.419388in}}%
\pgfpathlineto{\pgfqpoint{1.602341in}{2.397808in}}%
\pgfpathlineto{\pgfqpoint{2.238891in}{2.388949in}}%
\pgfpathlineto{\pgfqpoint{2.238589in}{2.367217in}}%
\pgfusepath{fill}%
\end{pgfscope}%
\begin{pgfscope}%
\pgfpathrectangle{\pgfqpoint{0.800000in}{1.363959in}}{\pgfqpoint{3.968000in}{2.024082in}} %
\pgfusepath{clip}%
\pgfsetbuttcap%
\pgfsetroundjoin%
\definecolor{currentfill}{rgb}{0.993248,0.906157,0.143936}%
\pgfsetfillcolor{currentfill}%
\pgfsetlinewidth{0.000000pt}%
\definecolor{currentstroke}{rgb}{0.000000,0.000000,0.000000}%
\pgfsetstrokecolor{currentstroke}%
\pgfsetdash{}{0pt}%
\pgfpathmoveto{\pgfqpoint{2.865687in}{3.213950in}}%
\pgfpathlineto{\pgfqpoint{3.599641in}{3.283093in}}%
\pgfpathlineto{\pgfqpoint{3.586784in}{3.303711in}}%
\pgfpathlineto{\pgfqpoint{3.698030in}{3.281447in}}%
\pgfpathlineto{\pgfqpoint{3.592899in}{3.238798in}}%
\pgfpathlineto{\pgfqpoint{3.601679in}{3.261455in}}%
\pgfpathlineto{\pgfqpoint{2.867726in}{3.192312in}}%
\pgfpathlineto{\pgfqpoint{2.865687in}{3.213950in}}%
\pgfusepath{fill}%
\end{pgfscope}%
\begin{pgfscope}%
\pgfpathrectangle{\pgfqpoint{0.800000in}{1.363959in}}{\pgfqpoint{3.968000in}{2.024082in}} %
\pgfusepath{clip}%
\pgfsetbuttcap%
\pgfsetroundjoin%
\definecolor{currentfill}{rgb}{0.804182,0.882046,0.114965}%
\pgfsetfillcolor{currentfill}%
\pgfsetlinewidth{0.000000pt}%
\definecolor{currentstroke}{rgb}{0.000000,0.000000,0.000000}%
\pgfsetstrokecolor{currentstroke}%
\pgfsetdash{}{0pt}%
\pgfpathmoveto{\pgfqpoint{2.947439in}{1.473735in}}%
\pgfpathlineto{\pgfqpoint{3.867362in}{1.502912in}}%
\pgfpathlineto{\pgfqpoint{3.855811in}{1.524291in}}%
\pgfpathlineto{\pgfqpoint{3.965458in}{1.495151in}}%
\pgfpathlineto{\pgfqpoint{3.857878in}{1.459122in}}%
\pgfpathlineto{\pgfqpoint{3.868051in}{1.481190in}}%
\pgfpathlineto{\pgfqpoint{2.948128in}{1.452012in}}%
\pgfpathlineto{\pgfqpoint{2.947439in}{1.473735in}}%
\pgfusepath{fill}%
\end{pgfscope}%
\begin{pgfscope}%
\pgfpathrectangle{\pgfqpoint{0.800000in}{1.363959in}}{\pgfqpoint{3.968000in}{2.024082in}} %
\pgfusepath{clip}%
\pgfsetbuttcap%
\pgfsetroundjoin%
\definecolor{currentfill}{rgb}{0.993248,0.906157,0.143936}%
\pgfsetfillcolor{currentfill}%
\pgfsetlinewidth{0.000000pt}%
\definecolor{currentstroke}{rgb}{0.000000,0.000000,0.000000}%
\pgfsetstrokecolor{currentstroke}%
\pgfsetdash{}{0pt}%
\pgfpathmoveto{\pgfqpoint{2.406288in}{1.752662in}}%
\pgfpathlineto{\pgfqpoint{1.664512in}{1.726619in}}%
\pgfpathlineto{\pgfqpoint{1.676135in}{1.705280in}}%
\pgfpathlineto{\pgfqpoint{1.566390in}{1.734048in}}%
\pgfpathlineto{\pgfqpoint{1.673847in}{1.770441in}}%
\pgfpathlineto{\pgfqpoint{1.663750in}{1.748339in}}%
\pgfpathlineto{\pgfqpoint{2.405525in}{1.774382in}}%
\pgfpathlineto{\pgfqpoint{2.406288in}{1.752662in}}%
\pgfusepath{fill}%
\end{pgfscope}%
\begin{pgfscope}%
\pgfpathrectangle{\pgfqpoint{0.800000in}{1.363959in}}{\pgfqpoint{3.968000in}{2.024082in}} %
\pgfusepath{clip}%
\pgfsetbuttcap%
\pgfsetroundjoin%
\definecolor{currentfill}{rgb}{0.157729,0.485932,0.558013}%
\pgfsetfillcolor{currentfill}%
\pgfsetlinewidth{0.000000pt}%
\definecolor{currentstroke}{rgb}{0.000000,0.000000,0.000000}%
\pgfsetstrokecolor{currentstroke}%
\pgfsetdash{}{0pt}%
\pgfpathmoveto{\pgfqpoint{2.780376in}{2.065113in}}%
\pgfpathlineto{\pgfqpoint{3.792351in}{2.006027in}}%
\pgfpathlineto{\pgfqpoint{3.782769in}{2.028357in}}%
\pgfpathlineto{\pgfqpoint{3.889352in}{1.989478in}}%
\pgfpathlineto{\pgfqpoint{3.778969in}{1.963267in}}%
\pgfpathlineto{\pgfqpoint{3.791084in}{1.984330in}}%
\pgfpathlineto{\pgfqpoint{2.779109in}{2.043417in}}%
\pgfpathlineto{\pgfqpoint{2.780376in}{2.065113in}}%
\pgfusepath{fill}%
\end{pgfscope}%
\begin{pgfscope}%
\pgfpathrectangle{\pgfqpoint{0.800000in}{1.363959in}}{\pgfqpoint{3.968000in}{2.024082in}} %
\pgfusepath{clip}%
\pgfsetbuttcap%
\pgfsetroundjoin%
\definecolor{currentfill}{rgb}{0.151918,0.500685,0.557587}%
\pgfsetfillcolor{currentfill}%
\pgfsetlinewidth{0.000000pt}%
\definecolor{currentstroke}{rgb}{0.000000,0.000000,0.000000}%
\pgfsetstrokecolor{currentstroke}%
\pgfsetdash{}{0pt}%
\pgfpathmoveto{\pgfqpoint{2.780954in}{2.065064in}}%
\pgfpathlineto{\pgfqpoint{3.771588in}{1.953929in}}%
\pgfpathlineto{\pgfqpoint{3.763212in}{1.976739in}}%
\pgfpathlineto{\pgfqpoint{3.867568in}{1.932226in}}%
\pgfpathlineto{\pgfqpoint{3.755943in}{1.911944in}}%
\pgfpathlineto{\pgfqpoint{3.769165in}{1.932331in}}%
\pgfpathlineto{\pgfqpoint{2.778531in}{2.043466in}}%
\pgfpathlineto{\pgfqpoint{2.780954in}{2.065064in}}%
\pgfusepath{fill}%
\end{pgfscope}%
\begin{pgfscope}%
\pgfpathrectangle{\pgfqpoint{0.800000in}{1.363959in}}{\pgfqpoint{3.968000in}{2.024082in}} %
\pgfusepath{clip}%
\pgfsetbuttcap%
\pgfsetroundjoin%
\definecolor{currentfill}{rgb}{0.168126,0.459988,0.558082}%
\pgfsetfillcolor{currentfill}%
\pgfsetlinewidth{0.000000pt}%
\definecolor{currentstroke}{rgb}{0.000000,0.000000,0.000000}%
\pgfsetstrokecolor{currentstroke}%
\pgfsetdash{}{0pt}%
\pgfpathmoveto{\pgfqpoint{2.463251in}{2.933957in}}%
\pgfpathlineto{\pgfqpoint{1.553970in}{2.915622in}}%
\pgfpathlineto{\pgfqpoint{1.565273in}{2.894112in}}%
\pgfpathlineto{\pgfqpoint{1.455969in}{2.924515in}}%
\pgfpathlineto{\pgfqpoint{1.563958in}{2.959300in}}%
\pgfpathlineto{\pgfqpoint{1.553532in}{2.937351in}}%
\pgfpathlineto{\pgfqpoint{2.462813in}{2.955686in}}%
\pgfpathlineto{\pgfqpoint{2.463251in}{2.933957in}}%
\pgfusepath{fill}%
\end{pgfscope}%
\begin{pgfscope}%
\pgfpathrectangle{\pgfqpoint{0.800000in}{1.363959in}}{\pgfqpoint{3.968000in}{2.024082in}} %
\pgfusepath{clip}%
\pgfsetbuttcap%
\pgfsetroundjoin%
\definecolor{currentfill}{rgb}{0.140536,0.530132,0.555659}%
\pgfsetfillcolor{currentfill}%
\pgfsetlinewidth{0.000000pt}%
\definecolor{currentstroke}{rgb}{0.000000,0.000000,0.000000}%
\pgfsetstrokecolor{currentstroke}%
\pgfsetdash{}{0pt}%
\pgfpathmoveto{\pgfqpoint{2.462210in}{2.933986in}}%
\pgfpathlineto{\pgfqpoint{1.665596in}{2.994413in}}%
\pgfpathlineto{\pgfqpoint{1.674788in}{2.971920in}}%
\pgfpathlineto{\pgfqpoint{1.568897in}{3.012646in}}%
\pgfpathlineto{\pgfqpoint{1.679719in}{3.036934in}}%
\pgfpathlineto{\pgfqpoint{1.667240in}{3.016085in}}%
\pgfpathlineto{\pgfqpoint{2.463854in}{2.955657in}}%
\pgfpathlineto{\pgfqpoint{2.462210in}{2.933986in}}%
\pgfusepath{fill}%
\end{pgfscope}%
\begin{pgfscope}%
\pgfpathrectangle{\pgfqpoint{0.800000in}{1.363959in}}{\pgfqpoint{3.968000in}{2.024082in}} %
\pgfusepath{clip}%
\pgfsetbuttcap%
\pgfsetroundjoin%
\definecolor{currentfill}{rgb}{0.993248,0.906157,0.143936}%
\pgfsetfillcolor{currentfill}%
\pgfsetlinewidth{0.000000pt}%
\definecolor{currentstroke}{rgb}{0.000000,0.000000,0.000000}%
\pgfsetstrokecolor{currentstroke}%
\pgfsetdash{}{0pt}%
\pgfpathmoveto{\pgfqpoint{3.493252in}{2.088598in}}%
\pgfpathlineto{\pgfqpoint{4.228158in}{1.955928in}}%
\pgfpathlineto{\pgfqpoint{4.221325in}{1.979246in}}%
\pgfpathlineto{\pgfqpoint{4.322473in}{1.927859in}}%
\pgfpathlineto{\pgfqpoint{4.209742in}{1.915083in}}%
\pgfpathlineto{\pgfqpoint{4.224297in}{1.934540in}}%
\pgfpathlineto{\pgfqpoint{3.489391in}{2.067210in}}%
\pgfpathlineto{\pgfqpoint{3.493252in}{2.088598in}}%
\pgfusepath{fill}%
\end{pgfscope}%
\begin{pgfscope}%
\pgfpathrectangle{\pgfqpoint{0.800000in}{1.363959in}}{\pgfqpoint{3.968000in}{2.024082in}} %
\pgfusepath{clip}%
\pgfsetbuttcap%
\pgfsetroundjoin%
\definecolor{currentfill}{rgb}{0.647257,0.858400,0.209861}%
\pgfsetfillcolor{currentfill}%
\pgfsetlinewidth{0.000000pt}%
\definecolor{currentstroke}{rgb}{0.000000,0.000000,0.000000}%
\pgfsetstrokecolor{currentstroke}%
\pgfsetdash{}{0pt}%
\pgfpathmoveto{\pgfqpoint{2.959918in}{3.199318in}}%
\pgfpathlineto{\pgfqpoint{3.792002in}{3.157862in}}%
\pgfpathlineto{\pgfqpoint{3.782230in}{3.180110in}}%
\pgfpathlineto{\pgfqpoint{3.889141in}{3.142143in}}%
\pgfpathlineto{\pgfqpoint{3.778985in}{3.114990in}}%
\pgfpathlineto{\pgfqpoint{3.790920in}{3.136156in}}%
\pgfpathlineto{\pgfqpoint{2.958836in}{3.177611in}}%
\pgfpathlineto{\pgfqpoint{2.959918in}{3.199318in}}%
\pgfusepath{fill}%
\end{pgfscope}%
\begin{pgfscope}%
\pgfpathrectangle{\pgfqpoint{0.800000in}{1.363959in}}{\pgfqpoint{3.968000in}{2.024082in}} %
\pgfusepath{clip}%
\pgfsetbuttcap%
\pgfsetroundjoin%
\definecolor{currentfill}{rgb}{0.327796,0.773980,0.406640}%
\pgfsetfillcolor{currentfill}%
\pgfsetlinewidth{0.000000pt}%
\definecolor{currentstroke}{rgb}{0.000000,0.000000,0.000000}%
\pgfsetstrokecolor{currentstroke}%
\pgfsetdash{}{0pt}%
\pgfpathmoveto{\pgfqpoint{1.828920in}{2.510004in}}%
\pgfpathlineto{\pgfqpoint{1.168092in}{2.333736in}}%
\pgfpathlineto{\pgfqpoint{1.184193in}{2.315537in}}%
\pgfpathlineto{\pgfqpoint{1.070793in}{2.319030in}}%
\pgfpathlineto{\pgfqpoint{1.167389in}{2.378536in}}%
\pgfpathlineto{\pgfqpoint{1.162490in}{2.354736in}}%
\pgfpathlineto{\pgfqpoint{1.823319in}{2.531004in}}%
\pgfpathlineto{\pgfqpoint{1.828920in}{2.510004in}}%
\pgfusepath{fill}%
\end{pgfscope}%
\begin{pgfscope}%
\pgfpathrectangle{\pgfqpoint{0.800000in}{1.363959in}}{\pgfqpoint{3.968000in}{2.024082in}} %
\pgfusepath{clip}%
\pgfsetbuttcap%
\pgfsetroundjoin%
\definecolor{currentfill}{rgb}{0.280255,0.165693,0.476498}%
\pgfsetfillcolor{currentfill}%
\pgfsetlinewidth{0.000000pt}%
\definecolor{currentstroke}{rgb}{0.000000,0.000000,0.000000}%
\pgfsetstrokecolor{currentstroke}%
\pgfsetdash{}{0pt}%
\pgfpathmoveto{\pgfqpoint{1.820603in}{2.511141in}}%
\pgfpathlineto{\pgfqpoint{1.227476in}{2.860588in}}%
\pgfpathlineto{\pgfqpoint{1.225806in}{2.836346in}}%
\pgfpathlineto{\pgfqpoint{1.148727in}{2.919596in}}%
\pgfpathlineto{\pgfqpoint{1.258903in}{2.892522in}}%
\pgfpathlineto{\pgfqpoint{1.238508in}{2.879313in}}%
\pgfpathlineto{\pgfqpoint{1.831636in}{2.529867in}}%
\pgfpathlineto{\pgfqpoint{1.820603in}{2.511141in}}%
\pgfusepath{fill}%
\end{pgfscope}%
\begin{pgfscope}%
\pgfpathrectangle{\pgfqpoint{0.800000in}{1.363959in}}{\pgfqpoint{3.968000in}{2.024082in}} %
\pgfusepath{clip}%
\pgfsetbuttcap%
\pgfsetroundjoin%
\definecolor{currentfill}{rgb}{0.280255,0.165693,0.476498}%
\pgfsetfillcolor{currentfill}%
\pgfsetlinewidth{0.000000pt}%
\definecolor{currentstroke}{rgb}{0.000000,0.000000,0.000000}%
\pgfsetstrokecolor{currentstroke}%
\pgfsetdash{}{0pt}%
\pgfpathmoveto{\pgfqpoint{3.084857in}{3.021418in}}%
\pgfpathlineto{\pgfqpoint{3.703097in}{3.049787in}}%
\pgfpathlineto{\pgfqpoint{3.691246in}{3.071000in}}%
\pgfpathlineto{\pgfqpoint{3.801294in}{3.043415in}}%
\pgfpathlineto{\pgfqpoint{3.694235in}{3.005867in}}%
\pgfpathlineto{\pgfqpoint{3.704094in}{3.028076in}}%
\pgfpathlineto{\pgfqpoint{3.085853in}{2.999707in}}%
\pgfpathlineto{\pgfqpoint{3.084857in}{3.021418in}}%
\pgfusepath{fill}%
\end{pgfscope}%
\begin{pgfscope}%
\pgfpathrectangle{\pgfqpoint{0.800000in}{1.363959in}}{\pgfqpoint{3.968000in}{2.024082in}} %
\pgfusepath{clip}%
\pgfsetbuttcap%
\pgfsetroundjoin%
\definecolor{currentfill}{rgb}{0.327796,0.773980,0.406640}%
\pgfsetfillcolor{currentfill}%
\pgfsetlinewidth{0.000000pt}%
\definecolor{currentstroke}{rgb}{0.000000,0.000000,0.000000}%
\pgfsetstrokecolor{currentstroke}%
\pgfsetdash{}{0pt}%
\pgfpathmoveto{\pgfqpoint{3.085375in}{3.021430in}}%
\pgfpathlineto{\pgfqpoint{3.895808in}{3.019942in}}%
\pgfpathlineto{\pgfqpoint{3.884981in}{3.041695in}}%
\pgfpathlineto{\pgfqpoint{3.993589in}{3.008895in}}%
\pgfpathlineto{\pgfqpoint{3.884862in}{2.976494in}}%
\pgfpathlineto{\pgfqpoint{3.895768in}{2.998208in}}%
\pgfpathlineto{\pgfqpoint{3.085335in}{2.999696in}}%
\pgfpathlineto{\pgfqpoint{3.085375in}{3.021430in}}%
\pgfusepath{fill}%
\end{pgfscope}%
\begin{pgfscope}%
\pgfpathrectangle{\pgfqpoint{0.800000in}{1.363959in}}{\pgfqpoint{3.968000in}{2.024082in}} %
\pgfusepath{clip}%
\pgfsetbuttcap%
\pgfsetroundjoin%
\definecolor{currentfill}{rgb}{0.187231,0.414746,0.556547}%
\pgfsetfillcolor{currentfill}%
\pgfsetlinewidth{0.000000pt}%
\definecolor{currentstroke}{rgb}{0.000000,0.000000,0.000000}%
\pgfsetstrokecolor{currentstroke}%
\pgfsetdash{}{0pt}%
\pgfpathmoveto{\pgfqpoint{2.117946in}{2.281819in}}%
\pgfpathlineto{\pgfqpoint{1.432353in}{2.290179in}}%
\pgfpathlineto{\pgfqpoint{1.442954in}{2.268314in}}%
\pgfpathlineto{\pgfqpoint{1.334691in}{2.302237in}}%
\pgfpathlineto{\pgfqpoint{1.443749in}{2.333510in}}%
\pgfpathlineto{\pgfqpoint{1.432618in}{2.311911in}}%
\pgfpathlineto{\pgfqpoint{2.118211in}{2.303551in}}%
\pgfpathlineto{\pgfqpoint{2.117946in}{2.281819in}}%
\pgfusepath{fill}%
\end{pgfscope}%
\begin{pgfscope}%
\pgfpathrectangle{\pgfqpoint{0.800000in}{1.363959in}}{\pgfqpoint{3.968000in}{2.024082in}} %
\pgfusepath{clip}%
\pgfsetbuttcap%
\pgfsetroundjoin%
\definecolor{currentfill}{rgb}{0.126453,0.570633,0.549841}%
\pgfsetfillcolor{currentfill}%
\pgfsetlinewidth{0.000000pt}%
\definecolor{currentstroke}{rgb}{0.000000,0.000000,0.000000}%
\pgfsetstrokecolor{currentstroke}%
\pgfsetdash{}{0pt}%
\pgfpathmoveto{\pgfqpoint{2.118770in}{2.281840in}}%
\pgfpathlineto{\pgfqpoint{1.446661in}{2.239022in}}%
\pgfpathlineto{\pgfqpoint{1.458888in}{2.218023in}}%
\pgfpathlineto{\pgfqpoint{1.348367in}{2.243649in}}%
\pgfpathlineto{\pgfqpoint{1.454743in}{2.283092in}}%
\pgfpathlineto{\pgfqpoint{1.445280in}{2.260712in}}%
\pgfpathlineto{\pgfqpoint{2.117388in}{2.303529in}}%
\pgfpathlineto{\pgfqpoint{2.118770in}{2.281840in}}%
\pgfusepath{fill}%
\end{pgfscope}%
\begin{pgfscope}%
\pgfpathrectangle{\pgfqpoint{0.800000in}{1.363959in}}{\pgfqpoint{3.968000in}{2.024082in}} %
\pgfusepath{clip}%
\pgfsetbuttcap%
\pgfsetroundjoin%
\definecolor{currentfill}{rgb}{0.993248,0.906157,0.143936}%
\pgfsetfillcolor{currentfill}%
\pgfsetlinewidth{0.000000pt}%
\definecolor{currentstroke}{rgb}{0.000000,0.000000,0.000000}%
\pgfsetstrokecolor{currentstroke}%
\pgfsetdash{}{0pt}%
\pgfpathmoveto{\pgfqpoint{2.052525in}{2.443107in}}%
\pgfpathlineto{\pgfqpoint{1.305095in}{2.435210in}}%
\pgfpathlineto{\pgfqpoint{1.316190in}{2.413592in}}%
\pgfpathlineto{\pgfqpoint{1.207184in}{2.445043in}}%
\pgfpathlineto{\pgfqpoint{1.315502in}{2.478789in}}%
\pgfpathlineto{\pgfqpoint{1.304865in}{2.456942in}}%
\pgfpathlineto{\pgfqpoint{2.052296in}{2.464839in}}%
\pgfpathlineto{\pgfqpoint{2.052525in}{2.443107in}}%
\pgfusepath{fill}%
\end{pgfscope}%
\begin{pgfscope}%
\pgfpathrectangle{\pgfqpoint{0.800000in}{1.363959in}}{\pgfqpoint{3.968000in}{2.024082in}} %
\pgfusepath{clip}%
\pgfsetbuttcap%
\pgfsetroundjoin%
\definecolor{currentfill}{rgb}{0.239374,0.735588,0.455688}%
\pgfsetfillcolor{currentfill}%
\pgfsetlinewidth{0.000000pt}%
\definecolor{currentstroke}{rgb}{0.000000,0.000000,0.000000}%
\pgfsetstrokecolor{currentstroke}%
\pgfsetdash{}{0pt}%
\pgfpathmoveto{\pgfqpoint{3.096735in}{1.978679in}}%
\pgfpathlineto{\pgfqpoint{4.128141in}{1.946907in}}%
\pgfpathlineto{\pgfqpoint{4.117949in}{1.968965in}}%
\pgfpathlineto{\pgfqpoint{4.225562in}{1.933035in}}%
\pgfpathlineto{\pgfqpoint{4.115941in}{1.903795in}}%
\pgfpathlineto{\pgfqpoint{4.127472in}{1.925184in}}%
\pgfpathlineto{\pgfqpoint{3.096066in}{1.956955in}}%
\pgfpathlineto{\pgfqpoint{3.096735in}{1.978679in}}%
\pgfusepath{fill}%
\end{pgfscope}%
\begin{pgfscope}%
\pgfpathrectangle{\pgfqpoint{0.800000in}{1.363959in}}{\pgfqpoint{3.968000in}{2.024082in}} %
\pgfusepath{clip}%
\pgfsetbuttcap%
\pgfsetroundjoin%
\definecolor{currentfill}{rgb}{0.267968,0.223549,0.512008}%
\pgfsetfillcolor{currentfill}%
\pgfsetlinewidth{0.000000pt}%
\definecolor{currentstroke}{rgb}{0.000000,0.000000,0.000000}%
\pgfsetstrokecolor{currentstroke}%
\pgfsetdash{}{0pt}%
\pgfpathmoveto{\pgfqpoint{3.095806in}{1.978668in}}%
\pgfpathlineto{\pgfqpoint{3.845916in}{2.019712in}}%
\pgfpathlineto{\pgfqpoint{3.833878in}{2.040819in}}%
\pgfpathlineto{\pgfqpoint{3.944165in}{2.014205in}}%
\pgfpathlineto{\pgfqpoint{3.837440in}{1.975716in}}%
\pgfpathlineto{\pgfqpoint{3.847103in}{1.998011in}}%
\pgfpathlineto{\pgfqpoint{3.096994in}{1.956966in}}%
\pgfpathlineto{\pgfqpoint{3.095806in}{1.978668in}}%
\pgfusepath{fill}%
\end{pgfscope}%
\begin{pgfscope}%
\pgfpathrectangle{\pgfqpoint{0.800000in}{1.363959in}}{\pgfqpoint{3.968000in}{2.024082in}} %
\pgfusepath{clip}%
\pgfsetbuttcap%
\pgfsetroundjoin%
\definecolor{currentfill}{rgb}{0.993248,0.906157,0.143936}%
\pgfsetfillcolor{currentfill}%
\pgfsetlinewidth{0.000000pt}%
\definecolor{currentstroke}{rgb}{0.000000,0.000000,0.000000}%
\pgfsetstrokecolor{currentstroke}%
\pgfsetdash{}{0pt}%
\pgfpathmoveto{\pgfqpoint{1.921082in}{2.338560in}}%
\pgfpathlineto{\pgfqpoint{1.110866in}{2.236136in}}%
\pgfpathlineto{\pgfqpoint{1.124372in}{2.215937in}}%
\pgfpathlineto{\pgfqpoint{1.012474in}{2.234652in}}%
\pgfpathlineto{\pgfqpoint{1.116195in}{2.280623in}}%
\pgfpathlineto{\pgfqpoint{1.108140in}{2.257698in}}%
\pgfpathlineto{\pgfqpoint{1.918357in}{2.360122in}}%
\pgfpathlineto{\pgfqpoint{1.921082in}{2.338560in}}%
\pgfusepath{fill}%
\end{pgfscope}%
\begin{pgfscope}%
\pgfpathrectangle{\pgfqpoint{0.800000in}{1.363959in}}{\pgfqpoint{3.968000in}{2.024082in}} %
\pgfusepath{clip}%
\pgfsetbuttcap%
\pgfsetroundjoin%
\definecolor{currentfill}{rgb}{0.146180,0.515413,0.556823}%
\pgfsetfillcolor{currentfill}%
\pgfsetlinewidth{0.000000pt}%
\definecolor{currentstroke}{rgb}{0.000000,0.000000,0.000000}%
\pgfsetstrokecolor{currentstroke}%
\pgfsetdash{}{0pt}%
\pgfpathmoveto{\pgfqpoint{3.206951in}{2.563199in}}%
\pgfpathlineto{\pgfqpoint{4.121042in}{2.465145in}}%
\pgfpathlineto{\pgfqpoint{4.112555in}{2.487914in}}%
\pgfpathlineto{\pgfqpoint{4.217126in}{2.443909in}}%
\pgfpathlineto{\pgfqpoint{4.105601in}{2.423085in}}%
\pgfpathlineto{\pgfqpoint{4.118724in}{2.443536in}}%
\pgfpathlineto{\pgfqpoint{3.204633in}{2.541589in}}%
\pgfpathlineto{\pgfqpoint{3.206951in}{2.563199in}}%
\pgfusepath{fill}%
\end{pgfscope}%
\begin{pgfscope}%
\pgfpathrectangle{\pgfqpoint{0.800000in}{1.363959in}}{\pgfqpoint{3.968000in}{2.024082in}} %
\pgfusepath{clip}%
\pgfsetbuttcap%
\pgfsetroundjoin%
\definecolor{currentfill}{rgb}{0.162142,0.474838,0.558140}%
\pgfsetfillcolor{currentfill}%
\pgfsetlinewidth{0.000000pt}%
\definecolor{currentstroke}{rgb}{0.000000,0.000000,0.000000}%
\pgfsetstrokecolor{currentstroke}%
\pgfsetdash{}{0pt}%
\pgfpathmoveto{\pgfqpoint{3.205280in}{2.563249in}}%
\pgfpathlineto{\pgfqpoint{4.032910in}{2.602265in}}%
\pgfpathlineto{\pgfqpoint{4.021031in}{2.623463in}}%
\pgfpathlineto{\pgfqpoint{4.131114in}{2.596016in}}%
\pgfpathlineto{\pgfqpoint{4.024102in}{2.558334in}}%
\pgfpathlineto{\pgfqpoint{4.033933in}{2.580555in}}%
\pgfpathlineto{\pgfqpoint{3.206304in}{2.541539in}}%
\pgfpathlineto{\pgfqpoint{3.205280in}{2.563249in}}%
\pgfusepath{fill}%
\end{pgfscope}%
\begin{pgfscope}%
\pgfpathrectangle{\pgfqpoint{0.800000in}{1.363959in}}{\pgfqpoint{3.968000in}{2.024082in}} %
\pgfusepath{clip}%
\pgfsetbuttcap%
\pgfsetroundjoin%
\definecolor{currentfill}{rgb}{0.267004,0.004874,0.329415}%
\pgfsetfillcolor{currentfill}%
\pgfsetlinewidth{0.000000pt}%
\definecolor{currentstroke}{rgb}{0.000000,0.000000,0.000000}%
\pgfsetstrokecolor{currentstroke}%
\pgfsetdash{}{0pt}%
\pgfpathmoveto{\pgfqpoint{3.518646in}{2.552757in}}%
\pgfpathlineto{\pgfqpoint{4.232837in}{2.533226in}}%
\pgfpathlineto{\pgfqpoint{4.222568in}{2.555248in}}%
\pgfpathlineto{\pgfqpoint{4.330305in}{2.519689in}}%
\pgfpathlineto{\pgfqpoint{4.220786in}{2.490072in}}%
\pgfpathlineto{\pgfqpoint{4.232243in}{2.511500in}}%
\pgfpathlineto{\pgfqpoint{3.518052in}{2.531031in}}%
\pgfpathlineto{\pgfqpoint{3.518646in}{2.552757in}}%
\pgfusepath{fill}%
\end{pgfscope}%
\begin{pgfscope}%
\pgfpathrectangle{\pgfqpoint{0.800000in}{1.363959in}}{\pgfqpoint{3.968000in}{2.024082in}} %
\pgfusepath{clip}%
\pgfsetbuttcap%
\pgfsetroundjoin%
\definecolor{currentfill}{rgb}{0.606045,0.850733,0.236712}%
\pgfsetfillcolor{currentfill}%
\pgfsetlinewidth{0.000000pt}%
\definecolor{currentstroke}{rgb}{0.000000,0.000000,0.000000}%
\pgfsetstrokecolor{currentstroke}%
\pgfsetdash{}{0pt}%
\pgfpathmoveto{\pgfqpoint{3.518350in}{2.552761in}}%
\pgfpathlineto{\pgfqpoint{4.447927in}{2.552678in}}%
\pgfpathlineto{\pgfqpoint{4.437063in}{2.574413in}}%
\pgfpathlineto{\pgfqpoint{4.545728in}{2.541803in}}%
\pgfpathlineto{\pgfqpoint{4.437057in}{2.509212in}}%
\pgfpathlineto{\pgfqpoint{4.447926in}{2.530945in}}%
\pgfpathlineto{\pgfqpoint{3.518348in}{2.531027in}}%
\pgfpathlineto{\pgfqpoint{3.518350in}{2.552761in}}%
\pgfusepath{fill}%
\end{pgfscope}%
\begin{pgfscope}%
\pgfpathrectangle{\pgfqpoint{0.800000in}{1.363959in}}{\pgfqpoint{3.968000in}{2.024082in}} %
\pgfusepath{clip}%
\pgfsetbuttcap%
\pgfsetroundjoin%
\definecolor{currentfill}{rgb}{0.168126,0.459988,0.558082}%
\pgfsetfillcolor{currentfill}%
\pgfsetlinewidth{0.000000pt}%
\definecolor{currentstroke}{rgb}{0.000000,0.000000,0.000000}%
\pgfsetstrokecolor{currentstroke}%
\pgfsetdash{}{0pt}%
\pgfpathmoveto{\pgfqpoint{3.125008in}{2.254609in}}%
\pgfpathlineto{\pgfqpoint{4.179284in}{2.139957in}}%
\pgfpathlineto{\pgfqpoint{4.170831in}{2.162738in}}%
\pgfpathlineto{\pgfqpoint{4.275337in}{2.118580in}}%
\pgfpathlineto{\pgfqpoint{4.163782in}{2.097919in}}%
\pgfpathlineto{\pgfqpoint{4.176934in}{2.118351in}}%
\pgfpathlineto{\pgfqpoint{3.122658in}{2.233003in}}%
\pgfpathlineto{\pgfqpoint{3.125008in}{2.254609in}}%
\pgfusepath{fill}%
\end{pgfscope}%
\begin{pgfscope}%
\pgfpathrectangle{\pgfqpoint{0.800000in}{1.363959in}}{\pgfqpoint{3.968000in}{2.024082in}} %
\pgfusepath{clip}%
\pgfsetbuttcap%
\pgfsetroundjoin%
\definecolor{currentfill}{rgb}{0.140536,0.530132,0.555659}%
\pgfsetfillcolor{currentfill}%
\pgfsetlinewidth{0.000000pt}%
\definecolor{currentstroke}{rgb}{0.000000,0.000000,0.000000}%
\pgfsetstrokecolor{currentstroke}%
\pgfsetdash{}{0pt}%
\pgfpathmoveto{\pgfqpoint{3.123938in}{2.254672in}}%
\pgfpathlineto{\pgfqpoint{3.999128in}{2.246186in}}%
\pgfpathlineto{\pgfqpoint{3.988473in}{2.268024in}}%
\pgfpathlineto{\pgfqpoint{4.096820in}{2.234372in}}%
\pgfpathlineto{\pgfqpoint{3.987840in}{2.202827in}}%
\pgfpathlineto{\pgfqpoint{3.998917in}{2.224454in}}%
\pgfpathlineto{\pgfqpoint{3.123727in}{2.232939in}}%
\pgfpathlineto{\pgfqpoint{3.123938in}{2.254672in}}%
\pgfusepath{fill}%
\end{pgfscope}%
\begin{pgfscope}%
\pgfpathrectangle{\pgfqpoint{0.800000in}{1.363959in}}{\pgfqpoint{3.968000in}{2.024082in}} %
\pgfusepath{clip}%
\pgfsetbuttcap%
\pgfsetroundjoin%
\definecolor{currentfill}{rgb}{0.993248,0.906157,0.143936}%
\pgfsetfillcolor{currentfill}%
\pgfsetlinewidth{0.000000pt}%
\definecolor{currentstroke}{rgb}{0.000000,0.000000,0.000000}%
\pgfsetstrokecolor{currentstroke}%
\pgfsetdash{}{0pt}%
\pgfpathmoveto{\pgfqpoint{2.025830in}{2.930293in}}%
\pgfpathlineto{\pgfqpoint{1.370553in}{3.090681in}}%
\pgfpathlineto{\pgfqpoint{1.375941in}{3.066987in}}%
\pgfpathlineto{\pgfqpoint{1.278139in}{3.124488in}}%
\pgfpathlineto{\pgfqpoint{1.391442in}{3.130318in}}%
\pgfpathlineto{\pgfqpoint{1.375720in}{3.111791in}}%
\pgfpathlineto{\pgfqpoint{2.030997in}{2.951404in}}%
\pgfpathlineto{\pgfqpoint{2.025830in}{2.930293in}}%
\pgfusepath{fill}%
\end{pgfscope}%
\begin{pgfscope}%
\pgfsetbuttcap%
\pgfsetroundjoin%
\definecolor{currentfill}{rgb}{0.000000,0.000000,0.000000}%
\pgfsetfillcolor{currentfill}%
\pgfsetlinewidth{0.803000pt}%
\definecolor{currentstroke}{rgb}{0.000000,0.000000,0.000000}%
\pgfsetstrokecolor{currentstroke}%
\pgfsetdash{}{0pt}%
\pgfsys@defobject{currentmarker}{\pgfqpoint{0.000000in}{-0.048611in}}{\pgfqpoint{0.000000in}{0.000000in}}{%
\pgfpathmoveto{\pgfqpoint{0.000000in}{0.000000in}}%
\pgfpathlineto{\pgfqpoint{0.000000in}{-0.048611in}}%
\pgfusepath{stroke,fill}%
}%
\begin{pgfscope}%
\pgfsys@transformshift{0.863945in}{1.363959in}%
\pgfsys@useobject{currentmarker}{}%
\end{pgfscope}%
\end{pgfscope}%
\begin{pgfscope}%
\pgftext[x=0.863945in,y=1.266737in,,top]{\rmfamily\fontsize{10.000000}{12.000000}\selectfont \(\displaystyle -2.0\)}%
\end{pgfscope}%
\begin{pgfscope}%
\pgfsetbuttcap%
\pgfsetroundjoin%
\definecolor{currentfill}{rgb}{0.000000,0.000000,0.000000}%
\pgfsetfillcolor{currentfill}%
\pgfsetlinewidth{0.803000pt}%
\definecolor{currentstroke}{rgb}{0.000000,0.000000,0.000000}%
\pgfsetstrokecolor{currentstroke}%
\pgfsetdash{}{0pt}%
\pgfsys@defobject{currentmarker}{\pgfqpoint{0.000000in}{-0.048611in}}{\pgfqpoint{0.000000in}{0.000000in}}{%
\pgfpathmoveto{\pgfqpoint{0.000000in}{0.000000in}}%
\pgfpathlineto{\pgfqpoint{0.000000in}{-0.048611in}}%
\pgfusepath{stroke,fill}%
}%
\begin{pgfscope}%
\pgfsys@transformshift{1.330276in}{1.363959in}%
\pgfsys@useobject{currentmarker}{}%
\end{pgfscope}%
\end{pgfscope}%
\begin{pgfscope}%
\pgftext[x=1.330276in,y=1.266737in,,top]{\rmfamily\fontsize{10.000000}{12.000000}\selectfont \(\displaystyle -1.5\)}%
\end{pgfscope}%
\begin{pgfscope}%
\pgfsetbuttcap%
\pgfsetroundjoin%
\definecolor{currentfill}{rgb}{0.000000,0.000000,0.000000}%
\pgfsetfillcolor{currentfill}%
\pgfsetlinewidth{0.803000pt}%
\definecolor{currentstroke}{rgb}{0.000000,0.000000,0.000000}%
\pgfsetstrokecolor{currentstroke}%
\pgfsetdash{}{0pt}%
\pgfsys@defobject{currentmarker}{\pgfqpoint{0.000000in}{-0.048611in}}{\pgfqpoint{0.000000in}{0.000000in}}{%
\pgfpathmoveto{\pgfqpoint{0.000000in}{0.000000in}}%
\pgfpathlineto{\pgfqpoint{0.000000in}{-0.048611in}}%
\pgfusepath{stroke,fill}%
}%
\begin{pgfscope}%
\pgfsys@transformshift{1.796608in}{1.363959in}%
\pgfsys@useobject{currentmarker}{}%
\end{pgfscope}%
\end{pgfscope}%
\begin{pgfscope}%
\pgftext[x=1.796608in,y=1.266737in,,top]{\rmfamily\fontsize{10.000000}{12.000000}\selectfont \(\displaystyle -1.0\)}%
\end{pgfscope}%
\begin{pgfscope}%
\pgfsetbuttcap%
\pgfsetroundjoin%
\definecolor{currentfill}{rgb}{0.000000,0.000000,0.000000}%
\pgfsetfillcolor{currentfill}%
\pgfsetlinewidth{0.803000pt}%
\definecolor{currentstroke}{rgb}{0.000000,0.000000,0.000000}%
\pgfsetstrokecolor{currentstroke}%
\pgfsetdash{}{0pt}%
\pgfsys@defobject{currentmarker}{\pgfqpoint{0.000000in}{-0.048611in}}{\pgfqpoint{0.000000in}{0.000000in}}{%
\pgfpathmoveto{\pgfqpoint{0.000000in}{0.000000in}}%
\pgfpathlineto{\pgfqpoint{0.000000in}{-0.048611in}}%
\pgfusepath{stroke,fill}%
}%
\begin{pgfscope}%
\pgfsys@transformshift{2.262939in}{1.363959in}%
\pgfsys@useobject{currentmarker}{}%
\end{pgfscope}%
\end{pgfscope}%
\begin{pgfscope}%
\pgftext[x=2.262939in,y=1.266737in,,top]{\rmfamily\fontsize{10.000000}{12.000000}\selectfont \(\displaystyle -0.5\)}%
\end{pgfscope}%
\begin{pgfscope}%
\pgfsetbuttcap%
\pgfsetroundjoin%
\definecolor{currentfill}{rgb}{0.000000,0.000000,0.000000}%
\pgfsetfillcolor{currentfill}%
\pgfsetlinewidth{0.803000pt}%
\definecolor{currentstroke}{rgb}{0.000000,0.000000,0.000000}%
\pgfsetstrokecolor{currentstroke}%
\pgfsetdash{}{0pt}%
\pgfsys@defobject{currentmarker}{\pgfqpoint{0.000000in}{-0.048611in}}{\pgfqpoint{0.000000in}{0.000000in}}{%
\pgfpathmoveto{\pgfqpoint{0.000000in}{0.000000in}}%
\pgfpathlineto{\pgfqpoint{0.000000in}{-0.048611in}}%
\pgfusepath{stroke,fill}%
}%
\begin{pgfscope}%
\pgfsys@transformshift{2.729270in}{1.363959in}%
\pgfsys@useobject{currentmarker}{}%
\end{pgfscope}%
\end{pgfscope}%
\begin{pgfscope}%
\pgftext[x=2.729270in,y=1.266737in,,top]{\rmfamily\fontsize{10.000000}{12.000000}\selectfont \(\displaystyle 0.0\)}%
\end{pgfscope}%
\begin{pgfscope}%
\pgfsetbuttcap%
\pgfsetroundjoin%
\definecolor{currentfill}{rgb}{0.000000,0.000000,0.000000}%
\pgfsetfillcolor{currentfill}%
\pgfsetlinewidth{0.803000pt}%
\definecolor{currentstroke}{rgb}{0.000000,0.000000,0.000000}%
\pgfsetstrokecolor{currentstroke}%
\pgfsetdash{}{0pt}%
\pgfsys@defobject{currentmarker}{\pgfqpoint{0.000000in}{-0.048611in}}{\pgfqpoint{0.000000in}{0.000000in}}{%
\pgfpathmoveto{\pgfqpoint{0.000000in}{0.000000in}}%
\pgfpathlineto{\pgfqpoint{0.000000in}{-0.048611in}}%
\pgfusepath{stroke,fill}%
}%
\begin{pgfscope}%
\pgfsys@transformshift{3.195602in}{1.363959in}%
\pgfsys@useobject{currentmarker}{}%
\end{pgfscope}%
\end{pgfscope}%
\begin{pgfscope}%
\pgftext[x=3.195602in,y=1.266737in,,top]{\rmfamily\fontsize{10.000000}{12.000000}\selectfont \(\displaystyle 0.5\)}%
\end{pgfscope}%
\begin{pgfscope}%
\pgfsetbuttcap%
\pgfsetroundjoin%
\definecolor{currentfill}{rgb}{0.000000,0.000000,0.000000}%
\pgfsetfillcolor{currentfill}%
\pgfsetlinewidth{0.803000pt}%
\definecolor{currentstroke}{rgb}{0.000000,0.000000,0.000000}%
\pgfsetstrokecolor{currentstroke}%
\pgfsetdash{}{0pt}%
\pgfsys@defobject{currentmarker}{\pgfqpoint{0.000000in}{-0.048611in}}{\pgfqpoint{0.000000in}{0.000000in}}{%
\pgfpathmoveto{\pgfqpoint{0.000000in}{0.000000in}}%
\pgfpathlineto{\pgfqpoint{0.000000in}{-0.048611in}}%
\pgfusepath{stroke,fill}%
}%
\begin{pgfscope}%
\pgfsys@transformshift{3.661933in}{1.363959in}%
\pgfsys@useobject{currentmarker}{}%
\end{pgfscope}%
\end{pgfscope}%
\begin{pgfscope}%
\pgftext[x=3.661933in,y=1.266737in,,top]{\rmfamily\fontsize{10.000000}{12.000000}\selectfont \(\displaystyle 1.0\)}%
\end{pgfscope}%
\begin{pgfscope}%
\pgfsetbuttcap%
\pgfsetroundjoin%
\definecolor{currentfill}{rgb}{0.000000,0.000000,0.000000}%
\pgfsetfillcolor{currentfill}%
\pgfsetlinewidth{0.803000pt}%
\definecolor{currentstroke}{rgb}{0.000000,0.000000,0.000000}%
\pgfsetstrokecolor{currentstroke}%
\pgfsetdash{}{0pt}%
\pgfsys@defobject{currentmarker}{\pgfqpoint{0.000000in}{-0.048611in}}{\pgfqpoint{0.000000in}{0.000000in}}{%
\pgfpathmoveto{\pgfqpoint{0.000000in}{0.000000in}}%
\pgfpathlineto{\pgfqpoint{0.000000in}{-0.048611in}}%
\pgfusepath{stroke,fill}%
}%
\begin{pgfscope}%
\pgfsys@transformshift{4.128264in}{1.363959in}%
\pgfsys@useobject{currentmarker}{}%
\end{pgfscope}%
\end{pgfscope}%
\begin{pgfscope}%
\pgftext[x=4.128264in,y=1.266737in,,top]{\rmfamily\fontsize{10.000000}{12.000000}\selectfont \(\displaystyle 1.5\)}%
\end{pgfscope}%
\begin{pgfscope}%
\pgfsetbuttcap%
\pgfsetroundjoin%
\definecolor{currentfill}{rgb}{0.000000,0.000000,0.000000}%
\pgfsetfillcolor{currentfill}%
\pgfsetlinewidth{0.803000pt}%
\definecolor{currentstroke}{rgb}{0.000000,0.000000,0.000000}%
\pgfsetstrokecolor{currentstroke}%
\pgfsetdash{}{0pt}%
\pgfsys@defobject{currentmarker}{\pgfqpoint{0.000000in}{-0.048611in}}{\pgfqpoint{0.000000in}{0.000000in}}{%
\pgfpathmoveto{\pgfqpoint{0.000000in}{0.000000in}}%
\pgfpathlineto{\pgfqpoint{0.000000in}{-0.048611in}}%
\pgfusepath{stroke,fill}%
}%
\begin{pgfscope}%
\pgfsys@transformshift{4.594595in}{1.363959in}%
\pgfsys@useobject{currentmarker}{}%
\end{pgfscope}%
\end{pgfscope}%
\begin{pgfscope}%
\pgftext[x=4.594595in,y=1.266737in,,top]{\rmfamily\fontsize{10.000000}{12.000000}\selectfont \(\displaystyle 2.0\)}%
\end{pgfscope}%
\begin{pgfscope}%
\pgfsetbuttcap%
\pgfsetroundjoin%
\definecolor{currentfill}{rgb}{0.000000,0.000000,0.000000}%
\pgfsetfillcolor{currentfill}%
\pgfsetlinewidth{0.803000pt}%
\definecolor{currentstroke}{rgb}{0.000000,0.000000,0.000000}%
\pgfsetstrokecolor{currentstroke}%
\pgfsetdash{}{0pt}%
\pgfsys@defobject{currentmarker}{\pgfqpoint{-0.048611in}{0.000000in}}{\pgfqpoint{0.000000in}{0.000000in}}{%
\pgfpathmoveto{\pgfqpoint{0.000000in}{0.000000in}}%
\pgfpathlineto{\pgfqpoint{-0.048611in}{0.000000in}}%
\pgfusepath{stroke,fill}%
}%
\begin{pgfscope}%
\pgfsys@transformshift{0.800000in}{1.434858in}%
\pgfsys@useobject{currentmarker}{}%
\end{pgfscope}%
\end{pgfscope}%
\begin{pgfscope}%
\pgftext[x=0.417283in,y=1.386664in,left,base]{\rmfamily\fontsize{10.000000}{12.000000}\selectfont \(\displaystyle -1.0\)}%
\end{pgfscope}%
\begin{pgfscope}%
\pgfsetbuttcap%
\pgfsetroundjoin%
\definecolor{currentfill}{rgb}{0.000000,0.000000,0.000000}%
\pgfsetfillcolor{currentfill}%
\pgfsetlinewidth{0.803000pt}%
\definecolor{currentstroke}{rgb}{0.000000,0.000000,0.000000}%
\pgfsetstrokecolor{currentstroke}%
\pgfsetdash{}{0pt}%
\pgfsys@defobject{currentmarker}{\pgfqpoint{-0.048611in}{0.000000in}}{\pgfqpoint{0.000000in}{0.000000in}}{%
\pgfpathmoveto{\pgfqpoint{0.000000in}{0.000000in}}%
\pgfpathlineto{\pgfqpoint{-0.048611in}{0.000000in}}%
\pgfusepath{stroke,fill}%
}%
\begin{pgfscope}%
\pgfsys@transformshift{0.800000in}{1.901190in}%
\pgfsys@useobject{currentmarker}{}%
\end{pgfscope}%
\end{pgfscope}%
\begin{pgfscope}%
\pgftext[x=0.417283in,y=1.852995in,left,base]{\rmfamily\fontsize{10.000000}{12.000000}\selectfont \(\displaystyle -0.5\)}%
\end{pgfscope}%
\begin{pgfscope}%
\pgfsetbuttcap%
\pgfsetroundjoin%
\definecolor{currentfill}{rgb}{0.000000,0.000000,0.000000}%
\pgfsetfillcolor{currentfill}%
\pgfsetlinewidth{0.803000pt}%
\definecolor{currentstroke}{rgb}{0.000000,0.000000,0.000000}%
\pgfsetstrokecolor{currentstroke}%
\pgfsetdash{}{0pt}%
\pgfsys@defobject{currentmarker}{\pgfqpoint{-0.048611in}{0.000000in}}{\pgfqpoint{0.000000in}{0.000000in}}{%
\pgfpathmoveto{\pgfqpoint{0.000000in}{0.000000in}}%
\pgfpathlineto{\pgfqpoint{-0.048611in}{0.000000in}}%
\pgfusepath{stroke,fill}%
}%
\begin{pgfscope}%
\pgfsys@transformshift{0.800000in}{2.367521in}%
\pgfsys@useobject{currentmarker}{}%
\end{pgfscope}%
\end{pgfscope}%
\begin{pgfscope}%
\pgftext[x=0.525308in,y=2.319327in,left,base]{\rmfamily\fontsize{10.000000}{12.000000}\selectfont \(\displaystyle 0.0\)}%
\end{pgfscope}%
\begin{pgfscope}%
\pgfsetbuttcap%
\pgfsetroundjoin%
\definecolor{currentfill}{rgb}{0.000000,0.000000,0.000000}%
\pgfsetfillcolor{currentfill}%
\pgfsetlinewidth{0.803000pt}%
\definecolor{currentstroke}{rgb}{0.000000,0.000000,0.000000}%
\pgfsetstrokecolor{currentstroke}%
\pgfsetdash{}{0pt}%
\pgfsys@defobject{currentmarker}{\pgfqpoint{-0.048611in}{0.000000in}}{\pgfqpoint{0.000000in}{0.000000in}}{%
\pgfpathmoveto{\pgfqpoint{0.000000in}{0.000000in}}%
\pgfpathlineto{\pgfqpoint{-0.048611in}{0.000000in}}%
\pgfusepath{stroke,fill}%
}%
\begin{pgfscope}%
\pgfsys@transformshift{0.800000in}{2.833852in}%
\pgfsys@useobject{currentmarker}{}%
\end{pgfscope}%
\end{pgfscope}%
\begin{pgfscope}%
\pgftext[x=0.525308in,y=2.785658in,left,base]{\rmfamily\fontsize{10.000000}{12.000000}\selectfont \(\displaystyle 0.5\)}%
\end{pgfscope}%
\begin{pgfscope}%
\pgfsetbuttcap%
\pgfsetroundjoin%
\definecolor{currentfill}{rgb}{0.000000,0.000000,0.000000}%
\pgfsetfillcolor{currentfill}%
\pgfsetlinewidth{0.803000pt}%
\definecolor{currentstroke}{rgb}{0.000000,0.000000,0.000000}%
\pgfsetstrokecolor{currentstroke}%
\pgfsetdash{}{0pt}%
\pgfsys@defobject{currentmarker}{\pgfqpoint{-0.048611in}{0.000000in}}{\pgfqpoint{0.000000in}{0.000000in}}{%
\pgfpathmoveto{\pgfqpoint{0.000000in}{0.000000in}}%
\pgfpathlineto{\pgfqpoint{-0.048611in}{0.000000in}}%
\pgfusepath{stroke,fill}%
}%
\begin{pgfscope}%
\pgfsys@transformshift{0.800000in}{3.300184in}%
\pgfsys@useobject{currentmarker}{}%
\end{pgfscope}%
\end{pgfscope}%
\begin{pgfscope}%
\pgftext[x=0.525308in,y=3.251989in,left,base]{\rmfamily\fontsize{10.000000}{12.000000}\selectfont \(\displaystyle 1.0\)}%
\end{pgfscope}%
\begin{pgfscope}%
\pgfsetrectcap%
\pgfsetmiterjoin%
\pgfsetlinewidth{0.803000pt}%
\definecolor{currentstroke}{rgb}{0.000000,0.000000,0.000000}%
\pgfsetstrokecolor{currentstroke}%
\pgfsetdash{}{0pt}%
\pgfpathmoveto{\pgfqpoint{0.800000in}{1.363959in}}%
\pgfpathlineto{\pgfqpoint{0.800000in}{3.388041in}}%
\pgfusepath{stroke}%
\end{pgfscope}%
\begin{pgfscope}%
\pgfsetrectcap%
\pgfsetmiterjoin%
\pgfsetlinewidth{0.803000pt}%
\definecolor{currentstroke}{rgb}{0.000000,0.000000,0.000000}%
\pgfsetstrokecolor{currentstroke}%
\pgfsetdash{}{0pt}%
\pgfpathmoveto{\pgfqpoint{4.768000in}{1.363959in}}%
\pgfpathlineto{\pgfqpoint{4.768000in}{3.388041in}}%
\pgfusepath{stroke}%
\end{pgfscope}%
\begin{pgfscope}%
\pgfsetrectcap%
\pgfsetmiterjoin%
\pgfsetlinewidth{0.803000pt}%
\definecolor{currentstroke}{rgb}{0.000000,0.000000,0.000000}%
\pgfsetstrokecolor{currentstroke}%
\pgfsetdash{}{0pt}%
\pgfpathmoveto{\pgfqpoint{0.800000in}{1.363959in}}%
\pgfpathlineto{\pgfqpoint{4.768000in}{1.363959in}}%
\pgfusepath{stroke}%
\end{pgfscope}%
\begin{pgfscope}%
\pgfsetrectcap%
\pgfsetmiterjoin%
\pgfsetlinewidth{0.803000pt}%
\definecolor{currentstroke}{rgb}{0.000000,0.000000,0.000000}%
\pgfsetstrokecolor{currentstroke}%
\pgfsetdash{}{0pt}%
\pgfpathmoveto{\pgfqpoint{0.800000in}{3.388041in}}%
\pgfpathlineto{\pgfqpoint{4.768000in}{3.388041in}}%
\pgfusepath{stroke}%
\end{pgfscope}%
\begin{pgfscope}%
\pgfpathrectangle{\pgfqpoint{5.016000in}{0.528000in}}{\pgfqpoint{0.184800in}{3.696000in}} %
\pgfusepath{clip}%
\pgfsetbuttcap%
\pgfsetmiterjoin%
\definecolor{currentfill}{rgb}{1.000000,1.000000,1.000000}%
\pgfsetfillcolor{currentfill}%
\pgfsetlinewidth{0.010037pt}%
\definecolor{currentstroke}{rgb}{1.000000,1.000000,1.000000}%
\pgfsetstrokecolor{currentstroke}%
\pgfsetdash{}{0pt}%
\pgfpathmoveto{\pgfqpoint{5.016000in}{0.528000in}}%
\pgfpathlineto{\pgfqpoint{5.016000in}{0.542438in}}%
\pgfpathlineto{\pgfqpoint{5.016000in}{4.209562in}}%
\pgfpathlineto{\pgfqpoint{5.016000in}{4.224000in}}%
\pgfpathlineto{\pgfqpoint{5.200800in}{4.224000in}}%
\pgfpathlineto{\pgfqpoint{5.200800in}{4.209562in}}%
\pgfpathlineto{\pgfqpoint{5.200800in}{0.542438in}}%
\pgfpathlineto{\pgfqpoint{5.200800in}{0.528000in}}%
\pgfpathclose%
\pgfusepath{stroke,fill}%
\end{pgfscope}%
\begin{pgfscope}%
\pgfsys@transformshift{5.020000in}{0.530000in}%
\pgftext[left,bottom]{\pgfimage[interpolate=true,width=0.180000in,height=3.690000in]{Figure-0003-20180109-004133-667379-img0.png}}%
\end{pgfscope}%
\begin{pgfscope}%
\pgfsetbuttcap%
\pgfsetroundjoin%
\definecolor{currentfill}{rgb}{0.000000,0.000000,0.000000}%
\pgfsetfillcolor{currentfill}%
\pgfsetlinewidth{0.803000pt}%
\definecolor{currentstroke}{rgb}{0.000000,0.000000,0.000000}%
\pgfsetstrokecolor{currentstroke}%
\pgfsetdash{}{0pt}%
\pgfsys@defobject{currentmarker}{\pgfqpoint{0.000000in}{0.000000in}}{\pgfqpoint{0.048611in}{0.000000in}}{%
\pgfpathmoveto{\pgfqpoint{0.000000in}{0.000000in}}%
\pgfpathlineto{\pgfqpoint{0.048611in}{0.000000in}}%
\pgfusepath{stroke,fill}%
}%
\begin{pgfscope}%
\pgfsys@transformshift{5.200800in}{0.831955in}%
\pgfsys@useobject{currentmarker}{}%
\end{pgfscope}%
\end{pgfscope}%
\begin{pgfscope}%
\pgftext[x=5.298022in,y=0.783760in,left,base]{\rmfamily\fontsize{10.000000}{12.000000}\selectfont \(\displaystyle 0.003\)}%
\end{pgfscope}%
\begin{pgfscope}%
\pgfsetbuttcap%
\pgfsetroundjoin%
\definecolor{currentfill}{rgb}{0.000000,0.000000,0.000000}%
\pgfsetfillcolor{currentfill}%
\pgfsetlinewidth{0.803000pt}%
\definecolor{currentstroke}{rgb}{0.000000,0.000000,0.000000}%
\pgfsetstrokecolor{currentstroke}%
\pgfsetdash{}{0pt}%
\pgfsys@defobject{currentmarker}{\pgfqpoint{0.000000in}{0.000000in}}{\pgfqpoint{0.048611in}{0.000000in}}{%
\pgfpathmoveto{\pgfqpoint{0.000000in}{0.000000in}}%
\pgfpathlineto{\pgfqpoint{0.048611in}{0.000000in}}%
\pgfusepath{stroke,fill}%
}%
\begin{pgfscope}%
\pgfsys@transformshift{5.200800in}{1.282646in}%
\pgfsys@useobject{currentmarker}{}%
\end{pgfscope}%
\end{pgfscope}%
\begin{pgfscope}%
\pgftext[x=5.298022in,y=1.234452in,left,base]{\rmfamily\fontsize{10.000000}{12.000000}\selectfont \(\displaystyle 0.004\)}%
\end{pgfscope}%
\begin{pgfscope}%
\pgfsetbuttcap%
\pgfsetroundjoin%
\definecolor{currentfill}{rgb}{0.000000,0.000000,0.000000}%
\pgfsetfillcolor{currentfill}%
\pgfsetlinewidth{0.803000pt}%
\definecolor{currentstroke}{rgb}{0.000000,0.000000,0.000000}%
\pgfsetstrokecolor{currentstroke}%
\pgfsetdash{}{0pt}%
\pgfsys@defobject{currentmarker}{\pgfqpoint{0.000000in}{0.000000in}}{\pgfqpoint{0.048611in}{0.000000in}}{%
\pgfpathmoveto{\pgfqpoint{0.000000in}{0.000000in}}%
\pgfpathlineto{\pgfqpoint{0.048611in}{0.000000in}}%
\pgfusepath{stroke,fill}%
}%
\begin{pgfscope}%
\pgfsys@transformshift{5.200800in}{1.733337in}%
\pgfsys@useobject{currentmarker}{}%
\end{pgfscope}%
\end{pgfscope}%
\begin{pgfscope}%
\pgftext[x=5.298022in,y=1.685143in,left,base]{\rmfamily\fontsize{10.000000}{12.000000}\selectfont \(\displaystyle 0.005\)}%
\end{pgfscope}%
\begin{pgfscope}%
\pgfsetbuttcap%
\pgfsetroundjoin%
\definecolor{currentfill}{rgb}{0.000000,0.000000,0.000000}%
\pgfsetfillcolor{currentfill}%
\pgfsetlinewidth{0.803000pt}%
\definecolor{currentstroke}{rgb}{0.000000,0.000000,0.000000}%
\pgfsetstrokecolor{currentstroke}%
\pgfsetdash{}{0pt}%
\pgfsys@defobject{currentmarker}{\pgfqpoint{0.000000in}{0.000000in}}{\pgfqpoint{0.048611in}{0.000000in}}{%
\pgfpathmoveto{\pgfqpoint{0.000000in}{0.000000in}}%
\pgfpathlineto{\pgfqpoint{0.048611in}{0.000000in}}%
\pgfusepath{stroke,fill}%
}%
\begin{pgfscope}%
\pgfsys@transformshift{5.200800in}{2.184029in}%
\pgfsys@useobject{currentmarker}{}%
\end{pgfscope}%
\end{pgfscope}%
\begin{pgfscope}%
\pgftext[x=5.298022in,y=2.135834in,left,base]{\rmfamily\fontsize{10.000000}{12.000000}\selectfont \(\displaystyle 0.006\)}%
\end{pgfscope}%
\begin{pgfscope}%
\pgfsetbuttcap%
\pgfsetroundjoin%
\definecolor{currentfill}{rgb}{0.000000,0.000000,0.000000}%
\pgfsetfillcolor{currentfill}%
\pgfsetlinewidth{0.803000pt}%
\definecolor{currentstroke}{rgb}{0.000000,0.000000,0.000000}%
\pgfsetstrokecolor{currentstroke}%
\pgfsetdash{}{0pt}%
\pgfsys@defobject{currentmarker}{\pgfqpoint{0.000000in}{0.000000in}}{\pgfqpoint{0.048611in}{0.000000in}}{%
\pgfpathmoveto{\pgfqpoint{0.000000in}{0.000000in}}%
\pgfpathlineto{\pgfqpoint{0.048611in}{0.000000in}}%
\pgfusepath{stroke,fill}%
}%
\begin{pgfscope}%
\pgfsys@transformshift{5.200800in}{2.634720in}%
\pgfsys@useobject{currentmarker}{}%
\end{pgfscope}%
\end{pgfscope}%
\begin{pgfscope}%
\pgftext[x=5.298022in,y=2.586526in,left,base]{\rmfamily\fontsize{10.000000}{12.000000}\selectfont \(\displaystyle 0.007\)}%
\end{pgfscope}%
\begin{pgfscope}%
\pgfsetbuttcap%
\pgfsetroundjoin%
\definecolor{currentfill}{rgb}{0.000000,0.000000,0.000000}%
\pgfsetfillcolor{currentfill}%
\pgfsetlinewidth{0.803000pt}%
\definecolor{currentstroke}{rgb}{0.000000,0.000000,0.000000}%
\pgfsetstrokecolor{currentstroke}%
\pgfsetdash{}{0pt}%
\pgfsys@defobject{currentmarker}{\pgfqpoint{0.000000in}{0.000000in}}{\pgfqpoint{0.048611in}{0.000000in}}{%
\pgfpathmoveto{\pgfqpoint{0.000000in}{0.000000in}}%
\pgfpathlineto{\pgfqpoint{0.048611in}{0.000000in}}%
\pgfusepath{stroke,fill}%
}%
\begin{pgfscope}%
\pgfsys@transformshift{5.200800in}{3.085411in}%
\pgfsys@useobject{currentmarker}{}%
\end{pgfscope}%
\end{pgfscope}%
\begin{pgfscope}%
\pgftext[x=5.298022in,y=3.037217in,left,base]{\rmfamily\fontsize{10.000000}{12.000000}\selectfont \(\displaystyle 0.008\)}%
\end{pgfscope}%
\begin{pgfscope}%
\pgfsetbuttcap%
\pgfsetroundjoin%
\definecolor{currentfill}{rgb}{0.000000,0.000000,0.000000}%
\pgfsetfillcolor{currentfill}%
\pgfsetlinewidth{0.803000pt}%
\definecolor{currentstroke}{rgb}{0.000000,0.000000,0.000000}%
\pgfsetstrokecolor{currentstroke}%
\pgfsetdash{}{0pt}%
\pgfsys@defobject{currentmarker}{\pgfqpoint{0.000000in}{0.000000in}}{\pgfqpoint{0.048611in}{0.000000in}}{%
\pgfpathmoveto{\pgfqpoint{0.000000in}{0.000000in}}%
\pgfpathlineto{\pgfqpoint{0.048611in}{0.000000in}}%
\pgfusepath{stroke,fill}%
}%
\begin{pgfscope}%
\pgfsys@transformshift{5.200800in}{3.536103in}%
\pgfsys@useobject{currentmarker}{}%
\end{pgfscope}%
\end{pgfscope}%
\begin{pgfscope}%
\pgftext[x=5.298022in,y=3.487908in,left,base]{\rmfamily\fontsize{10.000000}{12.000000}\selectfont \(\displaystyle 0.009\)}%
\end{pgfscope}%
\begin{pgfscope}%
\pgfsetbuttcap%
\pgfsetroundjoin%
\definecolor{currentfill}{rgb}{0.000000,0.000000,0.000000}%
\pgfsetfillcolor{currentfill}%
\pgfsetlinewidth{0.803000pt}%
\definecolor{currentstroke}{rgb}{0.000000,0.000000,0.000000}%
\pgfsetstrokecolor{currentstroke}%
\pgfsetdash{}{0pt}%
\pgfsys@defobject{currentmarker}{\pgfqpoint{0.000000in}{0.000000in}}{\pgfqpoint{0.048611in}{0.000000in}}{%
\pgfpathmoveto{\pgfqpoint{0.000000in}{0.000000in}}%
\pgfpathlineto{\pgfqpoint{0.048611in}{0.000000in}}%
\pgfusepath{stroke,fill}%
}%
\begin{pgfscope}%
\pgfsys@transformshift{5.200800in}{3.986794in}%
\pgfsys@useobject{currentmarker}{}%
\end{pgfscope}%
\end{pgfscope}%
\begin{pgfscope}%
\pgftext[x=5.298022in,y=3.938600in,left,base]{\rmfamily\fontsize{10.000000}{12.000000}\selectfont \(\displaystyle 0.010\)}%
\end{pgfscope}%
\begin{pgfscope}%
\pgfsetbuttcap%
\pgfsetmiterjoin%
\pgfsetlinewidth{0.803000pt}%
\definecolor{currentstroke}{rgb}{0.000000,0.000000,0.000000}%
\pgfsetstrokecolor{currentstroke}%
\pgfsetdash{}{0pt}%
\pgfpathmoveto{\pgfqpoint{5.016000in}{0.528000in}}%
\pgfpathlineto{\pgfqpoint{5.016000in}{0.542438in}}%
\pgfpathlineto{\pgfqpoint{5.016000in}{4.209562in}}%
\pgfpathlineto{\pgfqpoint{5.016000in}{4.224000in}}%
\pgfpathlineto{\pgfqpoint{5.200800in}{4.224000in}}%
\pgfpathlineto{\pgfqpoint{5.200800in}{4.209562in}}%
\pgfpathlineto{\pgfqpoint{5.200800in}{0.542438in}}%
\pgfpathlineto{\pgfqpoint{5.200800in}{0.528000in}}%
\pgfpathclose%
\pgfusepath{stroke}%
\end{pgfscope}%
\end{pgfpicture}%
\makeatother%
\endgroup%
} 
\caption{An transportation example on ellipses dataset} \label{Fig:Ellipse}
\end{figure}

\begin{figure}
\centering \scalebox{0.65}{%% Creator: Matplotlib, PGF backend
%%
%% To include the figure in your LaTeX document, write
%%   \input{<filename>.pgf}
%%
%% Make sure the required packages are loaded in your preamble
%%   \usepackage{pgf}
%%
%% Figures using additional raster images can only be included by \input if
%% they are in the same directory as the main LaTeX file. For loading figures
%% from other directories you can use the `import` package
%%   \usepackage{import}
%% and then include the figures with
%%   \import{<path to file>}{<filename>.pgf}
%%
%% Matplotlib used the following preamble
%%   \usepackage{fontspec}
%%
\begingroup%
\makeatletter%
\begin{pgfpicture}%
\pgfpathrectangle{\pgfpointorigin}{\pgfqpoint{6.400000in}{4.800000in}}%
\pgfusepath{use as bounding box, clip}%
\begin{pgfscope}%
\pgfsetbuttcap%
\pgfsetmiterjoin%
\definecolor{currentfill}{rgb}{1.000000,1.000000,1.000000}%
\pgfsetfillcolor{currentfill}%
\pgfsetlinewidth{0.000000pt}%
\definecolor{currentstroke}{rgb}{1.000000,1.000000,1.000000}%
\pgfsetstrokecolor{currentstroke}%
\pgfsetdash{}{0pt}%
\pgfpathmoveto{\pgfqpoint{0.000000in}{0.000000in}}%
\pgfpathlineto{\pgfqpoint{6.400000in}{0.000000in}}%
\pgfpathlineto{\pgfqpoint{6.400000in}{4.800000in}}%
\pgfpathlineto{\pgfqpoint{0.000000in}{4.800000in}}%
\pgfpathclose%
\pgfusepath{fill}%
\end{pgfscope}%
\begin{pgfscope}%
\pgfsetbuttcap%
\pgfsetmiterjoin%
\definecolor{currentfill}{rgb}{1.000000,1.000000,1.000000}%
\pgfsetfillcolor{currentfill}%
\pgfsetlinewidth{0.000000pt}%
\definecolor{currentstroke}{rgb}{0.000000,0.000000,0.000000}%
\pgfsetstrokecolor{currentstroke}%
\pgfsetstrokeopacity{0.000000}%
\pgfsetdash{}{0pt}%
\pgfpathmoveto{\pgfqpoint{1.280114in}{0.528000in}}%
\pgfpathlineto{\pgfqpoint{4.768000in}{0.528000in}}%
\pgfpathlineto{\pgfqpoint{4.768000in}{4.224000in}}%
\pgfpathlineto{\pgfqpoint{1.280114in}{4.224000in}}%
\pgfpathclose%
\pgfusepath{fill}%
\end{pgfscope}%
\begin{pgfscope}%
\pgfpathrectangle{\pgfqpoint{1.280114in}{0.528000in}}{\pgfqpoint{3.487886in}{3.696000in}} %
\pgfusepath{clip}%
\pgfsetbuttcap%
\pgfsetroundjoin%
\definecolor{currentfill}{rgb}{0.121569,0.466667,0.705882}%
\pgfsetfillcolor{currentfill}%
\pgfsetlinewidth{1.003750pt}%
\definecolor{currentstroke}{rgb}{0.121569,0.466667,0.705882}%
\pgfsetstrokecolor{currentstroke}%
\pgfsetdash{}{0pt}%
\pgfpathmoveto{\pgfqpoint{1.782364in}{2.571326in}}%
\pgfpathcurveto{\pgfqpoint{1.792452in}{2.571326in}}{\pgfqpoint{1.802127in}{2.575334in}}{\pgfqpoint{1.809260in}{2.582467in}}%
\pgfpathcurveto{\pgfqpoint{1.816393in}{2.589600in}}{\pgfqpoint{1.820401in}{2.599275in}}{\pgfqpoint{1.820401in}{2.609363in}}%
\pgfpathcurveto{\pgfqpoint{1.820401in}{2.619450in}}{\pgfqpoint{1.816393in}{2.629125in}}{\pgfqpoint{1.809260in}{2.636258in}}%
\pgfpathcurveto{\pgfqpoint{1.802127in}{2.643391in}}{\pgfqpoint{1.792452in}{2.647399in}}{\pgfqpoint{1.782364in}{2.647399in}}%
\pgfpathcurveto{\pgfqpoint{1.772277in}{2.647399in}}{\pgfqpoint{1.762602in}{2.643391in}}{\pgfqpoint{1.755469in}{2.636258in}}%
\pgfpathcurveto{\pgfqpoint{1.748336in}{2.629125in}}{\pgfqpoint{1.744328in}{2.619450in}}{\pgfqpoint{1.744328in}{2.609363in}}%
\pgfpathcurveto{\pgfqpoint{1.744328in}{2.599275in}}{\pgfqpoint{1.748336in}{2.589600in}}{\pgfqpoint{1.755469in}{2.582467in}}%
\pgfpathcurveto{\pgfqpoint{1.762602in}{2.575334in}}{\pgfqpoint{1.772277in}{2.571326in}}{\pgfqpoint{1.782364in}{2.571326in}}%
\pgfpathclose%
\pgfusepath{stroke,fill}%
\end{pgfscope}%
\begin{pgfscope}%
\pgfpathrectangle{\pgfqpoint{1.280114in}{0.528000in}}{\pgfqpoint{3.487886in}{3.696000in}} %
\pgfusepath{clip}%
\pgfsetbuttcap%
\pgfsetroundjoin%
\definecolor{currentfill}{rgb}{0.121569,0.466667,0.705882}%
\pgfsetfillcolor{currentfill}%
\pgfsetlinewidth{1.003750pt}%
\definecolor{currentstroke}{rgb}{0.121569,0.466667,0.705882}%
\pgfsetstrokecolor{currentstroke}%
\pgfsetdash{}{0pt}%
\pgfpathmoveto{\pgfqpoint{2.857146in}{2.002090in}}%
\pgfpathcurveto{\pgfqpoint{2.867233in}{2.002090in}}{\pgfqpoint{2.876909in}{2.006097in}}{\pgfqpoint{2.884042in}{2.013230in}}%
\pgfpathcurveto{\pgfqpoint{2.891175in}{2.020363in}}{\pgfqpoint{2.895182in}{2.030039in}}{\pgfqpoint{2.895182in}{2.040126in}}%
\pgfpathcurveto{\pgfqpoint{2.895182in}{2.050213in}}{\pgfqpoint{2.891175in}{2.059889in}}{\pgfqpoint{2.884042in}{2.067022in}}%
\pgfpathcurveto{\pgfqpoint{2.876909in}{2.074154in}}{\pgfqpoint{2.867233in}{2.078162in}}{\pgfqpoint{2.857146in}{2.078162in}}%
\pgfpathcurveto{\pgfqpoint{2.847059in}{2.078162in}}{\pgfqpoint{2.837383in}{2.074154in}}{\pgfqpoint{2.830250in}{2.067022in}}%
\pgfpathcurveto{\pgfqpoint{2.823117in}{2.059889in}}{\pgfqpoint{2.819110in}{2.050213in}}{\pgfqpoint{2.819110in}{2.040126in}}%
\pgfpathcurveto{\pgfqpoint{2.819110in}{2.030039in}}{\pgfqpoint{2.823117in}{2.020363in}}{\pgfqpoint{2.830250in}{2.013230in}}%
\pgfpathcurveto{\pgfqpoint{2.837383in}{2.006097in}}{\pgfqpoint{2.847059in}{2.002090in}}{\pgfqpoint{2.857146in}{2.002090in}}%
\pgfpathclose%
\pgfusepath{stroke,fill}%
\end{pgfscope}%
\begin{pgfscope}%
\pgfpathrectangle{\pgfqpoint{1.280114in}{0.528000in}}{\pgfqpoint{3.487886in}{3.696000in}} %
\pgfusepath{clip}%
\pgfsetbuttcap%
\pgfsetroundjoin%
\definecolor{currentfill}{rgb}{0.121569,0.466667,0.705882}%
\pgfsetfillcolor{currentfill}%
\pgfsetlinewidth{1.003750pt}%
\definecolor{currentstroke}{rgb}{0.121569,0.466667,0.705882}%
\pgfsetstrokecolor{currentstroke}%
\pgfsetdash{}{0pt}%
\pgfpathmoveto{\pgfqpoint{4.444400in}{2.360854in}}%
\pgfpathcurveto{\pgfqpoint{4.454487in}{2.360854in}}{\pgfqpoint{4.464163in}{2.364862in}}{\pgfqpoint{4.471296in}{2.371995in}}%
\pgfpathcurveto{\pgfqpoint{4.478428in}{2.379128in}}{\pgfqpoint{4.482436in}{2.388803in}}{\pgfqpoint{4.482436in}{2.398891in}}%
\pgfpathcurveto{\pgfqpoint{4.482436in}{2.408978in}}{\pgfqpoint{4.478428in}{2.418654in}}{\pgfqpoint{4.471296in}{2.425786in}}%
\pgfpathcurveto{\pgfqpoint{4.464163in}{2.432919in}}{\pgfqpoint{4.454487in}{2.436927in}}{\pgfqpoint{4.444400in}{2.436927in}}%
\pgfpathcurveto{\pgfqpoint{4.434313in}{2.436927in}}{\pgfqpoint{4.424637in}{2.432919in}}{\pgfqpoint{4.417504in}{2.425786in}}%
\pgfpathcurveto{\pgfqpoint{4.410371in}{2.418654in}}{\pgfqpoint{4.406364in}{2.408978in}}{\pgfqpoint{4.406364in}{2.398891in}}%
\pgfpathcurveto{\pgfqpoint{4.406364in}{2.388803in}}{\pgfqpoint{4.410371in}{2.379128in}}{\pgfqpoint{4.417504in}{2.371995in}}%
\pgfpathcurveto{\pgfqpoint{4.424637in}{2.364862in}}{\pgfqpoint{4.434313in}{2.360854in}}{\pgfqpoint{4.444400in}{2.360854in}}%
\pgfpathclose%
\pgfusepath{stroke,fill}%
\end{pgfscope}%
\begin{pgfscope}%
\pgfpathrectangle{\pgfqpoint{1.280114in}{0.528000in}}{\pgfqpoint{3.487886in}{3.696000in}} %
\pgfusepath{clip}%
\pgfsetbuttcap%
\pgfsetroundjoin%
\definecolor{currentfill}{rgb}{0.121569,0.466667,0.705882}%
\pgfsetfillcolor{currentfill}%
\pgfsetlinewidth{1.003750pt}%
\definecolor{currentstroke}{rgb}{0.121569,0.466667,0.705882}%
\pgfsetstrokecolor{currentstroke}%
\pgfsetdash{}{0pt}%
\pgfpathmoveto{\pgfqpoint{2.691598in}{2.713981in}}%
\pgfpathcurveto{\pgfqpoint{2.701685in}{2.713981in}}{\pgfqpoint{2.711361in}{2.717989in}}{\pgfqpoint{2.718493in}{2.725121in}}%
\pgfpathcurveto{\pgfqpoint{2.725626in}{2.732254in}}{\pgfqpoint{2.729634in}{2.741930in}}{\pgfqpoint{2.729634in}{2.752017in}}%
\pgfpathcurveto{\pgfqpoint{2.729634in}{2.762104in}}{\pgfqpoint{2.725626in}{2.771780in}}{\pgfqpoint{2.718493in}{2.778913in}}%
\pgfpathcurveto{\pgfqpoint{2.711361in}{2.786046in}}{\pgfqpoint{2.701685in}{2.790053in}}{\pgfqpoint{2.691598in}{2.790053in}}%
\pgfpathcurveto{\pgfqpoint{2.681510in}{2.790053in}}{\pgfqpoint{2.671835in}{2.786046in}}{\pgfqpoint{2.664702in}{2.778913in}}%
\pgfpathcurveto{\pgfqpoint{2.657569in}{2.771780in}}{\pgfqpoint{2.653561in}{2.762104in}}{\pgfqpoint{2.653561in}{2.752017in}}%
\pgfpathcurveto{\pgfqpoint{2.653561in}{2.741930in}}{\pgfqpoint{2.657569in}{2.732254in}}{\pgfqpoint{2.664702in}{2.725121in}}%
\pgfpathcurveto{\pgfqpoint{2.671835in}{2.717989in}}{\pgfqpoint{2.681510in}{2.713981in}}{\pgfqpoint{2.691598in}{2.713981in}}%
\pgfpathclose%
\pgfusepath{stroke,fill}%
\end{pgfscope}%
\begin{pgfscope}%
\pgfpathrectangle{\pgfqpoint{1.280114in}{0.528000in}}{\pgfqpoint{3.487886in}{3.696000in}} %
\pgfusepath{clip}%
\pgfsetbuttcap%
\pgfsetroundjoin%
\definecolor{currentfill}{rgb}{0.121569,0.466667,0.705882}%
\pgfsetfillcolor{currentfill}%
\pgfsetlinewidth{1.003750pt}%
\definecolor{currentstroke}{rgb}{0.121569,0.466667,0.705882}%
\pgfsetstrokecolor{currentstroke}%
\pgfsetdash{}{0pt}%
\pgfpathmoveto{\pgfqpoint{3.989893in}{2.683955in}}%
\pgfpathcurveto{\pgfqpoint{3.999980in}{2.683955in}}{\pgfqpoint{4.009656in}{2.687962in}}{\pgfqpoint{4.016788in}{2.695095in}}%
\pgfpathcurveto{\pgfqpoint{4.023921in}{2.702228in}}{\pgfqpoint{4.027929in}{2.711904in}}{\pgfqpoint{4.027929in}{2.721991in}}%
\pgfpathcurveto{\pgfqpoint{4.027929in}{2.732078in}}{\pgfqpoint{4.023921in}{2.741754in}}{\pgfqpoint{4.016788in}{2.748887in}}%
\pgfpathcurveto{\pgfqpoint{4.009656in}{2.756020in}}{\pgfqpoint{3.999980in}{2.760027in}}{\pgfqpoint{3.989893in}{2.760027in}}%
\pgfpathcurveto{\pgfqpoint{3.979805in}{2.760027in}}{\pgfqpoint{3.970130in}{2.756020in}}{\pgfqpoint{3.962997in}{2.748887in}}%
\pgfpathcurveto{\pgfqpoint{3.955864in}{2.741754in}}{\pgfqpoint{3.951856in}{2.732078in}}{\pgfqpoint{3.951856in}{2.721991in}}%
\pgfpathcurveto{\pgfqpoint{3.951856in}{2.711904in}}{\pgfqpoint{3.955864in}{2.702228in}}{\pgfqpoint{3.962997in}{2.695095in}}%
\pgfpathcurveto{\pgfqpoint{3.970130in}{2.687962in}}{\pgfqpoint{3.979805in}{2.683955in}}{\pgfqpoint{3.989893in}{2.683955in}}%
\pgfpathclose%
\pgfusepath{stroke,fill}%
\end{pgfscope}%
\begin{pgfscope}%
\pgfpathrectangle{\pgfqpoint{1.280114in}{0.528000in}}{\pgfqpoint{3.487886in}{3.696000in}} %
\pgfusepath{clip}%
\pgfsetbuttcap%
\pgfsetroundjoin%
\definecolor{currentfill}{rgb}{0.121569,0.466667,0.705882}%
\pgfsetfillcolor{currentfill}%
\pgfsetlinewidth{1.003750pt}%
\definecolor{currentstroke}{rgb}{0.121569,0.466667,0.705882}%
\pgfsetstrokecolor{currentstroke}%
\pgfsetdash{}{0pt}%
\pgfpathmoveto{\pgfqpoint{4.249422in}{2.588569in}}%
\pgfpathcurveto{\pgfqpoint{4.259509in}{2.588569in}}{\pgfqpoint{4.269185in}{2.592577in}}{\pgfqpoint{4.276318in}{2.599710in}}%
\pgfpathcurveto{\pgfqpoint{4.283451in}{2.606843in}}{\pgfqpoint{4.287458in}{2.616518in}}{\pgfqpoint{4.287458in}{2.626606in}}%
\pgfpathcurveto{\pgfqpoint{4.287458in}{2.636693in}}{\pgfqpoint{4.283451in}{2.646369in}}{\pgfqpoint{4.276318in}{2.653501in}}%
\pgfpathcurveto{\pgfqpoint{4.269185in}{2.660634in}}{\pgfqpoint{4.259509in}{2.664642in}}{\pgfqpoint{4.249422in}{2.664642in}}%
\pgfpathcurveto{\pgfqpoint{4.239335in}{2.664642in}}{\pgfqpoint{4.229659in}{2.660634in}}{\pgfqpoint{4.222526in}{2.653501in}}%
\pgfpathcurveto{\pgfqpoint{4.215393in}{2.646369in}}{\pgfqpoint{4.211386in}{2.636693in}}{\pgfqpoint{4.211386in}{2.626606in}}%
\pgfpathcurveto{\pgfqpoint{4.211386in}{2.616518in}}{\pgfqpoint{4.215393in}{2.606843in}}{\pgfqpoint{4.222526in}{2.599710in}}%
\pgfpathcurveto{\pgfqpoint{4.229659in}{2.592577in}}{\pgfqpoint{4.239335in}{2.588569in}}{\pgfqpoint{4.249422in}{2.588569in}}%
\pgfpathclose%
\pgfusepath{stroke,fill}%
\end{pgfscope}%
\begin{pgfscope}%
\pgfpathrectangle{\pgfqpoint{1.280114in}{0.528000in}}{\pgfqpoint{3.487886in}{3.696000in}} %
\pgfusepath{clip}%
\pgfsetbuttcap%
\pgfsetroundjoin%
\definecolor{currentfill}{rgb}{0.121569,0.466667,0.705882}%
\pgfsetfillcolor{currentfill}%
\pgfsetlinewidth{1.003750pt}%
\definecolor{currentstroke}{rgb}{0.121569,0.466667,0.705882}%
\pgfsetstrokecolor{currentstroke}%
\pgfsetdash{}{0pt}%
\pgfpathmoveto{\pgfqpoint{3.669294in}{2.721060in}}%
\pgfpathcurveto{\pgfqpoint{3.679381in}{2.721060in}}{\pgfqpoint{3.689057in}{2.725067in}}{\pgfqpoint{3.696190in}{2.732200in}}%
\pgfpathcurveto{\pgfqpoint{3.703322in}{2.739333in}}{\pgfqpoint{3.707330in}{2.749009in}}{\pgfqpoint{3.707330in}{2.759096in}}%
\pgfpathcurveto{\pgfqpoint{3.707330in}{2.769183in}}{\pgfqpoint{3.703322in}{2.778859in}}{\pgfqpoint{3.696190in}{2.785992in}}%
\pgfpathcurveto{\pgfqpoint{3.689057in}{2.793125in}}{\pgfqpoint{3.679381in}{2.797132in}}{\pgfqpoint{3.669294in}{2.797132in}}%
\pgfpathcurveto{\pgfqpoint{3.659206in}{2.797132in}}{\pgfqpoint{3.649531in}{2.793125in}}{\pgfqpoint{3.642398in}{2.785992in}}%
\pgfpathcurveto{\pgfqpoint{3.635265in}{2.778859in}}{\pgfqpoint{3.631258in}{2.769183in}}{\pgfqpoint{3.631258in}{2.759096in}}%
\pgfpathcurveto{\pgfqpoint{3.631258in}{2.749009in}}{\pgfqpoint{3.635265in}{2.739333in}}{\pgfqpoint{3.642398in}{2.732200in}}%
\pgfpathcurveto{\pgfqpoint{3.649531in}{2.725067in}}{\pgfqpoint{3.659206in}{2.721060in}}{\pgfqpoint{3.669294in}{2.721060in}}%
\pgfpathclose%
\pgfusepath{stroke,fill}%
\end{pgfscope}%
\begin{pgfscope}%
\pgfpathrectangle{\pgfqpoint{1.280114in}{0.528000in}}{\pgfqpoint{3.487886in}{3.696000in}} %
\pgfusepath{clip}%
\pgfsetbuttcap%
\pgfsetroundjoin%
\definecolor{currentfill}{rgb}{0.121569,0.466667,0.705882}%
\pgfsetfillcolor{currentfill}%
\pgfsetlinewidth{1.003750pt}%
\definecolor{currentstroke}{rgb}{0.121569,0.466667,0.705882}%
\pgfsetstrokecolor{currentstroke}%
\pgfsetdash{}{0pt}%
\pgfpathmoveto{\pgfqpoint{2.192902in}{2.675244in}}%
\pgfpathcurveto{\pgfqpoint{2.202989in}{2.675244in}}{\pgfqpoint{2.212665in}{2.679252in}}{\pgfqpoint{2.219797in}{2.686385in}}%
\pgfpathcurveto{\pgfqpoint{2.226930in}{2.693517in}}{\pgfqpoint{2.230938in}{2.703193in}}{\pgfqpoint{2.230938in}{2.713280in}}%
\pgfpathcurveto{\pgfqpoint{2.230938in}{2.723368in}}{\pgfqpoint{2.226930in}{2.733043in}}{\pgfqpoint{2.219797in}{2.740176in}}%
\pgfpathcurveto{\pgfqpoint{2.212665in}{2.747309in}}{\pgfqpoint{2.202989in}{2.751317in}}{\pgfqpoint{2.192902in}{2.751317in}}%
\pgfpathcurveto{\pgfqpoint{2.182814in}{2.751317in}}{\pgfqpoint{2.173139in}{2.747309in}}{\pgfqpoint{2.166006in}{2.740176in}}%
\pgfpathcurveto{\pgfqpoint{2.158873in}{2.733043in}}{\pgfqpoint{2.154865in}{2.723368in}}{\pgfqpoint{2.154865in}{2.713280in}}%
\pgfpathcurveto{\pgfqpoint{2.154865in}{2.703193in}}{\pgfqpoint{2.158873in}{2.693517in}}{\pgfqpoint{2.166006in}{2.686385in}}%
\pgfpathcurveto{\pgfqpoint{2.173139in}{2.679252in}}{\pgfqpoint{2.182814in}{2.675244in}}{\pgfqpoint{2.192902in}{2.675244in}}%
\pgfpathclose%
\pgfusepath{stroke,fill}%
\end{pgfscope}%
\begin{pgfscope}%
\pgfpathrectangle{\pgfqpoint{1.280114in}{0.528000in}}{\pgfqpoint{3.487886in}{3.696000in}} %
\pgfusepath{clip}%
\pgfsetbuttcap%
\pgfsetroundjoin%
\definecolor{currentfill}{rgb}{0.121569,0.466667,0.705882}%
\pgfsetfillcolor{currentfill}%
\pgfsetlinewidth{1.003750pt}%
\definecolor{currentstroke}{rgb}{0.121569,0.466667,0.705882}%
\pgfsetstrokecolor{currentstroke}%
\pgfsetdash{}{0pt}%
\pgfpathmoveto{\pgfqpoint{1.975384in}{2.591671in}}%
\pgfpathcurveto{\pgfqpoint{1.985472in}{2.591671in}}{\pgfqpoint{1.995147in}{2.595679in}}{\pgfqpoint{2.002280in}{2.602812in}}%
\pgfpathcurveto{\pgfqpoint{2.009413in}{2.609945in}}{\pgfqpoint{2.013421in}{2.619620in}}{\pgfqpoint{2.013421in}{2.629708in}}%
\pgfpathcurveto{\pgfqpoint{2.013421in}{2.639795in}}{\pgfqpoint{2.009413in}{2.649471in}}{\pgfqpoint{2.002280in}{2.656603in}}%
\pgfpathcurveto{\pgfqpoint{1.995147in}{2.663736in}}{\pgfqpoint{1.985472in}{2.667744in}}{\pgfqpoint{1.975384in}{2.667744in}}%
\pgfpathcurveto{\pgfqpoint{1.965297in}{2.667744in}}{\pgfqpoint{1.955621in}{2.663736in}}{\pgfqpoint{1.948489in}{2.656603in}}%
\pgfpathcurveto{\pgfqpoint{1.941356in}{2.649471in}}{\pgfqpoint{1.937348in}{2.639795in}}{\pgfqpoint{1.937348in}{2.629708in}}%
\pgfpathcurveto{\pgfqpoint{1.937348in}{2.619620in}}{\pgfqpoint{1.941356in}{2.609945in}}{\pgfqpoint{1.948489in}{2.602812in}}%
\pgfpathcurveto{\pgfqpoint{1.955621in}{2.595679in}}{\pgfqpoint{1.965297in}{2.591671in}}{\pgfqpoint{1.975384in}{2.591671in}}%
\pgfpathclose%
\pgfusepath{stroke,fill}%
\end{pgfscope}%
\begin{pgfscope}%
\pgfpathrectangle{\pgfqpoint{1.280114in}{0.528000in}}{\pgfqpoint{3.487886in}{3.696000in}} %
\pgfusepath{clip}%
\pgfsetbuttcap%
\pgfsetroundjoin%
\definecolor{currentfill}{rgb}{0.121569,0.466667,0.705882}%
\pgfsetfillcolor{currentfill}%
\pgfsetlinewidth{1.003750pt}%
\definecolor{currentstroke}{rgb}{0.121569,0.466667,0.705882}%
\pgfsetstrokecolor{currentstroke}%
\pgfsetdash{}{0pt}%
\pgfpathmoveto{\pgfqpoint{1.758629in}{2.292106in}}%
\pgfpathcurveto{\pgfqpoint{1.768716in}{2.292106in}}{\pgfqpoint{1.778391in}{2.296114in}}{\pgfqpoint{1.785524in}{2.303247in}}%
\pgfpathcurveto{\pgfqpoint{1.792657in}{2.310380in}}{\pgfqpoint{1.796665in}{2.320055in}}{\pgfqpoint{1.796665in}{2.330142in}}%
\pgfpathcurveto{\pgfqpoint{1.796665in}{2.340230in}}{\pgfqpoint{1.792657in}{2.349905in}}{\pgfqpoint{1.785524in}{2.357038in}}%
\pgfpathcurveto{\pgfqpoint{1.778391in}{2.364171in}}{\pgfqpoint{1.768716in}{2.368179in}}{\pgfqpoint{1.758629in}{2.368179in}}%
\pgfpathcurveto{\pgfqpoint{1.748541in}{2.368179in}}{\pgfqpoint{1.738866in}{2.364171in}}{\pgfqpoint{1.731733in}{2.357038in}}%
\pgfpathcurveto{\pgfqpoint{1.724600in}{2.349905in}}{\pgfqpoint{1.720592in}{2.340230in}}{\pgfqpoint{1.720592in}{2.330142in}}%
\pgfpathcurveto{\pgfqpoint{1.720592in}{2.320055in}}{\pgfqpoint{1.724600in}{2.310380in}}{\pgfqpoint{1.731733in}{2.303247in}}%
\pgfpathcurveto{\pgfqpoint{1.738866in}{2.296114in}}{\pgfqpoint{1.748541in}{2.292106in}}{\pgfqpoint{1.758629in}{2.292106in}}%
\pgfpathclose%
\pgfusepath{stroke,fill}%
\end{pgfscope}%
\begin{pgfscope}%
\pgfpathrectangle{\pgfqpoint{1.280114in}{0.528000in}}{\pgfqpoint{3.487886in}{3.696000in}} %
\pgfusepath{clip}%
\pgfsetbuttcap%
\pgfsetroundjoin%
\definecolor{currentfill}{rgb}{0.121569,0.466667,0.705882}%
\pgfsetfillcolor{currentfill}%
\pgfsetlinewidth{1.003750pt}%
\definecolor{currentstroke}{rgb}{0.121569,0.466667,0.705882}%
\pgfsetstrokecolor{currentstroke}%
\pgfsetdash{}{0pt}%
\pgfpathmoveto{\pgfqpoint{1.971793in}{2.551979in}}%
\pgfpathcurveto{\pgfqpoint{1.981881in}{2.551979in}}{\pgfqpoint{1.991556in}{2.555987in}}{\pgfqpoint{1.998689in}{2.563119in}}%
\pgfpathcurveto{\pgfqpoint{2.005822in}{2.570252in}}{\pgfqpoint{2.009830in}{2.579928in}}{\pgfqpoint{2.009830in}{2.590015in}}%
\pgfpathcurveto{\pgfqpoint{2.009830in}{2.600102in}}{\pgfqpoint{2.005822in}{2.609778in}}{\pgfqpoint{1.998689in}{2.616911in}}%
\pgfpathcurveto{\pgfqpoint{1.991556in}{2.624044in}}{\pgfqpoint{1.981881in}{2.628051in}}{\pgfqpoint{1.971793in}{2.628051in}}%
\pgfpathcurveto{\pgfqpoint{1.961706in}{2.628051in}}{\pgfqpoint{1.952031in}{2.624044in}}{\pgfqpoint{1.944898in}{2.616911in}}%
\pgfpathcurveto{\pgfqpoint{1.937765in}{2.609778in}}{\pgfqpoint{1.933757in}{2.600102in}}{\pgfqpoint{1.933757in}{2.590015in}}%
\pgfpathcurveto{\pgfqpoint{1.933757in}{2.579928in}}{\pgfqpoint{1.937765in}{2.570252in}}{\pgfqpoint{1.944898in}{2.563119in}}%
\pgfpathcurveto{\pgfqpoint{1.952031in}{2.555987in}}{\pgfqpoint{1.961706in}{2.551979in}}{\pgfqpoint{1.971793in}{2.551979in}}%
\pgfpathclose%
\pgfusepath{stroke,fill}%
\end{pgfscope}%
\begin{pgfscope}%
\pgfpathrectangle{\pgfqpoint{1.280114in}{0.528000in}}{\pgfqpoint{3.487886in}{3.696000in}} %
\pgfusepath{clip}%
\pgfsetbuttcap%
\pgfsetroundjoin%
\definecolor{currentfill}{rgb}{0.121569,0.466667,0.705882}%
\pgfsetfillcolor{currentfill}%
\pgfsetlinewidth{1.003750pt}%
\definecolor{currentstroke}{rgb}{0.121569,0.466667,0.705882}%
\pgfsetstrokecolor{currentstroke}%
\pgfsetdash{}{0pt}%
\pgfpathmoveto{\pgfqpoint{2.305267in}{2.042659in}}%
\pgfpathcurveto{\pgfqpoint{2.315355in}{2.042659in}}{\pgfqpoint{2.325030in}{2.046666in}}{\pgfqpoint{2.332163in}{2.053799in}}%
\pgfpathcurveto{\pgfqpoint{2.339296in}{2.060932in}}{\pgfqpoint{2.343304in}{2.070607in}}{\pgfqpoint{2.343304in}{2.080695in}}%
\pgfpathcurveto{\pgfqpoint{2.343304in}{2.090782in}}{\pgfqpoint{2.339296in}{2.100458in}}{\pgfqpoint{2.332163in}{2.107591in}}%
\pgfpathcurveto{\pgfqpoint{2.325030in}{2.114723in}}{\pgfqpoint{2.315355in}{2.118731in}}{\pgfqpoint{2.305267in}{2.118731in}}%
\pgfpathcurveto{\pgfqpoint{2.295180in}{2.118731in}}{\pgfqpoint{2.285504in}{2.114723in}}{\pgfqpoint{2.278371in}{2.107591in}}%
\pgfpathcurveto{\pgfqpoint{2.271239in}{2.100458in}}{\pgfqpoint{2.267231in}{2.090782in}}{\pgfqpoint{2.267231in}{2.080695in}}%
\pgfpathcurveto{\pgfqpoint{2.267231in}{2.070607in}}{\pgfqpoint{2.271239in}{2.060932in}}{\pgfqpoint{2.278371in}{2.053799in}}%
\pgfpathcurveto{\pgfqpoint{2.285504in}{2.046666in}}{\pgfqpoint{2.295180in}{2.042659in}}{\pgfqpoint{2.305267in}{2.042659in}}%
\pgfpathclose%
\pgfusepath{stroke,fill}%
\end{pgfscope}%
\begin{pgfscope}%
\pgfpathrectangle{\pgfqpoint{1.280114in}{0.528000in}}{\pgfqpoint{3.487886in}{3.696000in}} %
\pgfusepath{clip}%
\pgfsetbuttcap%
\pgfsetroundjoin%
\definecolor{currentfill}{rgb}{0.121569,0.466667,0.705882}%
\pgfsetfillcolor{currentfill}%
\pgfsetlinewidth{1.003750pt}%
\definecolor{currentstroke}{rgb}{0.121569,0.466667,0.705882}%
\pgfsetstrokecolor{currentstroke}%
\pgfsetdash{}{0pt}%
\pgfpathmoveto{\pgfqpoint{3.306809in}{2.713301in}}%
\pgfpathcurveto{\pgfqpoint{3.316896in}{2.713301in}}{\pgfqpoint{3.326571in}{2.717309in}}{\pgfqpoint{3.333704in}{2.724442in}}%
\pgfpathcurveto{\pgfqpoint{3.340837in}{2.731575in}}{\pgfqpoint{3.344845in}{2.741250in}}{\pgfqpoint{3.344845in}{2.751337in}}%
\pgfpathcurveto{\pgfqpoint{3.344845in}{2.761425in}}{\pgfqpoint{3.340837in}{2.771100in}}{\pgfqpoint{3.333704in}{2.778233in}}%
\pgfpathcurveto{\pgfqpoint{3.326571in}{2.785366in}}{\pgfqpoint{3.316896in}{2.789374in}}{\pgfqpoint{3.306809in}{2.789374in}}%
\pgfpathcurveto{\pgfqpoint{3.296721in}{2.789374in}}{\pgfqpoint{3.287046in}{2.785366in}}{\pgfqpoint{3.279913in}{2.778233in}}%
\pgfpathcurveto{\pgfqpoint{3.272780in}{2.771100in}}{\pgfqpoint{3.268772in}{2.761425in}}{\pgfqpoint{3.268772in}{2.751337in}}%
\pgfpathcurveto{\pgfqpoint{3.268772in}{2.741250in}}{\pgfqpoint{3.272780in}{2.731575in}}{\pgfqpoint{3.279913in}{2.724442in}}%
\pgfpathcurveto{\pgfqpoint{3.287046in}{2.717309in}}{\pgfqpoint{3.296721in}{2.713301in}}{\pgfqpoint{3.306809in}{2.713301in}}%
\pgfpathclose%
\pgfusepath{stroke,fill}%
\end{pgfscope}%
\begin{pgfscope}%
\pgfpathrectangle{\pgfqpoint{1.280114in}{0.528000in}}{\pgfqpoint{3.487886in}{3.696000in}} %
\pgfusepath{clip}%
\pgfsetbuttcap%
\pgfsetroundjoin%
\definecolor{currentfill}{rgb}{0.121569,0.466667,0.705882}%
\pgfsetfillcolor{currentfill}%
\pgfsetlinewidth{1.003750pt}%
\definecolor{currentstroke}{rgb}{0.121569,0.466667,0.705882}%
\pgfsetstrokecolor{currentstroke}%
\pgfsetdash{}{0pt}%
\pgfpathmoveto{\pgfqpoint{4.056276in}{2.044420in}}%
\pgfpathcurveto{\pgfqpoint{4.066364in}{2.044420in}}{\pgfqpoint{4.076039in}{2.048428in}}{\pgfqpoint{4.083172in}{2.055561in}}%
\pgfpathcurveto{\pgfqpoint{4.090305in}{2.062693in}}{\pgfqpoint{4.094313in}{2.072369in}}{\pgfqpoint{4.094313in}{2.082456in}}%
\pgfpathcurveto{\pgfqpoint{4.094313in}{2.092544in}}{\pgfqpoint{4.090305in}{2.102219in}}{\pgfqpoint{4.083172in}{2.109352in}}%
\pgfpathcurveto{\pgfqpoint{4.076039in}{2.116485in}}{\pgfqpoint{4.066364in}{2.120493in}}{\pgfqpoint{4.056276in}{2.120493in}}%
\pgfpathcurveto{\pgfqpoint{4.046189in}{2.120493in}}{\pgfqpoint{4.036514in}{2.116485in}}{\pgfqpoint{4.029381in}{2.109352in}}%
\pgfpathcurveto{\pgfqpoint{4.022248in}{2.102219in}}{\pgfqpoint{4.018240in}{2.092544in}}{\pgfqpoint{4.018240in}{2.082456in}}%
\pgfpathcurveto{\pgfqpoint{4.018240in}{2.072369in}}{\pgfqpoint{4.022248in}{2.062693in}}{\pgfqpoint{4.029381in}{2.055561in}}%
\pgfpathcurveto{\pgfqpoint{4.036514in}{2.048428in}}{\pgfqpoint{4.046189in}{2.044420in}}{\pgfqpoint{4.056276in}{2.044420in}}%
\pgfpathclose%
\pgfusepath{stroke,fill}%
\end{pgfscope}%
\begin{pgfscope}%
\pgfpathrectangle{\pgfqpoint{1.280114in}{0.528000in}}{\pgfqpoint{3.487886in}{3.696000in}} %
\pgfusepath{clip}%
\pgfsetbuttcap%
\pgfsetroundjoin%
\definecolor{currentfill}{rgb}{0.121569,0.466667,0.705882}%
\pgfsetfillcolor{currentfill}%
\pgfsetlinewidth{1.003750pt}%
\definecolor{currentstroke}{rgb}{0.121569,0.466667,0.705882}%
\pgfsetstrokecolor{currentstroke}%
\pgfsetdash{}{0pt}%
\pgfpathmoveto{\pgfqpoint{4.518447in}{2.456274in}}%
\pgfpathcurveto{\pgfqpoint{4.528535in}{2.456274in}}{\pgfqpoint{4.538210in}{2.460282in}}{\pgfqpoint{4.545343in}{2.467414in}}%
\pgfpathcurveto{\pgfqpoint{4.552476in}{2.474547in}}{\pgfqpoint{4.556484in}{2.484223in}}{\pgfqpoint{4.556484in}{2.494310in}}%
\pgfpathcurveto{\pgfqpoint{4.556484in}{2.504397in}}{\pgfqpoint{4.552476in}{2.514073in}}{\pgfqpoint{4.545343in}{2.521206in}}%
\pgfpathcurveto{\pgfqpoint{4.538210in}{2.528339in}}{\pgfqpoint{4.528535in}{2.532346in}}{\pgfqpoint{4.518447in}{2.532346in}}%
\pgfpathcurveto{\pgfqpoint{4.508360in}{2.532346in}}{\pgfqpoint{4.498685in}{2.528339in}}{\pgfqpoint{4.491552in}{2.521206in}}%
\pgfpathcurveto{\pgfqpoint{4.484419in}{2.514073in}}{\pgfqpoint{4.480411in}{2.504397in}}{\pgfqpoint{4.480411in}{2.494310in}}%
\pgfpathcurveto{\pgfqpoint{4.480411in}{2.484223in}}{\pgfqpoint{4.484419in}{2.474547in}}{\pgfqpoint{4.491552in}{2.467414in}}%
\pgfpathcurveto{\pgfqpoint{4.498685in}{2.460282in}}{\pgfqpoint{4.508360in}{2.456274in}}{\pgfqpoint{4.518447in}{2.456274in}}%
\pgfpathclose%
\pgfusepath{stroke,fill}%
\end{pgfscope}%
\begin{pgfscope}%
\pgfpathrectangle{\pgfqpoint{1.280114in}{0.528000in}}{\pgfqpoint{3.487886in}{3.696000in}} %
\pgfusepath{clip}%
\pgfsetbuttcap%
\pgfsetroundjoin%
\definecolor{currentfill}{rgb}{0.121569,0.466667,0.705882}%
\pgfsetfillcolor{currentfill}%
\pgfsetlinewidth{1.003750pt}%
\definecolor{currentstroke}{rgb}{0.121569,0.466667,0.705882}%
\pgfsetstrokecolor{currentstroke}%
\pgfsetdash{}{0pt}%
\pgfpathmoveto{\pgfqpoint{2.420152in}{2.097188in}}%
\pgfpathcurveto{\pgfqpoint{2.430239in}{2.097188in}}{\pgfqpoint{2.439915in}{2.101196in}}{\pgfqpoint{2.447048in}{2.108329in}}%
\pgfpathcurveto{\pgfqpoint{2.454181in}{2.115462in}}{\pgfqpoint{2.458188in}{2.125137in}}{\pgfqpoint{2.458188in}{2.135225in}}%
\pgfpathcurveto{\pgfqpoint{2.458188in}{2.145312in}}{\pgfqpoint{2.454181in}{2.154988in}}{\pgfqpoint{2.447048in}{2.162120in}}%
\pgfpathcurveto{\pgfqpoint{2.439915in}{2.169253in}}{\pgfqpoint{2.430239in}{2.173261in}}{\pgfqpoint{2.420152in}{2.173261in}}%
\pgfpathcurveto{\pgfqpoint{2.410065in}{2.173261in}}{\pgfqpoint{2.400389in}{2.169253in}}{\pgfqpoint{2.393256in}{2.162120in}}%
\pgfpathcurveto{\pgfqpoint{2.386123in}{2.154988in}}{\pgfqpoint{2.382116in}{2.145312in}}{\pgfqpoint{2.382116in}{2.135225in}}%
\pgfpathcurveto{\pgfqpoint{2.382116in}{2.125137in}}{\pgfqpoint{2.386123in}{2.115462in}}{\pgfqpoint{2.393256in}{2.108329in}}%
\pgfpathcurveto{\pgfqpoint{2.400389in}{2.101196in}}{\pgfqpoint{2.410065in}{2.097188in}}{\pgfqpoint{2.420152in}{2.097188in}}%
\pgfpathclose%
\pgfusepath{stroke,fill}%
\end{pgfscope}%
\begin{pgfscope}%
\pgfpathrectangle{\pgfqpoint{1.280114in}{0.528000in}}{\pgfqpoint{3.487886in}{3.696000in}} %
\pgfusepath{clip}%
\pgfsetbuttcap%
\pgfsetroundjoin%
\definecolor{currentfill}{rgb}{0.121569,0.466667,0.705882}%
\pgfsetfillcolor{currentfill}%
\pgfsetlinewidth{1.003750pt}%
\definecolor{currentstroke}{rgb}{0.121569,0.466667,0.705882}%
\pgfsetstrokecolor{currentstroke}%
\pgfsetdash{}{0pt}%
\pgfpathmoveto{\pgfqpoint{1.784206in}{2.561296in}}%
\pgfpathcurveto{\pgfqpoint{1.794294in}{2.561296in}}{\pgfqpoint{1.803969in}{2.565304in}}{\pgfqpoint{1.811102in}{2.572437in}}%
\pgfpathcurveto{\pgfqpoint{1.818235in}{2.579569in}}{\pgfqpoint{1.822243in}{2.589245in}}{\pgfqpoint{1.822243in}{2.599332in}}%
\pgfpathcurveto{\pgfqpoint{1.822243in}{2.609420in}}{\pgfqpoint{1.818235in}{2.619095in}}{\pgfqpoint{1.811102in}{2.626228in}}%
\pgfpathcurveto{\pgfqpoint{1.803969in}{2.633361in}}{\pgfqpoint{1.794294in}{2.637369in}}{\pgfqpoint{1.784206in}{2.637369in}}%
\pgfpathcurveto{\pgfqpoint{1.774119in}{2.637369in}}{\pgfqpoint{1.764444in}{2.633361in}}{\pgfqpoint{1.757311in}{2.626228in}}%
\pgfpathcurveto{\pgfqpoint{1.750178in}{2.619095in}}{\pgfqpoint{1.746170in}{2.609420in}}{\pgfqpoint{1.746170in}{2.599332in}}%
\pgfpathcurveto{\pgfqpoint{1.746170in}{2.589245in}}{\pgfqpoint{1.750178in}{2.579569in}}{\pgfqpoint{1.757311in}{2.572437in}}%
\pgfpathcurveto{\pgfqpoint{1.764444in}{2.565304in}}{\pgfqpoint{1.774119in}{2.561296in}}{\pgfqpoint{1.784206in}{2.561296in}}%
\pgfpathclose%
\pgfusepath{stroke,fill}%
\end{pgfscope}%
\begin{pgfscope}%
\pgfpathrectangle{\pgfqpoint{1.280114in}{0.528000in}}{\pgfqpoint{3.487886in}{3.696000in}} %
\pgfusepath{clip}%
\pgfsetbuttcap%
\pgfsetroundjoin%
\definecolor{currentfill}{rgb}{0.121569,0.466667,0.705882}%
\pgfsetfillcolor{currentfill}%
\pgfsetlinewidth{1.003750pt}%
\definecolor{currentstroke}{rgb}{0.121569,0.466667,0.705882}%
\pgfsetstrokecolor{currentstroke}%
\pgfsetdash{}{0pt}%
\pgfpathmoveto{\pgfqpoint{1.442752in}{2.229108in}}%
\pgfpathcurveto{\pgfqpoint{1.452839in}{2.229108in}}{\pgfqpoint{1.462515in}{2.233116in}}{\pgfqpoint{1.469647in}{2.240249in}}%
\pgfpathcurveto{\pgfqpoint{1.476780in}{2.247382in}}{\pgfqpoint{1.480788in}{2.257057in}}{\pgfqpoint{1.480788in}{2.267145in}}%
\pgfpathcurveto{\pgfqpoint{1.480788in}{2.277232in}}{\pgfqpoint{1.476780in}{2.286908in}}{\pgfqpoint{1.469647in}{2.294040in}}%
\pgfpathcurveto{\pgfqpoint{1.462515in}{2.301173in}}{\pgfqpoint{1.452839in}{2.305181in}}{\pgfqpoint{1.442752in}{2.305181in}}%
\pgfpathcurveto{\pgfqpoint{1.432664in}{2.305181in}}{\pgfqpoint{1.422989in}{2.301173in}}{\pgfqpoint{1.415856in}{2.294040in}}%
\pgfpathcurveto{\pgfqpoint{1.408723in}{2.286908in}}{\pgfqpoint{1.404715in}{2.277232in}}{\pgfqpoint{1.404715in}{2.267145in}}%
\pgfpathcurveto{\pgfqpoint{1.404715in}{2.257057in}}{\pgfqpoint{1.408723in}{2.247382in}}{\pgfqpoint{1.415856in}{2.240249in}}%
\pgfpathcurveto{\pgfqpoint{1.422989in}{2.233116in}}{\pgfqpoint{1.432664in}{2.229108in}}{\pgfqpoint{1.442752in}{2.229108in}}%
\pgfpathclose%
\pgfusepath{stroke,fill}%
\end{pgfscope}%
\begin{pgfscope}%
\pgfpathrectangle{\pgfqpoint{1.280114in}{0.528000in}}{\pgfqpoint{3.487886in}{3.696000in}} %
\pgfusepath{clip}%
\pgfsetbuttcap%
\pgfsetroundjoin%
\definecolor{currentfill}{rgb}{0.121569,0.466667,0.705882}%
\pgfsetfillcolor{currentfill}%
\pgfsetlinewidth{1.003750pt}%
\definecolor{currentstroke}{rgb}{0.121569,0.466667,0.705882}%
\pgfsetstrokecolor{currentstroke}%
\pgfsetdash{}{0pt}%
\pgfpathmoveto{\pgfqpoint{3.950524in}{2.648373in}}%
\pgfpathcurveto{\pgfqpoint{3.960612in}{2.648373in}}{\pgfqpoint{3.970287in}{2.652381in}}{\pgfqpoint{3.977420in}{2.659514in}}%
\pgfpathcurveto{\pgfqpoint{3.984553in}{2.666646in}}{\pgfqpoint{3.988561in}{2.676322in}}{\pgfqpoint{3.988561in}{2.686409in}}%
\pgfpathcurveto{\pgfqpoint{3.988561in}{2.696497in}}{\pgfqpoint{3.984553in}{2.706172in}}{\pgfqpoint{3.977420in}{2.713305in}}%
\pgfpathcurveto{\pgfqpoint{3.970287in}{2.720438in}}{\pgfqpoint{3.960612in}{2.724446in}}{\pgfqpoint{3.950524in}{2.724446in}}%
\pgfpathcurveto{\pgfqpoint{3.940437in}{2.724446in}}{\pgfqpoint{3.930761in}{2.720438in}}{\pgfqpoint{3.923629in}{2.713305in}}%
\pgfpathcurveto{\pgfqpoint{3.916496in}{2.706172in}}{\pgfqpoint{3.912488in}{2.696497in}}{\pgfqpoint{3.912488in}{2.686409in}}%
\pgfpathcurveto{\pgfqpoint{3.912488in}{2.676322in}}{\pgfqpoint{3.916496in}{2.666646in}}{\pgfqpoint{3.923629in}{2.659514in}}%
\pgfpathcurveto{\pgfqpoint{3.930761in}{2.652381in}}{\pgfqpoint{3.940437in}{2.648373in}}{\pgfqpoint{3.950524in}{2.648373in}}%
\pgfpathclose%
\pgfusepath{stroke,fill}%
\end{pgfscope}%
\begin{pgfscope}%
\pgfpathrectangle{\pgfqpoint{1.280114in}{0.528000in}}{\pgfqpoint{3.487886in}{3.696000in}} %
\pgfusepath{clip}%
\pgfsetbuttcap%
\pgfsetroundjoin%
\definecolor{currentfill}{rgb}{0.121569,0.466667,0.705882}%
\pgfsetfillcolor{currentfill}%
\pgfsetlinewidth{1.003750pt}%
\definecolor{currentstroke}{rgb}{0.121569,0.466667,0.705882}%
\pgfsetstrokecolor{currentstroke}%
\pgfsetdash{}{0pt}%
\pgfpathmoveto{\pgfqpoint{3.601725in}{2.713852in}}%
\pgfpathcurveto{\pgfqpoint{3.611812in}{2.713852in}}{\pgfqpoint{3.621488in}{2.717860in}}{\pgfqpoint{3.628621in}{2.724993in}}%
\pgfpathcurveto{\pgfqpoint{3.635754in}{2.732126in}}{\pgfqpoint{3.639761in}{2.741801in}}{\pgfqpoint{3.639761in}{2.751888in}}%
\pgfpathcurveto{\pgfqpoint{3.639761in}{2.761976in}}{\pgfqpoint{3.635754in}{2.771651in}}{\pgfqpoint{3.628621in}{2.778784in}}%
\pgfpathcurveto{\pgfqpoint{3.621488in}{2.785917in}}{\pgfqpoint{3.611812in}{2.789925in}}{\pgfqpoint{3.601725in}{2.789925in}}%
\pgfpathcurveto{\pgfqpoint{3.591638in}{2.789925in}}{\pgfqpoint{3.581962in}{2.785917in}}{\pgfqpoint{3.574829in}{2.778784in}}%
\pgfpathcurveto{\pgfqpoint{3.567697in}{2.771651in}}{\pgfqpoint{3.563689in}{2.761976in}}{\pgfqpoint{3.563689in}{2.751888in}}%
\pgfpathcurveto{\pgfqpoint{3.563689in}{2.741801in}}{\pgfqpoint{3.567697in}{2.732126in}}{\pgfqpoint{3.574829in}{2.724993in}}%
\pgfpathcurveto{\pgfqpoint{3.581962in}{2.717860in}}{\pgfqpoint{3.591638in}{2.713852in}}{\pgfqpoint{3.601725in}{2.713852in}}%
\pgfpathclose%
\pgfusepath{stroke,fill}%
\end{pgfscope}%
\begin{pgfscope}%
\pgfpathrectangle{\pgfqpoint{1.280114in}{0.528000in}}{\pgfqpoint{3.487886in}{3.696000in}} %
\pgfusepath{clip}%
\pgfsetbuttcap%
\pgfsetroundjoin%
\definecolor{currentfill}{rgb}{0.121569,0.466667,0.705882}%
\pgfsetfillcolor{currentfill}%
\pgfsetlinewidth{1.003750pt}%
\definecolor{currentstroke}{rgb}{0.121569,0.466667,0.705882}%
\pgfsetstrokecolor{currentstroke}%
\pgfsetdash{}{0pt}%
\pgfpathmoveto{\pgfqpoint{3.528773in}{2.013594in}}%
\pgfpathcurveto{\pgfqpoint{3.538860in}{2.013594in}}{\pgfqpoint{3.548536in}{2.017602in}}{\pgfqpoint{3.555668in}{2.024734in}}%
\pgfpathcurveto{\pgfqpoint{3.562801in}{2.031867in}}{\pgfqpoint{3.566809in}{2.041543in}}{\pgfqpoint{3.566809in}{2.051630in}}%
\pgfpathcurveto{\pgfqpoint{3.566809in}{2.061718in}}{\pgfqpoint{3.562801in}{2.071393in}}{\pgfqpoint{3.555668in}{2.078526in}}%
\pgfpathcurveto{\pgfqpoint{3.548536in}{2.085659in}}{\pgfqpoint{3.538860in}{2.089666in}}{\pgfqpoint{3.528773in}{2.089666in}}%
\pgfpathcurveto{\pgfqpoint{3.518685in}{2.089666in}}{\pgfqpoint{3.509010in}{2.085659in}}{\pgfqpoint{3.501877in}{2.078526in}}%
\pgfpathcurveto{\pgfqpoint{3.494744in}{2.071393in}}{\pgfqpoint{3.490736in}{2.061718in}}{\pgfqpoint{3.490736in}{2.051630in}}%
\pgfpathcurveto{\pgfqpoint{3.490736in}{2.041543in}}{\pgfqpoint{3.494744in}{2.031867in}}{\pgfqpoint{3.501877in}{2.024734in}}%
\pgfpathcurveto{\pgfqpoint{3.509010in}{2.017602in}}{\pgfqpoint{3.518685in}{2.013594in}}{\pgfqpoint{3.528773in}{2.013594in}}%
\pgfpathclose%
\pgfusepath{stroke,fill}%
\end{pgfscope}%
\begin{pgfscope}%
\pgfpathrectangle{\pgfqpoint{1.280114in}{0.528000in}}{\pgfqpoint{3.487886in}{3.696000in}} %
\pgfusepath{clip}%
\pgfsetbuttcap%
\pgfsetroundjoin%
\definecolor{currentfill}{rgb}{0.121569,0.466667,0.705882}%
\pgfsetfillcolor{currentfill}%
\pgfsetlinewidth{1.003750pt}%
\definecolor{currentstroke}{rgb}{0.121569,0.466667,0.705882}%
\pgfsetstrokecolor{currentstroke}%
\pgfsetdash{}{0pt}%
\pgfpathmoveto{\pgfqpoint{4.574130in}{2.262540in}}%
\pgfpathcurveto{\pgfqpoint{4.584218in}{2.262540in}}{\pgfqpoint{4.593893in}{2.266548in}}{\pgfqpoint{4.601026in}{2.273681in}}%
\pgfpathcurveto{\pgfqpoint{4.608159in}{2.280814in}}{\pgfqpoint{4.612167in}{2.290489in}}{\pgfqpoint{4.612167in}{2.300577in}}%
\pgfpathcurveto{\pgfqpoint{4.612167in}{2.310664in}}{\pgfqpoint{4.608159in}{2.320340in}}{\pgfqpoint{4.601026in}{2.327472in}}%
\pgfpathcurveto{\pgfqpoint{4.593893in}{2.334605in}}{\pgfqpoint{4.584218in}{2.338613in}}{\pgfqpoint{4.574130in}{2.338613in}}%
\pgfpathcurveto{\pgfqpoint{4.564043in}{2.338613in}}{\pgfqpoint{4.554367in}{2.334605in}}{\pgfqpoint{4.547235in}{2.327472in}}%
\pgfpathcurveto{\pgfqpoint{4.540102in}{2.320340in}}{\pgfqpoint{4.536094in}{2.310664in}}{\pgfqpoint{4.536094in}{2.300577in}}%
\pgfpathcurveto{\pgfqpoint{4.536094in}{2.290489in}}{\pgfqpoint{4.540102in}{2.280814in}}{\pgfqpoint{4.547235in}{2.273681in}}%
\pgfpathcurveto{\pgfqpoint{4.554367in}{2.266548in}}{\pgfqpoint{4.564043in}{2.262540in}}{\pgfqpoint{4.574130in}{2.262540in}}%
\pgfpathclose%
\pgfusepath{stroke,fill}%
\end{pgfscope}%
\begin{pgfscope}%
\pgfpathrectangle{\pgfqpoint{1.280114in}{0.528000in}}{\pgfqpoint{3.487886in}{3.696000in}} %
\pgfusepath{clip}%
\pgfsetbuttcap%
\pgfsetroundjoin%
\definecolor{currentfill}{rgb}{0.121569,0.466667,0.705882}%
\pgfsetfillcolor{currentfill}%
\pgfsetlinewidth{1.003750pt}%
\definecolor{currentstroke}{rgb}{0.121569,0.466667,0.705882}%
\pgfsetstrokecolor{currentstroke}%
\pgfsetdash{}{0pt}%
\pgfpathmoveto{\pgfqpoint{2.440130in}{2.693887in}}%
\pgfpathcurveto{\pgfqpoint{2.450217in}{2.693887in}}{\pgfqpoint{2.459892in}{2.697895in}}{\pgfqpoint{2.467025in}{2.705028in}}%
\pgfpathcurveto{\pgfqpoint{2.474158in}{2.712161in}}{\pgfqpoint{2.478166in}{2.721836in}}{\pgfqpoint{2.478166in}{2.731924in}}%
\pgfpathcurveto{\pgfqpoint{2.478166in}{2.742011in}}{\pgfqpoint{2.474158in}{2.751687in}}{\pgfqpoint{2.467025in}{2.758819in}}%
\pgfpathcurveto{\pgfqpoint{2.459892in}{2.765952in}}{\pgfqpoint{2.450217in}{2.769960in}}{\pgfqpoint{2.440130in}{2.769960in}}%
\pgfpathcurveto{\pgfqpoint{2.430042in}{2.769960in}}{\pgfqpoint{2.420367in}{2.765952in}}{\pgfqpoint{2.413234in}{2.758819in}}%
\pgfpathcurveto{\pgfqpoint{2.406101in}{2.751687in}}{\pgfqpoint{2.402093in}{2.742011in}}{\pgfqpoint{2.402093in}{2.731924in}}%
\pgfpathcurveto{\pgfqpoint{2.402093in}{2.721836in}}{\pgfqpoint{2.406101in}{2.712161in}}{\pgfqpoint{2.413234in}{2.705028in}}%
\pgfpathcurveto{\pgfqpoint{2.420367in}{2.697895in}}{\pgfqpoint{2.430042in}{2.693887in}}{\pgfqpoint{2.440130in}{2.693887in}}%
\pgfpathclose%
\pgfusepath{stroke,fill}%
\end{pgfscope}%
\begin{pgfscope}%
\pgfpathrectangle{\pgfqpoint{1.280114in}{0.528000in}}{\pgfqpoint{3.487886in}{3.696000in}} %
\pgfusepath{clip}%
\pgfsetbuttcap%
\pgfsetroundjoin%
\definecolor{currentfill}{rgb}{0.121569,0.466667,0.705882}%
\pgfsetfillcolor{currentfill}%
\pgfsetlinewidth{1.003750pt}%
\definecolor{currentstroke}{rgb}{0.121569,0.466667,0.705882}%
\pgfsetstrokecolor{currentstroke}%
\pgfsetdash{}{0pt}%
\pgfpathmoveto{\pgfqpoint{2.492150in}{1.992377in}}%
\pgfpathcurveto{\pgfqpoint{2.502237in}{1.992377in}}{\pgfqpoint{2.511913in}{1.996385in}}{\pgfqpoint{2.519046in}{2.003518in}}%
\pgfpathcurveto{\pgfqpoint{2.526178in}{2.010651in}}{\pgfqpoint{2.530186in}{2.020326in}}{\pgfqpoint{2.530186in}{2.030414in}}%
\pgfpathcurveto{\pgfqpoint{2.530186in}{2.040501in}}{\pgfqpoint{2.526178in}{2.050176in}}{\pgfqpoint{2.519046in}{2.057309in}}%
\pgfpathcurveto{\pgfqpoint{2.511913in}{2.064442in}}{\pgfqpoint{2.502237in}{2.068450in}}{\pgfqpoint{2.492150in}{2.068450in}}%
\pgfpathcurveto{\pgfqpoint{2.482063in}{2.068450in}}{\pgfqpoint{2.472387in}{2.064442in}}{\pgfqpoint{2.465254in}{2.057309in}}%
\pgfpathcurveto{\pgfqpoint{2.458121in}{2.050176in}}{\pgfqpoint{2.454114in}{2.040501in}}{\pgfqpoint{2.454114in}{2.030414in}}%
\pgfpathcurveto{\pgfqpoint{2.454114in}{2.020326in}}{\pgfqpoint{2.458121in}{2.010651in}}{\pgfqpoint{2.465254in}{2.003518in}}%
\pgfpathcurveto{\pgfqpoint{2.472387in}{1.996385in}}{\pgfqpoint{2.482063in}{1.992377in}}{\pgfqpoint{2.492150in}{1.992377in}}%
\pgfpathclose%
\pgfusepath{stroke,fill}%
\end{pgfscope}%
\begin{pgfscope}%
\pgfpathrectangle{\pgfqpoint{1.280114in}{0.528000in}}{\pgfqpoint{3.487886in}{3.696000in}} %
\pgfusepath{clip}%
\pgfsetbuttcap%
\pgfsetroundjoin%
\definecolor{currentfill}{rgb}{0.121569,0.466667,0.705882}%
\pgfsetfillcolor{currentfill}%
\pgfsetlinewidth{1.003750pt}%
\definecolor{currentstroke}{rgb}{0.121569,0.466667,0.705882}%
\pgfsetstrokecolor{currentstroke}%
\pgfsetdash{}{0pt}%
\pgfpathmoveto{\pgfqpoint{4.117564in}{2.111941in}}%
\pgfpathcurveto{\pgfqpoint{4.127652in}{2.111941in}}{\pgfqpoint{4.137327in}{2.115949in}}{\pgfqpoint{4.144460in}{2.123082in}}%
\pgfpathcurveto{\pgfqpoint{4.151593in}{2.130215in}}{\pgfqpoint{4.155601in}{2.139890in}}{\pgfqpoint{4.155601in}{2.149977in}}%
\pgfpathcurveto{\pgfqpoint{4.155601in}{2.160065in}}{\pgfqpoint{4.151593in}{2.169740in}}{\pgfqpoint{4.144460in}{2.176873in}}%
\pgfpathcurveto{\pgfqpoint{4.137327in}{2.184006in}}{\pgfqpoint{4.127652in}{2.188014in}}{\pgfqpoint{4.117564in}{2.188014in}}%
\pgfpathcurveto{\pgfqpoint{4.107477in}{2.188014in}}{\pgfqpoint{4.097801in}{2.184006in}}{\pgfqpoint{4.090669in}{2.176873in}}%
\pgfpathcurveto{\pgfqpoint{4.083536in}{2.169740in}}{\pgfqpoint{4.079528in}{2.160065in}}{\pgfqpoint{4.079528in}{2.149977in}}%
\pgfpathcurveto{\pgfqpoint{4.079528in}{2.139890in}}{\pgfqpoint{4.083536in}{2.130215in}}{\pgfqpoint{4.090669in}{2.123082in}}%
\pgfpathcurveto{\pgfqpoint{4.097801in}{2.115949in}}{\pgfqpoint{4.107477in}{2.111941in}}{\pgfqpoint{4.117564in}{2.111941in}}%
\pgfpathclose%
\pgfusepath{stroke,fill}%
\end{pgfscope}%
\begin{pgfscope}%
\pgfpathrectangle{\pgfqpoint{1.280114in}{0.528000in}}{\pgfqpoint{3.487886in}{3.696000in}} %
\pgfusepath{clip}%
\pgfsetbuttcap%
\pgfsetroundjoin%
\definecolor{currentfill}{rgb}{0.121569,0.466667,0.705882}%
\pgfsetfillcolor{currentfill}%
\pgfsetlinewidth{1.003750pt}%
\definecolor{currentstroke}{rgb}{0.121569,0.466667,0.705882}%
\pgfsetstrokecolor{currentstroke}%
\pgfsetdash{}{0pt}%
\pgfpathmoveto{\pgfqpoint{4.252906in}{2.129559in}}%
\pgfpathcurveto{\pgfqpoint{4.262994in}{2.129559in}}{\pgfqpoint{4.272669in}{2.133567in}}{\pgfqpoint{4.279802in}{2.140700in}}%
\pgfpathcurveto{\pgfqpoint{4.286935in}{2.147832in}}{\pgfqpoint{4.290942in}{2.157508in}}{\pgfqpoint{4.290942in}{2.167595in}}%
\pgfpathcurveto{\pgfqpoint{4.290942in}{2.177683in}}{\pgfqpoint{4.286935in}{2.187358in}}{\pgfqpoint{4.279802in}{2.194491in}}%
\pgfpathcurveto{\pgfqpoint{4.272669in}{2.201624in}}{\pgfqpoint{4.262994in}{2.205632in}}{\pgfqpoint{4.252906in}{2.205632in}}%
\pgfpathcurveto{\pgfqpoint{4.242819in}{2.205632in}}{\pgfqpoint{4.233143in}{2.201624in}}{\pgfqpoint{4.226010in}{2.194491in}}%
\pgfpathcurveto{\pgfqpoint{4.218878in}{2.187358in}}{\pgfqpoint{4.214870in}{2.177683in}}{\pgfqpoint{4.214870in}{2.167595in}}%
\pgfpathcurveto{\pgfqpoint{4.214870in}{2.157508in}}{\pgfqpoint{4.218878in}{2.147832in}}{\pgfqpoint{4.226010in}{2.140700in}}%
\pgfpathcurveto{\pgfqpoint{4.233143in}{2.133567in}}{\pgfqpoint{4.242819in}{2.129559in}}{\pgfqpoint{4.252906in}{2.129559in}}%
\pgfpathclose%
\pgfusepath{stroke,fill}%
\end{pgfscope}%
\begin{pgfscope}%
\pgfpathrectangle{\pgfqpoint{1.280114in}{0.528000in}}{\pgfqpoint{3.487886in}{3.696000in}} %
\pgfusepath{clip}%
\pgfsetbuttcap%
\pgfsetroundjoin%
\definecolor{currentfill}{rgb}{0.121569,0.466667,0.705882}%
\pgfsetfillcolor{currentfill}%
\pgfsetlinewidth{1.003750pt}%
\definecolor{currentstroke}{rgb}{0.121569,0.466667,0.705882}%
\pgfsetstrokecolor{currentstroke}%
\pgfsetdash{}{0pt}%
\pgfpathmoveto{\pgfqpoint{4.337489in}{2.567442in}}%
\pgfpathcurveto{\pgfqpoint{4.347577in}{2.567442in}}{\pgfqpoint{4.357252in}{2.571450in}}{\pgfqpoint{4.364385in}{2.578583in}}%
\pgfpathcurveto{\pgfqpoint{4.371518in}{2.585716in}}{\pgfqpoint{4.375526in}{2.595391in}}{\pgfqpoint{4.375526in}{2.605478in}}%
\pgfpathcurveto{\pgfqpoint{4.375526in}{2.615566in}}{\pgfqpoint{4.371518in}{2.625241in}}{\pgfqpoint{4.364385in}{2.632374in}}%
\pgfpathcurveto{\pgfqpoint{4.357252in}{2.639507in}}{\pgfqpoint{4.347577in}{2.643515in}}{\pgfqpoint{4.337489in}{2.643515in}}%
\pgfpathcurveto{\pgfqpoint{4.327402in}{2.643515in}}{\pgfqpoint{4.317726in}{2.639507in}}{\pgfqpoint{4.310594in}{2.632374in}}%
\pgfpathcurveto{\pgfqpoint{4.303461in}{2.625241in}}{\pgfqpoint{4.299453in}{2.615566in}}{\pgfqpoint{4.299453in}{2.605478in}}%
\pgfpathcurveto{\pgfqpoint{4.299453in}{2.595391in}}{\pgfqpoint{4.303461in}{2.585716in}}{\pgfqpoint{4.310594in}{2.578583in}}%
\pgfpathcurveto{\pgfqpoint{4.317726in}{2.571450in}}{\pgfqpoint{4.327402in}{2.567442in}}{\pgfqpoint{4.337489in}{2.567442in}}%
\pgfpathclose%
\pgfusepath{stroke,fill}%
\end{pgfscope}%
\begin{pgfscope}%
\pgfpathrectangle{\pgfqpoint{1.280114in}{0.528000in}}{\pgfqpoint{3.487886in}{3.696000in}} %
\pgfusepath{clip}%
\pgfsetbuttcap%
\pgfsetroundjoin%
\definecolor{currentfill}{rgb}{0.121569,0.466667,0.705882}%
\pgfsetfillcolor{currentfill}%
\pgfsetlinewidth{1.003750pt}%
\definecolor{currentstroke}{rgb}{0.121569,0.466667,0.705882}%
\pgfsetstrokecolor{currentstroke}%
\pgfsetdash{}{0pt}%
\pgfpathmoveto{\pgfqpoint{4.491601in}{2.451038in}}%
\pgfpathcurveto{\pgfqpoint{4.501688in}{2.451038in}}{\pgfqpoint{4.511364in}{2.455046in}}{\pgfqpoint{4.518496in}{2.462179in}}%
\pgfpathcurveto{\pgfqpoint{4.525629in}{2.469312in}}{\pgfqpoint{4.529637in}{2.478987in}}{\pgfqpoint{4.529637in}{2.489075in}}%
\pgfpathcurveto{\pgfqpoint{4.529637in}{2.499162in}}{\pgfqpoint{4.525629in}{2.508838in}}{\pgfqpoint{4.518496in}{2.515970in}}%
\pgfpathcurveto{\pgfqpoint{4.511364in}{2.523103in}}{\pgfqpoint{4.501688in}{2.527111in}}{\pgfqpoint{4.491601in}{2.527111in}}%
\pgfpathcurveto{\pgfqpoint{4.481513in}{2.527111in}}{\pgfqpoint{4.471838in}{2.523103in}}{\pgfqpoint{4.464705in}{2.515970in}}%
\pgfpathcurveto{\pgfqpoint{4.457572in}{2.508838in}}{\pgfqpoint{4.453564in}{2.499162in}}{\pgfqpoint{4.453564in}{2.489075in}}%
\pgfpathcurveto{\pgfqpoint{4.453564in}{2.478987in}}{\pgfqpoint{4.457572in}{2.469312in}}{\pgfqpoint{4.464705in}{2.462179in}}%
\pgfpathcurveto{\pgfqpoint{4.471838in}{2.455046in}}{\pgfqpoint{4.481513in}{2.451038in}}{\pgfqpoint{4.491601in}{2.451038in}}%
\pgfpathclose%
\pgfusepath{stroke,fill}%
\end{pgfscope}%
\begin{pgfscope}%
\pgfpathrectangle{\pgfqpoint{1.280114in}{0.528000in}}{\pgfqpoint{3.487886in}{3.696000in}} %
\pgfusepath{clip}%
\pgfsetbuttcap%
\pgfsetroundjoin%
\definecolor{currentfill}{rgb}{0.121569,0.466667,0.705882}%
\pgfsetfillcolor{currentfill}%
\pgfsetlinewidth{1.003750pt}%
\definecolor{currentstroke}{rgb}{0.121569,0.466667,0.705882}%
\pgfsetstrokecolor{currentstroke}%
\pgfsetdash{}{0pt}%
\pgfpathmoveto{\pgfqpoint{3.685400in}{2.716321in}}%
\pgfpathcurveto{\pgfqpoint{3.695488in}{2.716321in}}{\pgfqpoint{3.705163in}{2.720328in}}{\pgfqpoint{3.712296in}{2.727461in}}%
\pgfpathcurveto{\pgfqpoint{3.719429in}{2.734594in}}{\pgfqpoint{3.723437in}{2.744270in}}{\pgfqpoint{3.723437in}{2.754357in}}%
\pgfpathcurveto{\pgfqpoint{3.723437in}{2.764444in}}{\pgfqpoint{3.719429in}{2.774120in}}{\pgfqpoint{3.712296in}{2.781253in}}%
\pgfpathcurveto{\pgfqpoint{3.705163in}{2.788386in}}{\pgfqpoint{3.695488in}{2.792393in}}{\pgfqpoint{3.685400in}{2.792393in}}%
\pgfpathcurveto{\pgfqpoint{3.675313in}{2.792393in}}{\pgfqpoint{3.665637in}{2.788386in}}{\pgfqpoint{3.658505in}{2.781253in}}%
\pgfpathcurveto{\pgfqpoint{3.651372in}{2.774120in}}{\pgfqpoint{3.647364in}{2.764444in}}{\pgfqpoint{3.647364in}{2.754357in}}%
\pgfpathcurveto{\pgfqpoint{3.647364in}{2.744270in}}{\pgfqpoint{3.651372in}{2.734594in}}{\pgfqpoint{3.658505in}{2.727461in}}%
\pgfpathcurveto{\pgfqpoint{3.665637in}{2.720328in}}{\pgfqpoint{3.675313in}{2.716321in}}{\pgfqpoint{3.685400in}{2.716321in}}%
\pgfpathclose%
\pgfusepath{stroke,fill}%
\end{pgfscope}%
\begin{pgfscope}%
\pgfpathrectangle{\pgfqpoint{1.280114in}{0.528000in}}{\pgfqpoint{3.487886in}{3.696000in}} %
\pgfusepath{clip}%
\pgfsetbuttcap%
\pgfsetroundjoin%
\definecolor{currentfill}{rgb}{0.121569,0.466667,0.705882}%
\pgfsetfillcolor{currentfill}%
\pgfsetlinewidth{1.003750pt}%
\definecolor{currentstroke}{rgb}{0.121569,0.466667,0.705882}%
\pgfsetstrokecolor{currentstroke}%
\pgfsetdash{}{0pt}%
\pgfpathmoveto{\pgfqpoint{4.149005in}{2.113947in}}%
\pgfpathcurveto{\pgfqpoint{4.159093in}{2.113947in}}{\pgfqpoint{4.168768in}{2.117955in}}{\pgfqpoint{4.175901in}{2.125088in}}%
\pgfpathcurveto{\pgfqpoint{4.183034in}{2.132221in}}{\pgfqpoint{4.187042in}{2.141896in}}{\pgfqpoint{4.187042in}{2.151984in}}%
\pgfpathcurveto{\pgfqpoint{4.187042in}{2.162071in}}{\pgfqpoint{4.183034in}{2.171746in}}{\pgfqpoint{4.175901in}{2.178879in}}%
\pgfpathcurveto{\pgfqpoint{4.168768in}{2.186012in}}{\pgfqpoint{4.159093in}{2.190020in}}{\pgfqpoint{4.149005in}{2.190020in}}%
\pgfpathcurveto{\pgfqpoint{4.138918in}{2.190020in}}{\pgfqpoint{4.129243in}{2.186012in}}{\pgfqpoint{4.122110in}{2.178879in}}%
\pgfpathcurveto{\pgfqpoint{4.114977in}{2.171746in}}{\pgfqpoint{4.110969in}{2.162071in}}{\pgfqpoint{4.110969in}{2.151984in}}%
\pgfpathcurveto{\pgfqpoint{4.110969in}{2.141896in}}{\pgfqpoint{4.114977in}{2.132221in}}{\pgfqpoint{4.122110in}{2.125088in}}%
\pgfpathcurveto{\pgfqpoint{4.129243in}{2.117955in}}{\pgfqpoint{4.138918in}{2.113947in}}{\pgfqpoint{4.149005in}{2.113947in}}%
\pgfpathclose%
\pgfusepath{stroke,fill}%
\end{pgfscope}%
\begin{pgfscope}%
\pgfpathrectangle{\pgfqpoint{1.280114in}{0.528000in}}{\pgfqpoint{3.487886in}{3.696000in}} %
\pgfusepath{clip}%
\pgfsetbuttcap%
\pgfsetroundjoin%
\definecolor{currentfill}{rgb}{0.121569,0.466667,0.705882}%
\pgfsetfillcolor{currentfill}%
\pgfsetlinewidth{1.003750pt}%
\definecolor{currentstroke}{rgb}{0.121569,0.466667,0.705882}%
\pgfsetstrokecolor{currentstroke}%
\pgfsetdash{}{0pt}%
\pgfpathmoveto{\pgfqpoint{4.261900in}{2.637926in}}%
\pgfpathcurveto{\pgfqpoint{4.271987in}{2.637926in}}{\pgfqpoint{4.281663in}{2.641933in}}{\pgfqpoint{4.288796in}{2.649066in}}%
\pgfpathcurveto{\pgfqpoint{4.295929in}{2.656199in}}{\pgfqpoint{4.299936in}{2.665875in}}{\pgfqpoint{4.299936in}{2.675962in}}%
\pgfpathcurveto{\pgfqpoint{4.299936in}{2.686049in}}{\pgfqpoint{4.295929in}{2.695725in}}{\pgfqpoint{4.288796in}{2.702858in}}%
\pgfpathcurveto{\pgfqpoint{4.281663in}{2.709991in}}{\pgfqpoint{4.271987in}{2.713998in}}{\pgfqpoint{4.261900in}{2.713998in}}%
\pgfpathcurveto{\pgfqpoint{4.251813in}{2.713998in}}{\pgfqpoint{4.242137in}{2.709991in}}{\pgfqpoint{4.235004in}{2.702858in}}%
\pgfpathcurveto{\pgfqpoint{4.227872in}{2.695725in}}{\pgfqpoint{4.223864in}{2.686049in}}{\pgfqpoint{4.223864in}{2.675962in}}%
\pgfpathcurveto{\pgfqpoint{4.223864in}{2.665875in}}{\pgfqpoint{4.227872in}{2.656199in}}{\pgfqpoint{4.235004in}{2.649066in}}%
\pgfpathcurveto{\pgfqpoint{4.242137in}{2.641933in}}{\pgfqpoint{4.251813in}{2.637926in}}{\pgfqpoint{4.261900in}{2.637926in}}%
\pgfpathclose%
\pgfusepath{stroke,fill}%
\end{pgfscope}%
\begin{pgfscope}%
\pgfpathrectangle{\pgfqpoint{1.280114in}{0.528000in}}{\pgfqpoint{3.487886in}{3.696000in}} %
\pgfusepath{clip}%
\pgfsetbuttcap%
\pgfsetroundjoin%
\definecolor{currentfill}{rgb}{0.121569,0.466667,0.705882}%
\pgfsetfillcolor{currentfill}%
\pgfsetlinewidth{1.003750pt}%
\definecolor{currentstroke}{rgb}{0.121569,0.466667,0.705882}%
\pgfsetstrokecolor{currentstroke}%
\pgfsetdash{}{0pt}%
\pgfpathmoveto{\pgfqpoint{1.852379in}{2.492214in}}%
\pgfpathcurveto{\pgfqpoint{1.862467in}{2.492214in}}{\pgfqpoint{1.872142in}{2.496221in}}{\pgfqpoint{1.879275in}{2.503354in}}%
\pgfpathcurveto{\pgfqpoint{1.886408in}{2.510487in}}{\pgfqpoint{1.890416in}{2.520163in}}{\pgfqpoint{1.890416in}{2.530250in}}%
\pgfpathcurveto{\pgfqpoint{1.890416in}{2.540337in}}{\pgfqpoint{1.886408in}{2.550013in}}{\pgfqpoint{1.879275in}{2.557146in}}%
\pgfpathcurveto{\pgfqpoint{1.872142in}{2.564279in}}{\pgfqpoint{1.862467in}{2.568286in}}{\pgfqpoint{1.852379in}{2.568286in}}%
\pgfpathcurveto{\pgfqpoint{1.842292in}{2.568286in}}{\pgfqpoint{1.832616in}{2.564279in}}{\pgfqpoint{1.825484in}{2.557146in}}%
\pgfpathcurveto{\pgfqpoint{1.818351in}{2.550013in}}{\pgfqpoint{1.814343in}{2.540337in}}{\pgfqpoint{1.814343in}{2.530250in}}%
\pgfpathcurveto{\pgfqpoint{1.814343in}{2.520163in}}{\pgfqpoint{1.818351in}{2.510487in}}{\pgfqpoint{1.825484in}{2.503354in}}%
\pgfpathcurveto{\pgfqpoint{1.832616in}{2.496221in}}{\pgfqpoint{1.842292in}{2.492214in}}{\pgfqpoint{1.852379in}{2.492214in}}%
\pgfpathclose%
\pgfusepath{stroke,fill}%
\end{pgfscope}%
\begin{pgfscope}%
\pgfpathrectangle{\pgfqpoint{1.280114in}{0.528000in}}{\pgfqpoint{3.487886in}{3.696000in}} %
\pgfusepath{clip}%
\pgfsetbuttcap%
\pgfsetroundjoin%
\definecolor{currentfill}{rgb}{0.121569,0.466667,0.705882}%
\pgfsetfillcolor{currentfill}%
\pgfsetlinewidth{1.003750pt}%
\definecolor{currentstroke}{rgb}{0.121569,0.466667,0.705882}%
\pgfsetstrokecolor{currentstroke}%
\pgfsetdash{}{0pt}%
\pgfpathmoveto{\pgfqpoint{4.597747in}{2.251477in}}%
\pgfpathcurveto{\pgfqpoint{4.607835in}{2.251477in}}{\pgfqpoint{4.617510in}{2.255485in}}{\pgfqpoint{4.624643in}{2.262617in}}%
\pgfpathcurveto{\pgfqpoint{4.631776in}{2.269750in}}{\pgfqpoint{4.635783in}{2.279426in}}{\pgfqpoint{4.635783in}{2.289513in}}%
\pgfpathcurveto{\pgfqpoint{4.635783in}{2.299600in}}{\pgfqpoint{4.631776in}{2.309276in}}{\pgfqpoint{4.624643in}{2.316409in}}%
\pgfpathcurveto{\pgfqpoint{4.617510in}{2.323542in}}{\pgfqpoint{4.607835in}{2.327549in}}{\pgfqpoint{4.597747in}{2.327549in}}%
\pgfpathcurveto{\pgfqpoint{4.587660in}{2.327549in}}{\pgfqpoint{4.577984in}{2.323542in}}{\pgfqpoint{4.570851in}{2.316409in}}%
\pgfpathcurveto{\pgfqpoint{4.563719in}{2.309276in}}{\pgfqpoint{4.559711in}{2.299600in}}{\pgfqpoint{4.559711in}{2.289513in}}%
\pgfpathcurveto{\pgfqpoint{4.559711in}{2.279426in}}{\pgfqpoint{4.563719in}{2.269750in}}{\pgfqpoint{4.570851in}{2.262617in}}%
\pgfpathcurveto{\pgfqpoint{4.577984in}{2.255485in}}{\pgfqpoint{4.587660in}{2.251477in}}{\pgfqpoint{4.597747in}{2.251477in}}%
\pgfpathclose%
\pgfusepath{stroke,fill}%
\end{pgfscope}%
\begin{pgfscope}%
\pgfpathrectangle{\pgfqpoint{1.280114in}{0.528000in}}{\pgfqpoint{3.487886in}{3.696000in}} %
\pgfusepath{clip}%
\pgfsetbuttcap%
\pgfsetroundjoin%
\definecolor{currentfill}{rgb}{0.121569,0.466667,0.705882}%
\pgfsetfillcolor{currentfill}%
\pgfsetlinewidth{1.003750pt}%
\definecolor{currentstroke}{rgb}{0.121569,0.466667,0.705882}%
\pgfsetstrokecolor{currentstroke}%
\pgfsetdash{}{0pt}%
\pgfpathmoveto{\pgfqpoint{1.606743in}{2.312373in}}%
\pgfpathcurveto{\pgfqpoint{1.616830in}{2.312373in}}{\pgfqpoint{1.626506in}{2.316381in}}{\pgfqpoint{1.633639in}{2.323514in}}%
\pgfpathcurveto{\pgfqpoint{1.640772in}{2.330647in}}{\pgfqpoint{1.644779in}{2.340322in}}{\pgfqpoint{1.644779in}{2.350410in}}%
\pgfpathcurveto{\pgfqpoint{1.644779in}{2.360497in}}{\pgfqpoint{1.640772in}{2.370173in}}{\pgfqpoint{1.633639in}{2.377305in}}%
\pgfpathcurveto{\pgfqpoint{1.626506in}{2.384438in}}{\pgfqpoint{1.616830in}{2.388446in}}{\pgfqpoint{1.606743in}{2.388446in}}%
\pgfpathcurveto{\pgfqpoint{1.596656in}{2.388446in}}{\pgfqpoint{1.586980in}{2.384438in}}{\pgfqpoint{1.579847in}{2.377305in}}%
\pgfpathcurveto{\pgfqpoint{1.572714in}{2.370173in}}{\pgfqpoint{1.568707in}{2.360497in}}{\pgfqpoint{1.568707in}{2.350410in}}%
\pgfpathcurveto{\pgfqpoint{1.568707in}{2.340322in}}{\pgfqpoint{1.572714in}{2.330647in}}{\pgfqpoint{1.579847in}{2.323514in}}%
\pgfpathcurveto{\pgfqpoint{1.586980in}{2.316381in}}{\pgfqpoint{1.596656in}{2.312373in}}{\pgfqpoint{1.606743in}{2.312373in}}%
\pgfpathclose%
\pgfusepath{stroke,fill}%
\end{pgfscope}%
\begin{pgfscope}%
\pgfpathrectangle{\pgfqpoint{1.280114in}{0.528000in}}{\pgfqpoint{3.487886in}{3.696000in}} %
\pgfusepath{clip}%
\pgfsetbuttcap%
\pgfsetroundjoin%
\definecolor{currentfill}{rgb}{0.121569,0.466667,0.705882}%
\pgfsetfillcolor{currentfill}%
\pgfsetlinewidth{1.003750pt}%
\definecolor{currentstroke}{rgb}{0.121569,0.466667,0.705882}%
\pgfsetstrokecolor{currentstroke}%
\pgfsetdash{}{0pt}%
\pgfpathmoveto{\pgfqpoint{2.469951in}{2.085241in}}%
\pgfpathcurveto{\pgfqpoint{2.480038in}{2.085241in}}{\pgfqpoint{2.489714in}{2.089249in}}{\pgfqpoint{2.496847in}{2.096382in}}%
\pgfpathcurveto{\pgfqpoint{2.503980in}{2.103515in}}{\pgfqpoint{2.507987in}{2.113190in}}{\pgfqpoint{2.507987in}{2.123278in}}%
\pgfpathcurveto{\pgfqpoint{2.507987in}{2.133365in}}{\pgfqpoint{2.503980in}{2.143040in}}{\pgfqpoint{2.496847in}{2.150173in}}%
\pgfpathcurveto{\pgfqpoint{2.489714in}{2.157306in}}{\pgfqpoint{2.480038in}{2.161314in}}{\pgfqpoint{2.469951in}{2.161314in}}%
\pgfpathcurveto{\pgfqpoint{2.459864in}{2.161314in}}{\pgfqpoint{2.450188in}{2.157306in}}{\pgfqpoint{2.443055in}{2.150173in}}%
\pgfpathcurveto{\pgfqpoint{2.435923in}{2.143040in}}{\pgfqpoint{2.431915in}{2.133365in}}{\pgfqpoint{2.431915in}{2.123278in}}%
\pgfpathcurveto{\pgfqpoint{2.431915in}{2.113190in}}{\pgfqpoint{2.435923in}{2.103515in}}{\pgfqpoint{2.443055in}{2.096382in}}%
\pgfpathcurveto{\pgfqpoint{2.450188in}{2.089249in}}{\pgfqpoint{2.459864in}{2.085241in}}{\pgfqpoint{2.469951in}{2.085241in}}%
\pgfpathclose%
\pgfusepath{stroke,fill}%
\end{pgfscope}%
\begin{pgfscope}%
\pgfpathrectangle{\pgfqpoint{1.280114in}{0.528000in}}{\pgfqpoint{3.487886in}{3.696000in}} %
\pgfusepath{clip}%
\pgfsetbuttcap%
\pgfsetroundjoin%
\definecolor{currentfill}{rgb}{0.121569,0.466667,0.705882}%
\pgfsetfillcolor{currentfill}%
\pgfsetlinewidth{1.003750pt}%
\definecolor{currentstroke}{rgb}{0.121569,0.466667,0.705882}%
\pgfsetstrokecolor{currentstroke}%
\pgfsetdash{}{0pt}%
\pgfpathmoveto{\pgfqpoint{2.378504in}{2.699009in}}%
\pgfpathcurveto{\pgfqpoint{2.388591in}{2.699009in}}{\pgfqpoint{2.398267in}{2.703017in}}{\pgfqpoint{2.405400in}{2.710150in}}%
\pgfpathcurveto{\pgfqpoint{2.412533in}{2.717283in}}{\pgfqpoint{2.416540in}{2.726958in}}{\pgfqpoint{2.416540in}{2.737046in}}%
\pgfpathcurveto{\pgfqpoint{2.416540in}{2.747133in}}{\pgfqpoint{2.412533in}{2.756809in}}{\pgfqpoint{2.405400in}{2.763941in}}%
\pgfpathcurveto{\pgfqpoint{2.398267in}{2.771074in}}{\pgfqpoint{2.388591in}{2.775082in}}{\pgfqpoint{2.378504in}{2.775082in}}%
\pgfpathcurveto{\pgfqpoint{2.368417in}{2.775082in}}{\pgfqpoint{2.358741in}{2.771074in}}{\pgfqpoint{2.351608in}{2.763941in}}%
\pgfpathcurveto{\pgfqpoint{2.344475in}{2.756809in}}{\pgfqpoint{2.340468in}{2.747133in}}{\pgfqpoint{2.340468in}{2.737046in}}%
\pgfpathcurveto{\pgfqpoint{2.340468in}{2.726958in}}{\pgfqpoint{2.344475in}{2.717283in}}{\pgfqpoint{2.351608in}{2.710150in}}%
\pgfpathcurveto{\pgfqpoint{2.358741in}{2.703017in}}{\pgfqpoint{2.368417in}{2.699009in}}{\pgfqpoint{2.378504in}{2.699009in}}%
\pgfpathclose%
\pgfusepath{stroke,fill}%
\end{pgfscope}%
\begin{pgfscope}%
\pgfpathrectangle{\pgfqpoint{1.280114in}{0.528000in}}{\pgfqpoint{3.487886in}{3.696000in}} %
\pgfusepath{clip}%
\pgfsetbuttcap%
\pgfsetroundjoin%
\definecolor{currentfill}{rgb}{0.121569,0.466667,0.705882}%
\pgfsetfillcolor{currentfill}%
\pgfsetlinewidth{1.003750pt}%
\definecolor{currentstroke}{rgb}{0.121569,0.466667,0.705882}%
\pgfsetstrokecolor{currentstroke}%
\pgfsetdash{}{0pt}%
\pgfpathmoveto{\pgfqpoint{2.507264in}{2.024725in}}%
\pgfpathcurveto{\pgfqpoint{2.517351in}{2.024725in}}{\pgfqpoint{2.527026in}{2.028732in}}{\pgfqpoint{2.534159in}{2.035865in}}%
\pgfpathcurveto{\pgfqpoint{2.541292in}{2.042998in}}{\pgfqpoint{2.545300in}{2.052674in}}{\pgfqpoint{2.545300in}{2.062761in}}%
\pgfpathcurveto{\pgfqpoint{2.545300in}{2.072848in}}{\pgfqpoint{2.541292in}{2.082524in}}{\pgfqpoint{2.534159in}{2.089657in}}%
\pgfpathcurveto{\pgfqpoint{2.527026in}{2.096789in}}{\pgfqpoint{2.517351in}{2.100797in}}{\pgfqpoint{2.507264in}{2.100797in}}%
\pgfpathcurveto{\pgfqpoint{2.497176in}{2.100797in}}{\pgfqpoint{2.487501in}{2.096789in}}{\pgfqpoint{2.480368in}{2.089657in}}%
\pgfpathcurveto{\pgfqpoint{2.473235in}{2.082524in}}{\pgfqpoint{2.469227in}{2.072848in}}{\pgfqpoint{2.469227in}{2.062761in}}%
\pgfpathcurveto{\pgfqpoint{2.469227in}{2.052674in}}{\pgfqpoint{2.473235in}{2.042998in}}{\pgfqpoint{2.480368in}{2.035865in}}%
\pgfpathcurveto{\pgfqpoint{2.487501in}{2.028732in}}{\pgfqpoint{2.497176in}{2.024725in}}{\pgfqpoint{2.507264in}{2.024725in}}%
\pgfpathclose%
\pgfusepath{stroke,fill}%
\end{pgfscope}%
\begin{pgfscope}%
\pgfpathrectangle{\pgfqpoint{1.280114in}{0.528000in}}{\pgfqpoint{3.487886in}{3.696000in}} %
\pgfusepath{clip}%
\pgfsetbuttcap%
\pgfsetroundjoin%
\definecolor{currentfill}{rgb}{0.121569,0.466667,0.705882}%
\pgfsetfillcolor{currentfill}%
\pgfsetlinewidth{1.003750pt}%
\definecolor{currentstroke}{rgb}{0.121569,0.466667,0.705882}%
\pgfsetstrokecolor{currentstroke}%
\pgfsetdash{}{0pt}%
\pgfpathmoveto{\pgfqpoint{3.800009in}{2.039727in}}%
\pgfpathcurveto{\pgfqpoint{3.810097in}{2.039727in}}{\pgfqpoint{3.819772in}{2.043734in}}{\pgfqpoint{3.826905in}{2.050867in}}%
\pgfpathcurveto{\pgfqpoint{3.834038in}{2.058000in}}{\pgfqpoint{3.838046in}{2.067676in}}{\pgfqpoint{3.838046in}{2.077763in}}%
\pgfpathcurveto{\pgfqpoint{3.838046in}{2.087850in}}{\pgfqpoint{3.834038in}{2.097526in}}{\pgfqpoint{3.826905in}{2.104659in}}%
\pgfpathcurveto{\pgfqpoint{3.819772in}{2.111791in}}{\pgfqpoint{3.810097in}{2.115799in}}{\pgfqpoint{3.800009in}{2.115799in}}%
\pgfpathcurveto{\pgfqpoint{3.789922in}{2.115799in}}{\pgfqpoint{3.780246in}{2.111791in}}{\pgfqpoint{3.773114in}{2.104659in}}%
\pgfpathcurveto{\pgfqpoint{3.765981in}{2.097526in}}{\pgfqpoint{3.761973in}{2.087850in}}{\pgfqpoint{3.761973in}{2.077763in}}%
\pgfpathcurveto{\pgfqpoint{3.761973in}{2.067676in}}{\pgfqpoint{3.765981in}{2.058000in}}{\pgfqpoint{3.773114in}{2.050867in}}%
\pgfpathcurveto{\pgfqpoint{3.780246in}{2.043734in}}{\pgfqpoint{3.789922in}{2.039727in}}{\pgfqpoint{3.800009in}{2.039727in}}%
\pgfpathclose%
\pgfusepath{stroke,fill}%
\end{pgfscope}%
\begin{pgfscope}%
\pgfpathrectangle{\pgfqpoint{1.280114in}{0.528000in}}{\pgfqpoint{3.487886in}{3.696000in}} %
\pgfusepath{clip}%
\pgfsetbuttcap%
\pgfsetroundjoin%
\definecolor{currentfill}{rgb}{0.121569,0.466667,0.705882}%
\pgfsetfillcolor{currentfill}%
\pgfsetlinewidth{1.003750pt}%
\definecolor{currentstroke}{rgb}{0.121569,0.466667,0.705882}%
\pgfsetstrokecolor{currentstroke}%
\pgfsetdash{}{0pt}%
\pgfpathmoveto{\pgfqpoint{4.477698in}{2.442959in}}%
\pgfpathcurveto{\pgfqpoint{4.487786in}{2.442959in}}{\pgfqpoint{4.497461in}{2.446966in}}{\pgfqpoint{4.504594in}{2.454099in}}%
\pgfpathcurveto{\pgfqpoint{4.511727in}{2.461232in}}{\pgfqpoint{4.515734in}{2.470908in}}{\pgfqpoint{4.515734in}{2.480995in}}%
\pgfpathcurveto{\pgfqpoint{4.515734in}{2.491082in}}{\pgfqpoint{4.511727in}{2.500758in}}{\pgfqpoint{4.504594in}{2.507891in}}%
\pgfpathcurveto{\pgfqpoint{4.497461in}{2.515024in}}{\pgfqpoint{4.487786in}{2.519031in}}{\pgfqpoint{4.477698in}{2.519031in}}%
\pgfpathcurveto{\pgfqpoint{4.467611in}{2.519031in}}{\pgfqpoint{4.457935in}{2.515024in}}{\pgfqpoint{4.450802in}{2.507891in}}%
\pgfpathcurveto{\pgfqpoint{4.443670in}{2.500758in}}{\pgfqpoint{4.439662in}{2.491082in}}{\pgfqpoint{4.439662in}{2.480995in}}%
\pgfpathcurveto{\pgfqpoint{4.439662in}{2.470908in}}{\pgfqpoint{4.443670in}{2.461232in}}{\pgfqpoint{4.450802in}{2.454099in}}%
\pgfpathcurveto{\pgfqpoint{4.457935in}{2.446966in}}{\pgfqpoint{4.467611in}{2.442959in}}{\pgfqpoint{4.477698in}{2.442959in}}%
\pgfpathclose%
\pgfusepath{stroke,fill}%
\end{pgfscope}%
\begin{pgfscope}%
\pgfpathrectangle{\pgfqpoint{1.280114in}{0.528000in}}{\pgfqpoint{3.487886in}{3.696000in}} %
\pgfusepath{clip}%
\pgfsetbuttcap%
\pgfsetroundjoin%
\definecolor{currentfill}{rgb}{0.121569,0.466667,0.705882}%
\pgfsetfillcolor{currentfill}%
\pgfsetlinewidth{1.003750pt}%
\definecolor{currentstroke}{rgb}{0.121569,0.466667,0.705882}%
\pgfsetstrokecolor{currentstroke}%
\pgfsetdash{}{0pt}%
\pgfpathmoveto{\pgfqpoint{3.045309in}{1.986899in}}%
\pgfpathcurveto{\pgfqpoint{3.055396in}{1.986899in}}{\pgfqpoint{3.065072in}{1.990906in}}{\pgfqpoint{3.072205in}{1.998039in}}%
\pgfpathcurveto{\pgfqpoint{3.079338in}{2.005172in}}{\pgfqpoint{3.083345in}{2.014848in}}{\pgfqpoint{3.083345in}{2.024935in}}%
\pgfpathcurveto{\pgfqpoint{3.083345in}{2.035022in}}{\pgfqpoint{3.079338in}{2.044698in}}{\pgfqpoint{3.072205in}{2.051831in}}%
\pgfpathcurveto{\pgfqpoint{3.065072in}{2.058963in}}{\pgfqpoint{3.055396in}{2.062971in}}{\pgfqpoint{3.045309in}{2.062971in}}%
\pgfpathcurveto{\pgfqpoint{3.035222in}{2.062971in}}{\pgfqpoint{3.025546in}{2.058963in}}{\pgfqpoint{3.018413in}{2.051831in}}%
\pgfpathcurveto{\pgfqpoint{3.011281in}{2.044698in}}{\pgfqpoint{3.007273in}{2.035022in}}{\pgfqpoint{3.007273in}{2.024935in}}%
\pgfpathcurveto{\pgfqpoint{3.007273in}{2.014848in}}{\pgfqpoint{3.011281in}{2.005172in}}{\pgfqpoint{3.018413in}{1.998039in}}%
\pgfpathcurveto{\pgfqpoint{3.025546in}{1.990906in}}{\pgfqpoint{3.035222in}{1.986899in}}{\pgfqpoint{3.045309in}{1.986899in}}%
\pgfpathclose%
\pgfusepath{stroke,fill}%
\end{pgfscope}%
\begin{pgfscope}%
\pgfpathrectangle{\pgfqpoint{1.280114in}{0.528000in}}{\pgfqpoint{3.487886in}{3.696000in}} %
\pgfusepath{clip}%
\pgfsetbuttcap%
\pgfsetroundjoin%
\definecolor{currentfill}{rgb}{0.121569,0.466667,0.705882}%
\pgfsetfillcolor{currentfill}%
\pgfsetlinewidth{1.003750pt}%
\definecolor{currentstroke}{rgb}{0.121569,0.466667,0.705882}%
\pgfsetstrokecolor{currentstroke}%
\pgfsetdash{}{0pt}%
\pgfpathmoveto{\pgfqpoint{4.454871in}{2.357378in}}%
\pgfpathcurveto{\pgfqpoint{4.464959in}{2.357378in}}{\pgfqpoint{4.474634in}{2.361386in}}{\pgfqpoint{4.481767in}{2.368519in}}%
\pgfpathcurveto{\pgfqpoint{4.488900in}{2.375652in}}{\pgfqpoint{4.492908in}{2.385327in}}{\pgfqpoint{4.492908in}{2.395414in}}%
\pgfpathcurveto{\pgfqpoint{4.492908in}{2.405502in}}{\pgfqpoint{4.488900in}{2.415177in}}{\pgfqpoint{4.481767in}{2.422310in}}%
\pgfpathcurveto{\pgfqpoint{4.474634in}{2.429443in}}{\pgfqpoint{4.464959in}{2.433451in}}{\pgfqpoint{4.454871in}{2.433451in}}%
\pgfpathcurveto{\pgfqpoint{4.444784in}{2.433451in}}{\pgfqpoint{4.435109in}{2.429443in}}{\pgfqpoint{4.427976in}{2.422310in}}%
\pgfpathcurveto{\pgfqpoint{4.420843in}{2.415177in}}{\pgfqpoint{4.416835in}{2.405502in}}{\pgfqpoint{4.416835in}{2.395414in}}%
\pgfpathcurveto{\pgfqpoint{4.416835in}{2.385327in}}{\pgfqpoint{4.420843in}{2.375652in}}{\pgfqpoint{4.427976in}{2.368519in}}%
\pgfpathcurveto{\pgfqpoint{4.435109in}{2.361386in}}{\pgfqpoint{4.444784in}{2.357378in}}{\pgfqpoint{4.454871in}{2.357378in}}%
\pgfpathclose%
\pgfusepath{stroke,fill}%
\end{pgfscope}%
\begin{pgfscope}%
\pgfpathrectangle{\pgfqpoint{1.280114in}{0.528000in}}{\pgfqpoint{3.487886in}{3.696000in}} %
\pgfusepath{clip}%
\pgfsetbuttcap%
\pgfsetroundjoin%
\definecolor{currentfill}{rgb}{0.121569,0.466667,0.705882}%
\pgfsetfillcolor{currentfill}%
\pgfsetlinewidth{1.003750pt}%
\definecolor{currentstroke}{rgb}{0.121569,0.466667,0.705882}%
\pgfsetstrokecolor{currentstroke}%
\pgfsetdash{}{0pt}%
\pgfpathmoveto{\pgfqpoint{3.095539in}{1.986071in}}%
\pgfpathcurveto{\pgfqpoint{3.105627in}{1.986071in}}{\pgfqpoint{3.115302in}{1.990078in}}{\pgfqpoint{3.122435in}{1.997211in}}%
\pgfpathcurveto{\pgfqpoint{3.129568in}{2.004344in}}{\pgfqpoint{3.133575in}{2.014020in}}{\pgfqpoint{3.133575in}{2.024107in}}%
\pgfpathcurveto{\pgfqpoint{3.133575in}{2.034194in}}{\pgfqpoint{3.129568in}{2.043870in}}{\pgfqpoint{3.122435in}{2.051003in}}%
\pgfpathcurveto{\pgfqpoint{3.115302in}{2.058135in}}{\pgfqpoint{3.105627in}{2.062143in}}{\pgfqpoint{3.095539in}{2.062143in}}%
\pgfpathcurveto{\pgfqpoint{3.085452in}{2.062143in}}{\pgfqpoint{3.075776in}{2.058135in}}{\pgfqpoint{3.068643in}{2.051003in}}%
\pgfpathcurveto{\pgfqpoint{3.061511in}{2.043870in}}{\pgfqpoint{3.057503in}{2.034194in}}{\pgfqpoint{3.057503in}{2.024107in}}%
\pgfpathcurveto{\pgfqpoint{3.057503in}{2.014020in}}{\pgfqpoint{3.061511in}{2.004344in}}{\pgfqpoint{3.068643in}{1.997211in}}%
\pgfpathcurveto{\pgfqpoint{3.075776in}{1.990078in}}{\pgfqpoint{3.085452in}{1.986071in}}{\pgfqpoint{3.095539in}{1.986071in}}%
\pgfpathclose%
\pgfusepath{stroke,fill}%
\end{pgfscope}%
\begin{pgfscope}%
\pgfpathrectangle{\pgfqpoint{1.280114in}{0.528000in}}{\pgfqpoint{3.487886in}{3.696000in}} %
\pgfusepath{clip}%
\pgfsetbuttcap%
\pgfsetroundjoin%
\definecolor{currentfill}{rgb}{0.121569,0.466667,0.705882}%
\pgfsetfillcolor{currentfill}%
\pgfsetlinewidth{1.003750pt}%
\definecolor{currentstroke}{rgb}{0.121569,0.466667,0.705882}%
\pgfsetstrokecolor{currentstroke}%
\pgfsetdash{}{0pt}%
\pgfpathmoveto{\pgfqpoint{2.710288in}{2.696373in}}%
\pgfpathcurveto{\pgfqpoint{2.720376in}{2.696373in}}{\pgfqpoint{2.730051in}{2.700381in}}{\pgfqpoint{2.737184in}{2.707514in}}%
\pgfpathcurveto{\pgfqpoint{2.744317in}{2.714647in}}{\pgfqpoint{2.748325in}{2.724322in}}{\pgfqpoint{2.748325in}{2.734410in}}%
\pgfpathcurveto{\pgfqpoint{2.748325in}{2.744497in}}{\pgfqpoint{2.744317in}{2.754172in}}{\pgfqpoint{2.737184in}{2.761305in}}%
\pgfpathcurveto{\pgfqpoint{2.730051in}{2.768438in}}{\pgfqpoint{2.720376in}{2.772446in}}{\pgfqpoint{2.710288in}{2.772446in}}%
\pgfpathcurveto{\pgfqpoint{2.700201in}{2.772446in}}{\pgfqpoint{2.690526in}{2.768438in}}{\pgfqpoint{2.683393in}{2.761305in}}%
\pgfpathcurveto{\pgfqpoint{2.676260in}{2.754172in}}{\pgfqpoint{2.672252in}{2.744497in}}{\pgfqpoint{2.672252in}{2.734410in}}%
\pgfpathcurveto{\pgfqpoint{2.672252in}{2.724322in}}{\pgfqpoint{2.676260in}{2.714647in}}{\pgfqpoint{2.683393in}{2.707514in}}%
\pgfpathcurveto{\pgfqpoint{2.690526in}{2.700381in}}{\pgfqpoint{2.700201in}{2.696373in}}{\pgfqpoint{2.710288in}{2.696373in}}%
\pgfpathclose%
\pgfusepath{stroke,fill}%
\end{pgfscope}%
\begin{pgfscope}%
\pgfpathrectangle{\pgfqpoint{1.280114in}{0.528000in}}{\pgfqpoint{3.487886in}{3.696000in}} %
\pgfusepath{clip}%
\pgfsetbuttcap%
\pgfsetroundjoin%
\definecolor{currentfill}{rgb}{0.121569,0.466667,0.705882}%
\pgfsetfillcolor{currentfill}%
\pgfsetlinewidth{1.003750pt}%
\definecolor{currentstroke}{rgb}{0.121569,0.466667,0.705882}%
\pgfsetstrokecolor{currentstroke}%
\pgfsetdash{}{0pt}%
\pgfpathmoveto{\pgfqpoint{3.530486in}{2.013906in}}%
\pgfpathcurveto{\pgfqpoint{3.540573in}{2.013906in}}{\pgfqpoint{3.550249in}{2.017914in}}{\pgfqpoint{3.557381in}{2.025047in}}%
\pgfpathcurveto{\pgfqpoint{3.564514in}{2.032179in}}{\pgfqpoint{3.568522in}{2.041855in}}{\pgfqpoint{3.568522in}{2.051942in}}%
\pgfpathcurveto{\pgfqpoint{3.568522in}{2.062030in}}{\pgfqpoint{3.564514in}{2.071705in}}{\pgfqpoint{3.557381in}{2.078838in}}%
\pgfpathcurveto{\pgfqpoint{3.550249in}{2.085971in}}{\pgfqpoint{3.540573in}{2.089979in}}{\pgfqpoint{3.530486in}{2.089979in}}%
\pgfpathcurveto{\pgfqpoint{3.520398in}{2.089979in}}{\pgfqpoint{3.510723in}{2.085971in}}{\pgfqpoint{3.503590in}{2.078838in}}%
\pgfpathcurveto{\pgfqpoint{3.496457in}{2.071705in}}{\pgfqpoint{3.492449in}{2.062030in}}{\pgfqpoint{3.492449in}{2.051942in}}%
\pgfpathcurveto{\pgfqpoint{3.492449in}{2.041855in}}{\pgfqpoint{3.496457in}{2.032179in}}{\pgfqpoint{3.503590in}{2.025047in}}%
\pgfpathcurveto{\pgfqpoint{3.510723in}{2.017914in}}{\pgfqpoint{3.520398in}{2.013906in}}{\pgfqpoint{3.530486in}{2.013906in}}%
\pgfpathclose%
\pgfusepath{stroke,fill}%
\end{pgfscope}%
\begin{pgfscope}%
\pgfpathrectangle{\pgfqpoint{1.280114in}{0.528000in}}{\pgfqpoint{3.487886in}{3.696000in}} %
\pgfusepath{clip}%
\pgfsetbuttcap%
\pgfsetroundjoin%
\definecolor{currentfill}{rgb}{0.121569,0.466667,0.705882}%
\pgfsetfillcolor{currentfill}%
\pgfsetlinewidth{1.003750pt}%
\definecolor{currentstroke}{rgb}{0.121569,0.466667,0.705882}%
\pgfsetstrokecolor{currentstroke}%
\pgfsetdash{}{0pt}%
\pgfpathmoveto{\pgfqpoint{4.264536in}{2.602820in}}%
\pgfpathcurveto{\pgfqpoint{4.274623in}{2.602820in}}{\pgfqpoint{4.284299in}{2.606827in}}{\pgfqpoint{4.291432in}{2.613960in}}%
\pgfpathcurveto{\pgfqpoint{4.298564in}{2.621093in}}{\pgfqpoint{4.302572in}{2.630769in}}{\pgfqpoint{4.302572in}{2.640856in}}%
\pgfpathcurveto{\pgfqpoint{4.302572in}{2.650943in}}{\pgfqpoint{4.298564in}{2.660619in}}{\pgfqpoint{4.291432in}{2.667752in}}%
\pgfpathcurveto{\pgfqpoint{4.284299in}{2.674884in}}{\pgfqpoint{4.274623in}{2.678892in}}{\pgfqpoint{4.264536in}{2.678892in}}%
\pgfpathcurveto{\pgfqpoint{4.254448in}{2.678892in}}{\pgfqpoint{4.244773in}{2.674884in}}{\pgfqpoint{4.237640in}{2.667752in}}%
\pgfpathcurveto{\pgfqpoint{4.230507in}{2.660619in}}{\pgfqpoint{4.226500in}{2.650943in}}{\pgfqpoint{4.226500in}{2.640856in}}%
\pgfpathcurveto{\pgfqpoint{4.226500in}{2.630769in}}{\pgfqpoint{4.230507in}{2.621093in}}{\pgfqpoint{4.237640in}{2.613960in}}%
\pgfpathcurveto{\pgfqpoint{4.244773in}{2.606827in}}{\pgfqpoint{4.254448in}{2.602820in}}{\pgfqpoint{4.264536in}{2.602820in}}%
\pgfpathclose%
\pgfusepath{stroke,fill}%
\end{pgfscope}%
\begin{pgfscope}%
\pgfpathrectangle{\pgfqpoint{1.280114in}{0.528000in}}{\pgfqpoint{3.487886in}{3.696000in}} %
\pgfusepath{clip}%
\pgfsetbuttcap%
\pgfsetroundjoin%
\definecolor{currentfill}{rgb}{0.121569,0.466667,0.705882}%
\pgfsetfillcolor{currentfill}%
\pgfsetlinewidth{1.003750pt}%
\definecolor{currentstroke}{rgb}{0.121569,0.466667,0.705882}%
\pgfsetstrokecolor{currentstroke}%
\pgfsetdash{}{0pt}%
\pgfpathmoveto{\pgfqpoint{1.696333in}{2.518733in}}%
\pgfpathcurveto{\pgfqpoint{1.706420in}{2.518733in}}{\pgfqpoint{1.716096in}{2.522741in}}{\pgfqpoint{1.723229in}{2.529874in}}%
\pgfpathcurveto{\pgfqpoint{1.730361in}{2.537006in}}{\pgfqpoint{1.734369in}{2.546682in}}{\pgfqpoint{1.734369in}{2.556769in}}%
\pgfpathcurveto{\pgfqpoint{1.734369in}{2.566857in}}{\pgfqpoint{1.730361in}{2.576532in}}{\pgfqpoint{1.723229in}{2.583665in}}%
\pgfpathcurveto{\pgfqpoint{1.716096in}{2.590798in}}{\pgfqpoint{1.706420in}{2.594806in}}{\pgfqpoint{1.696333in}{2.594806in}}%
\pgfpathcurveto{\pgfqpoint{1.686246in}{2.594806in}}{\pgfqpoint{1.676570in}{2.590798in}}{\pgfqpoint{1.669437in}{2.583665in}}%
\pgfpathcurveto{\pgfqpoint{1.662304in}{2.576532in}}{\pgfqpoint{1.658297in}{2.566857in}}{\pgfqpoint{1.658297in}{2.556769in}}%
\pgfpathcurveto{\pgfqpoint{1.658297in}{2.546682in}}{\pgfqpoint{1.662304in}{2.537006in}}{\pgfqpoint{1.669437in}{2.529874in}}%
\pgfpathcurveto{\pgfqpoint{1.676570in}{2.522741in}}{\pgfqpoint{1.686246in}{2.518733in}}{\pgfqpoint{1.696333in}{2.518733in}}%
\pgfpathclose%
\pgfusepath{stroke,fill}%
\end{pgfscope}%
\begin{pgfscope}%
\pgfpathrectangle{\pgfqpoint{1.280114in}{0.528000in}}{\pgfqpoint{3.487886in}{3.696000in}} %
\pgfusepath{clip}%
\pgfsetbuttcap%
\pgfsetroundjoin%
\definecolor{currentfill}{rgb}{0.121569,0.466667,0.705882}%
\pgfsetfillcolor{currentfill}%
\pgfsetlinewidth{1.003750pt}%
\definecolor{currentstroke}{rgb}{0.121569,0.466667,0.705882}%
\pgfsetstrokecolor{currentstroke}%
\pgfsetdash{}{0pt}%
\pgfpathmoveto{\pgfqpoint{4.170551in}{2.161012in}}%
\pgfpathcurveto{\pgfqpoint{4.180639in}{2.161012in}}{\pgfqpoint{4.190314in}{2.165019in}}{\pgfqpoint{4.197447in}{2.172152in}}%
\pgfpathcurveto{\pgfqpoint{4.204580in}{2.179285in}}{\pgfqpoint{4.208588in}{2.188961in}}{\pgfqpoint{4.208588in}{2.199048in}}%
\pgfpathcurveto{\pgfqpoint{4.208588in}{2.209135in}}{\pgfqpoint{4.204580in}{2.218811in}}{\pgfqpoint{4.197447in}{2.225944in}}%
\pgfpathcurveto{\pgfqpoint{4.190314in}{2.233076in}}{\pgfqpoint{4.180639in}{2.237084in}}{\pgfqpoint{4.170551in}{2.237084in}}%
\pgfpathcurveto{\pgfqpoint{4.160464in}{2.237084in}}{\pgfqpoint{4.150788in}{2.233076in}}{\pgfqpoint{4.143656in}{2.225944in}}%
\pgfpathcurveto{\pgfqpoint{4.136523in}{2.218811in}}{\pgfqpoint{4.132515in}{2.209135in}}{\pgfqpoint{4.132515in}{2.199048in}}%
\pgfpathcurveto{\pgfqpoint{4.132515in}{2.188961in}}{\pgfqpoint{4.136523in}{2.179285in}}{\pgfqpoint{4.143656in}{2.172152in}}%
\pgfpathcurveto{\pgfqpoint{4.150788in}{2.165019in}}{\pgfqpoint{4.160464in}{2.161012in}}{\pgfqpoint{4.170551in}{2.161012in}}%
\pgfpathclose%
\pgfusepath{stroke,fill}%
\end{pgfscope}%
\begin{pgfscope}%
\pgfpathrectangle{\pgfqpoint{1.280114in}{0.528000in}}{\pgfqpoint{3.487886in}{3.696000in}} %
\pgfusepath{clip}%
\pgfsetbuttcap%
\pgfsetroundjoin%
\definecolor{currentfill}{rgb}{0.121569,0.466667,0.705882}%
\pgfsetfillcolor{currentfill}%
\pgfsetlinewidth{1.003750pt}%
\definecolor{currentstroke}{rgb}{0.121569,0.466667,0.705882}%
\pgfsetstrokecolor{currentstroke}%
\pgfsetdash{}{0pt}%
\pgfpathmoveto{\pgfqpoint{2.655516in}{2.706486in}}%
\pgfpathcurveto{\pgfqpoint{2.665603in}{2.706486in}}{\pgfqpoint{2.675279in}{2.710494in}}{\pgfqpoint{2.682412in}{2.717627in}}%
\pgfpathcurveto{\pgfqpoint{2.689545in}{2.724760in}}{\pgfqpoint{2.693552in}{2.734435in}}{\pgfqpoint{2.693552in}{2.744523in}}%
\pgfpathcurveto{\pgfqpoint{2.693552in}{2.754610in}}{\pgfqpoint{2.689545in}{2.764285in}}{\pgfqpoint{2.682412in}{2.771418in}}%
\pgfpathcurveto{\pgfqpoint{2.675279in}{2.778551in}}{\pgfqpoint{2.665603in}{2.782559in}}{\pgfqpoint{2.655516in}{2.782559in}}%
\pgfpathcurveto{\pgfqpoint{2.645429in}{2.782559in}}{\pgfqpoint{2.635753in}{2.778551in}}{\pgfqpoint{2.628620in}{2.771418in}}%
\pgfpathcurveto{\pgfqpoint{2.621487in}{2.764285in}}{\pgfqpoint{2.617480in}{2.754610in}}{\pgfqpoint{2.617480in}{2.744523in}}%
\pgfpathcurveto{\pgfqpoint{2.617480in}{2.734435in}}{\pgfqpoint{2.621487in}{2.724760in}}{\pgfqpoint{2.628620in}{2.717627in}}%
\pgfpathcurveto{\pgfqpoint{2.635753in}{2.710494in}}{\pgfqpoint{2.645429in}{2.706486in}}{\pgfqpoint{2.655516in}{2.706486in}}%
\pgfpathclose%
\pgfusepath{stroke,fill}%
\end{pgfscope}%
\begin{pgfscope}%
\pgfpathrectangle{\pgfqpoint{1.280114in}{0.528000in}}{\pgfqpoint{3.487886in}{3.696000in}} %
\pgfusepath{clip}%
\pgfsetbuttcap%
\pgfsetroundjoin%
\definecolor{currentfill}{rgb}{0.121569,0.466667,0.705882}%
\pgfsetfillcolor{currentfill}%
\pgfsetlinewidth{1.003750pt}%
\definecolor{currentstroke}{rgb}{0.121569,0.466667,0.705882}%
\pgfsetstrokecolor{currentstroke}%
\pgfsetdash{}{0pt}%
\pgfpathmoveto{\pgfqpoint{2.768135in}{2.740892in}}%
\pgfpathcurveto{\pgfqpoint{2.778223in}{2.740892in}}{\pgfqpoint{2.787898in}{2.744899in}}{\pgfqpoint{2.795031in}{2.752032in}}%
\pgfpathcurveto{\pgfqpoint{2.802164in}{2.759165in}}{\pgfqpoint{2.806172in}{2.768841in}}{\pgfqpoint{2.806172in}{2.778928in}}%
\pgfpathcurveto{\pgfqpoint{2.806172in}{2.789015in}}{\pgfqpoint{2.802164in}{2.798691in}}{\pgfqpoint{2.795031in}{2.805824in}}%
\pgfpathcurveto{\pgfqpoint{2.787898in}{2.812956in}}{\pgfqpoint{2.778223in}{2.816964in}}{\pgfqpoint{2.768135in}{2.816964in}}%
\pgfpathcurveto{\pgfqpoint{2.758048in}{2.816964in}}{\pgfqpoint{2.748373in}{2.812956in}}{\pgfqpoint{2.741240in}{2.805824in}}%
\pgfpathcurveto{\pgfqpoint{2.734107in}{2.798691in}}{\pgfqpoint{2.730099in}{2.789015in}}{\pgfqpoint{2.730099in}{2.778928in}}%
\pgfpathcurveto{\pgfqpoint{2.730099in}{2.768841in}}{\pgfqpoint{2.734107in}{2.759165in}}{\pgfqpoint{2.741240in}{2.752032in}}%
\pgfpathcurveto{\pgfqpoint{2.748373in}{2.744899in}}{\pgfqpoint{2.758048in}{2.740892in}}{\pgfqpoint{2.768135in}{2.740892in}}%
\pgfpathclose%
\pgfusepath{stroke,fill}%
\end{pgfscope}%
\begin{pgfscope}%
\pgfpathrectangle{\pgfqpoint{1.280114in}{0.528000in}}{\pgfqpoint{3.487886in}{3.696000in}} %
\pgfusepath{clip}%
\pgfsetbuttcap%
\pgfsetroundjoin%
\definecolor{currentfill}{rgb}{0.121569,0.466667,0.705882}%
\pgfsetfillcolor{currentfill}%
\pgfsetlinewidth{1.003750pt}%
\definecolor{currentstroke}{rgb}{0.121569,0.466667,0.705882}%
\pgfsetstrokecolor{currentstroke}%
\pgfsetdash{}{0pt}%
\pgfpathmoveto{\pgfqpoint{3.955099in}{2.642326in}}%
\pgfpathcurveto{\pgfqpoint{3.965187in}{2.642326in}}{\pgfqpoint{3.974862in}{2.646333in}}{\pgfqpoint{3.981995in}{2.653466in}}%
\pgfpathcurveto{\pgfqpoint{3.989128in}{2.660599in}}{\pgfqpoint{3.993136in}{2.670274in}}{\pgfqpoint{3.993136in}{2.680362in}}%
\pgfpathcurveto{\pgfqpoint{3.993136in}{2.690449in}}{\pgfqpoint{3.989128in}{2.700125in}}{\pgfqpoint{3.981995in}{2.707258in}}%
\pgfpathcurveto{\pgfqpoint{3.974862in}{2.714390in}}{\pgfqpoint{3.965187in}{2.718398in}}{\pgfqpoint{3.955099in}{2.718398in}}%
\pgfpathcurveto{\pgfqpoint{3.945012in}{2.718398in}}{\pgfqpoint{3.935336in}{2.714390in}}{\pgfqpoint{3.928204in}{2.707258in}}%
\pgfpathcurveto{\pgfqpoint{3.921071in}{2.700125in}}{\pgfqpoint{3.917063in}{2.690449in}}{\pgfqpoint{3.917063in}{2.680362in}}%
\pgfpathcurveto{\pgfqpoint{3.917063in}{2.670274in}}{\pgfqpoint{3.921071in}{2.660599in}}{\pgfqpoint{3.928204in}{2.653466in}}%
\pgfpathcurveto{\pgfqpoint{3.935336in}{2.646333in}}{\pgfqpoint{3.945012in}{2.642326in}}{\pgfqpoint{3.955099in}{2.642326in}}%
\pgfpathclose%
\pgfusepath{stroke,fill}%
\end{pgfscope}%
\begin{pgfscope}%
\pgfpathrectangle{\pgfqpoint{1.280114in}{0.528000in}}{\pgfqpoint{3.487886in}{3.696000in}} %
\pgfusepath{clip}%
\pgfsetbuttcap%
\pgfsetroundjoin%
\definecolor{currentfill}{rgb}{0.121569,0.466667,0.705882}%
\pgfsetfillcolor{currentfill}%
\pgfsetlinewidth{1.003750pt}%
\definecolor{currentstroke}{rgb}{0.121569,0.466667,0.705882}%
\pgfsetstrokecolor{currentstroke}%
\pgfsetdash{}{0pt}%
\pgfpathmoveto{\pgfqpoint{4.484659in}{2.417886in}}%
\pgfpathcurveto{\pgfqpoint{4.494746in}{2.417886in}}{\pgfqpoint{4.504421in}{2.421894in}}{\pgfqpoint{4.511554in}{2.429027in}}%
\pgfpathcurveto{\pgfqpoint{4.518687in}{2.436160in}}{\pgfqpoint{4.522695in}{2.445835in}}{\pgfqpoint{4.522695in}{2.455923in}}%
\pgfpathcurveto{\pgfqpoint{4.522695in}{2.466010in}}{\pgfqpoint{4.518687in}{2.475686in}}{\pgfqpoint{4.511554in}{2.482818in}}%
\pgfpathcurveto{\pgfqpoint{4.504421in}{2.489951in}}{\pgfqpoint{4.494746in}{2.493959in}}{\pgfqpoint{4.484659in}{2.493959in}}%
\pgfpathcurveto{\pgfqpoint{4.474571in}{2.493959in}}{\pgfqpoint{4.464896in}{2.489951in}}{\pgfqpoint{4.457763in}{2.482818in}}%
\pgfpathcurveto{\pgfqpoint{4.450630in}{2.475686in}}{\pgfqpoint{4.446622in}{2.466010in}}{\pgfqpoint{4.446622in}{2.455923in}}%
\pgfpathcurveto{\pgfqpoint{4.446622in}{2.445835in}}{\pgfqpoint{4.450630in}{2.436160in}}{\pgfqpoint{4.457763in}{2.429027in}}%
\pgfpathcurveto{\pgfqpoint{4.464896in}{2.421894in}}{\pgfqpoint{4.474571in}{2.417886in}}{\pgfqpoint{4.484659in}{2.417886in}}%
\pgfpathclose%
\pgfusepath{stroke,fill}%
\end{pgfscope}%
\begin{pgfscope}%
\pgfpathrectangle{\pgfqpoint{1.280114in}{0.528000in}}{\pgfqpoint{3.487886in}{3.696000in}} %
\pgfusepath{clip}%
\pgfsetbuttcap%
\pgfsetroundjoin%
\definecolor{currentfill}{rgb}{0.121569,0.466667,0.705882}%
\pgfsetfillcolor{currentfill}%
\pgfsetlinewidth{1.003750pt}%
\definecolor{currentstroke}{rgb}{0.121569,0.466667,0.705882}%
\pgfsetstrokecolor{currentstroke}%
\pgfsetdash{}{0pt}%
\pgfpathmoveto{\pgfqpoint{2.401067in}{2.028997in}}%
\pgfpathcurveto{\pgfqpoint{2.411154in}{2.028997in}}{\pgfqpoint{2.420830in}{2.033004in}}{\pgfqpoint{2.427963in}{2.040137in}}%
\pgfpathcurveto{\pgfqpoint{2.435096in}{2.047270in}}{\pgfqpoint{2.439103in}{2.056946in}}{\pgfqpoint{2.439103in}{2.067033in}}%
\pgfpathcurveto{\pgfqpoint{2.439103in}{2.077120in}}{\pgfqpoint{2.435096in}{2.086796in}}{\pgfqpoint{2.427963in}{2.093929in}}%
\pgfpathcurveto{\pgfqpoint{2.420830in}{2.101062in}}{\pgfqpoint{2.411154in}{2.105069in}}{\pgfqpoint{2.401067in}{2.105069in}}%
\pgfpathcurveto{\pgfqpoint{2.390980in}{2.105069in}}{\pgfqpoint{2.381304in}{2.101062in}}{\pgfqpoint{2.374171in}{2.093929in}}%
\pgfpathcurveto{\pgfqpoint{2.367039in}{2.086796in}}{\pgfqpoint{2.363031in}{2.077120in}}{\pgfqpoint{2.363031in}{2.067033in}}%
\pgfpathcurveto{\pgfqpoint{2.363031in}{2.056946in}}{\pgfqpoint{2.367039in}{2.047270in}}{\pgfqpoint{2.374171in}{2.040137in}}%
\pgfpathcurveto{\pgfqpoint{2.381304in}{2.033004in}}{\pgfqpoint{2.390980in}{2.028997in}}{\pgfqpoint{2.401067in}{2.028997in}}%
\pgfpathclose%
\pgfusepath{stroke,fill}%
\end{pgfscope}%
\begin{pgfscope}%
\pgfpathrectangle{\pgfqpoint{1.280114in}{0.528000in}}{\pgfqpoint{3.487886in}{3.696000in}} %
\pgfusepath{clip}%
\pgfsetbuttcap%
\pgfsetroundjoin%
\definecolor{currentfill}{rgb}{0.121569,0.466667,0.705882}%
\pgfsetfillcolor{currentfill}%
\pgfsetlinewidth{1.003750pt}%
\definecolor{currentstroke}{rgb}{0.121569,0.466667,0.705882}%
\pgfsetstrokecolor{currentstroke}%
\pgfsetdash{}{0pt}%
\pgfpathmoveto{\pgfqpoint{3.249659in}{2.756241in}}%
\pgfpathcurveto{\pgfqpoint{3.259747in}{2.756241in}}{\pgfqpoint{3.269422in}{2.760248in}}{\pgfqpoint{3.276555in}{2.767381in}}%
\pgfpathcurveto{\pgfqpoint{3.283688in}{2.774514in}}{\pgfqpoint{3.287696in}{2.784190in}}{\pgfqpoint{3.287696in}{2.794277in}}%
\pgfpathcurveto{\pgfqpoint{3.287696in}{2.804364in}}{\pgfqpoint{3.283688in}{2.814040in}}{\pgfqpoint{3.276555in}{2.821173in}}%
\pgfpathcurveto{\pgfqpoint{3.269422in}{2.828305in}}{\pgfqpoint{3.259747in}{2.832313in}}{\pgfqpoint{3.249659in}{2.832313in}}%
\pgfpathcurveto{\pgfqpoint{3.239572in}{2.832313in}}{\pgfqpoint{3.229896in}{2.828305in}}{\pgfqpoint{3.222764in}{2.821173in}}%
\pgfpathcurveto{\pgfqpoint{3.215631in}{2.814040in}}{\pgfqpoint{3.211623in}{2.804364in}}{\pgfqpoint{3.211623in}{2.794277in}}%
\pgfpathcurveto{\pgfqpoint{3.211623in}{2.784190in}}{\pgfqpoint{3.215631in}{2.774514in}}{\pgfqpoint{3.222764in}{2.767381in}}%
\pgfpathcurveto{\pgfqpoint{3.229896in}{2.760248in}}{\pgfqpoint{3.239572in}{2.756241in}}{\pgfqpoint{3.249659in}{2.756241in}}%
\pgfpathclose%
\pgfusepath{stroke,fill}%
\end{pgfscope}%
\begin{pgfscope}%
\pgfpathrectangle{\pgfqpoint{1.280114in}{0.528000in}}{\pgfqpoint{3.487886in}{3.696000in}} %
\pgfusepath{clip}%
\pgfsetbuttcap%
\pgfsetroundjoin%
\definecolor{currentfill}{rgb}{0.121569,0.466667,0.705882}%
\pgfsetfillcolor{currentfill}%
\pgfsetlinewidth{1.003750pt}%
\definecolor{currentstroke}{rgb}{0.121569,0.466667,0.705882}%
\pgfsetstrokecolor{currentstroke}%
\pgfsetdash{}{0pt}%
\pgfpathmoveto{\pgfqpoint{2.927715in}{2.744454in}}%
\pgfpathcurveto{\pgfqpoint{2.937802in}{2.744454in}}{\pgfqpoint{2.947478in}{2.748462in}}{\pgfqpoint{2.954610in}{2.755595in}}%
\pgfpathcurveto{\pgfqpoint{2.961743in}{2.762728in}}{\pgfqpoint{2.965751in}{2.772403in}}{\pgfqpoint{2.965751in}{2.782491in}}%
\pgfpathcurveto{\pgfqpoint{2.965751in}{2.792578in}}{\pgfqpoint{2.961743in}{2.802254in}}{\pgfqpoint{2.954610in}{2.809386in}}%
\pgfpathcurveto{\pgfqpoint{2.947478in}{2.816519in}}{\pgfqpoint{2.937802in}{2.820527in}}{\pgfqpoint{2.927715in}{2.820527in}}%
\pgfpathcurveto{\pgfqpoint{2.917627in}{2.820527in}}{\pgfqpoint{2.907952in}{2.816519in}}{\pgfqpoint{2.900819in}{2.809386in}}%
\pgfpathcurveto{\pgfqpoint{2.893686in}{2.802254in}}{\pgfqpoint{2.889678in}{2.792578in}}{\pgfqpoint{2.889678in}{2.782491in}}%
\pgfpathcurveto{\pgfqpoint{2.889678in}{2.772403in}}{\pgfqpoint{2.893686in}{2.762728in}}{\pgfqpoint{2.900819in}{2.755595in}}%
\pgfpathcurveto{\pgfqpoint{2.907952in}{2.748462in}}{\pgfqpoint{2.917627in}{2.744454in}}{\pgfqpoint{2.927715in}{2.744454in}}%
\pgfpathclose%
\pgfusepath{stroke,fill}%
\end{pgfscope}%
\begin{pgfscope}%
\pgfpathrectangle{\pgfqpoint{1.280114in}{0.528000in}}{\pgfqpoint{3.487886in}{3.696000in}} %
\pgfusepath{clip}%
\pgfsetbuttcap%
\pgfsetroundjoin%
\definecolor{currentfill}{rgb}{0.121569,0.466667,0.705882}%
\pgfsetfillcolor{currentfill}%
\pgfsetlinewidth{1.003750pt}%
\definecolor{currentstroke}{rgb}{0.121569,0.466667,0.705882}%
\pgfsetstrokecolor{currentstroke}%
\pgfsetdash{}{0pt}%
\pgfpathmoveto{\pgfqpoint{1.646086in}{2.435092in}}%
\pgfpathcurveto{\pgfqpoint{1.656173in}{2.435092in}}{\pgfqpoint{1.665849in}{2.439100in}}{\pgfqpoint{1.672982in}{2.446232in}}%
\pgfpathcurveto{\pgfqpoint{1.680114in}{2.453365in}}{\pgfqpoint{1.684122in}{2.463041in}}{\pgfqpoint{1.684122in}{2.473128in}}%
\pgfpathcurveto{\pgfqpoint{1.684122in}{2.483216in}}{\pgfqpoint{1.680114in}{2.492891in}}{\pgfqpoint{1.672982in}{2.500024in}}%
\pgfpathcurveto{\pgfqpoint{1.665849in}{2.507157in}}{\pgfqpoint{1.656173in}{2.511164in}}{\pgfqpoint{1.646086in}{2.511164in}}%
\pgfpathcurveto{\pgfqpoint{1.635998in}{2.511164in}}{\pgfqpoint{1.626323in}{2.507157in}}{\pgfqpoint{1.619190in}{2.500024in}}%
\pgfpathcurveto{\pgfqpoint{1.612057in}{2.492891in}}{\pgfqpoint{1.608050in}{2.483216in}}{\pgfqpoint{1.608050in}{2.473128in}}%
\pgfpathcurveto{\pgfqpoint{1.608050in}{2.463041in}}{\pgfqpoint{1.612057in}{2.453365in}}{\pgfqpoint{1.619190in}{2.446232in}}%
\pgfpathcurveto{\pgfqpoint{1.626323in}{2.439100in}}{\pgfqpoint{1.635998in}{2.435092in}}{\pgfqpoint{1.646086in}{2.435092in}}%
\pgfpathclose%
\pgfusepath{stroke,fill}%
\end{pgfscope}%
\begin{pgfscope}%
\pgfpathrectangle{\pgfqpoint{1.280114in}{0.528000in}}{\pgfqpoint{3.487886in}{3.696000in}} %
\pgfusepath{clip}%
\pgfsetbuttcap%
\pgfsetroundjoin%
\definecolor{currentfill}{rgb}{0.121569,0.466667,0.705882}%
\pgfsetfillcolor{currentfill}%
\pgfsetlinewidth{1.003750pt}%
\definecolor{currentstroke}{rgb}{0.121569,0.466667,0.705882}%
\pgfsetstrokecolor{currentstroke}%
\pgfsetdash{}{0pt}%
\pgfpathmoveto{\pgfqpoint{4.350723in}{2.516722in}}%
\pgfpathcurveto{\pgfqpoint{4.360810in}{2.516722in}}{\pgfqpoint{4.370486in}{2.520730in}}{\pgfqpoint{4.377619in}{2.527863in}}%
\pgfpathcurveto{\pgfqpoint{4.384752in}{2.534996in}}{\pgfqpoint{4.388759in}{2.544671in}}{\pgfqpoint{4.388759in}{2.554759in}}%
\pgfpathcurveto{\pgfqpoint{4.388759in}{2.564846in}}{\pgfqpoint{4.384752in}{2.574522in}}{\pgfqpoint{4.377619in}{2.581654in}}%
\pgfpathcurveto{\pgfqpoint{4.370486in}{2.588787in}}{\pgfqpoint{4.360810in}{2.592795in}}{\pgfqpoint{4.350723in}{2.592795in}}%
\pgfpathcurveto{\pgfqpoint{4.340636in}{2.592795in}}{\pgfqpoint{4.330960in}{2.588787in}}{\pgfqpoint{4.323827in}{2.581654in}}%
\pgfpathcurveto{\pgfqpoint{4.316695in}{2.574522in}}{\pgfqpoint{4.312687in}{2.564846in}}{\pgfqpoint{4.312687in}{2.554759in}}%
\pgfpathcurveto{\pgfqpoint{4.312687in}{2.544671in}}{\pgfqpoint{4.316695in}{2.534996in}}{\pgfqpoint{4.323827in}{2.527863in}}%
\pgfpathcurveto{\pgfqpoint{4.330960in}{2.520730in}}{\pgfqpoint{4.340636in}{2.516722in}}{\pgfqpoint{4.350723in}{2.516722in}}%
\pgfpathclose%
\pgfusepath{stroke,fill}%
\end{pgfscope}%
\begin{pgfscope}%
\pgfpathrectangle{\pgfqpoint{1.280114in}{0.528000in}}{\pgfqpoint{3.487886in}{3.696000in}} %
\pgfusepath{clip}%
\pgfsetbuttcap%
\pgfsetroundjoin%
\definecolor{currentfill}{rgb}{0.121569,0.466667,0.705882}%
\pgfsetfillcolor{currentfill}%
\pgfsetlinewidth{1.003750pt}%
\definecolor{currentstroke}{rgb}{0.121569,0.466667,0.705882}%
\pgfsetstrokecolor{currentstroke}%
\pgfsetdash{}{0pt}%
\pgfpathmoveto{\pgfqpoint{1.750043in}{2.216158in}}%
\pgfpathcurveto{\pgfqpoint{1.760131in}{2.216158in}}{\pgfqpoint{1.769806in}{2.220166in}}{\pgfqpoint{1.776939in}{2.227299in}}%
\pgfpathcurveto{\pgfqpoint{1.784072in}{2.234432in}}{\pgfqpoint{1.788080in}{2.244107in}}{\pgfqpoint{1.788080in}{2.254195in}}%
\pgfpathcurveto{\pgfqpoint{1.788080in}{2.264282in}}{\pgfqpoint{1.784072in}{2.273957in}}{\pgfqpoint{1.776939in}{2.281090in}}%
\pgfpathcurveto{\pgfqpoint{1.769806in}{2.288223in}}{\pgfqpoint{1.760131in}{2.292231in}}{\pgfqpoint{1.750043in}{2.292231in}}%
\pgfpathcurveto{\pgfqpoint{1.739956in}{2.292231in}}{\pgfqpoint{1.730280in}{2.288223in}}{\pgfqpoint{1.723148in}{2.281090in}}%
\pgfpathcurveto{\pgfqpoint{1.716015in}{2.273957in}}{\pgfqpoint{1.712007in}{2.264282in}}{\pgfqpoint{1.712007in}{2.254195in}}%
\pgfpathcurveto{\pgfqpoint{1.712007in}{2.244107in}}{\pgfqpoint{1.716015in}{2.234432in}}{\pgfqpoint{1.723148in}{2.227299in}}%
\pgfpathcurveto{\pgfqpoint{1.730280in}{2.220166in}}{\pgfqpoint{1.739956in}{2.216158in}}{\pgfqpoint{1.750043in}{2.216158in}}%
\pgfpathclose%
\pgfusepath{stroke,fill}%
\end{pgfscope}%
\begin{pgfscope}%
\pgfpathrectangle{\pgfqpoint{1.280114in}{0.528000in}}{\pgfqpoint{3.487886in}{3.696000in}} %
\pgfusepath{clip}%
\pgfsetbuttcap%
\pgfsetroundjoin%
\definecolor{currentfill}{rgb}{0.121569,0.466667,0.705882}%
\pgfsetfillcolor{currentfill}%
\pgfsetlinewidth{1.003750pt}%
\definecolor{currentstroke}{rgb}{0.121569,0.466667,0.705882}%
\pgfsetstrokecolor{currentstroke}%
\pgfsetdash{}{0pt}%
\pgfpathmoveto{\pgfqpoint{3.798544in}{2.618420in}}%
\pgfpathcurveto{\pgfqpoint{3.808631in}{2.618420in}}{\pgfqpoint{3.818307in}{2.622428in}}{\pgfqpoint{3.825439in}{2.629561in}}%
\pgfpathcurveto{\pgfqpoint{3.832572in}{2.636694in}}{\pgfqpoint{3.836580in}{2.646369in}}{\pgfqpoint{3.836580in}{2.656457in}}%
\pgfpathcurveto{\pgfqpoint{3.836580in}{2.666544in}}{\pgfqpoint{3.832572in}{2.676220in}}{\pgfqpoint{3.825439in}{2.683352in}}%
\pgfpathcurveto{\pgfqpoint{3.818307in}{2.690485in}}{\pgfqpoint{3.808631in}{2.694493in}}{\pgfqpoint{3.798544in}{2.694493in}}%
\pgfpathcurveto{\pgfqpoint{3.788456in}{2.694493in}}{\pgfqpoint{3.778781in}{2.690485in}}{\pgfqpoint{3.771648in}{2.683352in}}%
\pgfpathcurveto{\pgfqpoint{3.764515in}{2.676220in}}{\pgfqpoint{3.760507in}{2.666544in}}{\pgfqpoint{3.760507in}{2.656457in}}%
\pgfpathcurveto{\pgfqpoint{3.760507in}{2.646369in}}{\pgfqpoint{3.764515in}{2.636694in}}{\pgfqpoint{3.771648in}{2.629561in}}%
\pgfpathcurveto{\pgfqpoint{3.778781in}{2.622428in}}{\pgfqpoint{3.788456in}{2.618420in}}{\pgfqpoint{3.798544in}{2.618420in}}%
\pgfpathclose%
\pgfusepath{stroke,fill}%
\end{pgfscope}%
\begin{pgfscope}%
\pgfpathrectangle{\pgfqpoint{1.280114in}{0.528000in}}{\pgfqpoint{3.487886in}{3.696000in}} %
\pgfusepath{clip}%
\pgfsetbuttcap%
\pgfsetroundjoin%
\definecolor{currentfill}{rgb}{0.121569,0.466667,0.705882}%
\pgfsetfillcolor{currentfill}%
\pgfsetlinewidth{1.003750pt}%
\definecolor{currentstroke}{rgb}{0.121569,0.466667,0.705882}%
\pgfsetstrokecolor{currentstroke}%
\pgfsetdash{}{0pt}%
\pgfpathmoveto{\pgfqpoint{1.778730in}{2.183838in}}%
\pgfpathcurveto{\pgfqpoint{1.788818in}{2.183838in}}{\pgfqpoint{1.798493in}{2.187846in}}{\pgfqpoint{1.805626in}{2.194979in}}%
\pgfpathcurveto{\pgfqpoint{1.812759in}{2.202112in}}{\pgfqpoint{1.816767in}{2.211787in}}{\pgfqpoint{1.816767in}{2.221874in}}%
\pgfpathcurveto{\pgfqpoint{1.816767in}{2.231962in}}{\pgfqpoint{1.812759in}{2.241637in}}{\pgfqpoint{1.805626in}{2.248770in}}%
\pgfpathcurveto{\pgfqpoint{1.798493in}{2.255903in}}{\pgfqpoint{1.788818in}{2.259911in}}{\pgfqpoint{1.778730in}{2.259911in}}%
\pgfpathcurveto{\pgfqpoint{1.768643in}{2.259911in}}{\pgfqpoint{1.758967in}{2.255903in}}{\pgfqpoint{1.751835in}{2.248770in}}%
\pgfpathcurveto{\pgfqpoint{1.744702in}{2.241637in}}{\pgfqpoint{1.740694in}{2.231962in}}{\pgfqpoint{1.740694in}{2.221874in}}%
\pgfpathcurveto{\pgfqpoint{1.740694in}{2.211787in}}{\pgfqpoint{1.744702in}{2.202112in}}{\pgfqpoint{1.751835in}{2.194979in}}%
\pgfpathcurveto{\pgfqpoint{1.758967in}{2.187846in}}{\pgfqpoint{1.768643in}{2.183838in}}{\pgfqpoint{1.778730in}{2.183838in}}%
\pgfpathclose%
\pgfusepath{stroke,fill}%
\end{pgfscope}%
\begin{pgfscope}%
\pgfpathrectangle{\pgfqpoint{1.280114in}{0.528000in}}{\pgfqpoint{3.487886in}{3.696000in}} %
\pgfusepath{clip}%
\pgfsetbuttcap%
\pgfsetroundjoin%
\definecolor{currentfill}{rgb}{0.121569,0.466667,0.705882}%
\pgfsetfillcolor{currentfill}%
\pgfsetlinewidth{1.003750pt}%
\definecolor{currentstroke}{rgb}{0.121569,0.466667,0.705882}%
\pgfsetstrokecolor{currentstroke}%
\pgfsetdash{}{0pt}%
\pgfpathmoveto{\pgfqpoint{2.409036in}{2.023798in}}%
\pgfpathcurveto{\pgfqpoint{2.419123in}{2.023798in}}{\pgfqpoint{2.428798in}{2.027806in}}{\pgfqpoint{2.435931in}{2.034939in}}%
\pgfpathcurveto{\pgfqpoint{2.443064in}{2.042072in}}{\pgfqpoint{2.447072in}{2.051747in}}{\pgfqpoint{2.447072in}{2.061835in}}%
\pgfpathcurveto{\pgfqpoint{2.447072in}{2.071922in}}{\pgfqpoint{2.443064in}{2.081597in}}{\pgfqpoint{2.435931in}{2.088730in}}%
\pgfpathcurveto{\pgfqpoint{2.428798in}{2.095863in}}{\pgfqpoint{2.419123in}{2.099871in}}{\pgfqpoint{2.409036in}{2.099871in}}%
\pgfpathcurveto{\pgfqpoint{2.398948in}{2.099871in}}{\pgfqpoint{2.389273in}{2.095863in}}{\pgfqpoint{2.382140in}{2.088730in}}%
\pgfpathcurveto{\pgfqpoint{2.375007in}{2.081597in}}{\pgfqpoint{2.370999in}{2.071922in}}{\pgfqpoint{2.370999in}{2.061835in}}%
\pgfpathcurveto{\pgfqpoint{2.370999in}{2.051747in}}{\pgfqpoint{2.375007in}{2.042072in}}{\pgfqpoint{2.382140in}{2.034939in}}%
\pgfpathcurveto{\pgfqpoint{2.389273in}{2.027806in}}{\pgfqpoint{2.398948in}{2.023798in}}{\pgfqpoint{2.409036in}{2.023798in}}%
\pgfpathclose%
\pgfusepath{stroke,fill}%
\end{pgfscope}%
\begin{pgfscope}%
\pgfpathrectangle{\pgfqpoint{1.280114in}{0.528000in}}{\pgfqpoint{3.487886in}{3.696000in}} %
\pgfusepath{clip}%
\pgfsetbuttcap%
\pgfsetroundjoin%
\definecolor{currentfill}{rgb}{0.121569,0.466667,0.705882}%
\pgfsetfillcolor{currentfill}%
\pgfsetlinewidth{1.003750pt}%
\definecolor{currentstroke}{rgb}{0.121569,0.466667,0.705882}%
\pgfsetstrokecolor{currentstroke}%
\pgfsetdash{}{0pt}%
\pgfpathmoveto{\pgfqpoint{4.344884in}{2.609234in}}%
\pgfpathcurveto{\pgfqpoint{4.354972in}{2.609234in}}{\pgfqpoint{4.364647in}{2.613242in}}{\pgfqpoint{4.371780in}{2.620375in}}%
\pgfpathcurveto{\pgfqpoint{4.378913in}{2.627508in}}{\pgfqpoint{4.382921in}{2.637183in}}{\pgfqpoint{4.382921in}{2.647271in}}%
\pgfpathcurveto{\pgfqpoint{4.382921in}{2.657358in}}{\pgfqpoint{4.378913in}{2.667034in}}{\pgfqpoint{4.371780in}{2.674166in}}%
\pgfpathcurveto{\pgfqpoint{4.364647in}{2.681299in}}{\pgfqpoint{4.354972in}{2.685307in}}{\pgfqpoint{4.344884in}{2.685307in}}%
\pgfpathcurveto{\pgfqpoint{4.334797in}{2.685307in}}{\pgfqpoint{4.325121in}{2.681299in}}{\pgfqpoint{4.317989in}{2.674166in}}%
\pgfpathcurveto{\pgfqpoint{4.310856in}{2.667034in}}{\pgfqpoint{4.306848in}{2.657358in}}{\pgfqpoint{4.306848in}{2.647271in}}%
\pgfpathcurveto{\pgfqpoint{4.306848in}{2.637183in}}{\pgfqpoint{4.310856in}{2.627508in}}{\pgfqpoint{4.317989in}{2.620375in}}%
\pgfpathcurveto{\pgfqpoint{4.325121in}{2.613242in}}{\pgfqpoint{4.334797in}{2.609234in}}{\pgfqpoint{4.344884in}{2.609234in}}%
\pgfpathclose%
\pgfusepath{stroke,fill}%
\end{pgfscope}%
\begin{pgfscope}%
\pgfpathrectangle{\pgfqpoint{1.280114in}{0.528000in}}{\pgfqpoint{3.487886in}{3.696000in}} %
\pgfusepath{clip}%
\pgfsetbuttcap%
\pgfsetroundjoin%
\definecolor{currentfill}{rgb}{0.121569,0.466667,0.705882}%
\pgfsetfillcolor{currentfill}%
\pgfsetlinewidth{1.003750pt}%
\definecolor{currentstroke}{rgb}{0.121569,0.466667,0.705882}%
\pgfsetstrokecolor{currentstroke}%
\pgfsetdash{}{0pt}%
\pgfpathmoveto{\pgfqpoint{1.809305in}{2.583655in}}%
\pgfpathcurveto{\pgfqpoint{1.819392in}{2.583655in}}{\pgfqpoint{1.829068in}{2.587662in}}{\pgfqpoint{1.836200in}{2.594795in}}%
\pgfpathcurveto{\pgfqpoint{1.843333in}{2.601928in}}{\pgfqpoint{1.847341in}{2.611604in}}{\pgfqpoint{1.847341in}{2.621691in}}%
\pgfpathcurveto{\pgfqpoint{1.847341in}{2.631778in}}{\pgfqpoint{1.843333in}{2.641454in}}{\pgfqpoint{1.836200in}{2.648587in}}%
\pgfpathcurveto{\pgfqpoint{1.829068in}{2.655719in}}{\pgfqpoint{1.819392in}{2.659727in}}{\pgfqpoint{1.809305in}{2.659727in}}%
\pgfpathcurveto{\pgfqpoint{1.799217in}{2.659727in}}{\pgfqpoint{1.789542in}{2.655719in}}{\pgfqpoint{1.782409in}{2.648587in}}%
\pgfpathcurveto{\pgfqpoint{1.775276in}{2.641454in}}{\pgfqpoint{1.771268in}{2.631778in}}{\pgfqpoint{1.771268in}{2.621691in}}%
\pgfpathcurveto{\pgfqpoint{1.771268in}{2.611604in}}{\pgfqpoint{1.775276in}{2.601928in}}{\pgfqpoint{1.782409in}{2.594795in}}%
\pgfpathcurveto{\pgfqpoint{1.789542in}{2.587662in}}{\pgfqpoint{1.799217in}{2.583655in}}{\pgfqpoint{1.809305in}{2.583655in}}%
\pgfpathclose%
\pgfusepath{stroke,fill}%
\end{pgfscope}%
\begin{pgfscope}%
\pgfpathrectangle{\pgfqpoint{1.280114in}{0.528000in}}{\pgfqpoint{3.487886in}{3.696000in}} %
\pgfusepath{clip}%
\pgfsetbuttcap%
\pgfsetroundjoin%
\definecolor{currentfill}{rgb}{0.121569,0.466667,0.705882}%
\pgfsetfillcolor{currentfill}%
\pgfsetlinewidth{1.003750pt}%
\definecolor{currentstroke}{rgb}{0.121569,0.466667,0.705882}%
\pgfsetstrokecolor{currentstroke}%
\pgfsetdash{}{0pt}%
\pgfpathmoveto{\pgfqpoint{2.528795in}{2.006331in}}%
\pgfpathcurveto{\pgfqpoint{2.538883in}{2.006331in}}{\pgfqpoint{2.548558in}{2.010339in}}{\pgfqpoint{2.555691in}{2.017471in}}%
\pgfpathcurveto{\pgfqpoint{2.562824in}{2.024604in}}{\pgfqpoint{2.566832in}{2.034280in}}{\pgfqpoint{2.566832in}{2.044367in}}%
\pgfpathcurveto{\pgfqpoint{2.566832in}{2.054455in}}{\pgfqpoint{2.562824in}{2.064130in}}{\pgfqpoint{2.555691in}{2.071263in}}%
\pgfpathcurveto{\pgfqpoint{2.548558in}{2.078396in}}{\pgfqpoint{2.538883in}{2.082403in}}{\pgfqpoint{2.528795in}{2.082403in}}%
\pgfpathcurveto{\pgfqpoint{2.518708in}{2.082403in}}{\pgfqpoint{2.509032in}{2.078396in}}{\pgfqpoint{2.501900in}{2.071263in}}%
\pgfpathcurveto{\pgfqpoint{2.494767in}{2.064130in}}{\pgfqpoint{2.490759in}{2.054455in}}{\pgfqpoint{2.490759in}{2.044367in}}%
\pgfpathcurveto{\pgfqpoint{2.490759in}{2.034280in}}{\pgfqpoint{2.494767in}{2.024604in}}{\pgfqpoint{2.501900in}{2.017471in}}%
\pgfpathcurveto{\pgfqpoint{2.509032in}{2.010339in}}{\pgfqpoint{2.518708in}{2.006331in}}{\pgfqpoint{2.528795in}{2.006331in}}%
\pgfpathclose%
\pgfusepath{stroke,fill}%
\end{pgfscope}%
\begin{pgfscope}%
\pgfpathrectangle{\pgfqpoint{1.280114in}{0.528000in}}{\pgfqpoint{3.487886in}{3.696000in}} %
\pgfusepath{clip}%
\pgfsetbuttcap%
\pgfsetroundjoin%
\definecolor{currentfill}{rgb}{0.121569,0.466667,0.705882}%
\pgfsetfillcolor{currentfill}%
\pgfsetlinewidth{1.003750pt}%
\definecolor{currentstroke}{rgb}{0.121569,0.466667,0.705882}%
\pgfsetstrokecolor{currentstroke}%
\pgfsetdash{}{0pt}%
\pgfpathmoveto{\pgfqpoint{1.684449in}{2.577624in}}%
\pgfpathcurveto{\pgfqpoint{1.694537in}{2.577624in}}{\pgfqpoint{1.704212in}{2.581632in}}{\pgfqpoint{1.711345in}{2.588765in}}%
\pgfpathcurveto{\pgfqpoint{1.718478in}{2.595898in}}{\pgfqpoint{1.722486in}{2.605573in}}{\pgfqpoint{1.722486in}{2.615661in}}%
\pgfpathcurveto{\pgfqpoint{1.722486in}{2.625748in}}{\pgfqpoint{1.718478in}{2.635424in}}{\pgfqpoint{1.711345in}{2.642556in}}%
\pgfpathcurveto{\pgfqpoint{1.704212in}{2.649689in}}{\pgfqpoint{1.694537in}{2.653697in}}{\pgfqpoint{1.684449in}{2.653697in}}%
\pgfpathcurveto{\pgfqpoint{1.674362in}{2.653697in}}{\pgfqpoint{1.664687in}{2.649689in}}{\pgfqpoint{1.657554in}{2.642556in}}%
\pgfpathcurveto{\pgfqpoint{1.650421in}{2.635424in}}{\pgfqpoint{1.646413in}{2.625748in}}{\pgfqpoint{1.646413in}{2.615661in}}%
\pgfpathcurveto{\pgfqpoint{1.646413in}{2.605573in}}{\pgfqpoint{1.650421in}{2.595898in}}{\pgfqpoint{1.657554in}{2.588765in}}%
\pgfpathcurveto{\pgfqpoint{1.664687in}{2.581632in}}{\pgfqpoint{1.674362in}{2.577624in}}{\pgfqpoint{1.684449in}{2.577624in}}%
\pgfpathclose%
\pgfusepath{stroke,fill}%
\end{pgfscope}%
\begin{pgfscope}%
\pgfpathrectangle{\pgfqpoint{1.280114in}{0.528000in}}{\pgfqpoint{3.487886in}{3.696000in}} %
\pgfusepath{clip}%
\pgfsetbuttcap%
\pgfsetroundjoin%
\definecolor{currentfill}{rgb}{0.121569,0.466667,0.705882}%
\pgfsetfillcolor{currentfill}%
\pgfsetlinewidth{1.003750pt}%
\definecolor{currentstroke}{rgb}{0.121569,0.466667,0.705882}%
\pgfsetstrokecolor{currentstroke}%
\pgfsetdash{}{0pt}%
\pgfpathmoveto{\pgfqpoint{4.584743in}{2.465286in}}%
\pgfpathcurveto{\pgfqpoint{4.594831in}{2.465286in}}{\pgfqpoint{4.604506in}{2.469294in}}{\pgfqpoint{4.611639in}{2.476427in}}%
\pgfpathcurveto{\pgfqpoint{4.618772in}{2.483559in}}{\pgfqpoint{4.622780in}{2.493235in}}{\pgfqpoint{4.622780in}{2.503322in}}%
\pgfpathcurveto{\pgfqpoint{4.622780in}{2.513410in}}{\pgfqpoint{4.618772in}{2.523085in}}{\pgfqpoint{4.611639in}{2.530218in}}%
\pgfpathcurveto{\pgfqpoint{4.604506in}{2.537351in}}{\pgfqpoint{4.594831in}{2.541359in}}{\pgfqpoint{4.584743in}{2.541359in}}%
\pgfpathcurveto{\pgfqpoint{4.574656in}{2.541359in}}{\pgfqpoint{4.564980in}{2.537351in}}{\pgfqpoint{4.557848in}{2.530218in}}%
\pgfpathcurveto{\pgfqpoint{4.550715in}{2.523085in}}{\pgfqpoint{4.546707in}{2.513410in}}{\pgfqpoint{4.546707in}{2.503322in}}%
\pgfpathcurveto{\pgfqpoint{4.546707in}{2.493235in}}{\pgfqpoint{4.550715in}{2.483559in}}{\pgfqpoint{4.557848in}{2.476427in}}%
\pgfpathcurveto{\pgfqpoint{4.564980in}{2.469294in}}{\pgfqpoint{4.574656in}{2.465286in}}{\pgfqpoint{4.584743in}{2.465286in}}%
\pgfpathclose%
\pgfusepath{stroke,fill}%
\end{pgfscope}%
\begin{pgfscope}%
\pgfpathrectangle{\pgfqpoint{1.280114in}{0.528000in}}{\pgfqpoint{3.487886in}{3.696000in}} %
\pgfusepath{clip}%
\pgfsetbuttcap%
\pgfsetroundjoin%
\definecolor{currentfill}{rgb}{0.121569,0.466667,0.705882}%
\pgfsetfillcolor{currentfill}%
\pgfsetlinewidth{1.003750pt}%
\definecolor{currentstroke}{rgb}{0.121569,0.466667,0.705882}%
\pgfsetstrokecolor{currentstroke}%
\pgfsetdash{}{0pt}%
\pgfpathmoveto{\pgfqpoint{1.799237in}{2.333793in}}%
\pgfpathcurveto{\pgfqpoint{1.809324in}{2.333793in}}{\pgfqpoint{1.818999in}{2.337801in}}{\pgfqpoint{1.826132in}{2.344934in}}%
\pgfpathcurveto{\pgfqpoint{1.833265in}{2.352067in}}{\pgfqpoint{1.837273in}{2.361742in}}{\pgfqpoint{1.837273in}{2.371830in}}%
\pgfpathcurveto{\pgfqpoint{1.837273in}{2.381917in}}{\pgfqpoint{1.833265in}{2.391592in}}{\pgfqpoint{1.826132in}{2.398725in}}%
\pgfpathcurveto{\pgfqpoint{1.818999in}{2.405858in}}{\pgfqpoint{1.809324in}{2.409866in}}{\pgfqpoint{1.799237in}{2.409866in}}%
\pgfpathcurveto{\pgfqpoint{1.789149in}{2.409866in}}{\pgfqpoint{1.779474in}{2.405858in}}{\pgfqpoint{1.772341in}{2.398725in}}%
\pgfpathcurveto{\pgfqpoint{1.765208in}{2.391592in}}{\pgfqpoint{1.761200in}{2.381917in}}{\pgfqpoint{1.761200in}{2.371830in}}%
\pgfpathcurveto{\pgfqpoint{1.761200in}{2.361742in}}{\pgfqpoint{1.765208in}{2.352067in}}{\pgfqpoint{1.772341in}{2.344934in}}%
\pgfpathcurveto{\pgfqpoint{1.779474in}{2.337801in}}{\pgfqpoint{1.789149in}{2.333793in}}{\pgfqpoint{1.799237in}{2.333793in}}%
\pgfpathclose%
\pgfusepath{stroke,fill}%
\end{pgfscope}%
\begin{pgfscope}%
\pgfpathrectangle{\pgfqpoint{1.280114in}{0.528000in}}{\pgfqpoint{3.487886in}{3.696000in}} %
\pgfusepath{clip}%
\pgfsetbuttcap%
\pgfsetroundjoin%
\definecolor{currentfill}{rgb}{0.121569,0.466667,0.705882}%
\pgfsetfillcolor{currentfill}%
\pgfsetlinewidth{1.003750pt}%
\definecolor{currentstroke}{rgb}{0.121569,0.466667,0.705882}%
\pgfsetstrokecolor{currentstroke}%
\pgfsetdash{}{0pt}%
\pgfpathmoveto{\pgfqpoint{2.079232in}{2.012601in}}%
\pgfpathcurveto{\pgfqpoint{2.089320in}{2.012601in}}{\pgfqpoint{2.098995in}{2.016609in}}{\pgfqpoint{2.106128in}{2.023742in}}%
\pgfpathcurveto{\pgfqpoint{2.113261in}{2.030874in}}{\pgfqpoint{2.117269in}{2.040550in}}{\pgfqpoint{2.117269in}{2.050637in}}%
\pgfpathcurveto{\pgfqpoint{2.117269in}{2.060725in}}{\pgfqpoint{2.113261in}{2.070400in}}{\pgfqpoint{2.106128in}{2.077533in}}%
\pgfpathcurveto{\pgfqpoint{2.098995in}{2.084666in}}{\pgfqpoint{2.089320in}{2.088674in}}{\pgfqpoint{2.079232in}{2.088674in}}%
\pgfpathcurveto{\pgfqpoint{2.069145in}{2.088674in}}{\pgfqpoint{2.059469in}{2.084666in}}{\pgfqpoint{2.052337in}{2.077533in}}%
\pgfpathcurveto{\pgfqpoint{2.045204in}{2.070400in}}{\pgfqpoint{2.041196in}{2.060725in}}{\pgfqpoint{2.041196in}{2.050637in}}%
\pgfpathcurveto{\pgfqpoint{2.041196in}{2.040550in}}{\pgfqpoint{2.045204in}{2.030874in}}{\pgfqpoint{2.052337in}{2.023742in}}%
\pgfpathcurveto{\pgfqpoint{2.059469in}{2.016609in}}{\pgfqpoint{2.069145in}{2.012601in}}{\pgfqpoint{2.079232in}{2.012601in}}%
\pgfpathclose%
\pgfusepath{stroke,fill}%
\end{pgfscope}%
\begin{pgfscope}%
\pgfpathrectangle{\pgfqpoint{1.280114in}{0.528000in}}{\pgfqpoint{3.487886in}{3.696000in}} %
\pgfusepath{clip}%
\pgfsetbuttcap%
\pgfsetroundjoin%
\definecolor{currentfill}{rgb}{0.121569,0.466667,0.705882}%
\pgfsetfillcolor{currentfill}%
\pgfsetlinewidth{1.003750pt}%
\definecolor{currentstroke}{rgb}{0.121569,0.466667,0.705882}%
\pgfsetstrokecolor{currentstroke}%
\pgfsetdash{}{0pt}%
\pgfpathmoveto{\pgfqpoint{1.691426in}{2.284627in}}%
\pgfpathcurveto{\pgfqpoint{1.701513in}{2.284627in}}{\pgfqpoint{1.711189in}{2.288634in}}{\pgfqpoint{1.718322in}{2.295767in}}%
\pgfpathcurveto{\pgfqpoint{1.725454in}{2.302900in}}{\pgfqpoint{1.729462in}{2.312575in}}{\pgfqpoint{1.729462in}{2.322663in}}%
\pgfpathcurveto{\pgfqpoint{1.729462in}{2.332750in}}{\pgfqpoint{1.725454in}{2.342426in}}{\pgfqpoint{1.718322in}{2.349559in}}%
\pgfpathcurveto{\pgfqpoint{1.711189in}{2.356691in}}{\pgfqpoint{1.701513in}{2.360699in}}{\pgfqpoint{1.691426in}{2.360699in}}%
\pgfpathcurveto{\pgfqpoint{1.681338in}{2.360699in}}{\pgfqpoint{1.671663in}{2.356691in}}{\pgfqpoint{1.664530in}{2.349559in}}%
\pgfpathcurveto{\pgfqpoint{1.657397in}{2.342426in}}{\pgfqpoint{1.653390in}{2.332750in}}{\pgfqpoint{1.653390in}{2.322663in}}%
\pgfpathcurveto{\pgfqpoint{1.653390in}{2.312575in}}{\pgfqpoint{1.657397in}{2.302900in}}{\pgfqpoint{1.664530in}{2.295767in}}%
\pgfpathcurveto{\pgfqpoint{1.671663in}{2.288634in}}{\pgfqpoint{1.681338in}{2.284627in}}{\pgfqpoint{1.691426in}{2.284627in}}%
\pgfpathclose%
\pgfusepath{stroke,fill}%
\end{pgfscope}%
\begin{pgfscope}%
\pgfpathrectangle{\pgfqpoint{1.280114in}{0.528000in}}{\pgfqpoint{3.487886in}{3.696000in}} %
\pgfusepath{clip}%
\pgfsetbuttcap%
\pgfsetroundjoin%
\definecolor{currentfill}{rgb}{0.121569,0.466667,0.705882}%
\pgfsetfillcolor{currentfill}%
\pgfsetlinewidth{1.003750pt}%
\definecolor{currentstroke}{rgb}{0.121569,0.466667,0.705882}%
\pgfsetstrokecolor{currentstroke}%
\pgfsetdash{}{0pt}%
\pgfpathmoveto{\pgfqpoint{4.605362in}{2.195597in}}%
\pgfpathcurveto{\pgfqpoint{4.615450in}{2.195597in}}{\pgfqpoint{4.625125in}{2.199605in}}{\pgfqpoint{4.632258in}{2.206738in}}%
\pgfpathcurveto{\pgfqpoint{4.639391in}{2.213870in}}{\pgfqpoint{4.643399in}{2.223546in}}{\pgfqpoint{4.643399in}{2.233633in}}%
\pgfpathcurveto{\pgfqpoint{4.643399in}{2.243721in}}{\pgfqpoint{4.639391in}{2.253396in}}{\pgfqpoint{4.632258in}{2.260529in}}%
\pgfpathcurveto{\pgfqpoint{4.625125in}{2.267662in}}{\pgfqpoint{4.615450in}{2.271670in}}{\pgfqpoint{4.605362in}{2.271670in}}%
\pgfpathcurveto{\pgfqpoint{4.595275in}{2.271670in}}{\pgfqpoint{4.585599in}{2.267662in}}{\pgfqpoint{4.578467in}{2.260529in}}%
\pgfpathcurveto{\pgfqpoint{4.571334in}{2.253396in}}{\pgfqpoint{4.567326in}{2.243721in}}{\pgfqpoint{4.567326in}{2.233633in}}%
\pgfpathcurveto{\pgfqpoint{4.567326in}{2.223546in}}{\pgfqpoint{4.571334in}{2.213870in}}{\pgfqpoint{4.578467in}{2.206738in}}%
\pgfpathcurveto{\pgfqpoint{4.585599in}{2.199605in}}{\pgfqpoint{4.595275in}{2.195597in}}{\pgfqpoint{4.605362in}{2.195597in}}%
\pgfpathclose%
\pgfusepath{stroke,fill}%
\end{pgfscope}%
\begin{pgfscope}%
\pgfpathrectangle{\pgfqpoint{1.280114in}{0.528000in}}{\pgfqpoint{3.487886in}{3.696000in}} %
\pgfusepath{clip}%
\pgfsetbuttcap%
\pgfsetroundjoin%
\definecolor{currentfill}{rgb}{0.121569,0.466667,0.705882}%
\pgfsetfillcolor{currentfill}%
\pgfsetlinewidth{1.003750pt}%
\definecolor{currentstroke}{rgb}{0.121569,0.466667,0.705882}%
\pgfsetstrokecolor{currentstroke}%
\pgfsetdash{}{0pt}%
\pgfpathmoveto{\pgfqpoint{1.804804in}{2.150128in}}%
\pgfpathcurveto{\pgfqpoint{1.814891in}{2.150128in}}{\pgfqpoint{1.824567in}{2.154136in}}{\pgfqpoint{1.831700in}{2.161269in}}%
\pgfpathcurveto{\pgfqpoint{1.838833in}{2.168401in}}{\pgfqpoint{1.842840in}{2.178077in}}{\pgfqpoint{1.842840in}{2.188164in}}%
\pgfpathcurveto{\pgfqpoint{1.842840in}{2.198252in}}{\pgfqpoint{1.838833in}{2.207927in}}{\pgfqpoint{1.831700in}{2.215060in}}%
\pgfpathcurveto{\pgfqpoint{1.824567in}{2.222193in}}{\pgfqpoint{1.814891in}{2.226201in}}{\pgfqpoint{1.804804in}{2.226201in}}%
\pgfpathcurveto{\pgfqpoint{1.794717in}{2.226201in}}{\pgfqpoint{1.785041in}{2.222193in}}{\pgfqpoint{1.777908in}{2.215060in}}%
\pgfpathcurveto{\pgfqpoint{1.770776in}{2.207927in}}{\pgfqpoint{1.766768in}{2.198252in}}{\pgfqpoint{1.766768in}{2.188164in}}%
\pgfpathcurveto{\pgfqpoint{1.766768in}{2.178077in}}{\pgfqpoint{1.770776in}{2.168401in}}{\pgfqpoint{1.777908in}{2.161269in}}%
\pgfpathcurveto{\pgfqpoint{1.785041in}{2.154136in}}{\pgfqpoint{1.794717in}{2.150128in}}{\pgfqpoint{1.804804in}{2.150128in}}%
\pgfpathclose%
\pgfusepath{stroke,fill}%
\end{pgfscope}%
\begin{pgfscope}%
\pgfpathrectangle{\pgfqpoint{1.280114in}{0.528000in}}{\pgfqpoint{3.487886in}{3.696000in}} %
\pgfusepath{clip}%
\pgfsetbuttcap%
\pgfsetroundjoin%
\definecolor{currentfill}{rgb}{0.121569,0.466667,0.705882}%
\pgfsetfillcolor{currentfill}%
\pgfsetlinewidth{1.003750pt}%
\definecolor{currentstroke}{rgb}{0.121569,0.466667,0.705882}%
\pgfsetstrokecolor{currentstroke}%
\pgfsetdash{}{0pt}%
\pgfpathmoveto{\pgfqpoint{4.132629in}{2.124491in}}%
\pgfpathcurveto{\pgfqpoint{4.142717in}{2.124491in}}{\pgfqpoint{4.152392in}{2.128499in}}{\pgfqpoint{4.159525in}{2.135631in}}%
\pgfpathcurveto{\pgfqpoint{4.166658in}{2.142764in}}{\pgfqpoint{4.170666in}{2.152440in}}{\pgfqpoint{4.170666in}{2.162527in}}%
\pgfpathcurveto{\pgfqpoint{4.170666in}{2.172614in}}{\pgfqpoint{4.166658in}{2.182290in}}{\pgfqpoint{4.159525in}{2.189423in}}%
\pgfpathcurveto{\pgfqpoint{4.152392in}{2.196556in}}{\pgfqpoint{4.142717in}{2.200563in}}{\pgfqpoint{4.132629in}{2.200563in}}%
\pgfpathcurveto{\pgfqpoint{4.122542in}{2.200563in}}{\pgfqpoint{4.112867in}{2.196556in}}{\pgfqpoint{4.105734in}{2.189423in}}%
\pgfpathcurveto{\pgfqpoint{4.098601in}{2.182290in}}{\pgfqpoint{4.094593in}{2.172614in}}{\pgfqpoint{4.094593in}{2.162527in}}%
\pgfpathcurveto{\pgfqpoint{4.094593in}{2.152440in}}{\pgfqpoint{4.098601in}{2.142764in}}{\pgfqpoint{4.105734in}{2.135631in}}%
\pgfpathcurveto{\pgfqpoint{4.112867in}{2.128499in}}{\pgfqpoint{4.122542in}{2.124491in}}{\pgfqpoint{4.132629in}{2.124491in}}%
\pgfpathclose%
\pgfusepath{stroke,fill}%
\end{pgfscope}%
\begin{pgfscope}%
\pgfpathrectangle{\pgfqpoint{1.280114in}{0.528000in}}{\pgfqpoint{3.487886in}{3.696000in}} %
\pgfusepath{clip}%
\pgfsetbuttcap%
\pgfsetroundjoin%
\definecolor{currentfill}{rgb}{0.121569,0.466667,0.705882}%
\pgfsetfillcolor{currentfill}%
\pgfsetlinewidth{1.003750pt}%
\definecolor{currentstroke}{rgb}{0.121569,0.466667,0.705882}%
\pgfsetstrokecolor{currentstroke}%
\pgfsetdash{}{0pt}%
\pgfpathmoveto{\pgfqpoint{4.094074in}{2.698912in}}%
\pgfpathcurveto{\pgfqpoint{4.104162in}{2.698912in}}{\pgfqpoint{4.113837in}{2.702919in}}{\pgfqpoint{4.120970in}{2.710052in}}%
\pgfpathcurveto{\pgfqpoint{4.128103in}{2.717185in}}{\pgfqpoint{4.132111in}{2.726861in}}{\pgfqpoint{4.132111in}{2.736948in}}%
\pgfpathcurveto{\pgfqpoint{4.132111in}{2.747035in}}{\pgfqpoint{4.128103in}{2.756711in}}{\pgfqpoint{4.120970in}{2.763844in}}%
\pgfpathcurveto{\pgfqpoint{4.113837in}{2.770976in}}{\pgfqpoint{4.104162in}{2.774984in}}{\pgfqpoint{4.094074in}{2.774984in}}%
\pgfpathcurveto{\pgfqpoint{4.083987in}{2.774984in}}{\pgfqpoint{4.074311in}{2.770976in}}{\pgfqpoint{4.067179in}{2.763844in}}%
\pgfpathcurveto{\pgfqpoint{4.060046in}{2.756711in}}{\pgfqpoint{4.056038in}{2.747035in}}{\pgfqpoint{4.056038in}{2.736948in}}%
\pgfpathcurveto{\pgfqpoint{4.056038in}{2.726861in}}{\pgfqpoint{4.060046in}{2.717185in}}{\pgfqpoint{4.067179in}{2.710052in}}%
\pgfpathcurveto{\pgfqpoint{4.074311in}{2.702919in}}{\pgfqpoint{4.083987in}{2.698912in}}{\pgfqpoint{4.094074in}{2.698912in}}%
\pgfpathclose%
\pgfusepath{stroke,fill}%
\end{pgfscope}%
\begin{pgfscope}%
\pgfpathrectangle{\pgfqpoint{1.280114in}{0.528000in}}{\pgfqpoint{3.487886in}{3.696000in}} %
\pgfusepath{clip}%
\pgfsetbuttcap%
\pgfsetroundjoin%
\definecolor{currentfill}{rgb}{0.121569,0.466667,0.705882}%
\pgfsetfillcolor{currentfill}%
\pgfsetlinewidth{1.003750pt}%
\definecolor{currentstroke}{rgb}{0.121569,0.466667,0.705882}%
\pgfsetstrokecolor{currentstroke}%
\pgfsetdash{}{0pt}%
\pgfpathmoveto{\pgfqpoint{3.971646in}{2.641734in}}%
\pgfpathcurveto{\pgfqpoint{3.981733in}{2.641734in}}{\pgfqpoint{3.991408in}{2.645742in}}{\pgfqpoint{3.998541in}{2.652875in}}%
\pgfpathcurveto{\pgfqpoint{4.005674in}{2.660007in}}{\pgfqpoint{4.009682in}{2.669683in}}{\pgfqpoint{4.009682in}{2.679770in}}%
\pgfpathcurveto{\pgfqpoint{4.009682in}{2.689858in}}{\pgfqpoint{4.005674in}{2.699533in}}{\pgfqpoint{3.998541in}{2.706666in}}%
\pgfpathcurveto{\pgfqpoint{3.991408in}{2.713799in}}{\pgfqpoint{3.981733in}{2.717807in}}{\pgfqpoint{3.971646in}{2.717807in}}%
\pgfpathcurveto{\pgfqpoint{3.961558in}{2.717807in}}{\pgfqpoint{3.951883in}{2.713799in}}{\pgfqpoint{3.944750in}{2.706666in}}%
\pgfpathcurveto{\pgfqpoint{3.937617in}{2.699533in}}{\pgfqpoint{3.933609in}{2.689858in}}{\pgfqpoint{3.933609in}{2.679770in}}%
\pgfpathcurveto{\pgfqpoint{3.933609in}{2.669683in}}{\pgfqpoint{3.937617in}{2.660007in}}{\pgfqpoint{3.944750in}{2.652875in}}%
\pgfpathcurveto{\pgfqpoint{3.951883in}{2.645742in}}{\pgfqpoint{3.961558in}{2.641734in}}{\pgfqpoint{3.971646in}{2.641734in}}%
\pgfpathclose%
\pgfusepath{stroke,fill}%
\end{pgfscope}%
\begin{pgfscope}%
\pgfpathrectangle{\pgfqpoint{1.280114in}{0.528000in}}{\pgfqpoint{3.487886in}{3.696000in}} %
\pgfusepath{clip}%
\pgfsetbuttcap%
\pgfsetroundjoin%
\definecolor{currentfill}{rgb}{0.121569,0.466667,0.705882}%
\pgfsetfillcolor{currentfill}%
\pgfsetlinewidth{1.003750pt}%
\definecolor{currentstroke}{rgb}{0.121569,0.466667,0.705882}%
\pgfsetstrokecolor{currentstroke}%
\pgfsetdash{}{0pt}%
\pgfpathmoveto{\pgfqpoint{3.499681in}{1.983073in}}%
\pgfpathcurveto{\pgfqpoint{3.509769in}{1.983073in}}{\pgfqpoint{3.519444in}{1.987080in}}{\pgfqpoint{3.526577in}{1.994213in}}%
\pgfpathcurveto{\pgfqpoint{3.533710in}{2.001346in}}{\pgfqpoint{3.537718in}{2.011022in}}{\pgfqpoint{3.537718in}{2.021109in}}%
\pgfpathcurveto{\pgfqpoint{3.537718in}{2.031196in}}{\pgfqpoint{3.533710in}{2.040872in}}{\pgfqpoint{3.526577in}{2.048005in}}%
\pgfpathcurveto{\pgfqpoint{3.519444in}{2.055137in}}{\pgfqpoint{3.509769in}{2.059145in}}{\pgfqpoint{3.499681in}{2.059145in}}%
\pgfpathcurveto{\pgfqpoint{3.489594in}{2.059145in}}{\pgfqpoint{3.479918in}{2.055137in}}{\pgfqpoint{3.472786in}{2.048005in}}%
\pgfpathcurveto{\pgfqpoint{3.465653in}{2.040872in}}{\pgfqpoint{3.461645in}{2.031196in}}{\pgfqpoint{3.461645in}{2.021109in}}%
\pgfpathcurveto{\pgfqpoint{3.461645in}{2.011022in}}{\pgfqpoint{3.465653in}{2.001346in}}{\pgfqpoint{3.472786in}{1.994213in}}%
\pgfpathcurveto{\pgfqpoint{3.479918in}{1.987080in}}{\pgfqpoint{3.489594in}{1.983073in}}{\pgfqpoint{3.499681in}{1.983073in}}%
\pgfpathclose%
\pgfusepath{stroke,fill}%
\end{pgfscope}%
\begin{pgfscope}%
\pgfpathrectangle{\pgfqpoint{1.280114in}{0.528000in}}{\pgfqpoint{3.487886in}{3.696000in}} %
\pgfusepath{clip}%
\pgfsetbuttcap%
\pgfsetroundjoin%
\definecolor{currentfill}{rgb}{0.121569,0.466667,0.705882}%
\pgfsetfillcolor{currentfill}%
\pgfsetlinewidth{1.003750pt}%
\definecolor{currentstroke}{rgb}{0.121569,0.466667,0.705882}%
\pgfsetstrokecolor{currentstroke}%
\pgfsetdash{}{0pt}%
\pgfpathmoveto{\pgfqpoint{1.722427in}{2.617175in}}%
\pgfpathcurveto{\pgfqpoint{1.732514in}{2.617175in}}{\pgfqpoint{1.742190in}{2.621182in}}{\pgfqpoint{1.749323in}{2.628315in}}%
\pgfpathcurveto{\pgfqpoint{1.756456in}{2.635448in}}{\pgfqpoint{1.760463in}{2.645123in}}{\pgfqpoint{1.760463in}{2.655211in}}%
\pgfpathcurveto{\pgfqpoint{1.760463in}{2.665298in}}{\pgfqpoint{1.756456in}{2.674974in}}{\pgfqpoint{1.749323in}{2.682107in}}%
\pgfpathcurveto{\pgfqpoint{1.742190in}{2.689239in}}{\pgfqpoint{1.732514in}{2.693247in}}{\pgfqpoint{1.722427in}{2.693247in}}%
\pgfpathcurveto{\pgfqpoint{1.712340in}{2.693247in}}{\pgfqpoint{1.702664in}{2.689239in}}{\pgfqpoint{1.695531in}{2.682107in}}%
\pgfpathcurveto{\pgfqpoint{1.688399in}{2.674974in}}{\pgfqpoint{1.684391in}{2.665298in}}{\pgfqpoint{1.684391in}{2.655211in}}%
\pgfpathcurveto{\pgfqpoint{1.684391in}{2.645123in}}{\pgfqpoint{1.688399in}{2.635448in}}{\pgfqpoint{1.695531in}{2.628315in}}%
\pgfpathcurveto{\pgfqpoint{1.702664in}{2.621182in}}{\pgfqpoint{1.712340in}{2.617175in}}{\pgfqpoint{1.722427in}{2.617175in}}%
\pgfpathclose%
\pgfusepath{stroke,fill}%
\end{pgfscope}%
\begin{pgfscope}%
\pgfpathrectangle{\pgfqpoint{1.280114in}{0.528000in}}{\pgfqpoint{3.487886in}{3.696000in}} %
\pgfusepath{clip}%
\pgfsetbuttcap%
\pgfsetroundjoin%
\definecolor{currentfill}{rgb}{0.121569,0.466667,0.705882}%
\pgfsetfillcolor{currentfill}%
\pgfsetlinewidth{1.003750pt}%
\definecolor{currentstroke}{rgb}{0.121569,0.466667,0.705882}%
\pgfsetstrokecolor{currentstroke}%
\pgfsetdash{}{0pt}%
\pgfpathmoveto{\pgfqpoint{3.849832in}{2.669889in}}%
\pgfpathcurveto{\pgfqpoint{3.859919in}{2.669889in}}{\pgfqpoint{3.869595in}{2.673896in}}{\pgfqpoint{3.876728in}{2.681029in}}%
\pgfpathcurveto{\pgfqpoint{3.883861in}{2.688162in}}{\pgfqpoint{3.887868in}{2.697838in}}{\pgfqpoint{3.887868in}{2.707925in}}%
\pgfpathcurveto{\pgfqpoint{3.887868in}{2.718012in}}{\pgfqpoint{3.883861in}{2.727688in}}{\pgfqpoint{3.876728in}{2.734821in}}%
\pgfpathcurveto{\pgfqpoint{3.869595in}{2.741954in}}{\pgfqpoint{3.859919in}{2.745961in}}{\pgfqpoint{3.849832in}{2.745961in}}%
\pgfpathcurveto{\pgfqpoint{3.839745in}{2.745961in}}{\pgfqpoint{3.830069in}{2.741954in}}{\pgfqpoint{3.822936in}{2.734821in}}%
\pgfpathcurveto{\pgfqpoint{3.815804in}{2.727688in}}{\pgfqpoint{3.811796in}{2.718012in}}{\pgfqpoint{3.811796in}{2.707925in}}%
\pgfpathcurveto{\pgfqpoint{3.811796in}{2.697838in}}{\pgfqpoint{3.815804in}{2.688162in}}{\pgfqpoint{3.822936in}{2.681029in}}%
\pgfpathcurveto{\pgfqpoint{3.830069in}{2.673896in}}{\pgfqpoint{3.839745in}{2.669889in}}{\pgfqpoint{3.849832in}{2.669889in}}%
\pgfpathclose%
\pgfusepath{stroke,fill}%
\end{pgfscope}%
\begin{pgfscope}%
\pgfpathrectangle{\pgfqpoint{1.280114in}{0.528000in}}{\pgfqpoint{3.487886in}{3.696000in}} %
\pgfusepath{clip}%
\pgfsetbuttcap%
\pgfsetroundjoin%
\definecolor{currentfill}{rgb}{0.121569,0.466667,0.705882}%
\pgfsetfillcolor{currentfill}%
\pgfsetlinewidth{1.003750pt}%
\definecolor{currentstroke}{rgb}{0.121569,0.466667,0.705882}%
\pgfsetstrokecolor{currentstroke}%
\pgfsetdash{}{0pt}%
\pgfpathmoveto{\pgfqpoint{4.455465in}{2.169812in}}%
\pgfpathcurveto{\pgfqpoint{4.465553in}{2.169812in}}{\pgfqpoint{4.475228in}{2.173819in}}{\pgfqpoint{4.482361in}{2.180952in}}%
\pgfpathcurveto{\pgfqpoint{4.489494in}{2.188085in}}{\pgfqpoint{4.493501in}{2.197761in}}{\pgfqpoint{4.493501in}{2.207848in}}%
\pgfpathcurveto{\pgfqpoint{4.493501in}{2.217935in}}{\pgfqpoint{4.489494in}{2.227611in}}{\pgfqpoint{4.482361in}{2.234744in}}%
\pgfpathcurveto{\pgfqpoint{4.475228in}{2.241877in}}{\pgfqpoint{4.465553in}{2.245884in}}{\pgfqpoint{4.455465in}{2.245884in}}%
\pgfpathcurveto{\pgfqpoint{4.445378in}{2.245884in}}{\pgfqpoint{4.435702in}{2.241877in}}{\pgfqpoint{4.428569in}{2.234744in}}%
\pgfpathcurveto{\pgfqpoint{4.421437in}{2.227611in}}{\pgfqpoint{4.417429in}{2.217935in}}{\pgfqpoint{4.417429in}{2.207848in}}%
\pgfpathcurveto{\pgfqpoint{4.417429in}{2.197761in}}{\pgfqpoint{4.421437in}{2.188085in}}{\pgfqpoint{4.428569in}{2.180952in}}%
\pgfpathcurveto{\pgfqpoint{4.435702in}{2.173819in}}{\pgfqpoint{4.445378in}{2.169812in}}{\pgfqpoint{4.455465in}{2.169812in}}%
\pgfpathclose%
\pgfusepath{stroke,fill}%
\end{pgfscope}%
\begin{pgfscope}%
\pgfpathrectangle{\pgfqpoint{1.280114in}{0.528000in}}{\pgfqpoint{3.487886in}{3.696000in}} %
\pgfusepath{clip}%
\pgfsetbuttcap%
\pgfsetroundjoin%
\definecolor{currentfill}{rgb}{0.121569,0.466667,0.705882}%
\pgfsetfillcolor{currentfill}%
\pgfsetlinewidth{1.003750pt}%
\definecolor{currentstroke}{rgb}{0.121569,0.466667,0.705882}%
\pgfsetstrokecolor{currentstroke}%
\pgfsetdash{}{0pt}%
\pgfpathmoveto{\pgfqpoint{2.118509in}{2.691677in}}%
\pgfpathcurveto{\pgfqpoint{2.128596in}{2.691677in}}{\pgfqpoint{2.138272in}{2.695684in}}{\pgfqpoint{2.145404in}{2.702817in}}%
\pgfpathcurveto{\pgfqpoint{2.152537in}{2.709950in}}{\pgfqpoint{2.156545in}{2.719625in}}{\pgfqpoint{2.156545in}{2.729713in}}%
\pgfpathcurveto{\pgfqpoint{2.156545in}{2.739800in}}{\pgfqpoint{2.152537in}{2.749476in}}{\pgfqpoint{2.145404in}{2.756609in}}%
\pgfpathcurveto{\pgfqpoint{2.138272in}{2.763741in}}{\pgfqpoint{2.128596in}{2.767749in}}{\pgfqpoint{2.118509in}{2.767749in}}%
\pgfpathcurveto{\pgfqpoint{2.108421in}{2.767749in}}{\pgfqpoint{2.098746in}{2.763741in}}{\pgfqpoint{2.091613in}{2.756609in}}%
\pgfpathcurveto{\pgfqpoint{2.084480in}{2.749476in}}{\pgfqpoint{2.080472in}{2.739800in}}{\pgfqpoint{2.080472in}{2.729713in}}%
\pgfpathcurveto{\pgfqpoint{2.080472in}{2.719625in}}{\pgfqpoint{2.084480in}{2.709950in}}{\pgfqpoint{2.091613in}{2.702817in}}%
\pgfpathcurveto{\pgfqpoint{2.098746in}{2.695684in}}{\pgfqpoint{2.108421in}{2.691677in}}{\pgfqpoint{2.118509in}{2.691677in}}%
\pgfpathclose%
\pgfusepath{stroke,fill}%
\end{pgfscope}%
\begin{pgfscope}%
\pgfpathrectangle{\pgfqpoint{1.280114in}{0.528000in}}{\pgfqpoint{3.487886in}{3.696000in}} %
\pgfusepath{clip}%
\pgfsetbuttcap%
\pgfsetroundjoin%
\definecolor{currentfill}{rgb}{0.121569,0.466667,0.705882}%
\pgfsetfillcolor{currentfill}%
\pgfsetlinewidth{1.003750pt}%
\definecolor{currentstroke}{rgb}{0.121569,0.466667,0.705882}%
\pgfsetstrokecolor{currentstroke}%
\pgfsetdash{}{0pt}%
\pgfpathmoveto{\pgfqpoint{3.101594in}{2.010797in}}%
\pgfpathcurveto{\pgfqpoint{3.111682in}{2.010797in}}{\pgfqpoint{3.121357in}{2.014804in}}{\pgfqpoint{3.128490in}{2.021937in}}%
\pgfpathcurveto{\pgfqpoint{3.135623in}{2.029070in}}{\pgfqpoint{3.139631in}{2.038745in}}{\pgfqpoint{3.139631in}{2.048833in}}%
\pgfpathcurveto{\pgfqpoint{3.139631in}{2.058920in}}{\pgfqpoint{3.135623in}{2.068596in}}{\pgfqpoint{3.128490in}{2.075729in}}%
\pgfpathcurveto{\pgfqpoint{3.121357in}{2.082861in}}{\pgfqpoint{3.111682in}{2.086869in}}{\pgfqpoint{3.101594in}{2.086869in}}%
\pgfpathcurveto{\pgfqpoint{3.091507in}{2.086869in}}{\pgfqpoint{3.081832in}{2.082861in}}{\pgfqpoint{3.074699in}{2.075729in}}%
\pgfpathcurveto{\pgfqpoint{3.067566in}{2.068596in}}{\pgfqpoint{3.063558in}{2.058920in}}{\pgfqpoint{3.063558in}{2.048833in}}%
\pgfpathcurveto{\pgfqpoint{3.063558in}{2.038745in}}{\pgfqpoint{3.067566in}{2.029070in}}{\pgfqpoint{3.074699in}{2.021937in}}%
\pgfpathcurveto{\pgfqpoint{3.081832in}{2.014804in}}{\pgfqpoint{3.091507in}{2.010797in}}{\pgfqpoint{3.101594in}{2.010797in}}%
\pgfpathclose%
\pgfusepath{stroke,fill}%
\end{pgfscope}%
\begin{pgfscope}%
\pgfpathrectangle{\pgfqpoint{1.280114in}{0.528000in}}{\pgfqpoint{3.487886in}{3.696000in}} %
\pgfusepath{clip}%
\pgfsetbuttcap%
\pgfsetroundjoin%
\definecolor{currentfill}{rgb}{0.121569,0.466667,0.705882}%
\pgfsetfillcolor{currentfill}%
\pgfsetlinewidth{1.003750pt}%
\definecolor{currentstroke}{rgb}{0.121569,0.466667,0.705882}%
\pgfsetstrokecolor{currentstroke}%
\pgfsetdash{}{0pt}%
\pgfpathmoveto{\pgfqpoint{2.696311in}{1.987236in}}%
\pgfpathcurveto{\pgfqpoint{2.706398in}{1.987236in}}{\pgfqpoint{2.716074in}{1.991244in}}{\pgfqpoint{2.723207in}{1.998377in}}%
\pgfpathcurveto{\pgfqpoint{2.730339in}{2.005509in}}{\pgfqpoint{2.734347in}{2.015185in}}{\pgfqpoint{2.734347in}{2.025272in}}%
\pgfpathcurveto{\pgfqpoint{2.734347in}{2.035360in}}{\pgfqpoint{2.730339in}{2.045035in}}{\pgfqpoint{2.723207in}{2.052168in}}%
\pgfpathcurveto{\pgfqpoint{2.716074in}{2.059301in}}{\pgfqpoint{2.706398in}{2.063309in}}{\pgfqpoint{2.696311in}{2.063309in}}%
\pgfpathcurveto{\pgfqpoint{2.686224in}{2.063309in}}{\pgfqpoint{2.676548in}{2.059301in}}{\pgfqpoint{2.669415in}{2.052168in}}%
\pgfpathcurveto{\pgfqpoint{2.662282in}{2.045035in}}{\pgfqpoint{2.658275in}{2.035360in}}{\pgfqpoint{2.658275in}{2.025272in}}%
\pgfpathcurveto{\pgfqpoint{2.658275in}{2.015185in}}{\pgfqpoint{2.662282in}{2.005509in}}{\pgfqpoint{2.669415in}{1.998377in}}%
\pgfpathcurveto{\pgfqpoint{2.676548in}{1.991244in}}{\pgfqpoint{2.686224in}{1.987236in}}{\pgfqpoint{2.696311in}{1.987236in}}%
\pgfpathclose%
\pgfusepath{stroke,fill}%
\end{pgfscope}%
\begin{pgfscope}%
\pgfpathrectangle{\pgfqpoint{1.280114in}{0.528000in}}{\pgfqpoint{3.487886in}{3.696000in}} %
\pgfusepath{clip}%
\pgfsetbuttcap%
\pgfsetroundjoin%
\definecolor{currentfill}{rgb}{1.000000,0.498039,0.054902}%
\pgfsetfillcolor{currentfill}%
\pgfsetlinewidth{1.003750pt}%
\definecolor{currentstroke}{rgb}{1.000000,0.498039,0.054902}%
\pgfsetstrokecolor{currentstroke}%
\pgfsetdash{}{0pt}%
\pgfpathmoveto{\pgfqpoint{2.674387in}{2.162392in}}%
\pgfpathcurveto{\pgfqpoint{2.684474in}{2.162392in}}{\pgfqpoint{2.694150in}{2.166400in}}{\pgfqpoint{2.701283in}{2.173533in}}%
\pgfpathcurveto{\pgfqpoint{2.708415in}{2.180666in}}{\pgfqpoint{2.712423in}{2.190341in}}{\pgfqpoint{2.712423in}{2.200429in}}%
\pgfpathcurveto{\pgfqpoint{2.712423in}{2.210516in}}{\pgfqpoint{2.708415in}{2.220192in}}{\pgfqpoint{2.701283in}{2.227324in}}%
\pgfpathcurveto{\pgfqpoint{2.694150in}{2.234457in}}{\pgfqpoint{2.684474in}{2.238465in}}{\pgfqpoint{2.674387in}{2.238465in}}%
\pgfpathcurveto{\pgfqpoint{2.664300in}{2.238465in}}{\pgfqpoint{2.654624in}{2.234457in}}{\pgfqpoint{2.647491in}{2.227324in}}%
\pgfpathcurveto{\pgfqpoint{2.640358in}{2.220192in}}{\pgfqpoint{2.636351in}{2.210516in}}{\pgfqpoint{2.636351in}{2.200429in}}%
\pgfpathcurveto{\pgfqpoint{2.636351in}{2.190341in}}{\pgfqpoint{2.640358in}{2.180666in}}{\pgfqpoint{2.647491in}{2.173533in}}%
\pgfpathcurveto{\pgfqpoint{2.654624in}{2.166400in}}{\pgfqpoint{2.664300in}{2.162392in}}{\pgfqpoint{2.674387in}{2.162392in}}%
\pgfpathclose%
\pgfusepath{stroke,fill}%
\end{pgfscope}%
\begin{pgfscope}%
\pgfpathrectangle{\pgfqpoint{1.280114in}{0.528000in}}{\pgfqpoint{3.487886in}{3.696000in}} %
\pgfusepath{clip}%
\pgfsetbuttcap%
\pgfsetroundjoin%
\definecolor{currentfill}{rgb}{1.000000,0.498039,0.054902}%
\pgfsetfillcolor{currentfill}%
\pgfsetlinewidth{1.003750pt}%
\definecolor{currentstroke}{rgb}{1.000000,0.498039,0.054902}%
\pgfsetstrokecolor{currentstroke}%
\pgfsetdash{}{0pt}%
\pgfpathmoveto{\pgfqpoint{3.163477in}{0.668276in}}%
\pgfpathcurveto{\pgfqpoint{3.173565in}{0.668276in}}{\pgfqpoint{3.183240in}{0.672284in}}{\pgfqpoint{3.190373in}{0.679417in}}%
\pgfpathcurveto{\pgfqpoint{3.197506in}{0.686550in}}{\pgfqpoint{3.201514in}{0.696225in}}{\pgfqpoint{3.201514in}{0.706312in}}%
\pgfpathcurveto{\pgfqpoint{3.201514in}{0.716400in}}{\pgfqpoint{3.197506in}{0.726075in}}{\pgfqpoint{3.190373in}{0.733208in}}%
\pgfpathcurveto{\pgfqpoint{3.183240in}{0.740341in}}{\pgfqpoint{3.173565in}{0.744349in}}{\pgfqpoint{3.163477in}{0.744349in}}%
\pgfpathcurveto{\pgfqpoint{3.153390in}{0.744349in}}{\pgfqpoint{3.143714in}{0.740341in}}{\pgfqpoint{3.136582in}{0.733208in}}%
\pgfpathcurveto{\pgfqpoint{3.129449in}{0.726075in}}{\pgfqpoint{3.125441in}{0.716400in}}{\pgfqpoint{3.125441in}{0.706312in}}%
\pgfpathcurveto{\pgfqpoint{3.125441in}{0.696225in}}{\pgfqpoint{3.129449in}{0.686550in}}{\pgfqpoint{3.136582in}{0.679417in}}%
\pgfpathcurveto{\pgfqpoint{3.143714in}{0.672284in}}{\pgfqpoint{3.153390in}{0.668276in}}{\pgfqpoint{3.163477in}{0.668276in}}%
\pgfpathclose%
\pgfusepath{stroke,fill}%
\end{pgfscope}%
\begin{pgfscope}%
\pgfpathrectangle{\pgfqpoint{1.280114in}{0.528000in}}{\pgfqpoint{3.487886in}{3.696000in}} %
\pgfusepath{clip}%
\pgfsetbuttcap%
\pgfsetroundjoin%
\definecolor{currentfill}{rgb}{1.000000,0.498039,0.054902}%
\pgfsetfillcolor{currentfill}%
\pgfsetlinewidth{1.003750pt}%
\definecolor{currentstroke}{rgb}{1.000000,0.498039,0.054902}%
\pgfsetstrokecolor{currentstroke}%
\pgfsetdash{}{0pt}%
\pgfpathmoveto{\pgfqpoint{2.676874in}{1.773985in}}%
\pgfpathcurveto{\pgfqpoint{2.686961in}{1.773985in}}{\pgfqpoint{2.696637in}{1.777993in}}{\pgfqpoint{2.703769in}{1.785126in}}%
\pgfpathcurveto{\pgfqpoint{2.710902in}{1.792259in}}{\pgfqpoint{2.714910in}{1.801934in}}{\pgfqpoint{2.714910in}{1.812022in}}%
\pgfpathcurveto{\pgfqpoint{2.714910in}{1.822109in}}{\pgfqpoint{2.710902in}{1.831784in}}{\pgfqpoint{2.703769in}{1.838917in}}%
\pgfpathcurveto{\pgfqpoint{2.696637in}{1.846050in}}{\pgfqpoint{2.686961in}{1.850058in}}{\pgfqpoint{2.676874in}{1.850058in}}%
\pgfpathcurveto{\pgfqpoint{2.666786in}{1.850058in}}{\pgfqpoint{2.657111in}{1.846050in}}{\pgfqpoint{2.649978in}{1.838917in}}%
\pgfpathcurveto{\pgfqpoint{2.642845in}{1.831784in}}{\pgfqpoint{2.638837in}{1.822109in}}{\pgfqpoint{2.638837in}{1.812022in}}%
\pgfpathcurveto{\pgfqpoint{2.638837in}{1.801934in}}{\pgfqpoint{2.642845in}{1.792259in}}{\pgfqpoint{2.649978in}{1.785126in}}%
\pgfpathcurveto{\pgfqpoint{2.657111in}{1.777993in}}{\pgfqpoint{2.666786in}{1.773985in}}{\pgfqpoint{2.676874in}{1.773985in}}%
\pgfpathclose%
\pgfusepath{stroke,fill}%
\end{pgfscope}%
\begin{pgfscope}%
\pgfpathrectangle{\pgfqpoint{1.280114in}{0.528000in}}{\pgfqpoint{3.487886in}{3.696000in}} %
\pgfusepath{clip}%
\pgfsetbuttcap%
\pgfsetroundjoin%
\definecolor{currentfill}{rgb}{1.000000,0.498039,0.054902}%
\pgfsetfillcolor{currentfill}%
\pgfsetlinewidth{1.003750pt}%
\definecolor{currentstroke}{rgb}{1.000000,0.498039,0.054902}%
\pgfsetstrokecolor{currentstroke}%
\pgfsetdash{}{0pt}%
\pgfpathmoveto{\pgfqpoint{3.002284in}{0.923647in}}%
\pgfpathcurveto{\pgfqpoint{3.012371in}{0.923647in}}{\pgfqpoint{3.022047in}{0.927655in}}{\pgfqpoint{3.029180in}{0.934788in}}%
\pgfpathcurveto{\pgfqpoint{3.036313in}{0.941921in}}{\pgfqpoint{3.040320in}{0.951596in}}{\pgfqpoint{3.040320in}{0.961684in}}%
\pgfpathcurveto{\pgfqpoint{3.040320in}{0.971771in}}{\pgfqpoint{3.036313in}{0.981447in}}{\pgfqpoint{3.029180in}{0.988579in}}%
\pgfpathcurveto{\pgfqpoint{3.022047in}{0.995712in}}{\pgfqpoint{3.012371in}{0.999720in}}{\pgfqpoint{3.002284in}{0.999720in}}%
\pgfpathcurveto{\pgfqpoint{2.992197in}{0.999720in}}{\pgfqpoint{2.982521in}{0.995712in}}{\pgfqpoint{2.975388in}{0.988579in}}%
\pgfpathcurveto{\pgfqpoint{2.968255in}{0.981447in}}{\pgfqpoint{2.964248in}{0.971771in}}{\pgfqpoint{2.964248in}{0.961684in}}%
\pgfpathcurveto{\pgfqpoint{2.964248in}{0.951596in}}{\pgfqpoint{2.968255in}{0.941921in}}{\pgfqpoint{2.975388in}{0.934788in}}%
\pgfpathcurveto{\pgfqpoint{2.982521in}{0.927655in}}{\pgfqpoint{2.992197in}{0.923647in}}{\pgfqpoint{3.002284in}{0.923647in}}%
\pgfpathclose%
\pgfusepath{stroke,fill}%
\end{pgfscope}%
\begin{pgfscope}%
\pgfpathrectangle{\pgfqpoint{1.280114in}{0.528000in}}{\pgfqpoint{3.487886in}{3.696000in}} %
\pgfusepath{clip}%
\pgfsetbuttcap%
\pgfsetroundjoin%
\definecolor{currentfill}{rgb}{1.000000,0.498039,0.054902}%
\pgfsetfillcolor{currentfill}%
\pgfsetlinewidth{1.003750pt}%
\definecolor{currentstroke}{rgb}{1.000000,0.498039,0.054902}%
\pgfsetstrokecolor{currentstroke}%
\pgfsetdash{}{0pt}%
\pgfpathmoveto{\pgfqpoint{2.666091in}{2.190907in}}%
\pgfpathcurveto{\pgfqpoint{2.676178in}{2.190907in}}{\pgfqpoint{2.685854in}{2.194915in}}{\pgfqpoint{2.692986in}{2.202048in}}%
\pgfpathcurveto{\pgfqpoint{2.700119in}{2.209181in}}{\pgfqpoint{2.704127in}{2.218856in}}{\pgfqpoint{2.704127in}{2.228944in}}%
\pgfpathcurveto{\pgfqpoint{2.704127in}{2.239031in}}{\pgfqpoint{2.700119in}{2.248707in}}{\pgfqpoint{2.692986in}{2.255839in}}%
\pgfpathcurveto{\pgfqpoint{2.685854in}{2.262972in}}{\pgfqpoint{2.676178in}{2.266980in}}{\pgfqpoint{2.666091in}{2.266980in}}%
\pgfpathcurveto{\pgfqpoint{2.656003in}{2.266980in}}{\pgfqpoint{2.646328in}{2.262972in}}{\pgfqpoint{2.639195in}{2.255839in}}%
\pgfpathcurveto{\pgfqpoint{2.632062in}{2.248707in}}{\pgfqpoint{2.628054in}{2.239031in}}{\pgfqpoint{2.628054in}{2.228944in}}%
\pgfpathcurveto{\pgfqpoint{2.628054in}{2.218856in}}{\pgfqpoint{2.632062in}{2.209181in}}{\pgfqpoint{2.639195in}{2.202048in}}%
\pgfpathcurveto{\pgfqpoint{2.646328in}{2.194915in}}{\pgfqpoint{2.656003in}{2.190907in}}{\pgfqpoint{2.666091in}{2.190907in}}%
\pgfpathclose%
\pgfusepath{stroke,fill}%
\end{pgfscope}%
\begin{pgfscope}%
\pgfpathrectangle{\pgfqpoint{1.280114in}{0.528000in}}{\pgfqpoint{3.487886in}{3.696000in}} %
\pgfusepath{clip}%
\pgfsetbuttcap%
\pgfsetroundjoin%
\definecolor{currentfill}{rgb}{1.000000,0.498039,0.054902}%
\pgfsetfillcolor{currentfill}%
\pgfsetlinewidth{1.003750pt}%
\definecolor{currentstroke}{rgb}{1.000000,0.498039,0.054902}%
\pgfsetstrokecolor{currentstroke}%
\pgfsetdash{}{0pt}%
\pgfpathmoveto{\pgfqpoint{3.086445in}{0.844871in}}%
\pgfpathcurveto{\pgfqpoint{3.096533in}{0.844871in}}{\pgfqpoint{3.106208in}{0.848879in}}{\pgfqpoint{3.113341in}{0.856011in}}%
\pgfpathcurveto{\pgfqpoint{3.120474in}{0.863144in}}{\pgfqpoint{3.124482in}{0.872820in}}{\pgfqpoint{3.124482in}{0.882907in}}%
\pgfpathcurveto{\pgfqpoint{3.124482in}{0.892995in}}{\pgfqpoint{3.120474in}{0.902670in}}{\pgfqpoint{3.113341in}{0.909803in}}%
\pgfpathcurveto{\pgfqpoint{3.106208in}{0.916936in}}{\pgfqpoint{3.096533in}{0.920943in}}{\pgfqpoint{3.086445in}{0.920943in}}%
\pgfpathcurveto{\pgfqpoint{3.076358in}{0.920943in}}{\pgfqpoint{3.066682in}{0.916936in}}{\pgfqpoint{3.059549in}{0.909803in}}%
\pgfpathcurveto{\pgfqpoint{3.052417in}{0.902670in}}{\pgfqpoint{3.048409in}{0.892995in}}{\pgfqpoint{3.048409in}{0.882907in}}%
\pgfpathcurveto{\pgfqpoint{3.048409in}{0.872820in}}{\pgfqpoint{3.052417in}{0.863144in}}{\pgfqpoint{3.059549in}{0.856011in}}%
\pgfpathcurveto{\pgfqpoint{3.066682in}{0.848879in}}{\pgfqpoint{3.076358in}{0.844871in}}{\pgfqpoint{3.086445in}{0.844871in}}%
\pgfpathclose%
\pgfusepath{stroke,fill}%
\end{pgfscope}%
\begin{pgfscope}%
\pgfpathrectangle{\pgfqpoint{1.280114in}{0.528000in}}{\pgfqpoint{3.487886in}{3.696000in}} %
\pgfusepath{clip}%
\pgfsetbuttcap%
\pgfsetroundjoin%
\definecolor{currentfill}{rgb}{1.000000,0.498039,0.054902}%
\pgfsetfillcolor{currentfill}%
\pgfsetlinewidth{1.003750pt}%
\definecolor{currentstroke}{rgb}{1.000000,0.498039,0.054902}%
\pgfsetstrokecolor{currentstroke}%
\pgfsetdash{}{0pt}%
\pgfpathmoveto{\pgfqpoint{2.715448in}{1.832750in}}%
\pgfpathcurveto{\pgfqpoint{2.725535in}{1.832750in}}{\pgfqpoint{2.735211in}{1.836758in}}{\pgfqpoint{2.742344in}{1.843890in}}%
\pgfpathcurveto{\pgfqpoint{2.749476in}{1.851023in}}{\pgfqpoint{2.753484in}{1.860699in}}{\pgfqpoint{2.753484in}{1.870786in}}%
\pgfpathcurveto{\pgfqpoint{2.753484in}{1.880874in}}{\pgfqpoint{2.749476in}{1.890549in}}{\pgfqpoint{2.742344in}{1.897682in}}%
\pgfpathcurveto{\pgfqpoint{2.735211in}{1.904815in}}{\pgfqpoint{2.725535in}{1.908822in}}{\pgfqpoint{2.715448in}{1.908822in}}%
\pgfpathcurveto{\pgfqpoint{2.705360in}{1.908822in}}{\pgfqpoint{2.695685in}{1.904815in}}{\pgfqpoint{2.688552in}{1.897682in}}%
\pgfpathcurveto{\pgfqpoint{2.681419in}{1.890549in}}{\pgfqpoint{2.677412in}{1.880874in}}{\pgfqpoint{2.677412in}{1.870786in}}%
\pgfpathcurveto{\pgfqpoint{2.677412in}{1.860699in}}{\pgfqpoint{2.681419in}{1.851023in}}{\pgfqpoint{2.688552in}{1.843890in}}%
\pgfpathcurveto{\pgfqpoint{2.695685in}{1.836758in}}{\pgfqpoint{2.705360in}{1.832750in}}{\pgfqpoint{2.715448in}{1.832750in}}%
\pgfpathclose%
\pgfusepath{stroke,fill}%
\end{pgfscope}%
\begin{pgfscope}%
\pgfpathrectangle{\pgfqpoint{1.280114in}{0.528000in}}{\pgfqpoint{3.487886in}{3.696000in}} %
\pgfusepath{clip}%
\pgfsetbuttcap%
\pgfsetroundjoin%
\definecolor{currentfill}{rgb}{1.000000,0.498039,0.054902}%
\pgfsetfillcolor{currentfill}%
\pgfsetlinewidth{1.003750pt}%
\definecolor{currentstroke}{rgb}{1.000000,0.498039,0.054902}%
\pgfsetstrokecolor{currentstroke}%
\pgfsetdash{}{0pt}%
\pgfpathmoveto{\pgfqpoint{2.659061in}{2.639437in}}%
\pgfpathcurveto{\pgfqpoint{2.669149in}{2.639437in}}{\pgfqpoint{2.678824in}{2.643445in}}{\pgfqpoint{2.685957in}{2.650578in}}%
\pgfpathcurveto{\pgfqpoint{2.693090in}{2.657711in}}{\pgfqpoint{2.697098in}{2.667386in}}{\pgfqpoint{2.697098in}{2.677474in}}%
\pgfpathcurveto{\pgfqpoint{2.697098in}{2.687561in}}{\pgfqpoint{2.693090in}{2.697237in}}{\pgfqpoint{2.685957in}{2.704369in}}%
\pgfpathcurveto{\pgfqpoint{2.678824in}{2.711502in}}{\pgfqpoint{2.669149in}{2.715510in}}{\pgfqpoint{2.659061in}{2.715510in}}%
\pgfpathcurveto{\pgfqpoint{2.648974in}{2.715510in}}{\pgfqpoint{2.639299in}{2.711502in}}{\pgfqpoint{2.632166in}{2.704369in}}%
\pgfpathcurveto{\pgfqpoint{2.625033in}{2.697237in}}{\pgfqpoint{2.621025in}{2.687561in}}{\pgfqpoint{2.621025in}{2.677474in}}%
\pgfpathcurveto{\pgfqpoint{2.621025in}{2.667386in}}{\pgfqpoint{2.625033in}{2.657711in}}{\pgfqpoint{2.632166in}{2.650578in}}%
\pgfpathcurveto{\pgfqpoint{2.639299in}{2.643445in}}{\pgfqpoint{2.648974in}{2.639437in}}{\pgfqpoint{2.659061in}{2.639437in}}%
\pgfpathclose%
\pgfusepath{stroke,fill}%
\end{pgfscope}%
\begin{pgfscope}%
\pgfpathrectangle{\pgfqpoint{1.280114in}{0.528000in}}{\pgfqpoint{3.487886in}{3.696000in}} %
\pgfusepath{clip}%
\pgfsetbuttcap%
\pgfsetroundjoin%
\definecolor{currentfill}{rgb}{1.000000,0.498039,0.054902}%
\pgfsetfillcolor{currentfill}%
\pgfsetlinewidth{1.003750pt}%
\definecolor{currentstroke}{rgb}{1.000000,0.498039,0.054902}%
\pgfsetstrokecolor{currentstroke}%
\pgfsetdash{}{0pt}%
\pgfpathmoveto{\pgfqpoint{2.833090in}{3.797557in}}%
\pgfpathcurveto{\pgfqpoint{2.843177in}{3.797557in}}{\pgfqpoint{2.852853in}{3.801564in}}{\pgfqpoint{2.859986in}{3.808697in}}%
\pgfpathcurveto{\pgfqpoint{2.867119in}{3.815830in}}{\pgfqpoint{2.871126in}{3.825505in}}{\pgfqpoint{2.871126in}{3.835593in}}%
\pgfpathcurveto{\pgfqpoint{2.871126in}{3.845680in}}{\pgfqpoint{2.867119in}{3.855356in}}{\pgfqpoint{2.859986in}{3.862489in}}%
\pgfpathcurveto{\pgfqpoint{2.852853in}{3.869621in}}{\pgfqpoint{2.843177in}{3.873629in}}{\pgfqpoint{2.833090in}{3.873629in}}%
\pgfpathcurveto{\pgfqpoint{2.823003in}{3.873629in}}{\pgfqpoint{2.813327in}{3.869621in}}{\pgfqpoint{2.806194in}{3.862489in}}%
\pgfpathcurveto{\pgfqpoint{2.799061in}{3.855356in}}{\pgfqpoint{2.795054in}{3.845680in}}{\pgfqpoint{2.795054in}{3.835593in}}%
\pgfpathcurveto{\pgfqpoint{2.795054in}{3.825505in}}{\pgfqpoint{2.799061in}{3.815830in}}{\pgfqpoint{2.806194in}{3.808697in}}%
\pgfpathcurveto{\pgfqpoint{2.813327in}{3.801564in}}{\pgfqpoint{2.823003in}{3.797557in}}{\pgfqpoint{2.833090in}{3.797557in}}%
\pgfpathclose%
\pgfusepath{stroke,fill}%
\end{pgfscope}%
\begin{pgfscope}%
\pgfpathrectangle{\pgfqpoint{1.280114in}{0.528000in}}{\pgfqpoint{3.487886in}{3.696000in}} %
\pgfusepath{clip}%
\pgfsetbuttcap%
\pgfsetroundjoin%
\definecolor{currentfill}{rgb}{1.000000,0.498039,0.054902}%
\pgfsetfillcolor{currentfill}%
\pgfsetlinewidth{1.003750pt}%
\definecolor{currentstroke}{rgb}{1.000000,0.498039,0.054902}%
\pgfsetstrokecolor{currentstroke}%
\pgfsetdash{}{0pt}%
\pgfpathmoveto{\pgfqpoint{3.369075in}{3.041500in}}%
\pgfpathcurveto{\pgfqpoint{3.379162in}{3.041500in}}{\pgfqpoint{3.388838in}{3.045507in}}{\pgfqpoint{3.395971in}{3.052640in}}%
\pgfpathcurveto{\pgfqpoint{3.403103in}{3.059773in}}{\pgfqpoint{3.407111in}{3.069448in}}{\pgfqpoint{3.407111in}{3.079536in}}%
\pgfpathcurveto{\pgfqpoint{3.407111in}{3.089623in}}{\pgfqpoint{3.403103in}{3.099299in}}{\pgfqpoint{3.395971in}{3.106432in}}%
\pgfpathcurveto{\pgfqpoint{3.388838in}{3.113564in}}{\pgfqpoint{3.379162in}{3.117572in}}{\pgfqpoint{3.369075in}{3.117572in}}%
\pgfpathcurveto{\pgfqpoint{3.358988in}{3.117572in}}{\pgfqpoint{3.349312in}{3.113564in}}{\pgfqpoint{3.342179in}{3.106432in}}%
\pgfpathcurveto{\pgfqpoint{3.335046in}{3.099299in}}{\pgfqpoint{3.331039in}{3.089623in}}{\pgfqpoint{3.331039in}{3.079536in}}%
\pgfpathcurveto{\pgfqpoint{3.331039in}{3.069448in}}{\pgfqpoint{3.335046in}{3.059773in}}{\pgfqpoint{3.342179in}{3.052640in}}%
\pgfpathcurveto{\pgfqpoint{3.349312in}{3.045507in}}{\pgfqpoint{3.358988in}{3.041500in}}{\pgfqpoint{3.369075in}{3.041500in}}%
\pgfpathclose%
\pgfusepath{stroke,fill}%
\end{pgfscope}%
\begin{pgfscope}%
\pgfpathrectangle{\pgfqpoint{1.280114in}{0.528000in}}{\pgfqpoint{3.487886in}{3.696000in}} %
\pgfusepath{clip}%
\pgfsetbuttcap%
\pgfsetroundjoin%
\definecolor{currentfill}{rgb}{1.000000,0.498039,0.054902}%
\pgfsetfillcolor{currentfill}%
\pgfsetlinewidth{1.003750pt}%
\definecolor{currentstroke}{rgb}{1.000000,0.498039,0.054902}%
\pgfsetstrokecolor{currentstroke}%
\pgfsetdash{}{0pt}%
\pgfpathmoveto{\pgfqpoint{2.783191in}{3.619862in}}%
\pgfpathcurveto{\pgfqpoint{2.793279in}{3.619862in}}{\pgfqpoint{2.802954in}{3.623869in}}{\pgfqpoint{2.810087in}{3.631002in}}%
\pgfpathcurveto{\pgfqpoint{2.817220in}{3.638135in}}{\pgfqpoint{2.821228in}{3.647811in}}{\pgfqpoint{2.821228in}{3.657898in}}%
\pgfpathcurveto{\pgfqpoint{2.821228in}{3.667985in}}{\pgfqpoint{2.817220in}{3.677661in}}{\pgfqpoint{2.810087in}{3.684794in}}%
\pgfpathcurveto{\pgfqpoint{2.802954in}{3.691926in}}{\pgfqpoint{2.793279in}{3.695934in}}{\pgfqpoint{2.783191in}{3.695934in}}%
\pgfpathcurveto{\pgfqpoint{2.773104in}{3.695934in}}{\pgfqpoint{2.763429in}{3.691926in}}{\pgfqpoint{2.756296in}{3.684794in}}%
\pgfpathcurveto{\pgfqpoint{2.749163in}{3.677661in}}{\pgfqpoint{2.745155in}{3.667985in}}{\pgfqpoint{2.745155in}{3.657898in}}%
\pgfpathcurveto{\pgfqpoint{2.745155in}{3.647811in}}{\pgfqpoint{2.749163in}{3.638135in}}{\pgfqpoint{2.756296in}{3.631002in}}%
\pgfpathcurveto{\pgfqpoint{2.763429in}{3.623869in}}{\pgfqpoint{2.773104in}{3.619862in}}{\pgfqpoint{2.783191in}{3.619862in}}%
\pgfpathclose%
\pgfusepath{stroke,fill}%
\end{pgfscope}%
\begin{pgfscope}%
\pgfpathrectangle{\pgfqpoint{1.280114in}{0.528000in}}{\pgfqpoint{3.487886in}{3.696000in}} %
\pgfusepath{clip}%
\pgfsetbuttcap%
\pgfsetroundjoin%
\definecolor{currentfill}{rgb}{1.000000,0.498039,0.054902}%
\pgfsetfillcolor{currentfill}%
\pgfsetlinewidth{1.003750pt}%
\definecolor{currentstroke}{rgb}{1.000000,0.498039,0.054902}%
\pgfsetstrokecolor{currentstroke}%
\pgfsetdash{}{0pt}%
\pgfpathmoveto{\pgfqpoint{3.385547in}{2.994997in}}%
\pgfpathcurveto{\pgfqpoint{3.395635in}{2.994997in}}{\pgfqpoint{3.405310in}{2.999004in}}{\pgfqpoint{3.412443in}{3.006137in}}%
\pgfpathcurveto{\pgfqpoint{3.419576in}{3.013270in}}{\pgfqpoint{3.423584in}{3.022945in}}{\pgfqpoint{3.423584in}{3.033033in}}%
\pgfpathcurveto{\pgfqpoint{3.423584in}{3.043120in}}{\pgfqpoint{3.419576in}{3.052796in}}{\pgfqpoint{3.412443in}{3.059929in}}%
\pgfpathcurveto{\pgfqpoint{3.405310in}{3.067061in}}{\pgfqpoint{3.395635in}{3.071069in}}{\pgfqpoint{3.385547in}{3.071069in}}%
\pgfpathcurveto{\pgfqpoint{3.375460in}{3.071069in}}{\pgfqpoint{3.365784in}{3.067061in}}{\pgfqpoint{3.358652in}{3.059929in}}%
\pgfpathcurveto{\pgfqpoint{3.351519in}{3.052796in}}{\pgfqpoint{3.347511in}{3.043120in}}{\pgfqpoint{3.347511in}{3.033033in}}%
\pgfpathcurveto{\pgfqpoint{3.347511in}{3.022945in}}{\pgfqpoint{3.351519in}{3.013270in}}{\pgfqpoint{3.358652in}{3.006137in}}%
\pgfpathcurveto{\pgfqpoint{3.365784in}{2.999004in}}{\pgfqpoint{3.375460in}{2.994997in}}{\pgfqpoint{3.385547in}{2.994997in}}%
\pgfpathclose%
\pgfusepath{stroke,fill}%
\end{pgfscope}%
\begin{pgfscope}%
\pgfpathrectangle{\pgfqpoint{1.280114in}{0.528000in}}{\pgfqpoint{3.487886in}{3.696000in}} %
\pgfusepath{clip}%
\pgfsetbuttcap%
\pgfsetroundjoin%
\definecolor{currentfill}{rgb}{1.000000,0.498039,0.054902}%
\pgfsetfillcolor{currentfill}%
\pgfsetlinewidth{1.003750pt}%
\definecolor{currentstroke}{rgb}{1.000000,0.498039,0.054902}%
\pgfsetstrokecolor{currentstroke}%
\pgfsetdash{}{0pt}%
\pgfpathmoveto{\pgfqpoint{3.383206in}{2.021276in}}%
\pgfpathcurveto{\pgfqpoint{3.393294in}{2.021276in}}{\pgfqpoint{3.402969in}{2.025284in}}{\pgfqpoint{3.410102in}{2.032417in}}%
\pgfpathcurveto{\pgfqpoint{3.417235in}{2.039549in}}{\pgfqpoint{3.421243in}{2.049225in}}{\pgfqpoint{3.421243in}{2.059312in}}%
\pgfpathcurveto{\pgfqpoint{3.421243in}{2.069400in}}{\pgfqpoint{3.417235in}{2.079075in}}{\pgfqpoint{3.410102in}{2.086208in}}%
\pgfpathcurveto{\pgfqpoint{3.402969in}{2.093341in}}{\pgfqpoint{3.393294in}{2.097349in}}{\pgfqpoint{3.383206in}{2.097349in}}%
\pgfpathcurveto{\pgfqpoint{3.373119in}{2.097349in}}{\pgfqpoint{3.363443in}{2.093341in}}{\pgfqpoint{3.356311in}{2.086208in}}%
\pgfpathcurveto{\pgfqpoint{3.349178in}{2.079075in}}{\pgfqpoint{3.345170in}{2.069400in}}{\pgfqpoint{3.345170in}{2.059312in}}%
\pgfpathcurveto{\pgfqpoint{3.345170in}{2.049225in}}{\pgfqpoint{3.349178in}{2.039549in}}{\pgfqpoint{3.356311in}{2.032417in}}%
\pgfpathcurveto{\pgfqpoint{3.363443in}{2.025284in}}{\pgfqpoint{3.373119in}{2.021276in}}{\pgfqpoint{3.383206in}{2.021276in}}%
\pgfpathclose%
\pgfusepath{stroke,fill}%
\end{pgfscope}%
\begin{pgfscope}%
\pgfpathrectangle{\pgfqpoint{1.280114in}{0.528000in}}{\pgfqpoint{3.487886in}{3.696000in}} %
\pgfusepath{clip}%
\pgfsetbuttcap%
\pgfsetroundjoin%
\definecolor{currentfill}{rgb}{1.000000,0.498039,0.054902}%
\pgfsetfillcolor{currentfill}%
\pgfsetlinewidth{1.003750pt}%
\definecolor{currentstroke}{rgb}{1.000000,0.498039,0.054902}%
\pgfsetstrokecolor{currentstroke}%
\pgfsetdash{}{0pt}%
\pgfpathmoveto{\pgfqpoint{3.209334in}{3.789764in}}%
\pgfpathcurveto{\pgfqpoint{3.219421in}{3.789764in}}{\pgfqpoint{3.229097in}{3.793771in}}{\pgfqpoint{3.236230in}{3.800904in}}%
\pgfpathcurveto{\pgfqpoint{3.243362in}{3.808037in}}{\pgfqpoint{3.247370in}{3.817713in}}{\pgfqpoint{3.247370in}{3.827800in}}%
\pgfpathcurveto{\pgfqpoint{3.247370in}{3.837887in}}{\pgfqpoint{3.243362in}{3.847563in}}{\pgfqpoint{3.236230in}{3.854696in}}%
\pgfpathcurveto{\pgfqpoint{3.229097in}{3.861828in}}{\pgfqpoint{3.219421in}{3.865836in}}{\pgfqpoint{3.209334in}{3.865836in}}%
\pgfpathcurveto{\pgfqpoint{3.199247in}{3.865836in}}{\pgfqpoint{3.189571in}{3.861828in}}{\pgfqpoint{3.182438in}{3.854696in}}%
\pgfpathcurveto{\pgfqpoint{3.175305in}{3.847563in}}{\pgfqpoint{3.171298in}{3.837887in}}{\pgfqpoint{3.171298in}{3.827800in}}%
\pgfpathcurveto{\pgfqpoint{3.171298in}{3.817713in}}{\pgfqpoint{3.175305in}{3.808037in}}{\pgfqpoint{3.182438in}{3.800904in}}%
\pgfpathcurveto{\pgfqpoint{3.189571in}{3.793771in}}{\pgfqpoint{3.199247in}{3.789764in}}{\pgfqpoint{3.209334in}{3.789764in}}%
\pgfpathclose%
\pgfusepath{stroke,fill}%
\end{pgfscope}%
\begin{pgfscope}%
\pgfpathrectangle{\pgfqpoint{1.280114in}{0.528000in}}{\pgfqpoint{3.487886in}{3.696000in}} %
\pgfusepath{clip}%
\pgfsetbuttcap%
\pgfsetroundjoin%
\definecolor{currentfill}{rgb}{1.000000,0.498039,0.054902}%
\pgfsetfillcolor{currentfill}%
\pgfsetlinewidth{1.003750pt}%
\definecolor{currentstroke}{rgb}{1.000000,0.498039,0.054902}%
\pgfsetstrokecolor{currentstroke}%
\pgfsetdash{}{0pt}%
\pgfpathmoveto{\pgfqpoint{3.234199in}{1.081487in}}%
\pgfpathcurveto{\pgfqpoint{3.244286in}{1.081487in}}{\pgfqpoint{3.253961in}{1.085495in}}{\pgfqpoint{3.261094in}{1.092627in}}%
\pgfpathcurveto{\pgfqpoint{3.268227in}{1.099760in}}{\pgfqpoint{3.272235in}{1.109436in}}{\pgfqpoint{3.272235in}{1.119523in}}%
\pgfpathcurveto{\pgfqpoint{3.272235in}{1.129610in}}{\pgfqpoint{3.268227in}{1.139286in}}{\pgfqpoint{3.261094in}{1.146419in}}%
\pgfpathcurveto{\pgfqpoint{3.253961in}{1.153552in}}{\pgfqpoint{3.244286in}{1.157559in}}{\pgfqpoint{3.234199in}{1.157559in}}%
\pgfpathcurveto{\pgfqpoint{3.224111in}{1.157559in}}{\pgfqpoint{3.214436in}{1.153552in}}{\pgfqpoint{3.207303in}{1.146419in}}%
\pgfpathcurveto{\pgfqpoint{3.200170in}{1.139286in}}{\pgfqpoint{3.196162in}{1.129610in}}{\pgfqpoint{3.196162in}{1.119523in}}%
\pgfpathcurveto{\pgfqpoint{3.196162in}{1.109436in}}{\pgfqpoint{3.200170in}{1.099760in}}{\pgfqpoint{3.207303in}{1.092627in}}%
\pgfpathcurveto{\pgfqpoint{3.214436in}{1.085495in}}{\pgfqpoint{3.224111in}{1.081487in}}{\pgfqpoint{3.234199in}{1.081487in}}%
\pgfpathclose%
\pgfusepath{stroke,fill}%
\end{pgfscope}%
\begin{pgfscope}%
\pgfpathrectangle{\pgfqpoint{1.280114in}{0.528000in}}{\pgfqpoint{3.487886in}{3.696000in}} %
\pgfusepath{clip}%
\pgfsetbuttcap%
\pgfsetroundjoin%
\definecolor{currentfill}{rgb}{1.000000,0.498039,0.054902}%
\pgfsetfillcolor{currentfill}%
\pgfsetlinewidth{1.003750pt}%
\definecolor{currentstroke}{rgb}{1.000000,0.498039,0.054902}%
\pgfsetstrokecolor{currentstroke}%
\pgfsetdash{}{0pt}%
\pgfpathmoveto{\pgfqpoint{3.263095in}{1.319464in}}%
\pgfpathcurveto{\pgfqpoint{3.273183in}{1.319464in}}{\pgfqpoint{3.282858in}{1.323472in}}{\pgfqpoint{3.289991in}{1.330604in}}%
\pgfpathcurveto{\pgfqpoint{3.297124in}{1.337737in}}{\pgfqpoint{3.301132in}{1.347413in}}{\pgfqpoint{3.301132in}{1.357500in}}%
\pgfpathcurveto{\pgfqpoint{3.301132in}{1.367587in}}{\pgfqpoint{3.297124in}{1.377263in}}{\pgfqpoint{3.289991in}{1.384396in}}%
\pgfpathcurveto{\pgfqpoint{3.282858in}{1.391529in}}{\pgfqpoint{3.273183in}{1.395536in}}{\pgfqpoint{3.263095in}{1.395536in}}%
\pgfpathcurveto{\pgfqpoint{3.253008in}{1.395536in}}{\pgfqpoint{3.243332in}{1.391529in}}{\pgfqpoint{3.236200in}{1.384396in}}%
\pgfpathcurveto{\pgfqpoint{3.229067in}{1.377263in}}{\pgfqpoint{3.225059in}{1.367587in}}{\pgfqpoint{3.225059in}{1.357500in}}%
\pgfpathcurveto{\pgfqpoint{3.225059in}{1.347413in}}{\pgfqpoint{3.229067in}{1.337737in}}{\pgfqpoint{3.236200in}{1.330604in}}%
\pgfpathcurveto{\pgfqpoint{3.243332in}{1.323472in}}{\pgfqpoint{3.253008in}{1.319464in}}{\pgfqpoint{3.263095in}{1.319464in}}%
\pgfpathclose%
\pgfusepath{stroke,fill}%
\end{pgfscope}%
\begin{pgfscope}%
\pgfpathrectangle{\pgfqpoint{1.280114in}{0.528000in}}{\pgfqpoint{3.487886in}{3.696000in}} %
\pgfusepath{clip}%
\pgfsetbuttcap%
\pgfsetroundjoin%
\definecolor{currentfill}{rgb}{1.000000,0.498039,0.054902}%
\pgfsetfillcolor{currentfill}%
\pgfsetlinewidth{1.003750pt}%
\definecolor{currentstroke}{rgb}{1.000000,0.498039,0.054902}%
\pgfsetstrokecolor{currentstroke}%
\pgfsetdash{}{0pt}%
\pgfpathmoveto{\pgfqpoint{2.877583in}{0.829546in}}%
\pgfpathcurveto{\pgfqpoint{2.887670in}{0.829546in}}{\pgfqpoint{2.897346in}{0.833553in}}{\pgfqpoint{2.904479in}{0.840686in}}%
\pgfpathcurveto{\pgfqpoint{2.911612in}{0.847819in}}{\pgfqpoint{2.915619in}{0.857495in}}{\pgfqpoint{2.915619in}{0.867582in}}%
\pgfpathcurveto{\pgfqpoint{2.915619in}{0.877669in}}{\pgfqpoint{2.911612in}{0.887345in}}{\pgfqpoint{2.904479in}{0.894478in}}%
\pgfpathcurveto{\pgfqpoint{2.897346in}{0.901610in}}{\pgfqpoint{2.887670in}{0.905618in}}{\pgfqpoint{2.877583in}{0.905618in}}%
\pgfpathcurveto{\pgfqpoint{2.867496in}{0.905618in}}{\pgfqpoint{2.857820in}{0.901610in}}{\pgfqpoint{2.850687in}{0.894478in}}%
\pgfpathcurveto{\pgfqpoint{2.843554in}{0.887345in}}{\pgfqpoint{2.839547in}{0.877669in}}{\pgfqpoint{2.839547in}{0.867582in}}%
\pgfpathcurveto{\pgfqpoint{2.839547in}{0.857495in}}{\pgfqpoint{2.843554in}{0.847819in}}{\pgfqpoint{2.850687in}{0.840686in}}%
\pgfpathcurveto{\pgfqpoint{2.857820in}{0.833553in}}{\pgfqpoint{2.867496in}{0.829546in}}{\pgfqpoint{2.877583in}{0.829546in}}%
\pgfpathclose%
\pgfusepath{stroke,fill}%
\end{pgfscope}%
\begin{pgfscope}%
\pgfpathrectangle{\pgfqpoint{1.280114in}{0.528000in}}{\pgfqpoint{3.487886in}{3.696000in}} %
\pgfusepath{clip}%
\pgfsetbuttcap%
\pgfsetroundjoin%
\definecolor{currentfill}{rgb}{1.000000,0.498039,0.054902}%
\pgfsetfillcolor{currentfill}%
\pgfsetlinewidth{1.003750pt}%
\definecolor{currentstroke}{rgb}{1.000000,0.498039,0.054902}%
\pgfsetstrokecolor{currentstroke}%
\pgfsetdash{}{0pt}%
\pgfpathmoveto{\pgfqpoint{2.680052in}{1.690811in}}%
\pgfpathcurveto{\pgfqpoint{2.690139in}{1.690811in}}{\pgfqpoint{2.699814in}{1.694819in}}{\pgfqpoint{2.706947in}{1.701952in}}%
\pgfpathcurveto{\pgfqpoint{2.714080in}{1.709085in}}{\pgfqpoint{2.718088in}{1.718760in}}{\pgfqpoint{2.718088in}{1.728848in}}%
\pgfpathcurveto{\pgfqpoint{2.718088in}{1.738935in}}{\pgfqpoint{2.714080in}{1.748611in}}{\pgfqpoint{2.706947in}{1.755743in}}%
\pgfpathcurveto{\pgfqpoint{2.699814in}{1.762876in}}{\pgfqpoint{2.690139in}{1.766884in}}{\pgfqpoint{2.680052in}{1.766884in}}%
\pgfpathcurveto{\pgfqpoint{2.669964in}{1.766884in}}{\pgfqpoint{2.660289in}{1.762876in}}{\pgfqpoint{2.653156in}{1.755743in}}%
\pgfpathcurveto{\pgfqpoint{2.646023in}{1.748611in}}{\pgfqpoint{2.642015in}{1.738935in}}{\pgfqpoint{2.642015in}{1.728848in}}%
\pgfpathcurveto{\pgfqpoint{2.642015in}{1.718760in}}{\pgfqpoint{2.646023in}{1.709085in}}{\pgfqpoint{2.653156in}{1.701952in}}%
\pgfpathcurveto{\pgfqpoint{2.660289in}{1.694819in}}{\pgfqpoint{2.669964in}{1.690811in}}{\pgfqpoint{2.680052in}{1.690811in}}%
\pgfpathclose%
\pgfusepath{stroke,fill}%
\end{pgfscope}%
\begin{pgfscope}%
\pgfpathrectangle{\pgfqpoint{1.280114in}{0.528000in}}{\pgfqpoint{3.487886in}{3.696000in}} %
\pgfusepath{clip}%
\pgfsetbuttcap%
\pgfsetroundjoin%
\definecolor{currentfill}{rgb}{1.000000,0.498039,0.054902}%
\pgfsetfillcolor{currentfill}%
\pgfsetlinewidth{1.003750pt}%
\definecolor{currentstroke}{rgb}{1.000000,0.498039,0.054902}%
\pgfsetstrokecolor{currentstroke}%
\pgfsetdash{}{0pt}%
\pgfpathmoveto{\pgfqpoint{3.228966in}{3.688308in}}%
\pgfpathcurveto{\pgfqpoint{3.239053in}{3.688308in}}{\pgfqpoint{3.248729in}{3.692316in}}{\pgfqpoint{3.255862in}{3.699449in}}%
\pgfpathcurveto{\pgfqpoint{3.262994in}{3.706582in}}{\pgfqpoint{3.267002in}{3.716257in}}{\pgfqpoint{3.267002in}{3.726344in}}%
\pgfpathcurveto{\pgfqpoint{3.267002in}{3.736432in}}{\pgfqpoint{3.262994in}{3.746107in}}{\pgfqpoint{3.255862in}{3.753240in}}%
\pgfpathcurveto{\pgfqpoint{3.248729in}{3.760373in}}{\pgfqpoint{3.239053in}{3.764381in}}{\pgfqpoint{3.228966in}{3.764381in}}%
\pgfpathcurveto{\pgfqpoint{3.218879in}{3.764381in}}{\pgfqpoint{3.209203in}{3.760373in}}{\pgfqpoint{3.202070in}{3.753240in}}%
\pgfpathcurveto{\pgfqpoint{3.194937in}{3.746107in}}{\pgfqpoint{3.190930in}{3.736432in}}{\pgfqpoint{3.190930in}{3.726344in}}%
\pgfpathcurveto{\pgfqpoint{3.190930in}{3.716257in}}{\pgfqpoint{3.194937in}{3.706582in}}{\pgfqpoint{3.202070in}{3.699449in}}%
\pgfpathcurveto{\pgfqpoint{3.209203in}{3.692316in}}{\pgfqpoint{3.218879in}{3.688308in}}{\pgfqpoint{3.228966in}{3.688308in}}%
\pgfpathclose%
\pgfusepath{stroke,fill}%
\end{pgfscope}%
\begin{pgfscope}%
\pgfpathrectangle{\pgfqpoint{1.280114in}{0.528000in}}{\pgfqpoint{3.487886in}{3.696000in}} %
\pgfusepath{clip}%
\pgfsetbuttcap%
\pgfsetroundjoin%
\definecolor{currentfill}{rgb}{1.000000,0.498039,0.054902}%
\pgfsetfillcolor{currentfill}%
\pgfsetlinewidth{1.003750pt}%
\definecolor{currentstroke}{rgb}{1.000000,0.498039,0.054902}%
\pgfsetstrokecolor{currentstroke}%
\pgfsetdash{}{0pt}%
\pgfpathmoveto{\pgfqpoint{2.718926in}{2.652787in}}%
\pgfpathcurveto{\pgfqpoint{2.729013in}{2.652787in}}{\pgfqpoint{2.738688in}{2.656795in}}{\pgfqpoint{2.745821in}{2.663927in}}%
\pgfpathcurveto{\pgfqpoint{2.752954in}{2.671060in}}{\pgfqpoint{2.756962in}{2.680736in}}{\pgfqpoint{2.756962in}{2.690823in}}%
\pgfpathcurveto{\pgfqpoint{2.756962in}{2.700910in}}{\pgfqpoint{2.752954in}{2.710586in}}{\pgfqpoint{2.745821in}{2.717719in}}%
\pgfpathcurveto{\pgfqpoint{2.738688in}{2.724852in}}{\pgfqpoint{2.729013in}{2.728859in}}{\pgfqpoint{2.718926in}{2.728859in}}%
\pgfpathcurveto{\pgfqpoint{2.708838in}{2.728859in}}{\pgfqpoint{2.699163in}{2.724852in}}{\pgfqpoint{2.692030in}{2.717719in}}%
\pgfpathcurveto{\pgfqpoint{2.684897in}{2.710586in}}{\pgfqpoint{2.680889in}{2.700910in}}{\pgfqpoint{2.680889in}{2.690823in}}%
\pgfpathcurveto{\pgfqpoint{2.680889in}{2.680736in}}{\pgfqpoint{2.684897in}{2.671060in}}{\pgfqpoint{2.692030in}{2.663927in}}%
\pgfpathcurveto{\pgfqpoint{2.699163in}{2.656795in}}{\pgfqpoint{2.708838in}{2.652787in}}{\pgfqpoint{2.718926in}{2.652787in}}%
\pgfpathclose%
\pgfusepath{stroke,fill}%
\end{pgfscope}%
\begin{pgfscope}%
\pgfpathrectangle{\pgfqpoint{1.280114in}{0.528000in}}{\pgfqpoint{3.487886in}{3.696000in}} %
\pgfusepath{clip}%
\pgfsetbuttcap%
\pgfsetroundjoin%
\definecolor{currentfill}{rgb}{1.000000,0.498039,0.054902}%
\pgfsetfillcolor{currentfill}%
\pgfsetlinewidth{1.003750pt}%
\definecolor{currentstroke}{rgb}{1.000000,0.498039,0.054902}%
\pgfsetstrokecolor{currentstroke}%
\pgfsetdash{}{0pt}%
\pgfpathmoveto{\pgfqpoint{2.830848in}{3.579948in}}%
\pgfpathcurveto{\pgfqpoint{2.840935in}{3.579948in}}{\pgfqpoint{2.850611in}{3.583956in}}{\pgfqpoint{2.857744in}{3.591089in}}%
\pgfpathcurveto{\pgfqpoint{2.864876in}{3.598222in}}{\pgfqpoint{2.868884in}{3.607897in}}{\pgfqpoint{2.868884in}{3.617984in}}%
\pgfpathcurveto{\pgfqpoint{2.868884in}{3.628072in}}{\pgfqpoint{2.864876in}{3.637747in}}{\pgfqpoint{2.857744in}{3.644880in}}%
\pgfpathcurveto{\pgfqpoint{2.850611in}{3.652013in}}{\pgfqpoint{2.840935in}{3.656021in}}{\pgfqpoint{2.830848in}{3.656021in}}%
\pgfpathcurveto{\pgfqpoint{2.820760in}{3.656021in}}{\pgfqpoint{2.811085in}{3.652013in}}{\pgfqpoint{2.803952in}{3.644880in}}%
\pgfpathcurveto{\pgfqpoint{2.796819in}{3.637747in}}{\pgfqpoint{2.792811in}{3.628072in}}{\pgfqpoint{2.792811in}{3.617984in}}%
\pgfpathcurveto{\pgfqpoint{2.792811in}{3.607897in}}{\pgfqpoint{2.796819in}{3.598222in}}{\pgfqpoint{2.803952in}{3.591089in}}%
\pgfpathcurveto{\pgfqpoint{2.811085in}{3.583956in}}{\pgfqpoint{2.820760in}{3.579948in}}{\pgfqpoint{2.830848in}{3.579948in}}%
\pgfpathclose%
\pgfusepath{stroke,fill}%
\end{pgfscope}%
\begin{pgfscope}%
\pgfpathrectangle{\pgfqpoint{1.280114in}{0.528000in}}{\pgfqpoint{3.487886in}{3.696000in}} %
\pgfusepath{clip}%
\pgfsetbuttcap%
\pgfsetroundjoin%
\definecolor{currentfill}{rgb}{1.000000,0.498039,0.054902}%
\pgfsetfillcolor{currentfill}%
\pgfsetlinewidth{1.003750pt}%
\definecolor{currentstroke}{rgb}{1.000000,0.498039,0.054902}%
\pgfsetstrokecolor{currentstroke}%
\pgfsetdash{}{0pt}%
\pgfpathmoveto{\pgfqpoint{3.079114in}{3.813039in}}%
\pgfpathcurveto{\pgfqpoint{3.089201in}{3.813039in}}{\pgfqpoint{3.098877in}{3.817047in}}{\pgfqpoint{3.106009in}{3.824179in}}%
\pgfpathcurveto{\pgfqpoint{3.113142in}{3.831312in}}{\pgfqpoint{3.117150in}{3.840988in}}{\pgfqpoint{3.117150in}{3.851075in}}%
\pgfpathcurveto{\pgfqpoint{3.117150in}{3.861162in}}{\pgfqpoint{3.113142in}{3.870838in}}{\pgfqpoint{3.106009in}{3.877971in}}%
\pgfpathcurveto{\pgfqpoint{3.098877in}{3.885104in}}{\pgfqpoint{3.089201in}{3.889111in}}{\pgfqpoint{3.079114in}{3.889111in}}%
\pgfpathcurveto{\pgfqpoint{3.069026in}{3.889111in}}{\pgfqpoint{3.059351in}{3.885104in}}{\pgfqpoint{3.052218in}{3.877971in}}%
\pgfpathcurveto{\pgfqpoint{3.045085in}{3.870838in}}{\pgfqpoint{3.041077in}{3.861162in}}{\pgfqpoint{3.041077in}{3.851075in}}%
\pgfpathcurveto{\pgfqpoint{3.041077in}{3.840988in}}{\pgfqpoint{3.045085in}{3.831312in}}{\pgfqpoint{3.052218in}{3.824179in}}%
\pgfpathcurveto{\pgfqpoint{3.059351in}{3.817047in}}{\pgfqpoint{3.069026in}{3.813039in}}{\pgfqpoint{3.079114in}{3.813039in}}%
\pgfpathclose%
\pgfusepath{stroke,fill}%
\end{pgfscope}%
\begin{pgfscope}%
\pgfpathrectangle{\pgfqpoint{1.280114in}{0.528000in}}{\pgfqpoint{3.487886in}{3.696000in}} %
\pgfusepath{clip}%
\pgfsetbuttcap%
\pgfsetroundjoin%
\definecolor{currentfill}{rgb}{1.000000,0.498039,0.054902}%
\pgfsetfillcolor{currentfill}%
\pgfsetlinewidth{1.003750pt}%
\definecolor{currentstroke}{rgb}{1.000000,0.498039,0.054902}%
\pgfsetstrokecolor{currentstroke}%
\pgfsetdash{}{0pt}%
\pgfpathmoveto{\pgfqpoint{2.745594in}{1.452527in}}%
\pgfpathcurveto{\pgfqpoint{2.755682in}{1.452527in}}{\pgfqpoint{2.765357in}{1.456535in}}{\pgfqpoint{2.772490in}{1.463667in}}%
\pgfpathcurveto{\pgfqpoint{2.779623in}{1.470800in}}{\pgfqpoint{2.783631in}{1.480476in}}{\pgfqpoint{2.783631in}{1.490563in}}%
\pgfpathcurveto{\pgfqpoint{2.783631in}{1.500650in}}{\pgfqpoint{2.779623in}{1.510326in}}{\pgfqpoint{2.772490in}{1.517459in}}%
\pgfpathcurveto{\pgfqpoint{2.765357in}{1.524592in}}{\pgfqpoint{2.755682in}{1.528599in}}{\pgfqpoint{2.745594in}{1.528599in}}%
\pgfpathcurveto{\pgfqpoint{2.735507in}{1.528599in}}{\pgfqpoint{2.725831in}{1.524592in}}{\pgfqpoint{2.718699in}{1.517459in}}%
\pgfpathcurveto{\pgfqpoint{2.711566in}{1.510326in}}{\pgfqpoint{2.707558in}{1.500650in}}{\pgfqpoint{2.707558in}{1.490563in}}%
\pgfpathcurveto{\pgfqpoint{2.707558in}{1.480476in}}{\pgfqpoint{2.711566in}{1.470800in}}{\pgfqpoint{2.718699in}{1.463667in}}%
\pgfpathcurveto{\pgfqpoint{2.725831in}{1.456535in}}{\pgfqpoint{2.735507in}{1.452527in}}{\pgfqpoint{2.745594in}{1.452527in}}%
\pgfpathclose%
\pgfusepath{stroke,fill}%
\end{pgfscope}%
\begin{pgfscope}%
\pgfpathrectangle{\pgfqpoint{1.280114in}{0.528000in}}{\pgfqpoint{3.487886in}{3.696000in}} %
\pgfusepath{clip}%
\pgfsetbuttcap%
\pgfsetroundjoin%
\definecolor{currentfill}{rgb}{1.000000,0.498039,0.054902}%
\pgfsetfillcolor{currentfill}%
\pgfsetlinewidth{1.003750pt}%
\definecolor{currentstroke}{rgb}{1.000000,0.498039,0.054902}%
\pgfsetstrokecolor{currentstroke}%
\pgfsetdash{}{0pt}%
\pgfpathmoveto{\pgfqpoint{2.882154in}{3.681571in}}%
\pgfpathcurveto{\pgfqpoint{2.892241in}{3.681571in}}{\pgfqpoint{2.901917in}{3.685579in}}{\pgfqpoint{2.909049in}{3.692712in}}%
\pgfpathcurveto{\pgfqpoint{2.916182in}{3.699845in}}{\pgfqpoint{2.920190in}{3.709520in}}{\pgfqpoint{2.920190in}{3.719608in}}%
\pgfpathcurveto{\pgfqpoint{2.920190in}{3.729695in}}{\pgfqpoint{2.916182in}{3.739370in}}{\pgfqpoint{2.909049in}{3.746503in}}%
\pgfpathcurveto{\pgfqpoint{2.901917in}{3.753636in}}{\pgfqpoint{2.892241in}{3.757644in}}{\pgfqpoint{2.882154in}{3.757644in}}%
\pgfpathcurveto{\pgfqpoint{2.872066in}{3.757644in}}{\pgfqpoint{2.862391in}{3.753636in}}{\pgfqpoint{2.855258in}{3.746503in}}%
\pgfpathcurveto{\pgfqpoint{2.848125in}{3.739370in}}{\pgfqpoint{2.844117in}{3.729695in}}{\pgfqpoint{2.844117in}{3.719608in}}%
\pgfpathcurveto{\pgfqpoint{2.844117in}{3.709520in}}{\pgfqpoint{2.848125in}{3.699845in}}{\pgfqpoint{2.855258in}{3.692712in}}%
\pgfpathcurveto{\pgfqpoint{2.862391in}{3.685579in}}{\pgfqpoint{2.872066in}{3.681571in}}{\pgfqpoint{2.882154in}{3.681571in}}%
\pgfpathclose%
\pgfusepath{stroke,fill}%
\end{pgfscope}%
\begin{pgfscope}%
\pgfpathrectangle{\pgfqpoint{1.280114in}{0.528000in}}{\pgfqpoint{3.487886in}{3.696000in}} %
\pgfusepath{clip}%
\pgfsetbuttcap%
\pgfsetroundjoin%
\definecolor{currentfill}{rgb}{1.000000,0.498039,0.054902}%
\pgfsetfillcolor{currentfill}%
\pgfsetlinewidth{1.003750pt}%
\definecolor{currentstroke}{rgb}{1.000000,0.498039,0.054902}%
\pgfsetstrokecolor{currentstroke}%
\pgfsetdash{}{0pt}%
\pgfpathmoveto{\pgfqpoint{3.389776in}{1.470762in}}%
\pgfpathcurveto{\pgfqpoint{3.399864in}{1.470762in}}{\pgfqpoint{3.409539in}{1.474770in}}{\pgfqpoint{3.416672in}{1.481903in}}%
\pgfpathcurveto{\pgfqpoint{3.423805in}{1.489036in}}{\pgfqpoint{3.427813in}{1.498711in}}{\pgfqpoint{3.427813in}{1.508799in}}%
\pgfpathcurveto{\pgfqpoint{3.427813in}{1.518886in}}{\pgfqpoint{3.423805in}{1.528561in}}{\pgfqpoint{3.416672in}{1.535694in}}%
\pgfpathcurveto{\pgfqpoint{3.409539in}{1.542827in}}{\pgfqpoint{3.399864in}{1.546835in}}{\pgfqpoint{3.389776in}{1.546835in}}%
\pgfpathcurveto{\pgfqpoint{3.379689in}{1.546835in}}{\pgfqpoint{3.370014in}{1.542827in}}{\pgfqpoint{3.362881in}{1.535694in}}%
\pgfpathcurveto{\pgfqpoint{3.355748in}{1.528561in}}{\pgfqpoint{3.351740in}{1.518886in}}{\pgfqpoint{3.351740in}{1.508799in}}%
\pgfpathcurveto{\pgfqpoint{3.351740in}{1.498711in}}{\pgfqpoint{3.355748in}{1.489036in}}{\pgfqpoint{3.362881in}{1.481903in}}%
\pgfpathcurveto{\pgfqpoint{3.370014in}{1.474770in}}{\pgfqpoint{3.379689in}{1.470762in}}{\pgfqpoint{3.389776in}{1.470762in}}%
\pgfpathclose%
\pgfusepath{stroke,fill}%
\end{pgfscope}%
\begin{pgfscope}%
\pgfpathrectangle{\pgfqpoint{1.280114in}{0.528000in}}{\pgfqpoint{3.487886in}{3.696000in}} %
\pgfusepath{clip}%
\pgfsetbuttcap%
\pgfsetroundjoin%
\definecolor{currentfill}{rgb}{1.000000,0.498039,0.054902}%
\pgfsetfillcolor{currentfill}%
\pgfsetlinewidth{1.003750pt}%
\definecolor{currentstroke}{rgb}{1.000000,0.498039,0.054902}%
\pgfsetstrokecolor{currentstroke}%
\pgfsetdash{}{0pt}%
\pgfpathmoveto{\pgfqpoint{3.360226in}{1.752892in}}%
\pgfpathcurveto{\pgfqpoint{3.370313in}{1.752892in}}{\pgfqpoint{3.379989in}{1.756900in}}{\pgfqpoint{3.387121in}{1.764033in}}%
\pgfpathcurveto{\pgfqpoint{3.394254in}{1.771166in}}{\pgfqpoint{3.398262in}{1.780841in}}{\pgfqpoint{3.398262in}{1.790929in}}%
\pgfpathcurveto{\pgfqpoint{3.398262in}{1.801016in}}{\pgfqpoint{3.394254in}{1.810692in}}{\pgfqpoint{3.387121in}{1.817824in}}%
\pgfpathcurveto{\pgfqpoint{3.379989in}{1.824957in}}{\pgfqpoint{3.370313in}{1.828965in}}{\pgfqpoint{3.360226in}{1.828965in}}%
\pgfpathcurveto{\pgfqpoint{3.350138in}{1.828965in}}{\pgfqpoint{3.340463in}{1.824957in}}{\pgfqpoint{3.333330in}{1.817824in}}%
\pgfpathcurveto{\pgfqpoint{3.326197in}{1.810692in}}{\pgfqpoint{3.322189in}{1.801016in}}{\pgfqpoint{3.322189in}{1.790929in}}%
\pgfpathcurveto{\pgfqpoint{3.322189in}{1.780841in}}{\pgfqpoint{3.326197in}{1.771166in}}{\pgfqpoint{3.333330in}{1.764033in}}%
\pgfpathcurveto{\pgfqpoint{3.340463in}{1.756900in}}{\pgfqpoint{3.350138in}{1.752892in}}{\pgfqpoint{3.360226in}{1.752892in}}%
\pgfpathclose%
\pgfusepath{stroke,fill}%
\end{pgfscope}%
\begin{pgfscope}%
\pgfpathrectangle{\pgfqpoint{1.280114in}{0.528000in}}{\pgfqpoint{3.487886in}{3.696000in}} %
\pgfusepath{clip}%
\pgfsetbuttcap%
\pgfsetroundjoin%
\definecolor{currentfill}{rgb}{1.000000,0.498039,0.054902}%
\pgfsetfillcolor{currentfill}%
\pgfsetlinewidth{1.003750pt}%
\definecolor{currentstroke}{rgb}{1.000000,0.498039,0.054902}%
\pgfsetstrokecolor{currentstroke}%
\pgfsetdash{}{0pt}%
\pgfpathmoveto{\pgfqpoint{3.059236in}{3.798294in}}%
\pgfpathcurveto{\pgfqpoint{3.069323in}{3.798294in}}{\pgfqpoint{3.078998in}{3.802302in}}{\pgfqpoint{3.086131in}{3.809434in}}%
\pgfpathcurveto{\pgfqpoint{3.093264in}{3.816567in}}{\pgfqpoint{3.097272in}{3.826243in}}{\pgfqpoint{3.097272in}{3.836330in}}%
\pgfpathcurveto{\pgfqpoint{3.097272in}{3.846418in}}{\pgfqpoint{3.093264in}{3.856093in}}{\pgfqpoint{3.086131in}{3.863226in}}%
\pgfpathcurveto{\pgfqpoint{3.078998in}{3.870359in}}{\pgfqpoint{3.069323in}{3.874366in}}{\pgfqpoint{3.059236in}{3.874366in}}%
\pgfpathcurveto{\pgfqpoint{3.049148in}{3.874366in}}{\pgfqpoint{3.039473in}{3.870359in}}{\pgfqpoint{3.032340in}{3.863226in}}%
\pgfpathcurveto{\pgfqpoint{3.025207in}{3.856093in}}{\pgfqpoint{3.021199in}{3.846418in}}{\pgfqpoint{3.021199in}{3.836330in}}%
\pgfpathcurveto{\pgfqpoint{3.021199in}{3.826243in}}{\pgfqpoint{3.025207in}{3.816567in}}{\pgfqpoint{3.032340in}{3.809434in}}%
\pgfpathcurveto{\pgfqpoint{3.039473in}{3.802302in}}{\pgfqpoint{3.049148in}{3.798294in}}{\pgfqpoint{3.059236in}{3.798294in}}%
\pgfpathclose%
\pgfusepath{stroke,fill}%
\end{pgfscope}%
\begin{pgfscope}%
\pgfpathrectangle{\pgfqpoint{1.280114in}{0.528000in}}{\pgfqpoint{3.487886in}{3.696000in}} %
\pgfusepath{clip}%
\pgfsetbuttcap%
\pgfsetroundjoin%
\definecolor{currentfill}{rgb}{1.000000,0.498039,0.054902}%
\pgfsetfillcolor{currentfill}%
\pgfsetlinewidth{1.003750pt}%
\definecolor{currentstroke}{rgb}{1.000000,0.498039,0.054902}%
\pgfsetstrokecolor{currentstroke}%
\pgfsetdash{}{0pt}%
\pgfpathmoveto{\pgfqpoint{3.306603in}{3.368051in}}%
\pgfpathcurveto{\pgfqpoint{3.316691in}{3.368051in}}{\pgfqpoint{3.326366in}{3.372059in}}{\pgfqpoint{3.333499in}{3.379192in}}%
\pgfpathcurveto{\pgfqpoint{3.340632in}{3.386325in}}{\pgfqpoint{3.344640in}{3.396000in}}{\pgfqpoint{3.344640in}{3.406087in}}%
\pgfpathcurveto{\pgfqpoint{3.344640in}{3.416175in}}{\pgfqpoint{3.340632in}{3.425850in}}{\pgfqpoint{3.333499in}{3.432983in}}%
\pgfpathcurveto{\pgfqpoint{3.326366in}{3.440116in}}{\pgfqpoint{3.316691in}{3.444124in}}{\pgfqpoint{3.306603in}{3.444124in}}%
\pgfpathcurveto{\pgfqpoint{3.296516in}{3.444124in}}{\pgfqpoint{3.286840in}{3.440116in}}{\pgfqpoint{3.279708in}{3.432983in}}%
\pgfpathcurveto{\pgfqpoint{3.272575in}{3.425850in}}{\pgfqpoint{3.268567in}{3.416175in}}{\pgfqpoint{3.268567in}{3.406087in}}%
\pgfpathcurveto{\pgfqpoint{3.268567in}{3.396000in}}{\pgfqpoint{3.272575in}{3.386325in}}{\pgfqpoint{3.279708in}{3.379192in}}%
\pgfpathcurveto{\pgfqpoint{3.286840in}{3.372059in}}{\pgfqpoint{3.296516in}{3.368051in}}{\pgfqpoint{3.306603in}{3.368051in}}%
\pgfpathclose%
\pgfusepath{stroke,fill}%
\end{pgfscope}%
\begin{pgfscope}%
\pgfpathrectangle{\pgfqpoint{1.280114in}{0.528000in}}{\pgfqpoint{3.487886in}{3.696000in}} %
\pgfusepath{clip}%
\pgfsetbuttcap%
\pgfsetroundjoin%
\definecolor{currentfill}{rgb}{1.000000,0.498039,0.054902}%
\pgfsetfillcolor{currentfill}%
\pgfsetlinewidth{1.003750pt}%
\definecolor{currentstroke}{rgb}{1.000000,0.498039,0.054902}%
\pgfsetstrokecolor{currentstroke}%
\pgfsetdash{}{0pt}%
\pgfpathmoveto{\pgfqpoint{3.249054in}{3.648129in}}%
\pgfpathcurveto{\pgfqpoint{3.259141in}{3.648129in}}{\pgfqpoint{3.268817in}{3.652137in}}{\pgfqpoint{3.275950in}{3.659270in}}%
\pgfpathcurveto{\pgfqpoint{3.283083in}{3.666403in}}{\pgfqpoint{3.287090in}{3.676078in}}{\pgfqpoint{3.287090in}{3.686166in}}%
\pgfpathcurveto{\pgfqpoint{3.287090in}{3.696253in}}{\pgfqpoint{3.283083in}{3.705928in}}{\pgfqpoint{3.275950in}{3.713061in}}%
\pgfpathcurveto{\pgfqpoint{3.268817in}{3.720194in}}{\pgfqpoint{3.259141in}{3.724202in}}{\pgfqpoint{3.249054in}{3.724202in}}%
\pgfpathcurveto{\pgfqpoint{3.238967in}{3.724202in}}{\pgfqpoint{3.229291in}{3.720194in}}{\pgfqpoint{3.222158in}{3.713061in}}%
\pgfpathcurveto{\pgfqpoint{3.215026in}{3.705928in}}{\pgfqpoint{3.211018in}{3.696253in}}{\pgfqpoint{3.211018in}{3.686166in}}%
\pgfpathcurveto{\pgfqpoint{3.211018in}{3.676078in}}{\pgfqpoint{3.215026in}{3.666403in}}{\pgfqpoint{3.222158in}{3.659270in}}%
\pgfpathcurveto{\pgfqpoint{3.229291in}{3.652137in}}{\pgfqpoint{3.238967in}{3.648129in}}{\pgfqpoint{3.249054in}{3.648129in}}%
\pgfpathclose%
\pgfusepath{stroke,fill}%
\end{pgfscope}%
\begin{pgfscope}%
\pgfpathrectangle{\pgfqpoint{1.280114in}{0.528000in}}{\pgfqpoint{3.487886in}{3.696000in}} %
\pgfusepath{clip}%
\pgfsetbuttcap%
\pgfsetroundjoin%
\definecolor{currentfill}{rgb}{1.000000,0.498039,0.054902}%
\pgfsetfillcolor{currentfill}%
\pgfsetlinewidth{1.003750pt}%
\definecolor{currentstroke}{rgb}{1.000000,0.498039,0.054902}%
\pgfsetstrokecolor{currentstroke}%
\pgfsetdash{}{0pt}%
\pgfpathmoveto{\pgfqpoint{2.689978in}{2.452744in}}%
\pgfpathcurveto{\pgfqpoint{2.700065in}{2.452744in}}{\pgfqpoint{2.709741in}{2.456752in}}{\pgfqpoint{2.716873in}{2.463885in}}%
\pgfpathcurveto{\pgfqpoint{2.724006in}{2.471018in}}{\pgfqpoint{2.728014in}{2.480693in}}{\pgfqpoint{2.728014in}{2.490781in}}%
\pgfpathcurveto{\pgfqpoint{2.728014in}{2.500868in}}{\pgfqpoint{2.724006in}{2.510544in}}{\pgfqpoint{2.716873in}{2.517676in}}%
\pgfpathcurveto{\pgfqpoint{2.709741in}{2.524809in}}{\pgfqpoint{2.700065in}{2.528817in}}{\pgfqpoint{2.689978in}{2.528817in}}%
\pgfpathcurveto{\pgfqpoint{2.679890in}{2.528817in}}{\pgfqpoint{2.670215in}{2.524809in}}{\pgfqpoint{2.663082in}{2.517676in}}%
\pgfpathcurveto{\pgfqpoint{2.655949in}{2.510544in}}{\pgfqpoint{2.651941in}{2.500868in}}{\pgfqpoint{2.651941in}{2.490781in}}%
\pgfpathcurveto{\pgfqpoint{2.651941in}{2.480693in}}{\pgfqpoint{2.655949in}{2.471018in}}{\pgfqpoint{2.663082in}{2.463885in}}%
\pgfpathcurveto{\pgfqpoint{2.670215in}{2.456752in}}{\pgfqpoint{2.679890in}{2.452744in}}{\pgfqpoint{2.689978in}{2.452744in}}%
\pgfpathclose%
\pgfusepath{stroke,fill}%
\end{pgfscope}%
\begin{pgfscope}%
\pgfpathrectangle{\pgfqpoint{1.280114in}{0.528000in}}{\pgfqpoint{3.487886in}{3.696000in}} %
\pgfusepath{clip}%
\pgfsetbuttcap%
\pgfsetroundjoin%
\definecolor{currentfill}{rgb}{1.000000,0.498039,0.054902}%
\pgfsetfillcolor{currentfill}%
\pgfsetlinewidth{1.003750pt}%
\definecolor{currentstroke}{rgb}{1.000000,0.498039,0.054902}%
\pgfsetstrokecolor{currentstroke}%
\pgfsetdash{}{0pt}%
\pgfpathmoveto{\pgfqpoint{3.006788in}{0.698129in}}%
\pgfpathcurveto{\pgfqpoint{3.016876in}{0.698129in}}{\pgfqpoint{3.026551in}{0.702136in}}{\pgfqpoint{3.033684in}{0.709269in}}%
\pgfpathcurveto{\pgfqpoint{3.040817in}{0.716402in}}{\pgfqpoint{3.044825in}{0.726077in}}{\pgfqpoint{3.044825in}{0.736165in}}%
\pgfpathcurveto{\pgfqpoint{3.044825in}{0.746252in}}{\pgfqpoint{3.040817in}{0.755928in}}{\pgfqpoint{3.033684in}{0.763061in}}%
\pgfpathcurveto{\pgfqpoint{3.026551in}{0.770193in}}{\pgfqpoint{3.016876in}{0.774201in}}{\pgfqpoint{3.006788in}{0.774201in}}%
\pgfpathcurveto{\pgfqpoint{2.996701in}{0.774201in}}{\pgfqpoint{2.987025in}{0.770193in}}{\pgfqpoint{2.979893in}{0.763061in}}%
\pgfpathcurveto{\pgfqpoint{2.972760in}{0.755928in}}{\pgfqpoint{2.968752in}{0.746252in}}{\pgfqpoint{2.968752in}{0.736165in}}%
\pgfpathcurveto{\pgfqpoint{2.968752in}{0.726077in}}{\pgfqpoint{2.972760in}{0.716402in}}{\pgfqpoint{2.979893in}{0.709269in}}%
\pgfpathcurveto{\pgfqpoint{2.987025in}{0.702136in}}{\pgfqpoint{2.996701in}{0.698129in}}{\pgfqpoint{3.006788in}{0.698129in}}%
\pgfpathclose%
\pgfusepath{stroke,fill}%
\end{pgfscope}%
\begin{pgfscope}%
\pgfpathrectangle{\pgfqpoint{1.280114in}{0.528000in}}{\pgfqpoint{3.487886in}{3.696000in}} %
\pgfusepath{clip}%
\pgfsetbuttcap%
\pgfsetroundjoin%
\definecolor{currentfill}{rgb}{1.000000,0.498039,0.054902}%
\pgfsetfillcolor{currentfill}%
\pgfsetlinewidth{1.003750pt}%
\definecolor{currentstroke}{rgb}{1.000000,0.498039,0.054902}%
\pgfsetstrokecolor{currentstroke}%
\pgfsetdash{}{0pt}%
\pgfpathmoveto{\pgfqpoint{3.134391in}{3.895376in}}%
\pgfpathcurveto{\pgfqpoint{3.144479in}{3.895376in}}{\pgfqpoint{3.154154in}{3.899383in}}{\pgfqpoint{3.161287in}{3.906516in}}%
\pgfpathcurveto{\pgfqpoint{3.168420in}{3.913649in}}{\pgfqpoint{3.172428in}{3.923325in}}{\pgfqpoint{3.172428in}{3.933412in}}%
\pgfpathcurveto{\pgfqpoint{3.172428in}{3.943499in}}{\pgfqpoint{3.168420in}{3.953175in}}{\pgfqpoint{3.161287in}{3.960308in}}%
\pgfpathcurveto{\pgfqpoint{3.154154in}{3.967441in}}{\pgfqpoint{3.144479in}{3.971448in}}{\pgfqpoint{3.134391in}{3.971448in}}%
\pgfpathcurveto{\pgfqpoint{3.124304in}{3.971448in}}{\pgfqpoint{3.114629in}{3.967441in}}{\pgfqpoint{3.107496in}{3.960308in}}%
\pgfpathcurveto{\pgfqpoint{3.100363in}{3.953175in}}{\pgfqpoint{3.096355in}{3.943499in}}{\pgfqpoint{3.096355in}{3.933412in}}%
\pgfpathcurveto{\pgfqpoint{3.096355in}{3.923325in}}{\pgfqpoint{3.100363in}{3.913649in}}{\pgfqpoint{3.107496in}{3.906516in}}%
\pgfpathcurveto{\pgfqpoint{3.114629in}{3.899383in}}{\pgfqpoint{3.124304in}{3.895376in}}{\pgfqpoint{3.134391in}{3.895376in}}%
\pgfpathclose%
\pgfusepath{stroke,fill}%
\end{pgfscope}%
\begin{pgfscope}%
\pgfpathrectangle{\pgfqpoint{1.280114in}{0.528000in}}{\pgfqpoint{3.487886in}{3.696000in}} %
\pgfusepath{clip}%
\pgfsetbuttcap%
\pgfsetroundjoin%
\definecolor{currentfill}{rgb}{1.000000,0.498039,0.054902}%
\pgfsetfillcolor{currentfill}%
\pgfsetlinewidth{1.003750pt}%
\definecolor{currentstroke}{rgb}{1.000000,0.498039,0.054902}%
\pgfsetstrokecolor{currentstroke}%
\pgfsetdash{}{0pt}%
\pgfpathmoveto{\pgfqpoint{3.024935in}{3.845842in}}%
\pgfpathcurveto{\pgfqpoint{3.035022in}{3.845842in}}{\pgfqpoint{3.044698in}{3.849850in}}{\pgfqpoint{3.051831in}{3.856983in}}%
\pgfpathcurveto{\pgfqpoint{3.058963in}{3.864116in}}{\pgfqpoint{3.062971in}{3.873791in}}{\pgfqpoint{3.062971in}{3.883879in}}%
\pgfpathcurveto{\pgfqpoint{3.062971in}{3.893966in}}{\pgfqpoint{3.058963in}{3.903642in}}{\pgfqpoint{3.051831in}{3.910774in}}%
\pgfpathcurveto{\pgfqpoint{3.044698in}{3.917907in}}{\pgfqpoint{3.035022in}{3.921915in}}{\pgfqpoint{3.024935in}{3.921915in}}%
\pgfpathcurveto{\pgfqpoint{3.014848in}{3.921915in}}{\pgfqpoint{3.005172in}{3.917907in}}{\pgfqpoint{2.998039in}{3.910774in}}%
\pgfpathcurveto{\pgfqpoint{2.990906in}{3.903642in}}{\pgfqpoint{2.986899in}{3.893966in}}{\pgfqpoint{2.986899in}{3.883879in}}%
\pgfpathcurveto{\pgfqpoint{2.986899in}{3.873791in}}{\pgfqpoint{2.990906in}{3.864116in}}{\pgfqpoint{2.998039in}{3.856983in}}%
\pgfpathcurveto{\pgfqpoint{3.005172in}{3.849850in}}{\pgfqpoint{3.014848in}{3.845842in}}{\pgfqpoint{3.024935in}{3.845842in}}%
\pgfpathclose%
\pgfusepath{stroke,fill}%
\end{pgfscope}%
\begin{pgfscope}%
\pgfpathrectangle{\pgfqpoint{1.280114in}{0.528000in}}{\pgfqpoint{3.487886in}{3.696000in}} %
\pgfusepath{clip}%
\pgfsetbuttcap%
\pgfsetroundjoin%
\definecolor{currentfill}{rgb}{1.000000,0.498039,0.054902}%
\pgfsetfillcolor{currentfill}%
\pgfsetlinewidth{1.003750pt}%
\definecolor{currentstroke}{rgb}{1.000000,0.498039,0.054902}%
\pgfsetstrokecolor{currentstroke}%
\pgfsetdash{}{0pt}%
\pgfpathmoveto{\pgfqpoint{3.237857in}{1.395485in}}%
\pgfpathcurveto{\pgfqpoint{3.247944in}{1.395485in}}{\pgfqpoint{3.257620in}{1.399493in}}{\pgfqpoint{3.264753in}{1.406626in}}%
\pgfpathcurveto{\pgfqpoint{3.271886in}{1.413759in}}{\pgfqpoint{3.275893in}{1.423434in}}{\pgfqpoint{3.275893in}{1.433522in}}%
\pgfpathcurveto{\pgfqpoint{3.275893in}{1.443609in}}{\pgfqpoint{3.271886in}{1.453285in}}{\pgfqpoint{3.264753in}{1.460417in}}%
\pgfpathcurveto{\pgfqpoint{3.257620in}{1.467550in}}{\pgfqpoint{3.247944in}{1.471558in}}{\pgfqpoint{3.237857in}{1.471558in}}%
\pgfpathcurveto{\pgfqpoint{3.227770in}{1.471558in}}{\pgfqpoint{3.218094in}{1.467550in}}{\pgfqpoint{3.210961in}{1.460417in}}%
\pgfpathcurveto{\pgfqpoint{3.203829in}{1.453285in}}{\pgfqpoint{3.199821in}{1.443609in}}{\pgfqpoint{3.199821in}{1.433522in}}%
\pgfpathcurveto{\pgfqpoint{3.199821in}{1.423434in}}{\pgfqpoint{3.203829in}{1.413759in}}{\pgfqpoint{3.210961in}{1.406626in}}%
\pgfpathcurveto{\pgfqpoint{3.218094in}{1.399493in}}{\pgfqpoint{3.227770in}{1.395485in}}{\pgfqpoint{3.237857in}{1.395485in}}%
\pgfpathclose%
\pgfusepath{stroke,fill}%
\end{pgfscope}%
\begin{pgfscope}%
\pgfpathrectangle{\pgfqpoint{1.280114in}{0.528000in}}{\pgfqpoint{3.487886in}{3.696000in}} %
\pgfusepath{clip}%
\pgfsetbuttcap%
\pgfsetroundjoin%
\definecolor{currentfill}{rgb}{1.000000,0.498039,0.054902}%
\pgfsetfillcolor{currentfill}%
\pgfsetlinewidth{1.003750pt}%
\definecolor{currentstroke}{rgb}{1.000000,0.498039,0.054902}%
\pgfsetstrokecolor{currentstroke}%
\pgfsetdash{}{0pt}%
\pgfpathmoveto{\pgfqpoint{2.715443in}{2.977053in}}%
\pgfpathcurveto{\pgfqpoint{2.725530in}{2.977053in}}{\pgfqpoint{2.735206in}{2.981061in}}{\pgfqpoint{2.742338in}{2.988194in}}%
\pgfpathcurveto{\pgfqpoint{2.749471in}{2.995327in}}{\pgfqpoint{2.753479in}{3.005002in}}{\pgfqpoint{2.753479in}{3.015090in}}%
\pgfpathcurveto{\pgfqpoint{2.753479in}{3.025177in}}{\pgfqpoint{2.749471in}{3.034852in}}{\pgfqpoint{2.742338in}{3.041985in}}%
\pgfpathcurveto{\pgfqpoint{2.735206in}{3.049118in}}{\pgfqpoint{2.725530in}{3.053126in}}{\pgfqpoint{2.715443in}{3.053126in}}%
\pgfpathcurveto{\pgfqpoint{2.705355in}{3.053126in}}{\pgfqpoint{2.695680in}{3.049118in}}{\pgfqpoint{2.688547in}{3.041985in}}%
\pgfpathcurveto{\pgfqpoint{2.681414in}{3.034852in}}{\pgfqpoint{2.677406in}{3.025177in}}{\pgfqpoint{2.677406in}{3.015090in}}%
\pgfpathcurveto{\pgfqpoint{2.677406in}{3.005002in}}{\pgfqpoint{2.681414in}{2.995327in}}{\pgfqpoint{2.688547in}{2.988194in}}%
\pgfpathcurveto{\pgfqpoint{2.695680in}{2.981061in}}{\pgfqpoint{2.705355in}{2.977053in}}{\pgfqpoint{2.715443in}{2.977053in}}%
\pgfpathclose%
\pgfusepath{stroke,fill}%
\end{pgfscope}%
\begin{pgfscope}%
\pgfpathrectangle{\pgfqpoint{1.280114in}{0.528000in}}{\pgfqpoint{3.487886in}{3.696000in}} %
\pgfusepath{clip}%
\pgfsetbuttcap%
\pgfsetroundjoin%
\definecolor{currentfill}{rgb}{1.000000,0.498039,0.054902}%
\pgfsetfillcolor{currentfill}%
\pgfsetlinewidth{1.003750pt}%
\definecolor{currentstroke}{rgb}{1.000000,0.498039,0.054902}%
\pgfsetstrokecolor{currentstroke}%
\pgfsetdash{}{0pt}%
\pgfpathmoveto{\pgfqpoint{2.742758in}{1.380258in}}%
\pgfpathcurveto{\pgfqpoint{2.752845in}{1.380258in}}{\pgfqpoint{2.762521in}{1.384266in}}{\pgfqpoint{2.769653in}{1.391398in}}%
\pgfpathcurveto{\pgfqpoint{2.776786in}{1.398531in}}{\pgfqpoint{2.780794in}{1.408207in}}{\pgfqpoint{2.780794in}{1.418294in}}%
\pgfpathcurveto{\pgfqpoint{2.780794in}{1.428381in}}{\pgfqpoint{2.776786in}{1.438057in}}{\pgfqpoint{2.769653in}{1.445190in}}%
\pgfpathcurveto{\pgfqpoint{2.762521in}{1.452323in}}{\pgfqpoint{2.752845in}{1.456330in}}{\pgfqpoint{2.742758in}{1.456330in}}%
\pgfpathcurveto{\pgfqpoint{2.732670in}{1.456330in}}{\pgfqpoint{2.722995in}{1.452323in}}{\pgfqpoint{2.715862in}{1.445190in}}%
\pgfpathcurveto{\pgfqpoint{2.708729in}{1.438057in}}{\pgfqpoint{2.704721in}{1.428381in}}{\pgfqpoint{2.704721in}{1.418294in}}%
\pgfpathcurveto{\pgfqpoint{2.704721in}{1.408207in}}{\pgfqpoint{2.708729in}{1.398531in}}{\pgfqpoint{2.715862in}{1.391398in}}%
\pgfpathcurveto{\pgfqpoint{2.722995in}{1.384266in}}{\pgfqpoint{2.732670in}{1.380258in}}{\pgfqpoint{2.742758in}{1.380258in}}%
\pgfpathclose%
\pgfusepath{stroke,fill}%
\end{pgfscope}%
\begin{pgfscope}%
\pgfpathrectangle{\pgfqpoint{1.280114in}{0.528000in}}{\pgfqpoint{3.487886in}{3.696000in}} %
\pgfusepath{clip}%
\pgfsetbuttcap%
\pgfsetroundjoin%
\definecolor{currentfill}{rgb}{1.000000,0.498039,0.054902}%
\pgfsetfillcolor{currentfill}%
\pgfsetlinewidth{1.003750pt}%
\definecolor{currentstroke}{rgb}{1.000000,0.498039,0.054902}%
\pgfsetstrokecolor{currentstroke}%
\pgfsetdash{}{0pt}%
\pgfpathmoveto{\pgfqpoint{3.084818in}{3.880398in}}%
\pgfpathcurveto{\pgfqpoint{3.094905in}{3.880398in}}{\pgfqpoint{3.104581in}{3.884406in}}{\pgfqpoint{3.111713in}{3.891539in}}%
\pgfpathcurveto{\pgfqpoint{3.118846in}{3.898672in}}{\pgfqpoint{3.122854in}{3.908347in}}{\pgfqpoint{3.122854in}{3.918434in}}%
\pgfpathcurveto{\pgfqpoint{3.122854in}{3.928522in}}{\pgfqpoint{3.118846in}{3.938197in}}{\pgfqpoint{3.111713in}{3.945330in}}%
\pgfpathcurveto{\pgfqpoint{3.104581in}{3.952463in}}{\pgfqpoint{3.094905in}{3.956471in}}{\pgfqpoint{3.084818in}{3.956471in}}%
\pgfpathcurveto{\pgfqpoint{3.074730in}{3.956471in}}{\pgfqpoint{3.065055in}{3.952463in}}{\pgfqpoint{3.057922in}{3.945330in}}%
\pgfpathcurveto{\pgfqpoint{3.050789in}{3.938197in}}{\pgfqpoint{3.046781in}{3.928522in}}{\pgfqpoint{3.046781in}{3.918434in}}%
\pgfpathcurveto{\pgfqpoint{3.046781in}{3.908347in}}{\pgfqpoint{3.050789in}{3.898672in}}{\pgfqpoint{3.057922in}{3.891539in}}%
\pgfpathcurveto{\pgfqpoint{3.065055in}{3.884406in}}{\pgfqpoint{3.074730in}{3.880398in}}{\pgfqpoint{3.084818in}{3.880398in}}%
\pgfpathclose%
\pgfusepath{stroke,fill}%
\end{pgfscope}%
\begin{pgfscope}%
\pgfpathrectangle{\pgfqpoint{1.280114in}{0.528000in}}{\pgfqpoint{3.487886in}{3.696000in}} %
\pgfusepath{clip}%
\pgfsetbuttcap%
\pgfsetroundjoin%
\definecolor{currentfill}{rgb}{1.000000,0.498039,0.054902}%
\pgfsetfillcolor{currentfill}%
\pgfsetlinewidth{1.003750pt}%
\definecolor{currentstroke}{rgb}{1.000000,0.498039,0.054902}%
\pgfsetstrokecolor{currentstroke}%
\pgfsetdash{}{0pt}%
\pgfpathmoveto{\pgfqpoint{3.315178in}{3.382029in}}%
\pgfpathcurveto{\pgfqpoint{3.325265in}{3.382029in}}{\pgfqpoint{3.334941in}{3.386037in}}{\pgfqpoint{3.342073in}{3.393170in}}%
\pgfpathcurveto{\pgfqpoint{3.349206in}{3.400303in}}{\pgfqpoint{3.353214in}{3.409978in}}{\pgfqpoint{3.353214in}{3.420066in}}%
\pgfpathcurveto{\pgfqpoint{3.353214in}{3.430153in}}{\pgfqpoint{3.349206in}{3.439829in}}{\pgfqpoint{3.342073in}{3.446961in}}%
\pgfpathcurveto{\pgfqpoint{3.334941in}{3.454094in}}{\pgfqpoint{3.325265in}{3.458102in}}{\pgfqpoint{3.315178in}{3.458102in}}%
\pgfpathcurveto{\pgfqpoint{3.305090in}{3.458102in}}{\pgfqpoint{3.295415in}{3.454094in}}{\pgfqpoint{3.288282in}{3.446961in}}%
\pgfpathcurveto{\pgfqpoint{3.281149in}{3.439829in}}{\pgfqpoint{3.277141in}{3.430153in}}{\pgfqpoint{3.277141in}{3.420066in}}%
\pgfpathcurveto{\pgfqpoint{3.277141in}{3.409978in}}{\pgfqpoint{3.281149in}{3.400303in}}{\pgfqpoint{3.288282in}{3.393170in}}%
\pgfpathcurveto{\pgfqpoint{3.295415in}{3.386037in}}{\pgfqpoint{3.305090in}{3.382029in}}{\pgfqpoint{3.315178in}{3.382029in}}%
\pgfpathclose%
\pgfusepath{stroke,fill}%
\end{pgfscope}%
\begin{pgfscope}%
\pgfpathrectangle{\pgfqpoint{1.280114in}{0.528000in}}{\pgfqpoint{3.487886in}{3.696000in}} %
\pgfusepath{clip}%
\pgfsetbuttcap%
\pgfsetroundjoin%
\definecolor{currentfill}{rgb}{1.000000,0.498039,0.054902}%
\pgfsetfillcolor{currentfill}%
\pgfsetlinewidth{1.003750pt}%
\definecolor{currentstroke}{rgb}{1.000000,0.498039,0.054902}%
\pgfsetstrokecolor{currentstroke}%
\pgfsetdash{}{0pt}%
\pgfpathmoveto{\pgfqpoint{2.686477in}{2.173476in}}%
\pgfpathcurveto{\pgfqpoint{2.696565in}{2.173476in}}{\pgfqpoint{2.706240in}{2.177484in}}{\pgfqpoint{2.713373in}{2.184617in}}%
\pgfpathcurveto{\pgfqpoint{2.720506in}{2.191750in}}{\pgfqpoint{2.724513in}{2.201425in}}{\pgfqpoint{2.724513in}{2.211513in}}%
\pgfpathcurveto{\pgfqpoint{2.724513in}{2.221600in}}{\pgfqpoint{2.720506in}{2.231275in}}{\pgfqpoint{2.713373in}{2.238408in}}%
\pgfpathcurveto{\pgfqpoint{2.706240in}{2.245541in}}{\pgfqpoint{2.696565in}{2.249549in}}{\pgfqpoint{2.686477in}{2.249549in}}%
\pgfpathcurveto{\pgfqpoint{2.676390in}{2.249549in}}{\pgfqpoint{2.666714in}{2.245541in}}{\pgfqpoint{2.659581in}{2.238408in}}%
\pgfpathcurveto{\pgfqpoint{2.652449in}{2.231275in}}{\pgfqpoint{2.648441in}{2.221600in}}{\pgfqpoint{2.648441in}{2.211513in}}%
\pgfpathcurveto{\pgfqpoint{2.648441in}{2.201425in}}{\pgfqpoint{2.652449in}{2.191750in}}{\pgfqpoint{2.659581in}{2.184617in}}%
\pgfpathcurveto{\pgfqpoint{2.666714in}{2.177484in}}{\pgfqpoint{2.676390in}{2.173476in}}{\pgfqpoint{2.686477in}{2.173476in}}%
\pgfpathclose%
\pgfusepath{stroke,fill}%
\end{pgfscope}%
\begin{pgfscope}%
\pgfpathrectangle{\pgfqpoint{1.280114in}{0.528000in}}{\pgfqpoint{3.487886in}{3.696000in}} %
\pgfusepath{clip}%
\pgfsetbuttcap%
\pgfsetroundjoin%
\definecolor{currentfill}{rgb}{1.000000,0.498039,0.054902}%
\pgfsetfillcolor{currentfill}%
\pgfsetlinewidth{1.003750pt}%
\definecolor{currentstroke}{rgb}{1.000000,0.498039,0.054902}%
\pgfsetstrokecolor{currentstroke}%
\pgfsetdash{}{0pt}%
\pgfpathmoveto{\pgfqpoint{2.697248in}{2.586994in}}%
\pgfpathcurveto{\pgfqpoint{2.707335in}{2.586994in}}{\pgfqpoint{2.717011in}{2.591002in}}{\pgfqpoint{2.724144in}{2.598135in}}%
\pgfpathcurveto{\pgfqpoint{2.731277in}{2.605268in}}{\pgfqpoint{2.735284in}{2.614943in}}{\pgfqpoint{2.735284in}{2.625031in}}%
\pgfpathcurveto{\pgfqpoint{2.735284in}{2.635118in}}{\pgfqpoint{2.731277in}{2.644794in}}{\pgfqpoint{2.724144in}{2.651926in}}%
\pgfpathcurveto{\pgfqpoint{2.717011in}{2.659059in}}{\pgfqpoint{2.707335in}{2.663067in}}{\pgfqpoint{2.697248in}{2.663067in}}%
\pgfpathcurveto{\pgfqpoint{2.687161in}{2.663067in}}{\pgfqpoint{2.677485in}{2.659059in}}{\pgfqpoint{2.670352in}{2.651926in}}%
\pgfpathcurveto{\pgfqpoint{2.663220in}{2.644794in}}{\pgfqpoint{2.659212in}{2.635118in}}{\pgfqpoint{2.659212in}{2.625031in}}%
\pgfpathcurveto{\pgfqpoint{2.659212in}{2.614943in}}{\pgfqpoint{2.663220in}{2.605268in}}{\pgfqpoint{2.670352in}{2.598135in}}%
\pgfpathcurveto{\pgfqpoint{2.677485in}{2.591002in}}{\pgfqpoint{2.687161in}{2.586994in}}{\pgfqpoint{2.697248in}{2.586994in}}%
\pgfpathclose%
\pgfusepath{stroke,fill}%
\end{pgfscope}%
\begin{pgfscope}%
\pgfpathrectangle{\pgfqpoint{1.280114in}{0.528000in}}{\pgfqpoint{3.487886in}{3.696000in}} %
\pgfusepath{clip}%
\pgfsetbuttcap%
\pgfsetroundjoin%
\definecolor{currentfill}{rgb}{1.000000,0.498039,0.054902}%
\pgfsetfillcolor{currentfill}%
\pgfsetlinewidth{1.003750pt}%
\definecolor{currentstroke}{rgb}{1.000000,0.498039,0.054902}%
\pgfsetstrokecolor{currentstroke}%
\pgfsetdash{}{0pt}%
\pgfpathmoveto{\pgfqpoint{3.233895in}{3.355841in}}%
\pgfpathcurveto{\pgfqpoint{3.243982in}{3.355841in}}{\pgfqpoint{3.253658in}{3.359848in}}{\pgfqpoint{3.260791in}{3.366981in}}%
\pgfpathcurveto{\pgfqpoint{3.267923in}{3.374114in}}{\pgfqpoint{3.271931in}{3.383790in}}{\pgfqpoint{3.271931in}{3.393877in}}%
\pgfpathcurveto{\pgfqpoint{3.271931in}{3.403964in}}{\pgfqpoint{3.267923in}{3.413640in}}{\pgfqpoint{3.260791in}{3.420773in}}%
\pgfpathcurveto{\pgfqpoint{3.253658in}{3.427906in}}{\pgfqpoint{3.243982in}{3.431913in}}{\pgfqpoint{3.233895in}{3.431913in}}%
\pgfpathcurveto{\pgfqpoint{3.223808in}{3.431913in}}{\pgfqpoint{3.214132in}{3.427906in}}{\pgfqpoint{3.206999in}{3.420773in}}%
\pgfpathcurveto{\pgfqpoint{3.199866in}{3.413640in}}{\pgfqpoint{3.195859in}{3.403964in}}{\pgfqpoint{3.195859in}{3.393877in}}%
\pgfpathcurveto{\pgfqpoint{3.195859in}{3.383790in}}{\pgfqpoint{3.199866in}{3.374114in}}{\pgfqpoint{3.206999in}{3.366981in}}%
\pgfpathcurveto{\pgfqpoint{3.214132in}{3.359848in}}{\pgfqpoint{3.223808in}{3.355841in}}{\pgfqpoint{3.233895in}{3.355841in}}%
\pgfpathclose%
\pgfusepath{stroke,fill}%
\end{pgfscope}%
\begin{pgfscope}%
\pgfpathrectangle{\pgfqpoint{1.280114in}{0.528000in}}{\pgfqpoint{3.487886in}{3.696000in}} %
\pgfusepath{clip}%
\pgfsetbuttcap%
\pgfsetroundjoin%
\definecolor{currentfill}{rgb}{1.000000,0.498039,0.054902}%
\pgfsetfillcolor{currentfill}%
\pgfsetlinewidth{1.003750pt}%
\definecolor{currentstroke}{rgb}{1.000000,0.498039,0.054902}%
\pgfsetstrokecolor{currentstroke}%
\pgfsetdash{}{0pt}%
\pgfpathmoveto{\pgfqpoint{2.803099in}{1.250151in}}%
\pgfpathcurveto{\pgfqpoint{2.813187in}{1.250151in}}{\pgfqpoint{2.822862in}{1.254159in}}{\pgfqpoint{2.829995in}{1.261292in}}%
\pgfpathcurveto{\pgfqpoint{2.837128in}{1.268425in}}{\pgfqpoint{2.841136in}{1.278100in}}{\pgfqpoint{2.841136in}{1.288187in}}%
\pgfpathcurveto{\pgfqpoint{2.841136in}{1.298275in}}{\pgfqpoint{2.837128in}{1.307950in}}{\pgfqpoint{2.829995in}{1.315083in}}%
\pgfpathcurveto{\pgfqpoint{2.822862in}{1.322216in}}{\pgfqpoint{2.813187in}{1.326224in}}{\pgfqpoint{2.803099in}{1.326224in}}%
\pgfpathcurveto{\pgfqpoint{2.793012in}{1.326224in}}{\pgfqpoint{2.783336in}{1.322216in}}{\pgfqpoint{2.776204in}{1.315083in}}%
\pgfpathcurveto{\pgfqpoint{2.769071in}{1.307950in}}{\pgfqpoint{2.765063in}{1.298275in}}{\pgfqpoint{2.765063in}{1.288187in}}%
\pgfpathcurveto{\pgfqpoint{2.765063in}{1.278100in}}{\pgfqpoint{2.769071in}{1.268425in}}{\pgfqpoint{2.776204in}{1.261292in}}%
\pgfpathcurveto{\pgfqpoint{2.783336in}{1.254159in}}{\pgfqpoint{2.793012in}{1.250151in}}{\pgfqpoint{2.803099in}{1.250151in}}%
\pgfpathclose%
\pgfusepath{stroke,fill}%
\end{pgfscope}%
\begin{pgfscope}%
\pgfpathrectangle{\pgfqpoint{1.280114in}{0.528000in}}{\pgfqpoint{3.487886in}{3.696000in}} %
\pgfusepath{clip}%
\pgfsetbuttcap%
\pgfsetroundjoin%
\definecolor{currentfill}{rgb}{1.000000,0.498039,0.054902}%
\pgfsetfillcolor{currentfill}%
\pgfsetlinewidth{1.003750pt}%
\definecolor{currentstroke}{rgb}{1.000000,0.498039,0.054902}%
\pgfsetstrokecolor{currentstroke}%
\pgfsetdash{}{0pt}%
\pgfpathmoveto{\pgfqpoint{2.676195in}{2.007677in}}%
\pgfpathcurveto{\pgfqpoint{2.686282in}{2.007677in}}{\pgfqpoint{2.695958in}{2.011685in}}{\pgfqpoint{2.703091in}{2.018818in}}%
\pgfpathcurveto{\pgfqpoint{2.710223in}{2.025950in}}{\pgfqpoint{2.714231in}{2.035626in}}{\pgfqpoint{2.714231in}{2.045713in}}%
\pgfpathcurveto{\pgfqpoint{2.714231in}{2.055801in}}{\pgfqpoint{2.710223in}{2.065476in}}{\pgfqpoint{2.703091in}{2.072609in}}%
\pgfpathcurveto{\pgfqpoint{2.695958in}{2.079742in}}{\pgfqpoint{2.686282in}{2.083750in}}{\pgfqpoint{2.676195in}{2.083750in}}%
\pgfpathcurveto{\pgfqpoint{2.666108in}{2.083750in}}{\pgfqpoint{2.656432in}{2.079742in}}{\pgfqpoint{2.649299in}{2.072609in}}%
\pgfpathcurveto{\pgfqpoint{2.642166in}{2.065476in}}{\pgfqpoint{2.638159in}{2.055801in}}{\pgfqpoint{2.638159in}{2.045713in}}%
\pgfpathcurveto{\pgfqpoint{2.638159in}{2.035626in}}{\pgfqpoint{2.642166in}{2.025950in}}{\pgfqpoint{2.649299in}{2.018818in}}%
\pgfpathcurveto{\pgfqpoint{2.656432in}{2.011685in}}{\pgfqpoint{2.666108in}{2.007677in}}{\pgfqpoint{2.676195in}{2.007677in}}%
\pgfpathclose%
\pgfusepath{stroke,fill}%
\end{pgfscope}%
\begin{pgfscope}%
\pgfpathrectangle{\pgfqpoint{1.280114in}{0.528000in}}{\pgfqpoint{3.487886in}{3.696000in}} %
\pgfusepath{clip}%
\pgfsetbuttcap%
\pgfsetroundjoin%
\definecolor{currentfill}{rgb}{1.000000,0.498039,0.054902}%
\pgfsetfillcolor{currentfill}%
\pgfsetlinewidth{1.003750pt}%
\definecolor{currentstroke}{rgb}{1.000000,0.498039,0.054902}%
\pgfsetstrokecolor{currentstroke}%
\pgfsetdash{}{0pt}%
\pgfpathmoveto{\pgfqpoint{2.859480in}{1.258363in}}%
\pgfpathcurveto{\pgfqpoint{2.869567in}{1.258363in}}{\pgfqpoint{2.879243in}{1.262370in}}{\pgfqpoint{2.886376in}{1.269503in}}%
\pgfpathcurveto{\pgfqpoint{2.893509in}{1.276636in}}{\pgfqpoint{2.897516in}{1.286312in}}{\pgfqpoint{2.897516in}{1.296399in}}%
\pgfpathcurveto{\pgfqpoint{2.897516in}{1.306486in}}{\pgfqpoint{2.893509in}{1.316162in}}{\pgfqpoint{2.886376in}{1.323295in}}%
\pgfpathcurveto{\pgfqpoint{2.879243in}{1.330428in}}{\pgfqpoint{2.869567in}{1.334435in}}{\pgfqpoint{2.859480in}{1.334435in}}%
\pgfpathcurveto{\pgfqpoint{2.849393in}{1.334435in}}{\pgfqpoint{2.839717in}{1.330428in}}{\pgfqpoint{2.832584in}{1.323295in}}%
\pgfpathcurveto{\pgfqpoint{2.825451in}{1.316162in}}{\pgfqpoint{2.821444in}{1.306486in}}{\pgfqpoint{2.821444in}{1.296399in}}%
\pgfpathcurveto{\pgfqpoint{2.821444in}{1.286312in}}{\pgfqpoint{2.825451in}{1.276636in}}{\pgfqpoint{2.832584in}{1.269503in}}%
\pgfpathcurveto{\pgfqpoint{2.839717in}{1.262370in}}{\pgfqpoint{2.849393in}{1.258363in}}{\pgfqpoint{2.859480in}{1.258363in}}%
\pgfpathclose%
\pgfusepath{stroke,fill}%
\end{pgfscope}%
\begin{pgfscope}%
\pgfpathrectangle{\pgfqpoint{1.280114in}{0.528000in}}{\pgfqpoint{3.487886in}{3.696000in}} %
\pgfusepath{clip}%
\pgfsetbuttcap%
\pgfsetroundjoin%
\definecolor{currentfill}{rgb}{1.000000,0.498039,0.054902}%
\pgfsetfillcolor{currentfill}%
\pgfsetlinewidth{1.003750pt}%
\definecolor{currentstroke}{rgb}{1.000000,0.498039,0.054902}%
\pgfsetstrokecolor{currentstroke}%
\pgfsetdash{}{0pt}%
\pgfpathmoveto{\pgfqpoint{3.279137in}{3.506049in}}%
\pgfpathcurveto{\pgfqpoint{3.289225in}{3.506049in}}{\pgfqpoint{3.298900in}{3.510057in}}{\pgfqpoint{3.306033in}{3.517189in}}%
\pgfpathcurveto{\pgfqpoint{3.313166in}{3.524322in}}{\pgfqpoint{3.317173in}{3.533998in}}{\pgfqpoint{3.317173in}{3.544085in}}%
\pgfpathcurveto{\pgfqpoint{3.317173in}{3.554172in}}{\pgfqpoint{3.313166in}{3.563848in}}{\pgfqpoint{3.306033in}{3.570981in}}%
\pgfpathcurveto{\pgfqpoint{3.298900in}{3.578114in}}{\pgfqpoint{3.289225in}{3.582121in}}{\pgfqpoint{3.279137in}{3.582121in}}%
\pgfpathcurveto{\pgfqpoint{3.269050in}{3.582121in}}{\pgfqpoint{3.259374in}{3.578114in}}{\pgfqpoint{3.252241in}{3.570981in}}%
\pgfpathcurveto{\pgfqpoint{3.245109in}{3.563848in}}{\pgfqpoint{3.241101in}{3.554172in}}{\pgfqpoint{3.241101in}{3.544085in}}%
\pgfpathcurveto{\pgfqpoint{3.241101in}{3.533998in}}{\pgfqpoint{3.245109in}{3.524322in}}{\pgfqpoint{3.252241in}{3.517189in}}%
\pgfpathcurveto{\pgfqpoint{3.259374in}{3.510057in}}{\pgfqpoint{3.269050in}{3.506049in}}{\pgfqpoint{3.279137in}{3.506049in}}%
\pgfpathclose%
\pgfusepath{stroke,fill}%
\end{pgfscope}%
\begin{pgfscope}%
\pgfpathrectangle{\pgfqpoint{1.280114in}{0.528000in}}{\pgfqpoint{3.487886in}{3.696000in}} %
\pgfusepath{clip}%
\pgfsetbuttcap%
\pgfsetroundjoin%
\definecolor{currentfill}{rgb}{1.000000,0.498039,0.054902}%
\pgfsetfillcolor{currentfill}%
\pgfsetlinewidth{1.003750pt}%
\definecolor{currentstroke}{rgb}{1.000000,0.498039,0.054902}%
\pgfsetstrokecolor{currentstroke}%
\pgfsetdash{}{0pt}%
\pgfpathmoveto{\pgfqpoint{3.072396in}{1.097176in}}%
\pgfpathcurveto{\pgfqpoint{3.082484in}{1.097176in}}{\pgfqpoint{3.092159in}{1.101184in}}{\pgfqpoint{3.099292in}{1.108317in}}%
\pgfpathcurveto{\pgfqpoint{3.106425in}{1.115450in}}{\pgfqpoint{3.110433in}{1.125125in}}{\pgfqpoint{3.110433in}{1.135213in}}%
\pgfpathcurveto{\pgfqpoint{3.110433in}{1.145300in}}{\pgfqpoint{3.106425in}{1.154975in}}{\pgfqpoint{3.099292in}{1.162108in}}%
\pgfpathcurveto{\pgfqpoint{3.092159in}{1.169241in}}{\pgfqpoint{3.082484in}{1.173249in}}{\pgfqpoint{3.072396in}{1.173249in}}%
\pgfpathcurveto{\pgfqpoint{3.062309in}{1.173249in}}{\pgfqpoint{3.052633in}{1.169241in}}{\pgfqpoint{3.045501in}{1.162108in}}%
\pgfpathcurveto{\pgfqpoint{3.038368in}{1.154975in}}{\pgfqpoint{3.034360in}{1.145300in}}{\pgfqpoint{3.034360in}{1.135213in}}%
\pgfpathcurveto{\pgfqpoint{3.034360in}{1.125125in}}{\pgfqpoint{3.038368in}{1.115450in}}{\pgfqpoint{3.045501in}{1.108317in}}%
\pgfpathcurveto{\pgfqpoint{3.052633in}{1.101184in}}{\pgfqpoint{3.062309in}{1.097176in}}{\pgfqpoint{3.072396in}{1.097176in}}%
\pgfpathclose%
\pgfusepath{stroke,fill}%
\end{pgfscope}%
\begin{pgfscope}%
\pgfpathrectangle{\pgfqpoint{1.280114in}{0.528000in}}{\pgfqpoint{3.487886in}{3.696000in}} %
\pgfusepath{clip}%
\pgfsetbuttcap%
\pgfsetroundjoin%
\definecolor{currentfill}{rgb}{1.000000,0.498039,0.054902}%
\pgfsetfillcolor{currentfill}%
\pgfsetlinewidth{1.003750pt}%
\definecolor{currentstroke}{rgb}{1.000000,0.498039,0.054902}%
\pgfsetstrokecolor{currentstroke}%
\pgfsetdash{}{0pt}%
\pgfpathmoveto{\pgfqpoint{3.078058in}{4.011572in}}%
\pgfpathcurveto{\pgfqpoint{3.088145in}{4.011572in}}{\pgfqpoint{3.097821in}{4.015580in}}{\pgfqpoint{3.104954in}{4.022713in}}%
\pgfpathcurveto{\pgfqpoint{3.112087in}{4.029846in}}{\pgfqpoint{3.116094in}{4.039521in}}{\pgfqpoint{3.116094in}{4.049609in}}%
\pgfpathcurveto{\pgfqpoint{3.116094in}{4.059696in}}{\pgfqpoint{3.112087in}{4.069371in}}{\pgfqpoint{3.104954in}{4.076504in}}%
\pgfpathcurveto{\pgfqpoint{3.097821in}{4.083637in}}{\pgfqpoint{3.088145in}{4.087645in}}{\pgfqpoint{3.078058in}{4.087645in}}%
\pgfpathcurveto{\pgfqpoint{3.067971in}{4.087645in}}{\pgfqpoint{3.058295in}{4.083637in}}{\pgfqpoint{3.051162in}{4.076504in}}%
\pgfpathcurveto{\pgfqpoint{3.044030in}{4.069371in}}{\pgfqpoint{3.040022in}{4.059696in}}{\pgfqpoint{3.040022in}{4.049609in}}%
\pgfpathcurveto{\pgfqpoint{3.040022in}{4.039521in}}{\pgfqpoint{3.044030in}{4.029846in}}{\pgfqpoint{3.051162in}{4.022713in}}%
\pgfpathcurveto{\pgfqpoint{3.058295in}{4.015580in}}{\pgfqpoint{3.067971in}{4.011572in}}{\pgfqpoint{3.078058in}{4.011572in}}%
\pgfpathclose%
\pgfusepath{stroke,fill}%
\end{pgfscope}%
\begin{pgfscope}%
\pgfpathrectangle{\pgfqpoint{1.280114in}{0.528000in}}{\pgfqpoint{3.487886in}{3.696000in}} %
\pgfusepath{clip}%
\pgfsetbuttcap%
\pgfsetroundjoin%
\definecolor{currentfill}{rgb}{1.000000,0.498039,0.054902}%
\pgfsetfillcolor{currentfill}%
\pgfsetlinewidth{1.003750pt}%
\definecolor{currentstroke}{rgb}{1.000000,0.498039,0.054902}%
\pgfsetstrokecolor{currentstroke}%
\pgfsetdash{}{0pt}%
\pgfpathmoveto{\pgfqpoint{2.668279in}{2.179098in}}%
\pgfpathcurveto{\pgfqpoint{2.678367in}{2.179098in}}{\pgfqpoint{2.688042in}{2.183106in}}{\pgfqpoint{2.695175in}{2.190239in}}%
\pgfpathcurveto{\pgfqpoint{2.702308in}{2.197371in}}{\pgfqpoint{2.706316in}{2.207047in}}{\pgfqpoint{2.706316in}{2.217134in}}%
\pgfpathcurveto{\pgfqpoint{2.706316in}{2.227222in}}{\pgfqpoint{2.702308in}{2.236897in}}{\pgfqpoint{2.695175in}{2.244030in}}%
\pgfpathcurveto{\pgfqpoint{2.688042in}{2.251163in}}{\pgfqpoint{2.678367in}{2.255171in}}{\pgfqpoint{2.668279in}{2.255171in}}%
\pgfpathcurveto{\pgfqpoint{2.658192in}{2.255171in}}{\pgfqpoint{2.648516in}{2.251163in}}{\pgfqpoint{2.641384in}{2.244030in}}%
\pgfpathcurveto{\pgfqpoint{2.634251in}{2.236897in}}{\pgfqpoint{2.630243in}{2.227222in}}{\pgfqpoint{2.630243in}{2.217134in}}%
\pgfpathcurveto{\pgfqpoint{2.630243in}{2.207047in}}{\pgfqpoint{2.634251in}{2.197371in}}{\pgfqpoint{2.641384in}{2.190239in}}%
\pgfpathcurveto{\pgfqpoint{2.648516in}{2.183106in}}{\pgfqpoint{2.658192in}{2.179098in}}{\pgfqpoint{2.668279in}{2.179098in}}%
\pgfpathclose%
\pgfusepath{stroke,fill}%
\end{pgfscope}%
\begin{pgfscope}%
\pgfpathrectangle{\pgfqpoint{1.280114in}{0.528000in}}{\pgfqpoint{3.487886in}{3.696000in}} %
\pgfusepath{clip}%
\pgfsetbuttcap%
\pgfsetroundjoin%
\definecolor{currentfill}{rgb}{1.000000,0.498039,0.054902}%
\pgfsetfillcolor{currentfill}%
\pgfsetlinewidth{1.003750pt}%
\definecolor{currentstroke}{rgb}{1.000000,0.498039,0.054902}%
\pgfsetstrokecolor{currentstroke}%
\pgfsetdash{}{0pt}%
\pgfpathmoveto{\pgfqpoint{3.159084in}{0.664355in}}%
\pgfpathcurveto{\pgfqpoint{3.169171in}{0.664355in}}{\pgfqpoint{3.178846in}{0.668363in}}{\pgfqpoint{3.185979in}{0.675496in}}%
\pgfpathcurveto{\pgfqpoint{3.193112in}{0.682629in}}{\pgfqpoint{3.197120in}{0.692304in}}{\pgfqpoint{3.197120in}{0.702391in}}%
\pgfpathcurveto{\pgfqpoint{3.197120in}{0.712479in}}{\pgfqpoint{3.193112in}{0.722154in}}{\pgfqpoint{3.185979in}{0.729287in}}%
\pgfpathcurveto{\pgfqpoint{3.178846in}{0.736420in}}{\pgfqpoint{3.169171in}{0.740428in}}{\pgfqpoint{3.159084in}{0.740428in}}%
\pgfpathcurveto{\pgfqpoint{3.148996in}{0.740428in}}{\pgfqpoint{3.139321in}{0.736420in}}{\pgfqpoint{3.132188in}{0.729287in}}%
\pgfpathcurveto{\pgfqpoint{3.125055in}{0.722154in}}{\pgfqpoint{3.121047in}{0.712479in}}{\pgfqpoint{3.121047in}{0.702391in}}%
\pgfpathcurveto{\pgfqpoint{3.121047in}{0.692304in}}{\pgfqpoint{3.125055in}{0.682629in}}{\pgfqpoint{3.132188in}{0.675496in}}%
\pgfpathcurveto{\pgfqpoint{3.139321in}{0.668363in}}{\pgfqpoint{3.148996in}{0.664355in}}{\pgfqpoint{3.159084in}{0.664355in}}%
\pgfpathclose%
\pgfusepath{stroke,fill}%
\end{pgfscope}%
\begin{pgfscope}%
\pgfpathrectangle{\pgfqpoint{1.280114in}{0.528000in}}{\pgfqpoint{3.487886in}{3.696000in}} %
\pgfusepath{clip}%
\pgfsetbuttcap%
\pgfsetroundjoin%
\definecolor{currentfill}{rgb}{1.000000,0.498039,0.054902}%
\pgfsetfillcolor{currentfill}%
\pgfsetlinewidth{1.003750pt}%
\definecolor{currentstroke}{rgb}{1.000000,0.498039,0.054902}%
\pgfsetstrokecolor{currentstroke}%
\pgfsetdash{}{0pt}%
\pgfpathmoveto{\pgfqpoint{3.390573in}{2.386489in}}%
\pgfpathcurveto{\pgfqpoint{3.400660in}{2.386489in}}{\pgfqpoint{3.410335in}{2.390497in}}{\pgfqpoint{3.417468in}{2.397630in}}%
\pgfpathcurveto{\pgfqpoint{3.424601in}{2.404762in}}{\pgfqpoint{3.428609in}{2.414438in}}{\pgfqpoint{3.428609in}{2.424525in}}%
\pgfpathcurveto{\pgfqpoint{3.428609in}{2.434613in}}{\pgfqpoint{3.424601in}{2.444288in}}{\pgfqpoint{3.417468in}{2.451421in}}%
\pgfpathcurveto{\pgfqpoint{3.410335in}{2.458554in}}{\pgfqpoint{3.400660in}{2.462562in}}{\pgfqpoint{3.390573in}{2.462562in}}%
\pgfpathcurveto{\pgfqpoint{3.380485in}{2.462562in}}{\pgfqpoint{3.370810in}{2.458554in}}{\pgfqpoint{3.363677in}{2.451421in}}%
\pgfpathcurveto{\pgfqpoint{3.356544in}{2.444288in}}{\pgfqpoint{3.352536in}{2.434613in}}{\pgfqpoint{3.352536in}{2.424525in}}%
\pgfpathcurveto{\pgfqpoint{3.352536in}{2.414438in}}{\pgfqpoint{3.356544in}{2.404762in}}{\pgfqpoint{3.363677in}{2.397630in}}%
\pgfpathcurveto{\pgfqpoint{3.370810in}{2.390497in}}{\pgfqpoint{3.380485in}{2.386489in}}{\pgfqpoint{3.390573in}{2.386489in}}%
\pgfpathclose%
\pgfusepath{stroke,fill}%
\end{pgfscope}%
\begin{pgfscope}%
\pgfpathrectangle{\pgfqpoint{1.280114in}{0.528000in}}{\pgfqpoint{3.487886in}{3.696000in}} %
\pgfusepath{clip}%
\pgfsetbuttcap%
\pgfsetroundjoin%
\definecolor{currentfill}{rgb}{1.000000,0.498039,0.054902}%
\pgfsetfillcolor{currentfill}%
\pgfsetlinewidth{1.003750pt}%
\definecolor{currentstroke}{rgb}{1.000000,0.498039,0.054902}%
\pgfsetstrokecolor{currentstroke}%
\pgfsetdash{}{0pt}%
\pgfpathmoveto{\pgfqpoint{2.818812in}{3.608955in}}%
\pgfpathcurveto{\pgfqpoint{2.828900in}{3.608955in}}{\pgfqpoint{2.838575in}{3.612963in}}{\pgfqpoint{2.845708in}{3.620096in}}%
\pgfpathcurveto{\pgfqpoint{2.852841in}{3.627228in}}{\pgfqpoint{2.856849in}{3.636904in}}{\pgfqpoint{2.856849in}{3.646991in}}%
\pgfpathcurveto{\pgfqpoint{2.856849in}{3.657079in}}{\pgfqpoint{2.852841in}{3.666754in}}{\pgfqpoint{2.845708in}{3.673887in}}%
\pgfpathcurveto{\pgfqpoint{2.838575in}{3.681020in}}{\pgfqpoint{2.828900in}{3.685028in}}{\pgfqpoint{2.818812in}{3.685028in}}%
\pgfpathcurveto{\pgfqpoint{2.808725in}{3.685028in}}{\pgfqpoint{2.799049in}{3.681020in}}{\pgfqpoint{2.791917in}{3.673887in}}%
\pgfpathcurveto{\pgfqpoint{2.784784in}{3.666754in}}{\pgfqpoint{2.780776in}{3.657079in}}{\pgfqpoint{2.780776in}{3.646991in}}%
\pgfpathcurveto{\pgfqpoint{2.780776in}{3.636904in}}{\pgfqpoint{2.784784in}{3.627228in}}{\pgfqpoint{2.791917in}{3.620096in}}%
\pgfpathcurveto{\pgfqpoint{2.799049in}{3.612963in}}{\pgfqpoint{2.808725in}{3.608955in}}{\pgfqpoint{2.818812in}{3.608955in}}%
\pgfpathclose%
\pgfusepath{stroke,fill}%
\end{pgfscope}%
\begin{pgfscope}%
\pgfpathrectangle{\pgfqpoint{1.280114in}{0.528000in}}{\pgfqpoint{3.487886in}{3.696000in}} %
\pgfusepath{clip}%
\pgfsetbuttcap%
\pgfsetroundjoin%
\definecolor{currentfill}{rgb}{1.000000,0.498039,0.054902}%
\pgfsetfillcolor{currentfill}%
\pgfsetlinewidth{1.003750pt}%
\definecolor{currentstroke}{rgb}{1.000000,0.498039,0.054902}%
\pgfsetstrokecolor{currentstroke}%
\pgfsetdash{}{0pt}%
\pgfpathmoveto{\pgfqpoint{3.247342in}{1.419500in}}%
\pgfpathcurveto{\pgfqpoint{3.257430in}{1.419500in}}{\pgfqpoint{3.267105in}{1.423508in}}{\pgfqpoint{3.274238in}{1.430641in}}%
\pgfpathcurveto{\pgfqpoint{3.281371in}{1.437774in}}{\pgfqpoint{3.285379in}{1.447449in}}{\pgfqpoint{3.285379in}{1.457537in}}%
\pgfpathcurveto{\pgfqpoint{3.285379in}{1.467624in}}{\pgfqpoint{3.281371in}{1.477300in}}{\pgfqpoint{3.274238in}{1.484433in}}%
\pgfpathcurveto{\pgfqpoint{3.267105in}{1.491565in}}{\pgfqpoint{3.257430in}{1.495573in}}{\pgfqpoint{3.247342in}{1.495573in}}%
\pgfpathcurveto{\pgfqpoint{3.237255in}{1.495573in}}{\pgfqpoint{3.227580in}{1.491565in}}{\pgfqpoint{3.220447in}{1.484433in}}%
\pgfpathcurveto{\pgfqpoint{3.213314in}{1.477300in}}{\pgfqpoint{3.209306in}{1.467624in}}{\pgfqpoint{3.209306in}{1.457537in}}%
\pgfpathcurveto{\pgfqpoint{3.209306in}{1.447449in}}{\pgfqpoint{3.213314in}{1.437774in}}{\pgfqpoint{3.220447in}{1.430641in}}%
\pgfpathcurveto{\pgfqpoint{3.227580in}{1.423508in}}{\pgfqpoint{3.237255in}{1.419500in}}{\pgfqpoint{3.247342in}{1.419500in}}%
\pgfpathclose%
\pgfusepath{stroke,fill}%
\end{pgfscope}%
\begin{pgfscope}%
\pgfpathrectangle{\pgfqpoint{1.280114in}{0.528000in}}{\pgfqpoint{3.487886in}{3.696000in}} %
\pgfusepath{clip}%
\pgfsetbuttcap%
\pgfsetroundjoin%
\definecolor{currentfill}{rgb}{1.000000,0.498039,0.054902}%
\pgfsetfillcolor{currentfill}%
\pgfsetlinewidth{1.003750pt}%
\definecolor{currentstroke}{rgb}{1.000000,0.498039,0.054902}%
\pgfsetstrokecolor{currentstroke}%
\pgfsetdash{}{0pt}%
\pgfpathmoveto{\pgfqpoint{3.181625in}{1.168985in}}%
\pgfpathcurveto{\pgfqpoint{3.191713in}{1.168985in}}{\pgfqpoint{3.201388in}{1.172992in}}{\pgfqpoint{3.208521in}{1.180125in}}%
\pgfpathcurveto{\pgfqpoint{3.215654in}{1.187258in}}{\pgfqpoint{3.219662in}{1.196934in}}{\pgfqpoint{3.219662in}{1.207021in}}%
\pgfpathcurveto{\pgfqpoint{3.219662in}{1.217108in}}{\pgfqpoint{3.215654in}{1.226784in}}{\pgfqpoint{3.208521in}{1.233917in}}%
\pgfpathcurveto{\pgfqpoint{3.201388in}{1.241050in}}{\pgfqpoint{3.191713in}{1.245057in}}{\pgfqpoint{3.181625in}{1.245057in}}%
\pgfpathcurveto{\pgfqpoint{3.171538in}{1.245057in}}{\pgfqpoint{3.161862in}{1.241050in}}{\pgfqpoint{3.154730in}{1.233917in}}%
\pgfpathcurveto{\pgfqpoint{3.147597in}{1.226784in}}{\pgfqpoint{3.143589in}{1.217108in}}{\pgfqpoint{3.143589in}{1.207021in}}%
\pgfpathcurveto{\pgfqpoint{3.143589in}{1.196934in}}{\pgfqpoint{3.147597in}{1.187258in}}{\pgfqpoint{3.154730in}{1.180125in}}%
\pgfpathcurveto{\pgfqpoint{3.161862in}{1.172992in}}{\pgfqpoint{3.171538in}{1.168985in}}{\pgfqpoint{3.181625in}{1.168985in}}%
\pgfpathclose%
\pgfusepath{stroke,fill}%
\end{pgfscope}%
\begin{pgfscope}%
\pgfpathrectangle{\pgfqpoint{1.280114in}{0.528000in}}{\pgfqpoint{3.487886in}{3.696000in}} %
\pgfusepath{clip}%
\pgfsetbuttcap%
\pgfsetroundjoin%
\definecolor{currentfill}{rgb}{1.000000,0.498039,0.054902}%
\pgfsetfillcolor{currentfill}%
\pgfsetlinewidth{1.003750pt}%
\definecolor{currentstroke}{rgb}{1.000000,0.498039,0.054902}%
\pgfsetstrokecolor{currentstroke}%
\pgfsetdash{}{0pt}%
\pgfpathmoveto{\pgfqpoint{2.623643in}{1.790697in}}%
\pgfpathcurveto{\pgfqpoint{2.633730in}{1.790697in}}{\pgfqpoint{2.643406in}{1.794705in}}{\pgfqpoint{2.650539in}{1.801838in}}%
\pgfpathcurveto{\pgfqpoint{2.657671in}{1.808971in}}{\pgfqpoint{2.661679in}{1.818646in}}{\pgfqpoint{2.661679in}{1.828734in}}%
\pgfpathcurveto{\pgfqpoint{2.661679in}{1.838821in}}{\pgfqpoint{2.657671in}{1.848496in}}{\pgfqpoint{2.650539in}{1.855629in}}%
\pgfpathcurveto{\pgfqpoint{2.643406in}{1.862762in}}{\pgfqpoint{2.633730in}{1.866770in}}{\pgfqpoint{2.623643in}{1.866770in}}%
\pgfpathcurveto{\pgfqpoint{2.613556in}{1.866770in}}{\pgfqpoint{2.603880in}{1.862762in}}{\pgfqpoint{2.596747in}{1.855629in}}%
\pgfpathcurveto{\pgfqpoint{2.589614in}{1.848496in}}{\pgfqpoint{2.585607in}{1.838821in}}{\pgfqpoint{2.585607in}{1.828734in}}%
\pgfpathcurveto{\pgfqpoint{2.585607in}{1.818646in}}{\pgfqpoint{2.589614in}{1.808971in}}{\pgfqpoint{2.596747in}{1.801838in}}%
\pgfpathcurveto{\pgfqpoint{2.603880in}{1.794705in}}{\pgfqpoint{2.613556in}{1.790697in}}{\pgfqpoint{2.623643in}{1.790697in}}%
\pgfpathclose%
\pgfusepath{stroke,fill}%
\end{pgfscope}%
\begin{pgfscope}%
\pgfpathrectangle{\pgfqpoint{1.280114in}{0.528000in}}{\pgfqpoint{3.487886in}{3.696000in}} %
\pgfusepath{clip}%
\pgfsetbuttcap%
\pgfsetroundjoin%
\definecolor{currentfill}{rgb}{1.000000,0.498039,0.054902}%
\pgfsetfillcolor{currentfill}%
\pgfsetlinewidth{1.003750pt}%
\definecolor{currentstroke}{rgb}{1.000000,0.498039,0.054902}%
\pgfsetstrokecolor{currentstroke}%
\pgfsetdash{}{0pt}%
\pgfpathmoveto{\pgfqpoint{3.276765in}{1.238146in}}%
\pgfpathcurveto{\pgfqpoint{3.286852in}{1.238146in}}{\pgfqpoint{3.296528in}{1.242154in}}{\pgfqpoint{3.303661in}{1.249286in}}%
\pgfpathcurveto{\pgfqpoint{3.310794in}{1.256419in}}{\pgfqpoint{3.314801in}{1.266095in}}{\pgfqpoint{3.314801in}{1.276182in}}%
\pgfpathcurveto{\pgfqpoint{3.314801in}{1.286269in}}{\pgfqpoint{3.310794in}{1.295945in}}{\pgfqpoint{3.303661in}{1.303078in}}%
\pgfpathcurveto{\pgfqpoint{3.296528in}{1.310211in}}{\pgfqpoint{3.286852in}{1.314218in}}{\pgfqpoint{3.276765in}{1.314218in}}%
\pgfpathcurveto{\pgfqpoint{3.266678in}{1.314218in}}{\pgfqpoint{3.257002in}{1.310211in}}{\pgfqpoint{3.249869in}{1.303078in}}%
\pgfpathcurveto{\pgfqpoint{3.242736in}{1.295945in}}{\pgfqpoint{3.238729in}{1.286269in}}{\pgfqpoint{3.238729in}{1.276182in}}%
\pgfpathcurveto{\pgfqpoint{3.238729in}{1.266095in}}{\pgfqpoint{3.242736in}{1.256419in}}{\pgfqpoint{3.249869in}{1.249286in}}%
\pgfpathcurveto{\pgfqpoint{3.257002in}{1.242154in}}{\pgfqpoint{3.266678in}{1.238146in}}{\pgfqpoint{3.276765in}{1.238146in}}%
\pgfpathclose%
\pgfusepath{stroke,fill}%
\end{pgfscope}%
\begin{pgfscope}%
\pgfpathrectangle{\pgfqpoint{1.280114in}{0.528000in}}{\pgfqpoint{3.487886in}{3.696000in}} %
\pgfusepath{clip}%
\pgfsetbuttcap%
\pgfsetroundjoin%
\definecolor{currentfill}{rgb}{1.000000,0.498039,0.054902}%
\pgfsetfillcolor{currentfill}%
\pgfsetlinewidth{1.003750pt}%
\definecolor{currentstroke}{rgb}{1.000000,0.498039,0.054902}%
\pgfsetstrokecolor{currentstroke}%
\pgfsetdash{}{0pt}%
\pgfpathmoveto{\pgfqpoint{3.425678in}{2.074443in}}%
\pgfpathcurveto{\pgfqpoint{3.435766in}{2.074443in}}{\pgfqpoint{3.445441in}{2.078450in}}{\pgfqpoint{3.452574in}{2.085583in}}%
\pgfpathcurveto{\pgfqpoint{3.459707in}{2.092716in}}{\pgfqpoint{3.463715in}{2.102392in}}{\pgfqpoint{3.463715in}{2.112479in}}%
\pgfpathcurveto{\pgfqpoint{3.463715in}{2.122566in}}{\pgfqpoint{3.459707in}{2.132242in}}{\pgfqpoint{3.452574in}{2.139375in}}%
\pgfpathcurveto{\pgfqpoint{3.445441in}{2.146508in}}{\pgfqpoint{3.435766in}{2.150515in}}{\pgfqpoint{3.425678in}{2.150515in}}%
\pgfpathcurveto{\pgfqpoint{3.415591in}{2.150515in}}{\pgfqpoint{3.405916in}{2.146508in}}{\pgfqpoint{3.398783in}{2.139375in}}%
\pgfpathcurveto{\pgfqpoint{3.391650in}{2.132242in}}{\pgfqpoint{3.387642in}{2.122566in}}{\pgfqpoint{3.387642in}{2.112479in}}%
\pgfpathcurveto{\pgfqpoint{3.387642in}{2.102392in}}{\pgfqpoint{3.391650in}{2.092716in}}{\pgfqpoint{3.398783in}{2.085583in}}%
\pgfpathcurveto{\pgfqpoint{3.405916in}{2.078450in}}{\pgfqpoint{3.415591in}{2.074443in}}{\pgfqpoint{3.425678in}{2.074443in}}%
\pgfpathclose%
\pgfusepath{stroke,fill}%
\end{pgfscope}%
\begin{pgfscope}%
\pgfpathrectangle{\pgfqpoint{1.280114in}{0.528000in}}{\pgfqpoint{3.487886in}{3.696000in}} %
\pgfusepath{clip}%
\pgfsetbuttcap%
\pgfsetroundjoin%
\definecolor{currentfill}{rgb}{1.000000,0.498039,0.054902}%
\pgfsetfillcolor{currentfill}%
\pgfsetlinewidth{1.003750pt}%
\definecolor{currentstroke}{rgb}{1.000000,0.498039,0.054902}%
\pgfsetstrokecolor{currentstroke}%
\pgfsetdash{}{0pt}%
\pgfpathmoveto{\pgfqpoint{3.199153in}{1.009738in}}%
\pgfpathcurveto{\pgfqpoint{3.209240in}{1.009738in}}{\pgfqpoint{3.218916in}{1.013746in}}{\pgfqpoint{3.226048in}{1.020879in}}%
\pgfpathcurveto{\pgfqpoint{3.233181in}{1.028011in}}{\pgfqpoint{3.237189in}{1.037687in}}{\pgfqpoint{3.237189in}{1.047774in}}%
\pgfpathcurveto{\pgfqpoint{3.237189in}{1.057862in}}{\pgfqpoint{3.233181in}{1.067537in}}{\pgfqpoint{3.226048in}{1.074670in}}%
\pgfpathcurveto{\pgfqpoint{3.218916in}{1.081803in}}{\pgfqpoint{3.209240in}{1.085811in}}{\pgfqpoint{3.199153in}{1.085811in}}%
\pgfpathcurveto{\pgfqpoint{3.189065in}{1.085811in}}{\pgfqpoint{3.179390in}{1.081803in}}{\pgfqpoint{3.172257in}{1.074670in}}%
\pgfpathcurveto{\pgfqpoint{3.165124in}{1.067537in}}{\pgfqpoint{3.161116in}{1.057862in}}{\pgfqpoint{3.161116in}{1.047774in}}%
\pgfpathcurveto{\pgfqpoint{3.161116in}{1.037687in}}{\pgfqpoint{3.165124in}{1.028011in}}{\pgfqpoint{3.172257in}{1.020879in}}%
\pgfpathcurveto{\pgfqpoint{3.179390in}{1.013746in}}{\pgfqpoint{3.189065in}{1.009738in}}{\pgfqpoint{3.199153in}{1.009738in}}%
\pgfpathclose%
\pgfusepath{stroke,fill}%
\end{pgfscope}%
\begin{pgfscope}%
\pgfpathrectangle{\pgfqpoint{1.280114in}{0.528000in}}{\pgfqpoint{3.487886in}{3.696000in}} %
\pgfusepath{clip}%
\pgfsetbuttcap%
\pgfsetroundjoin%
\definecolor{currentfill}{rgb}{1.000000,0.498039,0.054902}%
\pgfsetfillcolor{currentfill}%
\pgfsetlinewidth{1.003750pt}%
\definecolor{currentstroke}{rgb}{1.000000,0.498039,0.054902}%
\pgfsetstrokecolor{currentstroke}%
\pgfsetdash{}{0pt}%
\pgfpathmoveto{\pgfqpoint{3.273660in}{1.109002in}}%
\pgfpathcurveto{\pgfqpoint{3.283748in}{1.109002in}}{\pgfqpoint{3.293423in}{1.113009in}}{\pgfqpoint{3.300556in}{1.120142in}}%
\pgfpathcurveto{\pgfqpoint{3.307689in}{1.127275in}}{\pgfqpoint{3.311696in}{1.136951in}}{\pgfqpoint{3.311696in}{1.147038in}}%
\pgfpathcurveto{\pgfqpoint{3.311696in}{1.157125in}}{\pgfqpoint{3.307689in}{1.166801in}}{\pgfqpoint{3.300556in}{1.173934in}}%
\pgfpathcurveto{\pgfqpoint{3.293423in}{1.181066in}}{\pgfqpoint{3.283748in}{1.185074in}}{\pgfqpoint{3.273660in}{1.185074in}}%
\pgfpathcurveto{\pgfqpoint{3.263573in}{1.185074in}}{\pgfqpoint{3.253897in}{1.181066in}}{\pgfqpoint{3.246764in}{1.173934in}}%
\pgfpathcurveto{\pgfqpoint{3.239632in}{1.166801in}}{\pgfqpoint{3.235624in}{1.157125in}}{\pgfqpoint{3.235624in}{1.147038in}}%
\pgfpathcurveto{\pgfqpoint{3.235624in}{1.136951in}}{\pgfqpoint{3.239632in}{1.127275in}}{\pgfqpoint{3.246764in}{1.120142in}}%
\pgfpathcurveto{\pgfqpoint{3.253897in}{1.113009in}}{\pgfqpoint{3.263573in}{1.109002in}}{\pgfqpoint{3.273660in}{1.109002in}}%
\pgfpathclose%
\pgfusepath{stroke,fill}%
\end{pgfscope}%
\begin{pgfscope}%
\pgfpathrectangle{\pgfqpoint{1.280114in}{0.528000in}}{\pgfqpoint{3.487886in}{3.696000in}} %
\pgfusepath{clip}%
\pgfsetbuttcap%
\pgfsetroundjoin%
\definecolor{currentfill}{rgb}{1.000000,0.498039,0.054902}%
\pgfsetfillcolor{currentfill}%
\pgfsetlinewidth{1.003750pt}%
\definecolor{currentstroke}{rgb}{1.000000,0.498039,0.054902}%
\pgfsetstrokecolor{currentstroke}%
\pgfsetdash{}{0pt}%
\pgfpathmoveto{\pgfqpoint{3.338397in}{3.443985in}}%
\pgfpathcurveto{\pgfqpoint{3.348485in}{3.443985in}}{\pgfqpoint{3.358160in}{3.447993in}}{\pgfqpoint{3.365293in}{3.455126in}}%
\pgfpathcurveto{\pgfqpoint{3.372426in}{3.462259in}}{\pgfqpoint{3.376434in}{3.471934in}}{\pgfqpoint{3.376434in}{3.482022in}}%
\pgfpathcurveto{\pgfqpoint{3.376434in}{3.492109in}}{\pgfqpoint{3.372426in}{3.501784in}}{\pgfqpoint{3.365293in}{3.508917in}}%
\pgfpathcurveto{\pgfqpoint{3.358160in}{3.516050in}}{\pgfqpoint{3.348485in}{3.520058in}}{\pgfqpoint{3.338397in}{3.520058in}}%
\pgfpathcurveto{\pgfqpoint{3.328310in}{3.520058in}}{\pgfqpoint{3.318634in}{3.516050in}}{\pgfqpoint{3.311502in}{3.508917in}}%
\pgfpathcurveto{\pgfqpoint{3.304369in}{3.501784in}}{\pgfqpoint{3.300361in}{3.492109in}}{\pgfqpoint{3.300361in}{3.482022in}}%
\pgfpathcurveto{\pgfqpoint{3.300361in}{3.471934in}}{\pgfqpoint{3.304369in}{3.462259in}}{\pgfqpoint{3.311502in}{3.455126in}}%
\pgfpathcurveto{\pgfqpoint{3.318634in}{3.447993in}}{\pgfqpoint{3.328310in}{3.443985in}}{\pgfqpoint{3.338397in}{3.443985in}}%
\pgfpathclose%
\pgfusepath{stroke,fill}%
\end{pgfscope}%
\begin{pgfscope}%
\pgfpathrectangle{\pgfqpoint{1.280114in}{0.528000in}}{\pgfqpoint{3.487886in}{3.696000in}} %
\pgfusepath{clip}%
\pgfsetbuttcap%
\pgfsetroundjoin%
\definecolor{currentfill}{rgb}{1.000000,0.498039,0.054902}%
\pgfsetfillcolor{currentfill}%
\pgfsetlinewidth{1.003750pt}%
\definecolor{currentstroke}{rgb}{1.000000,0.498039,0.054902}%
\pgfsetstrokecolor{currentstroke}%
\pgfsetdash{}{0pt}%
\pgfpathmoveto{\pgfqpoint{2.823984in}{1.163248in}}%
\pgfpathcurveto{\pgfqpoint{2.834072in}{1.163248in}}{\pgfqpoint{2.843747in}{1.167256in}}{\pgfqpoint{2.850880in}{1.174389in}}%
\pgfpathcurveto{\pgfqpoint{2.858013in}{1.181522in}}{\pgfqpoint{2.862020in}{1.191197in}}{\pgfqpoint{2.862020in}{1.201285in}}%
\pgfpathcurveto{\pgfqpoint{2.862020in}{1.211372in}}{\pgfqpoint{2.858013in}{1.221047in}}{\pgfqpoint{2.850880in}{1.228180in}}%
\pgfpathcurveto{\pgfqpoint{2.843747in}{1.235313in}}{\pgfqpoint{2.834072in}{1.239321in}}{\pgfqpoint{2.823984in}{1.239321in}}%
\pgfpathcurveto{\pgfqpoint{2.813897in}{1.239321in}}{\pgfqpoint{2.804221in}{1.235313in}}{\pgfqpoint{2.797088in}{1.228180in}}%
\pgfpathcurveto{\pgfqpoint{2.789956in}{1.221047in}}{\pgfqpoint{2.785948in}{1.211372in}}{\pgfqpoint{2.785948in}{1.201285in}}%
\pgfpathcurveto{\pgfqpoint{2.785948in}{1.191197in}}{\pgfqpoint{2.789956in}{1.181522in}}{\pgfqpoint{2.797088in}{1.174389in}}%
\pgfpathcurveto{\pgfqpoint{2.804221in}{1.167256in}}{\pgfqpoint{2.813897in}{1.163248in}}{\pgfqpoint{2.823984in}{1.163248in}}%
\pgfpathclose%
\pgfusepath{stroke,fill}%
\end{pgfscope}%
\begin{pgfscope}%
\pgfpathrectangle{\pgfqpoint{1.280114in}{0.528000in}}{\pgfqpoint{3.487886in}{3.696000in}} %
\pgfusepath{clip}%
\pgfsetbuttcap%
\pgfsetroundjoin%
\definecolor{currentfill}{rgb}{1.000000,0.498039,0.054902}%
\pgfsetfillcolor{currentfill}%
\pgfsetlinewidth{1.003750pt}%
\definecolor{currentstroke}{rgb}{1.000000,0.498039,0.054902}%
\pgfsetstrokecolor{currentstroke}%
\pgfsetdash{}{0pt}%
\pgfpathmoveto{\pgfqpoint{2.928141in}{0.893863in}}%
\pgfpathcurveto{\pgfqpoint{2.938228in}{0.893863in}}{\pgfqpoint{2.947904in}{0.897871in}}{\pgfqpoint{2.955036in}{0.905004in}}%
\pgfpathcurveto{\pgfqpoint{2.962169in}{0.912137in}}{\pgfqpoint{2.966177in}{0.921812in}}{\pgfqpoint{2.966177in}{0.931900in}}%
\pgfpathcurveto{\pgfqpoint{2.966177in}{0.941987in}}{\pgfqpoint{2.962169in}{0.951662in}}{\pgfqpoint{2.955036in}{0.958795in}}%
\pgfpathcurveto{\pgfqpoint{2.947904in}{0.965928in}}{\pgfqpoint{2.938228in}{0.969936in}}{\pgfqpoint{2.928141in}{0.969936in}}%
\pgfpathcurveto{\pgfqpoint{2.918053in}{0.969936in}}{\pgfqpoint{2.908378in}{0.965928in}}{\pgfqpoint{2.901245in}{0.958795in}}%
\pgfpathcurveto{\pgfqpoint{2.894112in}{0.951662in}}{\pgfqpoint{2.890104in}{0.941987in}}{\pgfqpoint{2.890104in}{0.931900in}}%
\pgfpathcurveto{\pgfqpoint{2.890104in}{0.921812in}}{\pgfqpoint{2.894112in}{0.912137in}}{\pgfqpoint{2.901245in}{0.905004in}}%
\pgfpathcurveto{\pgfqpoint{2.908378in}{0.897871in}}{\pgfqpoint{2.918053in}{0.893863in}}{\pgfqpoint{2.928141in}{0.893863in}}%
\pgfpathclose%
\pgfusepath{stroke,fill}%
\end{pgfscope}%
\begin{pgfscope}%
\pgfpathrectangle{\pgfqpoint{1.280114in}{0.528000in}}{\pgfqpoint{3.487886in}{3.696000in}} %
\pgfusepath{clip}%
\pgfsetbuttcap%
\pgfsetroundjoin%
\definecolor{currentfill}{rgb}{1.000000,0.498039,0.054902}%
\pgfsetfillcolor{currentfill}%
\pgfsetlinewidth{1.003750pt}%
\definecolor{currentstroke}{rgb}{1.000000,0.498039,0.054902}%
\pgfsetstrokecolor{currentstroke}%
\pgfsetdash{}{0pt}%
\pgfpathmoveto{\pgfqpoint{2.749280in}{1.454448in}}%
\pgfpathcurveto{\pgfqpoint{2.759368in}{1.454448in}}{\pgfqpoint{2.769043in}{1.458456in}}{\pgfqpoint{2.776176in}{1.465588in}}%
\pgfpathcurveto{\pgfqpoint{2.783309in}{1.472721in}}{\pgfqpoint{2.787317in}{1.482397in}}{\pgfqpoint{2.787317in}{1.492484in}}%
\pgfpathcurveto{\pgfqpoint{2.787317in}{1.502571in}}{\pgfqpoint{2.783309in}{1.512247in}}{\pgfqpoint{2.776176in}{1.519380in}}%
\pgfpathcurveto{\pgfqpoint{2.769043in}{1.526513in}}{\pgfqpoint{2.759368in}{1.530520in}}{\pgfqpoint{2.749280in}{1.530520in}}%
\pgfpathcurveto{\pgfqpoint{2.739193in}{1.530520in}}{\pgfqpoint{2.729517in}{1.526513in}}{\pgfqpoint{2.722384in}{1.519380in}}%
\pgfpathcurveto{\pgfqpoint{2.715252in}{1.512247in}}{\pgfqpoint{2.711244in}{1.502571in}}{\pgfqpoint{2.711244in}{1.492484in}}%
\pgfpathcurveto{\pgfqpoint{2.711244in}{1.482397in}}{\pgfqpoint{2.715252in}{1.472721in}}{\pgfqpoint{2.722384in}{1.465588in}}%
\pgfpathcurveto{\pgfqpoint{2.729517in}{1.458456in}}{\pgfqpoint{2.739193in}{1.454448in}}{\pgfqpoint{2.749280in}{1.454448in}}%
\pgfpathclose%
\pgfusepath{stroke,fill}%
\end{pgfscope}%
\begin{pgfscope}%
\pgfpathrectangle{\pgfqpoint{1.280114in}{0.528000in}}{\pgfqpoint{3.487886in}{3.696000in}} %
\pgfusepath{clip}%
\pgfsetbuttcap%
\pgfsetroundjoin%
\definecolor{currentfill}{rgb}{1.000000,0.498039,0.054902}%
\pgfsetfillcolor{currentfill}%
\pgfsetlinewidth{1.003750pt}%
\definecolor{currentstroke}{rgb}{1.000000,0.498039,0.054902}%
\pgfsetstrokecolor{currentstroke}%
\pgfsetdash{}{0pt}%
\pgfpathmoveto{\pgfqpoint{3.165294in}{0.954636in}}%
\pgfpathcurveto{\pgfqpoint{3.175381in}{0.954636in}}{\pgfqpoint{3.185057in}{0.958644in}}{\pgfqpoint{3.192189in}{0.965777in}}%
\pgfpathcurveto{\pgfqpoint{3.199322in}{0.972909in}}{\pgfqpoint{3.203330in}{0.982585in}}{\pgfqpoint{3.203330in}{0.992672in}}%
\pgfpathcurveto{\pgfqpoint{3.203330in}{1.002760in}}{\pgfqpoint{3.199322in}{1.012435in}}{\pgfqpoint{3.192189in}{1.019568in}}%
\pgfpathcurveto{\pgfqpoint{3.185057in}{1.026701in}}{\pgfqpoint{3.175381in}{1.030709in}}{\pgfqpoint{3.165294in}{1.030709in}}%
\pgfpathcurveto{\pgfqpoint{3.155206in}{1.030709in}}{\pgfqpoint{3.145531in}{1.026701in}}{\pgfqpoint{3.138398in}{1.019568in}}%
\pgfpathcurveto{\pgfqpoint{3.131265in}{1.012435in}}{\pgfqpoint{3.127257in}{1.002760in}}{\pgfqpoint{3.127257in}{0.992672in}}%
\pgfpathcurveto{\pgfqpoint{3.127257in}{0.982585in}}{\pgfqpoint{3.131265in}{0.972909in}}{\pgfqpoint{3.138398in}{0.965777in}}%
\pgfpathcurveto{\pgfqpoint{3.145531in}{0.958644in}}{\pgfqpoint{3.155206in}{0.954636in}}{\pgfqpoint{3.165294in}{0.954636in}}%
\pgfpathclose%
\pgfusepath{stroke,fill}%
\end{pgfscope}%
\begin{pgfscope}%
\pgfpathrectangle{\pgfqpoint{1.280114in}{0.528000in}}{\pgfqpoint{3.487886in}{3.696000in}} %
\pgfusepath{clip}%
\pgfsetbuttcap%
\pgfsetroundjoin%
\definecolor{currentfill}{rgb}{1.000000,0.498039,0.054902}%
\pgfsetfillcolor{currentfill}%
\pgfsetlinewidth{1.003750pt}%
\definecolor{currentstroke}{rgb}{1.000000,0.498039,0.054902}%
\pgfsetstrokecolor{currentstroke}%
\pgfsetdash{}{0pt}%
\pgfpathmoveto{\pgfqpoint{3.370861in}{2.640307in}}%
\pgfpathcurveto{\pgfqpoint{3.380948in}{2.640307in}}{\pgfqpoint{3.390624in}{2.644314in}}{\pgfqpoint{3.397756in}{2.651447in}}%
\pgfpathcurveto{\pgfqpoint{3.404889in}{2.658580in}}{\pgfqpoint{3.408897in}{2.668256in}}{\pgfqpoint{3.408897in}{2.678343in}}%
\pgfpathcurveto{\pgfqpoint{3.408897in}{2.688430in}}{\pgfqpoint{3.404889in}{2.698106in}}{\pgfqpoint{3.397756in}{2.705239in}}%
\pgfpathcurveto{\pgfqpoint{3.390624in}{2.712372in}}{\pgfqpoint{3.380948in}{2.716379in}}{\pgfqpoint{3.370861in}{2.716379in}}%
\pgfpathcurveto{\pgfqpoint{3.360773in}{2.716379in}}{\pgfqpoint{3.351098in}{2.712372in}}{\pgfqpoint{3.343965in}{2.705239in}}%
\pgfpathcurveto{\pgfqpoint{3.336832in}{2.698106in}}{\pgfqpoint{3.332824in}{2.688430in}}{\pgfqpoint{3.332824in}{2.678343in}}%
\pgfpathcurveto{\pgfqpoint{3.332824in}{2.668256in}}{\pgfqpoint{3.336832in}{2.658580in}}{\pgfqpoint{3.343965in}{2.651447in}}%
\pgfpathcurveto{\pgfqpoint{3.351098in}{2.644314in}}{\pgfqpoint{3.360773in}{2.640307in}}{\pgfqpoint{3.370861in}{2.640307in}}%
\pgfpathclose%
\pgfusepath{stroke,fill}%
\end{pgfscope}%
\begin{pgfscope}%
\pgfpathrectangle{\pgfqpoint{1.280114in}{0.528000in}}{\pgfqpoint{3.487886in}{3.696000in}} %
\pgfusepath{clip}%
\pgfsetbuttcap%
\pgfsetroundjoin%
\definecolor{currentfill}{rgb}{1.000000,0.498039,0.054902}%
\pgfsetfillcolor{currentfill}%
\pgfsetlinewidth{1.003750pt}%
\definecolor{currentstroke}{rgb}{1.000000,0.498039,0.054902}%
\pgfsetstrokecolor{currentstroke}%
\pgfsetdash{}{0pt}%
\pgfpathmoveto{\pgfqpoint{3.112947in}{1.020107in}}%
\pgfpathcurveto{\pgfqpoint{3.123034in}{1.020107in}}{\pgfqpoint{3.132710in}{1.024114in}}{\pgfqpoint{3.139843in}{1.031247in}}%
\pgfpathcurveto{\pgfqpoint{3.146976in}{1.038380in}}{\pgfqpoint{3.150983in}{1.048056in}}{\pgfqpoint{3.150983in}{1.058143in}}%
\pgfpathcurveto{\pgfqpoint{3.150983in}{1.068230in}}{\pgfqpoint{3.146976in}{1.077906in}}{\pgfqpoint{3.139843in}{1.085039in}}%
\pgfpathcurveto{\pgfqpoint{3.132710in}{1.092172in}}{\pgfqpoint{3.123034in}{1.096179in}}{\pgfqpoint{3.112947in}{1.096179in}}%
\pgfpathcurveto{\pgfqpoint{3.102860in}{1.096179in}}{\pgfqpoint{3.093184in}{1.092172in}}{\pgfqpoint{3.086051in}{1.085039in}}%
\pgfpathcurveto{\pgfqpoint{3.078919in}{1.077906in}}{\pgfqpoint{3.074911in}{1.068230in}}{\pgfqpoint{3.074911in}{1.058143in}}%
\pgfpathcurveto{\pgfqpoint{3.074911in}{1.048056in}}{\pgfqpoint{3.078919in}{1.038380in}}{\pgfqpoint{3.086051in}{1.031247in}}%
\pgfpathcurveto{\pgfqpoint{3.093184in}{1.024114in}}{\pgfqpoint{3.102860in}{1.020107in}}{\pgfqpoint{3.112947in}{1.020107in}}%
\pgfpathclose%
\pgfusepath{stroke,fill}%
\end{pgfscope}%
\begin{pgfscope}%
\pgfpathrectangle{\pgfqpoint{1.280114in}{0.528000in}}{\pgfqpoint{3.487886in}{3.696000in}} %
\pgfusepath{clip}%
\pgfsetbuttcap%
\pgfsetroundjoin%
\definecolor{currentfill}{rgb}{1.000000,0.498039,0.054902}%
\pgfsetfillcolor{currentfill}%
\pgfsetlinewidth{1.003750pt}%
\definecolor{currentstroke}{rgb}{1.000000,0.498039,0.054902}%
\pgfsetstrokecolor{currentstroke}%
\pgfsetdash{}{0pt}%
\pgfpathmoveto{\pgfqpoint{3.021023in}{1.008158in}}%
\pgfpathcurveto{\pgfqpoint{3.031110in}{1.008158in}}{\pgfqpoint{3.040786in}{1.012166in}}{\pgfqpoint{3.047919in}{1.019299in}}%
\pgfpathcurveto{\pgfqpoint{3.055051in}{1.026432in}}{\pgfqpoint{3.059059in}{1.036107in}}{\pgfqpoint{3.059059in}{1.046194in}}%
\pgfpathcurveto{\pgfqpoint{3.059059in}{1.056282in}}{\pgfqpoint{3.055051in}{1.065957in}}{\pgfqpoint{3.047919in}{1.073090in}}%
\pgfpathcurveto{\pgfqpoint{3.040786in}{1.080223in}}{\pgfqpoint{3.031110in}{1.084231in}}{\pgfqpoint{3.021023in}{1.084231in}}%
\pgfpathcurveto{\pgfqpoint{3.010936in}{1.084231in}}{\pgfqpoint{3.001260in}{1.080223in}}{\pgfqpoint{2.994127in}{1.073090in}}%
\pgfpathcurveto{\pgfqpoint{2.986994in}{1.065957in}}{\pgfqpoint{2.982987in}{1.056282in}}{\pgfqpoint{2.982987in}{1.046194in}}%
\pgfpathcurveto{\pgfqpoint{2.982987in}{1.036107in}}{\pgfqpoint{2.986994in}{1.026432in}}{\pgfqpoint{2.994127in}{1.019299in}}%
\pgfpathcurveto{\pgfqpoint{3.001260in}{1.012166in}}{\pgfqpoint{3.010936in}{1.008158in}}{\pgfqpoint{3.021023in}{1.008158in}}%
\pgfpathclose%
\pgfusepath{stroke,fill}%
\end{pgfscope}%
\begin{pgfscope}%
\pgfpathrectangle{\pgfqpoint{1.280114in}{0.528000in}}{\pgfqpoint{3.487886in}{3.696000in}} %
\pgfusepath{clip}%
\pgfsetbuttcap%
\pgfsetroundjoin%
\definecolor{currentfill}{rgb}{1.000000,0.498039,0.054902}%
\pgfsetfillcolor{currentfill}%
\pgfsetlinewidth{1.003750pt}%
\definecolor{currentstroke}{rgb}{1.000000,0.498039,0.054902}%
\pgfsetstrokecolor{currentstroke}%
\pgfsetdash{}{0pt}%
\pgfpathmoveto{\pgfqpoint{3.011674in}{3.935159in}}%
\pgfpathcurveto{\pgfqpoint{3.021761in}{3.935159in}}{\pgfqpoint{3.031437in}{3.939167in}}{\pgfqpoint{3.038570in}{3.946300in}}%
\pgfpathcurveto{\pgfqpoint{3.045702in}{3.953433in}}{\pgfqpoint{3.049710in}{3.963108in}}{\pgfqpoint{3.049710in}{3.973195in}}%
\pgfpathcurveto{\pgfqpoint{3.049710in}{3.983283in}}{\pgfqpoint{3.045702in}{3.992958in}}{\pgfqpoint{3.038570in}{4.000091in}}%
\pgfpathcurveto{\pgfqpoint{3.031437in}{4.007224in}}{\pgfqpoint{3.021761in}{4.011232in}}{\pgfqpoint{3.011674in}{4.011232in}}%
\pgfpathcurveto{\pgfqpoint{3.001587in}{4.011232in}}{\pgfqpoint{2.991911in}{4.007224in}}{\pgfqpoint{2.984778in}{4.000091in}}%
\pgfpathcurveto{\pgfqpoint{2.977645in}{3.992958in}}{\pgfqpoint{2.973638in}{3.983283in}}{\pgfqpoint{2.973638in}{3.973195in}}%
\pgfpathcurveto{\pgfqpoint{2.973638in}{3.963108in}}{\pgfqpoint{2.977645in}{3.953433in}}{\pgfqpoint{2.984778in}{3.946300in}}%
\pgfpathcurveto{\pgfqpoint{2.991911in}{3.939167in}}{\pgfqpoint{3.001587in}{3.935159in}}{\pgfqpoint{3.011674in}{3.935159in}}%
\pgfpathclose%
\pgfusepath{stroke,fill}%
\end{pgfscope}%
\begin{pgfscope}%
\pgfpathrectangle{\pgfqpoint{1.280114in}{0.528000in}}{\pgfqpoint{3.487886in}{3.696000in}} %
\pgfusepath{clip}%
\pgfsetbuttcap%
\pgfsetroundjoin%
\definecolor{currentfill}{rgb}{1.000000,0.498039,0.054902}%
\pgfsetfillcolor{currentfill}%
\pgfsetlinewidth{1.003750pt}%
\definecolor{currentstroke}{rgb}{1.000000,0.498039,0.054902}%
\pgfsetstrokecolor{currentstroke}%
\pgfsetdash{}{0pt}%
\pgfpathmoveto{\pgfqpoint{3.044043in}{3.837406in}}%
\pgfpathcurveto{\pgfqpoint{3.054131in}{3.837406in}}{\pgfqpoint{3.063806in}{3.841414in}}{\pgfqpoint{3.070939in}{3.848546in}}%
\pgfpathcurveto{\pgfqpoint{3.078072in}{3.855679in}}{\pgfqpoint{3.082080in}{3.865355in}}{\pgfqpoint{3.082080in}{3.875442in}}%
\pgfpathcurveto{\pgfqpoint{3.082080in}{3.885529in}}{\pgfqpoint{3.078072in}{3.895205in}}{\pgfqpoint{3.070939in}{3.902338in}}%
\pgfpathcurveto{\pgfqpoint{3.063806in}{3.909471in}}{\pgfqpoint{3.054131in}{3.913478in}}{\pgfqpoint{3.044043in}{3.913478in}}%
\pgfpathcurveto{\pgfqpoint{3.033956in}{3.913478in}}{\pgfqpoint{3.024281in}{3.909471in}}{\pgfqpoint{3.017148in}{3.902338in}}%
\pgfpathcurveto{\pgfqpoint{3.010015in}{3.895205in}}{\pgfqpoint{3.006007in}{3.885529in}}{\pgfqpoint{3.006007in}{3.875442in}}%
\pgfpathcurveto{\pgfqpoint{3.006007in}{3.865355in}}{\pgfqpoint{3.010015in}{3.855679in}}{\pgfqpoint{3.017148in}{3.848546in}}%
\pgfpathcurveto{\pgfqpoint{3.024281in}{3.841414in}}{\pgfqpoint{3.033956in}{3.837406in}}{\pgfqpoint{3.044043in}{3.837406in}}%
\pgfpathclose%
\pgfusepath{stroke,fill}%
\end{pgfscope}%
\begin{pgfscope}%
\pgfpathrectangle{\pgfqpoint{1.280114in}{0.528000in}}{\pgfqpoint{3.487886in}{3.696000in}} %
\pgfusepath{clip}%
\pgfsetbuttcap%
\pgfsetroundjoin%
\definecolor{currentfill}{rgb}{1.000000,0.498039,0.054902}%
\pgfsetfillcolor{currentfill}%
\pgfsetlinewidth{1.003750pt}%
\definecolor{currentstroke}{rgb}{1.000000,0.498039,0.054902}%
\pgfsetstrokecolor{currentstroke}%
\pgfsetdash{}{0pt}%
\pgfpathmoveto{\pgfqpoint{2.770929in}{1.389333in}}%
\pgfpathcurveto{\pgfqpoint{2.781017in}{1.389333in}}{\pgfqpoint{2.790692in}{1.393341in}}{\pgfqpoint{2.797825in}{1.400474in}}%
\pgfpathcurveto{\pgfqpoint{2.804958in}{1.407607in}}{\pgfqpoint{2.808965in}{1.417282in}}{\pgfqpoint{2.808965in}{1.427370in}}%
\pgfpathcurveto{\pgfqpoint{2.808965in}{1.437457in}}{\pgfqpoint{2.804958in}{1.447133in}}{\pgfqpoint{2.797825in}{1.454265in}}%
\pgfpathcurveto{\pgfqpoint{2.790692in}{1.461398in}}{\pgfqpoint{2.781017in}{1.465406in}}{\pgfqpoint{2.770929in}{1.465406in}}%
\pgfpathcurveto{\pgfqpoint{2.760842in}{1.465406in}}{\pgfqpoint{2.751166in}{1.461398in}}{\pgfqpoint{2.744033in}{1.454265in}}%
\pgfpathcurveto{\pgfqpoint{2.736901in}{1.447133in}}{\pgfqpoint{2.732893in}{1.437457in}}{\pgfqpoint{2.732893in}{1.427370in}}%
\pgfpathcurveto{\pgfqpoint{2.732893in}{1.417282in}}{\pgfqpoint{2.736901in}{1.407607in}}{\pgfqpoint{2.744033in}{1.400474in}}%
\pgfpathcurveto{\pgfqpoint{2.751166in}{1.393341in}}{\pgfqpoint{2.760842in}{1.389333in}}{\pgfqpoint{2.770929in}{1.389333in}}%
\pgfpathclose%
\pgfusepath{stroke,fill}%
\end{pgfscope}%
\begin{pgfscope}%
\pgfpathrectangle{\pgfqpoint{1.280114in}{0.528000in}}{\pgfqpoint{3.487886in}{3.696000in}} %
\pgfusepath{clip}%
\pgfsetbuttcap%
\pgfsetroundjoin%
\definecolor{currentfill}{rgb}{1.000000,0.498039,0.054902}%
\pgfsetfillcolor{currentfill}%
\pgfsetlinewidth{1.003750pt}%
\definecolor{currentstroke}{rgb}{1.000000,0.498039,0.054902}%
\pgfsetstrokecolor{currentstroke}%
\pgfsetdash{}{0pt}%
\pgfpathmoveto{\pgfqpoint{2.816161in}{3.586246in}}%
\pgfpathcurveto{\pgfqpoint{2.826248in}{3.586246in}}{\pgfqpoint{2.835924in}{3.590253in}}{\pgfqpoint{2.843057in}{3.597386in}}%
\pgfpathcurveto{\pgfqpoint{2.850189in}{3.604519in}}{\pgfqpoint{2.854197in}{3.614194in}}{\pgfqpoint{2.854197in}{3.624282in}}%
\pgfpathcurveto{\pgfqpoint{2.854197in}{3.634369in}}{\pgfqpoint{2.850189in}{3.644045in}}{\pgfqpoint{2.843057in}{3.651178in}}%
\pgfpathcurveto{\pgfqpoint{2.835924in}{3.658310in}}{\pgfqpoint{2.826248in}{3.662318in}}{\pgfqpoint{2.816161in}{3.662318in}}%
\pgfpathcurveto{\pgfqpoint{2.806073in}{3.662318in}}{\pgfqpoint{2.796398in}{3.658310in}}{\pgfqpoint{2.789265in}{3.651178in}}%
\pgfpathcurveto{\pgfqpoint{2.782132in}{3.644045in}}{\pgfqpoint{2.778125in}{3.634369in}}{\pgfqpoint{2.778125in}{3.624282in}}%
\pgfpathcurveto{\pgfqpoint{2.778125in}{3.614194in}}{\pgfqpoint{2.782132in}{3.604519in}}{\pgfqpoint{2.789265in}{3.597386in}}%
\pgfpathcurveto{\pgfqpoint{2.796398in}{3.590253in}}{\pgfqpoint{2.806073in}{3.586246in}}{\pgfqpoint{2.816161in}{3.586246in}}%
\pgfpathclose%
\pgfusepath{stroke,fill}%
\end{pgfscope}%
\begin{pgfscope}%
\pgfpathrectangle{\pgfqpoint{1.280114in}{0.528000in}}{\pgfqpoint{3.487886in}{3.696000in}} %
\pgfusepath{clip}%
\pgfsetbuttcap%
\pgfsetroundjoin%
\definecolor{currentfill}{rgb}{1.000000,0.498039,0.054902}%
\pgfsetfillcolor{currentfill}%
\pgfsetlinewidth{1.003750pt}%
\definecolor{currentstroke}{rgb}{1.000000,0.498039,0.054902}%
\pgfsetstrokecolor{currentstroke}%
\pgfsetdash{}{0pt}%
\pgfpathmoveto{\pgfqpoint{3.150558in}{0.993407in}}%
\pgfpathcurveto{\pgfqpoint{3.160645in}{0.993407in}}{\pgfqpoint{3.170321in}{0.997415in}}{\pgfqpoint{3.177454in}{1.004547in}}%
\pgfpathcurveto{\pgfqpoint{3.184586in}{1.011680in}}{\pgfqpoint{3.188594in}{1.021356in}}{\pgfqpoint{3.188594in}{1.031443in}}%
\pgfpathcurveto{\pgfqpoint{3.188594in}{1.041531in}}{\pgfqpoint{3.184586in}{1.051206in}}{\pgfqpoint{3.177454in}{1.058339in}}%
\pgfpathcurveto{\pgfqpoint{3.170321in}{1.065472in}}{\pgfqpoint{3.160645in}{1.069480in}}{\pgfqpoint{3.150558in}{1.069480in}}%
\pgfpathcurveto{\pgfqpoint{3.140471in}{1.069480in}}{\pgfqpoint{3.130795in}{1.065472in}}{\pgfqpoint{3.123662in}{1.058339in}}%
\pgfpathcurveto{\pgfqpoint{3.116529in}{1.051206in}}{\pgfqpoint{3.112522in}{1.041531in}}{\pgfqpoint{3.112522in}{1.031443in}}%
\pgfpathcurveto{\pgfqpoint{3.112522in}{1.021356in}}{\pgfqpoint{3.116529in}{1.011680in}}{\pgfqpoint{3.123662in}{1.004547in}}%
\pgfpathcurveto{\pgfqpoint{3.130795in}{0.997415in}}{\pgfqpoint{3.140471in}{0.993407in}}{\pgfqpoint{3.150558in}{0.993407in}}%
\pgfpathclose%
\pgfusepath{stroke,fill}%
\end{pgfscope}%
\begin{pgfscope}%
\pgfpathrectangle{\pgfqpoint{1.280114in}{0.528000in}}{\pgfqpoint{3.487886in}{3.696000in}} %
\pgfusepath{clip}%
\pgfsetbuttcap%
\pgfsetroundjoin%
\definecolor{currentfill}{rgb}{1.000000,0.498039,0.054902}%
\pgfsetfillcolor{currentfill}%
\pgfsetlinewidth{1.003750pt}%
\definecolor{currentstroke}{rgb}{1.000000,0.498039,0.054902}%
\pgfsetstrokecolor{currentstroke}%
\pgfsetdash{}{0pt}%
\pgfpathmoveto{\pgfqpoint{2.684391in}{2.966645in}}%
\pgfpathcurveto{\pgfqpoint{2.694478in}{2.966645in}}{\pgfqpoint{2.704154in}{2.970653in}}{\pgfqpoint{2.711287in}{2.977786in}}%
\pgfpathcurveto{\pgfqpoint{2.718419in}{2.984919in}}{\pgfqpoint{2.722427in}{2.994594in}}{\pgfqpoint{2.722427in}{3.004682in}}%
\pgfpathcurveto{\pgfqpoint{2.722427in}{3.014769in}}{\pgfqpoint{2.718419in}{3.024445in}}{\pgfqpoint{2.711287in}{3.031577in}}%
\pgfpathcurveto{\pgfqpoint{2.704154in}{3.038710in}}{\pgfqpoint{2.694478in}{3.042718in}}{\pgfqpoint{2.684391in}{3.042718in}}%
\pgfpathcurveto{\pgfqpoint{2.674303in}{3.042718in}}{\pgfqpoint{2.664628in}{3.038710in}}{\pgfqpoint{2.657495in}{3.031577in}}%
\pgfpathcurveto{\pgfqpoint{2.650362in}{3.024445in}}{\pgfqpoint{2.646355in}{3.014769in}}{\pgfqpoint{2.646355in}{3.004682in}}%
\pgfpathcurveto{\pgfqpoint{2.646355in}{2.994594in}}{\pgfqpoint{2.650362in}{2.984919in}}{\pgfqpoint{2.657495in}{2.977786in}}%
\pgfpathcurveto{\pgfqpoint{2.664628in}{2.970653in}}{\pgfqpoint{2.674303in}{2.966645in}}{\pgfqpoint{2.684391in}{2.966645in}}%
\pgfpathclose%
\pgfusepath{stroke,fill}%
\end{pgfscope}%
\begin{pgfscope}%
\pgfpathrectangle{\pgfqpoint{1.280114in}{0.528000in}}{\pgfqpoint{3.487886in}{3.696000in}} %
\pgfusepath{clip}%
\pgfsetbuttcap%
\pgfsetroundjoin%
\definecolor{currentfill}{rgb}{1.000000,0.498039,0.054902}%
\pgfsetfillcolor{currentfill}%
\pgfsetlinewidth{1.003750pt}%
\definecolor{currentstroke}{rgb}{1.000000,0.498039,0.054902}%
\pgfsetstrokecolor{currentstroke}%
\pgfsetdash{}{0pt}%
\pgfpathmoveto{\pgfqpoint{3.103427in}{0.997495in}}%
\pgfpathcurveto{\pgfqpoint{3.113515in}{0.997495in}}{\pgfqpoint{3.123190in}{1.001503in}}{\pgfqpoint{3.130323in}{1.008636in}}%
\pgfpathcurveto{\pgfqpoint{3.137456in}{1.015768in}}{\pgfqpoint{3.141463in}{1.025444in}}{\pgfqpoint{3.141463in}{1.035531in}}%
\pgfpathcurveto{\pgfqpoint{3.141463in}{1.045619in}}{\pgfqpoint{3.137456in}{1.055294in}}{\pgfqpoint{3.130323in}{1.062427in}}%
\pgfpathcurveto{\pgfqpoint{3.123190in}{1.069560in}}{\pgfqpoint{3.113515in}{1.073568in}}{\pgfqpoint{3.103427in}{1.073568in}}%
\pgfpathcurveto{\pgfqpoint{3.093340in}{1.073568in}}{\pgfqpoint{3.083664in}{1.069560in}}{\pgfqpoint{3.076531in}{1.062427in}}%
\pgfpathcurveto{\pgfqpoint{3.069399in}{1.055294in}}{\pgfqpoint{3.065391in}{1.045619in}}{\pgfqpoint{3.065391in}{1.035531in}}%
\pgfpathcurveto{\pgfqpoint{3.065391in}{1.025444in}}{\pgfqpoint{3.069399in}{1.015768in}}{\pgfqpoint{3.076531in}{1.008636in}}%
\pgfpathcurveto{\pgfqpoint{3.083664in}{1.001503in}}{\pgfqpoint{3.093340in}{0.997495in}}{\pgfqpoint{3.103427in}{0.997495in}}%
\pgfpathclose%
\pgfusepath{stroke,fill}%
\end{pgfscope}%
\begin{pgfscope}%
\pgfpathrectangle{\pgfqpoint{1.280114in}{0.528000in}}{\pgfqpoint{3.487886in}{3.696000in}} %
\pgfusepath{clip}%
\pgfsetbuttcap%
\pgfsetroundjoin%
\definecolor{currentfill}{rgb}{1.000000,0.498039,0.054902}%
\pgfsetfillcolor{currentfill}%
\pgfsetlinewidth{1.003750pt}%
\definecolor{currentstroke}{rgb}{1.000000,0.498039,0.054902}%
\pgfsetstrokecolor{currentstroke}%
\pgfsetdash{}{0pt}%
\pgfpathmoveto{\pgfqpoint{3.392126in}{2.464512in}}%
\pgfpathcurveto{\pgfqpoint{3.402214in}{2.464512in}}{\pgfqpoint{3.411889in}{2.468520in}}{\pgfqpoint{3.419022in}{2.475653in}}%
\pgfpathcurveto{\pgfqpoint{3.426155in}{2.482785in}}{\pgfqpoint{3.430163in}{2.492461in}}{\pgfqpoint{3.430163in}{2.502548in}}%
\pgfpathcurveto{\pgfqpoint{3.430163in}{2.512636in}}{\pgfqpoint{3.426155in}{2.522311in}}{\pgfqpoint{3.419022in}{2.529444in}}%
\pgfpathcurveto{\pgfqpoint{3.411889in}{2.536577in}}{\pgfqpoint{3.402214in}{2.540585in}}{\pgfqpoint{3.392126in}{2.540585in}}%
\pgfpathcurveto{\pgfqpoint{3.382039in}{2.540585in}}{\pgfqpoint{3.372364in}{2.536577in}}{\pgfqpoint{3.365231in}{2.529444in}}%
\pgfpathcurveto{\pgfqpoint{3.358098in}{2.522311in}}{\pgfqpoint{3.354090in}{2.512636in}}{\pgfqpoint{3.354090in}{2.502548in}}%
\pgfpathcurveto{\pgfqpoint{3.354090in}{2.492461in}}{\pgfqpoint{3.358098in}{2.482785in}}{\pgfqpoint{3.365231in}{2.475653in}}%
\pgfpathcurveto{\pgfqpoint{3.372364in}{2.468520in}}{\pgfqpoint{3.382039in}{2.464512in}}{\pgfqpoint{3.392126in}{2.464512in}}%
\pgfpathclose%
\pgfusepath{stroke,fill}%
\end{pgfscope}%
\begin{pgfscope}%
\pgfpathrectangle{\pgfqpoint{1.280114in}{0.528000in}}{\pgfqpoint{3.487886in}{3.696000in}} %
\pgfusepath{clip}%
\pgfsetbuttcap%
\pgfsetroundjoin%
\definecolor{currentfill}{rgb}{1.000000,0.498039,0.054902}%
\pgfsetfillcolor{currentfill}%
\pgfsetlinewidth{1.003750pt}%
\definecolor{currentstroke}{rgb}{1.000000,0.498039,0.054902}%
\pgfsetstrokecolor{currentstroke}%
\pgfsetdash{}{0pt}%
\pgfpathmoveto{\pgfqpoint{2.909551in}{3.823670in}}%
\pgfpathcurveto{\pgfqpoint{2.919638in}{3.823670in}}{\pgfqpoint{2.929314in}{3.827678in}}{\pgfqpoint{2.936447in}{3.834811in}}%
\pgfpathcurveto{\pgfqpoint{2.943580in}{3.841943in}}{\pgfqpoint{2.947587in}{3.851619in}}{\pgfqpoint{2.947587in}{3.861706in}}%
\pgfpathcurveto{\pgfqpoint{2.947587in}{3.871794in}}{\pgfqpoint{2.943580in}{3.881469in}}{\pgfqpoint{2.936447in}{3.888602in}}%
\pgfpathcurveto{\pgfqpoint{2.929314in}{3.895735in}}{\pgfqpoint{2.919638in}{3.899743in}}{\pgfqpoint{2.909551in}{3.899743in}}%
\pgfpathcurveto{\pgfqpoint{2.899464in}{3.899743in}}{\pgfqpoint{2.889788in}{3.895735in}}{\pgfqpoint{2.882655in}{3.888602in}}%
\pgfpathcurveto{\pgfqpoint{2.875522in}{3.881469in}}{\pgfqpoint{2.871515in}{3.871794in}}{\pgfqpoint{2.871515in}{3.861706in}}%
\pgfpathcurveto{\pgfqpoint{2.871515in}{3.851619in}}{\pgfqpoint{2.875522in}{3.841943in}}{\pgfqpoint{2.882655in}{3.834811in}}%
\pgfpathcurveto{\pgfqpoint{2.889788in}{3.827678in}}{\pgfqpoint{2.899464in}{3.823670in}}{\pgfqpoint{2.909551in}{3.823670in}}%
\pgfpathclose%
\pgfusepath{stroke,fill}%
\end{pgfscope}%
\begin{pgfscope}%
\pgfpathrectangle{\pgfqpoint{1.280114in}{0.528000in}}{\pgfqpoint{3.487886in}{3.696000in}} %
\pgfusepath{clip}%
\pgfsetbuttcap%
\pgfsetroundjoin%
\definecolor{currentfill}{rgb}{1.000000,0.498039,0.054902}%
\pgfsetfillcolor{currentfill}%
\pgfsetlinewidth{1.003750pt}%
\definecolor{currentstroke}{rgb}{1.000000,0.498039,0.054902}%
\pgfsetstrokecolor{currentstroke}%
\pgfsetdash{}{0pt}%
\pgfpathmoveto{\pgfqpoint{3.295101in}{3.349991in}}%
\pgfpathcurveto{\pgfqpoint{3.305189in}{3.349991in}}{\pgfqpoint{3.314864in}{3.353999in}}{\pgfqpoint{3.321997in}{3.361131in}}%
\pgfpathcurveto{\pgfqpoint{3.329130in}{3.368264in}}{\pgfqpoint{3.333138in}{3.377940in}}{\pgfqpoint{3.333138in}{3.388027in}}%
\pgfpathcurveto{\pgfqpoint{3.333138in}{3.398114in}}{\pgfqpoint{3.329130in}{3.407790in}}{\pgfqpoint{3.321997in}{3.414923in}}%
\pgfpathcurveto{\pgfqpoint{3.314864in}{3.422056in}}{\pgfqpoint{3.305189in}{3.426063in}}{\pgfqpoint{3.295101in}{3.426063in}}%
\pgfpathcurveto{\pgfqpoint{3.285014in}{3.426063in}}{\pgfqpoint{3.275338in}{3.422056in}}{\pgfqpoint{3.268206in}{3.414923in}}%
\pgfpathcurveto{\pgfqpoint{3.261073in}{3.407790in}}{\pgfqpoint{3.257065in}{3.398114in}}{\pgfqpoint{3.257065in}{3.388027in}}%
\pgfpathcurveto{\pgfqpoint{3.257065in}{3.377940in}}{\pgfqpoint{3.261073in}{3.368264in}}{\pgfqpoint{3.268206in}{3.361131in}}%
\pgfpathcurveto{\pgfqpoint{3.275338in}{3.353999in}}{\pgfqpoint{3.285014in}{3.349991in}}{\pgfqpoint{3.295101in}{3.349991in}}%
\pgfpathclose%
\pgfusepath{stroke,fill}%
\end{pgfscope}%
\begin{pgfscope}%
\pgfpathrectangle{\pgfqpoint{1.280114in}{0.528000in}}{\pgfqpoint{3.487886in}{3.696000in}} %
\pgfusepath{clip}%
\pgfsetbuttcap%
\pgfsetroundjoin%
\definecolor{currentfill}{rgb}{1.000000,0.498039,0.054902}%
\pgfsetfillcolor{currentfill}%
\pgfsetlinewidth{1.003750pt}%
\definecolor{currentstroke}{rgb}{1.000000,0.498039,0.054902}%
\pgfsetstrokecolor{currentstroke}%
\pgfsetdash{}{0pt}%
\pgfpathmoveto{\pgfqpoint{3.290367in}{1.381244in}}%
\pgfpathcurveto{\pgfqpoint{3.300454in}{1.381244in}}{\pgfqpoint{3.310130in}{1.385252in}}{\pgfqpoint{3.317262in}{1.392385in}}%
\pgfpathcurveto{\pgfqpoint{3.324395in}{1.399518in}}{\pgfqpoint{3.328403in}{1.409193in}}{\pgfqpoint{3.328403in}{1.419281in}}%
\pgfpathcurveto{\pgfqpoint{3.328403in}{1.429368in}}{\pgfqpoint{3.324395in}{1.439044in}}{\pgfqpoint{3.317262in}{1.446176in}}%
\pgfpathcurveto{\pgfqpoint{3.310130in}{1.453309in}}{\pgfqpoint{3.300454in}{1.457317in}}{\pgfqpoint{3.290367in}{1.457317in}}%
\pgfpathcurveto{\pgfqpoint{3.280279in}{1.457317in}}{\pgfqpoint{3.270604in}{1.453309in}}{\pgfqpoint{3.263471in}{1.446176in}}%
\pgfpathcurveto{\pgfqpoint{3.256338in}{1.439044in}}{\pgfqpoint{3.252330in}{1.429368in}}{\pgfqpoint{3.252330in}{1.419281in}}%
\pgfpathcurveto{\pgfqpoint{3.252330in}{1.409193in}}{\pgfqpoint{3.256338in}{1.399518in}}{\pgfqpoint{3.263471in}{1.392385in}}%
\pgfpathcurveto{\pgfqpoint{3.270604in}{1.385252in}}{\pgfqpoint{3.280279in}{1.381244in}}{\pgfqpoint{3.290367in}{1.381244in}}%
\pgfpathclose%
\pgfusepath{stroke,fill}%
\end{pgfscope}%
\begin{pgfscope}%
\pgfpathrectangle{\pgfqpoint{1.280114in}{0.528000in}}{\pgfqpoint{3.487886in}{3.696000in}} %
\pgfusepath{clip}%
\pgfsetbuttcap%
\pgfsetroundjoin%
\definecolor{currentfill}{rgb}{1.000000,0.498039,0.054902}%
\pgfsetfillcolor{currentfill}%
\pgfsetlinewidth{1.003750pt}%
\definecolor{currentstroke}{rgb}{1.000000,0.498039,0.054902}%
\pgfsetstrokecolor{currentstroke}%
\pgfsetdash{}{0pt}%
\pgfpathmoveto{\pgfqpoint{2.690169in}{2.124005in}}%
\pgfpathcurveto{\pgfqpoint{2.700256in}{2.124005in}}{\pgfqpoint{2.709931in}{2.128012in}}{\pgfqpoint{2.717064in}{2.135145in}}%
\pgfpathcurveto{\pgfqpoint{2.724197in}{2.142278in}}{\pgfqpoint{2.728205in}{2.151954in}}{\pgfqpoint{2.728205in}{2.162041in}}%
\pgfpathcurveto{\pgfqpoint{2.728205in}{2.172128in}}{\pgfqpoint{2.724197in}{2.181804in}}{\pgfqpoint{2.717064in}{2.188937in}}%
\pgfpathcurveto{\pgfqpoint{2.709931in}{2.196069in}}{\pgfqpoint{2.700256in}{2.200077in}}{\pgfqpoint{2.690169in}{2.200077in}}%
\pgfpathcurveto{\pgfqpoint{2.680081in}{2.200077in}}{\pgfqpoint{2.670406in}{2.196069in}}{\pgfqpoint{2.663273in}{2.188937in}}%
\pgfpathcurveto{\pgfqpoint{2.656140in}{2.181804in}}{\pgfqpoint{2.652132in}{2.172128in}}{\pgfqpoint{2.652132in}{2.162041in}}%
\pgfpathcurveto{\pgfqpoint{2.652132in}{2.151954in}}{\pgfqpoint{2.656140in}{2.142278in}}{\pgfqpoint{2.663273in}{2.135145in}}%
\pgfpathcurveto{\pgfqpoint{2.670406in}{2.128012in}}{\pgfqpoint{2.680081in}{2.124005in}}{\pgfqpoint{2.690169in}{2.124005in}}%
\pgfpathclose%
\pgfusepath{stroke,fill}%
\end{pgfscope}%
\begin{pgfscope}%
\pgfpathrectangle{\pgfqpoint{1.280114in}{0.528000in}}{\pgfqpoint{3.487886in}{3.696000in}} %
\pgfusepath{clip}%
\pgfsetbuttcap%
\pgfsetroundjoin%
\definecolor{currentfill}{rgb}{1.000000,0.498039,0.054902}%
\pgfsetfillcolor{currentfill}%
\pgfsetlinewidth{1.003750pt}%
\definecolor{currentstroke}{rgb}{1.000000,0.498039,0.054902}%
\pgfsetstrokecolor{currentstroke}%
\pgfsetdash{}{0pt}%
\pgfpathmoveto{\pgfqpoint{3.399567in}{2.096147in}}%
\pgfpathcurveto{\pgfqpoint{3.409655in}{2.096147in}}{\pgfqpoint{3.419330in}{2.100154in}}{\pgfqpoint{3.426463in}{2.107287in}}%
\pgfpathcurveto{\pgfqpoint{3.433596in}{2.114420in}}{\pgfqpoint{3.437604in}{2.124096in}}{\pgfqpoint{3.437604in}{2.134183in}}%
\pgfpathcurveto{\pgfqpoint{3.437604in}{2.144270in}}{\pgfqpoint{3.433596in}{2.153946in}}{\pgfqpoint{3.426463in}{2.161079in}}%
\pgfpathcurveto{\pgfqpoint{3.419330in}{2.168211in}}{\pgfqpoint{3.409655in}{2.172219in}}{\pgfqpoint{3.399567in}{2.172219in}}%
\pgfpathcurveto{\pgfqpoint{3.389480in}{2.172219in}}{\pgfqpoint{3.379804in}{2.168211in}}{\pgfqpoint{3.372672in}{2.161079in}}%
\pgfpathcurveto{\pgfqpoint{3.365539in}{2.153946in}}{\pgfqpoint{3.361531in}{2.144270in}}{\pgfqpoint{3.361531in}{2.134183in}}%
\pgfpathcurveto{\pgfqpoint{3.361531in}{2.124096in}}{\pgfqpoint{3.365539in}{2.114420in}}{\pgfqpoint{3.372672in}{2.107287in}}%
\pgfpathcurveto{\pgfqpoint{3.379804in}{2.100154in}}{\pgfqpoint{3.389480in}{2.096147in}}{\pgfqpoint{3.399567in}{2.096147in}}%
\pgfpathclose%
\pgfusepath{stroke,fill}%
\end{pgfscope}%
\begin{pgfscope}%
\pgfpathrectangle{\pgfqpoint{1.280114in}{0.528000in}}{\pgfqpoint{3.487886in}{3.696000in}} %
\pgfusepath{clip}%
\pgfsetbuttcap%
\pgfsetroundjoin%
\definecolor{currentfill}{rgb}{1.000000,0.498039,0.054902}%
\pgfsetfillcolor{currentfill}%
\pgfsetlinewidth{1.003750pt}%
\definecolor{currentstroke}{rgb}{1.000000,0.498039,0.054902}%
\pgfsetstrokecolor{currentstroke}%
\pgfsetdash{}{0pt}%
\pgfpathmoveto{\pgfqpoint{2.815924in}{1.282908in}}%
\pgfpathcurveto{\pgfqpoint{2.826012in}{1.282908in}}{\pgfqpoint{2.835687in}{1.286916in}}{\pgfqpoint{2.842820in}{1.294049in}}%
\pgfpathcurveto{\pgfqpoint{2.849953in}{1.301182in}}{\pgfqpoint{2.853961in}{1.310857in}}{\pgfqpoint{2.853961in}{1.320944in}}%
\pgfpathcurveto{\pgfqpoint{2.853961in}{1.331032in}}{\pgfqpoint{2.849953in}{1.340707in}}{\pgfqpoint{2.842820in}{1.347840in}}%
\pgfpathcurveto{\pgfqpoint{2.835687in}{1.354973in}}{\pgfqpoint{2.826012in}{1.358981in}}{\pgfqpoint{2.815924in}{1.358981in}}%
\pgfpathcurveto{\pgfqpoint{2.805837in}{1.358981in}}{\pgfqpoint{2.796162in}{1.354973in}}{\pgfqpoint{2.789029in}{1.347840in}}%
\pgfpathcurveto{\pgfqpoint{2.781896in}{1.340707in}}{\pgfqpoint{2.777888in}{1.331032in}}{\pgfqpoint{2.777888in}{1.320944in}}%
\pgfpathcurveto{\pgfqpoint{2.777888in}{1.310857in}}{\pgfqpoint{2.781896in}{1.301182in}}{\pgfqpoint{2.789029in}{1.294049in}}%
\pgfpathcurveto{\pgfqpoint{2.796162in}{1.286916in}}{\pgfqpoint{2.805837in}{1.282908in}}{\pgfqpoint{2.815924in}{1.282908in}}%
\pgfpathclose%
\pgfusepath{stroke,fill}%
\end{pgfscope}%
\begin{pgfscope}%
\pgfpathrectangle{\pgfqpoint{1.280114in}{0.528000in}}{\pgfqpoint{3.487886in}{3.696000in}} %
\pgfusepath{clip}%
\pgfsetbuttcap%
\pgfsetroundjoin%
\definecolor{currentfill}{rgb}{0.119483,0.614817,0.537692}%
\pgfsetfillcolor{currentfill}%
\pgfsetlinewidth{0.000000pt}%
\definecolor{currentstroke}{rgb}{0.000000,0.000000,0.000000}%
\pgfsetstrokecolor{currentstroke}%
\pgfsetdash{}{0pt}%
\pgfpathmoveto{\pgfqpoint{1.782164in}{2.621060in}}%
\pgfpathlineto{\pgfqpoint{2.591775in}{2.634925in}}%
\pgfpathlineto{\pgfqpoint{2.579677in}{2.658119in}}%
\pgfpathlineto{\pgfqpoint{2.697248in}{2.625031in}}%
\pgfpathlineto{\pgfqpoint{2.580879in}{2.587936in}}%
\pgfpathlineto{\pgfqpoint{2.592175in}{2.611531in}}%
\pgfpathlineto{\pgfqpoint{1.782565in}{2.597666in}}%
\pgfpathlineto{\pgfqpoint{1.782164in}{2.621060in}}%
\pgfusepath{fill}%
\end{pgfscope}%
\begin{pgfscope}%
\pgfpathrectangle{\pgfqpoint{1.280114in}{0.528000in}}{\pgfqpoint{3.487886in}{3.696000in}} %
\pgfusepath{clip}%
\pgfsetbuttcap%
\pgfsetroundjoin%
\definecolor{currentfill}{rgb}{0.993248,0.906157,0.143936}%
\pgfsetfillcolor{currentfill}%
\pgfsetlinewidth{0.000000pt}%
\definecolor{currentstroke}{rgb}{0.000000,0.000000,0.000000}%
\pgfsetstrokecolor{currentstroke}%
\pgfsetdash{}{0pt}%
\pgfpathmoveto{\pgfqpoint{2.868689in}{2.042029in}}%
\pgfpathlineto{\pgfqpoint{3.015437in}{1.151984in}}%
\pgfpathlineto{\pgfqpoint{3.036620in}{1.167333in}}%
\pgfpathlineto{\pgfqpoint{3.021023in}{1.046194in}}%
\pgfpathlineto{\pgfqpoint{2.967363in}{1.155914in}}%
\pgfpathlineto{\pgfqpoint{2.992352in}{1.148177in}}%
\pgfpathlineto{\pgfqpoint{2.845603in}{2.038223in}}%
\pgfpathlineto{\pgfqpoint{2.868689in}{2.042029in}}%
\pgfusepath{fill}%
\end{pgfscope}%
\begin{pgfscope}%
\pgfpathrectangle{\pgfqpoint{1.280114in}{0.528000in}}{\pgfqpoint{3.487886in}{3.696000in}} %
\pgfusepath{clip}%
\pgfsetbuttcap%
\pgfsetroundjoin%
\definecolor{currentfill}{rgb}{0.246811,0.283237,0.535941}%
\pgfsetfillcolor{currentfill}%
\pgfsetlinewidth{0.000000pt}%
\definecolor{currentstroke}{rgb}{0.000000,0.000000,0.000000}%
\pgfsetstrokecolor{currentstroke}%
\pgfsetdash{}{0pt}%
\pgfpathmoveto{\pgfqpoint{4.451631in}{2.389695in}}%
\pgfpathlineto{\pgfqpoint{3.337337in}{1.513425in}}%
\pgfpathlineto{\pgfqpoint{3.360996in}{1.502265in}}%
\pgfpathlineto{\pgfqpoint{3.247342in}{1.457537in}}%
\pgfpathlineto{\pgfqpoint{3.317607in}{1.557440in}}%
\pgfpathlineto{\pgfqpoint{3.322874in}{1.531817in}}%
\pgfpathlineto{\pgfqpoint{4.437168in}{2.408087in}}%
\pgfpathlineto{\pgfqpoint{4.451631in}{2.389695in}}%
\pgfusepath{fill}%
\end{pgfscope}%
\begin{pgfscope}%
\pgfpathrectangle{\pgfqpoint{1.280114in}{0.528000in}}{\pgfqpoint{3.487886in}{3.696000in}} %
\pgfusepath{clip}%
\pgfsetbuttcap%
\pgfsetroundjoin%
\definecolor{currentfill}{rgb}{0.993248,0.906157,0.143936}%
\pgfsetfillcolor{currentfill}%
\pgfsetlinewidth{0.000000pt}%
\definecolor{currentstroke}{rgb}{0.000000,0.000000,0.000000}%
\pgfsetstrokecolor{currentstroke}%
\pgfsetdash{}{0pt}%
\pgfpathmoveto{\pgfqpoint{2.680118in}{2.754272in}}%
\pgfpathlineto{\pgfqpoint{2.877780in}{3.760646in}}%
\pgfpathlineto{\pgfqpoint{2.852566in}{3.753676in}}%
\pgfpathlineto{\pgfqpoint{2.909551in}{3.861706in}}%
\pgfpathlineto{\pgfqpoint{2.921443in}{3.740148in}}%
\pgfpathlineto{\pgfqpoint{2.900738in}{3.756137in}}%
\pgfpathlineto{\pgfqpoint{2.703077in}{2.749762in}}%
\pgfpathlineto{\pgfqpoint{2.680118in}{2.754272in}}%
\pgfusepath{fill}%
\end{pgfscope}%
\begin{pgfscope}%
\pgfpathrectangle{\pgfqpoint{1.280114in}{0.528000in}}{\pgfqpoint{3.487886in}{3.696000in}} %
\pgfusepath{clip}%
\pgfsetbuttcap%
\pgfsetroundjoin%
\definecolor{currentfill}{rgb}{0.993248,0.906157,0.143936}%
\pgfsetfillcolor{currentfill}%
\pgfsetlinewidth{0.000000pt}%
\definecolor{currentstroke}{rgb}{0.000000,0.000000,0.000000}%
\pgfsetstrokecolor{currentstroke}%
\pgfsetdash{}{0pt}%
\pgfpathmoveto{\pgfqpoint{3.980335in}{2.715245in}}%
\pgfpathlineto{\pgfqpoint{3.260494in}{3.735036in}}%
\pgfpathlineto{\pgfqpoint{3.248125in}{3.711985in}}%
\pgfpathlineto{\pgfqpoint{3.209334in}{3.827800in}}%
\pgfpathlineto{\pgfqpoint{3.305470in}{3.752464in}}%
\pgfpathlineto{\pgfqpoint{3.279609in}{3.748528in}}%
\pgfpathlineto{\pgfqpoint{3.999450in}{2.728737in}}%
\pgfpathlineto{\pgfqpoint{3.980335in}{2.715245in}}%
\pgfusepath{fill}%
\end{pgfscope}%
\begin{pgfscope}%
\pgfpathrectangle{\pgfqpoint{1.280114in}{0.528000in}}{\pgfqpoint{3.487886in}{3.696000in}} %
\pgfusepath{clip}%
\pgfsetbuttcap%
\pgfsetroundjoin%
\definecolor{currentfill}{rgb}{0.993248,0.906157,0.143936}%
\pgfsetfillcolor{currentfill}%
\pgfsetlinewidth{0.000000pt}%
\definecolor{currentstroke}{rgb}{0.000000,0.000000,0.000000}%
\pgfsetstrokecolor{currentstroke}%
\pgfsetdash{}{0pt}%
\pgfpathmoveto{\pgfqpoint{4.244070in}{2.616203in}}%
\pgfpathlineto{\pgfqpoint{3.457347in}{3.020965in}}%
\pgfpathlineto{\pgfqpoint{3.457045in}{2.994807in}}%
\pgfpathlineto{\pgfqpoint{3.369075in}{3.079536in}}%
\pgfpathlineto{\pgfqpoint{3.489158in}{3.057223in}}%
\pgfpathlineto{\pgfqpoint{3.468051in}{3.041770in}}%
\pgfpathlineto{\pgfqpoint{4.254774in}{2.637008in}}%
\pgfpathlineto{\pgfqpoint{4.244070in}{2.616203in}}%
\pgfusepath{fill}%
\end{pgfscope}%
\begin{pgfscope}%
\pgfpathrectangle{\pgfqpoint{1.280114in}{0.528000in}}{\pgfqpoint{3.487886in}{3.696000in}} %
\pgfusepath{clip}%
\pgfsetbuttcap%
\pgfsetroundjoin%
\definecolor{currentfill}{rgb}{0.974417,0.903590,0.130215}%
\pgfsetfillcolor{currentfill}%
\pgfsetlinewidth{0.000000pt}%
\definecolor{currentstroke}{rgb}{0.000000,0.000000,0.000000}%
\pgfsetstrokecolor{currentstroke}%
\pgfsetdash{}{0pt}%
\pgfpathmoveto{\pgfqpoint{3.658658in}{2.754223in}}%
\pgfpathlineto{\pgfqpoint{3.111276in}{3.949015in}}%
\pgfpathlineto{\pgfqpoint{3.094877in}{3.928634in}}%
\pgfpathlineto{\pgfqpoint{3.078058in}{4.049609in}}%
\pgfpathlineto{\pgfqpoint{3.158691in}{3.957870in}}%
\pgfpathlineto{\pgfqpoint{3.132547in}{3.958760in}}%
\pgfpathlineto{\pgfqpoint{3.679930in}{2.763969in}}%
\pgfpathlineto{\pgfqpoint{3.658658in}{2.754223in}}%
\pgfusepath{fill}%
\end{pgfscope}%
\begin{pgfscope}%
\pgfpathrectangle{\pgfqpoint{1.280114in}{0.528000in}}{\pgfqpoint{3.487886in}{3.696000in}} %
\pgfusepath{clip}%
\pgfsetbuttcap%
\pgfsetroundjoin%
\definecolor{currentfill}{rgb}{0.993248,0.906157,0.143936}%
\pgfsetfillcolor{currentfill}%
\pgfsetlinewidth{0.000000pt}%
\definecolor{currentstroke}{rgb}{0.000000,0.000000,0.000000}%
\pgfsetstrokecolor{currentstroke}%
\pgfsetdash{}{0pt}%
\pgfpathmoveto{\pgfqpoint{2.183246in}{2.719886in}}%
\pgfpathlineto{\pgfqpoint{2.747054in}{3.543990in}}%
\pgfpathlineto{\pgfqpoint{2.721138in}{3.547546in}}%
\pgfpathlineto{\pgfqpoint{2.816161in}{3.624282in}}%
\pgfpathlineto{\pgfqpoint{2.779070in}{3.507912in}}%
\pgfpathlineto{\pgfqpoint{2.766365in}{3.530778in}}%
\pgfpathlineto{\pgfqpoint{2.202557in}{2.706675in}}%
\pgfpathlineto{\pgfqpoint{2.183246in}{2.719886in}}%
\pgfusepath{fill}%
\end{pgfscope}%
\begin{pgfscope}%
\pgfpathrectangle{\pgfqpoint{1.280114in}{0.528000in}}{\pgfqpoint{3.487886in}{3.696000in}} %
\pgfusepath{clip}%
\pgfsetbuttcap%
\pgfsetroundjoin%
\definecolor{currentfill}{rgb}{0.993248,0.906157,0.143936}%
\pgfsetfillcolor{currentfill}%
\pgfsetlinewidth{0.000000pt}%
\definecolor{currentstroke}{rgb}{0.000000,0.000000,0.000000}%
\pgfsetstrokecolor{currentstroke}%
\pgfsetdash{}{0pt}%
\pgfpathmoveto{\pgfqpoint{1.969981in}{2.640084in}}%
\pgfpathlineto{\pgfqpoint{2.616654in}{2.976836in}}%
\pgfpathlineto{\pgfqpoint{2.595471in}{2.992185in}}%
\pgfpathlineto{\pgfqpoint{2.715443in}{3.015090in}}%
\pgfpathlineto{\pgfqpoint{2.627891in}{2.929928in}}%
\pgfpathlineto{\pgfqpoint{2.627461in}{2.956084in}}%
\pgfpathlineto{\pgfqpoint{1.980788in}{2.619332in}}%
\pgfpathlineto{\pgfqpoint{1.969981in}{2.640084in}}%
\pgfusepath{fill}%
\end{pgfscope}%
\begin{pgfscope}%
\pgfpathrectangle{\pgfqpoint{1.280114in}{0.528000in}}{\pgfqpoint{3.487886in}{3.696000in}} %
\pgfusepath{clip}%
\pgfsetbuttcap%
\pgfsetroundjoin%
\definecolor{currentfill}{rgb}{0.993248,0.906157,0.143936}%
\pgfsetfillcolor{currentfill}%
\pgfsetlinewidth{0.000000pt}%
\definecolor{currentstroke}{rgb}{0.000000,0.000000,0.000000}%
\pgfsetstrokecolor{currentstroke}%
\pgfsetdash{}{0pt}%
\pgfpathmoveto{\pgfqpoint{1.763692in}{2.340689in}}%
\pgfpathlineto{\pgfqpoint{2.625594in}{1.926901in}}%
\pgfpathlineto{\pgfqpoint{2.625174in}{1.953057in}}%
\pgfpathlineto{\pgfqpoint{2.715448in}{1.870786in}}%
\pgfpathlineto{\pgfqpoint{2.594795in}{1.889779in}}%
\pgfpathlineto{\pgfqpoint{2.615468in}{1.905808in}}%
\pgfpathlineto{\pgfqpoint{1.753565in}{2.319596in}}%
\pgfpathlineto{\pgfqpoint{1.763692in}{2.340689in}}%
\pgfusepath{fill}%
\end{pgfscope}%
\begin{pgfscope}%
\pgfpathrectangle{\pgfqpoint{1.280114in}{0.528000in}}{\pgfqpoint{3.487886in}{3.696000in}} %
\pgfusepath{clip}%
\pgfsetbuttcap%
\pgfsetroundjoin%
\definecolor{currentfill}{rgb}{0.122606,0.585371,0.546557}%
\pgfsetfillcolor{currentfill}%
\pgfsetlinewidth{0.000000pt}%
\definecolor{currentstroke}{rgb}{0.000000,0.000000,0.000000}%
\pgfsetstrokecolor{currentstroke}%
\pgfsetdash{}{0pt}%
\pgfpathmoveto{\pgfqpoint{1.977269in}{2.600353in}}%
\pgfpathlineto{\pgfqpoint{2.598907in}{2.271128in}}%
\pgfpathlineto{\pgfqpoint{2.599520in}{2.297280in}}%
\pgfpathlineto{\pgfqpoint{2.686477in}{2.211513in}}%
\pgfpathlineto{\pgfqpoint{2.566668in}{2.235250in}}%
\pgfpathlineto{\pgfqpoint{2.587957in}{2.250452in}}%
\pgfpathlineto{\pgfqpoint{1.966318in}{2.579677in}}%
\pgfpathlineto{\pgfqpoint{1.977269in}{2.600353in}}%
\pgfusepath{fill}%
\end{pgfscope}%
\begin{pgfscope}%
\pgfpathrectangle{\pgfqpoint{1.280114in}{0.528000in}}{\pgfqpoint{3.487886in}{3.696000in}} %
\pgfusepath{clip}%
\pgfsetbuttcap%
\pgfsetroundjoin%
\definecolor{currentfill}{rgb}{0.993248,0.906157,0.143936}%
\pgfsetfillcolor{currentfill}%
\pgfsetlinewidth{0.000000pt}%
\definecolor{currentstroke}{rgb}{0.000000,0.000000,0.000000}%
\pgfsetstrokecolor{currentstroke}%
\pgfsetdash{}{0pt}%
\pgfpathmoveto{\pgfqpoint{2.315174in}{2.086918in}}%
\pgfpathlineto{\pgfqpoint{2.757000in}{1.383567in}}%
\pgfpathlineto{\pgfqpoint{2.770589in}{1.405920in}}%
\pgfpathlineto{\pgfqpoint{2.803099in}{1.288187in}}%
\pgfpathlineto{\pgfqpoint{2.711151in}{1.368582in}}%
\pgfpathlineto{\pgfqpoint{2.737187in}{1.371122in}}%
\pgfpathlineto{\pgfqpoint{2.295361in}{2.074472in}}%
\pgfpathlineto{\pgfqpoint{2.315174in}{2.086918in}}%
\pgfusepath{fill}%
\end{pgfscope}%
\begin{pgfscope}%
\pgfpathrectangle{\pgfqpoint{1.280114in}{0.528000in}}{\pgfqpoint{3.487886in}{3.696000in}} %
\pgfusepath{clip}%
\pgfsetbuttcap%
\pgfsetroundjoin%
\definecolor{currentfill}{rgb}{0.974417,0.903590,0.130215}%
\pgfsetfillcolor{currentfill}%
\pgfsetlinewidth{0.000000pt}%
\definecolor{currentstroke}{rgb}{0.000000,0.000000,0.000000}%
\pgfsetstrokecolor{currentstroke}%
\pgfsetdash{}{0pt}%
\pgfpathmoveto{\pgfqpoint{3.295403in}{2.748735in}}%
\pgfpathlineto{\pgfqpoint{3.071253in}{3.731078in}}%
\pgfpathlineto{\pgfqpoint{3.051044in}{3.714467in}}%
\pgfpathlineto{\pgfqpoint{3.059236in}{3.836330in}}%
\pgfpathlineto{\pgfqpoint{3.119477in}{3.730082in}}%
\pgfpathlineto{\pgfqpoint{3.094064in}{3.736283in}}%
\pgfpathlineto{\pgfqpoint{3.318214in}{2.753940in}}%
\pgfpathlineto{\pgfqpoint{3.295403in}{2.748735in}}%
\pgfusepath{fill}%
\end{pgfscope}%
\begin{pgfscope}%
\pgfpathrectangle{\pgfqpoint{1.280114in}{0.528000in}}{\pgfqpoint{3.487886in}{3.696000in}} %
\pgfusepath{clip}%
\pgfsetbuttcap%
\pgfsetroundjoin%
\definecolor{currentfill}{rgb}{0.993248,0.906157,0.143936}%
\pgfsetfillcolor{currentfill}%
\pgfsetlinewidth{0.000000pt}%
\definecolor{currentstroke}{rgb}{0.000000,0.000000,0.000000}%
\pgfsetstrokecolor{currentstroke}%
\pgfsetdash{}{0pt}%
\pgfpathmoveto{\pgfqpoint{4.065333in}{2.075051in}}%
\pgfpathlineto{\pgfqpoint{3.240994in}{1.066781in}}%
\pgfpathlineto{\pgfqpoint{3.266513in}{1.061028in}}%
\pgfpathlineto{\pgfqpoint{3.165294in}{0.992672in}}%
\pgfpathlineto{\pgfqpoint{3.212171in}{1.105457in}}%
\pgfpathlineto{\pgfqpoint{3.222880in}{1.081590in}}%
\pgfpathlineto{\pgfqpoint{4.047219in}{2.089861in}}%
\pgfpathlineto{\pgfqpoint{4.065333in}{2.075051in}}%
\pgfusepath{fill}%
\end{pgfscope}%
\begin{pgfscope}%
\pgfpathrectangle{\pgfqpoint{1.280114in}{0.528000in}}{\pgfqpoint{3.487886in}{3.696000in}} %
\pgfusepath{clip}%
\pgfsetbuttcap%
\pgfsetroundjoin%
\definecolor{currentfill}{rgb}{0.993248,0.906157,0.143936}%
\pgfsetfillcolor{currentfill}%
\pgfsetlinewidth{0.000000pt}%
\definecolor{currentstroke}{rgb}{0.000000,0.000000,0.000000}%
\pgfsetstrokecolor{currentstroke}%
\pgfsetdash{}{0pt}%
\pgfpathmoveto{\pgfqpoint{4.519170in}{2.482634in}}%
\pgfpathlineto{\pgfqpoint{3.496383in}{2.419351in}}%
\pgfpathlineto{\pgfqpoint{3.509504in}{2.396721in}}%
\pgfpathlineto{\pgfqpoint{3.390573in}{2.424525in}}%
\pgfpathlineto{\pgfqpoint{3.505169in}{2.466779in}}%
\pgfpathlineto{\pgfqpoint{3.494938in}{2.442704in}}%
\pgfpathlineto{\pgfqpoint{4.517725in}{2.505986in}}%
\pgfpathlineto{\pgfqpoint{4.519170in}{2.482634in}}%
\pgfusepath{fill}%
\end{pgfscope}%
\begin{pgfscope}%
\pgfpathrectangle{\pgfqpoint{1.280114in}{0.528000in}}{\pgfqpoint{3.487886in}{3.696000in}} %
\pgfusepath{clip}%
\pgfsetbuttcap%
\pgfsetroundjoin%
\definecolor{currentfill}{rgb}{0.993248,0.906157,0.143936}%
\pgfsetfillcolor{currentfill}%
\pgfsetlinewidth{0.000000pt}%
\definecolor{currentstroke}{rgb}{0.000000,0.000000,0.000000}%
\pgfsetstrokecolor{currentstroke}%
\pgfsetdash{}{0pt}%
\pgfpathmoveto{\pgfqpoint{2.430634in}{2.140419in}}%
\pgfpathlineto{\pgfqpoint{2.734661in}{1.526904in}}%
\pgfpathlineto{\pgfqpoint{2.750431in}{1.547776in}}%
\pgfpathlineto{\pgfqpoint{2.770929in}{1.427370in}}%
\pgfpathlineto{\pgfqpoint{2.687538in}{1.516609in}}%
\pgfpathlineto{\pgfqpoint{2.713697in}{1.516515in}}%
\pgfpathlineto{\pgfqpoint{2.409670in}{2.130030in}}%
\pgfpathlineto{\pgfqpoint{2.430634in}{2.140419in}}%
\pgfusepath{fill}%
\end{pgfscope}%
\begin{pgfscope}%
\pgfpathrectangle{\pgfqpoint{1.280114in}{0.528000in}}{\pgfqpoint{3.487886in}{3.696000in}} %
\pgfusepath{clip}%
\pgfsetbuttcap%
\pgfsetroundjoin%
\definecolor{currentfill}{rgb}{0.122606,0.585371,0.546557}%
\pgfsetfillcolor{currentfill}%
\pgfsetlinewidth{0.000000pt}%
\definecolor{currentstroke}{rgb}{0.000000,0.000000,0.000000}%
\pgfsetstrokecolor{currentstroke}%
\pgfsetdash{}{0pt}%
\pgfpathmoveto{\pgfqpoint{1.785598in}{2.610948in}}%
\pgfpathlineto{\pgfqpoint{2.586829in}{2.514925in}}%
\pgfpathlineto{\pgfqpoint{2.577998in}{2.539548in}}%
\pgfpathlineto{\pgfqpoint{2.689978in}{2.490781in}}%
\pgfpathlineto{\pgfqpoint{2.569645in}{2.469855in}}%
\pgfpathlineto{\pgfqpoint{2.584045in}{2.491694in}}%
\pgfpathlineto{\pgfqpoint{1.782814in}{2.587717in}}%
\pgfpathlineto{\pgfqpoint{1.785598in}{2.610948in}}%
\pgfusepath{fill}%
\end{pgfscope}%
\begin{pgfscope}%
\pgfpathrectangle{\pgfqpoint{1.280114in}{0.528000in}}{\pgfqpoint{3.487886in}{3.696000in}} %
\pgfusepath{clip}%
\pgfsetbuttcap%
\pgfsetroundjoin%
\definecolor{currentfill}{rgb}{0.993248,0.906157,0.143936}%
\pgfsetfillcolor{currentfill}%
\pgfsetlinewidth{0.000000pt}%
\definecolor{currentstroke}{rgb}{0.000000,0.000000,0.000000}%
\pgfsetstrokecolor{currentstroke}%
\pgfsetdash{}{0pt}%
\pgfpathmoveto{\pgfqpoint{1.446823in}{2.278112in}}%
\pgfpathlineto{\pgfqpoint{2.529009in}{1.876346in}}%
\pgfpathlineto{\pgfqpoint{2.526185in}{1.902352in}}%
\pgfpathlineto{\pgfqpoint{2.623643in}{1.828734in}}%
\pgfpathlineto{\pgfqpoint{2.501755in}{1.836548in}}%
\pgfpathlineto{\pgfqpoint{2.520865in}{1.854411in}}%
\pgfpathlineto{\pgfqpoint{1.438680in}{2.256177in}}%
\pgfpathlineto{\pgfqpoint{1.446823in}{2.278112in}}%
\pgfusepath{fill}%
\end{pgfscope}%
\begin{pgfscope}%
\pgfpathrectangle{\pgfqpoint{1.280114in}{0.528000in}}{\pgfqpoint{3.487886in}{3.696000in}} %
\pgfusepath{clip}%
\pgfsetbuttcap%
\pgfsetroundjoin%
\definecolor{currentfill}{rgb}{0.993248,0.906157,0.143936}%
\pgfsetfillcolor{currentfill}%
\pgfsetlinewidth{0.000000pt}%
\definecolor{currentstroke}{rgb}{0.000000,0.000000,0.000000}%
\pgfsetstrokecolor{currentstroke}%
\pgfsetdash{}{0pt}%
\pgfpathmoveto{\pgfqpoint{3.940913in}{2.679740in}}%
\pgfpathlineto{\pgfqpoint{3.279376in}{3.633170in}}%
\pgfpathlineto{\pgfqpoint{3.266821in}{3.610221in}}%
\pgfpathlineto{\pgfqpoint{3.228966in}{3.726344in}}%
\pgfpathlineto{\pgfqpoint{3.324491in}{3.650235in}}%
\pgfpathlineto{\pgfqpoint{3.298599in}{3.646509in}}%
\pgfpathlineto{\pgfqpoint{3.960136in}{2.693078in}}%
\pgfpathlineto{\pgfqpoint{3.940913in}{2.679740in}}%
\pgfusepath{fill}%
\end{pgfscope}%
\begin{pgfscope}%
\pgfpathrectangle{\pgfqpoint{1.280114in}{0.528000in}}{\pgfqpoint{3.487886in}{3.696000in}} %
\pgfusepath{clip}%
\pgfsetbuttcap%
\pgfsetroundjoin%
\definecolor{currentfill}{rgb}{0.974417,0.903590,0.130215}%
\pgfsetfillcolor{currentfill}%
\pgfsetlinewidth{0.000000pt}%
\definecolor{currentstroke}{rgb}{0.000000,0.000000,0.000000}%
\pgfsetstrokecolor{currentstroke}%
\pgfsetdash{}{0pt}%
\pgfpathmoveto{\pgfqpoint{3.591160in}{2.746865in}}%
\pgfpathlineto{\pgfqpoint{3.113758in}{3.750964in}}%
\pgfpathlineto{\pgfqpoint{3.097651in}{3.730352in}}%
\pgfpathlineto{\pgfqpoint{3.079114in}{3.851075in}}%
\pgfpathlineto{\pgfqpoint{3.161043in}{3.760492in}}%
\pgfpathlineto{\pgfqpoint{3.134889in}{3.761010in}}%
\pgfpathlineto{\pgfqpoint{3.612290in}{2.756912in}}%
\pgfpathlineto{\pgfqpoint{3.591160in}{2.746865in}}%
\pgfusepath{fill}%
\end{pgfscope}%
\begin{pgfscope}%
\pgfpathrectangle{\pgfqpoint{1.280114in}{0.528000in}}{\pgfqpoint{3.487886in}{3.696000in}} %
\pgfusepath{clip}%
\pgfsetbuttcap%
\pgfsetroundjoin%
\definecolor{currentfill}{rgb}{0.267004,0.004874,0.329415}%
\pgfsetfillcolor{currentfill}%
\pgfsetlinewidth{0.000000pt}%
\definecolor{currentstroke}{rgb}{0.000000,0.000000,0.000000}%
\pgfsetstrokecolor{currentstroke}%
\pgfsetdash{}{0pt}%
\pgfpathmoveto{\pgfqpoint{3.540063in}{2.048565in}}%
\pgfpathlineto{\pgfqpoint{3.202357in}{0.804856in}}%
\pgfpathlineto{\pgfqpoint{3.228003in}{0.810015in}}%
\pgfpathlineto{\pgfqpoint{3.163477in}{0.706312in}}%
\pgfpathlineto{\pgfqpoint{3.160263in}{0.828408in}}%
\pgfpathlineto{\pgfqpoint{3.179777in}{0.810987in}}%
\pgfpathlineto{\pgfqpoint{3.517483in}{2.054696in}}%
\pgfpathlineto{\pgfqpoint{3.540063in}{2.048565in}}%
\pgfusepath{fill}%
\end{pgfscope}%
\begin{pgfscope}%
\pgfpathrectangle{\pgfqpoint{1.280114in}{0.528000in}}{\pgfqpoint{3.487886in}{3.696000in}} %
\pgfusepath{clip}%
\pgfsetbuttcap%
\pgfsetroundjoin%
\definecolor{currentfill}{rgb}{0.993248,0.906157,0.143936}%
\pgfsetfillcolor{currentfill}%
\pgfsetlinewidth{0.000000pt}%
\definecolor{currentstroke}{rgb}{0.000000,0.000000,0.000000}%
\pgfsetstrokecolor{currentstroke}%
\pgfsetdash{}{0pt}%
\pgfpathmoveto{\pgfqpoint{4.580751in}{2.290932in}}%
\pgfpathlineto{\pgfqpoint{3.383791in}{1.469226in}}%
\pgfpathlineto{\pgfqpoint{3.406678in}{1.456557in}}%
\pgfpathlineto{\pgfqpoint{3.290367in}{1.419281in}}%
\pgfpathlineto{\pgfqpoint{3.366951in}{1.514426in}}%
\pgfpathlineto{\pgfqpoint{3.370548in}{1.488515in}}%
\pgfpathlineto{\pgfqpoint{4.567509in}{2.310221in}}%
\pgfpathlineto{\pgfqpoint{4.580751in}{2.290932in}}%
\pgfusepath{fill}%
\end{pgfscope}%
\begin{pgfscope}%
\pgfpathrectangle{\pgfqpoint{1.280114in}{0.528000in}}{\pgfqpoint{3.487886in}{3.696000in}} %
\pgfusepath{clip}%
\pgfsetbuttcap%
\pgfsetroundjoin%
\definecolor{currentfill}{rgb}{0.993248,0.906157,0.143936}%
\pgfsetfillcolor{currentfill}%
\pgfsetlinewidth{0.000000pt}%
\definecolor{currentstroke}{rgb}{0.000000,0.000000,0.000000}%
\pgfsetstrokecolor{currentstroke}%
\pgfsetdash{}{0pt}%
\pgfpathmoveto{\pgfqpoint{2.429425in}{2.736644in}}%
\pgfpathlineto{\pgfqpoint{2.777662in}{3.526367in}}%
\pgfpathlineto{\pgfqpoint{2.751534in}{3.525103in}}%
\pgfpathlineto{\pgfqpoint{2.830848in}{3.617984in}}%
\pgfpathlineto{\pgfqpoint{2.815759in}{3.496782in}}%
\pgfpathlineto{\pgfqpoint{2.799071in}{3.516926in}}%
\pgfpathlineto{\pgfqpoint{2.450834in}{2.727204in}}%
\pgfpathlineto{\pgfqpoint{2.429425in}{2.736644in}}%
\pgfusepath{fill}%
\end{pgfscope}%
\begin{pgfscope}%
\pgfpathrectangle{\pgfqpoint{1.280114in}{0.528000in}}{\pgfqpoint{3.487886in}{3.696000in}} %
\pgfusepath{clip}%
\pgfsetbuttcap%
\pgfsetroundjoin%
\definecolor{currentfill}{rgb}{0.993248,0.906157,0.143936}%
\pgfsetfillcolor{currentfill}%
\pgfsetlinewidth{0.000000pt}%
\definecolor{currentstroke}{rgb}{0.000000,0.000000,0.000000}%
\pgfsetstrokecolor{currentstroke}%
\pgfsetdash{}{0pt}%
\pgfpathmoveto{\pgfqpoint{2.503254in}{2.034094in}}%
\pgfpathlineto{\pgfqpoint{2.855561in}{0.971204in}}%
\pgfpathlineto{\pgfqpoint{2.874089in}{0.989670in}}%
\pgfpathlineto{\pgfqpoint{2.877583in}{0.867582in}}%
\pgfpathlineto{\pgfqpoint{2.807462in}{0.967586in}}%
\pgfpathlineto{\pgfqpoint{2.833352in}{0.963843in}}%
\pgfpathlineto{\pgfqpoint{2.481045in}{2.026733in}}%
\pgfpathlineto{\pgfqpoint{2.503254in}{2.034094in}}%
\pgfusepath{fill}%
\end{pgfscope}%
\begin{pgfscope}%
\pgfpathrectangle{\pgfqpoint{1.280114in}{0.528000in}}{\pgfqpoint{3.487886in}{3.696000in}} %
\pgfusepath{clip}%
\pgfsetbuttcap%
\pgfsetroundjoin%
\definecolor{currentfill}{rgb}{0.993248,0.906157,0.143936}%
\pgfsetfillcolor{currentfill}%
\pgfsetlinewidth{0.000000pt}%
\definecolor{currentstroke}{rgb}{0.000000,0.000000,0.000000}%
\pgfsetstrokecolor{currentstroke}%
\pgfsetdash{}{0pt}%
\pgfpathmoveto{\pgfqpoint{4.126552in}{2.142489in}}%
\pgfpathlineto{\pgfqpoint{3.275540in}{1.121174in}}%
\pgfpathlineto{\pgfqpoint{3.301004in}{1.115183in}}%
\pgfpathlineto{\pgfqpoint{3.199153in}{1.047774in}}%
\pgfpathlineto{\pgfqpoint{3.247079in}{1.160117in}}%
\pgfpathlineto{\pgfqpoint{3.257565in}{1.136151in}}%
\pgfpathlineto{\pgfqpoint{4.108577in}{2.157466in}}%
\pgfpathlineto{\pgfqpoint{4.126552in}{2.142489in}}%
\pgfusepath{fill}%
\end{pgfscope}%
\begin{pgfscope}%
\pgfpathrectangle{\pgfqpoint{1.280114in}{0.528000in}}{\pgfqpoint{3.487886in}{3.696000in}} %
\pgfusepath{clip}%
\pgfsetbuttcap%
\pgfsetroundjoin%
\definecolor{currentfill}{rgb}{0.993248,0.906157,0.143936}%
\pgfsetfillcolor{currentfill}%
\pgfsetlinewidth{0.000000pt}%
\definecolor{currentstroke}{rgb}{0.000000,0.000000,0.000000}%
\pgfsetstrokecolor{currentstroke}%
\pgfsetdash{}{0pt}%
\pgfpathmoveto{\pgfqpoint{4.261347in}{2.159496in}}%
\pgfpathlineto{\pgfqpoint{3.354998in}{1.214910in}}%
\pgfpathlineto{\pgfqpoint{3.379981in}{1.207152in}}%
\pgfpathlineto{\pgfqpoint{3.273660in}{1.147038in}}%
\pgfpathlineto{\pgfqpoint{3.329333in}{1.255750in}}%
\pgfpathlineto{\pgfqpoint{3.338116in}{1.231109in}}%
\pgfpathlineto{\pgfqpoint{4.244465in}{2.175695in}}%
\pgfpathlineto{\pgfqpoint{4.261347in}{2.159496in}}%
\pgfusepath{fill}%
\end{pgfscope}%
\begin{pgfscope}%
\pgfpathrectangle{\pgfqpoint{1.280114in}{0.528000in}}{\pgfqpoint{3.487886in}{3.696000in}} %
\pgfusepath{clip}%
\pgfsetbuttcap%
\pgfsetroundjoin%
\definecolor{currentfill}{rgb}{0.993248,0.906157,0.143936}%
\pgfsetfillcolor{currentfill}%
\pgfsetlinewidth{0.000000pt}%
\definecolor{currentstroke}{rgb}{0.000000,0.000000,0.000000}%
\pgfsetstrokecolor{currentstroke}%
\pgfsetdash{}{0pt}%
\pgfpathmoveto{\pgfqpoint{4.332696in}{2.594807in}}%
\pgfpathlineto{\pgfqpoint{3.476800in}{2.979223in}}%
\pgfpathlineto{\pgfqpoint{3.477885in}{2.953087in}}%
\pgfpathlineto{\pgfqpoint{3.385547in}{3.033033in}}%
\pgfpathlineto{\pgfqpoint{3.506644in}{3.017117in}}%
\pgfpathlineto{\pgfqpoint{3.486386in}{3.000567in}}%
\pgfpathlineto{\pgfqpoint{4.342282in}{2.616150in}}%
\pgfpathlineto{\pgfqpoint{4.332696in}{2.594807in}}%
\pgfusepath{fill}%
\end{pgfscope}%
\begin{pgfscope}%
\pgfpathrectangle{\pgfqpoint{1.280114in}{0.528000in}}{\pgfqpoint{3.487886in}{3.696000in}} %
\pgfusepath{clip}%
\pgfsetbuttcap%
\pgfsetroundjoin%
\definecolor{currentfill}{rgb}{0.964894,0.902323,0.123941}%
\pgfsetfillcolor{currentfill}%
\pgfsetlinewidth{0.000000pt}%
\definecolor{currentstroke}{rgb}{0.000000,0.000000,0.000000}%
\pgfsetstrokecolor{currentstroke}%
\pgfsetdash{}{0pt}%
\pgfpathmoveto{\pgfqpoint{4.495498in}{2.478044in}}%
\pgfpathlineto{\pgfqpoint{3.528850in}{2.136523in}}%
\pgfpathlineto{\pgfqpoint{3.547675in}{2.118359in}}%
\pgfpathlineto{\pgfqpoint{3.425678in}{2.112479in}}%
\pgfpathlineto{\pgfqpoint{3.524292in}{2.184542in}}%
\pgfpathlineto{\pgfqpoint{3.521056in}{2.158584in}}%
\pgfpathlineto{\pgfqpoint{4.487704in}{2.500105in}}%
\pgfpathlineto{\pgfqpoint{4.495498in}{2.478044in}}%
\pgfusepath{fill}%
\end{pgfscope}%
\begin{pgfscope}%
\pgfpathrectangle{\pgfqpoint{1.280114in}{0.528000in}}{\pgfqpoint{3.487886in}{3.696000in}} %
\pgfusepath{clip}%
\pgfsetbuttcap%
\pgfsetroundjoin%
\definecolor{currentfill}{rgb}{0.974417,0.903590,0.130215}%
\pgfsetfillcolor{currentfill}%
\pgfsetlinewidth{0.000000pt}%
\definecolor{currentstroke}{rgb}{0.000000,0.000000,0.000000}%
\pgfsetstrokecolor{currentstroke}%
\pgfsetdash{}{0pt}%
\pgfpathmoveto{\pgfqpoint{3.675004in}{2.748993in}}%
\pgfpathlineto{\pgfqpoint{3.122696in}{3.819501in}}%
\pgfpathlineto{\pgfqpoint{3.107267in}{3.798377in}}%
\pgfpathlineto{\pgfqpoint{3.084818in}{3.918434in}}%
\pgfpathlineto{\pgfqpoint{3.169646in}{3.830560in}}%
\pgfpathlineto{\pgfqpoint{3.143489in}{3.830229in}}%
\pgfpathlineto{\pgfqpoint{3.695797in}{2.759721in}}%
\pgfpathlineto{\pgfqpoint{3.675004in}{2.748993in}}%
\pgfusepath{fill}%
\end{pgfscope}%
\begin{pgfscope}%
\pgfpathrectangle{\pgfqpoint{1.280114in}{0.528000in}}{\pgfqpoint{3.487886in}{3.696000in}} %
\pgfusepath{clip}%
\pgfsetbuttcap%
\pgfsetroundjoin%
\definecolor{currentfill}{rgb}{0.993248,0.906157,0.143936}%
\pgfsetfillcolor{currentfill}%
\pgfsetlinewidth{0.000000pt}%
\definecolor{currentstroke}{rgb}{0.000000,0.000000,0.000000}%
\pgfsetstrokecolor{currentstroke}%
\pgfsetdash{}{0pt}%
\pgfpathmoveto{\pgfqpoint{4.157761in}{2.144225in}}%
\pgfpathlineto{\pgfqpoint{3.312779in}{1.190570in}}%
\pgfpathlineto{\pgfqpoint{3.338050in}{1.183809in}}%
\pgfpathlineto{\pgfqpoint{3.234199in}{1.119523in}}%
\pgfpathlineto{\pgfqpoint{3.285513in}{1.230359in}}%
\pgfpathlineto{\pgfqpoint{3.295267in}{1.206086in}}%
\pgfpathlineto{\pgfqpoint{4.140249in}{2.159742in}}%
\pgfpathlineto{\pgfqpoint{4.157761in}{2.144225in}}%
\pgfusepath{fill}%
\end{pgfscope}%
\begin{pgfscope}%
\pgfpathrectangle{\pgfqpoint{1.280114in}{0.528000in}}{\pgfqpoint{3.487886in}{3.696000in}} %
\pgfusepath{clip}%
\pgfsetbuttcap%
\pgfsetroundjoin%
\definecolor{currentfill}{rgb}{0.886271,0.892374,0.095374}%
\pgfsetfillcolor{currentfill}%
\pgfsetlinewidth{0.000000pt}%
\definecolor{currentstroke}{rgb}{0.000000,0.000000,0.000000}%
\pgfsetstrokecolor{currentstroke}%
\pgfsetdash{}{0pt}%
\pgfpathmoveto{\pgfqpoint{4.254671in}{2.666764in}}%
\pgfpathlineto{\pgfqpoint{3.390728in}{3.345805in}}%
\pgfpathlineto{\pgfqpoint{3.385467in}{3.320180in}}%
\pgfpathlineto{\pgfqpoint{3.315178in}{3.420066in}}%
\pgfpathlineto{\pgfqpoint{3.428843in}{3.375367in}}%
\pgfpathlineto{\pgfqpoint{3.405187in}{3.364200in}}%
\pgfpathlineto{\pgfqpoint{4.269129in}{2.685160in}}%
\pgfpathlineto{\pgfqpoint{4.254671in}{2.666764in}}%
\pgfusepath{fill}%
\end{pgfscope}%
\begin{pgfscope}%
\pgfpathrectangle{\pgfqpoint{1.280114in}{0.528000in}}{\pgfqpoint{3.487886in}{3.696000in}} %
\pgfusepath{clip}%
\pgfsetbuttcap%
\pgfsetroundjoin%
\definecolor{currentfill}{rgb}{0.124395,0.578002,0.548287}%
\pgfsetfillcolor{currentfill}%
\pgfsetlinewidth{0.000000pt}%
\definecolor{currentstroke}{rgb}{0.000000,0.000000,0.000000}%
\pgfsetstrokecolor{currentstroke}%
\pgfsetdash{}{0pt}%
\pgfpathmoveto{\pgfqpoint{1.856736in}{2.541107in}}%
\pgfpathlineto{\pgfqpoint{2.581027in}{2.250494in}}%
\pgfpathlineto{\pgfqpoint{2.578883in}{2.276565in}}%
\pgfpathlineto{\pgfqpoint{2.674387in}{2.200429in}}%
\pgfpathlineto{\pgfqpoint{2.552744in}{2.211421in}}%
\pgfpathlineto{\pgfqpoint{2.572314in}{2.228779in}}%
\pgfpathlineto{\pgfqpoint{1.848023in}{2.519393in}}%
\pgfpathlineto{\pgfqpoint{1.856736in}{2.541107in}}%
\pgfusepath{fill}%
\end{pgfscope}%
\begin{pgfscope}%
\pgfpathrectangle{\pgfqpoint{1.280114in}{0.528000in}}{\pgfqpoint{3.487886in}{3.696000in}} %
\pgfusepath{clip}%
\pgfsetbuttcap%
\pgfsetroundjoin%
\definecolor{currentfill}{rgb}{0.993248,0.906157,0.143936}%
\pgfsetfillcolor{currentfill}%
\pgfsetlinewidth{0.000000pt}%
\definecolor{currentstroke}{rgb}{0.000000,0.000000,0.000000}%
\pgfsetstrokecolor{currentstroke}%
\pgfsetdash{}{0pt}%
\pgfpathmoveto{\pgfqpoint{4.604097in}{2.279688in}}%
\pgfpathlineto{\pgfqpoint{3.484554in}{1.556124in}}%
\pgfpathlineto{\pgfqpoint{3.507080in}{1.542824in}}%
\pgfpathlineto{\pgfqpoint{3.389776in}{1.508799in}}%
\pgfpathlineto{\pgfqpoint{3.468979in}{1.601775in}}%
\pgfpathlineto{\pgfqpoint{3.471854in}{1.575775in}}%
\pgfpathlineto{\pgfqpoint{4.591397in}{2.299338in}}%
\pgfpathlineto{\pgfqpoint{4.604097in}{2.279688in}}%
\pgfusepath{fill}%
\end{pgfscope}%
\begin{pgfscope}%
\pgfpathrectangle{\pgfqpoint{1.280114in}{0.528000in}}{\pgfqpoint{3.487886in}{3.696000in}} %
\pgfusepath{clip}%
\pgfsetbuttcap%
\pgfsetroundjoin%
\definecolor{currentfill}{rgb}{0.993248,0.906157,0.143936}%
\pgfsetfillcolor{currentfill}%
\pgfsetlinewidth{0.000000pt}%
\definecolor{currentstroke}{rgb}{0.000000,0.000000,0.000000}%
\pgfsetstrokecolor{currentstroke}%
\pgfsetdash{}{0pt}%
\pgfpathmoveto{\pgfqpoint{1.609949in}{2.361661in}}%
\pgfpathlineto{\pgfqpoint{2.578141in}{2.085814in}}%
\pgfpathlineto{\pgfqpoint{2.573301in}{2.111521in}}%
\pgfpathlineto{\pgfqpoint{2.676195in}{2.045713in}}%
\pgfpathlineto{\pgfqpoint{2.554068in}{2.044015in}}%
\pgfpathlineto{\pgfqpoint{2.571730in}{2.063312in}}%
\pgfpathlineto{\pgfqpoint{1.603538in}{2.339159in}}%
\pgfpathlineto{\pgfqpoint{1.609949in}{2.361661in}}%
\pgfusepath{fill}%
\end{pgfscope}%
\begin{pgfscope}%
\pgfpathrectangle{\pgfqpoint{1.280114in}{0.528000in}}{\pgfqpoint{3.487886in}{3.696000in}} %
\pgfusepath{clip}%
\pgfsetbuttcap%
\pgfsetroundjoin%
\definecolor{currentfill}{rgb}{0.993248,0.906157,0.143936}%
\pgfsetfillcolor{currentfill}%
\pgfsetlinewidth{0.000000pt}%
\definecolor{currentstroke}{rgb}{0.000000,0.000000,0.000000}%
\pgfsetstrokecolor{currentstroke}%
\pgfsetdash{}{0pt}%
\pgfpathmoveto{\pgfqpoint{2.480534in}{2.128263in}}%
\pgfpathlineto{\pgfqpoint{2.825193in}{1.396633in}}%
\pgfpathlineto{\pgfqpoint{2.841374in}{1.417188in}}%
\pgfpathlineto{\pgfqpoint{2.859480in}{1.296399in}}%
\pgfpathlineto{\pgfqpoint{2.777875in}{1.387274in}}%
\pgfpathlineto{\pgfqpoint{2.804027in}{1.386662in}}%
\pgfpathlineto{\pgfqpoint{2.459368in}{2.118292in}}%
\pgfpathlineto{\pgfqpoint{2.480534in}{2.128263in}}%
\pgfusepath{fill}%
\end{pgfscope}%
\begin{pgfscope}%
\pgfpathrectangle{\pgfqpoint{1.280114in}{0.528000in}}{\pgfqpoint{3.487886in}{3.696000in}} %
\pgfusepath{clip}%
\pgfsetbuttcap%
\pgfsetroundjoin%
\definecolor{currentfill}{rgb}{0.993248,0.906157,0.143936}%
\pgfsetfillcolor{currentfill}%
\pgfsetlinewidth{0.000000pt}%
\definecolor{currentstroke}{rgb}{0.000000,0.000000,0.000000}%
\pgfsetstrokecolor{currentstroke}%
\pgfsetdash{}{0pt}%
\pgfpathmoveto{\pgfqpoint{2.367973in}{2.742141in}}%
\pgfpathlineto{\pgfqpoint{2.762421in}{3.557311in}}%
\pgfpathlineto{\pgfqpoint{2.736264in}{3.556972in}}%
\pgfpathlineto{\pgfqpoint{2.818812in}{3.646991in}}%
\pgfpathlineto{\pgfqpoint{2.799448in}{3.526398in}}%
\pgfpathlineto{\pgfqpoint{2.783482in}{3.547120in}}%
\pgfpathlineto{\pgfqpoint{2.389035in}{2.731950in}}%
\pgfpathlineto{\pgfqpoint{2.367973in}{2.742141in}}%
\pgfusepath{fill}%
\end{pgfscope}%
\begin{pgfscope}%
\pgfpathrectangle{\pgfqpoint{1.280114in}{0.528000in}}{\pgfqpoint{3.487886in}{3.696000in}} %
\pgfusepath{clip}%
\pgfsetbuttcap%
\pgfsetroundjoin%
\definecolor{currentfill}{rgb}{0.993248,0.906157,0.143936}%
\pgfsetfillcolor{currentfill}%
\pgfsetlinewidth{0.000000pt}%
\definecolor{currentstroke}{rgb}{0.000000,0.000000,0.000000}%
\pgfsetstrokecolor{currentstroke}%
\pgfsetdash{}{0pt}%
\pgfpathmoveto{\pgfqpoint{2.518228in}{2.066841in}}%
\pgfpathlineto{\pgfqpoint{2.902380in}{1.034656in}}%
\pgfpathlineto{\pgfqpoint{2.920228in}{1.053781in}}%
\pgfpathlineto{\pgfqpoint{2.928141in}{0.931900in}}%
\pgfpathlineto{\pgfqpoint{2.854444in}{1.029298in}}%
\pgfpathlineto{\pgfqpoint{2.880452in}{1.026495in}}%
\pgfpathlineto{\pgfqpoint{2.496300in}{2.058680in}}%
\pgfpathlineto{\pgfqpoint{2.518228in}{2.066841in}}%
\pgfusepath{fill}%
\end{pgfscope}%
\begin{pgfscope}%
\pgfpathrectangle{\pgfqpoint{1.280114in}{0.528000in}}{\pgfqpoint{3.487886in}{3.696000in}} %
\pgfusepath{clip}%
\pgfsetbuttcap%
\pgfsetroundjoin%
\definecolor{currentfill}{rgb}{0.993248,0.906157,0.143936}%
\pgfsetfillcolor{currentfill}%
\pgfsetlinewidth{0.000000pt}%
\definecolor{currentstroke}{rgb}{0.000000,0.000000,0.000000}%
\pgfsetstrokecolor{currentstroke}%
\pgfsetdash{}{0pt}%
\pgfpathmoveto{\pgfqpoint{3.809949in}{2.071593in}}%
\pgfpathlineto{\pgfqpoint{3.216024in}{1.114731in}}%
\pgfpathlineto{\pgfqpoint{3.242072in}{1.112331in}}%
\pgfpathlineto{\pgfqpoint{3.150558in}{1.031443in}}%
\pgfpathlineto{\pgfqpoint{3.182434in}{1.149348in}}%
\pgfpathlineto{\pgfqpoint{3.196144in}{1.127070in}}%
\pgfpathlineto{\pgfqpoint{3.790070in}{2.083932in}}%
\pgfpathlineto{\pgfqpoint{3.809949in}{2.071593in}}%
\pgfusepath{fill}%
\end{pgfscope}%
\begin{pgfscope}%
\pgfpathrectangle{\pgfqpoint{1.280114in}{0.528000in}}{\pgfqpoint{3.487886in}{3.696000in}} %
\pgfusepath{clip}%
\pgfsetbuttcap%
\pgfsetroundjoin%
\definecolor{currentfill}{rgb}{0.974417,0.903590,0.130215}%
\pgfsetfillcolor{currentfill}%
\pgfsetlinewidth{0.000000pt}%
\definecolor{currentstroke}{rgb}{0.000000,0.000000,0.000000}%
\pgfsetstrokecolor{currentstroke}%
\pgfsetdash{}{0pt}%
\pgfpathmoveto{\pgfqpoint{4.481281in}{2.469858in}}%
\pgfpathlineto{\pgfqpoint{3.503380in}{2.155288in}}%
\pgfpathlineto{\pgfqpoint{3.521682in}{2.136597in}}%
\pgfpathlineto{\pgfqpoint{3.399567in}{2.134183in}}%
\pgfpathlineto{\pgfqpoint{3.500187in}{2.203418in}}%
\pgfpathlineto{\pgfqpoint{3.496215in}{2.177562in}}%
\pgfpathlineto{\pgfqpoint{4.474116in}{2.492132in}}%
\pgfpathlineto{\pgfqpoint{4.481281in}{2.469858in}}%
\pgfusepath{fill}%
\end{pgfscope}%
\begin{pgfscope}%
\pgfpathrectangle{\pgfqpoint{1.280114in}{0.528000in}}{\pgfqpoint{3.487886in}{3.696000in}} %
\pgfusepath{clip}%
\pgfsetbuttcap%
\pgfsetroundjoin%
\definecolor{currentfill}{rgb}{0.993248,0.906157,0.143936}%
\pgfsetfillcolor{currentfill}%
\pgfsetlinewidth{0.000000pt}%
\definecolor{currentstroke}{rgb}{0.000000,0.000000,0.000000}%
\pgfsetstrokecolor{currentstroke}%
\pgfsetdash{}{0pt}%
\pgfpathmoveto{\pgfqpoint{3.057000in}{2.025356in}}%
\pgfpathlineto{\pgfqpoint{3.094346in}{0.988549in}}%
\pgfpathlineto{\pgfqpoint{3.117307in}{1.001082in}}%
\pgfpathlineto{\pgfqpoint{3.086445in}{0.882907in}}%
\pgfpathlineto{\pgfqpoint{3.047161in}{0.998555in}}%
\pgfpathlineto{\pgfqpoint{3.070964in}{0.987706in}}%
\pgfpathlineto{\pgfqpoint{3.033618in}{2.024514in}}%
\pgfpathlineto{\pgfqpoint{3.057000in}{2.025356in}}%
\pgfusepath{fill}%
\end{pgfscope}%
\begin{pgfscope}%
\pgfpathrectangle{\pgfqpoint{1.280114in}{0.528000in}}{\pgfqpoint{3.487886in}{3.696000in}} %
\pgfusepath{clip}%
\pgfsetbuttcap%
\pgfsetroundjoin%
\definecolor{currentfill}{rgb}{0.246811,0.283237,0.535941}%
\pgfsetfillcolor{currentfill}%
\pgfsetlinewidth{0.000000pt}%
\definecolor{currentstroke}{rgb}{0.000000,0.000000,0.000000}%
\pgfsetstrokecolor{currentstroke}%
\pgfsetdash{}{0pt}%
\pgfpathmoveto{\pgfqpoint{4.460527in}{2.385173in}}%
\pgfpathlineto{\pgfqpoint{3.458050in}{1.831585in}}%
\pgfpathlineto{\pgfqpoint{3.479602in}{1.816759in}}%
\pgfpathlineto{\pgfqpoint{3.360226in}{1.790929in}}%
\pgfpathlineto{\pgfqpoint{3.445670in}{1.878205in}}%
\pgfpathlineto{\pgfqpoint{3.446739in}{1.852067in}}%
\pgfpathlineto{\pgfqpoint{4.449216in}{2.405655in}}%
\pgfpathlineto{\pgfqpoint{4.460527in}{2.385173in}}%
\pgfusepath{fill}%
\end{pgfscope}%
\begin{pgfscope}%
\pgfpathrectangle{\pgfqpoint{1.280114in}{0.528000in}}{\pgfqpoint{3.487886in}{3.696000in}} %
\pgfusepath{clip}%
\pgfsetbuttcap%
\pgfsetroundjoin%
\definecolor{currentfill}{rgb}{0.993248,0.906157,0.143936}%
\pgfsetfillcolor{currentfill}%
\pgfsetlinewidth{0.000000pt}%
\definecolor{currentstroke}{rgb}{0.000000,0.000000,0.000000}%
\pgfsetstrokecolor{currentstroke}%
\pgfsetdash{}{0pt}%
\pgfpathmoveto{\pgfqpoint{3.107238in}{2.024200in}}%
\pgfpathlineto{\pgfqpoint{3.114285in}{1.140910in}}%
\pgfpathlineto{\pgfqpoint{3.137589in}{1.152795in}}%
\pgfpathlineto{\pgfqpoint{3.103427in}{1.035531in}}%
\pgfpathlineto{\pgfqpoint{3.067399in}{1.152235in}}%
\pgfpathlineto{\pgfqpoint{3.090889in}{1.140723in}}%
\pgfpathlineto{\pgfqpoint{3.083841in}{2.024014in}}%
\pgfpathlineto{\pgfqpoint{3.107238in}{2.024200in}}%
\pgfusepath{fill}%
\end{pgfscope}%
\begin{pgfscope}%
\pgfpathrectangle{\pgfqpoint{1.280114in}{0.528000in}}{\pgfqpoint{3.487886in}{3.696000in}} %
\pgfusepath{clip}%
\pgfsetbuttcap%
\pgfsetroundjoin%
\definecolor{currentfill}{rgb}{0.993248,0.906157,0.143936}%
\pgfsetfillcolor{currentfill}%
\pgfsetlinewidth{0.000000pt}%
\definecolor{currentstroke}{rgb}{0.000000,0.000000,0.000000}%
\pgfsetstrokecolor{currentstroke}%
\pgfsetdash{}{0pt}%
\pgfpathmoveto{\pgfqpoint{2.698764in}{2.736420in}}%
\pgfpathlineto{\pgfqpoint{2.852535in}{3.617896in}}%
\pgfpathlineto{\pgfqpoint{2.827475in}{3.610392in}}%
\pgfpathlineto{\pgfqpoint{2.882154in}{3.719608in}}%
\pgfpathlineto{\pgfqpoint{2.896623in}{3.598329in}}%
\pgfpathlineto{\pgfqpoint{2.875584in}{3.613875in}}%
\pgfpathlineto{\pgfqpoint{2.721813in}{2.732399in}}%
\pgfpathlineto{\pgfqpoint{2.698764in}{2.736420in}}%
\pgfusepath{fill}%
\end{pgfscope}%
\begin{pgfscope}%
\pgfpathrectangle{\pgfqpoint{1.280114in}{0.528000in}}{\pgfqpoint{3.487886in}{3.696000in}} %
\pgfusepath{clip}%
\pgfsetbuttcap%
\pgfsetroundjoin%
\definecolor{currentfill}{rgb}{0.268510,0.009605,0.335427}%
\pgfsetfillcolor{currentfill}%
\pgfsetlinewidth{0.000000pt}%
\definecolor{currentstroke}{rgb}{0.000000,0.000000,0.000000}%
\pgfsetstrokecolor{currentstroke}%
\pgfsetdash{}{0pt}%
\pgfpathmoveto{\pgfqpoint{3.541271in}{2.047411in}}%
\pgfpathlineto{\pgfqpoint{3.164516in}{1.150681in}}%
\pgfpathlineto{\pgfqpoint{3.190618in}{1.152403in}}%
\pgfpathlineto{\pgfqpoint{3.112947in}{1.058143in}}%
\pgfpathlineto{\pgfqpoint{3.125905in}{1.179592in}}%
\pgfpathlineto{\pgfqpoint{3.142945in}{1.159744in}}%
\pgfpathlineto{\pgfqpoint{3.519700in}{2.056474in}}%
\pgfpathlineto{\pgfqpoint{3.541271in}{2.047411in}}%
\pgfusepath{fill}%
\end{pgfscope}%
\begin{pgfscope}%
\pgfpathrectangle{\pgfqpoint{1.280114in}{0.528000in}}{\pgfqpoint{3.487886in}{3.696000in}} %
\pgfusepath{clip}%
\pgfsetbuttcap%
\pgfsetroundjoin%
\definecolor{currentfill}{rgb}{0.983868,0.904867,0.136897}%
\pgfsetfillcolor{currentfill}%
\pgfsetlinewidth{0.000000pt}%
\definecolor{currentstroke}{rgb}{0.000000,0.000000,0.000000}%
\pgfsetstrokecolor{currentstroke}%
\pgfsetdash{}{0pt}%
\pgfpathmoveto{\pgfqpoint{4.257234in}{2.631716in}}%
\pgfpathlineto{\pgfqpoint{3.381565in}{3.331232in}}%
\pgfpathlineto{\pgfqpoint{3.376102in}{3.305650in}}%
\pgfpathlineto{\pgfqpoint{3.306603in}{3.406087in}}%
\pgfpathlineto{\pgfqpoint{3.419912in}{3.360492in}}%
\pgfpathlineto{\pgfqpoint{3.396168in}{3.349513in}}%
\pgfpathlineto{\pgfqpoint{4.271837in}{2.649996in}}%
\pgfpathlineto{\pgfqpoint{4.257234in}{2.631716in}}%
\pgfusepath{fill}%
\end{pgfscope}%
\begin{pgfscope}%
\pgfpathrectangle{\pgfqpoint{1.280114in}{0.528000in}}{\pgfqpoint{3.487886in}{3.696000in}} %
\pgfusepath{clip}%
\pgfsetbuttcap%
\pgfsetroundjoin%
\definecolor{currentfill}{rgb}{0.121831,0.589055,0.545623}%
\pgfsetfillcolor{currentfill}%
\pgfsetlinewidth{0.000000pt}%
\definecolor{currentstroke}{rgb}{0.000000,0.000000,0.000000}%
\pgfsetstrokecolor{currentstroke}%
\pgfsetdash{}{0pt}%
\pgfpathmoveto{\pgfqpoint{1.700079in}{2.567852in}}%
\pgfpathlineto{\pgfqpoint{2.570094in}{2.273744in}}%
\pgfpathlineto{\pgfqpoint{2.566504in}{2.299656in}}%
\pgfpathlineto{\pgfqpoint{2.666091in}{2.228944in}}%
\pgfpathlineto{\pgfqpoint{2.544025in}{2.233160in}}%
\pgfpathlineto{\pgfqpoint{2.562601in}{2.251579in}}%
\pgfpathlineto{\pgfqpoint{1.692586in}{2.545687in}}%
\pgfpathlineto{\pgfqpoint{1.700079in}{2.567852in}}%
\pgfusepath{fill}%
\end{pgfscope}%
\begin{pgfscope}%
\pgfpathrectangle{\pgfqpoint{1.280114in}{0.528000in}}{\pgfqpoint{3.487886in}{3.696000in}} %
\pgfusepath{clip}%
\pgfsetbuttcap%
\pgfsetroundjoin%
\definecolor{currentfill}{rgb}{0.993248,0.906157,0.143936}%
\pgfsetfillcolor{currentfill}%
\pgfsetlinewidth{0.000000pt}%
\definecolor{currentstroke}{rgb}{0.000000,0.000000,0.000000}%
\pgfsetstrokecolor{currentstroke}%
\pgfsetdash{}{0pt}%
\pgfpathmoveto{\pgfqpoint{4.177973in}{2.190005in}}%
\pgfpathlineto{\pgfqpoint{3.326665in}{1.491278in}}%
\pgfpathlineto{\pgfqpoint{3.350552in}{1.480614in}}%
\pgfpathlineto{\pgfqpoint{3.237857in}{1.433522in}}%
\pgfpathlineto{\pgfqpoint{3.306019in}{1.534871in}}%
\pgfpathlineto{\pgfqpoint{3.311821in}{1.509363in}}%
\pgfpathlineto{\pgfqpoint{4.163129in}{2.208091in}}%
\pgfpathlineto{\pgfqpoint{4.177973in}{2.190005in}}%
\pgfusepath{fill}%
\end{pgfscope}%
\begin{pgfscope}%
\pgfpathrectangle{\pgfqpoint{1.280114in}{0.528000in}}{\pgfqpoint{3.487886in}{3.696000in}} %
\pgfusepath{clip}%
\pgfsetbuttcap%
\pgfsetroundjoin%
\definecolor{currentfill}{rgb}{0.993248,0.906157,0.143936}%
\pgfsetfillcolor{currentfill}%
\pgfsetlinewidth{0.000000pt}%
\definecolor{currentstroke}{rgb}{0.000000,0.000000,0.000000}%
\pgfsetstrokecolor{currentstroke}%
\pgfsetdash{}{0pt}%
\pgfpathmoveto{\pgfqpoint{2.643969in}{2.746402in}}%
\pgfpathlineto{\pgfqpoint{2.804630in}{3.733551in}}%
\pgfpathlineto{\pgfqpoint{2.779657in}{3.725763in}}%
\pgfpathlineto{\pgfqpoint{2.833090in}{3.835593in}}%
\pgfpathlineto{\pgfqpoint{2.848938in}{3.714487in}}%
\pgfpathlineto{\pgfqpoint{2.827723in}{3.729792in}}%
\pgfpathlineto{\pgfqpoint{2.667063in}{2.742643in}}%
\pgfpathlineto{\pgfqpoint{2.643969in}{2.746402in}}%
\pgfusepath{fill}%
\end{pgfscope}%
\begin{pgfscope}%
\pgfpathrectangle{\pgfqpoint{1.280114in}{0.528000in}}{\pgfqpoint{3.487886in}{3.696000in}} %
\pgfusepath{clip}%
\pgfsetbuttcap%
\pgfsetroundjoin%
\definecolor{currentfill}{rgb}{0.993248,0.906157,0.143936}%
\pgfsetfillcolor{currentfill}%
\pgfsetlinewidth{0.000000pt}%
\definecolor{currentstroke}{rgb}{0.000000,0.000000,0.000000}%
\pgfsetstrokecolor{currentstroke}%
\pgfsetdash{}{0pt}%
\pgfpathmoveto{\pgfqpoint{2.756673in}{2.781265in}}%
\pgfpathlineto{\pgfqpoint{2.979173in}{3.872368in}}%
\pgfpathlineto{\pgfqpoint{2.953910in}{3.865580in}}%
\pgfpathlineto{\pgfqpoint{3.011674in}{3.973195in}}%
\pgfpathlineto{\pgfqpoint{3.022687in}{3.851555in}}%
\pgfpathlineto{\pgfqpoint{3.002099in}{3.867693in}}%
\pgfpathlineto{\pgfqpoint{2.779598in}{2.776590in}}%
\pgfpathlineto{\pgfqpoint{2.756673in}{2.781265in}}%
\pgfusepath{fill}%
\end{pgfscope}%
\begin{pgfscope}%
\pgfpathrectangle{\pgfqpoint{1.280114in}{0.528000in}}{\pgfqpoint{3.487886in}{3.696000in}} %
\pgfusepath{clip}%
\pgfsetbuttcap%
\pgfsetroundjoin%
\definecolor{currentfill}{rgb}{0.876168,0.891125,0.095250}%
\pgfsetfillcolor{currentfill}%
\pgfsetlinewidth{0.000000pt}%
\definecolor{currentstroke}{rgb}{0.000000,0.000000,0.000000}%
\pgfsetstrokecolor{currentstroke}%
\pgfsetdash{}{0pt}%
\pgfpathmoveto{\pgfqpoint{3.945887in}{2.673152in}}%
\pgfpathlineto{\pgfqpoint{3.334815in}{3.453960in}}%
\pgfpathlineto{\pgfqpoint{3.323599in}{3.430327in}}%
\pgfpathlineto{\pgfqpoint{3.279137in}{3.544085in}}%
\pgfpathlineto{\pgfqpoint{3.378876in}{3.473587in}}%
\pgfpathlineto{\pgfqpoint{3.353240in}{3.468380in}}%
\pgfpathlineto{\pgfqpoint{3.964312in}{2.687572in}}%
\pgfpathlineto{\pgfqpoint{3.945887in}{2.673152in}}%
\pgfusepath{fill}%
\end{pgfscope}%
\begin{pgfscope}%
\pgfpathrectangle{\pgfqpoint{1.280114in}{0.528000in}}{\pgfqpoint{3.487886in}{3.696000in}} %
\pgfusepath{clip}%
\pgfsetbuttcap%
\pgfsetroundjoin%
\definecolor{currentfill}{rgb}{0.993248,0.906157,0.143936}%
\pgfsetfillcolor{currentfill}%
\pgfsetlinewidth{0.000000pt}%
\definecolor{currentstroke}{rgb}{0.000000,0.000000,0.000000}%
\pgfsetstrokecolor{currentstroke}%
\pgfsetdash{}{0pt}%
\pgfpathmoveto{\pgfqpoint{4.488622in}{2.444916in}}%
\pgfpathlineto{\pgfqpoint{3.486232in}{2.083976in}}%
\pgfpathlineto{\pgfqpoint{3.505165in}{2.065925in}}%
\pgfpathlineto{\pgfqpoint{3.383206in}{2.059312in}}%
\pgfpathlineto{\pgfqpoint{3.481385in}{2.131967in}}%
\pgfpathlineto{\pgfqpoint{3.478305in}{2.105989in}}%
\pgfpathlineto{\pgfqpoint{4.480695in}{2.466930in}}%
\pgfpathlineto{\pgfqpoint{4.488622in}{2.444916in}}%
\pgfusepath{fill}%
\end{pgfscope}%
\begin{pgfscope}%
\pgfpathrectangle{\pgfqpoint{1.280114in}{0.528000in}}{\pgfqpoint{3.487886in}{3.696000in}} %
\pgfusepath{clip}%
\pgfsetbuttcap%
\pgfsetroundjoin%
\definecolor{currentfill}{rgb}{0.983868,0.904867,0.136897}%
\pgfsetfillcolor{currentfill}%
\pgfsetlinewidth{0.000000pt}%
\definecolor{currentstroke}{rgb}{0.000000,0.000000,0.000000}%
\pgfsetstrokecolor{currentstroke}%
\pgfsetdash{}{0pt}%
\pgfpathmoveto{\pgfqpoint{2.411291in}{2.072718in}}%
\pgfpathlineto{\pgfqpoint{2.774982in}{1.418649in}}%
\pgfpathlineto{\pgfqpoint{2.789746in}{1.440244in}}%
\pgfpathlineto{\pgfqpoint{2.815924in}{1.320944in}}%
\pgfpathlineto{\pgfqpoint{2.728399in}{1.406133in}}%
\pgfpathlineto{\pgfqpoint{2.754533in}{1.407279in}}%
\pgfpathlineto{\pgfqpoint{2.390843in}{2.061348in}}%
\pgfpathlineto{\pgfqpoint{2.411291in}{2.072718in}}%
\pgfusepath{fill}%
\end{pgfscope}%
\begin{pgfscope}%
\pgfpathrectangle{\pgfqpoint{1.280114in}{0.528000in}}{\pgfqpoint{3.487886in}{3.696000in}} %
\pgfusepath{clip}%
\pgfsetbuttcap%
\pgfsetroundjoin%
\definecolor{currentfill}{rgb}{0.974417,0.903590,0.130215}%
\pgfsetfillcolor{currentfill}%
\pgfsetlinewidth{0.000000pt}%
\definecolor{currentstroke}{rgb}{0.000000,0.000000,0.000000}%
\pgfsetstrokecolor{currentstroke}%
\pgfsetdash{}{0pt}%
\pgfpathmoveto{\pgfqpoint{3.238167in}{2.792091in}}%
\pgfpathlineto{\pgfqpoint{3.052222in}{3.769822in}}%
\pgfpathlineto{\pgfqpoint{3.031422in}{3.753958in}}%
\pgfpathlineto{\pgfqpoint{3.044043in}{3.875442in}}%
\pgfpathlineto{\pgfqpoint{3.100378in}{3.767072in}}%
\pgfpathlineto{\pgfqpoint{3.075207in}{3.774193in}}%
\pgfpathlineto{\pgfqpoint{3.261152in}{2.796463in}}%
\pgfpathlineto{\pgfqpoint{3.238167in}{2.792091in}}%
\pgfusepath{fill}%
\end{pgfscope}%
\begin{pgfscope}%
\pgfpathrectangle{\pgfqpoint{1.280114in}{0.528000in}}{\pgfqpoint{3.487886in}{3.696000in}} %
\pgfusepath{clip}%
\pgfsetbuttcap%
\pgfsetroundjoin%
\definecolor{currentfill}{rgb}{0.993248,0.906157,0.143936}%
\pgfsetfillcolor{currentfill}%
\pgfsetlinewidth{0.000000pt}%
\definecolor{currentstroke}{rgb}{0.000000,0.000000,0.000000}%
\pgfsetstrokecolor{currentstroke}%
\pgfsetdash{}{0pt}%
\pgfpathmoveto{\pgfqpoint{2.916061in}{2.783519in}}%
\pgfpathlineto{\pgfqpoint{3.004024in}{3.780027in}}%
\pgfpathlineto{\pgfqpoint{2.979688in}{3.770430in}}%
\pgfpathlineto{\pgfqpoint{3.024935in}{3.883879in}}%
\pgfpathlineto{\pgfqpoint{3.049609in}{3.764259in}}%
\pgfpathlineto{\pgfqpoint{3.027330in}{3.777969in}}%
\pgfpathlineto{\pgfqpoint{2.939368in}{2.781462in}}%
\pgfpathlineto{\pgfqpoint{2.916061in}{2.783519in}}%
\pgfusepath{fill}%
\end{pgfscope}%
\begin{pgfscope}%
\pgfpathrectangle{\pgfqpoint{1.280114in}{0.528000in}}{\pgfqpoint{3.487886in}{3.696000in}} %
\pgfusepath{clip}%
\pgfsetbuttcap%
\pgfsetroundjoin%
\definecolor{currentfill}{rgb}{0.123463,0.581687,0.547445}%
\pgfsetfillcolor{currentfill}%
\pgfsetlinewidth{0.000000pt}%
\definecolor{currentstroke}{rgb}{0.000000,0.000000,0.000000}%
\pgfsetstrokecolor{currentstroke}%
\pgfsetdash{}{0pt}%
\pgfpathmoveto{\pgfqpoint{1.648928in}{2.484476in}}%
\pgfpathlineto{\pgfqpoint{2.568987in}{2.254061in}}%
\pgfpathlineto{\pgfqpoint{2.563323in}{2.279599in}}%
\pgfpathlineto{\pgfqpoint{2.668279in}{2.217134in}}%
\pgfpathlineto{\pgfqpoint{2.546271in}{2.211510in}}%
\pgfpathlineto{\pgfqpoint{2.563303in}{2.231364in}}%
\pgfpathlineto{\pgfqpoint{1.643244in}{2.461780in}}%
\pgfpathlineto{\pgfqpoint{1.648928in}{2.484476in}}%
\pgfusepath{fill}%
\end{pgfscope}%
\begin{pgfscope}%
\pgfpathrectangle{\pgfqpoint{1.280114in}{0.528000in}}{\pgfqpoint{3.487886in}{3.696000in}} %
\pgfusepath{clip}%
\pgfsetbuttcap%
\pgfsetroundjoin%
\definecolor{currentfill}{rgb}{0.993248,0.906157,0.143936}%
\pgfsetfillcolor{currentfill}%
\pgfsetlinewidth{0.000000pt}%
\definecolor{currentstroke}{rgb}{0.000000,0.000000,0.000000}%
\pgfsetstrokecolor{currentstroke}%
\pgfsetdash{}{0pt}%
\pgfpathmoveto{\pgfqpoint{4.349259in}{2.543152in}}%
\pgfpathlineto{\pgfqpoint{3.473858in}{2.653561in}}%
\pgfpathlineto{\pgfqpoint{3.482537in}{2.628884in}}%
\pgfpathlineto{\pgfqpoint{3.370861in}{2.678343in}}%
\pgfpathlineto{\pgfqpoint{3.491320in}{2.698524in}}%
\pgfpathlineto{\pgfqpoint{3.476786in}{2.676775in}}%
\pgfpathlineto{\pgfqpoint{4.352187in}{2.566366in}}%
\pgfpathlineto{\pgfqpoint{4.349259in}{2.543152in}}%
\pgfusepath{fill}%
\end{pgfscope}%
\begin{pgfscope}%
\pgfpathrectangle{\pgfqpoint{1.280114in}{0.528000in}}{\pgfqpoint{3.487886in}{3.696000in}} %
\pgfusepath{clip}%
\pgfsetbuttcap%
\pgfsetroundjoin%
\definecolor{currentfill}{rgb}{0.993248,0.906157,0.143936}%
\pgfsetfillcolor{currentfill}%
\pgfsetlinewidth{0.000000pt}%
\definecolor{currentstroke}{rgb}{0.000000,0.000000,0.000000}%
\pgfsetstrokecolor{currentstroke}%
\pgfsetdash{}{0pt}%
\pgfpathmoveto{\pgfqpoint{1.755797in}{2.264380in}}%
\pgfpathlineto{\pgfqpoint{2.594132in}{1.790818in}}%
\pgfpathlineto{\pgfqpoint{2.595454in}{1.816944in}}%
\pgfpathlineto{\pgfqpoint{2.680052in}{1.728848in}}%
\pgfpathlineto{\pgfqpoint{2.560931in}{1.755829in}}%
\pgfpathlineto{\pgfqpoint{2.582624in}{1.770447in}}%
\pgfpathlineto{\pgfqpoint{1.744289in}{2.244009in}}%
\pgfpathlineto{\pgfqpoint{1.755797in}{2.264380in}}%
\pgfusepath{fill}%
\end{pgfscope}%
\begin{pgfscope}%
\pgfpathrectangle{\pgfqpoint{1.280114in}{0.528000in}}{\pgfqpoint{3.487886in}{3.696000in}} %
\pgfusepath{clip}%
\pgfsetbuttcap%
\pgfsetroundjoin%
\definecolor{currentfill}{rgb}{0.993248,0.906157,0.143936}%
\pgfsetfillcolor{currentfill}%
\pgfsetlinewidth{0.000000pt}%
\definecolor{currentstroke}{rgb}{0.000000,0.000000,0.000000}%
\pgfsetstrokecolor{currentstroke}%
\pgfsetdash{}{0pt}%
\pgfpathmoveto{\pgfqpoint{3.789255in}{2.649344in}}%
\pgfpathlineto{\pgfqpoint{3.288617in}{3.303168in}}%
\pgfpathlineto{\pgfqpoint{3.277152in}{3.279655in}}%
\pgfpathlineto{\pgfqpoint{3.233895in}{3.393877in}}%
\pgfpathlineto{\pgfqpoint{3.332883in}{3.322329in}}%
\pgfpathlineto{\pgfqpoint{3.307194in}{3.317393in}}%
\pgfpathlineto{\pgfqpoint{3.807832in}{2.663569in}}%
\pgfpathlineto{\pgfqpoint{3.789255in}{2.649344in}}%
\pgfusepath{fill}%
\end{pgfscope}%
\begin{pgfscope}%
\pgfpathrectangle{\pgfqpoint{1.280114in}{0.528000in}}{\pgfqpoint{3.487886in}{3.696000in}} %
\pgfusepath{clip}%
\pgfsetbuttcap%
\pgfsetroundjoin%
\definecolor{currentfill}{rgb}{0.183898,0.422383,0.556944}%
\pgfsetfillcolor{currentfill}%
\pgfsetlinewidth{0.000000pt}%
\definecolor{currentstroke}{rgb}{0.000000,0.000000,0.000000}%
\pgfsetstrokecolor{currentstroke}%
\pgfsetdash{}{0pt}%
\pgfpathmoveto{\pgfqpoint{1.785787in}{2.231205in}}%
\pgfpathlineto{\pgfqpoint{2.668678in}{1.563409in}}%
\pgfpathlineto{\pgfqpoint{2.673462in}{1.589127in}}%
\pgfpathlineto{\pgfqpoint{2.745594in}{1.490563in}}%
\pgfpathlineto{\pgfqpoint{2.631119in}{1.533144in}}%
\pgfpathlineto{\pgfqpoint{2.654564in}{1.544748in}}%
\pgfpathlineto{\pgfqpoint{1.771673in}{2.212544in}}%
\pgfpathlineto{\pgfqpoint{1.785787in}{2.231205in}}%
\pgfusepath{fill}%
\end{pgfscope}%
\begin{pgfscope}%
\pgfpathrectangle{\pgfqpoint{1.280114in}{0.528000in}}{\pgfqpoint{3.487886in}{3.696000in}} %
\pgfusepath{clip}%
\pgfsetbuttcap%
\pgfsetroundjoin%
\definecolor{currentfill}{rgb}{0.983868,0.904867,0.136897}%
\pgfsetfillcolor{currentfill}%
\pgfsetlinewidth{0.000000pt}%
\definecolor{currentstroke}{rgb}{0.000000,0.000000,0.000000}%
\pgfsetstrokecolor{currentstroke}%
\pgfsetdash{}{0pt}%
\pgfpathmoveto{\pgfqpoint{2.419573in}{2.066916in}}%
\pgfpathlineto{\pgfqpoint{2.788791in}{1.301205in}}%
\pgfpathlineto{\pgfqpoint{2.804786in}{1.321904in}}%
\pgfpathlineto{\pgfqpoint{2.823984in}{1.201285in}}%
\pgfpathlineto{\pgfqpoint{2.741560in}{1.291418in}}%
\pgfpathlineto{\pgfqpoint{2.767716in}{1.291042in}}%
\pgfpathlineto{\pgfqpoint{2.398498in}{2.056753in}}%
\pgfpathlineto{\pgfqpoint{2.419573in}{2.066916in}}%
\pgfusepath{fill}%
\end{pgfscope}%
\begin{pgfscope}%
\pgfpathrectangle{\pgfqpoint{1.280114in}{0.528000in}}{\pgfqpoint{3.487886in}{3.696000in}} %
\pgfusepath{clip}%
\pgfsetbuttcap%
\pgfsetroundjoin%
\definecolor{currentfill}{rgb}{0.896320,0.893616,0.096335}%
\pgfsetfillcolor{currentfill}%
\pgfsetlinewidth{0.000000pt}%
\definecolor{currentstroke}{rgb}{0.000000,0.000000,0.000000}%
\pgfsetstrokecolor{currentstroke}%
\pgfsetdash{}{0pt}%
\pgfpathmoveto{\pgfqpoint{4.337416in}{2.638266in}}%
\pgfpathlineto{\pgfqpoint{3.411972in}{3.405802in}}%
\pgfpathlineto{\pgfqpoint{3.406040in}{3.380325in}}%
\pgfpathlineto{\pgfqpoint{3.338397in}{3.482022in}}%
\pgfpathlineto{\pgfqpoint{3.450849in}{3.434353in}}%
\pgfpathlineto{\pgfqpoint{3.426908in}{3.423812in}}%
\pgfpathlineto{\pgfqpoint{4.352353in}{2.656275in}}%
\pgfpathlineto{\pgfqpoint{4.337416in}{2.638266in}}%
\pgfusepath{fill}%
\end{pgfscope}%
\begin{pgfscope}%
\pgfpathrectangle{\pgfqpoint{1.280114in}{0.528000in}}{\pgfqpoint{3.487886in}{3.696000in}} %
\pgfusepath{clip}%
\pgfsetbuttcap%
\pgfsetroundjoin%
\definecolor{currentfill}{rgb}{0.119483,0.614817,0.537692}%
\pgfsetfillcolor{currentfill}%
\pgfsetlinewidth{0.000000pt}%
\definecolor{currentstroke}{rgb}{0.000000,0.000000,0.000000}%
\pgfsetstrokecolor{currentstroke}%
\pgfsetdash{}{0pt}%
\pgfpathmoveto{\pgfqpoint{1.808418in}{2.633356in}}%
\pgfpathlineto{\pgfqpoint{2.613053in}{2.694509in}}%
\pgfpathlineto{\pgfqpoint{2.599615in}{2.716953in}}%
\pgfpathlineto{\pgfqpoint{2.718926in}{2.690823in}}%
\pgfpathlineto{\pgfqpoint{2.604934in}{2.646962in}}%
\pgfpathlineto{\pgfqpoint{2.614826in}{2.671179in}}%
\pgfpathlineto{\pgfqpoint{1.810191in}{2.610026in}}%
\pgfpathlineto{\pgfqpoint{1.808418in}{2.633356in}}%
\pgfusepath{fill}%
\end{pgfscope}%
\begin{pgfscope}%
\pgfpathrectangle{\pgfqpoint{1.280114in}{0.528000in}}{\pgfqpoint{3.487886in}{3.696000in}} %
\pgfusepath{clip}%
\pgfsetbuttcap%
\pgfsetroundjoin%
\definecolor{currentfill}{rgb}{0.993248,0.906157,0.143936}%
\pgfsetfillcolor{currentfill}%
\pgfsetlinewidth{0.000000pt}%
\definecolor{currentstroke}{rgb}{0.000000,0.000000,0.000000}%
\pgfsetstrokecolor{currentstroke}%
\pgfsetdash{}{0pt}%
\pgfpathmoveto{\pgfqpoint{2.539514in}{2.049055in}}%
\pgfpathlineto{\pgfqpoint{2.970815in}{1.062838in}}%
\pgfpathlineto{\pgfqpoint{2.987564in}{1.082932in}}%
\pgfpathlineto{\pgfqpoint{3.002284in}{0.961684in}}%
\pgfpathlineto{\pgfqpoint{2.923253in}{1.054807in}}%
\pgfpathlineto{\pgfqpoint{2.949378in}{1.053463in}}%
\pgfpathlineto{\pgfqpoint{2.518077in}{2.039680in}}%
\pgfpathlineto{\pgfqpoint{2.539514in}{2.049055in}}%
\pgfusepath{fill}%
\end{pgfscope}%
\begin{pgfscope}%
\pgfpathrectangle{\pgfqpoint{1.280114in}{0.528000in}}{\pgfqpoint{3.487886in}{3.696000in}} %
\pgfusepath{clip}%
\pgfsetbuttcap%
\pgfsetroundjoin%
\definecolor{currentfill}{rgb}{0.993248,0.906157,0.143936}%
\pgfsetfillcolor{currentfill}%
\pgfsetlinewidth{0.000000pt}%
\definecolor{currentstroke}{rgb}{0.000000,0.000000,0.000000}%
\pgfsetstrokecolor{currentstroke}%
\pgfsetdash{}{0pt}%
\pgfpathmoveto{\pgfqpoint{1.683709in}{2.627336in}}%
\pgfpathlineto{\pgfqpoint{2.553244in}{2.682485in}}%
\pgfpathlineto{\pgfqpoint{2.540087in}{2.705095in}}%
\pgfpathlineto{\pgfqpoint{2.659061in}{2.677474in}}%
\pgfpathlineto{\pgfqpoint{2.544530in}{2.635043in}}%
\pgfpathlineto{\pgfqpoint{2.554725in}{2.659134in}}%
\pgfpathlineto{\pgfqpoint{1.685190in}{2.603985in}}%
\pgfpathlineto{\pgfqpoint{1.683709in}{2.627336in}}%
\pgfusepath{fill}%
\end{pgfscope}%
\begin{pgfscope}%
\pgfpathrectangle{\pgfqpoint{1.280114in}{0.528000in}}{\pgfqpoint{3.487886in}{3.696000in}} %
\pgfusepath{clip}%
\pgfsetbuttcap%
\pgfsetroundjoin%
\definecolor{currentfill}{rgb}{0.993248,0.906157,0.143936}%
\pgfsetfillcolor{currentfill}%
\pgfsetlinewidth{0.000000pt}%
\definecolor{currentstroke}{rgb}{0.000000,0.000000,0.000000}%
\pgfsetstrokecolor{currentstroke}%
\pgfsetdash{}{0pt}%
\pgfpathmoveto{\pgfqpoint{4.584751in}{2.491624in}}%
\pgfpathlineto{\pgfqpoint{3.497423in}{2.490918in}}%
\pgfpathlineto{\pgfqpoint{3.509136in}{2.467528in}}%
\pgfpathlineto{\pgfqpoint{3.392126in}{2.502548in}}%
\pgfpathlineto{\pgfqpoint{3.509091in}{2.537720in}}%
\pgfpathlineto{\pgfqpoint{3.497407in}{2.514315in}}%
\pgfpathlineto{\pgfqpoint{4.584736in}{2.515021in}}%
\pgfpathlineto{\pgfqpoint{4.584751in}{2.491624in}}%
\pgfusepath{fill}%
\end{pgfscope}%
\begin{pgfscope}%
\pgfpathrectangle{\pgfqpoint{1.280114in}{0.528000in}}{\pgfqpoint{3.487886in}{3.696000in}} %
\pgfusepath{clip}%
\pgfsetbuttcap%
\pgfsetroundjoin%
\definecolor{currentfill}{rgb}{0.993248,0.906157,0.143936}%
\pgfsetfillcolor{currentfill}%
\pgfsetlinewidth{0.000000pt}%
\definecolor{currentstroke}{rgb}{0.000000,0.000000,0.000000}%
\pgfsetstrokecolor{currentstroke}%
\pgfsetdash{}{0pt}%
\pgfpathmoveto{\pgfqpoint{1.801918in}{2.383217in}}%
\pgfpathlineto{\pgfqpoint{2.590364in}{2.197561in}}%
\pgfpathlineto{\pgfqpoint{2.584340in}{2.223017in}}%
\pgfpathlineto{\pgfqpoint{2.690169in}{2.162041in}}%
\pgfpathlineto{\pgfqpoint{2.568252in}{2.154693in}}%
\pgfpathlineto{\pgfqpoint{2.585002in}{2.174786in}}%
\pgfpathlineto{\pgfqpoint{1.796555in}{2.360442in}}%
\pgfpathlineto{\pgfqpoint{1.801918in}{2.383217in}}%
\pgfusepath{fill}%
\end{pgfscope}%
\begin{pgfscope}%
\pgfpathrectangle{\pgfqpoint{1.280114in}{0.528000in}}{\pgfqpoint{3.487886in}{3.696000in}} %
\pgfusepath{clip}%
\pgfsetbuttcap%
\pgfsetroundjoin%
\definecolor{currentfill}{rgb}{0.993248,0.906157,0.143936}%
\pgfsetfillcolor{currentfill}%
\pgfsetlinewidth{0.000000pt}%
\definecolor{currentstroke}{rgb}{0.000000,0.000000,0.000000}%
\pgfsetstrokecolor{currentstroke}%
\pgfsetdash{}{0pt}%
\pgfpathmoveto{\pgfqpoint{2.087303in}{2.059106in}}%
\pgfpathlineto{\pgfqpoint{2.674609in}{1.499401in}}%
\pgfpathlineto{\pgfqpoint{2.682282in}{1.524409in}}%
\pgfpathlineto{\pgfqpoint{2.742758in}{1.418294in}}%
\pgfpathlineto{\pgfqpoint{2.633857in}{1.473596in}}%
\pgfpathlineto{\pgfqpoint{2.658467in}{1.482463in}}%
\pgfpathlineto{\pgfqpoint{2.071161in}{2.042168in}}%
\pgfpathlineto{\pgfqpoint{2.087303in}{2.059106in}}%
\pgfusepath{fill}%
\end{pgfscope}%
\begin{pgfscope}%
\pgfpathrectangle{\pgfqpoint{1.280114in}{0.528000in}}{\pgfqpoint{3.487886in}{3.696000in}} %
\pgfusepath{clip}%
\pgfsetbuttcap%
\pgfsetroundjoin%
\definecolor{currentfill}{rgb}{0.993248,0.906157,0.143936}%
\pgfsetfillcolor{currentfill}%
\pgfsetlinewidth{0.000000pt}%
\definecolor{currentstroke}{rgb}{0.000000,0.000000,0.000000}%
\pgfsetstrokecolor{currentstroke}%
\pgfsetdash{}{0pt}%
\pgfpathmoveto{\pgfqpoint{1.696808in}{2.333050in}}%
\pgfpathlineto{\pgfqpoint{2.588773in}{1.870850in}}%
\pgfpathlineto{\pgfqpoint{2.589151in}{1.897006in}}%
\pgfpathlineto{\pgfqpoint{2.676874in}{1.812022in}}%
\pgfpathlineto{\pgfqpoint{2.556856in}{1.834684in}}%
\pgfpathlineto{\pgfqpoint{2.578008in}{1.850076in}}%
\pgfpathlineto{\pgfqpoint{1.686043in}{2.312276in}}%
\pgfpathlineto{\pgfqpoint{1.696808in}{2.333050in}}%
\pgfusepath{fill}%
\end{pgfscope}%
\begin{pgfscope}%
\pgfpathrectangle{\pgfqpoint{1.280114in}{0.528000in}}{\pgfqpoint{3.487886in}{3.696000in}} %
\pgfusepath{clip}%
\pgfsetbuttcap%
\pgfsetroundjoin%
\definecolor{currentfill}{rgb}{0.165117,0.467423,0.558141}%
\pgfsetfillcolor{currentfill}%
\pgfsetlinewidth{0.000000pt}%
\definecolor{currentstroke}{rgb}{0.000000,0.000000,0.000000}%
\pgfsetstrokecolor{currentstroke}%
\pgfsetdash{}{0pt}%
\pgfpathmoveto{\pgfqpoint{4.611757in}{2.223837in}}%
\pgfpathlineto{\pgfqpoint{3.357658in}{1.405253in}}%
\pgfpathlineto{\pgfqpoint{3.380244in}{1.392055in}}%
\pgfpathlineto{\pgfqpoint{3.263095in}{1.357500in}}%
\pgfpathlineto{\pgfqpoint{3.341877in}{1.450834in}}%
\pgfpathlineto{\pgfqpoint{3.344869in}{1.424846in}}%
\pgfpathlineto{\pgfqpoint{4.598968in}{2.243430in}}%
\pgfpathlineto{\pgfqpoint{4.611757in}{2.223837in}}%
\pgfusepath{fill}%
\end{pgfscope}%
\begin{pgfscope}%
\pgfpathrectangle{\pgfqpoint{1.280114in}{0.528000in}}{\pgfqpoint{3.487886in}{3.696000in}} %
\pgfusepath{clip}%
\pgfsetbuttcap%
\pgfsetroundjoin%
\definecolor{currentfill}{rgb}{0.183898,0.422383,0.556944}%
\pgfsetfillcolor{currentfill}%
\pgfsetlinewidth{0.000000pt}%
\definecolor{currentstroke}{rgb}{0.000000,0.000000,0.000000}%
\pgfsetstrokecolor{currentstroke}%
\pgfsetdash{}{0pt}%
\pgfpathmoveto{\pgfqpoint{1.811742in}{2.197584in}}%
\pgfpathlineto{\pgfqpoint{2.671444in}{1.564346in}}%
\pgfpathlineto{\pgfqpoint{2.675901in}{1.590123in}}%
\pgfpathlineto{\pgfqpoint{2.749280in}{1.492484in}}%
\pgfpathlineto{\pgfqpoint{2.634273in}{1.533607in}}%
\pgfpathlineto{\pgfqpoint{2.657568in}{1.545507in}}%
\pgfpathlineto{\pgfqpoint{1.797866in}{2.178745in}}%
\pgfpathlineto{\pgfqpoint{1.811742in}{2.197584in}}%
\pgfusepath{fill}%
\end{pgfscope}%
\begin{pgfscope}%
\pgfpathrectangle{\pgfqpoint{1.280114in}{0.528000in}}{\pgfqpoint{3.487886in}{3.696000in}} %
\pgfusepath{clip}%
\pgfsetbuttcap%
\pgfsetroundjoin%
\definecolor{currentfill}{rgb}{0.993248,0.906157,0.143936}%
\pgfsetfillcolor{currentfill}%
\pgfsetlinewidth{0.000000pt}%
\definecolor{currentstroke}{rgb}{0.000000,0.000000,0.000000}%
\pgfsetstrokecolor{currentstroke}%
\pgfsetdash{}{0pt}%
\pgfpathmoveto{\pgfqpoint{4.140921in}{2.154274in}}%
\pgfpathlineto{\pgfqpoint{3.264191in}{1.273394in}}%
\pgfpathlineto{\pgfqpoint{3.289027in}{1.265181in}}%
\pgfpathlineto{\pgfqpoint{3.181625in}{1.207021in}}%
\pgfpathlineto{\pgfqpoint{3.239277in}{1.314697in}}%
\pgfpathlineto{\pgfqpoint{3.247608in}{1.289900in}}%
\pgfpathlineto{\pgfqpoint{4.124338in}{2.170780in}}%
\pgfpathlineto{\pgfqpoint{4.140921in}{2.154274in}}%
\pgfusepath{fill}%
\end{pgfscope}%
\begin{pgfscope}%
\pgfpathrectangle{\pgfqpoint{1.280114in}{0.528000in}}{\pgfqpoint{3.487886in}{3.696000in}} %
\pgfusepath{clip}%
\pgfsetbuttcap%
\pgfsetroundjoin%
\definecolor{currentfill}{rgb}{0.993248,0.906157,0.143936}%
\pgfsetfillcolor{currentfill}%
\pgfsetlinewidth{0.000000pt}%
\definecolor{currentstroke}{rgb}{0.000000,0.000000,0.000000}%
\pgfsetstrokecolor{currentstroke}%
\pgfsetdash{}{0pt}%
\pgfpathmoveto{\pgfqpoint{4.085336in}{2.729169in}}%
\pgfpathlineto{\pgfqpoint{3.310325in}{3.599745in}}%
\pgfpathlineto{\pgfqpoint{3.300628in}{3.575450in}}%
\pgfpathlineto{\pgfqpoint{3.249054in}{3.686166in}}%
\pgfpathlineto{\pgfqpoint{3.353055in}{3.622123in}}%
\pgfpathlineto{\pgfqpoint{3.327801in}{3.615303in}}%
\pgfpathlineto{\pgfqpoint{4.102812in}{2.744727in}}%
\pgfpathlineto{\pgfqpoint{4.085336in}{2.729169in}}%
\pgfusepath{fill}%
\end{pgfscope}%
\begin{pgfscope}%
\pgfpathrectangle{\pgfqpoint{1.280114in}{0.528000in}}{\pgfqpoint{3.487886in}{3.696000in}} %
\pgfusepath{clip}%
\pgfsetbuttcap%
\pgfsetroundjoin%
\definecolor{currentfill}{rgb}{0.876168,0.891125,0.095250}%
\pgfsetfillcolor{currentfill}%
\pgfsetlinewidth{0.000000pt}%
\definecolor{currentstroke}{rgb}{0.000000,0.000000,0.000000}%
\pgfsetstrokecolor{currentstroke}%
\pgfsetdash{}{0pt}%
\pgfpathmoveto{\pgfqpoint{3.963186in}{2.671690in}}%
\pgfpathlineto{\pgfqpoint{3.359368in}{3.303811in}}%
\pgfpathlineto{\pgfqpoint{3.350530in}{3.279190in}}%
\pgfpathlineto{\pgfqpoint{3.295101in}{3.388027in}}%
\pgfpathlineto{\pgfqpoint{3.401287in}{3.327674in}}%
\pgfpathlineto{\pgfqpoint{3.376287in}{3.319973in}}%
\pgfpathlineto{\pgfqpoint{3.980105in}{2.687851in}}%
\pgfpathlineto{\pgfqpoint{3.963186in}{2.671690in}}%
\pgfusepath{fill}%
\end{pgfscope}%
\begin{pgfscope}%
\pgfpathrectangle{\pgfqpoint{1.280114in}{0.528000in}}{\pgfqpoint{3.487886in}{3.696000in}} %
\pgfusepath{clip}%
\pgfsetbuttcap%
\pgfsetroundjoin%
\definecolor{currentfill}{rgb}{0.974417,0.903590,0.130215}%
\pgfsetfillcolor{currentfill}%
\pgfsetlinewidth{0.000000pt}%
\definecolor{currentstroke}{rgb}{0.000000,0.000000,0.000000}%
\pgfsetstrokecolor{currentstroke}%
\pgfsetdash{}{0pt}%
\pgfpathmoveto{\pgfqpoint{3.511008in}{2.018183in}}%
\pgfpathlineto{\pgfqpoint{3.196740in}{0.801409in}}%
\pgfpathlineto{\pgfqpoint{3.222320in}{0.806885in}}%
\pgfpathlineto{\pgfqpoint{3.159084in}{0.702391in}}%
\pgfpathlineto{\pgfqpoint{3.154358in}{0.824438in}}%
\pgfpathlineto{\pgfqpoint{3.174086in}{0.807260in}}%
\pgfpathlineto{\pgfqpoint{3.488354in}{2.024034in}}%
\pgfpathlineto{\pgfqpoint{3.511008in}{2.018183in}}%
\pgfusepath{fill}%
\end{pgfscope}%
\begin{pgfscope}%
\pgfpathrectangle{\pgfqpoint{1.280114in}{0.528000in}}{\pgfqpoint{3.487886in}{3.696000in}} %
\pgfusepath{clip}%
\pgfsetbuttcap%
\pgfsetroundjoin%
\definecolor{currentfill}{rgb}{0.993248,0.906157,0.143936}%
\pgfsetfillcolor{currentfill}%
\pgfsetlinewidth{0.000000pt}%
\definecolor{currentstroke}{rgb}{0.000000,0.000000,0.000000}%
\pgfsetstrokecolor{currentstroke}%
\pgfsetdash{}{0pt}%
\pgfpathmoveto{\pgfqpoint{1.718432in}{2.666206in}}%
\pgfpathlineto{\pgfqpoint{2.581436in}{2.979726in}}%
\pgfpathlineto{\pgfqpoint{2.562451in}{2.997723in}}%
\pgfpathlineto{\pgfqpoint{2.684391in}{3.004682in}}%
\pgfpathlineto{\pgfqpoint{2.586418in}{2.931749in}}%
\pgfpathlineto{\pgfqpoint{2.589425in}{2.957735in}}%
\pgfpathlineto{\pgfqpoint{1.726422in}{2.644215in}}%
\pgfpathlineto{\pgfqpoint{1.718432in}{2.666206in}}%
\pgfusepath{fill}%
\end{pgfscope}%
\begin{pgfscope}%
\pgfpathrectangle{\pgfqpoint{1.280114in}{0.528000in}}{\pgfqpoint{3.487886in}{3.696000in}} %
\pgfusepath{clip}%
\pgfsetbuttcap%
\pgfsetroundjoin%
\definecolor{currentfill}{rgb}{0.993248,0.906157,0.143936}%
\pgfsetfillcolor{currentfill}%
\pgfsetlinewidth{0.000000pt}%
\definecolor{currentstroke}{rgb}{0.000000,0.000000,0.000000}%
\pgfsetstrokecolor{currentstroke}%
\pgfsetdash{}{0pt}%
\pgfpathmoveto{\pgfqpoint{3.839729in}{2.702027in}}%
\pgfpathlineto{\pgfqpoint{3.177372in}{3.836586in}}%
\pgfpathlineto{\pgfqpoint{3.163064in}{3.814687in}}%
\pgfpathlineto{\pgfqpoint{3.134391in}{3.933412in}}%
\pgfpathlineto{\pgfqpoint{3.223682in}{3.850076in}}%
\pgfpathlineto{\pgfqpoint{3.197578in}{3.848383in}}%
\pgfpathlineto{\pgfqpoint{3.859935in}{2.713823in}}%
\pgfpathlineto{\pgfqpoint{3.839729in}{2.702027in}}%
\pgfusepath{fill}%
\end{pgfscope}%
\begin{pgfscope}%
\pgfpathrectangle{\pgfqpoint{1.280114in}{0.528000in}}{\pgfqpoint{3.487886in}{3.696000in}} %
\pgfusepath{clip}%
\pgfsetbuttcap%
\pgfsetroundjoin%
\definecolor{currentfill}{rgb}{0.165117,0.467423,0.558141}%
\pgfsetfillcolor{currentfill}%
\pgfsetlinewidth{0.000000pt}%
\definecolor{currentstroke}{rgb}{0.000000,0.000000,0.000000}%
\pgfsetstrokecolor{currentstroke}%
\pgfsetdash{}{0pt}%
\pgfpathmoveto{\pgfqpoint{4.462720in}{2.198670in}}%
\pgfpathlineto{\pgfqpoint{3.366621in}{1.332294in}}%
\pgfpathlineto{\pgfqpoint{3.390307in}{1.321192in}}%
\pgfpathlineto{\pgfqpoint{3.276765in}{1.276182in}}%
\pgfpathlineto{\pgfqpoint{3.346781in}{1.376260in}}%
\pgfpathlineto{\pgfqpoint{3.352112in}{1.350650in}}%
\pgfpathlineto{\pgfqpoint{4.448211in}{2.217026in}}%
\pgfpathlineto{\pgfqpoint{4.462720in}{2.198670in}}%
\pgfusepath{fill}%
\end{pgfscope}%
\begin{pgfscope}%
\pgfpathrectangle{\pgfqpoint{1.280114in}{0.528000in}}{\pgfqpoint{3.487886in}{3.696000in}} %
\pgfusepath{clip}%
\pgfsetbuttcap%
\pgfsetroundjoin%
\definecolor{currentfill}{rgb}{0.993248,0.906157,0.143936}%
\pgfsetfillcolor{currentfill}%
\pgfsetlinewidth{0.000000pt}%
\definecolor{currentstroke}{rgb}{0.000000,0.000000,0.000000}%
\pgfsetstrokecolor{currentstroke}%
\pgfsetdash{}{0pt}%
\pgfpathmoveto{\pgfqpoint{2.108997in}{2.736524in}}%
\pgfpathlineto{\pgfqpoint{2.712379in}{3.579106in}}%
\pgfpathlineto{\pgfqpoint{2.686545in}{3.583217in}}%
\pgfpathlineto{\pgfqpoint{2.783191in}{3.657898in}}%
\pgfpathlineto{\pgfqpoint{2.743613in}{3.542350in}}%
\pgfpathlineto{\pgfqpoint{2.731402in}{3.565484in}}%
\pgfpathlineto{\pgfqpoint{2.128020in}{2.722902in}}%
\pgfpathlineto{\pgfqpoint{2.108997in}{2.736524in}}%
\pgfusepath{fill}%
\end{pgfscope}%
\begin{pgfscope}%
\pgfpathrectangle{\pgfqpoint{1.280114in}{0.528000in}}{\pgfqpoint{3.487886in}{3.696000in}} %
\pgfusepath{clip}%
\pgfsetbuttcap%
\pgfsetroundjoin%
\definecolor{currentfill}{rgb}{0.993248,0.906157,0.143936}%
\pgfsetfillcolor{currentfill}%
\pgfsetlinewidth{0.000000pt}%
\definecolor{currentstroke}{rgb}{0.000000,0.000000,0.000000}%
\pgfsetstrokecolor{currentstroke}%
\pgfsetdash{}{0pt}%
\pgfpathmoveto{\pgfqpoint{3.113287in}{2.048459in}}%
\pgfpathlineto{\pgfqpoint{3.087452in}{1.240074in}}%
\pgfpathlineto{\pgfqpoint{3.111211in}{1.251019in}}%
\pgfpathlineto{\pgfqpoint{3.072396in}{1.135213in}}%
\pgfpathlineto{\pgfqpoint{3.041055in}{1.253261in}}%
\pgfpathlineto{\pgfqpoint{3.064067in}{1.240821in}}%
\pgfpathlineto{\pgfqpoint{3.089902in}{2.049206in}}%
\pgfpathlineto{\pgfqpoint{3.113287in}{2.048459in}}%
\pgfusepath{fill}%
\end{pgfscope}%
\begin{pgfscope}%
\pgfpathrectangle{\pgfqpoint{1.280114in}{0.528000in}}{\pgfqpoint{3.487886in}{3.696000in}} %
\pgfusepath{clip}%
\pgfsetbuttcap%
\pgfsetroundjoin%
\definecolor{currentfill}{rgb}{0.993248,0.906157,0.143936}%
\pgfsetfillcolor{currentfill}%
\pgfsetlinewidth{0.000000pt}%
\definecolor{currentstroke}{rgb}{0.000000,0.000000,0.000000}%
\pgfsetstrokecolor{currentstroke}%
\pgfsetdash{}{0pt}%
\pgfpathmoveto{\pgfqpoint{2.707684in}{2.028012in}}%
\pgfpathlineto{\pgfqpoint{2.993508in}{0.841266in}}%
\pgfpathlineto{\pgfqpoint{3.013516in}{0.858118in}}%
\pgfpathlineto{\pgfqpoint{3.006788in}{0.736165in}}%
\pgfpathlineto{\pgfqpoint{2.945275in}{0.841682in}}%
\pgfpathlineto{\pgfqpoint{2.970761in}{0.835787in}}%
\pgfpathlineto{\pgfqpoint{2.684937in}{2.022533in}}%
\pgfpathlineto{\pgfqpoint{2.707684in}{2.028012in}}%
\pgfusepath{fill}%
\end{pgfscope}%
\begin{pgfscope}%
\pgfsetbuttcap%
\pgfsetroundjoin%
\definecolor{currentfill}{rgb}{0.000000,0.000000,0.000000}%
\pgfsetfillcolor{currentfill}%
\pgfsetlinewidth{0.803000pt}%
\definecolor{currentstroke}{rgb}{0.000000,0.000000,0.000000}%
\pgfsetstrokecolor{currentstroke}%
\pgfsetdash{}{0pt}%
\pgfsys@defobject{currentmarker}{\pgfqpoint{0.000000in}{-0.048611in}}{\pgfqpoint{0.000000in}{0.000000in}}{%
\pgfpathmoveto{\pgfqpoint{0.000000in}{0.000000in}}%
\pgfpathlineto{\pgfqpoint{0.000000in}{-0.048611in}}%
\pgfusepath{stroke,fill}%
}%
\begin{pgfscope}%
\pgfsys@transformshift{1.555203in}{0.528000in}%
\pgfsys@useobject{currentmarker}{}%
\end{pgfscope}%
\end{pgfscope}%
\begin{pgfscope}%
\pgftext[x=1.555203in,y=0.430778in,,top]{\rmfamily\fontsize{10.000000}{12.000000}\selectfont \(\displaystyle -2\)}%
\end{pgfscope}%
\begin{pgfscope}%
\pgfsetbuttcap%
\pgfsetroundjoin%
\definecolor{currentfill}{rgb}{0.000000,0.000000,0.000000}%
\pgfsetfillcolor{currentfill}%
\pgfsetlinewidth{0.803000pt}%
\definecolor{currentstroke}{rgb}{0.000000,0.000000,0.000000}%
\pgfsetstrokecolor{currentstroke}%
\pgfsetdash{}{0pt}%
\pgfsys@defobject{currentmarker}{\pgfqpoint{0.000000in}{-0.048611in}}{\pgfqpoint{0.000000in}{0.000000in}}{%
\pgfpathmoveto{\pgfqpoint{0.000000in}{0.000000in}}%
\pgfpathlineto{\pgfqpoint{0.000000in}{-0.048611in}}%
\pgfusepath{stroke,fill}%
}%
\begin{pgfscope}%
\pgfsys@transformshift{2.297296in}{0.528000in}%
\pgfsys@useobject{currentmarker}{}%
\end{pgfscope}%
\end{pgfscope}%
\begin{pgfscope}%
\pgftext[x=2.297296in,y=0.430778in,,top]{\rmfamily\fontsize{10.000000}{12.000000}\selectfont \(\displaystyle -1\)}%
\end{pgfscope}%
\begin{pgfscope}%
\pgfsetbuttcap%
\pgfsetroundjoin%
\definecolor{currentfill}{rgb}{0.000000,0.000000,0.000000}%
\pgfsetfillcolor{currentfill}%
\pgfsetlinewidth{0.803000pt}%
\definecolor{currentstroke}{rgb}{0.000000,0.000000,0.000000}%
\pgfsetstrokecolor{currentstroke}%
\pgfsetdash{}{0pt}%
\pgfsys@defobject{currentmarker}{\pgfqpoint{0.000000in}{-0.048611in}}{\pgfqpoint{0.000000in}{0.000000in}}{%
\pgfpathmoveto{\pgfqpoint{0.000000in}{0.000000in}}%
\pgfpathlineto{\pgfqpoint{0.000000in}{-0.048611in}}%
\pgfusepath{stroke,fill}%
}%
\begin{pgfscope}%
\pgfsys@transformshift{3.039388in}{0.528000in}%
\pgfsys@useobject{currentmarker}{}%
\end{pgfscope}%
\end{pgfscope}%
\begin{pgfscope}%
\pgftext[x=3.039388in,y=0.430778in,,top]{\rmfamily\fontsize{10.000000}{12.000000}\selectfont \(\displaystyle 0\)}%
\end{pgfscope}%
\begin{pgfscope}%
\pgfsetbuttcap%
\pgfsetroundjoin%
\definecolor{currentfill}{rgb}{0.000000,0.000000,0.000000}%
\pgfsetfillcolor{currentfill}%
\pgfsetlinewidth{0.803000pt}%
\definecolor{currentstroke}{rgb}{0.000000,0.000000,0.000000}%
\pgfsetstrokecolor{currentstroke}%
\pgfsetdash{}{0pt}%
\pgfsys@defobject{currentmarker}{\pgfqpoint{0.000000in}{-0.048611in}}{\pgfqpoint{0.000000in}{0.000000in}}{%
\pgfpathmoveto{\pgfqpoint{0.000000in}{0.000000in}}%
\pgfpathlineto{\pgfqpoint{0.000000in}{-0.048611in}}%
\pgfusepath{stroke,fill}%
}%
\begin{pgfscope}%
\pgfsys@transformshift{3.781480in}{0.528000in}%
\pgfsys@useobject{currentmarker}{}%
\end{pgfscope}%
\end{pgfscope}%
\begin{pgfscope}%
\pgftext[x=3.781480in,y=0.430778in,,top]{\rmfamily\fontsize{10.000000}{12.000000}\selectfont \(\displaystyle 1\)}%
\end{pgfscope}%
\begin{pgfscope}%
\pgfsetbuttcap%
\pgfsetroundjoin%
\definecolor{currentfill}{rgb}{0.000000,0.000000,0.000000}%
\pgfsetfillcolor{currentfill}%
\pgfsetlinewidth{0.803000pt}%
\definecolor{currentstroke}{rgb}{0.000000,0.000000,0.000000}%
\pgfsetstrokecolor{currentstroke}%
\pgfsetdash{}{0pt}%
\pgfsys@defobject{currentmarker}{\pgfqpoint{0.000000in}{-0.048611in}}{\pgfqpoint{0.000000in}{0.000000in}}{%
\pgfpathmoveto{\pgfqpoint{0.000000in}{0.000000in}}%
\pgfpathlineto{\pgfqpoint{0.000000in}{-0.048611in}}%
\pgfusepath{stroke,fill}%
}%
\begin{pgfscope}%
\pgfsys@transformshift{4.523573in}{0.528000in}%
\pgfsys@useobject{currentmarker}{}%
\end{pgfscope}%
\end{pgfscope}%
\begin{pgfscope}%
\pgftext[x=4.523573in,y=0.430778in,,top]{\rmfamily\fontsize{10.000000}{12.000000}\selectfont \(\displaystyle 2\)}%
\end{pgfscope}%
\begin{pgfscope}%
\pgfsetbuttcap%
\pgfsetroundjoin%
\definecolor{currentfill}{rgb}{0.000000,0.000000,0.000000}%
\pgfsetfillcolor{currentfill}%
\pgfsetlinewidth{0.803000pt}%
\definecolor{currentstroke}{rgb}{0.000000,0.000000,0.000000}%
\pgfsetstrokecolor{currentstroke}%
\pgfsetdash{}{0pt}%
\pgfsys@defobject{currentmarker}{\pgfqpoint{-0.048611in}{0.000000in}}{\pgfqpoint{0.000000in}{0.000000in}}{%
\pgfpathmoveto{\pgfqpoint{0.000000in}{0.000000in}}%
\pgfpathlineto{\pgfqpoint{-0.048611in}{0.000000in}}%
\pgfusepath{stroke,fill}%
}%
\begin{pgfscope}%
\pgfsys@transformshift{1.280114in}{0.919334in}%
\pgfsys@useobject{currentmarker}{}%
\end{pgfscope}%
\end{pgfscope}%
\begin{pgfscope}%
\pgftext[x=1.005422in,y=0.871140in,left,base]{\rmfamily\fontsize{10.000000}{12.000000}\selectfont \(\displaystyle -2\)}%
\end{pgfscope}%
\begin{pgfscope}%
\pgfsetbuttcap%
\pgfsetroundjoin%
\definecolor{currentfill}{rgb}{0.000000,0.000000,0.000000}%
\pgfsetfillcolor{currentfill}%
\pgfsetlinewidth{0.803000pt}%
\definecolor{currentstroke}{rgb}{0.000000,0.000000,0.000000}%
\pgfsetstrokecolor{currentstroke}%
\pgfsetdash{}{0pt}%
\pgfsys@defobject{currentmarker}{\pgfqpoint{-0.048611in}{0.000000in}}{\pgfqpoint{0.000000in}{0.000000in}}{%
\pgfpathmoveto{\pgfqpoint{0.000000in}{0.000000in}}%
\pgfpathlineto{\pgfqpoint{-0.048611in}{0.000000in}}%
\pgfusepath{stroke,fill}%
}%
\begin{pgfscope}%
\pgfsys@transformshift{1.280114in}{1.661427in}%
\pgfsys@useobject{currentmarker}{}%
\end{pgfscope}%
\end{pgfscope}%
\begin{pgfscope}%
\pgftext[x=1.005422in,y=1.613232in,left,base]{\rmfamily\fontsize{10.000000}{12.000000}\selectfont \(\displaystyle -1\)}%
\end{pgfscope}%
\begin{pgfscope}%
\pgfsetbuttcap%
\pgfsetroundjoin%
\definecolor{currentfill}{rgb}{0.000000,0.000000,0.000000}%
\pgfsetfillcolor{currentfill}%
\pgfsetlinewidth{0.803000pt}%
\definecolor{currentstroke}{rgb}{0.000000,0.000000,0.000000}%
\pgfsetstrokecolor{currentstroke}%
\pgfsetdash{}{0pt}%
\pgfsys@defobject{currentmarker}{\pgfqpoint{-0.048611in}{0.000000in}}{\pgfqpoint{0.000000in}{0.000000in}}{%
\pgfpathmoveto{\pgfqpoint{0.000000in}{0.000000in}}%
\pgfpathlineto{\pgfqpoint{-0.048611in}{0.000000in}}%
\pgfusepath{stroke,fill}%
}%
\begin{pgfscope}%
\pgfsys@transformshift{1.280114in}{2.403519in}%
\pgfsys@useobject{currentmarker}{}%
\end{pgfscope}%
\end{pgfscope}%
\begin{pgfscope}%
\pgftext[x=1.113447in,y=2.355325in,left,base]{\rmfamily\fontsize{10.000000}{12.000000}\selectfont \(\displaystyle 0\)}%
\end{pgfscope}%
\begin{pgfscope}%
\pgfsetbuttcap%
\pgfsetroundjoin%
\definecolor{currentfill}{rgb}{0.000000,0.000000,0.000000}%
\pgfsetfillcolor{currentfill}%
\pgfsetlinewidth{0.803000pt}%
\definecolor{currentstroke}{rgb}{0.000000,0.000000,0.000000}%
\pgfsetstrokecolor{currentstroke}%
\pgfsetdash{}{0pt}%
\pgfsys@defobject{currentmarker}{\pgfqpoint{-0.048611in}{0.000000in}}{\pgfqpoint{0.000000in}{0.000000in}}{%
\pgfpathmoveto{\pgfqpoint{0.000000in}{0.000000in}}%
\pgfpathlineto{\pgfqpoint{-0.048611in}{0.000000in}}%
\pgfusepath{stroke,fill}%
}%
\begin{pgfscope}%
\pgfsys@transformshift{1.280114in}{3.145611in}%
\pgfsys@useobject{currentmarker}{}%
\end{pgfscope}%
\end{pgfscope}%
\begin{pgfscope}%
\pgftext[x=1.113447in,y=3.097417in,left,base]{\rmfamily\fontsize{10.000000}{12.000000}\selectfont \(\displaystyle 1\)}%
\end{pgfscope}%
\begin{pgfscope}%
\pgfsetbuttcap%
\pgfsetroundjoin%
\definecolor{currentfill}{rgb}{0.000000,0.000000,0.000000}%
\pgfsetfillcolor{currentfill}%
\pgfsetlinewidth{0.803000pt}%
\definecolor{currentstroke}{rgb}{0.000000,0.000000,0.000000}%
\pgfsetstrokecolor{currentstroke}%
\pgfsetdash{}{0pt}%
\pgfsys@defobject{currentmarker}{\pgfqpoint{-0.048611in}{0.000000in}}{\pgfqpoint{0.000000in}{0.000000in}}{%
\pgfpathmoveto{\pgfqpoint{0.000000in}{0.000000in}}%
\pgfpathlineto{\pgfqpoint{-0.048611in}{0.000000in}}%
\pgfusepath{stroke,fill}%
}%
\begin{pgfscope}%
\pgfsys@transformshift{1.280114in}{3.887704in}%
\pgfsys@useobject{currentmarker}{}%
\end{pgfscope}%
\end{pgfscope}%
\begin{pgfscope}%
\pgftext[x=1.113447in,y=3.839509in,left,base]{\rmfamily\fontsize{10.000000}{12.000000}\selectfont \(\displaystyle 2\)}%
\end{pgfscope}%
\begin{pgfscope}%
\pgfsetrectcap%
\pgfsetmiterjoin%
\pgfsetlinewidth{0.803000pt}%
\definecolor{currentstroke}{rgb}{0.000000,0.000000,0.000000}%
\pgfsetstrokecolor{currentstroke}%
\pgfsetdash{}{0pt}%
\pgfpathmoveto{\pgfqpoint{1.280114in}{0.528000in}}%
\pgfpathlineto{\pgfqpoint{1.280114in}{4.224000in}}%
\pgfusepath{stroke}%
\end{pgfscope}%
\begin{pgfscope}%
\pgfsetrectcap%
\pgfsetmiterjoin%
\pgfsetlinewidth{0.803000pt}%
\definecolor{currentstroke}{rgb}{0.000000,0.000000,0.000000}%
\pgfsetstrokecolor{currentstroke}%
\pgfsetdash{}{0pt}%
\pgfpathmoveto{\pgfqpoint{4.768000in}{0.528000in}}%
\pgfpathlineto{\pgfqpoint{4.768000in}{4.224000in}}%
\pgfusepath{stroke}%
\end{pgfscope}%
\begin{pgfscope}%
\pgfsetrectcap%
\pgfsetmiterjoin%
\pgfsetlinewidth{0.803000pt}%
\definecolor{currentstroke}{rgb}{0.000000,0.000000,0.000000}%
\pgfsetstrokecolor{currentstroke}%
\pgfsetdash{}{0pt}%
\pgfpathmoveto{\pgfqpoint{1.280114in}{0.528000in}}%
\pgfpathlineto{\pgfqpoint{4.768000in}{0.528000in}}%
\pgfusepath{stroke}%
\end{pgfscope}%
\begin{pgfscope}%
\pgfsetrectcap%
\pgfsetmiterjoin%
\pgfsetlinewidth{0.803000pt}%
\definecolor{currentstroke}{rgb}{0.000000,0.000000,0.000000}%
\pgfsetstrokecolor{currentstroke}%
\pgfsetdash{}{0pt}%
\pgfpathmoveto{\pgfqpoint{1.280114in}{4.224000in}}%
\pgfpathlineto{\pgfqpoint{4.768000in}{4.224000in}}%
\pgfusepath{stroke}%
\end{pgfscope}%
\begin{pgfscope}%
\pgfpathrectangle{\pgfqpoint{5.016000in}{0.528000in}}{\pgfqpoint{0.184800in}{3.696000in}} %
\pgfusepath{clip}%
\pgfsetbuttcap%
\pgfsetmiterjoin%
\definecolor{currentfill}{rgb}{1.000000,1.000000,1.000000}%
\pgfsetfillcolor{currentfill}%
\pgfsetlinewidth{0.010037pt}%
\definecolor{currentstroke}{rgb}{1.000000,1.000000,1.000000}%
\pgfsetstrokecolor{currentstroke}%
\pgfsetdash{}{0pt}%
\pgfpathmoveto{\pgfqpoint{5.016000in}{0.528000in}}%
\pgfpathlineto{\pgfqpoint{5.016000in}{0.542438in}}%
\pgfpathlineto{\pgfqpoint{5.016000in}{4.209562in}}%
\pgfpathlineto{\pgfqpoint{5.016000in}{4.224000in}}%
\pgfpathlineto{\pgfqpoint{5.200800in}{4.224000in}}%
\pgfpathlineto{\pgfqpoint{5.200800in}{4.209562in}}%
\pgfpathlineto{\pgfqpoint{5.200800in}{0.542438in}}%
\pgfpathlineto{\pgfqpoint{5.200800in}{0.528000in}}%
\pgfpathclose%
\pgfusepath{stroke,fill}%
\end{pgfscope}%
\begin{pgfscope}%
\pgfsys@transformshift{5.020000in}{0.530000in}%
\pgftext[left,bottom]{\pgfimage[interpolate=true,width=0.180000in,height=3.690000in]{Figure-0004-20180109-004133-894347-img0.png}}%
\end{pgfscope}%
\begin{pgfscope}%
\pgfsetbuttcap%
\pgfsetroundjoin%
\definecolor{currentfill}{rgb}{0.000000,0.000000,0.000000}%
\pgfsetfillcolor{currentfill}%
\pgfsetlinewidth{0.803000pt}%
\definecolor{currentstroke}{rgb}{0.000000,0.000000,0.000000}%
\pgfsetstrokecolor{currentstroke}%
\pgfsetdash{}{0pt}%
\pgfsys@defobject{currentmarker}{\pgfqpoint{0.000000in}{0.000000in}}{\pgfqpoint{0.048611in}{0.000000in}}{%
\pgfpathmoveto{\pgfqpoint{0.000000in}{0.000000in}}%
\pgfpathlineto{\pgfqpoint{0.048611in}{0.000000in}}%
\pgfusepath{stroke,fill}%
}%
\begin{pgfscope}%
\pgfsys@transformshift{5.200800in}{0.589911in}%
\pgfsys@useobject{currentmarker}{}%
\end{pgfscope}%
\end{pgfscope}%
\begin{pgfscope}%
\pgftext[x=5.298022in,y=0.541717in,left,base]{\rmfamily\fontsize{10.000000}{12.000000}\selectfont \(\displaystyle 8.8\)}%
\end{pgfscope}%
\begin{pgfscope}%
\pgfsetbuttcap%
\pgfsetroundjoin%
\definecolor{currentfill}{rgb}{0.000000,0.000000,0.000000}%
\pgfsetfillcolor{currentfill}%
\pgfsetlinewidth{0.803000pt}%
\definecolor{currentstroke}{rgb}{0.000000,0.000000,0.000000}%
\pgfsetstrokecolor{currentstroke}%
\pgfsetdash{}{0pt}%
\pgfsys@defobject{currentmarker}{\pgfqpoint{0.000000in}{0.000000in}}{\pgfqpoint{0.048611in}{0.000000in}}{%
\pgfpathmoveto{\pgfqpoint{0.000000in}{0.000000in}}%
\pgfpathlineto{\pgfqpoint{0.048611in}{0.000000in}}%
\pgfusepath{stroke,fill}%
}%
\begin{pgfscope}%
\pgfsys@transformshift{5.200800in}{1.195614in}%
\pgfsys@useobject{currentmarker}{}%
\end{pgfscope}%
\end{pgfscope}%
\begin{pgfscope}%
\pgftext[x=5.298022in,y=1.147420in,left,base]{\rmfamily\fontsize{10.000000}{12.000000}\selectfont \(\displaystyle 9.0\)}%
\end{pgfscope}%
\begin{pgfscope}%
\pgfsetbuttcap%
\pgfsetroundjoin%
\definecolor{currentfill}{rgb}{0.000000,0.000000,0.000000}%
\pgfsetfillcolor{currentfill}%
\pgfsetlinewidth{0.803000pt}%
\definecolor{currentstroke}{rgb}{0.000000,0.000000,0.000000}%
\pgfsetstrokecolor{currentstroke}%
\pgfsetdash{}{0pt}%
\pgfsys@defobject{currentmarker}{\pgfqpoint{0.000000in}{0.000000in}}{\pgfqpoint{0.048611in}{0.000000in}}{%
\pgfpathmoveto{\pgfqpoint{0.000000in}{0.000000in}}%
\pgfpathlineto{\pgfqpoint{0.048611in}{0.000000in}}%
\pgfusepath{stroke,fill}%
}%
\begin{pgfscope}%
\pgfsys@transformshift{5.200800in}{1.801317in}%
\pgfsys@useobject{currentmarker}{}%
\end{pgfscope}%
\end{pgfscope}%
\begin{pgfscope}%
\pgftext[x=5.298022in,y=1.753123in,left,base]{\rmfamily\fontsize{10.000000}{12.000000}\selectfont \(\displaystyle 9.2\)}%
\end{pgfscope}%
\begin{pgfscope}%
\pgfsetbuttcap%
\pgfsetroundjoin%
\definecolor{currentfill}{rgb}{0.000000,0.000000,0.000000}%
\pgfsetfillcolor{currentfill}%
\pgfsetlinewidth{0.803000pt}%
\definecolor{currentstroke}{rgb}{0.000000,0.000000,0.000000}%
\pgfsetstrokecolor{currentstroke}%
\pgfsetdash{}{0pt}%
\pgfsys@defobject{currentmarker}{\pgfqpoint{0.000000in}{0.000000in}}{\pgfqpoint{0.048611in}{0.000000in}}{%
\pgfpathmoveto{\pgfqpoint{0.000000in}{0.000000in}}%
\pgfpathlineto{\pgfqpoint{0.048611in}{0.000000in}}%
\pgfusepath{stroke,fill}%
}%
\begin{pgfscope}%
\pgfsys@transformshift{5.200800in}{2.407020in}%
\pgfsys@useobject{currentmarker}{}%
\end{pgfscope}%
\end{pgfscope}%
\begin{pgfscope}%
\pgftext[x=5.298022in,y=2.358826in,left,base]{\rmfamily\fontsize{10.000000}{12.000000}\selectfont \(\displaystyle 9.4\)}%
\end{pgfscope}%
\begin{pgfscope}%
\pgfsetbuttcap%
\pgfsetroundjoin%
\definecolor{currentfill}{rgb}{0.000000,0.000000,0.000000}%
\pgfsetfillcolor{currentfill}%
\pgfsetlinewidth{0.803000pt}%
\definecolor{currentstroke}{rgb}{0.000000,0.000000,0.000000}%
\pgfsetstrokecolor{currentstroke}%
\pgfsetdash{}{0pt}%
\pgfsys@defobject{currentmarker}{\pgfqpoint{0.000000in}{0.000000in}}{\pgfqpoint{0.048611in}{0.000000in}}{%
\pgfpathmoveto{\pgfqpoint{0.000000in}{0.000000in}}%
\pgfpathlineto{\pgfqpoint{0.048611in}{0.000000in}}%
\pgfusepath{stroke,fill}%
}%
\begin{pgfscope}%
\pgfsys@transformshift{5.200800in}{3.012724in}%
\pgfsys@useobject{currentmarker}{}%
\end{pgfscope}%
\end{pgfscope}%
\begin{pgfscope}%
\pgftext[x=5.298022in,y=2.964529in,left,base]{\rmfamily\fontsize{10.000000}{12.000000}\selectfont \(\displaystyle 9.6\)}%
\end{pgfscope}%
\begin{pgfscope}%
\pgfsetbuttcap%
\pgfsetroundjoin%
\definecolor{currentfill}{rgb}{0.000000,0.000000,0.000000}%
\pgfsetfillcolor{currentfill}%
\pgfsetlinewidth{0.803000pt}%
\definecolor{currentstroke}{rgb}{0.000000,0.000000,0.000000}%
\pgfsetstrokecolor{currentstroke}%
\pgfsetdash{}{0pt}%
\pgfsys@defobject{currentmarker}{\pgfqpoint{0.000000in}{0.000000in}}{\pgfqpoint{0.048611in}{0.000000in}}{%
\pgfpathmoveto{\pgfqpoint{0.000000in}{0.000000in}}%
\pgfpathlineto{\pgfqpoint{0.048611in}{0.000000in}}%
\pgfusepath{stroke,fill}%
}%
\begin{pgfscope}%
\pgfsys@transformshift{5.200800in}{3.618427in}%
\pgfsys@useobject{currentmarker}{}%
\end{pgfscope}%
\end{pgfscope}%
\begin{pgfscope}%
\pgftext[x=5.298022in,y=3.570232in,left,base]{\rmfamily\fontsize{10.000000}{12.000000}\selectfont \(\displaystyle 9.8\)}%
\end{pgfscope}%
\begin{pgfscope}%
\pgftext[x=5.200800in,y=4.265667in,right,base]{\rmfamily\fontsize{10.000000}{12.000000}\selectfont \(\displaystyle \times10^{-7}+1.2499{\times}10^{-2}\)}%
\end{pgfscope}%
\begin{pgfscope}%
\pgfsetbuttcap%
\pgfsetmiterjoin%
\pgfsetlinewidth{0.803000pt}%
\definecolor{currentstroke}{rgb}{0.000000,0.000000,0.000000}%
\pgfsetstrokecolor{currentstroke}%
\pgfsetdash{}{0pt}%
\pgfpathmoveto{\pgfqpoint{5.016000in}{0.528000in}}%
\pgfpathlineto{\pgfqpoint{5.016000in}{0.542438in}}%
\pgfpathlineto{\pgfqpoint{5.016000in}{4.209562in}}%
\pgfpathlineto{\pgfqpoint{5.016000in}{4.224000in}}%
\pgfpathlineto{\pgfqpoint{5.200800in}{4.224000in}}%
\pgfpathlineto{\pgfqpoint{5.200800in}{4.209562in}}%
\pgfpathlineto{\pgfqpoint{5.200800in}{0.542438in}}%
\pgfpathlineto{\pgfqpoint{5.200800in}{0.528000in}}%
\pgfpathclose%
\pgfusepath{stroke}%
\end{pgfscope}%
\end{pgfpicture}%
\makeatother%
\endgroup%
}
\caption{An transportation example on  Caffarelli dataset} \label{Fig:Caff}
\end{figure}

The DOTmark dataset is used in the described way in \parencite{Schrieber2017}. Time limited, we only tested a randomly chosen pair of images from each class with size $ 32 \times 32 $. The classes are listed in Table \ref{Tbl:DOTmark}, and Figure \ref{Fig:DOTMark} shows the transportation in class 3.

\begin{table}[htbp]
\centering
\begin{tabular}{|c|c|}
\hline
1 & WhiteNoise \\ \hline
2 & GRFrough \\ \hline
3 & GRFmoderate \\ \hline
4 & GRFsmooth \\ \hline
5 & LogGRF \\ \hline
6 & LogitGRF \\ \hline
7 & CauchyDensity \\ \hline
8 & Shapes \\ \hline
9 & ClassicImages \\ \hline
10 & Microscopy \\ \hline
\end{tabular}
\caption{Classes in the DOTmark dataset} \label{Tbl:DOTmark}
\end{table}

\begin{figure}
\centering
\scalebox{0.33}{%% Creator: Matplotlib, PGF backend
%%
%% To include the figure in your LaTeX document, write
%%   \input{<filename>.pgf}
%%
%% Make sure the required packages are loaded in your preamble
%%   \usepackage{pgf}
%%
%% Figures using additional raster images can only be included by \input if
%% they are in the same directory as the main LaTeX file. For loading figures
%% from other directories you can use the `import` package
%%   \usepackage{import}
%% and then include the figures with
%%   \import{<path to file>}{<filename>.pgf}
%%
%% Matplotlib used the following preamble
%%   \usepackage{fontspec}
%%
\begingroup%
\makeatletter%
\begin{pgfpicture}%
\pgfpathrectangle{\pgfpointorigin}{\pgfqpoint{6.400000in}{4.800000in}}%
\pgfusepath{use as bounding box, clip}%
\begin{pgfscope}%
\pgfsetbuttcap%
\pgfsetmiterjoin%
\definecolor{currentfill}{rgb}{1.000000,1.000000,1.000000}%
\pgfsetfillcolor{currentfill}%
\pgfsetlinewidth{0.000000pt}%
\definecolor{currentstroke}{rgb}{1.000000,1.000000,1.000000}%
\pgfsetstrokecolor{currentstroke}%
\pgfsetdash{}{0pt}%
\pgfpathmoveto{\pgfqpoint{0.000000in}{0.000000in}}%
\pgfpathlineto{\pgfqpoint{6.400000in}{0.000000in}}%
\pgfpathlineto{\pgfqpoint{6.400000in}{4.800000in}}%
\pgfpathlineto{\pgfqpoint{0.000000in}{4.800000in}}%
\pgfpathclose%
\pgfusepath{fill}%
\end{pgfscope}%
\begin{pgfscope}%
\pgfsetbuttcap%
\pgfsetmiterjoin%
\definecolor{currentfill}{rgb}{1.000000,1.000000,1.000000}%
\pgfsetfillcolor{currentfill}%
\pgfsetlinewidth{0.000000pt}%
\definecolor{currentstroke}{rgb}{0.000000,0.000000,0.000000}%
\pgfsetstrokecolor{currentstroke}%
\pgfsetstrokeopacity{0.000000}%
\pgfsetdash{}{0pt}%
\pgfpathmoveto{\pgfqpoint{1.432000in}{0.528000in}}%
\pgfpathlineto{\pgfqpoint{5.128000in}{0.528000in}}%
\pgfpathlineto{\pgfqpoint{5.128000in}{4.224000in}}%
\pgfpathlineto{\pgfqpoint{1.432000in}{4.224000in}}%
\pgfpathclose%
\pgfusepath{fill}%
\end{pgfscope}%
\begin{pgfscope}%
\pgfpathrectangle{\pgfqpoint{1.432000in}{0.528000in}}{\pgfqpoint{3.696000in}{3.696000in}} %
\pgfusepath{clip}%
\pgfsys@transformshift{1.432000in}{0.528000in}%
\pgftext[left,bottom]{\pgfimage[interpolate=true,width=3.700000in,height=3.700000in]{Figure-0005-20180109-010236-058456-img0.png}}%
\end{pgfscope}%
\begin{pgfscope}%
\pgfsetbuttcap%
\pgfsetroundjoin%
\definecolor{currentfill}{rgb}{0.000000,0.000000,0.000000}%
\pgfsetfillcolor{currentfill}%
\pgfsetlinewidth{0.803000pt}%
\definecolor{currentstroke}{rgb}{0.000000,0.000000,0.000000}%
\pgfsetstrokecolor{currentstroke}%
\pgfsetdash{}{0pt}%
\pgfsys@defobject{currentmarker}{\pgfqpoint{0.000000in}{-0.048611in}}{\pgfqpoint{0.000000in}{0.000000in}}{%
\pgfpathmoveto{\pgfqpoint{0.000000in}{0.000000in}}%
\pgfpathlineto{\pgfqpoint{0.000000in}{-0.048611in}}%
\pgfusepath{stroke,fill}%
}%
\begin{pgfscope}%
\pgfsys@transformshift{1.432000in}{0.528000in}%
\pgfsys@useobject{currentmarker}{}%
\end{pgfscope}%
\end{pgfscope}%
\begin{pgfscope}%
\pgftext[x=1.432000in,y=0.430778in,,top]{\rmfamily\fontsize{10.000000}{12.000000}\selectfont \(\displaystyle 0.0\)}%
\end{pgfscope}%
\begin{pgfscope}%
\pgfsetbuttcap%
\pgfsetroundjoin%
\definecolor{currentfill}{rgb}{0.000000,0.000000,0.000000}%
\pgfsetfillcolor{currentfill}%
\pgfsetlinewidth{0.803000pt}%
\definecolor{currentstroke}{rgb}{0.000000,0.000000,0.000000}%
\pgfsetstrokecolor{currentstroke}%
\pgfsetdash{}{0pt}%
\pgfsys@defobject{currentmarker}{\pgfqpoint{0.000000in}{-0.048611in}}{\pgfqpoint{0.000000in}{0.000000in}}{%
\pgfpathmoveto{\pgfqpoint{0.000000in}{0.000000in}}%
\pgfpathlineto{\pgfqpoint{0.000000in}{-0.048611in}}%
\pgfusepath{stroke,fill}%
}%
\begin{pgfscope}%
\pgfsys@transformshift{2.171200in}{0.528000in}%
\pgfsys@useobject{currentmarker}{}%
\end{pgfscope}%
\end{pgfscope}%
\begin{pgfscope}%
\pgftext[x=2.171200in,y=0.430778in,,top]{\rmfamily\fontsize{10.000000}{12.000000}\selectfont \(\displaystyle 0.2\)}%
\end{pgfscope}%
\begin{pgfscope}%
\pgfsetbuttcap%
\pgfsetroundjoin%
\definecolor{currentfill}{rgb}{0.000000,0.000000,0.000000}%
\pgfsetfillcolor{currentfill}%
\pgfsetlinewidth{0.803000pt}%
\definecolor{currentstroke}{rgb}{0.000000,0.000000,0.000000}%
\pgfsetstrokecolor{currentstroke}%
\pgfsetdash{}{0pt}%
\pgfsys@defobject{currentmarker}{\pgfqpoint{0.000000in}{-0.048611in}}{\pgfqpoint{0.000000in}{0.000000in}}{%
\pgfpathmoveto{\pgfqpoint{0.000000in}{0.000000in}}%
\pgfpathlineto{\pgfqpoint{0.000000in}{-0.048611in}}%
\pgfusepath{stroke,fill}%
}%
\begin{pgfscope}%
\pgfsys@transformshift{2.910400in}{0.528000in}%
\pgfsys@useobject{currentmarker}{}%
\end{pgfscope}%
\end{pgfscope}%
\begin{pgfscope}%
\pgftext[x=2.910400in,y=0.430778in,,top]{\rmfamily\fontsize{10.000000}{12.000000}\selectfont \(\displaystyle 0.4\)}%
\end{pgfscope}%
\begin{pgfscope}%
\pgfsetbuttcap%
\pgfsetroundjoin%
\definecolor{currentfill}{rgb}{0.000000,0.000000,0.000000}%
\pgfsetfillcolor{currentfill}%
\pgfsetlinewidth{0.803000pt}%
\definecolor{currentstroke}{rgb}{0.000000,0.000000,0.000000}%
\pgfsetstrokecolor{currentstroke}%
\pgfsetdash{}{0pt}%
\pgfsys@defobject{currentmarker}{\pgfqpoint{0.000000in}{-0.048611in}}{\pgfqpoint{0.000000in}{0.000000in}}{%
\pgfpathmoveto{\pgfqpoint{0.000000in}{0.000000in}}%
\pgfpathlineto{\pgfqpoint{0.000000in}{-0.048611in}}%
\pgfusepath{stroke,fill}%
}%
\begin{pgfscope}%
\pgfsys@transformshift{3.649600in}{0.528000in}%
\pgfsys@useobject{currentmarker}{}%
\end{pgfscope}%
\end{pgfscope}%
\begin{pgfscope}%
\pgftext[x=3.649600in,y=0.430778in,,top]{\rmfamily\fontsize{10.000000}{12.000000}\selectfont \(\displaystyle 0.6\)}%
\end{pgfscope}%
\begin{pgfscope}%
\pgfsetbuttcap%
\pgfsetroundjoin%
\definecolor{currentfill}{rgb}{0.000000,0.000000,0.000000}%
\pgfsetfillcolor{currentfill}%
\pgfsetlinewidth{0.803000pt}%
\definecolor{currentstroke}{rgb}{0.000000,0.000000,0.000000}%
\pgfsetstrokecolor{currentstroke}%
\pgfsetdash{}{0pt}%
\pgfsys@defobject{currentmarker}{\pgfqpoint{0.000000in}{-0.048611in}}{\pgfqpoint{0.000000in}{0.000000in}}{%
\pgfpathmoveto{\pgfqpoint{0.000000in}{0.000000in}}%
\pgfpathlineto{\pgfqpoint{0.000000in}{-0.048611in}}%
\pgfusepath{stroke,fill}%
}%
\begin{pgfscope}%
\pgfsys@transformshift{4.388800in}{0.528000in}%
\pgfsys@useobject{currentmarker}{}%
\end{pgfscope}%
\end{pgfscope}%
\begin{pgfscope}%
\pgftext[x=4.388800in,y=0.430778in,,top]{\rmfamily\fontsize{10.000000}{12.000000}\selectfont \(\displaystyle 0.8\)}%
\end{pgfscope}%
\begin{pgfscope}%
\pgfsetbuttcap%
\pgfsetroundjoin%
\definecolor{currentfill}{rgb}{0.000000,0.000000,0.000000}%
\pgfsetfillcolor{currentfill}%
\pgfsetlinewidth{0.803000pt}%
\definecolor{currentstroke}{rgb}{0.000000,0.000000,0.000000}%
\pgfsetstrokecolor{currentstroke}%
\pgfsetdash{}{0pt}%
\pgfsys@defobject{currentmarker}{\pgfqpoint{0.000000in}{-0.048611in}}{\pgfqpoint{0.000000in}{0.000000in}}{%
\pgfpathmoveto{\pgfqpoint{0.000000in}{0.000000in}}%
\pgfpathlineto{\pgfqpoint{0.000000in}{-0.048611in}}%
\pgfusepath{stroke,fill}%
}%
\begin{pgfscope}%
\pgfsys@transformshift{5.128000in}{0.528000in}%
\pgfsys@useobject{currentmarker}{}%
\end{pgfscope}%
\end{pgfscope}%
\begin{pgfscope}%
\pgftext[x=5.128000in,y=0.430778in,,top]{\rmfamily\fontsize{10.000000}{12.000000}\selectfont \(\displaystyle 1.0\)}%
\end{pgfscope}%
\begin{pgfscope}%
\pgfsetbuttcap%
\pgfsetroundjoin%
\definecolor{currentfill}{rgb}{0.000000,0.000000,0.000000}%
\pgfsetfillcolor{currentfill}%
\pgfsetlinewidth{0.803000pt}%
\definecolor{currentstroke}{rgb}{0.000000,0.000000,0.000000}%
\pgfsetstrokecolor{currentstroke}%
\pgfsetdash{}{0pt}%
\pgfsys@defobject{currentmarker}{\pgfqpoint{-0.048611in}{0.000000in}}{\pgfqpoint{0.000000in}{0.000000in}}{%
\pgfpathmoveto{\pgfqpoint{0.000000in}{0.000000in}}%
\pgfpathlineto{\pgfqpoint{-0.048611in}{0.000000in}}%
\pgfusepath{stroke,fill}%
}%
\begin{pgfscope}%
\pgfsys@transformshift{1.432000in}{0.528000in}%
\pgfsys@useobject{currentmarker}{}%
\end{pgfscope}%
\end{pgfscope}%
\begin{pgfscope}%
\pgftext[x=1.157308in,y=0.479806in,left,base]{\rmfamily\fontsize{10.000000}{12.000000}\selectfont \(\displaystyle 0.0\)}%
\end{pgfscope}%
\begin{pgfscope}%
\pgfsetbuttcap%
\pgfsetroundjoin%
\definecolor{currentfill}{rgb}{0.000000,0.000000,0.000000}%
\pgfsetfillcolor{currentfill}%
\pgfsetlinewidth{0.803000pt}%
\definecolor{currentstroke}{rgb}{0.000000,0.000000,0.000000}%
\pgfsetstrokecolor{currentstroke}%
\pgfsetdash{}{0pt}%
\pgfsys@defobject{currentmarker}{\pgfqpoint{-0.048611in}{0.000000in}}{\pgfqpoint{0.000000in}{0.000000in}}{%
\pgfpathmoveto{\pgfqpoint{0.000000in}{0.000000in}}%
\pgfpathlineto{\pgfqpoint{-0.048611in}{0.000000in}}%
\pgfusepath{stroke,fill}%
}%
\begin{pgfscope}%
\pgfsys@transformshift{1.432000in}{1.267200in}%
\pgfsys@useobject{currentmarker}{}%
\end{pgfscope}%
\end{pgfscope}%
\begin{pgfscope}%
\pgftext[x=1.157308in,y=1.219006in,left,base]{\rmfamily\fontsize{10.000000}{12.000000}\selectfont \(\displaystyle 0.2\)}%
\end{pgfscope}%
\begin{pgfscope}%
\pgfsetbuttcap%
\pgfsetroundjoin%
\definecolor{currentfill}{rgb}{0.000000,0.000000,0.000000}%
\pgfsetfillcolor{currentfill}%
\pgfsetlinewidth{0.803000pt}%
\definecolor{currentstroke}{rgb}{0.000000,0.000000,0.000000}%
\pgfsetstrokecolor{currentstroke}%
\pgfsetdash{}{0pt}%
\pgfsys@defobject{currentmarker}{\pgfqpoint{-0.048611in}{0.000000in}}{\pgfqpoint{0.000000in}{0.000000in}}{%
\pgfpathmoveto{\pgfqpoint{0.000000in}{0.000000in}}%
\pgfpathlineto{\pgfqpoint{-0.048611in}{0.000000in}}%
\pgfusepath{stroke,fill}%
}%
\begin{pgfscope}%
\pgfsys@transformshift{1.432000in}{2.006400in}%
\pgfsys@useobject{currentmarker}{}%
\end{pgfscope}%
\end{pgfscope}%
\begin{pgfscope}%
\pgftext[x=1.157308in,y=1.958206in,left,base]{\rmfamily\fontsize{10.000000}{12.000000}\selectfont \(\displaystyle 0.4\)}%
\end{pgfscope}%
\begin{pgfscope}%
\pgfsetbuttcap%
\pgfsetroundjoin%
\definecolor{currentfill}{rgb}{0.000000,0.000000,0.000000}%
\pgfsetfillcolor{currentfill}%
\pgfsetlinewidth{0.803000pt}%
\definecolor{currentstroke}{rgb}{0.000000,0.000000,0.000000}%
\pgfsetstrokecolor{currentstroke}%
\pgfsetdash{}{0pt}%
\pgfsys@defobject{currentmarker}{\pgfqpoint{-0.048611in}{0.000000in}}{\pgfqpoint{0.000000in}{0.000000in}}{%
\pgfpathmoveto{\pgfqpoint{0.000000in}{0.000000in}}%
\pgfpathlineto{\pgfqpoint{-0.048611in}{0.000000in}}%
\pgfusepath{stroke,fill}%
}%
\begin{pgfscope}%
\pgfsys@transformshift{1.432000in}{2.745600in}%
\pgfsys@useobject{currentmarker}{}%
\end{pgfscope}%
\end{pgfscope}%
\begin{pgfscope}%
\pgftext[x=1.157308in,y=2.697406in,left,base]{\rmfamily\fontsize{10.000000}{12.000000}\selectfont \(\displaystyle 0.6\)}%
\end{pgfscope}%
\begin{pgfscope}%
\pgfsetbuttcap%
\pgfsetroundjoin%
\definecolor{currentfill}{rgb}{0.000000,0.000000,0.000000}%
\pgfsetfillcolor{currentfill}%
\pgfsetlinewidth{0.803000pt}%
\definecolor{currentstroke}{rgb}{0.000000,0.000000,0.000000}%
\pgfsetstrokecolor{currentstroke}%
\pgfsetdash{}{0pt}%
\pgfsys@defobject{currentmarker}{\pgfqpoint{-0.048611in}{0.000000in}}{\pgfqpoint{0.000000in}{0.000000in}}{%
\pgfpathmoveto{\pgfqpoint{0.000000in}{0.000000in}}%
\pgfpathlineto{\pgfqpoint{-0.048611in}{0.000000in}}%
\pgfusepath{stroke,fill}%
}%
\begin{pgfscope}%
\pgfsys@transformshift{1.432000in}{3.484800in}%
\pgfsys@useobject{currentmarker}{}%
\end{pgfscope}%
\end{pgfscope}%
\begin{pgfscope}%
\pgftext[x=1.157308in,y=3.436606in,left,base]{\rmfamily\fontsize{10.000000}{12.000000}\selectfont \(\displaystyle 0.8\)}%
\end{pgfscope}%
\begin{pgfscope}%
\pgfsetbuttcap%
\pgfsetroundjoin%
\definecolor{currentfill}{rgb}{0.000000,0.000000,0.000000}%
\pgfsetfillcolor{currentfill}%
\pgfsetlinewidth{0.803000pt}%
\definecolor{currentstroke}{rgb}{0.000000,0.000000,0.000000}%
\pgfsetstrokecolor{currentstroke}%
\pgfsetdash{}{0pt}%
\pgfsys@defobject{currentmarker}{\pgfqpoint{-0.048611in}{0.000000in}}{\pgfqpoint{0.000000in}{0.000000in}}{%
\pgfpathmoveto{\pgfqpoint{0.000000in}{0.000000in}}%
\pgfpathlineto{\pgfqpoint{-0.048611in}{0.000000in}}%
\pgfusepath{stroke,fill}%
}%
\begin{pgfscope}%
\pgfsys@transformshift{1.432000in}{4.224000in}%
\pgfsys@useobject{currentmarker}{}%
\end{pgfscope}%
\end{pgfscope}%
\begin{pgfscope}%
\pgftext[x=1.157308in,y=4.175806in,left,base]{\rmfamily\fontsize{10.000000}{12.000000}\selectfont \(\displaystyle 1.0\)}%
\end{pgfscope}%
\begin{pgfscope}%
\pgfsetrectcap%
\pgfsetmiterjoin%
\pgfsetlinewidth{0.803000pt}%
\definecolor{currentstroke}{rgb}{0.000000,0.000000,0.000000}%
\pgfsetstrokecolor{currentstroke}%
\pgfsetdash{}{0pt}%
\pgfpathmoveto{\pgfqpoint{1.432000in}{0.528000in}}%
\pgfpathlineto{\pgfqpoint{1.432000in}{4.224000in}}%
\pgfusepath{stroke}%
\end{pgfscope}%
\begin{pgfscope}%
\pgfsetrectcap%
\pgfsetmiterjoin%
\pgfsetlinewidth{0.803000pt}%
\definecolor{currentstroke}{rgb}{0.000000,0.000000,0.000000}%
\pgfsetstrokecolor{currentstroke}%
\pgfsetdash{}{0pt}%
\pgfpathmoveto{\pgfqpoint{5.128000in}{0.528000in}}%
\pgfpathlineto{\pgfqpoint{5.128000in}{4.224000in}}%
\pgfusepath{stroke}%
\end{pgfscope}%
\begin{pgfscope}%
\pgfsetrectcap%
\pgfsetmiterjoin%
\pgfsetlinewidth{0.803000pt}%
\definecolor{currentstroke}{rgb}{0.000000,0.000000,0.000000}%
\pgfsetstrokecolor{currentstroke}%
\pgfsetdash{}{0pt}%
\pgfpathmoveto{\pgfqpoint{1.432000in}{0.528000in}}%
\pgfpathlineto{\pgfqpoint{5.128000in}{0.528000in}}%
\pgfusepath{stroke}%
\end{pgfscope}%
\begin{pgfscope}%
\pgfsetrectcap%
\pgfsetmiterjoin%
\pgfsetlinewidth{0.803000pt}%
\definecolor{currentstroke}{rgb}{0.000000,0.000000,0.000000}%
\pgfsetstrokecolor{currentstroke}%
\pgfsetdash{}{0pt}%
\pgfpathmoveto{\pgfqpoint{1.432000in}{4.224000in}}%
\pgfpathlineto{\pgfqpoint{5.128000in}{4.224000in}}%
\pgfusepath{stroke}%
\end{pgfscope}%
\end{pgfpicture}%
\makeatother%
\endgroup%
} 
\hspace{-1cm}
\scalebox{0.33}{%% Creator: Matplotlib, PGF backend
%%
%% To include the figure in your LaTeX document, write
%%   \input{<filename>.pgf}
%%
%% Make sure the required packages are loaded in your preamble
%%   \usepackage{pgf}
%%
%% Figures using additional raster images can only be included by \input if
%% they are in the same directory as the main LaTeX file. For loading figures
%% from other directories you can use the `import` package
%%   \usepackage{import}
%% and then include the figures with
%%   \import{<path to file>}{<filename>.pgf}
%%
%% Matplotlib used the following preamble
%%   \usepackage{fontspec}
%%
\begingroup%
\makeatletter%
\begin{pgfpicture}%
\pgfpathrectangle{\pgfpointorigin}{\pgfqpoint{6.400000in}{4.800000in}}%
\pgfusepath{use as bounding box, clip}%
\begin{pgfscope}%
\pgfsetbuttcap%
\pgfsetmiterjoin%
\definecolor{currentfill}{rgb}{1.000000,1.000000,1.000000}%
\pgfsetfillcolor{currentfill}%
\pgfsetlinewidth{0.000000pt}%
\definecolor{currentstroke}{rgb}{1.000000,1.000000,1.000000}%
\pgfsetstrokecolor{currentstroke}%
\pgfsetdash{}{0pt}%
\pgfpathmoveto{\pgfqpoint{0.000000in}{0.000000in}}%
\pgfpathlineto{\pgfqpoint{6.400000in}{0.000000in}}%
\pgfpathlineto{\pgfqpoint{6.400000in}{4.800000in}}%
\pgfpathlineto{\pgfqpoint{0.000000in}{4.800000in}}%
\pgfpathclose%
\pgfusepath{fill}%
\end{pgfscope}%
\begin{pgfscope}%
\pgfsetbuttcap%
\pgfsetmiterjoin%
\definecolor{currentfill}{rgb}{1.000000,1.000000,1.000000}%
\pgfsetfillcolor{currentfill}%
\pgfsetlinewidth{0.000000pt}%
\definecolor{currentstroke}{rgb}{0.000000,0.000000,0.000000}%
\pgfsetstrokecolor{currentstroke}%
\pgfsetstrokeopacity{0.000000}%
\pgfsetdash{}{0pt}%
\pgfpathmoveto{\pgfqpoint{1.432000in}{0.528000in}}%
\pgfpathlineto{\pgfqpoint{5.128000in}{0.528000in}}%
\pgfpathlineto{\pgfqpoint{5.128000in}{4.224000in}}%
\pgfpathlineto{\pgfqpoint{1.432000in}{4.224000in}}%
\pgfpathclose%
\pgfusepath{fill}%
\end{pgfscope}%
\begin{pgfscope}%
\pgfpathrectangle{\pgfqpoint{1.432000in}{0.528000in}}{\pgfqpoint{3.696000in}{3.696000in}} %
\pgfusepath{clip}%
\pgfsys@transformshift{1.432000in}{0.528000in}%
\pgftext[left,bottom]{\pgfimage[interpolate=true,width=3.700000in,height=3.700000in]{Figure-0006-20180109-010236-127891-img0.png}}%
\end{pgfscope}%
\begin{pgfscope}%
\pgfsetbuttcap%
\pgfsetroundjoin%
\definecolor{currentfill}{rgb}{0.000000,0.000000,0.000000}%
\pgfsetfillcolor{currentfill}%
\pgfsetlinewidth{0.803000pt}%
\definecolor{currentstroke}{rgb}{0.000000,0.000000,0.000000}%
\pgfsetstrokecolor{currentstroke}%
\pgfsetdash{}{0pt}%
\pgfsys@defobject{currentmarker}{\pgfqpoint{0.000000in}{-0.048611in}}{\pgfqpoint{0.000000in}{0.000000in}}{%
\pgfpathmoveto{\pgfqpoint{0.000000in}{0.000000in}}%
\pgfpathlineto{\pgfqpoint{0.000000in}{-0.048611in}}%
\pgfusepath{stroke,fill}%
}%
\begin{pgfscope}%
\pgfsys@transformshift{1.432000in}{0.528000in}%
\pgfsys@useobject{currentmarker}{}%
\end{pgfscope}%
\end{pgfscope}%
\begin{pgfscope}%
\pgftext[x=1.432000in,y=0.430778in,,top]{\rmfamily\fontsize{10.000000}{12.000000}\selectfont \(\displaystyle 0.0\)}%
\end{pgfscope}%
\begin{pgfscope}%
\pgfsetbuttcap%
\pgfsetroundjoin%
\definecolor{currentfill}{rgb}{0.000000,0.000000,0.000000}%
\pgfsetfillcolor{currentfill}%
\pgfsetlinewidth{0.803000pt}%
\definecolor{currentstroke}{rgb}{0.000000,0.000000,0.000000}%
\pgfsetstrokecolor{currentstroke}%
\pgfsetdash{}{0pt}%
\pgfsys@defobject{currentmarker}{\pgfqpoint{0.000000in}{-0.048611in}}{\pgfqpoint{0.000000in}{0.000000in}}{%
\pgfpathmoveto{\pgfqpoint{0.000000in}{0.000000in}}%
\pgfpathlineto{\pgfqpoint{0.000000in}{-0.048611in}}%
\pgfusepath{stroke,fill}%
}%
\begin{pgfscope}%
\pgfsys@transformshift{2.171200in}{0.528000in}%
\pgfsys@useobject{currentmarker}{}%
\end{pgfscope}%
\end{pgfscope}%
\begin{pgfscope}%
\pgftext[x=2.171200in,y=0.430778in,,top]{\rmfamily\fontsize{10.000000}{12.000000}\selectfont \(\displaystyle 0.2\)}%
\end{pgfscope}%
\begin{pgfscope}%
\pgfsetbuttcap%
\pgfsetroundjoin%
\definecolor{currentfill}{rgb}{0.000000,0.000000,0.000000}%
\pgfsetfillcolor{currentfill}%
\pgfsetlinewidth{0.803000pt}%
\definecolor{currentstroke}{rgb}{0.000000,0.000000,0.000000}%
\pgfsetstrokecolor{currentstroke}%
\pgfsetdash{}{0pt}%
\pgfsys@defobject{currentmarker}{\pgfqpoint{0.000000in}{-0.048611in}}{\pgfqpoint{0.000000in}{0.000000in}}{%
\pgfpathmoveto{\pgfqpoint{0.000000in}{0.000000in}}%
\pgfpathlineto{\pgfqpoint{0.000000in}{-0.048611in}}%
\pgfusepath{stroke,fill}%
}%
\begin{pgfscope}%
\pgfsys@transformshift{2.910400in}{0.528000in}%
\pgfsys@useobject{currentmarker}{}%
\end{pgfscope}%
\end{pgfscope}%
\begin{pgfscope}%
\pgftext[x=2.910400in,y=0.430778in,,top]{\rmfamily\fontsize{10.000000}{12.000000}\selectfont \(\displaystyle 0.4\)}%
\end{pgfscope}%
\begin{pgfscope}%
\pgfsetbuttcap%
\pgfsetroundjoin%
\definecolor{currentfill}{rgb}{0.000000,0.000000,0.000000}%
\pgfsetfillcolor{currentfill}%
\pgfsetlinewidth{0.803000pt}%
\definecolor{currentstroke}{rgb}{0.000000,0.000000,0.000000}%
\pgfsetstrokecolor{currentstroke}%
\pgfsetdash{}{0pt}%
\pgfsys@defobject{currentmarker}{\pgfqpoint{0.000000in}{-0.048611in}}{\pgfqpoint{0.000000in}{0.000000in}}{%
\pgfpathmoveto{\pgfqpoint{0.000000in}{0.000000in}}%
\pgfpathlineto{\pgfqpoint{0.000000in}{-0.048611in}}%
\pgfusepath{stroke,fill}%
}%
\begin{pgfscope}%
\pgfsys@transformshift{3.649600in}{0.528000in}%
\pgfsys@useobject{currentmarker}{}%
\end{pgfscope}%
\end{pgfscope}%
\begin{pgfscope}%
\pgftext[x=3.649600in,y=0.430778in,,top]{\rmfamily\fontsize{10.000000}{12.000000}\selectfont \(\displaystyle 0.6\)}%
\end{pgfscope}%
\begin{pgfscope}%
\pgfsetbuttcap%
\pgfsetroundjoin%
\definecolor{currentfill}{rgb}{0.000000,0.000000,0.000000}%
\pgfsetfillcolor{currentfill}%
\pgfsetlinewidth{0.803000pt}%
\definecolor{currentstroke}{rgb}{0.000000,0.000000,0.000000}%
\pgfsetstrokecolor{currentstroke}%
\pgfsetdash{}{0pt}%
\pgfsys@defobject{currentmarker}{\pgfqpoint{0.000000in}{-0.048611in}}{\pgfqpoint{0.000000in}{0.000000in}}{%
\pgfpathmoveto{\pgfqpoint{0.000000in}{0.000000in}}%
\pgfpathlineto{\pgfqpoint{0.000000in}{-0.048611in}}%
\pgfusepath{stroke,fill}%
}%
\begin{pgfscope}%
\pgfsys@transformshift{4.388800in}{0.528000in}%
\pgfsys@useobject{currentmarker}{}%
\end{pgfscope}%
\end{pgfscope}%
\begin{pgfscope}%
\pgftext[x=4.388800in,y=0.430778in,,top]{\rmfamily\fontsize{10.000000}{12.000000}\selectfont \(\displaystyle 0.8\)}%
\end{pgfscope}%
\begin{pgfscope}%
\pgfsetbuttcap%
\pgfsetroundjoin%
\definecolor{currentfill}{rgb}{0.000000,0.000000,0.000000}%
\pgfsetfillcolor{currentfill}%
\pgfsetlinewidth{0.803000pt}%
\definecolor{currentstroke}{rgb}{0.000000,0.000000,0.000000}%
\pgfsetstrokecolor{currentstroke}%
\pgfsetdash{}{0pt}%
\pgfsys@defobject{currentmarker}{\pgfqpoint{0.000000in}{-0.048611in}}{\pgfqpoint{0.000000in}{0.000000in}}{%
\pgfpathmoveto{\pgfqpoint{0.000000in}{0.000000in}}%
\pgfpathlineto{\pgfqpoint{0.000000in}{-0.048611in}}%
\pgfusepath{stroke,fill}%
}%
\begin{pgfscope}%
\pgfsys@transformshift{5.128000in}{0.528000in}%
\pgfsys@useobject{currentmarker}{}%
\end{pgfscope}%
\end{pgfscope}%
\begin{pgfscope}%
\pgftext[x=5.128000in,y=0.430778in,,top]{\rmfamily\fontsize{10.000000}{12.000000}\selectfont \(\displaystyle 1.0\)}%
\end{pgfscope}%
\begin{pgfscope}%
\pgfsetbuttcap%
\pgfsetroundjoin%
\definecolor{currentfill}{rgb}{0.000000,0.000000,0.000000}%
\pgfsetfillcolor{currentfill}%
\pgfsetlinewidth{0.803000pt}%
\definecolor{currentstroke}{rgb}{0.000000,0.000000,0.000000}%
\pgfsetstrokecolor{currentstroke}%
\pgfsetdash{}{0pt}%
\pgfsys@defobject{currentmarker}{\pgfqpoint{-0.048611in}{0.000000in}}{\pgfqpoint{0.000000in}{0.000000in}}{%
\pgfpathmoveto{\pgfqpoint{0.000000in}{0.000000in}}%
\pgfpathlineto{\pgfqpoint{-0.048611in}{0.000000in}}%
\pgfusepath{stroke,fill}%
}%
\begin{pgfscope}%
\pgfsys@transformshift{1.432000in}{0.528000in}%
\pgfsys@useobject{currentmarker}{}%
\end{pgfscope}%
\end{pgfscope}%
\begin{pgfscope}%
\pgftext[x=1.157308in,y=0.479806in,left,base]{\rmfamily\fontsize{10.000000}{12.000000}\selectfont \(\displaystyle 0.0\)}%
\end{pgfscope}%
\begin{pgfscope}%
\pgfsetbuttcap%
\pgfsetroundjoin%
\definecolor{currentfill}{rgb}{0.000000,0.000000,0.000000}%
\pgfsetfillcolor{currentfill}%
\pgfsetlinewidth{0.803000pt}%
\definecolor{currentstroke}{rgb}{0.000000,0.000000,0.000000}%
\pgfsetstrokecolor{currentstroke}%
\pgfsetdash{}{0pt}%
\pgfsys@defobject{currentmarker}{\pgfqpoint{-0.048611in}{0.000000in}}{\pgfqpoint{0.000000in}{0.000000in}}{%
\pgfpathmoveto{\pgfqpoint{0.000000in}{0.000000in}}%
\pgfpathlineto{\pgfqpoint{-0.048611in}{0.000000in}}%
\pgfusepath{stroke,fill}%
}%
\begin{pgfscope}%
\pgfsys@transformshift{1.432000in}{1.267200in}%
\pgfsys@useobject{currentmarker}{}%
\end{pgfscope}%
\end{pgfscope}%
\begin{pgfscope}%
\pgftext[x=1.157308in,y=1.219006in,left,base]{\rmfamily\fontsize{10.000000}{12.000000}\selectfont \(\displaystyle 0.2\)}%
\end{pgfscope}%
\begin{pgfscope}%
\pgfsetbuttcap%
\pgfsetroundjoin%
\definecolor{currentfill}{rgb}{0.000000,0.000000,0.000000}%
\pgfsetfillcolor{currentfill}%
\pgfsetlinewidth{0.803000pt}%
\definecolor{currentstroke}{rgb}{0.000000,0.000000,0.000000}%
\pgfsetstrokecolor{currentstroke}%
\pgfsetdash{}{0pt}%
\pgfsys@defobject{currentmarker}{\pgfqpoint{-0.048611in}{0.000000in}}{\pgfqpoint{0.000000in}{0.000000in}}{%
\pgfpathmoveto{\pgfqpoint{0.000000in}{0.000000in}}%
\pgfpathlineto{\pgfqpoint{-0.048611in}{0.000000in}}%
\pgfusepath{stroke,fill}%
}%
\begin{pgfscope}%
\pgfsys@transformshift{1.432000in}{2.006400in}%
\pgfsys@useobject{currentmarker}{}%
\end{pgfscope}%
\end{pgfscope}%
\begin{pgfscope}%
\pgftext[x=1.157308in,y=1.958206in,left,base]{\rmfamily\fontsize{10.000000}{12.000000}\selectfont \(\displaystyle 0.4\)}%
\end{pgfscope}%
\begin{pgfscope}%
\pgfsetbuttcap%
\pgfsetroundjoin%
\definecolor{currentfill}{rgb}{0.000000,0.000000,0.000000}%
\pgfsetfillcolor{currentfill}%
\pgfsetlinewidth{0.803000pt}%
\definecolor{currentstroke}{rgb}{0.000000,0.000000,0.000000}%
\pgfsetstrokecolor{currentstroke}%
\pgfsetdash{}{0pt}%
\pgfsys@defobject{currentmarker}{\pgfqpoint{-0.048611in}{0.000000in}}{\pgfqpoint{0.000000in}{0.000000in}}{%
\pgfpathmoveto{\pgfqpoint{0.000000in}{0.000000in}}%
\pgfpathlineto{\pgfqpoint{-0.048611in}{0.000000in}}%
\pgfusepath{stroke,fill}%
}%
\begin{pgfscope}%
\pgfsys@transformshift{1.432000in}{2.745600in}%
\pgfsys@useobject{currentmarker}{}%
\end{pgfscope}%
\end{pgfscope}%
\begin{pgfscope}%
\pgftext[x=1.157308in,y=2.697406in,left,base]{\rmfamily\fontsize{10.000000}{12.000000}\selectfont \(\displaystyle 0.6\)}%
\end{pgfscope}%
\begin{pgfscope}%
\pgfsetbuttcap%
\pgfsetroundjoin%
\definecolor{currentfill}{rgb}{0.000000,0.000000,0.000000}%
\pgfsetfillcolor{currentfill}%
\pgfsetlinewidth{0.803000pt}%
\definecolor{currentstroke}{rgb}{0.000000,0.000000,0.000000}%
\pgfsetstrokecolor{currentstroke}%
\pgfsetdash{}{0pt}%
\pgfsys@defobject{currentmarker}{\pgfqpoint{-0.048611in}{0.000000in}}{\pgfqpoint{0.000000in}{0.000000in}}{%
\pgfpathmoveto{\pgfqpoint{0.000000in}{0.000000in}}%
\pgfpathlineto{\pgfqpoint{-0.048611in}{0.000000in}}%
\pgfusepath{stroke,fill}%
}%
\begin{pgfscope}%
\pgfsys@transformshift{1.432000in}{3.484800in}%
\pgfsys@useobject{currentmarker}{}%
\end{pgfscope}%
\end{pgfscope}%
\begin{pgfscope}%
\pgftext[x=1.157308in,y=3.436606in,left,base]{\rmfamily\fontsize{10.000000}{12.000000}\selectfont \(\displaystyle 0.8\)}%
\end{pgfscope}%
\begin{pgfscope}%
\pgfsetbuttcap%
\pgfsetroundjoin%
\definecolor{currentfill}{rgb}{0.000000,0.000000,0.000000}%
\pgfsetfillcolor{currentfill}%
\pgfsetlinewidth{0.803000pt}%
\definecolor{currentstroke}{rgb}{0.000000,0.000000,0.000000}%
\pgfsetstrokecolor{currentstroke}%
\pgfsetdash{}{0pt}%
\pgfsys@defobject{currentmarker}{\pgfqpoint{-0.048611in}{0.000000in}}{\pgfqpoint{0.000000in}{0.000000in}}{%
\pgfpathmoveto{\pgfqpoint{0.000000in}{0.000000in}}%
\pgfpathlineto{\pgfqpoint{-0.048611in}{0.000000in}}%
\pgfusepath{stroke,fill}%
}%
\begin{pgfscope}%
\pgfsys@transformshift{1.432000in}{4.224000in}%
\pgfsys@useobject{currentmarker}{}%
\end{pgfscope}%
\end{pgfscope}%
\begin{pgfscope}%
\pgftext[x=1.157308in,y=4.175806in,left,base]{\rmfamily\fontsize{10.000000}{12.000000}\selectfont \(\displaystyle 1.0\)}%
\end{pgfscope}%
\begin{pgfscope}%
\pgfsetrectcap%
\pgfsetmiterjoin%
\pgfsetlinewidth{0.803000pt}%
\definecolor{currentstroke}{rgb}{0.000000,0.000000,0.000000}%
\pgfsetstrokecolor{currentstroke}%
\pgfsetdash{}{0pt}%
\pgfpathmoveto{\pgfqpoint{1.432000in}{0.528000in}}%
\pgfpathlineto{\pgfqpoint{1.432000in}{4.224000in}}%
\pgfusepath{stroke}%
\end{pgfscope}%
\begin{pgfscope}%
\pgfsetrectcap%
\pgfsetmiterjoin%
\pgfsetlinewidth{0.803000pt}%
\definecolor{currentstroke}{rgb}{0.000000,0.000000,0.000000}%
\pgfsetstrokecolor{currentstroke}%
\pgfsetdash{}{0pt}%
\pgfpathmoveto{\pgfqpoint{5.128000in}{0.528000in}}%
\pgfpathlineto{\pgfqpoint{5.128000in}{4.224000in}}%
\pgfusepath{stroke}%
\end{pgfscope}%
\begin{pgfscope}%
\pgfsetrectcap%
\pgfsetmiterjoin%
\pgfsetlinewidth{0.803000pt}%
\definecolor{currentstroke}{rgb}{0.000000,0.000000,0.000000}%
\pgfsetstrokecolor{currentstroke}%
\pgfsetdash{}{0pt}%
\pgfpathmoveto{\pgfqpoint{1.432000in}{0.528000in}}%
\pgfpathlineto{\pgfqpoint{5.128000in}{0.528000in}}%
\pgfusepath{stroke}%
\end{pgfscope}%
\begin{pgfscope}%
\pgfsetrectcap%
\pgfsetmiterjoin%
\pgfsetlinewidth{0.803000pt}%
\definecolor{currentstroke}{rgb}{0.000000,0.000000,0.000000}%
\pgfsetstrokecolor{currentstroke}%
\pgfsetdash{}{0pt}%
\pgfpathmoveto{\pgfqpoint{1.432000in}{4.224000in}}%
\pgfpathlineto{\pgfqpoint{5.128000in}{4.224000in}}%
\pgfusepath{stroke}%
\end{pgfscope}%
\end{pgfpicture}%
\makeatother%
\endgroup%
} 
\hspace{-1cm}
\scalebox{0.33}
{%% Creator: Matplotlib, PGF backend
%%
%% To include the figure in your LaTeX document, write
%%   \input{<filename>.pgf}
%%
%% Make sure the required packages are loaded in your preamble
%%   \usepackage{pgf}
%%
%% Figures using additional raster images can only be included by \input if
%% they are in the same directory as the main LaTeX file. For loading figures
%% from other directories you can use the `import` package
%%   \usepackage{import}
%% and then include the figures with
%%   \import{<path to file>}{<filename>.pgf}
%%
%% Matplotlib used the following preamble
%%   \usepackage{fontspec}
%%
\begingroup%
\makeatletter%
\begin{pgfpicture}%
\pgfpathrectangle{\pgfpointorigin}{\pgfqpoint{6.400000in}{4.800000in}}%
\pgfusepath{use as bounding box, clip}%
\begin{pgfscope}%
\pgfsetbuttcap%
\pgfsetmiterjoin%
\definecolor{currentfill}{rgb}{1.000000,1.000000,1.000000}%
\pgfsetfillcolor{currentfill}%
\pgfsetlinewidth{0.000000pt}%
\definecolor{currentstroke}{rgb}{1.000000,1.000000,1.000000}%
\pgfsetstrokecolor{currentstroke}%
\pgfsetdash{}{0pt}%
\pgfpathmoveto{\pgfqpoint{0.000000in}{0.000000in}}%
\pgfpathlineto{\pgfqpoint{6.400000in}{0.000000in}}%
\pgfpathlineto{\pgfqpoint{6.400000in}{4.800000in}}%
\pgfpathlineto{\pgfqpoint{0.000000in}{4.800000in}}%
\pgfpathclose%
\pgfusepath{fill}%
\end{pgfscope}%
\begin{pgfscope}%
\pgfsetbuttcap%
\pgfsetmiterjoin%
\definecolor{currentfill}{rgb}{1.000000,1.000000,1.000000}%
\pgfsetfillcolor{currentfill}%
\pgfsetlinewidth{0.000000pt}%
\definecolor{currentstroke}{rgb}{0.000000,0.000000,0.000000}%
\pgfsetstrokecolor{currentstroke}%
\pgfsetstrokeopacity{0.000000}%
\pgfsetdash{}{0pt}%
\pgfpathmoveto{\pgfqpoint{1.432000in}{0.528000in}}%
\pgfpathlineto{\pgfqpoint{5.128000in}{0.528000in}}%
\pgfpathlineto{\pgfqpoint{5.128000in}{4.224000in}}%
\pgfpathlineto{\pgfqpoint{1.432000in}{4.224000in}}%
\pgfpathclose%
\pgfusepath{fill}%
\end{pgfscope}%
\begin{pgfscope}%
\pgfpathrectangle{\pgfqpoint{1.432000in}{0.528000in}}{\pgfqpoint{3.696000in}{3.696000in}} %
\pgfusepath{clip}%
\pgfsetbuttcap%
\pgfsetroundjoin%
\definecolor{currentfill}{rgb}{0.123463,0.581687,0.547445}%
\pgfsetfillcolor{currentfill}%
\pgfsetlinewidth{0.000000pt}%
\definecolor{currentstroke}{rgb}{0.000000,0.000000,0.000000}%
\pgfsetstrokecolor{currentstroke}%
\pgfsetdash{}{0pt}%
\pgfpathmoveto{\pgfqpoint{1.604435in}{0.696000in}}%
\pgfpathlineto{\pgfqpoint{1.602218in}{0.699841in}}%
\pgfpathlineto{\pgfqpoint{1.597782in}{0.699841in}}%
\pgfpathlineto{\pgfqpoint{1.595565in}{0.696000in}}%
\pgfpathlineto{\pgfqpoint{1.597782in}{0.692159in}}%
\pgfpathlineto{\pgfqpoint{1.602218in}{0.692159in}}%
\pgfpathlineto{\pgfqpoint{1.604435in}{0.696000in}}%
\pgfpathlineto{\pgfqpoint{1.602218in}{0.699841in}}%
\pgfusepath{fill}%
\end{pgfscope}%
\begin{pgfscope}%
\pgfpathrectangle{\pgfqpoint{1.432000in}{0.528000in}}{\pgfqpoint{3.696000in}{3.696000in}} %
\pgfusepath{clip}%
\pgfsetbuttcap%
\pgfsetroundjoin%
\definecolor{currentfill}{rgb}{0.175841,0.441290,0.557685}%
\pgfsetfillcolor{currentfill}%
\pgfsetlinewidth{0.000000pt}%
\definecolor{currentstroke}{rgb}{0.000000,0.000000,0.000000}%
\pgfsetstrokecolor{currentstroke}%
\pgfsetdash{}{0pt}%
\pgfpathmoveto{\pgfqpoint{1.712822in}{0.696000in}}%
\pgfpathlineto{\pgfqpoint{1.710605in}{0.699841in}}%
\pgfpathlineto{\pgfqpoint{1.706169in}{0.699841in}}%
\pgfpathlineto{\pgfqpoint{1.703952in}{0.696000in}}%
\pgfpathlineto{\pgfqpoint{1.706169in}{0.692159in}}%
\pgfpathlineto{\pgfqpoint{1.710605in}{0.692159in}}%
\pgfpathlineto{\pgfqpoint{1.712822in}{0.696000in}}%
\pgfpathlineto{\pgfqpoint{1.710605in}{0.699841in}}%
\pgfusepath{fill}%
\end{pgfscope}%
\begin{pgfscope}%
\pgfpathrectangle{\pgfqpoint{1.432000in}{0.528000in}}{\pgfqpoint{3.696000in}{3.696000in}} %
\pgfusepath{clip}%
\pgfsetbuttcap%
\pgfsetroundjoin%
\definecolor{currentfill}{rgb}{0.204903,0.375746,0.553533}%
\pgfsetfillcolor{currentfill}%
\pgfsetlinewidth{0.000000pt}%
\definecolor{currentstroke}{rgb}{0.000000,0.000000,0.000000}%
\pgfsetstrokecolor{currentstroke}%
\pgfsetdash{}{0pt}%
\pgfpathmoveto{\pgfqpoint{1.816774in}{0.691565in}}%
\pgfpathlineto{\pgfqpoint{1.748304in}{0.691565in}}%
\pgfpathlineto{\pgfqpoint{1.752739in}{0.682694in}}%
\pgfpathlineto{\pgfqpoint{1.708387in}{0.696000in}}%
\pgfpathlineto{\pgfqpoint{1.752739in}{0.709306in}}%
\pgfpathlineto{\pgfqpoint{1.748304in}{0.700435in}}%
\pgfpathlineto{\pgfqpoint{1.816774in}{0.700435in}}%
\pgfpathlineto{\pgfqpoint{1.816774in}{0.691565in}}%
\pgfusepath{fill}%
\end{pgfscope}%
\begin{pgfscope}%
\pgfpathrectangle{\pgfqpoint{1.432000in}{0.528000in}}{\pgfqpoint{3.696000in}{3.696000in}} %
\pgfusepath{clip}%
\pgfsetbuttcap%
\pgfsetroundjoin%
\definecolor{currentfill}{rgb}{0.271828,0.209303,0.504434}%
\pgfsetfillcolor{currentfill}%
\pgfsetlinewidth{0.000000pt}%
\definecolor{currentstroke}{rgb}{0.000000,0.000000,0.000000}%
\pgfsetstrokecolor{currentstroke}%
\pgfsetdash{}{0pt}%
\pgfpathmoveto{\pgfqpoint{1.821209in}{0.696000in}}%
\pgfpathlineto{\pgfqpoint{1.818992in}{0.699841in}}%
\pgfpathlineto{\pgfqpoint{1.814557in}{0.699841in}}%
\pgfpathlineto{\pgfqpoint{1.812339in}{0.696000in}}%
\pgfpathlineto{\pgfqpoint{1.814557in}{0.692159in}}%
\pgfpathlineto{\pgfqpoint{1.818992in}{0.692159in}}%
\pgfpathlineto{\pgfqpoint{1.821209in}{0.696000in}}%
\pgfpathlineto{\pgfqpoint{1.818992in}{0.699841in}}%
\pgfusepath{fill}%
\end{pgfscope}%
\begin{pgfscope}%
\pgfpathrectangle{\pgfqpoint{1.432000in}{0.528000in}}{\pgfqpoint{3.696000in}{3.696000in}} %
\pgfusepath{clip}%
\pgfsetbuttcap%
\pgfsetroundjoin%
\definecolor{currentfill}{rgb}{0.137339,0.662252,0.515571}%
\pgfsetfillcolor{currentfill}%
\pgfsetlinewidth{0.000000pt}%
\definecolor{currentstroke}{rgb}{0.000000,0.000000,0.000000}%
\pgfsetstrokecolor{currentstroke}%
\pgfsetdash{}{0pt}%
\pgfpathmoveto{\pgfqpoint{1.925161in}{0.691565in}}%
\pgfpathlineto{\pgfqpoint{1.856691in}{0.691565in}}%
\pgfpathlineto{\pgfqpoint{1.861126in}{0.682694in}}%
\pgfpathlineto{\pgfqpoint{1.816774in}{0.696000in}}%
\pgfpathlineto{\pgfqpoint{1.861126in}{0.709306in}}%
\pgfpathlineto{\pgfqpoint{1.856691in}{0.700435in}}%
\pgfpathlineto{\pgfqpoint{1.925161in}{0.700435in}}%
\pgfpathlineto{\pgfqpoint{1.925161in}{0.691565in}}%
\pgfusepath{fill}%
\end{pgfscope}%
\begin{pgfscope}%
\pgfpathrectangle{\pgfqpoint{1.432000in}{0.528000in}}{\pgfqpoint{3.696000in}{3.696000in}} %
\pgfusepath{clip}%
\pgfsetbuttcap%
\pgfsetroundjoin%
\definecolor{currentfill}{rgb}{0.128729,0.563265,0.551229}%
\pgfsetfillcolor{currentfill}%
\pgfsetlinewidth{0.000000pt}%
\definecolor{currentstroke}{rgb}{0.000000,0.000000,0.000000}%
\pgfsetstrokecolor{currentstroke}%
\pgfsetdash{}{0pt}%
\pgfpathmoveto{\pgfqpoint{2.033548in}{0.691565in}}%
\pgfpathlineto{\pgfqpoint{1.965078in}{0.691565in}}%
\pgfpathlineto{\pgfqpoint{1.969513in}{0.682694in}}%
\pgfpathlineto{\pgfqpoint{1.925161in}{0.696000in}}%
\pgfpathlineto{\pgfqpoint{1.969513in}{0.709306in}}%
\pgfpathlineto{\pgfqpoint{1.965078in}{0.700435in}}%
\pgfpathlineto{\pgfqpoint{2.033548in}{0.700435in}}%
\pgfpathlineto{\pgfqpoint{2.033548in}{0.691565in}}%
\pgfusepath{fill}%
\end{pgfscope}%
\begin{pgfscope}%
\pgfpathrectangle{\pgfqpoint{1.432000in}{0.528000in}}{\pgfqpoint{3.696000in}{3.696000in}} %
\pgfusepath{clip}%
\pgfsetbuttcap%
\pgfsetroundjoin%
\definecolor{currentfill}{rgb}{0.141935,0.526453,0.555991}%
\pgfsetfillcolor{currentfill}%
\pgfsetlinewidth{0.000000pt}%
\definecolor{currentstroke}{rgb}{0.000000,0.000000,0.000000}%
\pgfsetstrokecolor{currentstroke}%
\pgfsetdash{}{0pt}%
\pgfpathmoveto{\pgfqpoint{2.141935in}{0.691565in}}%
\pgfpathlineto{\pgfqpoint{1.965078in}{0.691565in}}%
\pgfpathlineto{\pgfqpoint{1.969513in}{0.682694in}}%
\pgfpathlineto{\pgfqpoint{1.925161in}{0.696000in}}%
\pgfpathlineto{\pgfqpoint{1.969513in}{0.709306in}}%
\pgfpathlineto{\pgfqpoint{1.965078in}{0.700435in}}%
\pgfpathlineto{\pgfqpoint{2.141935in}{0.700435in}}%
\pgfpathlineto{\pgfqpoint{2.141935in}{0.691565in}}%
\pgfusepath{fill}%
\end{pgfscope}%
\begin{pgfscope}%
\pgfpathrectangle{\pgfqpoint{1.432000in}{0.528000in}}{\pgfqpoint{3.696000in}{3.696000in}} %
\pgfusepath{clip}%
\pgfsetbuttcap%
\pgfsetroundjoin%
\definecolor{currentfill}{rgb}{0.221989,0.339161,0.548752}%
\pgfsetfillcolor{currentfill}%
\pgfsetlinewidth{0.000000pt}%
\definecolor{currentstroke}{rgb}{0.000000,0.000000,0.000000}%
\pgfsetstrokecolor{currentstroke}%
\pgfsetdash{}{0pt}%
\pgfpathmoveto{\pgfqpoint{2.250323in}{0.691565in}}%
\pgfpathlineto{\pgfqpoint{2.073465in}{0.691565in}}%
\pgfpathlineto{\pgfqpoint{2.077900in}{0.682694in}}%
\pgfpathlineto{\pgfqpoint{2.033548in}{0.696000in}}%
\pgfpathlineto{\pgfqpoint{2.077900in}{0.709306in}}%
\pgfpathlineto{\pgfqpoint{2.073465in}{0.700435in}}%
\pgfpathlineto{\pgfqpoint{2.250323in}{0.700435in}}%
\pgfpathlineto{\pgfqpoint{2.250323in}{0.691565in}}%
\pgfusepath{fill}%
\end{pgfscope}%
\begin{pgfscope}%
\pgfpathrectangle{\pgfqpoint{1.432000in}{0.528000in}}{\pgfqpoint{3.696000in}{3.696000in}} %
\pgfusepath{clip}%
\pgfsetbuttcap%
\pgfsetroundjoin%
\definecolor{currentfill}{rgb}{0.180629,0.429975,0.557282}%
\pgfsetfillcolor{currentfill}%
\pgfsetlinewidth{0.000000pt}%
\definecolor{currentstroke}{rgb}{0.000000,0.000000,0.000000}%
\pgfsetstrokecolor{currentstroke}%
\pgfsetdash{}{0pt}%
\pgfpathmoveto{\pgfqpoint{2.358710in}{0.691565in}}%
\pgfpathlineto{\pgfqpoint{2.073465in}{0.691565in}}%
\pgfpathlineto{\pgfqpoint{2.077900in}{0.682694in}}%
\pgfpathlineto{\pgfqpoint{2.033548in}{0.696000in}}%
\pgfpathlineto{\pgfqpoint{2.077900in}{0.709306in}}%
\pgfpathlineto{\pgfqpoint{2.073465in}{0.700435in}}%
\pgfpathlineto{\pgfqpoint{2.358710in}{0.700435in}}%
\pgfpathlineto{\pgfqpoint{2.358710in}{0.691565in}}%
\pgfusepath{fill}%
\end{pgfscope}%
\begin{pgfscope}%
\pgfpathrectangle{\pgfqpoint{1.432000in}{0.528000in}}{\pgfqpoint{3.696000in}{3.696000in}} %
\pgfusepath{clip}%
\pgfsetbuttcap%
\pgfsetroundjoin%
\definecolor{currentfill}{rgb}{0.268510,0.009605,0.335427}%
\pgfsetfillcolor{currentfill}%
\pgfsetlinewidth{0.000000pt}%
\definecolor{currentstroke}{rgb}{0.000000,0.000000,0.000000}%
\pgfsetstrokecolor{currentstroke}%
\pgfsetdash{}{0pt}%
\pgfpathmoveto{\pgfqpoint{2.467097in}{0.691565in}}%
\pgfpathlineto{\pgfqpoint{2.073465in}{0.691565in}}%
\pgfpathlineto{\pgfqpoint{2.077900in}{0.682694in}}%
\pgfpathlineto{\pgfqpoint{2.033548in}{0.696000in}}%
\pgfpathlineto{\pgfqpoint{2.077900in}{0.709306in}}%
\pgfpathlineto{\pgfqpoint{2.073465in}{0.700435in}}%
\pgfpathlineto{\pgfqpoint{2.467097in}{0.700435in}}%
\pgfpathlineto{\pgfqpoint{2.467097in}{0.691565in}}%
\pgfusepath{fill}%
\end{pgfscope}%
\begin{pgfscope}%
\pgfpathrectangle{\pgfqpoint{1.432000in}{0.528000in}}{\pgfqpoint{3.696000in}{3.696000in}} %
\pgfusepath{clip}%
\pgfsetbuttcap%
\pgfsetroundjoin%
\definecolor{currentfill}{rgb}{0.244972,0.287675,0.537260}%
\pgfsetfillcolor{currentfill}%
\pgfsetlinewidth{0.000000pt}%
\definecolor{currentstroke}{rgb}{0.000000,0.000000,0.000000}%
\pgfsetstrokecolor{currentstroke}%
\pgfsetdash{}{0pt}%
\pgfpathmoveto{\pgfqpoint{2.467097in}{0.691565in}}%
\pgfpathlineto{\pgfqpoint{2.181852in}{0.691565in}}%
\pgfpathlineto{\pgfqpoint{2.186287in}{0.682694in}}%
\pgfpathlineto{\pgfqpoint{2.141935in}{0.696000in}}%
\pgfpathlineto{\pgfqpoint{2.186287in}{0.709306in}}%
\pgfpathlineto{\pgfqpoint{2.181852in}{0.700435in}}%
\pgfpathlineto{\pgfqpoint{2.467097in}{0.700435in}}%
\pgfpathlineto{\pgfqpoint{2.467097in}{0.691565in}}%
\pgfusepath{fill}%
\end{pgfscope}%
\begin{pgfscope}%
\pgfpathrectangle{\pgfqpoint{1.432000in}{0.528000in}}{\pgfqpoint{3.696000in}{3.696000in}} %
\pgfusepath{clip}%
\pgfsetbuttcap%
\pgfsetroundjoin%
\definecolor{currentfill}{rgb}{0.174274,0.445044,0.557792}%
\pgfsetfillcolor{currentfill}%
\pgfsetlinewidth{0.000000pt}%
\definecolor{currentstroke}{rgb}{0.000000,0.000000,0.000000}%
\pgfsetstrokecolor{currentstroke}%
\pgfsetdash{}{0pt}%
\pgfpathmoveto{\pgfqpoint{2.575484in}{0.691565in}}%
\pgfpathlineto{\pgfqpoint{2.181852in}{0.691565in}}%
\pgfpathlineto{\pgfqpoint{2.186287in}{0.682694in}}%
\pgfpathlineto{\pgfqpoint{2.141935in}{0.696000in}}%
\pgfpathlineto{\pgfqpoint{2.186287in}{0.709306in}}%
\pgfpathlineto{\pgfqpoint{2.181852in}{0.700435in}}%
\pgfpathlineto{\pgfqpoint{2.575484in}{0.700435in}}%
\pgfpathlineto{\pgfqpoint{2.575484in}{0.691565in}}%
\pgfusepath{fill}%
\end{pgfscope}%
\begin{pgfscope}%
\pgfpathrectangle{\pgfqpoint{1.432000in}{0.528000in}}{\pgfqpoint{3.696000in}{3.696000in}} %
\pgfusepath{clip}%
\pgfsetbuttcap%
\pgfsetroundjoin%
\definecolor{currentfill}{rgb}{0.220057,0.343307,0.549413}%
\pgfsetfillcolor{currentfill}%
\pgfsetlinewidth{0.000000pt}%
\definecolor{currentstroke}{rgb}{0.000000,0.000000,0.000000}%
\pgfsetstrokecolor{currentstroke}%
\pgfsetdash{}{0pt}%
\pgfpathmoveto{\pgfqpoint{2.683871in}{0.691565in}}%
\pgfpathlineto{\pgfqpoint{2.290239in}{0.691565in}}%
\pgfpathlineto{\pgfqpoint{2.294675in}{0.682694in}}%
\pgfpathlineto{\pgfqpoint{2.250323in}{0.696000in}}%
\pgfpathlineto{\pgfqpoint{2.294675in}{0.709306in}}%
\pgfpathlineto{\pgfqpoint{2.290239in}{0.700435in}}%
\pgfpathlineto{\pgfqpoint{2.683871in}{0.700435in}}%
\pgfpathlineto{\pgfqpoint{2.683871in}{0.691565in}}%
\pgfusepath{fill}%
\end{pgfscope}%
\begin{pgfscope}%
\pgfpathrectangle{\pgfqpoint{1.432000in}{0.528000in}}{\pgfqpoint{3.696000in}{3.696000in}} %
\pgfusepath{clip}%
\pgfsetbuttcap%
\pgfsetroundjoin%
\definecolor{currentfill}{rgb}{0.282910,0.105393,0.426902}%
\pgfsetfillcolor{currentfill}%
\pgfsetlinewidth{0.000000pt}%
\definecolor{currentstroke}{rgb}{0.000000,0.000000,0.000000}%
\pgfsetstrokecolor{currentstroke}%
\pgfsetdash{}{0pt}%
\pgfpathmoveto{\pgfqpoint{2.792258in}{0.691565in}}%
\pgfpathlineto{\pgfqpoint{2.290239in}{0.691565in}}%
\pgfpathlineto{\pgfqpoint{2.294675in}{0.682694in}}%
\pgfpathlineto{\pgfqpoint{2.250323in}{0.696000in}}%
\pgfpathlineto{\pgfqpoint{2.294675in}{0.709306in}}%
\pgfpathlineto{\pgfqpoint{2.290239in}{0.700435in}}%
\pgfpathlineto{\pgfqpoint{2.792258in}{0.700435in}}%
\pgfpathlineto{\pgfqpoint{2.792258in}{0.691565in}}%
\pgfusepath{fill}%
\end{pgfscope}%
\begin{pgfscope}%
\pgfpathrectangle{\pgfqpoint{1.432000in}{0.528000in}}{\pgfqpoint{3.696000in}{3.696000in}} %
\pgfusepath{clip}%
\pgfsetbuttcap%
\pgfsetroundjoin%
\definecolor{currentfill}{rgb}{0.280255,0.165693,0.476498}%
\pgfsetfillcolor{currentfill}%
\pgfsetlinewidth{0.000000pt}%
\definecolor{currentstroke}{rgb}{0.000000,0.000000,0.000000}%
\pgfsetstrokecolor{currentstroke}%
\pgfsetdash{}{0pt}%
\pgfpathmoveto{\pgfqpoint{2.792258in}{0.691565in}}%
\pgfpathlineto{\pgfqpoint{2.398626in}{0.691565in}}%
\pgfpathlineto{\pgfqpoint{2.403062in}{0.682694in}}%
\pgfpathlineto{\pgfqpoint{2.358710in}{0.696000in}}%
\pgfpathlineto{\pgfqpoint{2.403062in}{0.709306in}}%
\pgfpathlineto{\pgfqpoint{2.398626in}{0.700435in}}%
\pgfpathlineto{\pgfqpoint{2.792258in}{0.700435in}}%
\pgfpathlineto{\pgfqpoint{2.792258in}{0.691565in}}%
\pgfusepath{fill}%
\end{pgfscope}%
\begin{pgfscope}%
\pgfpathrectangle{\pgfqpoint{1.432000in}{0.528000in}}{\pgfqpoint{3.696000in}{3.696000in}} %
\pgfusepath{clip}%
\pgfsetbuttcap%
\pgfsetroundjoin%
\definecolor{currentfill}{rgb}{0.280894,0.078907,0.402329}%
\pgfsetfillcolor{currentfill}%
\pgfsetlinewidth{0.000000pt}%
\definecolor{currentstroke}{rgb}{0.000000,0.000000,0.000000}%
\pgfsetstrokecolor{currentstroke}%
\pgfsetdash{}{0pt}%
\pgfpathmoveto{\pgfqpoint{2.900645in}{0.691565in}}%
\pgfpathlineto{\pgfqpoint{2.615401in}{0.691565in}}%
\pgfpathlineto{\pgfqpoint{2.619836in}{0.682694in}}%
\pgfpathlineto{\pgfqpoint{2.575484in}{0.696000in}}%
\pgfpathlineto{\pgfqpoint{2.619836in}{0.709306in}}%
\pgfpathlineto{\pgfqpoint{2.615401in}{0.700435in}}%
\pgfpathlineto{\pgfqpoint{2.900645in}{0.700435in}}%
\pgfpathlineto{\pgfqpoint{2.900645in}{0.691565in}}%
\pgfusepath{fill}%
\end{pgfscope}%
\begin{pgfscope}%
\pgfpathrectangle{\pgfqpoint{1.432000in}{0.528000in}}{\pgfqpoint{3.696000in}{3.696000in}} %
\pgfusepath{clip}%
\pgfsetbuttcap%
\pgfsetroundjoin%
\definecolor{currentfill}{rgb}{0.280868,0.160771,0.472899}%
\pgfsetfillcolor{currentfill}%
\pgfsetlinewidth{0.000000pt}%
\definecolor{currentstroke}{rgb}{0.000000,0.000000,0.000000}%
\pgfsetstrokecolor{currentstroke}%
\pgfsetdash{}{0pt}%
\pgfpathmoveto{\pgfqpoint{2.899243in}{0.691792in}}%
\pgfpathlineto{\pgfqpoint{2.611950in}{0.787557in}}%
\pgfpathlineto{\pgfqpoint{2.613352in}{0.777739in}}%
\pgfpathlineto{\pgfqpoint{2.575484in}{0.804387in}}%
\pgfpathlineto{\pgfqpoint{2.621767in}{0.802985in}}%
\pgfpathlineto{\pgfqpoint{2.614755in}{0.795972in}}%
\pgfpathlineto{\pgfqpoint{2.902048in}{0.700208in}}%
\pgfpathlineto{\pgfqpoint{2.899243in}{0.691792in}}%
\pgfusepath{fill}%
\end{pgfscope}%
\begin{pgfscope}%
\pgfpathrectangle{\pgfqpoint{1.432000in}{0.528000in}}{\pgfqpoint{3.696000in}{3.696000in}} %
\pgfusepath{clip}%
\pgfsetbuttcap%
\pgfsetroundjoin%
\definecolor{currentfill}{rgb}{0.278826,0.175490,0.483397}%
\pgfsetfillcolor{currentfill}%
\pgfsetlinewidth{0.000000pt}%
\definecolor{currentstroke}{rgb}{0.000000,0.000000,0.000000}%
\pgfsetstrokecolor{currentstroke}%
\pgfsetdash{}{0pt}%
\pgfpathmoveto{\pgfqpoint{3.009032in}{0.691565in}}%
\pgfpathlineto{\pgfqpoint{2.723788in}{0.691565in}}%
\pgfpathlineto{\pgfqpoint{2.728223in}{0.682694in}}%
\pgfpathlineto{\pgfqpoint{2.683871in}{0.696000in}}%
\pgfpathlineto{\pgfqpoint{2.728223in}{0.709306in}}%
\pgfpathlineto{\pgfqpoint{2.723788in}{0.700435in}}%
\pgfpathlineto{\pgfqpoint{3.009032in}{0.700435in}}%
\pgfpathlineto{\pgfqpoint{3.009032in}{0.691565in}}%
\pgfusepath{fill}%
\end{pgfscope}%
\begin{pgfscope}%
\pgfpathrectangle{\pgfqpoint{1.432000in}{0.528000in}}{\pgfqpoint{3.696000in}{3.696000in}} %
\pgfusepath{clip}%
\pgfsetbuttcap%
\pgfsetroundjoin%
\definecolor{currentfill}{rgb}{0.278791,0.062145,0.386592}%
\pgfsetfillcolor{currentfill}%
\pgfsetlinewidth{0.000000pt}%
\definecolor{currentstroke}{rgb}{0.000000,0.000000,0.000000}%
\pgfsetstrokecolor{currentstroke}%
\pgfsetdash{}{0pt}%
\pgfpathmoveto{\pgfqpoint{3.009032in}{0.691565in}}%
\pgfpathlineto{\pgfqpoint{2.832175in}{0.691565in}}%
\pgfpathlineto{\pgfqpoint{2.836610in}{0.682694in}}%
\pgfpathlineto{\pgfqpoint{2.792258in}{0.696000in}}%
\pgfpathlineto{\pgfqpoint{2.836610in}{0.709306in}}%
\pgfpathlineto{\pgfqpoint{2.832175in}{0.700435in}}%
\pgfpathlineto{\pgfqpoint{3.009032in}{0.700435in}}%
\pgfpathlineto{\pgfqpoint{3.009032in}{0.691565in}}%
\pgfusepath{fill}%
\end{pgfscope}%
\begin{pgfscope}%
\pgfpathrectangle{\pgfqpoint{1.432000in}{0.528000in}}{\pgfqpoint{3.696000in}{3.696000in}} %
\pgfusepath{clip}%
\pgfsetbuttcap%
\pgfsetroundjoin%
\definecolor{currentfill}{rgb}{0.277134,0.185228,0.489898}%
\pgfsetfillcolor{currentfill}%
\pgfsetlinewidth{0.000000pt}%
\definecolor{currentstroke}{rgb}{0.000000,0.000000,0.000000}%
\pgfsetstrokecolor{currentstroke}%
\pgfsetdash{}{0pt}%
\pgfpathmoveto{\pgfqpoint{3.007630in}{0.691792in}}%
\pgfpathlineto{\pgfqpoint{2.720337in}{0.787557in}}%
\pgfpathlineto{\pgfqpoint{2.721739in}{0.777739in}}%
\pgfpathlineto{\pgfqpoint{2.683871in}{0.804387in}}%
\pgfpathlineto{\pgfqpoint{2.730155in}{0.802985in}}%
\pgfpathlineto{\pgfqpoint{2.723142in}{0.795972in}}%
\pgfpathlineto{\pgfqpoint{3.010435in}{0.700208in}}%
\pgfpathlineto{\pgfqpoint{3.007630in}{0.691792in}}%
\pgfusepath{fill}%
\end{pgfscope}%
\begin{pgfscope}%
\pgfpathrectangle{\pgfqpoint{1.432000in}{0.528000in}}{\pgfqpoint{3.696000in}{3.696000in}} %
\pgfusepath{clip}%
\pgfsetbuttcap%
\pgfsetroundjoin%
\definecolor{currentfill}{rgb}{0.282290,0.145912,0.461510}%
\pgfsetfillcolor{currentfill}%
\pgfsetlinewidth{0.000000pt}%
\definecolor{currentstroke}{rgb}{0.000000,0.000000,0.000000}%
\pgfsetstrokecolor{currentstroke}%
\pgfsetdash{}{0pt}%
\pgfpathmoveto{\pgfqpoint{3.117419in}{0.691565in}}%
\pgfpathlineto{\pgfqpoint{2.940562in}{0.691565in}}%
\pgfpathlineto{\pgfqpoint{2.944997in}{0.682694in}}%
\pgfpathlineto{\pgfqpoint{2.900645in}{0.696000in}}%
\pgfpathlineto{\pgfqpoint{2.944997in}{0.709306in}}%
\pgfpathlineto{\pgfqpoint{2.940562in}{0.700435in}}%
\pgfpathlineto{\pgfqpoint{3.117419in}{0.700435in}}%
\pgfpathlineto{\pgfqpoint{3.117419in}{0.691565in}}%
\pgfusepath{fill}%
\end{pgfscope}%
\begin{pgfscope}%
\pgfpathrectangle{\pgfqpoint{1.432000in}{0.528000in}}{\pgfqpoint{3.696000in}{3.696000in}} %
\pgfusepath{clip}%
\pgfsetbuttcap%
\pgfsetroundjoin%
\definecolor{currentfill}{rgb}{0.277018,0.050344,0.375715}%
\pgfsetfillcolor{currentfill}%
\pgfsetlinewidth{0.000000pt}%
\definecolor{currentstroke}{rgb}{0.000000,0.000000,0.000000}%
\pgfsetstrokecolor{currentstroke}%
\pgfsetdash{}{0pt}%
\pgfpathmoveto{\pgfqpoint{3.115436in}{0.692033in}}%
\pgfpathlineto{\pgfqpoint{2.934364in}{0.782569in}}%
\pgfpathlineto{\pgfqpoint{2.934364in}{0.772651in}}%
\pgfpathlineto{\pgfqpoint{2.900645in}{0.804387in}}%
\pgfpathlineto{\pgfqpoint{2.946265in}{0.796453in}}%
\pgfpathlineto{\pgfqpoint{2.938331in}{0.790503in}}%
\pgfpathlineto{\pgfqpoint{3.119403in}{0.699967in}}%
\pgfpathlineto{\pgfqpoint{3.115436in}{0.692033in}}%
\pgfusepath{fill}%
\end{pgfscope}%
\begin{pgfscope}%
\pgfpathrectangle{\pgfqpoint{1.432000in}{0.528000in}}{\pgfqpoint{3.696000in}{3.696000in}} %
\pgfusepath{clip}%
\pgfsetbuttcap%
\pgfsetroundjoin%
\definecolor{currentfill}{rgb}{0.274952,0.037752,0.364543}%
\pgfsetfillcolor{currentfill}%
\pgfsetlinewidth{0.000000pt}%
\definecolor{currentstroke}{rgb}{0.000000,0.000000,0.000000}%
\pgfsetstrokecolor{currentstroke}%
\pgfsetdash{}{0pt}%
\pgfpathmoveto{\pgfqpoint{3.225806in}{0.691565in}}%
\pgfpathlineto{\pgfqpoint{3.048949in}{0.691565in}}%
\pgfpathlineto{\pgfqpoint{3.053384in}{0.682694in}}%
\pgfpathlineto{\pgfqpoint{3.009032in}{0.696000in}}%
\pgfpathlineto{\pgfqpoint{3.053384in}{0.709306in}}%
\pgfpathlineto{\pgfqpoint{3.048949in}{0.700435in}}%
\pgfpathlineto{\pgfqpoint{3.225806in}{0.700435in}}%
\pgfpathlineto{\pgfqpoint{3.225806in}{0.691565in}}%
\pgfusepath{fill}%
\end{pgfscope}%
\begin{pgfscope}%
\pgfpathrectangle{\pgfqpoint{1.432000in}{0.528000in}}{\pgfqpoint{3.696000in}{3.696000in}} %
\pgfusepath{clip}%
\pgfsetbuttcap%
\pgfsetroundjoin%
\definecolor{currentfill}{rgb}{0.273006,0.204520,0.501721}%
\pgfsetfillcolor{currentfill}%
\pgfsetlinewidth{0.000000pt}%
\definecolor{currentstroke}{rgb}{0.000000,0.000000,0.000000}%
\pgfsetstrokecolor{currentstroke}%
\pgfsetdash{}{0pt}%
\pgfpathmoveto{\pgfqpoint{3.225806in}{0.691565in}}%
\pgfpathlineto{\pgfqpoint{3.157336in}{0.691565in}}%
\pgfpathlineto{\pgfqpoint{3.161771in}{0.682694in}}%
\pgfpathlineto{\pgfqpoint{3.117419in}{0.696000in}}%
\pgfpathlineto{\pgfqpoint{3.161771in}{0.709306in}}%
\pgfpathlineto{\pgfqpoint{3.157336in}{0.700435in}}%
\pgfpathlineto{\pgfqpoint{3.225806in}{0.700435in}}%
\pgfpathlineto{\pgfqpoint{3.225806in}{0.691565in}}%
\pgfusepath{fill}%
\end{pgfscope}%
\begin{pgfscope}%
\pgfpathrectangle{\pgfqpoint{1.432000in}{0.528000in}}{\pgfqpoint{3.696000in}{3.696000in}} %
\pgfusepath{clip}%
\pgfsetbuttcap%
\pgfsetroundjoin%
\definecolor{currentfill}{rgb}{0.281924,0.089666,0.412415}%
\pgfsetfillcolor{currentfill}%
\pgfsetlinewidth{0.000000pt}%
\definecolor{currentstroke}{rgb}{0.000000,0.000000,0.000000}%
\pgfsetstrokecolor{currentstroke}%
\pgfsetdash{}{0pt}%
\pgfpathmoveto{\pgfqpoint{3.222670in}{0.692864in}}%
\pgfpathlineto{\pgfqpoint{3.142509in}{0.773025in}}%
\pgfpathlineto{\pgfqpoint{3.139372in}{0.763617in}}%
\pgfpathlineto{\pgfqpoint{3.117419in}{0.804387in}}%
\pgfpathlineto{\pgfqpoint{3.158189in}{0.782434in}}%
\pgfpathlineto{\pgfqpoint{3.148781in}{0.779298in}}%
\pgfpathlineto{\pgfqpoint{3.228943in}{0.699136in}}%
\pgfpathlineto{\pgfqpoint{3.222670in}{0.692864in}}%
\pgfusepath{fill}%
\end{pgfscope}%
\begin{pgfscope}%
\pgfpathrectangle{\pgfqpoint{1.432000in}{0.528000in}}{\pgfqpoint{3.696000in}{3.696000in}} %
\pgfusepath{clip}%
\pgfsetbuttcap%
\pgfsetroundjoin%
\definecolor{currentfill}{rgb}{0.169646,0.456262,0.558030}%
\pgfsetfillcolor{currentfill}%
\pgfsetlinewidth{0.000000pt}%
\definecolor{currentstroke}{rgb}{0.000000,0.000000,0.000000}%
\pgfsetstrokecolor{currentstroke}%
\pgfsetdash{}{0pt}%
\pgfpathmoveto{\pgfqpoint{3.334194in}{0.691565in}}%
\pgfpathlineto{\pgfqpoint{3.265723in}{0.691565in}}%
\pgfpathlineto{\pgfqpoint{3.270158in}{0.682694in}}%
\pgfpathlineto{\pgfqpoint{3.225806in}{0.696000in}}%
\pgfpathlineto{\pgfqpoint{3.270158in}{0.709306in}}%
\pgfpathlineto{\pgfqpoint{3.265723in}{0.700435in}}%
\pgfpathlineto{\pgfqpoint{3.334194in}{0.700435in}}%
\pgfpathlineto{\pgfqpoint{3.334194in}{0.691565in}}%
\pgfusepath{fill}%
\end{pgfscope}%
\begin{pgfscope}%
\pgfpathrectangle{\pgfqpoint{1.432000in}{0.528000in}}{\pgfqpoint{3.696000in}{3.696000in}} %
\pgfusepath{clip}%
\pgfsetbuttcap%
\pgfsetroundjoin%
\definecolor{currentfill}{rgb}{0.150476,0.504369,0.557430}%
\pgfsetfillcolor{currentfill}%
\pgfsetlinewidth{0.000000pt}%
\definecolor{currentstroke}{rgb}{0.000000,0.000000,0.000000}%
\pgfsetstrokecolor{currentstroke}%
\pgfsetdash{}{0pt}%
\pgfpathmoveto{\pgfqpoint{3.442581in}{0.691565in}}%
\pgfpathlineto{\pgfqpoint{3.374110in}{0.691565in}}%
\pgfpathlineto{\pgfqpoint{3.378546in}{0.682694in}}%
\pgfpathlineto{\pgfqpoint{3.334194in}{0.696000in}}%
\pgfpathlineto{\pgfqpoint{3.378546in}{0.709306in}}%
\pgfpathlineto{\pgfqpoint{3.374110in}{0.700435in}}%
\pgfpathlineto{\pgfqpoint{3.442581in}{0.700435in}}%
\pgfpathlineto{\pgfqpoint{3.442581in}{0.691565in}}%
\pgfusepath{fill}%
\end{pgfscope}%
\begin{pgfscope}%
\pgfpathrectangle{\pgfqpoint{1.432000in}{0.528000in}}{\pgfqpoint{3.696000in}{3.696000in}} %
\pgfusepath{clip}%
\pgfsetbuttcap%
\pgfsetroundjoin%
\definecolor{currentfill}{rgb}{0.131172,0.555899,0.552459}%
\pgfsetfillcolor{currentfill}%
\pgfsetlinewidth{0.000000pt}%
\definecolor{currentstroke}{rgb}{0.000000,0.000000,0.000000}%
\pgfsetstrokecolor{currentstroke}%
\pgfsetdash{}{0pt}%
\pgfpathmoveto{\pgfqpoint{3.550968in}{0.691565in}}%
\pgfpathlineto{\pgfqpoint{3.482497in}{0.691565in}}%
\pgfpathlineto{\pgfqpoint{3.486933in}{0.682694in}}%
\pgfpathlineto{\pgfqpoint{3.442581in}{0.696000in}}%
\pgfpathlineto{\pgfqpoint{3.486933in}{0.709306in}}%
\pgfpathlineto{\pgfqpoint{3.482497in}{0.700435in}}%
\pgfpathlineto{\pgfqpoint{3.550968in}{0.700435in}}%
\pgfpathlineto{\pgfqpoint{3.550968in}{0.691565in}}%
\pgfusepath{fill}%
\end{pgfscope}%
\begin{pgfscope}%
\pgfpathrectangle{\pgfqpoint{1.432000in}{0.528000in}}{\pgfqpoint{3.696000in}{3.696000in}} %
\pgfusepath{clip}%
\pgfsetbuttcap%
\pgfsetroundjoin%
\definecolor{currentfill}{rgb}{0.123444,0.636809,0.528763}%
\pgfsetfillcolor{currentfill}%
\pgfsetlinewidth{0.000000pt}%
\definecolor{currentstroke}{rgb}{0.000000,0.000000,0.000000}%
\pgfsetstrokecolor{currentstroke}%
\pgfsetdash{}{0pt}%
\pgfpathmoveto{\pgfqpoint{3.659355in}{0.691565in}}%
\pgfpathlineto{\pgfqpoint{3.590885in}{0.691565in}}%
\pgfpathlineto{\pgfqpoint{3.595320in}{0.682694in}}%
\pgfpathlineto{\pgfqpoint{3.550968in}{0.696000in}}%
\pgfpathlineto{\pgfqpoint{3.595320in}{0.709306in}}%
\pgfpathlineto{\pgfqpoint{3.590885in}{0.700435in}}%
\pgfpathlineto{\pgfqpoint{3.659355in}{0.700435in}}%
\pgfpathlineto{\pgfqpoint{3.659355in}{0.691565in}}%
\pgfusepath{fill}%
\end{pgfscope}%
\begin{pgfscope}%
\pgfpathrectangle{\pgfqpoint{1.432000in}{0.528000in}}{\pgfqpoint{3.696000in}{3.696000in}} %
\pgfusepath{clip}%
\pgfsetbuttcap%
\pgfsetroundjoin%
\definecolor{currentfill}{rgb}{0.121831,0.589055,0.545623}%
\pgfsetfillcolor{currentfill}%
\pgfsetlinewidth{0.000000pt}%
\definecolor{currentstroke}{rgb}{0.000000,0.000000,0.000000}%
\pgfsetstrokecolor{currentstroke}%
\pgfsetdash{}{0pt}%
\pgfpathmoveto{\pgfqpoint{3.767742in}{0.691565in}}%
\pgfpathlineto{\pgfqpoint{3.699272in}{0.691565in}}%
\pgfpathlineto{\pgfqpoint{3.703707in}{0.682694in}}%
\pgfpathlineto{\pgfqpoint{3.659355in}{0.696000in}}%
\pgfpathlineto{\pgfqpoint{3.703707in}{0.709306in}}%
\pgfpathlineto{\pgfqpoint{3.699272in}{0.700435in}}%
\pgfpathlineto{\pgfqpoint{3.767742in}{0.700435in}}%
\pgfpathlineto{\pgfqpoint{3.767742in}{0.691565in}}%
\pgfusepath{fill}%
\end{pgfscope}%
\begin{pgfscope}%
\pgfpathrectangle{\pgfqpoint{1.432000in}{0.528000in}}{\pgfqpoint{3.696000in}{3.696000in}} %
\pgfusepath{clip}%
\pgfsetbuttcap%
\pgfsetroundjoin%
\definecolor{currentfill}{rgb}{0.253935,0.265254,0.529983}%
\pgfsetfillcolor{currentfill}%
\pgfsetlinewidth{0.000000pt}%
\definecolor{currentstroke}{rgb}{0.000000,0.000000,0.000000}%
\pgfsetstrokecolor{currentstroke}%
\pgfsetdash{}{0pt}%
\pgfpathmoveto{\pgfqpoint{3.876129in}{0.691565in}}%
\pgfpathlineto{\pgfqpoint{3.807659in}{0.691565in}}%
\pgfpathlineto{\pgfqpoint{3.812094in}{0.682694in}}%
\pgfpathlineto{\pgfqpoint{3.767742in}{0.696000in}}%
\pgfpathlineto{\pgfqpoint{3.812094in}{0.709306in}}%
\pgfpathlineto{\pgfqpoint{3.807659in}{0.700435in}}%
\pgfpathlineto{\pgfqpoint{3.876129in}{0.700435in}}%
\pgfpathlineto{\pgfqpoint{3.876129in}{0.691565in}}%
\pgfusepath{fill}%
\end{pgfscope}%
\begin{pgfscope}%
\pgfpathrectangle{\pgfqpoint{1.432000in}{0.528000in}}{\pgfqpoint{3.696000in}{3.696000in}} %
\pgfusepath{clip}%
\pgfsetbuttcap%
\pgfsetroundjoin%
\definecolor{currentfill}{rgb}{0.282290,0.145912,0.461510}%
\pgfsetfillcolor{currentfill}%
\pgfsetlinewidth{0.000000pt}%
\definecolor{currentstroke}{rgb}{0.000000,0.000000,0.000000}%
\pgfsetstrokecolor{currentstroke}%
\pgfsetdash{}{0pt}%
\pgfpathmoveto{\pgfqpoint{3.984516in}{0.691565in}}%
\pgfpathlineto{\pgfqpoint{3.807659in}{0.691565in}}%
\pgfpathlineto{\pgfqpoint{3.812094in}{0.682694in}}%
\pgfpathlineto{\pgfqpoint{3.767742in}{0.696000in}}%
\pgfpathlineto{\pgfqpoint{3.812094in}{0.709306in}}%
\pgfpathlineto{\pgfqpoint{3.807659in}{0.700435in}}%
\pgfpathlineto{\pgfqpoint{3.984516in}{0.700435in}}%
\pgfpathlineto{\pgfqpoint{3.984516in}{0.691565in}}%
\pgfusepath{fill}%
\end{pgfscope}%
\begin{pgfscope}%
\pgfpathrectangle{\pgfqpoint{1.432000in}{0.528000in}}{\pgfqpoint{3.696000in}{3.696000in}} %
\pgfusepath{clip}%
\pgfsetbuttcap%
\pgfsetroundjoin%
\definecolor{currentfill}{rgb}{0.280894,0.078907,0.402329}%
\pgfsetfillcolor{currentfill}%
\pgfsetlinewidth{0.000000pt}%
\definecolor{currentstroke}{rgb}{0.000000,0.000000,0.000000}%
\pgfsetstrokecolor{currentstroke}%
\pgfsetdash{}{0pt}%
\pgfpathmoveto{\pgfqpoint{4.092903in}{0.691565in}}%
\pgfpathlineto{\pgfqpoint{3.916046in}{0.691565in}}%
\pgfpathlineto{\pgfqpoint{3.920481in}{0.682694in}}%
\pgfpathlineto{\pgfqpoint{3.876129in}{0.696000in}}%
\pgfpathlineto{\pgfqpoint{3.920481in}{0.709306in}}%
\pgfpathlineto{\pgfqpoint{3.916046in}{0.700435in}}%
\pgfpathlineto{\pgfqpoint{4.092903in}{0.700435in}}%
\pgfpathlineto{\pgfqpoint{4.092903in}{0.691565in}}%
\pgfusepath{fill}%
\end{pgfscope}%
\begin{pgfscope}%
\pgfpathrectangle{\pgfqpoint{1.432000in}{0.528000in}}{\pgfqpoint{3.696000in}{3.696000in}} %
\pgfusepath{clip}%
\pgfsetbuttcap%
\pgfsetroundjoin%
\definecolor{currentfill}{rgb}{0.275191,0.194905,0.496005}%
\pgfsetfillcolor{currentfill}%
\pgfsetlinewidth{0.000000pt}%
\definecolor{currentstroke}{rgb}{0.000000,0.000000,0.000000}%
\pgfsetstrokecolor{currentstroke}%
\pgfsetdash{}{0pt}%
\pgfpathmoveto{\pgfqpoint{4.201290in}{0.691565in}}%
\pgfpathlineto{\pgfqpoint{3.916046in}{0.691565in}}%
\pgfpathlineto{\pgfqpoint{3.920481in}{0.682694in}}%
\pgfpathlineto{\pgfqpoint{3.876129in}{0.696000in}}%
\pgfpathlineto{\pgfqpoint{3.920481in}{0.709306in}}%
\pgfpathlineto{\pgfqpoint{3.916046in}{0.700435in}}%
\pgfpathlineto{\pgfqpoint{4.201290in}{0.700435in}}%
\pgfpathlineto{\pgfqpoint{4.201290in}{0.691565in}}%
\pgfusepath{fill}%
\end{pgfscope}%
\begin{pgfscope}%
\pgfpathrectangle{\pgfqpoint{1.432000in}{0.528000in}}{\pgfqpoint{3.696000in}{3.696000in}} %
\pgfusepath{clip}%
\pgfsetbuttcap%
\pgfsetroundjoin%
\definecolor{currentfill}{rgb}{0.241237,0.296485,0.539709}%
\pgfsetfillcolor{currentfill}%
\pgfsetlinewidth{0.000000pt}%
\definecolor{currentstroke}{rgb}{0.000000,0.000000,0.000000}%
\pgfsetstrokecolor{currentstroke}%
\pgfsetdash{}{0pt}%
\pgfpathmoveto{\pgfqpoint{4.309677in}{0.691565in}}%
\pgfpathlineto{\pgfqpoint{4.024433in}{0.691565in}}%
\pgfpathlineto{\pgfqpoint{4.028868in}{0.682694in}}%
\pgfpathlineto{\pgfqpoint{3.984516in}{0.696000in}}%
\pgfpathlineto{\pgfqpoint{4.028868in}{0.709306in}}%
\pgfpathlineto{\pgfqpoint{4.024433in}{0.700435in}}%
\pgfpathlineto{\pgfqpoint{4.309677in}{0.700435in}}%
\pgfpathlineto{\pgfqpoint{4.309677in}{0.691565in}}%
\pgfusepath{fill}%
\end{pgfscope}%
\begin{pgfscope}%
\pgfpathrectangle{\pgfqpoint{1.432000in}{0.528000in}}{\pgfqpoint{3.696000in}{3.696000in}} %
\pgfusepath{clip}%
\pgfsetbuttcap%
\pgfsetroundjoin%
\definecolor{currentfill}{rgb}{0.218130,0.347432,0.550038}%
\pgfsetfillcolor{currentfill}%
\pgfsetlinewidth{0.000000pt}%
\definecolor{currentstroke}{rgb}{0.000000,0.000000,0.000000}%
\pgfsetstrokecolor{currentstroke}%
\pgfsetdash{}{0pt}%
\pgfpathmoveto{\pgfqpoint{4.418065in}{0.691565in}}%
\pgfpathlineto{\pgfqpoint{4.132820in}{0.691565in}}%
\pgfpathlineto{\pgfqpoint{4.137255in}{0.682694in}}%
\pgfpathlineto{\pgfqpoint{4.092903in}{0.696000in}}%
\pgfpathlineto{\pgfqpoint{4.137255in}{0.709306in}}%
\pgfpathlineto{\pgfqpoint{4.132820in}{0.700435in}}%
\pgfpathlineto{\pgfqpoint{4.418065in}{0.700435in}}%
\pgfpathlineto{\pgfqpoint{4.418065in}{0.691565in}}%
\pgfusepath{fill}%
\end{pgfscope}%
\begin{pgfscope}%
\pgfpathrectangle{\pgfqpoint{1.432000in}{0.528000in}}{\pgfqpoint{3.696000in}{3.696000in}} %
\pgfusepath{clip}%
\pgfsetbuttcap%
\pgfsetroundjoin%
\definecolor{currentfill}{rgb}{0.190631,0.407061,0.556089}%
\pgfsetfillcolor{currentfill}%
\pgfsetlinewidth{0.000000pt}%
\definecolor{currentstroke}{rgb}{0.000000,0.000000,0.000000}%
\pgfsetstrokecolor{currentstroke}%
\pgfsetdash{}{0pt}%
\pgfpathmoveto{\pgfqpoint{4.526452in}{0.691565in}}%
\pgfpathlineto{\pgfqpoint{4.241207in}{0.691565in}}%
\pgfpathlineto{\pgfqpoint{4.245642in}{0.682694in}}%
\pgfpathlineto{\pgfqpoint{4.201290in}{0.696000in}}%
\pgfpathlineto{\pgfqpoint{4.245642in}{0.709306in}}%
\pgfpathlineto{\pgfqpoint{4.241207in}{0.700435in}}%
\pgfpathlineto{\pgfqpoint{4.526452in}{0.700435in}}%
\pgfpathlineto{\pgfqpoint{4.526452in}{0.691565in}}%
\pgfusepath{fill}%
\end{pgfscope}%
\begin{pgfscope}%
\pgfpathrectangle{\pgfqpoint{1.432000in}{0.528000in}}{\pgfqpoint{3.696000in}{3.696000in}} %
\pgfusepath{clip}%
\pgfsetbuttcap%
\pgfsetroundjoin%
\definecolor{currentfill}{rgb}{0.282910,0.105393,0.426902}%
\pgfsetfillcolor{currentfill}%
\pgfsetlinewidth{0.000000pt}%
\definecolor{currentstroke}{rgb}{0.000000,0.000000,0.000000}%
\pgfsetstrokecolor{currentstroke}%
\pgfsetdash{}{0pt}%
\pgfpathmoveto{\pgfqpoint{4.526452in}{0.691565in}}%
\pgfpathlineto{\pgfqpoint{4.349594in}{0.691565in}}%
\pgfpathlineto{\pgfqpoint{4.354029in}{0.682694in}}%
\pgfpathlineto{\pgfqpoint{4.309677in}{0.696000in}}%
\pgfpathlineto{\pgfqpoint{4.354029in}{0.709306in}}%
\pgfpathlineto{\pgfqpoint{4.349594in}{0.700435in}}%
\pgfpathlineto{\pgfqpoint{4.526452in}{0.700435in}}%
\pgfpathlineto{\pgfqpoint{4.526452in}{0.691565in}}%
\pgfusepath{fill}%
\end{pgfscope}%
\begin{pgfscope}%
\pgfpathrectangle{\pgfqpoint{1.432000in}{0.528000in}}{\pgfqpoint{3.696000in}{3.696000in}} %
\pgfusepath{clip}%
\pgfsetbuttcap%
\pgfsetroundjoin%
\definecolor{currentfill}{rgb}{0.279566,0.067836,0.391917}%
\pgfsetfillcolor{currentfill}%
\pgfsetlinewidth{0.000000pt}%
\definecolor{currentstroke}{rgb}{0.000000,0.000000,0.000000}%
\pgfsetstrokecolor{currentstroke}%
\pgfsetdash{}{0pt}%
\pgfpathmoveto{\pgfqpoint{4.634839in}{0.691565in}}%
\pgfpathlineto{\pgfqpoint{4.349594in}{0.691565in}}%
\pgfpathlineto{\pgfqpoint{4.354029in}{0.682694in}}%
\pgfpathlineto{\pgfqpoint{4.309677in}{0.696000in}}%
\pgfpathlineto{\pgfqpoint{4.354029in}{0.709306in}}%
\pgfpathlineto{\pgfqpoint{4.349594in}{0.700435in}}%
\pgfpathlineto{\pgfqpoint{4.634839in}{0.700435in}}%
\pgfpathlineto{\pgfqpoint{4.634839in}{0.691565in}}%
\pgfusepath{fill}%
\end{pgfscope}%
\begin{pgfscope}%
\pgfpathrectangle{\pgfqpoint{1.432000in}{0.528000in}}{\pgfqpoint{3.696000in}{3.696000in}} %
\pgfusepath{clip}%
\pgfsetbuttcap%
\pgfsetroundjoin%
\definecolor{currentfill}{rgb}{0.179019,0.433756,0.557430}%
\pgfsetfillcolor{currentfill}%
\pgfsetlinewidth{0.000000pt}%
\definecolor{currentstroke}{rgb}{0.000000,0.000000,0.000000}%
\pgfsetstrokecolor{currentstroke}%
\pgfsetdash{}{0pt}%
\pgfpathmoveto{\pgfqpoint{4.634839in}{0.691565in}}%
\pgfpathlineto{\pgfqpoint{4.457981in}{0.691565in}}%
\pgfpathlineto{\pgfqpoint{4.462417in}{0.682694in}}%
\pgfpathlineto{\pgfqpoint{4.418065in}{0.696000in}}%
\pgfpathlineto{\pgfqpoint{4.462417in}{0.709306in}}%
\pgfpathlineto{\pgfqpoint{4.457981in}{0.700435in}}%
\pgfpathlineto{\pgfqpoint{4.634839in}{0.700435in}}%
\pgfpathlineto{\pgfqpoint{4.634839in}{0.691565in}}%
\pgfusepath{fill}%
\end{pgfscope}%
\begin{pgfscope}%
\pgfpathrectangle{\pgfqpoint{1.432000in}{0.528000in}}{\pgfqpoint{3.696000in}{3.696000in}} %
\pgfusepath{clip}%
\pgfsetbuttcap%
\pgfsetroundjoin%
\definecolor{currentfill}{rgb}{0.180629,0.429975,0.557282}%
\pgfsetfillcolor{currentfill}%
\pgfsetlinewidth{0.000000pt}%
\definecolor{currentstroke}{rgb}{0.000000,0.000000,0.000000}%
\pgfsetstrokecolor{currentstroke}%
\pgfsetdash{}{0pt}%
\pgfpathmoveto{\pgfqpoint{4.743226in}{0.691565in}}%
\pgfpathlineto{\pgfqpoint{4.566368in}{0.691565in}}%
\pgfpathlineto{\pgfqpoint{4.570804in}{0.682694in}}%
\pgfpathlineto{\pgfqpoint{4.526452in}{0.696000in}}%
\pgfpathlineto{\pgfqpoint{4.570804in}{0.709306in}}%
\pgfpathlineto{\pgfqpoint{4.566368in}{0.700435in}}%
\pgfpathlineto{\pgfqpoint{4.743226in}{0.700435in}}%
\pgfpathlineto{\pgfqpoint{4.743226in}{0.691565in}}%
\pgfusepath{fill}%
\end{pgfscope}%
\begin{pgfscope}%
\pgfpathrectangle{\pgfqpoint{1.432000in}{0.528000in}}{\pgfqpoint{3.696000in}{3.696000in}} %
\pgfusepath{clip}%
\pgfsetbuttcap%
\pgfsetroundjoin%
\definecolor{currentfill}{rgb}{0.282884,0.135920,0.453427}%
\pgfsetfillcolor{currentfill}%
\pgfsetlinewidth{0.000000pt}%
\definecolor{currentstroke}{rgb}{0.000000,0.000000,0.000000}%
\pgfsetstrokecolor{currentstroke}%
\pgfsetdash{}{0pt}%
\pgfpathmoveto{\pgfqpoint{4.743226in}{0.691565in}}%
\pgfpathlineto{\pgfqpoint{4.674756in}{0.691565in}}%
\pgfpathlineto{\pgfqpoint{4.679191in}{0.682694in}}%
\pgfpathlineto{\pgfqpoint{4.634839in}{0.696000in}}%
\pgfpathlineto{\pgfqpoint{4.679191in}{0.709306in}}%
\pgfpathlineto{\pgfqpoint{4.674756in}{0.700435in}}%
\pgfpathlineto{\pgfqpoint{4.743226in}{0.700435in}}%
\pgfpathlineto{\pgfqpoint{4.743226in}{0.691565in}}%
\pgfusepath{fill}%
\end{pgfscope}%
\begin{pgfscope}%
\pgfpathrectangle{\pgfqpoint{1.432000in}{0.528000in}}{\pgfqpoint{3.696000in}{3.696000in}} %
\pgfusepath{clip}%
\pgfsetbuttcap%
\pgfsetroundjoin%
\definecolor{currentfill}{rgb}{0.273006,0.204520,0.501721}%
\pgfsetfillcolor{currentfill}%
\pgfsetlinewidth{0.000000pt}%
\definecolor{currentstroke}{rgb}{0.000000,0.000000,0.000000}%
\pgfsetstrokecolor{currentstroke}%
\pgfsetdash{}{0pt}%
\pgfpathmoveto{\pgfqpoint{4.851613in}{0.691565in}}%
\pgfpathlineto{\pgfqpoint{4.674756in}{0.691565in}}%
\pgfpathlineto{\pgfqpoint{4.679191in}{0.682694in}}%
\pgfpathlineto{\pgfqpoint{4.634839in}{0.696000in}}%
\pgfpathlineto{\pgfqpoint{4.679191in}{0.709306in}}%
\pgfpathlineto{\pgfqpoint{4.674756in}{0.700435in}}%
\pgfpathlineto{\pgfqpoint{4.851613in}{0.700435in}}%
\pgfpathlineto{\pgfqpoint{4.851613in}{0.691565in}}%
\pgfusepath{fill}%
\end{pgfscope}%
\begin{pgfscope}%
\pgfpathrectangle{\pgfqpoint{1.432000in}{0.528000in}}{\pgfqpoint{3.696000in}{3.696000in}} %
\pgfusepath{clip}%
\pgfsetbuttcap%
\pgfsetroundjoin%
\definecolor{currentfill}{rgb}{0.203063,0.379716,0.553925}%
\pgfsetfillcolor{currentfill}%
\pgfsetlinewidth{0.000000pt}%
\definecolor{currentstroke}{rgb}{0.000000,0.000000,0.000000}%
\pgfsetstrokecolor{currentstroke}%
\pgfsetdash{}{0pt}%
\pgfpathmoveto{\pgfqpoint{4.851613in}{0.691565in}}%
\pgfpathlineto{\pgfqpoint{4.783143in}{0.691565in}}%
\pgfpathlineto{\pgfqpoint{4.787578in}{0.682694in}}%
\pgfpathlineto{\pgfqpoint{4.743226in}{0.696000in}}%
\pgfpathlineto{\pgfqpoint{4.787578in}{0.709306in}}%
\pgfpathlineto{\pgfqpoint{4.783143in}{0.700435in}}%
\pgfpathlineto{\pgfqpoint{4.851613in}{0.700435in}}%
\pgfpathlineto{\pgfqpoint{4.851613in}{0.691565in}}%
\pgfusepath{fill}%
\end{pgfscope}%
\begin{pgfscope}%
\pgfpathrectangle{\pgfqpoint{1.432000in}{0.528000in}}{\pgfqpoint{3.696000in}{3.696000in}} %
\pgfusepath{clip}%
\pgfsetbuttcap%
\pgfsetroundjoin%
\definecolor{currentfill}{rgb}{0.231674,0.318106,0.544834}%
\pgfsetfillcolor{currentfill}%
\pgfsetlinewidth{0.000000pt}%
\definecolor{currentstroke}{rgb}{0.000000,0.000000,0.000000}%
\pgfsetstrokecolor{currentstroke}%
\pgfsetdash{}{0pt}%
\pgfpathmoveto{\pgfqpoint{4.960000in}{0.691565in}}%
\pgfpathlineto{\pgfqpoint{4.891530in}{0.691565in}}%
\pgfpathlineto{\pgfqpoint{4.895965in}{0.682694in}}%
\pgfpathlineto{\pgfqpoint{4.851613in}{0.696000in}}%
\pgfpathlineto{\pgfqpoint{4.895965in}{0.709306in}}%
\pgfpathlineto{\pgfqpoint{4.891530in}{0.700435in}}%
\pgfpathlineto{\pgfqpoint{4.960000in}{0.700435in}}%
\pgfpathlineto{\pgfqpoint{4.960000in}{0.691565in}}%
\pgfusepath{fill}%
\end{pgfscope}%
\begin{pgfscope}%
\pgfpathrectangle{\pgfqpoint{1.432000in}{0.528000in}}{\pgfqpoint{3.696000in}{3.696000in}} %
\pgfusepath{clip}%
\pgfsetbuttcap%
\pgfsetroundjoin%
\definecolor{currentfill}{rgb}{0.162142,0.474838,0.558140}%
\pgfsetfillcolor{currentfill}%
\pgfsetlinewidth{0.000000pt}%
\definecolor{currentstroke}{rgb}{0.000000,0.000000,0.000000}%
\pgfsetstrokecolor{currentstroke}%
\pgfsetdash{}{0pt}%
\pgfpathmoveto{\pgfqpoint{4.964435in}{0.696000in}}%
\pgfpathlineto{\pgfqpoint{4.962218in}{0.699841in}}%
\pgfpathlineto{\pgfqpoint{4.957782in}{0.699841in}}%
\pgfpathlineto{\pgfqpoint{4.955565in}{0.696000in}}%
\pgfpathlineto{\pgfqpoint{4.957782in}{0.692159in}}%
\pgfpathlineto{\pgfqpoint{4.962218in}{0.692159in}}%
\pgfpathlineto{\pgfqpoint{4.964435in}{0.696000in}}%
\pgfpathlineto{\pgfqpoint{4.962218in}{0.699841in}}%
\pgfusepath{fill}%
\end{pgfscope}%
\begin{pgfscope}%
\pgfpathrectangle{\pgfqpoint{1.432000in}{0.528000in}}{\pgfqpoint{3.696000in}{3.696000in}} %
\pgfusepath{clip}%
\pgfsetbuttcap%
\pgfsetroundjoin%
\definecolor{currentfill}{rgb}{0.192357,0.403199,0.555836}%
\pgfsetfillcolor{currentfill}%
\pgfsetlinewidth{0.000000pt}%
\definecolor{currentstroke}{rgb}{0.000000,0.000000,0.000000}%
\pgfsetstrokecolor{currentstroke}%
\pgfsetdash{}{0pt}%
\pgfpathmoveto{\pgfqpoint{1.604435in}{0.804387in}}%
\pgfpathlineto{\pgfqpoint{1.602218in}{0.808228in}}%
\pgfpathlineto{\pgfqpoint{1.597782in}{0.808228in}}%
\pgfpathlineto{\pgfqpoint{1.595565in}{0.804387in}}%
\pgfpathlineto{\pgfqpoint{1.597782in}{0.800546in}}%
\pgfpathlineto{\pgfqpoint{1.602218in}{0.800546in}}%
\pgfpathlineto{\pgfqpoint{1.604435in}{0.804387in}}%
\pgfpathlineto{\pgfqpoint{1.602218in}{0.808228in}}%
\pgfusepath{fill}%
\end{pgfscope}%
\begin{pgfscope}%
\pgfpathrectangle{\pgfqpoint{1.432000in}{0.528000in}}{\pgfqpoint{3.696000in}{3.696000in}} %
\pgfusepath{clip}%
\pgfsetbuttcap%
\pgfsetroundjoin%
\definecolor{currentfill}{rgb}{0.283072,0.130895,0.449241}%
\pgfsetfillcolor{currentfill}%
\pgfsetlinewidth{0.000000pt}%
\definecolor{currentstroke}{rgb}{0.000000,0.000000,0.000000}%
\pgfsetstrokecolor{currentstroke}%
\pgfsetdash{}{0pt}%
\pgfpathmoveto{\pgfqpoint{1.595565in}{0.804387in}}%
\pgfpathlineto{\pgfqpoint{1.595565in}{0.872857in}}%
\pgfpathlineto{\pgfqpoint{1.586694in}{0.868422in}}%
\pgfpathlineto{\pgfqpoint{1.600000in}{0.912774in}}%
\pgfpathlineto{\pgfqpoint{1.613306in}{0.868422in}}%
\pgfpathlineto{\pgfqpoint{1.604435in}{0.872857in}}%
\pgfpathlineto{\pgfqpoint{1.604435in}{0.804387in}}%
\pgfpathlineto{\pgfqpoint{1.595565in}{0.804387in}}%
\pgfusepath{fill}%
\end{pgfscope}%
\begin{pgfscope}%
\pgfpathrectangle{\pgfqpoint{1.432000in}{0.528000in}}{\pgfqpoint{3.696000in}{3.696000in}} %
\pgfusepath{clip}%
\pgfsetbuttcap%
\pgfsetroundjoin%
\definecolor{currentfill}{rgb}{0.269944,0.014625,0.341379}%
\pgfsetfillcolor{currentfill}%
\pgfsetlinewidth{0.000000pt}%
\definecolor{currentstroke}{rgb}{0.000000,0.000000,0.000000}%
\pgfsetstrokecolor{currentstroke}%
\pgfsetdash{}{0pt}%
\pgfpathmoveto{\pgfqpoint{1.708387in}{0.799952in}}%
\pgfpathlineto{\pgfqpoint{1.639917in}{0.799952in}}%
\pgfpathlineto{\pgfqpoint{1.644352in}{0.791081in}}%
\pgfpathlineto{\pgfqpoint{1.600000in}{0.804387in}}%
\pgfpathlineto{\pgfqpoint{1.644352in}{0.817693in}}%
\pgfpathlineto{\pgfqpoint{1.639917in}{0.808822in}}%
\pgfpathlineto{\pgfqpoint{1.708387in}{0.808822in}}%
\pgfpathlineto{\pgfqpoint{1.708387in}{0.799952in}}%
\pgfusepath{fill}%
\end{pgfscope}%
\begin{pgfscope}%
\pgfpathrectangle{\pgfqpoint{1.432000in}{0.528000in}}{\pgfqpoint{3.696000in}{3.696000in}} %
\pgfusepath{clip}%
\pgfsetbuttcap%
\pgfsetroundjoin%
\definecolor{currentfill}{rgb}{0.276194,0.190074,0.493001}%
\pgfsetfillcolor{currentfill}%
\pgfsetlinewidth{0.000000pt}%
\definecolor{currentstroke}{rgb}{0.000000,0.000000,0.000000}%
\pgfsetstrokecolor{currentstroke}%
\pgfsetdash{}{0pt}%
\pgfpathmoveto{\pgfqpoint{1.712822in}{0.804387in}}%
\pgfpathlineto{\pgfqpoint{1.710605in}{0.808228in}}%
\pgfpathlineto{\pgfqpoint{1.706169in}{0.808228in}}%
\pgfpathlineto{\pgfqpoint{1.703952in}{0.804387in}}%
\pgfpathlineto{\pgfqpoint{1.706169in}{0.800546in}}%
\pgfpathlineto{\pgfqpoint{1.710605in}{0.800546in}}%
\pgfpathlineto{\pgfqpoint{1.712822in}{0.804387in}}%
\pgfpathlineto{\pgfqpoint{1.710605in}{0.808228in}}%
\pgfusepath{fill}%
\end{pgfscope}%
\begin{pgfscope}%
\pgfpathrectangle{\pgfqpoint{1.432000in}{0.528000in}}{\pgfqpoint{3.696000in}{3.696000in}} %
\pgfusepath{clip}%
\pgfsetbuttcap%
\pgfsetroundjoin%
\definecolor{currentfill}{rgb}{0.166617,0.463708,0.558119}%
\pgfsetfillcolor{currentfill}%
\pgfsetlinewidth{0.000000pt}%
\definecolor{currentstroke}{rgb}{0.000000,0.000000,0.000000}%
\pgfsetstrokecolor{currentstroke}%
\pgfsetdash{}{0pt}%
\pgfpathmoveto{\pgfqpoint{1.816774in}{0.799952in}}%
\pgfpathlineto{\pgfqpoint{1.748304in}{0.799952in}}%
\pgfpathlineto{\pgfqpoint{1.752739in}{0.791081in}}%
\pgfpathlineto{\pgfqpoint{1.708387in}{0.804387in}}%
\pgfpathlineto{\pgfqpoint{1.752739in}{0.817693in}}%
\pgfpathlineto{\pgfqpoint{1.748304in}{0.808822in}}%
\pgfpathlineto{\pgfqpoint{1.816774in}{0.808822in}}%
\pgfpathlineto{\pgfqpoint{1.816774in}{0.799952in}}%
\pgfusepath{fill}%
\end{pgfscope}%
\begin{pgfscope}%
\pgfpathrectangle{\pgfqpoint{1.432000in}{0.528000in}}{\pgfqpoint{3.696000in}{3.696000in}} %
\pgfusepath{clip}%
\pgfsetbuttcap%
\pgfsetroundjoin%
\definecolor{currentfill}{rgb}{0.121148,0.592739,0.544641}%
\pgfsetfillcolor{currentfill}%
\pgfsetlinewidth{0.000000pt}%
\definecolor{currentstroke}{rgb}{0.000000,0.000000,0.000000}%
\pgfsetstrokecolor{currentstroke}%
\pgfsetdash{}{0pt}%
\pgfpathmoveto{\pgfqpoint{1.925161in}{0.799952in}}%
\pgfpathlineto{\pgfqpoint{1.856691in}{0.799952in}}%
\pgfpathlineto{\pgfqpoint{1.861126in}{0.791081in}}%
\pgfpathlineto{\pgfqpoint{1.816774in}{0.804387in}}%
\pgfpathlineto{\pgfqpoint{1.861126in}{0.817693in}}%
\pgfpathlineto{\pgfqpoint{1.856691in}{0.808822in}}%
\pgfpathlineto{\pgfqpoint{1.925161in}{0.808822in}}%
\pgfpathlineto{\pgfqpoint{1.925161in}{0.799952in}}%
\pgfusepath{fill}%
\end{pgfscope}%
\begin{pgfscope}%
\pgfpathrectangle{\pgfqpoint{1.432000in}{0.528000in}}{\pgfqpoint{3.696000in}{3.696000in}} %
\pgfusepath{clip}%
\pgfsetbuttcap%
\pgfsetroundjoin%
\definecolor{currentfill}{rgb}{0.280267,0.073417,0.397163}%
\pgfsetfillcolor{currentfill}%
\pgfsetlinewidth{0.000000pt}%
\definecolor{currentstroke}{rgb}{0.000000,0.000000,0.000000}%
\pgfsetstrokecolor{currentstroke}%
\pgfsetdash{}{0pt}%
\pgfpathmoveto{\pgfqpoint{2.033548in}{0.799952in}}%
\pgfpathlineto{\pgfqpoint{1.856691in}{0.799952in}}%
\pgfpathlineto{\pgfqpoint{1.861126in}{0.791081in}}%
\pgfpathlineto{\pgfqpoint{1.816774in}{0.804387in}}%
\pgfpathlineto{\pgfqpoint{1.861126in}{0.817693in}}%
\pgfpathlineto{\pgfqpoint{1.856691in}{0.808822in}}%
\pgfpathlineto{\pgfqpoint{2.033548in}{0.808822in}}%
\pgfpathlineto{\pgfqpoint{2.033548in}{0.799952in}}%
\pgfusepath{fill}%
\end{pgfscope}%
\begin{pgfscope}%
\pgfpathrectangle{\pgfqpoint{1.432000in}{0.528000in}}{\pgfqpoint{3.696000in}{3.696000in}} %
\pgfusepath{clip}%
\pgfsetbuttcap%
\pgfsetroundjoin%
\definecolor{currentfill}{rgb}{0.223925,0.334994,0.548053}%
\pgfsetfillcolor{currentfill}%
\pgfsetlinewidth{0.000000pt}%
\definecolor{currentstroke}{rgb}{0.000000,0.000000,0.000000}%
\pgfsetstrokecolor{currentstroke}%
\pgfsetdash{}{0pt}%
\pgfpathmoveto{\pgfqpoint{2.033548in}{0.799952in}}%
\pgfpathlineto{\pgfqpoint{1.965078in}{0.799952in}}%
\pgfpathlineto{\pgfqpoint{1.969513in}{0.791081in}}%
\pgfpathlineto{\pgfqpoint{1.925161in}{0.804387in}}%
\pgfpathlineto{\pgfqpoint{1.969513in}{0.817693in}}%
\pgfpathlineto{\pgfqpoint{1.965078in}{0.808822in}}%
\pgfpathlineto{\pgfqpoint{2.033548in}{0.808822in}}%
\pgfpathlineto{\pgfqpoint{2.033548in}{0.799952in}}%
\pgfusepath{fill}%
\end{pgfscope}%
\begin{pgfscope}%
\pgfpathrectangle{\pgfqpoint{1.432000in}{0.528000in}}{\pgfqpoint{3.696000in}{3.696000in}} %
\pgfusepath{clip}%
\pgfsetbuttcap%
\pgfsetroundjoin%
\definecolor{currentfill}{rgb}{0.174274,0.445044,0.557792}%
\pgfsetfillcolor{currentfill}%
\pgfsetlinewidth{0.000000pt}%
\definecolor{currentstroke}{rgb}{0.000000,0.000000,0.000000}%
\pgfsetstrokecolor{currentstroke}%
\pgfsetdash{}{0pt}%
\pgfpathmoveto{\pgfqpoint{2.141935in}{0.799952in}}%
\pgfpathlineto{\pgfqpoint{1.965078in}{0.799952in}}%
\pgfpathlineto{\pgfqpoint{1.969513in}{0.791081in}}%
\pgfpathlineto{\pgfqpoint{1.925161in}{0.804387in}}%
\pgfpathlineto{\pgfqpoint{1.969513in}{0.817693in}}%
\pgfpathlineto{\pgfqpoint{1.965078in}{0.808822in}}%
\pgfpathlineto{\pgfqpoint{2.141935in}{0.808822in}}%
\pgfpathlineto{\pgfqpoint{2.141935in}{0.799952in}}%
\pgfusepath{fill}%
\end{pgfscope}%
\begin{pgfscope}%
\pgfpathrectangle{\pgfqpoint{1.432000in}{0.528000in}}{\pgfqpoint{3.696000in}{3.696000in}} %
\pgfusepath{clip}%
\pgfsetbuttcap%
\pgfsetroundjoin%
\definecolor{currentfill}{rgb}{0.257322,0.256130,0.526563}%
\pgfsetfillcolor{currentfill}%
\pgfsetlinewidth{0.000000pt}%
\definecolor{currentstroke}{rgb}{0.000000,0.000000,0.000000}%
\pgfsetstrokecolor{currentstroke}%
\pgfsetdash{}{0pt}%
\pgfpathmoveto{\pgfqpoint{2.250323in}{0.799952in}}%
\pgfpathlineto{\pgfqpoint{2.073465in}{0.799952in}}%
\pgfpathlineto{\pgfqpoint{2.077900in}{0.791081in}}%
\pgfpathlineto{\pgfqpoint{2.033548in}{0.804387in}}%
\pgfpathlineto{\pgfqpoint{2.077900in}{0.817693in}}%
\pgfpathlineto{\pgfqpoint{2.073465in}{0.808822in}}%
\pgfpathlineto{\pgfqpoint{2.250323in}{0.808822in}}%
\pgfpathlineto{\pgfqpoint{2.250323in}{0.799952in}}%
\pgfusepath{fill}%
\end{pgfscope}%
\begin{pgfscope}%
\pgfpathrectangle{\pgfqpoint{1.432000in}{0.528000in}}{\pgfqpoint{3.696000in}{3.696000in}} %
\pgfusepath{clip}%
\pgfsetbuttcap%
\pgfsetroundjoin%
\definecolor{currentfill}{rgb}{0.223925,0.334994,0.548053}%
\pgfsetfillcolor{currentfill}%
\pgfsetlinewidth{0.000000pt}%
\definecolor{currentstroke}{rgb}{0.000000,0.000000,0.000000}%
\pgfsetstrokecolor{currentstroke}%
\pgfsetdash{}{0pt}%
\pgfpathmoveto{\pgfqpoint{2.358710in}{0.799952in}}%
\pgfpathlineto{\pgfqpoint{2.073465in}{0.799952in}}%
\pgfpathlineto{\pgfqpoint{2.077900in}{0.791081in}}%
\pgfpathlineto{\pgfqpoint{2.033548in}{0.804387in}}%
\pgfpathlineto{\pgfqpoint{2.077900in}{0.817693in}}%
\pgfpathlineto{\pgfqpoint{2.073465in}{0.808822in}}%
\pgfpathlineto{\pgfqpoint{2.358710in}{0.808822in}}%
\pgfpathlineto{\pgfqpoint{2.358710in}{0.799952in}}%
\pgfusepath{fill}%
\end{pgfscope}%
\begin{pgfscope}%
\pgfpathrectangle{\pgfqpoint{1.432000in}{0.528000in}}{\pgfqpoint{3.696000in}{3.696000in}} %
\pgfusepath{clip}%
\pgfsetbuttcap%
\pgfsetroundjoin%
\definecolor{currentfill}{rgb}{0.201239,0.383670,0.554294}%
\pgfsetfillcolor{currentfill}%
\pgfsetlinewidth{0.000000pt}%
\definecolor{currentstroke}{rgb}{0.000000,0.000000,0.000000}%
\pgfsetstrokecolor{currentstroke}%
\pgfsetdash{}{0pt}%
\pgfpathmoveto{\pgfqpoint{2.467097in}{0.799952in}}%
\pgfpathlineto{\pgfqpoint{2.181852in}{0.799952in}}%
\pgfpathlineto{\pgfqpoint{2.186287in}{0.791081in}}%
\pgfpathlineto{\pgfqpoint{2.141935in}{0.804387in}}%
\pgfpathlineto{\pgfqpoint{2.186287in}{0.817693in}}%
\pgfpathlineto{\pgfqpoint{2.181852in}{0.808822in}}%
\pgfpathlineto{\pgfqpoint{2.467097in}{0.808822in}}%
\pgfpathlineto{\pgfqpoint{2.467097in}{0.799952in}}%
\pgfusepath{fill}%
\end{pgfscope}%
\begin{pgfscope}%
\pgfpathrectangle{\pgfqpoint{1.432000in}{0.528000in}}{\pgfqpoint{3.696000in}{3.696000in}} %
\pgfusepath{clip}%
\pgfsetbuttcap%
\pgfsetroundjoin%
\definecolor{currentfill}{rgb}{0.239346,0.300855,0.540844}%
\pgfsetfillcolor{currentfill}%
\pgfsetlinewidth{0.000000pt}%
\definecolor{currentstroke}{rgb}{0.000000,0.000000,0.000000}%
\pgfsetstrokecolor{currentstroke}%
\pgfsetdash{}{0pt}%
\pgfpathmoveto{\pgfqpoint{2.575484in}{0.799952in}}%
\pgfpathlineto{\pgfqpoint{2.181852in}{0.799952in}}%
\pgfpathlineto{\pgfqpoint{2.186287in}{0.791081in}}%
\pgfpathlineto{\pgfqpoint{2.141935in}{0.804387in}}%
\pgfpathlineto{\pgfqpoint{2.186287in}{0.817693in}}%
\pgfpathlineto{\pgfqpoint{2.181852in}{0.808822in}}%
\pgfpathlineto{\pgfqpoint{2.575484in}{0.808822in}}%
\pgfpathlineto{\pgfqpoint{2.575484in}{0.799952in}}%
\pgfusepath{fill}%
\end{pgfscope}%
\begin{pgfscope}%
\pgfpathrectangle{\pgfqpoint{1.432000in}{0.528000in}}{\pgfqpoint{3.696000in}{3.696000in}} %
\pgfusepath{clip}%
\pgfsetbuttcap%
\pgfsetroundjoin%
\definecolor{currentfill}{rgb}{0.183898,0.422383,0.556944}%
\pgfsetfillcolor{currentfill}%
\pgfsetlinewidth{0.000000pt}%
\definecolor{currentstroke}{rgb}{0.000000,0.000000,0.000000}%
\pgfsetstrokecolor{currentstroke}%
\pgfsetdash{}{0pt}%
\pgfpathmoveto{\pgfqpoint{2.683871in}{0.799952in}}%
\pgfpathlineto{\pgfqpoint{2.290239in}{0.799952in}}%
\pgfpathlineto{\pgfqpoint{2.294675in}{0.791081in}}%
\pgfpathlineto{\pgfqpoint{2.250323in}{0.804387in}}%
\pgfpathlineto{\pgfqpoint{2.294675in}{0.817693in}}%
\pgfpathlineto{\pgfqpoint{2.290239in}{0.808822in}}%
\pgfpathlineto{\pgfqpoint{2.683871in}{0.808822in}}%
\pgfpathlineto{\pgfqpoint{2.683871in}{0.799952in}}%
\pgfusepath{fill}%
\end{pgfscope}%
\begin{pgfscope}%
\pgfpathrectangle{\pgfqpoint{1.432000in}{0.528000in}}{\pgfqpoint{3.696000in}{3.696000in}} %
\pgfusepath{clip}%
\pgfsetbuttcap%
\pgfsetroundjoin%
\definecolor{currentfill}{rgb}{0.283187,0.125848,0.444960}%
\pgfsetfillcolor{currentfill}%
\pgfsetlinewidth{0.000000pt}%
\definecolor{currentstroke}{rgb}{0.000000,0.000000,0.000000}%
\pgfsetstrokecolor{currentstroke}%
\pgfsetdash{}{0pt}%
\pgfpathmoveto{\pgfqpoint{2.792258in}{0.799952in}}%
\pgfpathlineto{\pgfqpoint{2.398626in}{0.799952in}}%
\pgfpathlineto{\pgfqpoint{2.403062in}{0.791081in}}%
\pgfpathlineto{\pgfqpoint{2.358710in}{0.804387in}}%
\pgfpathlineto{\pgfqpoint{2.403062in}{0.817693in}}%
\pgfpathlineto{\pgfqpoint{2.398626in}{0.808822in}}%
\pgfpathlineto{\pgfqpoint{2.792258in}{0.808822in}}%
\pgfpathlineto{\pgfqpoint{2.792258in}{0.799952in}}%
\pgfusepath{fill}%
\end{pgfscope}%
\begin{pgfscope}%
\pgfpathrectangle{\pgfqpoint{1.432000in}{0.528000in}}{\pgfqpoint{3.696000in}{3.696000in}} %
\pgfusepath{clip}%
\pgfsetbuttcap%
\pgfsetroundjoin%
\definecolor{currentfill}{rgb}{0.227802,0.326594,0.546532}%
\pgfsetfillcolor{currentfill}%
\pgfsetlinewidth{0.000000pt}%
\definecolor{currentstroke}{rgb}{0.000000,0.000000,0.000000}%
\pgfsetstrokecolor{currentstroke}%
\pgfsetdash{}{0pt}%
\pgfpathmoveto{\pgfqpoint{2.899569in}{0.800084in}}%
\pgfpathlineto{\pgfqpoint{2.504746in}{0.898790in}}%
\pgfpathlineto{\pgfqpoint{2.506897in}{0.889109in}}%
\pgfpathlineto{\pgfqpoint{2.467097in}{0.912774in}}%
\pgfpathlineto{\pgfqpoint{2.513352in}{0.914926in}}%
\pgfpathlineto{\pgfqpoint{2.506897in}{0.907396in}}%
\pgfpathlineto{\pgfqpoint{2.901721in}{0.808690in}}%
\pgfpathlineto{\pgfqpoint{2.899569in}{0.800084in}}%
\pgfusepath{fill}%
\end{pgfscope}%
\begin{pgfscope}%
\pgfpathrectangle{\pgfqpoint{1.432000in}{0.528000in}}{\pgfqpoint{3.696000in}{3.696000in}} %
\pgfusepath{clip}%
\pgfsetbuttcap%
\pgfsetroundjoin%
\definecolor{currentfill}{rgb}{0.279566,0.067836,0.391917}%
\pgfsetfillcolor{currentfill}%
\pgfsetlinewidth{0.000000pt}%
\definecolor{currentstroke}{rgb}{0.000000,0.000000,0.000000}%
\pgfsetstrokecolor{currentstroke}%
\pgfsetdash{}{0pt}%
\pgfpathmoveto{\pgfqpoint{2.899243in}{0.800179in}}%
\pgfpathlineto{\pgfqpoint{2.611950in}{0.895944in}}%
\pgfpathlineto{\pgfqpoint{2.613352in}{0.886126in}}%
\pgfpathlineto{\pgfqpoint{2.575484in}{0.912774in}}%
\pgfpathlineto{\pgfqpoint{2.621767in}{0.911372in}}%
\pgfpathlineto{\pgfqpoint{2.614755in}{0.904359in}}%
\pgfpathlineto{\pgfqpoint{2.902048in}{0.808595in}}%
\pgfpathlineto{\pgfqpoint{2.899243in}{0.800179in}}%
\pgfusepath{fill}%
\end{pgfscope}%
\begin{pgfscope}%
\pgfpathrectangle{\pgfqpoint{1.432000in}{0.528000in}}{\pgfqpoint{3.696000in}{3.696000in}} %
\pgfusepath{clip}%
\pgfsetbuttcap%
\pgfsetroundjoin%
\definecolor{currentfill}{rgb}{0.244972,0.287675,0.537260}%
\pgfsetfillcolor{currentfill}%
\pgfsetlinewidth{0.000000pt}%
\definecolor{currentstroke}{rgb}{0.000000,0.000000,0.000000}%
\pgfsetstrokecolor{currentstroke}%
\pgfsetdash{}{0pt}%
\pgfpathmoveto{\pgfqpoint{3.007630in}{0.800179in}}%
\pgfpathlineto{\pgfqpoint{2.720337in}{0.895944in}}%
\pgfpathlineto{\pgfqpoint{2.721739in}{0.886126in}}%
\pgfpathlineto{\pgfqpoint{2.683871in}{0.912774in}}%
\pgfpathlineto{\pgfqpoint{2.730155in}{0.911372in}}%
\pgfpathlineto{\pgfqpoint{2.723142in}{0.904359in}}%
\pgfpathlineto{\pgfqpoint{3.010435in}{0.808595in}}%
\pgfpathlineto{\pgfqpoint{3.007630in}{0.800179in}}%
\pgfusepath{fill}%
\end{pgfscope}%
\begin{pgfscope}%
\pgfpathrectangle{\pgfqpoint{1.432000in}{0.528000in}}{\pgfqpoint{3.696000in}{3.696000in}} %
\pgfusepath{clip}%
\pgfsetbuttcap%
\pgfsetroundjoin%
\definecolor{currentfill}{rgb}{0.269944,0.014625,0.341379}%
\pgfsetfillcolor{currentfill}%
\pgfsetlinewidth{0.000000pt}%
\definecolor{currentstroke}{rgb}{0.000000,0.000000,0.000000}%
\pgfsetstrokecolor{currentstroke}%
\pgfsetdash{}{0pt}%
\pgfpathmoveto{\pgfqpoint{3.007049in}{0.800420in}}%
\pgfpathlineto{\pgfqpoint{2.825977in}{0.890956in}}%
\pgfpathlineto{\pgfqpoint{2.825977in}{0.881038in}}%
\pgfpathlineto{\pgfqpoint{2.792258in}{0.912774in}}%
\pgfpathlineto{\pgfqpoint{2.837878in}{0.904840in}}%
\pgfpathlineto{\pgfqpoint{2.829944in}{0.898890in}}%
\pgfpathlineto{\pgfqpoint{3.011016in}{0.808354in}}%
\pgfpathlineto{\pgfqpoint{3.007049in}{0.800420in}}%
\pgfusepath{fill}%
\end{pgfscope}%
\begin{pgfscope}%
\pgfpathrectangle{\pgfqpoint{1.432000in}{0.528000in}}{\pgfqpoint{3.696000in}{3.696000in}} %
\pgfusepath{clip}%
\pgfsetbuttcap%
\pgfsetroundjoin%
\definecolor{currentfill}{rgb}{0.280267,0.073417,0.397163}%
\pgfsetfillcolor{currentfill}%
\pgfsetlinewidth{0.000000pt}%
\definecolor{currentstroke}{rgb}{0.000000,0.000000,0.000000}%
\pgfsetstrokecolor{currentstroke}%
\pgfsetdash{}{0pt}%
\pgfpathmoveto{\pgfqpoint{3.116017in}{0.800179in}}%
\pgfpathlineto{\pgfqpoint{2.828724in}{0.895944in}}%
\pgfpathlineto{\pgfqpoint{2.830126in}{0.886126in}}%
\pgfpathlineto{\pgfqpoint{2.792258in}{0.912774in}}%
\pgfpathlineto{\pgfqpoint{2.838542in}{0.911372in}}%
\pgfpathlineto{\pgfqpoint{2.831529in}{0.904359in}}%
\pgfpathlineto{\pgfqpoint{3.118822in}{0.808595in}}%
\pgfpathlineto{\pgfqpoint{3.116017in}{0.800179in}}%
\pgfusepath{fill}%
\end{pgfscope}%
\begin{pgfscope}%
\pgfpathrectangle{\pgfqpoint{1.432000in}{0.528000in}}{\pgfqpoint{3.696000in}{3.696000in}} %
\pgfusepath{clip}%
\pgfsetbuttcap%
\pgfsetroundjoin%
\definecolor{currentfill}{rgb}{0.265145,0.232956,0.516599}%
\pgfsetfillcolor{currentfill}%
\pgfsetlinewidth{0.000000pt}%
\definecolor{currentstroke}{rgb}{0.000000,0.000000,0.000000}%
\pgfsetstrokecolor{currentstroke}%
\pgfsetdash{}{0pt}%
\pgfpathmoveto{\pgfqpoint{3.115436in}{0.800420in}}%
\pgfpathlineto{\pgfqpoint{2.934364in}{0.890956in}}%
\pgfpathlineto{\pgfqpoint{2.934364in}{0.881038in}}%
\pgfpathlineto{\pgfqpoint{2.900645in}{0.912774in}}%
\pgfpathlineto{\pgfqpoint{2.946265in}{0.904840in}}%
\pgfpathlineto{\pgfqpoint{2.938331in}{0.898890in}}%
\pgfpathlineto{\pgfqpoint{3.119403in}{0.808354in}}%
\pgfpathlineto{\pgfqpoint{3.115436in}{0.800420in}}%
\pgfusepath{fill}%
\end{pgfscope}%
\begin{pgfscope}%
\pgfpathrectangle{\pgfqpoint{1.432000in}{0.528000in}}{\pgfqpoint{3.696000in}{3.696000in}} %
\pgfusepath{clip}%
\pgfsetbuttcap%
\pgfsetroundjoin%
\definecolor{currentfill}{rgb}{0.272594,0.025563,0.353093}%
\pgfsetfillcolor{currentfill}%
\pgfsetlinewidth{0.000000pt}%
\definecolor{currentstroke}{rgb}{0.000000,0.000000,0.000000}%
\pgfsetstrokecolor{currentstroke}%
\pgfsetdash{}{0pt}%
\pgfpathmoveto{\pgfqpoint{3.225806in}{0.799952in}}%
\pgfpathlineto{\pgfqpoint{3.157336in}{0.799952in}}%
\pgfpathlineto{\pgfqpoint{3.161771in}{0.791081in}}%
\pgfpathlineto{\pgfqpoint{3.117419in}{0.804387in}}%
\pgfpathlineto{\pgfqpoint{3.161771in}{0.817693in}}%
\pgfpathlineto{\pgfqpoint{3.157336in}{0.808822in}}%
\pgfpathlineto{\pgfqpoint{3.225806in}{0.808822in}}%
\pgfpathlineto{\pgfqpoint{3.225806in}{0.799952in}}%
\pgfusepath{fill}%
\end{pgfscope}%
\begin{pgfscope}%
\pgfpathrectangle{\pgfqpoint{1.432000in}{0.528000in}}{\pgfqpoint{3.696000in}{3.696000in}} %
\pgfusepath{clip}%
\pgfsetbuttcap%
\pgfsetroundjoin%
\definecolor{currentfill}{rgb}{0.283229,0.120777,0.440584}%
\pgfsetfillcolor{currentfill}%
\pgfsetlinewidth{0.000000pt}%
\definecolor{currentstroke}{rgb}{0.000000,0.000000,0.000000}%
\pgfsetstrokecolor{currentstroke}%
\pgfsetdash{}{0pt}%
\pgfpathmoveto{\pgfqpoint{3.223823in}{0.800420in}}%
\pgfpathlineto{\pgfqpoint{3.042751in}{0.890956in}}%
\pgfpathlineto{\pgfqpoint{3.042751in}{0.881038in}}%
\pgfpathlineto{\pgfqpoint{3.009032in}{0.912774in}}%
\pgfpathlineto{\pgfqpoint{3.054652in}{0.904840in}}%
\pgfpathlineto{\pgfqpoint{3.046718in}{0.898890in}}%
\pgfpathlineto{\pgfqpoint{3.227790in}{0.808354in}}%
\pgfpathlineto{\pgfqpoint{3.223823in}{0.800420in}}%
\pgfusepath{fill}%
\end{pgfscope}%
\begin{pgfscope}%
\pgfpathrectangle{\pgfqpoint{1.432000in}{0.528000in}}{\pgfqpoint{3.696000in}{3.696000in}} %
\pgfusepath{clip}%
\pgfsetbuttcap%
\pgfsetroundjoin%
\definecolor{currentfill}{rgb}{0.280255,0.165693,0.476498}%
\pgfsetfillcolor{currentfill}%
\pgfsetlinewidth{0.000000pt}%
\definecolor{currentstroke}{rgb}{0.000000,0.000000,0.000000}%
\pgfsetstrokecolor{currentstroke}%
\pgfsetdash{}{0pt}%
\pgfpathmoveto{\pgfqpoint{3.222670in}{0.801251in}}%
\pgfpathlineto{\pgfqpoint{3.142509in}{0.881413in}}%
\pgfpathlineto{\pgfqpoint{3.139372in}{0.872004in}}%
\pgfpathlineto{\pgfqpoint{3.117419in}{0.912774in}}%
\pgfpathlineto{\pgfqpoint{3.158189in}{0.890821in}}%
\pgfpathlineto{\pgfqpoint{3.148781in}{0.887685in}}%
\pgfpathlineto{\pgfqpoint{3.228943in}{0.807523in}}%
\pgfpathlineto{\pgfqpoint{3.222670in}{0.801251in}}%
\pgfusepath{fill}%
\end{pgfscope}%
\begin{pgfscope}%
\pgfpathrectangle{\pgfqpoint{1.432000in}{0.528000in}}{\pgfqpoint{3.696000in}{3.696000in}} %
\pgfusepath{clip}%
\pgfsetbuttcap%
\pgfsetroundjoin%
\definecolor{currentfill}{rgb}{0.244972,0.287675,0.537260}%
\pgfsetfillcolor{currentfill}%
\pgfsetlinewidth{0.000000pt}%
\definecolor{currentstroke}{rgb}{0.000000,0.000000,0.000000}%
\pgfsetstrokecolor{currentstroke}%
\pgfsetdash{}{0pt}%
\pgfpathmoveto{\pgfqpoint{3.334194in}{0.799952in}}%
\pgfpathlineto{\pgfqpoint{3.265723in}{0.799952in}}%
\pgfpathlineto{\pgfqpoint{3.270158in}{0.791081in}}%
\pgfpathlineto{\pgfqpoint{3.225806in}{0.804387in}}%
\pgfpathlineto{\pgfqpoint{3.270158in}{0.817693in}}%
\pgfpathlineto{\pgfqpoint{3.265723in}{0.808822in}}%
\pgfpathlineto{\pgfqpoint{3.334194in}{0.808822in}}%
\pgfpathlineto{\pgfqpoint{3.334194in}{0.799952in}}%
\pgfusepath{fill}%
\end{pgfscope}%
\begin{pgfscope}%
\pgfpathrectangle{\pgfqpoint{1.432000in}{0.528000in}}{\pgfqpoint{3.696000in}{3.696000in}} %
\pgfusepath{clip}%
\pgfsetbuttcap%
\pgfsetroundjoin%
\definecolor{currentfill}{rgb}{0.278826,0.175490,0.483397}%
\pgfsetfillcolor{currentfill}%
\pgfsetlinewidth{0.000000pt}%
\definecolor{currentstroke}{rgb}{0.000000,0.000000,0.000000}%
\pgfsetstrokecolor{currentstroke}%
\pgfsetdash{}{0pt}%
\pgfpathmoveto{\pgfqpoint{3.331057in}{0.801251in}}%
\pgfpathlineto{\pgfqpoint{3.250896in}{0.881413in}}%
\pgfpathlineto{\pgfqpoint{3.247760in}{0.872004in}}%
\pgfpathlineto{\pgfqpoint{3.225806in}{0.912774in}}%
\pgfpathlineto{\pgfqpoint{3.266577in}{0.890821in}}%
\pgfpathlineto{\pgfqpoint{3.257168in}{0.887685in}}%
\pgfpathlineto{\pgfqpoint{3.337330in}{0.807523in}}%
\pgfpathlineto{\pgfqpoint{3.331057in}{0.801251in}}%
\pgfusepath{fill}%
\end{pgfscope}%
\begin{pgfscope}%
\pgfpathrectangle{\pgfqpoint{1.432000in}{0.528000in}}{\pgfqpoint{3.696000in}{3.696000in}} %
\pgfusepath{clip}%
\pgfsetbuttcap%
\pgfsetroundjoin%
\definecolor{currentfill}{rgb}{0.183898,0.422383,0.556944}%
\pgfsetfillcolor{currentfill}%
\pgfsetlinewidth{0.000000pt}%
\definecolor{currentstroke}{rgb}{0.000000,0.000000,0.000000}%
\pgfsetstrokecolor{currentstroke}%
\pgfsetdash{}{0pt}%
\pgfpathmoveto{\pgfqpoint{3.442581in}{0.799952in}}%
\pgfpathlineto{\pgfqpoint{3.374110in}{0.799952in}}%
\pgfpathlineto{\pgfqpoint{3.378546in}{0.791081in}}%
\pgfpathlineto{\pgfqpoint{3.334194in}{0.804387in}}%
\pgfpathlineto{\pgfqpoint{3.378546in}{0.817693in}}%
\pgfpathlineto{\pgfqpoint{3.374110in}{0.808822in}}%
\pgfpathlineto{\pgfqpoint{3.442581in}{0.808822in}}%
\pgfpathlineto{\pgfqpoint{3.442581in}{0.799952in}}%
\pgfusepath{fill}%
\end{pgfscope}%
\begin{pgfscope}%
\pgfpathrectangle{\pgfqpoint{1.432000in}{0.528000in}}{\pgfqpoint{3.696000in}{3.696000in}} %
\pgfusepath{clip}%
\pgfsetbuttcap%
\pgfsetroundjoin%
\definecolor{currentfill}{rgb}{0.276194,0.190074,0.493001}%
\pgfsetfillcolor{currentfill}%
\pgfsetlinewidth{0.000000pt}%
\definecolor{currentstroke}{rgb}{0.000000,0.000000,0.000000}%
\pgfsetstrokecolor{currentstroke}%
\pgfsetdash{}{0pt}%
\pgfpathmoveto{\pgfqpoint{3.439444in}{0.801251in}}%
\pgfpathlineto{\pgfqpoint{3.359283in}{0.881413in}}%
\pgfpathlineto{\pgfqpoint{3.356147in}{0.872004in}}%
\pgfpathlineto{\pgfqpoint{3.334194in}{0.912774in}}%
\pgfpathlineto{\pgfqpoint{3.374964in}{0.890821in}}%
\pgfpathlineto{\pgfqpoint{3.365555in}{0.887685in}}%
\pgfpathlineto{\pgfqpoint{3.445717in}{0.807523in}}%
\pgfpathlineto{\pgfqpoint{3.439444in}{0.801251in}}%
\pgfusepath{fill}%
\end{pgfscope}%
\begin{pgfscope}%
\pgfpathrectangle{\pgfqpoint{1.432000in}{0.528000in}}{\pgfqpoint{3.696000in}{3.696000in}} %
\pgfusepath{clip}%
\pgfsetbuttcap%
\pgfsetroundjoin%
\definecolor{currentfill}{rgb}{0.139147,0.533812,0.555298}%
\pgfsetfillcolor{currentfill}%
\pgfsetlinewidth{0.000000pt}%
\definecolor{currentstroke}{rgb}{0.000000,0.000000,0.000000}%
\pgfsetstrokecolor{currentstroke}%
\pgfsetdash{}{0pt}%
\pgfpathmoveto{\pgfqpoint{3.550968in}{0.799952in}}%
\pgfpathlineto{\pgfqpoint{3.482497in}{0.799952in}}%
\pgfpathlineto{\pgfqpoint{3.486933in}{0.791081in}}%
\pgfpathlineto{\pgfqpoint{3.442581in}{0.804387in}}%
\pgfpathlineto{\pgfqpoint{3.486933in}{0.817693in}}%
\pgfpathlineto{\pgfqpoint{3.482497in}{0.808822in}}%
\pgfpathlineto{\pgfqpoint{3.550968in}{0.808822in}}%
\pgfpathlineto{\pgfqpoint{3.550968in}{0.799952in}}%
\pgfusepath{fill}%
\end{pgfscope}%
\begin{pgfscope}%
\pgfpathrectangle{\pgfqpoint{1.432000in}{0.528000in}}{\pgfqpoint{3.696000in}{3.696000in}} %
\pgfusepath{clip}%
\pgfsetbuttcap%
\pgfsetroundjoin%
\definecolor{currentfill}{rgb}{0.273809,0.031497,0.358853}%
\pgfsetfillcolor{currentfill}%
\pgfsetlinewidth{0.000000pt}%
\definecolor{currentstroke}{rgb}{0.000000,0.000000,0.000000}%
\pgfsetstrokecolor{currentstroke}%
\pgfsetdash{}{0pt}%
\pgfpathmoveto{\pgfqpoint{3.547832in}{0.801251in}}%
\pgfpathlineto{\pgfqpoint{3.467670in}{0.881413in}}%
\pgfpathlineto{\pgfqpoint{3.464534in}{0.872004in}}%
\pgfpathlineto{\pgfqpoint{3.442581in}{0.912774in}}%
\pgfpathlineto{\pgfqpoint{3.483351in}{0.890821in}}%
\pgfpathlineto{\pgfqpoint{3.473942in}{0.887685in}}%
\pgfpathlineto{\pgfqpoint{3.554104in}{0.807523in}}%
\pgfpathlineto{\pgfqpoint{3.547832in}{0.801251in}}%
\pgfusepath{fill}%
\end{pgfscope}%
\begin{pgfscope}%
\pgfpathrectangle{\pgfqpoint{1.432000in}{0.528000in}}{\pgfqpoint{3.696000in}{3.696000in}} %
\pgfusepath{clip}%
\pgfsetbuttcap%
\pgfsetroundjoin%
\definecolor{currentfill}{rgb}{0.119699,0.618490,0.536347}%
\pgfsetfillcolor{currentfill}%
\pgfsetlinewidth{0.000000pt}%
\definecolor{currentstroke}{rgb}{0.000000,0.000000,0.000000}%
\pgfsetstrokecolor{currentstroke}%
\pgfsetdash{}{0pt}%
\pgfpathmoveto{\pgfqpoint{3.659355in}{0.799952in}}%
\pgfpathlineto{\pgfqpoint{3.590885in}{0.799952in}}%
\pgfpathlineto{\pgfqpoint{3.595320in}{0.791081in}}%
\pgfpathlineto{\pgfqpoint{3.550968in}{0.804387in}}%
\pgfpathlineto{\pgfqpoint{3.595320in}{0.817693in}}%
\pgfpathlineto{\pgfqpoint{3.590885in}{0.808822in}}%
\pgfpathlineto{\pgfqpoint{3.659355in}{0.808822in}}%
\pgfpathlineto{\pgfqpoint{3.659355in}{0.799952in}}%
\pgfusepath{fill}%
\end{pgfscope}%
\begin{pgfscope}%
\pgfpathrectangle{\pgfqpoint{1.432000in}{0.528000in}}{\pgfqpoint{3.696000in}{3.696000in}} %
\pgfusepath{clip}%
\pgfsetbuttcap%
\pgfsetroundjoin%
\definecolor{currentfill}{rgb}{0.146180,0.515413,0.556823}%
\pgfsetfillcolor{currentfill}%
\pgfsetlinewidth{0.000000pt}%
\definecolor{currentstroke}{rgb}{0.000000,0.000000,0.000000}%
\pgfsetstrokecolor{currentstroke}%
\pgfsetdash{}{0pt}%
\pgfpathmoveto{\pgfqpoint{3.767742in}{0.799952in}}%
\pgfpathlineto{\pgfqpoint{3.699272in}{0.799952in}}%
\pgfpathlineto{\pgfqpoint{3.703707in}{0.791081in}}%
\pgfpathlineto{\pgfqpoint{3.659355in}{0.804387in}}%
\pgfpathlineto{\pgfqpoint{3.703707in}{0.817693in}}%
\pgfpathlineto{\pgfqpoint{3.699272in}{0.808822in}}%
\pgfpathlineto{\pgfqpoint{3.767742in}{0.808822in}}%
\pgfpathlineto{\pgfqpoint{3.767742in}{0.799952in}}%
\pgfusepath{fill}%
\end{pgfscope}%
\begin{pgfscope}%
\pgfpathrectangle{\pgfqpoint{1.432000in}{0.528000in}}{\pgfqpoint{3.696000in}{3.696000in}} %
\pgfusepath{clip}%
\pgfsetbuttcap%
\pgfsetroundjoin%
\definecolor{currentfill}{rgb}{0.283197,0.115680,0.436115}%
\pgfsetfillcolor{currentfill}%
\pgfsetlinewidth{0.000000pt}%
\definecolor{currentstroke}{rgb}{0.000000,0.000000,0.000000}%
\pgfsetstrokecolor{currentstroke}%
\pgfsetdash{}{0pt}%
\pgfpathmoveto{\pgfqpoint{3.876129in}{0.799952in}}%
\pgfpathlineto{\pgfqpoint{3.807659in}{0.799952in}}%
\pgfpathlineto{\pgfqpoint{3.812094in}{0.791081in}}%
\pgfpathlineto{\pgfqpoint{3.767742in}{0.804387in}}%
\pgfpathlineto{\pgfqpoint{3.812094in}{0.817693in}}%
\pgfpathlineto{\pgfqpoint{3.807659in}{0.808822in}}%
\pgfpathlineto{\pgfqpoint{3.876129in}{0.808822in}}%
\pgfpathlineto{\pgfqpoint{3.876129in}{0.799952in}}%
\pgfusepath{fill}%
\end{pgfscope}%
\begin{pgfscope}%
\pgfpathrectangle{\pgfqpoint{1.432000in}{0.528000in}}{\pgfqpoint{3.696000in}{3.696000in}} %
\pgfusepath{clip}%
\pgfsetbuttcap%
\pgfsetroundjoin%
\definecolor{currentfill}{rgb}{0.279574,0.170599,0.479997}%
\pgfsetfillcolor{currentfill}%
\pgfsetlinewidth{0.000000pt}%
\definecolor{currentstroke}{rgb}{0.000000,0.000000,0.000000}%
\pgfsetstrokecolor{currentstroke}%
\pgfsetdash{}{0pt}%
\pgfpathmoveto{\pgfqpoint{3.984516in}{0.799952in}}%
\pgfpathlineto{\pgfqpoint{3.807659in}{0.799952in}}%
\pgfpathlineto{\pgfqpoint{3.812094in}{0.791081in}}%
\pgfpathlineto{\pgfqpoint{3.767742in}{0.804387in}}%
\pgfpathlineto{\pgfqpoint{3.812094in}{0.817693in}}%
\pgfpathlineto{\pgfqpoint{3.807659in}{0.808822in}}%
\pgfpathlineto{\pgfqpoint{3.984516in}{0.808822in}}%
\pgfpathlineto{\pgfqpoint{3.984516in}{0.799952in}}%
\pgfusepath{fill}%
\end{pgfscope}%
\begin{pgfscope}%
\pgfpathrectangle{\pgfqpoint{1.432000in}{0.528000in}}{\pgfqpoint{3.696000in}{3.696000in}} %
\pgfusepath{clip}%
\pgfsetbuttcap%
\pgfsetroundjoin%
\definecolor{currentfill}{rgb}{0.278791,0.062145,0.386592}%
\pgfsetfillcolor{currentfill}%
\pgfsetlinewidth{0.000000pt}%
\definecolor{currentstroke}{rgb}{0.000000,0.000000,0.000000}%
\pgfsetstrokecolor{currentstroke}%
\pgfsetdash{}{0pt}%
\pgfpathmoveto{\pgfqpoint{4.201290in}{0.799952in}}%
\pgfpathlineto{\pgfqpoint{3.916046in}{0.799952in}}%
\pgfpathlineto{\pgfqpoint{3.920481in}{0.791081in}}%
\pgfpathlineto{\pgfqpoint{3.876129in}{0.804387in}}%
\pgfpathlineto{\pgfqpoint{3.920481in}{0.817693in}}%
\pgfpathlineto{\pgfqpoint{3.916046in}{0.808822in}}%
\pgfpathlineto{\pgfqpoint{4.201290in}{0.808822in}}%
\pgfpathlineto{\pgfqpoint{4.201290in}{0.799952in}}%
\pgfusepath{fill}%
\end{pgfscope}%
\begin{pgfscope}%
\pgfpathrectangle{\pgfqpoint{1.432000in}{0.528000in}}{\pgfqpoint{3.696000in}{3.696000in}} %
\pgfusepath{clip}%
\pgfsetbuttcap%
\pgfsetroundjoin%
\definecolor{currentfill}{rgb}{0.280894,0.078907,0.402329}%
\pgfsetfillcolor{currentfill}%
\pgfsetlinewidth{0.000000pt}%
\definecolor{currentstroke}{rgb}{0.000000,0.000000,0.000000}%
\pgfsetstrokecolor{currentstroke}%
\pgfsetdash{}{0pt}%
\pgfpathmoveto{\pgfqpoint{4.309677in}{0.799952in}}%
\pgfpathlineto{\pgfqpoint{4.024433in}{0.799952in}}%
\pgfpathlineto{\pgfqpoint{4.028868in}{0.791081in}}%
\pgfpathlineto{\pgfqpoint{3.984516in}{0.804387in}}%
\pgfpathlineto{\pgfqpoint{4.028868in}{0.817693in}}%
\pgfpathlineto{\pgfqpoint{4.024433in}{0.808822in}}%
\pgfpathlineto{\pgfqpoint{4.309677in}{0.808822in}}%
\pgfpathlineto{\pgfqpoint{4.309677in}{0.799952in}}%
\pgfusepath{fill}%
\end{pgfscope}%
\begin{pgfscope}%
\pgfpathrectangle{\pgfqpoint{1.432000in}{0.528000in}}{\pgfqpoint{3.696000in}{3.696000in}} %
\pgfusepath{clip}%
\pgfsetbuttcap%
\pgfsetroundjoin%
\definecolor{currentfill}{rgb}{0.277018,0.050344,0.375715}%
\pgfsetfillcolor{currentfill}%
\pgfsetlinewidth{0.000000pt}%
\definecolor{currentstroke}{rgb}{0.000000,0.000000,0.000000}%
\pgfsetstrokecolor{currentstroke}%
\pgfsetdash{}{0pt}%
\pgfpathmoveto{\pgfqpoint{4.419467in}{0.800179in}}%
\pgfpathlineto{\pgfqpoint{4.132174in}{0.704415in}}%
\pgfpathlineto{\pgfqpoint{4.139187in}{0.697403in}}%
\pgfpathlineto{\pgfqpoint{4.092903in}{0.696000in}}%
\pgfpathlineto{\pgfqpoint{4.130772in}{0.722648in}}%
\pgfpathlineto{\pgfqpoint{4.129369in}{0.712830in}}%
\pgfpathlineto{\pgfqpoint{4.416662in}{0.808595in}}%
\pgfpathlineto{\pgfqpoint{4.419467in}{0.800179in}}%
\pgfusepath{fill}%
\end{pgfscope}%
\begin{pgfscope}%
\pgfpathrectangle{\pgfqpoint{1.432000in}{0.528000in}}{\pgfqpoint{3.696000in}{3.696000in}} %
\pgfusepath{clip}%
\pgfsetbuttcap%
\pgfsetroundjoin%
\definecolor{currentfill}{rgb}{0.283229,0.120777,0.440584}%
\pgfsetfillcolor{currentfill}%
\pgfsetlinewidth{0.000000pt}%
\definecolor{currentstroke}{rgb}{0.000000,0.000000,0.000000}%
\pgfsetstrokecolor{currentstroke}%
\pgfsetdash{}{0pt}%
\pgfpathmoveto{\pgfqpoint{4.418065in}{0.799952in}}%
\pgfpathlineto{\pgfqpoint{4.132820in}{0.799952in}}%
\pgfpathlineto{\pgfqpoint{4.137255in}{0.791081in}}%
\pgfpathlineto{\pgfqpoint{4.092903in}{0.804387in}}%
\pgfpathlineto{\pgfqpoint{4.137255in}{0.817693in}}%
\pgfpathlineto{\pgfqpoint{4.132820in}{0.808822in}}%
\pgfpathlineto{\pgfqpoint{4.418065in}{0.808822in}}%
\pgfpathlineto{\pgfqpoint{4.418065in}{0.799952in}}%
\pgfusepath{fill}%
\end{pgfscope}%
\begin{pgfscope}%
\pgfpathrectangle{\pgfqpoint{1.432000in}{0.528000in}}{\pgfqpoint{3.696000in}{3.696000in}} %
\pgfusepath{clip}%
\pgfsetbuttcap%
\pgfsetroundjoin%
\definecolor{currentfill}{rgb}{0.266580,0.228262,0.514349}%
\pgfsetfillcolor{currentfill}%
\pgfsetlinewidth{0.000000pt}%
\definecolor{currentstroke}{rgb}{0.000000,0.000000,0.000000}%
\pgfsetstrokecolor{currentstroke}%
\pgfsetdash{}{0pt}%
\pgfpathmoveto{\pgfqpoint{4.526452in}{0.799952in}}%
\pgfpathlineto{\pgfqpoint{4.241207in}{0.799952in}}%
\pgfpathlineto{\pgfqpoint{4.245642in}{0.791081in}}%
\pgfpathlineto{\pgfqpoint{4.201290in}{0.804387in}}%
\pgfpathlineto{\pgfqpoint{4.245642in}{0.817693in}}%
\pgfpathlineto{\pgfqpoint{4.241207in}{0.808822in}}%
\pgfpathlineto{\pgfqpoint{4.526452in}{0.808822in}}%
\pgfpathlineto{\pgfqpoint{4.526452in}{0.799952in}}%
\pgfusepath{fill}%
\end{pgfscope}%
\begin{pgfscope}%
\pgfpathrectangle{\pgfqpoint{1.432000in}{0.528000in}}{\pgfqpoint{3.696000in}{3.696000in}} %
\pgfusepath{clip}%
\pgfsetbuttcap%
\pgfsetroundjoin%
\definecolor{currentfill}{rgb}{0.283091,0.110553,0.431554}%
\pgfsetfillcolor{currentfill}%
\pgfsetlinewidth{0.000000pt}%
\definecolor{currentstroke}{rgb}{0.000000,0.000000,0.000000}%
\pgfsetstrokecolor{currentstroke}%
\pgfsetdash{}{0pt}%
\pgfpathmoveto{\pgfqpoint{4.634839in}{0.799952in}}%
\pgfpathlineto{\pgfqpoint{4.349594in}{0.799952in}}%
\pgfpathlineto{\pgfqpoint{4.354029in}{0.791081in}}%
\pgfpathlineto{\pgfqpoint{4.309677in}{0.804387in}}%
\pgfpathlineto{\pgfqpoint{4.354029in}{0.817693in}}%
\pgfpathlineto{\pgfqpoint{4.349594in}{0.808822in}}%
\pgfpathlineto{\pgfqpoint{4.634839in}{0.808822in}}%
\pgfpathlineto{\pgfqpoint{4.634839in}{0.799952in}}%
\pgfusepath{fill}%
\end{pgfscope}%
\begin{pgfscope}%
\pgfpathrectangle{\pgfqpoint{1.432000in}{0.528000in}}{\pgfqpoint{3.696000in}{3.696000in}} %
\pgfusepath{clip}%
\pgfsetbuttcap%
\pgfsetroundjoin%
\definecolor{currentfill}{rgb}{0.244972,0.287675,0.537260}%
\pgfsetfillcolor{currentfill}%
\pgfsetlinewidth{0.000000pt}%
\definecolor{currentstroke}{rgb}{0.000000,0.000000,0.000000}%
\pgfsetstrokecolor{currentstroke}%
\pgfsetdash{}{0pt}%
\pgfpathmoveto{\pgfqpoint{4.634839in}{0.799952in}}%
\pgfpathlineto{\pgfqpoint{4.457981in}{0.799952in}}%
\pgfpathlineto{\pgfqpoint{4.462417in}{0.791081in}}%
\pgfpathlineto{\pgfqpoint{4.418065in}{0.804387in}}%
\pgfpathlineto{\pgfqpoint{4.462417in}{0.817693in}}%
\pgfpathlineto{\pgfqpoint{4.457981in}{0.808822in}}%
\pgfpathlineto{\pgfqpoint{4.634839in}{0.808822in}}%
\pgfpathlineto{\pgfqpoint{4.634839in}{0.799952in}}%
\pgfusepath{fill}%
\end{pgfscope}%
\begin{pgfscope}%
\pgfpathrectangle{\pgfqpoint{1.432000in}{0.528000in}}{\pgfqpoint{3.696000in}{3.696000in}} %
\pgfusepath{clip}%
\pgfsetbuttcap%
\pgfsetroundjoin%
\definecolor{currentfill}{rgb}{0.244972,0.287675,0.537260}%
\pgfsetfillcolor{currentfill}%
\pgfsetlinewidth{0.000000pt}%
\definecolor{currentstroke}{rgb}{0.000000,0.000000,0.000000}%
\pgfsetstrokecolor{currentstroke}%
\pgfsetdash{}{0pt}%
\pgfpathmoveto{\pgfqpoint{4.743226in}{0.799952in}}%
\pgfpathlineto{\pgfqpoint{4.566368in}{0.799952in}}%
\pgfpathlineto{\pgfqpoint{4.570804in}{0.791081in}}%
\pgfpathlineto{\pgfqpoint{4.526452in}{0.804387in}}%
\pgfpathlineto{\pgfqpoint{4.570804in}{0.817693in}}%
\pgfpathlineto{\pgfqpoint{4.566368in}{0.808822in}}%
\pgfpathlineto{\pgfqpoint{4.743226in}{0.808822in}}%
\pgfpathlineto{\pgfqpoint{4.743226in}{0.799952in}}%
\pgfusepath{fill}%
\end{pgfscope}%
\begin{pgfscope}%
\pgfpathrectangle{\pgfqpoint{1.432000in}{0.528000in}}{\pgfqpoint{3.696000in}{3.696000in}} %
\pgfusepath{clip}%
\pgfsetbuttcap%
\pgfsetroundjoin%
\definecolor{currentfill}{rgb}{0.272594,0.025563,0.353093}%
\pgfsetfillcolor{currentfill}%
\pgfsetlinewidth{0.000000pt}%
\definecolor{currentstroke}{rgb}{0.000000,0.000000,0.000000}%
\pgfsetstrokecolor{currentstroke}%
\pgfsetdash{}{0pt}%
\pgfpathmoveto{\pgfqpoint{4.743226in}{0.799952in}}%
\pgfpathlineto{\pgfqpoint{4.674756in}{0.799952in}}%
\pgfpathlineto{\pgfqpoint{4.679191in}{0.791081in}}%
\pgfpathlineto{\pgfqpoint{4.634839in}{0.804387in}}%
\pgfpathlineto{\pgfqpoint{4.679191in}{0.817693in}}%
\pgfpathlineto{\pgfqpoint{4.674756in}{0.808822in}}%
\pgfpathlineto{\pgfqpoint{4.743226in}{0.808822in}}%
\pgfpathlineto{\pgfqpoint{4.743226in}{0.799952in}}%
\pgfusepath{fill}%
\end{pgfscope}%
\begin{pgfscope}%
\pgfpathrectangle{\pgfqpoint{1.432000in}{0.528000in}}{\pgfqpoint{3.696000in}{3.696000in}} %
\pgfusepath{clip}%
\pgfsetbuttcap%
\pgfsetroundjoin%
\definecolor{currentfill}{rgb}{0.280894,0.078907,0.402329}%
\pgfsetfillcolor{currentfill}%
\pgfsetlinewidth{0.000000pt}%
\definecolor{currentstroke}{rgb}{0.000000,0.000000,0.000000}%
\pgfsetstrokecolor{currentstroke}%
\pgfsetdash{}{0pt}%
\pgfpathmoveto{\pgfqpoint{4.851613in}{0.799952in}}%
\pgfpathlineto{\pgfqpoint{4.674756in}{0.799952in}}%
\pgfpathlineto{\pgfqpoint{4.679191in}{0.791081in}}%
\pgfpathlineto{\pgfqpoint{4.634839in}{0.804387in}}%
\pgfpathlineto{\pgfqpoint{4.679191in}{0.817693in}}%
\pgfpathlineto{\pgfqpoint{4.674756in}{0.808822in}}%
\pgfpathlineto{\pgfqpoint{4.851613in}{0.808822in}}%
\pgfpathlineto{\pgfqpoint{4.851613in}{0.799952in}}%
\pgfusepath{fill}%
\end{pgfscope}%
\begin{pgfscope}%
\pgfpathrectangle{\pgfqpoint{1.432000in}{0.528000in}}{\pgfqpoint{3.696000in}{3.696000in}} %
\pgfusepath{clip}%
\pgfsetbuttcap%
\pgfsetroundjoin%
\definecolor{currentfill}{rgb}{0.255645,0.260703,0.528312}%
\pgfsetfillcolor{currentfill}%
\pgfsetlinewidth{0.000000pt}%
\definecolor{currentstroke}{rgb}{0.000000,0.000000,0.000000}%
\pgfsetstrokecolor{currentstroke}%
\pgfsetdash{}{0pt}%
\pgfpathmoveto{\pgfqpoint{4.851613in}{0.799952in}}%
\pgfpathlineto{\pgfqpoint{4.783143in}{0.799952in}}%
\pgfpathlineto{\pgfqpoint{4.787578in}{0.791081in}}%
\pgfpathlineto{\pgfqpoint{4.743226in}{0.804387in}}%
\pgfpathlineto{\pgfqpoint{4.787578in}{0.817693in}}%
\pgfpathlineto{\pgfqpoint{4.783143in}{0.808822in}}%
\pgfpathlineto{\pgfqpoint{4.851613in}{0.808822in}}%
\pgfpathlineto{\pgfqpoint{4.851613in}{0.799952in}}%
\pgfusepath{fill}%
\end{pgfscope}%
\begin{pgfscope}%
\pgfpathrectangle{\pgfqpoint{1.432000in}{0.528000in}}{\pgfqpoint{3.696000in}{3.696000in}} %
\pgfusepath{clip}%
\pgfsetbuttcap%
\pgfsetroundjoin%
\definecolor{currentfill}{rgb}{0.273006,0.204520,0.501721}%
\pgfsetfillcolor{currentfill}%
\pgfsetlinewidth{0.000000pt}%
\definecolor{currentstroke}{rgb}{0.000000,0.000000,0.000000}%
\pgfsetstrokecolor{currentstroke}%
\pgfsetdash{}{0pt}%
\pgfpathmoveto{\pgfqpoint{4.960000in}{0.799952in}}%
\pgfpathlineto{\pgfqpoint{4.891530in}{0.799952in}}%
\pgfpathlineto{\pgfqpoint{4.895965in}{0.791081in}}%
\pgfpathlineto{\pgfqpoint{4.851613in}{0.804387in}}%
\pgfpathlineto{\pgfqpoint{4.895965in}{0.817693in}}%
\pgfpathlineto{\pgfqpoint{4.891530in}{0.808822in}}%
\pgfpathlineto{\pgfqpoint{4.960000in}{0.808822in}}%
\pgfpathlineto{\pgfqpoint{4.960000in}{0.799952in}}%
\pgfusepath{fill}%
\end{pgfscope}%
\begin{pgfscope}%
\pgfpathrectangle{\pgfqpoint{1.432000in}{0.528000in}}{\pgfqpoint{3.696000in}{3.696000in}} %
\pgfusepath{clip}%
\pgfsetbuttcap%
\pgfsetroundjoin%
\definecolor{currentfill}{rgb}{0.169646,0.456262,0.558030}%
\pgfsetfillcolor{currentfill}%
\pgfsetlinewidth{0.000000pt}%
\definecolor{currentstroke}{rgb}{0.000000,0.000000,0.000000}%
\pgfsetstrokecolor{currentstroke}%
\pgfsetdash{}{0pt}%
\pgfpathmoveto{\pgfqpoint{4.964435in}{0.804387in}}%
\pgfpathlineto{\pgfqpoint{4.962218in}{0.808228in}}%
\pgfpathlineto{\pgfqpoint{4.957782in}{0.808228in}}%
\pgfpathlineto{\pgfqpoint{4.955565in}{0.804387in}}%
\pgfpathlineto{\pgfqpoint{4.957782in}{0.800546in}}%
\pgfpathlineto{\pgfqpoint{4.962218in}{0.800546in}}%
\pgfpathlineto{\pgfqpoint{4.964435in}{0.804387in}}%
\pgfpathlineto{\pgfqpoint{4.962218in}{0.808228in}}%
\pgfusepath{fill}%
\end{pgfscope}%
\begin{pgfscope}%
\pgfpathrectangle{\pgfqpoint{1.432000in}{0.528000in}}{\pgfqpoint{3.696000in}{3.696000in}} %
\pgfusepath{clip}%
\pgfsetbuttcap%
\pgfsetroundjoin%
\definecolor{currentfill}{rgb}{0.279566,0.067836,0.391917}%
\pgfsetfillcolor{currentfill}%
\pgfsetlinewidth{0.000000pt}%
\definecolor{currentstroke}{rgb}{0.000000,0.000000,0.000000}%
\pgfsetstrokecolor{currentstroke}%
\pgfsetdash{}{0pt}%
\pgfpathmoveto{\pgfqpoint{1.604435in}{0.912774in}}%
\pgfpathlineto{\pgfqpoint{1.602218in}{0.916615in}}%
\pgfpathlineto{\pgfqpoint{1.597782in}{0.916615in}}%
\pgfpathlineto{\pgfqpoint{1.595565in}{0.912774in}}%
\pgfpathlineto{\pgfqpoint{1.597782in}{0.908933in}}%
\pgfpathlineto{\pgfqpoint{1.602218in}{0.908933in}}%
\pgfpathlineto{\pgfqpoint{1.604435in}{0.912774in}}%
\pgfpathlineto{\pgfqpoint{1.602218in}{0.916615in}}%
\pgfusepath{fill}%
\end{pgfscope}%
\begin{pgfscope}%
\pgfpathrectangle{\pgfqpoint{1.432000in}{0.528000in}}{\pgfqpoint{3.696000in}{3.696000in}} %
\pgfusepath{clip}%
\pgfsetbuttcap%
\pgfsetroundjoin%
\definecolor{currentfill}{rgb}{0.279566,0.067836,0.391917}%
\pgfsetfillcolor{currentfill}%
\pgfsetlinewidth{0.000000pt}%
\definecolor{currentstroke}{rgb}{0.000000,0.000000,0.000000}%
\pgfsetstrokecolor{currentstroke}%
\pgfsetdash{}{0pt}%
\pgfpathmoveto{\pgfqpoint{1.595565in}{0.912774in}}%
\pgfpathlineto{\pgfqpoint{1.595565in}{0.981244in}}%
\pgfpathlineto{\pgfqpoint{1.586694in}{0.976809in}}%
\pgfpathlineto{\pgfqpoint{1.600000in}{1.021161in}}%
\pgfpathlineto{\pgfqpoint{1.613306in}{0.976809in}}%
\pgfpathlineto{\pgfqpoint{1.604435in}{0.981244in}}%
\pgfpathlineto{\pgfqpoint{1.604435in}{0.912774in}}%
\pgfpathlineto{\pgfqpoint{1.595565in}{0.912774in}}%
\pgfusepath{fill}%
\end{pgfscope}%
\begin{pgfscope}%
\pgfpathrectangle{\pgfqpoint{1.432000in}{0.528000in}}{\pgfqpoint{3.696000in}{3.696000in}} %
\pgfusepath{clip}%
\pgfsetbuttcap%
\pgfsetroundjoin%
\definecolor{currentfill}{rgb}{0.255645,0.260703,0.528312}%
\pgfsetfillcolor{currentfill}%
\pgfsetlinewidth{0.000000pt}%
\definecolor{currentstroke}{rgb}{0.000000,0.000000,0.000000}%
\pgfsetstrokecolor{currentstroke}%
\pgfsetdash{}{0pt}%
\pgfpathmoveto{\pgfqpoint{1.816774in}{0.908339in}}%
\pgfpathlineto{\pgfqpoint{1.748304in}{0.908339in}}%
\pgfpathlineto{\pgfqpoint{1.752739in}{0.899469in}}%
\pgfpathlineto{\pgfqpoint{1.708387in}{0.912774in}}%
\pgfpathlineto{\pgfqpoint{1.752739in}{0.926080in}}%
\pgfpathlineto{\pgfqpoint{1.748304in}{0.917209in}}%
\pgfpathlineto{\pgfqpoint{1.816774in}{0.917209in}}%
\pgfpathlineto{\pgfqpoint{1.816774in}{0.908339in}}%
\pgfusepath{fill}%
\end{pgfscope}%
\begin{pgfscope}%
\pgfpathrectangle{\pgfqpoint{1.432000in}{0.528000in}}{\pgfqpoint{3.696000in}{3.696000in}} %
\pgfusepath{clip}%
\pgfsetbuttcap%
\pgfsetroundjoin%
\definecolor{currentfill}{rgb}{0.243113,0.292092,0.538516}%
\pgfsetfillcolor{currentfill}%
\pgfsetlinewidth{0.000000pt}%
\definecolor{currentstroke}{rgb}{0.000000,0.000000,0.000000}%
\pgfsetstrokecolor{currentstroke}%
\pgfsetdash{}{0pt}%
\pgfpathmoveto{\pgfqpoint{1.925161in}{0.908339in}}%
\pgfpathlineto{\pgfqpoint{1.856691in}{0.908339in}}%
\pgfpathlineto{\pgfqpoint{1.861126in}{0.899469in}}%
\pgfpathlineto{\pgfqpoint{1.816774in}{0.912774in}}%
\pgfpathlineto{\pgfqpoint{1.861126in}{0.926080in}}%
\pgfpathlineto{\pgfqpoint{1.856691in}{0.917209in}}%
\pgfpathlineto{\pgfqpoint{1.925161in}{0.917209in}}%
\pgfpathlineto{\pgfqpoint{1.925161in}{0.908339in}}%
\pgfusepath{fill}%
\end{pgfscope}%
\begin{pgfscope}%
\pgfpathrectangle{\pgfqpoint{1.432000in}{0.528000in}}{\pgfqpoint{3.696000in}{3.696000in}} %
\pgfusepath{clip}%
\pgfsetbuttcap%
\pgfsetroundjoin%
\definecolor{currentfill}{rgb}{0.199430,0.387607,0.554642}%
\pgfsetfillcolor{currentfill}%
\pgfsetlinewidth{0.000000pt}%
\definecolor{currentstroke}{rgb}{0.000000,0.000000,0.000000}%
\pgfsetstrokecolor{currentstroke}%
\pgfsetdash{}{0pt}%
\pgfpathmoveto{\pgfqpoint{2.033548in}{0.908339in}}%
\pgfpathlineto{\pgfqpoint{1.856691in}{0.908339in}}%
\pgfpathlineto{\pgfqpoint{1.861126in}{0.899469in}}%
\pgfpathlineto{\pgfqpoint{1.816774in}{0.912774in}}%
\pgfpathlineto{\pgfqpoint{1.861126in}{0.926080in}}%
\pgfpathlineto{\pgfqpoint{1.856691in}{0.917209in}}%
\pgfpathlineto{\pgfqpoint{2.033548in}{0.917209in}}%
\pgfpathlineto{\pgfqpoint{2.033548in}{0.908339in}}%
\pgfusepath{fill}%
\end{pgfscope}%
\begin{pgfscope}%
\pgfpathrectangle{\pgfqpoint{1.432000in}{0.528000in}}{\pgfqpoint{3.696000in}{3.696000in}} %
\pgfusepath{clip}%
\pgfsetbuttcap%
\pgfsetroundjoin%
\definecolor{currentfill}{rgb}{0.180629,0.429975,0.557282}%
\pgfsetfillcolor{currentfill}%
\pgfsetlinewidth{0.000000pt}%
\definecolor{currentstroke}{rgb}{0.000000,0.000000,0.000000}%
\pgfsetstrokecolor{currentstroke}%
\pgfsetdash{}{0pt}%
\pgfpathmoveto{\pgfqpoint{2.141935in}{0.908339in}}%
\pgfpathlineto{\pgfqpoint{1.965078in}{0.908339in}}%
\pgfpathlineto{\pgfqpoint{1.969513in}{0.899469in}}%
\pgfpathlineto{\pgfqpoint{1.925161in}{0.912774in}}%
\pgfpathlineto{\pgfqpoint{1.969513in}{0.926080in}}%
\pgfpathlineto{\pgfqpoint{1.965078in}{0.917209in}}%
\pgfpathlineto{\pgfqpoint{2.141935in}{0.917209in}}%
\pgfpathlineto{\pgfqpoint{2.141935in}{0.908339in}}%
\pgfusepath{fill}%
\end{pgfscope}%
\begin{pgfscope}%
\pgfpathrectangle{\pgfqpoint{1.432000in}{0.528000in}}{\pgfqpoint{3.696000in}{3.696000in}} %
\pgfusepath{clip}%
\pgfsetbuttcap%
\pgfsetroundjoin%
\definecolor{currentfill}{rgb}{0.283187,0.125848,0.444960}%
\pgfsetfillcolor{currentfill}%
\pgfsetlinewidth{0.000000pt}%
\definecolor{currentstroke}{rgb}{0.000000,0.000000,0.000000}%
\pgfsetstrokecolor{currentstroke}%
\pgfsetdash{}{0pt}%
\pgfpathmoveto{\pgfqpoint{2.250323in}{0.908339in}}%
\pgfpathlineto{\pgfqpoint{2.073465in}{0.908339in}}%
\pgfpathlineto{\pgfqpoint{2.077900in}{0.899469in}}%
\pgfpathlineto{\pgfqpoint{2.033548in}{0.912774in}}%
\pgfpathlineto{\pgfqpoint{2.077900in}{0.926080in}}%
\pgfpathlineto{\pgfqpoint{2.073465in}{0.917209in}}%
\pgfpathlineto{\pgfqpoint{2.250323in}{0.917209in}}%
\pgfpathlineto{\pgfqpoint{2.250323in}{0.908339in}}%
\pgfusepath{fill}%
\end{pgfscope}%
\begin{pgfscope}%
\pgfpathrectangle{\pgfqpoint{1.432000in}{0.528000in}}{\pgfqpoint{3.696000in}{3.696000in}} %
\pgfusepath{clip}%
\pgfsetbuttcap%
\pgfsetroundjoin%
\definecolor{currentfill}{rgb}{0.277018,0.050344,0.375715}%
\pgfsetfillcolor{currentfill}%
\pgfsetlinewidth{0.000000pt}%
\definecolor{currentstroke}{rgb}{0.000000,0.000000,0.000000}%
\pgfsetstrokecolor{currentstroke}%
\pgfsetdash{}{0pt}%
\pgfpathmoveto{\pgfqpoint{2.358710in}{0.908339in}}%
\pgfpathlineto{\pgfqpoint{2.073465in}{0.908339in}}%
\pgfpathlineto{\pgfqpoint{2.077900in}{0.899469in}}%
\pgfpathlineto{\pgfqpoint{2.033548in}{0.912774in}}%
\pgfpathlineto{\pgfqpoint{2.077900in}{0.926080in}}%
\pgfpathlineto{\pgfqpoint{2.073465in}{0.917209in}}%
\pgfpathlineto{\pgfqpoint{2.358710in}{0.917209in}}%
\pgfpathlineto{\pgfqpoint{2.358710in}{0.908339in}}%
\pgfusepath{fill}%
\end{pgfscope}%
\begin{pgfscope}%
\pgfpathrectangle{\pgfqpoint{1.432000in}{0.528000in}}{\pgfqpoint{3.696000in}{3.696000in}} %
\pgfusepath{clip}%
\pgfsetbuttcap%
\pgfsetroundjoin%
\definecolor{currentfill}{rgb}{0.216210,0.351535,0.550627}%
\pgfsetfillcolor{currentfill}%
\pgfsetlinewidth{0.000000pt}%
\definecolor{currentstroke}{rgb}{0.000000,0.000000,0.000000}%
\pgfsetstrokecolor{currentstroke}%
\pgfsetdash{}{0pt}%
\pgfpathmoveto{\pgfqpoint{2.467097in}{0.908339in}}%
\pgfpathlineto{\pgfqpoint{2.181852in}{0.908339in}}%
\pgfpathlineto{\pgfqpoint{2.186287in}{0.899469in}}%
\pgfpathlineto{\pgfqpoint{2.141935in}{0.912774in}}%
\pgfpathlineto{\pgfqpoint{2.186287in}{0.926080in}}%
\pgfpathlineto{\pgfqpoint{2.181852in}{0.917209in}}%
\pgfpathlineto{\pgfqpoint{2.467097in}{0.917209in}}%
\pgfpathlineto{\pgfqpoint{2.467097in}{0.908339in}}%
\pgfusepath{fill}%
\end{pgfscope}%
\begin{pgfscope}%
\pgfpathrectangle{\pgfqpoint{1.432000in}{0.528000in}}{\pgfqpoint{3.696000in}{3.696000in}} %
\pgfusepath{clip}%
\pgfsetbuttcap%
\pgfsetroundjoin%
\definecolor{currentfill}{rgb}{0.273006,0.204520,0.501721}%
\pgfsetfillcolor{currentfill}%
\pgfsetlinewidth{0.000000pt}%
\definecolor{currentstroke}{rgb}{0.000000,0.000000,0.000000}%
\pgfsetstrokecolor{currentstroke}%
\pgfsetdash{}{0pt}%
\pgfpathmoveto{\pgfqpoint{2.575484in}{0.908339in}}%
\pgfpathlineto{\pgfqpoint{2.181852in}{0.908339in}}%
\pgfpathlineto{\pgfqpoint{2.186287in}{0.899469in}}%
\pgfpathlineto{\pgfqpoint{2.141935in}{0.912774in}}%
\pgfpathlineto{\pgfqpoint{2.186287in}{0.926080in}}%
\pgfpathlineto{\pgfqpoint{2.181852in}{0.917209in}}%
\pgfpathlineto{\pgfqpoint{2.575484in}{0.917209in}}%
\pgfpathlineto{\pgfqpoint{2.575484in}{0.908339in}}%
\pgfusepath{fill}%
\end{pgfscope}%
\begin{pgfscope}%
\pgfpathrectangle{\pgfqpoint{1.432000in}{0.528000in}}{\pgfqpoint{3.696000in}{3.696000in}} %
\pgfusepath{clip}%
\pgfsetbuttcap%
\pgfsetroundjoin%
\definecolor{currentfill}{rgb}{0.280267,0.073417,0.397163}%
\pgfsetfillcolor{currentfill}%
\pgfsetlinewidth{0.000000pt}%
\definecolor{currentstroke}{rgb}{0.000000,0.000000,0.000000}%
\pgfsetstrokecolor{currentstroke}%
\pgfsetdash{}{0pt}%
\pgfpathmoveto{\pgfqpoint{2.575484in}{0.908339in}}%
\pgfpathlineto{\pgfqpoint{2.290239in}{0.908339in}}%
\pgfpathlineto{\pgfqpoint{2.294675in}{0.899469in}}%
\pgfpathlineto{\pgfqpoint{2.250323in}{0.912774in}}%
\pgfpathlineto{\pgfqpoint{2.294675in}{0.926080in}}%
\pgfpathlineto{\pgfqpoint{2.290239in}{0.917209in}}%
\pgfpathlineto{\pgfqpoint{2.575484in}{0.917209in}}%
\pgfpathlineto{\pgfqpoint{2.575484in}{0.908339in}}%
\pgfusepath{fill}%
\end{pgfscope}%
\begin{pgfscope}%
\pgfpathrectangle{\pgfqpoint{1.432000in}{0.528000in}}{\pgfqpoint{3.696000in}{3.696000in}} %
\pgfusepath{clip}%
\pgfsetbuttcap%
\pgfsetroundjoin%
\definecolor{currentfill}{rgb}{0.212395,0.359683,0.551710}%
\pgfsetfillcolor{currentfill}%
\pgfsetlinewidth{0.000000pt}%
\definecolor{currentstroke}{rgb}{0.000000,0.000000,0.000000}%
\pgfsetstrokecolor{currentstroke}%
\pgfsetdash{}{0pt}%
\pgfpathmoveto{\pgfqpoint{2.683871in}{0.908339in}}%
\pgfpathlineto{\pgfqpoint{2.290239in}{0.908339in}}%
\pgfpathlineto{\pgfqpoint{2.294675in}{0.899469in}}%
\pgfpathlineto{\pgfqpoint{2.250323in}{0.912774in}}%
\pgfpathlineto{\pgfqpoint{2.294675in}{0.926080in}}%
\pgfpathlineto{\pgfqpoint{2.290239in}{0.917209in}}%
\pgfpathlineto{\pgfqpoint{2.683871in}{0.917209in}}%
\pgfpathlineto{\pgfqpoint{2.683871in}{0.908339in}}%
\pgfusepath{fill}%
\end{pgfscope}%
\begin{pgfscope}%
\pgfpathrectangle{\pgfqpoint{1.432000in}{0.528000in}}{\pgfqpoint{3.696000in}{3.696000in}} %
\pgfusepath{clip}%
\pgfsetbuttcap%
\pgfsetroundjoin%
\definecolor{currentfill}{rgb}{0.229739,0.322361,0.545706}%
\pgfsetfillcolor{currentfill}%
\pgfsetlinewidth{0.000000pt}%
\definecolor{currentstroke}{rgb}{0.000000,0.000000,0.000000}%
\pgfsetstrokecolor{currentstroke}%
\pgfsetdash{}{0pt}%
\pgfpathmoveto{\pgfqpoint{2.792258in}{0.908339in}}%
\pgfpathlineto{\pgfqpoint{2.398626in}{0.908339in}}%
\pgfpathlineto{\pgfqpoint{2.403062in}{0.899469in}}%
\pgfpathlineto{\pgfqpoint{2.358710in}{0.912774in}}%
\pgfpathlineto{\pgfqpoint{2.403062in}{0.926080in}}%
\pgfpathlineto{\pgfqpoint{2.398626in}{0.917209in}}%
\pgfpathlineto{\pgfqpoint{2.792258in}{0.917209in}}%
\pgfpathlineto{\pgfqpoint{2.792258in}{0.908339in}}%
\pgfusepath{fill}%
\end{pgfscope}%
\begin{pgfscope}%
\pgfpathrectangle{\pgfqpoint{1.432000in}{0.528000in}}{\pgfqpoint{3.696000in}{3.696000in}} %
\pgfusepath{clip}%
\pgfsetbuttcap%
\pgfsetroundjoin%
\definecolor{currentfill}{rgb}{0.280894,0.078907,0.402329}%
\pgfsetfillcolor{currentfill}%
\pgfsetlinewidth{0.000000pt}%
\definecolor{currentstroke}{rgb}{0.000000,0.000000,0.000000}%
\pgfsetstrokecolor{currentstroke}%
\pgfsetdash{}{0pt}%
\pgfpathmoveto{\pgfqpoint{2.899569in}{0.908471in}}%
\pgfpathlineto{\pgfqpoint{2.504746in}{1.007177in}}%
\pgfpathlineto{\pgfqpoint{2.506897in}{0.997496in}}%
\pgfpathlineto{\pgfqpoint{2.467097in}{1.021161in}}%
\pgfpathlineto{\pgfqpoint{2.513352in}{1.023313in}}%
\pgfpathlineto{\pgfqpoint{2.506897in}{1.015783in}}%
\pgfpathlineto{\pgfqpoint{2.901721in}{0.917077in}}%
\pgfpathlineto{\pgfqpoint{2.899569in}{0.908471in}}%
\pgfusepath{fill}%
\end{pgfscope}%
\begin{pgfscope}%
\pgfpathrectangle{\pgfqpoint{1.432000in}{0.528000in}}{\pgfqpoint{3.696000in}{3.696000in}} %
\pgfusepath{clip}%
\pgfsetbuttcap%
\pgfsetroundjoin%
\definecolor{currentfill}{rgb}{0.229739,0.322361,0.545706}%
\pgfsetfillcolor{currentfill}%
\pgfsetlinewidth{0.000000pt}%
\definecolor{currentstroke}{rgb}{0.000000,0.000000,0.000000}%
\pgfsetstrokecolor{currentstroke}%
\pgfsetdash{}{0pt}%
\pgfpathmoveto{\pgfqpoint{2.899243in}{0.908567in}}%
\pgfpathlineto{\pgfqpoint{2.611950in}{1.004331in}}%
\pgfpathlineto{\pgfqpoint{2.613352in}{0.994513in}}%
\pgfpathlineto{\pgfqpoint{2.575484in}{1.021161in}}%
\pgfpathlineto{\pgfqpoint{2.621767in}{1.019759in}}%
\pgfpathlineto{\pgfqpoint{2.614755in}{1.012746in}}%
\pgfpathlineto{\pgfqpoint{2.902048in}{0.916982in}}%
\pgfpathlineto{\pgfqpoint{2.899243in}{0.908567in}}%
\pgfusepath{fill}%
\end{pgfscope}%
\begin{pgfscope}%
\pgfpathrectangle{\pgfqpoint{1.432000in}{0.528000in}}{\pgfqpoint{3.696000in}{3.696000in}} %
\pgfusepath{clip}%
\pgfsetbuttcap%
\pgfsetroundjoin%
\definecolor{currentfill}{rgb}{0.203063,0.379716,0.553925}%
\pgfsetfillcolor{currentfill}%
\pgfsetlinewidth{0.000000pt}%
\definecolor{currentstroke}{rgb}{0.000000,0.000000,0.000000}%
\pgfsetstrokecolor{currentstroke}%
\pgfsetdash{}{0pt}%
\pgfpathmoveto{\pgfqpoint{3.007630in}{0.908567in}}%
\pgfpathlineto{\pgfqpoint{2.720337in}{1.004331in}}%
\pgfpathlineto{\pgfqpoint{2.721739in}{0.994513in}}%
\pgfpathlineto{\pgfqpoint{2.683871in}{1.021161in}}%
\pgfpathlineto{\pgfqpoint{2.730155in}{1.019759in}}%
\pgfpathlineto{\pgfqpoint{2.723142in}{1.012746in}}%
\pgfpathlineto{\pgfqpoint{3.010435in}{0.916982in}}%
\pgfpathlineto{\pgfqpoint{3.007630in}{0.908567in}}%
\pgfusepath{fill}%
\end{pgfscope}%
\begin{pgfscope}%
\pgfpathrectangle{\pgfqpoint{1.432000in}{0.528000in}}{\pgfqpoint{3.696000in}{3.696000in}} %
\pgfusepath{clip}%
\pgfsetbuttcap%
\pgfsetroundjoin%
\definecolor{currentfill}{rgb}{0.231674,0.318106,0.544834}%
\pgfsetfillcolor{currentfill}%
\pgfsetlinewidth{0.000000pt}%
\definecolor{currentstroke}{rgb}{0.000000,0.000000,0.000000}%
\pgfsetstrokecolor{currentstroke}%
\pgfsetdash{}{0pt}%
\pgfpathmoveto{\pgfqpoint{3.116017in}{0.908567in}}%
\pgfpathlineto{\pgfqpoint{2.828724in}{1.004331in}}%
\pgfpathlineto{\pgfqpoint{2.830126in}{0.994513in}}%
\pgfpathlineto{\pgfqpoint{2.792258in}{1.021161in}}%
\pgfpathlineto{\pgfqpoint{2.838542in}{1.019759in}}%
\pgfpathlineto{\pgfqpoint{2.831529in}{1.012746in}}%
\pgfpathlineto{\pgfqpoint{3.118822in}{0.916982in}}%
\pgfpathlineto{\pgfqpoint{3.116017in}{0.908567in}}%
\pgfusepath{fill}%
\end{pgfscope}%
\begin{pgfscope}%
\pgfpathrectangle{\pgfqpoint{1.432000in}{0.528000in}}{\pgfqpoint{3.696000in}{3.696000in}} %
\pgfusepath{clip}%
\pgfsetbuttcap%
\pgfsetroundjoin%
\definecolor{currentfill}{rgb}{0.241237,0.296485,0.539709}%
\pgfsetfillcolor{currentfill}%
\pgfsetlinewidth{0.000000pt}%
\definecolor{currentstroke}{rgb}{0.000000,0.000000,0.000000}%
\pgfsetstrokecolor{currentstroke}%
\pgfsetdash{}{0pt}%
\pgfpathmoveto{\pgfqpoint{3.115436in}{0.908807in}}%
\pgfpathlineto{\pgfqpoint{2.934364in}{0.999343in}}%
\pgfpathlineto{\pgfqpoint{2.934364in}{0.989426in}}%
\pgfpathlineto{\pgfqpoint{2.900645in}{1.021161in}}%
\pgfpathlineto{\pgfqpoint{2.946265in}{1.013227in}}%
\pgfpathlineto{\pgfqpoint{2.938331in}{1.007277in}}%
\pgfpathlineto{\pgfqpoint{3.119403in}{0.916741in}}%
\pgfpathlineto{\pgfqpoint{3.115436in}{0.908807in}}%
\pgfusepath{fill}%
\end{pgfscope}%
\begin{pgfscope}%
\pgfpathrectangle{\pgfqpoint{1.432000in}{0.528000in}}{\pgfqpoint{3.696000in}{3.696000in}} %
\pgfusepath{clip}%
\pgfsetbuttcap%
\pgfsetroundjoin%
\definecolor{currentfill}{rgb}{0.263663,0.237631,0.518762}%
\pgfsetfillcolor{currentfill}%
\pgfsetlinewidth{0.000000pt}%
\definecolor{currentstroke}{rgb}{0.000000,0.000000,0.000000}%
\pgfsetstrokecolor{currentstroke}%
\pgfsetdash{}{0pt}%
\pgfpathmoveto{\pgfqpoint{3.223823in}{0.908807in}}%
\pgfpathlineto{\pgfqpoint{3.042751in}{0.999343in}}%
\pgfpathlineto{\pgfqpoint{3.042751in}{0.989426in}}%
\pgfpathlineto{\pgfqpoint{3.009032in}{1.021161in}}%
\pgfpathlineto{\pgfqpoint{3.054652in}{1.013227in}}%
\pgfpathlineto{\pgfqpoint{3.046718in}{1.007277in}}%
\pgfpathlineto{\pgfqpoint{3.227790in}{0.916741in}}%
\pgfpathlineto{\pgfqpoint{3.223823in}{0.908807in}}%
\pgfusepath{fill}%
\end{pgfscope}%
\begin{pgfscope}%
\pgfpathrectangle{\pgfqpoint{1.432000in}{0.528000in}}{\pgfqpoint{3.696000in}{3.696000in}} %
\pgfusepath{clip}%
\pgfsetbuttcap%
\pgfsetroundjoin%
\definecolor{currentfill}{rgb}{0.266580,0.228262,0.514349}%
\pgfsetfillcolor{currentfill}%
\pgfsetlinewidth{0.000000pt}%
\definecolor{currentstroke}{rgb}{0.000000,0.000000,0.000000}%
\pgfsetstrokecolor{currentstroke}%
\pgfsetdash{}{0pt}%
\pgfpathmoveto{\pgfqpoint{3.222670in}{0.909638in}}%
\pgfpathlineto{\pgfqpoint{3.142509in}{0.989800in}}%
\pgfpathlineto{\pgfqpoint{3.139372in}{0.980391in}}%
\pgfpathlineto{\pgfqpoint{3.117419in}{1.021161in}}%
\pgfpathlineto{\pgfqpoint{3.158189in}{0.999208in}}%
\pgfpathlineto{\pgfqpoint{3.148781in}{0.996072in}}%
\pgfpathlineto{\pgfqpoint{3.228943in}{0.915910in}}%
\pgfpathlineto{\pgfqpoint{3.222670in}{0.909638in}}%
\pgfusepath{fill}%
\end{pgfscope}%
\begin{pgfscope}%
\pgfpathrectangle{\pgfqpoint{1.432000in}{0.528000in}}{\pgfqpoint{3.696000in}{3.696000in}} %
\pgfusepath{clip}%
\pgfsetbuttcap%
\pgfsetroundjoin%
\definecolor{currentfill}{rgb}{0.282327,0.094955,0.417331}%
\pgfsetfillcolor{currentfill}%
\pgfsetlinewidth{0.000000pt}%
\definecolor{currentstroke}{rgb}{0.000000,0.000000,0.000000}%
\pgfsetstrokecolor{currentstroke}%
\pgfsetdash{}{0pt}%
\pgfpathmoveto{\pgfqpoint{3.334194in}{0.908339in}}%
\pgfpathlineto{\pgfqpoint{3.265723in}{0.908339in}}%
\pgfpathlineto{\pgfqpoint{3.270158in}{0.899469in}}%
\pgfpathlineto{\pgfqpoint{3.225806in}{0.912774in}}%
\pgfpathlineto{\pgfqpoint{3.270158in}{0.926080in}}%
\pgfpathlineto{\pgfqpoint{3.265723in}{0.917209in}}%
\pgfpathlineto{\pgfqpoint{3.334194in}{0.917209in}}%
\pgfpathlineto{\pgfqpoint{3.334194in}{0.908339in}}%
\pgfusepath{fill}%
\end{pgfscope}%
\begin{pgfscope}%
\pgfpathrectangle{\pgfqpoint{1.432000in}{0.528000in}}{\pgfqpoint{3.696000in}{3.696000in}} %
\pgfusepath{clip}%
\pgfsetbuttcap%
\pgfsetroundjoin%
\definecolor{currentfill}{rgb}{0.277018,0.050344,0.375715}%
\pgfsetfillcolor{currentfill}%
\pgfsetlinewidth{0.000000pt}%
\definecolor{currentstroke}{rgb}{0.000000,0.000000,0.000000}%
\pgfsetstrokecolor{currentstroke}%
\pgfsetdash{}{0pt}%
\pgfpathmoveto{\pgfqpoint{3.332210in}{0.908807in}}%
\pgfpathlineto{\pgfqpoint{3.151139in}{0.999343in}}%
\pgfpathlineto{\pgfqpoint{3.151139in}{0.989426in}}%
\pgfpathlineto{\pgfqpoint{3.117419in}{1.021161in}}%
\pgfpathlineto{\pgfqpoint{3.163039in}{1.013227in}}%
\pgfpathlineto{\pgfqpoint{3.155106in}{1.007277in}}%
\pgfpathlineto{\pgfqpoint{3.336177in}{0.916741in}}%
\pgfpathlineto{\pgfqpoint{3.332210in}{0.908807in}}%
\pgfusepath{fill}%
\end{pgfscope}%
\begin{pgfscope}%
\pgfpathrectangle{\pgfqpoint{1.432000in}{0.528000in}}{\pgfqpoint{3.696000in}{3.696000in}} %
\pgfusepath{clip}%
\pgfsetbuttcap%
\pgfsetroundjoin%
\definecolor{currentfill}{rgb}{0.239346,0.300855,0.540844}%
\pgfsetfillcolor{currentfill}%
\pgfsetlinewidth{0.000000pt}%
\definecolor{currentstroke}{rgb}{0.000000,0.000000,0.000000}%
\pgfsetstrokecolor{currentstroke}%
\pgfsetdash{}{0pt}%
\pgfpathmoveto{\pgfqpoint{3.331057in}{0.909638in}}%
\pgfpathlineto{\pgfqpoint{3.250896in}{0.989800in}}%
\pgfpathlineto{\pgfqpoint{3.247760in}{0.980391in}}%
\pgfpathlineto{\pgfqpoint{3.225806in}{1.021161in}}%
\pgfpathlineto{\pgfqpoint{3.266577in}{0.999208in}}%
\pgfpathlineto{\pgfqpoint{3.257168in}{0.996072in}}%
\pgfpathlineto{\pgfqpoint{3.337330in}{0.915910in}}%
\pgfpathlineto{\pgfqpoint{3.331057in}{0.909638in}}%
\pgfusepath{fill}%
\end{pgfscope}%
\begin{pgfscope}%
\pgfpathrectangle{\pgfqpoint{1.432000in}{0.528000in}}{\pgfqpoint{3.696000in}{3.696000in}} %
\pgfusepath{clip}%
\pgfsetbuttcap%
\pgfsetroundjoin%
\definecolor{currentfill}{rgb}{0.271828,0.209303,0.504434}%
\pgfsetfillcolor{currentfill}%
\pgfsetlinewidth{0.000000pt}%
\definecolor{currentstroke}{rgb}{0.000000,0.000000,0.000000}%
\pgfsetstrokecolor{currentstroke}%
\pgfsetdash{}{0pt}%
\pgfpathmoveto{\pgfqpoint{3.442581in}{0.908339in}}%
\pgfpathlineto{\pgfqpoint{3.374110in}{0.908339in}}%
\pgfpathlineto{\pgfqpoint{3.378546in}{0.899469in}}%
\pgfpathlineto{\pgfqpoint{3.334194in}{0.912774in}}%
\pgfpathlineto{\pgfqpoint{3.378546in}{0.926080in}}%
\pgfpathlineto{\pgfqpoint{3.374110in}{0.917209in}}%
\pgfpathlineto{\pgfqpoint{3.442581in}{0.917209in}}%
\pgfpathlineto{\pgfqpoint{3.442581in}{0.908339in}}%
\pgfusepath{fill}%
\end{pgfscope}%
\begin{pgfscope}%
\pgfpathrectangle{\pgfqpoint{1.432000in}{0.528000in}}{\pgfqpoint{3.696000in}{3.696000in}} %
\pgfusepath{clip}%
\pgfsetbuttcap%
\pgfsetroundjoin%
\definecolor{currentfill}{rgb}{0.195860,0.395433,0.555276}%
\pgfsetfillcolor{currentfill}%
\pgfsetlinewidth{0.000000pt}%
\definecolor{currentstroke}{rgb}{0.000000,0.000000,0.000000}%
\pgfsetstrokecolor{currentstroke}%
\pgfsetdash{}{0pt}%
\pgfpathmoveto{\pgfqpoint{3.439444in}{0.909638in}}%
\pgfpathlineto{\pgfqpoint{3.359283in}{0.989800in}}%
\pgfpathlineto{\pgfqpoint{3.356147in}{0.980391in}}%
\pgfpathlineto{\pgfqpoint{3.334194in}{1.021161in}}%
\pgfpathlineto{\pgfqpoint{3.374964in}{0.999208in}}%
\pgfpathlineto{\pgfqpoint{3.365555in}{0.996072in}}%
\pgfpathlineto{\pgfqpoint{3.445717in}{0.915910in}}%
\pgfpathlineto{\pgfqpoint{3.439444in}{0.909638in}}%
\pgfusepath{fill}%
\end{pgfscope}%
\begin{pgfscope}%
\pgfpathrectangle{\pgfqpoint{1.432000in}{0.528000in}}{\pgfqpoint{3.696000in}{3.696000in}} %
\pgfusepath{clip}%
\pgfsetbuttcap%
\pgfsetroundjoin%
\definecolor{currentfill}{rgb}{0.174274,0.445044,0.557792}%
\pgfsetfillcolor{currentfill}%
\pgfsetlinewidth{0.000000pt}%
\definecolor{currentstroke}{rgb}{0.000000,0.000000,0.000000}%
\pgfsetstrokecolor{currentstroke}%
\pgfsetdash{}{0pt}%
\pgfpathmoveto{\pgfqpoint{3.550968in}{0.908339in}}%
\pgfpathlineto{\pgfqpoint{3.482497in}{0.908339in}}%
\pgfpathlineto{\pgfqpoint{3.486933in}{0.899469in}}%
\pgfpathlineto{\pgfqpoint{3.442581in}{0.912774in}}%
\pgfpathlineto{\pgfqpoint{3.486933in}{0.926080in}}%
\pgfpathlineto{\pgfqpoint{3.482497in}{0.917209in}}%
\pgfpathlineto{\pgfqpoint{3.550968in}{0.917209in}}%
\pgfpathlineto{\pgfqpoint{3.550968in}{0.908339in}}%
\pgfusepath{fill}%
\end{pgfscope}%
\begin{pgfscope}%
\pgfpathrectangle{\pgfqpoint{1.432000in}{0.528000in}}{\pgfqpoint{3.696000in}{3.696000in}} %
\pgfusepath{clip}%
\pgfsetbuttcap%
\pgfsetroundjoin%
\definecolor{currentfill}{rgb}{0.274128,0.199721,0.498911}%
\pgfsetfillcolor{currentfill}%
\pgfsetlinewidth{0.000000pt}%
\definecolor{currentstroke}{rgb}{0.000000,0.000000,0.000000}%
\pgfsetstrokecolor{currentstroke}%
\pgfsetdash{}{0pt}%
\pgfpathmoveto{\pgfqpoint{3.547832in}{0.909638in}}%
\pgfpathlineto{\pgfqpoint{3.467670in}{0.989800in}}%
\pgfpathlineto{\pgfqpoint{3.464534in}{0.980391in}}%
\pgfpathlineto{\pgfqpoint{3.442581in}{1.021161in}}%
\pgfpathlineto{\pgfqpoint{3.483351in}{0.999208in}}%
\pgfpathlineto{\pgfqpoint{3.473942in}{0.996072in}}%
\pgfpathlineto{\pgfqpoint{3.554104in}{0.915910in}}%
\pgfpathlineto{\pgfqpoint{3.547832in}{0.909638in}}%
\pgfusepath{fill}%
\end{pgfscope}%
\begin{pgfscope}%
\pgfpathrectangle{\pgfqpoint{1.432000in}{0.528000in}}{\pgfqpoint{3.696000in}{3.696000in}} %
\pgfusepath{clip}%
\pgfsetbuttcap%
\pgfsetroundjoin%
\definecolor{currentfill}{rgb}{0.132444,0.552216,0.553018}%
\pgfsetfillcolor{currentfill}%
\pgfsetlinewidth{0.000000pt}%
\definecolor{currentstroke}{rgb}{0.000000,0.000000,0.000000}%
\pgfsetstrokecolor{currentstroke}%
\pgfsetdash{}{0pt}%
\pgfpathmoveto{\pgfqpoint{3.659355in}{0.908339in}}%
\pgfpathlineto{\pgfqpoint{3.590885in}{0.908339in}}%
\pgfpathlineto{\pgfqpoint{3.595320in}{0.899469in}}%
\pgfpathlineto{\pgfqpoint{3.550968in}{0.912774in}}%
\pgfpathlineto{\pgfqpoint{3.595320in}{0.926080in}}%
\pgfpathlineto{\pgfqpoint{3.590885in}{0.917209in}}%
\pgfpathlineto{\pgfqpoint{3.659355in}{0.917209in}}%
\pgfpathlineto{\pgfqpoint{3.659355in}{0.908339in}}%
\pgfusepath{fill}%
\end{pgfscope}%
\begin{pgfscope}%
\pgfpathrectangle{\pgfqpoint{1.432000in}{0.528000in}}{\pgfqpoint{3.696000in}{3.696000in}} %
\pgfusepath{clip}%
\pgfsetbuttcap%
\pgfsetroundjoin%
\definecolor{currentfill}{rgb}{0.216210,0.351535,0.550627}%
\pgfsetfillcolor{currentfill}%
\pgfsetlinewidth{0.000000pt}%
\definecolor{currentstroke}{rgb}{0.000000,0.000000,0.000000}%
\pgfsetstrokecolor{currentstroke}%
\pgfsetdash{}{0pt}%
\pgfpathmoveto{\pgfqpoint{3.767742in}{0.908339in}}%
\pgfpathlineto{\pgfqpoint{3.699272in}{0.908339in}}%
\pgfpathlineto{\pgfqpoint{3.703707in}{0.899469in}}%
\pgfpathlineto{\pgfqpoint{3.659355in}{0.912774in}}%
\pgfpathlineto{\pgfqpoint{3.703707in}{0.926080in}}%
\pgfpathlineto{\pgfqpoint{3.699272in}{0.917209in}}%
\pgfpathlineto{\pgfqpoint{3.767742in}{0.917209in}}%
\pgfpathlineto{\pgfqpoint{3.767742in}{0.908339in}}%
\pgfusepath{fill}%
\end{pgfscope}%
\begin{pgfscope}%
\pgfpathrectangle{\pgfqpoint{1.432000in}{0.528000in}}{\pgfqpoint{3.696000in}{3.696000in}} %
\pgfusepath{clip}%
\pgfsetbuttcap%
\pgfsetroundjoin%
\definecolor{currentfill}{rgb}{0.257322,0.256130,0.526563}%
\pgfsetfillcolor{currentfill}%
\pgfsetlinewidth{0.000000pt}%
\definecolor{currentstroke}{rgb}{0.000000,0.000000,0.000000}%
\pgfsetstrokecolor{currentstroke}%
\pgfsetdash{}{0pt}%
\pgfpathmoveto{\pgfqpoint{3.876129in}{0.908339in}}%
\pgfpathlineto{\pgfqpoint{3.699272in}{0.908339in}}%
\pgfpathlineto{\pgfqpoint{3.703707in}{0.899469in}}%
\pgfpathlineto{\pgfqpoint{3.659355in}{0.912774in}}%
\pgfpathlineto{\pgfqpoint{3.703707in}{0.926080in}}%
\pgfpathlineto{\pgfqpoint{3.699272in}{0.917209in}}%
\pgfpathlineto{\pgfqpoint{3.876129in}{0.917209in}}%
\pgfpathlineto{\pgfqpoint{3.876129in}{0.908339in}}%
\pgfusepath{fill}%
\end{pgfscope}%
\begin{pgfscope}%
\pgfpathrectangle{\pgfqpoint{1.432000in}{0.528000in}}{\pgfqpoint{3.696000in}{3.696000in}} %
\pgfusepath{clip}%
\pgfsetbuttcap%
\pgfsetroundjoin%
\definecolor{currentfill}{rgb}{0.276022,0.044167,0.370164}%
\pgfsetfillcolor{currentfill}%
\pgfsetlinewidth{0.000000pt}%
\definecolor{currentstroke}{rgb}{0.000000,0.000000,0.000000}%
\pgfsetstrokecolor{currentstroke}%
\pgfsetdash{}{0pt}%
\pgfpathmoveto{\pgfqpoint{3.986500in}{0.908807in}}%
\pgfpathlineto{\pgfqpoint{3.805428in}{0.818271in}}%
\pgfpathlineto{\pgfqpoint{3.813362in}{0.812321in}}%
\pgfpathlineto{\pgfqpoint{3.767742in}{0.804387in}}%
\pgfpathlineto{\pgfqpoint{3.801461in}{0.836123in}}%
\pgfpathlineto{\pgfqpoint{3.801461in}{0.826205in}}%
\pgfpathlineto{\pgfqpoint{3.982533in}{0.916741in}}%
\pgfpathlineto{\pgfqpoint{3.986500in}{0.908807in}}%
\pgfusepath{fill}%
\end{pgfscope}%
\begin{pgfscope}%
\pgfpathrectangle{\pgfqpoint{1.432000in}{0.528000in}}{\pgfqpoint{3.696000in}{3.696000in}} %
\pgfusepath{clip}%
\pgfsetbuttcap%
\pgfsetroundjoin%
\definecolor{currentfill}{rgb}{0.276194,0.190074,0.493001}%
\pgfsetfillcolor{currentfill}%
\pgfsetlinewidth{0.000000pt}%
\definecolor{currentstroke}{rgb}{0.000000,0.000000,0.000000}%
\pgfsetstrokecolor{currentstroke}%
\pgfsetdash{}{0pt}%
\pgfpathmoveto{\pgfqpoint{3.984516in}{0.908339in}}%
\pgfpathlineto{\pgfqpoint{3.807659in}{0.908339in}}%
\pgfpathlineto{\pgfqpoint{3.812094in}{0.899469in}}%
\pgfpathlineto{\pgfqpoint{3.767742in}{0.912774in}}%
\pgfpathlineto{\pgfqpoint{3.812094in}{0.926080in}}%
\pgfpathlineto{\pgfqpoint{3.807659in}{0.917209in}}%
\pgfpathlineto{\pgfqpoint{3.984516in}{0.917209in}}%
\pgfpathlineto{\pgfqpoint{3.984516in}{0.908339in}}%
\pgfusepath{fill}%
\end{pgfscope}%
\begin{pgfscope}%
\pgfpathrectangle{\pgfqpoint{1.432000in}{0.528000in}}{\pgfqpoint{3.696000in}{3.696000in}} %
\pgfusepath{clip}%
\pgfsetbuttcap%
\pgfsetroundjoin%
\definecolor{currentfill}{rgb}{0.280255,0.165693,0.476498}%
\pgfsetfillcolor{currentfill}%
\pgfsetlinewidth{0.000000pt}%
\definecolor{currentstroke}{rgb}{0.000000,0.000000,0.000000}%
\pgfsetstrokecolor{currentstroke}%
\pgfsetdash{}{0pt}%
\pgfpathmoveto{\pgfqpoint{4.202693in}{0.908567in}}%
\pgfpathlineto{\pgfqpoint{3.915400in}{0.812802in}}%
\pgfpathlineto{\pgfqpoint{3.922413in}{0.805790in}}%
\pgfpathlineto{\pgfqpoint{3.876129in}{0.804387in}}%
\pgfpathlineto{\pgfqpoint{3.913997in}{0.831035in}}%
\pgfpathlineto{\pgfqpoint{3.912595in}{0.821217in}}%
\pgfpathlineto{\pgfqpoint{4.199888in}{0.916982in}}%
\pgfpathlineto{\pgfqpoint{4.202693in}{0.908567in}}%
\pgfusepath{fill}%
\end{pgfscope}%
\begin{pgfscope}%
\pgfpathrectangle{\pgfqpoint{1.432000in}{0.528000in}}{\pgfqpoint{3.696000in}{3.696000in}} %
\pgfusepath{clip}%
\pgfsetbuttcap%
\pgfsetroundjoin%
\definecolor{currentfill}{rgb}{0.269308,0.218818,0.509577}%
\pgfsetfillcolor{currentfill}%
\pgfsetlinewidth{0.000000pt}%
\definecolor{currentstroke}{rgb}{0.000000,0.000000,0.000000}%
\pgfsetstrokecolor{currentstroke}%
\pgfsetdash{}{0pt}%
\pgfpathmoveto{\pgfqpoint{4.311080in}{0.908567in}}%
\pgfpathlineto{\pgfqpoint{4.023787in}{0.812802in}}%
\pgfpathlineto{\pgfqpoint{4.030800in}{0.805790in}}%
\pgfpathlineto{\pgfqpoint{3.984516in}{0.804387in}}%
\pgfpathlineto{\pgfqpoint{4.022385in}{0.831035in}}%
\pgfpathlineto{\pgfqpoint{4.020982in}{0.821217in}}%
\pgfpathlineto{\pgfqpoint{4.308275in}{0.916982in}}%
\pgfpathlineto{\pgfqpoint{4.311080in}{0.908567in}}%
\pgfusepath{fill}%
\end{pgfscope}%
\begin{pgfscope}%
\pgfpathrectangle{\pgfqpoint{1.432000in}{0.528000in}}{\pgfqpoint{3.696000in}{3.696000in}} %
\pgfusepath{clip}%
\pgfsetbuttcap%
\pgfsetroundjoin%
\definecolor{currentfill}{rgb}{0.248629,0.278775,0.534556}%
\pgfsetfillcolor{currentfill}%
\pgfsetlinewidth{0.000000pt}%
\definecolor{currentstroke}{rgb}{0.000000,0.000000,0.000000}%
\pgfsetstrokecolor{currentstroke}%
\pgfsetdash{}{0pt}%
\pgfpathmoveto{\pgfqpoint{4.419467in}{0.908567in}}%
\pgfpathlineto{\pgfqpoint{4.132174in}{0.812802in}}%
\pgfpathlineto{\pgfqpoint{4.139187in}{0.805790in}}%
\pgfpathlineto{\pgfqpoint{4.092903in}{0.804387in}}%
\pgfpathlineto{\pgfqpoint{4.130772in}{0.831035in}}%
\pgfpathlineto{\pgfqpoint{4.129369in}{0.821217in}}%
\pgfpathlineto{\pgfqpoint{4.416662in}{0.916982in}}%
\pgfpathlineto{\pgfqpoint{4.419467in}{0.908567in}}%
\pgfusepath{fill}%
\end{pgfscope}%
\begin{pgfscope}%
\pgfpathrectangle{\pgfqpoint{1.432000in}{0.528000in}}{\pgfqpoint{3.696000in}{3.696000in}} %
\pgfusepath{clip}%
\pgfsetbuttcap%
\pgfsetroundjoin%
\definecolor{currentfill}{rgb}{0.274128,0.199721,0.498911}%
\pgfsetfillcolor{currentfill}%
\pgfsetlinewidth{0.000000pt}%
\definecolor{currentstroke}{rgb}{0.000000,0.000000,0.000000}%
\pgfsetstrokecolor{currentstroke}%
\pgfsetdash{}{0pt}%
\pgfpathmoveto{\pgfqpoint{4.527854in}{0.908567in}}%
\pgfpathlineto{\pgfqpoint{4.240561in}{0.812802in}}%
\pgfpathlineto{\pgfqpoint{4.247574in}{0.805790in}}%
\pgfpathlineto{\pgfqpoint{4.201290in}{0.804387in}}%
\pgfpathlineto{\pgfqpoint{4.239159in}{0.831035in}}%
\pgfpathlineto{\pgfqpoint{4.237756in}{0.821217in}}%
\pgfpathlineto{\pgfqpoint{4.525049in}{0.916982in}}%
\pgfpathlineto{\pgfqpoint{4.527854in}{0.908567in}}%
\pgfusepath{fill}%
\end{pgfscope}%
\begin{pgfscope}%
\pgfpathrectangle{\pgfqpoint{1.432000in}{0.528000in}}{\pgfqpoint{3.696000in}{3.696000in}} %
\pgfusepath{clip}%
\pgfsetbuttcap%
\pgfsetroundjoin%
\definecolor{currentfill}{rgb}{0.268510,0.009605,0.335427}%
\pgfsetfillcolor{currentfill}%
\pgfsetlinewidth{0.000000pt}%
\definecolor{currentstroke}{rgb}{0.000000,0.000000,0.000000}%
\pgfsetstrokecolor{currentstroke}%
\pgfsetdash{}{0pt}%
\pgfpathmoveto{\pgfqpoint{4.528435in}{0.908807in}}%
\pgfpathlineto{\pgfqpoint{4.347364in}{0.818271in}}%
\pgfpathlineto{\pgfqpoint{4.355297in}{0.812321in}}%
\pgfpathlineto{\pgfqpoint{4.309677in}{0.804387in}}%
\pgfpathlineto{\pgfqpoint{4.343397in}{0.836123in}}%
\pgfpathlineto{\pgfqpoint{4.343397in}{0.826205in}}%
\pgfpathlineto{\pgfqpoint{4.524468in}{0.916741in}}%
\pgfpathlineto{\pgfqpoint{4.528435in}{0.908807in}}%
\pgfusepath{fill}%
\end{pgfscope}%
\begin{pgfscope}%
\pgfpathrectangle{\pgfqpoint{1.432000in}{0.528000in}}{\pgfqpoint{3.696000in}{3.696000in}} %
\pgfusepath{clip}%
\pgfsetbuttcap%
\pgfsetroundjoin%
\definecolor{currentfill}{rgb}{0.272594,0.025563,0.353093}%
\pgfsetfillcolor{currentfill}%
\pgfsetlinewidth{0.000000pt}%
\definecolor{currentstroke}{rgb}{0.000000,0.000000,0.000000}%
\pgfsetstrokecolor{currentstroke}%
\pgfsetdash{}{0pt}%
\pgfpathmoveto{\pgfqpoint{4.636241in}{0.908567in}}%
\pgfpathlineto{\pgfqpoint{4.348948in}{0.812802in}}%
\pgfpathlineto{\pgfqpoint{4.355961in}{0.805790in}}%
\pgfpathlineto{\pgfqpoint{4.309677in}{0.804387in}}%
\pgfpathlineto{\pgfqpoint{4.347546in}{0.831035in}}%
\pgfpathlineto{\pgfqpoint{4.346143in}{0.821217in}}%
\pgfpathlineto{\pgfqpoint{4.633436in}{0.916982in}}%
\pgfpathlineto{\pgfqpoint{4.636241in}{0.908567in}}%
\pgfusepath{fill}%
\end{pgfscope}%
\begin{pgfscope}%
\pgfpathrectangle{\pgfqpoint{1.432000in}{0.528000in}}{\pgfqpoint{3.696000in}{3.696000in}} %
\pgfusepath{clip}%
\pgfsetbuttcap%
\pgfsetroundjoin%
\definecolor{currentfill}{rgb}{0.275191,0.194905,0.496005}%
\pgfsetfillcolor{currentfill}%
\pgfsetlinewidth{0.000000pt}%
\definecolor{currentstroke}{rgb}{0.000000,0.000000,0.000000}%
\pgfsetstrokecolor{currentstroke}%
\pgfsetdash{}{0pt}%
\pgfpathmoveto{\pgfqpoint{4.636822in}{0.908807in}}%
\pgfpathlineto{\pgfqpoint{4.455751in}{0.818271in}}%
\pgfpathlineto{\pgfqpoint{4.463685in}{0.812321in}}%
\pgfpathlineto{\pgfqpoint{4.418065in}{0.804387in}}%
\pgfpathlineto{\pgfqpoint{4.451784in}{0.836123in}}%
\pgfpathlineto{\pgfqpoint{4.451784in}{0.826205in}}%
\pgfpathlineto{\pgfqpoint{4.632855in}{0.916741in}}%
\pgfpathlineto{\pgfqpoint{4.636822in}{0.908807in}}%
\pgfusepath{fill}%
\end{pgfscope}%
\begin{pgfscope}%
\pgfpathrectangle{\pgfqpoint{1.432000in}{0.528000in}}{\pgfqpoint{3.696000in}{3.696000in}} %
\pgfusepath{clip}%
\pgfsetbuttcap%
\pgfsetroundjoin%
\definecolor{currentfill}{rgb}{0.270595,0.214069,0.507052}%
\pgfsetfillcolor{currentfill}%
\pgfsetlinewidth{0.000000pt}%
\definecolor{currentstroke}{rgb}{0.000000,0.000000,0.000000}%
\pgfsetstrokecolor{currentstroke}%
\pgfsetdash{}{0pt}%
\pgfpathmoveto{\pgfqpoint{4.745209in}{0.908807in}}%
\pgfpathlineto{\pgfqpoint{4.564138in}{0.818271in}}%
\pgfpathlineto{\pgfqpoint{4.572072in}{0.812321in}}%
\pgfpathlineto{\pgfqpoint{4.526452in}{0.804387in}}%
\pgfpathlineto{\pgfqpoint{4.560171in}{0.836123in}}%
\pgfpathlineto{\pgfqpoint{4.560171in}{0.826205in}}%
\pgfpathlineto{\pgfqpoint{4.741242in}{0.916741in}}%
\pgfpathlineto{\pgfqpoint{4.745209in}{0.908807in}}%
\pgfusepath{fill}%
\end{pgfscope}%
\begin{pgfscope}%
\pgfpathrectangle{\pgfqpoint{1.432000in}{0.528000in}}{\pgfqpoint{3.696000in}{3.696000in}} %
\pgfusepath{clip}%
\pgfsetbuttcap%
\pgfsetroundjoin%
\definecolor{currentfill}{rgb}{0.277018,0.050344,0.375715}%
\pgfsetfillcolor{currentfill}%
\pgfsetlinewidth{0.000000pt}%
\definecolor{currentstroke}{rgb}{0.000000,0.000000,0.000000}%
\pgfsetstrokecolor{currentstroke}%
\pgfsetdash{}{0pt}%
\pgfpathmoveto{\pgfqpoint{4.743226in}{0.908339in}}%
\pgfpathlineto{\pgfqpoint{4.566368in}{0.908339in}}%
\pgfpathlineto{\pgfqpoint{4.570804in}{0.899469in}}%
\pgfpathlineto{\pgfqpoint{4.526452in}{0.912774in}}%
\pgfpathlineto{\pgfqpoint{4.570804in}{0.926080in}}%
\pgfpathlineto{\pgfqpoint{4.566368in}{0.917209in}}%
\pgfpathlineto{\pgfqpoint{4.743226in}{0.917209in}}%
\pgfpathlineto{\pgfqpoint{4.743226in}{0.908339in}}%
\pgfusepath{fill}%
\end{pgfscope}%
\begin{pgfscope}%
\pgfpathrectangle{\pgfqpoint{1.432000in}{0.528000in}}{\pgfqpoint{3.696000in}{3.696000in}} %
\pgfusepath{clip}%
\pgfsetbuttcap%
\pgfsetroundjoin%
\definecolor{currentfill}{rgb}{0.277018,0.050344,0.375715}%
\pgfsetfillcolor{currentfill}%
\pgfsetlinewidth{0.000000pt}%
\definecolor{currentstroke}{rgb}{0.000000,0.000000,0.000000}%
\pgfsetstrokecolor{currentstroke}%
\pgfsetdash{}{0pt}%
\pgfpathmoveto{\pgfqpoint{4.854749in}{0.909638in}}%
\pgfpathlineto{\pgfqpoint{4.774587in}{0.829476in}}%
\pgfpathlineto{\pgfqpoint{4.783996in}{0.826340in}}%
\pgfpathlineto{\pgfqpoint{4.743226in}{0.804387in}}%
\pgfpathlineto{\pgfqpoint{4.765179in}{0.845157in}}%
\pgfpathlineto{\pgfqpoint{4.768315in}{0.835749in}}%
\pgfpathlineto{\pgfqpoint{4.848477in}{0.915910in}}%
\pgfpathlineto{\pgfqpoint{4.854749in}{0.909638in}}%
\pgfusepath{fill}%
\end{pgfscope}%
\begin{pgfscope}%
\pgfpathrectangle{\pgfqpoint{1.432000in}{0.528000in}}{\pgfqpoint{3.696000in}{3.696000in}} %
\pgfusepath{clip}%
\pgfsetbuttcap%
\pgfsetroundjoin%
\definecolor{currentfill}{rgb}{0.271828,0.209303,0.504434}%
\pgfsetfillcolor{currentfill}%
\pgfsetlinewidth{0.000000pt}%
\definecolor{currentstroke}{rgb}{0.000000,0.000000,0.000000}%
\pgfsetstrokecolor{currentstroke}%
\pgfsetdash{}{0pt}%
\pgfpathmoveto{\pgfqpoint{4.851613in}{0.908339in}}%
\pgfpathlineto{\pgfqpoint{4.783143in}{0.908339in}}%
\pgfpathlineto{\pgfqpoint{4.787578in}{0.899469in}}%
\pgfpathlineto{\pgfqpoint{4.743226in}{0.912774in}}%
\pgfpathlineto{\pgfqpoint{4.787578in}{0.926080in}}%
\pgfpathlineto{\pgfqpoint{4.783143in}{0.917209in}}%
\pgfpathlineto{\pgfqpoint{4.851613in}{0.917209in}}%
\pgfpathlineto{\pgfqpoint{4.851613in}{0.908339in}}%
\pgfusepath{fill}%
\end{pgfscope}%
\begin{pgfscope}%
\pgfpathrectangle{\pgfqpoint{1.432000in}{0.528000in}}{\pgfqpoint{3.696000in}{3.696000in}} %
\pgfusepath{clip}%
\pgfsetbuttcap%
\pgfsetroundjoin%
\definecolor{currentfill}{rgb}{0.278826,0.175490,0.483397}%
\pgfsetfillcolor{currentfill}%
\pgfsetlinewidth{0.000000pt}%
\definecolor{currentstroke}{rgb}{0.000000,0.000000,0.000000}%
\pgfsetstrokecolor{currentstroke}%
\pgfsetdash{}{0pt}%
\pgfpathmoveto{\pgfqpoint{4.960000in}{0.908339in}}%
\pgfpathlineto{\pgfqpoint{4.891530in}{0.908339in}}%
\pgfpathlineto{\pgfqpoint{4.895965in}{0.899469in}}%
\pgfpathlineto{\pgfqpoint{4.851613in}{0.912774in}}%
\pgfpathlineto{\pgfqpoint{4.895965in}{0.926080in}}%
\pgfpathlineto{\pgfqpoint{4.891530in}{0.917209in}}%
\pgfpathlineto{\pgfqpoint{4.960000in}{0.917209in}}%
\pgfpathlineto{\pgfqpoint{4.960000in}{0.908339in}}%
\pgfusepath{fill}%
\end{pgfscope}%
\begin{pgfscope}%
\pgfpathrectangle{\pgfqpoint{1.432000in}{0.528000in}}{\pgfqpoint{3.696000in}{3.696000in}} %
\pgfusepath{clip}%
\pgfsetbuttcap%
\pgfsetroundjoin%
\definecolor{currentfill}{rgb}{0.190631,0.407061,0.556089}%
\pgfsetfillcolor{currentfill}%
\pgfsetlinewidth{0.000000pt}%
\definecolor{currentstroke}{rgb}{0.000000,0.000000,0.000000}%
\pgfsetstrokecolor{currentstroke}%
\pgfsetdash{}{0pt}%
\pgfpathmoveto{\pgfqpoint{4.964435in}{0.912774in}}%
\pgfpathlineto{\pgfqpoint{4.962218in}{0.916615in}}%
\pgfpathlineto{\pgfqpoint{4.957782in}{0.916615in}}%
\pgfpathlineto{\pgfqpoint{4.955565in}{0.912774in}}%
\pgfpathlineto{\pgfqpoint{4.957782in}{0.908933in}}%
\pgfpathlineto{\pgfqpoint{4.962218in}{0.908933in}}%
\pgfpathlineto{\pgfqpoint{4.964435in}{0.912774in}}%
\pgfpathlineto{\pgfqpoint{4.962218in}{0.916615in}}%
\pgfusepath{fill}%
\end{pgfscope}%
\begin{pgfscope}%
\pgfpathrectangle{\pgfqpoint{1.432000in}{0.528000in}}{\pgfqpoint{3.696000in}{3.696000in}} %
\pgfusepath{clip}%
\pgfsetbuttcap%
\pgfsetroundjoin%
\definecolor{currentfill}{rgb}{0.282656,0.100196,0.422160}%
\pgfsetfillcolor{currentfill}%
\pgfsetlinewidth{0.000000pt}%
\definecolor{currentstroke}{rgb}{0.000000,0.000000,0.000000}%
\pgfsetstrokecolor{currentstroke}%
\pgfsetdash{}{0pt}%
\pgfpathmoveto{\pgfqpoint{1.604435in}{1.021161in}}%
\pgfpathlineto{\pgfqpoint{1.602218in}{1.025002in}}%
\pgfpathlineto{\pgfqpoint{1.597782in}{1.025002in}}%
\pgfpathlineto{\pgfqpoint{1.595565in}{1.021161in}}%
\pgfpathlineto{\pgfqpoint{1.597782in}{1.017320in}}%
\pgfpathlineto{\pgfqpoint{1.602218in}{1.017320in}}%
\pgfpathlineto{\pgfqpoint{1.604435in}{1.021161in}}%
\pgfpathlineto{\pgfqpoint{1.602218in}{1.025002in}}%
\pgfusepath{fill}%
\end{pgfscope}%
\begin{pgfscope}%
\pgfpathrectangle{\pgfqpoint{1.432000in}{0.528000in}}{\pgfqpoint{3.696000in}{3.696000in}} %
\pgfusepath{clip}%
\pgfsetbuttcap%
\pgfsetroundjoin%
\definecolor{currentfill}{rgb}{0.276022,0.044167,0.370164}%
\pgfsetfillcolor{currentfill}%
\pgfsetlinewidth{0.000000pt}%
\definecolor{currentstroke}{rgb}{0.000000,0.000000,0.000000}%
\pgfsetstrokecolor{currentstroke}%
\pgfsetdash{}{0pt}%
\pgfpathmoveto{\pgfqpoint{1.708387in}{1.016726in}}%
\pgfpathlineto{\pgfqpoint{1.639917in}{1.016726in}}%
\pgfpathlineto{\pgfqpoint{1.644352in}{1.007856in}}%
\pgfpathlineto{\pgfqpoint{1.600000in}{1.021161in}}%
\pgfpathlineto{\pgfqpoint{1.644352in}{1.034467in}}%
\pgfpathlineto{\pgfqpoint{1.639917in}{1.025596in}}%
\pgfpathlineto{\pgfqpoint{1.708387in}{1.025596in}}%
\pgfpathlineto{\pgfqpoint{1.708387in}{1.016726in}}%
\pgfusepath{fill}%
\end{pgfscope}%
\begin{pgfscope}%
\pgfpathrectangle{\pgfqpoint{1.432000in}{0.528000in}}{\pgfqpoint{3.696000in}{3.696000in}} %
\pgfusepath{clip}%
\pgfsetbuttcap%
\pgfsetroundjoin%
\definecolor{currentfill}{rgb}{0.258965,0.251537,0.524736}%
\pgfsetfillcolor{currentfill}%
\pgfsetlinewidth{0.000000pt}%
\definecolor{currentstroke}{rgb}{0.000000,0.000000,0.000000}%
\pgfsetstrokecolor{currentstroke}%
\pgfsetdash{}{0pt}%
\pgfpathmoveto{\pgfqpoint{1.816774in}{1.016726in}}%
\pgfpathlineto{\pgfqpoint{1.748304in}{1.016726in}}%
\pgfpathlineto{\pgfqpoint{1.752739in}{1.007856in}}%
\pgfpathlineto{\pgfqpoint{1.708387in}{1.021161in}}%
\pgfpathlineto{\pgfqpoint{1.752739in}{1.034467in}}%
\pgfpathlineto{\pgfqpoint{1.748304in}{1.025596in}}%
\pgfpathlineto{\pgfqpoint{1.816774in}{1.025596in}}%
\pgfpathlineto{\pgfqpoint{1.816774in}{1.016726in}}%
\pgfusepath{fill}%
\end{pgfscope}%
\begin{pgfscope}%
\pgfpathrectangle{\pgfqpoint{1.432000in}{0.528000in}}{\pgfqpoint{3.696000in}{3.696000in}} %
\pgfusepath{clip}%
\pgfsetbuttcap%
\pgfsetroundjoin%
\definecolor{currentfill}{rgb}{0.266580,0.228262,0.514349}%
\pgfsetfillcolor{currentfill}%
\pgfsetlinewidth{0.000000pt}%
\definecolor{currentstroke}{rgb}{0.000000,0.000000,0.000000}%
\pgfsetstrokecolor{currentstroke}%
\pgfsetdash{}{0pt}%
\pgfpathmoveto{\pgfqpoint{1.925161in}{1.016726in}}%
\pgfpathlineto{\pgfqpoint{1.748304in}{1.016726in}}%
\pgfpathlineto{\pgfqpoint{1.752739in}{1.007856in}}%
\pgfpathlineto{\pgfqpoint{1.708387in}{1.021161in}}%
\pgfpathlineto{\pgfqpoint{1.752739in}{1.034467in}}%
\pgfpathlineto{\pgfqpoint{1.748304in}{1.025596in}}%
\pgfpathlineto{\pgfqpoint{1.925161in}{1.025596in}}%
\pgfpathlineto{\pgfqpoint{1.925161in}{1.016726in}}%
\pgfusepath{fill}%
\end{pgfscope}%
\begin{pgfscope}%
\pgfpathrectangle{\pgfqpoint{1.432000in}{0.528000in}}{\pgfqpoint{3.696000in}{3.696000in}} %
\pgfusepath{clip}%
\pgfsetbuttcap%
\pgfsetroundjoin%
\definecolor{currentfill}{rgb}{0.150476,0.504369,0.557430}%
\pgfsetfillcolor{currentfill}%
\pgfsetlinewidth{0.000000pt}%
\definecolor{currentstroke}{rgb}{0.000000,0.000000,0.000000}%
\pgfsetstrokecolor{currentstroke}%
\pgfsetdash{}{0pt}%
\pgfpathmoveto{\pgfqpoint{2.033548in}{1.016726in}}%
\pgfpathlineto{\pgfqpoint{1.856691in}{1.016726in}}%
\pgfpathlineto{\pgfqpoint{1.861126in}{1.007856in}}%
\pgfpathlineto{\pgfqpoint{1.816774in}{1.021161in}}%
\pgfpathlineto{\pgfqpoint{1.861126in}{1.034467in}}%
\pgfpathlineto{\pgfqpoint{1.856691in}{1.025596in}}%
\pgfpathlineto{\pgfqpoint{2.033548in}{1.025596in}}%
\pgfpathlineto{\pgfqpoint{2.033548in}{1.016726in}}%
\pgfusepath{fill}%
\end{pgfscope}%
\begin{pgfscope}%
\pgfpathrectangle{\pgfqpoint{1.432000in}{0.528000in}}{\pgfqpoint{3.696000in}{3.696000in}} %
\pgfusepath{clip}%
\pgfsetbuttcap%
\pgfsetroundjoin%
\definecolor{currentfill}{rgb}{0.274128,0.199721,0.498911}%
\pgfsetfillcolor{currentfill}%
\pgfsetlinewidth{0.000000pt}%
\definecolor{currentstroke}{rgb}{0.000000,0.000000,0.000000}%
\pgfsetstrokecolor{currentstroke}%
\pgfsetdash{}{0pt}%
\pgfpathmoveto{\pgfqpoint{2.143919in}{1.017194in}}%
\pgfpathlineto{\pgfqpoint{1.962847in}{0.926659in}}%
\pgfpathlineto{\pgfqpoint{1.970781in}{0.920708in}}%
\pgfpathlineto{\pgfqpoint{1.925161in}{0.912774in}}%
\pgfpathlineto{\pgfqpoint{1.958880in}{0.944510in}}%
\pgfpathlineto{\pgfqpoint{1.958880in}{0.934592in}}%
\pgfpathlineto{\pgfqpoint{2.139952in}{1.025128in}}%
\pgfpathlineto{\pgfqpoint{2.143919in}{1.017194in}}%
\pgfusepath{fill}%
\end{pgfscope}%
\begin{pgfscope}%
\pgfpathrectangle{\pgfqpoint{1.432000in}{0.528000in}}{\pgfqpoint{3.696000in}{3.696000in}} %
\pgfusepath{clip}%
\pgfsetbuttcap%
\pgfsetroundjoin%
\definecolor{currentfill}{rgb}{0.274128,0.199721,0.498911}%
\pgfsetfillcolor{currentfill}%
\pgfsetlinewidth{0.000000pt}%
\definecolor{currentstroke}{rgb}{0.000000,0.000000,0.000000}%
\pgfsetstrokecolor{currentstroke}%
\pgfsetdash{}{0pt}%
\pgfpathmoveto{\pgfqpoint{2.141935in}{1.016726in}}%
\pgfpathlineto{\pgfqpoint{1.965078in}{1.016726in}}%
\pgfpathlineto{\pgfqpoint{1.969513in}{1.007856in}}%
\pgfpathlineto{\pgfqpoint{1.925161in}{1.021161in}}%
\pgfpathlineto{\pgfqpoint{1.969513in}{1.034467in}}%
\pgfpathlineto{\pgfqpoint{1.965078in}{1.025596in}}%
\pgfpathlineto{\pgfqpoint{2.141935in}{1.025596in}}%
\pgfpathlineto{\pgfqpoint{2.141935in}{1.016726in}}%
\pgfusepath{fill}%
\end{pgfscope}%
\begin{pgfscope}%
\pgfpathrectangle{\pgfqpoint{1.432000in}{0.528000in}}{\pgfqpoint{3.696000in}{3.696000in}} %
\pgfusepath{clip}%
\pgfsetbuttcap%
\pgfsetroundjoin%
\definecolor{currentfill}{rgb}{0.283091,0.110553,0.431554}%
\pgfsetfillcolor{currentfill}%
\pgfsetlinewidth{0.000000pt}%
\definecolor{currentstroke}{rgb}{0.000000,0.000000,0.000000}%
\pgfsetstrokecolor{currentstroke}%
\pgfsetdash{}{0pt}%
\pgfpathmoveto{\pgfqpoint{2.252306in}{1.017194in}}%
\pgfpathlineto{\pgfqpoint{2.071235in}{0.926659in}}%
\pgfpathlineto{\pgfqpoint{2.079168in}{0.920708in}}%
\pgfpathlineto{\pgfqpoint{2.033548in}{0.912774in}}%
\pgfpathlineto{\pgfqpoint{2.067268in}{0.944510in}}%
\pgfpathlineto{\pgfqpoint{2.067268in}{0.934592in}}%
\pgfpathlineto{\pgfqpoint{2.248339in}{1.025128in}}%
\pgfpathlineto{\pgfqpoint{2.252306in}{1.017194in}}%
\pgfusepath{fill}%
\end{pgfscope}%
\begin{pgfscope}%
\pgfpathrectangle{\pgfqpoint{1.432000in}{0.528000in}}{\pgfqpoint{3.696000in}{3.696000in}} %
\pgfusepath{clip}%
\pgfsetbuttcap%
\pgfsetroundjoin%
\definecolor{currentfill}{rgb}{0.280894,0.078907,0.402329}%
\pgfsetfillcolor{currentfill}%
\pgfsetlinewidth{0.000000pt}%
\definecolor{currentstroke}{rgb}{0.000000,0.000000,0.000000}%
\pgfsetstrokecolor{currentstroke}%
\pgfsetdash{}{0pt}%
\pgfpathmoveto{\pgfqpoint{2.250323in}{1.016726in}}%
\pgfpathlineto{\pgfqpoint{2.073465in}{1.016726in}}%
\pgfpathlineto{\pgfqpoint{2.077900in}{1.007856in}}%
\pgfpathlineto{\pgfqpoint{2.033548in}{1.021161in}}%
\pgfpathlineto{\pgfqpoint{2.077900in}{1.034467in}}%
\pgfpathlineto{\pgfqpoint{2.073465in}{1.025596in}}%
\pgfpathlineto{\pgfqpoint{2.250323in}{1.025596in}}%
\pgfpathlineto{\pgfqpoint{2.250323in}{1.016726in}}%
\pgfusepath{fill}%
\end{pgfscope}%
\begin{pgfscope}%
\pgfpathrectangle{\pgfqpoint{1.432000in}{0.528000in}}{\pgfqpoint{3.696000in}{3.696000in}} %
\pgfusepath{clip}%
\pgfsetbuttcap%
\pgfsetroundjoin%
\definecolor{currentfill}{rgb}{0.282327,0.094955,0.417331}%
\pgfsetfillcolor{currentfill}%
\pgfsetlinewidth{0.000000pt}%
\definecolor{currentstroke}{rgb}{0.000000,0.000000,0.000000}%
\pgfsetstrokecolor{currentstroke}%
\pgfsetdash{}{0pt}%
\pgfpathmoveto{\pgfqpoint{2.360112in}{1.016954in}}%
\pgfpathlineto{\pgfqpoint{2.072819in}{0.921189in}}%
\pgfpathlineto{\pgfqpoint{2.079832in}{0.914177in}}%
\pgfpathlineto{\pgfqpoint{2.033548in}{0.912774in}}%
\pgfpathlineto{\pgfqpoint{2.071417in}{0.939422in}}%
\pgfpathlineto{\pgfqpoint{2.070014in}{0.929605in}}%
\pgfpathlineto{\pgfqpoint{2.357307in}{1.025369in}}%
\pgfpathlineto{\pgfqpoint{2.360112in}{1.016954in}}%
\pgfusepath{fill}%
\end{pgfscope}%
\begin{pgfscope}%
\pgfpathrectangle{\pgfqpoint{1.432000in}{0.528000in}}{\pgfqpoint{3.696000in}{3.696000in}} %
\pgfusepath{clip}%
\pgfsetbuttcap%
\pgfsetroundjoin%
\definecolor{currentfill}{rgb}{0.269944,0.014625,0.341379}%
\pgfsetfillcolor{currentfill}%
\pgfsetlinewidth{0.000000pt}%
\definecolor{currentstroke}{rgb}{0.000000,0.000000,0.000000}%
\pgfsetstrokecolor{currentstroke}%
\pgfsetdash{}{0pt}%
\pgfpathmoveto{\pgfqpoint{2.358710in}{1.016726in}}%
\pgfpathlineto{\pgfqpoint{2.073465in}{1.016726in}}%
\pgfpathlineto{\pgfqpoint{2.077900in}{1.007856in}}%
\pgfpathlineto{\pgfqpoint{2.033548in}{1.021161in}}%
\pgfpathlineto{\pgfqpoint{2.077900in}{1.034467in}}%
\pgfpathlineto{\pgfqpoint{2.073465in}{1.025596in}}%
\pgfpathlineto{\pgfqpoint{2.358710in}{1.025596in}}%
\pgfpathlineto{\pgfqpoint{2.358710in}{1.016726in}}%
\pgfusepath{fill}%
\end{pgfscope}%
\begin{pgfscope}%
\pgfpathrectangle{\pgfqpoint{1.432000in}{0.528000in}}{\pgfqpoint{3.696000in}{3.696000in}} %
\pgfusepath{clip}%
\pgfsetbuttcap%
\pgfsetroundjoin%
\definecolor{currentfill}{rgb}{0.233603,0.313828,0.543914}%
\pgfsetfillcolor{currentfill}%
\pgfsetlinewidth{0.000000pt}%
\definecolor{currentstroke}{rgb}{0.000000,0.000000,0.000000}%
\pgfsetstrokecolor{currentstroke}%
\pgfsetdash{}{0pt}%
\pgfpathmoveto{\pgfqpoint{2.467097in}{1.016726in}}%
\pgfpathlineto{\pgfqpoint{2.181852in}{1.016726in}}%
\pgfpathlineto{\pgfqpoint{2.186287in}{1.007856in}}%
\pgfpathlineto{\pgfqpoint{2.141935in}{1.021161in}}%
\pgfpathlineto{\pgfqpoint{2.186287in}{1.034467in}}%
\pgfpathlineto{\pgfqpoint{2.181852in}{1.025596in}}%
\pgfpathlineto{\pgfqpoint{2.467097in}{1.025596in}}%
\pgfpathlineto{\pgfqpoint{2.467097in}{1.016726in}}%
\pgfusepath{fill}%
\end{pgfscope}%
\begin{pgfscope}%
\pgfpathrectangle{\pgfqpoint{1.432000in}{0.528000in}}{\pgfqpoint{3.696000in}{3.696000in}} %
\pgfusepath{clip}%
\pgfsetbuttcap%
\pgfsetroundjoin%
\definecolor{currentfill}{rgb}{0.239346,0.300855,0.540844}%
\pgfsetfillcolor{currentfill}%
\pgfsetlinewidth{0.000000pt}%
\definecolor{currentstroke}{rgb}{0.000000,0.000000,0.000000}%
\pgfsetstrokecolor{currentstroke}%
\pgfsetdash{}{0pt}%
\pgfpathmoveto{\pgfqpoint{2.575484in}{1.016726in}}%
\pgfpathlineto{\pgfqpoint{2.290239in}{1.016726in}}%
\pgfpathlineto{\pgfqpoint{2.294675in}{1.007856in}}%
\pgfpathlineto{\pgfqpoint{2.250323in}{1.021161in}}%
\pgfpathlineto{\pgfqpoint{2.294675in}{1.034467in}}%
\pgfpathlineto{\pgfqpoint{2.290239in}{1.025596in}}%
\pgfpathlineto{\pgfqpoint{2.575484in}{1.025596in}}%
\pgfpathlineto{\pgfqpoint{2.575484in}{1.016726in}}%
\pgfusepath{fill}%
\end{pgfscope}%
\begin{pgfscope}%
\pgfpathrectangle{\pgfqpoint{1.432000in}{0.528000in}}{\pgfqpoint{3.696000in}{3.696000in}} %
\pgfusepath{clip}%
\pgfsetbuttcap%
\pgfsetroundjoin%
\definecolor{currentfill}{rgb}{0.275191,0.194905,0.496005}%
\pgfsetfillcolor{currentfill}%
\pgfsetlinewidth{0.000000pt}%
\definecolor{currentstroke}{rgb}{0.000000,0.000000,0.000000}%
\pgfsetstrokecolor{currentstroke}%
\pgfsetdash{}{0pt}%
\pgfpathmoveto{\pgfqpoint{2.683871in}{1.016726in}}%
\pgfpathlineto{\pgfqpoint{2.290239in}{1.016726in}}%
\pgfpathlineto{\pgfqpoint{2.294675in}{1.007856in}}%
\pgfpathlineto{\pgfqpoint{2.250323in}{1.021161in}}%
\pgfpathlineto{\pgfqpoint{2.294675in}{1.034467in}}%
\pgfpathlineto{\pgfqpoint{2.290239in}{1.025596in}}%
\pgfpathlineto{\pgfqpoint{2.683871in}{1.025596in}}%
\pgfpathlineto{\pgfqpoint{2.683871in}{1.016726in}}%
\pgfusepath{fill}%
\end{pgfscope}%
\begin{pgfscope}%
\pgfpathrectangle{\pgfqpoint{1.432000in}{0.528000in}}{\pgfqpoint{3.696000in}{3.696000in}} %
\pgfusepath{clip}%
\pgfsetbuttcap%
\pgfsetroundjoin%
\definecolor{currentfill}{rgb}{0.267968,0.223549,0.512008}%
\pgfsetfillcolor{currentfill}%
\pgfsetlinewidth{0.000000pt}%
\definecolor{currentstroke}{rgb}{0.000000,0.000000,0.000000}%
\pgfsetstrokecolor{currentstroke}%
\pgfsetdash{}{0pt}%
\pgfpathmoveto{\pgfqpoint{2.683871in}{1.016726in}}%
\pgfpathlineto{\pgfqpoint{2.398626in}{1.016726in}}%
\pgfpathlineto{\pgfqpoint{2.403062in}{1.007856in}}%
\pgfpathlineto{\pgfqpoint{2.358710in}{1.021161in}}%
\pgfpathlineto{\pgfqpoint{2.403062in}{1.034467in}}%
\pgfpathlineto{\pgfqpoint{2.398626in}{1.025596in}}%
\pgfpathlineto{\pgfqpoint{2.683871in}{1.025596in}}%
\pgfpathlineto{\pgfqpoint{2.683871in}{1.016726in}}%
\pgfusepath{fill}%
\end{pgfscope}%
\begin{pgfscope}%
\pgfpathrectangle{\pgfqpoint{1.432000in}{0.528000in}}{\pgfqpoint{3.696000in}{3.696000in}} %
\pgfusepath{clip}%
\pgfsetbuttcap%
\pgfsetroundjoin%
\definecolor{currentfill}{rgb}{0.267968,0.223549,0.512008}%
\pgfsetfillcolor{currentfill}%
\pgfsetlinewidth{0.000000pt}%
\definecolor{currentstroke}{rgb}{0.000000,0.000000,0.000000}%
\pgfsetstrokecolor{currentstroke}%
\pgfsetdash{}{0pt}%
\pgfpathmoveto{\pgfqpoint{2.792258in}{1.016726in}}%
\pgfpathlineto{\pgfqpoint{2.398626in}{1.016726in}}%
\pgfpathlineto{\pgfqpoint{2.403062in}{1.007856in}}%
\pgfpathlineto{\pgfqpoint{2.358710in}{1.021161in}}%
\pgfpathlineto{\pgfqpoint{2.403062in}{1.034467in}}%
\pgfpathlineto{\pgfqpoint{2.398626in}{1.025596in}}%
\pgfpathlineto{\pgfqpoint{2.792258in}{1.025596in}}%
\pgfpathlineto{\pgfqpoint{2.792258in}{1.016726in}}%
\pgfusepath{fill}%
\end{pgfscope}%
\begin{pgfscope}%
\pgfpathrectangle{\pgfqpoint{1.432000in}{0.528000in}}{\pgfqpoint{3.696000in}{3.696000in}} %
\pgfusepath{clip}%
\pgfsetbuttcap%
\pgfsetroundjoin%
\definecolor{currentfill}{rgb}{0.270595,0.214069,0.507052}%
\pgfsetfillcolor{currentfill}%
\pgfsetlinewidth{0.000000pt}%
\definecolor{currentstroke}{rgb}{0.000000,0.000000,0.000000}%
\pgfsetstrokecolor{currentstroke}%
\pgfsetdash{}{0pt}%
\pgfpathmoveto{\pgfqpoint{2.792258in}{1.016726in}}%
\pgfpathlineto{\pgfqpoint{2.507014in}{1.016726in}}%
\pgfpathlineto{\pgfqpoint{2.511449in}{1.007856in}}%
\pgfpathlineto{\pgfqpoint{2.467097in}{1.021161in}}%
\pgfpathlineto{\pgfqpoint{2.511449in}{1.034467in}}%
\pgfpathlineto{\pgfqpoint{2.507014in}{1.025596in}}%
\pgfpathlineto{\pgfqpoint{2.792258in}{1.025596in}}%
\pgfpathlineto{\pgfqpoint{2.792258in}{1.016726in}}%
\pgfusepath{fill}%
\end{pgfscope}%
\begin{pgfscope}%
\pgfpathrectangle{\pgfqpoint{1.432000in}{0.528000in}}{\pgfqpoint{3.696000in}{3.696000in}} %
\pgfusepath{clip}%
\pgfsetbuttcap%
\pgfsetroundjoin%
\definecolor{currentfill}{rgb}{0.133743,0.548535,0.553541}%
\pgfsetfillcolor{currentfill}%
\pgfsetlinewidth{0.000000pt}%
\definecolor{currentstroke}{rgb}{0.000000,0.000000,0.000000}%
\pgfsetstrokecolor{currentstroke}%
\pgfsetdash{}{0pt}%
\pgfpathmoveto{\pgfqpoint{2.899243in}{1.016954in}}%
\pgfpathlineto{\pgfqpoint{2.611950in}{1.112718in}}%
\pgfpathlineto{\pgfqpoint{2.613352in}{1.102900in}}%
\pgfpathlineto{\pgfqpoint{2.575484in}{1.129548in}}%
\pgfpathlineto{\pgfqpoint{2.621767in}{1.128146in}}%
\pgfpathlineto{\pgfqpoint{2.614755in}{1.121133in}}%
\pgfpathlineto{\pgfqpoint{2.902048in}{1.025369in}}%
\pgfpathlineto{\pgfqpoint{2.899243in}{1.016954in}}%
\pgfusepath{fill}%
\end{pgfscope}%
\begin{pgfscope}%
\pgfpathrectangle{\pgfqpoint{1.432000in}{0.528000in}}{\pgfqpoint{3.696000in}{3.696000in}} %
\pgfusepath{clip}%
\pgfsetbuttcap%
\pgfsetroundjoin%
\definecolor{currentfill}{rgb}{0.129933,0.559582,0.551864}%
\pgfsetfillcolor{currentfill}%
\pgfsetlinewidth{0.000000pt}%
\definecolor{currentstroke}{rgb}{0.000000,0.000000,0.000000}%
\pgfsetstrokecolor{currentstroke}%
\pgfsetdash{}{0pt}%
\pgfpathmoveto{\pgfqpoint{3.007630in}{1.016954in}}%
\pgfpathlineto{\pgfqpoint{2.720337in}{1.112718in}}%
\pgfpathlineto{\pgfqpoint{2.721739in}{1.102900in}}%
\pgfpathlineto{\pgfqpoint{2.683871in}{1.129548in}}%
\pgfpathlineto{\pgfqpoint{2.730155in}{1.128146in}}%
\pgfpathlineto{\pgfqpoint{2.723142in}{1.121133in}}%
\pgfpathlineto{\pgfqpoint{3.010435in}{1.025369in}}%
\pgfpathlineto{\pgfqpoint{3.007630in}{1.016954in}}%
\pgfusepath{fill}%
\end{pgfscope}%
\begin{pgfscope}%
\pgfpathrectangle{\pgfqpoint{1.432000in}{0.528000in}}{\pgfqpoint{3.696000in}{3.696000in}} %
\pgfusepath{clip}%
\pgfsetbuttcap%
\pgfsetroundjoin%
\definecolor{currentfill}{rgb}{0.274952,0.037752,0.364543}%
\pgfsetfillcolor{currentfill}%
\pgfsetlinewidth{0.000000pt}%
\definecolor{currentstroke}{rgb}{0.000000,0.000000,0.000000}%
\pgfsetstrokecolor{currentstroke}%
\pgfsetdash{}{0pt}%
\pgfpathmoveto{\pgfqpoint{3.007049in}{1.017194in}}%
\pgfpathlineto{\pgfqpoint{2.825977in}{1.107730in}}%
\pgfpathlineto{\pgfqpoint{2.825977in}{1.097813in}}%
\pgfpathlineto{\pgfqpoint{2.792258in}{1.129548in}}%
\pgfpathlineto{\pgfqpoint{2.837878in}{1.121614in}}%
\pgfpathlineto{\pgfqpoint{2.829944in}{1.115664in}}%
\pgfpathlineto{\pgfqpoint{3.011016in}{1.025128in}}%
\pgfpathlineto{\pgfqpoint{3.007049in}{1.017194in}}%
\pgfusepath{fill}%
\end{pgfscope}%
\begin{pgfscope}%
\pgfpathrectangle{\pgfqpoint{1.432000in}{0.528000in}}{\pgfqpoint{3.696000in}{3.696000in}} %
\pgfusepath{clip}%
\pgfsetbuttcap%
\pgfsetroundjoin%
\definecolor{currentfill}{rgb}{0.235526,0.309527,0.542944}%
\pgfsetfillcolor{currentfill}%
\pgfsetlinewidth{0.000000pt}%
\definecolor{currentstroke}{rgb}{0.000000,0.000000,0.000000}%
\pgfsetstrokecolor{currentstroke}%
\pgfsetdash{}{0pt}%
\pgfpathmoveto{\pgfqpoint{3.116017in}{1.016954in}}%
\pgfpathlineto{\pgfqpoint{2.828724in}{1.112718in}}%
\pgfpathlineto{\pgfqpoint{2.830126in}{1.102900in}}%
\pgfpathlineto{\pgfqpoint{2.792258in}{1.129548in}}%
\pgfpathlineto{\pgfqpoint{2.838542in}{1.128146in}}%
\pgfpathlineto{\pgfqpoint{2.831529in}{1.121133in}}%
\pgfpathlineto{\pgfqpoint{3.118822in}{1.025369in}}%
\pgfpathlineto{\pgfqpoint{3.116017in}{1.016954in}}%
\pgfusepath{fill}%
\end{pgfscope}%
\begin{pgfscope}%
\pgfpathrectangle{\pgfqpoint{1.432000in}{0.528000in}}{\pgfqpoint{3.696000in}{3.696000in}} %
\pgfusepath{clip}%
\pgfsetbuttcap%
\pgfsetroundjoin%
\definecolor{currentfill}{rgb}{0.192357,0.403199,0.555836}%
\pgfsetfillcolor{currentfill}%
\pgfsetlinewidth{0.000000pt}%
\definecolor{currentstroke}{rgb}{0.000000,0.000000,0.000000}%
\pgfsetstrokecolor{currentstroke}%
\pgfsetdash{}{0pt}%
\pgfpathmoveto{\pgfqpoint{3.115436in}{1.017194in}}%
\pgfpathlineto{\pgfqpoint{2.934364in}{1.107730in}}%
\pgfpathlineto{\pgfqpoint{2.934364in}{1.097813in}}%
\pgfpathlineto{\pgfqpoint{2.900645in}{1.129548in}}%
\pgfpathlineto{\pgfqpoint{2.946265in}{1.121614in}}%
\pgfpathlineto{\pgfqpoint{2.938331in}{1.115664in}}%
\pgfpathlineto{\pgfqpoint{3.119403in}{1.025128in}}%
\pgfpathlineto{\pgfqpoint{3.115436in}{1.017194in}}%
\pgfusepath{fill}%
\end{pgfscope}%
\begin{pgfscope}%
\pgfpathrectangle{\pgfqpoint{1.432000in}{0.528000in}}{\pgfqpoint{3.696000in}{3.696000in}} %
\pgfusepath{clip}%
\pgfsetbuttcap%
\pgfsetroundjoin%
\definecolor{currentfill}{rgb}{0.135066,0.544853,0.554029}%
\pgfsetfillcolor{currentfill}%
\pgfsetlinewidth{0.000000pt}%
\definecolor{currentstroke}{rgb}{0.000000,0.000000,0.000000}%
\pgfsetstrokecolor{currentstroke}%
\pgfsetdash{}{0pt}%
\pgfpathmoveto{\pgfqpoint{3.223823in}{1.017194in}}%
\pgfpathlineto{\pgfqpoint{3.042751in}{1.107730in}}%
\pgfpathlineto{\pgfqpoint{3.042751in}{1.097813in}}%
\pgfpathlineto{\pgfqpoint{3.009032in}{1.129548in}}%
\pgfpathlineto{\pgfqpoint{3.054652in}{1.121614in}}%
\pgfpathlineto{\pgfqpoint{3.046718in}{1.115664in}}%
\pgfpathlineto{\pgfqpoint{3.227790in}{1.025128in}}%
\pgfpathlineto{\pgfqpoint{3.223823in}{1.017194in}}%
\pgfusepath{fill}%
\end{pgfscope}%
\begin{pgfscope}%
\pgfpathrectangle{\pgfqpoint{1.432000in}{0.528000in}}{\pgfqpoint{3.696000in}{3.696000in}} %
\pgfusepath{clip}%
\pgfsetbuttcap%
\pgfsetroundjoin%
\definecolor{currentfill}{rgb}{0.273809,0.031497,0.358853}%
\pgfsetfillcolor{currentfill}%
\pgfsetlinewidth{0.000000pt}%
\definecolor{currentstroke}{rgb}{0.000000,0.000000,0.000000}%
\pgfsetstrokecolor{currentstroke}%
\pgfsetdash{}{0pt}%
\pgfpathmoveto{\pgfqpoint{3.222670in}{1.018025in}}%
\pgfpathlineto{\pgfqpoint{3.142509in}{1.098187in}}%
\pgfpathlineto{\pgfqpoint{3.139372in}{1.088778in}}%
\pgfpathlineto{\pgfqpoint{3.117419in}{1.129548in}}%
\pgfpathlineto{\pgfqpoint{3.158189in}{1.107595in}}%
\pgfpathlineto{\pgfqpoint{3.148781in}{1.104459in}}%
\pgfpathlineto{\pgfqpoint{3.228943in}{1.024297in}}%
\pgfpathlineto{\pgfqpoint{3.222670in}{1.018025in}}%
\pgfusepath{fill}%
\end{pgfscope}%
\begin{pgfscope}%
\pgfpathrectangle{\pgfqpoint{1.432000in}{0.528000in}}{\pgfqpoint{3.696000in}{3.696000in}} %
\pgfusepath{clip}%
\pgfsetbuttcap%
\pgfsetroundjoin%
\definecolor{currentfill}{rgb}{0.166617,0.463708,0.558119}%
\pgfsetfillcolor{currentfill}%
\pgfsetlinewidth{0.000000pt}%
\definecolor{currentstroke}{rgb}{0.000000,0.000000,0.000000}%
\pgfsetstrokecolor{currentstroke}%
\pgfsetdash{}{0pt}%
\pgfpathmoveto{\pgfqpoint{3.332210in}{1.017194in}}%
\pgfpathlineto{\pgfqpoint{3.151139in}{1.107730in}}%
\pgfpathlineto{\pgfqpoint{3.151139in}{1.097813in}}%
\pgfpathlineto{\pgfqpoint{3.117419in}{1.129548in}}%
\pgfpathlineto{\pgfqpoint{3.163039in}{1.121614in}}%
\pgfpathlineto{\pgfqpoint{3.155106in}{1.115664in}}%
\pgfpathlineto{\pgfqpoint{3.336177in}{1.025128in}}%
\pgfpathlineto{\pgfqpoint{3.332210in}{1.017194in}}%
\pgfusepath{fill}%
\end{pgfscope}%
\begin{pgfscope}%
\pgfpathrectangle{\pgfqpoint{1.432000in}{0.528000in}}{\pgfqpoint{3.696000in}{3.696000in}} %
\pgfusepath{clip}%
\pgfsetbuttcap%
\pgfsetroundjoin%
\definecolor{currentfill}{rgb}{0.280894,0.078907,0.402329}%
\pgfsetfillcolor{currentfill}%
\pgfsetlinewidth{0.000000pt}%
\definecolor{currentstroke}{rgb}{0.000000,0.000000,0.000000}%
\pgfsetstrokecolor{currentstroke}%
\pgfsetdash{}{0pt}%
\pgfpathmoveto{\pgfqpoint{3.331057in}{1.018025in}}%
\pgfpathlineto{\pgfqpoint{3.250896in}{1.098187in}}%
\pgfpathlineto{\pgfqpoint{3.247760in}{1.088778in}}%
\pgfpathlineto{\pgfqpoint{3.225806in}{1.129548in}}%
\pgfpathlineto{\pgfqpoint{3.266577in}{1.107595in}}%
\pgfpathlineto{\pgfqpoint{3.257168in}{1.104459in}}%
\pgfpathlineto{\pgfqpoint{3.337330in}{1.024297in}}%
\pgfpathlineto{\pgfqpoint{3.331057in}{1.018025in}}%
\pgfusepath{fill}%
\end{pgfscope}%
\begin{pgfscope}%
\pgfpathrectangle{\pgfqpoint{1.432000in}{0.528000in}}{\pgfqpoint{3.696000in}{3.696000in}} %
\pgfusepath{clip}%
\pgfsetbuttcap%
\pgfsetroundjoin%
\definecolor{currentfill}{rgb}{0.227802,0.326594,0.546532}%
\pgfsetfillcolor{currentfill}%
\pgfsetlinewidth{0.000000pt}%
\definecolor{currentstroke}{rgb}{0.000000,0.000000,0.000000}%
\pgfsetstrokecolor{currentstroke}%
\pgfsetdash{}{0pt}%
\pgfpathmoveto{\pgfqpoint{3.440597in}{1.017194in}}%
\pgfpathlineto{\pgfqpoint{3.259526in}{1.107730in}}%
\pgfpathlineto{\pgfqpoint{3.259526in}{1.097813in}}%
\pgfpathlineto{\pgfqpoint{3.225806in}{1.129548in}}%
\pgfpathlineto{\pgfqpoint{3.271427in}{1.121614in}}%
\pgfpathlineto{\pgfqpoint{3.263493in}{1.115664in}}%
\pgfpathlineto{\pgfqpoint{3.444564in}{1.025128in}}%
\pgfpathlineto{\pgfqpoint{3.440597in}{1.017194in}}%
\pgfusepath{fill}%
\end{pgfscope}%
\begin{pgfscope}%
\pgfpathrectangle{\pgfqpoint{1.432000in}{0.528000in}}{\pgfqpoint{3.696000in}{3.696000in}} %
\pgfusepath{clip}%
\pgfsetbuttcap%
\pgfsetroundjoin%
\definecolor{currentfill}{rgb}{0.260571,0.246922,0.522828}%
\pgfsetfillcolor{currentfill}%
\pgfsetlinewidth{0.000000pt}%
\definecolor{currentstroke}{rgb}{0.000000,0.000000,0.000000}%
\pgfsetstrokecolor{currentstroke}%
\pgfsetdash{}{0pt}%
\pgfpathmoveto{\pgfqpoint{3.439444in}{1.018025in}}%
\pgfpathlineto{\pgfqpoint{3.359283in}{1.098187in}}%
\pgfpathlineto{\pgfqpoint{3.356147in}{1.088778in}}%
\pgfpathlineto{\pgfqpoint{3.334194in}{1.129548in}}%
\pgfpathlineto{\pgfqpoint{3.374964in}{1.107595in}}%
\pgfpathlineto{\pgfqpoint{3.365555in}{1.104459in}}%
\pgfpathlineto{\pgfqpoint{3.445717in}{1.024297in}}%
\pgfpathlineto{\pgfqpoint{3.439444in}{1.018025in}}%
\pgfusepath{fill}%
\end{pgfscope}%
\begin{pgfscope}%
\pgfpathrectangle{\pgfqpoint{1.432000in}{0.528000in}}{\pgfqpoint{3.696000in}{3.696000in}} %
\pgfusepath{clip}%
\pgfsetbuttcap%
\pgfsetroundjoin%
\definecolor{currentfill}{rgb}{0.280255,0.165693,0.476498}%
\pgfsetfillcolor{currentfill}%
\pgfsetlinewidth{0.000000pt}%
\definecolor{currentstroke}{rgb}{0.000000,0.000000,0.000000}%
\pgfsetstrokecolor{currentstroke}%
\pgfsetdash{}{0pt}%
\pgfpathmoveto{\pgfqpoint{3.550968in}{1.016726in}}%
\pgfpathlineto{\pgfqpoint{3.482497in}{1.016726in}}%
\pgfpathlineto{\pgfqpoint{3.486933in}{1.007856in}}%
\pgfpathlineto{\pgfqpoint{3.442581in}{1.021161in}}%
\pgfpathlineto{\pgfqpoint{3.486933in}{1.034467in}}%
\pgfpathlineto{\pgfqpoint{3.482497in}{1.025596in}}%
\pgfpathlineto{\pgfqpoint{3.550968in}{1.025596in}}%
\pgfpathlineto{\pgfqpoint{3.550968in}{1.016726in}}%
\pgfusepath{fill}%
\end{pgfscope}%
\begin{pgfscope}%
\pgfpathrectangle{\pgfqpoint{1.432000in}{0.528000in}}{\pgfqpoint{3.696000in}{3.696000in}} %
\pgfusepath{clip}%
\pgfsetbuttcap%
\pgfsetroundjoin%
\definecolor{currentfill}{rgb}{0.269944,0.014625,0.341379}%
\pgfsetfillcolor{currentfill}%
\pgfsetlinewidth{0.000000pt}%
\definecolor{currentstroke}{rgb}{0.000000,0.000000,0.000000}%
\pgfsetstrokecolor{currentstroke}%
\pgfsetdash{}{0pt}%
\pgfpathmoveto{\pgfqpoint{3.548984in}{1.017194in}}%
\pgfpathlineto{\pgfqpoint{3.367913in}{1.107730in}}%
\pgfpathlineto{\pgfqpoint{3.367913in}{1.097813in}}%
\pgfpathlineto{\pgfqpoint{3.334194in}{1.129548in}}%
\pgfpathlineto{\pgfqpoint{3.379814in}{1.121614in}}%
\pgfpathlineto{\pgfqpoint{3.371880in}{1.115664in}}%
\pgfpathlineto{\pgfqpoint{3.552951in}{1.025128in}}%
\pgfpathlineto{\pgfqpoint{3.548984in}{1.017194in}}%
\pgfusepath{fill}%
\end{pgfscope}%
\begin{pgfscope}%
\pgfpathrectangle{\pgfqpoint{1.432000in}{0.528000in}}{\pgfqpoint{3.696000in}{3.696000in}} %
\pgfusepath{clip}%
\pgfsetbuttcap%
\pgfsetroundjoin%
\definecolor{currentfill}{rgb}{0.267968,0.223549,0.512008}%
\pgfsetfillcolor{currentfill}%
\pgfsetlinewidth{0.000000pt}%
\definecolor{currentstroke}{rgb}{0.000000,0.000000,0.000000}%
\pgfsetstrokecolor{currentstroke}%
\pgfsetdash{}{0pt}%
\pgfpathmoveto{\pgfqpoint{3.547832in}{1.018025in}}%
\pgfpathlineto{\pgfqpoint{3.467670in}{1.098187in}}%
\pgfpathlineto{\pgfqpoint{3.464534in}{1.088778in}}%
\pgfpathlineto{\pgfqpoint{3.442581in}{1.129548in}}%
\pgfpathlineto{\pgfqpoint{3.483351in}{1.107595in}}%
\pgfpathlineto{\pgfqpoint{3.473942in}{1.104459in}}%
\pgfpathlineto{\pgfqpoint{3.554104in}{1.024297in}}%
\pgfpathlineto{\pgfqpoint{3.547832in}{1.018025in}}%
\pgfusepath{fill}%
\end{pgfscope}%
\begin{pgfscope}%
\pgfpathrectangle{\pgfqpoint{1.432000in}{0.528000in}}{\pgfqpoint{3.696000in}{3.696000in}} %
\pgfusepath{clip}%
\pgfsetbuttcap%
\pgfsetroundjoin%
\definecolor{currentfill}{rgb}{0.274128,0.199721,0.498911}%
\pgfsetfillcolor{currentfill}%
\pgfsetlinewidth{0.000000pt}%
\definecolor{currentstroke}{rgb}{0.000000,0.000000,0.000000}%
\pgfsetstrokecolor{currentstroke}%
\pgfsetdash{}{0pt}%
\pgfpathmoveto{\pgfqpoint{3.659355in}{1.016726in}}%
\pgfpathlineto{\pgfqpoint{3.590885in}{1.016726in}}%
\pgfpathlineto{\pgfqpoint{3.595320in}{1.007856in}}%
\pgfpathlineto{\pgfqpoint{3.550968in}{1.021161in}}%
\pgfpathlineto{\pgfqpoint{3.595320in}{1.034467in}}%
\pgfpathlineto{\pgfqpoint{3.590885in}{1.025596in}}%
\pgfpathlineto{\pgfqpoint{3.659355in}{1.025596in}}%
\pgfpathlineto{\pgfqpoint{3.659355in}{1.016726in}}%
\pgfusepath{fill}%
\end{pgfscope}%
\begin{pgfscope}%
\pgfpathrectangle{\pgfqpoint{1.432000in}{0.528000in}}{\pgfqpoint{3.696000in}{3.696000in}} %
\pgfusepath{clip}%
\pgfsetbuttcap%
\pgfsetroundjoin%
\definecolor{currentfill}{rgb}{0.281446,0.084320,0.407414}%
\pgfsetfillcolor{currentfill}%
\pgfsetlinewidth{0.000000pt}%
\definecolor{currentstroke}{rgb}{0.000000,0.000000,0.000000}%
\pgfsetstrokecolor{currentstroke}%
\pgfsetdash{}{0pt}%
\pgfpathmoveto{\pgfqpoint{3.656219in}{1.018025in}}%
\pgfpathlineto{\pgfqpoint{3.576057in}{1.098187in}}%
\pgfpathlineto{\pgfqpoint{3.572921in}{1.088778in}}%
\pgfpathlineto{\pgfqpoint{3.550968in}{1.129548in}}%
\pgfpathlineto{\pgfqpoint{3.591738in}{1.107595in}}%
\pgfpathlineto{\pgfqpoint{3.582329in}{1.104459in}}%
\pgfpathlineto{\pgfqpoint{3.662491in}{1.024297in}}%
\pgfpathlineto{\pgfqpoint{3.656219in}{1.018025in}}%
\pgfusepath{fill}%
\end{pgfscope}%
\begin{pgfscope}%
\pgfpathrectangle{\pgfqpoint{1.432000in}{0.528000in}}{\pgfqpoint{3.696000in}{3.696000in}} %
\pgfusepath{clip}%
\pgfsetbuttcap%
\pgfsetroundjoin%
\definecolor{currentfill}{rgb}{0.267968,0.223549,0.512008}%
\pgfsetfillcolor{currentfill}%
\pgfsetlinewidth{0.000000pt}%
\definecolor{currentstroke}{rgb}{0.000000,0.000000,0.000000}%
\pgfsetstrokecolor{currentstroke}%
\pgfsetdash{}{0pt}%
\pgfpathmoveto{\pgfqpoint{3.767742in}{1.016726in}}%
\pgfpathlineto{\pgfqpoint{3.590885in}{1.016726in}}%
\pgfpathlineto{\pgfqpoint{3.595320in}{1.007856in}}%
\pgfpathlineto{\pgfqpoint{3.550968in}{1.021161in}}%
\pgfpathlineto{\pgfqpoint{3.595320in}{1.034467in}}%
\pgfpathlineto{\pgfqpoint{3.590885in}{1.025596in}}%
\pgfpathlineto{\pgfqpoint{3.767742in}{1.025596in}}%
\pgfpathlineto{\pgfqpoint{3.767742in}{1.016726in}}%
\pgfusepath{fill}%
\end{pgfscope}%
\begin{pgfscope}%
\pgfpathrectangle{\pgfqpoint{1.432000in}{0.528000in}}{\pgfqpoint{3.696000in}{3.696000in}} %
\pgfusepath{clip}%
\pgfsetbuttcap%
\pgfsetroundjoin%
\definecolor{currentfill}{rgb}{0.282656,0.100196,0.422160}%
\pgfsetfillcolor{currentfill}%
\pgfsetlinewidth{0.000000pt}%
\definecolor{currentstroke}{rgb}{0.000000,0.000000,0.000000}%
\pgfsetstrokecolor{currentstroke}%
\pgfsetdash{}{0pt}%
\pgfpathmoveto{\pgfqpoint{3.765758in}{1.017194in}}%
\pgfpathlineto{\pgfqpoint{3.584687in}{1.107730in}}%
\pgfpathlineto{\pgfqpoint{3.584687in}{1.097813in}}%
\pgfpathlineto{\pgfqpoint{3.550968in}{1.129548in}}%
\pgfpathlineto{\pgfqpoint{3.596588in}{1.121614in}}%
\pgfpathlineto{\pgfqpoint{3.588654in}{1.115664in}}%
\pgfpathlineto{\pgfqpoint{3.769725in}{1.025128in}}%
\pgfpathlineto{\pgfqpoint{3.765758in}{1.017194in}}%
\pgfusepath{fill}%
\end{pgfscope}%
\begin{pgfscope}%
\pgfpathrectangle{\pgfqpoint{1.432000in}{0.528000in}}{\pgfqpoint{3.696000in}{3.696000in}} %
\pgfusepath{clip}%
\pgfsetbuttcap%
\pgfsetroundjoin%
\definecolor{currentfill}{rgb}{0.171176,0.452530,0.557965}%
\pgfsetfillcolor{currentfill}%
\pgfsetlinewidth{0.000000pt}%
\definecolor{currentstroke}{rgb}{0.000000,0.000000,0.000000}%
\pgfsetstrokecolor{currentstroke}%
\pgfsetdash{}{0pt}%
\pgfpathmoveto{\pgfqpoint{3.876129in}{1.016726in}}%
\pgfpathlineto{\pgfqpoint{3.699272in}{1.016726in}}%
\pgfpathlineto{\pgfqpoint{3.703707in}{1.007856in}}%
\pgfpathlineto{\pgfqpoint{3.659355in}{1.021161in}}%
\pgfpathlineto{\pgfqpoint{3.703707in}{1.034467in}}%
\pgfpathlineto{\pgfqpoint{3.699272in}{1.025596in}}%
\pgfpathlineto{\pgfqpoint{3.876129in}{1.025596in}}%
\pgfpathlineto{\pgfqpoint{3.876129in}{1.016726in}}%
\pgfusepath{fill}%
\end{pgfscope}%
\begin{pgfscope}%
\pgfpathrectangle{\pgfqpoint{1.432000in}{0.528000in}}{\pgfqpoint{3.696000in}{3.696000in}} %
\pgfusepath{clip}%
\pgfsetbuttcap%
\pgfsetroundjoin%
\definecolor{currentfill}{rgb}{0.255645,0.260703,0.528312}%
\pgfsetfillcolor{currentfill}%
\pgfsetlinewidth{0.000000pt}%
\definecolor{currentstroke}{rgb}{0.000000,0.000000,0.000000}%
\pgfsetstrokecolor{currentstroke}%
\pgfsetdash{}{0pt}%
\pgfpathmoveto{\pgfqpoint{3.984516in}{1.016726in}}%
\pgfpathlineto{\pgfqpoint{3.807659in}{1.016726in}}%
\pgfpathlineto{\pgfqpoint{3.812094in}{1.007856in}}%
\pgfpathlineto{\pgfqpoint{3.767742in}{1.021161in}}%
\pgfpathlineto{\pgfqpoint{3.812094in}{1.034467in}}%
\pgfpathlineto{\pgfqpoint{3.807659in}{1.025596in}}%
\pgfpathlineto{\pgfqpoint{3.984516in}{1.025596in}}%
\pgfpathlineto{\pgfqpoint{3.984516in}{1.016726in}}%
\pgfusepath{fill}%
\end{pgfscope}%
\begin{pgfscope}%
\pgfpathrectangle{\pgfqpoint{1.432000in}{0.528000in}}{\pgfqpoint{3.696000in}{3.696000in}} %
\pgfusepath{clip}%
\pgfsetbuttcap%
\pgfsetroundjoin%
\definecolor{currentfill}{rgb}{0.276022,0.044167,0.370164}%
\pgfsetfillcolor{currentfill}%
\pgfsetlinewidth{0.000000pt}%
\definecolor{currentstroke}{rgb}{0.000000,0.000000,0.000000}%
\pgfsetstrokecolor{currentstroke}%
\pgfsetdash{}{0pt}%
\pgfpathmoveto{\pgfqpoint{4.094306in}{1.016954in}}%
\pgfpathlineto{\pgfqpoint{3.807013in}{0.921189in}}%
\pgfpathlineto{\pgfqpoint{3.814026in}{0.914177in}}%
\pgfpathlineto{\pgfqpoint{3.767742in}{0.912774in}}%
\pgfpathlineto{\pgfqpoint{3.805610in}{0.939422in}}%
\pgfpathlineto{\pgfqpoint{3.804208in}{0.929605in}}%
\pgfpathlineto{\pgfqpoint{4.091501in}{1.025369in}}%
\pgfpathlineto{\pgfqpoint{4.094306in}{1.016954in}}%
\pgfusepath{fill}%
\end{pgfscope}%
\begin{pgfscope}%
\pgfpathrectangle{\pgfqpoint{1.432000in}{0.528000in}}{\pgfqpoint{3.696000in}{3.696000in}} %
\pgfusepath{clip}%
\pgfsetbuttcap%
\pgfsetroundjoin%
\definecolor{currentfill}{rgb}{0.281412,0.155834,0.469201}%
\pgfsetfillcolor{currentfill}%
\pgfsetlinewidth{0.000000pt}%
\definecolor{currentstroke}{rgb}{0.000000,0.000000,0.000000}%
\pgfsetstrokecolor{currentstroke}%
\pgfsetdash{}{0pt}%
\pgfpathmoveto{\pgfqpoint{4.092903in}{1.016726in}}%
\pgfpathlineto{\pgfqpoint{3.807659in}{1.016726in}}%
\pgfpathlineto{\pgfqpoint{3.812094in}{1.007856in}}%
\pgfpathlineto{\pgfqpoint{3.767742in}{1.021161in}}%
\pgfpathlineto{\pgfqpoint{3.812094in}{1.034467in}}%
\pgfpathlineto{\pgfqpoint{3.807659in}{1.025596in}}%
\pgfpathlineto{\pgfqpoint{4.092903in}{1.025596in}}%
\pgfpathlineto{\pgfqpoint{4.092903in}{1.016726in}}%
\pgfusepath{fill}%
\end{pgfscope}%
\begin{pgfscope}%
\pgfpathrectangle{\pgfqpoint{1.432000in}{0.528000in}}{\pgfqpoint{3.696000in}{3.696000in}} %
\pgfusepath{clip}%
\pgfsetbuttcap%
\pgfsetroundjoin%
\definecolor{currentfill}{rgb}{0.233603,0.313828,0.543914}%
\pgfsetfillcolor{currentfill}%
\pgfsetlinewidth{0.000000pt}%
\definecolor{currentstroke}{rgb}{0.000000,0.000000,0.000000}%
\pgfsetstrokecolor{currentstroke}%
\pgfsetdash{}{0pt}%
\pgfpathmoveto{\pgfqpoint{4.202693in}{1.016954in}}%
\pgfpathlineto{\pgfqpoint{3.915400in}{0.921189in}}%
\pgfpathlineto{\pgfqpoint{3.922413in}{0.914177in}}%
\pgfpathlineto{\pgfqpoint{3.876129in}{0.912774in}}%
\pgfpathlineto{\pgfqpoint{3.913997in}{0.939422in}}%
\pgfpathlineto{\pgfqpoint{3.912595in}{0.929605in}}%
\pgfpathlineto{\pgfqpoint{4.199888in}{1.025369in}}%
\pgfpathlineto{\pgfqpoint{4.202693in}{1.016954in}}%
\pgfusepath{fill}%
\end{pgfscope}%
\begin{pgfscope}%
\pgfpathrectangle{\pgfqpoint{1.432000in}{0.528000in}}{\pgfqpoint{3.696000in}{3.696000in}} %
\pgfusepath{clip}%
\pgfsetbuttcap%
\pgfsetroundjoin%
\definecolor{currentfill}{rgb}{0.237441,0.305202,0.541921}%
\pgfsetfillcolor{currentfill}%
\pgfsetlinewidth{0.000000pt}%
\definecolor{currentstroke}{rgb}{0.000000,0.000000,0.000000}%
\pgfsetstrokecolor{currentstroke}%
\pgfsetdash{}{0pt}%
\pgfpathmoveto{\pgfqpoint{4.311080in}{1.016954in}}%
\pgfpathlineto{\pgfqpoint{4.023787in}{0.921189in}}%
\pgfpathlineto{\pgfqpoint{4.030800in}{0.914177in}}%
\pgfpathlineto{\pgfqpoint{3.984516in}{0.912774in}}%
\pgfpathlineto{\pgfqpoint{4.022385in}{0.939422in}}%
\pgfpathlineto{\pgfqpoint{4.020982in}{0.929605in}}%
\pgfpathlineto{\pgfqpoint{4.308275in}{1.025369in}}%
\pgfpathlineto{\pgfqpoint{4.311080in}{1.016954in}}%
\pgfusepath{fill}%
\end{pgfscope}%
\begin{pgfscope}%
\pgfpathrectangle{\pgfqpoint{1.432000in}{0.528000in}}{\pgfqpoint{3.696000in}{3.696000in}} %
\pgfusepath{clip}%
\pgfsetbuttcap%
\pgfsetroundjoin%
\definecolor{currentfill}{rgb}{0.243113,0.292092,0.538516}%
\pgfsetfillcolor{currentfill}%
\pgfsetlinewidth{0.000000pt}%
\definecolor{currentstroke}{rgb}{0.000000,0.000000,0.000000}%
\pgfsetstrokecolor{currentstroke}%
\pgfsetdash{}{0pt}%
\pgfpathmoveto{\pgfqpoint{4.419467in}{1.016954in}}%
\pgfpathlineto{\pgfqpoint{4.132174in}{0.921189in}}%
\pgfpathlineto{\pgfqpoint{4.139187in}{0.914177in}}%
\pgfpathlineto{\pgfqpoint{4.092903in}{0.912774in}}%
\pgfpathlineto{\pgfqpoint{4.130772in}{0.939422in}}%
\pgfpathlineto{\pgfqpoint{4.129369in}{0.929605in}}%
\pgfpathlineto{\pgfqpoint{4.416662in}{1.025369in}}%
\pgfpathlineto{\pgfqpoint{4.419467in}{1.016954in}}%
\pgfusepath{fill}%
\end{pgfscope}%
\begin{pgfscope}%
\pgfpathrectangle{\pgfqpoint{1.432000in}{0.528000in}}{\pgfqpoint{3.696000in}{3.696000in}} %
\pgfusepath{clip}%
\pgfsetbuttcap%
\pgfsetroundjoin%
\definecolor{currentfill}{rgb}{0.263663,0.237631,0.518762}%
\pgfsetfillcolor{currentfill}%
\pgfsetlinewidth{0.000000pt}%
\definecolor{currentstroke}{rgb}{0.000000,0.000000,0.000000}%
\pgfsetstrokecolor{currentstroke}%
\pgfsetdash{}{0pt}%
\pgfpathmoveto{\pgfqpoint{4.527854in}{1.016954in}}%
\pgfpathlineto{\pgfqpoint{4.240561in}{0.921189in}}%
\pgfpathlineto{\pgfqpoint{4.247574in}{0.914177in}}%
\pgfpathlineto{\pgfqpoint{4.201290in}{0.912774in}}%
\pgfpathlineto{\pgfqpoint{4.239159in}{0.939422in}}%
\pgfpathlineto{\pgfqpoint{4.237756in}{0.929605in}}%
\pgfpathlineto{\pgfqpoint{4.525049in}{1.025369in}}%
\pgfpathlineto{\pgfqpoint{4.527854in}{1.016954in}}%
\pgfusepath{fill}%
\end{pgfscope}%
\begin{pgfscope}%
\pgfpathrectangle{\pgfqpoint{1.432000in}{0.528000in}}{\pgfqpoint{3.696000in}{3.696000in}} %
\pgfusepath{clip}%
\pgfsetbuttcap%
\pgfsetroundjoin%
\definecolor{currentfill}{rgb}{0.282910,0.105393,0.426902}%
\pgfsetfillcolor{currentfill}%
\pgfsetlinewidth{0.000000pt}%
\definecolor{currentstroke}{rgb}{0.000000,0.000000,0.000000}%
\pgfsetstrokecolor{currentstroke}%
\pgfsetdash{}{0pt}%
\pgfpathmoveto{\pgfqpoint{4.528435in}{1.017194in}}%
\pgfpathlineto{\pgfqpoint{4.347364in}{0.926659in}}%
\pgfpathlineto{\pgfqpoint{4.355297in}{0.920708in}}%
\pgfpathlineto{\pgfqpoint{4.309677in}{0.912774in}}%
\pgfpathlineto{\pgfqpoint{4.343397in}{0.944510in}}%
\pgfpathlineto{\pgfqpoint{4.343397in}{0.934592in}}%
\pgfpathlineto{\pgfqpoint{4.524468in}{1.025128in}}%
\pgfpathlineto{\pgfqpoint{4.528435in}{1.017194in}}%
\pgfusepath{fill}%
\end{pgfscope}%
\begin{pgfscope}%
\pgfpathrectangle{\pgfqpoint{1.432000in}{0.528000in}}{\pgfqpoint{3.696000in}{3.696000in}} %
\pgfusepath{clip}%
\pgfsetbuttcap%
\pgfsetroundjoin%
\definecolor{currentfill}{rgb}{0.257322,0.256130,0.526563}%
\pgfsetfillcolor{currentfill}%
\pgfsetlinewidth{0.000000pt}%
\definecolor{currentstroke}{rgb}{0.000000,0.000000,0.000000}%
\pgfsetstrokecolor{currentstroke}%
\pgfsetdash{}{0pt}%
\pgfpathmoveto{\pgfqpoint{4.636822in}{1.017194in}}%
\pgfpathlineto{\pgfqpoint{4.455751in}{0.926659in}}%
\pgfpathlineto{\pgfqpoint{4.463685in}{0.920708in}}%
\pgfpathlineto{\pgfqpoint{4.418065in}{0.912774in}}%
\pgfpathlineto{\pgfqpoint{4.451784in}{0.944510in}}%
\pgfpathlineto{\pgfqpoint{4.451784in}{0.934592in}}%
\pgfpathlineto{\pgfqpoint{4.632855in}{1.025128in}}%
\pgfpathlineto{\pgfqpoint{4.636822in}{1.017194in}}%
\pgfusepath{fill}%
\end{pgfscope}%
\begin{pgfscope}%
\pgfpathrectangle{\pgfqpoint{1.432000in}{0.528000in}}{\pgfqpoint{3.696000in}{3.696000in}} %
\pgfusepath{clip}%
\pgfsetbuttcap%
\pgfsetroundjoin%
\definecolor{currentfill}{rgb}{0.278826,0.175490,0.483397}%
\pgfsetfillcolor{currentfill}%
\pgfsetlinewidth{0.000000pt}%
\definecolor{currentstroke}{rgb}{0.000000,0.000000,0.000000}%
\pgfsetstrokecolor{currentstroke}%
\pgfsetdash{}{0pt}%
\pgfpathmoveto{\pgfqpoint{4.745209in}{1.017194in}}%
\pgfpathlineto{\pgfqpoint{4.564138in}{0.926659in}}%
\pgfpathlineto{\pgfqpoint{4.572072in}{0.920708in}}%
\pgfpathlineto{\pgfqpoint{4.526452in}{0.912774in}}%
\pgfpathlineto{\pgfqpoint{4.560171in}{0.944510in}}%
\pgfpathlineto{\pgfqpoint{4.560171in}{0.934592in}}%
\pgfpathlineto{\pgfqpoint{4.741242in}{1.025128in}}%
\pgfpathlineto{\pgfqpoint{4.745209in}{1.017194in}}%
\pgfusepath{fill}%
\end{pgfscope}%
\begin{pgfscope}%
\pgfpathrectangle{\pgfqpoint{1.432000in}{0.528000in}}{\pgfqpoint{3.696000in}{3.696000in}} %
\pgfusepath{clip}%
\pgfsetbuttcap%
\pgfsetroundjoin%
\definecolor{currentfill}{rgb}{0.277018,0.050344,0.375715}%
\pgfsetfillcolor{currentfill}%
\pgfsetlinewidth{0.000000pt}%
\definecolor{currentstroke}{rgb}{0.000000,0.000000,0.000000}%
\pgfsetstrokecolor{currentstroke}%
\pgfsetdash{}{0pt}%
\pgfpathmoveto{\pgfqpoint{4.746362in}{1.018025in}}%
\pgfpathlineto{\pgfqpoint{4.666200in}{0.937863in}}%
\pgfpathlineto{\pgfqpoint{4.675609in}{0.934727in}}%
\pgfpathlineto{\pgfqpoint{4.634839in}{0.912774in}}%
\pgfpathlineto{\pgfqpoint{4.656792in}{0.953544in}}%
\pgfpathlineto{\pgfqpoint{4.659928in}{0.944136in}}%
\pgfpathlineto{\pgfqpoint{4.740090in}{1.024297in}}%
\pgfpathlineto{\pgfqpoint{4.746362in}{1.018025in}}%
\pgfusepath{fill}%
\end{pgfscope}%
\begin{pgfscope}%
\pgfpathrectangle{\pgfqpoint{1.432000in}{0.528000in}}{\pgfqpoint{3.696000in}{3.696000in}} %
\pgfusepath{clip}%
\pgfsetbuttcap%
\pgfsetroundjoin%
\definecolor{currentfill}{rgb}{0.267004,0.004874,0.329415}%
\pgfsetfillcolor{currentfill}%
\pgfsetlinewidth{0.000000pt}%
\definecolor{currentstroke}{rgb}{0.000000,0.000000,0.000000}%
\pgfsetstrokecolor{currentstroke}%
\pgfsetdash{}{0pt}%
\pgfpathmoveto{\pgfqpoint{4.743226in}{1.016726in}}%
\pgfpathlineto{\pgfqpoint{4.674756in}{1.016726in}}%
\pgfpathlineto{\pgfqpoint{4.679191in}{1.007856in}}%
\pgfpathlineto{\pgfqpoint{4.634839in}{1.021161in}}%
\pgfpathlineto{\pgfqpoint{4.679191in}{1.034467in}}%
\pgfpathlineto{\pgfqpoint{4.674756in}{1.025596in}}%
\pgfpathlineto{\pgfqpoint{4.743226in}{1.025596in}}%
\pgfpathlineto{\pgfqpoint{4.743226in}{1.016726in}}%
\pgfusepath{fill}%
\end{pgfscope}%
\begin{pgfscope}%
\pgfpathrectangle{\pgfqpoint{1.432000in}{0.528000in}}{\pgfqpoint{3.696000in}{3.696000in}} %
\pgfusepath{clip}%
\pgfsetbuttcap%
\pgfsetroundjoin%
\definecolor{currentfill}{rgb}{0.282656,0.100196,0.422160}%
\pgfsetfillcolor{currentfill}%
\pgfsetlinewidth{0.000000pt}%
\definecolor{currentstroke}{rgb}{0.000000,0.000000,0.000000}%
\pgfsetstrokecolor{currentstroke}%
\pgfsetdash{}{0pt}%
\pgfpathmoveto{\pgfqpoint{4.854749in}{1.018025in}}%
\pgfpathlineto{\pgfqpoint{4.774587in}{0.937863in}}%
\pgfpathlineto{\pgfqpoint{4.783996in}{0.934727in}}%
\pgfpathlineto{\pgfqpoint{4.743226in}{0.912774in}}%
\pgfpathlineto{\pgfqpoint{4.765179in}{0.953544in}}%
\pgfpathlineto{\pgfqpoint{4.768315in}{0.944136in}}%
\pgfpathlineto{\pgfqpoint{4.848477in}{1.024297in}}%
\pgfpathlineto{\pgfqpoint{4.854749in}{1.018025in}}%
\pgfusepath{fill}%
\end{pgfscope}%
\begin{pgfscope}%
\pgfpathrectangle{\pgfqpoint{1.432000in}{0.528000in}}{\pgfqpoint{3.696000in}{3.696000in}} %
\pgfusepath{clip}%
\pgfsetbuttcap%
\pgfsetroundjoin%
\definecolor{currentfill}{rgb}{0.270595,0.214069,0.507052}%
\pgfsetfillcolor{currentfill}%
\pgfsetlinewidth{0.000000pt}%
\definecolor{currentstroke}{rgb}{0.000000,0.000000,0.000000}%
\pgfsetstrokecolor{currentstroke}%
\pgfsetdash{}{0pt}%
\pgfpathmoveto{\pgfqpoint{4.851613in}{1.016726in}}%
\pgfpathlineto{\pgfqpoint{4.783143in}{1.016726in}}%
\pgfpathlineto{\pgfqpoint{4.787578in}{1.007856in}}%
\pgfpathlineto{\pgfqpoint{4.743226in}{1.021161in}}%
\pgfpathlineto{\pgfqpoint{4.787578in}{1.034467in}}%
\pgfpathlineto{\pgfqpoint{4.783143in}{1.025596in}}%
\pgfpathlineto{\pgfqpoint{4.851613in}{1.025596in}}%
\pgfpathlineto{\pgfqpoint{4.851613in}{1.016726in}}%
\pgfusepath{fill}%
\end{pgfscope}%
\begin{pgfscope}%
\pgfpathrectangle{\pgfqpoint{1.432000in}{0.528000in}}{\pgfqpoint{3.696000in}{3.696000in}} %
\pgfusepath{clip}%
\pgfsetbuttcap%
\pgfsetroundjoin%
\definecolor{currentfill}{rgb}{0.276022,0.044167,0.370164}%
\pgfsetfillcolor{currentfill}%
\pgfsetlinewidth{0.000000pt}%
\definecolor{currentstroke}{rgb}{0.000000,0.000000,0.000000}%
\pgfsetstrokecolor{currentstroke}%
\pgfsetdash{}{0pt}%
\pgfpathmoveto{\pgfqpoint{4.856048in}{1.021161in}}%
\pgfpathlineto{\pgfqpoint{4.853831in}{1.025002in}}%
\pgfpathlineto{\pgfqpoint{4.849395in}{1.025002in}}%
\pgfpathlineto{\pgfqpoint{4.847178in}{1.021161in}}%
\pgfpathlineto{\pgfqpoint{4.849395in}{1.017320in}}%
\pgfpathlineto{\pgfqpoint{4.853831in}{1.017320in}}%
\pgfpathlineto{\pgfqpoint{4.856048in}{1.021161in}}%
\pgfpathlineto{\pgfqpoint{4.853831in}{1.025002in}}%
\pgfusepath{fill}%
\end{pgfscope}%
\begin{pgfscope}%
\pgfpathrectangle{\pgfqpoint{1.432000in}{0.528000in}}{\pgfqpoint{3.696000in}{3.696000in}} %
\pgfusepath{clip}%
\pgfsetbuttcap%
\pgfsetroundjoin%
\definecolor{currentfill}{rgb}{0.278012,0.180367,0.486697}%
\pgfsetfillcolor{currentfill}%
\pgfsetlinewidth{0.000000pt}%
\definecolor{currentstroke}{rgb}{0.000000,0.000000,0.000000}%
\pgfsetstrokecolor{currentstroke}%
\pgfsetdash{}{0pt}%
\pgfpathmoveto{\pgfqpoint{4.960000in}{1.016726in}}%
\pgfpathlineto{\pgfqpoint{4.891530in}{1.016726in}}%
\pgfpathlineto{\pgfqpoint{4.895965in}{1.007856in}}%
\pgfpathlineto{\pgfqpoint{4.851613in}{1.021161in}}%
\pgfpathlineto{\pgfqpoint{4.895965in}{1.034467in}}%
\pgfpathlineto{\pgfqpoint{4.891530in}{1.025596in}}%
\pgfpathlineto{\pgfqpoint{4.960000in}{1.025596in}}%
\pgfpathlineto{\pgfqpoint{4.960000in}{1.016726in}}%
\pgfusepath{fill}%
\end{pgfscope}%
\begin{pgfscope}%
\pgfpathrectangle{\pgfqpoint{1.432000in}{0.528000in}}{\pgfqpoint{3.696000in}{3.696000in}} %
\pgfusepath{clip}%
\pgfsetbuttcap%
\pgfsetroundjoin%
\definecolor{currentfill}{rgb}{0.199430,0.387607,0.554642}%
\pgfsetfillcolor{currentfill}%
\pgfsetlinewidth{0.000000pt}%
\definecolor{currentstroke}{rgb}{0.000000,0.000000,0.000000}%
\pgfsetstrokecolor{currentstroke}%
\pgfsetdash{}{0pt}%
\pgfpathmoveto{\pgfqpoint{4.964435in}{1.021161in}}%
\pgfpathlineto{\pgfqpoint{4.962218in}{1.025002in}}%
\pgfpathlineto{\pgfqpoint{4.957782in}{1.025002in}}%
\pgfpathlineto{\pgfqpoint{4.955565in}{1.021161in}}%
\pgfpathlineto{\pgfqpoint{4.957782in}{1.017320in}}%
\pgfpathlineto{\pgfqpoint{4.962218in}{1.017320in}}%
\pgfpathlineto{\pgfqpoint{4.964435in}{1.021161in}}%
\pgfpathlineto{\pgfqpoint{4.962218in}{1.025002in}}%
\pgfusepath{fill}%
\end{pgfscope}%
\begin{pgfscope}%
\pgfpathrectangle{\pgfqpoint{1.432000in}{0.528000in}}{\pgfqpoint{3.696000in}{3.696000in}} %
\pgfusepath{clip}%
\pgfsetbuttcap%
\pgfsetroundjoin%
\definecolor{currentfill}{rgb}{0.280267,0.073417,0.397163}%
\pgfsetfillcolor{currentfill}%
\pgfsetlinewidth{0.000000pt}%
\definecolor{currentstroke}{rgb}{0.000000,0.000000,0.000000}%
\pgfsetstrokecolor{currentstroke}%
\pgfsetdash{}{0pt}%
\pgfpathmoveto{\pgfqpoint{1.604435in}{1.129548in}}%
\pgfpathlineto{\pgfqpoint{1.602218in}{1.133389in}}%
\pgfpathlineto{\pgfqpoint{1.597782in}{1.133389in}}%
\pgfpathlineto{\pgfqpoint{1.595565in}{1.129548in}}%
\pgfpathlineto{\pgfqpoint{1.597782in}{1.125707in}}%
\pgfpathlineto{\pgfqpoint{1.602218in}{1.125707in}}%
\pgfpathlineto{\pgfqpoint{1.604435in}{1.129548in}}%
\pgfpathlineto{\pgfqpoint{1.602218in}{1.133389in}}%
\pgfusepath{fill}%
\end{pgfscope}%
\begin{pgfscope}%
\pgfpathrectangle{\pgfqpoint{1.432000in}{0.528000in}}{\pgfqpoint{3.696000in}{3.696000in}} %
\pgfusepath{clip}%
\pgfsetbuttcap%
\pgfsetroundjoin%
\definecolor{currentfill}{rgb}{0.268510,0.009605,0.335427}%
\pgfsetfillcolor{currentfill}%
\pgfsetlinewidth{0.000000pt}%
\definecolor{currentstroke}{rgb}{0.000000,0.000000,0.000000}%
\pgfsetstrokecolor{currentstroke}%
\pgfsetdash{}{0pt}%
\pgfpathmoveto{\pgfqpoint{1.816774in}{1.125113in}}%
\pgfpathlineto{\pgfqpoint{1.748304in}{1.125113in}}%
\pgfpathlineto{\pgfqpoint{1.752739in}{1.116243in}}%
\pgfpathlineto{\pgfqpoint{1.708387in}{1.129548in}}%
\pgfpathlineto{\pgfqpoint{1.752739in}{1.142854in}}%
\pgfpathlineto{\pgfqpoint{1.748304in}{1.133984in}}%
\pgfpathlineto{\pgfqpoint{1.816774in}{1.133984in}}%
\pgfpathlineto{\pgfqpoint{1.816774in}{1.125113in}}%
\pgfusepath{fill}%
\end{pgfscope}%
\begin{pgfscope}%
\pgfpathrectangle{\pgfqpoint{1.432000in}{0.528000in}}{\pgfqpoint{3.696000in}{3.696000in}} %
\pgfusepath{clip}%
\pgfsetbuttcap%
\pgfsetroundjoin%
\definecolor{currentfill}{rgb}{0.262138,0.242286,0.520837}%
\pgfsetfillcolor{currentfill}%
\pgfsetlinewidth{0.000000pt}%
\definecolor{currentstroke}{rgb}{0.000000,0.000000,0.000000}%
\pgfsetstrokecolor{currentstroke}%
\pgfsetdash{}{0pt}%
\pgfpathmoveto{\pgfqpoint{1.925161in}{1.125113in}}%
\pgfpathlineto{\pgfqpoint{1.748304in}{1.125113in}}%
\pgfpathlineto{\pgfqpoint{1.752739in}{1.116243in}}%
\pgfpathlineto{\pgfqpoint{1.708387in}{1.129548in}}%
\pgfpathlineto{\pgfqpoint{1.752739in}{1.142854in}}%
\pgfpathlineto{\pgfqpoint{1.748304in}{1.133984in}}%
\pgfpathlineto{\pgfqpoint{1.925161in}{1.133984in}}%
\pgfpathlineto{\pgfqpoint{1.925161in}{1.125113in}}%
\pgfusepath{fill}%
\end{pgfscope}%
\begin{pgfscope}%
\pgfpathrectangle{\pgfqpoint{1.432000in}{0.528000in}}{\pgfqpoint{3.696000in}{3.696000in}} %
\pgfusepath{clip}%
\pgfsetbuttcap%
\pgfsetroundjoin%
\definecolor{currentfill}{rgb}{0.281887,0.150881,0.465405}%
\pgfsetfillcolor{currentfill}%
\pgfsetlinewidth{0.000000pt}%
\definecolor{currentstroke}{rgb}{0.000000,0.000000,0.000000}%
\pgfsetstrokecolor{currentstroke}%
\pgfsetdash{}{0pt}%
\pgfpathmoveto{\pgfqpoint{2.035532in}{1.125581in}}%
\pgfpathlineto{\pgfqpoint{1.854460in}{1.035046in}}%
\pgfpathlineto{\pgfqpoint{1.862394in}{1.029095in}}%
\pgfpathlineto{\pgfqpoint{1.816774in}{1.021161in}}%
\pgfpathlineto{\pgfqpoint{1.850493in}{1.052897in}}%
\pgfpathlineto{\pgfqpoint{1.850493in}{1.042980in}}%
\pgfpathlineto{\pgfqpoint{2.031565in}{1.133515in}}%
\pgfpathlineto{\pgfqpoint{2.035532in}{1.125581in}}%
\pgfusepath{fill}%
\end{pgfscope}%
\begin{pgfscope}%
\pgfpathrectangle{\pgfqpoint{1.432000in}{0.528000in}}{\pgfqpoint{3.696000in}{3.696000in}} %
\pgfusepath{clip}%
\pgfsetbuttcap%
\pgfsetroundjoin%
\definecolor{currentfill}{rgb}{0.225863,0.330805,0.547314}%
\pgfsetfillcolor{currentfill}%
\pgfsetlinewidth{0.000000pt}%
\definecolor{currentstroke}{rgb}{0.000000,0.000000,0.000000}%
\pgfsetstrokecolor{currentstroke}%
\pgfsetdash{}{0pt}%
\pgfpathmoveto{\pgfqpoint{2.033548in}{1.125113in}}%
\pgfpathlineto{\pgfqpoint{1.856691in}{1.125113in}}%
\pgfpathlineto{\pgfqpoint{1.861126in}{1.116243in}}%
\pgfpathlineto{\pgfqpoint{1.816774in}{1.129548in}}%
\pgfpathlineto{\pgfqpoint{1.861126in}{1.142854in}}%
\pgfpathlineto{\pgfqpoint{1.856691in}{1.133984in}}%
\pgfpathlineto{\pgfqpoint{2.033548in}{1.133984in}}%
\pgfpathlineto{\pgfqpoint{2.033548in}{1.125113in}}%
\pgfusepath{fill}%
\end{pgfscope}%
\begin{pgfscope}%
\pgfpathrectangle{\pgfqpoint{1.432000in}{0.528000in}}{\pgfqpoint{3.696000in}{3.696000in}} %
\pgfusepath{clip}%
\pgfsetbuttcap%
\pgfsetroundjoin%
\definecolor{currentfill}{rgb}{0.175841,0.441290,0.557685}%
\pgfsetfillcolor{currentfill}%
\pgfsetlinewidth{0.000000pt}%
\definecolor{currentstroke}{rgb}{0.000000,0.000000,0.000000}%
\pgfsetstrokecolor{currentstroke}%
\pgfsetdash{}{0pt}%
\pgfpathmoveto{\pgfqpoint{2.143919in}{1.125581in}}%
\pgfpathlineto{\pgfqpoint{1.962847in}{1.035046in}}%
\pgfpathlineto{\pgfqpoint{1.970781in}{1.029095in}}%
\pgfpathlineto{\pgfqpoint{1.925161in}{1.021161in}}%
\pgfpathlineto{\pgfqpoint{1.958880in}{1.052897in}}%
\pgfpathlineto{\pgfqpoint{1.958880in}{1.042980in}}%
\pgfpathlineto{\pgfqpoint{2.139952in}{1.133515in}}%
\pgfpathlineto{\pgfqpoint{2.143919in}{1.125581in}}%
\pgfusepath{fill}%
\end{pgfscope}%
\begin{pgfscope}%
\pgfpathrectangle{\pgfqpoint{1.432000in}{0.528000in}}{\pgfqpoint{3.696000in}{3.696000in}} %
\pgfusepath{clip}%
\pgfsetbuttcap%
\pgfsetroundjoin%
\definecolor{currentfill}{rgb}{0.280267,0.073417,0.397163}%
\pgfsetfillcolor{currentfill}%
\pgfsetlinewidth{0.000000pt}%
\definecolor{currentstroke}{rgb}{0.000000,0.000000,0.000000}%
\pgfsetstrokecolor{currentstroke}%
\pgfsetdash{}{0pt}%
\pgfpathmoveto{\pgfqpoint{2.141935in}{1.125113in}}%
\pgfpathlineto{\pgfqpoint{1.965078in}{1.125113in}}%
\pgfpathlineto{\pgfqpoint{1.969513in}{1.116243in}}%
\pgfpathlineto{\pgfqpoint{1.925161in}{1.129548in}}%
\pgfpathlineto{\pgfqpoint{1.969513in}{1.142854in}}%
\pgfpathlineto{\pgfqpoint{1.965078in}{1.133984in}}%
\pgfpathlineto{\pgfqpoint{2.141935in}{1.133984in}}%
\pgfpathlineto{\pgfqpoint{2.141935in}{1.125113in}}%
\pgfusepath{fill}%
\end{pgfscope}%
\begin{pgfscope}%
\pgfpathrectangle{\pgfqpoint{1.432000in}{0.528000in}}{\pgfqpoint{3.696000in}{3.696000in}} %
\pgfusepath{clip}%
\pgfsetbuttcap%
\pgfsetroundjoin%
\definecolor{currentfill}{rgb}{0.274128,0.199721,0.498911}%
\pgfsetfillcolor{currentfill}%
\pgfsetlinewidth{0.000000pt}%
\definecolor{currentstroke}{rgb}{0.000000,0.000000,0.000000}%
\pgfsetstrokecolor{currentstroke}%
\pgfsetdash{}{0pt}%
\pgfpathmoveto{\pgfqpoint{2.252306in}{1.125581in}}%
\pgfpathlineto{\pgfqpoint{2.071235in}{1.035046in}}%
\pgfpathlineto{\pgfqpoint{2.079168in}{1.029095in}}%
\pgfpathlineto{\pgfqpoint{2.033548in}{1.021161in}}%
\pgfpathlineto{\pgfqpoint{2.067268in}{1.052897in}}%
\pgfpathlineto{\pgfqpoint{2.067268in}{1.042980in}}%
\pgfpathlineto{\pgfqpoint{2.248339in}{1.133515in}}%
\pgfpathlineto{\pgfqpoint{2.252306in}{1.125581in}}%
\pgfusepath{fill}%
\end{pgfscope}%
\begin{pgfscope}%
\pgfpathrectangle{\pgfqpoint{1.432000in}{0.528000in}}{\pgfqpoint{3.696000in}{3.696000in}} %
\pgfusepath{clip}%
\pgfsetbuttcap%
\pgfsetroundjoin%
\definecolor{currentfill}{rgb}{0.267004,0.004874,0.329415}%
\pgfsetfillcolor{currentfill}%
\pgfsetlinewidth{0.000000pt}%
\definecolor{currentstroke}{rgb}{0.000000,0.000000,0.000000}%
\pgfsetstrokecolor{currentstroke}%
\pgfsetdash{}{0pt}%
\pgfpathmoveto{\pgfqpoint{2.250323in}{1.125113in}}%
\pgfpathlineto{\pgfqpoint{2.073465in}{1.125113in}}%
\pgfpathlineto{\pgfqpoint{2.077900in}{1.116243in}}%
\pgfpathlineto{\pgfqpoint{2.033548in}{1.129548in}}%
\pgfpathlineto{\pgfqpoint{2.077900in}{1.142854in}}%
\pgfpathlineto{\pgfqpoint{2.073465in}{1.133984in}}%
\pgfpathlineto{\pgfqpoint{2.250323in}{1.133984in}}%
\pgfpathlineto{\pgfqpoint{2.250323in}{1.125113in}}%
\pgfusepath{fill}%
\end{pgfscope}%
\begin{pgfscope}%
\pgfpathrectangle{\pgfqpoint{1.432000in}{0.528000in}}{\pgfqpoint{3.696000in}{3.696000in}} %
\pgfusepath{clip}%
\pgfsetbuttcap%
\pgfsetroundjoin%
\definecolor{currentfill}{rgb}{0.282910,0.105393,0.426902}%
\pgfsetfillcolor{currentfill}%
\pgfsetlinewidth{0.000000pt}%
\definecolor{currentstroke}{rgb}{0.000000,0.000000,0.000000}%
\pgfsetstrokecolor{currentstroke}%
\pgfsetdash{}{0pt}%
\pgfpathmoveto{\pgfqpoint{2.360112in}{1.125341in}}%
\pgfpathlineto{\pgfqpoint{2.072819in}{1.029576in}}%
\pgfpathlineto{\pgfqpoint{2.079832in}{1.022564in}}%
\pgfpathlineto{\pgfqpoint{2.033548in}{1.021161in}}%
\pgfpathlineto{\pgfqpoint{2.071417in}{1.047809in}}%
\pgfpathlineto{\pgfqpoint{2.070014in}{1.037992in}}%
\pgfpathlineto{\pgfqpoint{2.357307in}{1.133756in}}%
\pgfpathlineto{\pgfqpoint{2.360112in}{1.125341in}}%
\pgfusepath{fill}%
\end{pgfscope}%
\begin{pgfscope}%
\pgfpathrectangle{\pgfqpoint{1.432000in}{0.528000in}}{\pgfqpoint{3.696000in}{3.696000in}} %
\pgfusepath{clip}%
\pgfsetbuttcap%
\pgfsetroundjoin%
\definecolor{currentfill}{rgb}{0.281446,0.084320,0.407414}%
\pgfsetfillcolor{currentfill}%
\pgfsetlinewidth{0.000000pt}%
\definecolor{currentstroke}{rgb}{0.000000,0.000000,0.000000}%
\pgfsetstrokecolor{currentstroke}%
\pgfsetdash{}{0pt}%
\pgfpathmoveto{\pgfqpoint{2.468499in}{1.125341in}}%
\pgfpathlineto{\pgfqpoint{2.181206in}{1.029576in}}%
\pgfpathlineto{\pgfqpoint{2.188219in}{1.022564in}}%
\pgfpathlineto{\pgfqpoint{2.141935in}{1.021161in}}%
\pgfpathlineto{\pgfqpoint{2.179804in}{1.047809in}}%
\pgfpathlineto{\pgfqpoint{2.178401in}{1.037992in}}%
\pgfpathlineto{\pgfqpoint{2.465694in}{1.133756in}}%
\pgfpathlineto{\pgfqpoint{2.468499in}{1.125341in}}%
\pgfusepath{fill}%
\end{pgfscope}%
\begin{pgfscope}%
\pgfpathrectangle{\pgfqpoint{1.432000in}{0.528000in}}{\pgfqpoint{3.696000in}{3.696000in}} %
\pgfusepath{clip}%
\pgfsetbuttcap%
\pgfsetroundjoin%
\definecolor{currentfill}{rgb}{0.252194,0.269783,0.531579}%
\pgfsetfillcolor{currentfill}%
\pgfsetlinewidth{0.000000pt}%
\definecolor{currentstroke}{rgb}{0.000000,0.000000,0.000000}%
\pgfsetstrokecolor{currentstroke}%
\pgfsetdash{}{0pt}%
\pgfpathmoveto{\pgfqpoint{2.467097in}{1.125113in}}%
\pgfpathlineto{\pgfqpoint{2.181852in}{1.125113in}}%
\pgfpathlineto{\pgfqpoint{2.186287in}{1.116243in}}%
\pgfpathlineto{\pgfqpoint{2.141935in}{1.129548in}}%
\pgfpathlineto{\pgfqpoint{2.186287in}{1.142854in}}%
\pgfpathlineto{\pgfqpoint{2.181852in}{1.133984in}}%
\pgfpathlineto{\pgfqpoint{2.467097in}{1.133984in}}%
\pgfpathlineto{\pgfqpoint{2.467097in}{1.125113in}}%
\pgfusepath{fill}%
\end{pgfscope}%
\begin{pgfscope}%
\pgfpathrectangle{\pgfqpoint{1.432000in}{0.528000in}}{\pgfqpoint{3.696000in}{3.696000in}} %
\pgfusepath{clip}%
\pgfsetbuttcap%
\pgfsetroundjoin%
\definecolor{currentfill}{rgb}{0.149039,0.508051,0.557250}%
\pgfsetfillcolor{currentfill}%
\pgfsetlinewidth{0.000000pt}%
\definecolor{currentstroke}{rgb}{0.000000,0.000000,0.000000}%
\pgfsetstrokecolor{currentstroke}%
\pgfsetdash{}{0pt}%
\pgfpathmoveto{\pgfqpoint{2.575484in}{1.125113in}}%
\pgfpathlineto{\pgfqpoint{2.290239in}{1.125113in}}%
\pgfpathlineto{\pgfqpoint{2.294675in}{1.116243in}}%
\pgfpathlineto{\pgfqpoint{2.250323in}{1.129548in}}%
\pgfpathlineto{\pgfqpoint{2.294675in}{1.142854in}}%
\pgfpathlineto{\pgfqpoint{2.290239in}{1.133984in}}%
\pgfpathlineto{\pgfqpoint{2.575484in}{1.133984in}}%
\pgfpathlineto{\pgfqpoint{2.575484in}{1.125113in}}%
\pgfusepath{fill}%
\end{pgfscope}%
\begin{pgfscope}%
\pgfpathrectangle{\pgfqpoint{1.432000in}{0.528000in}}{\pgfqpoint{3.696000in}{3.696000in}} %
\pgfusepath{clip}%
\pgfsetbuttcap%
\pgfsetroundjoin%
\definecolor{currentfill}{rgb}{0.120092,0.600104,0.542530}%
\pgfsetfillcolor{currentfill}%
\pgfsetlinewidth{0.000000pt}%
\definecolor{currentstroke}{rgb}{0.000000,0.000000,0.000000}%
\pgfsetstrokecolor{currentstroke}%
\pgfsetdash{}{0pt}%
\pgfpathmoveto{\pgfqpoint{2.683871in}{1.125113in}}%
\pgfpathlineto{\pgfqpoint{2.398626in}{1.125113in}}%
\pgfpathlineto{\pgfqpoint{2.403062in}{1.116243in}}%
\pgfpathlineto{\pgfqpoint{2.358710in}{1.129548in}}%
\pgfpathlineto{\pgfqpoint{2.403062in}{1.142854in}}%
\pgfpathlineto{\pgfqpoint{2.398626in}{1.133984in}}%
\pgfpathlineto{\pgfqpoint{2.683871in}{1.133984in}}%
\pgfpathlineto{\pgfqpoint{2.683871in}{1.125113in}}%
\pgfusepath{fill}%
\end{pgfscope}%
\begin{pgfscope}%
\pgfpathrectangle{\pgfqpoint{1.432000in}{0.528000in}}{\pgfqpoint{3.696000in}{3.696000in}} %
\pgfusepath{clip}%
\pgfsetbuttcap%
\pgfsetroundjoin%
\definecolor{currentfill}{rgb}{0.120565,0.596422,0.543611}%
\pgfsetfillcolor{currentfill}%
\pgfsetlinewidth{0.000000pt}%
\definecolor{currentstroke}{rgb}{0.000000,0.000000,0.000000}%
\pgfsetstrokecolor{currentstroke}%
\pgfsetdash{}{0pt}%
\pgfpathmoveto{\pgfqpoint{2.792258in}{1.125113in}}%
\pgfpathlineto{\pgfqpoint{2.507014in}{1.125113in}}%
\pgfpathlineto{\pgfqpoint{2.511449in}{1.116243in}}%
\pgfpathlineto{\pgfqpoint{2.467097in}{1.129548in}}%
\pgfpathlineto{\pgfqpoint{2.511449in}{1.142854in}}%
\pgfpathlineto{\pgfqpoint{2.507014in}{1.133984in}}%
\pgfpathlineto{\pgfqpoint{2.792258in}{1.133984in}}%
\pgfpathlineto{\pgfqpoint{2.792258in}{1.125113in}}%
\pgfusepath{fill}%
\end{pgfscope}%
\begin{pgfscope}%
\pgfpathrectangle{\pgfqpoint{1.432000in}{0.528000in}}{\pgfqpoint{3.696000in}{3.696000in}} %
\pgfusepath{clip}%
\pgfsetbuttcap%
\pgfsetroundjoin%
\definecolor{currentfill}{rgb}{0.120092,0.600104,0.542530}%
\pgfsetfillcolor{currentfill}%
\pgfsetlinewidth{0.000000pt}%
\definecolor{currentstroke}{rgb}{0.000000,0.000000,0.000000}%
\pgfsetstrokecolor{currentstroke}%
\pgfsetdash{}{0pt}%
\pgfpathmoveto{\pgfqpoint{2.899243in}{1.125341in}}%
\pgfpathlineto{\pgfqpoint{2.611950in}{1.221105in}}%
\pgfpathlineto{\pgfqpoint{2.613352in}{1.211287in}}%
\pgfpathlineto{\pgfqpoint{2.575484in}{1.237935in}}%
\pgfpathlineto{\pgfqpoint{2.621767in}{1.236533in}}%
\pgfpathlineto{\pgfqpoint{2.614755in}{1.229520in}}%
\pgfpathlineto{\pgfqpoint{2.902048in}{1.133756in}}%
\pgfpathlineto{\pgfqpoint{2.899243in}{1.125341in}}%
\pgfusepath{fill}%
\end{pgfscope}%
\begin{pgfscope}%
\pgfpathrectangle{\pgfqpoint{1.432000in}{0.528000in}}{\pgfqpoint{3.696000in}{3.696000in}} %
\pgfusepath{clip}%
\pgfsetbuttcap%
\pgfsetroundjoin%
\definecolor{currentfill}{rgb}{0.269944,0.014625,0.341379}%
\pgfsetfillcolor{currentfill}%
\pgfsetlinewidth{0.000000pt}%
\definecolor{currentstroke}{rgb}{0.000000,0.000000,0.000000}%
\pgfsetstrokecolor{currentstroke}%
\pgfsetdash{}{0pt}%
\pgfpathmoveto{\pgfqpoint{2.898662in}{1.125581in}}%
\pgfpathlineto{\pgfqpoint{2.717590in}{1.216117in}}%
\pgfpathlineto{\pgfqpoint{2.717590in}{1.206200in}}%
\pgfpathlineto{\pgfqpoint{2.683871in}{1.237935in}}%
\pgfpathlineto{\pgfqpoint{2.729491in}{1.230002in}}%
\pgfpathlineto{\pgfqpoint{2.721557in}{1.224051in}}%
\pgfpathlineto{\pgfqpoint{2.902629in}{1.133515in}}%
\pgfpathlineto{\pgfqpoint{2.898662in}{1.125581in}}%
\pgfusepath{fill}%
\end{pgfscope}%
\begin{pgfscope}%
\pgfpathrectangle{\pgfqpoint{1.432000in}{0.528000in}}{\pgfqpoint{3.696000in}{3.696000in}} %
\pgfusepath{clip}%
\pgfsetbuttcap%
\pgfsetroundjoin%
\definecolor{currentfill}{rgb}{0.132444,0.552216,0.553018}%
\pgfsetfillcolor{currentfill}%
\pgfsetlinewidth{0.000000pt}%
\definecolor{currentstroke}{rgb}{0.000000,0.000000,0.000000}%
\pgfsetstrokecolor{currentstroke}%
\pgfsetdash{}{0pt}%
\pgfpathmoveto{\pgfqpoint{3.007630in}{1.125341in}}%
\pgfpathlineto{\pgfqpoint{2.720337in}{1.221105in}}%
\pgfpathlineto{\pgfqpoint{2.721739in}{1.211287in}}%
\pgfpathlineto{\pgfqpoint{2.683871in}{1.237935in}}%
\pgfpathlineto{\pgfqpoint{2.730155in}{1.236533in}}%
\pgfpathlineto{\pgfqpoint{2.723142in}{1.229520in}}%
\pgfpathlineto{\pgfqpoint{3.010435in}{1.133756in}}%
\pgfpathlineto{\pgfqpoint{3.007630in}{1.125341in}}%
\pgfusepath{fill}%
\end{pgfscope}%
\begin{pgfscope}%
\pgfpathrectangle{\pgfqpoint{1.432000in}{0.528000in}}{\pgfqpoint{3.696000in}{3.696000in}} %
\pgfusepath{clip}%
\pgfsetbuttcap%
\pgfsetroundjoin%
\definecolor{currentfill}{rgb}{0.269308,0.218818,0.509577}%
\pgfsetfillcolor{currentfill}%
\pgfsetlinewidth{0.000000pt}%
\definecolor{currentstroke}{rgb}{0.000000,0.000000,0.000000}%
\pgfsetstrokecolor{currentstroke}%
\pgfsetdash{}{0pt}%
\pgfpathmoveto{\pgfqpoint{3.007049in}{1.125581in}}%
\pgfpathlineto{\pgfqpoint{2.825977in}{1.216117in}}%
\pgfpathlineto{\pgfqpoint{2.825977in}{1.206200in}}%
\pgfpathlineto{\pgfqpoint{2.792258in}{1.237935in}}%
\pgfpathlineto{\pgfqpoint{2.837878in}{1.230002in}}%
\pgfpathlineto{\pgfqpoint{2.829944in}{1.224051in}}%
\pgfpathlineto{\pgfqpoint{3.011016in}{1.133515in}}%
\pgfpathlineto{\pgfqpoint{3.007049in}{1.125581in}}%
\pgfusepath{fill}%
\end{pgfscope}%
\begin{pgfscope}%
\pgfpathrectangle{\pgfqpoint{1.432000in}{0.528000in}}{\pgfqpoint{3.696000in}{3.696000in}} %
\pgfusepath{clip}%
\pgfsetbuttcap%
\pgfsetroundjoin%
\definecolor{currentfill}{rgb}{0.269308,0.218818,0.509577}%
\pgfsetfillcolor{currentfill}%
\pgfsetlinewidth{0.000000pt}%
\definecolor{currentstroke}{rgb}{0.000000,0.000000,0.000000}%
\pgfsetstrokecolor{currentstroke}%
\pgfsetdash{}{0pt}%
\pgfpathmoveto{\pgfqpoint{3.116017in}{1.125341in}}%
\pgfpathlineto{\pgfqpoint{2.828724in}{1.221105in}}%
\pgfpathlineto{\pgfqpoint{2.830126in}{1.211287in}}%
\pgfpathlineto{\pgfqpoint{2.792258in}{1.237935in}}%
\pgfpathlineto{\pgfqpoint{2.838542in}{1.236533in}}%
\pgfpathlineto{\pgfqpoint{2.831529in}{1.229520in}}%
\pgfpathlineto{\pgfqpoint{3.118822in}{1.133756in}}%
\pgfpathlineto{\pgfqpoint{3.116017in}{1.125341in}}%
\pgfusepath{fill}%
\end{pgfscope}%
\begin{pgfscope}%
\pgfpathrectangle{\pgfqpoint{1.432000in}{0.528000in}}{\pgfqpoint{3.696000in}{3.696000in}} %
\pgfusepath{clip}%
\pgfsetbuttcap%
\pgfsetroundjoin%
\definecolor{currentfill}{rgb}{0.123463,0.581687,0.547445}%
\pgfsetfillcolor{currentfill}%
\pgfsetlinewidth{0.000000pt}%
\definecolor{currentstroke}{rgb}{0.000000,0.000000,0.000000}%
\pgfsetstrokecolor{currentstroke}%
\pgfsetdash{}{0pt}%
\pgfpathmoveto{\pgfqpoint{3.115436in}{1.125581in}}%
\pgfpathlineto{\pgfqpoint{2.934364in}{1.216117in}}%
\pgfpathlineto{\pgfqpoint{2.934364in}{1.206200in}}%
\pgfpathlineto{\pgfqpoint{2.900645in}{1.237935in}}%
\pgfpathlineto{\pgfqpoint{2.946265in}{1.230002in}}%
\pgfpathlineto{\pgfqpoint{2.938331in}{1.224051in}}%
\pgfpathlineto{\pgfqpoint{3.119403in}{1.133515in}}%
\pgfpathlineto{\pgfqpoint{3.115436in}{1.125581in}}%
\pgfusepath{fill}%
\end{pgfscope}%
\begin{pgfscope}%
\pgfpathrectangle{\pgfqpoint{1.432000in}{0.528000in}}{\pgfqpoint{3.696000in}{3.696000in}} %
\pgfusepath{clip}%
\pgfsetbuttcap%
\pgfsetroundjoin%
\definecolor{currentfill}{rgb}{0.180653,0.701402,0.488189}%
\pgfsetfillcolor{currentfill}%
\pgfsetlinewidth{0.000000pt}%
\definecolor{currentstroke}{rgb}{0.000000,0.000000,0.000000}%
\pgfsetstrokecolor{currentstroke}%
\pgfsetdash{}{0pt}%
\pgfpathmoveto{\pgfqpoint{3.223823in}{1.125581in}}%
\pgfpathlineto{\pgfqpoint{3.042751in}{1.216117in}}%
\pgfpathlineto{\pgfqpoint{3.042751in}{1.206200in}}%
\pgfpathlineto{\pgfqpoint{3.009032in}{1.237935in}}%
\pgfpathlineto{\pgfqpoint{3.054652in}{1.230002in}}%
\pgfpathlineto{\pgfqpoint{3.046718in}{1.224051in}}%
\pgfpathlineto{\pgfqpoint{3.227790in}{1.133515in}}%
\pgfpathlineto{\pgfqpoint{3.223823in}{1.125581in}}%
\pgfusepath{fill}%
\end{pgfscope}%
\begin{pgfscope}%
\pgfpathrectangle{\pgfqpoint{1.432000in}{0.528000in}}{\pgfqpoint{3.696000in}{3.696000in}} %
\pgfusepath{clip}%
\pgfsetbuttcap%
\pgfsetroundjoin%
\definecolor{currentfill}{rgb}{0.144759,0.519093,0.556572}%
\pgfsetfillcolor{currentfill}%
\pgfsetlinewidth{0.000000pt}%
\definecolor{currentstroke}{rgb}{0.000000,0.000000,0.000000}%
\pgfsetstrokecolor{currentstroke}%
\pgfsetdash{}{0pt}%
\pgfpathmoveto{\pgfqpoint{3.332210in}{1.125581in}}%
\pgfpathlineto{\pgfqpoint{3.151139in}{1.216117in}}%
\pgfpathlineto{\pgfqpoint{3.151139in}{1.206200in}}%
\pgfpathlineto{\pgfqpoint{3.117419in}{1.237935in}}%
\pgfpathlineto{\pgfqpoint{3.163039in}{1.230002in}}%
\pgfpathlineto{\pgfqpoint{3.155106in}{1.224051in}}%
\pgfpathlineto{\pgfqpoint{3.336177in}{1.133515in}}%
\pgfpathlineto{\pgfqpoint{3.332210in}{1.125581in}}%
\pgfusepath{fill}%
\end{pgfscope}%
\begin{pgfscope}%
\pgfpathrectangle{\pgfqpoint{1.432000in}{0.528000in}}{\pgfqpoint{3.696000in}{3.696000in}} %
\pgfusepath{clip}%
\pgfsetbuttcap%
\pgfsetroundjoin%
\definecolor{currentfill}{rgb}{0.280267,0.073417,0.397163}%
\pgfsetfillcolor{currentfill}%
\pgfsetlinewidth{0.000000pt}%
\definecolor{currentstroke}{rgb}{0.000000,0.000000,0.000000}%
\pgfsetstrokecolor{currentstroke}%
\pgfsetdash{}{0pt}%
\pgfpathmoveto{\pgfqpoint{3.331057in}{1.126412in}}%
\pgfpathlineto{\pgfqpoint{3.250896in}{1.206574in}}%
\pgfpathlineto{\pgfqpoint{3.247760in}{1.197165in}}%
\pgfpathlineto{\pgfqpoint{3.225806in}{1.237935in}}%
\pgfpathlineto{\pgfqpoint{3.266577in}{1.215982in}}%
\pgfpathlineto{\pgfqpoint{3.257168in}{1.212846in}}%
\pgfpathlineto{\pgfqpoint{3.337330in}{1.132685in}}%
\pgfpathlineto{\pgfqpoint{3.331057in}{1.126412in}}%
\pgfusepath{fill}%
\end{pgfscope}%
\begin{pgfscope}%
\pgfpathrectangle{\pgfqpoint{1.432000in}{0.528000in}}{\pgfqpoint{3.696000in}{3.696000in}} %
\pgfusepath{clip}%
\pgfsetbuttcap%
\pgfsetroundjoin%
\definecolor{currentfill}{rgb}{0.269308,0.218818,0.509577}%
\pgfsetfillcolor{currentfill}%
\pgfsetlinewidth{0.000000pt}%
\definecolor{currentstroke}{rgb}{0.000000,0.000000,0.000000}%
\pgfsetstrokecolor{currentstroke}%
\pgfsetdash{}{0pt}%
\pgfpathmoveto{\pgfqpoint{3.440597in}{1.125581in}}%
\pgfpathlineto{\pgfqpoint{3.259526in}{1.216117in}}%
\pgfpathlineto{\pgfqpoint{3.259526in}{1.206200in}}%
\pgfpathlineto{\pgfqpoint{3.225806in}{1.237935in}}%
\pgfpathlineto{\pgfqpoint{3.271427in}{1.230002in}}%
\pgfpathlineto{\pgfqpoint{3.263493in}{1.224051in}}%
\pgfpathlineto{\pgfqpoint{3.444564in}{1.133515in}}%
\pgfpathlineto{\pgfqpoint{3.440597in}{1.125581in}}%
\pgfusepath{fill}%
\end{pgfscope}%
\begin{pgfscope}%
\pgfpathrectangle{\pgfqpoint{1.432000in}{0.528000in}}{\pgfqpoint{3.696000in}{3.696000in}} %
\pgfusepath{clip}%
\pgfsetbuttcap%
\pgfsetroundjoin%
\definecolor{currentfill}{rgb}{0.276194,0.190074,0.493001}%
\pgfsetfillcolor{currentfill}%
\pgfsetlinewidth{0.000000pt}%
\definecolor{currentstroke}{rgb}{0.000000,0.000000,0.000000}%
\pgfsetstrokecolor{currentstroke}%
\pgfsetdash{}{0pt}%
\pgfpathmoveto{\pgfqpoint{3.439444in}{1.126412in}}%
\pgfpathlineto{\pgfqpoint{3.359283in}{1.206574in}}%
\pgfpathlineto{\pgfqpoint{3.356147in}{1.197165in}}%
\pgfpathlineto{\pgfqpoint{3.334194in}{1.237935in}}%
\pgfpathlineto{\pgfqpoint{3.374964in}{1.215982in}}%
\pgfpathlineto{\pgfqpoint{3.365555in}{1.212846in}}%
\pgfpathlineto{\pgfqpoint{3.445717in}{1.132685in}}%
\pgfpathlineto{\pgfqpoint{3.439444in}{1.126412in}}%
\pgfusepath{fill}%
\end{pgfscope}%
\begin{pgfscope}%
\pgfpathrectangle{\pgfqpoint{1.432000in}{0.528000in}}{\pgfqpoint{3.696000in}{3.696000in}} %
\pgfusepath{clip}%
\pgfsetbuttcap%
\pgfsetroundjoin%
\definecolor{currentfill}{rgb}{0.269944,0.014625,0.341379}%
\pgfsetfillcolor{currentfill}%
\pgfsetlinewidth{0.000000pt}%
\definecolor{currentstroke}{rgb}{0.000000,0.000000,0.000000}%
\pgfsetstrokecolor{currentstroke}%
\pgfsetdash{}{0pt}%
\pgfpathmoveto{\pgfqpoint{3.547832in}{1.126412in}}%
\pgfpathlineto{\pgfqpoint{3.467670in}{1.206574in}}%
\pgfpathlineto{\pgfqpoint{3.464534in}{1.197165in}}%
\pgfpathlineto{\pgfqpoint{3.442581in}{1.237935in}}%
\pgfpathlineto{\pgfqpoint{3.483351in}{1.215982in}}%
\pgfpathlineto{\pgfqpoint{3.473942in}{1.212846in}}%
\pgfpathlineto{\pgfqpoint{3.554104in}{1.132685in}}%
\pgfpathlineto{\pgfqpoint{3.547832in}{1.126412in}}%
\pgfusepath{fill}%
\end{pgfscope}%
\begin{pgfscope}%
\pgfpathrectangle{\pgfqpoint{1.432000in}{0.528000in}}{\pgfqpoint{3.696000in}{3.696000in}} %
\pgfusepath{clip}%
\pgfsetbuttcap%
\pgfsetroundjoin%
\definecolor{currentfill}{rgb}{0.272594,0.025563,0.353093}%
\pgfsetfillcolor{currentfill}%
\pgfsetlinewidth{0.000000pt}%
\definecolor{currentstroke}{rgb}{0.000000,0.000000,0.000000}%
\pgfsetstrokecolor{currentstroke}%
\pgfsetdash{}{0pt}%
\pgfpathmoveto{\pgfqpoint{3.767742in}{1.125113in}}%
\pgfpathlineto{\pgfqpoint{3.590885in}{1.125113in}}%
\pgfpathlineto{\pgfqpoint{3.595320in}{1.116243in}}%
\pgfpathlineto{\pgfqpoint{3.550968in}{1.129548in}}%
\pgfpathlineto{\pgfqpoint{3.595320in}{1.142854in}}%
\pgfpathlineto{\pgfqpoint{3.590885in}{1.133984in}}%
\pgfpathlineto{\pgfqpoint{3.767742in}{1.133984in}}%
\pgfpathlineto{\pgfqpoint{3.767742in}{1.125113in}}%
\pgfusepath{fill}%
\end{pgfscope}%
\begin{pgfscope}%
\pgfpathrectangle{\pgfqpoint{1.432000in}{0.528000in}}{\pgfqpoint{3.696000in}{3.696000in}} %
\pgfusepath{clip}%
\pgfsetbuttcap%
\pgfsetroundjoin%
\definecolor{currentfill}{rgb}{0.281446,0.084320,0.407414}%
\pgfsetfillcolor{currentfill}%
\pgfsetlinewidth{0.000000pt}%
\definecolor{currentstroke}{rgb}{0.000000,0.000000,0.000000}%
\pgfsetstrokecolor{currentstroke}%
\pgfsetdash{}{0pt}%
\pgfpathmoveto{\pgfqpoint{3.765758in}{1.125581in}}%
\pgfpathlineto{\pgfqpoint{3.584687in}{1.216117in}}%
\pgfpathlineto{\pgfqpoint{3.584687in}{1.206200in}}%
\pgfpathlineto{\pgfqpoint{3.550968in}{1.237935in}}%
\pgfpathlineto{\pgfqpoint{3.596588in}{1.230002in}}%
\pgfpathlineto{\pgfqpoint{3.588654in}{1.224051in}}%
\pgfpathlineto{\pgfqpoint{3.769725in}{1.133515in}}%
\pgfpathlineto{\pgfqpoint{3.765758in}{1.125581in}}%
\pgfusepath{fill}%
\end{pgfscope}%
\begin{pgfscope}%
\pgfpathrectangle{\pgfqpoint{1.432000in}{0.528000in}}{\pgfqpoint{3.696000in}{3.696000in}} %
\pgfusepath{clip}%
\pgfsetbuttcap%
\pgfsetroundjoin%
\definecolor{currentfill}{rgb}{0.274128,0.199721,0.498911}%
\pgfsetfillcolor{currentfill}%
\pgfsetlinewidth{0.000000pt}%
\definecolor{currentstroke}{rgb}{0.000000,0.000000,0.000000}%
\pgfsetstrokecolor{currentstroke}%
\pgfsetdash{}{0pt}%
\pgfpathmoveto{\pgfqpoint{3.876129in}{1.125113in}}%
\pgfpathlineto{\pgfqpoint{3.699272in}{1.125113in}}%
\pgfpathlineto{\pgfqpoint{3.703707in}{1.116243in}}%
\pgfpathlineto{\pgfqpoint{3.659355in}{1.129548in}}%
\pgfpathlineto{\pgfqpoint{3.703707in}{1.142854in}}%
\pgfpathlineto{\pgfqpoint{3.699272in}{1.133984in}}%
\pgfpathlineto{\pgfqpoint{3.876129in}{1.133984in}}%
\pgfpathlineto{\pgfqpoint{3.876129in}{1.125113in}}%
\pgfusepath{fill}%
\end{pgfscope}%
\begin{pgfscope}%
\pgfpathrectangle{\pgfqpoint{1.432000in}{0.528000in}}{\pgfqpoint{3.696000in}{3.696000in}} %
\pgfusepath{clip}%
\pgfsetbuttcap%
\pgfsetroundjoin%
\definecolor{currentfill}{rgb}{0.248629,0.278775,0.534556}%
\pgfsetfillcolor{currentfill}%
\pgfsetlinewidth{0.000000pt}%
\definecolor{currentstroke}{rgb}{0.000000,0.000000,0.000000}%
\pgfsetstrokecolor{currentstroke}%
\pgfsetdash{}{0pt}%
\pgfpathmoveto{\pgfqpoint{3.984516in}{1.125113in}}%
\pgfpathlineto{\pgfqpoint{3.699272in}{1.125113in}}%
\pgfpathlineto{\pgfqpoint{3.703707in}{1.116243in}}%
\pgfpathlineto{\pgfqpoint{3.659355in}{1.129548in}}%
\pgfpathlineto{\pgfqpoint{3.703707in}{1.142854in}}%
\pgfpathlineto{\pgfqpoint{3.699272in}{1.133984in}}%
\pgfpathlineto{\pgfqpoint{3.984516in}{1.133984in}}%
\pgfpathlineto{\pgfqpoint{3.984516in}{1.125113in}}%
\pgfusepath{fill}%
\end{pgfscope}%
\begin{pgfscope}%
\pgfpathrectangle{\pgfqpoint{1.432000in}{0.528000in}}{\pgfqpoint{3.696000in}{3.696000in}} %
\pgfusepath{clip}%
\pgfsetbuttcap%
\pgfsetroundjoin%
\definecolor{currentfill}{rgb}{0.235526,0.309527,0.542944}%
\pgfsetfillcolor{currentfill}%
\pgfsetlinewidth{0.000000pt}%
\definecolor{currentstroke}{rgb}{0.000000,0.000000,0.000000}%
\pgfsetstrokecolor{currentstroke}%
\pgfsetdash{}{0pt}%
\pgfpathmoveto{\pgfqpoint{4.092903in}{1.125113in}}%
\pgfpathlineto{\pgfqpoint{3.807659in}{1.125113in}}%
\pgfpathlineto{\pgfqpoint{3.812094in}{1.116243in}}%
\pgfpathlineto{\pgfqpoint{3.767742in}{1.129548in}}%
\pgfpathlineto{\pgfqpoint{3.812094in}{1.142854in}}%
\pgfpathlineto{\pgfqpoint{3.807659in}{1.133984in}}%
\pgfpathlineto{\pgfqpoint{4.092903in}{1.133984in}}%
\pgfpathlineto{\pgfqpoint{4.092903in}{1.125113in}}%
\pgfusepath{fill}%
\end{pgfscope}%
\begin{pgfscope}%
\pgfpathrectangle{\pgfqpoint{1.432000in}{0.528000in}}{\pgfqpoint{3.696000in}{3.696000in}} %
\pgfusepath{clip}%
\pgfsetbuttcap%
\pgfsetroundjoin%
\definecolor{currentfill}{rgb}{0.280868,0.160771,0.472899}%
\pgfsetfillcolor{currentfill}%
\pgfsetlinewidth{0.000000pt}%
\definecolor{currentstroke}{rgb}{0.000000,0.000000,0.000000}%
\pgfsetstrokecolor{currentstroke}%
\pgfsetdash{}{0pt}%
\pgfpathmoveto{\pgfqpoint{4.202693in}{1.125341in}}%
\pgfpathlineto{\pgfqpoint{3.915400in}{1.029576in}}%
\pgfpathlineto{\pgfqpoint{3.922413in}{1.022564in}}%
\pgfpathlineto{\pgfqpoint{3.876129in}{1.021161in}}%
\pgfpathlineto{\pgfqpoint{3.913997in}{1.047809in}}%
\pgfpathlineto{\pgfqpoint{3.912595in}{1.037992in}}%
\pgfpathlineto{\pgfqpoint{4.199888in}{1.133756in}}%
\pgfpathlineto{\pgfqpoint{4.202693in}{1.125341in}}%
\pgfusepath{fill}%
\end{pgfscope}%
\begin{pgfscope}%
\pgfpathrectangle{\pgfqpoint{1.432000in}{0.528000in}}{\pgfqpoint{3.696000in}{3.696000in}} %
\pgfusepath{clip}%
\pgfsetbuttcap%
\pgfsetroundjoin%
\definecolor{currentfill}{rgb}{0.275191,0.194905,0.496005}%
\pgfsetfillcolor{currentfill}%
\pgfsetlinewidth{0.000000pt}%
\definecolor{currentstroke}{rgb}{0.000000,0.000000,0.000000}%
\pgfsetstrokecolor{currentstroke}%
\pgfsetdash{}{0pt}%
\pgfpathmoveto{\pgfqpoint{4.310753in}{1.125246in}}%
\pgfpathlineto{\pgfqpoint{3.915930in}{1.026540in}}%
\pgfpathlineto{\pgfqpoint{3.922384in}{1.019010in}}%
\pgfpathlineto{\pgfqpoint{3.876129in}{1.021161in}}%
\pgfpathlineto{\pgfqpoint{3.915930in}{1.044827in}}%
\pgfpathlineto{\pgfqpoint{3.913778in}{1.035145in}}%
\pgfpathlineto{\pgfqpoint{4.308602in}{1.133851in}}%
\pgfpathlineto{\pgfqpoint{4.310753in}{1.125246in}}%
\pgfusepath{fill}%
\end{pgfscope}%
\begin{pgfscope}%
\pgfpathrectangle{\pgfqpoint{1.432000in}{0.528000in}}{\pgfqpoint{3.696000in}{3.696000in}} %
\pgfusepath{clip}%
\pgfsetbuttcap%
\pgfsetroundjoin%
\definecolor{currentfill}{rgb}{0.269944,0.014625,0.341379}%
\pgfsetfillcolor{currentfill}%
\pgfsetlinewidth{0.000000pt}%
\definecolor{currentstroke}{rgb}{0.000000,0.000000,0.000000}%
\pgfsetstrokecolor{currentstroke}%
\pgfsetdash{}{0pt}%
\pgfpathmoveto{\pgfqpoint{4.311080in}{1.125341in}}%
\pgfpathlineto{\pgfqpoint{4.023787in}{1.029576in}}%
\pgfpathlineto{\pgfqpoint{4.030800in}{1.022564in}}%
\pgfpathlineto{\pgfqpoint{3.984516in}{1.021161in}}%
\pgfpathlineto{\pgfqpoint{4.022385in}{1.047809in}}%
\pgfpathlineto{\pgfqpoint{4.020982in}{1.037992in}}%
\pgfpathlineto{\pgfqpoint{4.308275in}{1.133756in}}%
\pgfpathlineto{\pgfqpoint{4.311080in}{1.125341in}}%
\pgfusepath{fill}%
\end{pgfscope}%
\begin{pgfscope}%
\pgfpathrectangle{\pgfqpoint{1.432000in}{0.528000in}}{\pgfqpoint{3.696000in}{3.696000in}} %
\pgfusepath{clip}%
\pgfsetbuttcap%
\pgfsetroundjoin%
\definecolor{currentfill}{rgb}{0.283197,0.115680,0.436115}%
\pgfsetfillcolor{currentfill}%
\pgfsetlinewidth{0.000000pt}%
\definecolor{currentstroke}{rgb}{0.000000,0.000000,0.000000}%
\pgfsetstrokecolor{currentstroke}%
\pgfsetdash{}{0pt}%
\pgfpathmoveto{\pgfqpoint{4.419140in}{1.125246in}}%
\pgfpathlineto{\pgfqpoint{4.024317in}{1.026540in}}%
\pgfpathlineto{\pgfqpoint{4.030771in}{1.019010in}}%
\pgfpathlineto{\pgfqpoint{3.984516in}{1.021161in}}%
\pgfpathlineto{\pgfqpoint{4.024317in}{1.044827in}}%
\pgfpathlineto{\pgfqpoint{4.022165in}{1.035145in}}%
\pgfpathlineto{\pgfqpoint{4.416989in}{1.133851in}}%
\pgfpathlineto{\pgfqpoint{4.419140in}{1.125246in}}%
\pgfusepath{fill}%
\end{pgfscope}%
\begin{pgfscope}%
\pgfpathrectangle{\pgfqpoint{1.432000in}{0.528000in}}{\pgfqpoint{3.696000in}{3.696000in}} %
\pgfusepath{clip}%
\pgfsetbuttcap%
\pgfsetroundjoin%
\definecolor{currentfill}{rgb}{0.283229,0.120777,0.440584}%
\pgfsetfillcolor{currentfill}%
\pgfsetlinewidth{0.000000pt}%
\definecolor{currentstroke}{rgb}{0.000000,0.000000,0.000000}%
\pgfsetstrokecolor{currentstroke}%
\pgfsetdash{}{0pt}%
\pgfpathmoveto{\pgfqpoint{4.419467in}{1.125341in}}%
\pgfpathlineto{\pgfqpoint{4.132174in}{1.029576in}}%
\pgfpathlineto{\pgfqpoint{4.139187in}{1.022564in}}%
\pgfpathlineto{\pgfqpoint{4.092903in}{1.021161in}}%
\pgfpathlineto{\pgfqpoint{4.130772in}{1.047809in}}%
\pgfpathlineto{\pgfqpoint{4.129369in}{1.037992in}}%
\pgfpathlineto{\pgfqpoint{4.416662in}{1.133756in}}%
\pgfpathlineto{\pgfqpoint{4.419467in}{1.125341in}}%
\pgfusepath{fill}%
\end{pgfscope}%
\begin{pgfscope}%
\pgfpathrectangle{\pgfqpoint{1.432000in}{0.528000in}}{\pgfqpoint{3.696000in}{3.696000in}} %
\pgfusepath{clip}%
\pgfsetbuttcap%
\pgfsetroundjoin%
\definecolor{currentfill}{rgb}{0.269308,0.218818,0.509577}%
\pgfsetfillcolor{currentfill}%
\pgfsetlinewidth{0.000000pt}%
\definecolor{currentstroke}{rgb}{0.000000,0.000000,0.000000}%
\pgfsetstrokecolor{currentstroke}%
\pgfsetdash{}{0pt}%
\pgfpathmoveto{\pgfqpoint{4.528435in}{1.125581in}}%
\pgfpathlineto{\pgfqpoint{4.347364in}{1.035046in}}%
\pgfpathlineto{\pgfqpoint{4.355297in}{1.029095in}}%
\pgfpathlineto{\pgfqpoint{4.309677in}{1.021161in}}%
\pgfpathlineto{\pgfqpoint{4.343397in}{1.052897in}}%
\pgfpathlineto{\pgfqpoint{4.343397in}{1.042980in}}%
\pgfpathlineto{\pgfqpoint{4.524468in}{1.133515in}}%
\pgfpathlineto{\pgfqpoint{4.528435in}{1.125581in}}%
\pgfusepath{fill}%
\end{pgfscope}%
\begin{pgfscope}%
\pgfpathrectangle{\pgfqpoint{1.432000in}{0.528000in}}{\pgfqpoint{3.696000in}{3.696000in}} %
\pgfusepath{clip}%
\pgfsetbuttcap%
\pgfsetroundjoin%
\definecolor{currentfill}{rgb}{0.267968,0.223549,0.512008}%
\pgfsetfillcolor{currentfill}%
\pgfsetlinewidth{0.000000pt}%
\definecolor{currentstroke}{rgb}{0.000000,0.000000,0.000000}%
\pgfsetstrokecolor{currentstroke}%
\pgfsetdash{}{0pt}%
\pgfpathmoveto{\pgfqpoint{4.636822in}{1.125581in}}%
\pgfpathlineto{\pgfqpoint{4.455751in}{1.035046in}}%
\pgfpathlineto{\pgfqpoint{4.463685in}{1.029095in}}%
\pgfpathlineto{\pgfqpoint{4.418065in}{1.021161in}}%
\pgfpathlineto{\pgfqpoint{4.451784in}{1.052897in}}%
\pgfpathlineto{\pgfqpoint{4.451784in}{1.042980in}}%
\pgfpathlineto{\pgfqpoint{4.632855in}{1.133515in}}%
\pgfpathlineto{\pgfqpoint{4.636822in}{1.125581in}}%
\pgfusepath{fill}%
\end{pgfscope}%
\begin{pgfscope}%
\pgfpathrectangle{\pgfqpoint{1.432000in}{0.528000in}}{\pgfqpoint{3.696000in}{3.696000in}} %
\pgfusepath{clip}%
\pgfsetbuttcap%
\pgfsetroundjoin%
\definecolor{currentfill}{rgb}{0.277018,0.050344,0.375715}%
\pgfsetfillcolor{currentfill}%
\pgfsetlinewidth{0.000000pt}%
\definecolor{currentstroke}{rgb}{0.000000,0.000000,0.000000}%
\pgfsetstrokecolor{currentstroke}%
\pgfsetdash{}{0pt}%
\pgfpathmoveto{\pgfqpoint{4.745209in}{1.125581in}}%
\pgfpathlineto{\pgfqpoint{4.564138in}{1.035046in}}%
\pgfpathlineto{\pgfqpoint{4.572072in}{1.029095in}}%
\pgfpathlineto{\pgfqpoint{4.526452in}{1.021161in}}%
\pgfpathlineto{\pgfqpoint{4.560171in}{1.052897in}}%
\pgfpathlineto{\pgfqpoint{4.560171in}{1.042980in}}%
\pgfpathlineto{\pgfqpoint{4.741242in}{1.133515in}}%
\pgfpathlineto{\pgfqpoint{4.745209in}{1.125581in}}%
\pgfusepath{fill}%
\end{pgfscope}%
\begin{pgfscope}%
\pgfpathrectangle{\pgfqpoint{1.432000in}{0.528000in}}{\pgfqpoint{3.696000in}{3.696000in}} %
\pgfusepath{clip}%
\pgfsetbuttcap%
\pgfsetroundjoin%
\definecolor{currentfill}{rgb}{0.278791,0.062145,0.386592}%
\pgfsetfillcolor{currentfill}%
\pgfsetlinewidth{0.000000pt}%
\definecolor{currentstroke}{rgb}{0.000000,0.000000,0.000000}%
\pgfsetstrokecolor{currentstroke}%
\pgfsetdash{}{0pt}%
\pgfpathmoveto{\pgfqpoint{4.746362in}{1.126412in}}%
\pgfpathlineto{\pgfqpoint{4.666200in}{1.046251in}}%
\pgfpathlineto{\pgfqpoint{4.675609in}{1.043114in}}%
\pgfpathlineto{\pgfqpoint{4.634839in}{1.021161in}}%
\pgfpathlineto{\pgfqpoint{4.656792in}{1.061931in}}%
\pgfpathlineto{\pgfqpoint{4.659928in}{1.052523in}}%
\pgfpathlineto{\pgfqpoint{4.740090in}{1.132685in}}%
\pgfpathlineto{\pgfqpoint{4.746362in}{1.126412in}}%
\pgfusepath{fill}%
\end{pgfscope}%
\begin{pgfscope}%
\pgfpathrectangle{\pgfqpoint{1.432000in}{0.528000in}}{\pgfqpoint{3.696000in}{3.696000in}} %
\pgfusepath{clip}%
\pgfsetbuttcap%
\pgfsetroundjoin%
\definecolor{currentfill}{rgb}{0.281446,0.084320,0.407414}%
\pgfsetfillcolor{currentfill}%
\pgfsetlinewidth{0.000000pt}%
\definecolor{currentstroke}{rgb}{0.000000,0.000000,0.000000}%
\pgfsetstrokecolor{currentstroke}%
\pgfsetdash{}{0pt}%
\pgfpathmoveto{\pgfqpoint{4.743226in}{1.125113in}}%
\pgfpathlineto{\pgfqpoint{4.674756in}{1.125113in}}%
\pgfpathlineto{\pgfqpoint{4.679191in}{1.116243in}}%
\pgfpathlineto{\pgfqpoint{4.634839in}{1.129548in}}%
\pgfpathlineto{\pgfqpoint{4.679191in}{1.142854in}}%
\pgfpathlineto{\pgfqpoint{4.674756in}{1.133984in}}%
\pgfpathlineto{\pgfqpoint{4.743226in}{1.133984in}}%
\pgfpathlineto{\pgfqpoint{4.743226in}{1.125113in}}%
\pgfusepath{fill}%
\end{pgfscope}%
\begin{pgfscope}%
\pgfpathrectangle{\pgfqpoint{1.432000in}{0.528000in}}{\pgfqpoint{3.696000in}{3.696000in}} %
\pgfusepath{clip}%
\pgfsetbuttcap%
\pgfsetroundjoin%
\definecolor{currentfill}{rgb}{0.269308,0.218818,0.509577}%
\pgfsetfillcolor{currentfill}%
\pgfsetlinewidth{0.000000pt}%
\definecolor{currentstroke}{rgb}{0.000000,0.000000,0.000000}%
\pgfsetstrokecolor{currentstroke}%
\pgfsetdash{}{0pt}%
\pgfpathmoveto{\pgfqpoint{4.851613in}{1.125113in}}%
\pgfpathlineto{\pgfqpoint{4.783143in}{1.125113in}}%
\pgfpathlineto{\pgfqpoint{4.787578in}{1.116243in}}%
\pgfpathlineto{\pgfqpoint{4.743226in}{1.129548in}}%
\pgfpathlineto{\pgfqpoint{4.787578in}{1.142854in}}%
\pgfpathlineto{\pgfqpoint{4.783143in}{1.133984in}}%
\pgfpathlineto{\pgfqpoint{4.851613in}{1.133984in}}%
\pgfpathlineto{\pgfqpoint{4.851613in}{1.125113in}}%
\pgfusepath{fill}%
\end{pgfscope}%
\begin{pgfscope}%
\pgfpathrectangle{\pgfqpoint{1.432000in}{0.528000in}}{\pgfqpoint{3.696000in}{3.696000in}} %
\pgfusepath{clip}%
\pgfsetbuttcap%
\pgfsetroundjoin%
\definecolor{currentfill}{rgb}{0.283187,0.125848,0.444960}%
\pgfsetfillcolor{currentfill}%
\pgfsetlinewidth{0.000000pt}%
\definecolor{currentstroke}{rgb}{0.000000,0.000000,0.000000}%
\pgfsetstrokecolor{currentstroke}%
\pgfsetdash{}{0pt}%
\pgfpathmoveto{\pgfqpoint{4.856048in}{1.129548in}}%
\pgfpathlineto{\pgfqpoint{4.853831in}{1.133389in}}%
\pgfpathlineto{\pgfqpoint{4.849395in}{1.133389in}}%
\pgfpathlineto{\pgfqpoint{4.847178in}{1.129548in}}%
\pgfpathlineto{\pgfqpoint{4.849395in}{1.125707in}}%
\pgfpathlineto{\pgfqpoint{4.853831in}{1.125707in}}%
\pgfpathlineto{\pgfqpoint{4.856048in}{1.129548in}}%
\pgfpathlineto{\pgfqpoint{4.853831in}{1.133389in}}%
\pgfusepath{fill}%
\end{pgfscope}%
\begin{pgfscope}%
\pgfpathrectangle{\pgfqpoint{1.432000in}{0.528000in}}{\pgfqpoint{3.696000in}{3.696000in}} %
\pgfusepath{clip}%
\pgfsetbuttcap%
\pgfsetroundjoin%
\definecolor{currentfill}{rgb}{0.282623,0.140926,0.457517}%
\pgfsetfillcolor{currentfill}%
\pgfsetlinewidth{0.000000pt}%
\definecolor{currentstroke}{rgb}{0.000000,0.000000,0.000000}%
\pgfsetstrokecolor{currentstroke}%
\pgfsetdash{}{0pt}%
\pgfpathmoveto{\pgfqpoint{4.960000in}{1.125113in}}%
\pgfpathlineto{\pgfqpoint{4.891530in}{1.125113in}}%
\pgfpathlineto{\pgfqpoint{4.895965in}{1.116243in}}%
\pgfpathlineto{\pgfqpoint{4.851613in}{1.129548in}}%
\pgfpathlineto{\pgfqpoint{4.895965in}{1.142854in}}%
\pgfpathlineto{\pgfqpoint{4.891530in}{1.133984in}}%
\pgfpathlineto{\pgfqpoint{4.960000in}{1.133984in}}%
\pgfpathlineto{\pgfqpoint{4.960000in}{1.125113in}}%
\pgfusepath{fill}%
\end{pgfscope}%
\begin{pgfscope}%
\pgfpathrectangle{\pgfqpoint{1.432000in}{0.528000in}}{\pgfqpoint{3.696000in}{3.696000in}} %
\pgfusepath{clip}%
\pgfsetbuttcap%
\pgfsetroundjoin%
\definecolor{currentfill}{rgb}{0.162142,0.474838,0.558140}%
\pgfsetfillcolor{currentfill}%
\pgfsetlinewidth{0.000000pt}%
\definecolor{currentstroke}{rgb}{0.000000,0.000000,0.000000}%
\pgfsetstrokecolor{currentstroke}%
\pgfsetdash{}{0pt}%
\pgfpathmoveto{\pgfqpoint{4.964435in}{1.129548in}}%
\pgfpathlineto{\pgfqpoint{4.962218in}{1.133389in}}%
\pgfpathlineto{\pgfqpoint{4.957782in}{1.133389in}}%
\pgfpathlineto{\pgfqpoint{4.955565in}{1.129548in}}%
\pgfpathlineto{\pgfqpoint{4.957782in}{1.125707in}}%
\pgfpathlineto{\pgfqpoint{4.962218in}{1.125707in}}%
\pgfpathlineto{\pgfqpoint{4.964435in}{1.129548in}}%
\pgfpathlineto{\pgfqpoint{4.962218in}{1.133389in}}%
\pgfusepath{fill}%
\end{pgfscope}%
\begin{pgfscope}%
\pgfpathrectangle{\pgfqpoint{1.432000in}{0.528000in}}{\pgfqpoint{3.696000in}{3.696000in}} %
\pgfusepath{clip}%
\pgfsetbuttcap%
\pgfsetroundjoin%
\definecolor{currentfill}{rgb}{0.248629,0.278775,0.534556}%
\pgfsetfillcolor{currentfill}%
\pgfsetlinewidth{0.000000pt}%
\definecolor{currentstroke}{rgb}{0.000000,0.000000,0.000000}%
\pgfsetstrokecolor{currentstroke}%
\pgfsetdash{}{0pt}%
\pgfpathmoveto{\pgfqpoint{1.604435in}{1.237935in}}%
\pgfpathlineto{\pgfqpoint{1.604435in}{1.169465in}}%
\pgfpathlineto{\pgfqpoint{1.613306in}{1.173900in}}%
\pgfpathlineto{\pgfqpoint{1.600000in}{1.129548in}}%
\pgfpathlineto{\pgfqpoint{1.586694in}{1.173900in}}%
\pgfpathlineto{\pgfqpoint{1.595565in}{1.169465in}}%
\pgfpathlineto{\pgfqpoint{1.595565in}{1.237935in}}%
\pgfpathlineto{\pgfqpoint{1.604435in}{1.237935in}}%
\pgfusepath{fill}%
\end{pgfscope}%
\begin{pgfscope}%
\pgfpathrectangle{\pgfqpoint{1.432000in}{0.528000in}}{\pgfqpoint{3.696000in}{3.696000in}} %
\pgfusepath{clip}%
\pgfsetbuttcap%
\pgfsetroundjoin%
\definecolor{currentfill}{rgb}{0.282623,0.140926,0.457517}%
\pgfsetfillcolor{currentfill}%
\pgfsetlinewidth{0.000000pt}%
\definecolor{currentstroke}{rgb}{0.000000,0.000000,0.000000}%
\pgfsetstrokecolor{currentstroke}%
\pgfsetdash{}{0pt}%
\pgfpathmoveto{\pgfqpoint{1.711523in}{1.234799in}}%
\pgfpathlineto{\pgfqpoint{1.631362in}{1.154638in}}%
\pgfpathlineto{\pgfqpoint{1.640770in}{1.151502in}}%
\pgfpathlineto{\pgfqpoint{1.600000in}{1.129548in}}%
\pgfpathlineto{\pgfqpoint{1.621953in}{1.170318in}}%
\pgfpathlineto{\pgfqpoint{1.625089in}{1.160910in}}%
\pgfpathlineto{\pgfqpoint{1.705251in}{1.241072in}}%
\pgfpathlineto{\pgfqpoint{1.711523in}{1.234799in}}%
\pgfusepath{fill}%
\end{pgfscope}%
\begin{pgfscope}%
\pgfpathrectangle{\pgfqpoint{1.432000in}{0.528000in}}{\pgfqpoint{3.696000in}{3.696000in}} %
\pgfusepath{clip}%
\pgfsetbuttcap%
\pgfsetroundjoin%
\definecolor{currentfill}{rgb}{0.282623,0.140926,0.457517}%
\pgfsetfillcolor{currentfill}%
\pgfsetlinewidth{0.000000pt}%
\definecolor{currentstroke}{rgb}{0.000000,0.000000,0.000000}%
\pgfsetstrokecolor{currentstroke}%
\pgfsetdash{}{0pt}%
\pgfpathmoveto{\pgfqpoint{1.819910in}{1.234799in}}%
\pgfpathlineto{\pgfqpoint{1.739749in}{1.154638in}}%
\pgfpathlineto{\pgfqpoint{1.749157in}{1.151502in}}%
\pgfpathlineto{\pgfqpoint{1.708387in}{1.129548in}}%
\pgfpathlineto{\pgfqpoint{1.730340in}{1.170318in}}%
\pgfpathlineto{\pgfqpoint{1.733476in}{1.160910in}}%
\pgfpathlineto{\pgfqpoint{1.813638in}{1.241072in}}%
\pgfpathlineto{\pgfqpoint{1.819910in}{1.234799in}}%
\pgfusepath{fill}%
\end{pgfscope}%
\begin{pgfscope}%
\pgfpathrectangle{\pgfqpoint{1.432000in}{0.528000in}}{\pgfqpoint{3.696000in}{3.696000in}} %
\pgfusepath{clip}%
\pgfsetbuttcap%
\pgfsetroundjoin%
\definecolor{currentfill}{rgb}{0.282623,0.140926,0.457517}%
\pgfsetfillcolor{currentfill}%
\pgfsetlinewidth{0.000000pt}%
\definecolor{currentstroke}{rgb}{0.000000,0.000000,0.000000}%
\pgfsetstrokecolor{currentstroke}%
\pgfsetdash{}{0pt}%
\pgfpathmoveto{\pgfqpoint{1.927145in}{1.233969in}}%
\pgfpathlineto{\pgfqpoint{1.746073in}{1.143433in}}%
\pgfpathlineto{\pgfqpoint{1.754007in}{1.137482in}}%
\pgfpathlineto{\pgfqpoint{1.708387in}{1.129548in}}%
\pgfpathlineto{\pgfqpoint{1.742106in}{1.161284in}}%
\pgfpathlineto{\pgfqpoint{1.742106in}{1.151367in}}%
\pgfpathlineto{\pgfqpoint{1.923178in}{1.241902in}}%
\pgfpathlineto{\pgfqpoint{1.927145in}{1.233969in}}%
\pgfusepath{fill}%
\end{pgfscope}%
\begin{pgfscope}%
\pgfpathrectangle{\pgfqpoint{1.432000in}{0.528000in}}{\pgfqpoint{3.696000in}{3.696000in}} %
\pgfusepath{clip}%
\pgfsetbuttcap%
\pgfsetroundjoin%
\definecolor{currentfill}{rgb}{0.199430,0.387607,0.554642}%
\pgfsetfillcolor{currentfill}%
\pgfsetlinewidth{0.000000pt}%
\definecolor{currentstroke}{rgb}{0.000000,0.000000,0.000000}%
\pgfsetstrokecolor{currentstroke}%
\pgfsetdash{}{0pt}%
\pgfpathmoveto{\pgfqpoint{2.035532in}{1.233969in}}%
\pgfpathlineto{\pgfqpoint{1.854460in}{1.143433in}}%
\pgfpathlineto{\pgfqpoint{1.862394in}{1.137482in}}%
\pgfpathlineto{\pgfqpoint{1.816774in}{1.129548in}}%
\pgfpathlineto{\pgfqpoint{1.850493in}{1.161284in}}%
\pgfpathlineto{\pgfqpoint{1.850493in}{1.151367in}}%
\pgfpathlineto{\pgfqpoint{2.031565in}{1.241902in}}%
\pgfpathlineto{\pgfqpoint{2.035532in}{1.233969in}}%
\pgfusepath{fill}%
\end{pgfscope}%
\begin{pgfscope}%
\pgfpathrectangle{\pgfqpoint{1.432000in}{0.528000in}}{\pgfqpoint{3.696000in}{3.696000in}} %
\pgfusepath{clip}%
\pgfsetbuttcap%
\pgfsetroundjoin%
\definecolor{currentfill}{rgb}{0.137770,0.537492,0.554906}%
\pgfsetfillcolor{currentfill}%
\pgfsetlinewidth{0.000000pt}%
\definecolor{currentstroke}{rgb}{0.000000,0.000000,0.000000}%
\pgfsetstrokecolor{currentstroke}%
\pgfsetdash{}{0pt}%
\pgfpathmoveto{\pgfqpoint{2.143919in}{1.233969in}}%
\pgfpathlineto{\pgfqpoint{1.962847in}{1.143433in}}%
\pgfpathlineto{\pgfqpoint{1.970781in}{1.137482in}}%
\pgfpathlineto{\pgfqpoint{1.925161in}{1.129548in}}%
\pgfpathlineto{\pgfqpoint{1.958880in}{1.161284in}}%
\pgfpathlineto{\pgfqpoint{1.958880in}{1.151367in}}%
\pgfpathlineto{\pgfqpoint{2.139952in}{1.241902in}}%
\pgfpathlineto{\pgfqpoint{2.143919in}{1.233969in}}%
\pgfusepath{fill}%
\end{pgfscope}%
\begin{pgfscope}%
\pgfpathrectangle{\pgfqpoint{1.432000in}{0.528000in}}{\pgfqpoint{3.696000in}{3.696000in}} %
\pgfusepath{clip}%
\pgfsetbuttcap%
\pgfsetroundjoin%
\definecolor{currentfill}{rgb}{0.252194,0.269783,0.531579}%
\pgfsetfillcolor{currentfill}%
\pgfsetlinewidth{0.000000pt}%
\definecolor{currentstroke}{rgb}{0.000000,0.000000,0.000000}%
\pgfsetstrokecolor{currentstroke}%
\pgfsetdash{}{0pt}%
\pgfpathmoveto{\pgfqpoint{2.252306in}{1.233969in}}%
\pgfpathlineto{\pgfqpoint{2.071235in}{1.143433in}}%
\pgfpathlineto{\pgfqpoint{2.079168in}{1.137482in}}%
\pgfpathlineto{\pgfqpoint{2.033548in}{1.129548in}}%
\pgfpathlineto{\pgfqpoint{2.067268in}{1.161284in}}%
\pgfpathlineto{\pgfqpoint{2.067268in}{1.151367in}}%
\pgfpathlineto{\pgfqpoint{2.248339in}{1.241902in}}%
\pgfpathlineto{\pgfqpoint{2.252306in}{1.233969in}}%
\pgfusepath{fill}%
\end{pgfscope}%
\begin{pgfscope}%
\pgfpathrectangle{\pgfqpoint{1.432000in}{0.528000in}}{\pgfqpoint{3.696000in}{3.696000in}} %
\pgfusepath{clip}%
\pgfsetbuttcap%
\pgfsetroundjoin%
\definecolor{currentfill}{rgb}{0.263663,0.237631,0.518762}%
\pgfsetfillcolor{currentfill}%
\pgfsetlinewidth{0.000000pt}%
\definecolor{currentstroke}{rgb}{0.000000,0.000000,0.000000}%
\pgfsetstrokecolor{currentstroke}%
\pgfsetdash{}{0pt}%
\pgfpathmoveto{\pgfqpoint{2.360112in}{1.233728in}}%
\pgfpathlineto{\pgfqpoint{2.072819in}{1.137964in}}%
\pgfpathlineto{\pgfqpoint{2.079832in}{1.130951in}}%
\pgfpathlineto{\pgfqpoint{2.033548in}{1.129548in}}%
\pgfpathlineto{\pgfqpoint{2.071417in}{1.156197in}}%
\pgfpathlineto{\pgfqpoint{2.070014in}{1.146379in}}%
\pgfpathlineto{\pgfqpoint{2.357307in}{1.242143in}}%
\pgfpathlineto{\pgfqpoint{2.360112in}{1.233728in}}%
\pgfusepath{fill}%
\end{pgfscope}%
\begin{pgfscope}%
\pgfpathrectangle{\pgfqpoint{1.432000in}{0.528000in}}{\pgfqpoint{3.696000in}{3.696000in}} %
\pgfusepath{clip}%
\pgfsetbuttcap%
\pgfsetroundjoin%
\definecolor{currentfill}{rgb}{0.265145,0.232956,0.516599}%
\pgfsetfillcolor{currentfill}%
\pgfsetlinewidth{0.000000pt}%
\definecolor{currentstroke}{rgb}{0.000000,0.000000,0.000000}%
\pgfsetstrokecolor{currentstroke}%
\pgfsetdash{}{0pt}%
\pgfpathmoveto{\pgfqpoint{2.468499in}{1.233728in}}%
\pgfpathlineto{\pgfqpoint{2.181206in}{1.137964in}}%
\pgfpathlineto{\pgfqpoint{2.188219in}{1.130951in}}%
\pgfpathlineto{\pgfqpoint{2.141935in}{1.129548in}}%
\pgfpathlineto{\pgfqpoint{2.179804in}{1.156197in}}%
\pgfpathlineto{\pgfqpoint{2.178401in}{1.146379in}}%
\pgfpathlineto{\pgfqpoint{2.465694in}{1.242143in}}%
\pgfpathlineto{\pgfqpoint{2.468499in}{1.233728in}}%
\pgfusepath{fill}%
\end{pgfscope}%
\begin{pgfscope}%
\pgfpathrectangle{\pgfqpoint{1.432000in}{0.528000in}}{\pgfqpoint{3.696000in}{3.696000in}} %
\pgfusepath{clip}%
\pgfsetbuttcap%
\pgfsetroundjoin%
\definecolor{currentfill}{rgb}{0.237441,0.305202,0.541921}%
\pgfsetfillcolor{currentfill}%
\pgfsetlinewidth{0.000000pt}%
\definecolor{currentstroke}{rgb}{0.000000,0.000000,0.000000}%
\pgfsetstrokecolor{currentstroke}%
\pgfsetdash{}{0pt}%
\pgfpathmoveto{\pgfqpoint{2.467097in}{1.233500in}}%
\pgfpathlineto{\pgfqpoint{2.181852in}{1.233500in}}%
\pgfpathlineto{\pgfqpoint{2.186287in}{1.224630in}}%
\pgfpathlineto{\pgfqpoint{2.141935in}{1.237935in}}%
\pgfpathlineto{\pgfqpoint{2.186287in}{1.251241in}}%
\pgfpathlineto{\pgfqpoint{2.181852in}{1.242371in}}%
\pgfpathlineto{\pgfqpoint{2.467097in}{1.242371in}}%
\pgfpathlineto{\pgfqpoint{2.467097in}{1.233500in}}%
\pgfusepath{fill}%
\end{pgfscope}%
\begin{pgfscope}%
\pgfpathrectangle{\pgfqpoint{1.432000in}{0.528000in}}{\pgfqpoint{3.696000in}{3.696000in}} %
\pgfusepath{clip}%
\pgfsetbuttcap%
\pgfsetroundjoin%
\definecolor{currentfill}{rgb}{0.133743,0.548535,0.553541}%
\pgfsetfillcolor{currentfill}%
\pgfsetlinewidth{0.000000pt}%
\definecolor{currentstroke}{rgb}{0.000000,0.000000,0.000000}%
\pgfsetstrokecolor{currentstroke}%
\pgfsetdash{}{0pt}%
\pgfpathmoveto{\pgfqpoint{2.575484in}{1.233500in}}%
\pgfpathlineto{\pgfqpoint{2.290239in}{1.233500in}}%
\pgfpathlineto{\pgfqpoint{2.294675in}{1.224630in}}%
\pgfpathlineto{\pgfqpoint{2.250323in}{1.237935in}}%
\pgfpathlineto{\pgfqpoint{2.294675in}{1.251241in}}%
\pgfpathlineto{\pgfqpoint{2.290239in}{1.242371in}}%
\pgfpathlineto{\pgfqpoint{2.575484in}{1.242371in}}%
\pgfpathlineto{\pgfqpoint{2.575484in}{1.233500in}}%
\pgfusepath{fill}%
\end{pgfscope}%
\begin{pgfscope}%
\pgfpathrectangle{\pgfqpoint{1.432000in}{0.528000in}}{\pgfqpoint{3.696000in}{3.696000in}} %
\pgfusepath{clip}%
\pgfsetbuttcap%
\pgfsetroundjoin%
\definecolor{currentfill}{rgb}{0.129933,0.559582,0.551864}%
\pgfsetfillcolor{currentfill}%
\pgfsetlinewidth{0.000000pt}%
\definecolor{currentstroke}{rgb}{0.000000,0.000000,0.000000}%
\pgfsetstrokecolor{currentstroke}%
\pgfsetdash{}{0pt}%
\pgfpathmoveto{\pgfqpoint{2.683871in}{1.233500in}}%
\pgfpathlineto{\pgfqpoint{2.398626in}{1.233500in}}%
\pgfpathlineto{\pgfqpoint{2.403062in}{1.224630in}}%
\pgfpathlineto{\pgfqpoint{2.358710in}{1.237935in}}%
\pgfpathlineto{\pgfqpoint{2.403062in}{1.251241in}}%
\pgfpathlineto{\pgfqpoint{2.398626in}{1.242371in}}%
\pgfpathlineto{\pgfqpoint{2.683871in}{1.242371in}}%
\pgfpathlineto{\pgfqpoint{2.683871in}{1.233500in}}%
\pgfusepath{fill}%
\end{pgfscope}%
\begin{pgfscope}%
\pgfpathrectangle{\pgfqpoint{1.432000in}{0.528000in}}{\pgfqpoint{3.696000in}{3.696000in}} %
\pgfusepath{clip}%
\pgfsetbuttcap%
\pgfsetroundjoin%
\definecolor{currentfill}{rgb}{0.131172,0.555899,0.552459}%
\pgfsetfillcolor{currentfill}%
\pgfsetlinewidth{0.000000pt}%
\definecolor{currentstroke}{rgb}{0.000000,0.000000,0.000000}%
\pgfsetstrokecolor{currentstroke}%
\pgfsetdash{}{0pt}%
\pgfpathmoveto{\pgfqpoint{2.792258in}{1.233500in}}%
\pgfpathlineto{\pgfqpoint{2.507014in}{1.233500in}}%
\pgfpathlineto{\pgfqpoint{2.511449in}{1.224630in}}%
\pgfpathlineto{\pgfqpoint{2.467097in}{1.237935in}}%
\pgfpathlineto{\pgfqpoint{2.511449in}{1.251241in}}%
\pgfpathlineto{\pgfqpoint{2.507014in}{1.242371in}}%
\pgfpathlineto{\pgfqpoint{2.792258in}{1.242371in}}%
\pgfpathlineto{\pgfqpoint{2.792258in}{1.233500in}}%
\pgfusepath{fill}%
\end{pgfscope}%
\begin{pgfscope}%
\pgfpathrectangle{\pgfqpoint{1.432000in}{0.528000in}}{\pgfqpoint{3.696000in}{3.696000in}} %
\pgfusepath{clip}%
\pgfsetbuttcap%
\pgfsetroundjoin%
\definecolor{currentfill}{rgb}{0.120081,0.622161,0.534946}%
\pgfsetfillcolor{currentfill}%
\pgfsetlinewidth{0.000000pt}%
\definecolor{currentstroke}{rgb}{0.000000,0.000000,0.000000}%
\pgfsetstrokecolor{currentstroke}%
\pgfsetdash{}{0pt}%
\pgfpathmoveto{\pgfqpoint{2.899243in}{1.233728in}}%
\pgfpathlineto{\pgfqpoint{2.611950in}{1.329492in}}%
\pgfpathlineto{\pgfqpoint{2.613352in}{1.319674in}}%
\pgfpathlineto{\pgfqpoint{2.575484in}{1.346323in}}%
\pgfpathlineto{\pgfqpoint{2.621767in}{1.344920in}}%
\pgfpathlineto{\pgfqpoint{2.614755in}{1.337907in}}%
\pgfpathlineto{\pgfqpoint{2.902048in}{1.242143in}}%
\pgfpathlineto{\pgfqpoint{2.899243in}{1.233728in}}%
\pgfusepath{fill}%
\end{pgfscope}%
\begin{pgfscope}%
\pgfpathrectangle{\pgfqpoint{1.432000in}{0.528000in}}{\pgfqpoint{3.696000in}{3.696000in}} %
\pgfusepath{clip}%
\pgfsetbuttcap%
\pgfsetroundjoin%
\definecolor{currentfill}{rgb}{0.146180,0.515413,0.556823}%
\pgfsetfillcolor{currentfill}%
\pgfsetlinewidth{0.000000pt}%
\definecolor{currentstroke}{rgb}{0.000000,0.000000,0.000000}%
\pgfsetstrokecolor{currentstroke}%
\pgfsetdash{}{0pt}%
\pgfpathmoveto{\pgfqpoint{3.007630in}{1.233728in}}%
\pgfpathlineto{\pgfqpoint{2.720337in}{1.329492in}}%
\pgfpathlineto{\pgfqpoint{2.721739in}{1.319674in}}%
\pgfpathlineto{\pgfqpoint{2.683871in}{1.346323in}}%
\pgfpathlineto{\pgfqpoint{2.730155in}{1.344920in}}%
\pgfpathlineto{\pgfqpoint{2.723142in}{1.337907in}}%
\pgfpathlineto{\pgfqpoint{3.010435in}{1.242143in}}%
\pgfpathlineto{\pgfqpoint{3.007630in}{1.233728in}}%
\pgfusepath{fill}%
\end{pgfscope}%
\begin{pgfscope}%
\pgfpathrectangle{\pgfqpoint{1.432000in}{0.528000in}}{\pgfqpoint{3.696000in}{3.696000in}} %
\pgfusepath{clip}%
\pgfsetbuttcap%
\pgfsetroundjoin%
\definecolor{currentfill}{rgb}{0.283229,0.120777,0.440584}%
\pgfsetfillcolor{currentfill}%
\pgfsetlinewidth{0.000000pt}%
\definecolor{currentstroke}{rgb}{0.000000,0.000000,0.000000}%
\pgfsetstrokecolor{currentstroke}%
\pgfsetdash{}{0pt}%
\pgfpathmoveto{\pgfqpoint{3.007049in}{1.233969in}}%
\pgfpathlineto{\pgfqpoint{2.825977in}{1.324504in}}%
\pgfpathlineto{\pgfqpoint{2.825977in}{1.314587in}}%
\pgfpathlineto{\pgfqpoint{2.792258in}{1.346323in}}%
\pgfpathlineto{\pgfqpoint{2.837878in}{1.338389in}}%
\pgfpathlineto{\pgfqpoint{2.829944in}{1.332438in}}%
\pgfpathlineto{\pgfqpoint{3.011016in}{1.241902in}}%
\pgfpathlineto{\pgfqpoint{3.007049in}{1.233969in}}%
\pgfusepath{fill}%
\end{pgfscope}%
\begin{pgfscope}%
\pgfpathrectangle{\pgfqpoint{1.432000in}{0.528000in}}{\pgfqpoint{3.696000in}{3.696000in}} %
\pgfusepath{clip}%
\pgfsetbuttcap%
\pgfsetroundjoin%
\definecolor{currentfill}{rgb}{0.253935,0.265254,0.529983}%
\pgfsetfillcolor{currentfill}%
\pgfsetlinewidth{0.000000pt}%
\definecolor{currentstroke}{rgb}{0.000000,0.000000,0.000000}%
\pgfsetstrokecolor{currentstroke}%
\pgfsetdash{}{0pt}%
\pgfpathmoveto{\pgfqpoint{3.116017in}{1.233728in}}%
\pgfpathlineto{\pgfqpoint{2.828724in}{1.329492in}}%
\pgfpathlineto{\pgfqpoint{2.830126in}{1.319674in}}%
\pgfpathlineto{\pgfqpoint{2.792258in}{1.346323in}}%
\pgfpathlineto{\pgfqpoint{2.838542in}{1.344920in}}%
\pgfpathlineto{\pgfqpoint{2.831529in}{1.337907in}}%
\pgfpathlineto{\pgfqpoint{3.118822in}{1.242143in}}%
\pgfpathlineto{\pgfqpoint{3.116017in}{1.233728in}}%
\pgfusepath{fill}%
\end{pgfscope}%
\begin{pgfscope}%
\pgfpathrectangle{\pgfqpoint{1.432000in}{0.528000in}}{\pgfqpoint{3.696000in}{3.696000in}} %
\pgfusepath{clip}%
\pgfsetbuttcap%
\pgfsetroundjoin%
\definecolor{currentfill}{rgb}{0.172719,0.448791,0.557885}%
\pgfsetfillcolor{currentfill}%
\pgfsetlinewidth{0.000000pt}%
\definecolor{currentstroke}{rgb}{0.000000,0.000000,0.000000}%
\pgfsetstrokecolor{currentstroke}%
\pgfsetdash{}{0pt}%
\pgfpathmoveto{\pgfqpoint{3.115436in}{1.233969in}}%
\pgfpathlineto{\pgfqpoint{2.934364in}{1.324504in}}%
\pgfpathlineto{\pgfqpoint{2.934364in}{1.314587in}}%
\pgfpathlineto{\pgfqpoint{2.900645in}{1.346323in}}%
\pgfpathlineto{\pgfqpoint{2.946265in}{1.338389in}}%
\pgfpathlineto{\pgfqpoint{2.938331in}{1.332438in}}%
\pgfpathlineto{\pgfqpoint{3.119403in}{1.241902in}}%
\pgfpathlineto{\pgfqpoint{3.115436in}{1.233969in}}%
\pgfusepath{fill}%
\end{pgfscope}%
\begin{pgfscope}%
\pgfpathrectangle{\pgfqpoint{1.432000in}{0.528000in}}{\pgfqpoint{3.696000in}{3.696000in}} %
\pgfusepath{clip}%
\pgfsetbuttcap%
\pgfsetroundjoin%
\definecolor{currentfill}{rgb}{0.277018,0.050344,0.375715}%
\pgfsetfillcolor{currentfill}%
\pgfsetlinewidth{0.000000pt}%
\definecolor{currentstroke}{rgb}{0.000000,0.000000,0.000000}%
\pgfsetstrokecolor{currentstroke}%
\pgfsetdash{}{0pt}%
\pgfpathmoveto{\pgfqpoint{3.225806in}{1.233500in}}%
\pgfpathlineto{\pgfqpoint{3.048949in}{1.233500in}}%
\pgfpathlineto{\pgfqpoint{3.053384in}{1.224630in}}%
\pgfpathlineto{\pgfqpoint{3.009032in}{1.237935in}}%
\pgfpathlineto{\pgfqpoint{3.053384in}{1.251241in}}%
\pgfpathlineto{\pgfqpoint{3.048949in}{1.242371in}}%
\pgfpathlineto{\pgfqpoint{3.225806in}{1.242371in}}%
\pgfpathlineto{\pgfqpoint{3.225806in}{1.233500in}}%
\pgfusepath{fill}%
\end{pgfscope}%
\begin{pgfscope}%
\pgfpathrectangle{\pgfqpoint{1.432000in}{0.528000in}}{\pgfqpoint{3.696000in}{3.696000in}} %
\pgfusepath{clip}%
\pgfsetbuttcap%
\pgfsetroundjoin%
\definecolor{currentfill}{rgb}{0.123463,0.581687,0.547445}%
\pgfsetfillcolor{currentfill}%
\pgfsetlinewidth{0.000000pt}%
\definecolor{currentstroke}{rgb}{0.000000,0.000000,0.000000}%
\pgfsetstrokecolor{currentstroke}%
\pgfsetdash{}{0pt}%
\pgfpathmoveto{\pgfqpoint{3.223823in}{1.233969in}}%
\pgfpathlineto{\pgfqpoint{3.042751in}{1.324504in}}%
\pgfpathlineto{\pgfqpoint{3.042751in}{1.314587in}}%
\pgfpathlineto{\pgfqpoint{3.009032in}{1.346323in}}%
\pgfpathlineto{\pgfqpoint{3.054652in}{1.338389in}}%
\pgfpathlineto{\pgfqpoint{3.046718in}{1.332438in}}%
\pgfpathlineto{\pgfqpoint{3.227790in}{1.241902in}}%
\pgfpathlineto{\pgfqpoint{3.223823in}{1.233969in}}%
\pgfusepath{fill}%
\end{pgfscope}%
\begin{pgfscope}%
\pgfpathrectangle{\pgfqpoint{1.432000in}{0.528000in}}{\pgfqpoint{3.696000in}{3.696000in}} %
\pgfusepath{clip}%
\pgfsetbuttcap%
\pgfsetroundjoin%
\definecolor{currentfill}{rgb}{0.262138,0.242286,0.520837}%
\pgfsetfillcolor{currentfill}%
\pgfsetlinewidth{0.000000pt}%
\definecolor{currentstroke}{rgb}{0.000000,0.000000,0.000000}%
\pgfsetstrokecolor{currentstroke}%
\pgfsetdash{}{0pt}%
\pgfpathmoveto{\pgfqpoint{3.334194in}{1.233500in}}%
\pgfpathlineto{\pgfqpoint{3.157336in}{1.233500in}}%
\pgfpathlineto{\pgfqpoint{3.161771in}{1.224630in}}%
\pgfpathlineto{\pgfqpoint{3.117419in}{1.237935in}}%
\pgfpathlineto{\pgfqpoint{3.161771in}{1.251241in}}%
\pgfpathlineto{\pgfqpoint{3.157336in}{1.242371in}}%
\pgfpathlineto{\pgfqpoint{3.334194in}{1.242371in}}%
\pgfpathlineto{\pgfqpoint{3.334194in}{1.233500in}}%
\pgfusepath{fill}%
\end{pgfscope}%
\begin{pgfscope}%
\pgfpathrectangle{\pgfqpoint{1.432000in}{0.528000in}}{\pgfqpoint{3.696000in}{3.696000in}} %
\pgfusepath{clip}%
\pgfsetbuttcap%
\pgfsetroundjoin%
\definecolor{currentfill}{rgb}{0.244972,0.287675,0.537260}%
\pgfsetfillcolor{currentfill}%
\pgfsetlinewidth{0.000000pt}%
\definecolor{currentstroke}{rgb}{0.000000,0.000000,0.000000}%
\pgfsetstrokecolor{currentstroke}%
\pgfsetdash{}{0pt}%
\pgfpathmoveto{\pgfqpoint{3.332210in}{1.233969in}}%
\pgfpathlineto{\pgfqpoint{3.151139in}{1.324504in}}%
\pgfpathlineto{\pgfqpoint{3.151139in}{1.314587in}}%
\pgfpathlineto{\pgfqpoint{3.117419in}{1.346323in}}%
\pgfpathlineto{\pgfqpoint{3.163039in}{1.338389in}}%
\pgfpathlineto{\pgfqpoint{3.155106in}{1.332438in}}%
\pgfpathlineto{\pgfqpoint{3.336177in}{1.241902in}}%
\pgfpathlineto{\pgfqpoint{3.332210in}{1.233969in}}%
\pgfusepath{fill}%
\end{pgfscope}%
\begin{pgfscope}%
\pgfpathrectangle{\pgfqpoint{1.432000in}{0.528000in}}{\pgfqpoint{3.696000in}{3.696000in}} %
\pgfusepath{clip}%
\pgfsetbuttcap%
\pgfsetroundjoin%
\definecolor{currentfill}{rgb}{0.283187,0.125848,0.444960}%
\pgfsetfillcolor{currentfill}%
\pgfsetlinewidth{0.000000pt}%
\definecolor{currentstroke}{rgb}{0.000000,0.000000,0.000000}%
\pgfsetstrokecolor{currentstroke}%
\pgfsetdash{}{0pt}%
\pgfpathmoveto{\pgfqpoint{3.442581in}{1.233500in}}%
\pgfpathlineto{\pgfqpoint{3.265723in}{1.233500in}}%
\pgfpathlineto{\pgfqpoint{3.270158in}{1.224630in}}%
\pgfpathlineto{\pgfqpoint{3.225806in}{1.237935in}}%
\pgfpathlineto{\pgfqpoint{3.270158in}{1.251241in}}%
\pgfpathlineto{\pgfqpoint{3.265723in}{1.242371in}}%
\pgfpathlineto{\pgfqpoint{3.442581in}{1.242371in}}%
\pgfpathlineto{\pgfqpoint{3.442581in}{1.233500in}}%
\pgfusepath{fill}%
\end{pgfscope}%
\begin{pgfscope}%
\pgfpathrectangle{\pgfqpoint{1.432000in}{0.528000in}}{\pgfqpoint{3.696000in}{3.696000in}} %
\pgfusepath{clip}%
\pgfsetbuttcap%
\pgfsetroundjoin%
\definecolor{currentfill}{rgb}{0.282327,0.094955,0.417331}%
\pgfsetfillcolor{currentfill}%
\pgfsetlinewidth{0.000000pt}%
\definecolor{currentstroke}{rgb}{0.000000,0.000000,0.000000}%
\pgfsetstrokecolor{currentstroke}%
\pgfsetdash{}{0pt}%
\pgfpathmoveto{\pgfqpoint{3.442581in}{1.233500in}}%
\pgfpathlineto{\pgfqpoint{3.374110in}{1.233500in}}%
\pgfpathlineto{\pgfqpoint{3.378546in}{1.224630in}}%
\pgfpathlineto{\pgfqpoint{3.334194in}{1.237935in}}%
\pgfpathlineto{\pgfqpoint{3.378546in}{1.251241in}}%
\pgfpathlineto{\pgfqpoint{3.374110in}{1.242371in}}%
\pgfpathlineto{\pgfqpoint{3.442581in}{1.242371in}}%
\pgfpathlineto{\pgfqpoint{3.442581in}{1.233500in}}%
\pgfusepath{fill}%
\end{pgfscope}%
\begin{pgfscope}%
\pgfpathrectangle{\pgfqpoint{1.432000in}{0.528000in}}{\pgfqpoint{3.696000in}{3.696000in}} %
\pgfusepath{clip}%
\pgfsetbuttcap%
\pgfsetroundjoin%
\definecolor{currentfill}{rgb}{0.267004,0.004874,0.329415}%
\pgfsetfillcolor{currentfill}%
\pgfsetlinewidth{0.000000pt}%
\definecolor{currentstroke}{rgb}{0.000000,0.000000,0.000000}%
\pgfsetstrokecolor{currentstroke}%
\pgfsetdash{}{0pt}%
\pgfpathmoveto{\pgfqpoint{3.440597in}{1.233969in}}%
\pgfpathlineto{\pgfqpoint{3.259526in}{1.324504in}}%
\pgfpathlineto{\pgfqpoint{3.259526in}{1.314587in}}%
\pgfpathlineto{\pgfqpoint{3.225806in}{1.346323in}}%
\pgfpathlineto{\pgfqpoint{3.271427in}{1.338389in}}%
\pgfpathlineto{\pgfqpoint{3.263493in}{1.332438in}}%
\pgfpathlineto{\pgfqpoint{3.444564in}{1.241902in}}%
\pgfpathlineto{\pgfqpoint{3.440597in}{1.233969in}}%
\pgfusepath{fill}%
\end{pgfscope}%
\begin{pgfscope}%
\pgfpathrectangle{\pgfqpoint{1.432000in}{0.528000in}}{\pgfqpoint{3.696000in}{3.696000in}} %
\pgfusepath{clip}%
\pgfsetbuttcap%
\pgfsetroundjoin%
\definecolor{currentfill}{rgb}{0.282910,0.105393,0.426902}%
\pgfsetfillcolor{currentfill}%
\pgfsetlinewidth{0.000000pt}%
\definecolor{currentstroke}{rgb}{0.000000,0.000000,0.000000}%
\pgfsetstrokecolor{currentstroke}%
\pgfsetdash{}{0pt}%
\pgfpathmoveto{\pgfqpoint{3.550968in}{1.233500in}}%
\pgfpathlineto{\pgfqpoint{3.482497in}{1.233500in}}%
\pgfpathlineto{\pgfqpoint{3.486933in}{1.224630in}}%
\pgfpathlineto{\pgfqpoint{3.442581in}{1.237935in}}%
\pgfpathlineto{\pgfqpoint{3.486933in}{1.251241in}}%
\pgfpathlineto{\pgfqpoint{3.482497in}{1.242371in}}%
\pgfpathlineto{\pgfqpoint{3.550968in}{1.242371in}}%
\pgfpathlineto{\pgfqpoint{3.550968in}{1.233500in}}%
\pgfusepath{fill}%
\end{pgfscope}%
\begin{pgfscope}%
\pgfpathrectangle{\pgfqpoint{1.432000in}{0.528000in}}{\pgfqpoint{3.696000in}{3.696000in}} %
\pgfusepath{clip}%
\pgfsetbuttcap%
\pgfsetroundjoin%
\definecolor{currentfill}{rgb}{0.267004,0.004874,0.329415}%
\pgfsetfillcolor{currentfill}%
\pgfsetlinewidth{0.000000pt}%
\definecolor{currentstroke}{rgb}{0.000000,0.000000,0.000000}%
\pgfsetstrokecolor{currentstroke}%
\pgfsetdash{}{0pt}%
\pgfpathmoveto{\pgfqpoint{3.659355in}{1.233500in}}%
\pgfpathlineto{\pgfqpoint{3.482497in}{1.233500in}}%
\pgfpathlineto{\pgfqpoint{3.486933in}{1.224630in}}%
\pgfpathlineto{\pgfqpoint{3.442581in}{1.237935in}}%
\pgfpathlineto{\pgfqpoint{3.486933in}{1.251241in}}%
\pgfpathlineto{\pgfqpoint{3.482497in}{1.242371in}}%
\pgfpathlineto{\pgfqpoint{3.659355in}{1.242371in}}%
\pgfpathlineto{\pgfqpoint{3.659355in}{1.233500in}}%
\pgfusepath{fill}%
\end{pgfscope}%
\begin{pgfscope}%
\pgfpathrectangle{\pgfqpoint{1.432000in}{0.528000in}}{\pgfqpoint{3.696000in}{3.696000in}} %
\pgfusepath{clip}%
\pgfsetbuttcap%
\pgfsetroundjoin%
\definecolor{currentfill}{rgb}{0.282910,0.105393,0.426902}%
\pgfsetfillcolor{currentfill}%
\pgfsetlinewidth{0.000000pt}%
\definecolor{currentstroke}{rgb}{0.000000,0.000000,0.000000}%
\pgfsetstrokecolor{currentstroke}%
\pgfsetdash{}{0pt}%
\pgfpathmoveto{\pgfqpoint{3.767742in}{1.233500in}}%
\pgfpathlineto{\pgfqpoint{3.590885in}{1.233500in}}%
\pgfpathlineto{\pgfqpoint{3.595320in}{1.224630in}}%
\pgfpathlineto{\pgfqpoint{3.550968in}{1.237935in}}%
\pgfpathlineto{\pgfqpoint{3.595320in}{1.251241in}}%
\pgfpathlineto{\pgfqpoint{3.590885in}{1.242371in}}%
\pgfpathlineto{\pgfqpoint{3.767742in}{1.242371in}}%
\pgfpathlineto{\pgfqpoint{3.767742in}{1.233500in}}%
\pgfusepath{fill}%
\end{pgfscope}%
\begin{pgfscope}%
\pgfpathrectangle{\pgfqpoint{1.432000in}{0.528000in}}{\pgfqpoint{3.696000in}{3.696000in}} %
\pgfusepath{clip}%
\pgfsetbuttcap%
\pgfsetroundjoin%
\definecolor{currentfill}{rgb}{0.235526,0.309527,0.542944}%
\pgfsetfillcolor{currentfill}%
\pgfsetlinewidth{0.000000pt}%
\definecolor{currentstroke}{rgb}{0.000000,0.000000,0.000000}%
\pgfsetstrokecolor{currentstroke}%
\pgfsetdash{}{0pt}%
\pgfpathmoveto{\pgfqpoint{4.310753in}{1.233633in}}%
\pgfpathlineto{\pgfqpoint{3.915930in}{1.134927in}}%
\pgfpathlineto{\pgfqpoint{3.922384in}{1.127397in}}%
\pgfpathlineto{\pgfqpoint{3.876129in}{1.129548in}}%
\pgfpathlineto{\pgfqpoint{3.915930in}{1.153214in}}%
\pgfpathlineto{\pgfqpoint{3.913778in}{1.143532in}}%
\pgfpathlineto{\pgfqpoint{4.308602in}{1.242238in}}%
\pgfpathlineto{\pgfqpoint{4.310753in}{1.233633in}}%
\pgfusepath{fill}%
\end{pgfscope}%
\begin{pgfscope}%
\pgfpathrectangle{\pgfqpoint{1.432000in}{0.528000in}}{\pgfqpoint{3.696000in}{3.696000in}} %
\pgfusepath{clip}%
\pgfsetbuttcap%
\pgfsetroundjoin%
\definecolor{currentfill}{rgb}{0.283072,0.130895,0.449241}%
\pgfsetfillcolor{currentfill}%
\pgfsetlinewidth{0.000000pt}%
\definecolor{currentstroke}{rgb}{0.000000,0.000000,0.000000}%
\pgfsetstrokecolor{currentstroke}%
\pgfsetdash{}{0pt}%
\pgfpathmoveto{\pgfqpoint{4.419140in}{1.233633in}}%
\pgfpathlineto{\pgfqpoint{4.024317in}{1.134927in}}%
\pgfpathlineto{\pgfqpoint{4.030771in}{1.127397in}}%
\pgfpathlineto{\pgfqpoint{3.984516in}{1.129548in}}%
\pgfpathlineto{\pgfqpoint{4.024317in}{1.153214in}}%
\pgfpathlineto{\pgfqpoint{4.022165in}{1.143532in}}%
\pgfpathlineto{\pgfqpoint{4.416989in}{1.242238in}}%
\pgfpathlineto{\pgfqpoint{4.419140in}{1.233633in}}%
\pgfusepath{fill}%
\end{pgfscope}%
\begin{pgfscope}%
\pgfpathrectangle{\pgfqpoint{1.432000in}{0.528000in}}{\pgfqpoint{3.696000in}{3.696000in}} %
\pgfusepath{clip}%
\pgfsetbuttcap%
\pgfsetroundjoin%
\definecolor{currentfill}{rgb}{0.280267,0.073417,0.397163}%
\pgfsetfillcolor{currentfill}%
\pgfsetlinewidth{0.000000pt}%
\definecolor{currentstroke}{rgb}{0.000000,0.000000,0.000000}%
\pgfsetstrokecolor{currentstroke}%
\pgfsetdash{}{0pt}%
\pgfpathmoveto{\pgfqpoint{4.528435in}{1.233969in}}%
\pgfpathlineto{\pgfqpoint{4.347364in}{1.143433in}}%
\pgfpathlineto{\pgfqpoint{4.355297in}{1.137482in}}%
\pgfpathlineto{\pgfqpoint{4.309677in}{1.129548in}}%
\pgfpathlineto{\pgfqpoint{4.343397in}{1.161284in}}%
\pgfpathlineto{\pgfqpoint{4.343397in}{1.151367in}}%
\pgfpathlineto{\pgfqpoint{4.524468in}{1.241902in}}%
\pgfpathlineto{\pgfqpoint{4.528435in}{1.233969in}}%
\pgfusepath{fill}%
\end{pgfscope}%
\begin{pgfscope}%
\pgfpathrectangle{\pgfqpoint{1.432000in}{0.528000in}}{\pgfqpoint{3.696000in}{3.696000in}} %
\pgfusepath{clip}%
\pgfsetbuttcap%
\pgfsetroundjoin%
\definecolor{currentfill}{rgb}{0.267004,0.004874,0.329415}%
\pgfsetfillcolor{currentfill}%
\pgfsetlinewidth{0.000000pt}%
\definecolor{currentstroke}{rgb}{0.000000,0.000000,0.000000}%
\pgfsetstrokecolor{currentstroke}%
\pgfsetdash{}{0pt}%
\pgfpathmoveto{\pgfqpoint{4.636822in}{1.233969in}}%
\pgfpathlineto{\pgfqpoint{4.455751in}{1.143433in}}%
\pgfpathlineto{\pgfqpoint{4.463685in}{1.137482in}}%
\pgfpathlineto{\pgfqpoint{4.418065in}{1.129548in}}%
\pgfpathlineto{\pgfqpoint{4.451784in}{1.161284in}}%
\pgfpathlineto{\pgfqpoint{4.451784in}{1.151367in}}%
\pgfpathlineto{\pgfqpoint{4.632855in}{1.241902in}}%
\pgfpathlineto{\pgfqpoint{4.636822in}{1.233969in}}%
\pgfusepath{fill}%
\end{pgfscope}%
\begin{pgfscope}%
\pgfpathrectangle{\pgfqpoint{1.432000in}{0.528000in}}{\pgfqpoint{3.696000in}{3.696000in}} %
\pgfusepath{clip}%
\pgfsetbuttcap%
\pgfsetroundjoin%
\definecolor{currentfill}{rgb}{0.266580,0.228262,0.514349}%
\pgfsetfillcolor{currentfill}%
\pgfsetlinewidth{0.000000pt}%
\definecolor{currentstroke}{rgb}{0.000000,0.000000,0.000000}%
\pgfsetstrokecolor{currentstroke}%
\pgfsetdash{}{0pt}%
\pgfpathmoveto{\pgfqpoint{4.634839in}{1.233500in}}%
\pgfpathlineto{\pgfqpoint{4.457981in}{1.233500in}}%
\pgfpathlineto{\pgfqpoint{4.462417in}{1.224630in}}%
\pgfpathlineto{\pgfqpoint{4.418065in}{1.237935in}}%
\pgfpathlineto{\pgfqpoint{4.462417in}{1.251241in}}%
\pgfpathlineto{\pgfqpoint{4.457981in}{1.242371in}}%
\pgfpathlineto{\pgfqpoint{4.634839in}{1.242371in}}%
\pgfpathlineto{\pgfqpoint{4.634839in}{1.233500in}}%
\pgfusepath{fill}%
\end{pgfscope}%
\begin{pgfscope}%
\pgfpathrectangle{\pgfqpoint{1.432000in}{0.528000in}}{\pgfqpoint{3.696000in}{3.696000in}} %
\pgfusepath{clip}%
\pgfsetbuttcap%
\pgfsetroundjoin%
\definecolor{currentfill}{rgb}{0.281412,0.155834,0.469201}%
\pgfsetfillcolor{currentfill}%
\pgfsetlinewidth{0.000000pt}%
\definecolor{currentstroke}{rgb}{0.000000,0.000000,0.000000}%
\pgfsetstrokecolor{currentstroke}%
\pgfsetdash{}{0pt}%
\pgfpathmoveto{\pgfqpoint{4.743226in}{1.233500in}}%
\pgfpathlineto{\pgfqpoint{4.566368in}{1.233500in}}%
\pgfpathlineto{\pgfqpoint{4.570804in}{1.224630in}}%
\pgfpathlineto{\pgfqpoint{4.526452in}{1.237935in}}%
\pgfpathlineto{\pgfqpoint{4.570804in}{1.251241in}}%
\pgfpathlineto{\pgfqpoint{4.566368in}{1.242371in}}%
\pgfpathlineto{\pgfqpoint{4.743226in}{1.242371in}}%
\pgfpathlineto{\pgfqpoint{4.743226in}{1.233500in}}%
\pgfusepath{fill}%
\end{pgfscope}%
\begin{pgfscope}%
\pgfpathrectangle{\pgfqpoint{1.432000in}{0.528000in}}{\pgfqpoint{3.696000in}{3.696000in}} %
\pgfusepath{clip}%
\pgfsetbuttcap%
\pgfsetroundjoin%
\definecolor{currentfill}{rgb}{0.233603,0.313828,0.543914}%
\pgfsetfillcolor{currentfill}%
\pgfsetlinewidth{0.000000pt}%
\definecolor{currentstroke}{rgb}{0.000000,0.000000,0.000000}%
\pgfsetstrokecolor{currentstroke}%
\pgfsetdash{}{0pt}%
\pgfpathmoveto{\pgfqpoint{4.743226in}{1.233500in}}%
\pgfpathlineto{\pgfqpoint{4.674756in}{1.233500in}}%
\pgfpathlineto{\pgfqpoint{4.679191in}{1.224630in}}%
\pgfpathlineto{\pgfqpoint{4.634839in}{1.237935in}}%
\pgfpathlineto{\pgfqpoint{4.679191in}{1.251241in}}%
\pgfpathlineto{\pgfqpoint{4.674756in}{1.242371in}}%
\pgfpathlineto{\pgfqpoint{4.743226in}{1.242371in}}%
\pgfpathlineto{\pgfqpoint{4.743226in}{1.233500in}}%
\pgfusepath{fill}%
\end{pgfscope}%
\begin{pgfscope}%
\pgfpathrectangle{\pgfqpoint{1.432000in}{0.528000in}}{\pgfqpoint{3.696000in}{3.696000in}} %
\pgfusepath{clip}%
\pgfsetbuttcap%
\pgfsetroundjoin%
\definecolor{currentfill}{rgb}{0.171176,0.452530,0.557965}%
\pgfsetfillcolor{currentfill}%
\pgfsetlinewidth{0.000000pt}%
\definecolor{currentstroke}{rgb}{0.000000,0.000000,0.000000}%
\pgfsetstrokecolor{currentstroke}%
\pgfsetdash{}{0pt}%
\pgfpathmoveto{\pgfqpoint{4.851613in}{1.233500in}}%
\pgfpathlineto{\pgfqpoint{4.783143in}{1.233500in}}%
\pgfpathlineto{\pgfqpoint{4.787578in}{1.224630in}}%
\pgfpathlineto{\pgfqpoint{4.743226in}{1.237935in}}%
\pgfpathlineto{\pgfqpoint{4.787578in}{1.251241in}}%
\pgfpathlineto{\pgfqpoint{4.783143in}{1.242371in}}%
\pgfpathlineto{\pgfqpoint{4.851613in}{1.242371in}}%
\pgfpathlineto{\pgfqpoint{4.851613in}{1.233500in}}%
\pgfusepath{fill}%
\end{pgfscope}%
\begin{pgfscope}%
\pgfpathrectangle{\pgfqpoint{1.432000in}{0.528000in}}{\pgfqpoint{3.696000in}{3.696000in}} %
\pgfusepath{clip}%
\pgfsetbuttcap%
\pgfsetroundjoin%
\definecolor{currentfill}{rgb}{0.260571,0.246922,0.522828}%
\pgfsetfillcolor{currentfill}%
\pgfsetlinewidth{0.000000pt}%
\definecolor{currentstroke}{rgb}{0.000000,0.000000,0.000000}%
\pgfsetstrokecolor{currentstroke}%
\pgfsetdash{}{0pt}%
\pgfpathmoveto{\pgfqpoint{4.856048in}{1.237935in}}%
\pgfpathlineto{\pgfqpoint{4.853831in}{1.241776in}}%
\pgfpathlineto{\pgfqpoint{4.849395in}{1.241776in}}%
\pgfpathlineto{\pgfqpoint{4.847178in}{1.237935in}}%
\pgfpathlineto{\pgfqpoint{4.849395in}{1.234094in}}%
\pgfpathlineto{\pgfqpoint{4.853831in}{1.234094in}}%
\pgfpathlineto{\pgfqpoint{4.856048in}{1.237935in}}%
\pgfpathlineto{\pgfqpoint{4.853831in}{1.241776in}}%
\pgfusepath{fill}%
\end{pgfscope}%
\begin{pgfscope}%
\pgfpathrectangle{\pgfqpoint{1.432000in}{0.528000in}}{\pgfqpoint{3.696000in}{3.696000in}} %
\pgfusepath{clip}%
\pgfsetbuttcap%
\pgfsetroundjoin%
\definecolor{currentfill}{rgb}{0.278826,0.175490,0.483397}%
\pgfsetfillcolor{currentfill}%
\pgfsetlinewidth{0.000000pt}%
\definecolor{currentstroke}{rgb}{0.000000,0.000000,0.000000}%
\pgfsetstrokecolor{currentstroke}%
\pgfsetdash{}{0pt}%
\pgfpathmoveto{\pgfqpoint{4.960000in}{1.233500in}}%
\pgfpathlineto{\pgfqpoint{4.891530in}{1.233500in}}%
\pgfpathlineto{\pgfqpoint{4.895965in}{1.224630in}}%
\pgfpathlineto{\pgfqpoint{4.851613in}{1.237935in}}%
\pgfpathlineto{\pgfqpoint{4.895965in}{1.251241in}}%
\pgfpathlineto{\pgfqpoint{4.891530in}{1.242371in}}%
\pgfpathlineto{\pgfqpoint{4.960000in}{1.242371in}}%
\pgfpathlineto{\pgfqpoint{4.960000in}{1.233500in}}%
\pgfusepath{fill}%
\end{pgfscope}%
\begin{pgfscope}%
\pgfpathrectangle{\pgfqpoint{1.432000in}{0.528000in}}{\pgfqpoint{3.696000in}{3.696000in}} %
\pgfusepath{clip}%
\pgfsetbuttcap%
\pgfsetroundjoin%
\definecolor{currentfill}{rgb}{0.136408,0.541173,0.554483}%
\pgfsetfillcolor{currentfill}%
\pgfsetlinewidth{0.000000pt}%
\definecolor{currentstroke}{rgb}{0.000000,0.000000,0.000000}%
\pgfsetstrokecolor{currentstroke}%
\pgfsetdash{}{0pt}%
\pgfpathmoveto{\pgfqpoint{4.964435in}{1.237935in}}%
\pgfpathlineto{\pgfqpoint{4.962218in}{1.241776in}}%
\pgfpathlineto{\pgfqpoint{4.957782in}{1.241776in}}%
\pgfpathlineto{\pgfqpoint{4.955565in}{1.237935in}}%
\pgfpathlineto{\pgfqpoint{4.957782in}{1.234094in}}%
\pgfpathlineto{\pgfqpoint{4.962218in}{1.234094in}}%
\pgfpathlineto{\pgfqpoint{4.964435in}{1.237935in}}%
\pgfpathlineto{\pgfqpoint{4.962218in}{1.241776in}}%
\pgfusepath{fill}%
\end{pgfscope}%
\begin{pgfscope}%
\pgfpathrectangle{\pgfqpoint{1.432000in}{0.528000in}}{\pgfqpoint{3.696000in}{3.696000in}} %
\pgfusepath{clip}%
\pgfsetbuttcap%
\pgfsetroundjoin%
\definecolor{currentfill}{rgb}{0.271828,0.209303,0.504434}%
\pgfsetfillcolor{currentfill}%
\pgfsetlinewidth{0.000000pt}%
\definecolor{currentstroke}{rgb}{0.000000,0.000000,0.000000}%
\pgfsetstrokecolor{currentstroke}%
\pgfsetdash{}{0pt}%
\pgfpathmoveto{\pgfqpoint{1.604435in}{1.346323in}}%
\pgfpathlineto{\pgfqpoint{1.604435in}{1.277852in}}%
\pgfpathlineto{\pgfqpoint{1.613306in}{1.282287in}}%
\pgfpathlineto{\pgfqpoint{1.600000in}{1.237935in}}%
\pgfpathlineto{\pgfqpoint{1.586694in}{1.282287in}}%
\pgfpathlineto{\pgfqpoint{1.595565in}{1.277852in}}%
\pgfpathlineto{\pgfqpoint{1.595565in}{1.346323in}}%
\pgfpathlineto{\pgfqpoint{1.604435in}{1.346323in}}%
\pgfusepath{fill}%
\end{pgfscope}%
\begin{pgfscope}%
\pgfpathrectangle{\pgfqpoint{1.432000in}{0.528000in}}{\pgfqpoint{3.696000in}{3.696000in}} %
\pgfusepath{clip}%
\pgfsetbuttcap%
\pgfsetroundjoin%
\definecolor{currentfill}{rgb}{0.253935,0.265254,0.529983}%
\pgfsetfillcolor{currentfill}%
\pgfsetlinewidth{0.000000pt}%
\definecolor{currentstroke}{rgb}{0.000000,0.000000,0.000000}%
\pgfsetstrokecolor{currentstroke}%
\pgfsetdash{}{0pt}%
\pgfpathmoveto{\pgfqpoint{1.711523in}{1.343186in}}%
\pgfpathlineto{\pgfqpoint{1.631362in}{1.263025in}}%
\pgfpathlineto{\pgfqpoint{1.640770in}{1.259889in}}%
\pgfpathlineto{\pgfqpoint{1.600000in}{1.237935in}}%
\pgfpathlineto{\pgfqpoint{1.621953in}{1.278706in}}%
\pgfpathlineto{\pgfqpoint{1.625089in}{1.269297in}}%
\pgfpathlineto{\pgfqpoint{1.705251in}{1.349459in}}%
\pgfpathlineto{\pgfqpoint{1.711523in}{1.343186in}}%
\pgfusepath{fill}%
\end{pgfscope}%
\begin{pgfscope}%
\pgfpathrectangle{\pgfqpoint{1.432000in}{0.528000in}}{\pgfqpoint{3.696000in}{3.696000in}} %
\pgfusepath{clip}%
\pgfsetbuttcap%
\pgfsetroundjoin%
\definecolor{currentfill}{rgb}{0.172719,0.448791,0.557885}%
\pgfsetfillcolor{currentfill}%
\pgfsetlinewidth{0.000000pt}%
\definecolor{currentstroke}{rgb}{0.000000,0.000000,0.000000}%
\pgfsetstrokecolor{currentstroke}%
\pgfsetdash{}{0pt}%
\pgfpathmoveto{\pgfqpoint{1.819910in}{1.343186in}}%
\pgfpathlineto{\pgfqpoint{1.739749in}{1.263025in}}%
\pgfpathlineto{\pgfqpoint{1.749157in}{1.259889in}}%
\pgfpathlineto{\pgfqpoint{1.708387in}{1.237935in}}%
\pgfpathlineto{\pgfqpoint{1.730340in}{1.278706in}}%
\pgfpathlineto{\pgfqpoint{1.733476in}{1.269297in}}%
\pgfpathlineto{\pgfqpoint{1.813638in}{1.349459in}}%
\pgfpathlineto{\pgfqpoint{1.819910in}{1.343186in}}%
\pgfusepath{fill}%
\end{pgfscope}%
\begin{pgfscope}%
\pgfpathrectangle{\pgfqpoint{1.432000in}{0.528000in}}{\pgfqpoint{3.696000in}{3.696000in}} %
\pgfusepath{clip}%
\pgfsetbuttcap%
\pgfsetroundjoin%
\definecolor{currentfill}{rgb}{0.274128,0.199721,0.498911}%
\pgfsetfillcolor{currentfill}%
\pgfsetlinewidth{0.000000pt}%
\definecolor{currentstroke}{rgb}{0.000000,0.000000,0.000000}%
\pgfsetstrokecolor{currentstroke}%
\pgfsetdash{}{0pt}%
\pgfpathmoveto{\pgfqpoint{1.927145in}{1.342356in}}%
\pgfpathlineto{\pgfqpoint{1.746073in}{1.251820in}}%
\pgfpathlineto{\pgfqpoint{1.754007in}{1.245869in}}%
\pgfpathlineto{\pgfqpoint{1.708387in}{1.237935in}}%
\pgfpathlineto{\pgfqpoint{1.742106in}{1.269671in}}%
\pgfpathlineto{\pgfqpoint{1.742106in}{1.259754in}}%
\pgfpathlineto{\pgfqpoint{1.923178in}{1.350290in}}%
\pgfpathlineto{\pgfqpoint{1.927145in}{1.342356in}}%
\pgfusepath{fill}%
\end{pgfscope}%
\begin{pgfscope}%
\pgfpathrectangle{\pgfqpoint{1.432000in}{0.528000in}}{\pgfqpoint{3.696000in}{3.696000in}} %
\pgfusepath{clip}%
\pgfsetbuttcap%
\pgfsetroundjoin%
\definecolor{currentfill}{rgb}{0.282910,0.105393,0.426902}%
\pgfsetfillcolor{currentfill}%
\pgfsetlinewidth{0.000000pt}%
\definecolor{currentstroke}{rgb}{0.000000,0.000000,0.000000}%
\pgfsetstrokecolor{currentstroke}%
\pgfsetdash{}{0pt}%
\pgfpathmoveto{\pgfqpoint{1.928297in}{1.343186in}}%
\pgfpathlineto{\pgfqpoint{1.848136in}{1.263025in}}%
\pgfpathlineto{\pgfqpoint{1.857544in}{1.259889in}}%
\pgfpathlineto{\pgfqpoint{1.816774in}{1.237935in}}%
\pgfpathlineto{\pgfqpoint{1.838727in}{1.278706in}}%
\pgfpathlineto{\pgfqpoint{1.841863in}{1.269297in}}%
\pgfpathlineto{\pgfqpoint{1.922025in}{1.349459in}}%
\pgfpathlineto{\pgfqpoint{1.928297in}{1.343186in}}%
\pgfusepath{fill}%
\end{pgfscope}%
\begin{pgfscope}%
\pgfpathrectangle{\pgfqpoint{1.432000in}{0.528000in}}{\pgfqpoint{3.696000in}{3.696000in}} %
\pgfusepath{clip}%
\pgfsetbuttcap%
\pgfsetroundjoin%
\definecolor{currentfill}{rgb}{0.169646,0.456262,0.558030}%
\pgfsetfillcolor{currentfill}%
\pgfsetlinewidth{0.000000pt}%
\definecolor{currentstroke}{rgb}{0.000000,0.000000,0.000000}%
\pgfsetstrokecolor{currentstroke}%
\pgfsetdash{}{0pt}%
\pgfpathmoveto{\pgfqpoint{2.035532in}{1.342356in}}%
\pgfpathlineto{\pgfqpoint{1.854460in}{1.251820in}}%
\pgfpathlineto{\pgfqpoint{1.862394in}{1.245869in}}%
\pgfpathlineto{\pgfqpoint{1.816774in}{1.237935in}}%
\pgfpathlineto{\pgfqpoint{1.850493in}{1.269671in}}%
\pgfpathlineto{\pgfqpoint{1.850493in}{1.259754in}}%
\pgfpathlineto{\pgfqpoint{2.031565in}{1.350290in}}%
\pgfpathlineto{\pgfqpoint{2.035532in}{1.342356in}}%
\pgfusepath{fill}%
\end{pgfscope}%
\begin{pgfscope}%
\pgfpathrectangle{\pgfqpoint{1.432000in}{0.528000in}}{\pgfqpoint{3.696000in}{3.696000in}} %
\pgfusepath{clip}%
\pgfsetbuttcap%
\pgfsetroundjoin%
\definecolor{currentfill}{rgb}{0.132444,0.552216,0.553018}%
\pgfsetfillcolor{currentfill}%
\pgfsetlinewidth{0.000000pt}%
\definecolor{currentstroke}{rgb}{0.000000,0.000000,0.000000}%
\pgfsetstrokecolor{currentstroke}%
\pgfsetdash{}{0pt}%
\pgfpathmoveto{\pgfqpoint{2.143919in}{1.342356in}}%
\pgfpathlineto{\pgfqpoint{1.962847in}{1.251820in}}%
\pgfpathlineto{\pgfqpoint{1.970781in}{1.245869in}}%
\pgfpathlineto{\pgfqpoint{1.925161in}{1.237935in}}%
\pgfpathlineto{\pgfqpoint{1.958880in}{1.269671in}}%
\pgfpathlineto{\pgfqpoint{1.958880in}{1.259754in}}%
\pgfpathlineto{\pgfqpoint{2.139952in}{1.350290in}}%
\pgfpathlineto{\pgfqpoint{2.143919in}{1.342356in}}%
\pgfusepath{fill}%
\end{pgfscope}%
\begin{pgfscope}%
\pgfpathrectangle{\pgfqpoint{1.432000in}{0.528000in}}{\pgfqpoint{3.696000in}{3.696000in}} %
\pgfusepath{clip}%
\pgfsetbuttcap%
\pgfsetroundjoin%
\definecolor{currentfill}{rgb}{0.163625,0.471133,0.558148}%
\pgfsetfillcolor{currentfill}%
\pgfsetlinewidth{0.000000pt}%
\definecolor{currentstroke}{rgb}{0.000000,0.000000,0.000000}%
\pgfsetstrokecolor{currentstroke}%
\pgfsetdash{}{0pt}%
\pgfpathmoveto{\pgfqpoint{2.252306in}{1.342356in}}%
\pgfpathlineto{\pgfqpoint{2.071235in}{1.251820in}}%
\pgfpathlineto{\pgfqpoint{2.079168in}{1.245869in}}%
\pgfpathlineto{\pgfqpoint{2.033548in}{1.237935in}}%
\pgfpathlineto{\pgfqpoint{2.067268in}{1.269671in}}%
\pgfpathlineto{\pgfqpoint{2.067268in}{1.259754in}}%
\pgfpathlineto{\pgfqpoint{2.248339in}{1.350290in}}%
\pgfpathlineto{\pgfqpoint{2.252306in}{1.342356in}}%
\pgfusepath{fill}%
\end{pgfscope}%
\begin{pgfscope}%
\pgfpathrectangle{\pgfqpoint{1.432000in}{0.528000in}}{\pgfqpoint{3.696000in}{3.696000in}} %
\pgfusepath{clip}%
\pgfsetbuttcap%
\pgfsetroundjoin%
\definecolor{currentfill}{rgb}{0.282327,0.094955,0.417331}%
\pgfsetfillcolor{currentfill}%
\pgfsetlinewidth{0.000000pt}%
\definecolor{currentstroke}{rgb}{0.000000,0.000000,0.000000}%
\pgfsetstrokecolor{currentstroke}%
\pgfsetdash{}{0pt}%
\pgfpathmoveto{\pgfqpoint{2.360693in}{1.342356in}}%
\pgfpathlineto{\pgfqpoint{2.179622in}{1.251820in}}%
\pgfpathlineto{\pgfqpoint{2.187556in}{1.245869in}}%
\pgfpathlineto{\pgfqpoint{2.141935in}{1.237935in}}%
\pgfpathlineto{\pgfqpoint{2.175655in}{1.269671in}}%
\pgfpathlineto{\pgfqpoint{2.175655in}{1.259754in}}%
\pgfpathlineto{\pgfqpoint{2.356726in}{1.350290in}}%
\pgfpathlineto{\pgfqpoint{2.360693in}{1.342356in}}%
\pgfusepath{fill}%
\end{pgfscope}%
\begin{pgfscope}%
\pgfpathrectangle{\pgfqpoint{1.432000in}{0.528000in}}{\pgfqpoint{3.696000in}{3.696000in}} %
\pgfusepath{clip}%
\pgfsetbuttcap%
\pgfsetroundjoin%
\definecolor{currentfill}{rgb}{0.269944,0.014625,0.341379}%
\pgfsetfillcolor{currentfill}%
\pgfsetlinewidth{0.000000pt}%
\definecolor{currentstroke}{rgb}{0.000000,0.000000,0.000000}%
\pgfsetstrokecolor{currentstroke}%
\pgfsetdash{}{0pt}%
\pgfpathmoveto{\pgfqpoint{2.358710in}{1.341887in}}%
\pgfpathlineto{\pgfqpoint{2.181852in}{1.341887in}}%
\pgfpathlineto{\pgfqpoint{2.186287in}{1.333017in}}%
\pgfpathlineto{\pgfqpoint{2.141935in}{1.346323in}}%
\pgfpathlineto{\pgfqpoint{2.186287in}{1.359628in}}%
\pgfpathlineto{\pgfqpoint{2.181852in}{1.350758in}}%
\pgfpathlineto{\pgfqpoint{2.358710in}{1.350758in}}%
\pgfpathlineto{\pgfqpoint{2.358710in}{1.341887in}}%
\pgfusepath{fill}%
\end{pgfscope}%
\begin{pgfscope}%
\pgfpathrectangle{\pgfqpoint{1.432000in}{0.528000in}}{\pgfqpoint{3.696000in}{3.696000in}} %
\pgfusepath{clip}%
\pgfsetbuttcap%
\pgfsetroundjoin%
\definecolor{currentfill}{rgb}{0.279566,0.067836,0.391917}%
\pgfsetfillcolor{currentfill}%
\pgfsetlinewidth{0.000000pt}%
\definecolor{currentstroke}{rgb}{0.000000,0.000000,0.000000}%
\pgfsetstrokecolor{currentstroke}%
\pgfsetdash{}{0pt}%
\pgfpathmoveto{\pgfqpoint{2.467097in}{1.341887in}}%
\pgfpathlineto{\pgfqpoint{2.181852in}{1.341887in}}%
\pgfpathlineto{\pgfqpoint{2.186287in}{1.333017in}}%
\pgfpathlineto{\pgfqpoint{2.141935in}{1.346323in}}%
\pgfpathlineto{\pgfqpoint{2.186287in}{1.359628in}}%
\pgfpathlineto{\pgfqpoint{2.181852in}{1.350758in}}%
\pgfpathlineto{\pgfqpoint{2.467097in}{1.350758in}}%
\pgfpathlineto{\pgfqpoint{2.467097in}{1.341887in}}%
\pgfusepath{fill}%
\end{pgfscope}%
\begin{pgfscope}%
\pgfpathrectangle{\pgfqpoint{1.432000in}{0.528000in}}{\pgfqpoint{3.696000in}{3.696000in}} %
\pgfusepath{clip}%
\pgfsetbuttcap%
\pgfsetroundjoin%
\definecolor{currentfill}{rgb}{0.279566,0.067836,0.391917}%
\pgfsetfillcolor{currentfill}%
\pgfsetlinewidth{0.000000pt}%
\definecolor{currentstroke}{rgb}{0.000000,0.000000,0.000000}%
\pgfsetstrokecolor{currentstroke}%
\pgfsetdash{}{0pt}%
\pgfpathmoveto{\pgfqpoint{2.467097in}{1.341887in}}%
\pgfpathlineto{\pgfqpoint{2.290239in}{1.341887in}}%
\pgfpathlineto{\pgfqpoint{2.294675in}{1.333017in}}%
\pgfpathlineto{\pgfqpoint{2.250323in}{1.346323in}}%
\pgfpathlineto{\pgfqpoint{2.294675in}{1.359628in}}%
\pgfpathlineto{\pgfqpoint{2.290239in}{1.350758in}}%
\pgfpathlineto{\pgfqpoint{2.467097in}{1.350758in}}%
\pgfpathlineto{\pgfqpoint{2.467097in}{1.341887in}}%
\pgfusepath{fill}%
\end{pgfscope}%
\begin{pgfscope}%
\pgfpathrectangle{\pgfqpoint{1.432000in}{0.528000in}}{\pgfqpoint{3.696000in}{3.696000in}} %
\pgfusepath{clip}%
\pgfsetbuttcap%
\pgfsetroundjoin%
\definecolor{currentfill}{rgb}{0.241237,0.296485,0.539709}%
\pgfsetfillcolor{currentfill}%
\pgfsetlinewidth{0.000000pt}%
\definecolor{currentstroke}{rgb}{0.000000,0.000000,0.000000}%
\pgfsetstrokecolor{currentstroke}%
\pgfsetdash{}{0pt}%
\pgfpathmoveto{\pgfqpoint{2.575484in}{1.341887in}}%
\pgfpathlineto{\pgfqpoint{2.290239in}{1.341887in}}%
\pgfpathlineto{\pgfqpoint{2.294675in}{1.333017in}}%
\pgfpathlineto{\pgfqpoint{2.250323in}{1.346323in}}%
\pgfpathlineto{\pgfqpoint{2.294675in}{1.359628in}}%
\pgfpathlineto{\pgfqpoint{2.290239in}{1.350758in}}%
\pgfpathlineto{\pgfqpoint{2.575484in}{1.350758in}}%
\pgfpathlineto{\pgfqpoint{2.575484in}{1.341887in}}%
\pgfusepath{fill}%
\end{pgfscope}%
\begin{pgfscope}%
\pgfpathrectangle{\pgfqpoint{1.432000in}{0.528000in}}{\pgfqpoint{3.696000in}{3.696000in}} %
\pgfusepath{clip}%
\pgfsetbuttcap%
\pgfsetroundjoin%
\definecolor{currentfill}{rgb}{0.187231,0.414746,0.556547}%
\pgfsetfillcolor{currentfill}%
\pgfsetlinewidth{0.000000pt}%
\definecolor{currentstroke}{rgb}{0.000000,0.000000,0.000000}%
\pgfsetstrokecolor{currentstroke}%
\pgfsetdash{}{0pt}%
\pgfpathmoveto{\pgfqpoint{2.683871in}{1.341887in}}%
\pgfpathlineto{\pgfqpoint{2.398626in}{1.341887in}}%
\pgfpathlineto{\pgfqpoint{2.403062in}{1.333017in}}%
\pgfpathlineto{\pgfqpoint{2.358710in}{1.346323in}}%
\pgfpathlineto{\pgfqpoint{2.403062in}{1.359628in}}%
\pgfpathlineto{\pgfqpoint{2.398626in}{1.350758in}}%
\pgfpathlineto{\pgfqpoint{2.683871in}{1.350758in}}%
\pgfpathlineto{\pgfqpoint{2.683871in}{1.341887in}}%
\pgfusepath{fill}%
\end{pgfscope}%
\begin{pgfscope}%
\pgfpathrectangle{\pgfqpoint{1.432000in}{0.528000in}}{\pgfqpoint{3.696000in}{3.696000in}} %
\pgfusepath{clip}%
\pgfsetbuttcap%
\pgfsetroundjoin%
\definecolor{currentfill}{rgb}{0.123463,0.581687,0.547445}%
\pgfsetfillcolor{currentfill}%
\pgfsetlinewidth{0.000000pt}%
\definecolor{currentstroke}{rgb}{0.000000,0.000000,0.000000}%
\pgfsetstrokecolor{currentstroke}%
\pgfsetdash{}{0pt}%
\pgfpathmoveto{\pgfqpoint{2.792258in}{1.341887in}}%
\pgfpathlineto{\pgfqpoint{2.507014in}{1.341887in}}%
\pgfpathlineto{\pgfqpoint{2.511449in}{1.333017in}}%
\pgfpathlineto{\pgfqpoint{2.467097in}{1.346323in}}%
\pgfpathlineto{\pgfqpoint{2.511449in}{1.359628in}}%
\pgfpathlineto{\pgfqpoint{2.507014in}{1.350758in}}%
\pgfpathlineto{\pgfqpoint{2.792258in}{1.350758in}}%
\pgfpathlineto{\pgfqpoint{2.792258in}{1.341887in}}%
\pgfusepath{fill}%
\end{pgfscope}%
\begin{pgfscope}%
\pgfpathrectangle{\pgfqpoint{1.432000in}{0.528000in}}{\pgfqpoint{3.696000in}{3.696000in}} %
\pgfusepath{clip}%
\pgfsetbuttcap%
\pgfsetroundjoin%
\definecolor{currentfill}{rgb}{0.263663,0.237631,0.518762}%
\pgfsetfillcolor{currentfill}%
\pgfsetlinewidth{0.000000pt}%
\definecolor{currentstroke}{rgb}{0.000000,0.000000,0.000000}%
\pgfsetstrokecolor{currentstroke}%
\pgfsetdash{}{0pt}%
\pgfpathmoveto{\pgfqpoint{2.900645in}{1.341887in}}%
\pgfpathlineto{\pgfqpoint{2.615401in}{1.341887in}}%
\pgfpathlineto{\pgfqpoint{2.619836in}{1.333017in}}%
\pgfpathlineto{\pgfqpoint{2.575484in}{1.346323in}}%
\pgfpathlineto{\pgfqpoint{2.619836in}{1.359628in}}%
\pgfpathlineto{\pgfqpoint{2.615401in}{1.350758in}}%
\pgfpathlineto{\pgfqpoint{2.900645in}{1.350758in}}%
\pgfpathlineto{\pgfqpoint{2.900645in}{1.341887in}}%
\pgfusepath{fill}%
\end{pgfscope}%
\begin{pgfscope}%
\pgfpathrectangle{\pgfqpoint{1.432000in}{0.528000in}}{\pgfqpoint{3.696000in}{3.696000in}} %
\pgfusepath{clip}%
\pgfsetbuttcap%
\pgfsetroundjoin%
\definecolor{currentfill}{rgb}{0.210503,0.363727,0.552206}%
\pgfsetfillcolor{currentfill}%
\pgfsetlinewidth{0.000000pt}%
\definecolor{currentstroke}{rgb}{0.000000,0.000000,0.000000}%
\pgfsetstrokecolor{currentstroke}%
\pgfsetdash{}{0pt}%
\pgfpathmoveto{\pgfqpoint{2.899243in}{1.342115in}}%
\pgfpathlineto{\pgfqpoint{2.611950in}{1.437879in}}%
\pgfpathlineto{\pgfqpoint{2.613352in}{1.428062in}}%
\pgfpathlineto{\pgfqpoint{2.575484in}{1.454710in}}%
\pgfpathlineto{\pgfqpoint{2.621767in}{1.453307in}}%
\pgfpathlineto{\pgfqpoint{2.614755in}{1.446294in}}%
\pgfpathlineto{\pgfqpoint{2.902048in}{1.350530in}}%
\pgfpathlineto{\pgfqpoint{2.899243in}{1.342115in}}%
\pgfusepath{fill}%
\end{pgfscope}%
\begin{pgfscope}%
\pgfpathrectangle{\pgfqpoint{1.432000in}{0.528000in}}{\pgfqpoint{3.696000in}{3.696000in}} %
\pgfusepath{clip}%
\pgfsetbuttcap%
\pgfsetroundjoin%
\definecolor{currentfill}{rgb}{0.279574,0.170599,0.479997}%
\pgfsetfillcolor{currentfill}%
\pgfsetlinewidth{0.000000pt}%
\definecolor{currentstroke}{rgb}{0.000000,0.000000,0.000000}%
\pgfsetstrokecolor{currentstroke}%
\pgfsetdash{}{0pt}%
\pgfpathmoveto{\pgfqpoint{3.009032in}{1.341887in}}%
\pgfpathlineto{\pgfqpoint{2.723788in}{1.341887in}}%
\pgfpathlineto{\pgfqpoint{2.728223in}{1.333017in}}%
\pgfpathlineto{\pgfqpoint{2.683871in}{1.346323in}}%
\pgfpathlineto{\pgfqpoint{2.728223in}{1.359628in}}%
\pgfpathlineto{\pgfqpoint{2.723788in}{1.350758in}}%
\pgfpathlineto{\pgfqpoint{3.009032in}{1.350758in}}%
\pgfpathlineto{\pgfqpoint{3.009032in}{1.341887in}}%
\pgfusepath{fill}%
\end{pgfscope}%
\begin{pgfscope}%
\pgfpathrectangle{\pgfqpoint{1.432000in}{0.528000in}}{\pgfqpoint{3.696000in}{3.696000in}} %
\pgfusepath{clip}%
\pgfsetbuttcap%
\pgfsetroundjoin%
\definecolor{currentfill}{rgb}{0.283187,0.125848,0.444960}%
\pgfsetfillcolor{currentfill}%
\pgfsetlinewidth{0.000000pt}%
\definecolor{currentstroke}{rgb}{0.000000,0.000000,0.000000}%
\pgfsetstrokecolor{currentstroke}%
\pgfsetdash{}{0pt}%
\pgfpathmoveto{\pgfqpoint{3.009032in}{1.341887in}}%
\pgfpathlineto{\pgfqpoint{2.832175in}{1.341887in}}%
\pgfpathlineto{\pgfqpoint{2.836610in}{1.333017in}}%
\pgfpathlineto{\pgfqpoint{2.792258in}{1.346323in}}%
\pgfpathlineto{\pgfqpoint{2.836610in}{1.359628in}}%
\pgfpathlineto{\pgfqpoint{2.832175in}{1.350758in}}%
\pgfpathlineto{\pgfqpoint{3.009032in}{1.350758in}}%
\pgfpathlineto{\pgfqpoint{3.009032in}{1.341887in}}%
\pgfusepath{fill}%
\end{pgfscope}%
\begin{pgfscope}%
\pgfpathrectangle{\pgfqpoint{1.432000in}{0.528000in}}{\pgfqpoint{3.696000in}{3.696000in}} %
\pgfusepath{clip}%
\pgfsetbuttcap%
\pgfsetroundjoin%
\definecolor{currentfill}{rgb}{0.281412,0.155834,0.469201}%
\pgfsetfillcolor{currentfill}%
\pgfsetlinewidth{0.000000pt}%
\definecolor{currentstroke}{rgb}{0.000000,0.000000,0.000000}%
\pgfsetstrokecolor{currentstroke}%
\pgfsetdash{}{0pt}%
\pgfpathmoveto{\pgfqpoint{3.007630in}{1.342115in}}%
\pgfpathlineto{\pgfqpoint{2.720337in}{1.437879in}}%
\pgfpathlineto{\pgfqpoint{2.721739in}{1.428062in}}%
\pgfpathlineto{\pgfqpoint{2.683871in}{1.454710in}}%
\pgfpathlineto{\pgfqpoint{2.730155in}{1.453307in}}%
\pgfpathlineto{\pgfqpoint{2.723142in}{1.446294in}}%
\pgfpathlineto{\pgfqpoint{3.010435in}{1.350530in}}%
\pgfpathlineto{\pgfqpoint{3.007630in}{1.342115in}}%
\pgfusepath{fill}%
\end{pgfscope}%
\begin{pgfscope}%
\pgfpathrectangle{\pgfqpoint{1.432000in}{0.528000in}}{\pgfqpoint{3.696000in}{3.696000in}} %
\pgfusepath{clip}%
\pgfsetbuttcap%
\pgfsetroundjoin%
\definecolor{currentfill}{rgb}{0.276022,0.044167,0.370164}%
\pgfsetfillcolor{currentfill}%
\pgfsetlinewidth{0.000000pt}%
\definecolor{currentstroke}{rgb}{0.000000,0.000000,0.000000}%
\pgfsetstrokecolor{currentstroke}%
\pgfsetdash{}{0pt}%
\pgfpathmoveto{\pgfqpoint{3.007049in}{1.342356in}}%
\pgfpathlineto{\pgfqpoint{2.825977in}{1.432891in}}%
\pgfpathlineto{\pgfqpoint{2.825977in}{1.422974in}}%
\pgfpathlineto{\pgfqpoint{2.792258in}{1.454710in}}%
\pgfpathlineto{\pgfqpoint{2.837878in}{1.446776in}}%
\pgfpathlineto{\pgfqpoint{2.829944in}{1.440825in}}%
\pgfpathlineto{\pgfqpoint{3.011016in}{1.350290in}}%
\pgfpathlineto{\pgfqpoint{3.007049in}{1.342356in}}%
\pgfusepath{fill}%
\end{pgfscope}%
\begin{pgfscope}%
\pgfpathrectangle{\pgfqpoint{1.432000in}{0.528000in}}{\pgfqpoint{3.696000in}{3.696000in}} %
\pgfusepath{clip}%
\pgfsetbuttcap%
\pgfsetroundjoin%
\definecolor{currentfill}{rgb}{0.279566,0.067836,0.391917}%
\pgfsetfillcolor{currentfill}%
\pgfsetlinewidth{0.000000pt}%
\definecolor{currentstroke}{rgb}{0.000000,0.000000,0.000000}%
\pgfsetstrokecolor{currentstroke}%
\pgfsetdash{}{0pt}%
\pgfpathmoveto{\pgfqpoint{3.117419in}{1.341887in}}%
\pgfpathlineto{\pgfqpoint{2.832175in}{1.341887in}}%
\pgfpathlineto{\pgfqpoint{2.836610in}{1.333017in}}%
\pgfpathlineto{\pgfqpoint{2.792258in}{1.346323in}}%
\pgfpathlineto{\pgfqpoint{2.836610in}{1.359628in}}%
\pgfpathlineto{\pgfqpoint{2.832175in}{1.350758in}}%
\pgfpathlineto{\pgfqpoint{3.117419in}{1.350758in}}%
\pgfpathlineto{\pgfqpoint{3.117419in}{1.341887in}}%
\pgfusepath{fill}%
\end{pgfscope}%
\begin{pgfscope}%
\pgfpathrectangle{\pgfqpoint{1.432000in}{0.528000in}}{\pgfqpoint{3.696000in}{3.696000in}} %
\pgfusepath{clip}%
\pgfsetbuttcap%
\pgfsetroundjoin%
\definecolor{currentfill}{rgb}{0.257322,0.256130,0.526563}%
\pgfsetfillcolor{currentfill}%
\pgfsetlinewidth{0.000000pt}%
\definecolor{currentstroke}{rgb}{0.000000,0.000000,0.000000}%
\pgfsetstrokecolor{currentstroke}%
\pgfsetdash{}{0pt}%
\pgfpathmoveto{\pgfqpoint{3.117419in}{1.341887in}}%
\pgfpathlineto{\pgfqpoint{2.940562in}{1.341887in}}%
\pgfpathlineto{\pgfqpoint{2.944997in}{1.333017in}}%
\pgfpathlineto{\pgfqpoint{2.900645in}{1.346323in}}%
\pgfpathlineto{\pgfqpoint{2.944997in}{1.359628in}}%
\pgfpathlineto{\pgfqpoint{2.940562in}{1.350758in}}%
\pgfpathlineto{\pgfqpoint{3.117419in}{1.350758in}}%
\pgfpathlineto{\pgfqpoint{3.117419in}{1.341887in}}%
\pgfusepath{fill}%
\end{pgfscope}%
\begin{pgfscope}%
\pgfpathrectangle{\pgfqpoint{1.432000in}{0.528000in}}{\pgfqpoint{3.696000in}{3.696000in}} %
\pgfusepath{clip}%
\pgfsetbuttcap%
\pgfsetroundjoin%
\definecolor{currentfill}{rgb}{0.260571,0.246922,0.522828}%
\pgfsetfillcolor{currentfill}%
\pgfsetlinewidth{0.000000pt}%
\definecolor{currentstroke}{rgb}{0.000000,0.000000,0.000000}%
\pgfsetstrokecolor{currentstroke}%
\pgfsetdash{}{0pt}%
\pgfpathmoveto{\pgfqpoint{3.115436in}{1.342356in}}%
\pgfpathlineto{\pgfqpoint{2.934364in}{1.432891in}}%
\pgfpathlineto{\pgfqpoint{2.934364in}{1.422974in}}%
\pgfpathlineto{\pgfqpoint{2.900645in}{1.454710in}}%
\pgfpathlineto{\pgfqpoint{2.946265in}{1.446776in}}%
\pgfpathlineto{\pgfqpoint{2.938331in}{1.440825in}}%
\pgfpathlineto{\pgfqpoint{3.119403in}{1.350290in}}%
\pgfpathlineto{\pgfqpoint{3.115436in}{1.342356in}}%
\pgfusepath{fill}%
\end{pgfscope}%
\begin{pgfscope}%
\pgfpathrectangle{\pgfqpoint{1.432000in}{0.528000in}}{\pgfqpoint{3.696000in}{3.696000in}} %
\pgfusepath{clip}%
\pgfsetbuttcap%
\pgfsetroundjoin%
\definecolor{currentfill}{rgb}{0.159194,0.482237,0.558073}%
\pgfsetfillcolor{currentfill}%
\pgfsetlinewidth{0.000000pt}%
\definecolor{currentstroke}{rgb}{0.000000,0.000000,0.000000}%
\pgfsetstrokecolor{currentstroke}%
\pgfsetdash{}{0pt}%
\pgfpathmoveto{\pgfqpoint{3.225806in}{1.341887in}}%
\pgfpathlineto{\pgfqpoint{3.048949in}{1.341887in}}%
\pgfpathlineto{\pgfqpoint{3.053384in}{1.333017in}}%
\pgfpathlineto{\pgfqpoint{3.009032in}{1.346323in}}%
\pgfpathlineto{\pgfqpoint{3.053384in}{1.359628in}}%
\pgfpathlineto{\pgfqpoint{3.048949in}{1.350758in}}%
\pgfpathlineto{\pgfqpoint{3.225806in}{1.350758in}}%
\pgfpathlineto{\pgfqpoint{3.225806in}{1.341887in}}%
\pgfusepath{fill}%
\end{pgfscope}%
\begin{pgfscope}%
\pgfpathrectangle{\pgfqpoint{1.432000in}{0.528000in}}{\pgfqpoint{3.696000in}{3.696000in}} %
\pgfusepath{clip}%
\pgfsetbuttcap%
\pgfsetroundjoin%
\definecolor{currentfill}{rgb}{0.280267,0.073417,0.397163}%
\pgfsetfillcolor{currentfill}%
\pgfsetlinewidth{0.000000pt}%
\definecolor{currentstroke}{rgb}{0.000000,0.000000,0.000000}%
\pgfsetstrokecolor{currentstroke}%
\pgfsetdash{}{0pt}%
\pgfpathmoveto{\pgfqpoint{3.223823in}{1.342356in}}%
\pgfpathlineto{\pgfqpoint{3.042751in}{1.432891in}}%
\pgfpathlineto{\pgfqpoint{3.042751in}{1.422974in}}%
\pgfpathlineto{\pgfqpoint{3.009032in}{1.454710in}}%
\pgfpathlineto{\pgfqpoint{3.054652in}{1.446776in}}%
\pgfpathlineto{\pgfqpoint{3.046718in}{1.440825in}}%
\pgfpathlineto{\pgfqpoint{3.227790in}{1.350290in}}%
\pgfpathlineto{\pgfqpoint{3.223823in}{1.342356in}}%
\pgfusepath{fill}%
\end{pgfscope}%
\begin{pgfscope}%
\pgfpathrectangle{\pgfqpoint{1.432000in}{0.528000in}}{\pgfqpoint{3.696000in}{3.696000in}} %
\pgfusepath{clip}%
\pgfsetbuttcap%
\pgfsetroundjoin%
\definecolor{currentfill}{rgb}{0.123444,0.636809,0.528763}%
\pgfsetfillcolor{currentfill}%
\pgfsetlinewidth{0.000000pt}%
\definecolor{currentstroke}{rgb}{0.000000,0.000000,0.000000}%
\pgfsetstrokecolor{currentstroke}%
\pgfsetdash{}{0pt}%
\pgfpathmoveto{\pgfqpoint{3.334194in}{1.341887in}}%
\pgfpathlineto{\pgfqpoint{3.157336in}{1.341887in}}%
\pgfpathlineto{\pgfqpoint{3.161771in}{1.333017in}}%
\pgfpathlineto{\pgfqpoint{3.117419in}{1.346323in}}%
\pgfpathlineto{\pgfqpoint{3.161771in}{1.359628in}}%
\pgfpathlineto{\pgfqpoint{3.157336in}{1.350758in}}%
\pgfpathlineto{\pgfqpoint{3.334194in}{1.350758in}}%
\pgfpathlineto{\pgfqpoint{3.334194in}{1.341887in}}%
\pgfusepath{fill}%
\end{pgfscope}%
\begin{pgfscope}%
\pgfpathrectangle{\pgfqpoint{1.432000in}{0.528000in}}{\pgfqpoint{3.696000in}{3.696000in}} %
\pgfusepath{clip}%
\pgfsetbuttcap%
\pgfsetroundjoin%
\definecolor{currentfill}{rgb}{0.175841,0.441290,0.557685}%
\pgfsetfillcolor{currentfill}%
\pgfsetlinewidth{0.000000pt}%
\definecolor{currentstroke}{rgb}{0.000000,0.000000,0.000000}%
\pgfsetstrokecolor{currentstroke}%
\pgfsetdash{}{0pt}%
\pgfpathmoveto{\pgfqpoint{3.442581in}{1.341887in}}%
\pgfpathlineto{\pgfqpoint{3.265723in}{1.341887in}}%
\pgfpathlineto{\pgfqpoint{3.270158in}{1.333017in}}%
\pgfpathlineto{\pgfqpoint{3.225806in}{1.346323in}}%
\pgfpathlineto{\pgfqpoint{3.270158in}{1.359628in}}%
\pgfpathlineto{\pgfqpoint{3.265723in}{1.350758in}}%
\pgfpathlineto{\pgfqpoint{3.442581in}{1.350758in}}%
\pgfpathlineto{\pgfqpoint{3.442581in}{1.341887in}}%
\pgfusepath{fill}%
\end{pgfscope}%
\begin{pgfscope}%
\pgfpathrectangle{\pgfqpoint{1.432000in}{0.528000in}}{\pgfqpoint{3.696000in}{3.696000in}} %
\pgfusepath{clip}%
\pgfsetbuttcap%
\pgfsetroundjoin%
\definecolor{currentfill}{rgb}{0.282290,0.145912,0.461510}%
\pgfsetfillcolor{currentfill}%
\pgfsetlinewidth{0.000000pt}%
\definecolor{currentstroke}{rgb}{0.000000,0.000000,0.000000}%
\pgfsetstrokecolor{currentstroke}%
\pgfsetdash{}{0pt}%
\pgfpathmoveto{\pgfqpoint{3.442581in}{1.341887in}}%
\pgfpathlineto{\pgfqpoint{3.374110in}{1.341887in}}%
\pgfpathlineto{\pgfqpoint{3.378546in}{1.333017in}}%
\pgfpathlineto{\pgfqpoint{3.334194in}{1.346323in}}%
\pgfpathlineto{\pgfqpoint{3.378546in}{1.359628in}}%
\pgfpathlineto{\pgfqpoint{3.374110in}{1.350758in}}%
\pgfpathlineto{\pgfqpoint{3.442581in}{1.350758in}}%
\pgfpathlineto{\pgfqpoint{3.442581in}{1.341887in}}%
\pgfusepath{fill}%
\end{pgfscope}%
\begin{pgfscope}%
\pgfpathrectangle{\pgfqpoint{1.432000in}{0.528000in}}{\pgfqpoint{3.696000in}{3.696000in}} %
\pgfusepath{clip}%
\pgfsetbuttcap%
\pgfsetroundjoin%
\definecolor{currentfill}{rgb}{0.229739,0.322361,0.545706}%
\pgfsetfillcolor{currentfill}%
\pgfsetlinewidth{0.000000pt}%
\definecolor{currentstroke}{rgb}{0.000000,0.000000,0.000000}%
\pgfsetstrokecolor{currentstroke}%
\pgfsetdash{}{0pt}%
\pgfpathmoveto{\pgfqpoint{3.550968in}{1.341887in}}%
\pgfpathlineto{\pgfqpoint{3.374110in}{1.341887in}}%
\pgfpathlineto{\pgfqpoint{3.378546in}{1.333017in}}%
\pgfpathlineto{\pgfqpoint{3.334194in}{1.346323in}}%
\pgfpathlineto{\pgfqpoint{3.378546in}{1.359628in}}%
\pgfpathlineto{\pgfqpoint{3.374110in}{1.350758in}}%
\pgfpathlineto{\pgfqpoint{3.550968in}{1.350758in}}%
\pgfpathlineto{\pgfqpoint{3.550968in}{1.341887in}}%
\pgfusepath{fill}%
\end{pgfscope}%
\begin{pgfscope}%
\pgfpathrectangle{\pgfqpoint{1.432000in}{0.528000in}}{\pgfqpoint{3.696000in}{3.696000in}} %
\pgfusepath{clip}%
\pgfsetbuttcap%
\pgfsetroundjoin%
\definecolor{currentfill}{rgb}{0.231674,0.318106,0.544834}%
\pgfsetfillcolor{currentfill}%
\pgfsetlinewidth{0.000000pt}%
\definecolor{currentstroke}{rgb}{0.000000,0.000000,0.000000}%
\pgfsetstrokecolor{currentstroke}%
\pgfsetdash{}{0pt}%
\pgfpathmoveto{\pgfqpoint{3.659355in}{1.341887in}}%
\pgfpathlineto{\pgfqpoint{3.482497in}{1.341887in}}%
\pgfpathlineto{\pgfqpoint{3.486933in}{1.333017in}}%
\pgfpathlineto{\pgfqpoint{3.442581in}{1.346323in}}%
\pgfpathlineto{\pgfqpoint{3.486933in}{1.359628in}}%
\pgfpathlineto{\pgfqpoint{3.482497in}{1.350758in}}%
\pgfpathlineto{\pgfqpoint{3.659355in}{1.350758in}}%
\pgfpathlineto{\pgfqpoint{3.659355in}{1.341887in}}%
\pgfusepath{fill}%
\end{pgfscope}%
\begin{pgfscope}%
\pgfpathrectangle{\pgfqpoint{1.432000in}{0.528000in}}{\pgfqpoint{3.696000in}{3.696000in}} %
\pgfusepath{clip}%
\pgfsetbuttcap%
\pgfsetroundjoin%
\definecolor{currentfill}{rgb}{0.260571,0.246922,0.522828}%
\pgfsetfillcolor{currentfill}%
\pgfsetlinewidth{0.000000pt}%
\definecolor{currentstroke}{rgb}{0.000000,0.000000,0.000000}%
\pgfsetstrokecolor{currentstroke}%
\pgfsetdash{}{0pt}%
\pgfpathmoveto{\pgfqpoint{3.767742in}{1.341887in}}%
\pgfpathlineto{\pgfqpoint{3.590885in}{1.341887in}}%
\pgfpathlineto{\pgfqpoint{3.595320in}{1.333017in}}%
\pgfpathlineto{\pgfqpoint{3.550968in}{1.346323in}}%
\pgfpathlineto{\pgfqpoint{3.595320in}{1.359628in}}%
\pgfpathlineto{\pgfqpoint{3.590885in}{1.350758in}}%
\pgfpathlineto{\pgfqpoint{3.767742in}{1.350758in}}%
\pgfpathlineto{\pgfqpoint{3.767742in}{1.341887in}}%
\pgfusepath{fill}%
\end{pgfscope}%
\begin{pgfscope}%
\pgfpathrectangle{\pgfqpoint{1.432000in}{0.528000in}}{\pgfqpoint{3.696000in}{3.696000in}} %
\pgfusepath{clip}%
\pgfsetbuttcap%
\pgfsetroundjoin%
\definecolor{currentfill}{rgb}{0.268510,0.009605,0.335427}%
\pgfsetfillcolor{currentfill}%
\pgfsetlinewidth{0.000000pt}%
\definecolor{currentstroke}{rgb}{0.000000,0.000000,0.000000}%
\pgfsetstrokecolor{currentstroke}%
\pgfsetdash{}{0pt}%
\pgfpathmoveto{\pgfqpoint{3.985919in}{1.342115in}}%
\pgfpathlineto{\pgfqpoint{3.698626in}{1.246351in}}%
\pgfpathlineto{\pgfqpoint{3.705638in}{1.239338in}}%
\pgfpathlineto{\pgfqpoint{3.659355in}{1.237935in}}%
\pgfpathlineto{\pgfqpoint{3.697223in}{1.264584in}}%
\pgfpathlineto{\pgfqpoint{3.695821in}{1.254766in}}%
\pgfpathlineto{\pgfqpoint{3.983114in}{1.350530in}}%
\pgfpathlineto{\pgfqpoint{3.985919in}{1.342115in}}%
\pgfusepath{fill}%
\end{pgfscope}%
\begin{pgfscope}%
\pgfpathrectangle{\pgfqpoint{1.432000in}{0.528000in}}{\pgfqpoint{3.696000in}{3.696000in}} %
\pgfusepath{clip}%
\pgfsetbuttcap%
\pgfsetroundjoin%
\definecolor{currentfill}{rgb}{0.268510,0.009605,0.335427}%
\pgfsetfillcolor{currentfill}%
\pgfsetlinewidth{0.000000pt}%
\definecolor{currentstroke}{rgb}{0.000000,0.000000,0.000000}%
\pgfsetstrokecolor{currentstroke}%
\pgfsetdash{}{0pt}%
\pgfpathmoveto{\pgfqpoint{4.093979in}{1.342020in}}%
\pgfpathlineto{\pgfqpoint{3.699156in}{1.243314in}}%
\pgfpathlineto{\pgfqpoint{3.705610in}{1.235784in}}%
\pgfpathlineto{\pgfqpoint{3.659355in}{1.237935in}}%
\pgfpathlineto{\pgfqpoint{3.699156in}{1.261601in}}%
\pgfpathlineto{\pgfqpoint{3.697004in}{1.251920in}}%
\pgfpathlineto{\pgfqpoint{4.091828in}{1.350625in}}%
\pgfpathlineto{\pgfqpoint{4.093979in}{1.342020in}}%
\pgfusepath{fill}%
\end{pgfscope}%
\begin{pgfscope}%
\pgfpathrectangle{\pgfqpoint{1.432000in}{0.528000in}}{\pgfqpoint{3.696000in}{3.696000in}} %
\pgfusepath{clip}%
\pgfsetbuttcap%
\pgfsetroundjoin%
\definecolor{currentfill}{rgb}{0.283229,0.120777,0.440584}%
\pgfsetfillcolor{currentfill}%
\pgfsetlinewidth{0.000000pt}%
\definecolor{currentstroke}{rgb}{0.000000,0.000000,0.000000}%
\pgfsetstrokecolor{currentstroke}%
\pgfsetdash{}{0pt}%
\pgfpathmoveto{\pgfqpoint{4.202366in}{1.342020in}}%
\pgfpathlineto{\pgfqpoint{3.807543in}{1.243314in}}%
\pgfpathlineto{\pgfqpoint{3.813997in}{1.235784in}}%
\pgfpathlineto{\pgfqpoint{3.767742in}{1.237935in}}%
\pgfpathlineto{\pgfqpoint{3.807543in}{1.261601in}}%
\pgfpathlineto{\pgfqpoint{3.805391in}{1.251920in}}%
\pgfpathlineto{\pgfqpoint{4.200215in}{1.350625in}}%
\pgfpathlineto{\pgfqpoint{4.202366in}{1.342020in}}%
\pgfusepath{fill}%
\end{pgfscope}%
\begin{pgfscope}%
\pgfpathrectangle{\pgfqpoint{1.432000in}{0.528000in}}{\pgfqpoint{3.696000in}{3.696000in}} %
\pgfusepath{clip}%
\pgfsetbuttcap%
\pgfsetroundjoin%
\definecolor{currentfill}{rgb}{0.260571,0.246922,0.522828}%
\pgfsetfillcolor{currentfill}%
\pgfsetlinewidth{0.000000pt}%
\definecolor{currentstroke}{rgb}{0.000000,0.000000,0.000000}%
\pgfsetstrokecolor{currentstroke}%
\pgfsetdash{}{0pt}%
\pgfpathmoveto{\pgfqpoint{4.310753in}{1.342020in}}%
\pgfpathlineto{\pgfqpoint{3.915930in}{1.243314in}}%
\pgfpathlineto{\pgfqpoint{3.922384in}{1.235784in}}%
\pgfpathlineto{\pgfqpoint{3.876129in}{1.237935in}}%
\pgfpathlineto{\pgfqpoint{3.915930in}{1.261601in}}%
\pgfpathlineto{\pgfqpoint{3.913778in}{1.251920in}}%
\pgfpathlineto{\pgfqpoint{4.308602in}{1.350625in}}%
\pgfpathlineto{\pgfqpoint{4.310753in}{1.342020in}}%
\pgfusepath{fill}%
\end{pgfscope}%
\begin{pgfscope}%
\pgfpathrectangle{\pgfqpoint{1.432000in}{0.528000in}}{\pgfqpoint{3.696000in}{3.696000in}} %
\pgfusepath{clip}%
\pgfsetbuttcap%
\pgfsetroundjoin%
\definecolor{currentfill}{rgb}{0.281924,0.089666,0.412415}%
\pgfsetfillcolor{currentfill}%
\pgfsetlinewidth{0.000000pt}%
\definecolor{currentstroke}{rgb}{0.000000,0.000000,0.000000}%
\pgfsetstrokecolor{currentstroke}%
\pgfsetdash{}{0pt}%
\pgfpathmoveto{\pgfqpoint{4.419140in}{1.342020in}}%
\pgfpathlineto{\pgfqpoint{4.024317in}{1.243314in}}%
\pgfpathlineto{\pgfqpoint{4.030771in}{1.235784in}}%
\pgfpathlineto{\pgfqpoint{3.984516in}{1.237935in}}%
\pgfpathlineto{\pgfqpoint{4.024317in}{1.261601in}}%
\pgfpathlineto{\pgfqpoint{4.022165in}{1.251920in}}%
\pgfpathlineto{\pgfqpoint{4.416989in}{1.350625in}}%
\pgfpathlineto{\pgfqpoint{4.419140in}{1.342020in}}%
\pgfusepath{fill}%
\end{pgfscope}%
\begin{pgfscope}%
\pgfpathrectangle{\pgfqpoint{1.432000in}{0.528000in}}{\pgfqpoint{3.696000in}{3.696000in}} %
\pgfusepath{clip}%
\pgfsetbuttcap%
\pgfsetroundjoin%
\definecolor{currentfill}{rgb}{0.280267,0.073417,0.397163}%
\pgfsetfillcolor{currentfill}%
\pgfsetlinewidth{0.000000pt}%
\definecolor{currentstroke}{rgb}{0.000000,0.000000,0.000000}%
\pgfsetstrokecolor{currentstroke}%
\pgfsetdash{}{0pt}%
\pgfpathmoveto{\pgfqpoint{4.528435in}{1.342356in}}%
\pgfpathlineto{\pgfqpoint{4.347364in}{1.251820in}}%
\pgfpathlineto{\pgfqpoint{4.355297in}{1.245869in}}%
\pgfpathlineto{\pgfqpoint{4.309677in}{1.237935in}}%
\pgfpathlineto{\pgfqpoint{4.343397in}{1.269671in}}%
\pgfpathlineto{\pgfqpoint{4.343397in}{1.259754in}}%
\pgfpathlineto{\pgfqpoint{4.524468in}{1.350290in}}%
\pgfpathlineto{\pgfqpoint{4.528435in}{1.342356in}}%
\pgfusepath{fill}%
\end{pgfscope}%
\begin{pgfscope}%
\pgfpathrectangle{\pgfqpoint{1.432000in}{0.528000in}}{\pgfqpoint{3.696000in}{3.696000in}} %
\pgfusepath{clip}%
\pgfsetbuttcap%
\pgfsetroundjoin%
\definecolor{currentfill}{rgb}{0.266580,0.228262,0.514349}%
\pgfsetfillcolor{currentfill}%
\pgfsetlinewidth{0.000000pt}%
\definecolor{currentstroke}{rgb}{0.000000,0.000000,0.000000}%
\pgfsetstrokecolor{currentstroke}%
\pgfsetdash{}{0pt}%
\pgfpathmoveto{\pgfqpoint{4.634839in}{1.341887in}}%
\pgfpathlineto{\pgfqpoint{4.457981in}{1.341887in}}%
\pgfpathlineto{\pgfqpoint{4.462417in}{1.333017in}}%
\pgfpathlineto{\pgfqpoint{4.418065in}{1.346323in}}%
\pgfpathlineto{\pgfqpoint{4.462417in}{1.359628in}}%
\pgfpathlineto{\pgfqpoint{4.457981in}{1.350758in}}%
\pgfpathlineto{\pgfqpoint{4.634839in}{1.350758in}}%
\pgfpathlineto{\pgfqpoint{4.634839in}{1.341887in}}%
\pgfusepath{fill}%
\end{pgfscope}%
\begin{pgfscope}%
\pgfpathrectangle{\pgfqpoint{1.432000in}{0.528000in}}{\pgfqpoint{3.696000in}{3.696000in}} %
\pgfusepath{clip}%
\pgfsetbuttcap%
\pgfsetroundjoin%
\definecolor{currentfill}{rgb}{0.281887,0.150881,0.465405}%
\pgfsetfillcolor{currentfill}%
\pgfsetlinewidth{0.000000pt}%
\definecolor{currentstroke}{rgb}{0.000000,0.000000,0.000000}%
\pgfsetstrokecolor{currentstroke}%
\pgfsetdash{}{0pt}%
\pgfpathmoveto{\pgfqpoint{4.634839in}{1.341887in}}%
\pgfpathlineto{\pgfqpoint{4.566368in}{1.341887in}}%
\pgfpathlineto{\pgfqpoint{4.570804in}{1.333017in}}%
\pgfpathlineto{\pgfqpoint{4.526452in}{1.346323in}}%
\pgfpathlineto{\pgfqpoint{4.570804in}{1.359628in}}%
\pgfpathlineto{\pgfqpoint{4.566368in}{1.350758in}}%
\pgfpathlineto{\pgfqpoint{4.634839in}{1.350758in}}%
\pgfpathlineto{\pgfqpoint{4.634839in}{1.341887in}}%
\pgfusepath{fill}%
\end{pgfscope}%
\begin{pgfscope}%
\pgfpathrectangle{\pgfqpoint{1.432000in}{0.528000in}}{\pgfqpoint{3.696000in}{3.696000in}} %
\pgfusepath{clip}%
\pgfsetbuttcap%
\pgfsetroundjoin%
\definecolor{currentfill}{rgb}{0.271828,0.209303,0.504434}%
\pgfsetfillcolor{currentfill}%
\pgfsetlinewidth{0.000000pt}%
\definecolor{currentstroke}{rgb}{0.000000,0.000000,0.000000}%
\pgfsetstrokecolor{currentstroke}%
\pgfsetdash{}{0pt}%
\pgfpathmoveto{\pgfqpoint{4.743226in}{1.341887in}}%
\pgfpathlineto{\pgfqpoint{4.566368in}{1.341887in}}%
\pgfpathlineto{\pgfqpoint{4.570804in}{1.333017in}}%
\pgfpathlineto{\pgfqpoint{4.526452in}{1.346323in}}%
\pgfpathlineto{\pgfqpoint{4.570804in}{1.359628in}}%
\pgfpathlineto{\pgfqpoint{4.566368in}{1.350758in}}%
\pgfpathlineto{\pgfqpoint{4.743226in}{1.350758in}}%
\pgfpathlineto{\pgfqpoint{4.743226in}{1.341887in}}%
\pgfusepath{fill}%
\end{pgfscope}%
\begin{pgfscope}%
\pgfpathrectangle{\pgfqpoint{1.432000in}{0.528000in}}{\pgfqpoint{3.696000in}{3.696000in}} %
\pgfusepath{clip}%
\pgfsetbuttcap%
\pgfsetroundjoin%
\definecolor{currentfill}{rgb}{0.153364,0.497000,0.557724}%
\pgfsetfillcolor{currentfill}%
\pgfsetlinewidth{0.000000pt}%
\definecolor{currentstroke}{rgb}{0.000000,0.000000,0.000000}%
\pgfsetstrokecolor{currentstroke}%
\pgfsetdash{}{0pt}%
\pgfpathmoveto{\pgfqpoint{4.743226in}{1.341887in}}%
\pgfpathlineto{\pgfqpoint{4.674756in}{1.341887in}}%
\pgfpathlineto{\pgfqpoint{4.679191in}{1.333017in}}%
\pgfpathlineto{\pgfqpoint{4.634839in}{1.346323in}}%
\pgfpathlineto{\pgfqpoint{4.679191in}{1.359628in}}%
\pgfpathlineto{\pgfqpoint{4.674756in}{1.350758in}}%
\pgfpathlineto{\pgfqpoint{4.743226in}{1.350758in}}%
\pgfpathlineto{\pgfqpoint{4.743226in}{1.341887in}}%
\pgfusepath{fill}%
\end{pgfscope}%
\begin{pgfscope}%
\pgfpathrectangle{\pgfqpoint{1.432000in}{0.528000in}}{\pgfqpoint{3.696000in}{3.696000in}} %
\pgfusepath{clip}%
\pgfsetbuttcap%
\pgfsetroundjoin%
\definecolor{currentfill}{rgb}{0.151918,0.500685,0.557587}%
\pgfsetfillcolor{currentfill}%
\pgfsetlinewidth{0.000000pt}%
\definecolor{currentstroke}{rgb}{0.000000,0.000000,0.000000}%
\pgfsetstrokecolor{currentstroke}%
\pgfsetdash{}{0pt}%
\pgfpathmoveto{\pgfqpoint{4.851613in}{1.341887in}}%
\pgfpathlineto{\pgfqpoint{4.783143in}{1.341887in}}%
\pgfpathlineto{\pgfqpoint{4.787578in}{1.333017in}}%
\pgfpathlineto{\pgfqpoint{4.743226in}{1.346323in}}%
\pgfpathlineto{\pgfqpoint{4.787578in}{1.359628in}}%
\pgfpathlineto{\pgfqpoint{4.783143in}{1.350758in}}%
\pgfpathlineto{\pgfqpoint{4.851613in}{1.350758in}}%
\pgfpathlineto{\pgfqpoint{4.851613in}{1.341887in}}%
\pgfusepath{fill}%
\end{pgfscope}%
\begin{pgfscope}%
\pgfpathrectangle{\pgfqpoint{1.432000in}{0.528000in}}{\pgfqpoint{3.696000in}{3.696000in}} %
\pgfusepath{clip}%
\pgfsetbuttcap%
\pgfsetroundjoin%
\definecolor{currentfill}{rgb}{0.282910,0.105393,0.426902}%
\pgfsetfillcolor{currentfill}%
\pgfsetlinewidth{0.000000pt}%
\definecolor{currentstroke}{rgb}{0.000000,0.000000,0.000000}%
\pgfsetstrokecolor{currentstroke}%
\pgfsetdash{}{0pt}%
\pgfpathmoveto{\pgfqpoint{4.856048in}{1.346323in}}%
\pgfpathlineto{\pgfqpoint{4.853831in}{1.350164in}}%
\pgfpathlineto{\pgfqpoint{4.849395in}{1.350164in}}%
\pgfpathlineto{\pgfqpoint{4.847178in}{1.346323in}}%
\pgfpathlineto{\pgfqpoint{4.849395in}{1.342482in}}%
\pgfpathlineto{\pgfqpoint{4.853831in}{1.342482in}}%
\pgfpathlineto{\pgfqpoint{4.856048in}{1.346323in}}%
\pgfpathlineto{\pgfqpoint{4.853831in}{1.350164in}}%
\pgfusepath{fill}%
\end{pgfscope}%
\begin{pgfscope}%
\pgfpathrectangle{\pgfqpoint{1.432000in}{0.528000in}}{\pgfqpoint{3.696000in}{3.696000in}} %
\pgfusepath{clip}%
\pgfsetbuttcap%
\pgfsetroundjoin%
\definecolor{currentfill}{rgb}{0.263663,0.237631,0.518762}%
\pgfsetfillcolor{currentfill}%
\pgfsetlinewidth{0.000000pt}%
\definecolor{currentstroke}{rgb}{0.000000,0.000000,0.000000}%
\pgfsetstrokecolor{currentstroke}%
\pgfsetdash{}{0pt}%
\pgfpathmoveto{\pgfqpoint{4.960000in}{1.341887in}}%
\pgfpathlineto{\pgfqpoint{4.891530in}{1.341887in}}%
\pgfpathlineto{\pgfqpoint{4.895965in}{1.333017in}}%
\pgfpathlineto{\pgfqpoint{4.851613in}{1.346323in}}%
\pgfpathlineto{\pgfqpoint{4.895965in}{1.359628in}}%
\pgfpathlineto{\pgfqpoint{4.891530in}{1.350758in}}%
\pgfpathlineto{\pgfqpoint{4.960000in}{1.350758in}}%
\pgfpathlineto{\pgfqpoint{4.960000in}{1.341887in}}%
\pgfusepath{fill}%
\end{pgfscope}%
\begin{pgfscope}%
\pgfpathrectangle{\pgfqpoint{1.432000in}{0.528000in}}{\pgfqpoint{3.696000in}{3.696000in}} %
\pgfusepath{clip}%
\pgfsetbuttcap%
\pgfsetroundjoin%
\definecolor{currentfill}{rgb}{0.174274,0.445044,0.557792}%
\pgfsetfillcolor{currentfill}%
\pgfsetlinewidth{0.000000pt}%
\definecolor{currentstroke}{rgb}{0.000000,0.000000,0.000000}%
\pgfsetstrokecolor{currentstroke}%
\pgfsetdash{}{0pt}%
\pgfpathmoveto{\pgfqpoint{4.964435in}{1.346323in}}%
\pgfpathlineto{\pgfqpoint{4.962218in}{1.350164in}}%
\pgfpathlineto{\pgfqpoint{4.957782in}{1.350164in}}%
\pgfpathlineto{\pgfqpoint{4.955565in}{1.346323in}}%
\pgfpathlineto{\pgfqpoint{4.957782in}{1.342482in}}%
\pgfpathlineto{\pgfqpoint{4.962218in}{1.342482in}}%
\pgfpathlineto{\pgfqpoint{4.964435in}{1.346323in}}%
\pgfpathlineto{\pgfqpoint{4.962218in}{1.350164in}}%
\pgfusepath{fill}%
\end{pgfscope}%
\begin{pgfscope}%
\pgfpathrectangle{\pgfqpoint{1.432000in}{0.528000in}}{\pgfqpoint{3.696000in}{3.696000in}} %
\pgfusepath{clip}%
\pgfsetbuttcap%
\pgfsetroundjoin%
\definecolor{currentfill}{rgb}{0.278791,0.062145,0.386592}%
\pgfsetfillcolor{currentfill}%
\pgfsetlinewidth{0.000000pt}%
\definecolor{currentstroke}{rgb}{0.000000,0.000000,0.000000}%
\pgfsetstrokecolor{currentstroke}%
\pgfsetdash{}{0pt}%
\pgfpathmoveto{\pgfqpoint{1.604435in}{1.454710in}}%
\pgfpathlineto{\pgfqpoint{1.604435in}{1.277852in}}%
\pgfpathlineto{\pgfqpoint{1.613306in}{1.282287in}}%
\pgfpathlineto{\pgfqpoint{1.600000in}{1.237935in}}%
\pgfpathlineto{\pgfqpoint{1.586694in}{1.282287in}}%
\pgfpathlineto{\pgfqpoint{1.595565in}{1.277852in}}%
\pgfpathlineto{\pgfqpoint{1.595565in}{1.454710in}}%
\pgfpathlineto{\pgfqpoint{1.604435in}{1.454710in}}%
\pgfusepath{fill}%
\end{pgfscope}%
\begin{pgfscope}%
\pgfpathrectangle{\pgfqpoint{1.432000in}{0.528000in}}{\pgfqpoint{3.696000in}{3.696000in}} %
\pgfusepath{clip}%
\pgfsetbuttcap%
\pgfsetroundjoin%
\definecolor{currentfill}{rgb}{0.278826,0.175490,0.483397}%
\pgfsetfillcolor{currentfill}%
\pgfsetlinewidth{0.000000pt}%
\definecolor{currentstroke}{rgb}{0.000000,0.000000,0.000000}%
\pgfsetstrokecolor{currentstroke}%
\pgfsetdash{}{0pt}%
\pgfpathmoveto{\pgfqpoint{1.604435in}{1.454710in}}%
\pgfpathlineto{\pgfqpoint{1.604435in}{1.386239in}}%
\pgfpathlineto{\pgfqpoint{1.613306in}{1.390675in}}%
\pgfpathlineto{\pgfqpoint{1.600000in}{1.346323in}}%
\pgfpathlineto{\pgfqpoint{1.586694in}{1.390675in}}%
\pgfpathlineto{\pgfqpoint{1.595565in}{1.386239in}}%
\pgfpathlineto{\pgfqpoint{1.595565in}{1.454710in}}%
\pgfpathlineto{\pgfqpoint{1.604435in}{1.454710in}}%
\pgfusepath{fill}%
\end{pgfscope}%
\begin{pgfscope}%
\pgfpathrectangle{\pgfqpoint{1.432000in}{0.528000in}}{\pgfqpoint{3.696000in}{3.696000in}} %
\pgfusepath{clip}%
\pgfsetbuttcap%
\pgfsetroundjoin%
\definecolor{currentfill}{rgb}{0.281924,0.089666,0.412415}%
\pgfsetfillcolor{currentfill}%
\pgfsetlinewidth{0.000000pt}%
\definecolor{currentstroke}{rgb}{0.000000,0.000000,0.000000}%
\pgfsetstrokecolor{currentstroke}%
\pgfsetdash{}{0pt}%
\pgfpathmoveto{\pgfqpoint{1.712354in}{1.452726in}}%
\pgfpathlineto{\pgfqpoint{1.621818in}{1.271655in}}%
\pgfpathlineto{\pgfqpoint{1.631736in}{1.271655in}}%
\pgfpathlineto{\pgfqpoint{1.600000in}{1.237935in}}%
\pgfpathlineto{\pgfqpoint{1.607934in}{1.283556in}}%
\pgfpathlineto{\pgfqpoint{1.613884in}{1.275622in}}%
\pgfpathlineto{\pgfqpoint{1.704420in}{1.456693in}}%
\pgfpathlineto{\pgfqpoint{1.712354in}{1.452726in}}%
\pgfusepath{fill}%
\end{pgfscope}%
\begin{pgfscope}%
\pgfpathrectangle{\pgfqpoint{1.432000in}{0.528000in}}{\pgfqpoint{3.696000in}{3.696000in}} %
\pgfusepath{clip}%
\pgfsetbuttcap%
\pgfsetroundjoin%
\definecolor{currentfill}{rgb}{0.253935,0.265254,0.529983}%
\pgfsetfillcolor{currentfill}%
\pgfsetlinewidth{0.000000pt}%
\definecolor{currentstroke}{rgb}{0.000000,0.000000,0.000000}%
\pgfsetstrokecolor{currentstroke}%
\pgfsetdash{}{0pt}%
\pgfpathmoveto{\pgfqpoint{1.711523in}{1.451574in}}%
\pgfpathlineto{\pgfqpoint{1.631362in}{1.371412in}}%
\pgfpathlineto{\pgfqpoint{1.640770in}{1.368276in}}%
\pgfpathlineto{\pgfqpoint{1.600000in}{1.346323in}}%
\pgfpathlineto{\pgfqpoint{1.621953in}{1.387093in}}%
\pgfpathlineto{\pgfqpoint{1.625089in}{1.377684in}}%
\pgfpathlineto{\pgfqpoint{1.705251in}{1.457846in}}%
\pgfpathlineto{\pgfqpoint{1.711523in}{1.451574in}}%
\pgfusepath{fill}%
\end{pgfscope}%
\begin{pgfscope}%
\pgfpathrectangle{\pgfqpoint{1.432000in}{0.528000in}}{\pgfqpoint{3.696000in}{3.696000in}} %
\pgfusepath{clip}%
\pgfsetbuttcap%
\pgfsetroundjoin%
\definecolor{currentfill}{rgb}{0.182256,0.426184,0.557120}%
\pgfsetfillcolor{currentfill}%
\pgfsetlinewidth{0.000000pt}%
\definecolor{currentstroke}{rgb}{0.000000,0.000000,0.000000}%
\pgfsetstrokecolor{currentstroke}%
\pgfsetdash{}{0pt}%
\pgfpathmoveto{\pgfqpoint{1.819910in}{1.451574in}}%
\pgfpathlineto{\pgfqpoint{1.739749in}{1.371412in}}%
\pgfpathlineto{\pgfqpoint{1.749157in}{1.368276in}}%
\pgfpathlineto{\pgfqpoint{1.708387in}{1.346323in}}%
\pgfpathlineto{\pgfqpoint{1.730340in}{1.387093in}}%
\pgfpathlineto{\pgfqpoint{1.733476in}{1.377684in}}%
\pgfpathlineto{\pgfqpoint{1.813638in}{1.457846in}}%
\pgfpathlineto{\pgfqpoint{1.819910in}{1.451574in}}%
\pgfusepath{fill}%
\end{pgfscope}%
\begin{pgfscope}%
\pgfpathrectangle{\pgfqpoint{1.432000in}{0.528000in}}{\pgfqpoint{3.696000in}{3.696000in}} %
\pgfusepath{clip}%
\pgfsetbuttcap%
\pgfsetroundjoin%
\definecolor{currentfill}{rgb}{0.270595,0.214069,0.507052}%
\pgfsetfillcolor{currentfill}%
\pgfsetlinewidth{0.000000pt}%
\definecolor{currentstroke}{rgb}{0.000000,0.000000,0.000000}%
\pgfsetstrokecolor{currentstroke}%
\pgfsetdash{}{0pt}%
\pgfpathmoveto{\pgfqpoint{1.927145in}{1.450743in}}%
\pgfpathlineto{\pgfqpoint{1.746073in}{1.360207in}}%
\pgfpathlineto{\pgfqpoint{1.754007in}{1.354257in}}%
\pgfpathlineto{\pgfqpoint{1.708387in}{1.346323in}}%
\pgfpathlineto{\pgfqpoint{1.742106in}{1.378058in}}%
\pgfpathlineto{\pgfqpoint{1.742106in}{1.368141in}}%
\pgfpathlineto{\pgfqpoint{1.923178in}{1.458677in}}%
\pgfpathlineto{\pgfqpoint{1.927145in}{1.450743in}}%
\pgfusepath{fill}%
\end{pgfscope}%
\begin{pgfscope}%
\pgfpathrectangle{\pgfqpoint{1.432000in}{0.528000in}}{\pgfqpoint{3.696000in}{3.696000in}} %
\pgfusepath{clip}%
\pgfsetbuttcap%
\pgfsetroundjoin%
\definecolor{currentfill}{rgb}{0.281446,0.084320,0.407414}%
\pgfsetfillcolor{currentfill}%
\pgfsetlinewidth{0.000000pt}%
\definecolor{currentstroke}{rgb}{0.000000,0.000000,0.000000}%
\pgfsetstrokecolor{currentstroke}%
\pgfsetdash{}{0pt}%
\pgfpathmoveto{\pgfqpoint{1.928297in}{1.451574in}}%
\pgfpathlineto{\pgfqpoint{1.848136in}{1.371412in}}%
\pgfpathlineto{\pgfqpoint{1.857544in}{1.368276in}}%
\pgfpathlineto{\pgfqpoint{1.816774in}{1.346323in}}%
\pgfpathlineto{\pgfqpoint{1.838727in}{1.387093in}}%
\pgfpathlineto{\pgfqpoint{1.841863in}{1.377684in}}%
\pgfpathlineto{\pgfqpoint{1.922025in}{1.457846in}}%
\pgfpathlineto{\pgfqpoint{1.928297in}{1.451574in}}%
\pgfusepath{fill}%
\end{pgfscope}%
\begin{pgfscope}%
\pgfpathrectangle{\pgfqpoint{1.432000in}{0.528000in}}{\pgfqpoint{3.696000in}{3.696000in}} %
\pgfusepath{clip}%
\pgfsetbuttcap%
\pgfsetroundjoin%
\definecolor{currentfill}{rgb}{0.135066,0.544853,0.554029}%
\pgfsetfillcolor{currentfill}%
\pgfsetlinewidth{0.000000pt}%
\definecolor{currentstroke}{rgb}{0.000000,0.000000,0.000000}%
\pgfsetstrokecolor{currentstroke}%
\pgfsetdash{}{0pt}%
\pgfpathmoveto{\pgfqpoint{2.035532in}{1.450743in}}%
\pgfpathlineto{\pgfqpoint{1.854460in}{1.360207in}}%
\pgfpathlineto{\pgfqpoint{1.862394in}{1.354257in}}%
\pgfpathlineto{\pgfqpoint{1.816774in}{1.346323in}}%
\pgfpathlineto{\pgfqpoint{1.850493in}{1.378058in}}%
\pgfpathlineto{\pgfqpoint{1.850493in}{1.368141in}}%
\pgfpathlineto{\pgfqpoint{2.031565in}{1.458677in}}%
\pgfpathlineto{\pgfqpoint{2.035532in}{1.450743in}}%
\pgfusepath{fill}%
\end{pgfscope}%
\begin{pgfscope}%
\pgfpathrectangle{\pgfqpoint{1.432000in}{0.528000in}}{\pgfqpoint{3.696000in}{3.696000in}} %
\pgfusepath{clip}%
\pgfsetbuttcap%
\pgfsetroundjoin%
\definecolor{currentfill}{rgb}{0.124395,0.578002,0.548287}%
\pgfsetfillcolor{currentfill}%
\pgfsetlinewidth{0.000000pt}%
\definecolor{currentstroke}{rgb}{0.000000,0.000000,0.000000}%
\pgfsetstrokecolor{currentstroke}%
\pgfsetdash{}{0pt}%
\pgfpathmoveto{\pgfqpoint{2.143919in}{1.450743in}}%
\pgfpathlineto{\pgfqpoint{1.962847in}{1.360207in}}%
\pgfpathlineto{\pgfqpoint{1.970781in}{1.354257in}}%
\pgfpathlineto{\pgfqpoint{1.925161in}{1.346323in}}%
\pgfpathlineto{\pgfqpoint{1.958880in}{1.378058in}}%
\pgfpathlineto{\pgfqpoint{1.958880in}{1.368141in}}%
\pgfpathlineto{\pgfqpoint{2.139952in}{1.458677in}}%
\pgfpathlineto{\pgfqpoint{2.143919in}{1.450743in}}%
\pgfusepath{fill}%
\end{pgfscope}%
\begin{pgfscope}%
\pgfpathrectangle{\pgfqpoint{1.432000in}{0.528000in}}{\pgfqpoint{3.696000in}{3.696000in}} %
\pgfusepath{clip}%
\pgfsetbuttcap%
\pgfsetroundjoin%
\definecolor{currentfill}{rgb}{0.179019,0.433756,0.557430}%
\pgfsetfillcolor{currentfill}%
\pgfsetlinewidth{0.000000pt}%
\definecolor{currentstroke}{rgb}{0.000000,0.000000,0.000000}%
\pgfsetstrokecolor{currentstroke}%
\pgfsetdash{}{0pt}%
\pgfpathmoveto{\pgfqpoint{2.252306in}{1.450743in}}%
\pgfpathlineto{\pgfqpoint{2.071235in}{1.360207in}}%
\pgfpathlineto{\pgfqpoint{2.079168in}{1.354257in}}%
\pgfpathlineto{\pgfqpoint{2.033548in}{1.346323in}}%
\pgfpathlineto{\pgfqpoint{2.067268in}{1.378058in}}%
\pgfpathlineto{\pgfqpoint{2.067268in}{1.368141in}}%
\pgfpathlineto{\pgfqpoint{2.248339in}{1.458677in}}%
\pgfpathlineto{\pgfqpoint{2.252306in}{1.450743in}}%
\pgfusepath{fill}%
\end{pgfscope}%
\begin{pgfscope}%
\pgfpathrectangle{\pgfqpoint{1.432000in}{0.528000in}}{\pgfqpoint{3.696000in}{3.696000in}} %
\pgfusepath{clip}%
\pgfsetbuttcap%
\pgfsetroundjoin%
\definecolor{currentfill}{rgb}{0.273809,0.031497,0.358853}%
\pgfsetfillcolor{currentfill}%
\pgfsetlinewidth{0.000000pt}%
\definecolor{currentstroke}{rgb}{0.000000,0.000000,0.000000}%
\pgfsetstrokecolor{currentstroke}%
\pgfsetdash{}{0pt}%
\pgfpathmoveto{\pgfqpoint{2.360693in}{1.450743in}}%
\pgfpathlineto{\pgfqpoint{2.179622in}{1.360207in}}%
\pgfpathlineto{\pgfqpoint{2.187556in}{1.354257in}}%
\pgfpathlineto{\pgfqpoint{2.141935in}{1.346323in}}%
\pgfpathlineto{\pgfqpoint{2.175655in}{1.378058in}}%
\pgfpathlineto{\pgfqpoint{2.175655in}{1.368141in}}%
\pgfpathlineto{\pgfqpoint{2.356726in}{1.458677in}}%
\pgfpathlineto{\pgfqpoint{2.360693in}{1.450743in}}%
\pgfusepath{fill}%
\end{pgfscope}%
\begin{pgfscope}%
\pgfpathrectangle{\pgfqpoint{1.432000in}{0.528000in}}{\pgfqpoint{3.696000in}{3.696000in}} %
\pgfusepath{clip}%
\pgfsetbuttcap%
\pgfsetroundjoin%
\definecolor{currentfill}{rgb}{0.221989,0.339161,0.548752}%
\pgfsetfillcolor{currentfill}%
\pgfsetlinewidth{0.000000pt}%
\definecolor{currentstroke}{rgb}{0.000000,0.000000,0.000000}%
\pgfsetstrokecolor{currentstroke}%
\pgfsetdash{}{0pt}%
\pgfpathmoveto{\pgfqpoint{2.358710in}{1.450274in}}%
\pgfpathlineto{\pgfqpoint{2.181852in}{1.450274in}}%
\pgfpathlineto{\pgfqpoint{2.186287in}{1.441404in}}%
\pgfpathlineto{\pgfqpoint{2.141935in}{1.454710in}}%
\pgfpathlineto{\pgfqpoint{2.186287in}{1.468015in}}%
\pgfpathlineto{\pgfqpoint{2.181852in}{1.459145in}}%
\pgfpathlineto{\pgfqpoint{2.358710in}{1.459145in}}%
\pgfpathlineto{\pgfqpoint{2.358710in}{1.450274in}}%
\pgfusepath{fill}%
\end{pgfscope}%
\begin{pgfscope}%
\pgfpathrectangle{\pgfqpoint{1.432000in}{0.528000in}}{\pgfqpoint{3.696000in}{3.696000in}} %
\pgfusepath{clip}%
\pgfsetbuttcap%
\pgfsetroundjoin%
\definecolor{currentfill}{rgb}{0.172719,0.448791,0.557885}%
\pgfsetfillcolor{currentfill}%
\pgfsetlinewidth{0.000000pt}%
\definecolor{currentstroke}{rgb}{0.000000,0.000000,0.000000}%
\pgfsetstrokecolor{currentstroke}%
\pgfsetdash{}{0pt}%
\pgfpathmoveto{\pgfqpoint{2.467097in}{1.450274in}}%
\pgfpathlineto{\pgfqpoint{2.290239in}{1.450274in}}%
\pgfpathlineto{\pgfqpoint{2.294675in}{1.441404in}}%
\pgfpathlineto{\pgfqpoint{2.250323in}{1.454710in}}%
\pgfpathlineto{\pgfqpoint{2.294675in}{1.468015in}}%
\pgfpathlineto{\pgfqpoint{2.290239in}{1.459145in}}%
\pgfpathlineto{\pgfqpoint{2.467097in}{1.459145in}}%
\pgfpathlineto{\pgfqpoint{2.467097in}{1.450274in}}%
\pgfusepath{fill}%
\end{pgfscope}%
\begin{pgfscope}%
\pgfpathrectangle{\pgfqpoint{1.432000in}{0.528000in}}{\pgfqpoint{3.696000in}{3.696000in}} %
\pgfusepath{clip}%
\pgfsetbuttcap%
\pgfsetroundjoin%
\definecolor{currentfill}{rgb}{0.267004,0.004874,0.329415}%
\pgfsetfillcolor{currentfill}%
\pgfsetlinewidth{0.000000pt}%
\definecolor{currentstroke}{rgb}{0.000000,0.000000,0.000000}%
\pgfsetstrokecolor{currentstroke}%
\pgfsetdash{}{0pt}%
\pgfpathmoveto{\pgfqpoint{2.575484in}{1.450274in}}%
\pgfpathlineto{\pgfqpoint{2.290239in}{1.450274in}}%
\pgfpathlineto{\pgfqpoint{2.294675in}{1.441404in}}%
\pgfpathlineto{\pgfqpoint{2.250323in}{1.454710in}}%
\pgfpathlineto{\pgfqpoint{2.294675in}{1.468015in}}%
\pgfpathlineto{\pgfqpoint{2.290239in}{1.459145in}}%
\pgfpathlineto{\pgfqpoint{2.575484in}{1.459145in}}%
\pgfpathlineto{\pgfqpoint{2.575484in}{1.450274in}}%
\pgfusepath{fill}%
\end{pgfscope}%
\begin{pgfscope}%
\pgfpathrectangle{\pgfqpoint{1.432000in}{0.528000in}}{\pgfqpoint{3.696000in}{3.696000in}} %
\pgfusepath{clip}%
\pgfsetbuttcap%
\pgfsetroundjoin%
\definecolor{currentfill}{rgb}{0.231674,0.318106,0.544834}%
\pgfsetfillcolor{currentfill}%
\pgfsetlinewidth{0.000000pt}%
\definecolor{currentstroke}{rgb}{0.000000,0.000000,0.000000}%
\pgfsetstrokecolor{currentstroke}%
\pgfsetdash{}{0pt}%
\pgfpathmoveto{\pgfqpoint{2.575484in}{1.450274in}}%
\pgfpathlineto{\pgfqpoint{2.398626in}{1.450274in}}%
\pgfpathlineto{\pgfqpoint{2.403062in}{1.441404in}}%
\pgfpathlineto{\pgfqpoint{2.358710in}{1.454710in}}%
\pgfpathlineto{\pgfqpoint{2.403062in}{1.468015in}}%
\pgfpathlineto{\pgfqpoint{2.398626in}{1.459145in}}%
\pgfpathlineto{\pgfqpoint{2.575484in}{1.459145in}}%
\pgfpathlineto{\pgfqpoint{2.575484in}{1.450274in}}%
\pgfusepath{fill}%
\end{pgfscope}%
\begin{pgfscope}%
\pgfpathrectangle{\pgfqpoint{1.432000in}{0.528000in}}{\pgfqpoint{3.696000in}{3.696000in}} %
\pgfusepath{clip}%
\pgfsetbuttcap%
\pgfsetroundjoin%
\definecolor{currentfill}{rgb}{0.218130,0.347432,0.550038}%
\pgfsetfillcolor{currentfill}%
\pgfsetlinewidth{0.000000pt}%
\definecolor{currentstroke}{rgb}{0.000000,0.000000,0.000000}%
\pgfsetstrokecolor{currentstroke}%
\pgfsetdash{}{0pt}%
\pgfpathmoveto{\pgfqpoint{2.683871in}{1.450274in}}%
\pgfpathlineto{\pgfqpoint{2.398626in}{1.450274in}}%
\pgfpathlineto{\pgfqpoint{2.403062in}{1.441404in}}%
\pgfpathlineto{\pgfqpoint{2.358710in}{1.454710in}}%
\pgfpathlineto{\pgfqpoint{2.403062in}{1.468015in}}%
\pgfpathlineto{\pgfqpoint{2.398626in}{1.459145in}}%
\pgfpathlineto{\pgfqpoint{2.683871in}{1.459145in}}%
\pgfpathlineto{\pgfqpoint{2.683871in}{1.450274in}}%
\pgfusepath{fill}%
\end{pgfscope}%
\begin{pgfscope}%
\pgfpathrectangle{\pgfqpoint{1.432000in}{0.528000in}}{\pgfqpoint{3.696000in}{3.696000in}} %
\pgfusepath{clip}%
\pgfsetbuttcap%
\pgfsetroundjoin%
\definecolor{currentfill}{rgb}{0.121831,0.589055,0.545623}%
\pgfsetfillcolor{currentfill}%
\pgfsetlinewidth{0.000000pt}%
\definecolor{currentstroke}{rgb}{0.000000,0.000000,0.000000}%
\pgfsetstrokecolor{currentstroke}%
\pgfsetdash{}{0pt}%
\pgfpathmoveto{\pgfqpoint{2.792258in}{1.450274in}}%
\pgfpathlineto{\pgfqpoint{2.507014in}{1.450274in}}%
\pgfpathlineto{\pgfqpoint{2.511449in}{1.441404in}}%
\pgfpathlineto{\pgfqpoint{2.467097in}{1.454710in}}%
\pgfpathlineto{\pgfqpoint{2.511449in}{1.468015in}}%
\pgfpathlineto{\pgfqpoint{2.507014in}{1.459145in}}%
\pgfpathlineto{\pgfqpoint{2.792258in}{1.459145in}}%
\pgfpathlineto{\pgfqpoint{2.792258in}{1.450274in}}%
\pgfusepath{fill}%
\end{pgfscope}%
\begin{pgfscope}%
\pgfpathrectangle{\pgfqpoint{1.432000in}{0.528000in}}{\pgfqpoint{3.696000in}{3.696000in}} %
\pgfusepath{clip}%
\pgfsetbuttcap%
\pgfsetroundjoin%
\definecolor{currentfill}{rgb}{0.144759,0.519093,0.556572}%
\pgfsetfillcolor{currentfill}%
\pgfsetlinewidth{0.000000pt}%
\definecolor{currentstroke}{rgb}{0.000000,0.000000,0.000000}%
\pgfsetstrokecolor{currentstroke}%
\pgfsetdash{}{0pt}%
\pgfpathmoveto{\pgfqpoint{2.900645in}{1.450274in}}%
\pgfpathlineto{\pgfqpoint{2.615401in}{1.450274in}}%
\pgfpathlineto{\pgfqpoint{2.619836in}{1.441404in}}%
\pgfpathlineto{\pgfqpoint{2.575484in}{1.454710in}}%
\pgfpathlineto{\pgfqpoint{2.619836in}{1.468015in}}%
\pgfpathlineto{\pgfqpoint{2.615401in}{1.459145in}}%
\pgfpathlineto{\pgfqpoint{2.900645in}{1.459145in}}%
\pgfpathlineto{\pgfqpoint{2.900645in}{1.450274in}}%
\pgfusepath{fill}%
\end{pgfscope}%
\begin{pgfscope}%
\pgfpathrectangle{\pgfqpoint{1.432000in}{0.528000in}}{\pgfqpoint{3.696000in}{3.696000in}} %
\pgfusepath{clip}%
\pgfsetbuttcap%
\pgfsetroundjoin%
\definecolor{currentfill}{rgb}{0.273809,0.031497,0.358853}%
\pgfsetfillcolor{currentfill}%
\pgfsetlinewidth{0.000000pt}%
\definecolor{currentstroke}{rgb}{0.000000,0.000000,0.000000}%
\pgfsetstrokecolor{currentstroke}%
\pgfsetdash{}{0pt}%
\pgfpathmoveto{\pgfqpoint{2.900645in}{1.450274in}}%
\pgfpathlineto{\pgfqpoint{2.723788in}{1.450274in}}%
\pgfpathlineto{\pgfqpoint{2.728223in}{1.441404in}}%
\pgfpathlineto{\pgfqpoint{2.683871in}{1.454710in}}%
\pgfpathlineto{\pgfqpoint{2.728223in}{1.468015in}}%
\pgfpathlineto{\pgfqpoint{2.723788in}{1.459145in}}%
\pgfpathlineto{\pgfqpoint{2.900645in}{1.459145in}}%
\pgfpathlineto{\pgfqpoint{2.900645in}{1.450274in}}%
\pgfusepath{fill}%
\end{pgfscope}%
\begin{pgfscope}%
\pgfpathrectangle{\pgfqpoint{1.432000in}{0.528000in}}{\pgfqpoint{3.696000in}{3.696000in}} %
\pgfusepath{clip}%
\pgfsetbuttcap%
\pgfsetroundjoin%
\definecolor{currentfill}{rgb}{0.185556,0.418570,0.556753}%
\pgfsetfillcolor{currentfill}%
\pgfsetlinewidth{0.000000pt}%
\definecolor{currentstroke}{rgb}{0.000000,0.000000,0.000000}%
\pgfsetstrokecolor{currentstroke}%
\pgfsetdash{}{0pt}%
\pgfpathmoveto{\pgfqpoint{3.009032in}{1.450274in}}%
\pgfpathlineto{\pgfqpoint{2.723788in}{1.450274in}}%
\pgfpathlineto{\pgfqpoint{2.728223in}{1.441404in}}%
\pgfpathlineto{\pgfqpoint{2.683871in}{1.454710in}}%
\pgfpathlineto{\pgfqpoint{2.728223in}{1.468015in}}%
\pgfpathlineto{\pgfqpoint{2.723788in}{1.459145in}}%
\pgfpathlineto{\pgfqpoint{3.009032in}{1.459145in}}%
\pgfpathlineto{\pgfqpoint{3.009032in}{1.450274in}}%
\pgfusepath{fill}%
\end{pgfscope}%
\begin{pgfscope}%
\pgfpathrectangle{\pgfqpoint{1.432000in}{0.528000in}}{\pgfqpoint{3.696000in}{3.696000in}} %
\pgfusepath{clip}%
\pgfsetbuttcap%
\pgfsetroundjoin%
\definecolor{currentfill}{rgb}{0.257322,0.256130,0.526563}%
\pgfsetfillcolor{currentfill}%
\pgfsetlinewidth{0.000000pt}%
\definecolor{currentstroke}{rgb}{0.000000,0.000000,0.000000}%
\pgfsetstrokecolor{currentstroke}%
\pgfsetdash{}{0pt}%
\pgfpathmoveto{\pgfqpoint{3.009032in}{1.450274in}}%
\pgfpathlineto{\pgfqpoint{2.832175in}{1.450274in}}%
\pgfpathlineto{\pgfqpoint{2.836610in}{1.441404in}}%
\pgfpathlineto{\pgfqpoint{2.792258in}{1.454710in}}%
\pgfpathlineto{\pgfqpoint{2.836610in}{1.468015in}}%
\pgfpathlineto{\pgfqpoint{2.832175in}{1.459145in}}%
\pgfpathlineto{\pgfqpoint{3.009032in}{1.459145in}}%
\pgfpathlineto{\pgfqpoint{3.009032in}{1.450274in}}%
\pgfusepath{fill}%
\end{pgfscope}%
\begin{pgfscope}%
\pgfpathrectangle{\pgfqpoint{1.432000in}{0.528000in}}{\pgfqpoint{3.696000in}{3.696000in}} %
\pgfusepath{clip}%
\pgfsetbuttcap%
\pgfsetroundjoin%
\definecolor{currentfill}{rgb}{0.270595,0.214069,0.507052}%
\pgfsetfillcolor{currentfill}%
\pgfsetlinewidth{0.000000pt}%
\definecolor{currentstroke}{rgb}{0.000000,0.000000,0.000000}%
\pgfsetstrokecolor{currentstroke}%
\pgfsetdash{}{0pt}%
\pgfpathmoveto{\pgfqpoint{3.117419in}{1.450274in}}%
\pgfpathlineto{\pgfqpoint{2.832175in}{1.450274in}}%
\pgfpathlineto{\pgfqpoint{2.836610in}{1.441404in}}%
\pgfpathlineto{\pgfqpoint{2.792258in}{1.454710in}}%
\pgfpathlineto{\pgfqpoint{2.836610in}{1.468015in}}%
\pgfpathlineto{\pgfqpoint{2.832175in}{1.459145in}}%
\pgfpathlineto{\pgfqpoint{3.117419in}{1.459145in}}%
\pgfpathlineto{\pgfqpoint{3.117419in}{1.450274in}}%
\pgfusepath{fill}%
\end{pgfscope}%
\begin{pgfscope}%
\pgfpathrectangle{\pgfqpoint{1.432000in}{0.528000in}}{\pgfqpoint{3.696000in}{3.696000in}} %
\pgfusepath{clip}%
\pgfsetbuttcap%
\pgfsetroundjoin%
\definecolor{currentfill}{rgb}{0.201239,0.383670,0.554294}%
\pgfsetfillcolor{currentfill}%
\pgfsetlinewidth{0.000000pt}%
\definecolor{currentstroke}{rgb}{0.000000,0.000000,0.000000}%
\pgfsetstrokecolor{currentstroke}%
\pgfsetdash{}{0pt}%
\pgfpathmoveto{\pgfqpoint{3.117419in}{1.450274in}}%
\pgfpathlineto{\pgfqpoint{2.940562in}{1.450274in}}%
\pgfpathlineto{\pgfqpoint{2.944997in}{1.441404in}}%
\pgfpathlineto{\pgfqpoint{2.900645in}{1.454710in}}%
\pgfpathlineto{\pgfqpoint{2.944997in}{1.468015in}}%
\pgfpathlineto{\pgfqpoint{2.940562in}{1.459145in}}%
\pgfpathlineto{\pgfqpoint{3.117419in}{1.459145in}}%
\pgfpathlineto{\pgfqpoint{3.117419in}{1.450274in}}%
\pgfusepath{fill}%
\end{pgfscope}%
\begin{pgfscope}%
\pgfpathrectangle{\pgfqpoint{1.432000in}{0.528000in}}{\pgfqpoint{3.696000in}{3.696000in}} %
\pgfusepath{clip}%
\pgfsetbuttcap%
\pgfsetroundjoin%
\definecolor{currentfill}{rgb}{0.126326,0.644107,0.525311}%
\pgfsetfillcolor{currentfill}%
\pgfsetlinewidth{0.000000pt}%
\definecolor{currentstroke}{rgb}{0.000000,0.000000,0.000000}%
\pgfsetstrokecolor{currentstroke}%
\pgfsetdash{}{0pt}%
\pgfpathmoveto{\pgfqpoint{3.225806in}{1.450274in}}%
\pgfpathlineto{\pgfqpoint{3.048949in}{1.450274in}}%
\pgfpathlineto{\pgfqpoint{3.053384in}{1.441404in}}%
\pgfpathlineto{\pgfqpoint{3.009032in}{1.454710in}}%
\pgfpathlineto{\pgfqpoint{3.053384in}{1.468015in}}%
\pgfpathlineto{\pgfqpoint{3.048949in}{1.459145in}}%
\pgfpathlineto{\pgfqpoint{3.225806in}{1.459145in}}%
\pgfpathlineto{\pgfqpoint{3.225806in}{1.450274in}}%
\pgfusepath{fill}%
\end{pgfscope}%
\begin{pgfscope}%
\pgfpathrectangle{\pgfqpoint{1.432000in}{0.528000in}}{\pgfqpoint{3.696000in}{3.696000in}} %
\pgfusepath{clip}%
\pgfsetbuttcap%
\pgfsetroundjoin%
\definecolor{currentfill}{rgb}{0.157851,0.683765,0.501686}%
\pgfsetfillcolor{currentfill}%
\pgfsetlinewidth{0.000000pt}%
\definecolor{currentstroke}{rgb}{0.000000,0.000000,0.000000}%
\pgfsetstrokecolor{currentstroke}%
\pgfsetdash{}{0pt}%
\pgfpathmoveto{\pgfqpoint{3.334194in}{1.450274in}}%
\pgfpathlineto{\pgfqpoint{3.157336in}{1.450274in}}%
\pgfpathlineto{\pgfqpoint{3.161771in}{1.441404in}}%
\pgfpathlineto{\pgfqpoint{3.117419in}{1.454710in}}%
\pgfpathlineto{\pgfqpoint{3.161771in}{1.468015in}}%
\pgfpathlineto{\pgfqpoint{3.157336in}{1.459145in}}%
\pgfpathlineto{\pgfqpoint{3.334194in}{1.459145in}}%
\pgfpathlineto{\pgfqpoint{3.334194in}{1.450274in}}%
\pgfusepath{fill}%
\end{pgfscope}%
\begin{pgfscope}%
\pgfpathrectangle{\pgfqpoint{1.432000in}{0.528000in}}{\pgfqpoint{3.696000in}{3.696000in}} %
\pgfusepath{clip}%
\pgfsetbuttcap%
\pgfsetroundjoin%
\definecolor{currentfill}{rgb}{0.125394,0.574318,0.549086}%
\pgfsetfillcolor{currentfill}%
\pgfsetlinewidth{0.000000pt}%
\definecolor{currentstroke}{rgb}{0.000000,0.000000,0.000000}%
\pgfsetstrokecolor{currentstroke}%
\pgfsetdash{}{0pt}%
\pgfpathmoveto{\pgfqpoint{3.442581in}{1.450274in}}%
\pgfpathlineto{\pgfqpoint{3.265723in}{1.450274in}}%
\pgfpathlineto{\pgfqpoint{3.270158in}{1.441404in}}%
\pgfpathlineto{\pgfqpoint{3.225806in}{1.454710in}}%
\pgfpathlineto{\pgfqpoint{3.270158in}{1.468015in}}%
\pgfpathlineto{\pgfqpoint{3.265723in}{1.459145in}}%
\pgfpathlineto{\pgfqpoint{3.442581in}{1.459145in}}%
\pgfpathlineto{\pgfqpoint{3.442581in}{1.450274in}}%
\pgfusepath{fill}%
\end{pgfscope}%
\begin{pgfscope}%
\pgfpathrectangle{\pgfqpoint{1.432000in}{0.528000in}}{\pgfqpoint{3.696000in}{3.696000in}} %
\pgfusepath{clip}%
\pgfsetbuttcap%
\pgfsetroundjoin%
\definecolor{currentfill}{rgb}{0.141935,0.526453,0.555991}%
\pgfsetfillcolor{currentfill}%
\pgfsetlinewidth{0.000000pt}%
\definecolor{currentstroke}{rgb}{0.000000,0.000000,0.000000}%
\pgfsetstrokecolor{currentstroke}%
\pgfsetdash{}{0pt}%
\pgfpathmoveto{\pgfqpoint{3.550968in}{1.450274in}}%
\pgfpathlineto{\pgfqpoint{3.374110in}{1.450274in}}%
\pgfpathlineto{\pgfqpoint{3.378546in}{1.441404in}}%
\pgfpathlineto{\pgfqpoint{3.334194in}{1.454710in}}%
\pgfpathlineto{\pgfqpoint{3.378546in}{1.468015in}}%
\pgfpathlineto{\pgfqpoint{3.374110in}{1.459145in}}%
\pgfpathlineto{\pgfqpoint{3.550968in}{1.459145in}}%
\pgfpathlineto{\pgfqpoint{3.550968in}{1.450274in}}%
\pgfusepath{fill}%
\end{pgfscope}%
\begin{pgfscope}%
\pgfpathrectangle{\pgfqpoint{1.432000in}{0.528000in}}{\pgfqpoint{3.696000in}{3.696000in}} %
\pgfusepath{clip}%
\pgfsetbuttcap%
\pgfsetroundjoin%
\definecolor{currentfill}{rgb}{0.206756,0.371758,0.553117}%
\pgfsetfillcolor{currentfill}%
\pgfsetlinewidth{0.000000pt}%
\definecolor{currentstroke}{rgb}{0.000000,0.000000,0.000000}%
\pgfsetstrokecolor{currentstroke}%
\pgfsetdash{}{0pt}%
\pgfpathmoveto{\pgfqpoint{3.659355in}{1.450274in}}%
\pgfpathlineto{\pgfqpoint{3.482497in}{1.450274in}}%
\pgfpathlineto{\pgfqpoint{3.486933in}{1.441404in}}%
\pgfpathlineto{\pgfqpoint{3.442581in}{1.454710in}}%
\pgfpathlineto{\pgfqpoint{3.486933in}{1.468015in}}%
\pgfpathlineto{\pgfqpoint{3.482497in}{1.459145in}}%
\pgfpathlineto{\pgfqpoint{3.659355in}{1.459145in}}%
\pgfpathlineto{\pgfqpoint{3.659355in}{1.450274in}}%
\pgfusepath{fill}%
\end{pgfscope}%
\begin{pgfscope}%
\pgfpathrectangle{\pgfqpoint{1.432000in}{0.528000in}}{\pgfqpoint{3.696000in}{3.696000in}} %
\pgfusepath{clip}%
\pgfsetbuttcap%
\pgfsetroundjoin%
\definecolor{currentfill}{rgb}{0.277018,0.050344,0.375715}%
\pgfsetfillcolor{currentfill}%
\pgfsetlinewidth{0.000000pt}%
\definecolor{currentstroke}{rgb}{0.000000,0.000000,0.000000}%
\pgfsetstrokecolor{currentstroke}%
\pgfsetdash{}{0pt}%
\pgfpathmoveto{\pgfqpoint{3.767742in}{1.450274in}}%
\pgfpathlineto{\pgfqpoint{3.482497in}{1.450274in}}%
\pgfpathlineto{\pgfqpoint{3.486933in}{1.441404in}}%
\pgfpathlineto{\pgfqpoint{3.442581in}{1.454710in}}%
\pgfpathlineto{\pgfqpoint{3.486933in}{1.468015in}}%
\pgfpathlineto{\pgfqpoint{3.482497in}{1.459145in}}%
\pgfpathlineto{\pgfqpoint{3.767742in}{1.459145in}}%
\pgfpathlineto{\pgfqpoint{3.767742in}{1.450274in}}%
\pgfusepath{fill}%
\end{pgfscope}%
\begin{pgfscope}%
\pgfpathrectangle{\pgfqpoint{1.432000in}{0.528000in}}{\pgfqpoint{3.696000in}{3.696000in}} %
\pgfusepath{clip}%
\pgfsetbuttcap%
\pgfsetroundjoin%
\definecolor{currentfill}{rgb}{0.280267,0.073417,0.397163}%
\pgfsetfillcolor{currentfill}%
\pgfsetlinewidth{0.000000pt}%
\definecolor{currentstroke}{rgb}{0.000000,0.000000,0.000000}%
\pgfsetstrokecolor{currentstroke}%
\pgfsetdash{}{0pt}%
\pgfpathmoveto{\pgfqpoint{3.877532in}{1.450502in}}%
\pgfpathlineto{\pgfqpoint{3.590239in}{1.354738in}}%
\pgfpathlineto{\pgfqpoint{3.597251in}{1.347725in}}%
\pgfpathlineto{\pgfqpoint{3.550968in}{1.346323in}}%
\pgfpathlineto{\pgfqpoint{3.588836in}{1.372971in}}%
\pgfpathlineto{\pgfqpoint{3.587434in}{1.363153in}}%
\pgfpathlineto{\pgfqpoint{3.874726in}{1.458917in}}%
\pgfpathlineto{\pgfqpoint{3.877532in}{1.450502in}}%
\pgfusepath{fill}%
\end{pgfscope}%
\begin{pgfscope}%
\pgfpathrectangle{\pgfqpoint{1.432000in}{0.528000in}}{\pgfqpoint{3.696000in}{3.696000in}} %
\pgfusepath{clip}%
\pgfsetbuttcap%
\pgfsetroundjoin%
\definecolor{currentfill}{rgb}{0.262138,0.242286,0.520837}%
\pgfsetfillcolor{currentfill}%
\pgfsetlinewidth{0.000000pt}%
\definecolor{currentstroke}{rgb}{0.000000,0.000000,0.000000}%
\pgfsetstrokecolor{currentstroke}%
\pgfsetdash{}{0pt}%
\pgfpathmoveto{\pgfqpoint{3.985919in}{1.450502in}}%
\pgfpathlineto{\pgfqpoint{3.698626in}{1.354738in}}%
\pgfpathlineto{\pgfqpoint{3.705638in}{1.347725in}}%
\pgfpathlineto{\pgfqpoint{3.659355in}{1.346323in}}%
\pgfpathlineto{\pgfqpoint{3.697223in}{1.372971in}}%
\pgfpathlineto{\pgfqpoint{3.695821in}{1.363153in}}%
\pgfpathlineto{\pgfqpoint{3.983114in}{1.458917in}}%
\pgfpathlineto{\pgfqpoint{3.985919in}{1.450502in}}%
\pgfusepath{fill}%
\end{pgfscope}%
\begin{pgfscope}%
\pgfpathrectangle{\pgfqpoint{1.432000in}{0.528000in}}{\pgfqpoint{3.696000in}{3.696000in}} %
\pgfusepath{clip}%
\pgfsetbuttcap%
\pgfsetroundjoin%
\definecolor{currentfill}{rgb}{0.282884,0.135920,0.453427}%
\pgfsetfillcolor{currentfill}%
\pgfsetlinewidth{0.000000pt}%
\definecolor{currentstroke}{rgb}{0.000000,0.000000,0.000000}%
\pgfsetstrokecolor{currentstroke}%
\pgfsetdash{}{0pt}%
\pgfpathmoveto{\pgfqpoint{4.094306in}{1.450502in}}%
\pgfpathlineto{\pgfqpoint{3.807013in}{1.354738in}}%
\pgfpathlineto{\pgfqpoint{3.814026in}{1.347725in}}%
\pgfpathlineto{\pgfqpoint{3.767742in}{1.346323in}}%
\pgfpathlineto{\pgfqpoint{3.805610in}{1.372971in}}%
\pgfpathlineto{\pgfqpoint{3.804208in}{1.363153in}}%
\pgfpathlineto{\pgfqpoint{4.091501in}{1.458917in}}%
\pgfpathlineto{\pgfqpoint{4.094306in}{1.450502in}}%
\pgfusepath{fill}%
\end{pgfscope}%
\begin{pgfscope}%
\pgfpathrectangle{\pgfqpoint{1.432000in}{0.528000in}}{\pgfqpoint{3.696000in}{3.696000in}} %
\pgfusepath{clip}%
\pgfsetbuttcap%
\pgfsetroundjoin%
\definecolor{currentfill}{rgb}{0.278012,0.180367,0.486697}%
\pgfsetfillcolor{currentfill}%
\pgfsetlinewidth{0.000000pt}%
\definecolor{currentstroke}{rgb}{0.000000,0.000000,0.000000}%
\pgfsetstrokecolor{currentstroke}%
\pgfsetdash{}{0pt}%
\pgfpathmoveto{\pgfqpoint{4.202693in}{1.450502in}}%
\pgfpathlineto{\pgfqpoint{3.915400in}{1.354738in}}%
\pgfpathlineto{\pgfqpoint{3.922413in}{1.347725in}}%
\pgfpathlineto{\pgfqpoint{3.876129in}{1.346323in}}%
\pgfpathlineto{\pgfqpoint{3.913997in}{1.372971in}}%
\pgfpathlineto{\pgfqpoint{3.912595in}{1.363153in}}%
\pgfpathlineto{\pgfqpoint{4.199888in}{1.458917in}}%
\pgfpathlineto{\pgfqpoint{4.202693in}{1.450502in}}%
\pgfusepath{fill}%
\end{pgfscope}%
\begin{pgfscope}%
\pgfpathrectangle{\pgfqpoint{1.432000in}{0.528000in}}{\pgfqpoint{3.696000in}{3.696000in}} %
\pgfusepath{clip}%
\pgfsetbuttcap%
\pgfsetroundjoin%
\definecolor{currentfill}{rgb}{0.282884,0.135920,0.453427}%
\pgfsetfillcolor{currentfill}%
\pgfsetlinewidth{0.000000pt}%
\definecolor{currentstroke}{rgb}{0.000000,0.000000,0.000000}%
\pgfsetstrokecolor{currentstroke}%
\pgfsetdash{}{0pt}%
\pgfpathmoveto{\pgfqpoint{4.311080in}{1.450502in}}%
\pgfpathlineto{\pgfqpoint{4.023787in}{1.354738in}}%
\pgfpathlineto{\pgfqpoint{4.030800in}{1.347725in}}%
\pgfpathlineto{\pgfqpoint{3.984516in}{1.346323in}}%
\pgfpathlineto{\pgfqpoint{4.022385in}{1.372971in}}%
\pgfpathlineto{\pgfqpoint{4.020982in}{1.363153in}}%
\pgfpathlineto{\pgfqpoint{4.308275in}{1.458917in}}%
\pgfpathlineto{\pgfqpoint{4.311080in}{1.450502in}}%
\pgfusepath{fill}%
\end{pgfscope}%
\begin{pgfscope}%
\pgfpathrectangle{\pgfqpoint{1.432000in}{0.528000in}}{\pgfqpoint{3.696000in}{3.696000in}} %
\pgfusepath{clip}%
\pgfsetbuttcap%
\pgfsetroundjoin%
\definecolor{currentfill}{rgb}{0.277018,0.050344,0.375715}%
\pgfsetfillcolor{currentfill}%
\pgfsetlinewidth{0.000000pt}%
\definecolor{currentstroke}{rgb}{0.000000,0.000000,0.000000}%
\pgfsetstrokecolor{currentstroke}%
\pgfsetdash{}{0pt}%
\pgfpathmoveto{\pgfqpoint{4.419467in}{1.450502in}}%
\pgfpathlineto{\pgfqpoint{4.132174in}{1.354738in}}%
\pgfpathlineto{\pgfqpoint{4.139187in}{1.347725in}}%
\pgfpathlineto{\pgfqpoint{4.092903in}{1.346323in}}%
\pgfpathlineto{\pgfqpoint{4.130772in}{1.372971in}}%
\pgfpathlineto{\pgfqpoint{4.129369in}{1.363153in}}%
\pgfpathlineto{\pgfqpoint{4.416662in}{1.458917in}}%
\pgfpathlineto{\pgfqpoint{4.419467in}{1.450502in}}%
\pgfusepath{fill}%
\end{pgfscope}%
\begin{pgfscope}%
\pgfpathrectangle{\pgfqpoint{1.432000in}{0.528000in}}{\pgfqpoint{3.696000in}{3.696000in}} %
\pgfusepath{clip}%
\pgfsetbuttcap%
\pgfsetroundjoin%
\definecolor{currentfill}{rgb}{0.282656,0.100196,0.422160}%
\pgfsetfillcolor{currentfill}%
\pgfsetlinewidth{0.000000pt}%
\definecolor{currentstroke}{rgb}{0.000000,0.000000,0.000000}%
\pgfsetstrokecolor{currentstroke}%
\pgfsetdash{}{0pt}%
\pgfpathmoveto{\pgfqpoint{4.528435in}{1.450743in}}%
\pgfpathlineto{\pgfqpoint{4.347364in}{1.360207in}}%
\pgfpathlineto{\pgfqpoint{4.355297in}{1.354257in}}%
\pgfpathlineto{\pgfqpoint{4.309677in}{1.346323in}}%
\pgfpathlineto{\pgfqpoint{4.343397in}{1.378058in}}%
\pgfpathlineto{\pgfqpoint{4.343397in}{1.368141in}}%
\pgfpathlineto{\pgfqpoint{4.524468in}{1.458677in}}%
\pgfpathlineto{\pgfqpoint{4.528435in}{1.450743in}}%
\pgfusepath{fill}%
\end{pgfscope}%
\begin{pgfscope}%
\pgfpathrectangle{\pgfqpoint{1.432000in}{0.528000in}}{\pgfqpoint{3.696000in}{3.696000in}} %
\pgfusepath{clip}%
\pgfsetbuttcap%
\pgfsetroundjoin%
\definecolor{currentfill}{rgb}{0.281446,0.084320,0.407414}%
\pgfsetfillcolor{currentfill}%
\pgfsetlinewidth{0.000000pt}%
\definecolor{currentstroke}{rgb}{0.000000,0.000000,0.000000}%
\pgfsetstrokecolor{currentstroke}%
\pgfsetdash{}{0pt}%
\pgfpathmoveto{\pgfqpoint{4.526452in}{1.450274in}}%
\pgfpathlineto{\pgfqpoint{4.349594in}{1.450274in}}%
\pgfpathlineto{\pgfqpoint{4.354029in}{1.441404in}}%
\pgfpathlineto{\pgfqpoint{4.309677in}{1.454710in}}%
\pgfpathlineto{\pgfqpoint{4.354029in}{1.468015in}}%
\pgfpathlineto{\pgfqpoint{4.349594in}{1.459145in}}%
\pgfpathlineto{\pgfqpoint{4.526452in}{1.459145in}}%
\pgfpathlineto{\pgfqpoint{4.526452in}{1.450274in}}%
\pgfusepath{fill}%
\end{pgfscope}%
\begin{pgfscope}%
\pgfpathrectangle{\pgfqpoint{1.432000in}{0.528000in}}{\pgfqpoint{3.696000in}{3.696000in}} %
\pgfusepath{clip}%
\pgfsetbuttcap%
\pgfsetroundjoin%
\definecolor{currentfill}{rgb}{0.279574,0.170599,0.479997}%
\pgfsetfillcolor{currentfill}%
\pgfsetlinewidth{0.000000pt}%
\definecolor{currentstroke}{rgb}{0.000000,0.000000,0.000000}%
\pgfsetstrokecolor{currentstroke}%
\pgfsetdash{}{0pt}%
\pgfpathmoveto{\pgfqpoint{4.634839in}{1.450274in}}%
\pgfpathlineto{\pgfqpoint{4.457981in}{1.450274in}}%
\pgfpathlineto{\pgfqpoint{4.462417in}{1.441404in}}%
\pgfpathlineto{\pgfqpoint{4.418065in}{1.454710in}}%
\pgfpathlineto{\pgfqpoint{4.462417in}{1.468015in}}%
\pgfpathlineto{\pgfqpoint{4.457981in}{1.459145in}}%
\pgfpathlineto{\pgfqpoint{4.634839in}{1.459145in}}%
\pgfpathlineto{\pgfqpoint{4.634839in}{1.450274in}}%
\pgfusepath{fill}%
\end{pgfscope}%
\begin{pgfscope}%
\pgfpathrectangle{\pgfqpoint{1.432000in}{0.528000in}}{\pgfqpoint{3.696000in}{3.696000in}} %
\pgfusepath{clip}%
\pgfsetbuttcap%
\pgfsetroundjoin%
\definecolor{currentfill}{rgb}{0.281887,0.150881,0.465405}%
\pgfsetfillcolor{currentfill}%
\pgfsetlinewidth{0.000000pt}%
\definecolor{currentstroke}{rgb}{0.000000,0.000000,0.000000}%
\pgfsetstrokecolor{currentstroke}%
\pgfsetdash{}{0pt}%
\pgfpathmoveto{\pgfqpoint{4.634839in}{1.450274in}}%
\pgfpathlineto{\pgfqpoint{4.566368in}{1.450274in}}%
\pgfpathlineto{\pgfqpoint{4.570804in}{1.441404in}}%
\pgfpathlineto{\pgfqpoint{4.526452in}{1.454710in}}%
\pgfpathlineto{\pgfqpoint{4.570804in}{1.468015in}}%
\pgfpathlineto{\pgfqpoint{4.566368in}{1.459145in}}%
\pgfpathlineto{\pgfqpoint{4.634839in}{1.459145in}}%
\pgfpathlineto{\pgfqpoint{4.634839in}{1.450274in}}%
\pgfusepath{fill}%
\end{pgfscope}%
\begin{pgfscope}%
\pgfpathrectangle{\pgfqpoint{1.432000in}{0.528000in}}{\pgfqpoint{3.696000in}{3.696000in}} %
\pgfusepath{clip}%
\pgfsetbuttcap%
\pgfsetroundjoin%
\definecolor{currentfill}{rgb}{0.283229,0.120777,0.440584}%
\pgfsetfillcolor{currentfill}%
\pgfsetlinewidth{0.000000pt}%
\definecolor{currentstroke}{rgb}{0.000000,0.000000,0.000000}%
\pgfsetstrokecolor{currentstroke}%
\pgfsetdash{}{0pt}%
\pgfpathmoveto{\pgfqpoint{4.743226in}{1.450274in}}%
\pgfpathlineto{\pgfqpoint{4.566368in}{1.450274in}}%
\pgfpathlineto{\pgfqpoint{4.570804in}{1.441404in}}%
\pgfpathlineto{\pgfqpoint{4.526452in}{1.454710in}}%
\pgfpathlineto{\pgfqpoint{4.570804in}{1.468015in}}%
\pgfpathlineto{\pgfqpoint{4.566368in}{1.459145in}}%
\pgfpathlineto{\pgfqpoint{4.743226in}{1.459145in}}%
\pgfpathlineto{\pgfqpoint{4.743226in}{1.450274in}}%
\pgfusepath{fill}%
\end{pgfscope}%
\begin{pgfscope}%
\pgfpathrectangle{\pgfqpoint{1.432000in}{0.528000in}}{\pgfqpoint{3.696000in}{3.696000in}} %
\pgfusepath{clip}%
\pgfsetbuttcap%
\pgfsetroundjoin%
\definecolor{currentfill}{rgb}{0.204903,0.375746,0.553533}%
\pgfsetfillcolor{currentfill}%
\pgfsetlinewidth{0.000000pt}%
\definecolor{currentstroke}{rgb}{0.000000,0.000000,0.000000}%
\pgfsetstrokecolor{currentstroke}%
\pgfsetdash{}{0pt}%
\pgfpathmoveto{\pgfqpoint{4.743226in}{1.450274in}}%
\pgfpathlineto{\pgfqpoint{4.674756in}{1.450274in}}%
\pgfpathlineto{\pgfqpoint{4.679191in}{1.441404in}}%
\pgfpathlineto{\pgfqpoint{4.634839in}{1.454710in}}%
\pgfpathlineto{\pgfqpoint{4.679191in}{1.468015in}}%
\pgfpathlineto{\pgfqpoint{4.674756in}{1.459145in}}%
\pgfpathlineto{\pgfqpoint{4.743226in}{1.459145in}}%
\pgfpathlineto{\pgfqpoint{4.743226in}{1.450274in}}%
\pgfusepath{fill}%
\end{pgfscope}%
\begin{pgfscope}%
\pgfpathrectangle{\pgfqpoint{1.432000in}{0.528000in}}{\pgfqpoint{3.696000in}{3.696000in}} %
\pgfusepath{clip}%
\pgfsetbuttcap%
\pgfsetroundjoin%
\definecolor{currentfill}{rgb}{0.147607,0.511733,0.557049}%
\pgfsetfillcolor{currentfill}%
\pgfsetlinewidth{0.000000pt}%
\definecolor{currentstroke}{rgb}{0.000000,0.000000,0.000000}%
\pgfsetstrokecolor{currentstroke}%
\pgfsetdash{}{0pt}%
\pgfpathmoveto{\pgfqpoint{4.851613in}{1.450274in}}%
\pgfpathlineto{\pgfqpoint{4.783143in}{1.450274in}}%
\pgfpathlineto{\pgfqpoint{4.787578in}{1.441404in}}%
\pgfpathlineto{\pgfqpoint{4.743226in}{1.454710in}}%
\pgfpathlineto{\pgfqpoint{4.787578in}{1.468015in}}%
\pgfpathlineto{\pgfqpoint{4.783143in}{1.459145in}}%
\pgfpathlineto{\pgfqpoint{4.851613in}{1.459145in}}%
\pgfpathlineto{\pgfqpoint{4.851613in}{1.450274in}}%
\pgfusepath{fill}%
\end{pgfscope}%
\begin{pgfscope}%
\pgfpathrectangle{\pgfqpoint{1.432000in}{0.528000in}}{\pgfqpoint{3.696000in}{3.696000in}} %
\pgfusepath{clip}%
\pgfsetbuttcap%
\pgfsetroundjoin%
\definecolor{currentfill}{rgb}{0.277941,0.056324,0.381191}%
\pgfsetfillcolor{currentfill}%
\pgfsetlinewidth{0.000000pt}%
\definecolor{currentstroke}{rgb}{0.000000,0.000000,0.000000}%
\pgfsetstrokecolor{currentstroke}%
\pgfsetdash{}{0pt}%
\pgfpathmoveto{\pgfqpoint{4.856048in}{1.454710in}}%
\pgfpathlineto{\pgfqpoint{4.853831in}{1.458551in}}%
\pgfpathlineto{\pgfqpoint{4.849395in}{1.458551in}}%
\pgfpathlineto{\pgfqpoint{4.847178in}{1.454710in}}%
\pgfpathlineto{\pgfqpoint{4.849395in}{1.450869in}}%
\pgfpathlineto{\pgfqpoint{4.853831in}{1.450869in}}%
\pgfpathlineto{\pgfqpoint{4.856048in}{1.454710in}}%
\pgfpathlineto{\pgfqpoint{4.853831in}{1.458551in}}%
\pgfusepath{fill}%
\end{pgfscope}%
\begin{pgfscope}%
\pgfpathrectangle{\pgfqpoint{1.432000in}{0.528000in}}{\pgfqpoint{3.696000in}{3.696000in}} %
\pgfusepath{clip}%
\pgfsetbuttcap%
\pgfsetroundjoin%
\definecolor{currentfill}{rgb}{0.225863,0.330805,0.547314}%
\pgfsetfillcolor{currentfill}%
\pgfsetlinewidth{0.000000pt}%
\definecolor{currentstroke}{rgb}{0.000000,0.000000,0.000000}%
\pgfsetstrokecolor{currentstroke}%
\pgfsetdash{}{0pt}%
\pgfpathmoveto{\pgfqpoint{4.960000in}{1.450274in}}%
\pgfpathlineto{\pgfqpoint{4.891530in}{1.450274in}}%
\pgfpathlineto{\pgfqpoint{4.895965in}{1.441404in}}%
\pgfpathlineto{\pgfqpoint{4.851613in}{1.454710in}}%
\pgfpathlineto{\pgfqpoint{4.895965in}{1.468015in}}%
\pgfpathlineto{\pgfqpoint{4.891530in}{1.459145in}}%
\pgfpathlineto{\pgfqpoint{4.960000in}{1.459145in}}%
\pgfpathlineto{\pgfqpoint{4.960000in}{1.450274in}}%
\pgfusepath{fill}%
\end{pgfscope}%
\begin{pgfscope}%
\pgfpathrectangle{\pgfqpoint{1.432000in}{0.528000in}}{\pgfqpoint{3.696000in}{3.696000in}} %
\pgfusepath{clip}%
\pgfsetbuttcap%
\pgfsetroundjoin%
\definecolor{currentfill}{rgb}{0.169646,0.456262,0.558030}%
\pgfsetfillcolor{currentfill}%
\pgfsetlinewidth{0.000000pt}%
\definecolor{currentstroke}{rgb}{0.000000,0.000000,0.000000}%
\pgfsetstrokecolor{currentstroke}%
\pgfsetdash{}{0pt}%
\pgfpathmoveto{\pgfqpoint{4.964435in}{1.454710in}}%
\pgfpathlineto{\pgfqpoint{4.962218in}{1.458551in}}%
\pgfpathlineto{\pgfqpoint{4.957782in}{1.458551in}}%
\pgfpathlineto{\pgfqpoint{4.955565in}{1.454710in}}%
\pgfpathlineto{\pgfqpoint{4.957782in}{1.450869in}}%
\pgfpathlineto{\pgfqpoint{4.962218in}{1.450869in}}%
\pgfpathlineto{\pgfqpoint{4.964435in}{1.454710in}}%
\pgfpathlineto{\pgfqpoint{4.962218in}{1.458551in}}%
\pgfusepath{fill}%
\end{pgfscope}%
\begin{pgfscope}%
\pgfpathrectangle{\pgfqpoint{1.432000in}{0.528000in}}{\pgfqpoint{3.696000in}{3.696000in}} %
\pgfusepath{clip}%
\pgfsetbuttcap%
\pgfsetroundjoin%
\definecolor{currentfill}{rgb}{0.282290,0.145912,0.461510}%
\pgfsetfillcolor{currentfill}%
\pgfsetlinewidth{0.000000pt}%
\definecolor{currentstroke}{rgb}{0.000000,0.000000,0.000000}%
\pgfsetstrokecolor{currentstroke}%
\pgfsetdash{}{0pt}%
\pgfpathmoveto{\pgfqpoint{1.604435in}{1.563097in}}%
\pgfpathlineto{\pgfqpoint{1.604435in}{1.386239in}}%
\pgfpathlineto{\pgfqpoint{1.613306in}{1.390675in}}%
\pgfpathlineto{\pgfqpoint{1.600000in}{1.346323in}}%
\pgfpathlineto{\pgfqpoint{1.586694in}{1.390675in}}%
\pgfpathlineto{\pgfqpoint{1.595565in}{1.386239in}}%
\pgfpathlineto{\pgfqpoint{1.595565in}{1.563097in}}%
\pgfpathlineto{\pgfqpoint{1.604435in}{1.563097in}}%
\pgfusepath{fill}%
\end{pgfscope}%
\begin{pgfscope}%
\pgfpathrectangle{\pgfqpoint{1.432000in}{0.528000in}}{\pgfqpoint{3.696000in}{3.696000in}} %
\pgfusepath{clip}%
\pgfsetbuttcap%
\pgfsetroundjoin%
\definecolor{currentfill}{rgb}{0.267968,0.223549,0.512008}%
\pgfsetfillcolor{currentfill}%
\pgfsetlinewidth{0.000000pt}%
\definecolor{currentstroke}{rgb}{0.000000,0.000000,0.000000}%
\pgfsetstrokecolor{currentstroke}%
\pgfsetdash{}{0pt}%
\pgfpathmoveto{\pgfqpoint{1.604435in}{1.563097in}}%
\pgfpathlineto{\pgfqpoint{1.604435in}{1.494626in}}%
\pgfpathlineto{\pgfqpoint{1.613306in}{1.499062in}}%
\pgfpathlineto{\pgfqpoint{1.600000in}{1.454710in}}%
\pgfpathlineto{\pgfqpoint{1.586694in}{1.499062in}}%
\pgfpathlineto{\pgfqpoint{1.595565in}{1.494626in}}%
\pgfpathlineto{\pgfqpoint{1.595565in}{1.563097in}}%
\pgfpathlineto{\pgfqpoint{1.604435in}{1.563097in}}%
\pgfusepath{fill}%
\end{pgfscope}%
\begin{pgfscope}%
\pgfpathrectangle{\pgfqpoint{1.432000in}{0.528000in}}{\pgfqpoint{3.696000in}{3.696000in}} %
\pgfusepath{clip}%
\pgfsetbuttcap%
\pgfsetroundjoin%
\definecolor{currentfill}{rgb}{0.283197,0.115680,0.436115}%
\pgfsetfillcolor{currentfill}%
\pgfsetlinewidth{0.000000pt}%
\definecolor{currentstroke}{rgb}{0.000000,0.000000,0.000000}%
\pgfsetstrokecolor{currentstroke}%
\pgfsetdash{}{0pt}%
\pgfpathmoveto{\pgfqpoint{1.712354in}{1.561113in}}%
\pgfpathlineto{\pgfqpoint{1.621818in}{1.380042in}}%
\pgfpathlineto{\pgfqpoint{1.631736in}{1.380042in}}%
\pgfpathlineto{\pgfqpoint{1.600000in}{1.346323in}}%
\pgfpathlineto{\pgfqpoint{1.607934in}{1.391943in}}%
\pgfpathlineto{\pgfqpoint{1.613884in}{1.384009in}}%
\pgfpathlineto{\pgfqpoint{1.704420in}{1.565080in}}%
\pgfpathlineto{\pgfqpoint{1.712354in}{1.561113in}}%
\pgfusepath{fill}%
\end{pgfscope}%
\begin{pgfscope}%
\pgfpathrectangle{\pgfqpoint{1.432000in}{0.528000in}}{\pgfqpoint{3.696000in}{3.696000in}} %
\pgfusepath{clip}%
\pgfsetbuttcap%
\pgfsetroundjoin%
\definecolor{currentfill}{rgb}{0.241237,0.296485,0.539709}%
\pgfsetfillcolor{currentfill}%
\pgfsetlinewidth{0.000000pt}%
\definecolor{currentstroke}{rgb}{0.000000,0.000000,0.000000}%
\pgfsetstrokecolor{currentstroke}%
\pgfsetdash{}{0pt}%
\pgfpathmoveto{\pgfqpoint{1.711523in}{1.559961in}}%
\pgfpathlineto{\pgfqpoint{1.631362in}{1.479799in}}%
\pgfpathlineto{\pgfqpoint{1.640770in}{1.476663in}}%
\pgfpathlineto{\pgfqpoint{1.600000in}{1.454710in}}%
\pgfpathlineto{\pgfqpoint{1.621953in}{1.495480in}}%
\pgfpathlineto{\pgfqpoint{1.625089in}{1.486071in}}%
\pgfpathlineto{\pgfqpoint{1.705251in}{1.566233in}}%
\pgfpathlineto{\pgfqpoint{1.711523in}{1.559961in}}%
\pgfusepath{fill}%
\end{pgfscope}%
\begin{pgfscope}%
\pgfpathrectangle{\pgfqpoint{1.432000in}{0.528000in}}{\pgfqpoint{3.696000in}{3.696000in}} %
\pgfusepath{clip}%
\pgfsetbuttcap%
\pgfsetroundjoin%
\definecolor{currentfill}{rgb}{0.268510,0.009605,0.335427}%
\pgfsetfillcolor{currentfill}%
\pgfsetlinewidth{0.000000pt}%
\definecolor{currentstroke}{rgb}{0.000000,0.000000,0.000000}%
\pgfsetstrokecolor{currentstroke}%
\pgfsetdash{}{0pt}%
\pgfpathmoveto{\pgfqpoint{1.820741in}{1.561113in}}%
\pgfpathlineto{\pgfqpoint{1.730205in}{1.380042in}}%
\pgfpathlineto{\pgfqpoint{1.740123in}{1.380042in}}%
\pgfpathlineto{\pgfqpoint{1.708387in}{1.346323in}}%
\pgfpathlineto{\pgfqpoint{1.716321in}{1.391943in}}%
\pgfpathlineto{\pgfqpoint{1.722271in}{1.384009in}}%
\pgfpathlineto{\pgfqpoint{1.812807in}{1.565080in}}%
\pgfpathlineto{\pgfqpoint{1.820741in}{1.561113in}}%
\pgfusepath{fill}%
\end{pgfscope}%
\begin{pgfscope}%
\pgfpathrectangle{\pgfqpoint{1.432000in}{0.528000in}}{\pgfqpoint{3.696000in}{3.696000in}} %
\pgfusepath{clip}%
\pgfsetbuttcap%
\pgfsetroundjoin%
\definecolor{currentfill}{rgb}{0.140536,0.530132,0.555659}%
\pgfsetfillcolor{currentfill}%
\pgfsetlinewidth{0.000000pt}%
\definecolor{currentstroke}{rgb}{0.000000,0.000000,0.000000}%
\pgfsetstrokecolor{currentstroke}%
\pgfsetdash{}{0pt}%
\pgfpathmoveto{\pgfqpoint{1.819910in}{1.559961in}}%
\pgfpathlineto{\pgfqpoint{1.739749in}{1.479799in}}%
\pgfpathlineto{\pgfqpoint{1.749157in}{1.476663in}}%
\pgfpathlineto{\pgfqpoint{1.708387in}{1.454710in}}%
\pgfpathlineto{\pgfqpoint{1.730340in}{1.495480in}}%
\pgfpathlineto{\pgfqpoint{1.733476in}{1.486071in}}%
\pgfpathlineto{\pgfqpoint{1.813638in}{1.566233in}}%
\pgfpathlineto{\pgfqpoint{1.819910in}{1.559961in}}%
\pgfusepath{fill}%
\end{pgfscope}%
\begin{pgfscope}%
\pgfpathrectangle{\pgfqpoint{1.432000in}{0.528000in}}{\pgfqpoint{3.696000in}{3.696000in}} %
\pgfusepath{clip}%
\pgfsetbuttcap%
\pgfsetroundjoin%
\definecolor{currentfill}{rgb}{0.272594,0.025563,0.353093}%
\pgfsetfillcolor{currentfill}%
\pgfsetlinewidth{0.000000pt}%
\definecolor{currentstroke}{rgb}{0.000000,0.000000,0.000000}%
\pgfsetstrokecolor{currentstroke}%
\pgfsetdash{}{0pt}%
\pgfpathmoveto{\pgfqpoint{1.927145in}{1.559130in}}%
\pgfpathlineto{\pgfqpoint{1.746073in}{1.468594in}}%
\pgfpathlineto{\pgfqpoint{1.754007in}{1.462644in}}%
\pgfpathlineto{\pgfqpoint{1.708387in}{1.454710in}}%
\pgfpathlineto{\pgfqpoint{1.742106in}{1.486445in}}%
\pgfpathlineto{\pgfqpoint{1.742106in}{1.476528in}}%
\pgfpathlineto{\pgfqpoint{1.923178in}{1.567064in}}%
\pgfpathlineto{\pgfqpoint{1.927145in}{1.559130in}}%
\pgfusepath{fill}%
\end{pgfscope}%
\begin{pgfscope}%
\pgfpathrectangle{\pgfqpoint{1.432000in}{0.528000in}}{\pgfqpoint{3.696000in}{3.696000in}} %
\pgfusepath{clip}%
\pgfsetbuttcap%
\pgfsetroundjoin%
\definecolor{currentfill}{rgb}{0.237441,0.305202,0.541921}%
\pgfsetfillcolor{currentfill}%
\pgfsetlinewidth{0.000000pt}%
\definecolor{currentstroke}{rgb}{0.000000,0.000000,0.000000}%
\pgfsetstrokecolor{currentstroke}%
\pgfsetdash{}{0pt}%
\pgfpathmoveto{\pgfqpoint{1.928297in}{1.559961in}}%
\pgfpathlineto{\pgfqpoint{1.848136in}{1.479799in}}%
\pgfpathlineto{\pgfqpoint{1.857544in}{1.476663in}}%
\pgfpathlineto{\pgfqpoint{1.816774in}{1.454710in}}%
\pgfpathlineto{\pgfqpoint{1.838727in}{1.495480in}}%
\pgfpathlineto{\pgfqpoint{1.841863in}{1.486071in}}%
\pgfpathlineto{\pgfqpoint{1.922025in}{1.566233in}}%
\pgfpathlineto{\pgfqpoint{1.928297in}{1.559961in}}%
\pgfusepath{fill}%
\end{pgfscope}%
\begin{pgfscope}%
\pgfpathrectangle{\pgfqpoint{1.432000in}{0.528000in}}{\pgfqpoint{3.696000in}{3.696000in}} %
\pgfusepath{clip}%
\pgfsetbuttcap%
\pgfsetroundjoin%
\definecolor{currentfill}{rgb}{0.169646,0.456262,0.558030}%
\pgfsetfillcolor{currentfill}%
\pgfsetlinewidth{0.000000pt}%
\definecolor{currentstroke}{rgb}{0.000000,0.000000,0.000000}%
\pgfsetstrokecolor{currentstroke}%
\pgfsetdash{}{0pt}%
\pgfpathmoveto{\pgfqpoint{2.035532in}{1.559130in}}%
\pgfpathlineto{\pgfqpoint{1.854460in}{1.468594in}}%
\pgfpathlineto{\pgfqpoint{1.862394in}{1.462644in}}%
\pgfpathlineto{\pgfqpoint{1.816774in}{1.454710in}}%
\pgfpathlineto{\pgfqpoint{1.850493in}{1.486445in}}%
\pgfpathlineto{\pgfqpoint{1.850493in}{1.476528in}}%
\pgfpathlineto{\pgfqpoint{2.031565in}{1.567064in}}%
\pgfpathlineto{\pgfqpoint{2.035532in}{1.559130in}}%
\pgfusepath{fill}%
\end{pgfscope}%
\begin{pgfscope}%
\pgfpathrectangle{\pgfqpoint{1.432000in}{0.528000in}}{\pgfqpoint{3.696000in}{3.696000in}} %
\pgfusepath{clip}%
\pgfsetbuttcap%
\pgfsetroundjoin%
\definecolor{currentfill}{rgb}{0.156270,0.489624,0.557936}%
\pgfsetfillcolor{currentfill}%
\pgfsetlinewidth{0.000000pt}%
\definecolor{currentstroke}{rgb}{0.000000,0.000000,0.000000}%
\pgfsetstrokecolor{currentstroke}%
\pgfsetdash{}{0pt}%
\pgfpathmoveto{\pgfqpoint{2.143919in}{1.559130in}}%
\pgfpathlineto{\pgfqpoint{1.962847in}{1.468594in}}%
\pgfpathlineto{\pgfqpoint{1.970781in}{1.462644in}}%
\pgfpathlineto{\pgfqpoint{1.925161in}{1.454710in}}%
\pgfpathlineto{\pgfqpoint{1.958880in}{1.486445in}}%
\pgfpathlineto{\pgfqpoint{1.958880in}{1.476528in}}%
\pgfpathlineto{\pgfqpoint{2.139952in}{1.567064in}}%
\pgfpathlineto{\pgfqpoint{2.143919in}{1.559130in}}%
\pgfusepath{fill}%
\end{pgfscope}%
\begin{pgfscope}%
\pgfpathrectangle{\pgfqpoint{1.432000in}{0.528000in}}{\pgfqpoint{3.696000in}{3.696000in}} %
\pgfusepath{clip}%
\pgfsetbuttcap%
\pgfsetroundjoin%
\definecolor{currentfill}{rgb}{0.185556,0.418570,0.556753}%
\pgfsetfillcolor{currentfill}%
\pgfsetlinewidth{0.000000pt}%
\definecolor{currentstroke}{rgb}{0.000000,0.000000,0.000000}%
\pgfsetstrokecolor{currentstroke}%
\pgfsetdash{}{0pt}%
\pgfpathmoveto{\pgfqpoint{2.252306in}{1.559130in}}%
\pgfpathlineto{\pgfqpoint{2.071235in}{1.468594in}}%
\pgfpathlineto{\pgfqpoint{2.079168in}{1.462644in}}%
\pgfpathlineto{\pgfqpoint{2.033548in}{1.454710in}}%
\pgfpathlineto{\pgfqpoint{2.067268in}{1.486445in}}%
\pgfpathlineto{\pgfqpoint{2.067268in}{1.476528in}}%
\pgfpathlineto{\pgfqpoint{2.248339in}{1.567064in}}%
\pgfpathlineto{\pgfqpoint{2.252306in}{1.559130in}}%
\pgfusepath{fill}%
\end{pgfscope}%
\begin{pgfscope}%
\pgfpathrectangle{\pgfqpoint{1.432000in}{0.528000in}}{\pgfqpoint{3.696000in}{3.696000in}} %
\pgfusepath{clip}%
\pgfsetbuttcap%
\pgfsetroundjoin%
\definecolor{currentfill}{rgb}{0.235526,0.309527,0.542944}%
\pgfsetfillcolor{currentfill}%
\pgfsetlinewidth{0.000000pt}%
\definecolor{currentstroke}{rgb}{0.000000,0.000000,0.000000}%
\pgfsetstrokecolor{currentstroke}%
\pgfsetdash{}{0pt}%
\pgfpathmoveto{\pgfqpoint{2.358710in}{1.558662in}}%
\pgfpathlineto{\pgfqpoint{2.181852in}{1.558662in}}%
\pgfpathlineto{\pgfqpoint{2.186287in}{1.549791in}}%
\pgfpathlineto{\pgfqpoint{2.141935in}{1.563097in}}%
\pgfpathlineto{\pgfqpoint{2.186287in}{1.576402in}}%
\pgfpathlineto{\pgfqpoint{2.181852in}{1.567532in}}%
\pgfpathlineto{\pgfqpoint{2.358710in}{1.567532in}}%
\pgfpathlineto{\pgfqpoint{2.358710in}{1.558662in}}%
\pgfusepath{fill}%
\end{pgfscope}%
\begin{pgfscope}%
\pgfpathrectangle{\pgfqpoint{1.432000in}{0.528000in}}{\pgfqpoint{3.696000in}{3.696000in}} %
\pgfusepath{clip}%
\pgfsetbuttcap%
\pgfsetroundjoin%
\definecolor{currentfill}{rgb}{0.149039,0.508051,0.557250}%
\pgfsetfillcolor{currentfill}%
\pgfsetlinewidth{0.000000pt}%
\definecolor{currentstroke}{rgb}{0.000000,0.000000,0.000000}%
\pgfsetstrokecolor{currentstroke}%
\pgfsetdash{}{0pt}%
\pgfpathmoveto{\pgfqpoint{2.467097in}{1.558662in}}%
\pgfpathlineto{\pgfqpoint{2.290239in}{1.558662in}}%
\pgfpathlineto{\pgfqpoint{2.294675in}{1.549791in}}%
\pgfpathlineto{\pgfqpoint{2.250323in}{1.563097in}}%
\pgfpathlineto{\pgfqpoint{2.294675in}{1.576402in}}%
\pgfpathlineto{\pgfqpoint{2.290239in}{1.567532in}}%
\pgfpathlineto{\pgfqpoint{2.467097in}{1.567532in}}%
\pgfpathlineto{\pgfqpoint{2.467097in}{1.558662in}}%
\pgfusepath{fill}%
\end{pgfscope}%
\begin{pgfscope}%
\pgfpathrectangle{\pgfqpoint{1.432000in}{0.528000in}}{\pgfqpoint{3.696000in}{3.696000in}} %
\pgfusepath{clip}%
\pgfsetbuttcap%
\pgfsetroundjoin%
\definecolor{currentfill}{rgb}{0.127568,0.566949,0.550556}%
\pgfsetfillcolor{currentfill}%
\pgfsetlinewidth{0.000000pt}%
\definecolor{currentstroke}{rgb}{0.000000,0.000000,0.000000}%
\pgfsetstrokecolor{currentstroke}%
\pgfsetdash{}{0pt}%
\pgfpathmoveto{\pgfqpoint{2.575484in}{1.558662in}}%
\pgfpathlineto{\pgfqpoint{2.398626in}{1.558662in}}%
\pgfpathlineto{\pgfqpoint{2.403062in}{1.549791in}}%
\pgfpathlineto{\pgfqpoint{2.358710in}{1.563097in}}%
\pgfpathlineto{\pgfqpoint{2.403062in}{1.576402in}}%
\pgfpathlineto{\pgfqpoint{2.398626in}{1.567532in}}%
\pgfpathlineto{\pgfqpoint{2.575484in}{1.567532in}}%
\pgfpathlineto{\pgfqpoint{2.575484in}{1.558662in}}%
\pgfusepath{fill}%
\end{pgfscope}%
\begin{pgfscope}%
\pgfpathrectangle{\pgfqpoint{1.432000in}{0.528000in}}{\pgfqpoint{3.696000in}{3.696000in}} %
\pgfusepath{clip}%
\pgfsetbuttcap%
\pgfsetroundjoin%
\definecolor{currentfill}{rgb}{0.280894,0.078907,0.402329}%
\pgfsetfillcolor{currentfill}%
\pgfsetlinewidth{0.000000pt}%
\definecolor{currentstroke}{rgb}{0.000000,0.000000,0.000000}%
\pgfsetstrokecolor{currentstroke}%
\pgfsetdash{}{0pt}%
\pgfpathmoveto{\pgfqpoint{2.683871in}{1.558662in}}%
\pgfpathlineto{\pgfqpoint{2.398626in}{1.558662in}}%
\pgfpathlineto{\pgfqpoint{2.403062in}{1.549791in}}%
\pgfpathlineto{\pgfqpoint{2.358710in}{1.563097in}}%
\pgfpathlineto{\pgfqpoint{2.403062in}{1.576402in}}%
\pgfpathlineto{\pgfqpoint{2.398626in}{1.567532in}}%
\pgfpathlineto{\pgfqpoint{2.683871in}{1.567532in}}%
\pgfpathlineto{\pgfqpoint{2.683871in}{1.558662in}}%
\pgfusepath{fill}%
\end{pgfscope}%
\begin{pgfscope}%
\pgfpathrectangle{\pgfqpoint{1.432000in}{0.528000in}}{\pgfqpoint{3.696000in}{3.696000in}} %
\pgfusepath{clip}%
\pgfsetbuttcap%
\pgfsetroundjoin%
\definecolor{currentfill}{rgb}{0.277134,0.185228,0.489898}%
\pgfsetfillcolor{currentfill}%
\pgfsetlinewidth{0.000000pt}%
\definecolor{currentstroke}{rgb}{0.000000,0.000000,0.000000}%
\pgfsetstrokecolor{currentstroke}%
\pgfsetdash{}{0pt}%
\pgfpathmoveto{\pgfqpoint{2.683871in}{1.558662in}}%
\pgfpathlineto{\pgfqpoint{2.507014in}{1.558662in}}%
\pgfpathlineto{\pgfqpoint{2.511449in}{1.549791in}}%
\pgfpathlineto{\pgfqpoint{2.467097in}{1.563097in}}%
\pgfpathlineto{\pgfqpoint{2.511449in}{1.576402in}}%
\pgfpathlineto{\pgfqpoint{2.507014in}{1.567532in}}%
\pgfpathlineto{\pgfqpoint{2.683871in}{1.567532in}}%
\pgfpathlineto{\pgfqpoint{2.683871in}{1.558662in}}%
\pgfusepath{fill}%
\end{pgfscope}%
\begin{pgfscope}%
\pgfpathrectangle{\pgfqpoint{1.432000in}{0.528000in}}{\pgfqpoint{3.696000in}{3.696000in}} %
\pgfusepath{clip}%
\pgfsetbuttcap%
\pgfsetroundjoin%
\definecolor{currentfill}{rgb}{0.156270,0.489624,0.557936}%
\pgfsetfillcolor{currentfill}%
\pgfsetlinewidth{0.000000pt}%
\definecolor{currentstroke}{rgb}{0.000000,0.000000,0.000000}%
\pgfsetstrokecolor{currentstroke}%
\pgfsetdash{}{0pt}%
\pgfpathmoveto{\pgfqpoint{2.792258in}{1.558662in}}%
\pgfpathlineto{\pgfqpoint{2.507014in}{1.558662in}}%
\pgfpathlineto{\pgfqpoint{2.511449in}{1.549791in}}%
\pgfpathlineto{\pgfqpoint{2.467097in}{1.563097in}}%
\pgfpathlineto{\pgfqpoint{2.511449in}{1.576402in}}%
\pgfpathlineto{\pgfqpoint{2.507014in}{1.567532in}}%
\pgfpathlineto{\pgfqpoint{2.792258in}{1.567532in}}%
\pgfpathlineto{\pgfqpoint{2.792258in}{1.558662in}}%
\pgfusepath{fill}%
\end{pgfscope}%
\begin{pgfscope}%
\pgfpathrectangle{\pgfqpoint{1.432000in}{0.528000in}}{\pgfqpoint{3.696000in}{3.696000in}} %
\pgfusepath{clip}%
\pgfsetbuttcap%
\pgfsetroundjoin%
\definecolor{currentfill}{rgb}{0.121831,0.589055,0.545623}%
\pgfsetfillcolor{currentfill}%
\pgfsetlinewidth{0.000000pt}%
\definecolor{currentstroke}{rgb}{0.000000,0.000000,0.000000}%
\pgfsetstrokecolor{currentstroke}%
\pgfsetdash{}{0pt}%
\pgfpathmoveto{\pgfqpoint{2.900645in}{1.558662in}}%
\pgfpathlineto{\pgfqpoint{2.615401in}{1.558662in}}%
\pgfpathlineto{\pgfqpoint{2.619836in}{1.549791in}}%
\pgfpathlineto{\pgfqpoint{2.575484in}{1.563097in}}%
\pgfpathlineto{\pgfqpoint{2.619836in}{1.576402in}}%
\pgfpathlineto{\pgfqpoint{2.615401in}{1.567532in}}%
\pgfpathlineto{\pgfqpoint{2.900645in}{1.567532in}}%
\pgfpathlineto{\pgfqpoint{2.900645in}{1.558662in}}%
\pgfusepath{fill}%
\end{pgfscope}%
\begin{pgfscope}%
\pgfpathrectangle{\pgfqpoint{1.432000in}{0.528000in}}{\pgfqpoint{3.696000in}{3.696000in}} %
\pgfusepath{clip}%
\pgfsetbuttcap%
\pgfsetroundjoin%
\definecolor{currentfill}{rgb}{0.273809,0.031497,0.358853}%
\pgfsetfillcolor{currentfill}%
\pgfsetlinewidth{0.000000pt}%
\definecolor{currentstroke}{rgb}{0.000000,0.000000,0.000000}%
\pgfsetstrokecolor{currentstroke}%
\pgfsetdash{}{0pt}%
\pgfpathmoveto{\pgfqpoint{2.900645in}{1.558662in}}%
\pgfpathlineto{\pgfqpoint{2.723788in}{1.558662in}}%
\pgfpathlineto{\pgfqpoint{2.728223in}{1.549791in}}%
\pgfpathlineto{\pgfqpoint{2.683871in}{1.563097in}}%
\pgfpathlineto{\pgfqpoint{2.728223in}{1.576402in}}%
\pgfpathlineto{\pgfqpoint{2.723788in}{1.567532in}}%
\pgfpathlineto{\pgfqpoint{2.900645in}{1.567532in}}%
\pgfpathlineto{\pgfqpoint{2.900645in}{1.558662in}}%
\pgfusepath{fill}%
\end{pgfscope}%
\begin{pgfscope}%
\pgfpathrectangle{\pgfqpoint{1.432000in}{0.528000in}}{\pgfqpoint{3.696000in}{3.696000in}} %
\pgfusepath{clip}%
\pgfsetbuttcap%
\pgfsetroundjoin%
\definecolor{currentfill}{rgb}{0.154815,0.493313,0.557840}%
\pgfsetfillcolor{currentfill}%
\pgfsetlinewidth{0.000000pt}%
\definecolor{currentstroke}{rgb}{0.000000,0.000000,0.000000}%
\pgfsetstrokecolor{currentstroke}%
\pgfsetdash{}{0pt}%
\pgfpathmoveto{\pgfqpoint{3.009032in}{1.558662in}}%
\pgfpathlineto{\pgfqpoint{2.723788in}{1.558662in}}%
\pgfpathlineto{\pgfqpoint{2.728223in}{1.549791in}}%
\pgfpathlineto{\pgfqpoint{2.683871in}{1.563097in}}%
\pgfpathlineto{\pgfqpoint{2.728223in}{1.576402in}}%
\pgfpathlineto{\pgfqpoint{2.723788in}{1.567532in}}%
\pgfpathlineto{\pgfqpoint{3.009032in}{1.567532in}}%
\pgfpathlineto{\pgfqpoint{3.009032in}{1.558662in}}%
\pgfusepath{fill}%
\end{pgfscope}%
\begin{pgfscope}%
\pgfpathrectangle{\pgfqpoint{1.432000in}{0.528000in}}{\pgfqpoint{3.696000in}{3.696000in}} %
\pgfusepath{clip}%
\pgfsetbuttcap%
\pgfsetroundjoin%
\definecolor{currentfill}{rgb}{0.260571,0.246922,0.522828}%
\pgfsetfillcolor{currentfill}%
\pgfsetlinewidth{0.000000pt}%
\definecolor{currentstroke}{rgb}{0.000000,0.000000,0.000000}%
\pgfsetstrokecolor{currentstroke}%
\pgfsetdash{}{0pt}%
\pgfpathmoveto{\pgfqpoint{3.009032in}{1.558662in}}%
\pgfpathlineto{\pgfqpoint{2.832175in}{1.558662in}}%
\pgfpathlineto{\pgfqpoint{2.836610in}{1.549791in}}%
\pgfpathlineto{\pgfqpoint{2.792258in}{1.563097in}}%
\pgfpathlineto{\pgfqpoint{2.836610in}{1.576402in}}%
\pgfpathlineto{\pgfqpoint{2.832175in}{1.567532in}}%
\pgfpathlineto{\pgfqpoint{3.009032in}{1.567532in}}%
\pgfpathlineto{\pgfqpoint{3.009032in}{1.558662in}}%
\pgfusepath{fill}%
\end{pgfscope}%
\begin{pgfscope}%
\pgfpathrectangle{\pgfqpoint{1.432000in}{0.528000in}}{\pgfqpoint{3.696000in}{3.696000in}} %
\pgfusepath{clip}%
\pgfsetbuttcap%
\pgfsetroundjoin%
\definecolor{currentfill}{rgb}{0.237441,0.305202,0.541921}%
\pgfsetfillcolor{currentfill}%
\pgfsetlinewidth{0.000000pt}%
\definecolor{currentstroke}{rgb}{0.000000,0.000000,0.000000}%
\pgfsetstrokecolor{currentstroke}%
\pgfsetdash{}{0pt}%
\pgfpathmoveto{\pgfqpoint{3.117419in}{1.558662in}}%
\pgfpathlineto{\pgfqpoint{2.832175in}{1.558662in}}%
\pgfpathlineto{\pgfqpoint{2.836610in}{1.549791in}}%
\pgfpathlineto{\pgfqpoint{2.792258in}{1.563097in}}%
\pgfpathlineto{\pgfqpoint{2.836610in}{1.576402in}}%
\pgfpathlineto{\pgfqpoint{2.832175in}{1.567532in}}%
\pgfpathlineto{\pgfqpoint{3.117419in}{1.567532in}}%
\pgfpathlineto{\pgfqpoint{3.117419in}{1.558662in}}%
\pgfusepath{fill}%
\end{pgfscope}%
\begin{pgfscope}%
\pgfpathrectangle{\pgfqpoint{1.432000in}{0.528000in}}{\pgfqpoint{3.696000in}{3.696000in}} %
\pgfusepath{clip}%
\pgfsetbuttcap%
\pgfsetroundjoin%
\definecolor{currentfill}{rgb}{0.175841,0.441290,0.557685}%
\pgfsetfillcolor{currentfill}%
\pgfsetlinewidth{0.000000pt}%
\definecolor{currentstroke}{rgb}{0.000000,0.000000,0.000000}%
\pgfsetstrokecolor{currentstroke}%
\pgfsetdash{}{0pt}%
\pgfpathmoveto{\pgfqpoint{3.117419in}{1.558662in}}%
\pgfpathlineto{\pgfqpoint{2.940562in}{1.558662in}}%
\pgfpathlineto{\pgfqpoint{2.944997in}{1.549791in}}%
\pgfpathlineto{\pgfqpoint{2.900645in}{1.563097in}}%
\pgfpathlineto{\pgfqpoint{2.944997in}{1.576402in}}%
\pgfpathlineto{\pgfqpoint{2.940562in}{1.567532in}}%
\pgfpathlineto{\pgfqpoint{3.117419in}{1.567532in}}%
\pgfpathlineto{\pgfqpoint{3.117419in}{1.558662in}}%
\pgfusepath{fill}%
\end{pgfscope}%
\begin{pgfscope}%
\pgfpathrectangle{\pgfqpoint{1.432000in}{0.528000in}}{\pgfqpoint{3.696000in}{3.696000in}} %
\pgfusepath{clip}%
\pgfsetbuttcap%
\pgfsetroundjoin%
\definecolor{currentfill}{rgb}{0.277134,0.185228,0.489898}%
\pgfsetfillcolor{currentfill}%
\pgfsetlinewidth{0.000000pt}%
\definecolor{currentstroke}{rgb}{0.000000,0.000000,0.000000}%
\pgfsetstrokecolor{currentstroke}%
\pgfsetdash{}{0pt}%
\pgfpathmoveto{\pgfqpoint{3.225806in}{1.558662in}}%
\pgfpathlineto{\pgfqpoint{2.940562in}{1.558662in}}%
\pgfpathlineto{\pgfqpoint{2.944997in}{1.549791in}}%
\pgfpathlineto{\pgfqpoint{2.900645in}{1.563097in}}%
\pgfpathlineto{\pgfqpoint{2.944997in}{1.576402in}}%
\pgfpathlineto{\pgfqpoint{2.940562in}{1.567532in}}%
\pgfpathlineto{\pgfqpoint{3.225806in}{1.567532in}}%
\pgfpathlineto{\pgfqpoint{3.225806in}{1.558662in}}%
\pgfusepath{fill}%
\end{pgfscope}%
\begin{pgfscope}%
\pgfpathrectangle{\pgfqpoint{1.432000in}{0.528000in}}{\pgfqpoint{3.696000in}{3.696000in}} %
\pgfusepath{clip}%
\pgfsetbuttcap%
\pgfsetroundjoin%
\definecolor{currentfill}{rgb}{0.119738,0.603785,0.541400}%
\pgfsetfillcolor{currentfill}%
\pgfsetlinewidth{0.000000pt}%
\definecolor{currentstroke}{rgb}{0.000000,0.000000,0.000000}%
\pgfsetstrokecolor{currentstroke}%
\pgfsetdash{}{0pt}%
\pgfpathmoveto{\pgfqpoint{3.225806in}{1.558662in}}%
\pgfpathlineto{\pgfqpoint{3.048949in}{1.558662in}}%
\pgfpathlineto{\pgfqpoint{3.053384in}{1.549791in}}%
\pgfpathlineto{\pgfqpoint{3.009032in}{1.563097in}}%
\pgfpathlineto{\pgfqpoint{3.053384in}{1.576402in}}%
\pgfpathlineto{\pgfqpoint{3.048949in}{1.567532in}}%
\pgfpathlineto{\pgfqpoint{3.225806in}{1.567532in}}%
\pgfpathlineto{\pgfqpoint{3.225806in}{1.558662in}}%
\pgfusepath{fill}%
\end{pgfscope}%
\begin{pgfscope}%
\pgfpathrectangle{\pgfqpoint{1.432000in}{0.528000in}}{\pgfqpoint{3.696000in}{3.696000in}} %
\pgfusepath{clip}%
\pgfsetbuttcap%
\pgfsetroundjoin%
\definecolor{currentfill}{rgb}{0.149039,0.508051,0.557250}%
\pgfsetfillcolor{currentfill}%
\pgfsetlinewidth{0.000000pt}%
\definecolor{currentstroke}{rgb}{0.000000,0.000000,0.000000}%
\pgfsetstrokecolor{currentstroke}%
\pgfsetdash{}{0pt}%
\pgfpathmoveto{\pgfqpoint{3.334194in}{1.558662in}}%
\pgfpathlineto{\pgfqpoint{3.157336in}{1.558662in}}%
\pgfpathlineto{\pgfqpoint{3.161771in}{1.549791in}}%
\pgfpathlineto{\pgfqpoint{3.117419in}{1.563097in}}%
\pgfpathlineto{\pgfqpoint{3.161771in}{1.576402in}}%
\pgfpathlineto{\pgfqpoint{3.157336in}{1.567532in}}%
\pgfpathlineto{\pgfqpoint{3.334194in}{1.567532in}}%
\pgfpathlineto{\pgfqpoint{3.334194in}{1.558662in}}%
\pgfusepath{fill}%
\end{pgfscope}%
\begin{pgfscope}%
\pgfpathrectangle{\pgfqpoint{1.432000in}{0.528000in}}{\pgfqpoint{3.696000in}{3.696000in}} %
\pgfusepath{clip}%
\pgfsetbuttcap%
\pgfsetroundjoin%
\definecolor{currentfill}{rgb}{0.147607,0.511733,0.557049}%
\pgfsetfillcolor{currentfill}%
\pgfsetlinewidth{0.000000pt}%
\definecolor{currentstroke}{rgb}{0.000000,0.000000,0.000000}%
\pgfsetstrokecolor{currentstroke}%
\pgfsetdash{}{0pt}%
\pgfpathmoveto{\pgfqpoint{3.442581in}{1.558662in}}%
\pgfpathlineto{\pgfqpoint{3.265723in}{1.558662in}}%
\pgfpathlineto{\pgfqpoint{3.270158in}{1.549791in}}%
\pgfpathlineto{\pgfqpoint{3.225806in}{1.563097in}}%
\pgfpathlineto{\pgfqpoint{3.270158in}{1.576402in}}%
\pgfpathlineto{\pgfqpoint{3.265723in}{1.567532in}}%
\pgfpathlineto{\pgfqpoint{3.442581in}{1.567532in}}%
\pgfpathlineto{\pgfqpoint{3.442581in}{1.558662in}}%
\pgfusepath{fill}%
\end{pgfscope}%
\begin{pgfscope}%
\pgfpathrectangle{\pgfqpoint{1.432000in}{0.528000in}}{\pgfqpoint{3.696000in}{3.696000in}} %
\pgfusepath{clip}%
\pgfsetbuttcap%
\pgfsetroundjoin%
\definecolor{currentfill}{rgb}{0.187231,0.414746,0.556547}%
\pgfsetfillcolor{currentfill}%
\pgfsetlinewidth{0.000000pt}%
\definecolor{currentstroke}{rgb}{0.000000,0.000000,0.000000}%
\pgfsetstrokecolor{currentstroke}%
\pgfsetdash{}{0pt}%
\pgfpathmoveto{\pgfqpoint{3.550968in}{1.558662in}}%
\pgfpathlineto{\pgfqpoint{3.374110in}{1.558662in}}%
\pgfpathlineto{\pgfqpoint{3.378546in}{1.549791in}}%
\pgfpathlineto{\pgfqpoint{3.334194in}{1.563097in}}%
\pgfpathlineto{\pgfqpoint{3.378546in}{1.576402in}}%
\pgfpathlineto{\pgfqpoint{3.374110in}{1.567532in}}%
\pgfpathlineto{\pgfqpoint{3.550968in}{1.567532in}}%
\pgfpathlineto{\pgfqpoint{3.550968in}{1.558662in}}%
\pgfusepath{fill}%
\end{pgfscope}%
\begin{pgfscope}%
\pgfpathrectangle{\pgfqpoint{1.432000in}{0.528000in}}{\pgfqpoint{3.696000in}{3.696000in}} %
\pgfusepath{clip}%
\pgfsetbuttcap%
\pgfsetroundjoin%
\definecolor{currentfill}{rgb}{0.280267,0.073417,0.397163}%
\pgfsetfillcolor{currentfill}%
\pgfsetlinewidth{0.000000pt}%
\definecolor{currentstroke}{rgb}{0.000000,0.000000,0.000000}%
\pgfsetstrokecolor{currentstroke}%
\pgfsetdash{}{0pt}%
\pgfpathmoveto{\pgfqpoint{3.659355in}{1.558662in}}%
\pgfpathlineto{\pgfqpoint{3.374110in}{1.558662in}}%
\pgfpathlineto{\pgfqpoint{3.378546in}{1.549791in}}%
\pgfpathlineto{\pgfqpoint{3.334194in}{1.563097in}}%
\pgfpathlineto{\pgfqpoint{3.378546in}{1.576402in}}%
\pgfpathlineto{\pgfqpoint{3.374110in}{1.567532in}}%
\pgfpathlineto{\pgfqpoint{3.659355in}{1.567532in}}%
\pgfpathlineto{\pgfqpoint{3.659355in}{1.558662in}}%
\pgfusepath{fill}%
\end{pgfscope}%
\begin{pgfscope}%
\pgfpathrectangle{\pgfqpoint{1.432000in}{0.528000in}}{\pgfqpoint{3.696000in}{3.696000in}} %
\pgfusepath{clip}%
\pgfsetbuttcap%
\pgfsetroundjoin%
\definecolor{currentfill}{rgb}{0.277018,0.050344,0.375715}%
\pgfsetfillcolor{currentfill}%
\pgfsetlinewidth{0.000000pt}%
\definecolor{currentstroke}{rgb}{0.000000,0.000000,0.000000}%
\pgfsetstrokecolor{currentstroke}%
\pgfsetdash{}{0pt}%
\pgfpathmoveto{\pgfqpoint{3.767742in}{1.558662in}}%
\pgfpathlineto{\pgfqpoint{3.482497in}{1.558662in}}%
\pgfpathlineto{\pgfqpoint{3.486933in}{1.549791in}}%
\pgfpathlineto{\pgfqpoint{3.442581in}{1.563097in}}%
\pgfpathlineto{\pgfqpoint{3.486933in}{1.576402in}}%
\pgfpathlineto{\pgfqpoint{3.482497in}{1.567532in}}%
\pgfpathlineto{\pgfqpoint{3.767742in}{1.567532in}}%
\pgfpathlineto{\pgfqpoint{3.767742in}{1.558662in}}%
\pgfusepath{fill}%
\end{pgfscope}%
\begin{pgfscope}%
\pgfpathrectangle{\pgfqpoint{1.432000in}{0.528000in}}{\pgfqpoint{3.696000in}{3.696000in}} %
\pgfusepath{clip}%
\pgfsetbuttcap%
\pgfsetroundjoin%
\definecolor{currentfill}{rgb}{0.210503,0.363727,0.552206}%
\pgfsetfillcolor{currentfill}%
\pgfsetlinewidth{0.000000pt}%
\definecolor{currentstroke}{rgb}{0.000000,0.000000,0.000000}%
\pgfsetstrokecolor{currentstroke}%
\pgfsetdash{}{0pt}%
\pgfpathmoveto{\pgfqpoint{3.877532in}{1.558889in}}%
\pgfpathlineto{\pgfqpoint{3.590239in}{1.463125in}}%
\pgfpathlineto{\pgfqpoint{3.597251in}{1.456112in}}%
\pgfpathlineto{\pgfqpoint{3.550968in}{1.454710in}}%
\pgfpathlineto{\pgfqpoint{3.588836in}{1.481358in}}%
\pgfpathlineto{\pgfqpoint{3.587434in}{1.471540in}}%
\pgfpathlineto{\pgfqpoint{3.874726in}{1.567304in}}%
\pgfpathlineto{\pgfqpoint{3.877532in}{1.558889in}}%
\pgfusepath{fill}%
\end{pgfscope}%
\begin{pgfscope}%
\pgfpathrectangle{\pgfqpoint{1.432000in}{0.528000in}}{\pgfqpoint{3.696000in}{3.696000in}} %
\pgfusepath{clip}%
\pgfsetbuttcap%
\pgfsetroundjoin%
\definecolor{currentfill}{rgb}{0.274952,0.037752,0.364543}%
\pgfsetfillcolor{currentfill}%
\pgfsetlinewidth{0.000000pt}%
\definecolor{currentstroke}{rgb}{0.000000,0.000000,0.000000}%
\pgfsetstrokecolor{currentstroke}%
\pgfsetdash{}{0pt}%
\pgfpathmoveto{\pgfqpoint{3.985592in}{1.558794in}}%
\pgfpathlineto{\pgfqpoint{3.590768in}{1.460088in}}%
\pgfpathlineto{\pgfqpoint{3.597223in}{1.452558in}}%
\pgfpathlineto{\pgfqpoint{3.550968in}{1.454710in}}%
\pgfpathlineto{\pgfqpoint{3.590768in}{1.478375in}}%
\pgfpathlineto{\pgfqpoint{3.588617in}{1.468694in}}%
\pgfpathlineto{\pgfqpoint{3.983440in}{1.567400in}}%
\pgfpathlineto{\pgfqpoint{3.985592in}{1.558794in}}%
\pgfusepath{fill}%
\end{pgfscope}%
\begin{pgfscope}%
\pgfpathrectangle{\pgfqpoint{1.432000in}{0.528000in}}{\pgfqpoint{3.696000in}{3.696000in}} %
\pgfusepath{clip}%
\pgfsetbuttcap%
\pgfsetroundjoin%
\definecolor{currentfill}{rgb}{0.283187,0.125848,0.444960}%
\pgfsetfillcolor{currentfill}%
\pgfsetlinewidth{0.000000pt}%
\definecolor{currentstroke}{rgb}{0.000000,0.000000,0.000000}%
\pgfsetstrokecolor{currentstroke}%
\pgfsetdash{}{0pt}%
\pgfpathmoveto{\pgfqpoint{3.985919in}{1.558889in}}%
\pgfpathlineto{\pgfqpoint{3.698626in}{1.463125in}}%
\pgfpathlineto{\pgfqpoint{3.705638in}{1.456112in}}%
\pgfpathlineto{\pgfqpoint{3.659355in}{1.454710in}}%
\pgfpathlineto{\pgfqpoint{3.697223in}{1.481358in}}%
\pgfpathlineto{\pgfqpoint{3.695821in}{1.471540in}}%
\pgfpathlineto{\pgfqpoint{3.983114in}{1.567304in}}%
\pgfpathlineto{\pgfqpoint{3.985919in}{1.558889in}}%
\pgfusepath{fill}%
\end{pgfscope}%
\begin{pgfscope}%
\pgfpathrectangle{\pgfqpoint{1.432000in}{0.528000in}}{\pgfqpoint{3.696000in}{3.696000in}} %
\pgfusepath{clip}%
\pgfsetbuttcap%
\pgfsetroundjoin%
\definecolor{currentfill}{rgb}{0.282884,0.135920,0.453427}%
\pgfsetfillcolor{currentfill}%
\pgfsetlinewidth{0.000000pt}%
\definecolor{currentstroke}{rgb}{0.000000,0.000000,0.000000}%
\pgfsetstrokecolor{currentstroke}%
\pgfsetdash{}{0pt}%
\pgfpathmoveto{\pgfqpoint{4.093979in}{1.558794in}}%
\pgfpathlineto{\pgfqpoint{3.699156in}{1.460088in}}%
\pgfpathlineto{\pgfqpoint{3.705610in}{1.452558in}}%
\pgfpathlineto{\pgfqpoint{3.659355in}{1.454710in}}%
\pgfpathlineto{\pgfqpoint{3.699156in}{1.478375in}}%
\pgfpathlineto{\pgfqpoint{3.697004in}{1.468694in}}%
\pgfpathlineto{\pgfqpoint{4.091828in}{1.567400in}}%
\pgfpathlineto{\pgfqpoint{4.093979in}{1.558794in}}%
\pgfusepath{fill}%
\end{pgfscope}%
\begin{pgfscope}%
\pgfpathrectangle{\pgfqpoint{1.432000in}{0.528000in}}{\pgfqpoint{3.696000in}{3.696000in}} %
\pgfusepath{clip}%
\pgfsetbuttcap%
\pgfsetroundjoin%
\definecolor{currentfill}{rgb}{0.279566,0.067836,0.391917}%
\pgfsetfillcolor{currentfill}%
\pgfsetlinewidth{0.000000pt}%
\definecolor{currentstroke}{rgb}{0.000000,0.000000,0.000000}%
\pgfsetstrokecolor{currentstroke}%
\pgfsetdash{}{0pt}%
\pgfpathmoveto{\pgfqpoint{4.202366in}{1.558794in}}%
\pgfpathlineto{\pgfqpoint{3.807543in}{1.460088in}}%
\pgfpathlineto{\pgfqpoint{3.813997in}{1.452558in}}%
\pgfpathlineto{\pgfqpoint{3.767742in}{1.454710in}}%
\pgfpathlineto{\pgfqpoint{3.807543in}{1.478375in}}%
\pgfpathlineto{\pgfqpoint{3.805391in}{1.468694in}}%
\pgfpathlineto{\pgfqpoint{4.200215in}{1.567400in}}%
\pgfpathlineto{\pgfqpoint{4.202366in}{1.558794in}}%
\pgfusepath{fill}%
\end{pgfscope}%
\begin{pgfscope}%
\pgfpathrectangle{\pgfqpoint{1.432000in}{0.528000in}}{\pgfqpoint{3.696000in}{3.696000in}} %
\pgfusepath{clip}%
\pgfsetbuttcap%
\pgfsetroundjoin%
\definecolor{currentfill}{rgb}{0.274128,0.199721,0.498911}%
\pgfsetfillcolor{currentfill}%
\pgfsetlinewidth{0.000000pt}%
\definecolor{currentstroke}{rgb}{0.000000,0.000000,0.000000}%
\pgfsetstrokecolor{currentstroke}%
\pgfsetdash{}{0pt}%
\pgfpathmoveto{\pgfqpoint{4.202693in}{1.558889in}}%
\pgfpathlineto{\pgfqpoint{3.915400in}{1.463125in}}%
\pgfpathlineto{\pgfqpoint{3.922413in}{1.456112in}}%
\pgfpathlineto{\pgfqpoint{3.876129in}{1.454710in}}%
\pgfpathlineto{\pgfqpoint{3.913997in}{1.481358in}}%
\pgfpathlineto{\pgfqpoint{3.912595in}{1.471540in}}%
\pgfpathlineto{\pgfqpoint{4.199888in}{1.567304in}}%
\pgfpathlineto{\pgfqpoint{4.202693in}{1.558889in}}%
\pgfusepath{fill}%
\end{pgfscope}%
\begin{pgfscope}%
\pgfpathrectangle{\pgfqpoint{1.432000in}{0.528000in}}{\pgfqpoint{3.696000in}{3.696000in}} %
\pgfusepath{clip}%
\pgfsetbuttcap%
\pgfsetroundjoin%
\definecolor{currentfill}{rgb}{0.199430,0.387607,0.554642}%
\pgfsetfillcolor{currentfill}%
\pgfsetlinewidth{0.000000pt}%
\definecolor{currentstroke}{rgb}{0.000000,0.000000,0.000000}%
\pgfsetstrokecolor{currentstroke}%
\pgfsetdash{}{0pt}%
\pgfpathmoveto{\pgfqpoint{4.311080in}{1.558889in}}%
\pgfpathlineto{\pgfqpoint{4.023787in}{1.463125in}}%
\pgfpathlineto{\pgfqpoint{4.030800in}{1.456112in}}%
\pgfpathlineto{\pgfqpoint{3.984516in}{1.454710in}}%
\pgfpathlineto{\pgfqpoint{4.022385in}{1.481358in}}%
\pgfpathlineto{\pgfqpoint{4.020982in}{1.471540in}}%
\pgfpathlineto{\pgfqpoint{4.308275in}{1.567304in}}%
\pgfpathlineto{\pgfqpoint{4.311080in}{1.558889in}}%
\pgfusepath{fill}%
\end{pgfscope}%
\begin{pgfscope}%
\pgfpathrectangle{\pgfqpoint{1.432000in}{0.528000in}}{\pgfqpoint{3.696000in}{3.696000in}} %
\pgfusepath{clip}%
\pgfsetbuttcap%
\pgfsetroundjoin%
\definecolor{currentfill}{rgb}{0.265145,0.232956,0.516599}%
\pgfsetfillcolor{currentfill}%
\pgfsetlinewidth{0.000000pt}%
\definecolor{currentstroke}{rgb}{0.000000,0.000000,0.000000}%
\pgfsetstrokecolor{currentstroke}%
\pgfsetdash{}{0pt}%
\pgfpathmoveto{\pgfqpoint{4.419467in}{1.558889in}}%
\pgfpathlineto{\pgfqpoint{4.132174in}{1.463125in}}%
\pgfpathlineto{\pgfqpoint{4.139187in}{1.456112in}}%
\pgfpathlineto{\pgfqpoint{4.092903in}{1.454710in}}%
\pgfpathlineto{\pgfqpoint{4.130772in}{1.481358in}}%
\pgfpathlineto{\pgfqpoint{4.129369in}{1.471540in}}%
\pgfpathlineto{\pgfqpoint{4.416662in}{1.567304in}}%
\pgfpathlineto{\pgfqpoint{4.419467in}{1.558889in}}%
\pgfusepath{fill}%
\end{pgfscope}%
\begin{pgfscope}%
\pgfpathrectangle{\pgfqpoint{1.432000in}{0.528000in}}{\pgfqpoint{3.696000in}{3.696000in}} %
\pgfusepath{clip}%
\pgfsetbuttcap%
\pgfsetroundjoin%
\definecolor{currentfill}{rgb}{0.280894,0.078907,0.402329}%
\pgfsetfillcolor{currentfill}%
\pgfsetlinewidth{0.000000pt}%
\definecolor{currentstroke}{rgb}{0.000000,0.000000,0.000000}%
\pgfsetstrokecolor{currentstroke}%
\pgfsetdash{}{0pt}%
\pgfpathmoveto{\pgfqpoint{4.528435in}{1.559130in}}%
\pgfpathlineto{\pgfqpoint{4.347364in}{1.468594in}}%
\pgfpathlineto{\pgfqpoint{4.355297in}{1.462644in}}%
\pgfpathlineto{\pgfqpoint{4.309677in}{1.454710in}}%
\pgfpathlineto{\pgfqpoint{4.343397in}{1.486445in}}%
\pgfpathlineto{\pgfqpoint{4.343397in}{1.476528in}}%
\pgfpathlineto{\pgfqpoint{4.524468in}{1.567064in}}%
\pgfpathlineto{\pgfqpoint{4.528435in}{1.559130in}}%
\pgfusepath{fill}%
\end{pgfscope}%
\begin{pgfscope}%
\pgfpathrectangle{\pgfqpoint{1.432000in}{0.528000in}}{\pgfqpoint{3.696000in}{3.696000in}} %
\pgfusepath{clip}%
\pgfsetbuttcap%
\pgfsetroundjoin%
\definecolor{currentfill}{rgb}{0.272594,0.025563,0.353093}%
\pgfsetfillcolor{currentfill}%
\pgfsetlinewidth{0.000000pt}%
\definecolor{currentstroke}{rgb}{0.000000,0.000000,0.000000}%
\pgfsetstrokecolor{currentstroke}%
\pgfsetdash{}{0pt}%
\pgfpathmoveto{\pgfqpoint{4.634839in}{1.558662in}}%
\pgfpathlineto{\pgfqpoint{4.457981in}{1.558662in}}%
\pgfpathlineto{\pgfqpoint{4.462417in}{1.549791in}}%
\pgfpathlineto{\pgfqpoint{4.418065in}{1.563097in}}%
\pgfpathlineto{\pgfqpoint{4.462417in}{1.576402in}}%
\pgfpathlineto{\pgfqpoint{4.457981in}{1.567532in}}%
\pgfpathlineto{\pgfqpoint{4.634839in}{1.567532in}}%
\pgfpathlineto{\pgfqpoint{4.634839in}{1.558662in}}%
\pgfusepath{fill}%
\end{pgfscope}%
\begin{pgfscope}%
\pgfpathrectangle{\pgfqpoint{1.432000in}{0.528000in}}{\pgfqpoint{3.696000in}{3.696000in}} %
\pgfusepath{clip}%
\pgfsetbuttcap%
\pgfsetroundjoin%
\definecolor{currentfill}{rgb}{0.283072,0.130895,0.449241}%
\pgfsetfillcolor{currentfill}%
\pgfsetlinewidth{0.000000pt}%
\definecolor{currentstroke}{rgb}{0.000000,0.000000,0.000000}%
\pgfsetstrokecolor{currentstroke}%
\pgfsetdash{}{0pt}%
\pgfpathmoveto{\pgfqpoint{4.746362in}{1.559961in}}%
\pgfpathlineto{\pgfqpoint{4.666200in}{1.479799in}}%
\pgfpathlineto{\pgfqpoint{4.675609in}{1.476663in}}%
\pgfpathlineto{\pgfqpoint{4.634839in}{1.454710in}}%
\pgfpathlineto{\pgfqpoint{4.656792in}{1.495480in}}%
\pgfpathlineto{\pgfqpoint{4.659928in}{1.486071in}}%
\pgfpathlineto{\pgfqpoint{4.740090in}{1.566233in}}%
\pgfpathlineto{\pgfqpoint{4.746362in}{1.559961in}}%
\pgfusepath{fill}%
\end{pgfscope}%
\begin{pgfscope}%
\pgfpathrectangle{\pgfqpoint{1.432000in}{0.528000in}}{\pgfqpoint{3.696000in}{3.696000in}} %
\pgfusepath{clip}%
\pgfsetbuttcap%
\pgfsetroundjoin%
\definecolor{currentfill}{rgb}{0.255645,0.260703,0.528312}%
\pgfsetfillcolor{currentfill}%
\pgfsetlinewidth{0.000000pt}%
\definecolor{currentstroke}{rgb}{0.000000,0.000000,0.000000}%
\pgfsetstrokecolor{currentstroke}%
\pgfsetdash{}{0pt}%
\pgfpathmoveto{\pgfqpoint{4.743226in}{1.558662in}}%
\pgfpathlineto{\pgfqpoint{4.674756in}{1.558662in}}%
\pgfpathlineto{\pgfqpoint{4.679191in}{1.549791in}}%
\pgfpathlineto{\pgfqpoint{4.634839in}{1.563097in}}%
\pgfpathlineto{\pgfqpoint{4.679191in}{1.576402in}}%
\pgfpathlineto{\pgfqpoint{4.674756in}{1.567532in}}%
\pgfpathlineto{\pgfqpoint{4.743226in}{1.567532in}}%
\pgfpathlineto{\pgfqpoint{4.743226in}{1.558662in}}%
\pgfusepath{fill}%
\end{pgfscope}%
\begin{pgfscope}%
\pgfpathrectangle{\pgfqpoint{1.432000in}{0.528000in}}{\pgfqpoint{3.696000in}{3.696000in}} %
\pgfusepath{clip}%
\pgfsetbuttcap%
\pgfsetroundjoin%
\definecolor{currentfill}{rgb}{0.143343,0.522773,0.556295}%
\pgfsetfillcolor{currentfill}%
\pgfsetlinewidth{0.000000pt}%
\definecolor{currentstroke}{rgb}{0.000000,0.000000,0.000000}%
\pgfsetstrokecolor{currentstroke}%
\pgfsetdash{}{0pt}%
\pgfpathmoveto{\pgfqpoint{4.851613in}{1.558662in}}%
\pgfpathlineto{\pgfqpoint{4.783143in}{1.558662in}}%
\pgfpathlineto{\pgfqpoint{4.787578in}{1.549791in}}%
\pgfpathlineto{\pgfqpoint{4.743226in}{1.563097in}}%
\pgfpathlineto{\pgfqpoint{4.787578in}{1.576402in}}%
\pgfpathlineto{\pgfqpoint{4.783143in}{1.567532in}}%
\pgfpathlineto{\pgfqpoint{4.851613in}{1.567532in}}%
\pgfpathlineto{\pgfqpoint{4.851613in}{1.558662in}}%
\pgfusepath{fill}%
\end{pgfscope}%
\begin{pgfscope}%
\pgfpathrectangle{\pgfqpoint{1.432000in}{0.528000in}}{\pgfqpoint{3.696000in}{3.696000in}} %
\pgfusepath{clip}%
\pgfsetbuttcap%
\pgfsetroundjoin%
\definecolor{currentfill}{rgb}{0.266580,0.228262,0.514349}%
\pgfsetfillcolor{currentfill}%
\pgfsetlinewidth{0.000000pt}%
\definecolor{currentstroke}{rgb}{0.000000,0.000000,0.000000}%
\pgfsetstrokecolor{currentstroke}%
\pgfsetdash{}{0pt}%
\pgfpathmoveto{\pgfqpoint{4.856048in}{1.563097in}}%
\pgfpathlineto{\pgfqpoint{4.853831in}{1.566938in}}%
\pgfpathlineto{\pgfqpoint{4.849395in}{1.566938in}}%
\pgfpathlineto{\pgfqpoint{4.847178in}{1.563097in}}%
\pgfpathlineto{\pgfqpoint{4.849395in}{1.559256in}}%
\pgfpathlineto{\pgfqpoint{4.853831in}{1.559256in}}%
\pgfpathlineto{\pgfqpoint{4.856048in}{1.563097in}}%
\pgfpathlineto{\pgfqpoint{4.853831in}{1.566938in}}%
\pgfusepath{fill}%
\end{pgfscope}%
\begin{pgfscope}%
\pgfpathrectangle{\pgfqpoint{1.432000in}{0.528000in}}{\pgfqpoint{3.696000in}{3.696000in}} %
\pgfusepath{clip}%
\pgfsetbuttcap%
\pgfsetroundjoin%
\definecolor{currentfill}{rgb}{0.250425,0.274290,0.533103}%
\pgfsetfillcolor{currentfill}%
\pgfsetlinewidth{0.000000pt}%
\definecolor{currentstroke}{rgb}{0.000000,0.000000,0.000000}%
\pgfsetstrokecolor{currentstroke}%
\pgfsetdash{}{0pt}%
\pgfpathmoveto{\pgfqpoint{4.960000in}{1.558662in}}%
\pgfpathlineto{\pgfqpoint{4.891530in}{1.558662in}}%
\pgfpathlineto{\pgfqpoint{4.895965in}{1.549791in}}%
\pgfpathlineto{\pgfqpoint{4.851613in}{1.563097in}}%
\pgfpathlineto{\pgfqpoint{4.895965in}{1.576402in}}%
\pgfpathlineto{\pgfqpoint{4.891530in}{1.567532in}}%
\pgfpathlineto{\pgfqpoint{4.960000in}{1.567532in}}%
\pgfpathlineto{\pgfqpoint{4.960000in}{1.558662in}}%
\pgfusepath{fill}%
\end{pgfscope}%
\begin{pgfscope}%
\pgfpathrectangle{\pgfqpoint{1.432000in}{0.528000in}}{\pgfqpoint{3.696000in}{3.696000in}} %
\pgfusepath{clip}%
\pgfsetbuttcap%
\pgfsetroundjoin%
\definecolor{currentfill}{rgb}{0.146616,0.673050,0.508936}%
\pgfsetfillcolor{currentfill}%
\pgfsetlinewidth{0.000000pt}%
\definecolor{currentstroke}{rgb}{0.000000,0.000000,0.000000}%
\pgfsetstrokecolor{currentstroke}%
\pgfsetdash{}{0pt}%
\pgfpathmoveto{\pgfqpoint{4.964435in}{1.563097in}}%
\pgfpathlineto{\pgfqpoint{4.962218in}{1.566938in}}%
\pgfpathlineto{\pgfqpoint{4.957782in}{1.566938in}}%
\pgfpathlineto{\pgfqpoint{4.955565in}{1.563097in}}%
\pgfpathlineto{\pgfqpoint{4.957782in}{1.559256in}}%
\pgfpathlineto{\pgfqpoint{4.962218in}{1.559256in}}%
\pgfpathlineto{\pgfqpoint{4.964435in}{1.563097in}}%
\pgfpathlineto{\pgfqpoint{4.962218in}{1.566938in}}%
\pgfusepath{fill}%
\end{pgfscope}%
\begin{pgfscope}%
\pgfpathrectangle{\pgfqpoint{1.432000in}{0.528000in}}{\pgfqpoint{3.696000in}{3.696000in}} %
\pgfusepath{clip}%
\pgfsetbuttcap%
\pgfsetroundjoin%
\definecolor{currentfill}{rgb}{0.280267,0.073417,0.397163}%
\pgfsetfillcolor{currentfill}%
\pgfsetlinewidth{0.000000pt}%
\definecolor{currentstroke}{rgb}{0.000000,0.000000,0.000000}%
\pgfsetstrokecolor{currentstroke}%
\pgfsetdash{}{0pt}%
\pgfpathmoveto{\pgfqpoint{1.604435in}{1.671484in}}%
\pgfpathlineto{\pgfqpoint{1.604435in}{1.494626in}}%
\pgfpathlineto{\pgfqpoint{1.613306in}{1.499062in}}%
\pgfpathlineto{\pgfqpoint{1.600000in}{1.454710in}}%
\pgfpathlineto{\pgfqpoint{1.586694in}{1.499062in}}%
\pgfpathlineto{\pgfqpoint{1.595565in}{1.494626in}}%
\pgfpathlineto{\pgfqpoint{1.595565in}{1.671484in}}%
\pgfpathlineto{\pgfqpoint{1.604435in}{1.671484in}}%
\pgfusepath{fill}%
\end{pgfscope}%
\begin{pgfscope}%
\pgfpathrectangle{\pgfqpoint{1.432000in}{0.528000in}}{\pgfqpoint{3.696000in}{3.696000in}} %
\pgfusepath{clip}%
\pgfsetbuttcap%
\pgfsetroundjoin%
\definecolor{currentfill}{rgb}{0.257322,0.256130,0.526563}%
\pgfsetfillcolor{currentfill}%
\pgfsetlinewidth{0.000000pt}%
\definecolor{currentstroke}{rgb}{0.000000,0.000000,0.000000}%
\pgfsetstrokecolor{currentstroke}%
\pgfsetdash{}{0pt}%
\pgfpathmoveto{\pgfqpoint{1.604435in}{1.671484in}}%
\pgfpathlineto{\pgfqpoint{1.604435in}{1.603014in}}%
\pgfpathlineto{\pgfqpoint{1.613306in}{1.607449in}}%
\pgfpathlineto{\pgfqpoint{1.600000in}{1.563097in}}%
\pgfpathlineto{\pgfqpoint{1.586694in}{1.607449in}}%
\pgfpathlineto{\pgfqpoint{1.595565in}{1.603014in}}%
\pgfpathlineto{\pgfqpoint{1.595565in}{1.671484in}}%
\pgfpathlineto{\pgfqpoint{1.604435in}{1.671484in}}%
\pgfusepath{fill}%
\end{pgfscope}%
\begin{pgfscope}%
\pgfpathrectangle{\pgfqpoint{1.432000in}{0.528000in}}{\pgfqpoint{3.696000in}{3.696000in}} %
\pgfusepath{clip}%
\pgfsetbuttcap%
\pgfsetroundjoin%
\definecolor{currentfill}{rgb}{0.282884,0.135920,0.453427}%
\pgfsetfillcolor{currentfill}%
\pgfsetlinewidth{0.000000pt}%
\definecolor{currentstroke}{rgb}{0.000000,0.000000,0.000000}%
\pgfsetstrokecolor{currentstroke}%
\pgfsetdash{}{0pt}%
\pgfpathmoveto{\pgfqpoint{1.712354in}{1.669500in}}%
\pgfpathlineto{\pgfqpoint{1.621818in}{1.488429in}}%
\pgfpathlineto{\pgfqpoint{1.631736in}{1.488429in}}%
\pgfpathlineto{\pgfqpoint{1.600000in}{1.454710in}}%
\pgfpathlineto{\pgfqpoint{1.607934in}{1.500330in}}%
\pgfpathlineto{\pgfqpoint{1.613884in}{1.492396in}}%
\pgfpathlineto{\pgfqpoint{1.704420in}{1.673467in}}%
\pgfpathlineto{\pgfqpoint{1.712354in}{1.669500in}}%
\pgfusepath{fill}%
\end{pgfscope}%
\begin{pgfscope}%
\pgfpathrectangle{\pgfqpoint{1.432000in}{0.528000in}}{\pgfqpoint{3.696000in}{3.696000in}} %
\pgfusepath{clip}%
\pgfsetbuttcap%
\pgfsetroundjoin%
\definecolor{currentfill}{rgb}{0.233603,0.313828,0.543914}%
\pgfsetfillcolor{currentfill}%
\pgfsetlinewidth{0.000000pt}%
\definecolor{currentstroke}{rgb}{0.000000,0.000000,0.000000}%
\pgfsetstrokecolor{currentstroke}%
\pgfsetdash{}{0pt}%
\pgfpathmoveto{\pgfqpoint{1.711523in}{1.668348in}}%
\pgfpathlineto{\pgfqpoint{1.631362in}{1.588186in}}%
\pgfpathlineto{\pgfqpoint{1.640770in}{1.585050in}}%
\pgfpathlineto{\pgfqpoint{1.600000in}{1.563097in}}%
\pgfpathlineto{\pgfqpoint{1.621953in}{1.603867in}}%
\pgfpathlineto{\pgfqpoint{1.625089in}{1.594458in}}%
\pgfpathlineto{\pgfqpoint{1.705251in}{1.674620in}}%
\pgfpathlineto{\pgfqpoint{1.711523in}{1.668348in}}%
\pgfusepath{fill}%
\end{pgfscope}%
\begin{pgfscope}%
\pgfpathrectangle{\pgfqpoint{1.432000in}{0.528000in}}{\pgfqpoint{3.696000in}{3.696000in}} %
\pgfusepath{clip}%
\pgfsetbuttcap%
\pgfsetroundjoin%
\definecolor{currentfill}{rgb}{0.267004,0.004874,0.329415}%
\pgfsetfillcolor{currentfill}%
\pgfsetlinewidth{0.000000pt}%
\definecolor{currentstroke}{rgb}{0.000000,0.000000,0.000000}%
\pgfsetstrokecolor{currentstroke}%
\pgfsetdash{}{0pt}%
\pgfpathmoveto{\pgfqpoint{1.820741in}{1.669500in}}%
\pgfpathlineto{\pgfqpoint{1.730205in}{1.488429in}}%
\pgfpathlineto{\pgfqpoint{1.740123in}{1.488429in}}%
\pgfpathlineto{\pgfqpoint{1.708387in}{1.454710in}}%
\pgfpathlineto{\pgfqpoint{1.716321in}{1.500330in}}%
\pgfpathlineto{\pgfqpoint{1.722271in}{1.492396in}}%
\pgfpathlineto{\pgfqpoint{1.812807in}{1.673467in}}%
\pgfpathlineto{\pgfqpoint{1.820741in}{1.669500in}}%
\pgfusepath{fill}%
\end{pgfscope}%
\begin{pgfscope}%
\pgfpathrectangle{\pgfqpoint{1.432000in}{0.528000in}}{\pgfqpoint{3.696000in}{3.696000in}} %
\pgfusepath{clip}%
\pgfsetbuttcap%
\pgfsetroundjoin%
\definecolor{currentfill}{rgb}{0.124780,0.640461,0.527068}%
\pgfsetfillcolor{currentfill}%
\pgfsetlinewidth{0.000000pt}%
\definecolor{currentstroke}{rgb}{0.000000,0.000000,0.000000}%
\pgfsetstrokecolor{currentstroke}%
\pgfsetdash{}{0pt}%
\pgfpathmoveto{\pgfqpoint{1.819910in}{1.668348in}}%
\pgfpathlineto{\pgfqpoint{1.739749in}{1.588186in}}%
\pgfpathlineto{\pgfqpoint{1.749157in}{1.585050in}}%
\pgfpathlineto{\pgfqpoint{1.708387in}{1.563097in}}%
\pgfpathlineto{\pgfqpoint{1.730340in}{1.603867in}}%
\pgfpathlineto{\pgfqpoint{1.733476in}{1.594458in}}%
\pgfpathlineto{\pgfqpoint{1.813638in}{1.674620in}}%
\pgfpathlineto{\pgfqpoint{1.819910in}{1.668348in}}%
\pgfusepath{fill}%
\end{pgfscope}%
\begin{pgfscope}%
\pgfpathrectangle{\pgfqpoint{1.432000in}{0.528000in}}{\pgfqpoint{3.696000in}{3.696000in}} %
\pgfusepath{clip}%
\pgfsetbuttcap%
\pgfsetroundjoin%
\definecolor{currentfill}{rgb}{0.124395,0.578002,0.548287}%
\pgfsetfillcolor{currentfill}%
\pgfsetlinewidth{0.000000pt}%
\definecolor{currentstroke}{rgb}{0.000000,0.000000,0.000000}%
\pgfsetstrokecolor{currentstroke}%
\pgfsetdash{}{0pt}%
\pgfpathmoveto{\pgfqpoint{1.928297in}{1.668348in}}%
\pgfpathlineto{\pgfqpoint{1.848136in}{1.588186in}}%
\pgfpathlineto{\pgfqpoint{1.857544in}{1.585050in}}%
\pgfpathlineto{\pgfqpoint{1.816774in}{1.563097in}}%
\pgfpathlineto{\pgfqpoint{1.838727in}{1.603867in}}%
\pgfpathlineto{\pgfqpoint{1.841863in}{1.594458in}}%
\pgfpathlineto{\pgfqpoint{1.922025in}{1.674620in}}%
\pgfpathlineto{\pgfqpoint{1.928297in}{1.668348in}}%
\pgfusepath{fill}%
\end{pgfscope}%
\begin{pgfscope}%
\pgfpathrectangle{\pgfqpoint{1.432000in}{0.528000in}}{\pgfqpoint{3.696000in}{3.696000in}} %
\pgfusepath{clip}%
\pgfsetbuttcap%
\pgfsetroundjoin%
\definecolor{currentfill}{rgb}{0.280267,0.073417,0.397163}%
\pgfsetfillcolor{currentfill}%
\pgfsetlinewidth{0.000000pt}%
\definecolor{currentstroke}{rgb}{0.000000,0.000000,0.000000}%
\pgfsetstrokecolor{currentstroke}%
\pgfsetdash{}{0pt}%
\pgfpathmoveto{\pgfqpoint{2.035532in}{1.667517in}}%
\pgfpathlineto{\pgfqpoint{1.854460in}{1.576981in}}%
\pgfpathlineto{\pgfqpoint{1.862394in}{1.571031in}}%
\pgfpathlineto{\pgfqpoint{1.816774in}{1.563097in}}%
\pgfpathlineto{\pgfqpoint{1.850493in}{1.594832in}}%
\pgfpathlineto{\pgfqpoint{1.850493in}{1.584915in}}%
\pgfpathlineto{\pgfqpoint{2.031565in}{1.675451in}}%
\pgfpathlineto{\pgfqpoint{2.035532in}{1.667517in}}%
\pgfusepath{fill}%
\end{pgfscope}%
\begin{pgfscope}%
\pgfpathrectangle{\pgfqpoint{1.432000in}{0.528000in}}{\pgfqpoint{3.696000in}{3.696000in}} %
\pgfusepath{clip}%
\pgfsetbuttcap%
\pgfsetroundjoin%
\definecolor{currentfill}{rgb}{0.157729,0.485932,0.558013}%
\pgfsetfillcolor{currentfill}%
\pgfsetlinewidth{0.000000pt}%
\definecolor{currentstroke}{rgb}{0.000000,0.000000,0.000000}%
\pgfsetstrokecolor{currentstroke}%
\pgfsetdash{}{0pt}%
\pgfpathmoveto{\pgfqpoint{2.036685in}{1.668348in}}%
\pgfpathlineto{\pgfqpoint{1.956523in}{1.588186in}}%
\pgfpathlineto{\pgfqpoint{1.965931in}{1.585050in}}%
\pgfpathlineto{\pgfqpoint{1.925161in}{1.563097in}}%
\pgfpathlineto{\pgfqpoint{1.947114in}{1.603867in}}%
\pgfpathlineto{\pgfqpoint{1.950251in}{1.594458in}}%
\pgfpathlineto{\pgfqpoint{2.030412in}{1.674620in}}%
\pgfpathlineto{\pgfqpoint{2.036685in}{1.668348in}}%
\pgfusepath{fill}%
\end{pgfscope}%
\begin{pgfscope}%
\pgfpathrectangle{\pgfqpoint{1.432000in}{0.528000in}}{\pgfqpoint{3.696000in}{3.696000in}} %
\pgfusepath{clip}%
\pgfsetbuttcap%
\pgfsetroundjoin%
\definecolor{currentfill}{rgb}{0.220057,0.343307,0.549413}%
\pgfsetfillcolor{currentfill}%
\pgfsetlinewidth{0.000000pt}%
\definecolor{currentstroke}{rgb}{0.000000,0.000000,0.000000}%
\pgfsetstrokecolor{currentstroke}%
\pgfsetdash{}{0pt}%
\pgfpathmoveto{\pgfqpoint{2.145072in}{1.668348in}}%
\pgfpathlineto{\pgfqpoint{2.064910in}{1.588186in}}%
\pgfpathlineto{\pgfqpoint{2.074318in}{1.585050in}}%
\pgfpathlineto{\pgfqpoint{2.033548in}{1.563097in}}%
\pgfpathlineto{\pgfqpoint{2.055502in}{1.603867in}}%
\pgfpathlineto{\pgfqpoint{2.058638in}{1.594458in}}%
\pgfpathlineto{\pgfqpoint{2.138799in}{1.674620in}}%
\pgfpathlineto{\pgfqpoint{2.145072in}{1.668348in}}%
\pgfusepath{fill}%
\end{pgfscope}%
\begin{pgfscope}%
\pgfpathrectangle{\pgfqpoint{1.432000in}{0.528000in}}{\pgfqpoint{3.696000in}{3.696000in}} %
\pgfusepath{clip}%
\pgfsetbuttcap%
\pgfsetroundjoin%
\definecolor{currentfill}{rgb}{0.279574,0.170599,0.479997}%
\pgfsetfillcolor{currentfill}%
\pgfsetlinewidth{0.000000pt}%
\definecolor{currentstroke}{rgb}{0.000000,0.000000,0.000000}%
\pgfsetstrokecolor{currentstroke}%
\pgfsetdash{}{0pt}%
\pgfpathmoveto{\pgfqpoint{2.252306in}{1.667517in}}%
\pgfpathlineto{\pgfqpoint{2.071235in}{1.576981in}}%
\pgfpathlineto{\pgfqpoint{2.079168in}{1.571031in}}%
\pgfpathlineto{\pgfqpoint{2.033548in}{1.563097in}}%
\pgfpathlineto{\pgfqpoint{2.067268in}{1.594832in}}%
\pgfpathlineto{\pgfqpoint{2.067268in}{1.584915in}}%
\pgfpathlineto{\pgfqpoint{2.248339in}{1.675451in}}%
\pgfpathlineto{\pgfqpoint{2.252306in}{1.667517in}}%
\pgfusepath{fill}%
\end{pgfscope}%
\begin{pgfscope}%
\pgfpathrectangle{\pgfqpoint{1.432000in}{0.528000in}}{\pgfqpoint{3.696000in}{3.696000in}} %
\pgfusepath{clip}%
\pgfsetbuttcap%
\pgfsetroundjoin%
\definecolor{currentfill}{rgb}{0.279566,0.067836,0.391917}%
\pgfsetfillcolor{currentfill}%
\pgfsetlinewidth{0.000000pt}%
\definecolor{currentstroke}{rgb}{0.000000,0.000000,0.000000}%
\pgfsetstrokecolor{currentstroke}%
\pgfsetdash{}{0pt}%
\pgfpathmoveto{\pgfqpoint{2.250323in}{1.667049in}}%
\pgfpathlineto{\pgfqpoint{2.181852in}{1.667049in}}%
\pgfpathlineto{\pgfqpoint{2.186287in}{1.658178in}}%
\pgfpathlineto{\pgfqpoint{2.141935in}{1.671484in}}%
\pgfpathlineto{\pgfqpoint{2.186287in}{1.684789in}}%
\pgfpathlineto{\pgfqpoint{2.181852in}{1.675919in}}%
\pgfpathlineto{\pgfqpoint{2.250323in}{1.675919in}}%
\pgfpathlineto{\pgfqpoint{2.250323in}{1.667049in}}%
\pgfusepath{fill}%
\end{pgfscope}%
\begin{pgfscope}%
\pgfpathrectangle{\pgfqpoint{1.432000in}{0.528000in}}{\pgfqpoint{3.696000in}{3.696000in}} %
\pgfusepath{clip}%
\pgfsetbuttcap%
\pgfsetroundjoin%
\definecolor{currentfill}{rgb}{0.283091,0.110553,0.431554}%
\pgfsetfillcolor{currentfill}%
\pgfsetlinewidth{0.000000pt}%
\definecolor{currentstroke}{rgb}{0.000000,0.000000,0.000000}%
\pgfsetstrokecolor{currentstroke}%
\pgfsetdash{}{0pt}%
\pgfpathmoveto{\pgfqpoint{2.360693in}{1.667517in}}%
\pgfpathlineto{\pgfqpoint{2.179622in}{1.576981in}}%
\pgfpathlineto{\pgfqpoint{2.187556in}{1.571031in}}%
\pgfpathlineto{\pgfqpoint{2.141935in}{1.563097in}}%
\pgfpathlineto{\pgfqpoint{2.175655in}{1.594832in}}%
\pgfpathlineto{\pgfqpoint{2.175655in}{1.584915in}}%
\pgfpathlineto{\pgfqpoint{2.356726in}{1.675451in}}%
\pgfpathlineto{\pgfqpoint{2.360693in}{1.667517in}}%
\pgfusepath{fill}%
\end{pgfscope}%
\begin{pgfscope}%
\pgfpathrectangle{\pgfqpoint{1.432000in}{0.528000in}}{\pgfqpoint{3.696000in}{3.696000in}} %
\pgfusepath{clip}%
\pgfsetbuttcap%
\pgfsetroundjoin%
\definecolor{currentfill}{rgb}{0.282623,0.140926,0.457517}%
\pgfsetfillcolor{currentfill}%
\pgfsetlinewidth{0.000000pt}%
\definecolor{currentstroke}{rgb}{0.000000,0.000000,0.000000}%
\pgfsetstrokecolor{currentstroke}%
\pgfsetdash{}{0pt}%
\pgfpathmoveto{\pgfqpoint{2.358710in}{1.667049in}}%
\pgfpathlineto{\pgfqpoint{2.181852in}{1.667049in}}%
\pgfpathlineto{\pgfqpoint{2.186287in}{1.658178in}}%
\pgfpathlineto{\pgfqpoint{2.141935in}{1.671484in}}%
\pgfpathlineto{\pgfqpoint{2.186287in}{1.684789in}}%
\pgfpathlineto{\pgfqpoint{2.181852in}{1.675919in}}%
\pgfpathlineto{\pgfqpoint{2.358710in}{1.675919in}}%
\pgfpathlineto{\pgfqpoint{2.358710in}{1.667049in}}%
\pgfusepath{fill}%
\end{pgfscope}%
\begin{pgfscope}%
\pgfpathrectangle{\pgfqpoint{1.432000in}{0.528000in}}{\pgfqpoint{3.696000in}{3.696000in}} %
\pgfusepath{clip}%
\pgfsetbuttcap%
\pgfsetroundjoin%
\definecolor{currentfill}{rgb}{0.268510,0.009605,0.335427}%
\pgfsetfillcolor{currentfill}%
\pgfsetlinewidth{0.000000pt}%
\definecolor{currentstroke}{rgb}{0.000000,0.000000,0.000000}%
\pgfsetstrokecolor{currentstroke}%
\pgfsetdash{}{0pt}%
\pgfpathmoveto{\pgfqpoint{2.358710in}{1.667049in}}%
\pgfpathlineto{\pgfqpoint{2.290239in}{1.667049in}}%
\pgfpathlineto{\pgfqpoint{2.294675in}{1.658178in}}%
\pgfpathlineto{\pgfqpoint{2.250323in}{1.671484in}}%
\pgfpathlineto{\pgfqpoint{2.294675in}{1.684789in}}%
\pgfpathlineto{\pgfqpoint{2.290239in}{1.675919in}}%
\pgfpathlineto{\pgfqpoint{2.358710in}{1.675919in}}%
\pgfpathlineto{\pgfqpoint{2.358710in}{1.667049in}}%
\pgfusepath{fill}%
\end{pgfscope}%
\begin{pgfscope}%
\pgfpathrectangle{\pgfqpoint{1.432000in}{0.528000in}}{\pgfqpoint{3.696000in}{3.696000in}} %
\pgfusepath{clip}%
\pgfsetbuttcap%
\pgfsetroundjoin%
\definecolor{currentfill}{rgb}{0.216210,0.351535,0.550627}%
\pgfsetfillcolor{currentfill}%
\pgfsetlinewidth{0.000000pt}%
\definecolor{currentstroke}{rgb}{0.000000,0.000000,0.000000}%
\pgfsetstrokecolor{currentstroke}%
\pgfsetdash{}{0pt}%
\pgfpathmoveto{\pgfqpoint{2.467097in}{1.667049in}}%
\pgfpathlineto{\pgfqpoint{2.290239in}{1.667049in}}%
\pgfpathlineto{\pgfqpoint{2.294675in}{1.658178in}}%
\pgfpathlineto{\pgfqpoint{2.250323in}{1.671484in}}%
\pgfpathlineto{\pgfqpoint{2.294675in}{1.684789in}}%
\pgfpathlineto{\pgfqpoint{2.290239in}{1.675919in}}%
\pgfpathlineto{\pgfqpoint{2.467097in}{1.675919in}}%
\pgfpathlineto{\pgfqpoint{2.467097in}{1.667049in}}%
\pgfusepath{fill}%
\end{pgfscope}%
\begin{pgfscope}%
\pgfpathrectangle{\pgfqpoint{1.432000in}{0.528000in}}{\pgfqpoint{3.696000in}{3.696000in}} %
\pgfusepath{clip}%
\pgfsetbuttcap%
\pgfsetroundjoin%
\definecolor{currentfill}{rgb}{0.141935,0.526453,0.555991}%
\pgfsetfillcolor{currentfill}%
\pgfsetlinewidth{0.000000pt}%
\definecolor{currentstroke}{rgb}{0.000000,0.000000,0.000000}%
\pgfsetstrokecolor{currentstroke}%
\pgfsetdash{}{0pt}%
\pgfpathmoveto{\pgfqpoint{2.575484in}{1.667049in}}%
\pgfpathlineto{\pgfqpoint{2.398626in}{1.667049in}}%
\pgfpathlineto{\pgfqpoint{2.403062in}{1.658178in}}%
\pgfpathlineto{\pgfqpoint{2.358710in}{1.671484in}}%
\pgfpathlineto{\pgfqpoint{2.403062in}{1.684789in}}%
\pgfpathlineto{\pgfqpoint{2.398626in}{1.675919in}}%
\pgfpathlineto{\pgfqpoint{2.575484in}{1.675919in}}%
\pgfpathlineto{\pgfqpoint{2.575484in}{1.667049in}}%
\pgfusepath{fill}%
\end{pgfscope}%
\begin{pgfscope}%
\pgfpathrectangle{\pgfqpoint{1.432000in}{0.528000in}}{\pgfqpoint{3.696000in}{3.696000in}} %
\pgfusepath{clip}%
\pgfsetbuttcap%
\pgfsetroundjoin%
\definecolor{currentfill}{rgb}{0.216210,0.351535,0.550627}%
\pgfsetfillcolor{currentfill}%
\pgfsetlinewidth{0.000000pt}%
\definecolor{currentstroke}{rgb}{0.000000,0.000000,0.000000}%
\pgfsetstrokecolor{currentstroke}%
\pgfsetdash{}{0pt}%
\pgfpathmoveto{\pgfqpoint{2.683871in}{1.667049in}}%
\pgfpathlineto{\pgfqpoint{2.507014in}{1.667049in}}%
\pgfpathlineto{\pgfqpoint{2.511449in}{1.658178in}}%
\pgfpathlineto{\pgfqpoint{2.467097in}{1.671484in}}%
\pgfpathlineto{\pgfqpoint{2.511449in}{1.684789in}}%
\pgfpathlineto{\pgfqpoint{2.507014in}{1.675919in}}%
\pgfpathlineto{\pgfqpoint{2.683871in}{1.675919in}}%
\pgfpathlineto{\pgfqpoint{2.683871in}{1.667049in}}%
\pgfusepath{fill}%
\end{pgfscope}%
\begin{pgfscope}%
\pgfpathrectangle{\pgfqpoint{1.432000in}{0.528000in}}{\pgfqpoint{3.696000in}{3.696000in}} %
\pgfusepath{clip}%
\pgfsetbuttcap%
\pgfsetroundjoin%
\definecolor{currentfill}{rgb}{0.271305,0.019942,0.347269}%
\pgfsetfillcolor{currentfill}%
\pgfsetlinewidth{0.000000pt}%
\definecolor{currentstroke}{rgb}{0.000000,0.000000,0.000000}%
\pgfsetstrokecolor{currentstroke}%
\pgfsetdash{}{0pt}%
\pgfpathmoveto{\pgfqpoint{2.792258in}{1.667049in}}%
\pgfpathlineto{\pgfqpoint{2.507014in}{1.667049in}}%
\pgfpathlineto{\pgfqpoint{2.511449in}{1.658178in}}%
\pgfpathlineto{\pgfqpoint{2.467097in}{1.671484in}}%
\pgfpathlineto{\pgfqpoint{2.511449in}{1.684789in}}%
\pgfpathlineto{\pgfqpoint{2.507014in}{1.675919in}}%
\pgfpathlineto{\pgfqpoint{2.792258in}{1.675919in}}%
\pgfpathlineto{\pgfqpoint{2.792258in}{1.667049in}}%
\pgfusepath{fill}%
\end{pgfscope}%
\begin{pgfscope}%
\pgfpathrectangle{\pgfqpoint{1.432000in}{0.528000in}}{\pgfqpoint{3.696000in}{3.696000in}} %
\pgfusepath{clip}%
\pgfsetbuttcap%
\pgfsetroundjoin%
\definecolor{currentfill}{rgb}{0.203063,0.379716,0.553925}%
\pgfsetfillcolor{currentfill}%
\pgfsetlinewidth{0.000000pt}%
\definecolor{currentstroke}{rgb}{0.000000,0.000000,0.000000}%
\pgfsetstrokecolor{currentstroke}%
\pgfsetdash{}{0pt}%
\pgfpathmoveto{\pgfqpoint{2.792258in}{1.667049in}}%
\pgfpathlineto{\pgfqpoint{2.615401in}{1.667049in}}%
\pgfpathlineto{\pgfqpoint{2.619836in}{1.658178in}}%
\pgfpathlineto{\pgfqpoint{2.575484in}{1.671484in}}%
\pgfpathlineto{\pgfqpoint{2.619836in}{1.684789in}}%
\pgfpathlineto{\pgfqpoint{2.615401in}{1.675919in}}%
\pgfpathlineto{\pgfqpoint{2.792258in}{1.675919in}}%
\pgfpathlineto{\pgfqpoint{2.792258in}{1.667049in}}%
\pgfusepath{fill}%
\end{pgfscope}%
\begin{pgfscope}%
\pgfpathrectangle{\pgfqpoint{1.432000in}{0.528000in}}{\pgfqpoint{3.696000in}{3.696000in}} %
\pgfusepath{clip}%
\pgfsetbuttcap%
\pgfsetroundjoin%
\definecolor{currentfill}{rgb}{0.269308,0.218818,0.509577}%
\pgfsetfillcolor{currentfill}%
\pgfsetlinewidth{0.000000pt}%
\definecolor{currentstroke}{rgb}{0.000000,0.000000,0.000000}%
\pgfsetstrokecolor{currentstroke}%
\pgfsetdash{}{0pt}%
\pgfpathmoveto{\pgfqpoint{2.900645in}{1.667049in}}%
\pgfpathlineto{\pgfqpoint{2.615401in}{1.667049in}}%
\pgfpathlineto{\pgfqpoint{2.619836in}{1.658178in}}%
\pgfpathlineto{\pgfqpoint{2.575484in}{1.671484in}}%
\pgfpathlineto{\pgfqpoint{2.619836in}{1.684789in}}%
\pgfpathlineto{\pgfqpoint{2.615401in}{1.675919in}}%
\pgfpathlineto{\pgfqpoint{2.900645in}{1.675919in}}%
\pgfpathlineto{\pgfqpoint{2.900645in}{1.667049in}}%
\pgfusepath{fill}%
\end{pgfscope}%
\begin{pgfscope}%
\pgfpathrectangle{\pgfqpoint{1.432000in}{0.528000in}}{\pgfqpoint{3.696000in}{3.696000in}} %
\pgfusepath{clip}%
\pgfsetbuttcap%
\pgfsetroundjoin%
\definecolor{currentfill}{rgb}{0.157729,0.485932,0.558013}%
\pgfsetfillcolor{currentfill}%
\pgfsetlinewidth{0.000000pt}%
\definecolor{currentstroke}{rgb}{0.000000,0.000000,0.000000}%
\pgfsetstrokecolor{currentstroke}%
\pgfsetdash{}{0pt}%
\pgfpathmoveto{\pgfqpoint{2.900645in}{1.667049in}}%
\pgfpathlineto{\pgfqpoint{2.723788in}{1.667049in}}%
\pgfpathlineto{\pgfqpoint{2.728223in}{1.658178in}}%
\pgfpathlineto{\pgfqpoint{2.683871in}{1.671484in}}%
\pgfpathlineto{\pgfqpoint{2.728223in}{1.684789in}}%
\pgfpathlineto{\pgfqpoint{2.723788in}{1.675919in}}%
\pgfpathlineto{\pgfqpoint{2.900645in}{1.675919in}}%
\pgfpathlineto{\pgfqpoint{2.900645in}{1.667049in}}%
\pgfusepath{fill}%
\end{pgfscope}%
\begin{pgfscope}%
\pgfpathrectangle{\pgfqpoint{1.432000in}{0.528000in}}{\pgfqpoint{3.696000in}{3.696000in}} %
\pgfusepath{clip}%
\pgfsetbuttcap%
\pgfsetroundjoin%
\definecolor{currentfill}{rgb}{0.273809,0.031497,0.358853}%
\pgfsetfillcolor{currentfill}%
\pgfsetlinewidth{0.000000pt}%
\definecolor{currentstroke}{rgb}{0.000000,0.000000,0.000000}%
\pgfsetstrokecolor{currentstroke}%
\pgfsetdash{}{0pt}%
\pgfpathmoveto{\pgfqpoint{3.009032in}{1.667049in}}%
\pgfpathlineto{\pgfqpoint{2.723788in}{1.667049in}}%
\pgfpathlineto{\pgfqpoint{2.728223in}{1.658178in}}%
\pgfpathlineto{\pgfqpoint{2.683871in}{1.671484in}}%
\pgfpathlineto{\pgfqpoint{2.728223in}{1.684789in}}%
\pgfpathlineto{\pgfqpoint{2.723788in}{1.675919in}}%
\pgfpathlineto{\pgfqpoint{3.009032in}{1.675919in}}%
\pgfpathlineto{\pgfqpoint{3.009032in}{1.667049in}}%
\pgfusepath{fill}%
\end{pgfscope}%
\begin{pgfscope}%
\pgfpathrectangle{\pgfqpoint{1.432000in}{0.528000in}}{\pgfqpoint{3.696000in}{3.696000in}} %
\pgfusepath{clip}%
\pgfsetbuttcap%
\pgfsetroundjoin%
\definecolor{currentfill}{rgb}{0.150148,0.676631,0.506589}%
\pgfsetfillcolor{currentfill}%
\pgfsetlinewidth{0.000000pt}%
\definecolor{currentstroke}{rgb}{0.000000,0.000000,0.000000}%
\pgfsetstrokecolor{currentstroke}%
\pgfsetdash{}{0pt}%
\pgfpathmoveto{\pgfqpoint{3.009032in}{1.667049in}}%
\pgfpathlineto{\pgfqpoint{2.832175in}{1.667049in}}%
\pgfpathlineto{\pgfqpoint{2.836610in}{1.658178in}}%
\pgfpathlineto{\pgfqpoint{2.792258in}{1.671484in}}%
\pgfpathlineto{\pgfqpoint{2.836610in}{1.684789in}}%
\pgfpathlineto{\pgfqpoint{2.832175in}{1.675919in}}%
\pgfpathlineto{\pgfqpoint{3.009032in}{1.675919in}}%
\pgfpathlineto{\pgfqpoint{3.009032in}{1.667049in}}%
\pgfusepath{fill}%
\end{pgfscope}%
\begin{pgfscope}%
\pgfpathrectangle{\pgfqpoint{1.432000in}{0.528000in}}{\pgfqpoint{3.696000in}{3.696000in}} %
\pgfusepath{clip}%
\pgfsetbuttcap%
\pgfsetroundjoin%
\definecolor{currentfill}{rgb}{0.208030,0.718701,0.472873}%
\pgfsetfillcolor{currentfill}%
\pgfsetlinewidth{0.000000pt}%
\definecolor{currentstroke}{rgb}{0.000000,0.000000,0.000000}%
\pgfsetstrokecolor{currentstroke}%
\pgfsetdash{}{0pt}%
\pgfpathmoveto{\pgfqpoint{3.117419in}{1.667049in}}%
\pgfpathlineto{\pgfqpoint{2.940562in}{1.667049in}}%
\pgfpathlineto{\pgfqpoint{2.944997in}{1.658178in}}%
\pgfpathlineto{\pgfqpoint{2.900645in}{1.671484in}}%
\pgfpathlineto{\pgfqpoint{2.944997in}{1.684789in}}%
\pgfpathlineto{\pgfqpoint{2.940562in}{1.675919in}}%
\pgfpathlineto{\pgfqpoint{3.117419in}{1.675919in}}%
\pgfpathlineto{\pgfqpoint{3.117419in}{1.667049in}}%
\pgfusepath{fill}%
\end{pgfscope}%
\begin{pgfscope}%
\pgfpathrectangle{\pgfqpoint{1.432000in}{0.528000in}}{\pgfqpoint{3.696000in}{3.696000in}} %
\pgfusepath{clip}%
\pgfsetbuttcap%
\pgfsetroundjoin%
\definecolor{currentfill}{rgb}{0.124395,0.578002,0.548287}%
\pgfsetfillcolor{currentfill}%
\pgfsetlinewidth{0.000000pt}%
\definecolor{currentstroke}{rgb}{0.000000,0.000000,0.000000}%
\pgfsetstrokecolor{currentstroke}%
\pgfsetdash{}{0pt}%
\pgfpathmoveto{\pgfqpoint{3.225806in}{1.667049in}}%
\pgfpathlineto{\pgfqpoint{3.048949in}{1.667049in}}%
\pgfpathlineto{\pgfqpoint{3.053384in}{1.658178in}}%
\pgfpathlineto{\pgfqpoint{3.009032in}{1.671484in}}%
\pgfpathlineto{\pgfqpoint{3.053384in}{1.684789in}}%
\pgfpathlineto{\pgfqpoint{3.048949in}{1.675919in}}%
\pgfpathlineto{\pgfqpoint{3.225806in}{1.675919in}}%
\pgfpathlineto{\pgfqpoint{3.225806in}{1.667049in}}%
\pgfusepath{fill}%
\end{pgfscope}%
\begin{pgfscope}%
\pgfpathrectangle{\pgfqpoint{1.432000in}{0.528000in}}{\pgfqpoint{3.696000in}{3.696000in}} %
\pgfusepath{clip}%
\pgfsetbuttcap%
\pgfsetroundjoin%
\definecolor{currentfill}{rgb}{0.141935,0.526453,0.555991}%
\pgfsetfillcolor{currentfill}%
\pgfsetlinewidth{0.000000pt}%
\definecolor{currentstroke}{rgb}{0.000000,0.000000,0.000000}%
\pgfsetstrokecolor{currentstroke}%
\pgfsetdash{}{0pt}%
\pgfpathmoveto{\pgfqpoint{3.334194in}{1.667049in}}%
\pgfpathlineto{\pgfqpoint{3.157336in}{1.667049in}}%
\pgfpathlineto{\pgfqpoint{3.161771in}{1.658178in}}%
\pgfpathlineto{\pgfqpoint{3.117419in}{1.671484in}}%
\pgfpathlineto{\pgfqpoint{3.161771in}{1.684789in}}%
\pgfpathlineto{\pgfqpoint{3.157336in}{1.675919in}}%
\pgfpathlineto{\pgfqpoint{3.334194in}{1.675919in}}%
\pgfpathlineto{\pgfqpoint{3.334194in}{1.667049in}}%
\pgfusepath{fill}%
\end{pgfscope}%
\begin{pgfscope}%
\pgfpathrectangle{\pgfqpoint{1.432000in}{0.528000in}}{\pgfqpoint{3.696000in}{3.696000in}} %
\pgfusepath{clip}%
\pgfsetbuttcap%
\pgfsetroundjoin%
\definecolor{currentfill}{rgb}{0.137770,0.537492,0.554906}%
\pgfsetfillcolor{currentfill}%
\pgfsetlinewidth{0.000000pt}%
\definecolor{currentstroke}{rgb}{0.000000,0.000000,0.000000}%
\pgfsetstrokecolor{currentstroke}%
\pgfsetdash{}{0pt}%
\pgfpathmoveto{\pgfqpoint{3.442581in}{1.667049in}}%
\pgfpathlineto{\pgfqpoint{3.265723in}{1.667049in}}%
\pgfpathlineto{\pgfqpoint{3.270158in}{1.658178in}}%
\pgfpathlineto{\pgfqpoint{3.225806in}{1.671484in}}%
\pgfpathlineto{\pgfqpoint{3.270158in}{1.684789in}}%
\pgfpathlineto{\pgfqpoint{3.265723in}{1.675919in}}%
\pgfpathlineto{\pgfqpoint{3.442581in}{1.675919in}}%
\pgfpathlineto{\pgfqpoint{3.442581in}{1.667049in}}%
\pgfusepath{fill}%
\end{pgfscope}%
\begin{pgfscope}%
\pgfpathrectangle{\pgfqpoint{1.432000in}{0.528000in}}{\pgfqpoint{3.696000in}{3.696000in}} %
\pgfusepath{clip}%
\pgfsetbuttcap%
\pgfsetroundjoin%
\definecolor{currentfill}{rgb}{0.204903,0.375746,0.553533}%
\pgfsetfillcolor{currentfill}%
\pgfsetlinewidth{0.000000pt}%
\definecolor{currentstroke}{rgb}{0.000000,0.000000,0.000000}%
\pgfsetstrokecolor{currentstroke}%
\pgfsetdash{}{0pt}%
\pgfpathmoveto{\pgfqpoint{3.550968in}{1.667049in}}%
\pgfpathlineto{\pgfqpoint{3.374110in}{1.667049in}}%
\pgfpathlineto{\pgfqpoint{3.378546in}{1.658178in}}%
\pgfpathlineto{\pgfqpoint{3.334194in}{1.671484in}}%
\pgfpathlineto{\pgfqpoint{3.378546in}{1.684789in}}%
\pgfpathlineto{\pgfqpoint{3.374110in}{1.675919in}}%
\pgfpathlineto{\pgfqpoint{3.550968in}{1.675919in}}%
\pgfpathlineto{\pgfqpoint{3.550968in}{1.667049in}}%
\pgfusepath{fill}%
\end{pgfscope}%
\begin{pgfscope}%
\pgfpathrectangle{\pgfqpoint{1.432000in}{0.528000in}}{\pgfqpoint{3.696000in}{3.696000in}} %
\pgfusepath{clip}%
\pgfsetbuttcap%
\pgfsetroundjoin%
\definecolor{currentfill}{rgb}{0.283072,0.130895,0.449241}%
\pgfsetfillcolor{currentfill}%
\pgfsetlinewidth{0.000000pt}%
\definecolor{currentstroke}{rgb}{0.000000,0.000000,0.000000}%
\pgfsetstrokecolor{currentstroke}%
\pgfsetdash{}{0pt}%
\pgfpathmoveto{\pgfqpoint{3.659355in}{1.667049in}}%
\pgfpathlineto{\pgfqpoint{3.374110in}{1.667049in}}%
\pgfpathlineto{\pgfqpoint{3.378546in}{1.658178in}}%
\pgfpathlineto{\pgfqpoint{3.334194in}{1.671484in}}%
\pgfpathlineto{\pgfqpoint{3.378546in}{1.684789in}}%
\pgfpathlineto{\pgfqpoint{3.374110in}{1.675919in}}%
\pgfpathlineto{\pgfqpoint{3.659355in}{1.675919in}}%
\pgfpathlineto{\pgfqpoint{3.659355in}{1.667049in}}%
\pgfusepath{fill}%
\end{pgfscope}%
\begin{pgfscope}%
\pgfpathrectangle{\pgfqpoint{1.432000in}{0.528000in}}{\pgfqpoint{3.696000in}{3.696000in}} %
\pgfusepath{clip}%
\pgfsetbuttcap%
\pgfsetroundjoin%
\definecolor{currentfill}{rgb}{0.282327,0.094955,0.417331}%
\pgfsetfillcolor{currentfill}%
\pgfsetlinewidth{0.000000pt}%
\definecolor{currentstroke}{rgb}{0.000000,0.000000,0.000000}%
\pgfsetstrokecolor{currentstroke}%
\pgfsetdash{}{0pt}%
\pgfpathmoveto{\pgfqpoint{3.767742in}{1.667049in}}%
\pgfpathlineto{\pgfqpoint{3.482497in}{1.667049in}}%
\pgfpathlineto{\pgfqpoint{3.486933in}{1.658178in}}%
\pgfpathlineto{\pgfqpoint{3.442581in}{1.671484in}}%
\pgfpathlineto{\pgfqpoint{3.486933in}{1.684789in}}%
\pgfpathlineto{\pgfqpoint{3.482497in}{1.675919in}}%
\pgfpathlineto{\pgfqpoint{3.767742in}{1.675919in}}%
\pgfpathlineto{\pgfqpoint{3.767742in}{1.667049in}}%
\pgfusepath{fill}%
\end{pgfscope}%
\begin{pgfscope}%
\pgfpathrectangle{\pgfqpoint{1.432000in}{0.528000in}}{\pgfqpoint{3.696000in}{3.696000in}} %
\pgfusepath{clip}%
\pgfsetbuttcap%
\pgfsetroundjoin%
\definecolor{currentfill}{rgb}{0.276022,0.044167,0.370164}%
\pgfsetfillcolor{currentfill}%
\pgfsetlinewidth{0.000000pt}%
\definecolor{currentstroke}{rgb}{0.000000,0.000000,0.000000}%
\pgfsetstrokecolor{currentstroke}%
\pgfsetdash{}{0pt}%
\pgfpathmoveto{\pgfqpoint{3.877205in}{1.667181in}}%
\pgfpathlineto{\pgfqpoint{3.482381in}{1.568475in}}%
\pgfpathlineto{\pgfqpoint{3.488835in}{1.560945in}}%
\pgfpathlineto{\pgfqpoint{3.442581in}{1.563097in}}%
\pgfpathlineto{\pgfqpoint{3.482381in}{1.586762in}}%
\pgfpathlineto{\pgfqpoint{3.480230in}{1.577081in}}%
\pgfpathlineto{\pgfqpoint{3.875053in}{1.675787in}}%
\pgfpathlineto{\pgfqpoint{3.877205in}{1.667181in}}%
\pgfusepath{fill}%
\end{pgfscope}%
\begin{pgfscope}%
\pgfpathrectangle{\pgfqpoint{1.432000in}{0.528000in}}{\pgfqpoint{3.696000in}{3.696000in}} %
\pgfusepath{clip}%
\pgfsetbuttcap%
\pgfsetroundjoin%
\definecolor{currentfill}{rgb}{0.269944,0.014625,0.341379}%
\pgfsetfillcolor{currentfill}%
\pgfsetlinewidth{0.000000pt}%
\definecolor{currentstroke}{rgb}{0.000000,0.000000,0.000000}%
\pgfsetstrokecolor{currentstroke}%
\pgfsetdash{}{0pt}%
\pgfpathmoveto{\pgfqpoint{3.877532in}{1.667276in}}%
\pgfpathlineto{\pgfqpoint{3.590239in}{1.571512in}}%
\pgfpathlineto{\pgfqpoint{3.597251in}{1.564499in}}%
\pgfpathlineto{\pgfqpoint{3.550968in}{1.563097in}}%
\pgfpathlineto{\pgfqpoint{3.588836in}{1.589745in}}%
\pgfpathlineto{\pgfqpoint{3.587434in}{1.579927in}}%
\pgfpathlineto{\pgfqpoint{3.874726in}{1.675691in}}%
\pgfpathlineto{\pgfqpoint{3.877532in}{1.667276in}}%
\pgfusepath{fill}%
\end{pgfscope}%
\begin{pgfscope}%
\pgfpathrectangle{\pgfqpoint{1.432000in}{0.528000in}}{\pgfqpoint{3.696000in}{3.696000in}} %
\pgfusepath{clip}%
\pgfsetbuttcap%
\pgfsetroundjoin%
\definecolor{currentfill}{rgb}{0.282656,0.100196,0.422160}%
\pgfsetfillcolor{currentfill}%
\pgfsetlinewidth{0.000000pt}%
\definecolor{currentstroke}{rgb}{0.000000,0.000000,0.000000}%
\pgfsetstrokecolor{currentstroke}%
\pgfsetdash{}{0pt}%
\pgfpathmoveto{\pgfqpoint{3.876129in}{1.667049in}}%
\pgfpathlineto{\pgfqpoint{3.482497in}{1.667049in}}%
\pgfpathlineto{\pgfqpoint{3.486933in}{1.658178in}}%
\pgfpathlineto{\pgfqpoint{3.442581in}{1.671484in}}%
\pgfpathlineto{\pgfqpoint{3.486933in}{1.684789in}}%
\pgfpathlineto{\pgfqpoint{3.482497in}{1.675919in}}%
\pgfpathlineto{\pgfqpoint{3.876129in}{1.675919in}}%
\pgfpathlineto{\pgfqpoint{3.876129in}{1.667049in}}%
\pgfusepath{fill}%
\end{pgfscope}%
\begin{pgfscope}%
\pgfpathrectangle{\pgfqpoint{1.432000in}{0.528000in}}{\pgfqpoint{3.696000in}{3.696000in}} %
\pgfusepath{clip}%
\pgfsetbuttcap%
\pgfsetroundjoin%
\definecolor{currentfill}{rgb}{0.246811,0.283237,0.535941}%
\pgfsetfillcolor{currentfill}%
\pgfsetlinewidth{0.000000pt}%
\definecolor{currentstroke}{rgb}{0.000000,0.000000,0.000000}%
\pgfsetstrokecolor{currentstroke}%
\pgfsetdash{}{0pt}%
\pgfpathmoveto{\pgfqpoint{3.985592in}{1.667181in}}%
\pgfpathlineto{\pgfqpoint{3.590768in}{1.568475in}}%
\pgfpathlineto{\pgfqpoint{3.597223in}{1.560945in}}%
\pgfpathlineto{\pgfqpoint{3.550968in}{1.563097in}}%
\pgfpathlineto{\pgfqpoint{3.590768in}{1.586762in}}%
\pgfpathlineto{\pgfqpoint{3.588617in}{1.577081in}}%
\pgfpathlineto{\pgfqpoint{3.983440in}{1.675787in}}%
\pgfpathlineto{\pgfqpoint{3.985592in}{1.667181in}}%
\pgfusepath{fill}%
\end{pgfscope}%
\begin{pgfscope}%
\pgfpathrectangle{\pgfqpoint{1.432000in}{0.528000in}}{\pgfqpoint{3.696000in}{3.696000in}} %
\pgfusepath{clip}%
\pgfsetbuttcap%
\pgfsetroundjoin%
\definecolor{currentfill}{rgb}{0.276194,0.190074,0.493001}%
\pgfsetfillcolor{currentfill}%
\pgfsetlinewidth{0.000000pt}%
\definecolor{currentstroke}{rgb}{0.000000,0.000000,0.000000}%
\pgfsetstrokecolor{currentstroke}%
\pgfsetdash{}{0pt}%
\pgfpathmoveto{\pgfqpoint{4.093979in}{1.667181in}}%
\pgfpathlineto{\pgfqpoint{3.699156in}{1.568475in}}%
\pgfpathlineto{\pgfqpoint{3.705610in}{1.560945in}}%
\pgfpathlineto{\pgfqpoint{3.659355in}{1.563097in}}%
\pgfpathlineto{\pgfqpoint{3.699156in}{1.586762in}}%
\pgfpathlineto{\pgfqpoint{3.697004in}{1.577081in}}%
\pgfpathlineto{\pgfqpoint{4.091828in}{1.675787in}}%
\pgfpathlineto{\pgfqpoint{4.093979in}{1.667181in}}%
\pgfusepath{fill}%
\end{pgfscope}%
\begin{pgfscope}%
\pgfpathrectangle{\pgfqpoint{1.432000in}{0.528000in}}{\pgfqpoint{3.696000in}{3.696000in}} %
\pgfusepath{clip}%
\pgfsetbuttcap%
\pgfsetroundjoin%
\definecolor{currentfill}{rgb}{0.257322,0.256130,0.526563}%
\pgfsetfillcolor{currentfill}%
\pgfsetlinewidth{0.000000pt}%
\definecolor{currentstroke}{rgb}{0.000000,0.000000,0.000000}%
\pgfsetstrokecolor{currentstroke}%
\pgfsetdash{}{0pt}%
\pgfpathmoveto{\pgfqpoint{4.202693in}{1.667276in}}%
\pgfpathlineto{\pgfqpoint{3.915400in}{1.571512in}}%
\pgfpathlineto{\pgfqpoint{3.922413in}{1.564499in}}%
\pgfpathlineto{\pgfqpoint{3.876129in}{1.563097in}}%
\pgfpathlineto{\pgfqpoint{3.913997in}{1.589745in}}%
\pgfpathlineto{\pgfqpoint{3.912595in}{1.579927in}}%
\pgfpathlineto{\pgfqpoint{4.199888in}{1.675691in}}%
\pgfpathlineto{\pgfqpoint{4.202693in}{1.667276in}}%
\pgfusepath{fill}%
\end{pgfscope}%
\begin{pgfscope}%
\pgfpathrectangle{\pgfqpoint{1.432000in}{0.528000in}}{\pgfqpoint{3.696000in}{3.696000in}} %
\pgfusepath{clip}%
\pgfsetbuttcap%
\pgfsetroundjoin%
\definecolor{currentfill}{rgb}{0.243113,0.292092,0.538516}%
\pgfsetfillcolor{currentfill}%
\pgfsetlinewidth{0.000000pt}%
\definecolor{currentstroke}{rgb}{0.000000,0.000000,0.000000}%
\pgfsetstrokecolor{currentstroke}%
\pgfsetdash{}{0pt}%
\pgfpathmoveto{\pgfqpoint{4.311080in}{1.667276in}}%
\pgfpathlineto{\pgfqpoint{4.023787in}{1.571512in}}%
\pgfpathlineto{\pgfqpoint{4.030800in}{1.564499in}}%
\pgfpathlineto{\pgfqpoint{3.984516in}{1.563097in}}%
\pgfpathlineto{\pgfqpoint{4.022385in}{1.589745in}}%
\pgfpathlineto{\pgfqpoint{4.020982in}{1.579927in}}%
\pgfpathlineto{\pgfqpoint{4.308275in}{1.675691in}}%
\pgfpathlineto{\pgfqpoint{4.311080in}{1.667276in}}%
\pgfusepath{fill}%
\end{pgfscope}%
\begin{pgfscope}%
\pgfpathrectangle{\pgfqpoint{1.432000in}{0.528000in}}{\pgfqpoint{3.696000in}{3.696000in}} %
\pgfusepath{clip}%
\pgfsetbuttcap%
\pgfsetroundjoin%
\definecolor{currentfill}{rgb}{0.263663,0.237631,0.518762}%
\pgfsetfillcolor{currentfill}%
\pgfsetlinewidth{0.000000pt}%
\definecolor{currentstroke}{rgb}{0.000000,0.000000,0.000000}%
\pgfsetstrokecolor{currentstroke}%
\pgfsetdash{}{0pt}%
\pgfpathmoveto{\pgfqpoint{4.419467in}{1.667276in}}%
\pgfpathlineto{\pgfqpoint{4.132174in}{1.571512in}}%
\pgfpathlineto{\pgfqpoint{4.139187in}{1.564499in}}%
\pgfpathlineto{\pgfqpoint{4.092903in}{1.563097in}}%
\pgfpathlineto{\pgfqpoint{4.130772in}{1.589745in}}%
\pgfpathlineto{\pgfqpoint{4.129369in}{1.579927in}}%
\pgfpathlineto{\pgfqpoint{4.416662in}{1.675691in}}%
\pgfpathlineto{\pgfqpoint{4.419467in}{1.667276in}}%
\pgfusepath{fill}%
\end{pgfscope}%
\begin{pgfscope}%
\pgfpathrectangle{\pgfqpoint{1.432000in}{0.528000in}}{\pgfqpoint{3.696000in}{3.696000in}} %
\pgfusepath{clip}%
\pgfsetbuttcap%
\pgfsetroundjoin%
\definecolor{currentfill}{rgb}{0.283197,0.115680,0.436115}%
\pgfsetfillcolor{currentfill}%
\pgfsetlinewidth{0.000000pt}%
\definecolor{currentstroke}{rgb}{0.000000,0.000000,0.000000}%
\pgfsetstrokecolor{currentstroke}%
\pgfsetdash{}{0pt}%
\pgfpathmoveto{\pgfqpoint{4.528435in}{1.667517in}}%
\pgfpathlineto{\pgfqpoint{4.347364in}{1.576981in}}%
\pgfpathlineto{\pgfqpoint{4.355297in}{1.571031in}}%
\pgfpathlineto{\pgfqpoint{4.309677in}{1.563097in}}%
\pgfpathlineto{\pgfqpoint{4.343397in}{1.594832in}}%
\pgfpathlineto{\pgfqpoint{4.343397in}{1.584915in}}%
\pgfpathlineto{\pgfqpoint{4.524468in}{1.675451in}}%
\pgfpathlineto{\pgfqpoint{4.528435in}{1.667517in}}%
\pgfusepath{fill}%
\end{pgfscope}%
\begin{pgfscope}%
\pgfpathrectangle{\pgfqpoint{1.432000in}{0.528000in}}{\pgfqpoint{3.696000in}{3.696000in}} %
\pgfusepath{clip}%
\pgfsetbuttcap%
\pgfsetroundjoin%
\definecolor{currentfill}{rgb}{0.282884,0.135920,0.453427}%
\pgfsetfillcolor{currentfill}%
\pgfsetlinewidth{0.000000pt}%
\definecolor{currentstroke}{rgb}{0.000000,0.000000,0.000000}%
\pgfsetstrokecolor{currentstroke}%
\pgfsetdash{}{0pt}%
\pgfpathmoveto{\pgfqpoint{4.529588in}{1.668348in}}%
\pgfpathlineto{\pgfqpoint{4.449426in}{1.588186in}}%
\pgfpathlineto{\pgfqpoint{4.458835in}{1.585050in}}%
\pgfpathlineto{\pgfqpoint{4.418065in}{1.563097in}}%
\pgfpathlineto{\pgfqpoint{4.440018in}{1.603867in}}%
\pgfpathlineto{\pgfqpoint{4.443154in}{1.594458in}}%
\pgfpathlineto{\pgfqpoint{4.523315in}{1.674620in}}%
\pgfpathlineto{\pgfqpoint{4.529588in}{1.668348in}}%
\pgfusepath{fill}%
\end{pgfscope}%
\begin{pgfscope}%
\pgfpathrectangle{\pgfqpoint{1.432000in}{0.528000in}}{\pgfqpoint{3.696000in}{3.696000in}} %
\pgfusepath{clip}%
\pgfsetbuttcap%
\pgfsetroundjoin%
\definecolor{currentfill}{rgb}{0.280894,0.078907,0.402329}%
\pgfsetfillcolor{currentfill}%
\pgfsetlinewidth{0.000000pt}%
\definecolor{currentstroke}{rgb}{0.000000,0.000000,0.000000}%
\pgfsetstrokecolor{currentstroke}%
\pgfsetdash{}{0pt}%
\pgfpathmoveto{\pgfqpoint{4.636822in}{1.667517in}}%
\pgfpathlineto{\pgfqpoint{4.455751in}{1.576981in}}%
\pgfpathlineto{\pgfqpoint{4.463685in}{1.571031in}}%
\pgfpathlineto{\pgfqpoint{4.418065in}{1.563097in}}%
\pgfpathlineto{\pgfqpoint{4.451784in}{1.594832in}}%
\pgfpathlineto{\pgfqpoint{4.451784in}{1.584915in}}%
\pgfpathlineto{\pgfqpoint{4.632855in}{1.675451in}}%
\pgfpathlineto{\pgfqpoint{4.636822in}{1.667517in}}%
\pgfusepath{fill}%
\end{pgfscope}%
\begin{pgfscope}%
\pgfpathrectangle{\pgfqpoint{1.432000in}{0.528000in}}{\pgfqpoint{3.696000in}{3.696000in}} %
\pgfusepath{clip}%
\pgfsetbuttcap%
\pgfsetroundjoin%
\definecolor{currentfill}{rgb}{0.279574,0.170599,0.479997}%
\pgfsetfillcolor{currentfill}%
\pgfsetlinewidth{0.000000pt}%
\definecolor{currentstroke}{rgb}{0.000000,0.000000,0.000000}%
\pgfsetstrokecolor{currentstroke}%
\pgfsetdash{}{0pt}%
\pgfpathmoveto{\pgfqpoint{4.637975in}{1.668348in}}%
\pgfpathlineto{\pgfqpoint{4.557813in}{1.588186in}}%
\pgfpathlineto{\pgfqpoint{4.567222in}{1.585050in}}%
\pgfpathlineto{\pgfqpoint{4.526452in}{1.563097in}}%
\pgfpathlineto{\pgfqpoint{4.548405in}{1.603867in}}%
\pgfpathlineto{\pgfqpoint{4.551541in}{1.594458in}}%
\pgfpathlineto{\pgfqpoint{4.631703in}{1.674620in}}%
\pgfpathlineto{\pgfqpoint{4.637975in}{1.668348in}}%
\pgfusepath{fill}%
\end{pgfscope}%
\begin{pgfscope}%
\pgfpathrectangle{\pgfqpoint{1.432000in}{0.528000in}}{\pgfqpoint{3.696000in}{3.696000in}} %
\pgfusepath{clip}%
\pgfsetbuttcap%
\pgfsetroundjoin%
\definecolor{currentfill}{rgb}{0.241237,0.296485,0.539709}%
\pgfsetfillcolor{currentfill}%
\pgfsetlinewidth{0.000000pt}%
\definecolor{currentstroke}{rgb}{0.000000,0.000000,0.000000}%
\pgfsetstrokecolor{currentstroke}%
\pgfsetdash{}{0pt}%
\pgfpathmoveto{\pgfqpoint{4.746362in}{1.668348in}}%
\pgfpathlineto{\pgfqpoint{4.666200in}{1.588186in}}%
\pgfpathlineto{\pgfqpoint{4.675609in}{1.585050in}}%
\pgfpathlineto{\pgfqpoint{4.634839in}{1.563097in}}%
\pgfpathlineto{\pgfqpoint{4.656792in}{1.603867in}}%
\pgfpathlineto{\pgfqpoint{4.659928in}{1.594458in}}%
\pgfpathlineto{\pgfqpoint{4.740090in}{1.674620in}}%
\pgfpathlineto{\pgfqpoint{4.746362in}{1.668348in}}%
\pgfusepath{fill}%
\end{pgfscope}%
\begin{pgfscope}%
\pgfpathrectangle{\pgfqpoint{1.432000in}{0.528000in}}{\pgfqpoint{3.696000in}{3.696000in}} %
\pgfusepath{clip}%
\pgfsetbuttcap%
\pgfsetroundjoin%
\definecolor{currentfill}{rgb}{0.281887,0.150881,0.465405}%
\pgfsetfillcolor{currentfill}%
\pgfsetlinewidth{0.000000pt}%
\definecolor{currentstroke}{rgb}{0.000000,0.000000,0.000000}%
\pgfsetstrokecolor{currentstroke}%
\pgfsetdash{}{0pt}%
\pgfpathmoveto{\pgfqpoint{4.743226in}{1.667049in}}%
\pgfpathlineto{\pgfqpoint{4.674756in}{1.667049in}}%
\pgfpathlineto{\pgfqpoint{4.679191in}{1.658178in}}%
\pgfpathlineto{\pgfqpoint{4.634839in}{1.671484in}}%
\pgfpathlineto{\pgfqpoint{4.679191in}{1.684789in}}%
\pgfpathlineto{\pgfqpoint{4.674756in}{1.675919in}}%
\pgfpathlineto{\pgfqpoint{4.743226in}{1.675919in}}%
\pgfpathlineto{\pgfqpoint{4.743226in}{1.667049in}}%
\pgfusepath{fill}%
\end{pgfscope}%
\begin{pgfscope}%
\pgfpathrectangle{\pgfqpoint{1.432000in}{0.528000in}}{\pgfqpoint{3.696000in}{3.696000in}} %
\pgfusepath{clip}%
\pgfsetbuttcap%
\pgfsetroundjoin%
\definecolor{currentfill}{rgb}{0.131172,0.555899,0.552459}%
\pgfsetfillcolor{currentfill}%
\pgfsetlinewidth{0.000000pt}%
\definecolor{currentstroke}{rgb}{0.000000,0.000000,0.000000}%
\pgfsetstrokecolor{currentstroke}%
\pgfsetdash{}{0pt}%
\pgfpathmoveto{\pgfqpoint{4.851613in}{1.667049in}}%
\pgfpathlineto{\pgfqpoint{4.783143in}{1.667049in}}%
\pgfpathlineto{\pgfqpoint{4.787578in}{1.658178in}}%
\pgfpathlineto{\pgfqpoint{4.743226in}{1.671484in}}%
\pgfpathlineto{\pgfqpoint{4.787578in}{1.684789in}}%
\pgfpathlineto{\pgfqpoint{4.783143in}{1.675919in}}%
\pgfpathlineto{\pgfqpoint{4.851613in}{1.675919in}}%
\pgfpathlineto{\pgfqpoint{4.851613in}{1.667049in}}%
\pgfusepath{fill}%
\end{pgfscope}%
\begin{pgfscope}%
\pgfpathrectangle{\pgfqpoint{1.432000in}{0.528000in}}{\pgfqpoint{3.696000in}{3.696000in}} %
\pgfusepath{clip}%
\pgfsetbuttcap%
\pgfsetroundjoin%
\definecolor{currentfill}{rgb}{0.280894,0.078907,0.402329}%
\pgfsetfillcolor{currentfill}%
\pgfsetlinewidth{0.000000pt}%
\definecolor{currentstroke}{rgb}{0.000000,0.000000,0.000000}%
\pgfsetstrokecolor{currentstroke}%
\pgfsetdash{}{0pt}%
\pgfpathmoveto{\pgfqpoint{4.856048in}{1.671484in}}%
\pgfpathlineto{\pgfqpoint{4.853831in}{1.675325in}}%
\pgfpathlineto{\pgfqpoint{4.849395in}{1.675325in}}%
\pgfpathlineto{\pgfqpoint{4.847178in}{1.671484in}}%
\pgfpathlineto{\pgfqpoint{4.849395in}{1.667643in}}%
\pgfpathlineto{\pgfqpoint{4.853831in}{1.667643in}}%
\pgfpathlineto{\pgfqpoint{4.856048in}{1.671484in}}%
\pgfpathlineto{\pgfqpoint{4.853831in}{1.675325in}}%
\pgfusepath{fill}%
\end{pgfscope}%
\begin{pgfscope}%
\pgfpathrectangle{\pgfqpoint{1.432000in}{0.528000in}}{\pgfqpoint{3.696000in}{3.696000in}} %
\pgfusepath{clip}%
\pgfsetbuttcap%
\pgfsetroundjoin%
\definecolor{currentfill}{rgb}{0.244972,0.287675,0.537260}%
\pgfsetfillcolor{currentfill}%
\pgfsetlinewidth{0.000000pt}%
\definecolor{currentstroke}{rgb}{0.000000,0.000000,0.000000}%
\pgfsetstrokecolor{currentstroke}%
\pgfsetdash{}{0pt}%
\pgfpathmoveto{\pgfqpoint{4.960000in}{1.667049in}}%
\pgfpathlineto{\pgfqpoint{4.891530in}{1.667049in}}%
\pgfpathlineto{\pgfqpoint{4.895965in}{1.658178in}}%
\pgfpathlineto{\pgfqpoint{4.851613in}{1.671484in}}%
\pgfpathlineto{\pgfqpoint{4.895965in}{1.684789in}}%
\pgfpathlineto{\pgfqpoint{4.891530in}{1.675919in}}%
\pgfpathlineto{\pgfqpoint{4.960000in}{1.675919in}}%
\pgfpathlineto{\pgfqpoint{4.960000in}{1.667049in}}%
\pgfusepath{fill}%
\end{pgfscope}%
\begin{pgfscope}%
\pgfpathrectangle{\pgfqpoint{1.432000in}{0.528000in}}{\pgfqpoint{3.696000in}{3.696000in}} %
\pgfusepath{clip}%
\pgfsetbuttcap%
\pgfsetroundjoin%
\definecolor{currentfill}{rgb}{0.120092,0.600104,0.542530}%
\pgfsetfillcolor{currentfill}%
\pgfsetlinewidth{0.000000pt}%
\definecolor{currentstroke}{rgb}{0.000000,0.000000,0.000000}%
\pgfsetstrokecolor{currentstroke}%
\pgfsetdash{}{0pt}%
\pgfpathmoveto{\pgfqpoint{4.964435in}{1.671484in}}%
\pgfpathlineto{\pgfqpoint{4.962218in}{1.675325in}}%
\pgfpathlineto{\pgfqpoint{4.957782in}{1.675325in}}%
\pgfpathlineto{\pgfqpoint{4.955565in}{1.671484in}}%
\pgfpathlineto{\pgfqpoint{4.957782in}{1.667643in}}%
\pgfpathlineto{\pgfqpoint{4.962218in}{1.667643in}}%
\pgfpathlineto{\pgfqpoint{4.964435in}{1.671484in}}%
\pgfpathlineto{\pgfqpoint{4.962218in}{1.675325in}}%
\pgfusepath{fill}%
\end{pgfscope}%
\begin{pgfscope}%
\pgfpathrectangle{\pgfqpoint{1.432000in}{0.528000in}}{\pgfqpoint{3.696000in}{3.696000in}} %
\pgfusepath{clip}%
\pgfsetbuttcap%
\pgfsetroundjoin%
\definecolor{currentfill}{rgb}{0.274952,0.037752,0.364543}%
\pgfsetfillcolor{currentfill}%
\pgfsetlinewidth{0.000000pt}%
\definecolor{currentstroke}{rgb}{0.000000,0.000000,0.000000}%
\pgfsetstrokecolor{currentstroke}%
\pgfsetdash{}{0pt}%
\pgfpathmoveto{\pgfqpoint{1.604435in}{1.779871in}}%
\pgfpathlineto{\pgfqpoint{1.604435in}{1.603014in}}%
\pgfpathlineto{\pgfqpoint{1.613306in}{1.607449in}}%
\pgfpathlineto{\pgfqpoint{1.600000in}{1.563097in}}%
\pgfpathlineto{\pgfqpoint{1.586694in}{1.607449in}}%
\pgfpathlineto{\pgfqpoint{1.595565in}{1.603014in}}%
\pgfpathlineto{\pgfqpoint{1.595565in}{1.779871in}}%
\pgfpathlineto{\pgfqpoint{1.604435in}{1.779871in}}%
\pgfusepath{fill}%
\end{pgfscope}%
\begin{pgfscope}%
\pgfpathrectangle{\pgfqpoint{1.432000in}{0.528000in}}{\pgfqpoint{3.696000in}{3.696000in}} %
\pgfusepath{clip}%
\pgfsetbuttcap%
\pgfsetroundjoin%
\definecolor{currentfill}{rgb}{0.177423,0.437527,0.557565}%
\pgfsetfillcolor{currentfill}%
\pgfsetlinewidth{0.000000pt}%
\definecolor{currentstroke}{rgb}{0.000000,0.000000,0.000000}%
\pgfsetstrokecolor{currentstroke}%
\pgfsetdash{}{0pt}%
\pgfpathmoveto{\pgfqpoint{1.604435in}{1.779871in}}%
\pgfpathlineto{\pgfqpoint{1.604435in}{1.711401in}}%
\pgfpathlineto{\pgfqpoint{1.613306in}{1.715836in}}%
\pgfpathlineto{\pgfqpoint{1.600000in}{1.671484in}}%
\pgfpathlineto{\pgfqpoint{1.586694in}{1.715836in}}%
\pgfpathlineto{\pgfqpoint{1.595565in}{1.711401in}}%
\pgfpathlineto{\pgfqpoint{1.595565in}{1.779871in}}%
\pgfpathlineto{\pgfqpoint{1.604435in}{1.779871in}}%
\pgfusepath{fill}%
\end{pgfscope}%
\begin{pgfscope}%
\pgfpathrectangle{\pgfqpoint{1.432000in}{0.528000in}}{\pgfqpoint{3.696000in}{3.696000in}} %
\pgfusepath{clip}%
\pgfsetbuttcap%
\pgfsetroundjoin%
\definecolor{currentfill}{rgb}{0.267004,0.004874,0.329415}%
\pgfsetfillcolor{currentfill}%
\pgfsetlinewidth{0.000000pt}%
\definecolor{currentstroke}{rgb}{0.000000,0.000000,0.000000}%
\pgfsetstrokecolor{currentstroke}%
\pgfsetdash{}{0pt}%
\pgfpathmoveto{\pgfqpoint{1.712354in}{1.777887in}}%
\pgfpathlineto{\pgfqpoint{1.621818in}{1.596816in}}%
\pgfpathlineto{\pgfqpoint{1.631736in}{1.596816in}}%
\pgfpathlineto{\pgfqpoint{1.600000in}{1.563097in}}%
\pgfpathlineto{\pgfqpoint{1.607934in}{1.608717in}}%
\pgfpathlineto{\pgfqpoint{1.613884in}{1.600783in}}%
\pgfpathlineto{\pgfqpoint{1.704420in}{1.781854in}}%
\pgfpathlineto{\pgfqpoint{1.712354in}{1.777887in}}%
\pgfusepath{fill}%
\end{pgfscope}%
\begin{pgfscope}%
\pgfpathrectangle{\pgfqpoint{1.432000in}{0.528000in}}{\pgfqpoint{3.696000in}{3.696000in}} %
\pgfusepath{clip}%
\pgfsetbuttcap%
\pgfsetroundjoin%
\definecolor{currentfill}{rgb}{0.199430,0.387607,0.554642}%
\pgfsetfillcolor{currentfill}%
\pgfsetlinewidth{0.000000pt}%
\definecolor{currentstroke}{rgb}{0.000000,0.000000,0.000000}%
\pgfsetstrokecolor{currentstroke}%
\pgfsetdash{}{0pt}%
\pgfpathmoveto{\pgfqpoint{1.711523in}{1.776735in}}%
\pgfpathlineto{\pgfqpoint{1.631362in}{1.696573in}}%
\pgfpathlineto{\pgfqpoint{1.640770in}{1.693437in}}%
\pgfpathlineto{\pgfqpoint{1.600000in}{1.671484in}}%
\pgfpathlineto{\pgfqpoint{1.621953in}{1.712254in}}%
\pgfpathlineto{\pgfqpoint{1.625089in}{1.702845in}}%
\pgfpathlineto{\pgfqpoint{1.705251in}{1.783007in}}%
\pgfpathlineto{\pgfqpoint{1.711523in}{1.776735in}}%
\pgfusepath{fill}%
\end{pgfscope}%
\begin{pgfscope}%
\pgfpathrectangle{\pgfqpoint{1.432000in}{0.528000in}}{\pgfqpoint{3.696000in}{3.696000in}} %
\pgfusepath{clip}%
\pgfsetbuttcap%
\pgfsetroundjoin%
\definecolor{currentfill}{rgb}{0.135066,0.544853,0.554029}%
\pgfsetfillcolor{currentfill}%
\pgfsetlinewidth{0.000000pt}%
\definecolor{currentstroke}{rgb}{0.000000,0.000000,0.000000}%
\pgfsetstrokecolor{currentstroke}%
\pgfsetdash{}{0pt}%
\pgfpathmoveto{\pgfqpoint{1.819910in}{1.776735in}}%
\pgfpathlineto{\pgfqpoint{1.739749in}{1.696573in}}%
\pgfpathlineto{\pgfqpoint{1.749157in}{1.693437in}}%
\pgfpathlineto{\pgfqpoint{1.708387in}{1.671484in}}%
\pgfpathlineto{\pgfqpoint{1.730340in}{1.712254in}}%
\pgfpathlineto{\pgfqpoint{1.733476in}{1.702845in}}%
\pgfpathlineto{\pgfqpoint{1.813638in}{1.783007in}}%
\pgfpathlineto{\pgfqpoint{1.819910in}{1.776735in}}%
\pgfusepath{fill}%
\end{pgfscope}%
\begin{pgfscope}%
\pgfpathrectangle{\pgfqpoint{1.432000in}{0.528000in}}{\pgfqpoint{3.696000in}{3.696000in}} %
\pgfusepath{clip}%
\pgfsetbuttcap%
\pgfsetroundjoin%
\definecolor{currentfill}{rgb}{0.120092,0.600104,0.542530}%
\pgfsetfillcolor{currentfill}%
\pgfsetlinewidth{0.000000pt}%
\definecolor{currentstroke}{rgb}{0.000000,0.000000,0.000000}%
\pgfsetstrokecolor{currentstroke}%
\pgfsetdash{}{0pt}%
\pgfpathmoveto{\pgfqpoint{1.928297in}{1.776735in}}%
\pgfpathlineto{\pgfqpoint{1.848136in}{1.696573in}}%
\pgfpathlineto{\pgfqpoint{1.857544in}{1.693437in}}%
\pgfpathlineto{\pgfqpoint{1.816774in}{1.671484in}}%
\pgfpathlineto{\pgfqpoint{1.838727in}{1.712254in}}%
\pgfpathlineto{\pgfqpoint{1.841863in}{1.702845in}}%
\pgfpathlineto{\pgfqpoint{1.922025in}{1.783007in}}%
\pgfpathlineto{\pgfqpoint{1.928297in}{1.776735in}}%
\pgfusepath{fill}%
\end{pgfscope}%
\begin{pgfscope}%
\pgfpathrectangle{\pgfqpoint{1.432000in}{0.528000in}}{\pgfqpoint{3.696000in}{3.696000in}} %
\pgfusepath{clip}%
\pgfsetbuttcap%
\pgfsetroundjoin%
\definecolor{currentfill}{rgb}{0.129933,0.559582,0.551864}%
\pgfsetfillcolor{currentfill}%
\pgfsetlinewidth{0.000000pt}%
\definecolor{currentstroke}{rgb}{0.000000,0.000000,0.000000}%
\pgfsetstrokecolor{currentstroke}%
\pgfsetdash{}{0pt}%
\pgfpathmoveto{\pgfqpoint{2.036685in}{1.776735in}}%
\pgfpathlineto{\pgfqpoint{1.956523in}{1.696573in}}%
\pgfpathlineto{\pgfqpoint{1.965931in}{1.693437in}}%
\pgfpathlineto{\pgfqpoint{1.925161in}{1.671484in}}%
\pgfpathlineto{\pgfqpoint{1.947114in}{1.712254in}}%
\pgfpathlineto{\pgfqpoint{1.950251in}{1.702845in}}%
\pgfpathlineto{\pgfqpoint{2.030412in}{1.783007in}}%
\pgfpathlineto{\pgfqpoint{2.036685in}{1.776735in}}%
\pgfusepath{fill}%
\end{pgfscope}%
\begin{pgfscope}%
\pgfpathrectangle{\pgfqpoint{1.432000in}{0.528000in}}{\pgfqpoint{3.696000in}{3.696000in}} %
\pgfusepath{clip}%
\pgfsetbuttcap%
\pgfsetroundjoin%
\definecolor{currentfill}{rgb}{0.225863,0.330805,0.547314}%
\pgfsetfillcolor{currentfill}%
\pgfsetlinewidth{0.000000pt}%
\definecolor{currentstroke}{rgb}{0.000000,0.000000,0.000000}%
\pgfsetstrokecolor{currentstroke}%
\pgfsetdash{}{0pt}%
\pgfpathmoveto{\pgfqpoint{2.145072in}{1.776735in}}%
\pgfpathlineto{\pgfqpoint{2.064910in}{1.696573in}}%
\pgfpathlineto{\pgfqpoint{2.074318in}{1.693437in}}%
\pgfpathlineto{\pgfqpoint{2.033548in}{1.671484in}}%
\pgfpathlineto{\pgfqpoint{2.055502in}{1.712254in}}%
\pgfpathlineto{\pgfqpoint{2.058638in}{1.702845in}}%
\pgfpathlineto{\pgfqpoint{2.138799in}{1.783007in}}%
\pgfpathlineto{\pgfqpoint{2.145072in}{1.776735in}}%
\pgfusepath{fill}%
\end{pgfscope}%
\begin{pgfscope}%
\pgfpathrectangle{\pgfqpoint{1.432000in}{0.528000in}}{\pgfqpoint{3.696000in}{3.696000in}} %
\pgfusepath{clip}%
\pgfsetbuttcap%
\pgfsetroundjoin%
\definecolor{currentfill}{rgb}{0.274128,0.199721,0.498911}%
\pgfsetfillcolor{currentfill}%
\pgfsetlinewidth{0.000000pt}%
\definecolor{currentstroke}{rgb}{0.000000,0.000000,0.000000}%
\pgfsetstrokecolor{currentstroke}%
\pgfsetdash{}{0pt}%
\pgfpathmoveto{\pgfqpoint{2.141935in}{1.775436in}}%
\pgfpathlineto{\pgfqpoint{2.073465in}{1.775436in}}%
\pgfpathlineto{\pgfqpoint{2.077900in}{1.766565in}}%
\pgfpathlineto{\pgfqpoint{2.033548in}{1.779871in}}%
\pgfpathlineto{\pgfqpoint{2.077900in}{1.793177in}}%
\pgfpathlineto{\pgfqpoint{2.073465in}{1.784306in}}%
\pgfpathlineto{\pgfqpoint{2.141935in}{1.784306in}}%
\pgfpathlineto{\pgfqpoint{2.141935in}{1.775436in}}%
\pgfusepath{fill}%
\end{pgfscope}%
\begin{pgfscope}%
\pgfpathrectangle{\pgfqpoint{1.432000in}{0.528000in}}{\pgfqpoint{3.696000in}{3.696000in}} %
\pgfusepath{clip}%
\pgfsetbuttcap%
\pgfsetroundjoin%
\definecolor{currentfill}{rgb}{0.282623,0.140926,0.457517}%
\pgfsetfillcolor{currentfill}%
\pgfsetlinewidth{0.000000pt}%
\definecolor{currentstroke}{rgb}{0.000000,0.000000,0.000000}%
\pgfsetstrokecolor{currentstroke}%
\pgfsetdash{}{0pt}%
\pgfpathmoveto{\pgfqpoint{2.250323in}{1.775436in}}%
\pgfpathlineto{\pgfqpoint{2.181852in}{1.775436in}}%
\pgfpathlineto{\pgfqpoint{2.186287in}{1.766565in}}%
\pgfpathlineto{\pgfqpoint{2.141935in}{1.779871in}}%
\pgfpathlineto{\pgfqpoint{2.186287in}{1.793177in}}%
\pgfpathlineto{\pgfqpoint{2.181852in}{1.784306in}}%
\pgfpathlineto{\pgfqpoint{2.250323in}{1.784306in}}%
\pgfpathlineto{\pgfqpoint{2.250323in}{1.775436in}}%
\pgfusepath{fill}%
\end{pgfscope}%
\begin{pgfscope}%
\pgfpathrectangle{\pgfqpoint{1.432000in}{0.528000in}}{\pgfqpoint{3.696000in}{3.696000in}} %
\pgfusepath{clip}%
\pgfsetbuttcap%
\pgfsetroundjoin%
\definecolor{currentfill}{rgb}{0.273809,0.031497,0.358853}%
\pgfsetfillcolor{currentfill}%
\pgfsetlinewidth{0.000000pt}%
\definecolor{currentstroke}{rgb}{0.000000,0.000000,0.000000}%
\pgfsetstrokecolor{currentstroke}%
\pgfsetdash{}{0pt}%
\pgfpathmoveto{\pgfqpoint{2.360693in}{1.775904in}}%
\pgfpathlineto{\pgfqpoint{2.179622in}{1.685368in}}%
\pgfpathlineto{\pgfqpoint{2.187556in}{1.679418in}}%
\pgfpathlineto{\pgfqpoint{2.141935in}{1.671484in}}%
\pgfpathlineto{\pgfqpoint{2.175655in}{1.703220in}}%
\pgfpathlineto{\pgfqpoint{2.175655in}{1.693302in}}%
\pgfpathlineto{\pgfqpoint{2.356726in}{1.783838in}}%
\pgfpathlineto{\pgfqpoint{2.360693in}{1.775904in}}%
\pgfusepath{fill}%
\end{pgfscope}%
\begin{pgfscope}%
\pgfpathrectangle{\pgfqpoint{1.432000in}{0.528000in}}{\pgfqpoint{3.696000in}{3.696000in}} %
\pgfusepath{clip}%
\pgfsetbuttcap%
\pgfsetroundjoin%
\definecolor{currentfill}{rgb}{0.283187,0.125848,0.444960}%
\pgfsetfillcolor{currentfill}%
\pgfsetlinewidth{0.000000pt}%
\definecolor{currentstroke}{rgb}{0.000000,0.000000,0.000000}%
\pgfsetstrokecolor{currentstroke}%
\pgfsetdash{}{0pt}%
\pgfpathmoveto{\pgfqpoint{2.358710in}{1.775436in}}%
\pgfpathlineto{\pgfqpoint{2.181852in}{1.775436in}}%
\pgfpathlineto{\pgfqpoint{2.186287in}{1.766565in}}%
\pgfpathlineto{\pgfqpoint{2.141935in}{1.779871in}}%
\pgfpathlineto{\pgfqpoint{2.186287in}{1.793177in}}%
\pgfpathlineto{\pgfqpoint{2.181852in}{1.784306in}}%
\pgfpathlineto{\pgfqpoint{2.358710in}{1.784306in}}%
\pgfpathlineto{\pgfqpoint{2.358710in}{1.775436in}}%
\pgfusepath{fill}%
\end{pgfscope}%
\begin{pgfscope}%
\pgfpathrectangle{\pgfqpoint{1.432000in}{0.528000in}}{\pgfqpoint{3.696000in}{3.696000in}} %
\pgfusepath{clip}%
\pgfsetbuttcap%
\pgfsetroundjoin%
\definecolor{currentfill}{rgb}{0.283072,0.130895,0.449241}%
\pgfsetfillcolor{currentfill}%
\pgfsetlinewidth{0.000000pt}%
\definecolor{currentstroke}{rgb}{0.000000,0.000000,0.000000}%
\pgfsetstrokecolor{currentstroke}%
\pgfsetdash{}{0pt}%
\pgfpathmoveto{\pgfqpoint{2.469080in}{1.775904in}}%
\pgfpathlineto{\pgfqpoint{2.288009in}{1.685368in}}%
\pgfpathlineto{\pgfqpoint{2.295943in}{1.679418in}}%
\pgfpathlineto{\pgfqpoint{2.250323in}{1.671484in}}%
\pgfpathlineto{\pgfqpoint{2.284042in}{1.703220in}}%
\pgfpathlineto{\pgfqpoint{2.284042in}{1.693302in}}%
\pgfpathlineto{\pgfqpoint{2.465113in}{1.783838in}}%
\pgfpathlineto{\pgfqpoint{2.469080in}{1.775904in}}%
\pgfusepath{fill}%
\end{pgfscope}%
\begin{pgfscope}%
\pgfpathrectangle{\pgfqpoint{1.432000in}{0.528000in}}{\pgfqpoint{3.696000in}{3.696000in}} %
\pgfusepath{clip}%
\pgfsetbuttcap%
\pgfsetroundjoin%
\definecolor{currentfill}{rgb}{0.225863,0.330805,0.547314}%
\pgfsetfillcolor{currentfill}%
\pgfsetlinewidth{0.000000pt}%
\definecolor{currentstroke}{rgb}{0.000000,0.000000,0.000000}%
\pgfsetstrokecolor{currentstroke}%
\pgfsetdash{}{0pt}%
\pgfpathmoveto{\pgfqpoint{2.467097in}{1.775436in}}%
\pgfpathlineto{\pgfqpoint{2.290239in}{1.775436in}}%
\pgfpathlineto{\pgfqpoint{2.294675in}{1.766565in}}%
\pgfpathlineto{\pgfqpoint{2.250323in}{1.779871in}}%
\pgfpathlineto{\pgfqpoint{2.294675in}{1.793177in}}%
\pgfpathlineto{\pgfqpoint{2.290239in}{1.784306in}}%
\pgfpathlineto{\pgfqpoint{2.467097in}{1.784306in}}%
\pgfpathlineto{\pgfqpoint{2.467097in}{1.775436in}}%
\pgfusepath{fill}%
\end{pgfscope}%
\begin{pgfscope}%
\pgfpathrectangle{\pgfqpoint{1.432000in}{0.528000in}}{\pgfqpoint{3.696000in}{3.696000in}} %
\pgfusepath{clip}%
\pgfsetbuttcap%
\pgfsetroundjoin%
\definecolor{currentfill}{rgb}{0.127568,0.566949,0.550556}%
\pgfsetfillcolor{currentfill}%
\pgfsetlinewidth{0.000000pt}%
\definecolor{currentstroke}{rgb}{0.000000,0.000000,0.000000}%
\pgfsetstrokecolor{currentstroke}%
\pgfsetdash{}{0pt}%
\pgfpathmoveto{\pgfqpoint{2.575484in}{1.775436in}}%
\pgfpathlineto{\pgfqpoint{2.398626in}{1.775436in}}%
\pgfpathlineto{\pgfqpoint{2.403062in}{1.766565in}}%
\pgfpathlineto{\pgfqpoint{2.358710in}{1.779871in}}%
\pgfpathlineto{\pgfqpoint{2.403062in}{1.793177in}}%
\pgfpathlineto{\pgfqpoint{2.398626in}{1.784306in}}%
\pgfpathlineto{\pgfqpoint{2.575484in}{1.784306in}}%
\pgfpathlineto{\pgfqpoint{2.575484in}{1.775436in}}%
\pgfusepath{fill}%
\end{pgfscope}%
\begin{pgfscope}%
\pgfpathrectangle{\pgfqpoint{1.432000in}{0.528000in}}{\pgfqpoint{3.696000in}{3.696000in}} %
\pgfusepath{clip}%
\pgfsetbuttcap%
\pgfsetroundjoin%
\definecolor{currentfill}{rgb}{0.269944,0.014625,0.341379}%
\pgfsetfillcolor{currentfill}%
\pgfsetlinewidth{0.000000pt}%
\definecolor{currentstroke}{rgb}{0.000000,0.000000,0.000000}%
\pgfsetstrokecolor{currentstroke}%
\pgfsetdash{}{0pt}%
\pgfpathmoveto{\pgfqpoint{2.683871in}{1.775436in}}%
\pgfpathlineto{\pgfqpoint{2.398626in}{1.775436in}}%
\pgfpathlineto{\pgfqpoint{2.403062in}{1.766565in}}%
\pgfpathlineto{\pgfqpoint{2.358710in}{1.779871in}}%
\pgfpathlineto{\pgfqpoint{2.403062in}{1.793177in}}%
\pgfpathlineto{\pgfqpoint{2.398626in}{1.784306in}}%
\pgfpathlineto{\pgfqpoint{2.683871in}{1.784306in}}%
\pgfpathlineto{\pgfqpoint{2.683871in}{1.775436in}}%
\pgfusepath{fill}%
\end{pgfscope}%
\begin{pgfscope}%
\pgfpathrectangle{\pgfqpoint{1.432000in}{0.528000in}}{\pgfqpoint{3.696000in}{3.696000in}} %
\pgfusepath{clip}%
\pgfsetbuttcap%
\pgfsetroundjoin%
\definecolor{currentfill}{rgb}{0.150476,0.504369,0.557430}%
\pgfsetfillcolor{currentfill}%
\pgfsetlinewidth{0.000000pt}%
\definecolor{currentstroke}{rgb}{0.000000,0.000000,0.000000}%
\pgfsetstrokecolor{currentstroke}%
\pgfsetdash{}{0pt}%
\pgfpathmoveto{\pgfqpoint{2.683871in}{1.775436in}}%
\pgfpathlineto{\pgfqpoint{2.507014in}{1.775436in}}%
\pgfpathlineto{\pgfqpoint{2.511449in}{1.766565in}}%
\pgfpathlineto{\pgfqpoint{2.467097in}{1.779871in}}%
\pgfpathlineto{\pgfqpoint{2.511449in}{1.793177in}}%
\pgfpathlineto{\pgfqpoint{2.507014in}{1.784306in}}%
\pgfpathlineto{\pgfqpoint{2.683871in}{1.784306in}}%
\pgfpathlineto{\pgfqpoint{2.683871in}{1.775436in}}%
\pgfusepath{fill}%
\end{pgfscope}%
\begin{pgfscope}%
\pgfpathrectangle{\pgfqpoint{1.432000in}{0.528000in}}{\pgfqpoint{3.696000in}{3.696000in}} %
\pgfusepath{clip}%
\pgfsetbuttcap%
\pgfsetroundjoin%
\definecolor{currentfill}{rgb}{0.123444,0.636809,0.528763}%
\pgfsetfillcolor{currentfill}%
\pgfsetlinewidth{0.000000pt}%
\definecolor{currentstroke}{rgb}{0.000000,0.000000,0.000000}%
\pgfsetstrokecolor{currentstroke}%
\pgfsetdash{}{0pt}%
\pgfpathmoveto{\pgfqpoint{2.792258in}{1.775436in}}%
\pgfpathlineto{\pgfqpoint{2.615401in}{1.775436in}}%
\pgfpathlineto{\pgfqpoint{2.619836in}{1.766565in}}%
\pgfpathlineto{\pgfqpoint{2.575484in}{1.779871in}}%
\pgfpathlineto{\pgfqpoint{2.619836in}{1.793177in}}%
\pgfpathlineto{\pgfqpoint{2.615401in}{1.784306in}}%
\pgfpathlineto{\pgfqpoint{2.792258in}{1.784306in}}%
\pgfpathlineto{\pgfqpoint{2.792258in}{1.775436in}}%
\pgfusepath{fill}%
\end{pgfscope}%
\begin{pgfscope}%
\pgfpathrectangle{\pgfqpoint{1.432000in}{0.528000in}}{\pgfqpoint{3.696000in}{3.696000in}} %
\pgfusepath{clip}%
\pgfsetbuttcap%
\pgfsetroundjoin%
\definecolor{currentfill}{rgb}{0.202219,0.715272,0.476084}%
\pgfsetfillcolor{currentfill}%
\pgfsetlinewidth{0.000000pt}%
\definecolor{currentstroke}{rgb}{0.000000,0.000000,0.000000}%
\pgfsetstrokecolor{currentstroke}%
\pgfsetdash{}{0pt}%
\pgfpathmoveto{\pgfqpoint{2.900645in}{1.775436in}}%
\pgfpathlineto{\pgfqpoint{2.723788in}{1.775436in}}%
\pgfpathlineto{\pgfqpoint{2.728223in}{1.766565in}}%
\pgfpathlineto{\pgfqpoint{2.683871in}{1.779871in}}%
\pgfpathlineto{\pgfqpoint{2.728223in}{1.793177in}}%
\pgfpathlineto{\pgfqpoint{2.723788in}{1.784306in}}%
\pgfpathlineto{\pgfqpoint{2.900645in}{1.784306in}}%
\pgfpathlineto{\pgfqpoint{2.900645in}{1.775436in}}%
\pgfusepath{fill}%
\end{pgfscope}%
\begin{pgfscope}%
\pgfpathrectangle{\pgfqpoint{1.432000in}{0.528000in}}{\pgfqpoint{3.696000in}{3.696000in}} %
\pgfusepath{clip}%
\pgfsetbuttcap%
\pgfsetroundjoin%
\definecolor{currentfill}{rgb}{0.191090,0.708366,0.482284}%
\pgfsetfillcolor{currentfill}%
\pgfsetlinewidth{0.000000pt}%
\definecolor{currentstroke}{rgb}{0.000000,0.000000,0.000000}%
\pgfsetstrokecolor{currentstroke}%
\pgfsetdash{}{0pt}%
\pgfpathmoveto{\pgfqpoint{3.009032in}{1.775436in}}%
\pgfpathlineto{\pgfqpoint{2.832175in}{1.775436in}}%
\pgfpathlineto{\pgfqpoint{2.836610in}{1.766565in}}%
\pgfpathlineto{\pgfqpoint{2.792258in}{1.779871in}}%
\pgfpathlineto{\pgfqpoint{2.836610in}{1.793177in}}%
\pgfpathlineto{\pgfqpoint{2.832175in}{1.784306in}}%
\pgfpathlineto{\pgfqpoint{3.009032in}{1.784306in}}%
\pgfpathlineto{\pgfqpoint{3.009032in}{1.775436in}}%
\pgfusepath{fill}%
\end{pgfscope}%
\begin{pgfscope}%
\pgfpathrectangle{\pgfqpoint{1.432000in}{0.528000in}}{\pgfqpoint{3.696000in}{3.696000in}} %
\pgfusepath{clip}%
\pgfsetbuttcap%
\pgfsetroundjoin%
\definecolor{currentfill}{rgb}{0.282327,0.094955,0.417331}%
\pgfsetfillcolor{currentfill}%
\pgfsetlinewidth{0.000000pt}%
\definecolor{currentstroke}{rgb}{0.000000,0.000000,0.000000}%
\pgfsetstrokecolor{currentstroke}%
\pgfsetdash{}{0pt}%
\pgfpathmoveto{\pgfqpoint{3.007049in}{1.775904in}}%
\pgfpathlineto{\pgfqpoint{2.825977in}{1.866440in}}%
\pgfpathlineto{\pgfqpoint{2.825977in}{1.856522in}}%
\pgfpathlineto{\pgfqpoint{2.792258in}{1.888258in}}%
\pgfpathlineto{\pgfqpoint{2.837878in}{1.880324in}}%
\pgfpathlineto{\pgfqpoint{2.829944in}{1.874374in}}%
\pgfpathlineto{\pgfqpoint{3.011016in}{1.783838in}}%
\pgfpathlineto{\pgfqpoint{3.007049in}{1.775904in}}%
\pgfusepath{fill}%
\end{pgfscope}%
\begin{pgfscope}%
\pgfpathrectangle{\pgfqpoint{1.432000in}{0.528000in}}{\pgfqpoint{3.696000in}{3.696000in}} %
\pgfusepath{clip}%
\pgfsetbuttcap%
\pgfsetroundjoin%
\definecolor{currentfill}{rgb}{0.122312,0.633153,0.530398}%
\pgfsetfillcolor{currentfill}%
\pgfsetlinewidth{0.000000pt}%
\definecolor{currentstroke}{rgb}{0.000000,0.000000,0.000000}%
\pgfsetstrokecolor{currentstroke}%
\pgfsetdash{}{0pt}%
\pgfpathmoveto{\pgfqpoint{3.117419in}{1.775436in}}%
\pgfpathlineto{\pgfqpoint{2.940562in}{1.775436in}}%
\pgfpathlineto{\pgfqpoint{2.944997in}{1.766565in}}%
\pgfpathlineto{\pgfqpoint{2.900645in}{1.779871in}}%
\pgfpathlineto{\pgfqpoint{2.944997in}{1.793177in}}%
\pgfpathlineto{\pgfqpoint{2.940562in}{1.784306in}}%
\pgfpathlineto{\pgfqpoint{3.117419in}{1.784306in}}%
\pgfpathlineto{\pgfqpoint{3.117419in}{1.775436in}}%
\pgfusepath{fill}%
\end{pgfscope}%
\begin{pgfscope}%
\pgfpathrectangle{\pgfqpoint{1.432000in}{0.528000in}}{\pgfqpoint{3.696000in}{3.696000in}} %
\pgfusepath{clip}%
\pgfsetbuttcap%
\pgfsetroundjoin%
\definecolor{currentfill}{rgb}{0.120638,0.625828,0.533488}%
\pgfsetfillcolor{currentfill}%
\pgfsetlinewidth{0.000000pt}%
\definecolor{currentstroke}{rgb}{0.000000,0.000000,0.000000}%
\pgfsetstrokecolor{currentstroke}%
\pgfsetdash{}{0pt}%
\pgfpathmoveto{\pgfqpoint{3.225806in}{1.775436in}}%
\pgfpathlineto{\pgfqpoint{3.048949in}{1.775436in}}%
\pgfpathlineto{\pgfqpoint{3.053384in}{1.766565in}}%
\pgfpathlineto{\pgfqpoint{3.009032in}{1.779871in}}%
\pgfpathlineto{\pgfqpoint{3.053384in}{1.793177in}}%
\pgfpathlineto{\pgfqpoint{3.048949in}{1.784306in}}%
\pgfpathlineto{\pgfqpoint{3.225806in}{1.784306in}}%
\pgfpathlineto{\pgfqpoint{3.225806in}{1.775436in}}%
\pgfusepath{fill}%
\end{pgfscope}%
\begin{pgfscope}%
\pgfpathrectangle{\pgfqpoint{1.432000in}{0.528000in}}{\pgfqpoint{3.696000in}{3.696000in}} %
\pgfusepath{clip}%
\pgfsetbuttcap%
\pgfsetroundjoin%
\definecolor{currentfill}{rgb}{0.137770,0.537492,0.554906}%
\pgfsetfillcolor{currentfill}%
\pgfsetlinewidth{0.000000pt}%
\definecolor{currentstroke}{rgb}{0.000000,0.000000,0.000000}%
\pgfsetstrokecolor{currentstroke}%
\pgfsetdash{}{0pt}%
\pgfpathmoveto{\pgfqpoint{3.334194in}{1.775436in}}%
\pgfpathlineto{\pgfqpoint{3.157336in}{1.775436in}}%
\pgfpathlineto{\pgfqpoint{3.161771in}{1.766565in}}%
\pgfpathlineto{\pgfqpoint{3.117419in}{1.779871in}}%
\pgfpathlineto{\pgfqpoint{3.161771in}{1.793177in}}%
\pgfpathlineto{\pgfqpoint{3.157336in}{1.784306in}}%
\pgfpathlineto{\pgfqpoint{3.334194in}{1.784306in}}%
\pgfpathlineto{\pgfqpoint{3.334194in}{1.775436in}}%
\pgfusepath{fill}%
\end{pgfscope}%
\begin{pgfscope}%
\pgfpathrectangle{\pgfqpoint{1.432000in}{0.528000in}}{\pgfqpoint{3.696000in}{3.696000in}} %
\pgfusepath{clip}%
\pgfsetbuttcap%
\pgfsetroundjoin%
\definecolor{currentfill}{rgb}{0.280894,0.078907,0.402329}%
\pgfsetfillcolor{currentfill}%
\pgfsetlinewidth{0.000000pt}%
\definecolor{currentstroke}{rgb}{0.000000,0.000000,0.000000}%
\pgfsetstrokecolor{currentstroke}%
\pgfsetdash{}{0pt}%
\pgfpathmoveto{\pgfqpoint{3.332210in}{1.775904in}}%
\pgfpathlineto{\pgfqpoint{3.151139in}{1.866440in}}%
\pgfpathlineto{\pgfqpoint{3.151139in}{1.856522in}}%
\pgfpathlineto{\pgfqpoint{3.117419in}{1.888258in}}%
\pgfpathlineto{\pgfqpoint{3.163039in}{1.880324in}}%
\pgfpathlineto{\pgfqpoint{3.155106in}{1.874374in}}%
\pgfpathlineto{\pgfqpoint{3.336177in}{1.783838in}}%
\pgfpathlineto{\pgfqpoint{3.332210in}{1.775904in}}%
\pgfusepath{fill}%
\end{pgfscope}%
\begin{pgfscope}%
\pgfpathrectangle{\pgfqpoint{1.432000in}{0.528000in}}{\pgfqpoint{3.696000in}{3.696000in}} %
\pgfusepath{clip}%
\pgfsetbuttcap%
\pgfsetroundjoin%
\definecolor{currentfill}{rgb}{0.147607,0.511733,0.557049}%
\pgfsetfillcolor{currentfill}%
\pgfsetlinewidth{0.000000pt}%
\definecolor{currentstroke}{rgb}{0.000000,0.000000,0.000000}%
\pgfsetstrokecolor{currentstroke}%
\pgfsetdash{}{0pt}%
\pgfpathmoveto{\pgfqpoint{3.442581in}{1.775436in}}%
\pgfpathlineto{\pgfqpoint{3.265723in}{1.775436in}}%
\pgfpathlineto{\pgfqpoint{3.270158in}{1.766565in}}%
\pgfpathlineto{\pgfqpoint{3.225806in}{1.779871in}}%
\pgfpathlineto{\pgfqpoint{3.270158in}{1.793177in}}%
\pgfpathlineto{\pgfqpoint{3.265723in}{1.784306in}}%
\pgfpathlineto{\pgfqpoint{3.442581in}{1.784306in}}%
\pgfpathlineto{\pgfqpoint{3.442581in}{1.775436in}}%
\pgfusepath{fill}%
\end{pgfscope}%
\begin{pgfscope}%
\pgfpathrectangle{\pgfqpoint{1.432000in}{0.528000in}}{\pgfqpoint{3.696000in}{3.696000in}} %
\pgfusepath{clip}%
\pgfsetbuttcap%
\pgfsetroundjoin%
\definecolor{currentfill}{rgb}{0.260571,0.246922,0.522828}%
\pgfsetfillcolor{currentfill}%
\pgfsetlinewidth{0.000000pt}%
\definecolor{currentstroke}{rgb}{0.000000,0.000000,0.000000}%
\pgfsetstrokecolor{currentstroke}%
\pgfsetdash{}{0pt}%
\pgfpathmoveto{\pgfqpoint{3.550968in}{1.775436in}}%
\pgfpathlineto{\pgfqpoint{3.374110in}{1.775436in}}%
\pgfpathlineto{\pgfqpoint{3.378546in}{1.766565in}}%
\pgfpathlineto{\pgfqpoint{3.334194in}{1.779871in}}%
\pgfpathlineto{\pgfqpoint{3.378546in}{1.793177in}}%
\pgfpathlineto{\pgfqpoint{3.374110in}{1.784306in}}%
\pgfpathlineto{\pgfqpoint{3.550968in}{1.784306in}}%
\pgfpathlineto{\pgfqpoint{3.550968in}{1.775436in}}%
\pgfusepath{fill}%
\end{pgfscope}%
\begin{pgfscope}%
\pgfpathrectangle{\pgfqpoint{1.432000in}{0.528000in}}{\pgfqpoint{3.696000in}{3.696000in}} %
\pgfusepath{clip}%
\pgfsetbuttcap%
\pgfsetroundjoin%
\definecolor{currentfill}{rgb}{0.246811,0.283237,0.535941}%
\pgfsetfillcolor{currentfill}%
\pgfsetlinewidth{0.000000pt}%
\definecolor{currentstroke}{rgb}{0.000000,0.000000,0.000000}%
\pgfsetstrokecolor{currentstroke}%
\pgfsetdash{}{0pt}%
\pgfpathmoveto{\pgfqpoint{3.659355in}{1.775436in}}%
\pgfpathlineto{\pgfqpoint{3.374110in}{1.775436in}}%
\pgfpathlineto{\pgfqpoint{3.378546in}{1.766565in}}%
\pgfpathlineto{\pgfqpoint{3.334194in}{1.779871in}}%
\pgfpathlineto{\pgfqpoint{3.378546in}{1.793177in}}%
\pgfpathlineto{\pgfqpoint{3.374110in}{1.784306in}}%
\pgfpathlineto{\pgfqpoint{3.659355in}{1.784306in}}%
\pgfpathlineto{\pgfqpoint{3.659355in}{1.775436in}}%
\pgfusepath{fill}%
\end{pgfscope}%
\begin{pgfscope}%
\pgfpathrectangle{\pgfqpoint{1.432000in}{0.528000in}}{\pgfqpoint{3.696000in}{3.696000in}} %
\pgfusepath{clip}%
\pgfsetbuttcap%
\pgfsetroundjoin%
\definecolor{currentfill}{rgb}{0.231674,0.318106,0.544834}%
\pgfsetfillcolor{currentfill}%
\pgfsetlinewidth{0.000000pt}%
\definecolor{currentstroke}{rgb}{0.000000,0.000000,0.000000}%
\pgfsetstrokecolor{currentstroke}%
\pgfsetdash{}{0pt}%
\pgfpathmoveto{\pgfqpoint{3.767742in}{1.775436in}}%
\pgfpathlineto{\pgfqpoint{3.482497in}{1.775436in}}%
\pgfpathlineto{\pgfqpoint{3.486933in}{1.766565in}}%
\pgfpathlineto{\pgfqpoint{3.442581in}{1.779871in}}%
\pgfpathlineto{\pgfqpoint{3.486933in}{1.793177in}}%
\pgfpathlineto{\pgfqpoint{3.482497in}{1.784306in}}%
\pgfpathlineto{\pgfqpoint{3.767742in}{1.784306in}}%
\pgfpathlineto{\pgfqpoint{3.767742in}{1.775436in}}%
\pgfusepath{fill}%
\end{pgfscope}%
\begin{pgfscope}%
\pgfpathrectangle{\pgfqpoint{1.432000in}{0.528000in}}{\pgfqpoint{3.696000in}{3.696000in}} %
\pgfusepath{clip}%
\pgfsetbuttcap%
\pgfsetroundjoin%
\definecolor{currentfill}{rgb}{0.282623,0.140926,0.457517}%
\pgfsetfillcolor{currentfill}%
\pgfsetlinewidth{0.000000pt}%
\definecolor{currentstroke}{rgb}{0.000000,0.000000,0.000000}%
\pgfsetstrokecolor{currentstroke}%
\pgfsetdash{}{0pt}%
\pgfpathmoveto{\pgfqpoint{3.876129in}{1.775436in}}%
\pgfpathlineto{\pgfqpoint{3.482497in}{1.775436in}}%
\pgfpathlineto{\pgfqpoint{3.486933in}{1.766565in}}%
\pgfpathlineto{\pgfqpoint{3.442581in}{1.779871in}}%
\pgfpathlineto{\pgfqpoint{3.486933in}{1.793177in}}%
\pgfpathlineto{\pgfqpoint{3.482497in}{1.784306in}}%
\pgfpathlineto{\pgfqpoint{3.876129in}{1.784306in}}%
\pgfpathlineto{\pgfqpoint{3.876129in}{1.775436in}}%
\pgfusepath{fill}%
\end{pgfscope}%
\begin{pgfscope}%
\pgfpathrectangle{\pgfqpoint{1.432000in}{0.528000in}}{\pgfqpoint{3.696000in}{3.696000in}} %
\pgfusepath{clip}%
\pgfsetbuttcap%
\pgfsetroundjoin%
\definecolor{currentfill}{rgb}{0.283187,0.125848,0.444960}%
\pgfsetfillcolor{currentfill}%
\pgfsetlinewidth{0.000000pt}%
\definecolor{currentstroke}{rgb}{0.000000,0.000000,0.000000}%
\pgfsetstrokecolor{currentstroke}%
\pgfsetdash{}{0pt}%
\pgfpathmoveto{\pgfqpoint{3.985592in}{1.775568in}}%
\pgfpathlineto{\pgfqpoint{3.590768in}{1.676862in}}%
\pgfpathlineto{\pgfqpoint{3.597223in}{1.669332in}}%
\pgfpathlineto{\pgfqpoint{3.550968in}{1.671484in}}%
\pgfpathlineto{\pgfqpoint{3.590768in}{1.695149in}}%
\pgfpathlineto{\pgfqpoint{3.588617in}{1.685468in}}%
\pgfpathlineto{\pgfqpoint{3.983440in}{1.784174in}}%
\pgfpathlineto{\pgfqpoint{3.985592in}{1.775568in}}%
\pgfusepath{fill}%
\end{pgfscope}%
\begin{pgfscope}%
\pgfpathrectangle{\pgfqpoint{1.432000in}{0.528000in}}{\pgfqpoint{3.696000in}{3.696000in}} %
\pgfusepath{clip}%
\pgfsetbuttcap%
\pgfsetroundjoin%
\definecolor{currentfill}{rgb}{0.252194,0.269783,0.531579}%
\pgfsetfillcolor{currentfill}%
\pgfsetlinewidth{0.000000pt}%
\definecolor{currentstroke}{rgb}{0.000000,0.000000,0.000000}%
\pgfsetstrokecolor{currentstroke}%
\pgfsetdash{}{0pt}%
\pgfpathmoveto{\pgfqpoint{3.984516in}{1.775436in}}%
\pgfpathlineto{\pgfqpoint{3.590885in}{1.775436in}}%
\pgfpathlineto{\pgfqpoint{3.595320in}{1.766565in}}%
\pgfpathlineto{\pgfqpoint{3.550968in}{1.779871in}}%
\pgfpathlineto{\pgfqpoint{3.595320in}{1.793177in}}%
\pgfpathlineto{\pgfqpoint{3.590885in}{1.784306in}}%
\pgfpathlineto{\pgfqpoint{3.984516in}{1.784306in}}%
\pgfpathlineto{\pgfqpoint{3.984516in}{1.775436in}}%
\pgfusepath{fill}%
\end{pgfscope}%
\begin{pgfscope}%
\pgfpathrectangle{\pgfqpoint{1.432000in}{0.528000in}}{\pgfqpoint{3.696000in}{3.696000in}} %
\pgfusepath{clip}%
\pgfsetbuttcap%
\pgfsetroundjoin%
\definecolor{currentfill}{rgb}{0.262138,0.242286,0.520837}%
\pgfsetfillcolor{currentfill}%
\pgfsetlinewidth{0.000000pt}%
\definecolor{currentstroke}{rgb}{0.000000,0.000000,0.000000}%
\pgfsetstrokecolor{currentstroke}%
\pgfsetdash{}{0pt}%
\pgfpathmoveto{\pgfqpoint{4.092903in}{1.775436in}}%
\pgfpathlineto{\pgfqpoint{3.699272in}{1.775436in}}%
\pgfpathlineto{\pgfqpoint{3.703707in}{1.766565in}}%
\pgfpathlineto{\pgfqpoint{3.659355in}{1.779871in}}%
\pgfpathlineto{\pgfqpoint{3.703707in}{1.793177in}}%
\pgfpathlineto{\pgfqpoint{3.699272in}{1.784306in}}%
\pgfpathlineto{\pgfqpoint{4.092903in}{1.784306in}}%
\pgfpathlineto{\pgfqpoint{4.092903in}{1.775436in}}%
\pgfusepath{fill}%
\end{pgfscope}%
\begin{pgfscope}%
\pgfpathrectangle{\pgfqpoint{1.432000in}{0.528000in}}{\pgfqpoint{3.696000in}{3.696000in}} %
\pgfusepath{clip}%
\pgfsetbuttcap%
\pgfsetroundjoin%
\definecolor{currentfill}{rgb}{0.283091,0.110553,0.431554}%
\pgfsetfillcolor{currentfill}%
\pgfsetlinewidth{0.000000pt}%
\definecolor{currentstroke}{rgb}{0.000000,0.000000,0.000000}%
\pgfsetstrokecolor{currentstroke}%
\pgfsetdash{}{0pt}%
\pgfpathmoveto{\pgfqpoint{4.202693in}{1.775663in}}%
\pgfpathlineto{\pgfqpoint{3.915400in}{1.679899in}}%
\pgfpathlineto{\pgfqpoint{3.922413in}{1.672886in}}%
\pgfpathlineto{\pgfqpoint{3.876129in}{1.671484in}}%
\pgfpathlineto{\pgfqpoint{3.913997in}{1.698132in}}%
\pgfpathlineto{\pgfqpoint{3.912595in}{1.688314in}}%
\pgfpathlineto{\pgfqpoint{4.199888in}{1.784079in}}%
\pgfpathlineto{\pgfqpoint{4.202693in}{1.775663in}}%
\pgfusepath{fill}%
\end{pgfscope}%
\begin{pgfscope}%
\pgfpathrectangle{\pgfqpoint{1.432000in}{0.528000in}}{\pgfqpoint{3.696000in}{3.696000in}} %
\pgfusepath{clip}%
\pgfsetbuttcap%
\pgfsetroundjoin%
\definecolor{currentfill}{rgb}{0.282327,0.094955,0.417331}%
\pgfsetfillcolor{currentfill}%
\pgfsetlinewidth{0.000000pt}%
\definecolor{currentstroke}{rgb}{0.000000,0.000000,0.000000}%
\pgfsetstrokecolor{currentstroke}%
\pgfsetdash{}{0pt}%
\pgfpathmoveto{\pgfqpoint{4.201290in}{1.775436in}}%
\pgfpathlineto{\pgfqpoint{3.916046in}{1.775436in}}%
\pgfpathlineto{\pgfqpoint{3.920481in}{1.766565in}}%
\pgfpathlineto{\pgfqpoint{3.876129in}{1.779871in}}%
\pgfpathlineto{\pgfqpoint{3.920481in}{1.793177in}}%
\pgfpathlineto{\pgfqpoint{3.916046in}{1.784306in}}%
\pgfpathlineto{\pgfqpoint{4.201290in}{1.784306in}}%
\pgfpathlineto{\pgfqpoint{4.201290in}{1.775436in}}%
\pgfusepath{fill}%
\end{pgfscope}%
\begin{pgfscope}%
\pgfpathrectangle{\pgfqpoint{1.432000in}{0.528000in}}{\pgfqpoint{3.696000in}{3.696000in}} %
\pgfusepath{clip}%
\pgfsetbuttcap%
\pgfsetroundjoin%
\definecolor{currentfill}{rgb}{0.282290,0.145912,0.461510}%
\pgfsetfillcolor{currentfill}%
\pgfsetlinewidth{0.000000pt}%
\definecolor{currentstroke}{rgb}{0.000000,0.000000,0.000000}%
\pgfsetstrokecolor{currentstroke}%
\pgfsetdash{}{0pt}%
\pgfpathmoveto{\pgfqpoint{4.311080in}{1.775663in}}%
\pgfpathlineto{\pgfqpoint{4.023787in}{1.679899in}}%
\pgfpathlineto{\pgfqpoint{4.030800in}{1.672886in}}%
\pgfpathlineto{\pgfqpoint{3.984516in}{1.671484in}}%
\pgfpathlineto{\pgfqpoint{4.022385in}{1.698132in}}%
\pgfpathlineto{\pgfqpoint{4.020982in}{1.688314in}}%
\pgfpathlineto{\pgfqpoint{4.308275in}{1.784079in}}%
\pgfpathlineto{\pgfqpoint{4.311080in}{1.775663in}}%
\pgfusepath{fill}%
\end{pgfscope}%
\begin{pgfscope}%
\pgfpathrectangle{\pgfqpoint{1.432000in}{0.528000in}}{\pgfqpoint{3.696000in}{3.696000in}} %
\pgfusepath{clip}%
\pgfsetbuttcap%
\pgfsetroundjoin%
\definecolor{currentfill}{rgb}{0.250425,0.274290,0.533103}%
\pgfsetfillcolor{currentfill}%
\pgfsetlinewidth{0.000000pt}%
\definecolor{currentstroke}{rgb}{0.000000,0.000000,0.000000}%
\pgfsetstrokecolor{currentstroke}%
\pgfsetdash{}{0pt}%
\pgfpathmoveto{\pgfqpoint{4.309677in}{1.775436in}}%
\pgfpathlineto{\pgfqpoint{4.024433in}{1.775436in}}%
\pgfpathlineto{\pgfqpoint{4.028868in}{1.766565in}}%
\pgfpathlineto{\pgfqpoint{3.984516in}{1.779871in}}%
\pgfpathlineto{\pgfqpoint{4.028868in}{1.793177in}}%
\pgfpathlineto{\pgfqpoint{4.024433in}{1.784306in}}%
\pgfpathlineto{\pgfqpoint{4.309677in}{1.784306in}}%
\pgfpathlineto{\pgfqpoint{4.309677in}{1.775436in}}%
\pgfusepath{fill}%
\end{pgfscope}%
\begin{pgfscope}%
\pgfpathrectangle{\pgfqpoint{1.432000in}{0.528000in}}{\pgfqpoint{3.696000in}{3.696000in}} %
\pgfusepath{clip}%
\pgfsetbuttcap%
\pgfsetroundjoin%
\definecolor{currentfill}{rgb}{0.282656,0.100196,0.422160}%
\pgfsetfillcolor{currentfill}%
\pgfsetlinewidth{0.000000pt}%
\definecolor{currentstroke}{rgb}{0.000000,0.000000,0.000000}%
\pgfsetstrokecolor{currentstroke}%
\pgfsetdash{}{0pt}%
\pgfpathmoveto{\pgfqpoint{4.419467in}{1.775663in}}%
\pgfpathlineto{\pgfqpoint{4.132174in}{1.679899in}}%
\pgfpathlineto{\pgfqpoint{4.139187in}{1.672886in}}%
\pgfpathlineto{\pgfqpoint{4.092903in}{1.671484in}}%
\pgfpathlineto{\pgfqpoint{4.130772in}{1.698132in}}%
\pgfpathlineto{\pgfqpoint{4.129369in}{1.688314in}}%
\pgfpathlineto{\pgfqpoint{4.416662in}{1.784079in}}%
\pgfpathlineto{\pgfqpoint{4.419467in}{1.775663in}}%
\pgfusepath{fill}%
\end{pgfscope}%
\begin{pgfscope}%
\pgfpathrectangle{\pgfqpoint{1.432000in}{0.528000in}}{\pgfqpoint{3.696000in}{3.696000in}} %
\pgfusepath{clip}%
\pgfsetbuttcap%
\pgfsetroundjoin%
\definecolor{currentfill}{rgb}{0.274952,0.037752,0.364543}%
\pgfsetfillcolor{currentfill}%
\pgfsetlinewidth{0.000000pt}%
\definecolor{currentstroke}{rgb}{0.000000,0.000000,0.000000}%
\pgfsetstrokecolor{currentstroke}%
\pgfsetdash{}{0pt}%
\pgfpathmoveto{\pgfqpoint{4.420048in}{1.775904in}}%
\pgfpathlineto{\pgfqpoint{4.238976in}{1.685368in}}%
\pgfpathlineto{\pgfqpoint{4.246910in}{1.679418in}}%
\pgfpathlineto{\pgfqpoint{4.201290in}{1.671484in}}%
\pgfpathlineto{\pgfqpoint{4.235010in}{1.703220in}}%
\pgfpathlineto{\pgfqpoint{4.235010in}{1.693302in}}%
\pgfpathlineto{\pgfqpoint{4.416081in}{1.783838in}}%
\pgfpathlineto{\pgfqpoint{4.420048in}{1.775904in}}%
\pgfusepath{fill}%
\end{pgfscope}%
\begin{pgfscope}%
\pgfpathrectangle{\pgfqpoint{1.432000in}{0.528000in}}{\pgfqpoint{3.696000in}{3.696000in}} %
\pgfusepath{clip}%
\pgfsetbuttcap%
\pgfsetroundjoin%
\definecolor{currentfill}{rgb}{0.283197,0.115680,0.436115}%
\pgfsetfillcolor{currentfill}%
\pgfsetlinewidth{0.000000pt}%
\definecolor{currentstroke}{rgb}{0.000000,0.000000,0.000000}%
\pgfsetstrokecolor{currentstroke}%
\pgfsetdash{}{0pt}%
\pgfpathmoveto{\pgfqpoint{4.418065in}{1.775436in}}%
\pgfpathlineto{\pgfqpoint{4.132820in}{1.775436in}}%
\pgfpathlineto{\pgfqpoint{4.137255in}{1.766565in}}%
\pgfpathlineto{\pgfqpoint{4.092903in}{1.779871in}}%
\pgfpathlineto{\pgfqpoint{4.137255in}{1.793177in}}%
\pgfpathlineto{\pgfqpoint{4.132820in}{1.784306in}}%
\pgfpathlineto{\pgfqpoint{4.418065in}{1.784306in}}%
\pgfpathlineto{\pgfqpoint{4.418065in}{1.775436in}}%
\pgfusepath{fill}%
\end{pgfscope}%
\begin{pgfscope}%
\pgfpathrectangle{\pgfqpoint{1.432000in}{0.528000in}}{\pgfqpoint{3.696000in}{3.696000in}} %
\pgfusepath{clip}%
\pgfsetbuttcap%
\pgfsetroundjoin%
\definecolor{currentfill}{rgb}{0.277018,0.050344,0.375715}%
\pgfsetfillcolor{currentfill}%
\pgfsetlinewidth{0.000000pt}%
\definecolor{currentstroke}{rgb}{0.000000,0.000000,0.000000}%
\pgfsetstrokecolor{currentstroke}%
\pgfsetdash{}{0pt}%
\pgfpathmoveto{\pgfqpoint{4.528435in}{1.775904in}}%
\pgfpathlineto{\pgfqpoint{4.347364in}{1.685368in}}%
\pgfpathlineto{\pgfqpoint{4.355297in}{1.679418in}}%
\pgfpathlineto{\pgfqpoint{4.309677in}{1.671484in}}%
\pgfpathlineto{\pgfqpoint{4.343397in}{1.703220in}}%
\pgfpathlineto{\pgfqpoint{4.343397in}{1.693302in}}%
\pgfpathlineto{\pgfqpoint{4.524468in}{1.783838in}}%
\pgfpathlineto{\pgfqpoint{4.528435in}{1.775904in}}%
\pgfusepath{fill}%
\end{pgfscope}%
\begin{pgfscope}%
\pgfpathrectangle{\pgfqpoint{1.432000in}{0.528000in}}{\pgfqpoint{3.696000in}{3.696000in}} %
\pgfusepath{clip}%
\pgfsetbuttcap%
\pgfsetroundjoin%
\definecolor{currentfill}{rgb}{0.277941,0.056324,0.381191}%
\pgfsetfillcolor{currentfill}%
\pgfsetlinewidth{0.000000pt}%
\definecolor{currentstroke}{rgb}{0.000000,0.000000,0.000000}%
\pgfsetstrokecolor{currentstroke}%
\pgfsetdash{}{0pt}%
\pgfpathmoveto{\pgfqpoint{4.529588in}{1.776735in}}%
\pgfpathlineto{\pgfqpoint{4.449426in}{1.696573in}}%
\pgfpathlineto{\pgfqpoint{4.458835in}{1.693437in}}%
\pgfpathlineto{\pgfqpoint{4.418065in}{1.671484in}}%
\pgfpathlineto{\pgfqpoint{4.440018in}{1.712254in}}%
\pgfpathlineto{\pgfqpoint{4.443154in}{1.702845in}}%
\pgfpathlineto{\pgfqpoint{4.523315in}{1.783007in}}%
\pgfpathlineto{\pgfqpoint{4.529588in}{1.776735in}}%
\pgfusepath{fill}%
\end{pgfscope}%
\begin{pgfscope}%
\pgfpathrectangle{\pgfqpoint{1.432000in}{0.528000in}}{\pgfqpoint{3.696000in}{3.696000in}} %
\pgfusepath{clip}%
\pgfsetbuttcap%
\pgfsetroundjoin%
\definecolor{currentfill}{rgb}{0.283091,0.110553,0.431554}%
\pgfsetfillcolor{currentfill}%
\pgfsetlinewidth{0.000000pt}%
\definecolor{currentstroke}{rgb}{0.000000,0.000000,0.000000}%
\pgfsetstrokecolor{currentstroke}%
\pgfsetdash{}{0pt}%
\pgfpathmoveto{\pgfqpoint{4.526452in}{1.775436in}}%
\pgfpathlineto{\pgfqpoint{4.349594in}{1.775436in}}%
\pgfpathlineto{\pgfqpoint{4.354029in}{1.766565in}}%
\pgfpathlineto{\pgfqpoint{4.309677in}{1.779871in}}%
\pgfpathlineto{\pgfqpoint{4.354029in}{1.793177in}}%
\pgfpathlineto{\pgfqpoint{4.349594in}{1.784306in}}%
\pgfpathlineto{\pgfqpoint{4.526452in}{1.784306in}}%
\pgfpathlineto{\pgfqpoint{4.526452in}{1.775436in}}%
\pgfusepath{fill}%
\end{pgfscope}%
\begin{pgfscope}%
\pgfpathrectangle{\pgfqpoint{1.432000in}{0.528000in}}{\pgfqpoint{3.696000in}{3.696000in}} %
\pgfusepath{clip}%
\pgfsetbuttcap%
\pgfsetroundjoin%
\definecolor{currentfill}{rgb}{0.274128,0.199721,0.498911}%
\pgfsetfillcolor{currentfill}%
\pgfsetlinewidth{0.000000pt}%
\definecolor{currentstroke}{rgb}{0.000000,0.000000,0.000000}%
\pgfsetstrokecolor{currentstroke}%
\pgfsetdash{}{0pt}%
\pgfpathmoveto{\pgfqpoint{4.637975in}{1.776735in}}%
\pgfpathlineto{\pgfqpoint{4.557813in}{1.696573in}}%
\pgfpathlineto{\pgfqpoint{4.567222in}{1.693437in}}%
\pgfpathlineto{\pgfqpoint{4.526452in}{1.671484in}}%
\pgfpathlineto{\pgfqpoint{4.548405in}{1.712254in}}%
\pgfpathlineto{\pgfqpoint{4.551541in}{1.702845in}}%
\pgfpathlineto{\pgfqpoint{4.631703in}{1.783007in}}%
\pgfpathlineto{\pgfqpoint{4.637975in}{1.776735in}}%
\pgfusepath{fill}%
\end{pgfscope}%
\begin{pgfscope}%
\pgfpathrectangle{\pgfqpoint{1.432000in}{0.528000in}}{\pgfqpoint{3.696000in}{3.696000in}} %
\pgfusepath{clip}%
\pgfsetbuttcap%
\pgfsetroundjoin%
\definecolor{currentfill}{rgb}{0.282656,0.100196,0.422160}%
\pgfsetfillcolor{currentfill}%
\pgfsetlinewidth{0.000000pt}%
\definecolor{currentstroke}{rgb}{0.000000,0.000000,0.000000}%
\pgfsetstrokecolor{currentstroke}%
\pgfsetdash{}{0pt}%
\pgfpathmoveto{\pgfqpoint{4.634839in}{1.775436in}}%
\pgfpathlineto{\pgfqpoint{4.566368in}{1.775436in}}%
\pgfpathlineto{\pgfqpoint{4.570804in}{1.766565in}}%
\pgfpathlineto{\pgfqpoint{4.526452in}{1.779871in}}%
\pgfpathlineto{\pgfqpoint{4.570804in}{1.793177in}}%
\pgfpathlineto{\pgfqpoint{4.566368in}{1.784306in}}%
\pgfpathlineto{\pgfqpoint{4.634839in}{1.784306in}}%
\pgfpathlineto{\pgfqpoint{4.634839in}{1.775436in}}%
\pgfusepath{fill}%
\end{pgfscope}%
\begin{pgfscope}%
\pgfpathrectangle{\pgfqpoint{1.432000in}{0.528000in}}{\pgfqpoint{3.696000in}{3.696000in}} %
\pgfusepath{clip}%
\pgfsetbuttcap%
\pgfsetroundjoin%
\definecolor{currentfill}{rgb}{0.258965,0.251537,0.524736}%
\pgfsetfillcolor{currentfill}%
\pgfsetlinewidth{0.000000pt}%
\definecolor{currentstroke}{rgb}{0.000000,0.000000,0.000000}%
\pgfsetstrokecolor{currentstroke}%
\pgfsetdash{}{0pt}%
\pgfpathmoveto{\pgfqpoint{4.746362in}{1.776735in}}%
\pgfpathlineto{\pgfqpoint{4.666200in}{1.696573in}}%
\pgfpathlineto{\pgfqpoint{4.675609in}{1.693437in}}%
\pgfpathlineto{\pgfqpoint{4.634839in}{1.671484in}}%
\pgfpathlineto{\pgfqpoint{4.656792in}{1.712254in}}%
\pgfpathlineto{\pgfqpoint{4.659928in}{1.702845in}}%
\pgfpathlineto{\pgfqpoint{4.740090in}{1.783007in}}%
\pgfpathlineto{\pgfqpoint{4.746362in}{1.776735in}}%
\pgfusepath{fill}%
\end{pgfscope}%
\begin{pgfscope}%
\pgfpathrectangle{\pgfqpoint{1.432000in}{0.528000in}}{\pgfqpoint{3.696000in}{3.696000in}} %
\pgfusepath{clip}%
\pgfsetbuttcap%
\pgfsetroundjoin%
\definecolor{currentfill}{rgb}{0.270595,0.214069,0.507052}%
\pgfsetfillcolor{currentfill}%
\pgfsetlinewidth{0.000000pt}%
\definecolor{currentstroke}{rgb}{0.000000,0.000000,0.000000}%
\pgfsetstrokecolor{currentstroke}%
\pgfsetdash{}{0pt}%
\pgfpathmoveto{\pgfqpoint{4.743226in}{1.775436in}}%
\pgfpathlineto{\pgfqpoint{4.674756in}{1.775436in}}%
\pgfpathlineto{\pgfqpoint{4.679191in}{1.766565in}}%
\pgfpathlineto{\pgfqpoint{4.634839in}{1.779871in}}%
\pgfpathlineto{\pgfqpoint{4.679191in}{1.793177in}}%
\pgfpathlineto{\pgfqpoint{4.674756in}{1.784306in}}%
\pgfpathlineto{\pgfqpoint{4.743226in}{1.784306in}}%
\pgfpathlineto{\pgfqpoint{4.743226in}{1.775436in}}%
\pgfusepath{fill}%
\end{pgfscope}%
\begin{pgfscope}%
\pgfpathrectangle{\pgfqpoint{1.432000in}{0.528000in}}{\pgfqpoint{3.696000in}{3.696000in}} %
\pgfusepath{clip}%
\pgfsetbuttcap%
\pgfsetroundjoin%
\definecolor{currentfill}{rgb}{0.160665,0.478540,0.558115}%
\pgfsetfillcolor{currentfill}%
\pgfsetlinewidth{0.000000pt}%
\definecolor{currentstroke}{rgb}{0.000000,0.000000,0.000000}%
\pgfsetstrokecolor{currentstroke}%
\pgfsetdash{}{0pt}%
\pgfpathmoveto{\pgfqpoint{4.851613in}{1.775436in}}%
\pgfpathlineto{\pgfqpoint{4.783143in}{1.775436in}}%
\pgfpathlineto{\pgfqpoint{4.787578in}{1.766565in}}%
\pgfpathlineto{\pgfqpoint{4.743226in}{1.779871in}}%
\pgfpathlineto{\pgfqpoint{4.787578in}{1.793177in}}%
\pgfpathlineto{\pgfqpoint{4.783143in}{1.784306in}}%
\pgfpathlineto{\pgfqpoint{4.851613in}{1.784306in}}%
\pgfpathlineto{\pgfqpoint{4.851613in}{1.775436in}}%
\pgfusepath{fill}%
\end{pgfscope}%
\begin{pgfscope}%
\pgfpathrectangle{\pgfqpoint{1.432000in}{0.528000in}}{\pgfqpoint{3.696000in}{3.696000in}} %
\pgfusepath{clip}%
\pgfsetbuttcap%
\pgfsetroundjoin%
\definecolor{currentfill}{rgb}{0.272594,0.025563,0.353093}%
\pgfsetfillcolor{currentfill}%
\pgfsetlinewidth{0.000000pt}%
\definecolor{currentstroke}{rgb}{0.000000,0.000000,0.000000}%
\pgfsetstrokecolor{currentstroke}%
\pgfsetdash{}{0pt}%
\pgfpathmoveto{\pgfqpoint{4.856048in}{1.779871in}}%
\pgfpathlineto{\pgfqpoint{4.853831in}{1.783712in}}%
\pgfpathlineto{\pgfqpoint{4.849395in}{1.783712in}}%
\pgfpathlineto{\pgfqpoint{4.847178in}{1.779871in}}%
\pgfpathlineto{\pgfqpoint{4.849395in}{1.776030in}}%
\pgfpathlineto{\pgfqpoint{4.853831in}{1.776030in}}%
\pgfpathlineto{\pgfqpoint{4.856048in}{1.779871in}}%
\pgfpathlineto{\pgfqpoint{4.853831in}{1.783712in}}%
\pgfusepath{fill}%
\end{pgfscope}%
\begin{pgfscope}%
\pgfpathrectangle{\pgfqpoint{1.432000in}{0.528000in}}{\pgfqpoint{3.696000in}{3.696000in}} %
\pgfusepath{clip}%
\pgfsetbuttcap%
\pgfsetroundjoin%
\definecolor{currentfill}{rgb}{0.280868,0.160771,0.472899}%
\pgfsetfillcolor{currentfill}%
\pgfsetlinewidth{0.000000pt}%
\definecolor{currentstroke}{rgb}{0.000000,0.000000,0.000000}%
\pgfsetstrokecolor{currentstroke}%
\pgfsetdash{}{0pt}%
\pgfpathmoveto{\pgfqpoint{4.960000in}{1.775436in}}%
\pgfpathlineto{\pgfqpoint{4.891530in}{1.775436in}}%
\pgfpathlineto{\pgfqpoint{4.895965in}{1.766565in}}%
\pgfpathlineto{\pgfqpoint{4.851613in}{1.779871in}}%
\pgfpathlineto{\pgfqpoint{4.895965in}{1.793177in}}%
\pgfpathlineto{\pgfqpoint{4.891530in}{1.784306in}}%
\pgfpathlineto{\pgfqpoint{4.960000in}{1.784306in}}%
\pgfpathlineto{\pgfqpoint{4.960000in}{1.775436in}}%
\pgfusepath{fill}%
\end{pgfscope}%
\begin{pgfscope}%
\pgfpathrectangle{\pgfqpoint{1.432000in}{0.528000in}}{\pgfqpoint{3.696000in}{3.696000in}} %
\pgfusepath{clip}%
\pgfsetbuttcap%
\pgfsetroundjoin%
\definecolor{currentfill}{rgb}{0.162142,0.474838,0.558140}%
\pgfsetfillcolor{currentfill}%
\pgfsetlinewidth{0.000000pt}%
\definecolor{currentstroke}{rgb}{0.000000,0.000000,0.000000}%
\pgfsetstrokecolor{currentstroke}%
\pgfsetdash{}{0pt}%
\pgfpathmoveto{\pgfqpoint{4.964435in}{1.779871in}}%
\pgfpathlineto{\pgfqpoint{4.962218in}{1.783712in}}%
\pgfpathlineto{\pgfqpoint{4.957782in}{1.783712in}}%
\pgfpathlineto{\pgfqpoint{4.955565in}{1.779871in}}%
\pgfpathlineto{\pgfqpoint{4.957782in}{1.776030in}}%
\pgfpathlineto{\pgfqpoint{4.962218in}{1.776030in}}%
\pgfpathlineto{\pgfqpoint{4.964435in}{1.779871in}}%
\pgfpathlineto{\pgfqpoint{4.962218in}{1.783712in}}%
\pgfusepath{fill}%
\end{pgfscope}%
\begin{pgfscope}%
\pgfpathrectangle{\pgfqpoint{1.432000in}{0.528000in}}{\pgfqpoint{3.696000in}{3.696000in}} %
\pgfusepath{clip}%
\pgfsetbuttcap%
\pgfsetroundjoin%
\definecolor{currentfill}{rgb}{0.160665,0.478540,0.558115}%
\pgfsetfillcolor{currentfill}%
\pgfsetlinewidth{0.000000pt}%
\definecolor{currentstroke}{rgb}{0.000000,0.000000,0.000000}%
\pgfsetstrokecolor{currentstroke}%
\pgfsetdash{}{0pt}%
\pgfpathmoveto{\pgfqpoint{1.604435in}{1.888258in}}%
\pgfpathlineto{\pgfqpoint{1.604435in}{1.819788in}}%
\pgfpathlineto{\pgfqpoint{1.613306in}{1.824223in}}%
\pgfpathlineto{\pgfqpoint{1.600000in}{1.779871in}}%
\pgfpathlineto{\pgfqpoint{1.586694in}{1.824223in}}%
\pgfpathlineto{\pgfqpoint{1.595565in}{1.819788in}}%
\pgfpathlineto{\pgfqpoint{1.595565in}{1.888258in}}%
\pgfpathlineto{\pgfqpoint{1.604435in}{1.888258in}}%
\pgfusepath{fill}%
\end{pgfscope}%
\begin{pgfscope}%
\pgfpathrectangle{\pgfqpoint{1.432000in}{0.528000in}}{\pgfqpoint{3.696000in}{3.696000in}} %
\pgfusepath{clip}%
\pgfsetbuttcap%
\pgfsetroundjoin%
\definecolor{currentfill}{rgb}{0.223925,0.334994,0.548053}%
\pgfsetfillcolor{currentfill}%
\pgfsetlinewidth{0.000000pt}%
\definecolor{currentstroke}{rgb}{0.000000,0.000000,0.000000}%
\pgfsetstrokecolor{currentstroke}%
\pgfsetdash{}{0pt}%
\pgfpathmoveto{\pgfqpoint{1.604435in}{1.888258in}}%
\pgfpathlineto{\pgfqpoint{1.602218in}{1.892099in}}%
\pgfpathlineto{\pgfqpoint{1.597782in}{1.892099in}}%
\pgfpathlineto{\pgfqpoint{1.595565in}{1.888258in}}%
\pgfpathlineto{\pgfqpoint{1.597782in}{1.884417in}}%
\pgfpathlineto{\pgfqpoint{1.602218in}{1.884417in}}%
\pgfpathlineto{\pgfqpoint{1.604435in}{1.888258in}}%
\pgfpathlineto{\pgfqpoint{1.602218in}{1.892099in}}%
\pgfusepath{fill}%
\end{pgfscope}%
\begin{pgfscope}%
\pgfpathrectangle{\pgfqpoint{1.432000in}{0.528000in}}{\pgfqpoint{3.696000in}{3.696000in}} %
\pgfusepath{clip}%
\pgfsetbuttcap%
\pgfsetroundjoin%
\definecolor{currentfill}{rgb}{0.274128,0.199721,0.498911}%
\pgfsetfillcolor{currentfill}%
\pgfsetlinewidth{0.000000pt}%
\definecolor{currentstroke}{rgb}{0.000000,0.000000,0.000000}%
\pgfsetstrokecolor{currentstroke}%
\pgfsetdash{}{0pt}%
\pgfpathmoveto{\pgfqpoint{1.711523in}{1.885122in}}%
\pgfpathlineto{\pgfqpoint{1.631362in}{1.804960in}}%
\pgfpathlineto{\pgfqpoint{1.640770in}{1.801824in}}%
\pgfpathlineto{\pgfqpoint{1.600000in}{1.779871in}}%
\pgfpathlineto{\pgfqpoint{1.621953in}{1.820641in}}%
\pgfpathlineto{\pgfqpoint{1.625089in}{1.811233in}}%
\pgfpathlineto{\pgfqpoint{1.705251in}{1.891394in}}%
\pgfpathlineto{\pgfqpoint{1.711523in}{1.885122in}}%
\pgfusepath{fill}%
\end{pgfscope}%
\begin{pgfscope}%
\pgfpathrectangle{\pgfqpoint{1.432000in}{0.528000in}}{\pgfqpoint{3.696000in}{3.696000in}} %
\pgfusepath{clip}%
\pgfsetbuttcap%
\pgfsetroundjoin%
\definecolor{currentfill}{rgb}{0.273006,0.204520,0.501721}%
\pgfsetfillcolor{currentfill}%
\pgfsetlinewidth{0.000000pt}%
\definecolor{currentstroke}{rgb}{0.000000,0.000000,0.000000}%
\pgfsetstrokecolor{currentstroke}%
\pgfsetdash{}{0pt}%
\pgfpathmoveto{\pgfqpoint{1.712822in}{1.888258in}}%
\pgfpathlineto{\pgfqpoint{1.712822in}{1.819788in}}%
\pgfpathlineto{\pgfqpoint{1.721693in}{1.824223in}}%
\pgfpathlineto{\pgfqpoint{1.708387in}{1.779871in}}%
\pgfpathlineto{\pgfqpoint{1.695081in}{1.824223in}}%
\pgfpathlineto{\pgfqpoint{1.703952in}{1.819788in}}%
\pgfpathlineto{\pgfqpoint{1.703952in}{1.888258in}}%
\pgfpathlineto{\pgfqpoint{1.712822in}{1.888258in}}%
\pgfusepath{fill}%
\end{pgfscope}%
\begin{pgfscope}%
\pgfpathrectangle{\pgfqpoint{1.432000in}{0.528000in}}{\pgfqpoint{3.696000in}{3.696000in}} %
\pgfusepath{clip}%
\pgfsetbuttcap%
\pgfsetroundjoin%
\definecolor{currentfill}{rgb}{0.282656,0.100196,0.422160}%
\pgfsetfillcolor{currentfill}%
\pgfsetlinewidth{0.000000pt}%
\definecolor{currentstroke}{rgb}{0.000000,0.000000,0.000000}%
\pgfsetstrokecolor{currentstroke}%
\pgfsetdash{}{0pt}%
\pgfpathmoveto{\pgfqpoint{1.708387in}{1.883823in}}%
\pgfpathlineto{\pgfqpoint{1.639917in}{1.883823in}}%
\pgfpathlineto{\pgfqpoint{1.644352in}{1.874952in}}%
\pgfpathlineto{\pgfqpoint{1.600000in}{1.888258in}}%
\pgfpathlineto{\pgfqpoint{1.644352in}{1.901564in}}%
\pgfpathlineto{\pgfqpoint{1.639917in}{1.892693in}}%
\pgfpathlineto{\pgfqpoint{1.708387in}{1.892693in}}%
\pgfpathlineto{\pgfqpoint{1.708387in}{1.883823in}}%
\pgfusepath{fill}%
\end{pgfscope}%
\begin{pgfscope}%
\pgfpathrectangle{\pgfqpoint{1.432000in}{0.528000in}}{\pgfqpoint{3.696000in}{3.696000in}} %
\pgfusepath{clip}%
\pgfsetbuttcap%
\pgfsetroundjoin%
\definecolor{currentfill}{rgb}{0.282656,0.100196,0.422160}%
\pgfsetfillcolor{currentfill}%
\pgfsetlinewidth{0.000000pt}%
\definecolor{currentstroke}{rgb}{0.000000,0.000000,0.000000}%
\pgfsetstrokecolor{currentstroke}%
\pgfsetdash{}{0pt}%
\pgfpathmoveto{\pgfqpoint{1.712822in}{1.888258in}}%
\pgfpathlineto{\pgfqpoint{1.710605in}{1.892099in}}%
\pgfpathlineto{\pgfqpoint{1.706169in}{1.892099in}}%
\pgfpathlineto{\pgfqpoint{1.703952in}{1.888258in}}%
\pgfpathlineto{\pgfqpoint{1.706169in}{1.884417in}}%
\pgfpathlineto{\pgfqpoint{1.710605in}{1.884417in}}%
\pgfpathlineto{\pgfqpoint{1.712822in}{1.888258in}}%
\pgfpathlineto{\pgfqpoint{1.710605in}{1.892099in}}%
\pgfusepath{fill}%
\end{pgfscope}%
\begin{pgfscope}%
\pgfpathrectangle{\pgfqpoint{1.432000in}{0.528000in}}{\pgfqpoint{3.696000in}{3.696000in}} %
\pgfusepath{clip}%
\pgfsetbuttcap%
\pgfsetroundjoin%
\definecolor{currentfill}{rgb}{0.257322,0.256130,0.526563}%
\pgfsetfillcolor{currentfill}%
\pgfsetlinewidth{0.000000pt}%
\definecolor{currentstroke}{rgb}{0.000000,0.000000,0.000000}%
\pgfsetstrokecolor{currentstroke}%
\pgfsetdash{}{0pt}%
\pgfpathmoveto{\pgfqpoint{1.819910in}{1.885122in}}%
\pgfpathlineto{\pgfqpoint{1.739749in}{1.804960in}}%
\pgfpathlineto{\pgfqpoint{1.749157in}{1.801824in}}%
\pgfpathlineto{\pgfqpoint{1.708387in}{1.779871in}}%
\pgfpathlineto{\pgfqpoint{1.730340in}{1.820641in}}%
\pgfpathlineto{\pgfqpoint{1.733476in}{1.811233in}}%
\pgfpathlineto{\pgfqpoint{1.813638in}{1.891394in}}%
\pgfpathlineto{\pgfqpoint{1.819910in}{1.885122in}}%
\pgfusepath{fill}%
\end{pgfscope}%
\begin{pgfscope}%
\pgfpathrectangle{\pgfqpoint{1.432000in}{0.528000in}}{\pgfqpoint{3.696000in}{3.696000in}} %
\pgfusepath{clip}%
\pgfsetbuttcap%
\pgfsetroundjoin%
\definecolor{currentfill}{rgb}{0.267004,0.004874,0.329415}%
\pgfsetfillcolor{currentfill}%
\pgfsetlinewidth{0.000000pt}%
\definecolor{currentstroke}{rgb}{0.000000,0.000000,0.000000}%
\pgfsetstrokecolor{currentstroke}%
\pgfsetdash{}{0pt}%
\pgfpathmoveto{\pgfqpoint{1.821209in}{1.888258in}}%
\pgfpathlineto{\pgfqpoint{1.821209in}{1.819788in}}%
\pgfpathlineto{\pgfqpoint{1.830080in}{1.824223in}}%
\pgfpathlineto{\pgfqpoint{1.816774in}{1.779871in}}%
\pgfpathlineto{\pgfqpoint{1.803469in}{1.824223in}}%
\pgfpathlineto{\pgfqpoint{1.812339in}{1.819788in}}%
\pgfpathlineto{\pgfqpoint{1.812339in}{1.888258in}}%
\pgfpathlineto{\pgfqpoint{1.821209in}{1.888258in}}%
\pgfusepath{fill}%
\end{pgfscope}%
\begin{pgfscope}%
\pgfpathrectangle{\pgfqpoint{1.432000in}{0.528000in}}{\pgfqpoint{3.696000in}{3.696000in}} %
\pgfusepath{clip}%
\pgfsetbuttcap%
\pgfsetroundjoin%
\definecolor{currentfill}{rgb}{0.280868,0.160771,0.472899}%
\pgfsetfillcolor{currentfill}%
\pgfsetlinewidth{0.000000pt}%
\definecolor{currentstroke}{rgb}{0.000000,0.000000,0.000000}%
\pgfsetstrokecolor{currentstroke}%
\pgfsetdash{}{0pt}%
\pgfpathmoveto{\pgfqpoint{1.816774in}{1.883823in}}%
\pgfpathlineto{\pgfqpoint{1.748304in}{1.883823in}}%
\pgfpathlineto{\pgfqpoint{1.752739in}{1.874952in}}%
\pgfpathlineto{\pgfqpoint{1.708387in}{1.888258in}}%
\pgfpathlineto{\pgfqpoint{1.752739in}{1.901564in}}%
\pgfpathlineto{\pgfqpoint{1.748304in}{1.892693in}}%
\pgfpathlineto{\pgfqpoint{1.816774in}{1.892693in}}%
\pgfpathlineto{\pgfqpoint{1.816774in}{1.883823in}}%
\pgfusepath{fill}%
\end{pgfscope}%
\begin{pgfscope}%
\pgfpathrectangle{\pgfqpoint{1.432000in}{0.528000in}}{\pgfqpoint{3.696000in}{3.696000in}} %
\pgfusepath{clip}%
\pgfsetbuttcap%
\pgfsetroundjoin%
\definecolor{currentfill}{rgb}{0.231674,0.318106,0.544834}%
\pgfsetfillcolor{currentfill}%
\pgfsetlinewidth{0.000000pt}%
\definecolor{currentstroke}{rgb}{0.000000,0.000000,0.000000}%
\pgfsetstrokecolor{currentstroke}%
\pgfsetdash{}{0pt}%
\pgfpathmoveto{\pgfqpoint{1.928297in}{1.885122in}}%
\pgfpathlineto{\pgfqpoint{1.848136in}{1.804960in}}%
\pgfpathlineto{\pgfqpoint{1.857544in}{1.801824in}}%
\pgfpathlineto{\pgfqpoint{1.816774in}{1.779871in}}%
\pgfpathlineto{\pgfqpoint{1.838727in}{1.820641in}}%
\pgfpathlineto{\pgfqpoint{1.841863in}{1.811233in}}%
\pgfpathlineto{\pgfqpoint{1.922025in}{1.891394in}}%
\pgfpathlineto{\pgfqpoint{1.928297in}{1.885122in}}%
\pgfusepath{fill}%
\end{pgfscope}%
\begin{pgfscope}%
\pgfpathrectangle{\pgfqpoint{1.432000in}{0.528000in}}{\pgfqpoint{3.696000in}{3.696000in}} %
\pgfusepath{clip}%
\pgfsetbuttcap%
\pgfsetroundjoin%
\definecolor{currentfill}{rgb}{0.253935,0.265254,0.529983}%
\pgfsetfillcolor{currentfill}%
\pgfsetlinewidth{0.000000pt}%
\definecolor{currentstroke}{rgb}{0.000000,0.000000,0.000000}%
\pgfsetstrokecolor{currentstroke}%
\pgfsetdash{}{0pt}%
\pgfpathmoveto{\pgfqpoint{1.925161in}{1.883823in}}%
\pgfpathlineto{\pgfqpoint{1.856691in}{1.883823in}}%
\pgfpathlineto{\pgfqpoint{1.861126in}{1.874952in}}%
\pgfpathlineto{\pgfqpoint{1.816774in}{1.888258in}}%
\pgfpathlineto{\pgfqpoint{1.861126in}{1.901564in}}%
\pgfpathlineto{\pgfqpoint{1.856691in}{1.892693in}}%
\pgfpathlineto{\pgfqpoint{1.925161in}{1.892693in}}%
\pgfpathlineto{\pgfqpoint{1.925161in}{1.883823in}}%
\pgfusepath{fill}%
\end{pgfscope}%
\begin{pgfscope}%
\pgfpathrectangle{\pgfqpoint{1.432000in}{0.528000in}}{\pgfqpoint{3.696000in}{3.696000in}} %
\pgfusepath{clip}%
\pgfsetbuttcap%
\pgfsetroundjoin%
\definecolor{currentfill}{rgb}{0.273006,0.204520,0.501721}%
\pgfsetfillcolor{currentfill}%
\pgfsetlinewidth{0.000000pt}%
\definecolor{currentstroke}{rgb}{0.000000,0.000000,0.000000}%
\pgfsetstrokecolor{currentstroke}%
\pgfsetdash{}{0pt}%
\pgfpathmoveto{\pgfqpoint{2.036685in}{1.885122in}}%
\pgfpathlineto{\pgfqpoint{1.956523in}{1.804960in}}%
\pgfpathlineto{\pgfqpoint{1.965931in}{1.801824in}}%
\pgfpathlineto{\pgfqpoint{1.925161in}{1.779871in}}%
\pgfpathlineto{\pgfqpoint{1.947114in}{1.820641in}}%
\pgfpathlineto{\pgfqpoint{1.950251in}{1.811233in}}%
\pgfpathlineto{\pgfqpoint{2.030412in}{1.891394in}}%
\pgfpathlineto{\pgfqpoint{2.036685in}{1.885122in}}%
\pgfusepath{fill}%
\end{pgfscope}%
\begin{pgfscope}%
\pgfpathrectangle{\pgfqpoint{1.432000in}{0.528000in}}{\pgfqpoint{3.696000in}{3.696000in}} %
\pgfusepath{clip}%
\pgfsetbuttcap%
\pgfsetroundjoin%
\definecolor{currentfill}{rgb}{0.201239,0.383670,0.554294}%
\pgfsetfillcolor{currentfill}%
\pgfsetlinewidth{0.000000pt}%
\definecolor{currentstroke}{rgb}{0.000000,0.000000,0.000000}%
\pgfsetstrokecolor{currentstroke}%
\pgfsetdash{}{0pt}%
\pgfpathmoveto{\pgfqpoint{2.033548in}{1.883823in}}%
\pgfpathlineto{\pgfqpoint{1.965078in}{1.883823in}}%
\pgfpathlineto{\pgfqpoint{1.969513in}{1.874952in}}%
\pgfpathlineto{\pgfqpoint{1.925161in}{1.888258in}}%
\pgfpathlineto{\pgfqpoint{1.969513in}{1.901564in}}%
\pgfpathlineto{\pgfqpoint{1.965078in}{1.892693in}}%
\pgfpathlineto{\pgfqpoint{2.033548in}{1.892693in}}%
\pgfpathlineto{\pgfqpoint{2.033548in}{1.883823in}}%
\pgfusepath{fill}%
\end{pgfscope}%
\begin{pgfscope}%
\pgfpathrectangle{\pgfqpoint{1.432000in}{0.528000in}}{\pgfqpoint{3.696000in}{3.696000in}} %
\pgfusepath{clip}%
\pgfsetbuttcap%
\pgfsetroundjoin%
\definecolor{currentfill}{rgb}{0.190631,0.407061,0.556089}%
\pgfsetfillcolor{currentfill}%
\pgfsetlinewidth{0.000000pt}%
\definecolor{currentstroke}{rgb}{0.000000,0.000000,0.000000}%
\pgfsetstrokecolor{currentstroke}%
\pgfsetdash{}{0pt}%
\pgfpathmoveto{\pgfqpoint{2.141935in}{1.883823in}}%
\pgfpathlineto{\pgfqpoint{2.073465in}{1.883823in}}%
\pgfpathlineto{\pgfqpoint{2.077900in}{1.874952in}}%
\pgfpathlineto{\pgfqpoint{2.033548in}{1.888258in}}%
\pgfpathlineto{\pgfqpoint{2.077900in}{1.901564in}}%
\pgfpathlineto{\pgfqpoint{2.073465in}{1.892693in}}%
\pgfpathlineto{\pgfqpoint{2.141935in}{1.892693in}}%
\pgfpathlineto{\pgfqpoint{2.141935in}{1.883823in}}%
\pgfusepath{fill}%
\end{pgfscope}%
\begin{pgfscope}%
\pgfpathrectangle{\pgfqpoint{1.432000in}{0.528000in}}{\pgfqpoint{3.696000in}{3.696000in}} %
\pgfusepath{clip}%
\pgfsetbuttcap%
\pgfsetroundjoin%
\definecolor{currentfill}{rgb}{0.225863,0.330805,0.547314}%
\pgfsetfillcolor{currentfill}%
\pgfsetlinewidth{0.000000pt}%
\definecolor{currentstroke}{rgb}{0.000000,0.000000,0.000000}%
\pgfsetstrokecolor{currentstroke}%
\pgfsetdash{}{0pt}%
\pgfpathmoveto{\pgfqpoint{2.250323in}{1.883823in}}%
\pgfpathlineto{\pgfqpoint{2.181852in}{1.883823in}}%
\pgfpathlineto{\pgfqpoint{2.186287in}{1.874952in}}%
\pgfpathlineto{\pgfqpoint{2.141935in}{1.888258in}}%
\pgfpathlineto{\pgfqpoint{2.186287in}{1.901564in}}%
\pgfpathlineto{\pgfqpoint{2.181852in}{1.892693in}}%
\pgfpathlineto{\pgfqpoint{2.250323in}{1.892693in}}%
\pgfpathlineto{\pgfqpoint{2.250323in}{1.883823in}}%
\pgfusepath{fill}%
\end{pgfscope}%
\begin{pgfscope}%
\pgfpathrectangle{\pgfqpoint{1.432000in}{0.528000in}}{\pgfqpoint{3.696000in}{3.696000in}} %
\pgfusepath{clip}%
\pgfsetbuttcap%
\pgfsetroundjoin%
\definecolor{currentfill}{rgb}{0.265145,0.232956,0.516599}%
\pgfsetfillcolor{currentfill}%
\pgfsetlinewidth{0.000000pt}%
\definecolor{currentstroke}{rgb}{0.000000,0.000000,0.000000}%
\pgfsetstrokecolor{currentstroke}%
\pgfsetdash{}{0pt}%
\pgfpathmoveto{\pgfqpoint{2.358710in}{1.883823in}}%
\pgfpathlineto{\pgfqpoint{2.181852in}{1.883823in}}%
\pgfpathlineto{\pgfqpoint{2.186287in}{1.874952in}}%
\pgfpathlineto{\pgfqpoint{2.141935in}{1.888258in}}%
\pgfpathlineto{\pgfqpoint{2.186287in}{1.901564in}}%
\pgfpathlineto{\pgfqpoint{2.181852in}{1.892693in}}%
\pgfpathlineto{\pgfqpoint{2.358710in}{1.892693in}}%
\pgfpathlineto{\pgfqpoint{2.358710in}{1.883823in}}%
\pgfusepath{fill}%
\end{pgfscope}%
\begin{pgfscope}%
\pgfpathrectangle{\pgfqpoint{1.432000in}{0.528000in}}{\pgfqpoint{3.696000in}{3.696000in}} %
\pgfusepath{clip}%
\pgfsetbuttcap%
\pgfsetroundjoin%
\definecolor{currentfill}{rgb}{0.282884,0.135920,0.453427}%
\pgfsetfillcolor{currentfill}%
\pgfsetlinewidth{0.000000pt}%
\definecolor{currentstroke}{rgb}{0.000000,0.000000,0.000000}%
\pgfsetstrokecolor{currentstroke}%
\pgfsetdash{}{0pt}%
\pgfpathmoveto{\pgfqpoint{2.358710in}{1.883823in}}%
\pgfpathlineto{\pgfqpoint{2.290239in}{1.883823in}}%
\pgfpathlineto{\pgfqpoint{2.294675in}{1.874952in}}%
\pgfpathlineto{\pgfqpoint{2.250323in}{1.888258in}}%
\pgfpathlineto{\pgfqpoint{2.294675in}{1.901564in}}%
\pgfpathlineto{\pgfqpoint{2.290239in}{1.892693in}}%
\pgfpathlineto{\pgfqpoint{2.358710in}{1.892693in}}%
\pgfpathlineto{\pgfqpoint{2.358710in}{1.883823in}}%
\pgfusepath{fill}%
\end{pgfscope}%
\begin{pgfscope}%
\pgfpathrectangle{\pgfqpoint{1.432000in}{0.528000in}}{\pgfqpoint{3.696000in}{3.696000in}} %
\pgfusepath{clip}%
\pgfsetbuttcap%
\pgfsetroundjoin%
\definecolor{currentfill}{rgb}{0.278791,0.062145,0.386592}%
\pgfsetfillcolor{currentfill}%
\pgfsetlinewidth{0.000000pt}%
\definecolor{currentstroke}{rgb}{0.000000,0.000000,0.000000}%
\pgfsetstrokecolor{currentstroke}%
\pgfsetdash{}{0pt}%
\pgfpathmoveto{\pgfqpoint{2.469080in}{1.884291in}}%
\pgfpathlineto{\pgfqpoint{2.288009in}{1.793755in}}%
\pgfpathlineto{\pgfqpoint{2.295943in}{1.787805in}}%
\pgfpathlineto{\pgfqpoint{2.250323in}{1.779871in}}%
\pgfpathlineto{\pgfqpoint{2.284042in}{1.811607in}}%
\pgfpathlineto{\pgfqpoint{2.284042in}{1.801689in}}%
\pgfpathlineto{\pgfqpoint{2.465113in}{1.892225in}}%
\pgfpathlineto{\pgfqpoint{2.469080in}{1.884291in}}%
\pgfusepath{fill}%
\end{pgfscope}%
\begin{pgfscope}%
\pgfpathrectangle{\pgfqpoint{1.432000in}{0.528000in}}{\pgfqpoint{3.696000in}{3.696000in}} %
\pgfusepath{clip}%
\pgfsetbuttcap%
\pgfsetroundjoin%
\definecolor{currentfill}{rgb}{0.165117,0.467423,0.558141}%
\pgfsetfillcolor{currentfill}%
\pgfsetlinewidth{0.000000pt}%
\definecolor{currentstroke}{rgb}{0.000000,0.000000,0.000000}%
\pgfsetstrokecolor{currentstroke}%
\pgfsetdash{}{0pt}%
\pgfpathmoveto{\pgfqpoint{2.467097in}{1.883823in}}%
\pgfpathlineto{\pgfqpoint{2.290239in}{1.883823in}}%
\pgfpathlineto{\pgfqpoint{2.294675in}{1.874952in}}%
\pgfpathlineto{\pgfqpoint{2.250323in}{1.888258in}}%
\pgfpathlineto{\pgfqpoint{2.294675in}{1.901564in}}%
\pgfpathlineto{\pgfqpoint{2.290239in}{1.892693in}}%
\pgfpathlineto{\pgfqpoint{2.467097in}{1.892693in}}%
\pgfpathlineto{\pgfqpoint{2.467097in}{1.883823in}}%
\pgfusepath{fill}%
\end{pgfscope}%
\begin{pgfscope}%
\pgfpathrectangle{\pgfqpoint{1.432000in}{0.528000in}}{\pgfqpoint{3.696000in}{3.696000in}} %
\pgfusepath{clip}%
\pgfsetbuttcap%
\pgfsetroundjoin%
\definecolor{currentfill}{rgb}{0.151918,0.500685,0.557587}%
\pgfsetfillcolor{currentfill}%
\pgfsetlinewidth{0.000000pt}%
\definecolor{currentstroke}{rgb}{0.000000,0.000000,0.000000}%
\pgfsetstrokecolor{currentstroke}%
\pgfsetdash{}{0pt}%
\pgfpathmoveto{\pgfqpoint{2.575484in}{1.883823in}}%
\pgfpathlineto{\pgfqpoint{2.398626in}{1.883823in}}%
\pgfpathlineto{\pgfqpoint{2.403062in}{1.874952in}}%
\pgfpathlineto{\pgfqpoint{2.358710in}{1.888258in}}%
\pgfpathlineto{\pgfqpoint{2.403062in}{1.901564in}}%
\pgfpathlineto{\pgfqpoint{2.398626in}{1.892693in}}%
\pgfpathlineto{\pgfqpoint{2.575484in}{1.892693in}}%
\pgfpathlineto{\pgfqpoint{2.575484in}{1.883823in}}%
\pgfusepath{fill}%
\end{pgfscope}%
\begin{pgfscope}%
\pgfpathrectangle{\pgfqpoint{1.432000in}{0.528000in}}{\pgfqpoint{3.696000in}{3.696000in}} %
\pgfusepath{clip}%
\pgfsetbuttcap%
\pgfsetroundjoin%
\definecolor{currentfill}{rgb}{0.277018,0.050344,0.375715}%
\pgfsetfillcolor{currentfill}%
\pgfsetlinewidth{0.000000pt}%
\definecolor{currentstroke}{rgb}{0.000000,0.000000,0.000000}%
\pgfsetstrokecolor{currentstroke}%
\pgfsetdash{}{0pt}%
\pgfpathmoveto{\pgfqpoint{2.573500in}{1.884291in}}%
\pgfpathlineto{\pgfqpoint{2.392429in}{1.974827in}}%
\pgfpathlineto{\pgfqpoint{2.392429in}{1.964909in}}%
\pgfpathlineto{\pgfqpoint{2.358710in}{1.996645in}}%
\pgfpathlineto{\pgfqpoint{2.404330in}{1.988711in}}%
\pgfpathlineto{\pgfqpoint{2.396396in}{1.982761in}}%
\pgfpathlineto{\pgfqpoint{2.577467in}{1.892225in}}%
\pgfpathlineto{\pgfqpoint{2.573500in}{1.884291in}}%
\pgfusepath{fill}%
\end{pgfscope}%
\begin{pgfscope}%
\pgfpathrectangle{\pgfqpoint{1.432000in}{0.528000in}}{\pgfqpoint{3.696000in}{3.696000in}} %
\pgfusepath{clip}%
\pgfsetbuttcap%
\pgfsetroundjoin%
\definecolor{currentfill}{rgb}{0.129933,0.559582,0.551864}%
\pgfsetfillcolor{currentfill}%
\pgfsetlinewidth{0.000000pt}%
\definecolor{currentstroke}{rgb}{0.000000,0.000000,0.000000}%
\pgfsetstrokecolor{currentstroke}%
\pgfsetdash{}{0pt}%
\pgfpathmoveto{\pgfqpoint{2.683871in}{1.883823in}}%
\pgfpathlineto{\pgfqpoint{2.507014in}{1.883823in}}%
\pgfpathlineto{\pgfqpoint{2.511449in}{1.874952in}}%
\pgfpathlineto{\pgfqpoint{2.467097in}{1.888258in}}%
\pgfpathlineto{\pgfqpoint{2.511449in}{1.901564in}}%
\pgfpathlineto{\pgfqpoint{2.507014in}{1.892693in}}%
\pgfpathlineto{\pgfqpoint{2.683871in}{1.892693in}}%
\pgfpathlineto{\pgfqpoint{2.683871in}{1.883823in}}%
\pgfusepath{fill}%
\end{pgfscope}%
\begin{pgfscope}%
\pgfpathrectangle{\pgfqpoint{1.432000in}{0.528000in}}{\pgfqpoint{3.696000in}{3.696000in}} %
\pgfusepath{clip}%
\pgfsetbuttcap%
\pgfsetroundjoin%
\definecolor{currentfill}{rgb}{0.282656,0.100196,0.422160}%
\pgfsetfillcolor{currentfill}%
\pgfsetlinewidth{0.000000pt}%
\definecolor{currentstroke}{rgb}{0.000000,0.000000,0.000000}%
\pgfsetstrokecolor{currentstroke}%
\pgfsetdash{}{0pt}%
\pgfpathmoveto{\pgfqpoint{2.681887in}{1.884291in}}%
\pgfpathlineto{\pgfqpoint{2.500816in}{1.974827in}}%
\pgfpathlineto{\pgfqpoint{2.500816in}{1.964909in}}%
\pgfpathlineto{\pgfqpoint{2.467097in}{1.996645in}}%
\pgfpathlineto{\pgfqpoint{2.512717in}{1.988711in}}%
\pgfpathlineto{\pgfqpoint{2.504783in}{1.982761in}}%
\pgfpathlineto{\pgfqpoint{2.685854in}{1.892225in}}%
\pgfpathlineto{\pgfqpoint{2.681887in}{1.884291in}}%
\pgfusepath{fill}%
\end{pgfscope}%
\begin{pgfscope}%
\pgfpathrectangle{\pgfqpoint{1.432000in}{0.528000in}}{\pgfqpoint{3.696000in}{3.696000in}} %
\pgfusepath{clip}%
\pgfsetbuttcap%
\pgfsetroundjoin%
\definecolor{currentfill}{rgb}{0.119483,0.614817,0.537692}%
\pgfsetfillcolor{currentfill}%
\pgfsetlinewidth{0.000000pt}%
\definecolor{currentstroke}{rgb}{0.000000,0.000000,0.000000}%
\pgfsetstrokecolor{currentstroke}%
\pgfsetdash{}{0pt}%
\pgfpathmoveto{\pgfqpoint{2.792258in}{1.883823in}}%
\pgfpathlineto{\pgfqpoint{2.615401in}{1.883823in}}%
\pgfpathlineto{\pgfqpoint{2.619836in}{1.874952in}}%
\pgfpathlineto{\pgfqpoint{2.575484in}{1.888258in}}%
\pgfpathlineto{\pgfqpoint{2.619836in}{1.901564in}}%
\pgfpathlineto{\pgfqpoint{2.615401in}{1.892693in}}%
\pgfpathlineto{\pgfqpoint{2.792258in}{1.892693in}}%
\pgfpathlineto{\pgfqpoint{2.792258in}{1.883823in}}%
\pgfusepath{fill}%
\end{pgfscope}%
\begin{pgfscope}%
\pgfpathrectangle{\pgfqpoint{1.432000in}{0.528000in}}{\pgfqpoint{3.696000in}{3.696000in}} %
\pgfusepath{clip}%
\pgfsetbuttcap%
\pgfsetroundjoin%
\definecolor{currentfill}{rgb}{0.280868,0.160771,0.472899}%
\pgfsetfillcolor{currentfill}%
\pgfsetlinewidth{0.000000pt}%
\definecolor{currentstroke}{rgb}{0.000000,0.000000,0.000000}%
\pgfsetstrokecolor{currentstroke}%
\pgfsetdash{}{0pt}%
\pgfpathmoveto{\pgfqpoint{2.790275in}{1.884291in}}%
\pgfpathlineto{\pgfqpoint{2.609203in}{1.974827in}}%
\pgfpathlineto{\pgfqpoint{2.609203in}{1.964909in}}%
\pgfpathlineto{\pgfqpoint{2.575484in}{1.996645in}}%
\pgfpathlineto{\pgfqpoint{2.621104in}{1.988711in}}%
\pgfpathlineto{\pgfqpoint{2.613170in}{1.982761in}}%
\pgfpathlineto{\pgfqpoint{2.794242in}{1.892225in}}%
\pgfpathlineto{\pgfqpoint{2.790275in}{1.884291in}}%
\pgfusepath{fill}%
\end{pgfscope}%
\begin{pgfscope}%
\pgfpathrectangle{\pgfqpoint{1.432000in}{0.528000in}}{\pgfqpoint{3.696000in}{3.696000in}} %
\pgfusepath{clip}%
\pgfsetbuttcap%
\pgfsetroundjoin%
\definecolor{currentfill}{rgb}{0.122312,0.633153,0.530398}%
\pgfsetfillcolor{currentfill}%
\pgfsetlinewidth{0.000000pt}%
\definecolor{currentstroke}{rgb}{0.000000,0.000000,0.000000}%
\pgfsetstrokecolor{currentstroke}%
\pgfsetdash{}{0pt}%
\pgfpathmoveto{\pgfqpoint{2.900645in}{1.883823in}}%
\pgfpathlineto{\pgfqpoint{2.723788in}{1.883823in}}%
\pgfpathlineto{\pgfqpoint{2.728223in}{1.874952in}}%
\pgfpathlineto{\pgfqpoint{2.683871in}{1.888258in}}%
\pgfpathlineto{\pgfqpoint{2.728223in}{1.901564in}}%
\pgfpathlineto{\pgfqpoint{2.723788in}{1.892693in}}%
\pgfpathlineto{\pgfqpoint{2.900645in}{1.892693in}}%
\pgfpathlineto{\pgfqpoint{2.900645in}{1.883823in}}%
\pgfusepath{fill}%
\end{pgfscope}%
\begin{pgfscope}%
\pgfpathrectangle{\pgfqpoint{1.432000in}{0.528000in}}{\pgfqpoint{3.696000in}{3.696000in}} %
\pgfusepath{clip}%
\pgfsetbuttcap%
\pgfsetroundjoin%
\definecolor{currentfill}{rgb}{0.257322,0.256130,0.526563}%
\pgfsetfillcolor{currentfill}%
\pgfsetlinewidth{0.000000pt}%
\definecolor{currentstroke}{rgb}{0.000000,0.000000,0.000000}%
\pgfsetstrokecolor{currentstroke}%
\pgfsetdash{}{0pt}%
\pgfpathmoveto{\pgfqpoint{2.898662in}{1.884291in}}%
\pgfpathlineto{\pgfqpoint{2.717590in}{1.974827in}}%
\pgfpathlineto{\pgfqpoint{2.717590in}{1.964909in}}%
\pgfpathlineto{\pgfqpoint{2.683871in}{1.996645in}}%
\pgfpathlineto{\pgfqpoint{2.729491in}{1.988711in}}%
\pgfpathlineto{\pgfqpoint{2.721557in}{1.982761in}}%
\pgfpathlineto{\pgfqpoint{2.902629in}{1.892225in}}%
\pgfpathlineto{\pgfqpoint{2.898662in}{1.884291in}}%
\pgfusepath{fill}%
\end{pgfscope}%
\begin{pgfscope}%
\pgfpathrectangle{\pgfqpoint{1.432000in}{0.528000in}}{\pgfqpoint{3.696000in}{3.696000in}} %
\pgfusepath{clip}%
\pgfsetbuttcap%
\pgfsetroundjoin%
\definecolor{currentfill}{rgb}{0.190631,0.407061,0.556089}%
\pgfsetfillcolor{currentfill}%
\pgfsetlinewidth{0.000000pt}%
\definecolor{currentstroke}{rgb}{0.000000,0.000000,0.000000}%
\pgfsetstrokecolor{currentstroke}%
\pgfsetdash{}{0pt}%
\pgfpathmoveto{\pgfqpoint{3.009032in}{1.883823in}}%
\pgfpathlineto{\pgfqpoint{2.832175in}{1.883823in}}%
\pgfpathlineto{\pgfqpoint{2.836610in}{1.874952in}}%
\pgfpathlineto{\pgfqpoint{2.792258in}{1.888258in}}%
\pgfpathlineto{\pgfqpoint{2.836610in}{1.901564in}}%
\pgfpathlineto{\pgfqpoint{2.832175in}{1.892693in}}%
\pgfpathlineto{\pgfqpoint{3.009032in}{1.892693in}}%
\pgfpathlineto{\pgfqpoint{3.009032in}{1.883823in}}%
\pgfusepath{fill}%
\end{pgfscope}%
\begin{pgfscope}%
\pgfpathrectangle{\pgfqpoint{1.432000in}{0.528000in}}{\pgfqpoint{3.696000in}{3.696000in}} %
\pgfusepath{clip}%
\pgfsetbuttcap%
\pgfsetroundjoin%
\definecolor{currentfill}{rgb}{0.153364,0.497000,0.557724}%
\pgfsetfillcolor{currentfill}%
\pgfsetlinewidth{0.000000pt}%
\definecolor{currentstroke}{rgb}{0.000000,0.000000,0.000000}%
\pgfsetstrokecolor{currentstroke}%
\pgfsetdash{}{0pt}%
\pgfpathmoveto{\pgfqpoint{3.007049in}{1.884291in}}%
\pgfpathlineto{\pgfqpoint{2.825977in}{1.974827in}}%
\pgfpathlineto{\pgfqpoint{2.825977in}{1.964909in}}%
\pgfpathlineto{\pgfqpoint{2.792258in}{1.996645in}}%
\pgfpathlineto{\pgfqpoint{2.837878in}{1.988711in}}%
\pgfpathlineto{\pgfqpoint{2.829944in}{1.982761in}}%
\pgfpathlineto{\pgfqpoint{3.011016in}{1.892225in}}%
\pgfpathlineto{\pgfqpoint{3.007049in}{1.884291in}}%
\pgfusepath{fill}%
\end{pgfscope}%
\begin{pgfscope}%
\pgfpathrectangle{\pgfqpoint{1.432000in}{0.528000in}}{\pgfqpoint{3.696000in}{3.696000in}} %
\pgfusepath{clip}%
\pgfsetbuttcap%
\pgfsetroundjoin%
\definecolor{currentfill}{rgb}{0.187231,0.414746,0.556547}%
\pgfsetfillcolor{currentfill}%
\pgfsetlinewidth{0.000000pt}%
\definecolor{currentstroke}{rgb}{0.000000,0.000000,0.000000}%
\pgfsetstrokecolor{currentstroke}%
\pgfsetdash{}{0pt}%
\pgfpathmoveto{\pgfqpoint{3.117419in}{1.883823in}}%
\pgfpathlineto{\pgfqpoint{2.940562in}{1.883823in}}%
\pgfpathlineto{\pgfqpoint{2.944997in}{1.874952in}}%
\pgfpathlineto{\pgfqpoint{2.900645in}{1.888258in}}%
\pgfpathlineto{\pgfqpoint{2.944997in}{1.901564in}}%
\pgfpathlineto{\pgfqpoint{2.940562in}{1.892693in}}%
\pgfpathlineto{\pgfqpoint{3.117419in}{1.892693in}}%
\pgfpathlineto{\pgfqpoint{3.117419in}{1.883823in}}%
\pgfusepath{fill}%
\end{pgfscope}%
\begin{pgfscope}%
\pgfpathrectangle{\pgfqpoint{1.432000in}{0.528000in}}{\pgfqpoint{3.696000in}{3.696000in}} %
\pgfusepath{clip}%
\pgfsetbuttcap%
\pgfsetroundjoin%
\definecolor{currentfill}{rgb}{0.223925,0.334994,0.548053}%
\pgfsetfillcolor{currentfill}%
\pgfsetlinewidth{0.000000pt}%
\definecolor{currentstroke}{rgb}{0.000000,0.000000,0.000000}%
\pgfsetstrokecolor{currentstroke}%
\pgfsetdash{}{0pt}%
\pgfpathmoveto{\pgfqpoint{3.115436in}{1.884291in}}%
\pgfpathlineto{\pgfqpoint{2.934364in}{1.974827in}}%
\pgfpathlineto{\pgfqpoint{2.934364in}{1.964909in}}%
\pgfpathlineto{\pgfqpoint{2.900645in}{1.996645in}}%
\pgfpathlineto{\pgfqpoint{2.946265in}{1.988711in}}%
\pgfpathlineto{\pgfqpoint{2.938331in}{1.982761in}}%
\pgfpathlineto{\pgfqpoint{3.119403in}{1.892225in}}%
\pgfpathlineto{\pgfqpoint{3.115436in}{1.884291in}}%
\pgfusepath{fill}%
\end{pgfscope}%
\begin{pgfscope}%
\pgfpathrectangle{\pgfqpoint{1.432000in}{0.528000in}}{\pgfqpoint{3.696000in}{3.696000in}} %
\pgfusepath{clip}%
\pgfsetbuttcap%
\pgfsetroundjoin%
\definecolor{currentfill}{rgb}{0.177423,0.437527,0.557565}%
\pgfsetfillcolor{currentfill}%
\pgfsetlinewidth{0.000000pt}%
\definecolor{currentstroke}{rgb}{0.000000,0.000000,0.000000}%
\pgfsetstrokecolor{currentstroke}%
\pgfsetdash{}{0pt}%
\pgfpathmoveto{\pgfqpoint{3.225806in}{1.883823in}}%
\pgfpathlineto{\pgfqpoint{3.048949in}{1.883823in}}%
\pgfpathlineto{\pgfqpoint{3.053384in}{1.874952in}}%
\pgfpathlineto{\pgfqpoint{3.009032in}{1.888258in}}%
\pgfpathlineto{\pgfqpoint{3.053384in}{1.901564in}}%
\pgfpathlineto{\pgfqpoint{3.048949in}{1.892693in}}%
\pgfpathlineto{\pgfqpoint{3.225806in}{1.892693in}}%
\pgfpathlineto{\pgfqpoint{3.225806in}{1.883823in}}%
\pgfusepath{fill}%
\end{pgfscope}%
\begin{pgfscope}%
\pgfpathrectangle{\pgfqpoint{1.432000in}{0.528000in}}{\pgfqpoint{3.696000in}{3.696000in}} %
\pgfusepath{clip}%
\pgfsetbuttcap%
\pgfsetroundjoin%
\definecolor{currentfill}{rgb}{0.250425,0.274290,0.533103}%
\pgfsetfillcolor{currentfill}%
\pgfsetlinewidth{0.000000pt}%
\definecolor{currentstroke}{rgb}{0.000000,0.000000,0.000000}%
\pgfsetstrokecolor{currentstroke}%
\pgfsetdash{}{0pt}%
\pgfpathmoveto{\pgfqpoint{3.223823in}{1.884291in}}%
\pgfpathlineto{\pgfqpoint{3.042751in}{1.974827in}}%
\pgfpathlineto{\pgfqpoint{3.042751in}{1.964909in}}%
\pgfpathlineto{\pgfqpoint{3.009032in}{1.996645in}}%
\pgfpathlineto{\pgfqpoint{3.054652in}{1.988711in}}%
\pgfpathlineto{\pgfqpoint{3.046718in}{1.982761in}}%
\pgfpathlineto{\pgfqpoint{3.227790in}{1.892225in}}%
\pgfpathlineto{\pgfqpoint{3.223823in}{1.884291in}}%
\pgfusepath{fill}%
\end{pgfscope}%
\begin{pgfscope}%
\pgfpathrectangle{\pgfqpoint{1.432000in}{0.528000in}}{\pgfqpoint{3.696000in}{3.696000in}} %
\pgfusepath{clip}%
\pgfsetbuttcap%
\pgfsetroundjoin%
\definecolor{currentfill}{rgb}{0.185556,0.418570,0.556753}%
\pgfsetfillcolor{currentfill}%
\pgfsetlinewidth{0.000000pt}%
\definecolor{currentstroke}{rgb}{0.000000,0.000000,0.000000}%
\pgfsetstrokecolor{currentstroke}%
\pgfsetdash{}{0pt}%
\pgfpathmoveto{\pgfqpoint{3.334194in}{1.883823in}}%
\pgfpathlineto{\pgfqpoint{3.157336in}{1.883823in}}%
\pgfpathlineto{\pgfqpoint{3.161771in}{1.874952in}}%
\pgfpathlineto{\pgfqpoint{3.117419in}{1.888258in}}%
\pgfpathlineto{\pgfqpoint{3.161771in}{1.901564in}}%
\pgfpathlineto{\pgfqpoint{3.157336in}{1.892693in}}%
\pgfpathlineto{\pgfqpoint{3.334194in}{1.892693in}}%
\pgfpathlineto{\pgfqpoint{3.334194in}{1.883823in}}%
\pgfusepath{fill}%
\end{pgfscope}%
\begin{pgfscope}%
\pgfpathrectangle{\pgfqpoint{1.432000in}{0.528000in}}{\pgfqpoint{3.696000in}{3.696000in}} %
\pgfusepath{clip}%
\pgfsetbuttcap%
\pgfsetroundjoin%
\definecolor{currentfill}{rgb}{0.262138,0.242286,0.520837}%
\pgfsetfillcolor{currentfill}%
\pgfsetlinewidth{0.000000pt}%
\definecolor{currentstroke}{rgb}{0.000000,0.000000,0.000000}%
\pgfsetstrokecolor{currentstroke}%
\pgfsetdash{}{0pt}%
\pgfpathmoveto{\pgfqpoint{3.332210in}{1.884291in}}%
\pgfpathlineto{\pgfqpoint{3.151139in}{1.974827in}}%
\pgfpathlineto{\pgfqpoint{3.151139in}{1.964909in}}%
\pgfpathlineto{\pgfqpoint{3.117419in}{1.996645in}}%
\pgfpathlineto{\pgfqpoint{3.163039in}{1.988711in}}%
\pgfpathlineto{\pgfqpoint{3.155106in}{1.982761in}}%
\pgfpathlineto{\pgfqpoint{3.336177in}{1.892225in}}%
\pgfpathlineto{\pgfqpoint{3.332210in}{1.884291in}}%
\pgfusepath{fill}%
\end{pgfscope}%
\begin{pgfscope}%
\pgfpathrectangle{\pgfqpoint{1.432000in}{0.528000in}}{\pgfqpoint{3.696000in}{3.696000in}} %
\pgfusepath{clip}%
\pgfsetbuttcap%
\pgfsetroundjoin%
\definecolor{currentfill}{rgb}{0.206756,0.371758,0.553117}%
\pgfsetfillcolor{currentfill}%
\pgfsetlinewidth{0.000000pt}%
\definecolor{currentstroke}{rgb}{0.000000,0.000000,0.000000}%
\pgfsetstrokecolor{currentstroke}%
\pgfsetdash{}{0pt}%
\pgfpathmoveto{\pgfqpoint{3.442581in}{1.883823in}}%
\pgfpathlineto{\pgfqpoint{3.265723in}{1.883823in}}%
\pgfpathlineto{\pgfqpoint{3.270158in}{1.874952in}}%
\pgfpathlineto{\pgfqpoint{3.225806in}{1.888258in}}%
\pgfpathlineto{\pgfqpoint{3.270158in}{1.901564in}}%
\pgfpathlineto{\pgfqpoint{3.265723in}{1.892693in}}%
\pgfpathlineto{\pgfqpoint{3.442581in}{1.892693in}}%
\pgfpathlineto{\pgfqpoint{3.442581in}{1.883823in}}%
\pgfusepath{fill}%
\end{pgfscope}%
\begin{pgfscope}%
\pgfpathrectangle{\pgfqpoint{1.432000in}{0.528000in}}{\pgfqpoint{3.696000in}{3.696000in}} %
\pgfusepath{clip}%
\pgfsetbuttcap%
\pgfsetroundjoin%
\definecolor{currentfill}{rgb}{0.282623,0.140926,0.457517}%
\pgfsetfillcolor{currentfill}%
\pgfsetlinewidth{0.000000pt}%
\definecolor{currentstroke}{rgb}{0.000000,0.000000,0.000000}%
\pgfsetstrokecolor{currentstroke}%
\pgfsetdash{}{0pt}%
\pgfpathmoveto{\pgfqpoint{3.440597in}{1.884291in}}%
\pgfpathlineto{\pgfqpoint{3.259526in}{1.974827in}}%
\pgfpathlineto{\pgfqpoint{3.259526in}{1.964909in}}%
\pgfpathlineto{\pgfqpoint{3.225806in}{1.996645in}}%
\pgfpathlineto{\pgfqpoint{3.271427in}{1.988711in}}%
\pgfpathlineto{\pgfqpoint{3.263493in}{1.982761in}}%
\pgfpathlineto{\pgfqpoint{3.444564in}{1.892225in}}%
\pgfpathlineto{\pgfqpoint{3.440597in}{1.884291in}}%
\pgfusepath{fill}%
\end{pgfscope}%
\begin{pgfscope}%
\pgfpathrectangle{\pgfqpoint{1.432000in}{0.528000in}}{\pgfqpoint{3.696000in}{3.696000in}} %
\pgfusepath{clip}%
\pgfsetbuttcap%
\pgfsetroundjoin%
\definecolor{currentfill}{rgb}{0.269944,0.014625,0.341379}%
\pgfsetfillcolor{currentfill}%
\pgfsetlinewidth{0.000000pt}%
\definecolor{currentstroke}{rgb}{0.000000,0.000000,0.000000}%
\pgfsetstrokecolor{currentstroke}%
\pgfsetdash{}{0pt}%
\pgfpathmoveto{\pgfqpoint{3.549565in}{1.884050in}}%
\pgfpathlineto{\pgfqpoint{3.262272in}{1.979815in}}%
\pgfpathlineto{\pgfqpoint{3.263675in}{1.969997in}}%
\pgfpathlineto{\pgfqpoint{3.225806in}{1.996645in}}%
\pgfpathlineto{\pgfqpoint{3.272090in}{1.995243in}}%
\pgfpathlineto{\pgfqpoint{3.265077in}{1.988230in}}%
\pgfpathlineto{\pgfqpoint{3.552370in}{1.892466in}}%
\pgfpathlineto{\pgfqpoint{3.549565in}{1.884050in}}%
\pgfusepath{fill}%
\end{pgfscope}%
\begin{pgfscope}%
\pgfpathrectangle{\pgfqpoint{1.432000in}{0.528000in}}{\pgfqpoint{3.696000in}{3.696000in}} %
\pgfusepath{clip}%
\pgfsetbuttcap%
\pgfsetroundjoin%
\definecolor{currentfill}{rgb}{0.278791,0.062145,0.386592}%
\pgfsetfillcolor{currentfill}%
\pgfsetlinewidth{0.000000pt}%
\definecolor{currentstroke}{rgb}{0.000000,0.000000,0.000000}%
\pgfsetstrokecolor{currentstroke}%
\pgfsetdash{}{0pt}%
\pgfpathmoveto{\pgfqpoint{3.548984in}{1.884291in}}%
\pgfpathlineto{\pgfqpoint{3.367913in}{1.974827in}}%
\pgfpathlineto{\pgfqpoint{3.367913in}{1.964909in}}%
\pgfpathlineto{\pgfqpoint{3.334194in}{1.996645in}}%
\pgfpathlineto{\pgfqpoint{3.379814in}{1.988711in}}%
\pgfpathlineto{\pgfqpoint{3.371880in}{1.982761in}}%
\pgfpathlineto{\pgfqpoint{3.552951in}{1.892225in}}%
\pgfpathlineto{\pgfqpoint{3.548984in}{1.884291in}}%
\pgfusepath{fill}%
\end{pgfscope}%
\begin{pgfscope}%
\pgfpathrectangle{\pgfqpoint{1.432000in}{0.528000in}}{\pgfqpoint{3.696000in}{3.696000in}} %
\pgfusepath{clip}%
\pgfsetbuttcap%
\pgfsetroundjoin%
\definecolor{currentfill}{rgb}{0.282910,0.105393,0.426902}%
\pgfsetfillcolor{currentfill}%
\pgfsetlinewidth{0.000000pt}%
\definecolor{currentstroke}{rgb}{0.000000,0.000000,0.000000}%
\pgfsetstrokecolor{currentstroke}%
\pgfsetdash{}{0pt}%
\pgfpathmoveto{\pgfqpoint{3.659355in}{1.883823in}}%
\pgfpathlineto{\pgfqpoint{3.374110in}{1.883823in}}%
\pgfpathlineto{\pgfqpoint{3.378546in}{1.874952in}}%
\pgfpathlineto{\pgfqpoint{3.334194in}{1.888258in}}%
\pgfpathlineto{\pgfqpoint{3.378546in}{1.901564in}}%
\pgfpathlineto{\pgfqpoint{3.374110in}{1.892693in}}%
\pgfpathlineto{\pgfqpoint{3.659355in}{1.892693in}}%
\pgfpathlineto{\pgfqpoint{3.659355in}{1.883823in}}%
\pgfusepath{fill}%
\end{pgfscope}%
\begin{pgfscope}%
\pgfpathrectangle{\pgfqpoint{1.432000in}{0.528000in}}{\pgfqpoint{3.696000in}{3.696000in}} %
\pgfusepath{clip}%
\pgfsetbuttcap%
\pgfsetroundjoin%
\definecolor{currentfill}{rgb}{0.270595,0.214069,0.507052}%
\pgfsetfillcolor{currentfill}%
\pgfsetlinewidth{0.000000pt}%
\definecolor{currentstroke}{rgb}{0.000000,0.000000,0.000000}%
\pgfsetstrokecolor{currentstroke}%
\pgfsetdash{}{0pt}%
\pgfpathmoveto{\pgfqpoint{3.657952in}{1.884050in}}%
\pgfpathlineto{\pgfqpoint{3.370659in}{1.979815in}}%
\pgfpathlineto{\pgfqpoint{3.372062in}{1.969997in}}%
\pgfpathlineto{\pgfqpoint{3.334194in}{1.996645in}}%
\pgfpathlineto{\pgfqpoint{3.380477in}{1.995243in}}%
\pgfpathlineto{\pgfqpoint{3.373464in}{1.988230in}}%
\pgfpathlineto{\pgfqpoint{3.660757in}{1.892466in}}%
\pgfpathlineto{\pgfqpoint{3.657952in}{1.884050in}}%
\pgfusepath{fill}%
\end{pgfscope}%
\begin{pgfscope}%
\pgfpathrectangle{\pgfqpoint{1.432000in}{0.528000in}}{\pgfqpoint{3.696000in}{3.696000in}} %
\pgfusepath{clip}%
\pgfsetbuttcap%
\pgfsetroundjoin%
\definecolor{currentfill}{rgb}{0.282327,0.094955,0.417331}%
\pgfsetfillcolor{currentfill}%
\pgfsetlinewidth{0.000000pt}%
\definecolor{currentstroke}{rgb}{0.000000,0.000000,0.000000}%
\pgfsetstrokecolor{currentstroke}%
\pgfsetdash{}{0pt}%
\pgfpathmoveto{\pgfqpoint{3.767742in}{1.883823in}}%
\pgfpathlineto{\pgfqpoint{3.482497in}{1.883823in}}%
\pgfpathlineto{\pgfqpoint{3.486933in}{1.874952in}}%
\pgfpathlineto{\pgfqpoint{3.442581in}{1.888258in}}%
\pgfpathlineto{\pgfqpoint{3.486933in}{1.901564in}}%
\pgfpathlineto{\pgfqpoint{3.482497in}{1.892693in}}%
\pgfpathlineto{\pgfqpoint{3.767742in}{1.892693in}}%
\pgfpathlineto{\pgfqpoint{3.767742in}{1.883823in}}%
\pgfusepath{fill}%
\end{pgfscope}%
\begin{pgfscope}%
\pgfpathrectangle{\pgfqpoint{1.432000in}{0.528000in}}{\pgfqpoint{3.696000in}{3.696000in}} %
\pgfusepath{clip}%
\pgfsetbuttcap%
\pgfsetroundjoin%
\definecolor{currentfill}{rgb}{0.281887,0.150881,0.465405}%
\pgfsetfillcolor{currentfill}%
\pgfsetlinewidth{0.000000pt}%
\definecolor{currentstroke}{rgb}{0.000000,0.000000,0.000000}%
\pgfsetstrokecolor{currentstroke}%
\pgfsetdash{}{0pt}%
\pgfpathmoveto{\pgfqpoint{3.766339in}{1.884050in}}%
\pgfpathlineto{\pgfqpoint{3.479047in}{1.979815in}}%
\pgfpathlineto{\pgfqpoint{3.480449in}{1.969997in}}%
\pgfpathlineto{\pgfqpoint{3.442581in}{1.996645in}}%
\pgfpathlineto{\pgfqpoint{3.488864in}{1.995243in}}%
\pgfpathlineto{\pgfqpoint{3.481852in}{1.988230in}}%
\pgfpathlineto{\pgfqpoint{3.769144in}{1.892466in}}%
\pgfpathlineto{\pgfqpoint{3.766339in}{1.884050in}}%
\pgfusepath{fill}%
\end{pgfscope}%
\begin{pgfscope}%
\pgfpathrectangle{\pgfqpoint{1.432000in}{0.528000in}}{\pgfqpoint{3.696000in}{3.696000in}} %
\pgfusepath{clip}%
\pgfsetbuttcap%
\pgfsetroundjoin%
\definecolor{currentfill}{rgb}{0.269944,0.014625,0.341379}%
\pgfsetfillcolor{currentfill}%
\pgfsetlinewidth{0.000000pt}%
\definecolor{currentstroke}{rgb}{0.000000,0.000000,0.000000}%
\pgfsetstrokecolor{currentstroke}%
\pgfsetdash{}{0pt}%
\pgfpathmoveto{\pgfqpoint{3.765282in}{1.884568in}}%
\pgfpathlineto{\pgfqpoint{3.473333in}{2.079200in}}%
\pgfpathlineto{\pgfqpoint{3.472103in}{2.069359in}}%
\pgfpathlineto{\pgfqpoint{3.442581in}{2.105032in}}%
\pgfpathlineto{\pgfqpoint{3.486864in}{2.091501in}}%
\pgfpathlineto{\pgfqpoint{3.478254in}{2.086581in}}%
\pgfpathlineto{\pgfqpoint{3.770202in}{1.891948in}}%
\pgfpathlineto{\pgfqpoint{3.765282in}{1.884568in}}%
\pgfusepath{fill}%
\end{pgfscope}%
\begin{pgfscope}%
\pgfpathrectangle{\pgfqpoint{1.432000in}{0.528000in}}{\pgfqpoint{3.696000in}{3.696000in}} %
\pgfusepath{clip}%
\pgfsetbuttcap%
\pgfsetroundjoin%
\definecolor{currentfill}{rgb}{0.268510,0.009605,0.335427}%
\pgfsetfillcolor{currentfill}%
\pgfsetlinewidth{0.000000pt}%
\definecolor{currentstroke}{rgb}{0.000000,0.000000,0.000000}%
\pgfsetstrokecolor{currentstroke}%
\pgfsetdash{}{0pt}%
\pgfpathmoveto{\pgfqpoint{3.875053in}{1.883955in}}%
\pgfpathlineto{\pgfqpoint{3.480230in}{1.982661in}}%
\pgfpathlineto{\pgfqpoint{3.482381in}{1.972980in}}%
\pgfpathlineto{\pgfqpoint{3.442581in}{1.996645in}}%
\pgfpathlineto{\pgfqpoint{3.488835in}{1.998797in}}%
\pgfpathlineto{\pgfqpoint{3.482381in}{1.991267in}}%
\pgfpathlineto{\pgfqpoint{3.877205in}{1.892561in}}%
\pgfpathlineto{\pgfqpoint{3.875053in}{1.883955in}}%
\pgfusepath{fill}%
\end{pgfscope}%
\begin{pgfscope}%
\pgfpathrectangle{\pgfqpoint{1.432000in}{0.528000in}}{\pgfqpoint{3.696000in}{3.696000in}} %
\pgfusepath{clip}%
\pgfsetbuttcap%
\pgfsetroundjoin%
\definecolor{currentfill}{rgb}{0.283197,0.115680,0.436115}%
\pgfsetfillcolor{currentfill}%
\pgfsetlinewidth{0.000000pt}%
\definecolor{currentstroke}{rgb}{0.000000,0.000000,0.000000}%
\pgfsetstrokecolor{currentstroke}%
\pgfsetdash{}{0pt}%
\pgfpathmoveto{\pgfqpoint{3.984516in}{1.883823in}}%
\pgfpathlineto{\pgfqpoint{3.590885in}{1.883823in}}%
\pgfpathlineto{\pgfqpoint{3.595320in}{1.874952in}}%
\pgfpathlineto{\pgfqpoint{3.550968in}{1.888258in}}%
\pgfpathlineto{\pgfqpoint{3.595320in}{1.901564in}}%
\pgfpathlineto{\pgfqpoint{3.590885in}{1.892693in}}%
\pgfpathlineto{\pgfqpoint{3.984516in}{1.892693in}}%
\pgfpathlineto{\pgfqpoint{3.984516in}{1.883823in}}%
\pgfusepath{fill}%
\end{pgfscope}%
\begin{pgfscope}%
\pgfpathrectangle{\pgfqpoint{1.432000in}{0.528000in}}{\pgfqpoint{3.696000in}{3.696000in}} %
\pgfusepath{clip}%
\pgfsetbuttcap%
\pgfsetroundjoin%
\definecolor{currentfill}{rgb}{0.273006,0.204520,0.501721}%
\pgfsetfillcolor{currentfill}%
\pgfsetlinewidth{0.000000pt}%
\definecolor{currentstroke}{rgb}{0.000000,0.000000,0.000000}%
\pgfsetstrokecolor{currentstroke}%
\pgfsetdash{}{0pt}%
\pgfpathmoveto{\pgfqpoint{3.983440in}{1.883955in}}%
\pgfpathlineto{\pgfqpoint{3.588617in}{1.982661in}}%
\pgfpathlineto{\pgfqpoint{3.590768in}{1.972980in}}%
\pgfpathlineto{\pgfqpoint{3.550968in}{1.996645in}}%
\pgfpathlineto{\pgfqpoint{3.597223in}{1.998797in}}%
\pgfpathlineto{\pgfqpoint{3.590768in}{1.991267in}}%
\pgfpathlineto{\pgfqpoint{3.985592in}{1.892561in}}%
\pgfpathlineto{\pgfqpoint{3.983440in}{1.883955in}}%
\pgfusepath{fill}%
\end{pgfscope}%
\begin{pgfscope}%
\pgfpathrectangle{\pgfqpoint{1.432000in}{0.528000in}}{\pgfqpoint{3.696000in}{3.696000in}} %
\pgfusepath{clip}%
\pgfsetbuttcap%
\pgfsetroundjoin%
\definecolor{currentfill}{rgb}{0.278791,0.062145,0.386592}%
\pgfsetfillcolor{currentfill}%
\pgfsetlinewidth{0.000000pt}%
\definecolor{currentstroke}{rgb}{0.000000,0.000000,0.000000}%
\pgfsetstrokecolor{currentstroke}%
\pgfsetdash{}{0pt}%
\pgfpathmoveto{\pgfqpoint{3.982533in}{1.884291in}}%
\pgfpathlineto{\pgfqpoint{3.584687in}{2.083214in}}%
\pgfpathlineto{\pgfqpoint{3.584687in}{2.073297in}}%
\pgfpathlineto{\pgfqpoint{3.550968in}{2.105032in}}%
\pgfpathlineto{\pgfqpoint{3.596588in}{2.097098in}}%
\pgfpathlineto{\pgfqpoint{3.588654in}{2.091148in}}%
\pgfpathlineto{\pgfqpoint{3.986500in}{1.892225in}}%
\pgfpathlineto{\pgfqpoint{3.982533in}{1.884291in}}%
\pgfusepath{fill}%
\end{pgfscope}%
\begin{pgfscope}%
\pgfpathrectangle{\pgfqpoint{1.432000in}{0.528000in}}{\pgfqpoint{3.696000in}{3.696000in}} %
\pgfusepath{clip}%
\pgfsetbuttcap%
\pgfsetroundjoin%
\definecolor{currentfill}{rgb}{0.282290,0.145912,0.461510}%
\pgfsetfillcolor{currentfill}%
\pgfsetlinewidth{0.000000pt}%
\definecolor{currentstroke}{rgb}{0.000000,0.000000,0.000000}%
\pgfsetstrokecolor{currentstroke}%
\pgfsetdash{}{0pt}%
\pgfpathmoveto{\pgfqpoint{4.092903in}{1.883823in}}%
\pgfpathlineto{\pgfqpoint{3.699272in}{1.883823in}}%
\pgfpathlineto{\pgfqpoint{3.703707in}{1.874952in}}%
\pgfpathlineto{\pgfqpoint{3.659355in}{1.888258in}}%
\pgfpathlineto{\pgfqpoint{3.703707in}{1.901564in}}%
\pgfpathlineto{\pgfqpoint{3.699272in}{1.892693in}}%
\pgfpathlineto{\pgfqpoint{4.092903in}{1.892693in}}%
\pgfpathlineto{\pgfqpoint{4.092903in}{1.883823in}}%
\pgfusepath{fill}%
\end{pgfscope}%
\begin{pgfscope}%
\pgfpathrectangle{\pgfqpoint{1.432000in}{0.528000in}}{\pgfqpoint{3.696000in}{3.696000in}} %
\pgfusepath{clip}%
\pgfsetbuttcap%
\pgfsetroundjoin%
\definecolor{currentfill}{rgb}{0.223925,0.334994,0.548053}%
\pgfsetfillcolor{currentfill}%
\pgfsetlinewidth{0.000000pt}%
\definecolor{currentstroke}{rgb}{0.000000,0.000000,0.000000}%
\pgfsetstrokecolor{currentstroke}%
\pgfsetdash{}{0pt}%
\pgfpathmoveto{\pgfqpoint{4.091828in}{1.883955in}}%
\pgfpathlineto{\pgfqpoint{3.697004in}{1.982661in}}%
\pgfpathlineto{\pgfqpoint{3.699156in}{1.972980in}}%
\pgfpathlineto{\pgfqpoint{3.659355in}{1.996645in}}%
\pgfpathlineto{\pgfqpoint{3.705610in}{1.998797in}}%
\pgfpathlineto{\pgfqpoint{3.699156in}{1.991267in}}%
\pgfpathlineto{\pgfqpoint{4.093979in}{1.892561in}}%
\pgfpathlineto{\pgfqpoint{4.091828in}{1.883955in}}%
\pgfusepath{fill}%
\end{pgfscope}%
\begin{pgfscope}%
\pgfpathrectangle{\pgfqpoint{1.432000in}{0.528000in}}{\pgfqpoint{3.696000in}{3.696000in}} %
\pgfusepath{clip}%
\pgfsetbuttcap%
\pgfsetroundjoin%
\definecolor{currentfill}{rgb}{0.282327,0.094955,0.417331}%
\pgfsetfillcolor{currentfill}%
\pgfsetlinewidth{0.000000pt}%
\definecolor{currentstroke}{rgb}{0.000000,0.000000,0.000000}%
\pgfsetstrokecolor{currentstroke}%
\pgfsetdash{}{0pt}%
\pgfpathmoveto{\pgfqpoint{4.090920in}{1.884291in}}%
\pgfpathlineto{\pgfqpoint{3.693074in}{2.083214in}}%
\pgfpathlineto{\pgfqpoint{3.693074in}{2.073297in}}%
\pgfpathlineto{\pgfqpoint{3.659355in}{2.105032in}}%
\pgfpathlineto{\pgfqpoint{3.704975in}{2.097098in}}%
\pgfpathlineto{\pgfqpoint{3.697041in}{2.091148in}}%
\pgfpathlineto{\pgfqpoint{4.094887in}{1.892225in}}%
\pgfpathlineto{\pgfqpoint{4.090920in}{1.884291in}}%
\pgfusepath{fill}%
\end{pgfscope}%
\begin{pgfscope}%
\pgfpathrectangle{\pgfqpoint{1.432000in}{0.528000in}}{\pgfqpoint{3.696000in}{3.696000in}} %
\pgfusepath{clip}%
\pgfsetbuttcap%
\pgfsetroundjoin%
\definecolor{currentfill}{rgb}{0.280894,0.078907,0.402329}%
\pgfsetfillcolor{currentfill}%
\pgfsetlinewidth{0.000000pt}%
\definecolor{currentstroke}{rgb}{0.000000,0.000000,0.000000}%
\pgfsetstrokecolor{currentstroke}%
\pgfsetdash{}{0pt}%
\pgfpathmoveto{\pgfqpoint{4.200215in}{1.883955in}}%
\pgfpathlineto{\pgfqpoint{3.805391in}{1.982661in}}%
\pgfpathlineto{\pgfqpoint{3.807543in}{1.972980in}}%
\pgfpathlineto{\pgfqpoint{3.767742in}{1.996645in}}%
\pgfpathlineto{\pgfqpoint{3.813997in}{1.998797in}}%
\pgfpathlineto{\pgfqpoint{3.807543in}{1.991267in}}%
\pgfpathlineto{\pgfqpoint{4.202366in}{1.892561in}}%
\pgfpathlineto{\pgfqpoint{4.200215in}{1.883955in}}%
\pgfusepath{fill}%
\end{pgfscope}%
\begin{pgfscope}%
\pgfpathrectangle{\pgfqpoint{1.432000in}{0.528000in}}{\pgfqpoint{3.696000in}{3.696000in}} %
\pgfusepath{clip}%
\pgfsetbuttcap%
\pgfsetroundjoin%
\definecolor{currentfill}{rgb}{0.265145,0.232956,0.516599}%
\pgfsetfillcolor{currentfill}%
\pgfsetlinewidth{0.000000pt}%
\definecolor{currentstroke}{rgb}{0.000000,0.000000,0.000000}%
\pgfsetstrokecolor{currentstroke}%
\pgfsetdash{}{0pt}%
\pgfpathmoveto{\pgfqpoint{4.199307in}{1.884291in}}%
\pgfpathlineto{\pgfqpoint{3.801461in}{2.083214in}}%
\pgfpathlineto{\pgfqpoint{3.801461in}{2.073297in}}%
\pgfpathlineto{\pgfqpoint{3.767742in}{2.105032in}}%
\pgfpathlineto{\pgfqpoint{3.813362in}{2.097098in}}%
\pgfpathlineto{\pgfqpoint{3.805428in}{2.091148in}}%
\pgfpathlineto{\pgfqpoint{4.203274in}{1.892225in}}%
\pgfpathlineto{\pgfqpoint{4.199307in}{1.884291in}}%
\pgfusepath{fill}%
\end{pgfscope}%
\begin{pgfscope}%
\pgfpathrectangle{\pgfqpoint{1.432000in}{0.528000in}}{\pgfqpoint{3.696000in}{3.696000in}} %
\pgfusepath{clip}%
\pgfsetbuttcap%
\pgfsetroundjoin%
\definecolor{currentfill}{rgb}{0.255645,0.260703,0.528312}%
\pgfsetfillcolor{currentfill}%
\pgfsetlinewidth{0.000000pt}%
\definecolor{currentstroke}{rgb}{0.000000,0.000000,0.000000}%
\pgfsetstrokecolor{currentstroke}%
\pgfsetdash{}{0pt}%
\pgfpathmoveto{\pgfqpoint{4.309677in}{1.883823in}}%
\pgfpathlineto{\pgfqpoint{4.024433in}{1.883823in}}%
\pgfpathlineto{\pgfqpoint{4.028868in}{1.874952in}}%
\pgfpathlineto{\pgfqpoint{3.984516in}{1.888258in}}%
\pgfpathlineto{\pgfqpoint{4.028868in}{1.901564in}}%
\pgfpathlineto{\pgfqpoint{4.024433in}{1.892693in}}%
\pgfpathlineto{\pgfqpoint{4.309677in}{1.892693in}}%
\pgfpathlineto{\pgfqpoint{4.309677in}{1.883823in}}%
\pgfusepath{fill}%
\end{pgfscope}%
\begin{pgfscope}%
\pgfpathrectangle{\pgfqpoint{1.432000in}{0.528000in}}{\pgfqpoint{3.696000in}{3.696000in}} %
\pgfusepath{clip}%
\pgfsetbuttcap%
\pgfsetroundjoin%
\definecolor{currentfill}{rgb}{0.271305,0.019942,0.347269}%
\pgfsetfillcolor{currentfill}%
\pgfsetlinewidth{0.000000pt}%
\definecolor{currentstroke}{rgb}{0.000000,0.000000,0.000000}%
\pgfsetstrokecolor{currentstroke}%
\pgfsetdash{}{0pt}%
\pgfpathmoveto{\pgfqpoint{4.308275in}{1.884050in}}%
\pgfpathlineto{\pgfqpoint{4.020982in}{1.979815in}}%
\pgfpathlineto{\pgfqpoint{4.022385in}{1.969997in}}%
\pgfpathlineto{\pgfqpoint{3.984516in}{1.996645in}}%
\pgfpathlineto{\pgfqpoint{4.030800in}{1.995243in}}%
\pgfpathlineto{\pgfqpoint{4.023787in}{1.988230in}}%
\pgfpathlineto{\pgfqpoint{4.311080in}{1.892466in}}%
\pgfpathlineto{\pgfqpoint{4.308275in}{1.884050in}}%
\pgfusepath{fill}%
\end{pgfscope}%
\begin{pgfscope}%
\pgfpathrectangle{\pgfqpoint{1.432000in}{0.528000in}}{\pgfqpoint{3.696000in}{3.696000in}} %
\pgfusepath{clip}%
\pgfsetbuttcap%
\pgfsetroundjoin%
\definecolor{currentfill}{rgb}{0.274952,0.037752,0.364543}%
\pgfsetfillcolor{currentfill}%
\pgfsetlinewidth{0.000000pt}%
\definecolor{currentstroke}{rgb}{0.000000,0.000000,0.000000}%
\pgfsetstrokecolor{currentstroke}%
\pgfsetdash{}{0pt}%
\pgfpathmoveto{\pgfqpoint{4.418065in}{1.883823in}}%
\pgfpathlineto{\pgfqpoint{4.241207in}{1.883823in}}%
\pgfpathlineto{\pgfqpoint{4.245642in}{1.874952in}}%
\pgfpathlineto{\pgfqpoint{4.201290in}{1.888258in}}%
\pgfpathlineto{\pgfqpoint{4.245642in}{1.901564in}}%
\pgfpathlineto{\pgfqpoint{4.241207in}{1.892693in}}%
\pgfpathlineto{\pgfqpoint{4.418065in}{1.892693in}}%
\pgfpathlineto{\pgfqpoint{4.418065in}{1.883823in}}%
\pgfusepath{fill}%
\end{pgfscope}%
\begin{pgfscope}%
\pgfpathrectangle{\pgfqpoint{1.432000in}{0.528000in}}{\pgfqpoint{3.696000in}{3.696000in}} %
\pgfusepath{clip}%
\pgfsetbuttcap%
\pgfsetroundjoin%
\definecolor{currentfill}{rgb}{0.283091,0.110553,0.431554}%
\pgfsetfillcolor{currentfill}%
\pgfsetlinewidth{0.000000pt}%
\definecolor{currentstroke}{rgb}{0.000000,0.000000,0.000000}%
\pgfsetstrokecolor{currentstroke}%
\pgfsetdash{}{0pt}%
\pgfpathmoveto{\pgfqpoint{4.416081in}{1.884291in}}%
\pgfpathlineto{\pgfqpoint{4.235010in}{1.974827in}}%
\pgfpathlineto{\pgfqpoint{4.235010in}{1.964909in}}%
\pgfpathlineto{\pgfqpoint{4.201290in}{1.996645in}}%
\pgfpathlineto{\pgfqpoint{4.246910in}{1.988711in}}%
\pgfpathlineto{\pgfqpoint{4.238976in}{1.982761in}}%
\pgfpathlineto{\pgfqpoint{4.420048in}{1.892225in}}%
\pgfpathlineto{\pgfqpoint{4.416081in}{1.884291in}}%
\pgfusepath{fill}%
\end{pgfscope}%
\begin{pgfscope}%
\pgfpathrectangle{\pgfqpoint{1.432000in}{0.528000in}}{\pgfqpoint{3.696000in}{3.696000in}} %
\pgfusepath{clip}%
\pgfsetbuttcap%
\pgfsetroundjoin%
\definecolor{currentfill}{rgb}{0.273006,0.204520,0.501721}%
\pgfsetfillcolor{currentfill}%
\pgfsetlinewidth{0.000000pt}%
\definecolor{currentstroke}{rgb}{0.000000,0.000000,0.000000}%
\pgfsetstrokecolor{currentstroke}%
\pgfsetdash{}{0pt}%
\pgfpathmoveto{\pgfqpoint{4.414928in}{1.885122in}}%
\pgfpathlineto{\pgfqpoint{4.226380in}{2.073671in}}%
\pgfpathlineto{\pgfqpoint{4.223243in}{2.064262in}}%
\pgfpathlineto{\pgfqpoint{4.201290in}{2.105032in}}%
\pgfpathlineto{\pgfqpoint{4.242060in}{2.083079in}}%
\pgfpathlineto{\pgfqpoint{4.232652in}{2.079943in}}%
\pgfpathlineto{\pgfqpoint{4.421201in}{1.891394in}}%
\pgfpathlineto{\pgfqpoint{4.414928in}{1.885122in}}%
\pgfusepath{fill}%
\end{pgfscope}%
\begin{pgfscope}%
\pgfpathrectangle{\pgfqpoint{1.432000in}{0.528000in}}{\pgfqpoint{3.696000in}{3.696000in}} %
\pgfusepath{clip}%
\pgfsetbuttcap%
\pgfsetroundjoin%
\definecolor{currentfill}{rgb}{0.282290,0.145912,0.461510}%
\pgfsetfillcolor{currentfill}%
\pgfsetlinewidth{0.000000pt}%
\definecolor{currentstroke}{rgb}{0.000000,0.000000,0.000000}%
\pgfsetstrokecolor{currentstroke}%
\pgfsetdash{}{0pt}%
\pgfpathmoveto{\pgfqpoint{4.526452in}{1.883823in}}%
\pgfpathlineto{\pgfqpoint{4.349594in}{1.883823in}}%
\pgfpathlineto{\pgfqpoint{4.354029in}{1.874952in}}%
\pgfpathlineto{\pgfqpoint{4.309677in}{1.888258in}}%
\pgfpathlineto{\pgfqpoint{4.354029in}{1.901564in}}%
\pgfpathlineto{\pgfqpoint{4.349594in}{1.892693in}}%
\pgfpathlineto{\pgfqpoint{4.526452in}{1.892693in}}%
\pgfpathlineto{\pgfqpoint{4.526452in}{1.883823in}}%
\pgfusepath{fill}%
\end{pgfscope}%
\begin{pgfscope}%
\pgfpathrectangle{\pgfqpoint{1.432000in}{0.528000in}}{\pgfqpoint{3.696000in}{3.696000in}} %
\pgfusepath{clip}%
\pgfsetbuttcap%
\pgfsetroundjoin%
\definecolor{currentfill}{rgb}{0.221989,0.339161,0.548752}%
\pgfsetfillcolor{currentfill}%
\pgfsetlinewidth{0.000000pt}%
\definecolor{currentstroke}{rgb}{0.000000,0.000000,0.000000}%
\pgfsetstrokecolor{currentstroke}%
\pgfsetdash{}{0pt}%
\pgfpathmoveto{\pgfqpoint{4.524468in}{1.884291in}}%
\pgfpathlineto{\pgfqpoint{4.343397in}{1.974827in}}%
\pgfpathlineto{\pgfqpoint{4.343397in}{1.964909in}}%
\pgfpathlineto{\pgfqpoint{4.309677in}{1.996645in}}%
\pgfpathlineto{\pgfqpoint{4.355297in}{1.988711in}}%
\pgfpathlineto{\pgfqpoint{4.347364in}{1.982761in}}%
\pgfpathlineto{\pgfqpoint{4.528435in}{1.892225in}}%
\pgfpathlineto{\pgfqpoint{4.524468in}{1.884291in}}%
\pgfusepath{fill}%
\end{pgfscope}%
\begin{pgfscope}%
\pgfpathrectangle{\pgfqpoint{1.432000in}{0.528000in}}{\pgfqpoint{3.696000in}{3.696000in}} %
\pgfusepath{clip}%
\pgfsetbuttcap%
\pgfsetroundjoin%
\definecolor{currentfill}{rgb}{0.255645,0.260703,0.528312}%
\pgfsetfillcolor{currentfill}%
\pgfsetlinewidth{0.000000pt}%
\definecolor{currentstroke}{rgb}{0.000000,0.000000,0.000000}%
\pgfsetstrokecolor{currentstroke}%
\pgfsetdash{}{0pt}%
\pgfpathmoveto{\pgfqpoint{4.523315in}{1.885122in}}%
\pgfpathlineto{\pgfqpoint{4.334767in}{2.073671in}}%
\pgfpathlineto{\pgfqpoint{4.331631in}{2.064262in}}%
\pgfpathlineto{\pgfqpoint{4.309677in}{2.105032in}}%
\pgfpathlineto{\pgfqpoint{4.350447in}{2.083079in}}%
\pgfpathlineto{\pgfqpoint{4.341039in}{2.079943in}}%
\pgfpathlineto{\pgfqpoint{4.529588in}{1.891394in}}%
\pgfpathlineto{\pgfqpoint{4.523315in}{1.885122in}}%
\pgfusepath{fill}%
\end{pgfscope}%
\begin{pgfscope}%
\pgfpathrectangle{\pgfqpoint{1.432000in}{0.528000in}}{\pgfqpoint{3.696000in}{3.696000in}} %
\pgfusepath{clip}%
\pgfsetbuttcap%
\pgfsetroundjoin%
\definecolor{currentfill}{rgb}{0.269944,0.014625,0.341379}%
\pgfsetfillcolor{currentfill}%
\pgfsetlinewidth{0.000000pt}%
\definecolor{currentstroke}{rgb}{0.000000,0.000000,0.000000}%
\pgfsetstrokecolor{currentstroke}%
\pgfsetdash{}{0pt}%
\pgfpathmoveto{\pgfqpoint{4.634839in}{1.883823in}}%
\pgfpathlineto{\pgfqpoint{4.457981in}{1.883823in}}%
\pgfpathlineto{\pgfqpoint{4.462417in}{1.874952in}}%
\pgfpathlineto{\pgfqpoint{4.418065in}{1.888258in}}%
\pgfpathlineto{\pgfqpoint{4.462417in}{1.901564in}}%
\pgfpathlineto{\pgfqpoint{4.457981in}{1.892693in}}%
\pgfpathlineto{\pgfqpoint{4.634839in}{1.892693in}}%
\pgfpathlineto{\pgfqpoint{4.634839in}{1.883823in}}%
\pgfusepath{fill}%
\end{pgfscope}%
\begin{pgfscope}%
\pgfpathrectangle{\pgfqpoint{1.432000in}{0.528000in}}{\pgfqpoint{3.696000in}{3.696000in}} %
\pgfusepath{clip}%
\pgfsetbuttcap%
\pgfsetroundjoin%
\definecolor{currentfill}{rgb}{0.235526,0.309527,0.542944}%
\pgfsetfillcolor{currentfill}%
\pgfsetlinewidth{0.000000pt}%
\definecolor{currentstroke}{rgb}{0.000000,0.000000,0.000000}%
\pgfsetstrokecolor{currentstroke}%
\pgfsetdash{}{0pt}%
\pgfpathmoveto{\pgfqpoint{4.632855in}{1.884291in}}%
\pgfpathlineto{\pgfqpoint{4.451784in}{1.974827in}}%
\pgfpathlineto{\pgfqpoint{4.451784in}{1.964909in}}%
\pgfpathlineto{\pgfqpoint{4.418065in}{1.996645in}}%
\pgfpathlineto{\pgfqpoint{4.463685in}{1.988711in}}%
\pgfpathlineto{\pgfqpoint{4.455751in}{1.982761in}}%
\pgfpathlineto{\pgfqpoint{4.636822in}{1.892225in}}%
\pgfpathlineto{\pgfqpoint{4.632855in}{1.884291in}}%
\pgfusepath{fill}%
\end{pgfscope}%
\begin{pgfscope}%
\pgfpathrectangle{\pgfqpoint{1.432000in}{0.528000in}}{\pgfqpoint{3.696000in}{3.696000in}} %
\pgfusepath{clip}%
\pgfsetbuttcap%
\pgfsetroundjoin%
\definecolor{currentfill}{rgb}{0.274128,0.199721,0.498911}%
\pgfsetfillcolor{currentfill}%
\pgfsetlinewidth{0.000000pt}%
\definecolor{currentstroke}{rgb}{0.000000,0.000000,0.000000}%
\pgfsetstrokecolor{currentstroke}%
\pgfsetdash{}{0pt}%
\pgfpathmoveto{\pgfqpoint{4.631703in}{1.885122in}}%
\pgfpathlineto{\pgfqpoint{4.443154in}{2.073671in}}%
\pgfpathlineto{\pgfqpoint{4.440018in}{2.064262in}}%
\pgfpathlineto{\pgfqpoint{4.418065in}{2.105032in}}%
\pgfpathlineto{\pgfqpoint{4.458835in}{2.083079in}}%
\pgfpathlineto{\pgfqpoint{4.449426in}{2.079943in}}%
\pgfpathlineto{\pgfqpoint{4.637975in}{1.891394in}}%
\pgfpathlineto{\pgfqpoint{4.631703in}{1.885122in}}%
\pgfusepath{fill}%
\end{pgfscope}%
\begin{pgfscope}%
\pgfpathrectangle{\pgfqpoint{1.432000in}{0.528000in}}{\pgfqpoint{3.696000in}{3.696000in}} %
\pgfusepath{clip}%
\pgfsetbuttcap%
\pgfsetroundjoin%
\definecolor{currentfill}{rgb}{0.273809,0.031497,0.358853}%
\pgfsetfillcolor{currentfill}%
\pgfsetlinewidth{0.000000pt}%
\definecolor{currentstroke}{rgb}{0.000000,0.000000,0.000000}%
\pgfsetstrokecolor{currentstroke}%
\pgfsetdash{}{0pt}%
\pgfpathmoveto{\pgfqpoint{4.746362in}{1.885122in}}%
\pgfpathlineto{\pgfqpoint{4.666200in}{1.804960in}}%
\pgfpathlineto{\pgfqpoint{4.675609in}{1.801824in}}%
\pgfpathlineto{\pgfqpoint{4.634839in}{1.779871in}}%
\pgfpathlineto{\pgfqpoint{4.656792in}{1.820641in}}%
\pgfpathlineto{\pgfqpoint{4.659928in}{1.811233in}}%
\pgfpathlineto{\pgfqpoint{4.740090in}{1.891394in}}%
\pgfpathlineto{\pgfqpoint{4.746362in}{1.885122in}}%
\pgfusepath{fill}%
\end{pgfscope}%
\begin{pgfscope}%
\pgfpathrectangle{\pgfqpoint{1.432000in}{0.528000in}}{\pgfqpoint{3.696000in}{3.696000in}} %
\pgfusepath{clip}%
\pgfsetbuttcap%
\pgfsetroundjoin%
\definecolor{currentfill}{rgb}{0.278791,0.062145,0.386592}%
\pgfsetfillcolor{currentfill}%
\pgfsetlinewidth{0.000000pt}%
\definecolor{currentstroke}{rgb}{0.000000,0.000000,0.000000}%
\pgfsetstrokecolor{currentstroke}%
\pgfsetdash{}{0pt}%
\pgfpathmoveto{\pgfqpoint{4.743226in}{1.883823in}}%
\pgfpathlineto{\pgfqpoint{4.674756in}{1.883823in}}%
\pgfpathlineto{\pgfqpoint{4.679191in}{1.874952in}}%
\pgfpathlineto{\pgfqpoint{4.634839in}{1.888258in}}%
\pgfpathlineto{\pgfqpoint{4.679191in}{1.901564in}}%
\pgfpathlineto{\pgfqpoint{4.674756in}{1.892693in}}%
\pgfpathlineto{\pgfqpoint{4.743226in}{1.892693in}}%
\pgfpathlineto{\pgfqpoint{4.743226in}{1.883823in}}%
\pgfusepath{fill}%
\end{pgfscope}%
\begin{pgfscope}%
\pgfpathrectangle{\pgfqpoint{1.432000in}{0.528000in}}{\pgfqpoint{3.696000in}{3.696000in}} %
\pgfusepath{clip}%
\pgfsetbuttcap%
\pgfsetroundjoin%
\definecolor{currentfill}{rgb}{0.274952,0.037752,0.364543}%
\pgfsetfillcolor{currentfill}%
\pgfsetlinewidth{0.000000pt}%
\definecolor{currentstroke}{rgb}{0.000000,0.000000,0.000000}%
\pgfsetstrokecolor{currentstroke}%
\pgfsetdash{}{0pt}%
\pgfpathmoveto{\pgfqpoint{4.741242in}{1.884291in}}%
\pgfpathlineto{\pgfqpoint{4.560171in}{1.974827in}}%
\pgfpathlineto{\pgfqpoint{4.560171in}{1.964909in}}%
\pgfpathlineto{\pgfqpoint{4.526452in}{1.996645in}}%
\pgfpathlineto{\pgfqpoint{4.572072in}{1.988711in}}%
\pgfpathlineto{\pgfqpoint{4.564138in}{1.982761in}}%
\pgfpathlineto{\pgfqpoint{4.745209in}{1.892225in}}%
\pgfpathlineto{\pgfqpoint{4.741242in}{1.884291in}}%
\pgfusepath{fill}%
\end{pgfscope}%
\begin{pgfscope}%
\pgfpathrectangle{\pgfqpoint{1.432000in}{0.528000in}}{\pgfqpoint{3.696000in}{3.696000in}} %
\pgfusepath{clip}%
\pgfsetbuttcap%
\pgfsetroundjoin%
\definecolor{currentfill}{rgb}{0.268510,0.009605,0.335427}%
\pgfsetfillcolor{currentfill}%
\pgfsetlinewidth{0.000000pt}%
\definecolor{currentstroke}{rgb}{0.000000,0.000000,0.000000}%
\pgfsetstrokecolor{currentstroke}%
\pgfsetdash{}{0pt}%
\pgfpathmoveto{\pgfqpoint{4.740090in}{1.885122in}}%
\pgfpathlineto{\pgfqpoint{4.659928in}{1.965284in}}%
\pgfpathlineto{\pgfqpoint{4.656792in}{1.955875in}}%
\pgfpathlineto{\pgfqpoint{4.634839in}{1.996645in}}%
\pgfpathlineto{\pgfqpoint{4.675609in}{1.974692in}}%
\pgfpathlineto{\pgfqpoint{4.666200in}{1.971556in}}%
\pgfpathlineto{\pgfqpoint{4.746362in}{1.891394in}}%
\pgfpathlineto{\pgfqpoint{4.740090in}{1.885122in}}%
\pgfusepath{fill}%
\end{pgfscope}%
\begin{pgfscope}%
\pgfpathrectangle{\pgfqpoint{1.432000in}{0.528000in}}{\pgfqpoint{3.696000in}{3.696000in}} %
\pgfusepath{clip}%
\pgfsetbuttcap%
\pgfsetroundjoin%
\definecolor{currentfill}{rgb}{0.282327,0.094955,0.417331}%
\pgfsetfillcolor{currentfill}%
\pgfsetlinewidth{0.000000pt}%
\definecolor{currentstroke}{rgb}{0.000000,0.000000,0.000000}%
\pgfsetstrokecolor{currentstroke}%
\pgfsetdash{}{0pt}%
\pgfpathmoveto{\pgfqpoint{4.851613in}{1.883823in}}%
\pgfpathlineto{\pgfqpoint{4.783143in}{1.883823in}}%
\pgfpathlineto{\pgfqpoint{4.787578in}{1.874952in}}%
\pgfpathlineto{\pgfqpoint{4.743226in}{1.888258in}}%
\pgfpathlineto{\pgfqpoint{4.787578in}{1.901564in}}%
\pgfpathlineto{\pgfqpoint{4.783143in}{1.892693in}}%
\pgfpathlineto{\pgfqpoint{4.851613in}{1.892693in}}%
\pgfpathlineto{\pgfqpoint{4.851613in}{1.883823in}}%
\pgfusepath{fill}%
\end{pgfscope}%
\begin{pgfscope}%
\pgfpathrectangle{\pgfqpoint{1.432000in}{0.528000in}}{\pgfqpoint{3.696000in}{3.696000in}} %
\pgfusepath{clip}%
\pgfsetbuttcap%
\pgfsetroundjoin%
\definecolor{currentfill}{rgb}{0.269308,0.218818,0.509577}%
\pgfsetfillcolor{currentfill}%
\pgfsetlinewidth{0.000000pt}%
\definecolor{currentstroke}{rgb}{0.000000,0.000000,0.000000}%
\pgfsetstrokecolor{currentstroke}%
\pgfsetdash{}{0pt}%
\pgfpathmoveto{\pgfqpoint{4.856048in}{1.888258in}}%
\pgfpathlineto{\pgfqpoint{4.853831in}{1.892099in}}%
\pgfpathlineto{\pgfqpoint{4.849395in}{1.892099in}}%
\pgfpathlineto{\pgfqpoint{4.847178in}{1.888258in}}%
\pgfpathlineto{\pgfqpoint{4.849395in}{1.884417in}}%
\pgfpathlineto{\pgfqpoint{4.853831in}{1.884417in}}%
\pgfpathlineto{\pgfqpoint{4.856048in}{1.888258in}}%
\pgfpathlineto{\pgfqpoint{4.853831in}{1.892099in}}%
\pgfusepath{fill}%
\end{pgfscope}%
\begin{pgfscope}%
\pgfpathrectangle{\pgfqpoint{1.432000in}{0.528000in}}{\pgfqpoint{3.696000in}{3.696000in}} %
\pgfusepath{clip}%
\pgfsetbuttcap%
\pgfsetroundjoin%
\definecolor{currentfill}{rgb}{0.277134,0.185228,0.489898}%
\pgfsetfillcolor{currentfill}%
\pgfsetlinewidth{0.000000pt}%
\definecolor{currentstroke}{rgb}{0.000000,0.000000,0.000000}%
\pgfsetstrokecolor{currentstroke}%
\pgfsetdash{}{0pt}%
\pgfpathmoveto{\pgfqpoint{4.848477in}{1.885122in}}%
\pgfpathlineto{\pgfqpoint{4.768315in}{1.965284in}}%
\pgfpathlineto{\pgfqpoint{4.765179in}{1.955875in}}%
\pgfpathlineto{\pgfqpoint{4.743226in}{1.996645in}}%
\pgfpathlineto{\pgfqpoint{4.783996in}{1.974692in}}%
\pgfpathlineto{\pgfqpoint{4.774587in}{1.971556in}}%
\pgfpathlineto{\pgfqpoint{4.854749in}{1.891394in}}%
\pgfpathlineto{\pgfqpoint{4.848477in}{1.885122in}}%
\pgfusepath{fill}%
\end{pgfscope}%
\begin{pgfscope}%
\pgfpathrectangle{\pgfqpoint{1.432000in}{0.528000in}}{\pgfqpoint{3.696000in}{3.696000in}} %
\pgfusepath{clip}%
\pgfsetbuttcap%
\pgfsetroundjoin%
\definecolor{currentfill}{rgb}{0.246811,0.283237,0.535941}%
\pgfsetfillcolor{currentfill}%
\pgfsetlinewidth{0.000000pt}%
\definecolor{currentstroke}{rgb}{0.000000,0.000000,0.000000}%
\pgfsetstrokecolor{currentstroke}%
\pgfsetdash{}{0pt}%
\pgfpathmoveto{\pgfqpoint{4.847178in}{1.888258in}}%
\pgfpathlineto{\pgfqpoint{4.847178in}{1.956728in}}%
\pgfpathlineto{\pgfqpoint{4.838307in}{1.952293in}}%
\pgfpathlineto{\pgfqpoint{4.851613in}{1.996645in}}%
\pgfpathlineto{\pgfqpoint{4.864919in}{1.952293in}}%
\pgfpathlineto{\pgfqpoint{4.856048in}{1.956728in}}%
\pgfpathlineto{\pgfqpoint{4.856048in}{1.888258in}}%
\pgfpathlineto{\pgfqpoint{4.847178in}{1.888258in}}%
\pgfusepath{fill}%
\end{pgfscope}%
\begin{pgfscope}%
\pgfpathrectangle{\pgfqpoint{1.432000in}{0.528000in}}{\pgfqpoint{3.696000in}{3.696000in}} %
\pgfusepath{clip}%
\pgfsetbuttcap%
\pgfsetroundjoin%
\definecolor{currentfill}{rgb}{0.216210,0.351535,0.550627}%
\pgfsetfillcolor{currentfill}%
\pgfsetlinewidth{0.000000pt}%
\definecolor{currentstroke}{rgb}{0.000000,0.000000,0.000000}%
\pgfsetstrokecolor{currentstroke}%
\pgfsetdash{}{0pt}%
\pgfpathmoveto{\pgfqpoint{4.964435in}{1.888258in}}%
\pgfpathlineto{\pgfqpoint{4.962218in}{1.892099in}}%
\pgfpathlineto{\pgfqpoint{4.957782in}{1.892099in}}%
\pgfpathlineto{\pgfqpoint{4.955565in}{1.888258in}}%
\pgfpathlineto{\pgfqpoint{4.957782in}{1.884417in}}%
\pgfpathlineto{\pgfqpoint{4.962218in}{1.884417in}}%
\pgfpathlineto{\pgfqpoint{4.964435in}{1.888258in}}%
\pgfpathlineto{\pgfqpoint{4.962218in}{1.892099in}}%
\pgfusepath{fill}%
\end{pgfscope}%
\begin{pgfscope}%
\pgfpathrectangle{\pgfqpoint{1.432000in}{0.528000in}}{\pgfqpoint{3.696000in}{3.696000in}} %
\pgfusepath{clip}%
\pgfsetbuttcap%
\pgfsetroundjoin%
\definecolor{currentfill}{rgb}{0.187231,0.414746,0.556547}%
\pgfsetfillcolor{currentfill}%
\pgfsetlinewidth{0.000000pt}%
\definecolor{currentstroke}{rgb}{0.000000,0.000000,0.000000}%
\pgfsetstrokecolor{currentstroke}%
\pgfsetdash{}{0pt}%
\pgfpathmoveto{\pgfqpoint{4.955565in}{1.888258in}}%
\pgfpathlineto{\pgfqpoint{4.955565in}{1.956728in}}%
\pgfpathlineto{\pgfqpoint{4.946694in}{1.952293in}}%
\pgfpathlineto{\pgfqpoint{4.960000in}{1.996645in}}%
\pgfpathlineto{\pgfqpoint{4.973306in}{1.952293in}}%
\pgfpathlineto{\pgfqpoint{4.964435in}{1.956728in}}%
\pgfpathlineto{\pgfqpoint{4.964435in}{1.888258in}}%
\pgfpathlineto{\pgfqpoint{4.955565in}{1.888258in}}%
\pgfusepath{fill}%
\end{pgfscope}%
\begin{pgfscope}%
\pgfpathrectangle{\pgfqpoint{1.432000in}{0.528000in}}{\pgfqpoint{3.696000in}{3.696000in}} %
\pgfusepath{clip}%
\pgfsetbuttcap%
\pgfsetroundjoin%
\definecolor{currentfill}{rgb}{0.277018,0.050344,0.375715}%
\pgfsetfillcolor{currentfill}%
\pgfsetlinewidth{0.000000pt}%
\definecolor{currentstroke}{rgb}{0.000000,0.000000,0.000000}%
\pgfsetstrokecolor{currentstroke}%
\pgfsetdash{}{0pt}%
\pgfpathmoveto{\pgfqpoint{1.604435in}{1.996645in}}%
\pgfpathlineto{\pgfqpoint{1.604435in}{1.928175in}}%
\pgfpathlineto{\pgfqpoint{1.613306in}{1.932610in}}%
\pgfpathlineto{\pgfqpoint{1.600000in}{1.888258in}}%
\pgfpathlineto{\pgfqpoint{1.586694in}{1.932610in}}%
\pgfpathlineto{\pgfqpoint{1.595565in}{1.928175in}}%
\pgfpathlineto{\pgfqpoint{1.595565in}{1.996645in}}%
\pgfpathlineto{\pgfqpoint{1.604435in}{1.996645in}}%
\pgfusepath{fill}%
\end{pgfscope}%
\begin{pgfscope}%
\pgfpathrectangle{\pgfqpoint{1.432000in}{0.528000in}}{\pgfqpoint{3.696000in}{3.696000in}} %
\pgfusepath{clip}%
\pgfsetbuttcap%
\pgfsetroundjoin%
\definecolor{currentfill}{rgb}{0.170948,0.694384,0.493803}%
\pgfsetfillcolor{currentfill}%
\pgfsetlinewidth{0.000000pt}%
\definecolor{currentstroke}{rgb}{0.000000,0.000000,0.000000}%
\pgfsetstrokecolor{currentstroke}%
\pgfsetdash{}{0pt}%
\pgfpathmoveto{\pgfqpoint{1.604435in}{1.996645in}}%
\pgfpathlineto{\pgfqpoint{1.602218in}{2.000486in}}%
\pgfpathlineto{\pgfqpoint{1.597782in}{2.000486in}}%
\pgfpathlineto{\pgfqpoint{1.595565in}{1.996645in}}%
\pgfpathlineto{\pgfqpoint{1.597782in}{1.992804in}}%
\pgfpathlineto{\pgfqpoint{1.602218in}{1.992804in}}%
\pgfpathlineto{\pgfqpoint{1.604435in}{1.996645in}}%
\pgfpathlineto{\pgfqpoint{1.602218in}{2.000486in}}%
\pgfusepath{fill}%
\end{pgfscope}%
\begin{pgfscope}%
\pgfpathrectangle{\pgfqpoint{1.432000in}{0.528000in}}{\pgfqpoint{3.696000in}{3.696000in}} %
\pgfusepath{clip}%
\pgfsetbuttcap%
\pgfsetroundjoin%
\definecolor{currentfill}{rgb}{0.269944,0.014625,0.341379}%
\pgfsetfillcolor{currentfill}%
\pgfsetlinewidth{0.000000pt}%
\definecolor{currentstroke}{rgb}{0.000000,0.000000,0.000000}%
\pgfsetstrokecolor{currentstroke}%
\pgfsetdash{}{0pt}%
\pgfpathmoveto{\pgfqpoint{1.595565in}{1.996645in}}%
\pgfpathlineto{\pgfqpoint{1.595565in}{2.065115in}}%
\pgfpathlineto{\pgfqpoint{1.586694in}{2.060680in}}%
\pgfpathlineto{\pgfqpoint{1.600000in}{2.105032in}}%
\pgfpathlineto{\pgfqpoint{1.613306in}{2.060680in}}%
\pgfpathlineto{\pgfqpoint{1.604435in}{2.065115in}}%
\pgfpathlineto{\pgfqpoint{1.604435in}{1.996645in}}%
\pgfpathlineto{\pgfqpoint{1.595565in}{1.996645in}}%
\pgfusepath{fill}%
\end{pgfscope}%
\begin{pgfscope}%
\pgfpathrectangle{\pgfqpoint{1.432000in}{0.528000in}}{\pgfqpoint{3.696000in}{3.696000in}} %
\pgfusepath{clip}%
\pgfsetbuttcap%
\pgfsetroundjoin%
\definecolor{currentfill}{rgb}{0.283072,0.130895,0.449241}%
\pgfsetfillcolor{currentfill}%
\pgfsetlinewidth{0.000000pt}%
\definecolor{currentstroke}{rgb}{0.000000,0.000000,0.000000}%
\pgfsetstrokecolor{currentstroke}%
\pgfsetdash{}{0pt}%
\pgfpathmoveto{\pgfqpoint{1.708387in}{1.992210in}}%
\pgfpathlineto{\pgfqpoint{1.639917in}{1.992210in}}%
\pgfpathlineto{\pgfqpoint{1.644352in}{1.983340in}}%
\pgfpathlineto{\pgfqpoint{1.600000in}{1.996645in}}%
\pgfpathlineto{\pgfqpoint{1.644352in}{2.009951in}}%
\pgfpathlineto{\pgfqpoint{1.639917in}{2.001080in}}%
\pgfpathlineto{\pgfqpoint{1.708387in}{2.001080in}}%
\pgfpathlineto{\pgfqpoint{1.708387in}{1.992210in}}%
\pgfusepath{fill}%
\end{pgfscope}%
\begin{pgfscope}%
\pgfpathrectangle{\pgfqpoint{1.432000in}{0.528000in}}{\pgfqpoint{3.696000in}{3.696000in}} %
\pgfusepath{clip}%
\pgfsetbuttcap%
\pgfsetroundjoin%
\definecolor{currentfill}{rgb}{0.190631,0.407061,0.556089}%
\pgfsetfillcolor{currentfill}%
\pgfsetlinewidth{0.000000pt}%
\definecolor{currentstroke}{rgb}{0.000000,0.000000,0.000000}%
\pgfsetstrokecolor{currentstroke}%
\pgfsetdash{}{0pt}%
\pgfpathmoveto{\pgfqpoint{1.712822in}{1.996645in}}%
\pgfpathlineto{\pgfqpoint{1.710605in}{2.000486in}}%
\pgfpathlineto{\pgfqpoint{1.706169in}{2.000486in}}%
\pgfpathlineto{\pgfqpoint{1.703952in}{1.996645in}}%
\pgfpathlineto{\pgfqpoint{1.706169in}{1.992804in}}%
\pgfpathlineto{\pgfqpoint{1.710605in}{1.992804in}}%
\pgfpathlineto{\pgfqpoint{1.712822in}{1.996645in}}%
\pgfpathlineto{\pgfqpoint{1.710605in}{2.000486in}}%
\pgfusepath{fill}%
\end{pgfscope}%
\begin{pgfscope}%
\pgfpathrectangle{\pgfqpoint{1.432000in}{0.528000in}}{\pgfqpoint{3.696000in}{3.696000in}} %
\pgfusepath{clip}%
\pgfsetbuttcap%
\pgfsetroundjoin%
\definecolor{currentfill}{rgb}{0.244972,0.287675,0.537260}%
\pgfsetfillcolor{currentfill}%
\pgfsetlinewidth{0.000000pt}%
\definecolor{currentstroke}{rgb}{0.000000,0.000000,0.000000}%
\pgfsetstrokecolor{currentstroke}%
\pgfsetdash{}{0pt}%
\pgfpathmoveto{\pgfqpoint{1.816774in}{1.992210in}}%
\pgfpathlineto{\pgfqpoint{1.748304in}{1.992210in}}%
\pgfpathlineto{\pgfqpoint{1.752739in}{1.983340in}}%
\pgfpathlineto{\pgfqpoint{1.708387in}{1.996645in}}%
\pgfpathlineto{\pgfqpoint{1.752739in}{2.009951in}}%
\pgfpathlineto{\pgfqpoint{1.748304in}{2.001080in}}%
\pgfpathlineto{\pgfqpoint{1.816774in}{2.001080in}}%
\pgfpathlineto{\pgfqpoint{1.816774in}{1.992210in}}%
\pgfusepath{fill}%
\end{pgfscope}%
\begin{pgfscope}%
\pgfpathrectangle{\pgfqpoint{1.432000in}{0.528000in}}{\pgfqpoint{3.696000in}{3.696000in}} %
\pgfusepath{clip}%
\pgfsetbuttcap%
\pgfsetroundjoin%
\definecolor{currentfill}{rgb}{0.281412,0.155834,0.469201}%
\pgfsetfillcolor{currentfill}%
\pgfsetlinewidth{0.000000pt}%
\definecolor{currentstroke}{rgb}{0.000000,0.000000,0.000000}%
\pgfsetstrokecolor{currentstroke}%
\pgfsetdash{}{0pt}%
\pgfpathmoveto{\pgfqpoint{1.821209in}{1.996645in}}%
\pgfpathlineto{\pgfqpoint{1.818992in}{2.000486in}}%
\pgfpathlineto{\pgfqpoint{1.814557in}{2.000486in}}%
\pgfpathlineto{\pgfqpoint{1.812339in}{1.996645in}}%
\pgfpathlineto{\pgfqpoint{1.814557in}{1.992804in}}%
\pgfpathlineto{\pgfqpoint{1.818992in}{1.992804in}}%
\pgfpathlineto{\pgfqpoint{1.821209in}{1.996645in}}%
\pgfpathlineto{\pgfqpoint{1.818992in}{2.000486in}}%
\pgfusepath{fill}%
\end{pgfscope}%
\begin{pgfscope}%
\pgfpathrectangle{\pgfqpoint{1.432000in}{0.528000in}}{\pgfqpoint{3.696000in}{3.696000in}} %
\pgfusepath{clip}%
\pgfsetbuttcap%
\pgfsetroundjoin%
\definecolor{currentfill}{rgb}{0.235526,0.309527,0.542944}%
\pgfsetfillcolor{currentfill}%
\pgfsetlinewidth{0.000000pt}%
\definecolor{currentstroke}{rgb}{0.000000,0.000000,0.000000}%
\pgfsetstrokecolor{currentstroke}%
\pgfsetdash{}{0pt}%
\pgfpathmoveto{\pgfqpoint{1.925161in}{1.992210in}}%
\pgfpathlineto{\pgfqpoint{1.856691in}{1.992210in}}%
\pgfpathlineto{\pgfqpoint{1.861126in}{1.983340in}}%
\pgfpathlineto{\pgfqpoint{1.816774in}{1.996645in}}%
\pgfpathlineto{\pgfqpoint{1.861126in}{2.009951in}}%
\pgfpathlineto{\pgfqpoint{1.856691in}{2.001080in}}%
\pgfpathlineto{\pgfqpoint{1.925161in}{2.001080in}}%
\pgfpathlineto{\pgfqpoint{1.925161in}{1.992210in}}%
\pgfusepath{fill}%
\end{pgfscope}%
\begin{pgfscope}%
\pgfpathrectangle{\pgfqpoint{1.432000in}{0.528000in}}{\pgfqpoint{3.696000in}{3.696000in}} %
\pgfusepath{clip}%
\pgfsetbuttcap%
\pgfsetroundjoin%
\definecolor{currentfill}{rgb}{0.203063,0.379716,0.553925}%
\pgfsetfillcolor{currentfill}%
\pgfsetlinewidth{0.000000pt}%
\definecolor{currentstroke}{rgb}{0.000000,0.000000,0.000000}%
\pgfsetstrokecolor{currentstroke}%
\pgfsetdash{}{0pt}%
\pgfpathmoveto{\pgfqpoint{2.033548in}{1.992210in}}%
\pgfpathlineto{\pgfqpoint{1.965078in}{1.992210in}}%
\pgfpathlineto{\pgfqpoint{1.969513in}{1.983340in}}%
\pgfpathlineto{\pgfqpoint{1.925161in}{1.996645in}}%
\pgfpathlineto{\pgfqpoint{1.969513in}{2.009951in}}%
\pgfpathlineto{\pgfqpoint{1.965078in}{2.001080in}}%
\pgfpathlineto{\pgfqpoint{2.033548in}{2.001080in}}%
\pgfpathlineto{\pgfqpoint{2.033548in}{1.992210in}}%
\pgfusepath{fill}%
\end{pgfscope}%
\begin{pgfscope}%
\pgfpathrectangle{\pgfqpoint{1.432000in}{0.528000in}}{\pgfqpoint{3.696000in}{3.696000in}} %
\pgfusepath{clip}%
\pgfsetbuttcap%
\pgfsetroundjoin%
\definecolor{currentfill}{rgb}{0.276022,0.044167,0.370164}%
\pgfsetfillcolor{currentfill}%
\pgfsetlinewidth{0.000000pt}%
\definecolor{currentstroke}{rgb}{0.000000,0.000000,0.000000}%
\pgfsetstrokecolor{currentstroke}%
\pgfsetdash{}{0pt}%
\pgfpathmoveto{\pgfqpoint{2.030412in}{1.993509in}}%
\pgfpathlineto{\pgfqpoint{1.950251in}{2.073671in}}%
\pgfpathlineto{\pgfqpoint{1.947114in}{2.064262in}}%
\pgfpathlineto{\pgfqpoint{1.925161in}{2.105032in}}%
\pgfpathlineto{\pgfqpoint{1.965931in}{2.083079in}}%
\pgfpathlineto{\pgfqpoint{1.956523in}{2.079943in}}%
\pgfpathlineto{\pgfqpoint{2.036685in}{1.999781in}}%
\pgfpathlineto{\pgfqpoint{2.030412in}{1.993509in}}%
\pgfusepath{fill}%
\end{pgfscope}%
\begin{pgfscope}%
\pgfpathrectangle{\pgfqpoint{1.432000in}{0.528000in}}{\pgfqpoint{3.696000in}{3.696000in}} %
\pgfusepath{clip}%
\pgfsetbuttcap%
\pgfsetroundjoin%
\definecolor{currentfill}{rgb}{0.206756,0.371758,0.553117}%
\pgfsetfillcolor{currentfill}%
\pgfsetlinewidth{0.000000pt}%
\definecolor{currentstroke}{rgb}{0.000000,0.000000,0.000000}%
\pgfsetstrokecolor{currentstroke}%
\pgfsetdash{}{0pt}%
\pgfpathmoveto{\pgfqpoint{2.141935in}{1.992210in}}%
\pgfpathlineto{\pgfqpoint{2.073465in}{1.992210in}}%
\pgfpathlineto{\pgfqpoint{2.077900in}{1.983340in}}%
\pgfpathlineto{\pgfqpoint{2.033548in}{1.996645in}}%
\pgfpathlineto{\pgfqpoint{2.077900in}{2.009951in}}%
\pgfpathlineto{\pgfqpoint{2.073465in}{2.001080in}}%
\pgfpathlineto{\pgfqpoint{2.141935in}{2.001080in}}%
\pgfpathlineto{\pgfqpoint{2.141935in}{1.992210in}}%
\pgfusepath{fill}%
\end{pgfscope}%
\begin{pgfscope}%
\pgfpathrectangle{\pgfqpoint{1.432000in}{0.528000in}}{\pgfqpoint{3.696000in}{3.696000in}} %
\pgfusepath{clip}%
\pgfsetbuttcap%
\pgfsetroundjoin%
\definecolor{currentfill}{rgb}{0.280267,0.073417,0.397163}%
\pgfsetfillcolor{currentfill}%
\pgfsetlinewidth{0.000000pt}%
\definecolor{currentstroke}{rgb}{0.000000,0.000000,0.000000}%
\pgfsetstrokecolor{currentstroke}%
\pgfsetdash{}{0pt}%
\pgfpathmoveto{\pgfqpoint{2.138799in}{1.993509in}}%
\pgfpathlineto{\pgfqpoint{2.058638in}{2.073671in}}%
\pgfpathlineto{\pgfqpoint{2.055502in}{2.064262in}}%
\pgfpathlineto{\pgfqpoint{2.033548in}{2.105032in}}%
\pgfpathlineto{\pgfqpoint{2.074318in}{2.083079in}}%
\pgfpathlineto{\pgfqpoint{2.064910in}{2.079943in}}%
\pgfpathlineto{\pgfqpoint{2.145072in}{1.999781in}}%
\pgfpathlineto{\pgfqpoint{2.138799in}{1.993509in}}%
\pgfusepath{fill}%
\end{pgfscope}%
\begin{pgfscope}%
\pgfpathrectangle{\pgfqpoint{1.432000in}{0.528000in}}{\pgfqpoint{3.696000in}{3.696000in}} %
\pgfusepath{clip}%
\pgfsetbuttcap%
\pgfsetroundjoin%
\definecolor{currentfill}{rgb}{0.260571,0.246922,0.522828}%
\pgfsetfillcolor{currentfill}%
\pgfsetlinewidth{0.000000pt}%
\definecolor{currentstroke}{rgb}{0.000000,0.000000,0.000000}%
\pgfsetstrokecolor{currentstroke}%
\pgfsetdash{}{0pt}%
\pgfpathmoveto{\pgfqpoint{2.250323in}{1.992210in}}%
\pgfpathlineto{\pgfqpoint{2.181852in}{1.992210in}}%
\pgfpathlineto{\pgfqpoint{2.186287in}{1.983340in}}%
\pgfpathlineto{\pgfqpoint{2.141935in}{1.996645in}}%
\pgfpathlineto{\pgfqpoint{2.186287in}{2.009951in}}%
\pgfpathlineto{\pgfqpoint{2.181852in}{2.001080in}}%
\pgfpathlineto{\pgfqpoint{2.250323in}{2.001080in}}%
\pgfpathlineto{\pgfqpoint{2.250323in}{1.992210in}}%
\pgfusepath{fill}%
\end{pgfscope}%
\begin{pgfscope}%
\pgfpathrectangle{\pgfqpoint{1.432000in}{0.528000in}}{\pgfqpoint{3.696000in}{3.696000in}} %
\pgfusepath{clip}%
\pgfsetbuttcap%
\pgfsetroundjoin%
\definecolor{currentfill}{rgb}{0.273809,0.031497,0.358853}%
\pgfsetfillcolor{currentfill}%
\pgfsetlinewidth{0.000000pt}%
\definecolor{currentstroke}{rgb}{0.000000,0.000000,0.000000}%
\pgfsetstrokecolor{currentstroke}%
\pgfsetdash{}{0pt}%
\pgfpathmoveto{\pgfqpoint{2.247186in}{1.993509in}}%
\pgfpathlineto{\pgfqpoint{2.167025in}{2.073671in}}%
\pgfpathlineto{\pgfqpoint{2.163889in}{2.064262in}}%
\pgfpathlineto{\pgfqpoint{2.141935in}{2.105032in}}%
\pgfpathlineto{\pgfqpoint{2.182706in}{2.083079in}}%
\pgfpathlineto{\pgfqpoint{2.173297in}{2.079943in}}%
\pgfpathlineto{\pgfqpoint{2.253459in}{1.999781in}}%
\pgfpathlineto{\pgfqpoint{2.247186in}{1.993509in}}%
\pgfusepath{fill}%
\end{pgfscope}%
\begin{pgfscope}%
\pgfpathrectangle{\pgfqpoint{1.432000in}{0.528000in}}{\pgfqpoint{3.696000in}{3.696000in}} %
\pgfusepath{clip}%
\pgfsetbuttcap%
\pgfsetroundjoin%
\definecolor{currentfill}{rgb}{0.266580,0.228262,0.514349}%
\pgfsetfillcolor{currentfill}%
\pgfsetlinewidth{0.000000pt}%
\definecolor{currentstroke}{rgb}{0.000000,0.000000,0.000000}%
\pgfsetstrokecolor{currentstroke}%
\pgfsetdash{}{0pt}%
\pgfpathmoveto{\pgfqpoint{2.358710in}{1.992210in}}%
\pgfpathlineto{\pgfqpoint{2.181852in}{1.992210in}}%
\pgfpathlineto{\pgfqpoint{2.186287in}{1.983340in}}%
\pgfpathlineto{\pgfqpoint{2.141935in}{1.996645in}}%
\pgfpathlineto{\pgfqpoint{2.186287in}{2.009951in}}%
\pgfpathlineto{\pgfqpoint{2.181852in}{2.001080in}}%
\pgfpathlineto{\pgfqpoint{2.358710in}{2.001080in}}%
\pgfpathlineto{\pgfqpoint{2.358710in}{1.992210in}}%
\pgfusepath{fill}%
\end{pgfscope}%
\begin{pgfscope}%
\pgfpathrectangle{\pgfqpoint{1.432000in}{0.528000in}}{\pgfqpoint{3.696000in}{3.696000in}} %
\pgfusepath{clip}%
\pgfsetbuttcap%
\pgfsetroundjoin%
\definecolor{currentfill}{rgb}{0.280267,0.073417,0.397163}%
\pgfsetfillcolor{currentfill}%
\pgfsetlinewidth{0.000000pt}%
\definecolor{currentstroke}{rgb}{0.000000,0.000000,0.000000}%
\pgfsetstrokecolor{currentstroke}%
\pgfsetdash{}{0pt}%
\pgfpathmoveto{\pgfqpoint{2.358710in}{1.992210in}}%
\pgfpathlineto{\pgfqpoint{2.290239in}{1.992210in}}%
\pgfpathlineto{\pgfqpoint{2.294675in}{1.983340in}}%
\pgfpathlineto{\pgfqpoint{2.250323in}{1.996645in}}%
\pgfpathlineto{\pgfqpoint{2.294675in}{2.009951in}}%
\pgfpathlineto{\pgfqpoint{2.290239in}{2.001080in}}%
\pgfpathlineto{\pgfqpoint{2.358710in}{2.001080in}}%
\pgfpathlineto{\pgfqpoint{2.358710in}{1.992210in}}%
\pgfusepath{fill}%
\end{pgfscope}%
\begin{pgfscope}%
\pgfpathrectangle{\pgfqpoint{1.432000in}{0.528000in}}{\pgfqpoint{3.696000in}{3.696000in}} %
\pgfusepath{clip}%
\pgfsetbuttcap%
\pgfsetroundjoin%
\definecolor{currentfill}{rgb}{0.269944,0.014625,0.341379}%
\pgfsetfillcolor{currentfill}%
\pgfsetlinewidth{0.000000pt}%
\definecolor{currentstroke}{rgb}{0.000000,0.000000,0.000000}%
\pgfsetstrokecolor{currentstroke}%
\pgfsetdash{}{0pt}%
\pgfpathmoveto{\pgfqpoint{2.356726in}{1.992678in}}%
\pgfpathlineto{\pgfqpoint{2.175655in}{2.083214in}}%
\pgfpathlineto{\pgfqpoint{2.175655in}{2.073297in}}%
\pgfpathlineto{\pgfqpoint{2.141935in}{2.105032in}}%
\pgfpathlineto{\pgfqpoint{2.187556in}{2.097098in}}%
\pgfpathlineto{\pgfqpoint{2.179622in}{2.091148in}}%
\pgfpathlineto{\pgfqpoint{2.360693in}{2.000612in}}%
\pgfpathlineto{\pgfqpoint{2.356726in}{1.992678in}}%
\pgfusepath{fill}%
\end{pgfscope}%
\begin{pgfscope}%
\pgfpathrectangle{\pgfqpoint{1.432000in}{0.528000in}}{\pgfqpoint{3.696000in}{3.696000in}} %
\pgfusepath{clip}%
\pgfsetbuttcap%
\pgfsetroundjoin%
\definecolor{currentfill}{rgb}{0.195860,0.395433,0.555276}%
\pgfsetfillcolor{currentfill}%
\pgfsetlinewidth{0.000000pt}%
\definecolor{currentstroke}{rgb}{0.000000,0.000000,0.000000}%
\pgfsetstrokecolor{currentstroke}%
\pgfsetdash{}{0pt}%
\pgfpathmoveto{\pgfqpoint{2.467097in}{1.992210in}}%
\pgfpathlineto{\pgfqpoint{2.290239in}{1.992210in}}%
\pgfpathlineto{\pgfqpoint{2.294675in}{1.983340in}}%
\pgfpathlineto{\pgfqpoint{2.250323in}{1.996645in}}%
\pgfpathlineto{\pgfqpoint{2.294675in}{2.009951in}}%
\pgfpathlineto{\pgfqpoint{2.290239in}{2.001080in}}%
\pgfpathlineto{\pgfqpoint{2.467097in}{2.001080in}}%
\pgfpathlineto{\pgfqpoint{2.467097in}{1.992210in}}%
\pgfusepath{fill}%
\end{pgfscope}%
\begin{pgfscope}%
\pgfpathrectangle{\pgfqpoint{1.432000in}{0.528000in}}{\pgfqpoint{3.696000in}{3.696000in}} %
\pgfusepath{clip}%
\pgfsetbuttcap%
\pgfsetroundjoin%
\definecolor{currentfill}{rgb}{0.277941,0.056324,0.381191}%
\pgfsetfillcolor{currentfill}%
\pgfsetlinewidth{0.000000pt}%
\definecolor{currentstroke}{rgb}{0.000000,0.000000,0.000000}%
\pgfsetstrokecolor{currentstroke}%
\pgfsetdash{}{0pt}%
\pgfpathmoveto{\pgfqpoint{2.465113in}{1.992678in}}%
\pgfpathlineto{\pgfqpoint{2.284042in}{2.083214in}}%
\pgfpathlineto{\pgfqpoint{2.284042in}{2.073297in}}%
\pgfpathlineto{\pgfqpoint{2.250323in}{2.105032in}}%
\pgfpathlineto{\pgfqpoint{2.295943in}{2.097098in}}%
\pgfpathlineto{\pgfqpoint{2.288009in}{2.091148in}}%
\pgfpathlineto{\pgfqpoint{2.469080in}{2.000612in}}%
\pgfpathlineto{\pgfqpoint{2.465113in}{1.992678in}}%
\pgfusepath{fill}%
\end{pgfscope}%
\begin{pgfscope}%
\pgfpathrectangle{\pgfqpoint{1.432000in}{0.528000in}}{\pgfqpoint{3.696000in}{3.696000in}} %
\pgfusepath{clip}%
\pgfsetbuttcap%
\pgfsetroundjoin%
\definecolor{currentfill}{rgb}{0.237441,0.305202,0.541921}%
\pgfsetfillcolor{currentfill}%
\pgfsetlinewidth{0.000000pt}%
\definecolor{currentstroke}{rgb}{0.000000,0.000000,0.000000}%
\pgfsetstrokecolor{currentstroke}%
\pgfsetdash{}{0pt}%
\pgfpathmoveto{\pgfqpoint{2.575484in}{1.992210in}}%
\pgfpathlineto{\pgfqpoint{2.398626in}{1.992210in}}%
\pgfpathlineto{\pgfqpoint{2.403062in}{1.983340in}}%
\pgfpathlineto{\pgfqpoint{2.358710in}{1.996645in}}%
\pgfpathlineto{\pgfqpoint{2.403062in}{2.009951in}}%
\pgfpathlineto{\pgfqpoint{2.398626in}{2.001080in}}%
\pgfpathlineto{\pgfqpoint{2.575484in}{2.001080in}}%
\pgfpathlineto{\pgfqpoint{2.575484in}{1.992210in}}%
\pgfusepath{fill}%
\end{pgfscope}%
\begin{pgfscope}%
\pgfpathrectangle{\pgfqpoint{1.432000in}{0.528000in}}{\pgfqpoint{3.696000in}{3.696000in}} %
\pgfusepath{clip}%
\pgfsetbuttcap%
\pgfsetroundjoin%
\definecolor{currentfill}{rgb}{0.283072,0.130895,0.449241}%
\pgfsetfillcolor{currentfill}%
\pgfsetlinewidth{0.000000pt}%
\definecolor{currentstroke}{rgb}{0.000000,0.000000,0.000000}%
\pgfsetstrokecolor{currentstroke}%
\pgfsetdash{}{0pt}%
\pgfpathmoveto{\pgfqpoint{2.573500in}{1.992678in}}%
\pgfpathlineto{\pgfqpoint{2.392429in}{2.083214in}}%
\pgfpathlineto{\pgfqpoint{2.392429in}{2.073297in}}%
\pgfpathlineto{\pgfqpoint{2.358710in}{2.105032in}}%
\pgfpathlineto{\pgfqpoint{2.404330in}{2.097098in}}%
\pgfpathlineto{\pgfqpoint{2.396396in}{2.091148in}}%
\pgfpathlineto{\pgfqpoint{2.577467in}{2.000612in}}%
\pgfpathlineto{\pgfqpoint{2.573500in}{1.992678in}}%
\pgfusepath{fill}%
\end{pgfscope}%
\begin{pgfscope}%
\pgfpathrectangle{\pgfqpoint{1.432000in}{0.528000in}}{\pgfqpoint{3.696000in}{3.696000in}} %
\pgfusepath{clip}%
\pgfsetbuttcap%
\pgfsetroundjoin%
\definecolor{currentfill}{rgb}{0.280255,0.165693,0.476498}%
\pgfsetfillcolor{currentfill}%
\pgfsetlinewidth{0.000000pt}%
\definecolor{currentstroke}{rgb}{0.000000,0.000000,0.000000}%
\pgfsetstrokecolor{currentstroke}%
\pgfsetdash{}{0pt}%
\pgfpathmoveto{\pgfqpoint{2.683871in}{1.992210in}}%
\pgfpathlineto{\pgfqpoint{2.507014in}{1.992210in}}%
\pgfpathlineto{\pgfqpoint{2.511449in}{1.983340in}}%
\pgfpathlineto{\pgfqpoint{2.467097in}{1.996645in}}%
\pgfpathlineto{\pgfqpoint{2.511449in}{2.009951in}}%
\pgfpathlineto{\pgfqpoint{2.507014in}{2.001080in}}%
\pgfpathlineto{\pgfqpoint{2.683871in}{2.001080in}}%
\pgfpathlineto{\pgfqpoint{2.683871in}{1.992210in}}%
\pgfusepath{fill}%
\end{pgfscope}%
\begin{pgfscope}%
\pgfpathrectangle{\pgfqpoint{1.432000in}{0.528000in}}{\pgfqpoint{3.696000in}{3.696000in}} %
\pgfusepath{clip}%
\pgfsetbuttcap%
\pgfsetroundjoin%
\definecolor{currentfill}{rgb}{0.192357,0.403199,0.555836}%
\pgfsetfillcolor{currentfill}%
\pgfsetlinewidth{0.000000pt}%
\definecolor{currentstroke}{rgb}{0.000000,0.000000,0.000000}%
\pgfsetstrokecolor{currentstroke}%
\pgfsetdash{}{0pt}%
\pgfpathmoveto{\pgfqpoint{2.681887in}{1.992678in}}%
\pgfpathlineto{\pgfqpoint{2.500816in}{2.083214in}}%
\pgfpathlineto{\pgfqpoint{2.500816in}{2.073297in}}%
\pgfpathlineto{\pgfqpoint{2.467097in}{2.105032in}}%
\pgfpathlineto{\pgfqpoint{2.512717in}{2.097098in}}%
\pgfpathlineto{\pgfqpoint{2.504783in}{2.091148in}}%
\pgfpathlineto{\pgfqpoint{2.685854in}{2.000612in}}%
\pgfpathlineto{\pgfqpoint{2.681887in}{1.992678in}}%
\pgfusepath{fill}%
\end{pgfscope}%
\begin{pgfscope}%
\pgfpathrectangle{\pgfqpoint{1.432000in}{0.528000in}}{\pgfqpoint{3.696000in}{3.696000in}} %
\pgfusepath{clip}%
\pgfsetbuttcap%
\pgfsetroundjoin%
\definecolor{currentfill}{rgb}{0.250425,0.274290,0.533103}%
\pgfsetfillcolor{currentfill}%
\pgfsetlinewidth{0.000000pt}%
\definecolor{currentstroke}{rgb}{0.000000,0.000000,0.000000}%
\pgfsetstrokecolor{currentstroke}%
\pgfsetdash{}{0pt}%
\pgfpathmoveto{\pgfqpoint{2.792258in}{1.992210in}}%
\pgfpathlineto{\pgfqpoint{2.615401in}{1.992210in}}%
\pgfpathlineto{\pgfqpoint{2.619836in}{1.983340in}}%
\pgfpathlineto{\pgfqpoint{2.575484in}{1.996645in}}%
\pgfpathlineto{\pgfqpoint{2.619836in}{2.009951in}}%
\pgfpathlineto{\pgfqpoint{2.615401in}{2.001080in}}%
\pgfpathlineto{\pgfqpoint{2.792258in}{2.001080in}}%
\pgfpathlineto{\pgfqpoint{2.792258in}{1.992210in}}%
\pgfusepath{fill}%
\end{pgfscope}%
\begin{pgfscope}%
\pgfpathrectangle{\pgfqpoint{1.432000in}{0.528000in}}{\pgfqpoint{3.696000in}{3.696000in}} %
\pgfusepath{clip}%
\pgfsetbuttcap%
\pgfsetroundjoin%
\definecolor{currentfill}{rgb}{0.153364,0.497000,0.557724}%
\pgfsetfillcolor{currentfill}%
\pgfsetlinewidth{0.000000pt}%
\definecolor{currentstroke}{rgb}{0.000000,0.000000,0.000000}%
\pgfsetstrokecolor{currentstroke}%
\pgfsetdash{}{0pt}%
\pgfpathmoveto{\pgfqpoint{2.790275in}{1.992678in}}%
\pgfpathlineto{\pgfqpoint{2.609203in}{2.083214in}}%
\pgfpathlineto{\pgfqpoint{2.609203in}{2.073297in}}%
\pgfpathlineto{\pgfqpoint{2.575484in}{2.105032in}}%
\pgfpathlineto{\pgfqpoint{2.621104in}{2.097098in}}%
\pgfpathlineto{\pgfqpoint{2.613170in}{2.091148in}}%
\pgfpathlineto{\pgfqpoint{2.794242in}{2.000612in}}%
\pgfpathlineto{\pgfqpoint{2.790275in}{1.992678in}}%
\pgfusepath{fill}%
\end{pgfscope}%
\begin{pgfscope}%
\pgfpathrectangle{\pgfqpoint{1.432000in}{0.528000in}}{\pgfqpoint{3.696000in}{3.696000in}} %
\pgfusepath{clip}%
\pgfsetbuttcap%
\pgfsetroundjoin%
\definecolor{currentfill}{rgb}{0.271828,0.209303,0.504434}%
\pgfsetfillcolor{currentfill}%
\pgfsetlinewidth{0.000000pt}%
\definecolor{currentstroke}{rgb}{0.000000,0.000000,0.000000}%
\pgfsetstrokecolor{currentstroke}%
\pgfsetdash{}{0pt}%
\pgfpathmoveto{\pgfqpoint{2.900645in}{1.992210in}}%
\pgfpathlineto{\pgfqpoint{2.723788in}{1.992210in}}%
\pgfpathlineto{\pgfqpoint{2.728223in}{1.983340in}}%
\pgfpathlineto{\pgfqpoint{2.683871in}{1.996645in}}%
\pgfpathlineto{\pgfqpoint{2.728223in}{2.009951in}}%
\pgfpathlineto{\pgfqpoint{2.723788in}{2.001080in}}%
\pgfpathlineto{\pgfqpoint{2.900645in}{2.001080in}}%
\pgfpathlineto{\pgfqpoint{2.900645in}{1.992210in}}%
\pgfusepath{fill}%
\end{pgfscope}%
\begin{pgfscope}%
\pgfpathrectangle{\pgfqpoint{1.432000in}{0.528000in}}{\pgfqpoint{3.696000in}{3.696000in}} %
\pgfusepath{clip}%
\pgfsetbuttcap%
\pgfsetroundjoin%
\definecolor{currentfill}{rgb}{0.121148,0.592739,0.544641}%
\pgfsetfillcolor{currentfill}%
\pgfsetlinewidth{0.000000pt}%
\definecolor{currentstroke}{rgb}{0.000000,0.000000,0.000000}%
\pgfsetstrokecolor{currentstroke}%
\pgfsetdash{}{0pt}%
\pgfpathmoveto{\pgfqpoint{2.898662in}{1.992678in}}%
\pgfpathlineto{\pgfqpoint{2.717590in}{2.083214in}}%
\pgfpathlineto{\pgfqpoint{2.717590in}{2.073297in}}%
\pgfpathlineto{\pgfqpoint{2.683871in}{2.105032in}}%
\pgfpathlineto{\pgfqpoint{2.729491in}{2.097098in}}%
\pgfpathlineto{\pgfqpoint{2.721557in}{2.091148in}}%
\pgfpathlineto{\pgfqpoint{2.902629in}{2.000612in}}%
\pgfpathlineto{\pgfqpoint{2.898662in}{1.992678in}}%
\pgfusepath{fill}%
\end{pgfscope}%
\begin{pgfscope}%
\pgfpathrectangle{\pgfqpoint{1.432000in}{0.528000in}}{\pgfqpoint{3.696000in}{3.696000in}} %
\pgfusepath{clip}%
\pgfsetbuttcap%
\pgfsetroundjoin%
\definecolor{currentfill}{rgb}{0.146180,0.515413,0.556823}%
\pgfsetfillcolor{currentfill}%
\pgfsetlinewidth{0.000000pt}%
\definecolor{currentstroke}{rgb}{0.000000,0.000000,0.000000}%
\pgfsetstrokecolor{currentstroke}%
\pgfsetdash{}{0pt}%
\pgfpathmoveto{\pgfqpoint{3.007049in}{1.992678in}}%
\pgfpathlineto{\pgfqpoint{2.825977in}{2.083214in}}%
\pgfpathlineto{\pgfqpoint{2.825977in}{2.073297in}}%
\pgfpathlineto{\pgfqpoint{2.792258in}{2.105032in}}%
\pgfpathlineto{\pgfqpoint{2.837878in}{2.097098in}}%
\pgfpathlineto{\pgfqpoint{2.829944in}{2.091148in}}%
\pgfpathlineto{\pgfqpoint{3.011016in}{2.000612in}}%
\pgfpathlineto{\pgfqpoint{3.007049in}{1.992678in}}%
\pgfusepath{fill}%
\end{pgfscope}%
\begin{pgfscope}%
\pgfpathrectangle{\pgfqpoint{1.432000in}{0.528000in}}{\pgfqpoint{3.696000in}{3.696000in}} %
\pgfusepath{clip}%
\pgfsetbuttcap%
\pgfsetroundjoin%
\definecolor{currentfill}{rgb}{0.258965,0.251537,0.524736}%
\pgfsetfillcolor{currentfill}%
\pgfsetlinewidth{0.000000pt}%
\definecolor{currentstroke}{rgb}{0.000000,0.000000,0.000000}%
\pgfsetstrokecolor{currentstroke}%
\pgfsetdash{}{0pt}%
\pgfpathmoveto{\pgfqpoint{3.005896in}{1.993509in}}%
\pgfpathlineto{\pgfqpoint{2.817347in}{2.182058in}}%
\pgfpathlineto{\pgfqpoint{2.814211in}{2.172649in}}%
\pgfpathlineto{\pgfqpoint{2.792258in}{2.213419in}}%
\pgfpathlineto{\pgfqpoint{2.833028in}{2.191466in}}%
\pgfpathlineto{\pgfqpoint{2.823620in}{2.188330in}}%
\pgfpathlineto{\pgfqpoint{3.012168in}{1.999781in}}%
\pgfpathlineto{\pgfqpoint{3.005896in}{1.993509in}}%
\pgfusepath{fill}%
\end{pgfscope}%
\begin{pgfscope}%
\pgfpathrectangle{\pgfqpoint{1.432000in}{0.528000in}}{\pgfqpoint{3.696000in}{3.696000in}} %
\pgfusepath{clip}%
\pgfsetbuttcap%
\pgfsetroundjoin%
\definecolor{currentfill}{rgb}{0.192357,0.403199,0.555836}%
\pgfsetfillcolor{currentfill}%
\pgfsetlinewidth{0.000000pt}%
\definecolor{currentstroke}{rgb}{0.000000,0.000000,0.000000}%
\pgfsetstrokecolor{currentstroke}%
\pgfsetdash{}{0pt}%
\pgfpathmoveto{\pgfqpoint{3.115436in}{1.992678in}}%
\pgfpathlineto{\pgfqpoint{2.934364in}{2.083214in}}%
\pgfpathlineto{\pgfqpoint{2.934364in}{2.073297in}}%
\pgfpathlineto{\pgfqpoint{2.900645in}{2.105032in}}%
\pgfpathlineto{\pgfqpoint{2.946265in}{2.097098in}}%
\pgfpathlineto{\pgfqpoint{2.938331in}{2.091148in}}%
\pgfpathlineto{\pgfqpoint{3.119403in}{2.000612in}}%
\pgfpathlineto{\pgfqpoint{3.115436in}{1.992678in}}%
\pgfusepath{fill}%
\end{pgfscope}%
\begin{pgfscope}%
\pgfpathrectangle{\pgfqpoint{1.432000in}{0.528000in}}{\pgfqpoint{3.696000in}{3.696000in}} %
\pgfusepath{clip}%
\pgfsetbuttcap%
\pgfsetroundjoin%
\definecolor{currentfill}{rgb}{0.250425,0.274290,0.533103}%
\pgfsetfillcolor{currentfill}%
\pgfsetlinewidth{0.000000pt}%
\definecolor{currentstroke}{rgb}{0.000000,0.000000,0.000000}%
\pgfsetstrokecolor{currentstroke}%
\pgfsetdash{}{0pt}%
\pgfpathmoveto{\pgfqpoint{3.114283in}{1.993509in}}%
\pgfpathlineto{\pgfqpoint{2.925734in}{2.182058in}}%
\pgfpathlineto{\pgfqpoint{2.922598in}{2.172649in}}%
\pgfpathlineto{\pgfqpoint{2.900645in}{2.213419in}}%
\pgfpathlineto{\pgfqpoint{2.941415in}{2.191466in}}%
\pgfpathlineto{\pgfqpoint{2.932007in}{2.188330in}}%
\pgfpathlineto{\pgfqpoint{3.120556in}{1.999781in}}%
\pgfpathlineto{\pgfqpoint{3.114283in}{1.993509in}}%
\pgfusepath{fill}%
\end{pgfscope}%
\begin{pgfscope}%
\pgfpathrectangle{\pgfqpoint{1.432000in}{0.528000in}}{\pgfqpoint{3.696000in}{3.696000in}} %
\pgfusepath{clip}%
\pgfsetbuttcap%
\pgfsetroundjoin%
\definecolor{currentfill}{rgb}{0.257322,0.256130,0.526563}%
\pgfsetfillcolor{currentfill}%
\pgfsetlinewidth{0.000000pt}%
\definecolor{currentstroke}{rgb}{0.000000,0.000000,0.000000}%
\pgfsetstrokecolor{currentstroke}%
\pgfsetdash{}{0pt}%
\pgfpathmoveto{\pgfqpoint{3.223823in}{1.992678in}}%
\pgfpathlineto{\pgfqpoint{3.042751in}{2.083214in}}%
\pgfpathlineto{\pgfqpoint{3.042751in}{2.073297in}}%
\pgfpathlineto{\pgfqpoint{3.009032in}{2.105032in}}%
\pgfpathlineto{\pgfqpoint{3.054652in}{2.097098in}}%
\pgfpathlineto{\pgfqpoint{3.046718in}{2.091148in}}%
\pgfpathlineto{\pgfqpoint{3.227790in}{2.000612in}}%
\pgfpathlineto{\pgfqpoint{3.223823in}{1.992678in}}%
\pgfusepath{fill}%
\end{pgfscope}%
\begin{pgfscope}%
\pgfpathrectangle{\pgfqpoint{1.432000in}{0.528000in}}{\pgfqpoint{3.696000in}{3.696000in}} %
\pgfusepath{clip}%
\pgfsetbuttcap%
\pgfsetroundjoin%
\definecolor{currentfill}{rgb}{0.172719,0.448791,0.557885}%
\pgfsetfillcolor{currentfill}%
\pgfsetlinewidth{0.000000pt}%
\definecolor{currentstroke}{rgb}{0.000000,0.000000,0.000000}%
\pgfsetstrokecolor{currentstroke}%
\pgfsetdash{}{0pt}%
\pgfpathmoveto{\pgfqpoint{3.222670in}{1.993509in}}%
\pgfpathlineto{\pgfqpoint{3.034122in}{2.182058in}}%
\pgfpathlineto{\pgfqpoint{3.030985in}{2.172649in}}%
\pgfpathlineto{\pgfqpoint{3.009032in}{2.213419in}}%
\pgfpathlineto{\pgfqpoint{3.049802in}{2.191466in}}%
\pgfpathlineto{\pgfqpoint{3.040394in}{2.188330in}}%
\pgfpathlineto{\pgfqpoint{3.228943in}{1.999781in}}%
\pgfpathlineto{\pgfqpoint{3.222670in}{1.993509in}}%
\pgfusepath{fill}%
\end{pgfscope}%
\begin{pgfscope}%
\pgfpathrectangle{\pgfqpoint{1.432000in}{0.528000in}}{\pgfqpoint{3.696000in}{3.696000in}} %
\pgfusepath{clip}%
\pgfsetbuttcap%
\pgfsetroundjoin%
\definecolor{currentfill}{rgb}{0.274128,0.199721,0.498911}%
\pgfsetfillcolor{currentfill}%
\pgfsetlinewidth{0.000000pt}%
\definecolor{currentstroke}{rgb}{0.000000,0.000000,0.000000}%
\pgfsetstrokecolor{currentstroke}%
\pgfsetdash{}{0pt}%
\pgfpathmoveto{\pgfqpoint{3.332210in}{1.992678in}}%
\pgfpathlineto{\pgfqpoint{3.151139in}{2.083214in}}%
\pgfpathlineto{\pgfqpoint{3.151139in}{2.073297in}}%
\pgfpathlineto{\pgfqpoint{3.117419in}{2.105032in}}%
\pgfpathlineto{\pgfqpoint{3.163039in}{2.097098in}}%
\pgfpathlineto{\pgfqpoint{3.155106in}{2.091148in}}%
\pgfpathlineto{\pgfqpoint{3.336177in}{2.000612in}}%
\pgfpathlineto{\pgfqpoint{3.332210in}{1.992678in}}%
\pgfusepath{fill}%
\end{pgfscope}%
\begin{pgfscope}%
\pgfpathrectangle{\pgfqpoint{1.432000in}{0.528000in}}{\pgfqpoint{3.696000in}{3.696000in}} %
\pgfusepath{clip}%
\pgfsetbuttcap%
\pgfsetroundjoin%
\definecolor{currentfill}{rgb}{0.177423,0.437527,0.557565}%
\pgfsetfillcolor{currentfill}%
\pgfsetlinewidth{0.000000pt}%
\definecolor{currentstroke}{rgb}{0.000000,0.000000,0.000000}%
\pgfsetstrokecolor{currentstroke}%
\pgfsetdash{}{0pt}%
\pgfpathmoveto{\pgfqpoint{3.331057in}{1.993509in}}%
\pgfpathlineto{\pgfqpoint{3.142509in}{2.182058in}}%
\pgfpathlineto{\pgfqpoint{3.139372in}{2.172649in}}%
\pgfpathlineto{\pgfqpoint{3.117419in}{2.213419in}}%
\pgfpathlineto{\pgfqpoint{3.158189in}{2.191466in}}%
\pgfpathlineto{\pgfqpoint{3.148781in}{2.188330in}}%
\pgfpathlineto{\pgfqpoint{3.337330in}{1.999781in}}%
\pgfpathlineto{\pgfqpoint{3.331057in}{1.993509in}}%
\pgfusepath{fill}%
\end{pgfscope}%
\begin{pgfscope}%
\pgfpathrectangle{\pgfqpoint{1.432000in}{0.528000in}}{\pgfqpoint{3.696000in}{3.696000in}} %
\pgfusepath{clip}%
\pgfsetbuttcap%
\pgfsetroundjoin%
\definecolor{currentfill}{rgb}{0.283187,0.125848,0.444960}%
\pgfsetfillcolor{currentfill}%
\pgfsetlinewidth{0.000000pt}%
\definecolor{currentstroke}{rgb}{0.000000,0.000000,0.000000}%
\pgfsetstrokecolor{currentstroke}%
\pgfsetdash{}{0pt}%
\pgfpathmoveto{\pgfqpoint{3.440597in}{1.992678in}}%
\pgfpathlineto{\pgfqpoint{3.259526in}{2.083214in}}%
\pgfpathlineto{\pgfqpoint{3.259526in}{2.073297in}}%
\pgfpathlineto{\pgfqpoint{3.225806in}{2.105032in}}%
\pgfpathlineto{\pgfqpoint{3.271427in}{2.097098in}}%
\pgfpathlineto{\pgfqpoint{3.263493in}{2.091148in}}%
\pgfpathlineto{\pgfqpoint{3.444564in}{2.000612in}}%
\pgfpathlineto{\pgfqpoint{3.440597in}{1.992678in}}%
\pgfusepath{fill}%
\end{pgfscope}%
\begin{pgfscope}%
\pgfpathrectangle{\pgfqpoint{1.432000in}{0.528000in}}{\pgfqpoint{3.696000in}{3.696000in}} %
\pgfusepath{clip}%
\pgfsetbuttcap%
\pgfsetroundjoin%
\definecolor{currentfill}{rgb}{0.183898,0.422383,0.556944}%
\pgfsetfillcolor{currentfill}%
\pgfsetlinewidth{0.000000pt}%
\definecolor{currentstroke}{rgb}{0.000000,0.000000,0.000000}%
\pgfsetstrokecolor{currentstroke}%
\pgfsetdash{}{0pt}%
\pgfpathmoveto{\pgfqpoint{3.439444in}{1.993509in}}%
\pgfpathlineto{\pgfqpoint{3.250896in}{2.182058in}}%
\pgfpathlineto{\pgfqpoint{3.247760in}{2.172649in}}%
\pgfpathlineto{\pgfqpoint{3.225806in}{2.213419in}}%
\pgfpathlineto{\pgfqpoint{3.266577in}{2.191466in}}%
\pgfpathlineto{\pgfqpoint{3.257168in}{2.188330in}}%
\pgfpathlineto{\pgfqpoint{3.445717in}{1.999781in}}%
\pgfpathlineto{\pgfqpoint{3.439444in}{1.993509in}}%
\pgfusepath{fill}%
\end{pgfscope}%
\begin{pgfscope}%
\pgfpathrectangle{\pgfqpoint{1.432000in}{0.528000in}}{\pgfqpoint{3.696000in}{3.696000in}} %
\pgfusepath{clip}%
\pgfsetbuttcap%
\pgfsetroundjoin%
\definecolor{currentfill}{rgb}{0.276022,0.044167,0.370164}%
\pgfsetfillcolor{currentfill}%
\pgfsetlinewidth{0.000000pt}%
\definecolor{currentstroke}{rgb}{0.000000,0.000000,0.000000}%
\pgfsetstrokecolor{currentstroke}%
\pgfsetdash{}{0pt}%
\pgfpathmoveto{\pgfqpoint{3.548984in}{1.992678in}}%
\pgfpathlineto{\pgfqpoint{3.367913in}{2.083214in}}%
\pgfpathlineto{\pgfqpoint{3.367913in}{2.073297in}}%
\pgfpathlineto{\pgfqpoint{3.334194in}{2.105032in}}%
\pgfpathlineto{\pgfqpoint{3.379814in}{2.097098in}}%
\pgfpathlineto{\pgfqpoint{3.371880in}{2.091148in}}%
\pgfpathlineto{\pgfqpoint{3.552951in}{2.000612in}}%
\pgfpathlineto{\pgfqpoint{3.548984in}{1.992678in}}%
\pgfusepath{fill}%
\end{pgfscope}%
\begin{pgfscope}%
\pgfpathrectangle{\pgfqpoint{1.432000in}{0.528000in}}{\pgfqpoint{3.696000in}{3.696000in}} %
\pgfusepath{clip}%
\pgfsetbuttcap%
\pgfsetroundjoin%
\definecolor{currentfill}{rgb}{0.252194,0.269783,0.531579}%
\pgfsetfillcolor{currentfill}%
\pgfsetlinewidth{0.000000pt}%
\definecolor{currentstroke}{rgb}{0.000000,0.000000,0.000000}%
\pgfsetstrokecolor{currentstroke}%
\pgfsetdash{}{0pt}%
\pgfpathmoveto{\pgfqpoint{3.547832in}{1.993509in}}%
\pgfpathlineto{\pgfqpoint{3.359283in}{2.182058in}}%
\pgfpathlineto{\pgfqpoint{3.356147in}{2.172649in}}%
\pgfpathlineto{\pgfqpoint{3.334194in}{2.213419in}}%
\pgfpathlineto{\pgfqpoint{3.374964in}{2.191466in}}%
\pgfpathlineto{\pgfqpoint{3.365555in}{2.188330in}}%
\pgfpathlineto{\pgfqpoint{3.554104in}{1.999781in}}%
\pgfpathlineto{\pgfqpoint{3.547832in}{1.993509in}}%
\pgfusepath{fill}%
\end{pgfscope}%
\begin{pgfscope}%
\pgfpathrectangle{\pgfqpoint{1.432000in}{0.528000in}}{\pgfqpoint{3.696000in}{3.696000in}} %
\pgfusepath{clip}%
\pgfsetbuttcap%
\pgfsetroundjoin%
\definecolor{currentfill}{rgb}{0.271828,0.209303,0.504434}%
\pgfsetfillcolor{currentfill}%
\pgfsetlinewidth{0.000000pt}%
\definecolor{currentstroke}{rgb}{0.000000,0.000000,0.000000}%
\pgfsetstrokecolor{currentstroke}%
\pgfsetdash{}{0pt}%
\pgfpathmoveto{\pgfqpoint{3.657952in}{1.992438in}}%
\pgfpathlineto{\pgfqpoint{3.370659in}{2.088202in}}%
\pgfpathlineto{\pgfqpoint{3.372062in}{2.078384in}}%
\pgfpathlineto{\pgfqpoint{3.334194in}{2.105032in}}%
\pgfpathlineto{\pgfqpoint{3.380477in}{2.103630in}}%
\pgfpathlineto{\pgfqpoint{3.373464in}{2.096617in}}%
\pgfpathlineto{\pgfqpoint{3.660757in}{2.000853in}}%
\pgfpathlineto{\pgfqpoint{3.657952in}{1.992438in}}%
\pgfusepath{fill}%
\end{pgfscope}%
\begin{pgfscope}%
\pgfpathrectangle{\pgfqpoint{1.432000in}{0.528000in}}{\pgfqpoint{3.696000in}{3.696000in}} %
\pgfusepath{clip}%
\pgfsetbuttcap%
\pgfsetroundjoin%
\definecolor{currentfill}{rgb}{0.282910,0.105393,0.426902}%
\pgfsetfillcolor{currentfill}%
\pgfsetlinewidth{0.000000pt}%
\definecolor{currentstroke}{rgb}{0.000000,0.000000,0.000000}%
\pgfsetstrokecolor{currentstroke}%
\pgfsetdash{}{0pt}%
\pgfpathmoveto{\pgfqpoint{3.656895in}{1.992955in}}%
\pgfpathlineto{\pgfqpoint{3.364946in}{2.187587in}}%
\pgfpathlineto{\pgfqpoint{3.363716in}{2.177746in}}%
\pgfpathlineto{\pgfqpoint{3.334194in}{2.213419in}}%
\pgfpathlineto{\pgfqpoint{3.378477in}{2.199888in}}%
\pgfpathlineto{\pgfqpoint{3.369867in}{2.194968in}}%
\pgfpathlineto{\pgfqpoint{3.661815in}{2.000335in}}%
\pgfpathlineto{\pgfqpoint{3.656895in}{1.992955in}}%
\pgfusepath{fill}%
\end{pgfscope}%
\begin{pgfscope}%
\pgfpathrectangle{\pgfqpoint{1.432000in}{0.528000in}}{\pgfqpoint{3.696000in}{3.696000in}} %
\pgfusepath{clip}%
\pgfsetbuttcap%
\pgfsetroundjoin%
\definecolor{currentfill}{rgb}{0.201239,0.383670,0.554294}%
\pgfsetfillcolor{currentfill}%
\pgfsetlinewidth{0.000000pt}%
\definecolor{currentstroke}{rgb}{0.000000,0.000000,0.000000}%
\pgfsetstrokecolor{currentstroke}%
\pgfsetdash{}{0pt}%
\pgfpathmoveto{\pgfqpoint{3.765282in}{1.992955in}}%
\pgfpathlineto{\pgfqpoint{3.473333in}{2.187587in}}%
\pgfpathlineto{\pgfqpoint{3.472103in}{2.177746in}}%
\pgfpathlineto{\pgfqpoint{3.442581in}{2.213419in}}%
\pgfpathlineto{\pgfqpoint{3.486864in}{2.199888in}}%
\pgfpathlineto{\pgfqpoint{3.478254in}{2.194968in}}%
\pgfpathlineto{\pgfqpoint{3.770202in}{2.000335in}}%
\pgfpathlineto{\pgfqpoint{3.765282in}{1.992955in}}%
\pgfusepath{fill}%
\end{pgfscope}%
\begin{pgfscope}%
\pgfpathrectangle{\pgfqpoint{1.432000in}{0.528000in}}{\pgfqpoint{3.696000in}{3.696000in}} %
\pgfusepath{clip}%
\pgfsetbuttcap%
\pgfsetroundjoin%
\definecolor{currentfill}{rgb}{0.269944,0.014625,0.341379}%
\pgfsetfillcolor{currentfill}%
\pgfsetlinewidth{0.000000pt}%
\definecolor{currentstroke}{rgb}{0.000000,0.000000,0.000000}%
\pgfsetstrokecolor{currentstroke}%
\pgfsetdash{}{0pt}%
\pgfpathmoveto{\pgfqpoint{3.875053in}{1.992342in}}%
\pgfpathlineto{\pgfqpoint{3.480230in}{2.091048in}}%
\pgfpathlineto{\pgfqpoint{3.482381in}{2.081367in}}%
\pgfpathlineto{\pgfqpoint{3.442581in}{2.105032in}}%
\pgfpathlineto{\pgfqpoint{3.488835in}{2.107184in}}%
\pgfpathlineto{\pgfqpoint{3.482381in}{2.099654in}}%
\pgfpathlineto{\pgfqpoint{3.877205in}{2.000948in}}%
\pgfpathlineto{\pgfqpoint{3.875053in}{1.992342in}}%
\pgfusepath{fill}%
\end{pgfscope}%
\begin{pgfscope}%
\pgfpathrectangle{\pgfqpoint{1.432000in}{0.528000in}}{\pgfqpoint{3.696000in}{3.696000in}} %
\pgfusepath{clip}%
\pgfsetbuttcap%
\pgfsetroundjoin%
\definecolor{currentfill}{rgb}{0.277018,0.050344,0.375715}%
\pgfsetfillcolor{currentfill}%
\pgfsetlinewidth{0.000000pt}%
\definecolor{currentstroke}{rgb}{0.000000,0.000000,0.000000}%
\pgfsetstrokecolor{currentstroke}%
\pgfsetdash{}{0pt}%
\pgfpathmoveto{\pgfqpoint{3.873669in}{1.992955in}}%
\pgfpathlineto{\pgfqpoint{3.581720in}{2.187587in}}%
\pgfpathlineto{\pgfqpoint{3.580490in}{2.177746in}}%
\pgfpathlineto{\pgfqpoint{3.550968in}{2.213419in}}%
\pgfpathlineto{\pgfqpoint{3.595251in}{2.199888in}}%
\pgfpathlineto{\pgfqpoint{3.586641in}{2.194968in}}%
\pgfpathlineto{\pgfqpoint{3.878589in}{2.000335in}}%
\pgfpathlineto{\pgfqpoint{3.873669in}{1.992955in}}%
\pgfusepath{fill}%
\end{pgfscope}%
\begin{pgfscope}%
\pgfpathrectangle{\pgfqpoint{1.432000in}{0.528000in}}{\pgfqpoint{3.696000in}{3.696000in}} %
\pgfusepath{clip}%
\pgfsetbuttcap%
\pgfsetroundjoin%
\definecolor{currentfill}{rgb}{0.267004,0.004874,0.329415}%
\pgfsetfillcolor{currentfill}%
\pgfsetlinewidth{0.000000pt}%
\definecolor{currentstroke}{rgb}{0.000000,0.000000,0.000000}%
\pgfsetstrokecolor{currentstroke}%
\pgfsetdash{}{0pt}%
\pgfpathmoveto{\pgfqpoint{3.983440in}{1.992342in}}%
\pgfpathlineto{\pgfqpoint{3.588617in}{2.091048in}}%
\pgfpathlineto{\pgfqpoint{3.590768in}{2.081367in}}%
\pgfpathlineto{\pgfqpoint{3.550968in}{2.105032in}}%
\pgfpathlineto{\pgfqpoint{3.597223in}{2.107184in}}%
\pgfpathlineto{\pgfqpoint{3.590768in}{2.099654in}}%
\pgfpathlineto{\pgfqpoint{3.985592in}{2.000948in}}%
\pgfpathlineto{\pgfqpoint{3.983440in}{1.992342in}}%
\pgfusepath{fill}%
\end{pgfscope}%
\begin{pgfscope}%
\pgfpathrectangle{\pgfqpoint{1.432000in}{0.528000in}}{\pgfqpoint{3.696000in}{3.696000in}} %
\pgfusepath{clip}%
\pgfsetbuttcap%
\pgfsetroundjoin%
\definecolor{currentfill}{rgb}{0.188923,0.410910,0.556326}%
\pgfsetfillcolor{currentfill}%
\pgfsetlinewidth{0.000000pt}%
\definecolor{currentstroke}{rgb}{0.000000,0.000000,0.000000}%
\pgfsetstrokecolor{currentstroke}%
\pgfsetdash{}{0pt}%
\pgfpathmoveto{\pgfqpoint{3.982533in}{1.992678in}}%
\pgfpathlineto{\pgfqpoint{3.584687in}{2.191601in}}%
\pgfpathlineto{\pgfqpoint{3.584687in}{2.181684in}}%
\pgfpathlineto{\pgfqpoint{3.550968in}{2.213419in}}%
\pgfpathlineto{\pgfqpoint{3.596588in}{2.205485in}}%
\pgfpathlineto{\pgfqpoint{3.588654in}{2.199535in}}%
\pgfpathlineto{\pgfqpoint{3.986500in}{2.000612in}}%
\pgfpathlineto{\pgfqpoint{3.982533in}{1.992678in}}%
\pgfusepath{fill}%
\end{pgfscope}%
\begin{pgfscope}%
\pgfpathrectangle{\pgfqpoint{1.432000in}{0.528000in}}{\pgfqpoint{3.696000in}{3.696000in}} %
\pgfusepath{clip}%
\pgfsetbuttcap%
\pgfsetroundjoin%
\definecolor{currentfill}{rgb}{0.277941,0.056324,0.381191}%
\pgfsetfillcolor{currentfill}%
\pgfsetlinewidth{0.000000pt}%
\definecolor{currentstroke}{rgb}{0.000000,0.000000,0.000000}%
\pgfsetstrokecolor{currentstroke}%
\pgfsetdash{}{0pt}%
\pgfpathmoveto{\pgfqpoint{4.091828in}{1.992342in}}%
\pgfpathlineto{\pgfqpoint{3.697004in}{2.091048in}}%
\pgfpathlineto{\pgfqpoint{3.699156in}{2.081367in}}%
\pgfpathlineto{\pgfqpoint{3.659355in}{2.105032in}}%
\pgfpathlineto{\pgfqpoint{3.705610in}{2.107184in}}%
\pgfpathlineto{\pgfqpoint{3.699156in}{2.099654in}}%
\pgfpathlineto{\pgfqpoint{4.093979in}{2.000948in}}%
\pgfpathlineto{\pgfqpoint{4.091828in}{1.992342in}}%
\pgfusepath{fill}%
\end{pgfscope}%
\begin{pgfscope}%
\pgfpathrectangle{\pgfqpoint{1.432000in}{0.528000in}}{\pgfqpoint{3.696000in}{3.696000in}} %
\pgfusepath{clip}%
\pgfsetbuttcap%
\pgfsetroundjoin%
\definecolor{currentfill}{rgb}{0.123463,0.581687,0.547445}%
\pgfsetfillcolor{currentfill}%
\pgfsetlinewidth{0.000000pt}%
\definecolor{currentstroke}{rgb}{0.000000,0.000000,0.000000}%
\pgfsetstrokecolor{currentstroke}%
\pgfsetdash{}{0pt}%
\pgfpathmoveto{\pgfqpoint{4.090920in}{1.992678in}}%
\pgfpathlineto{\pgfqpoint{3.693074in}{2.191601in}}%
\pgfpathlineto{\pgfqpoint{3.693074in}{2.181684in}}%
\pgfpathlineto{\pgfqpoint{3.659355in}{2.213419in}}%
\pgfpathlineto{\pgfqpoint{3.704975in}{2.205485in}}%
\pgfpathlineto{\pgfqpoint{3.697041in}{2.199535in}}%
\pgfpathlineto{\pgfqpoint{4.094887in}{2.000612in}}%
\pgfpathlineto{\pgfqpoint{4.090920in}{1.992678in}}%
\pgfusepath{fill}%
\end{pgfscope}%
\begin{pgfscope}%
\pgfpathrectangle{\pgfqpoint{1.432000in}{0.528000in}}{\pgfqpoint{3.696000in}{3.696000in}} %
\pgfusepath{clip}%
\pgfsetbuttcap%
\pgfsetroundjoin%
\definecolor{currentfill}{rgb}{0.135066,0.544853,0.554029}%
\pgfsetfillcolor{currentfill}%
\pgfsetlinewidth{0.000000pt}%
\definecolor{currentstroke}{rgb}{0.000000,0.000000,0.000000}%
\pgfsetstrokecolor{currentstroke}%
\pgfsetdash{}{0pt}%
\pgfpathmoveto{\pgfqpoint{4.199307in}{1.992678in}}%
\pgfpathlineto{\pgfqpoint{3.801461in}{2.191601in}}%
\pgfpathlineto{\pgfqpoint{3.801461in}{2.181684in}}%
\pgfpathlineto{\pgfqpoint{3.767742in}{2.213419in}}%
\pgfpathlineto{\pgfqpoint{3.813362in}{2.205485in}}%
\pgfpathlineto{\pgfqpoint{3.805428in}{2.199535in}}%
\pgfpathlineto{\pgfqpoint{4.203274in}{2.000612in}}%
\pgfpathlineto{\pgfqpoint{4.199307in}{1.992678in}}%
\pgfusepath{fill}%
\end{pgfscope}%
\begin{pgfscope}%
\pgfpathrectangle{\pgfqpoint{1.432000in}{0.528000in}}{\pgfqpoint{3.696000in}{3.696000in}} %
\pgfusepath{clip}%
\pgfsetbuttcap%
\pgfsetroundjoin%
\definecolor{currentfill}{rgb}{0.269944,0.014625,0.341379}%
\pgfsetfillcolor{currentfill}%
\pgfsetlinewidth{0.000000pt}%
\definecolor{currentstroke}{rgb}{0.000000,0.000000,0.000000}%
\pgfsetstrokecolor{currentstroke}%
\pgfsetdash{}{0pt}%
\pgfpathmoveto{\pgfqpoint{4.198629in}{1.993097in}}%
\pgfpathlineto{\pgfqpoint{3.797014in}{2.294308in}}%
\pgfpathlineto{\pgfqpoint{3.795240in}{2.284551in}}%
\pgfpathlineto{\pgfqpoint{3.767742in}{2.321806in}}%
\pgfpathlineto{\pgfqpoint{3.811207in}{2.305840in}}%
\pgfpathlineto{\pgfqpoint{3.802336in}{2.301405in}}%
\pgfpathlineto{\pgfqpoint{4.203951in}{2.000193in}}%
\pgfpathlineto{\pgfqpoint{4.198629in}{1.993097in}}%
\pgfusepath{fill}%
\end{pgfscope}%
\begin{pgfscope}%
\pgfpathrectangle{\pgfqpoint{1.432000in}{0.528000in}}{\pgfqpoint{3.696000in}{3.696000in}} %
\pgfusepath{clip}%
\pgfsetbuttcap%
\pgfsetroundjoin%
\definecolor{currentfill}{rgb}{0.283197,0.115680,0.436115}%
\pgfsetfillcolor{currentfill}%
\pgfsetlinewidth{0.000000pt}%
\definecolor{currentstroke}{rgb}{0.000000,0.000000,0.000000}%
\pgfsetstrokecolor{currentstroke}%
\pgfsetdash{}{0pt}%
\pgfpathmoveto{\pgfqpoint{4.307694in}{1.992678in}}%
\pgfpathlineto{\pgfqpoint{3.909848in}{2.191601in}}%
\pgfpathlineto{\pgfqpoint{3.909848in}{2.181684in}}%
\pgfpathlineto{\pgfqpoint{3.876129in}{2.213419in}}%
\pgfpathlineto{\pgfqpoint{3.921749in}{2.205485in}}%
\pgfpathlineto{\pgfqpoint{3.913815in}{2.199535in}}%
\pgfpathlineto{\pgfqpoint{4.311661in}{2.000612in}}%
\pgfpathlineto{\pgfqpoint{4.307694in}{1.992678in}}%
\pgfusepath{fill}%
\end{pgfscope}%
\begin{pgfscope}%
\pgfpathrectangle{\pgfqpoint{1.432000in}{0.528000in}}{\pgfqpoint{3.696000in}{3.696000in}} %
\pgfusepath{clip}%
\pgfsetbuttcap%
\pgfsetroundjoin%
\definecolor{currentfill}{rgb}{0.267004,0.004874,0.329415}%
\pgfsetfillcolor{currentfill}%
\pgfsetlinewidth{0.000000pt}%
\definecolor{currentstroke}{rgb}{0.000000,0.000000,0.000000}%
\pgfsetstrokecolor{currentstroke}%
\pgfsetdash{}{0pt}%
\pgfpathmoveto{\pgfqpoint{4.307217in}{1.992955in}}%
\pgfpathlineto{\pgfqpoint{4.015269in}{2.187587in}}%
\pgfpathlineto{\pgfqpoint{4.014039in}{2.177746in}}%
\pgfpathlineto{\pgfqpoint{3.984516in}{2.213419in}}%
\pgfpathlineto{\pgfqpoint{4.028800in}{2.199888in}}%
\pgfpathlineto{\pgfqpoint{4.020189in}{2.194968in}}%
\pgfpathlineto{\pgfqpoint{4.312138in}{2.000335in}}%
\pgfpathlineto{\pgfqpoint{4.307217in}{1.992955in}}%
\pgfusepath{fill}%
\end{pgfscope}%
\begin{pgfscope}%
\pgfpathrectangle{\pgfqpoint{1.432000in}{0.528000in}}{\pgfqpoint{3.696000in}{3.696000in}} %
\pgfusepath{clip}%
\pgfsetbuttcap%
\pgfsetroundjoin%
\definecolor{currentfill}{rgb}{0.274952,0.037752,0.364543}%
\pgfsetfillcolor{currentfill}%
\pgfsetlinewidth{0.000000pt}%
\definecolor{currentstroke}{rgb}{0.000000,0.000000,0.000000}%
\pgfsetstrokecolor{currentstroke}%
\pgfsetdash{}{0pt}%
\pgfpathmoveto{\pgfqpoint{4.415604in}{1.992955in}}%
\pgfpathlineto{\pgfqpoint{4.123656in}{2.187587in}}%
\pgfpathlineto{\pgfqpoint{4.122426in}{2.177746in}}%
\pgfpathlineto{\pgfqpoint{4.092903in}{2.213419in}}%
\pgfpathlineto{\pgfqpoint{4.137187in}{2.199888in}}%
\pgfpathlineto{\pgfqpoint{4.128576in}{2.194968in}}%
\pgfpathlineto{\pgfqpoint{4.420525in}{2.000335in}}%
\pgfpathlineto{\pgfqpoint{4.415604in}{1.992955in}}%
\pgfusepath{fill}%
\end{pgfscope}%
\begin{pgfscope}%
\pgfpathrectangle{\pgfqpoint{1.432000in}{0.528000in}}{\pgfqpoint{3.696000in}{3.696000in}} %
\pgfusepath{clip}%
\pgfsetbuttcap%
\pgfsetroundjoin%
\definecolor{currentfill}{rgb}{0.119512,0.607464,0.540218}%
\pgfsetfillcolor{currentfill}%
\pgfsetlinewidth{0.000000pt}%
\definecolor{currentstroke}{rgb}{0.000000,0.000000,0.000000}%
\pgfsetstrokecolor{currentstroke}%
\pgfsetdash{}{0pt}%
\pgfpathmoveto{\pgfqpoint{4.414928in}{1.993509in}}%
\pgfpathlineto{\pgfqpoint{4.226380in}{2.182058in}}%
\pgfpathlineto{\pgfqpoint{4.223243in}{2.172649in}}%
\pgfpathlineto{\pgfqpoint{4.201290in}{2.213419in}}%
\pgfpathlineto{\pgfqpoint{4.242060in}{2.191466in}}%
\pgfpathlineto{\pgfqpoint{4.232652in}{2.188330in}}%
\pgfpathlineto{\pgfqpoint{4.421201in}{1.999781in}}%
\pgfpathlineto{\pgfqpoint{4.414928in}{1.993509in}}%
\pgfusepath{fill}%
\end{pgfscope}%
\begin{pgfscope}%
\pgfpathrectangle{\pgfqpoint{1.432000in}{0.528000in}}{\pgfqpoint{3.696000in}{3.696000in}} %
\pgfusepath{clip}%
\pgfsetbuttcap%
\pgfsetroundjoin%
\definecolor{currentfill}{rgb}{0.271828,0.209303,0.504434}%
\pgfsetfillcolor{currentfill}%
\pgfsetlinewidth{0.000000pt}%
\definecolor{currentstroke}{rgb}{0.000000,0.000000,0.000000}%
\pgfsetstrokecolor{currentstroke}%
\pgfsetdash{}{0pt}%
\pgfpathmoveto{\pgfqpoint{4.524468in}{1.992678in}}%
\pgfpathlineto{\pgfqpoint{4.343397in}{2.083214in}}%
\pgfpathlineto{\pgfqpoint{4.343397in}{2.073297in}}%
\pgfpathlineto{\pgfqpoint{4.309677in}{2.105032in}}%
\pgfpathlineto{\pgfqpoint{4.355297in}{2.097098in}}%
\pgfpathlineto{\pgfqpoint{4.347364in}{2.091148in}}%
\pgfpathlineto{\pgfqpoint{4.528435in}{2.000612in}}%
\pgfpathlineto{\pgfqpoint{4.524468in}{1.992678in}}%
\pgfusepath{fill}%
\end{pgfscope}%
\begin{pgfscope}%
\pgfpathrectangle{\pgfqpoint{1.432000in}{0.528000in}}{\pgfqpoint{3.696000in}{3.696000in}} %
\pgfusepath{clip}%
\pgfsetbuttcap%
\pgfsetroundjoin%
\definecolor{currentfill}{rgb}{0.246070,0.738910,0.452024}%
\pgfsetfillcolor{currentfill}%
\pgfsetlinewidth{0.000000pt}%
\definecolor{currentstroke}{rgb}{0.000000,0.000000,0.000000}%
\pgfsetstrokecolor{currentstroke}%
\pgfsetdash{}{0pt}%
\pgfpathmoveto{\pgfqpoint{4.523315in}{1.993509in}}%
\pgfpathlineto{\pgfqpoint{4.334767in}{2.182058in}}%
\pgfpathlineto{\pgfqpoint{4.331631in}{2.172649in}}%
\pgfpathlineto{\pgfqpoint{4.309677in}{2.213419in}}%
\pgfpathlineto{\pgfqpoint{4.350447in}{2.191466in}}%
\pgfpathlineto{\pgfqpoint{4.341039in}{2.188330in}}%
\pgfpathlineto{\pgfqpoint{4.529588in}{1.999781in}}%
\pgfpathlineto{\pgfqpoint{4.523315in}{1.993509in}}%
\pgfusepath{fill}%
\end{pgfscope}%
\begin{pgfscope}%
\pgfpathrectangle{\pgfqpoint{1.432000in}{0.528000in}}{\pgfqpoint{3.696000in}{3.696000in}} %
\pgfusepath{clip}%
\pgfsetbuttcap%
\pgfsetroundjoin%
\definecolor{currentfill}{rgb}{0.274128,0.199721,0.498911}%
\pgfsetfillcolor{currentfill}%
\pgfsetlinewidth{0.000000pt}%
\definecolor{currentstroke}{rgb}{0.000000,0.000000,0.000000}%
\pgfsetstrokecolor{currentstroke}%
\pgfsetdash{}{0pt}%
\pgfpathmoveto{\pgfqpoint{4.632855in}{1.992678in}}%
\pgfpathlineto{\pgfqpoint{4.451784in}{2.083214in}}%
\pgfpathlineto{\pgfqpoint{4.451784in}{2.073297in}}%
\pgfpathlineto{\pgfqpoint{4.418065in}{2.105032in}}%
\pgfpathlineto{\pgfqpoint{4.463685in}{2.097098in}}%
\pgfpathlineto{\pgfqpoint{4.455751in}{2.091148in}}%
\pgfpathlineto{\pgfqpoint{4.636822in}{2.000612in}}%
\pgfpathlineto{\pgfqpoint{4.632855in}{1.992678in}}%
\pgfusepath{fill}%
\end{pgfscope}%
\begin{pgfscope}%
\pgfpathrectangle{\pgfqpoint{1.432000in}{0.528000in}}{\pgfqpoint{3.696000in}{3.696000in}} %
\pgfusepath{clip}%
\pgfsetbuttcap%
\pgfsetroundjoin%
\definecolor{currentfill}{rgb}{0.157851,0.683765,0.501686}%
\pgfsetfillcolor{currentfill}%
\pgfsetlinewidth{0.000000pt}%
\definecolor{currentstroke}{rgb}{0.000000,0.000000,0.000000}%
\pgfsetstrokecolor{currentstroke}%
\pgfsetdash{}{0pt}%
\pgfpathmoveto{\pgfqpoint{4.631703in}{1.993509in}}%
\pgfpathlineto{\pgfqpoint{4.443154in}{2.182058in}}%
\pgfpathlineto{\pgfqpoint{4.440018in}{2.172649in}}%
\pgfpathlineto{\pgfqpoint{4.418065in}{2.213419in}}%
\pgfpathlineto{\pgfqpoint{4.458835in}{2.191466in}}%
\pgfpathlineto{\pgfqpoint{4.449426in}{2.188330in}}%
\pgfpathlineto{\pgfqpoint{4.637975in}{1.999781in}}%
\pgfpathlineto{\pgfqpoint{4.631703in}{1.993509in}}%
\pgfusepath{fill}%
\end{pgfscope}%
\begin{pgfscope}%
\pgfpathrectangle{\pgfqpoint{1.432000in}{0.528000in}}{\pgfqpoint{3.696000in}{3.696000in}} %
\pgfusepath{clip}%
\pgfsetbuttcap%
\pgfsetroundjoin%
\definecolor{currentfill}{rgb}{0.277018,0.050344,0.375715}%
\pgfsetfillcolor{currentfill}%
\pgfsetlinewidth{0.000000pt}%
\definecolor{currentstroke}{rgb}{0.000000,0.000000,0.000000}%
\pgfsetstrokecolor{currentstroke}%
\pgfsetdash{}{0pt}%
\pgfpathmoveto{\pgfqpoint{4.741242in}{1.992678in}}%
\pgfpathlineto{\pgfqpoint{4.560171in}{2.083214in}}%
\pgfpathlineto{\pgfqpoint{4.560171in}{2.073297in}}%
\pgfpathlineto{\pgfqpoint{4.526452in}{2.105032in}}%
\pgfpathlineto{\pgfqpoint{4.572072in}{2.097098in}}%
\pgfpathlineto{\pgfqpoint{4.564138in}{2.091148in}}%
\pgfpathlineto{\pgfqpoint{4.745209in}{2.000612in}}%
\pgfpathlineto{\pgfqpoint{4.741242in}{1.992678in}}%
\pgfusepath{fill}%
\end{pgfscope}%
\begin{pgfscope}%
\pgfpathrectangle{\pgfqpoint{1.432000in}{0.528000in}}{\pgfqpoint{3.696000in}{3.696000in}} %
\pgfusepath{clip}%
\pgfsetbuttcap%
\pgfsetroundjoin%
\definecolor{currentfill}{rgb}{0.274952,0.037752,0.364543}%
\pgfsetfillcolor{currentfill}%
\pgfsetlinewidth{0.000000pt}%
\definecolor{currentstroke}{rgb}{0.000000,0.000000,0.000000}%
\pgfsetstrokecolor{currentstroke}%
\pgfsetdash{}{0pt}%
\pgfpathmoveto{\pgfqpoint{4.740090in}{1.993509in}}%
\pgfpathlineto{\pgfqpoint{4.659928in}{2.073671in}}%
\pgfpathlineto{\pgfqpoint{4.656792in}{2.064262in}}%
\pgfpathlineto{\pgfqpoint{4.634839in}{2.105032in}}%
\pgfpathlineto{\pgfqpoint{4.675609in}{2.083079in}}%
\pgfpathlineto{\pgfqpoint{4.666200in}{2.079943in}}%
\pgfpathlineto{\pgfqpoint{4.746362in}{1.999781in}}%
\pgfpathlineto{\pgfqpoint{4.740090in}{1.993509in}}%
\pgfusepath{fill}%
\end{pgfscope}%
\begin{pgfscope}%
\pgfpathrectangle{\pgfqpoint{1.432000in}{0.528000in}}{\pgfqpoint{3.696000in}{3.696000in}} %
\pgfusepath{clip}%
\pgfsetbuttcap%
\pgfsetroundjoin%
\definecolor{currentfill}{rgb}{0.225863,0.330805,0.547314}%
\pgfsetfillcolor{currentfill}%
\pgfsetlinewidth{0.000000pt}%
\definecolor{currentstroke}{rgb}{0.000000,0.000000,0.000000}%
\pgfsetstrokecolor{currentstroke}%
\pgfsetdash{}{0pt}%
\pgfpathmoveto{\pgfqpoint{4.740090in}{1.993509in}}%
\pgfpathlineto{\pgfqpoint{4.551541in}{2.182058in}}%
\pgfpathlineto{\pgfqpoint{4.548405in}{2.172649in}}%
\pgfpathlineto{\pgfqpoint{4.526452in}{2.213419in}}%
\pgfpathlineto{\pgfqpoint{4.567222in}{2.191466in}}%
\pgfpathlineto{\pgfqpoint{4.557813in}{2.188330in}}%
\pgfpathlineto{\pgfqpoint{4.746362in}{1.999781in}}%
\pgfpathlineto{\pgfqpoint{4.740090in}{1.993509in}}%
\pgfusepath{fill}%
\end{pgfscope}%
\begin{pgfscope}%
\pgfpathrectangle{\pgfqpoint{1.432000in}{0.528000in}}{\pgfqpoint{3.696000in}{3.696000in}} %
\pgfusepath{clip}%
\pgfsetbuttcap%
\pgfsetroundjoin%
\definecolor{currentfill}{rgb}{0.248629,0.278775,0.534556}%
\pgfsetfillcolor{currentfill}%
\pgfsetlinewidth{0.000000pt}%
\definecolor{currentstroke}{rgb}{0.000000,0.000000,0.000000}%
\pgfsetstrokecolor{currentstroke}%
\pgfsetdash{}{0pt}%
\pgfpathmoveto{\pgfqpoint{4.739259in}{1.994662in}}%
\pgfpathlineto{\pgfqpoint{4.648723in}{2.175733in}}%
\pgfpathlineto{\pgfqpoint{4.642773in}{2.167799in}}%
\pgfpathlineto{\pgfqpoint{4.634839in}{2.213419in}}%
\pgfpathlineto{\pgfqpoint{4.666574in}{2.179700in}}%
\pgfpathlineto{\pgfqpoint{4.656657in}{2.179700in}}%
\pgfpathlineto{\pgfqpoint{4.747193in}{1.998629in}}%
\pgfpathlineto{\pgfqpoint{4.739259in}{1.994662in}}%
\pgfusepath{fill}%
\end{pgfscope}%
\begin{pgfscope}%
\pgfpathrectangle{\pgfqpoint{1.432000in}{0.528000in}}{\pgfqpoint{3.696000in}{3.696000in}} %
\pgfusepath{clip}%
\pgfsetbuttcap%
\pgfsetroundjoin%
\definecolor{currentfill}{rgb}{0.281412,0.155834,0.469201}%
\pgfsetfillcolor{currentfill}%
\pgfsetlinewidth{0.000000pt}%
\definecolor{currentstroke}{rgb}{0.000000,0.000000,0.000000}%
\pgfsetstrokecolor{currentstroke}%
\pgfsetdash{}{0pt}%
\pgfpathmoveto{\pgfqpoint{4.848477in}{1.993509in}}%
\pgfpathlineto{\pgfqpoint{4.768315in}{2.073671in}}%
\pgfpathlineto{\pgfqpoint{4.765179in}{2.064262in}}%
\pgfpathlineto{\pgfqpoint{4.743226in}{2.105032in}}%
\pgfpathlineto{\pgfqpoint{4.783996in}{2.083079in}}%
\pgfpathlineto{\pgfqpoint{4.774587in}{2.079943in}}%
\pgfpathlineto{\pgfqpoint{4.854749in}{1.999781in}}%
\pgfpathlineto{\pgfqpoint{4.848477in}{1.993509in}}%
\pgfusepath{fill}%
\end{pgfscope}%
\begin{pgfscope}%
\pgfpathrectangle{\pgfqpoint{1.432000in}{0.528000in}}{\pgfqpoint{3.696000in}{3.696000in}} %
\pgfusepath{clip}%
\pgfsetbuttcap%
\pgfsetroundjoin%
\definecolor{currentfill}{rgb}{0.275191,0.194905,0.496005}%
\pgfsetfillcolor{currentfill}%
\pgfsetlinewidth{0.000000pt}%
\definecolor{currentstroke}{rgb}{0.000000,0.000000,0.000000}%
\pgfsetstrokecolor{currentstroke}%
\pgfsetdash{}{0pt}%
\pgfpathmoveto{\pgfqpoint{4.847178in}{1.996645in}}%
\pgfpathlineto{\pgfqpoint{4.847178in}{2.065115in}}%
\pgfpathlineto{\pgfqpoint{4.838307in}{2.060680in}}%
\pgfpathlineto{\pgfqpoint{4.851613in}{2.105032in}}%
\pgfpathlineto{\pgfqpoint{4.864919in}{2.060680in}}%
\pgfpathlineto{\pgfqpoint{4.856048in}{2.065115in}}%
\pgfpathlineto{\pgfqpoint{4.856048in}{1.996645in}}%
\pgfpathlineto{\pgfqpoint{4.847178in}{1.996645in}}%
\pgfusepath{fill}%
\end{pgfscope}%
\begin{pgfscope}%
\pgfpathrectangle{\pgfqpoint{1.432000in}{0.528000in}}{\pgfqpoint{3.696000in}{3.696000in}} %
\pgfusepath{clip}%
\pgfsetbuttcap%
\pgfsetroundjoin%
\definecolor{currentfill}{rgb}{0.282656,0.100196,0.422160}%
\pgfsetfillcolor{currentfill}%
\pgfsetlinewidth{0.000000pt}%
\definecolor{currentstroke}{rgb}{0.000000,0.000000,0.000000}%
\pgfsetstrokecolor{currentstroke}%
\pgfsetdash{}{0pt}%
\pgfpathmoveto{\pgfqpoint{4.847646in}{1.994662in}}%
\pgfpathlineto{\pgfqpoint{4.757110in}{2.175733in}}%
\pgfpathlineto{\pgfqpoint{4.751160in}{2.167799in}}%
\pgfpathlineto{\pgfqpoint{4.743226in}{2.213419in}}%
\pgfpathlineto{\pgfqpoint{4.774962in}{2.179700in}}%
\pgfpathlineto{\pgfqpoint{4.765044in}{2.179700in}}%
\pgfpathlineto{\pgfqpoint{4.855580in}{1.998629in}}%
\pgfpathlineto{\pgfqpoint{4.847646in}{1.994662in}}%
\pgfusepath{fill}%
\end{pgfscope}%
\begin{pgfscope}%
\pgfpathrectangle{\pgfqpoint{1.432000in}{0.528000in}}{\pgfqpoint{3.696000in}{3.696000in}} %
\pgfusepath{clip}%
\pgfsetbuttcap%
\pgfsetroundjoin%
\definecolor{currentfill}{rgb}{0.273006,0.204520,0.501721}%
\pgfsetfillcolor{currentfill}%
\pgfsetlinewidth{0.000000pt}%
\definecolor{currentstroke}{rgb}{0.000000,0.000000,0.000000}%
\pgfsetstrokecolor{currentstroke}%
\pgfsetdash{}{0pt}%
\pgfpathmoveto{\pgfqpoint{4.847178in}{1.996645in}}%
\pgfpathlineto{\pgfqpoint{4.847178in}{2.173503in}}%
\pgfpathlineto{\pgfqpoint{4.838307in}{2.169067in}}%
\pgfpathlineto{\pgfqpoint{4.851613in}{2.213419in}}%
\pgfpathlineto{\pgfqpoint{4.864919in}{2.169067in}}%
\pgfpathlineto{\pgfqpoint{4.856048in}{2.173503in}}%
\pgfpathlineto{\pgfqpoint{4.856048in}{1.996645in}}%
\pgfpathlineto{\pgfqpoint{4.847178in}{1.996645in}}%
\pgfusepath{fill}%
\end{pgfscope}%
\begin{pgfscope}%
\pgfpathrectangle{\pgfqpoint{1.432000in}{0.528000in}}{\pgfqpoint{3.696000in}{3.696000in}} %
\pgfusepath{clip}%
\pgfsetbuttcap%
\pgfsetroundjoin%
\definecolor{currentfill}{rgb}{0.221989,0.339161,0.548752}%
\pgfsetfillcolor{currentfill}%
\pgfsetlinewidth{0.000000pt}%
\definecolor{currentstroke}{rgb}{0.000000,0.000000,0.000000}%
\pgfsetstrokecolor{currentstroke}%
\pgfsetdash{}{0pt}%
\pgfpathmoveto{\pgfqpoint{4.955565in}{1.996645in}}%
\pgfpathlineto{\pgfqpoint{4.955565in}{2.065115in}}%
\pgfpathlineto{\pgfqpoint{4.946694in}{2.060680in}}%
\pgfpathlineto{\pgfqpoint{4.960000in}{2.105032in}}%
\pgfpathlineto{\pgfqpoint{4.973306in}{2.060680in}}%
\pgfpathlineto{\pgfqpoint{4.964435in}{2.065115in}}%
\pgfpathlineto{\pgfqpoint{4.964435in}{1.996645in}}%
\pgfpathlineto{\pgfqpoint{4.955565in}{1.996645in}}%
\pgfusepath{fill}%
\end{pgfscope}%
\begin{pgfscope}%
\pgfpathrectangle{\pgfqpoint{1.432000in}{0.528000in}}{\pgfqpoint{3.696000in}{3.696000in}} %
\pgfusepath{clip}%
\pgfsetbuttcap%
\pgfsetroundjoin%
\definecolor{currentfill}{rgb}{0.174274,0.445044,0.557792}%
\pgfsetfillcolor{currentfill}%
\pgfsetlinewidth{0.000000pt}%
\definecolor{currentstroke}{rgb}{0.000000,0.000000,0.000000}%
\pgfsetstrokecolor{currentstroke}%
\pgfsetdash{}{0pt}%
\pgfpathmoveto{\pgfqpoint{4.955565in}{1.996645in}}%
\pgfpathlineto{\pgfqpoint{4.955565in}{2.173503in}}%
\pgfpathlineto{\pgfqpoint{4.946694in}{2.169067in}}%
\pgfpathlineto{\pgfqpoint{4.960000in}{2.213419in}}%
\pgfpathlineto{\pgfqpoint{4.973306in}{2.169067in}}%
\pgfpathlineto{\pgfqpoint{4.964435in}{2.173503in}}%
\pgfpathlineto{\pgfqpoint{4.964435in}{1.996645in}}%
\pgfpathlineto{\pgfqpoint{4.955565in}{1.996645in}}%
\pgfusepath{fill}%
\end{pgfscope}%
\begin{pgfscope}%
\pgfpathrectangle{\pgfqpoint{1.432000in}{0.528000in}}{\pgfqpoint{3.696000in}{3.696000in}} %
\pgfusepath{clip}%
\pgfsetbuttcap%
\pgfsetroundjoin%
\definecolor{currentfill}{rgb}{0.132444,0.552216,0.553018}%
\pgfsetfillcolor{currentfill}%
\pgfsetlinewidth{0.000000pt}%
\definecolor{currentstroke}{rgb}{0.000000,0.000000,0.000000}%
\pgfsetstrokecolor{currentstroke}%
\pgfsetdash{}{0pt}%
\pgfpathmoveto{\pgfqpoint{1.604435in}{2.105032in}}%
\pgfpathlineto{\pgfqpoint{1.602218in}{2.108873in}}%
\pgfpathlineto{\pgfqpoint{1.597782in}{2.108873in}}%
\pgfpathlineto{\pgfqpoint{1.595565in}{2.105032in}}%
\pgfpathlineto{\pgfqpoint{1.597782in}{2.101191in}}%
\pgfpathlineto{\pgfqpoint{1.602218in}{2.101191in}}%
\pgfpathlineto{\pgfqpoint{1.604435in}{2.105032in}}%
\pgfpathlineto{\pgfqpoint{1.602218in}{2.108873in}}%
\pgfusepath{fill}%
\end{pgfscope}%
\begin{pgfscope}%
\pgfpathrectangle{\pgfqpoint{1.432000in}{0.528000in}}{\pgfqpoint{3.696000in}{3.696000in}} %
\pgfusepath{clip}%
\pgfsetbuttcap%
\pgfsetroundjoin%
\definecolor{currentfill}{rgb}{0.126453,0.570633,0.549841}%
\pgfsetfillcolor{currentfill}%
\pgfsetlinewidth{0.000000pt}%
\definecolor{currentstroke}{rgb}{0.000000,0.000000,0.000000}%
\pgfsetstrokecolor{currentstroke}%
\pgfsetdash{}{0pt}%
\pgfpathmoveto{\pgfqpoint{1.595565in}{2.105032in}}%
\pgfpathlineto{\pgfqpoint{1.595565in}{2.173503in}}%
\pgfpathlineto{\pgfqpoint{1.586694in}{2.169067in}}%
\pgfpathlineto{\pgfqpoint{1.600000in}{2.213419in}}%
\pgfpathlineto{\pgfqpoint{1.613306in}{2.169067in}}%
\pgfpathlineto{\pgfqpoint{1.604435in}{2.173503in}}%
\pgfpathlineto{\pgfqpoint{1.604435in}{2.105032in}}%
\pgfpathlineto{\pgfqpoint{1.595565in}{2.105032in}}%
\pgfusepath{fill}%
\end{pgfscope}%
\begin{pgfscope}%
\pgfpathrectangle{\pgfqpoint{1.432000in}{0.528000in}}{\pgfqpoint{3.696000in}{3.696000in}} %
\pgfusepath{clip}%
\pgfsetbuttcap%
\pgfsetroundjoin%
\definecolor{currentfill}{rgb}{0.273809,0.031497,0.358853}%
\pgfsetfillcolor{currentfill}%
\pgfsetlinewidth{0.000000pt}%
\definecolor{currentstroke}{rgb}{0.000000,0.000000,0.000000}%
\pgfsetstrokecolor{currentstroke}%
\pgfsetdash{}{0pt}%
\pgfpathmoveto{\pgfqpoint{1.708387in}{2.100597in}}%
\pgfpathlineto{\pgfqpoint{1.639917in}{2.100597in}}%
\pgfpathlineto{\pgfqpoint{1.644352in}{2.091727in}}%
\pgfpathlineto{\pgfqpoint{1.600000in}{2.105032in}}%
\pgfpathlineto{\pgfqpoint{1.644352in}{2.118338in}}%
\pgfpathlineto{\pgfqpoint{1.639917in}{2.109467in}}%
\pgfpathlineto{\pgfqpoint{1.708387in}{2.109467in}}%
\pgfpathlineto{\pgfqpoint{1.708387in}{2.100597in}}%
\pgfusepath{fill}%
\end{pgfscope}%
\begin{pgfscope}%
\pgfpathrectangle{\pgfqpoint{1.432000in}{0.528000in}}{\pgfqpoint{3.696000in}{3.696000in}} %
\pgfusepath{clip}%
\pgfsetbuttcap%
\pgfsetroundjoin%
\definecolor{currentfill}{rgb}{0.212395,0.359683,0.551710}%
\pgfsetfillcolor{currentfill}%
\pgfsetlinewidth{0.000000pt}%
\definecolor{currentstroke}{rgb}{0.000000,0.000000,0.000000}%
\pgfsetstrokecolor{currentstroke}%
\pgfsetdash{}{0pt}%
\pgfpathmoveto{\pgfqpoint{1.712822in}{2.105032in}}%
\pgfpathlineto{\pgfqpoint{1.710605in}{2.108873in}}%
\pgfpathlineto{\pgfqpoint{1.706169in}{2.108873in}}%
\pgfpathlineto{\pgfqpoint{1.703952in}{2.105032in}}%
\pgfpathlineto{\pgfqpoint{1.706169in}{2.101191in}}%
\pgfpathlineto{\pgfqpoint{1.710605in}{2.101191in}}%
\pgfpathlineto{\pgfqpoint{1.712822in}{2.105032in}}%
\pgfpathlineto{\pgfqpoint{1.710605in}{2.108873in}}%
\pgfusepath{fill}%
\end{pgfscope}%
\begin{pgfscope}%
\pgfpathrectangle{\pgfqpoint{1.432000in}{0.528000in}}{\pgfqpoint{3.696000in}{3.696000in}} %
\pgfusepath{clip}%
\pgfsetbuttcap%
\pgfsetroundjoin%
\definecolor{currentfill}{rgb}{0.272594,0.025563,0.353093}%
\pgfsetfillcolor{currentfill}%
\pgfsetlinewidth{0.000000pt}%
\definecolor{currentstroke}{rgb}{0.000000,0.000000,0.000000}%
\pgfsetstrokecolor{currentstroke}%
\pgfsetdash{}{0pt}%
\pgfpathmoveto{\pgfqpoint{1.705251in}{2.101896in}}%
\pgfpathlineto{\pgfqpoint{1.625089in}{2.182058in}}%
\pgfpathlineto{\pgfqpoint{1.621953in}{2.172649in}}%
\pgfpathlineto{\pgfqpoint{1.600000in}{2.213419in}}%
\pgfpathlineto{\pgfqpoint{1.640770in}{2.191466in}}%
\pgfpathlineto{\pgfqpoint{1.631362in}{2.188330in}}%
\pgfpathlineto{\pgfqpoint{1.711523in}{2.108168in}}%
\pgfpathlineto{\pgfqpoint{1.705251in}{2.101896in}}%
\pgfusepath{fill}%
\end{pgfscope}%
\begin{pgfscope}%
\pgfpathrectangle{\pgfqpoint{1.432000in}{0.528000in}}{\pgfqpoint{3.696000in}{3.696000in}} %
\pgfusepath{clip}%
\pgfsetbuttcap%
\pgfsetroundjoin%
\definecolor{currentfill}{rgb}{0.192357,0.403199,0.555836}%
\pgfsetfillcolor{currentfill}%
\pgfsetlinewidth{0.000000pt}%
\definecolor{currentstroke}{rgb}{0.000000,0.000000,0.000000}%
\pgfsetstrokecolor{currentstroke}%
\pgfsetdash{}{0pt}%
\pgfpathmoveto{\pgfqpoint{1.703952in}{2.105032in}}%
\pgfpathlineto{\pgfqpoint{1.703952in}{2.173503in}}%
\pgfpathlineto{\pgfqpoint{1.695081in}{2.169067in}}%
\pgfpathlineto{\pgfqpoint{1.708387in}{2.213419in}}%
\pgfpathlineto{\pgfqpoint{1.721693in}{2.169067in}}%
\pgfpathlineto{\pgfqpoint{1.712822in}{2.173503in}}%
\pgfpathlineto{\pgfqpoint{1.712822in}{2.105032in}}%
\pgfpathlineto{\pgfqpoint{1.703952in}{2.105032in}}%
\pgfusepath{fill}%
\end{pgfscope}%
\begin{pgfscope}%
\pgfpathrectangle{\pgfqpoint{1.432000in}{0.528000in}}{\pgfqpoint{3.696000in}{3.696000in}} %
\pgfusepath{clip}%
\pgfsetbuttcap%
\pgfsetroundjoin%
\definecolor{currentfill}{rgb}{0.283187,0.125848,0.444960}%
\pgfsetfillcolor{currentfill}%
\pgfsetlinewidth{0.000000pt}%
\definecolor{currentstroke}{rgb}{0.000000,0.000000,0.000000}%
\pgfsetstrokecolor{currentstroke}%
\pgfsetdash{}{0pt}%
\pgfpathmoveto{\pgfqpoint{1.816774in}{2.100597in}}%
\pgfpathlineto{\pgfqpoint{1.748304in}{2.100597in}}%
\pgfpathlineto{\pgfqpoint{1.752739in}{2.091727in}}%
\pgfpathlineto{\pgfqpoint{1.708387in}{2.105032in}}%
\pgfpathlineto{\pgfqpoint{1.752739in}{2.118338in}}%
\pgfpathlineto{\pgfqpoint{1.748304in}{2.109467in}}%
\pgfpathlineto{\pgfqpoint{1.816774in}{2.109467in}}%
\pgfpathlineto{\pgfqpoint{1.816774in}{2.100597in}}%
\pgfusepath{fill}%
\end{pgfscope}%
\begin{pgfscope}%
\pgfpathrectangle{\pgfqpoint{1.432000in}{0.528000in}}{\pgfqpoint{3.696000in}{3.696000in}} %
\pgfusepath{clip}%
\pgfsetbuttcap%
\pgfsetroundjoin%
\definecolor{currentfill}{rgb}{0.280868,0.160771,0.472899}%
\pgfsetfillcolor{currentfill}%
\pgfsetlinewidth{0.000000pt}%
\definecolor{currentstroke}{rgb}{0.000000,0.000000,0.000000}%
\pgfsetstrokecolor{currentstroke}%
\pgfsetdash{}{0pt}%
\pgfpathmoveto{\pgfqpoint{1.821209in}{2.105032in}}%
\pgfpathlineto{\pgfqpoint{1.818992in}{2.108873in}}%
\pgfpathlineto{\pgfqpoint{1.814557in}{2.108873in}}%
\pgfpathlineto{\pgfqpoint{1.812339in}{2.105032in}}%
\pgfpathlineto{\pgfqpoint{1.814557in}{2.101191in}}%
\pgfpathlineto{\pgfqpoint{1.818992in}{2.101191in}}%
\pgfpathlineto{\pgfqpoint{1.821209in}{2.105032in}}%
\pgfpathlineto{\pgfqpoint{1.818992in}{2.108873in}}%
\pgfusepath{fill}%
\end{pgfscope}%
\begin{pgfscope}%
\pgfpathrectangle{\pgfqpoint{1.432000in}{0.528000in}}{\pgfqpoint{3.696000in}{3.696000in}} %
\pgfusepath{clip}%
\pgfsetbuttcap%
\pgfsetroundjoin%
\definecolor{currentfill}{rgb}{0.282290,0.145912,0.461510}%
\pgfsetfillcolor{currentfill}%
\pgfsetlinewidth{0.000000pt}%
\definecolor{currentstroke}{rgb}{0.000000,0.000000,0.000000}%
\pgfsetstrokecolor{currentstroke}%
\pgfsetdash{}{0pt}%
\pgfpathmoveto{\pgfqpoint{1.813638in}{2.101896in}}%
\pgfpathlineto{\pgfqpoint{1.733476in}{2.182058in}}%
\pgfpathlineto{\pgfqpoint{1.730340in}{2.172649in}}%
\pgfpathlineto{\pgfqpoint{1.708387in}{2.213419in}}%
\pgfpathlineto{\pgfqpoint{1.749157in}{2.191466in}}%
\pgfpathlineto{\pgfqpoint{1.739749in}{2.188330in}}%
\pgfpathlineto{\pgfqpoint{1.819910in}{2.108168in}}%
\pgfpathlineto{\pgfqpoint{1.813638in}{2.101896in}}%
\pgfusepath{fill}%
\end{pgfscope}%
\begin{pgfscope}%
\pgfpathrectangle{\pgfqpoint{1.432000in}{0.528000in}}{\pgfqpoint{3.696000in}{3.696000in}} %
\pgfusepath{clip}%
\pgfsetbuttcap%
\pgfsetroundjoin%
\definecolor{currentfill}{rgb}{0.274128,0.199721,0.498911}%
\pgfsetfillcolor{currentfill}%
\pgfsetlinewidth{0.000000pt}%
\definecolor{currentstroke}{rgb}{0.000000,0.000000,0.000000}%
\pgfsetstrokecolor{currentstroke}%
\pgfsetdash{}{0pt}%
\pgfpathmoveto{\pgfqpoint{1.812339in}{2.105032in}}%
\pgfpathlineto{\pgfqpoint{1.812339in}{2.173503in}}%
\pgfpathlineto{\pgfqpoint{1.803469in}{2.169067in}}%
\pgfpathlineto{\pgfqpoint{1.816774in}{2.213419in}}%
\pgfpathlineto{\pgfqpoint{1.830080in}{2.169067in}}%
\pgfpathlineto{\pgfqpoint{1.821209in}{2.173503in}}%
\pgfpathlineto{\pgfqpoint{1.821209in}{2.105032in}}%
\pgfpathlineto{\pgfqpoint{1.812339in}{2.105032in}}%
\pgfusepath{fill}%
\end{pgfscope}%
\begin{pgfscope}%
\pgfpathrectangle{\pgfqpoint{1.432000in}{0.528000in}}{\pgfqpoint{3.696000in}{3.696000in}} %
\pgfusepath{clip}%
\pgfsetbuttcap%
\pgfsetroundjoin%
\definecolor{currentfill}{rgb}{0.281887,0.150881,0.465405}%
\pgfsetfillcolor{currentfill}%
\pgfsetlinewidth{0.000000pt}%
\definecolor{currentstroke}{rgb}{0.000000,0.000000,0.000000}%
\pgfsetstrokecolor{currentstroke}%
\pgfsetdash{}{0pt}%
\pgfpathmoveto{\pgfqpoint{1.925161in}{2.100597in}}%
\pgfpathlineto{\pgfqpoint{1.856691in}{2.100597in}}%
\pgfpathlineto{\pgfqpoint{1.861126in}{2.091727in}}%
\pgfpathlineto{\pgfqpoint{1.816774in}{2.105032in}}%
\pgfpathlineto{\pgfqpoint{1.861126in}{2.118338in}}%
\pgfpathlineto{\pgfqpoint{1.856691in}{2.109467in}}%
\pgfpathlineto{\pgfqpoint{1.925161in}{2.109467in}}%
\pgfpathlineto{\pgfqpoint{1.925161in}{2.100597in}}%
\pgfusepath{fill}%
\end{pgfscope}%
\begin{pgfscope}%
\pgfpathrectangle{\pgfqpoint{1.432000in}{0.528000in}}{\pgfqpoint{3.696000in}{3.696000in}} %
\pgfusepath{clip}%
\pgfsetbuttcap%
\pgfsetroundjoin%
\definecolor{currentfill}{rgb}{0.267004,0.004874,0.329415}%
\pgfsetfillcolor{currentfill}%
\pgfsetlinewidth{0.000000pt}%
\definecolor{currentstroke}{rgb}{0.000000,0.000000,0.000000}%
\pgfsetstrokecolor{currentstroke}%
\pgfsetdash{}{0pt}%
\pgfpathmoveto{\pgfqpoint{1.929596in}{2.105032in}}%
\pgfpathlineto{\pgfqpoint{1.927379in}{2.108873in}}%
\pgfpathlineto{\pgfqpoint{1.922944in}{2.108873in}}%
\pgfpathlineto{\pgfqpoint{1.920726in}{2.105032in}}%
\pgfpathlineto{\pgfqpoint{1.922944in}{2.101191in}}%
\pgfpathlineto{\pgfqpoint{1.927379in}{2.101191in}}%
\pgfpathlineto{\pgfqpoint{1.929596in}{2.105032in}}%
\pgfpathlineto{\pgfqpoint{1.927379in}{2.108873in}}%
\pgfusepath{fill}%
\end{pgfscope}%
\begin{pgfscope}%
\pgfpathrectangle{\pgfqpoint{1.432000in}{0.528000in}}{\pgfqpoint{3.696000in}{3.696000in}} %
\pgfusepath{clip}%
\pgfsetbuttcap%
\pgfsetroundjoin%
\definecolor{currentfill}{rgb}{0.275191,0.194905,0.496005}%
\pgfsetfillcolor{currentfill}%
\pgfsetlinewidth{0.000000pt}%
\definecolor{currentstroke}{rgb}{0.000000,0.000000,0.000000}%
\pgfsetstrokecolor{currentstroke}%
\pgfsetdash{}{0pt}%
\pgfpathmoveto{\pgfqpoint{1.922025in}{2.101896in}}%
\pgfpathlineto{\pgfqpoint{1.841863in}{2.182058in}}%
\pgfpathlineto{\pgfqpoint{1.838727in}{2.172649in}}%
\pgfpathlineto{\pgfqpoint{1.816774in}{2.213419in}}%
\pgfpathlineto{\pgfqpoint{1.857544in}{2.191466in}}%
\pgfpathlineto{\pgfqpoint{1.848136in}{2.188330in}}%
\pgfpathlineto{\pgfqpoint{1.928297in}{2.108168in}}%
\pgfpathlineto{\pgfqpoint{1.922025in}{2.101896in}}%
\pgfusepath{fill}%
\end{pgfscope}%
\begin{pgfscope}%
\pgfpathrectangle{\pgfqpoint{1.432000in}{0.528000in}}{\pgfqpoint{3.696000in}{3.696000in}} %
\pgfusepath{clip}%
\pgfsetbuttcap%
\pgfsetroundjoin%
\definecolor{currentfill}{rgb}{0.278791,0.062145,0.386592}%
\pgfsetfillcolor{currentfill}%
\pgfsetlinewidth{0.000000pt}%
\definecolor{currentstroke}{rgb}{0.000000,0.000000,0.000000}%
\pgfsetstrokecolor{currentstroke}%
\pgfsetdash{}{0pt}%
\pgfpathmoveto{\pgfqpoint{1.920726in}{2.105032in}}%
\pgfpathlineto{\pgfqpoint{1.920726in}{2.173503in}}%
\pgfpathlineto{\pgfqpoint{1.911856in}{2.169067in}}%
\pgfpathlineto{\pgfqpoint{1.925161in}{2.213419in}}%
\pgfpathlineto{\pgfqpoint{1.938467in}{2.169067in}}%
\pgfpathlineto{\pgfqpoint{1.929596in}{2.173503in}}%
\pgfpathlineto{\pgfqpoint{1.929596in}{2.105032in}}%
\pgfpathlineto{\pgfqpoint{1.920726in}{2.105032in}}%
\pgfusepath{fill}%
\end{pgfscope}%
\begin{pgfscope}%
\pgfpathrectangle{\pgfqpoint{1.432000in}{0.528000in}}{\pgfqpoint{3.696000in}{3.696000in}} %
\pgfusepath{clip}%
\pgfsetbuttcap%
\pgfsetroundjoin%
\definecolor{currentfill}{rgb}{0.280255,0.165693,0.476498}%
\pgfsetfillcolor{currentfill}%
\pgfsetlinewidth{0.000000pt}%
\definecolor{currentstroke}{rgb}{0.000000,0.000000,0.000000}%
\pgfsetstrokecolor{currentstroke}%
\pgfsetdash{}{0pt}%
\pgfpathmoveto{\pgfqpoint{2.033548in}{2.100597in}}%
\pgfpathlineto{\pgfqpoint{1.965078in}{2.100597in}}%
\pgfpathlineto{\pgfqpoint{1.969513in}{2.091727in}}%
\pgfpathlineto{\pgfqpoint{1.925161in}{2.105032in}}%
\pgfpathlineto{\pgfqpoint{1.969513in}{2.118338in}}%
\pgfpathlineto{\pgfqpoint{1.965078in}{2.109467in}}%
\pgfpathlineto{\pgfqpoint{2.033548in}{2.109467in}}%
\pgfpathlineto{\pgfqpoint{2.033548in}{2.100597in}}%
\pgfusepath{fill}%
\end{pgfscope}%
\begin{pgfscope}%
\pgfpathrectangle{\pgfqpoint{1.432000in}{0.528000in}}{\pgfqpoint{3.696000in}{3.696000in}} %
\pgfusepath{clip}%
\pgfsetbuttcap%
\pgfsetroundjoin%
\definecolor{currentfill}{rgb}{0.255645,0.260703,0.528312}%
\pgfsetfillcolor{currentfill}%
\pgfsetlinewidth{0.000000pt}%
\definecolor{currentstroke}{rgb}{0.000000,0.000000,0.000000}%
\pgfsetstrokecolor{currentstroke}%
\pgfsetdash{}{0pt}%
\pgfpathmoveto{\pgfqpoint{2.030412in}{2.101896in}}%
\pgfpathlineto{\pgfqpoint{1.950251in}{2.182058in}}%
\pgfpathlineto{\pgfqpoint{1.947114in}{2.172649in}}%
\pgfpathlineto{\pgfqpoint{1.925161in}{2.213419in}}%
\pgfpathlineto{\pgfqpoint{1.965931in}{2.191466in}}%
\pgfpathlineto{\pgfqpoint{1.956523in}{2.188330in}}%
\pgfpathlineto{\pgfqpoint{2.036685in}{2.108168in}}%
\pgfpathlineto{\pgfqpoint{2.030412in}{2.101896in}}%
\pgfusepath{fill}%
\end{pgfscope}%
\begin{pgfscope}%
\pgfpathrectangle{\pgfqpoint{1.432000in}{0.528000in}}{\pgfqpoint{3.696000in}{3.696000in}} %
\pgfusepath{clip}%
\pgfsetbuttcap%
\pgfsetroundjoin%
\definecolor{currentfill}{rgb}{0.275191,0.194905,0.496005}%
\pgfsetfillcolor{currentfill}%
\pgfsetlinewidth{0.000000pt}%
\definecolor{currentstroke}{rgb}{0.000000,0.000000,0.000000}%
\pgfsetstrokecolor{currentstroke}%
\pgfsetdash{}{0pt}%
\pgfpathmoveto{\pgfqpoint{2.141935in}{2.100597in}}%
\pgfpathlineto{\pgfqpoint{2.073465in}{2.100597in}}%
\pgfpathlineto{\pgfqpoint{2.077900in}{2.091727in}}%
\pgfpathlineto{\pgfqpoint{2.033548in}{2.105032in}}%
\pgfpathlineto{\pgfqpoint{2.077900in}{2.118338in}}%
\pgfpathlineto{\pgfqpoint{2.073465in}{2.109467in}}%
\pgfpathlineto{\pgfqpoint{2.141935in}{2.109467in}}%
\pgfpathlineto{\pgfqpoint{2.141935in}{2.100597in}}%
\pgfusepath{fill}%
\end{pgfscope}%
\begin{pgfscope}%
\pgfpathrectangle{\pgfqpoint{1.432000in}{0.528000in}}{\pgfqpoint{3.696000in}{3.696000in}} %
\pgfusepath{clip}%
\pgfsetbuttcap%
\pgfsetroundjoin%
\definecolor{currentfill}{rgb}{0.257322,0.256130,0.526563}%
\pgfsetfillcolor{currentfill}%
\pgfsetlinewidth{0.000000pt}%
\definecolor{currentstroke}{rgb}{0.000000,0.000000,0.000000}%
\pgfsetstrokecolor{currentstroke}%
\pgfsetdash{}{0pt}%
\pgfpathmoveto{\pgfqpoint{2.138799in}{2.101896in}}%
\pgfpathlineto{\pgfqpoint{2.058638in}{2.182058in}}%
\pgfpathlineto{\pgfqpoint{2.055502in}{2.172649in}}%
\pgfpathlineto{\pgfqpoint{2.033548in}{2.213419in}}%
\pgfpathlineto{\pgfqpoint{2.074318in}{2.191466in}}%
\pgfpathlineto{\pgfqpoint{2.064910in}{2.188330in}}%
\pgfpathlineto{\pgfqpoint{2.145072in}{2.108168in}}%
\pgfpathlineto{\pgfqpoint{2.138799in}{2.101896in}}%
\pgfusepath{fill}%
\end{pgfscope}%
\begin{pgfscope}%
\pgfpathrectangle{\pgfqpoint{1.432000in}{0.528000in}}{\pgfqpoint{3.696000in}{3.696000in}} %
\pgfusepath{clip}%
\pgfsetbuttcap%
\pgfsetroundjoin%
\definecolor{currentfill}{rgb}{0.277941,0.056324,0.381191}%
\pgfsetfillcolor{currentfill}%
\pgfsetlinewidth{0.000000pt}%
\definecolor{currentstroke}{rgb}{0.000000,0.000000,0.000000}%
\pgfsetstrokecolor{currentstroke}%
\pgfsetdash{}{0pt}%
\pgfpathmoveto{\pgfqpoint{2.250323in}{2.100597in}}%
\pgfpathlineto{\pgfqpoint{2.181852in}{2.100597in}}%
\pgfpathlineto{\pgfqpoint{2.186287in}{2.091727in}}%
\pgfpathlineto{\pgfqpoint{2.141935in}{2.105032in}}%
\pgfpathlineto{\pgfqpoint{2.186287in}{2.118338in}}%
\pgfpathlineto{\pgfqpoint{2.181852in}{2.109467in}}%
\pgfpathlineto{\pgfqpoint{2.250323in}{2.109467in}}%
\pgfpathlineto{\pgfqpoint{2.250323in}{2.100597in}}%
\pgfusepath{fill}%
\end{pgfscope}%
\begin{pgfscope}%
\pgfpathrectangle{\pgfqpoint{1.432000in}{0.528000in}}{\pgfqpoint{3.696000in}{3.696000in}} %
\pgfusepath{clip}%
\pgfsetbuttcap%
\pgfsetroundjoin%
\definecolor{currentfill}{rgb}{0.283229,0.120777,0.440584}%
\pgfsetfillcolor{currentfill}%
\pgfsetlinewidth{0.000000pt}%
\definecolor{currentstroke}{rgb}{0.000000,0.000000,0.000000}%
\pgfsetstrokecolor{currentstroke}%
\pgfsetdash{}{0pt}%
\pgfpathmoveto{\pgfqpoint{2.247186in}{2.101896in}}%
\pgfpathlineto{\pgfqpoint{2.167025in}{2.182058in}}%
\pgfpathlineto{\pgfqpoint{2.163889in}{2.172649in}}%
\pgfpathlineto{\pgfqpoint{2.141935in}{2.213419in}}%
\pgfpathlineto{\pgfqpoint{2.182706in}{2.191466in}}%
\pgfpathlineto{\pgfqpoint{2.173297in}{2.188330in}}%
\pgfpathlineto{\pgfqpoint{2.253459in}{2.108168in}}%
\pgfpathlineto{\pgfqpoint{2.247186in}{2.101896in}}%
\pgfusepath{fill}%
\end{pgfscope}%
\begin{pgfscope}%
\pgfpathrectangle{\pgfqpoint{1.432000in}{0.528000in}}{\pgfqpoint{3.696000in}{3.696000in}} %
\pgfusepath{clip}%
\pgfsetbuttcap%
\pgfsetroundjoin%
\definecolor{currentfill}{rgb}{0.281446,0.084320,0.407414}%
\pgfsetfillcolor{currentfill}%
\pgfsetlinewidth{0.000000pt}%
\definecolor{currentstroke}{rgb}{0.000000,0.000000,0.000000}%
\pgfsetstrokecolor{currentstroke}%
\pgfsetdash{}{0pt}%
\pgfpathmoveto{\pgfqpoint{2.358710in}{2.100597in}}%
\pgfpathlineto{\pgfqpoint{2.181852in}{2.100597in}}%
\pgfpathlineto{\pgfqpoint{2.186287in}{2.091727in}}%
\pgfpathlineto{\pgfqpoint{2.141935in}{2.105032in}}%
\pgfpathlineto{\pgfqpoint{2.186287in}{2.118338in}}%
\pgfpathlineto{\pgfqpoint{2.181852in}{2.109467in}}%
\pgfpathlineto{\pgfqpoint{2.358710in}{2.109467in}}%
\pgfpathlineto{\pgfqpoint{2.358710in}{2.100597in}}%
\pgfusepath{fill}%
\end{pgfscope}%
\begin{pgfscope}%
\pgfpathrectangle{\pgfqpoint{1.432000in}{0.528000in}}{\pgfqpoint{3.696000in}{3.696000in}} %
\pgfusepath{clip}%
\pgfsetbuttcap%
\pgfsetroundjoin%
\definecolor{currentfill}{rgb}{0.283072,0.130895,0.449241}%
\pgfsetfillcolor{currentfill}%
\pgfsetlinewidth{0.000000pt}%
\definecolor{currentstroke}{rgb}{0.000000,0.000000,0.000000}%
\pgfsetstrokecolor{currentstroke}%
\pgfsetdash{}{0pt}%
\pgfpathmoveto{\pgfqpoint{2.356726in}{2.101065in}}%
\pgfpathlineto{\pgfqpoint{2.175655in}{2.191601in}}%
\pgfpathlineto{\pgfqpoint{2.175655in}{2.181684in}}%
\pgfpathlineto{\pgfqpoint{2.141935in}{2.213419in}}%
\pgfpathlineto{\pgfqpoint{2.187556in}{2.205485in}}%
\pgfpathlineto{\pgfqpoint{2.179622in}{2.199535in}}%
\pgfpathlineto{\pgfqpoint{2.360693in}{2.108999in}}%
\pgfpathlineto{\pgfqpoint{2.356726in}{2.101065in}}%
\pgfusepath{fill}%
\end{pgfscope}%
\begin{pgfscope}%
\pgfpathrectangle{\pgfqpoint{1.432000in}{0.528000in}}{\pgfqpoint{3.696000in}{3.696000in}} %
\pgfusepath{clip}%
\pgfsetbuttcap%
\pgfsetroundjoin%
\definecolor{currentfill}{rgb}{0.267968,0.223549,0.512008}%
\pgfsetfillcolor{currentfill}%
\pgfsetlinewidth{0.000000pt}%
\definecolor{currentstroke}{rgb}{0.000000,0.000000,0.000000}%
\pgfsetstrokecolor{currentstroke}%
\pgfsetdash{}{0pt}%
\pgfpathmoveto{\pgfqpoint{2.467097in}{2.100597in}}%
\pgfpathlineto{\pgfqpoint{2.290239in}{2.100597in}}%
\pgfpathlineto{\pgfqpoint{2.294675in}{2.091727in}}%
\pgfpathlineto{\pgfqpoint{2.250323in}{2.105032in}}%
\pgfpathlineto{\pgfqpoint{2.294675in}{2.118338in}}%
\pgfpathlineto{\pgfqpoint{2.290239in}{2.109467in}}%
\pgfpathlineto{\pgfqpoint{2.467097in}{2.109467in}}%
\pgfpathlineto{\pgfqpoint{2.467097in}{2.100597in}}%
\pgfusepath{fill}%
\end{pgfscope}%
\begin{pgfscope}%
\pgfpathrectangle{\pgfqpoint{1.432000in}{0.528000in}}{\pgfqpoint{3.696000in}{3.696000in}} %
\pgfusepath{clip}%
\pgfsetbuttcap%
\pgfsetroundjoin%
\definecolor{currentfill}{rgb}{0.266580,0.228262,0.514349}%
\pgfsetfillcolor{currentfill}%
\pgfsetlinewidth{0.000000pt}%
\definecolor{currentstroke}{rgb}{0.000000,0.000000,0.000000}%
\pgfsetstrokecolor{currentstroke}%
\pgfsetdash{}{0pt}%
\pgfpathmoveto{\pgfqpoint{2.465113in}{2.101065in}}%
\pgfpathlineto{\pgfqpoint{2.284042in}{2.191601in}}%
\pgfpathlineto{\pgfqpoint{2.284042in}{2.181684in}}%
\pgfpathlineto{\pgfqpoint{2.250323in}{2.213419in}}%
\pgfpathlineto{\pgfqpoint{2.295943in}{2.205485in}}%
\pgfpathlineto{\pgfqpoint{2.288009in}{2.199535in}}%
\pgfpathlineto{\pgfqpoint{2.469080in}{2.108999in}}%
\pgfpathlineto{\pgfqpoint{2.465113in}{2.101065in}}%
\pgfusepath{fill}%
\end{pgfscope}%
\begin{pgfscope}%
\pgfpathrectangle{\pgfqpoint{1.432000in}{0.528000in}}{\pgfqpoint{3.696000in}{3.696000in}} %
\pgfusepath{clip}%
\pgfsetbuttcap%
\pgfsetroundjoin%
\definecolor{currentfill}{rgb}{0.270595,0.214069,0.507052}%
\pgfsetfillcolor{currentfill}%
\pgfsetlinewidth{0.000000pt}%
\definecolor{currentstroke}{rgb}{0.000000,0.000000,0.000000}%
\pgfsetstrokecolor{currentstroke}%
\pgfsetdash{}{0pt}%
\pgfpathmoveto{\pgfqpoint{2.575484in}{2.100597in}}%
\pgfpathlineto{\pgfqpoint{2.398626in}{2.100597in}}%
\pgfpathlineto{\pgfqpoint{2.403062in}{2.091727in}}%
\pgfpathlineto{\pgfqpoint{2.358710in}{2.105032in}}%
\pgfpathlineto{\pgfqpoint{2.403062in}{2.118338in}}%
\pgfpathlineto{\pgfqpoint{2.398626in}{2.109467in}}%
\pgfpathlineto{\pgfqpoint{2.575484in}{2.109467in}}%
\pgfpathlineto{\pgfqpoint{2.575484in}{2.100597in}}%
\pgfusepath{fill}%
\end{pgfscope}%
\begin{pgfscope}%
\pgfpathrectangle{\pgfqpoint{1.432000in}{0.528000in}}{\pgfqpoint{3.696000in}{3.696000in}} %
\pgfusepath{clip}%
\pgfsetbuttcap%
\pgfsetroundjoin%
\definecolor{currentfill}{rgb}{0.266580,0.228262,0.514349}%
\pgfsetfillcolor{currentfill}%
\pgfsetlinewidth{0.000000pt}%
\definecolor{currentstroke}{rgb}{0.000000,0.000000,0.000000}%
\pgfsetstrokecolor{currentstroke}%
\pgfsetdash{}{0pt}%
\pgfpathmoveto{\pgfqpoint{2.573500in}{2.101065in}}%
\pgfpathlineto{\pgfqpoint{2.392429in}{2.191601in}}%
\pgfpathlineto{\pgfqpoint{2.392429in}{2.181684in}}%
\pgfpathlineto{\pgfqpoint{2.358710in}{2.213419in}}%
\pgfpathlineto{\pgfqpoint{2.404330in}{2.205485in}}%
\pgfpathlineto{\pgfqpoint{2.396396in}{2.199535in}}%
\pgfpathlineto{\pgfqpoint{2.577467in}{2.108999in}}%
\pgfpathlineto{\pgfqpoint{2.573500in}{2.101065in}}%
\pgfusepath{fill}%
\end{pgfscope}%
\begin{pgfscope}%
\pgfpathrectangle{\pgfqpoint{1.432000in}{0.528000in}}{\pgfqpoint{3.696000in}{3.696000in}} %
\pgfusepath{clip}%
\pgfsetbuttcap%
\pgfsetroundjoin%
\definecolor{currentfill}{rgb}{0.119699,0.618490,0.536347}%
\pgfsetfillcolor{currentfill}%
\pgfsetlinewidth{0.000000pt}%
\definecolor{currentstroke}{rgb}{0.000000,0.000000,0.000000}%
\pgfsetstrokecolor{currentstroke}%
\pgfsetdash{}{0pt}%
\pgfpathmoveto{\pgfqpoint{2.681887in}{2.101065in}}%
\pgfpathlineto{\pgfqpoint{2.500816in}{2.191601in}}%
\pgfpathlineto{\pgfqpoint{2.500816in}{2.181684in}}%
\pgfpathlineto{\pgfqpoint{2.467097in}{2.213419in}}%
\pgfpathlineto{\pgfqpoint{2.512717in}{2.205485in}}%
\pgfpathlineto{\pgfqpoint{2.504783in}{2.199535in}}%
\pgfpathlineto{\pgfqpoint{2.685854in}{2.108999in}}%
\pgfpathlineto{\pgfqpoint{2.681887in}{2.101065in}}%
\pgfusepath{fill}%
\end{pgfscope}%
\begin{pgfscope}%
\pgfpathrectangle{\pgfqpoint{1.432000in}{0.528000in}}{\pgfqpoint{3.696000in}{3.696000in}} %
\pgfusepath{clip}%
\pgfsetbuttcap%
\pgfsetroundjoin%
\definecolor{currentfill}{rgb}{0.163625,0.471133,0.558148}%
\pgfsetfillcolor{currentfill}%
\pgfsetlinewidth{0.000000pt}%
\definecolor{currentstroke}{rgb}{0.000000,0.000000,0.000000}%
\pgfsetstrokecolor{currentstroke}%
\pgfsetdash{}{0pt}%
\pgfpathmoveto{\pgfqpoint{2.790275in}{2.101065in}}%
\pgfpathlineto{\pgfqpoint{2.609203in}{2.191601in}}%
\pgfpathlineto{\pgfqpoint{2.609203in}{2.181684in}}%
\pgfpathlineto{\pgfqpoint{2.575484in}{2.213419in}}%
\pgfpathlineto{\pgfqpoint{2.621104in}{2.205485in}}%
\pgfpathlineto{\pgfqpoint{2.613170in}{2.199535in}}%
\pgfpathlineto{\pgfqpoint{2.794242in}{2.108999in}}%
\pgfpathlineto{\pgfqpoint{2.790275in}{2.101065in}}%
\pgfusepath{fill}%
\end{pgfscope}%
\begin{pgfscope}%
\pgfpathrectangle{\pgfqpoint{1.432000in}{0.528000in}}{\pgfqpoint{3.696000in}{3.696000in}} %
\pgfusepath{clip}%
\pgfsetbuttcap%
\pgfsetroundjoin%
\definecolor{currentfill}{rgb}{0.280894,0.078907,0.402329}%
\pgfsetfillcolor{currentfill}%
\pgfsetlinewidth{0.000000pt}%
\definecolor{currentstroke}{rgb}{0.000000,0.000000,0.000000}%
\pgfsetstrokecolor{currentstroke}%
\pgfsetdash{}{0pt}%
\pgfpathmoveto{\pgfqpoint{2.789122in}{2.101896in}}%
\pgfpathlineto{\pgfqpoint{2.600573in}{2.290445in}}%
\pgfpathlineto{\pgfqpoint{2.597437in}{2.281036in}}%
\pgfpathlineto{\pgfqpoint{2.575484in}{2.321806in}}%
\pgfpathlineto{\pgfqpoint{2.616254in}{2.299853in}}%
\pgfpathlineto{\pgfqpoint{2.606845in}{2.296717in}}%
\pgfpathlineto{\pgfqpoint{2.795394in}{2.108168in}}%
\pgfpathlineto{\pgfqpoint{2.789122in}{2.101896in}}%
\pgfusepath{fill}%
\end{pgfscope}%
\begin{pgfscope}%
\pgfpathrectangle{\pgfqpoint{1.432000in}{0.528000in}}{\pgfqpoint{3.696000in}{3.696000in}} %
\pgfusepath{clip}%
\pgfsetbuttcap%
\pgfsetroundjoin%
\definecolor{currentfill}{rgb}{0.151918,0.500685,0.557587}%
\pgfsetfillcolor{currentfill}%
\pgfsetlinewidth{0.000000pt}%
\definecolor{currentstroke}{rgb}{0.000000,0.000000,0.000000}%
\pgfsetstrokecolor{currentstroke}%
\pgfsetdash{}{0pt}%
\pgfpathmoveto{\pgfqpoint{2.898662in}{2.101065in}}%
\pgfpathlineto{\pgfqpoint{2.717590in}{2.191601in}}%
\pgfpathlineto{\pgfqpoint{2.717590in}{2.181684in}}%
\pgfpathlineto{\pgfqpoint{2.683871in}{2.213419in}}%
\pgfpathlineto{\pgfqpoint{2.729491in}{2.205485in}}%
\pgfpathlineto{\pgfqpoint{2.721557in}{2.199535in}}%
\pgfpathlineto{\pgfqpoint{2.902629in}{2.108999in}}%
\pgfpathlineto{\pgfqpoint{2.898662in}{2.101065in}}%
\pgfusepath{fill}%
\end{pgfscope}%
\begin{pgfscope}%
\pgfpathrectangle{\pgfqpoint{1.432000in}{0.528000in}}{\pgfqpoint{3.696000in}{3.696000in}} %
\pgfusepath{clip}%
\pgfsetbuttcap%
\pgfsetroundjoin%
\definecolor{currentfill}{rgb}{0.283229,0.120777,0.440584}%
\pgfsetfillcolor{currentfill}%
\pgfsetlinewidth{0.000000pt}%
\definecolor{currentstroke}{rgb}{0.000000,0.000000,0.000000}%
\pgfsetstrokecolor{currentstroke}%
\pgfsetdash{}{0pt}%
\pgfpathmoveto{\pgfqpoint{2.897509in}{2.101896in}}%
\pgfpathlineto{\pgfqpoint{2.708960in}{2.290445in}}%
\pgfpathlineto{\pgfqpoint{2.705824in}{2.281036in}}%
\pgfpathlineto{\pgfqpoint{2.683871in}{2.321806in}}%
\pgfpathlineto{\pgfqpoint{2.724641in}{2.299853in}}%
\pgfpathlineto{\pgfqpoint{2.715233in}{2.296717in}}%
\pgfpathlineto{\pgfqpoint{2.903781in}{2.108168in}}%
\pgfpathlineto{\pgfqpoint{2.897509in}{2.101896in}}%
\pgfusepath{fill}%
\end{pgfscope}%
\begin{pgfscope}%
\pgfpathrectangle{\pgfqpoint{1.432000in}{0.528000in}}{\pgfqpoint{3.696000in}{3.696000in}} %
\pgfusepath{clip}%
\pgfsetbuttcap%
\pgfsetroundjoin%
\definecolor{currentfill}{rgb}{0.282910,0.105393,0.426902}%
\pgfsetfillcolor{currentfill}%
\pgfsetlinewidth{0.000000pt}%
\definecolor{currentstroke}{rgb}{0.000000,0.000000,0.000000}%
\pgfsetstrokecolor{currentstroke}%
\pgfsetdash{}{0pt}%
\pgfpathmoveto{\pgfqpoint{3.007049in}{2.101065in}}%
\pgfpathlineto{\pgfqpoint{2.825977in}{2.191601in}}%
\pgfpathlineto{\pgfqpoint{2.825977in}{2.181684in}}%
\pgfpathlineto{\pgfqpoint{2.792258in}{2.213419in}}%
\pgfpathlineto{\pgfqpoint{2.837878in}{2.205485in}}%
\pgfpathlineto{\pgfqpoint{2.829944in}{2.199535in}}%
\pgfpathlineto{\pgfqpoint{3.011016in}{2.108999in}}%
\pgfpathlineto{\pgfqpoint{3.007049in}{2.101065in}}%
\pgfusepath{fill}%
\end{pgfscope}%
\begin{pgfscope}%
\pgfpathrectangle{\pgfqpoint{1.432000in}{0.528000in}}{\pgfqpoint{3.696000in}{3.696000in}} %
\pgfusepath{clip}%
\pgfsetbuttcap%
\pgfsetroundjoin%
\definecolor{currentfill}{rgb}{0.129933,0.559582,0.551864}%
\pgfsetfillcolor{currentfill}%
\pgfsetlinewidth{0.000000pt}%
\definecolor{currentstroke}{rgb}{0.000000,0.000000,0.000000}%
\pgfsetstrokecolor{currentstroke}%
\pgfsetdash{}{0pt}%
\pgfpathmoveto{\pgfqpoint{3.005896in}{2.101896in}}%
\pgfpathlineto{\pgfqpoint{2.817347in}{2.290445in}}%
\pgfpathlineto{\pgfqpoint{2.814211in}{2.281036in}}%
\pgfpathlineto{\pgfqpoint{2.792258in}{2.321806in}}%
\pgfpathlineto{\pgfqpoint{2.833028in}{2.299853in}}%
\pgfpathlineto{\pgfqpoint{2.823620in}{2.296717in}}%
\pgfpathlineto{\pgfqpoint{3.012168in}{2.108168in}}%
\pgfpathlineto{\pgfqpoint{3.005896in}{2.101896in}}%
\pgfusepath{fill}%
\end{pgfscope}%
\begin{pgfscope}%
\pgfpathrectangle{\pgfqpoint{1.432000in}{0.528000in}}{\pgfqpoint{3.696000in}{3.696000in}} %
\pgfusepath{clip}%
\pgfsetbuttcap%
\pgfsetroundjoin%
\definecolor{currentfill}{rgb}{0.163625,0.471133,0.558148}%
\pgfsetfillcolor{currentfill}%
\pgfsetlinewidth{0.000000pt}%
\definecolor{currentstroke}{rgb}{0.000000,0.000000,0.000000}%
\pgfsetstrokecolor{currentstroke}%
\pgfsetdash{}{0pt}%
\pgfpathmoveto{\pgfqpoint{3.114283in}{2.101896in}}%
\pgfpathlineto{\pgfqpoint{2.925734in}{2.290445in}}%
\pgfpathlineto{\pgfqpoint{2.922598in}{2.281036in}}%
\pgfpathlineto{\pgfqpoint{2.900645in}{2.321806in}}%
\pgfpathlineto{\pgfqpoint{2.941415in}{2.299853in}}%
\pgfpathlineto{\pgfqpoint{2.932007in}{2.296717in}}%
\pgfpathlineto{\pgfqpoint{3.120556in}{2.108168in}}%
\pgfpathlineto{\pgfqpoint{3.114283in}{2.101896in}}%
\pgfusepath{fill}%
\end{pgfscope}%
\begin{pgfscope}%
\pgfpathrectangle{\pgfqpoint{1.432000in}{0.528000in}}{\pgfqpoint{3.696000in}{3.696000in}} %
\pgfusepath{clip}%
\pgfsetbuttcap%
\pgfsetroundjoin%
\definecolor{currentfill}{rgb}{0.233603,0.313828,0.543914}%
\pgfsetfillcolor{currentfill}%
\pgfsetlinewidth{0.000000pt}%
\definecolor{currentstroke}{rgb}{0.000000,0.000000,0.000000}%
\pgfsetstrokecolor{currentstroke}%
\pgfsetdash{}{0pt}%
\pgfpathmoveto{\pgfqpoint{3.222670in}{2.101896in}}%
\pgfpathlineto{\pgfqpoint{3.034122in}{2.290445in}}%
\pgfpathlineto{\pgfqpoint{3.030985in}{2.281036in}}%
\pgfpathlineto{\pgfqpoint{3.009032in}{2.321806in}}%
\pgfpathlineto{\pgfqpoint{3.049802in}{2.299853in}}%
\pgfpathlineto{\pgfqpoint{3.040394in}{2.296717in}}%
\pgfpathlineto{\pgfqpoint{3.228943in}{2.108168in}}%
\pgfpathlineto{\pgfqpoint{3.222670in}{2.101896in}}%
\pgfusepath{fill}%
\end{pgfscope}%
\begin{pgfscope}%
\pgfpathrectangle{\pgfqpoint{1.432000in}{0.528000in}}{\pgfqpoint{3.696000in}{3.696000in}} %
\pgfusepath{clip}%
\pgfsetbuttcap%
\pgfsetroundjoin%
\definecolor{currentfill}{rgb}{0.281924,0.089666,0.412415}%
\pgfsetfillcolor{currentfill}%
\pgfsetlinewidth{0.000000pt}%
\definecolor{currentstroke}{rgb}{0.000000,0.000000,0.000000}%
\pgfsetstrokecolor{currentstroke}%
\pgfsetdash{}{0pt}%
\pgfpathmoveto{\pgfqpoint{3.222116in}{2.102572in}}%
\pgfpathlineto{\pgfqpoint{3.027484in}{2.394521in}}%
\pgfpathlineto{\pgfqpoint{3.022563in}{2.385910in}}%
\pgfpathlineto{\pgfqpoint{3.009032in}{2.430194in}}%
\pgfpathlineto{\pgfqpoint{3.044705in}{2.400671in}}%
\pgfpathlineto{\pgfqpoint{3.034864in}{2.399441in}}%
\pgfpathlineto{\pgfqpoint{3.229497in}{2.107492in}}%
\pgfpathlineto{\pgfqpoint{3.222116in}{2.102572in}}%
\pgfusepath{fill}%
\end{pgfscope}%
\begin{pgfscope}%
\pgfpathrectangle{\pgfqpoint{1.432000in}{0.528000in}}{\pgfqpoint{3.696000in}{3.696000in}} %
\pgfusepath{clip}%
\pgfsetbuttcap%
\pgfsetroundjoin%
\definecolor{currentfill}{rgb}{0.174274,0.445044,0.557792}%
\pgfsetfillcolor{currentfill}%
\pgfsetlinewidth{0.000000pt}%
\definecolor{currentstroke}{rgb}{0.000000,0.000000,0.000000}%
\pgfsetstrokecolor{currentstroke}%
\pgfsetdash{}{0pt}%
\pgfpathmoveto{\pgfqpoint{3.331057in}{2.101896in}}%
\pgfpathlineto{\pgfqpoint{3.142509in}{2.290445in}}%
\pgfpathlineto{\pgfqpoint{3.139372in}{2.281036in}}%
\pgfpathlineto{\pgfqpoint{3.117419in}{2.321806in}}%
\pgfpathlineto{\pgfqpoint{3.158189in}{2.299853in}}%
\pgfpathlineto{\pgfqpoint{3.148781in}{2.296717in}}%
\pgfpathlineto{\pgfqpoint{3.337330in}{2.108168in}}%
\pgfpathlineto{\pgfqpoint{3.331057in}{2.101896in}}%
\pgfusepath{fill}%
\end{pgfscope}%
\begin{pgfscope}%
\pgfpathrectangle{\pgfqpoint{1.432000in}{0.528000in}}{\pgfqpoint{3.696000in}{3.696000in}} %
\pgfusepath{clip}%
\pgfsetbuttcap%
\pgfsetroundjoin%
\definecolor{currentfill}{rgb}{0.282327,0.094955,0.417331}%
\pgfsetfillcolor{currentfill}%
\pgfsetlinewidth{0.000000pt}%
\definecolor{currentstroke}{rgb}{0.000000,0.000000,0.000000}%
\pgfsetstrokecolor{currentstroke}%
\pgfsetdash{}{0pt}%
\pgfpathmoveto{\pgfqpoint{3.330503in}{2.102572in}}%
\pgfpathlineto{\pgfqpoint{3.135871in}{2.394521in}}%
\pgfpathlineto{\pgfqpoint{3.130950in}{2.385910in}}%
\pgfpathlineto{\pgfqpoint{3.117419in}{2.430194in}}%
\pgfpathlineto{\pgfqpoint{3.153092in}{2.400671in}}%
\pgfpathlineto{\pgfqpoint{3.143252in}{2.399441in}}%
\pgfpathlineto{\pgfqpoint{3.337884in}{2.107492in}}%
\pgfpathlineto{\pgfqpoint{3.330503in}{2.102572in}}%
\pgfusepath{fill}%
\end{pgfscope}%
\begin{pgfscope}%
\pgfpathrectangle{\pgfqpoint{1.432000in}{0.528000in}}{\pgfqpoint{3.696000in}{3.696000in}} %
\pgfusepath{clip}%
\pgfsetbuttcap%
\pgfsetroundjoin%
\definecolor{currentfill}{rgb}{0.187231,0.414746,0.556547}%
\pgfsetfillcolor{currentfill}%
\pgfsetlinewidth{0.000000pt}%
\definecolor{currentstroke}{rgb}{0.000000,0.000000,0.000000}%
\pgfsetstrokecolor{currentstroke}%
\pgfsetdash{}{0pt}%
\pgfpathmoveto{\pgfqpoint{3.439444in}{2.101896in}}%
\pgfpathlineto{\pgfqpoint{3.250896in}{2.290445in}}%
\pgfpathlineto{\pgfqpoint{3.247760in}{2.281036in}}%
\pgfpathlineto{\pgfqpoint{3.225806in}{2.321806in}}%
\pgfpathlineto{\pgfqpoint{3.266577in}{2.299853in}}%
\pgfpathlineto{\pgfqpoint{3.257168in}{2.296717in}}%
\pgfpathlineto{\pgfqpoint{3.445717in}{2.108168in}}%
\pgfpathlineto{\pgfqpoint{3.439444in}{2.101896in}}%
\pgfusepath{fill}%
\end{pgfscope}%
\begin{pgfscope}%
\pgfpathrectangle{\pgfqpoint{1.432000in}{0.528000in}}{\pgfqpoint{3.696000in}{3.696000in}} %
\pgfusepath{clip}%
\pgfsetbuttcap%
\pgfsetroundjoin%
\definecolor{currentfill}{rgb}{0.281924,0.089666,0.412415}%
\pgfsetfillcolor{currentfill}%
\pgfsetlinewidth{0.000000pt}%
\definecolor{currentstroke}{rgb}{0.000000,0.000000,0.000000}%
\pgfsetstrokecolor{currentstroke}%
\pgfsetdash{}{0pt}%
\pgfpathmoveto{\pgfqpoint{3.548508in}{2.101342in}}%
\pgfpathlineto{\pgfqpoint{3.256559in}{2.295974in}}%
\pgfpathlineto{\pgfqpoint{3.255329in}{2.286133in}}%
\pgfpathlineto{\pgfqpoint{3.225806in}{2.321806in}}%
\pgfpathlineto{\pgfqpoint{3.270090in}{2.308275in}}%
\pgfpathlineto{\pgfqpoint{3.261479in}{2.303355in}}%
\pgfpathlineto{\pgfqpoint{3.553428in}{2.108723in}}%
\pgfpathlineto{\pgfqpoint{3.548508in}{2.101342in}}%
\pgfusepath{fill}%
\end{pgfscope}%
\begin{pgfscope}%
\pgfpathrectangle{\pgfqpoint{1.432000in}{0.528000in}}{\pgfqpoint{3.696000in}{3.696000in}} %
\pgfusepath{clip}%
\pgfsetbuttcap%
\pgfsetroundjoin%
\definecolor{currentfill}{rgb}{0.276194,0.190074,0.493001}%
\pgfsetfillcolor{currentfill}%
\pgfsetlinewidth{0.000000pt}%
\definecolor{currentstroke}{rgb}{0.000000,0.000000,0.000000}%
\pgfsetstrokecolor{currentstroke}%
\pgfsetdash{}{0pt}%
\pgfpathmoveto{\pgfqpoint{3.547832in}{2.101896in}}%
\pgfpathlineto{\pgfqpoint{3.359283in}{2.290445in}}%
\pgfpathlineto{\pgfqpoint{3.356147in}{2.281036in}}%
\pgfpathlineto{\pgfqpoint{3.334194in}{2.321806in}}%
\pgfpathlineto{\pgfqpoint{3.374964in}{2.299853in}}%
\pgfpathlineto{\pgfqpoint{3.365555in}{2.296717in}}%
\pgfpathlineto{\pgfqpoint{3.554104in}{2.108168in}}%
\pgfpathlineto{\pgfqpoint{3.547832in}{2.101896in}}%
\pgfusepath{fill}%
\end{pgfscope}%
\begin{pgfscope}%
\pgfpathrectangle{\pgfqpoint{1.432000in}{0.528000in}}{\pgfqpoint{3.696000in}{3.696000in}} %
\pgfusepath{clip}%
\pgfsetbuttcap%
\pgfsetroundjoin%
\definecolor{currentfill}{rgb}{0.177423,0.437527,0.557565}%
\pgfsetfillcolor{currentfill}%
\pgfsetlinewidth{0.000000pt}%
\definecolor{currentstroke}{rgb}{0.000000,0.000000,0.000000}%
\pgfsetstrokecolor{currentstroke}%
\pgfsetdash{}{0pt}%
\pgfpathmoveto{\pgfqpoint{3.656895in}{2.101342in}}%
\pgfpathlineto{\pgfqpoint{3.364946in}{2.295974in}}%
\pgfpathlineto{\pgfqpoint{3.363716in}{2.286133in}}%
\pgfpathlineto{\pgfqpoint{3.334194in}{2.321806in}}%
\pgfpathlineto{\pgfqpoint{3.378477in}{2.308275in}}%
\pgfpathlineto{\pgfqpoint{3.369867in}{2.303355in}}%
\pgfpathlineto{\pgfqpoint{3.661815in}{2.108723in}}%
\pgfpathlineto{\pgfqpoint{3.656895in}{2.101342in}}%
\pgfusepath{fill}%
\end{pgfscope}%
\begin{pgfscope}%
\pgfpathrectangle{\pgfqpoint{1.432000in}{0.528000in}}{\pgfqpoint{3.696000in}{3.696000in}} %
\pgfusepath{clip}%
\pgfsetbuttcap%
\pgfsetroundjoin%
\definecolor{currentfill}{rgb}{0.120565,0.596422,0.543611}%
\pgfsetfillcolor{currentfill}%
\pgfsetlinewidth{0.000000pt}%
\definecolor{currentstroke}{rgb}{0.000000,0.000000,0.000000}%
\pgfsetstrokecolor{currentstroke}%
\pgfsetdash{}{0pt}%
\pgfpathmoveto{\pgfqpoint{3.765282in}{2.101342in}}%
\pgfpathlineto{\pgfqpoint{3.473333in}{2.295974in}}%
\pgfpathlineto{\pgfqpoint{3.472103in}{2.286133in}}%
\pgfpathlineto{\pgfqpoint{3.442581in}{2.321806in}}%
\pgfpathlineto{\pgfqpoint{3.486864in}{2.308275in}}%
\pgfpathlineto{\pgfqpoint{3.478254in}{2.303355in}}%
\pgfpathlineto{\pgfqpoint{3.770202in}{2.108723in}}%
\pgfpathlineto{\pgfqpoint{3.765282in}{2.101342in}}%
\pgfusepath{fill}%
\end{pgfscope}%
\begin{pgfscope}%
\pgfpathrectangle{\pgfqpoint{1.432000in}{0.528000in}}{\pgfqpoint{3.696000in}{3.696000in}} %
\pgfusepath{clip}%
\pgfsetbuttcap%
\pgfsetroundjoin%
\definecolor{currentfill}{rgb}{0.283229,0.120777,0.440584}%
\pgfsetfillcolor{currentfill}%
\pgfsetlinewidth{0.000000pt}%
\definecolor{currentstroke}{rgb}{0.000000,0.000000,0.000000}%
\pgfsetstrokecolor{currentstroke}%
\pgfsetdash{}{0pt}%
\pgfpathmoveto{\pgfqpoint{3.874146in}{2.101065in}}%
\pgfpathlineto{\pgfqpoint{3.476300in}{2.299988in}}%
\pgfpathlineto{\pgfqpoint{3.476300in}{2.290071in}}%
\pgfpathlineto{\pgfqpoint{3.442581in}{2.321806in}}%
\pgfpathlineto{\pgfqpoint{3.488201in}{2.313873in}}%
\pgfpathlineto{\pgfqpoint{3.480267in}{2.307922in}}%
\pgfpathlineto{\pgfqpoint{3.878113in}{2.108999in}}%
\pgfpathlineto{\pgfqpoint{3.874146in}{2.101065in}}%
\pgfusepath{fill}%
\end{pgfscope}%
\begin{pgfscope}%
\pgfpathrectangle{\pgfqpoint{1.432000in}{0.528000in}}{\pgfqpoint{3.696000in}{3.696000in}} %
\pgfusepath{clip}%
\pgfsetbuttcap%
\pgfsetroundjoin%
\definecolor{currentfill}{rgb}{0.182256,0.426184,0.557120}%
\pgfsetfillcolor{currentfill}%
\pgfsetlinewidth{0.000000pt}%
\definecolor{currentstroke}{rgb}{0.000000,0.000000,0.000000}%
\pgfsetstrokecolor{currentstroke}%
\pgfsetdash{}{0pt}%
\pgfpathmoveto{\pgfqpoint{3.873669in}{2.101342in}}%
\pgfpathlineto{\pgfqpoint{3.581720in}{2.295974in}}%
\pgfpathlineto{\pgfqpoint{3.580490in}{2.286133in}}%
\pgfpathlineto{\pgfqpoint{3.550968in}{2.321806in}}%
\pgfpathlineto{\pgfqpoint{3.595251in}{2.308275in}}%
\pgfpathlineto{\pgfqpoint{3.586641in}{2.303355in}}%
\pgfpathlineto{\pgfqpoint{3.878589in}{2.108723in}}%
\pgfpathlineto{\pgfqpoint{3.873669in}{2.101342in}}%
\pgfusepath{fill}%
\end{pgfscope}%
\begin{pgfscope}%
\pgfpathrectangle{\pgfqpoint{1.432000in}{0.528000in}}{\pgfqpoint{3.696000in}{3.696000in}} %
\pgfusepath{clip}%
\pgfsetbuttcap%
\pgfsetroundjoin%
\definecolor{currentfill}{rgb}{0.172719,0.448791,0.557885}%
\pgfsetfillcolor{currentfill}%
\pgfsetlinewidth{0.000000pt}%
\definecolor{currentstroke}{rgb}{0.000000,0.000000,0.000000}%
\pgfsetstrokecolor{currentstroke}%
\pgfsetdash{}{0pt}%
\pgfpathmoveto{\pgfqpoint{3.982533in}{2.101065in}}%
\pgfpathlineto{\pgfqpoint{3.584687in}{2.299988in}}%
\pgfpathlineto{\pgfqpoint{3.584687in}{2.290071in}}%
\pgfpathlineto{\pgfqpoint{3.550968in}{2.321806in}}%
\pgfpathlineto{\pgfqpoint{3.596588in}{2.313873in}}%
\pgfpathlineto{\pgfqpoint{3.588654in}{2.307922in}}%
\pgfpathlineto{\pgfqpoint{3.986500in}{2.108999in}}%
\pgfpathlineto{\pgfqpoint{3.982533in}{2.101065in}}%
\pgfusepath{fill}%
\end{pgfscope}%
\begin{pgfscope}%
\pgfpathrectangle{\pgfqpoint{1.432000in}{0.528000in}}{\pgfqpoint{3.696000in}{3.696000in}} %
\pgfusepath{clip}%
\pgfsetbuttcap%
\pgfsetroundjoin%
\definecolor{currentfill}{rgb}{0.281412,0.155834,0.469201}%
\pgfsetfillcolor{currentfill}%
\pgfsetlinewidth{0.000000pt}%
\definecolor{currentstroke}{rgb}{0.000000,0.000000,0.000000}%
\pgfsetstrokecolor{currentstroke}%
\pgfsetdash{}{0pt}%
\pgfpathmoveto{\pgfqpoint{3.982056in}{2.101342in}}%
\pgfpathlineto{\pgfqpoint{3.690107in}{2.295974in}}%
\pgfpathlineto{\pgfqpoint{3.688877in}{2.286133in}}%
\pgfpathlineto{\pgfqpoint{3.659355in}{2.321806in}}%
\pgfpathlineto{\pgfqpoint{3.703639in}{2.308275in}}%
\pgfpathlineto{\pgfqpoint{3.695028in}{2.303355in}}%
\pgfpathlineto{\pgfqpoint{3.986976in}{2.108723in}}%
\pgfpathlineto{\pgfqpoint{3.982056in}{2.101342in}}%
\pgfusepath{fill}%
\end{pgfscope}%
\begin{pgfscope}%
\pgfpathrectangle{\pgfqpoint{1.432000in}{0.528000in}}{\pgfqpoint{3.696000in}{3.696000in}} %
\pgfusepath{clip}%
\pgfsetbuttcap%
\pgfsetroundjoin%
\definecolor{currentfill}{rgb}{0.185783,0.704891,0.485273}%
\pgfsetfillcolor{currentfill}%
\pgfsetlinewidth{0.000000pt}%
\definecolor{currentstroke}{rgb}{0.000000,0.000000,0.000000}%
\pgfsetstrokecolor{currentstroke}%
\pgfsetdash{}{0pt}%
\pgfpathmoveto{\pgfqpoint{4.090920in}{2.101065in}}%
\pgfpathlineto{\pgfqpoint{3.693074in}{2.299988in}}%
\pgfpathlineto{\pgfqpoint{3.693074in}{2.290071in}}%
\pgfpathlineto{\pgfqpoint{3.659355in}{2.321806in}}%
\pgfpathlineto{\pgfqpoint{3.704975in}{2.313873in}}%
\pgfpathlineto{\pgfqpoint{3.697041in}{2.307922in}}%
\pgfpathlineto{\pgfqpoint{4.094887in}{2.108999in}}%
\pgfpathlineto{\pgfqpoint{4.090920in}{2.101065in}}%
\pgfusepath{fill}%
\end{pgfscope}%
\begin{pgfscope}%
\pgfpathrectangle{\pgfqpoint{1.432000in}{0.528000in}}{\pgfqpoint{3.696000in}{3.696000in}} %
\pgfusepath{clip}%
\pgfsetbuttcap%
\pgfsetroundjoin%
\definecolor{currentfill}{rgb}{0.146616,0.673050,0.508936}%
\pgfsetfillcolor{currentfill}%
\pgfsetlinewidth{0.000000pt}%
\definecolor{currentstroke}{rgb}{0.000000,0.000000,0.000000}%
\pgfsetstrokecolor{currentstroke}%
\pgfsetdash{}{0pt}%
\pgfpathmoveto{\pgfqpoint{4.199307in}{2.101065in}}%
\pgfpathlineto{\pgfqpoint{3.801461in}{2.299988in}}%
\pgfpathlineto{\pgfqpoint{3.801461in}{2.290071in}}%
\pgfpathlineto{\pgfqpoint{3.767742in}{2.321806in}}%
\pgfpathlineto{\pgfqpoint{3.813362in}{2.313873in}}%
\pgfpathlineto{\pgfqpoint{3.805428in}{2.307922in}}%
\pgfpathlineto{\pgfqpoint{4.203274in}{2.108999in}}%
\pgfpathlineto{\pgfqpoint{4.199307in}{2.101065in}}%
\pgfusepath{fill}%
\end{pgfscope}%
\begin{pgfscope}%
\pgfpathrectangle{\pgfqpoint{1.432000in}{0.528000in}}{\pgfqpoint{3.696000in}{3.696000in}} %
\pgfusepath{clip}%
\pgfsetbuttcap%
\pgfsetroundjoin%
\definecolor{currentfill}{rgb}{0.216210,0.351535,0.550627}%
\pgfsetfillcolor{currentfill}%
\pgfsetlinewidth{0.000000pt}%
\definecolor{currentstroke}{rgb}{0.000000,0.000000,0.000000}%
\pgfsetstrokecolor{currentstroke}%
\pgfsetdash{}{0pt}%
\pgfpathmoveto{\pgfqpoint{4.307694in}{2.101065in}}%
\pgfpathlineto{\pgfqpoint{3.909848in}{2.299988in}}%
\pgfpathlineto{\pgfqpoint{3.909848in}{2.290071in}}%
\pgfpathlineto{\pgfqpoint{3.876129in}{2.321806in}}%
\pgfpathlineto{\pgfqpoint{3.921749in}{2.313873in}}%
\pgfpathlineto{\pgfqpoint{3.913815in}{2.307922in}}%
\pgfpathlineto{\pgfqpoint{4.311661in}{2.108999in}}%
\pgfpathlineto{\pgfqpoint{4.307694in}{2.101065in}}%
\pgfusepath{fill}%
\end{pgfscope}%
\begin{pgfscope}%
\pgfpathrectangle{\pgfqpoint{1.432000in}{0.528000in}}{\pgfqpoint{3.696000in}{3.696000in}} %
\pgfusepath{clip}%
\pgfsetbuttcap%
\pgfsetroundjoin%
\definecolor{currentfill}{rgb}{0.282884,0.135920,0.453427}%
\pgfsetfillcolor{currentfill}%
\pgfsetlinewidth{0.000000pt}%
\definecolor{currentstroke}{rgb}{0.000000,0.000000,0.000000}%
\pgfsetstrokecolor{currentstroke}%
\pgfsetdash{}{0pt}%
\pgfpathmoveto{\pgfqpoint{4.307217in}{2.101342in}}%
\pgfpathlineto{\pgfqpoint{4.015269in}{2.295974in}}%
\pgfpathlineto{\pgfqpoint{4.014039in}{2.286133in}}%
\pgfpathlineto{\pgfqpoint{3.984516in}{2.321806in}}%
\pgfpathlineto{\pgfqpoint{4.028800in}{2.308275in}}%
\pgfpathlineto{\pgfqpoint{4.020189in}{2.303355in}}%
\pgfpathlineto{\pgfqpoint{4.312138in}{2.108723in}}%
\pgfpathlineto{\pgfqpoint{4.307217in}{2.101342in}}%
\pgfusepath{fill}%
\end{pgfscope}%
\begin{pgfscope}%
\pgfpathrectangle{\pgfqpoint{1.432000in}{0.528000in}}{\pgfqpoint{3.696000in}{3.696000in}} %
\pgfusepath{clip}%
\pgfsetbuttcap%
\pgfsetroundjoin%
\definecolor{currentfill}{rgb}{0.283091,0.110553,0.431554}%
\pgfsetfillcolor{currentfill}%
\pgfsetlinewidth{0.000000pt}%
\definecolor{currentstroke}{rgb}{0.000000,0.000000,0.000000}%
\pgfsetstrokecolor{currentstroke}%
\pgfsetdash{}{0pt}%
\pgfpathmoveto{\pgfqpoint{4.306541in}{2.101896in}}%
\pgfpathlineto{\pgfqpoint{4.009605in}{2.398832in}}%
\pgfpathlineto{\pgfqpoint{4.006469in}{2.389423in}}%
\pgfpathlineto{\pgfqpoint{3.984516in}{2.430194in}}%
\pgfpathlineto{\pgfqpoint{4.025286in}{2.408240in}}%
\pgfpathlineto{\pgfqpoint{4.015878in}{2.405104in}}%
\pgfpathlineto{\pgfqpoint{4.312814in}{2.108168in}}%
\pgfpathlineto{\pgfqpoint{4.306541in}{2.101896in}}%
\pgfusepath{fill}%
\end{pgfscope}%
\begin{pgfscope}%
\pgfpathrectangle{\pgfqpoint{1.432000in}{0.528000in}}{\pgfqpoint{3.696000in}{3.696000in}} %
\pgfusepath{clip}%
\pgfsetbuttcap%
\pgfsetroundjoin%
\definecolor{currentfill}{rgb}{0.243113,0.292092,0.538516}%
\pgfsetfillcolor{currentfill}%
\pgfsetlinewidth{0.000000pt}%
\definecolor{currentstroke}{rgb}{0.000000,0.000000,0.000000}%
\pgfsetstrokecolor{currentstroke}%
\pgfsetdash{}{0pt}%
\pgfpathmoveto{\pgfqpoint{4.415604in}{2.101342in}}%
\pgfpathlineto{\pgfqpoint{4.123656in}{2.295974in}}%
\pgfpathlineto{\pgfqpoint{4.122426in}{2.286133in}}%
\pgfpathlineto{\pgfqpoint{4.092903in}{2.321806in}}%
\pgfpathlineto{\pgfqpoint{4.137187in}{2.308275in}}%
\pgfpathlineto{\pgfqpoint{4.128576in}{2.303355in}}%
\pgfpathlineto{\pgfqpoint{4.420525in}{2.108723in}}%
\pgfpathlineto{\pgfqpoint{4.415604in}{2.101342in}}%
\pgfusepath{fill}%
\end{pgfscope}%
\begin{pgfscope}%
\pgfpathrectangle{\pgfqpoint{1.432000in}{0.528000in}}{\pgfqpoint{3.696000in}{3.696000in}} %
\pgfusepath{clip}%
\pgfsetbuttcap%
\pgfsetroundjoin%
\definecolor{currentfill}{rgb}{0.163625,0.471133,0.558148}%
\pgfsetfillcolor{currentfill}%
\pgfsetlinewidth{0.000000pt}%
\definecolor{currentstroke}{rgb}{0.000000,0.000000,0.000000}%
\pgfsetstrokecolor{currentstroke}%
\pgfsetdash{}{0pt}%
\pgfpathmoveto{\pgfqpoint{4.414928in}{2.101896in}}%
\pgfpathlineto{\pgfqpoint{4.226380in}{2.290445in}}%
\pgfpathlineto{\pgfqpoint{4.223243in}{2.281036in}}%
\pgfpathlineto{\pgfqpoint{4.201290in}{2.321806in}}%
\pgfpathlineto{\pgfqpoint{4.242060in}{2.299853in}}%
\pgfpathlineto{\pgfqpoint{4.232652in}{2.296717in}}%
\pgfpathlineto{\pgfqpoint{4.421201in}{2.108168in}}%
\pgfpathlineto{\pgfqpoint{4.414928in}{2.101896in}}%
\pgfusepath{fill}%
\end{pgfscope}%
\begin{pgfscope}%
\pgfpathrectangle{\pgfqpoint{1.432000in}{0.528000in}}{\pgfqpoint{3.696000in}{3.696000in}} %
\pgfusepath{clip}%
\pgfsetbuttcap%
\pgfsetroundjoin%
\definecolor{currentfill}{rgb}{0.281446,0.084320,0.407414}%
\pgfsetfillcolor{currentfill}%
\pgfsetlinewidth{0.000000pt}%
\definecolor{currentstroke}{rgb}{0.000000,0.000000,0.000000}%
\pgfsetstrokecolor{currentstroke}%
\pgfsetdash{}{0pt}%
\pgfpathmoveto{\pgfqpoint{4.523991in}{2.101342in}}%
\pgfpathlineto{\pgfqpoint{4.232043in}{2.295974in}}%
\pgfpathlineto{\pgfqpoint{4.230813in}{2.286133in}}%
\pgfpathlineto{\pgfqpoint{4.201290in}{2.321806in}}%
\pgfpathlineto{\pgfqpoint{4.245574in}{2.308275in}}%
\pgfpathlineto{\pgfqpoint{4.236963in}{2.303355in}}%
\pgfpathlineto{\pgfqpoint{4.528912in}{2.108723in}}%
\pgfpathlineto{\pgfqpoint{4.523991in}{2.101342in}}%
\pgfusepath{fill}%
\end{pgfscope}%
\begin{pgfscope}%
\pgfpathrectangle{\pgfqpoint{1.432000in}{0.528000in}}{\pgfqpoint{3.696000in}{3.696000in}} %
\pgfusepath{clip}%
\pgfsetbuttcap%
\pgfsetroundjoin%
\definecolor{currentfill}{rgb}{0.140210,0.665859,0.513427}%
\pgfsetfillcolor{currentfill}%
\pgfsetlinewidth{0.000000pt}%
\definecolor{currentstroke}{rgb}{0.000000,0.000000,0.000000}%
\pgfsetstrokecolor{currentstroke}%
\pgfsetdash{}{0pt}%
\pgfpathmoveto{\pgfqpoint{4.523315in}{2.101896in}}%
\pgfpathlineto{\pgfqpoint{4.334767in}{2.290445in}}%
\pgfpathlineto{\pgfqpoint{4.331631in}{2.281036in}}%
\pgfpathlineto{\pgfqpoint{4.309677in}{2.321806in}}%
\pgfpathlineto{\pgfqpoint{4.350447in}{2.299853in}}%
\pgfpathlineto{\pgfqpoint{4.341039in}{2.296717in}}%
\pgfpathlineto{\pgfqpoint{4.529588in}{2.108168in}}%
\pgfpathlineto{\pgfqpoint{4.523315in}{2.101896in}}%
\pgfusepath{fill}%
\end{pgfscope}%
\begin{pgfscope}%
\pgfpathrectangle{\pgfqpoint{1.432000in}{0.528000in}}{\pgfqpoint{3.696000in}{3.696000in}} %
\pgfusepath{clip}%
\pgfsetbuttcap%
\pgfsetroundjoin%
\definecolor{currentfill}{rgb}{0.130067,0.651384,0.521608}%
\pgfsetfillcolor{currentfill}%
\pgfsetlinewidth{0.000000pt}%
\definecolor{currentstroke}{rgb}{0.000000,0.000000,0.000000}%
\pgfsetstrokecolor{currentstroke}%
\pgfsetdash{}{0pt}%
\pgfpathmoveto{\pgfqpoint{4.631703in}{2.101896in}}%
\pgfpathlineto{\pgfqpoint{4.443154in}{2.290445in}}%
\pgfpathlineto{\pgfqpoint{4.440018in}{2.281036in}}%
\pgfpathlineto{\pgfqpoint{4.418065in}{2.321806in}}%
\pgfpathlineto{\pgfqpoint{4.458835in}{2.299853in}}%
\pgfpathlineto{\pgfqpoint{4.449426in}{2.296717in}}%
\pgfpathlineto{\pgfqpoint{4.637975in}{2.108168in}}%
\pgfpathlineto{\pgfqpoint{4.631703in}{2.101896in}}%
\pgfusepath{fill}%
\end{pgfscope}%
\begin{pgfscope}%
\pgfpathrectangle{\pgfqpoint{1.432000in}{0.528000in}}{\pgfqpoint{3.696000in}{3.696000in}} %
\pgfusepath{clip}%
\pgfsetbuttcap%
\pgfsetroundjoin%
\definecolor{currentfill}{rgb}{0.194100,0.399323,0.555565}%
\pgfsetfillcolor{currentfill}%
\pgfsetlinewidth{0.000000pt}%
\definecolor{currentstroke}{rgb}{0.000000,0.000000,0.000000}%
\pgfsetstrokecolor{currentstroke}%
\pgfsetdash{}{0pt}%
\pgfpathmoveto{\pgfqpoint{4.740090in}{2.101896in}}%
\pgfpathlineto{\pgfqpoint{4.551541in}{2.290445in}}%
\pgfpathlineto{\pgfqpoint{4.548405in}{2.281036in}}%
\pgfpathlineto{\pgfqpoint{4.526452in}{2.321806in}}%
\pgfpathlineto{\pgfqpoint{4.567222in}{2.299853in}}%
\pgfpathlineto{\pgfqpoint{4.557813in}{2.296717in}}%
\pgfpathlineto{\pgfqpoint{4.746362in}{2.108168in}}%
\pgfpathlineto{\pgfqpoint{4.740090in}{2.101896in}}%
\pgfusepath{fill}%
\end{pgfscope}%
\begin{pgfscope}%
\pgfpathrectangle{\pgfqpoint{1.432000in}{0.528000in}}{\pgfqpoint{3.696000in}{3.696000in}} %
\pgfusepath{clip}%
\pgfsetbuttcap%
\pgfsetroundjoin%
\definecolor{currentfill}{rgb}{0.221989,0.339161,0.548752}%
\pgfsetfillcolor{currentfill}%
\pgfsetlinewidth{0.000000pt}%
\definecolor{currentstroke}{rgb}{0.000000,0.000000,0.000000}%
\pgfsetstrokecolor{currentstroke}%
\pgfsetdash{}{0pt}%
\pgfpathmoveto{\pgfqpoint{4.739259in}{2.103049in}}%
\pgfpathlineto{\pgfqpoint{4.648723in}{2.284120in}}%
\pgfpathlineto{\pgfqpoint{4.642773in}{2.276186in}}%
\pgfpathlineto{\pgfqpoint{4.634839in}{2.321806in}}%
\pgfpathlineto{\pgfqpoint{4.666574in}{2.288087in}}%
\pgfpathlineto{\pgfqpoint{4.656657in}{2.288087in}}%
\pgfpathlineto{\pgfqpoint{4.747193in}{2.107016in}}%
\pgfpathlineto{\pgfqpoint{4.739259in}{2.103049in}}%
\pgfusepath{fill}%
\end{pgfscope}%
\begin{pgfscope}%
\pgfpathrectangle{\pgfqpoint{1.432000in}{0.528000in}}{\pgfqpoint{3.696000in}{3.696000in}} %
\pgfusepath{clip}%
\pgfsetbuttcap%
\pgfsetroundjoin%
\definecolor{currentfill}{rgb}{0.183898,0.422383,0.556944}%
\pgfsetfillcolor{currentfill}%
\pgfsetlinewidth{0.000000pt}%
\definecolor{currentstroke}{rgb}{0.000000,0.000000,0.000000}%
\pgfsetstrokecolor{currentstroke}%
\pgfsetdash{}{0pt}%
\pgfpathmoveto{\pgfqpoint{4.847646in}{2.103049in}}%
\pgfpathlineto{\pgfqpoint{4.757110in}{2.284120in}}%
\pgfpathlineto{\pgfqpoint{4.751160in}{2.276186in}}%
\pgfpathlineto{\pgfqpoint{4.743226in}{2.321806in}}%
\pgfpathlineto{\pgfqpoint{4.774962in}{2.288087in}}%
\pgfpathlineto{\pgfqpoint{4.765044in}{2.288087in}}%
\pgfpathlineto{\pgfqpoint{4.855580in}{2.107016in}}%
\pgfpathlineto{\pgfqpoint{4.847646in}{2.103049in}}%
\pgfusepath{fill}%
\end{pgfscope}%
\begin{pgfscope}%
\pgfpathrectangle{\pgfqpoint{1.432000in}{0.528000in}}{\pgfqpoint{3.696000in}{3.696000in}} %
\pgfusepath{clip}%
\pgfsetbuttcap%
\pgfsetroundjoin%
\definecolor{currentfill}{rgb}{0.281412,0.155834,0.469201}%
\pgfsetfillcolor{currentfill}%
\pgfsetlinewidth{0.000000pt}%
\definecolor{currentstroke}{rgb}{0.000000,0.000000,0.000000}%
\pgfsetstrokecolor{currentstroke}%
\pgfsetdash{}{0pt}%
\pgfpathmoveto{\pgfqpoint{4.847178in}{2.105032in}}%
\pgfpathlineto{\pgfqpoint{4.847178in}{2.281890in}}%
\pgfpathlineto{\pgfqpoint{4.838307in}{2.277454in}}%
\pgfpathlineto{\pgfqpoint{4.851613in}{2.321806in}}%
\pgfpathlineto{\pgfqpoint{4.864919in}{2.277454in}}%
\pgfpathlineto{\pgfqpoint{4.856048in}{2.281890in}}%
\pgfpathlineto{\pgfqpoint{4.856048in}{2.105032in}}%
\pgfpathlineto{\pgfqpoint{4.847178in}{2.105032in}}%
\pgfusepath{fill}%
\end{pgfscope}%
\begin{pgfscope}%
\pgfpathrectangle{\pgfqpoint{1.432000in}{0.528000in}}{\pgfqpoint{3.696000in}{3.696000in}} %
\pgfusepath{clip}%
\pgfsetbuttcap%
\pgfsetroundjoin%
\definecolor{currentfill}{rgb}{0.260571,0.246922,0.522828}%
\pgfsetfillcolor{currentfill}%
\pgfsetlinewidth{0.000000pt}%
\definecolor{currentstroke}{rgb}{0.000000,0.000000,0.000000}%
\pgfsetstrokecolor{currentstroke}%
\pgfsetdash{}{0pt}%
\pgfpathmoveto{\pgfqpoint{4.956033in}{2.103049in}}%
\pgfpathlineto{\pgfqpoint{4.865497in}{2.284120in}}%
\pgfpathlineto{\pgfqpoint{4.859547in}{2.276186in}}%
\pgfpathlineto{\pgfqpoint{4.851613in}{2.321806in}}%
\pgfpathlineto{\pgfqpoint{4.883349in}{2.288087in}}%
\pgfpathlineto{\pgfqpoint{4.873431in}{2.288087in}}%
\pgfpathlineto{\pgfqpoint{4.963967in}{2.107016in}}%
\pgfpathlineto{\pgfqpoint{4.956033in}{2.103049in}}%
\pgfusepath{fill}%
\end{pgfscope}%
\begin{pgfscope}%
\pgfpathrectangle{\pgfqpoint{1.432000in}{0.528000in}}{\pgfqpoint{3.696000in}{3.696000in}} %
\pgfusepath{clip}%
\pgfsetbuttcap%
\pgfsetroundjoin%
\definecolor{currentfill}{rgb}{0.175841,0.441290,0.557685}%
\pgfsetfillcolor{currentfill}%
\pgfsetlinewidth{0.000000pt}%
\definecolor{currentstroke}{rgb}{0.000000,0.000000,0.000000}%
\pgfsetstrokecolor{currentstroke}%
\pgfsetdash{}{0pt}%
\pgfpathmoveto{\pgfqpoint{4.955565in}{2.105032in}}%
\pgfpathlineto{\pgfqpoint{4.955565in}{2.281890in}}%
\pgfpathlineto{\pgfqpoint{4.946694in}{2.277454in}}%
\pgfpathlineto{\pgfqpoint{4.960000in}{2.321806in}}%
\pgfpathlineto{\pgfqpoint{4.973306in}{2.277454in}}%
\pgfpathlineto{\pgfqpoint{4.964435in}{2.281890in}}%
\pgfpathlineto{\pgfqpoint{4.964435in}{2.105032in}}%
\pgfpathlineto{\pgfqpoint{4.955565in}{2.105032in}}%
\pgfusepath{fill}%
\end{pgfscope}%
\begin{pgfscope}%
\pgfpathrectangle{\pgfqpoint{1.432000in}{0.528000in}}{\pgfqpoint{3.696000in}{3.696000in}} %
\pgfusepath{clip}%
\pgfsetbuttcap%
\pgfsetroundjoin%
\definecolor{currentfill}{rgb}{0.272594,0.025563,0.353093}%
\pgfsetfillcolor{currentfill}%
\pgfsetlinewidth{0.000000pt}%
\definecolor{currentstroke}{rgb}{0.000000,0.000000,0.000000}%
\pgfsetstrokecolor{currentstroke}%
\pgfsetdash{}{0pt}%
\pgfpathmoveto{\pgfqpoint{1.604435in}{2.213419in}}%
\pgfpathlineto{\pgfqpoint{1.602218in}{2.217260in}}%
\pgfpathlineto{\pgfqpoint{1.597782in}{2.217260in}}%
\pgfpathlineto{\pgfqpoint{1.595565in}{2.213419in}}%
\pgfpathlineto{\pgfqpoint{1.597782in}{2.209578in}}%
\pgfpathlineto{\pgfqpoint{1.602218in}{2.209578in}}%
\pgfpathlineto{\pgfqpoint{1.604435in}{2.213419in}}%
\pgfpathlineto{\pgfqpoint{1.602218in}{2.217260in}}%
\pgfusepath{fill}%
\end{pgfscope}%
\begin{pgfscope}%
\pgfpathrectangle{\pgfqpoint{1.432000in}{0.528000in}}{\pgfqpoint{3.696000in}{3.696000in}} %
\pgfusepath{clip}%
\pgfsetbuttcap%
\pgfsetroundjoin%
\definecolor{currentfill}{rgb}{0.468053,0.818921,0.323998}%
\pgfsetfillcolor{currentfill}%
\pgfsetlinewidth{0.000000pt}%
\definecolor{currentstroke}{rgb}{0.000000,0.000000,0.000000}%
\pgfsetstrokecolor{currentstroke}%
\pgfsetdash{}{0pt}%
\pgfpathmoveto{\pgfqpoint{1.595565in}{2.213419in}}%
\pgfpathlineto{\pgfqpoint{1.595565in}{2.281890in}}%
\pgfpathlineto{\pgfqpoint{1.586694in}{2.277454in}}%
\pgfpathlineto{\pgfqpoint{1.600000in}{2.321806in}}%
\pgfpathlineto{\pgfqpoint{1.613306in}{2.277454in}}%
\pgfpathlineto{\pgfqpoint{1.604435in}{2.281890in}}%
\pgfpathlineto{\pgfqpoint{1.604435in}{2.213419in}}%
\pgfpathlineto{\pgfqpoint{1.595565in}{2.213419in}}%
\pgfusepath{fill}%
\end{pgfscope}%
\begin{pgfscope}%
\pgfpathrectangle{\pgfqpoint{1.432000in}{0.528000in}}{\pgfqpoint{3.696000in}{3.696000in}} %
\pgfusepath{clip}%
\pgfsetbuttcap%
\pgfsetroundjoin%
\definecolor{currentfill}{rgb}{0.281924,0.089666,0.412415}%
\pgfsetfillcolor{currentfill}%
\pgfsetlinewidth{0.000000pt}%
\definecolor{currentstroke}{rgb}{0.000000,0.000000,0.000000}%
\pgfsetstrokecolor{currentstroke}%
\pgfsetdash{}{0pt}%
\pgfpathmoveto{\pgfqpoint{1.705251in}{2.210283in}}%
\pgfpathlineto{\pgfqpoint{1.625089in}{2.290445in}}%
\pgfpathlineto{\pgfqpoint{1.621953in}{2.281036in}}%
\pgfpathlineto{\pgfqpoint{1.600000in}{2.321806in}}%
\pgfpathlineto{\pgfqpoint{1.640770in}{2.299853in}}%
\pgfpathlineto{\pgfqpoint{1.631362in}{2.296717in}}%
\pgfpathlineto{\pgfqpoint{1.711523in}{2.216556in}}%
\pgfpathlineto{\pgfqpoint{1.705251in}{2.210283in}}%
\pgfusepath{fill}%
\end{pgfscope}%
\begin{pgfscope}%
\pgfpathrectangle{\pgfqpoint{1.432000in}{0.528000in}}{\pgfqpoint{3.696000in}{3.696000in}} %
\pgfusepath{clip}%
\pgfsetbuttcap%
\pgfsetroundjoin%
\definecolor{currentfill}{rgb}{0.121831,0.589055,0.545623}%
\pgfsetfillcolor{currentfill}%
\pgfsetlinewidth{0.000000pt}%
\definecolor{currentstroke}{rgb}{0.000000,0.000000,0.000000}%
\pgfsetstrokecolor{currentstroke}%
\pgfsetdash{}{0pt}%
\pgfpathmoveto{\pgfqpoint{1.703952in}{2.213419in}}%
\pgfpathlineto{\pgfqpoint{1.703952in}{2.281890in}}%
\pgfpathlineto{\pgfqpoint{1.695081in}{2.277454in}}%
\pgfpathlineto{\pgfqpoint{1.708387in}{2.321806in}}%
\pgfpathlineto{\pgfqpoint{1.721693in}{2.277454in}}%
\pgfpathlineto{\pgfqpoint{1.712822in}{2.281890in}}%
\pgfpathlineto{\pgfqpoint{1.712822in}{2.213419in}}%
\pgfpathlineto{\pgfqpoint{1.703952in}{2.213419in}}%
\pgfusepath{fill}%
\end{pgfscope}%
\begin{pgfscope}%
\pgfpathrectangle{\pgfqpoint{1.432000in}{0.528000in}}{\pgfqpoint{3.696000in}{3.696000in}} %
\pgfusepath{clip}%
\pgfsetbuttcap%
\pgfsetroundjoin%
\definecolor{currentfill}{rgb}{0.278012,0.180367,0.486697}%
\pgfsetfillcolor{currentfill}%
\pgfsetlinewidth{0.000000pt}%
\definecolor{currentstroke}{rgb}{0.000000,0.000000,0.000000}%
\pgfsetstrokecolor{currentstroke}%
\pgfsetdash{}{0pt}%
\pgfpathmoveto{\pgfqpoint{1.813638in}{2.210283in}}%
\pgfpathlineto{\pgfqpoint{1.733476in}{2.290445in}}%
\pgfpathlineto{\pgfqpoint{1.730340in}{2.281036in}}%
\pgfpathlineto{\pgfqpoint{1.708387in}{2.321806in}}%
\pgfpathlineto{\pgfqpoint{1.749157in}{2.299853in}}%
\pgfpathlineto{\pgfqpoint{1.739749in}{2.296717in}}%
\pgfpathlineto{\pgfqpoint{1.819910in}{2.216556in}}%
\pgfpathlineto{\pgfqpoint{1.813638in}{2.210283in}}%
\pgfusepath{fill}%
\end{pgfscope}%
\begin{pgfscope}%
\pgfpathrectangle{\pgfqpoint{1.432000in}{0.528000in}}{\pgfqpoint{3.696000in}{3.696000in}} %
\pgfusepath{clip}%
\pgfsetbuttcap%
\pgfsetroundjoin%
\definecolor{currentfill}{rgb}{0.225863,0.330805,0.547314}%
\pgfsetfillcolor{currentfill}%
\pgfsetlinewidth{0.000000pt}%
\definecolor{currentstroke}{rgb}{0.000000,0.000000,0.000000}%
\pgfsetstrokecolor{currentstroke}%
\pgfsetdash{}{0pt}%
\pgfpathmoveto{\pgfqpoint{1.812339in}{2.213419in}}%
\pgfpathlineto{\pgfqpoint{1.812339in}{2.281890in}}%
\pgfpathlineto{\pgfqpoint{1.803469in}{2.277454in}}%
\pgfpathlineto{\pgfqpoint{1.816774in}{2.321806in}}%
\pgfpathlineto{\pgfqpoint{1.830080in}{2.277454in}}%
\pgfpathlineto{\pgfqpoint{1.821209in}{2.281890in}}%
\pgfpathlineto{\pgfqpoint{1.821209in}{2.213419in}}%
\pgfpathlineto{\pgfqpoint{1.812339in}{2.213419in}}%
\pgfusepath{fill}%
\end{pgfscope}%
\begin{pgfscope}%
\pgfpathrectangle{\pgfqpoint{1.432000in}{0.528000in}}{\pgfqpoint{3.696000in}{3.696000in}} %
\pgfusepath{clip}%
\pgfsetbuttcap%
\pgfsetroundjoin%
\definecolor{currentfill}{rgb}{0.220057,0.343307,0.549413}%
\pgfsetfillcolor{currentfill}%
\pgfsetlinewidth{0.000000pt}%
\definecolor{currentstroke}{rgb}{0.000000,0.000000,0.000000}%
\pgfsetstrokecolor{currentstroke}%
\pgfsetdash{}{0pt}%
\pgfpathmoveto{\pgfqpoint{1.922025in}{2.210283in}}%
\pgfpathlineto{\pgfqpoint{1.841863in}{2.290445in}}%
\pgfpathlineto{\pgfqpoint{1.838727in}{2.281036in}}%
\pgfpathlineto{\pgfqpoint{1.816774in}{2.321806in}}%
\pgfpathlineto{\pgfqpoint{1.857544in}{2.299853in}}%
\pgfpathlineto{\pgfqpoint{1.848136in}{2.296717in}}%
\pgfpathlineto{\pgfqpoint{1.928297in}{2.216556in}}%
\pgfpathlineto{\pgfqpoint{1.922025in}{2.210283in}}%
\pgfusepath{fill}%
\end{pgfscope}%
\begin{pgfscope}%
\pgfpathrectangle{\pgfqpoint{1.432000in}{0.528000in}}{\pgfqpoint{3.696000in}{3.696000in}} %
\pgfusepath{clip}%
\pgfsetbuttcap%
\pgfsetroundjoin%
\definecolor{currentfill}{rgb}{0.278012,0.180367,0.486697}%
\pgfsetfillcolor{currentfill}%
\pgfsetlinewidth{0.000000pt}%
\definecolor{currentstroke}{rgb}{0.000000,0.000000,0.000000}%
\pgfsetstrokecolor{currentstroke}%
\pgfsetdash{}{0pt}%
\pgfpathmoveto{\pgfqpoint{1.920726in}{2.213419in}}%
\pgfpathlineto{\pgfqpoint{1.920726in}{2.281890in}}%
\pgfpathlineto{\pgfqpoint{1.911856in}{2.277454in}}%
\pgfpathlineto{\pgfqpoint{1.925161in}{2.321806in}}%
\pgfpathlineto{\pgfqpoint{1.938467in}{2.277454in}}%
\pgfpathlineto{\pgfqpoint{1.929596in}{2.281890in}}%
\pgfpathlineto{\pgfqpoint{1.929596in}{2.213419in}}%
\pgfpathlineto{\pgfqpoint{1.920726in}{2.213419in}}%
\pgfusepath{fill}%
\end{pgfscope}%
\begin{pgfscope}%
\pgfpathrectangle{\pgfqpoint{1.432000in}{0.528000in}}{\pgfqpoint{3.696000in}{3.696000in}} %
\pgfusepath{clip}%
\pgfsetbuttcap%
\pgfsetroundjoin%
\definecolor{currentfill}{rgb}{0.159194,0.482237,0.558073}%
\pgfsetfillcolor{currentfill}%
\pgfsetlinewidth{0.000000pt}%
\definecolor{currentstroke}{rgb}{0.000000,0.000000,0.000000}%
\pgfsetstrokecolor{currentstroke}%
\pgfsetdash{}{0pt}%
\pgfpathmoveto{\pgfqpoint{2.030412in}{2.210283in}}%
\pgfpathlineto{\pgfqpoint{1.950251in}{2.290445in}}%
\pgfpathlineto{\pgfqpoint{1.947114in}{2.281036in}}%
\pgfpathlineto{\pgfqpoint{1.925161in}{2.321806in}}%
\pgfpathlineto{\pgfqpoint{1.965931in}{2.299853in}}%
\pgfpathlineto{\pgfqpoint{1.956523in}{2.296717in}}%
\pgfpathlineto{\pgfqpoint{2.036685in}{2.216556in}}%
\pgfpathlineto{\pgfqpoint{2.030412in}{2.210283in}}%
\pgfusepath{fill}%
\end{pgfscope}%
\begin{pgfscope}%
\pgfpathrectangle{\pgfqpoint{1.432000in}{0.528000in}}{\pgfqpoint{3.696000in}{3.696000in}} %
\pgfusepath{clip}%
\pgfsetbuttcap%
\pgfsetroundjoin%
\definecolor{currentfill}{rgb}{0.175841,0.441290,0.557685}%
\pgfsetfillcolor{currentfill}%
\pgfsetlinewidth{0.000000pt}%
\definecolor{currentstroke}{rgb}{0.000000,0.000000,0.000000}%
\pgfsetstrokecolor{currentstroke}%
\pgfsetdash{}{0pt}%
\pgfpathmoveto{\pgfqpoint{2.138799in}{2.210283in}}%
\pgfpathlineto{\pgfqpoint{2.058638in}{2.290445in}}%
\pgfpathlineto{\pgfqpoint{2.055502in}{2.281036in}}%
\pgfpathlineto{\pgfqpoint{2.033548in}{2.321806in}}%
\pgfpathlineto{\pgfqpoint{2.074318in}{2.299853in}}%
\pgfpathlineto{\pgfqpoint{2.064910in}{2.296717in}}%
\pgfpathlineto{\pgfqpoint{2.145072in}{2.216556in}}%
\pgfpathlineto{\pgfqpoint{2.138799in}{2.210283in}}%
\pgfusepath{fill}%
\end{pgfscope}%
\begin{pgfscope}%
\pgfpathrectangle{\pgfqpoint{1.432000in}{0.528000in}}{\pgfqpoint{3.696000in}{3.696000in}} %
\pgfusepath{clip}%
\pgfsetbuttcap%
\pgfsetroundjoin%
\definecolor{currentfill}{rgb}{0.268510,0.009605,0.335427}%
\pgfsetfillcolor{currentfill}%
\pgfsetlinewidth{0.000000pt}%
\definecolor{currentstroke}{rgb}{0.000000,0.000000,0.000000}%
\pgfsetstrokecolor{currentstroke}%
\pgfsetdash{}{0pt}%
\pgfpathmoveto{\pgfqpoint{2.248339in}{2.209452in}}%
\pgfpathlineto{\pgfqpoint{2.067268in}{2.299988in}}%
\pgfpathlineto{\pgfqpoint{2.067268in}{2.290071in}}%
\pgfpathlineto{\pgfqpoint{2.033548in}{2.321806in}}%
\pgfpathlineto{\pgfqpoint{2.079168in}{2.313873in}}%
\pgfpathlineto{\pgfqpoint{2.071235in}{2.307922in}}%
\pgfpathlineto{\pgfqpoint{2.252306in}{2.217386in}}%
\pgfpathlineto{\pgfqpoint{2.248339in}{2.209452in}}%
\pgfusepath{fill}%
\end{pgfscope}%
\begin{pgfscope}%
\pgfpathrectangle{\pgfqpoint{1.432000in}{0.528000in}}{\pgfqpoint{3.696000in}{3.696000in}} %
\pgfusepath{clip}%
\pgfsetbuttcap%
\pgfsetroundjoin%
\definecolor{currentfill}{rgb}{0.269308,0.218818,0.509577}%
\pgfsetfillcolor{currentfill}%
\pgfsetlinewidth{0.000000pt}%
\definecolor{currentstroke}{rgb}{0.000000,0.000000,0.000000}%
\pgfsetstrokecolor{currentstroke}%
\pgfsetdash{}{0pt}%
\pgfpathmoveto{\pgfqpoint{2.247186in}{2.210283in}}%
\pgfpathlineto{\pgfqpoint{2.167025in}{2.290445in}}%
\pgfpathlineto{\pgfqpoint{2.163889in}{2.281036in}}%
\pgfpathlineto{\pgfqpoint{2.141935in}{2.321806in}}%
\pgfpathlineto{\pgfqpoint{2.182706in}{2.299853in}}%
\pgfpathlineto{\pgfqpoint{2.173297in}{2.296717in}}%
\pgfpathlineto{\pgfqpoint{2.253459in}{2.216556in}}%
\pgfpathlineto{\pgfqpoint{2.247186in}{2.210283in}}%
\pgfusepath{fill}%
\end{pgfscope}%
\begin{pgfscope}%
\pgfpathrectangle{\pgfqpoint{1.432000in}{0.528000in}}{\pgfqpoint{3.696000in}{3.696000in}} %
\pgfusepath{clip}%
\pgfsetbuttcap%
\pgfsetroundjoin%
\definecolor{currentfill}{rgb}{0.267968,0.223549,0.512008}%
\pgfsetfillcolor{currentfill}%
\pgfsetlinewidth{0.000000pt}%
\definecolor{currentstroke}{rgb}{0.000000,0.000000,0.000000}%
\pgfsetstrokecolor{currentstroke}%
\pgfsetdash{}{0pt}%
\pgfpathmoveto{\pgfqpoint{2.356726in}{2.209452in}}%
\pgfpathlineto{\pgfqpoint{2.175655in}{2.299988in}}%
\pgfpathlineto{\pgfqpoint{2.175655in}{2.290071in}}%
\pgfpathlineto{\pgfqpoint{2.141935in}{2.321806in}}%
\pgfpathlineto{\pgfqpoint{2.187556in}{2.313873in}}%
\pgfpathlineto{\pgfqpoint{2.179622in}{2.307922in}}%
\pgfpathlineto{\pgfqpoint{2.360693in}{2.217386in}}%
\pgfpathlineto{\pgfqpoint{2.356726in}{2.209452in}}%
\pgfusepath{fill}%
\end{pgfscope}%
\begin{pgfscope}%
\pgfpathrectangle{\pgfqpoint{1.432000in}{0.528000in}}{\pgfqpoint{3.696000in}{3.696000in}} %
\pgfusepath{clip}%
\pgfsetbuttcap%
\pgfsetroundjoin%
\definecolor{currentfill}{rgb}{0.271305,0.019942,0.347269}%
\pgfsetfillcolor{currentfill}%
\pgfsetlinewidth{0.000000pt}%
\definecolor{currentstroke}{rgb}{0.000000,0.000000,0.000000}%
\pgfsetstrokecolor{currentstroke}%
\pgfsetdash{}{0pt}%
\pgfpathmoveto{\pgfqpoint{2.355574in}{2.210283in}}%
\pgfpathlineto{\pgfqpoint{2.275412in}{2.290445in}}%
\pgfpathlineto{\pgfqpoint{2.272276in}{2.281036in}}%
\pgfpathlineto{\pgfqpoint{2.250323in}{2.321806in}}%
\pgfpathlineto{\pgfqpoint{2.291093in}{2.299853in}}%
\pgfpathlineto{\pgfqpoint{2.281684in}{2.296717in}}%
\pgfpathlineto{\pgfqpoint{2.361846in}{2.216556in}}%
\pgfpathlineto{\pgfqpoint{2.355574in}{2.210283in}}%
\pgfusepath{fill}%
\end{pgfscope}%
\begin{pgfscope}%
\pgfpathrectangle{\pgfqpoint{1.432000in}{0.528000in}}{\pgfqpoint{3.696000in}{3.696000in}} %
\pgfusepath{clip}%
\pgfsetbuttcap%
\pgfsetroundjoin%
\definecolor{currentfill}{rgb}{0.274952,0.037752,0.364543}%
\pgfsetfillcolor{currentfill}%
\pgfsetlinewidth{0.000000pt}%
\definecolor{currentstroke}{rgb}{0.000000,0.000000,0.000000}%
\pgfsetstrokecolor{currentstroke}%
\pgfsetdash{}{0pt}%
\pgfpathmoveto{\pgfqpoint{2.467097in}{2.208984in}}%
\pgfpathlineto{\pgfqpoint{2.290239in}{2.208984in}}%
\pgfpathlineto{\pgfqpoint{2.294675in}{2.200114in}}%
\pgfpathlineto{\pgfqpoint{2.250323in}{2.213419in}}%
\pgfpathlineto{\pgfqpoint{2.294675in}{2.226725in}}%
\pgfpathlineto{\pgfqpoint{2.290239in}{2.217855in}}%
\pgfpathlineto{\pgfqpoint{2.467097in}{2.217855in}}%
\pgfpathlineto{\pgfqpoint{2.467097in}{2.208984in}}%
\pgfusepath{fill}%
\end{pgfscope}%
\begin{pgfscope}%
\pgfpathrectangle{\pgfqpoint{1.432000in}{0.528000in}}{\pgfqpoint{3.696000in}{3.696000in}} %
\pgfusepath{clip}%
\pgfsetbuttcap%
\pgfsetroundjoin%
\definecolor{currentfill}{rgb}{0.201239,0.383670,0.554294}%
\pgfsetfillcolor{currentfill}%
\pgfsetlinewidth{0.000000pt}%
\definecolor{currentstroke}{rgb}{0.000000,0.000000,0.000000}%
\pgfsetstrokecolor{currentstroke}%
\pgfsetdash{}{0pt}%
\pgfpathmoveto{\pgfqpoint{2.465113in}{2.209452in}}%
\pgfpathlineto{\pgfqpoint{2.284042in}{2.299988in}}%
\pgfpathlineto{\pgfqpoint{2.284042in}{2.290071in}}%
\pgfpathlineto{\pgfqpoint{2.250323in}{2.321806in}}%
\pgfpathlineto{\pgfqpoint{2.295943in}{2.313873in}}%
\pgfpathlineto{\pgfqpoint{2.288009in}{2.307922in}}%
\pgfpathlineto{\pgfqpoint{2.469080in}{2.217386in}}%
\pgfpathlineto{\pgfqpoint{2.465113in}{2.209452in}}%
\pgfusepath{fill}%
\end{pgfscope}%
\begin{pgfscope}%
\pgfpathrectangle{\pgfqpoint{1.432000in}{0.528000in}}{\pgfqpoint{3.696000in}{3.696000in}} %
\pgfusepath{clip}%
\pgfsetbuttcap%
\pgfsetroundjoin%
\definecolor{currentfill}{rgb}{0.282910,0.105393,0.426902}%
\pgfsetfillcolor{currentfill}%
\pgfsetlinewidth{0.000000pt}%
\definecolor{currentstroke}{rgb}{0.000000,0.000000,0.000000}%
\pgfsetstrokecolor{currentstroke}%
\pgfsetdash{}{0pt}%
\pgfpathmoveto{\pgfqpoint{2.575484in}{2.208984in}}%
\pgfpathlineto{\pgfqpoint{2.398626in}{2.208984in}}%
\pgfpathlineto{\pgfqpoint{2.403062in}{2.200114in}}%
\pgfpathlineto{\pgfqpoint{2.358710in}{2.213419in}}%
\pgfpathlineto{\pgfqpoint{2.403062in}{2.226725in}}%
\pgfpathlineto{\pgfqpoint{2.398626in}{2.217855in}}%
\pgfpathlineto{\pgfqpoint{2.575484in}{2.217855in}}%
\pgfpathlineto{\pgfqpoint{2.575484in}{2.208984in}}%
\pgfusepath{fill}%
\end{pgfscope}%
\begin{pgfscope}%
\pgfpathrectangle{\pgfqpoint{1.432000in}{0.528000in}}{\pgfqpoint{3.696000in}{3.696000in}} %
\pgfusepath{clip}%
\pgfsetbuttcap%
\pgfsetroundjoin%
\definecolor{currentfill}{rgb}{0.185556,0.418570,0.556753}%
\pgfsetfillcolor{currentfill}%
\pgfsetlinewidth{0.000000pt}%
\definecolor{currentstroke}{rgb}{0.000000,0.000000,0.000000}%
\pgfsetstrokecolor{currentstroke}%
\pgfsetdash{}{0pt}%
\pgfpathmoveto{\pgfqpoint{2.573500in}{2.209452in}}%
\pgfpathlineto{\pgfqpoint{2.392429in}{2.299988in}}%
\pgfpathlineto{\pgfqpoint{2.392429in}{2.290071in}}%
\pgfpathlineto{\pgfqpoint{2.358710in}{2.321806in}}%
\pgfpathlineto{\pgfqpoint{2.404330in}{2.313873in}}%
\pgfpathlineto{\pgfqpoint{2.396396in}{2.307922in}}%
\pgfpathlineto{\pgfqpoint{2.577467in}{2.217386in}}%
\pgfpathlineto{\pgfqpoint{2.573500in}{2.209452in}}%
\pgfusepath{fill}%
\end{pgfscope}%
\begin{pgfscope}%
\pgfpathrectangle{\pgfqpoint{1.432000in}{0.528000in}}{\pgfqpoint{3.696000in}{3.696000in}} %
\pgfusepath{clip}%
\pgfsetbuttcap%
\pgfsetroundjoin%
\definecolor{currentfill}{rgb}{0.146180,0.515413,0.556823}%
\pgfsetfillcolor{currentfill}%
\pgfsetlinewidth{0.000000pt}%
\definecolor{currentstroke}{rgb}{0.000000,0.000000,0.000000}%
\pgfsetstrokecolor{currentstroke}%
\pgfsetdash{}{0pt}%
\pgfpathmoveto{\pgfqpoint{2.681887in}{2.209452in}}%
\pgfpathlineto{\pgfqpoint{2.500816in}{2.299988in}}%
\pgfpathlineto{\pgfqpoint{2.500816in}{2.290071in}}%
\pgfpathlineto{\pgfqpoint{2.467097in}{2.321806in}}%
\pgfpathlineto{\pgfqpoint{2.512717in}{2.313873in}}%
\pgfpathlineto{\pgfqpoint{2.504783in}{2.307922in}}%
\pgfpathlineto{\pgfqpoint{2.685854in}{2.217386in}}%
\pgfpathlineto{\pgfqpoint{2.681887in}{2.209452in}}%
\pgfusepath{fill}%
\end{pgfscope}%
\begin{pgfscope}%
\pgfpathrectangle{\pgfqpoint{1.432000in}{0.528000in}}{\pgfqpoint{3.696000in}{3.696000in}} %
\pgfusepath{clip}%
\pgfsetbuttcap%
\pgfsetroundjoin%
\definecolor{currentfill}{rgb}{0.221989,0.339161,0.548752}%
\pgfsetfillcolor{currentfill}%
\pgfsetlinewidth{0.000000pt}%
\definecolor{currentstroke}{rgb}{0.000000,0.000000,0.000000}%
\pgfsetstrokecolor{currentstroke}%
\pgfsetdash{}{0pt}%
\pgfpathmoveto{\pgfqpoint{2.790275in}{2.209452in}}%
\pgfpathlineto{\pgfqpoint{2.609203in}{2.299988in}}%
\pgfpathlineto{\pgfqpoint{2.609203in}{2.290071in}}%
\pgfpathlineto{\pgfqpoint{2.575484in}{2.321806in}}%
\pgfpathlineto{\pgfqpoint{2.621104in}{2.313873in}}%
\pgfpathlineto{\pgfqpoint{2.613170in}{2.307922in}}%
\pgfpathlineto{\pgfqpoint{2.794242in}{2.217386in}}%
\pgfpathlineto{\pgfqpoint{2.790275in}{2.209452in}}%
\pgfusepath{fill}%
\end{pgfscope}%
\begin{pgfscope}%
\pgfpathrectangle{\pgfqpoint{1.432000in}{0.528000in}}{\pgfqpoint{3.696000in}{3.696000in}} %
\pgfusepath{clip}%
\pgfsetbuttcap%
\pgfsetroundjoin%
\definecolor{currentfill}{rgb}{0.273006,0.204520,0.501721}%
\pgfsetfillcolor{currentfill}%
\pgfsetlinewidth{0.000000pt}%
\definecolor{currentstroke}{rgb}{0.000000,0.000000,0.000000}%
\pgfsetstrokecolor{currentstroke}%
\pgfsetdash{}{0pt}%
\pgfpathmoveto{\pgfqpoint{2.898662in}{2.209452in}}%
\pgfpathlineto{\pgfqpoint{2.717590in}{2.299988in}}%
\pgfpathlineto{\pgfqpoint{2.717590in}{2.290071in}}%
\pgfpathlineto{\pgfqpoint{2.683871in}{2.321806in}}%
\pgfpathlineto{\pgfqpoint{2.729491in}{2.313873in}}%
\pgfpathlineto{\pgfqpoint{2.721557in}{2.307922in}}%
\pgfpathlineto{\pgfqpoint{2.902629in}{2.217386in}}%
\pgfpathlineto{\pgfqpoint{2.898662in}{2.209452in}}%
\pgfusepath{fill}%
\end{pgfscope}%
\begin{pgfscope}%
\pgfpathrectangle{\pgfqpoint{1.432000in}{0.528000in}}{\pgfqpoint{3.696000in}{3.696000in}} %
\pgfusepath{clip}%
\pgfsetbuttcap%
\pgfsetroundjoin%
\definecolor{currentfill}{rgb}{0.235526,0.309527,0.542944}%
\pgfsetfillcolor{currentfill}%
\pgfsetlinewidth{0.000000pt}%
\definecolor{currentstroke}{rgb}{0.000000,0.000000,0.000000}%
\pgfsetstrokecolor{currentstroke}%
\pgfsetdash{}{0pt}%
\pgfpathmoveto{\pgfqpoint{2.897509in}{2.210283in}}%
\pgfpathlineto{\pgfqpoint{2.708960in}{2.398832in}}%
\pgfpathlineto{\pgfqpoint{2.705824in}{2.389423in}}%
\pgfpathlineto{\pgfqpoint{2.683871in}{2.430194in}}%
\pgfpathlineto{\pgfqpoint{2.724641in}{2.408240in}}%
\pgfpathlineto{\pgfqpoint{2.715233in}{2.405104in}}%
\pgfpathlineto{\pgfqpoint{2.903781in}{2.216556in}}%
\pgfpathlineto{\pgfqpoint{2.897509in}{2.210283in}}%
\pgfusepath{fill}%
\end{pgfscope}%
\begin{pgfscope}%
\pgfpathrectangle{\pgfqpoint{1.432000in}{0.528000in}}{\pgfqpoint{3.696000in}{3.696000in}} %
\pgfusepath{clip}%
\pgfsetbuttcap%
\pgfsetroundjoin%
\definecolor{currentfill}{rgb}{0.122312,0.633153,0.530398}%
\pgfsetfillcolor{currentfill}%
\pgfsetlinewidth{0.000000pt}%
\definecolor{currentstroke}{rgb}{0.000000,0.000000,0.000000}%
\pgfsetstrokecolor{currentstroke}%
\pgfsetdash{}{0pt}%
\pgfpathmoveto{\pgfqpoint{3.005896in}{2.210283in}}%
\pgfpathlineto{\pgfqpoint{2.817347in}{2.398832in}}%
\pgfpathlineto{\pgfqpoint{2.814211in}{2.389423in}}%
\pgfpathlineto{\pgfqpoint{2.792258in}{2.430194in}}%
\pgfpathlineto{\pgfqpoint{2.833028in}{2.408240in}}%
\pgfpathlineto{\pgfqpoint{2.823620in}{2.405104in}}%
\pgfpathlineto{\pgfqpoint{3.012168in}{2.216556in}}%
\pgfpathlineto{\pgfqpoint{3.005896in}{2.210283in}}%
\pgfusepath{fill}%
\end{pgfscope}%
\begin{pgfscope}%
\pgfpathrectangle{\pgfqpoint{1.432000in}{0.528000in}}{\pgfqpoint{3.696000in}{3.696000in}} %
\pgfusepath{clip}%
\pgfsetbuttcap%
\pgfsetroundjoin%
\definecolor{currentfill}{rgb}{0.121380,0.629492,0.531973}%
\pgfsetfillcolor{currentfill}%
\pgfsetlinewidth{0.000000pt}%
\definecolor{currentstroke}{rgb}{0.000000,0.000000,0.000000}%
\pgfsetstrokecolor{currentstroke}%
\pgfsetdash{}{0pt}%
\pgfpathmoveto{\pgfqpoint{3.114283in}{2.210283in}}%
\pgfpathlineto{\pgfqpoint{2.925734in}{2.398832in}}%
\pgfpathlineto{\pgfqpoint{2.922598in}{2.389423in}}%
\pgfpathlineto{\pgfqpoint{2.900645in}{2.430194in}}%
\pgfpathlineto{\pgfqpoint{2.941415in}{2.408240in}}%
\pgfpathlineto{\pgfqpoint{2.932007in}{2.405104in}}%
\pgfpathlineto{\pgfqpoint{3.120556in}{2.216556in}}%
\pgfpathlineto{\pgfqpoint{3.114283in}{2.210283in}}%
\pgfusepath{fill}%
\end{pgfscope}%
\begin{pgfscope}%
\pgfpathrectangle{\pgfqpoint{1.432000in}{0.528000in}}{\pgfqpoint{3.696000in}{3.696000in}} %
\pgfusepath{clip}%
\pgfsetbuttcap%
\pgfsetroundjoin%
\definecolor{currentfill}{rgb}{0.188923,0.410910,0.556326}%
\pgfsetfillcolor{currentfill}%
\pgfsetlinewidth{0.000000pt}%
\definecolor{currentstroke}{rgb}{0.000000,0.000000,0.000000}%
\pgfsetstrokecolor{currentstroke}%
\pgfsetdash{}{0pt}%
\pgfpathmoveto{\pgfqpoint{3.222670in}{2.210283in}}%
\pgfpathlineto{\pgfqpoint{3.034122in}{2.398832in}}%
\pgfpathlineto{\pgfqpoint{3.030985in}{2.389423in}}%
\pgfpathlineto{\pgfqpoint{3.009032in}{2.430194in}}%
\pgfpathlineto{\pgfqpoint{3.049802in}{2.408240in}}%
\pgfpathlineto{\pgfqpoint{3.040394in}{2.405104in}}%
\pgfpathlineto{\pgfqpoint{3.228943in}{2.216556in}}%
\pgfpathlineto{\pgfqpoint{3.222670in}{2.210283in}}%
\pgfusepath{fill}%
\end{pgfscope}%
\begin{pgfscope}%
\pgfpathrectangle{\pgfqpoint{1.432000in}{0.528000in}}{\pgfqpoint{3.696000in}{3.696000in}} %
\pgfusepath{clip}%
\pgfsetbuttcap%
\pgfsetroundjoin%
\definecolor{currentfill}{rgb}{0.237441,0.305202,0.541921}%
\pgfsetfillcolor{currentfill}%
\pgfsetlinewidth{0.000000pt}%
\definecolor{currentstroke}{rgb}{0.000000,0.000000,0.000000}%
\pgfsetstrokecolor{currentstroke}%
\pgfsetdash{}{0pt}%
\pgfpathmoveto{\pgfqpoint{3.331057in}{2.210283in}}%
\pgfpathlineto{\pgfqpoint{3.142509in}{2.398832in}}%
\pgfpathlineto{\pgfqpoint{3.139372in}{2.389423in}}%
\pgfpathlineto{\pgfqpoint{3.117419in}{2.430194in}}%
\pgfpathlineto{\pgfqpoint{3.158189in}{2.408240in}}%
\pgfpathlineto{\pgfqpoint{3.148781in}{2.405104in}}%
\pgfpathlineto{\pgfqpoint{3.337330in}{2.216556in}}%
\pgfpathlineto{\pgfqpoint{3.331057in}{2.210283in}}%
\pgfusepath{fill}%
\end{pgfscope}%
\begin{pgfscope}%
\pgfpathrectangle{\pgfqpoint{1.432000in}{0.528000in}}{\pgfqpoint{3.696000in}{3.696000in}} %
\pgfusepath{clip}%
\pgfsetbuttcap%
\pgfsetroundjoin%
\definecolor{currentfill}{rgb}{0.257322,0.256130,0.526563}%
\pgfsetfillcolor{currentfill}%
\pgfsetlinewidth{0.000000pt}%
\definecolor{currentstroke}{rgb}{0.000000,0.000000,0.000000}%
\pgfsetstrokecolor{currentstroke}%
\pgfsetdash{}{0pt}%
\pgfpathmoveto{\pgfqpoint{3.330503in}{2.210959in}}%
\pgfpathlineto{\pgfqpoint{3.135871in}{2.502908in}}%
\pgfpathlineto{\pgfqpoint{3.130950in}{2.494297in}}%
\pgfpathlineto{\pgfqpoint{3.117419in}{2.538581in}}%
\pgfpathlineto{\pgfqpoint{3.153092in}{2.509058in}}%
\pgfpathlineto{\pgfqpoint{3.143252in}{2.507828in}}%
\pgfpathlineto{\pgfqpoint{3.337884in}{2.215880in}}%
\pgfpathlineto{\pgfqpoint{3.330503in}{2.210959in}}%
\pgfusepath{fill}%
\end{pgfscope}%
\begin{pgfscope}%
\pgfpathrectangle{\pgfqpoint{1.432000in}{0.528000in}}{\pgfqpoint{3.696000in}{3.696000in}} %
\pgfusepath{clip}%
\pgfsetbuttcap%
\pgfsetroundjoin%
\definecolor{currentfill}{rgb}{0.192357,0.403199,0.555836}%
\pgfsetfillcolor{currentfill}%
\pgfsetlinewidth{0.000000pt}%
\definecolor{currentstroke}{rgb}{0.000000,0.000000,0.000000}%
\pgfsetstrokecolor{currentstroke}%
\pgfsetdash{}{0pt}%
\pgfpathmoveto{\pgfqpoint{3.439444in}{2.210283in}}%
\pgfpathlineto{\pgfqpoint{3.250896in}{2.398832in}}%
\pgfpathlineto{\pgfqpoint{3.247760in}{2.389423in}}%
\pgfpathlineto{\pgfqpoint{3.225806in}{2.430194in}}%
\pgfpathlineto{\pgfqpoint{3.266577in}{2.408240in}}%
\pgfpathlineto{\pgfqpoint{3.257168in}{2.405104in}}%
\pgfpathlineto{\pgfqpoint{3.445717in}{2.216556in}}%
\pgfpathlineto{\pgfqpoint{3.439444in}{2.210283in}}%
\pgfusepath{fill}%
\end{pgfscope}%
\begin{pgfscope}%
\pgfpathrectangle{\pgfqpoint{1.432000in}{0.528000in}}{\pgfqpoint{3.696000in}{3.696000in}} %
\pgfusepath{clip}%
\pgfsetbuttcap%
\pgfsetroundjoin%
\definecolor{currentfill}{rgb}{0.265145,0.232956,0.516599}%
\pgfsetfillcolor{currentfill}%
\pgfsetlinewidth{0.000000pt}%
\definecolor{currentstroke}{rgb}{0.000000,0.000000,0.000000}%
\pgfsetstrokecolor{currentstroke}%
\pgfsetdash{}{0pt}%
\pgfpathmoveto{\pgfqpoint{3.548508in}{2.209729in}}%
\pgfpathlineto{\pgfqpoint{3.256559in}{2.404361in}}%
\pgfpathlineto{\pgfqpoint{3.255329in}{2.394521in}}%
\pgfpathlineto{\pgfqpoint{3.225806in}{2.430194in}}%
\pgfpathlineto{\pgfqpoint{3.270090in}{2.416662in}}%
\pgfpathlineto{\pgfqpoint{3.261479in}{2.411742in}}%
\pgfpathlineto{\pgfqpoint{3.553428in}{2.217110in}}%
\pgfpathlineto{\pgfqpoint{3.548508in}{2.209729in}}%
\pgfusepath{fill}%
\end{pgfscope}%
\begin{pgfscope}%
\pgfpathrectangle{\pgfqpoint{1.432000in}{0.528000in}}{\pgfqpoint{3.696000in}{3.696000in}} %
\pgfusepath{clip}%
\pgfsetbuttcap%
\pgfsetroundjoin%
\definecolor{currentfill}{rgb}{0.282290,0.145912,0.461510}%
\pgfsetfillcolor{currentfill}%
\pgfsetlinewidth{0.000000pt}%
\definecolor{currentstroke}{rgb}{0.000000,0.000000,0.000000}%
\pgfsetstrokecolor{currentstroke}%
\pgfsetdash{}{0pt}%
\pgfpathmoveto{\pgfqpoint{3.547832in}{2.210283in}}%
\pgfpathlineto{\pgfqpoint{3.359283in}{2.398832in}}%
\pgfpathlineto{\pgfqpoint{3.356147in}{2.389423in}}%
\pgfpathlineto{\pgfqpoint{3.334194in}{2.430194in}}%
\pgfpathlineto{\pgfqpoint{3.374964in}{2.408240in}}%
\pgfpathlineto{\pgfqpoint{3.365555in}{2.405104in}}%
\pgfpathlineto{\pgfqpoint{3.554104in}{2.216556in}}%
\pgfpathlineto{\pgfqpoint{3.547832in}{2.210283in}}%
\pgfusepath{fill}%
\end{pgfscope}%
\begin{pgfscope}%
\pgfpathrectangle{\pgfqpoint{1.432000in}{0.528000in}}{\pgfqpoint{3.696000in}{3.696000in}} %
\pgfusepath{clip}%
\pgfsetbuttcap%
\pgfsetroundjoin%
\definecolor{currentfill}{rgb}{0.163625,0.471133,0.558148}%
\pgfsetfillcolor{currentfill}%
\pgfsetlinewidth{0.000000pt}%
\definecolor{currentstroke}{rgb}{0.000000,0.000000,0.000000}%
\pgfsetstrokecolor{currentstroke}%
\pgfsetdash{}{0pt}%
\pgfpathmoveto{\pgfqpoint{3.656895in}{2.209729in}}%
\pgfpathlineto{\pgfqpoint{3.364946in}{2.404361in}}%
\pgfpathlineto{\pgfqpoint{3.363716in}{2.394521in}}%
\pgfpathlineto{\pgfqpoint{3.334194in}{2.430194in}}%
\pgfpathlineto{\pgfqpoint{3.378477in}{2.416662in}}%
\pgfpathlineto{\pgfqpoint{3.369867in}{2.411742in}}%
\pgfpathlineto{\pgfqpoint{3.661815in}{2.217110in}}%
\pgfpathlineto{\pgfqpoint{3.656895in}{2.209729in}}%
\pgfusepath{fill}%
\end{pgfscope}%
\begin{pgfscope}%
\pgfpathrectangle{\pgfqpoint{1.432000in}{0.528000in}}{\pgfqpoint{3.696000in}{3.696000in}} %
\pgfusepath{clip}%
\pgfsetbuttcap%
\pgfsetroundjoin%
\definecolor{currentfill}{rgb}{0.377779,0.791781,0.377939}%
\pgfsetfillcolor{currentfill}%
\pgfsetlinewidth{0.000000pt}%
\definecolor{currentstroke}{rgb}{0.000000,0.000000,0.000000}%
\pgfsetstrokecolor{currentstroke}%
\pgfsetdash{}{0pt}%
\pgfpathmoveto{\pgfqpoint{3.765282in}{2.209729in}}%
\pgfpathlineto{\pgfqpoint{3.473333in}{2.404361in}}%
\pgfpathlineto{\pgfqpoint{3.472103in}{2.394521in}}%
\pgfpathlineto{\pgfqpoint{3.442581in}{2.430194in}}%
\pgfpathlineto{\pgfqpoint{3.486864in}{2.416662in}}%
\pgfpathlineto{\pgfqpoint{3.478254in}{2.411742in}}%
\pgfpathlineto{\pgfqpoint{3.770202in}{2.217110in}}%
\pgfpathlineto{\pgfqpoint{3.765282in}{2.209729in}}%
\pgfusepath{fill}%
\end{pgfscope}%
\begin{pgfscope}%
\pgfpathrectangle{\pgfqpoint{1.432000in}{0.528000in}}{\pgfqpoint{3.696000in}{3.696000in}} %
\pgfusepath{clip}%
\pgfsetbuttcap%
\pgfsetroundjoin%
\definecolor{currentfill}{rgb}{0.252899,0.742211,0.448284}%
\pgfsetfillcolor{currentfill}%
\pgfsetlinewidth{0.000000pt}%
\definecolor{currentstroke}{rgb}{0.000000,0.000000,0.000000}%
\pgfsetstrokecolor{currentstroke}%
\pgfsetdash{}{0pt}%
\pgfpathmoveto{\pgfqpoint{3.873669in}{2.209729in}}%
\pgfpathlineto{\pgfqpoint{3.581720in}{2.404361in}}%
\pgfpathlineto{\pgfqpoint{3.580490in}{2.394521in}}%
\pgfpathlineto{\pgfqpoint{3.550968in}{2.430194in}}%
\pgfpathlineto{\pgfqpoint{3.595251in}{2.416662in}}%
\pgfpathlineto{\pgfqpoint{3.586641in}{2.411742in}}%
\pgfpathlineto{\pgfqpoint{3.878589in}{2.217110in}}%
\pgfpathlineto{\pgfqpoint{3.873669in}{2.209729in}}%
\pgfusepath{fill}%
\end{pgfscope}%
\begin{pgfscope}%
\pgfpathrectangle{\pgfqpoint{1.432000in}{0.528000in}}{\pgfqpoint{3.696000in}{3.696000in}} %
\pgfusepath{clip}%
\pgfsetbuttcap%
\pgfsetroundjoin%
\definecolor{currentfill}{rgb}{0.266580,0.228262,0.514349}%
\pgfsetfillcolor{currentfill}%
\pgfsetlinewidth{0.000000pt}%
\definecolor{currentstroke}{rgb}{0.000000,0.000000,0.000000}%
\pgfsetstrokecolor{currentstroke}%
\pgfsetdash{}{0pt}%
\pgfpathmoveto{\pgfqpoint{3.982533in}{2.209452in}}%
\pgfpathlineto{\pgfqpoint{3.584687in}{2.408375in}}%
\pgfpathlineto{\pgfqpoint{3.584687in}{2.398458in}}%
\pgfpathlineto{\pgfqpoint{3.550968in}{2.430194in}}%
\pgfpathlineto{\pgfqpoint{3.596588in}{2.422260in}}%
\pgfpathlineto{\pgfqpoint{3.588654in}{2.416309in}}%
\pgfpathlineto{\pgfqpoint{3.986500in}{2.217386in}}%
\pgfpathlineto{\pgfqpoint{3.982533in}{2.209452in}}%
\pgfusepath{fill}%
\end{pgfscope}%
\begin{pgfscope}%
\pgfpathrectangle{\pgfqpoint{1.432000in}{0.528000in}}{\pgfqpoint{3.696000in}{3.696000in}} %
\pgfusepath{clip}%
\pgfsetbuttcap%
\pgfsetroundjoin%
\definecolor{currentfill}{rgb}{0.119512,0.607464,0.540218}%
\pgfsetfillcolor{currentfill}%
\pgfsetlinewidth{0.000000pt}%
\definecolor{currentstroke}{rgb}{0.000000,0.000000,0.000000}%
\pgfsetstrokecolor{currentstroke}%
\pgfsetdash{}{0pt}%
\pgfpathmoveto{\pgfqpoint{3.982056in}{2.209729in}}%
\pgfpathlineto{\pgfqpoint{3.690107in}{2.404361in}}%
\pgfpathlineto{\pgfqpoint{3.688877in}{2.394521in}}%
\pgfpathlineto{\pgfqpoint{3.659355in}{2.430194in}}%
\pgfpathlineto{\pgfqpoint{3.703639in}{2.416662in}}%
\pgfpathlineto{\pgfqpoint{3.695028in}{2.411742in}}%
\pgfpathlineto{\pgfqpoint{3.986976in}{2.217110in}}%
\pgfpathlineto{\pgfqpoint{3.982056in}{2.209729in}}%
\pgfusepath{fill}%
\end{pgfscope}%
\begin{pgfscope}%
\pgfpathrectangle{\pgfqpoint{1.432000in}{0.528000in}}{\pgfqpoint{3.696000in}{3.696000in}} %
\pgfusepath{clip}%
\pgfsetbuttcap%
\pgfsetroundjoin%
\definecolor{currentfill}{rgb}{0.149039,0.508051,0.557250}%
\pgfsetfillcolor{currentfill}%
\pgfsetlinewidth{0.000000pt}%
\definecolor{currentstroke}{rgb}{0.000000,0.000000,0.000000}%
\pgfsetstrokecolor{currentstroke}%
\pgfsetdash{}{0pt}%
\pgfpathmoveto{\pgfqpoint{4.090920in}{2.209452in}}%
\pgfpathlineto{\pgfqpoint{3.693074in}{2.408375in}}%
\pgfpathlineto{\pgfqpoint{3.693074in}{2.398458in}}%
\pgfpathlineto{\pgfqpoint{3.659355in}{2.430194in}}%
\pgfpathlineto{\pgfqpoint{3.704975in}{2.422260in}}%
\pgfpathlineto{\pgfqpoint{3.697041in}{2.416309in}}%
\pgfpathlineto{\pgfqpoint{4.094887in}{2.217386in}}%
\pgfpathlineto{\pgfqpoint{4.090920in}{2.209452in}}%
\pgfusepath{fill}%
\end{pgfscope}%
\begin{pgfscope}%
\pgfpathrectangle{\pgfqpoint{1.432000in}{0.528000in}}{\pgfqpoint{3.696000in}{3.696000in}} %
\pgfusepath{clip}%
\pgfsetbuttcap%
\pgfsetroundjoin%
\definecolor{currentfill}{rgb}{0.183898,0.422383,0.556944}%
\pgfsetfillcolor{currentfill}%
\pgfsetlinewidth{0.000000pt}%
\definecolor{currentstroke}{rgb}{0.000000,0.000000,0.000000}%
\pgfsetstrokecolor{currentstroke}%
\pgfsetdash{}{0pt}%
\pgfpathmoveto{\pgfqpoint{4.090443in}{2.209729in}}%
\pgfpathlineto{\pgfqpoint{3.798495in}{2.404361in}}%
\pgfpathlineto{\pgfqpoint{3.797264in}{2.394521in}}%
\pgfpathlineto{\pgfqpoint{3.767742in}{2.430194in}}%
\pgfpathlineto{\pgfqpoint{3.812026in}{2.416662in}}%
\pgfpathlineto{\pgfqpoint{3.803415in}{2.411742in}}%
\pgfpathlineto{\pgfqpoint{4.095363in}{2.217110in}}%
\pgfpathlineto{\pgfqpoint{4.090443in}{2.209729in}}%
\pgfusepath{fill}%
\end{pgfscope}%
\begin{pgfscope}%
\pgfpathrectangle{\pgfqpoint{1.432000in}{0.528000in}}{\pgfqpoint{3.696000in}{3.696000in}} %
\pgfusepath{clip}%
\pgfsetbuttcap%
\pgfsetroundjoin%
\definecolor{currentfill}{rgb}{0.121148,0.592739,0.544641}%
\pgfsetfillcolor{currentfill}%
\pgfsetlinewidth{0.000000pt}%
\definecolor{currentstroke}{rgb}{0.000000,0.000000,0.000000}%
\pgfsetstrokecolor{currentstroke}%
\pgfsetdash{}{0pt}%
\pgfpathmoveto{\pgfqpoint{4.199307in}{2.209452in}}%
\pgfpathlineto{\pgfqpoint{3.801461in}{2.408375in}}%
\pgfpathlineto{\pgfqpoint{3.801461in}{2.398458in}}%
\pgfpathlineto{\pgfqpoint{3.767742in}{2.430194in}}%
\pgfpathlineto{\pgfqpoint{3.813362in}{2.422260in}}%
\pgfpathlineto{\pgfqpoint{3.805428in}{2.416309in}}%
\pgfpathlineto{\pgfqpoint{4.203274in}{2.217386in}}%
\pgfpathlineto{\pgfqpoint{4.199307in}{2.209452in}}%
\pgfusepath{fill}%
\end{pgfscope}%
\begin{pgfscope}%
\pgfpathrectangle{\pgfqpoint{1.432000in}{0.528000in}}{\pgfqpoint{3.696000in}{3.696000in}} %
\pgfusepath{clip}%
\pgfsetbuttcap%
\pgfsetroundjoin%
\definecolor{currentfill}{rgb}{0.216210,0.351535,0.550627}%
\pgfsetfillcolor{currentfill}%
\pgfsetlinewidth{0.000000pt}%
\definecolor{currentstroke}{rgb}{0.000000,0.000000,0.000000}%
\pgfsetstrokecolor{currentstroke}%
\pgfsetdash{}{0pt}%
\pgfpathmoveto{\pgfqpoint{4.198830in}{2.209729in}}%
\pgfpathlineto{\pgfqpoint{3.906882in}{2.404361in}}%
\pgfpathlineto{\pgfqpoint{3.905652in}{2.394521in}}%
\pgfpathlineto{\pgfqpoint{3.876129in}{2.430194in}}%
\pgfpathlineto{\pgfqpoint{3.920413in}{2.416662in}}%
\pgfpathlineto{\pgfqpoint{3.911802in}{2.411742in}}%
\pgfpathlineto{\pgfqpoint{4.203751in}{2.217110in}}%
\pgfpathlineto{\pgfqpoint{4.198830in}{2.209729in}}%
\pgfusepath{fill}%
\end{pgfscope}%
\begin{pgfscope}%
\pgfpathrectangle{\pgfqpoint{1.432000in}{0.528000in}}{\pgfqpoint{3.696000in}{3.696000in}} %
\pgfusepath{clip}%
\pgfsetbuttcap%
\pgfsetroundjoin%
\definecolor{currentfill}{rgb}{0.190631,0.407061,0.556089}%
\pgfsetfillcolor{currentfill}%
\pgfsetlinewidth{0.000000pt}%
\definecolor{currentstroke}{rgb}{0.000000,0.000000,0.000000}%
\pgfsetstrokecolor{currentstroke}%
\pgfsetdash{}{0pt}%
\pgfpathmoveto{\pgfqpoint{4.307694in}{2.209452in}}%
\pgfpathlineto{\pgfqpoint{3.909848in}{2.408375in}}%
\pgfpathlineto{\pgfqpoint{3.909848in}{2.398458in}}%
\pgfpathlineto{\pgfqpoint{3.876129in}{2.430194in}}%
\pgfpathlineto{\pgfqpoint{3.921749in}{2.422260in}}%
\pgfpathlineto{\pgfqpoint{3.913815in}{2.416309in}}%
\pgfpathlineto{\pgfqpoint{4.311661in}{2.217386in}}%
\pgfpathlineto{\pgfqpoint{4.307694in}{2.209452in}}%
\pgfusepath{fill}%
\end{pgfscope}%
\begin{pgfscope}%
\pgfpathrectangle{\pgfqpoint{1.432000in}{0.528000in}}{\pgfqpoint{3.696000in}{3.696000in}} %
\pgfusepath{clip}%
\pgfsetbuttcap%
\pgfsetroundjoin%
\definecolor{currentfill}{rgb}{0.132444,0.552216,0.553018}%
\pgfsetfillcolor{currentfill}%
\pgfsetlinewidth{0.000000pt}%
\definecolor{currentstroke}{rgb}{0.000000,0.000000,0.000000}%
\pgfsetstrokecolor{currentstroke}%
\pgfsetdash{}{0pt}%
\pgfpathmoveto{\pgfqpoint{4.307217in}{2.209729in}}%
\pgfpathlineto{\pgfqpoint{4.015269in}{2.404361in}}%
\pgfpathlineto{\pgfqpoint{4.014039in}{2.394521in}}%
\pgfpathlineto{\pgfqpoint{3.984516in}{2.430194in}}%
\pgfpathlineto{\pgfqpoint{4.028800in}{2.416662in}}%
\pgfpathlineto{\pgfqpoint{4.020189in}{2.411742in}}%
\pgfpathlineto{\pgfqpoint{4.312138in}{2.217110in}}%
\pgfpathlineto{\pgfqpoint{4.307217in}{2.209729in}}%
\pgfusepath{fill}%
\end{pgfscope}%
\begin{pgfscope}%
\pgfpathrectangle{\pgfqpoint{1.432000in}{0.528000in}}{\pgfqpoint{3.696000in}{3.696000in}} %
\pgfusepath{clip}%
\pgfsetbuttcap%
\pgfsetroundjoin%
\definecolor{currentfill}{rgb}{0.125394,0.574318,0.549086}%
\pgfsetfillcolor{currentfill}%
\pgfsetlinewidth{0.000000pt}%
\definecolor{currentstroke}{rgb}{0.000000,0.000000,0.000000}%
\pgfsetstrokecolor{currentstroke}%
\pgfsetdash{}{0pt}%
\pgfpathmoveto{\pgfqpoint{4.415604in}{2.209729in}}%
\pgfpathlineto{\pgfqpoint{4.123656in}{2.404361in}}%
\pgfpathlineto{\pgfqpoint{4.122426in}{2.394521in}}%
\pgfpathlineto{\pgfqpoint{4.092903in}{2.430194in}}%
\pgfpathlineto{\pgfqpoint{4.137187in}{2.416662in}}%
\pgfpathlineto{\pgfqpoint{4.128576in}{2.411742in}}%
\pgfpathlineto{\pgfqpoint{4.420525in}{2.217110in}}%
\pgfpathlineto{\pgfqpoint{4.415604in}{2.209729in}}%
\pgfusepath{fill}%
\end{pgfscope}%
\begin{pgfscope}%
\pgfpathrectangle{\pgfqpoint{1.432000in}{0.528000in}}{\pgfqpoint{3.696000in}{3.696000in}} %
\pgfusepath{clip}%
\pgfsetbuttcap%
\pgfsetroundjoin%
\definecolor{currentfill}{rgb}{0.218130,0.347432,0.550038}%
\pgfsetfillcolor{currentfill}%
\pgfsetlinewidth{0.000000pt}%
\definecolor{currentstroke}{rgb}{0.000000,0.000000,0.000000}%
\pgfsetstrokecolor{currentstroke}%
\pgfsetdash{}{0pt}%
\pgfpathmoveto{\pgfqpoint{4.414928in}{2.210283in}}%
\pgfpathlineto{\pgfqpoint{4.226380in}{2.398832in}}%
\pgfpathlineto{\pgfqpoint{4.223243in}{2.389423in}}%
\pgfpathlineto{\pgfqpoint{4.201290in}{2.430194in}}%
\pgfpathlineto{\pgfqpoint{4.242060in}{2.408240in}}%
\pgfpathlineto{\pgfqpoint{4.232652in}{2.405104in}}%
\pgfpathlineto{\pgfqpoint{4.421201in}{2.216556in}}%
\pgfpathlineto{\pgfqpoint{4.414928in}{2.210283in}}%
\pgfusepath{fill}%
\end{pgfscope}%
\begin{pgfscope}%
\pgfpathrectangle{\pgfqpoint{1.432000in}{0.528000in}}{\pgfqpoint{3.696000in}{3.696000in}} %
\pgfusepath{clip}%
\pgfsetbuttcap%
\pgfsetroundjoin%
\definecolor{currentfill}{rgb}{0.276022,0.044167,0.370164}%
\pgfsetfillcolor{currentfill}%
\pgfsetlinewidth{0.000000pt}%
\definecolor{currentstroke}{rgb}{0.000000,0.000000,0.000000}%
\pgfsetstrokecolor{currentstroke}%
\pgfsetdash{}{0pt}%
\pgfpathmoveto{\pgfqpoint{4.414928in}{2.210283in}}%
\pgfpathlineto{\pgfqpoint{4.117993in}{2.507219in}}%
\pgfpathlineto{\pgfqpoint{4.114856in}{2.497811in}}%
\pgfpathlineto{\pgfqpoint{4.092903in}{2.538581in}}%
\pgfpathlineto{\pgfqpoint{4.133673in}{2.516628in}}%
\pgfpathlineto{\pgfqpoint{4.124265in}{2.513491in}}%
\pgfpathlineto{\pgfqpoint{4.421201in}{2.216556in}}%
\pgfpathlineto{\pgfqpoint{4.414928in}{2.210283in}}%
\pgfusepath{fill}%
\end{pgfscope}%
\begin{pgfscope}%
\pgfpathrectangle{\pgfqpoint{1.432000in}{0.528000in}}{\pgfqpoint{3.696000in}{3.696000in}} %
\pgfusepath{clip}%
\pgfsetbuttcap%
\pgfsetroundjoin%
\definecolor{currentfill}{rgb}{0.277134,0.185228,0.489898}%
\pgfsetfillcolor{currentfill}%
\pgfsetlinewidth{0.000000pt}%
\definecolor{currentstroke}{rgb}{0.000000,0.000000,0.000000}%
\pgfsetstrokecolor{currentstroke}%
\pgfsetdash{}{0pt}%
\pgfpathmoveto{\pgfqpoint{4.523991in}{2.209729in}}%
\pgfpathlineto{\pgfqpoint{4.232043in}{2.404361in}}%
\pgfpathlineto{\pgfqpoint{4.230813in}{2.394521in}}%
\pgfpathlineto{\pgfqpoint{4.201290in}{2.430194in}}%
\pgfpathlineto{\pgfqpoint{4.245574in}{2.416662in}}%
\pgfpathlineto{\pgfqpoint{4.236963in}{2.411742in}}%
\pgfpathlineto{\pgfqpoint{4.528912in}{2.217110in}}%
\pgfpathlineto{\pgfqpoint{4.523991in}{2.209729in}}%
\pgfusepath{fill}%
\end{pgfscope}%
\begin{pgfscope}%
\pgfpathrectangle{\pgfqpoint{1.432000in}{0.528000in}}{\pgfqpoint{3.696000in}{3.696000in}} %
\pgfusepath{clip}%
\pgfsetbuttcap%
\pgfsetroundjoin%
\definecolor{currentfill}{rgb}{0.137339,0.662252,0.515571}%
\pgfsetfillcolor{currentfill}%
\pgfsetlinewidth{0.000000pt}%
\definecolor{currentstroke}{rgb}{0.000000,0.000000,0.000000}%
\pgfsetstrokecolor{currentstroke}%
\pgfsetdash{}{0pt}%
\pgfpathmoveto{\pgfqpoint{4.523315in}{2.210283in}}%
\pgfpathlineto{\pgfqpoint{4.334767in}{2.398832in}}%
\pgfpathlineto{\pgfqpoint{4.331631in}{2.389423in}}%
\pgfpathlineto{\pgfqpoint{4.309677in}{2.430194in}}%
\pgfpathlineto{\pgfqpoint{4.350447in}{2.408240in}}%
\pgfpathlineto{\pgfqpoint{4.341039in}{2.405104in}}%
\pgfpathlineto{\pgfqpoint{4.529588in}{2.216556in}}%
\pgfpathlineto{\pgfqpoint{4.523315in}{2.210283in}}%
\pgfusepath{fill}%
\end{pgfscope}%
\begin{pgfscope}%
\pgfpathrectangle{\pgfqpoint{1.432000in}{0.528000in}}{\pgfqpoint{3.696000in}{3.696000in}} %
\pgfusepath{clip}%
\pgfsetbuttcap%
\pgfsetroundjoin%
\definecolor{currentfill}{rgb}{0.196571,0.711827,0.479221}%
\pgfsetfillcolor{currentfill}%
\pgfsetlinewidth{0.000000pt}%
\definecolor{currentstroke}{rgb}{0.000000,0.000000,0.000000}%
\pgfsetstrokecolor{currentstroke}%
\pgfsetdash{}{0pt}%
\pgfpathmoveto{\pgfqpoint{4.631703in}{2.210283in}}%
\pgfpathlineto{\pgfqpoint{4.443154in}{2.398832in}}%
\pgfpathlineto{\pgfqpoint{4.440018in}{2.389423in}}%
\pgfpathlineto{\pgfqpoint{4.418065in}{2.430194in}}%
\pgfpathlineto{\pgfqpoint{4.458835in}{2.408240in}}%
\pgfpathlineto{\pgfqpoint{4.449426in}{2.405104in}}%
\pgfpathlineto{\pgfqpoint{4.637975in}{2.216556in}}%
\pgfpathlineto{\pgfqpoint{4.631703in}{2.210283in}}%
\pgfusepath{fill}%
\end{pgfscope}%
\begin{pgfscope}%
\pgfpathrectangle{\pgfqpoint{1.432000in}{0.528000in}}{\pgfqpoint{3.696000in}{3.696000in}} %
\pgfusepath{clip}%
\pgfsetbuttcap%
\pgfsetroundjoin%
\definecolor{currentfill}{rgb}{0.146180,0.515413,0.556823}%
\pgfsetfillcolor{currentfill}%
\pgfsetlinewidth{0.000000pt}%
\definecolor{currentstroke}{rgb}{0.000000,0.000000,0.000000}%
\pgfsetstrokecolor{currentstroke}%
\pgfsetdash{}{0pt}%
\pgfpathmoveto{\pgfqpoint{4.740090in}{2.210283in}}%
\pgfpathlineto{\pgfqpoint{4.551541in}{2.398832in}}%
\pgfpathlineto{\pgfqpoint{4.548405in}{2.389423in}}%
\pgfpathlineto{\pgfqpoint{4.526452in}{2.430194in}}%
\pgfpathlineto{\pgfqpoint{4.567222in}{2.408240in}}%
\pgfpathlineto{\pgfqpoint{4.557813in}{2.405104in}}%
\pgfpathlineto{\pgfqpoint{4.746362in}{2.216556in}}%
\pgfpathlineto{\pgfqpoint{4.740090in}{2.210283in}}%
\pgfusepath{fill}%
\end{pgfscope}%
\begin{pgfscope}%
\pgfpathrectangle{\pgfqpoint{1.432000in}{0.528000in}}{\pgfqpoint{3.696000in}{3.696000in}} %
\pgfusepath{clip}%
\pgfsetbuttcap%
\pgfsetroundjoin%
\definecolor{currentfill}{rgb}{0.243113,0.292092,0.538516}%
\pgfsetfillcolor{currentfill}%
\pgfsetlinewidth{0.000000pt}%
\definecolor{currentstroke}{rgb}{0.000000,0.000000,0.000000}%
\pgfsetstrokecolor{currentstroke}%
\pgfsetdash{}{0pt}%
\pgfpathmoveto{\pgfqpoint{4.739259in}{2.211436in}}%
\pgfpathlineto{\pgfqpoint{4.648723in}{2.392507in}}%
\pgfpathlineto{\pgfqpoint{4.642773in}{2.384573in}}%
\pgfpathlineto{\pgfqpoint{4.634839in}{2.430194in}}%
\pgfpathlineto{\pgfqpoint{4.666574in}{2.396474in}}%
\pgfpathlineto{\pgfqpoint{4.656657in}{2.396474in}}%
\pgfpathlineto{\pgfqpoint{4.747193in}{2.215403in}}%
\pgfpathlineto{\pgfqpoint{4.739259in}{2.211436in}}%
\pgfusepath{fill}%
\end{pgfscope}%
\begin{pgfscope}%
\pgfpathrectangle{\pgfqpoint{1.432000in}{0.528000in}}{\pgfqpoint{3.696000in}{3.696000in}} %
\pgfusepath{clip}%
\pgfsetbuttcap%
\pgfsetroundjoin%
\definecolor{currentfill}{rgb}{0.267004,0.004874,0.329415}%
\pgfsetfillcolor{currentfill}%
\pgfsetlinewidth{0.000000pt}%
\definecolor{currentstroke}{rgb}{0.000000,0.000000,0.000000}%
\pgfsetstrokecolor{currentstroke}%
\pgfsetdash{}{0pt}%
\pgfpathmoveto{\pgfqpoint{4.739535in}{2.210959in}}%
\pgfpathlineto{\pgfqpoint{4.544903in}{2.502908in}}%
\pgfpathlineto{\pgfqpoint{4.539983in}{2.494297in}}%
\pgfpathlineto{\pgfqpoint{4.526452in}{2.538581in}}%
\pgfpathlineto{\pgfqpoint{4.562125in}{2.509058in}}%
\pgfpathlineto{\pgfqpoint{4.552284in}{2.507828in}}%
\pgfpathlineto{\pgfqpoint{4.746916in}{2.215880in}}%
\pgfpathlineto{\pgfqpoint{4.739535in}{2.210959in}}%
\pgfusepath{fill}%
\end{pgfscope}%
\begin{pgfscope}%
\pgfpathrectangle{\pgfqpoint{1.432000in}{0.528000in}}{\pgfqpoint{3.696000in}{3.696000in}} %
\pgfusepath{clip}%
\pgfsetbuttcap%
\pgfsetroundjoin%
\definecolor{currentfill}{rgb}{0.281887,0.150881,0.465405}%
\pgfsetfillcolor{currentfill}%
\pgfsetlinewidth{0.000000pt}%
\definecolor{currentstroke}{rgb}{0.000000,0.000000,0.000000}%
\pgfsetstrokecolor{currentstroke}%
\pgfsetdash{}{0pt}%
\pgfpathmoveto{\pgfqpoint{4.848477in}{2.210283in}}%
\pgfpathlineto{\pgfqpoint{4.659928in}{2.398832in}}%
\pgfpathlineto{\pgfqpoint{4.656792in}{2.389423in}}%
\pgfpathlineto{\pgfqpoint{4.634839in}{2.430194in}}%
\pgfpathlineto{\pgfqpoint{4.675609in}{2.408240in}}%
\pgfpathlineto{\pgfqpoint{4.666200in}{2.405104in}}%
\pgfpathlineto{\pgfqpoint{4.854749in}{2.216556in}}%
\pgfpathlineto{\pgfqpoint{4.848477in}{2.210283in}}%
\pgfusepath{fill}%
\end{pgfscope}%
\begin{pgfscope}%
\pgfpathrectangle{\pgfqpoint{1.432000in}{0.528000in}}{\pgfqpoint{3.696000in}{3.696000in}} %
\pgfusepath{clip}%
\pgfsetbuttcap%
\pgfsetroundjoin%
\definecolor{currentfill}{rgb}{0.172719,0.448791,0.557885}%
\pgfsetfillcolor{currentfill}%
\pgfsetlinewidth{0.000000pt}%
\definecolor{currentstroke}{rgb}{0.000000,0.000000,0.000000}%
\pgfsetstrokecolor{currentstroke}%
\pgfsetdash{}{0pt}%
\pgfpathmoveto{\pgfqpoint{4.847646in}{2.211436in}}%
\pgfpathlineto{\pgfqpoint{4.757110in}{2.392507in}}%
\pgfpathlineto{\pgfqpoint{4.751160in}{2.384573in}}%
\pgfpathlineto{\pgfqpoint{4.743226in}{2.430194in}}%
\pgfpathlineto{\pgfqpoint{4.774962in}{2.396474in}}%
\pgfpathlineto{\pgfqpoint{4.765044in}{2.396474in}}%
\pgfpathlineto{\pgfqpoint{4.855580in}{2.215403in}}%
\pgfpathlineto{\pgfqpoint{4.847646in}{2.211436in}}%
\pgfusepath{fill}%
\end{pgfscope}%
\begin{pgfscope}%
\pgfpathrectangle{\pgfqpoint{1.432000in}{0.528000in}}{\pgfqpoint{3.696000in}{3.696000in}} %
\pgfusepath{clip}%
\pgfsetbuttcap%
\pgfsetroundjoin%
\definecolor{currentfill}{rgb}{0.179019,0.433756,0.557430}%
\pgfsetfillcolor{currentfill}%
\pgfsetlinewidth{0.000000pt}%
\definecolor{currentstroke}{rgb}{0.000000,0.000000,0.000000}%
\pgfsetstrokecolor{currentstroke}%
\pgfsetdash{}{0pt}%
\pgfpathmoveto{\pgfqpoint{4.956033in}{2.211436in}}%
\pgfpathlineto{\pgfqpoint{4.865497in}{2.392507in}}%
\pgfpathlineto{\pgfqpoint{4.859547in}{2.384573in}}%
\pgfpathlineto{\pgfqpoint{4.851613in}{2.430194in}}%
\pgfpathlineto{\pgfqpoint{4.883349in}{2.396474in}}%
\pgfpathlineto{\pgfqpoint{4.873431in}{2.396474in}}%
\pgfpathlineto{\pgfqpoint{4.963967in}{2.215403in}}%
\pgfpathlineto{\pgfqpoint{4.956033in}{2.211436in}}%
\pgfusepath{fill}%
\end{pgfscope}%
\begin{pgfscope}%
\pgfpathrectangle{\pgfqpoint{1.432000in}{0.528000in}}{\pgfqpoint{3.696000in}{3.696000in}} %
\pgfusepath{clip}%
\pgfsetbuttcap%
\pgfsetroundjoin%
\definecolor{currentfill}{rgb}{0.243113,0.292092,0.538516}%
\pgfsetfillcolor{currentfill}%
\pgfsetlinewidth{0.000000pt}%
\definecolor{currentstroke}{rgb}{0.000000,0.000000,0.000000}%
\pgfsetstrokecolor{currentstroke}%
\pgfsetdash{}{0pt}%
\pgfpathmoveto{\pgfqpoint{4.955565in}{2.213419in}}%
\pgfpathlineto{\pgfqpoint{4.955565in}{2.390277in}}%
\pgfpathlineto{\pgfqpoint{4.946694in}{2.385842in}}%
\pgfpathlineto{\pgfqpoint{4.960000in}{2.430194in}}%
\pgfpathlineto{\pgfqpoint{4.973306in}{2.385842in}}%
\pgfpathlineto{\pgfqpoint{4.964435in}{2.390277in}}%
\pgfpathlineto{\pgfqpoint{4.964435in}{2.213419in}}%
\pgfpathlineto{\pgfqpoint{4.955565in}{2.213419in}}%
\pgfusepath{fill}%
\end{pgfscope}%
\begin{pgfscope}%
\pgfpathrectangle{\pgfqpoint{1.432000in}{0.528000in}}{\pgfqpoint{3.696000in}{3.696000in}} %
\pgfusepath{clip}%
\pgfsetbuttcap%
\pgfsetroundjoin%
\definecolor{currentfill}{rgb}{0.377779,0.791781,0.377939}%
\pgfsetfillcolor{currentfill}%
\pgfsetlinewidth{0.000000pt}%
\definecolor{currentstroke}{rgb}{0.000000,0.000000,0.000000}%
\pgfsetstrokecolor{currentstroke}%
\pgfsetdash{}{0pt}%
\pgfpathmoveto{\pgfqpoint{1.595565in}{2.321806in}}%
\pgfpathlineto{\pgfqpoint{1.595565in}{2.390277in}}%
\pgfpathlineto{\pgfqpoint{1.586694in}{2.385842in}}%
\pgfpathlineto{\pgfqpoint{1.600000in}{2.430194in}}%
\pgfpathlineto{\pgfqpoint{1.613306in}{2.385842in}}%
\pgfpathlineto{\pgfqpoint{1.604435in}{2.390277in}}%
\pgfpathlineto{\pgfqpoint{1.604435in}{2.321806in}}%
\pgfpathlineto{\pgfqpoint{1.595565in}{2.321806in}}%
\pgfusepath{fill}%
\end{pgfscope}%
\begin{pgfscope}%
\pgfpathrectangle{\pgfqpoint{1.432000in}{0.528000in}}{\pgfqpoint{3.696000in}{3.696000in}} %
\pgfusepath{clip}%
\pgfsetbuttcap%
\pgfsetroundjoin%
\definecolor{currentfill}{rgb}{0.282884,0.135920,0.453427}%
\pgfsetfillcolor{currentfill}%
\pgfsetlinewidth{0.000000pt}%
\definecolor{currentstroke}{rgb}{0.000000,0.000000,0.000000}%
\pgfsetstrokecolor{currentstroke}%
\pgfsetdash{}{0pt}%
\pgfpathmoveto{\pgfqpoint{1.705251in}{2.318670in}}%
\pgfpathlineto{\pgfqpoint{1.625089in}{2.398832in}}%
\pgfpathlineto{\pgfqpoint{1.621953in}{2.389423in}}%
\pgfpathlineto{\pgfqpoint{1.600000in}{2.430194in}}%
\pgfpathlineto{\pgfqpoint{1.640770in}{2.408240in}}%
\pgfpathlineto{\pgfqpoint{1.631362in}{2.405104in}}%
\pgfpathlineto{\pgfqpoint{1.711523in}{2.324943in}}%
\pgfpathlineto{\pgfqpoint{1.705251in}{2.318670in}}%
\pgfusepath{fill}%
\end{pgfscope}%
\begin{pgfscope}%
\pgfpathrectangle{\pgfqpoint{1.432000in}{0.528000in}}{\pgfqpoint{3.696000in}{3.696000in}} %
\pgfusepath{clip}%
\pgfsetbuttcap%
\pgfsetroundjoin%
\definecolor{currentfill}{rgb}{0.119738,0.603785,0.541400}%
\pgfsetfillcolor{currentfill}%
\pgfsetlinewidth{0.000000pt}%
\definecolor{currentstroke}{rgb}{0.000000,0.000000,0.000000}%
\pgfsetstrokecolor{currentstroke}%
\pgfsetdash{}{0pt}%
\pgfpathmoveto{\pgfqpoint{1.703952in}{2.321806in}}%
\pgfpathlineto{\pgfqpoint{1.703952in}{2.390277in}}%
\pgfpathlineto{\pgfqpoint{1.695081in}{2.385842in}}%
\pgfpathlineto{\pgfqpoint{1.708387in}{2.430194in}}%
\pgfpathlineto{\pgfqpoint{1.721693in}{2.385842in}}%
\pgfpathlineto{\pgfqpoint{1.712822in}{2.390277in}}%
\pgfpathlineto{\pgfqpoint{1.712822in}{2.321806in}}%
\pgfpathlineto{\pgfqpoint{1.703952in}{2.321806in}}%
\pgfusepath{fill}%
\end{pgfscope}%
\begin{pgfscope}%
\pgfpathrectangle{\pgfqpoint{1.432000in}{0.528000in}}{\pgfqpoint{3.696000in}{3.696000in}} %
\pgfusepath{clip}%
\pgfsetbuttcap%
\pgfsetroundjoin%
\definecolor{currentfill}{rgb}{0.195860,0.395433,0.555276}%
\pgfsetfillcolor{currentfill}%
\pgfsetlinewidth{0.000000pt}%
\definecolor{currentstroke}{rgb}{0.000000,0.000000,0.000000}%
\pgfsetstrokecolor{currentstroke}%
\pgfsetdash{}{0pt}%
\pgfpathmoveto{\pgfqpoint{1.813638in}{2.318670in}}%
\pgfpathlineto{\pgfqpoint{1.733476in}{2.398832in}}%
\pgfpathlineto{\pgfqpoint{1.730340in}{2.389423in}}%
\pgfpathlineto{\pgfqpoint{1.708387in}{2.430194in}}%
\pgfpathlineto{\pgfqpoint{1.749157in}{2.408240in}}%
\pgfpathlineto{\pgfqpoint{1.739749in}{2.405104in}}%
\pgfpathlineto{\pgfqpoint{1.819910in}{2.324943in}}%
\pgfpathlineto{\pgfqpoint{1.813638in}{2.318670in}}%
\pgfusepath{fill}%
\end{pgfscope}%
\begin{pgfscope}%
\pgfpathrectangle{\pgfqpoint{1.432000in}{0.528000in}}{\pgfqpoint{3.696000in}{3.696000in}} %
\pgfusepath{clip}%
\pgfsetbuttcap%
\pgfsetroundjoin%
\definecolor{currentfill}{rgb}{0.252194,0.269783,0.531579}%
\pgfsetfillcolor{currentfill}%
\pgfsetlinewidth{0.000000pt}%
\definecolor{currentstroke}{rgb}{0.000000,0.000000,0.000000}%
\pgfsetstrokecolor{currentstroke}%
\pgfsetdash{}{0pt}%
\pgfpathmoveto{\pgfqpoint{1.812339in}{2.321806in}}%
\pgfpathlineto{\pgfqpoint{1.812339in}{2.390277in}}%
\pgfpathlineto{\pgfqpoint{1.803469in}{2.385842in}}%
\pgfpathlineto{\pgfqpoint{1.816774in}{2.430194in}}%
\pgfpathlineto{\pgfqpoint{1.830080in}{2.385842in}}%
\pgfpathlineto{\pgfqpoint{1.821209in}{2.390277in}}%
\pgfpathlineto{\pgfqpoint{1.821209in}{2.321806in}}%
\pgfpathlineto{\pgfqpoint{1.812339in}{2.321806in}}%
\pgfusepath{fill}%
\end{pgfscope}%
\begin{pgfscope}%
\pgfpathrectangle{\pgfqpoint{1.432000in}{0.528000in}}{\pgfqpoint{3.696000in}{3.696000in}} %
\pgfusepath{clip}%
\pgfsetbuttcap%
\pgfsetroundjoin%
\definecolor{currentfill}{rgb}{0.165117,0.467423,0.558141}%
\pgfsetfillcolor{currentfill}%
\pgfsetlinewidth{0.000000pt}%
\definecolor{currentstroke}{rgb}{0.000000,0.000000,0.000000}%
\pgfsetstrokecolor{currentstroke}%
\pgfsetdash{}{0pt}%
\pgfpathmoveto{\pgfqpoint{1.922025in}{2.318670in}}%
\pgfpathlineto{\pgfqpoint{1.841863in}{2.398832in}}%
\pgfpathlineto{\pgfqpoint{1.838727in}{2.389423in}}%
\pgfpathlineto{\pgfqpoint{1.816774in}{2.430194in}}%
\pgfpathlineto{\pgfqpoint{1.857544in}{2.408240in}}%
\pgfpathlineto{\pgfqpoint{1.848136in}{2.405104in}}%
\pgfpathlineto{\pgfqpoint{1.928297in}{2.324943in}}%
\pgfpathlineto{\pgfqpoint{1.922025in}{2.318670in}}%
\pgfusepath{fill}%
\end{pgfscope}%
\begin{pgfscope}%
\pgfpathrectangle{\pgfqpoint{1.432000in}{0.528000in}}{\pgfqpoint{3.696000in}{3.696000in}} %
\pgfusepath{clip}%
\pgfsetbuttcap%
\pgfsetroundjoin%
\definecolor{currentfill}{rgb}{0.281446,0.084320,0.407414}%
\pgfsetfillcolor{currentfill}%
\pgfsetlinewidth{0.000000pt}%
\definecolor{currentstroke}{rgb}{0.000000,0.000000,0.000000}%
\pgfsetstrokecolor{currentstroke}%
\pgfsetdash{}{0pt}%
\pgfpathmoveto{\pgfqpoint{1.920726in}{2.321806in}}%
\pgfpathlineto{\pgfqpoint{1.920726in}{2.390277in}}%
\pgfpathlineto{\pgfqpoint{1.911856in}{2.385842in}}%
\pgfpathlineto{\pgfqpoint{1.925161in}{2.430194in}}%
\pgfpathlineto{\pgfqpoint{1.938467in}{2.385842in}}%
\pgfpathlineto{\pgfqpoint{1.929596in}{2.390277in}}%
\pgfpathlineto{\pgfqpoint{1.929596in}{2.321806in}}%
\pgfpathlineto{\pgfqpoint{1.920726in}{2.321806in}}%
\pgfusepath{fill}%
\end{pgfscope}%
\begin{pgfscope}%
\pgfpathrectangle{\pgfqpoint{1.432000in}{0.528000in}}{\pgfqpoint{3.696000in}{3.696000in}} %
\pgfusepath{clip}%
\pgfsetbuttcap%
\pgfsetroundjoin%
\definecolor{currentfill}{rgb}{0.143343,0.522773,0.556295}%
\pgfsetfillcolor{currentfill}%
\pgfsetlinewidth{0.000000pt}%
\definecolor{currentstroke}{rgb}{0.000000,0.000000,0.000000}%
\pgfsetstrokecolor{currentstroke}%
\pgfsetdash{}{0pt}%
\pgfpathmoveto{\pgfqpoint{2.030412in}{2.318670in}}%
\pgfpathlineto{\pgfqpoint{1.950251in}{2.398832in}}%
\pgfpathlineto{\pgfqpoint{1.947114in}{2.389423in}}%
\pgfpathlineto{\pgfqpoint{1.925161in}{2.430194in}}%
\pgfpathlineto{\pgfqpoint{1.965931in}{2.408240in}}%
\pgfpathlineto{\pgfqpoint{1.956523in}{2.405104in}}%
\pgfpathlineto{\pgfqpoint{2.036685in}{2.324943in}}%
\pgfpathlineto{\pgfqpoint{2.030412in}{2.318670in}}%
\pgfusepath{fill}%
\end{pgfscope}%
\begin{pgfscope}%
\pgfpathrectangle{\pgfqpoint{1.432000in}{0.528000in}}{\pgfqpoint{3.696000in}{3.696000in}} %
\pgfusepath{clip}%
\pgfsetbuttcap%
\pgfsetroundjoin%
\definecolor{currentfill}{rgb}{0.274952,0.037752,0.364543}%
\pgfsetfillcolor{currentfill}%
\pgfsetlinewidth{0.000000pt}%
\definecolor{currentstroke}{rgb}{0.000000,0.000000,0.000000}%
\pgfsetstrokecolor{currentstroke}%
\pgfsetdash{}{0pt}%
\pgfpathmoveto{\pgfqpoint{2.029581in}{2.319823in}}%
\pgfpathlineto{\pgfqpoint{1.939046in}{2.500894in}}%
\pgfpathlineto{\pgfqpoint{1.933095in}{2.492961in}}%
\pgfpathlineto{\pgfqpoint{1.925161in}{2.538581in}}%
\pgfpathlineto{\pgfqpoint{1.956897in}{2.504861in}}%
\pgfpathlineto{\pgfqpoint{1.946980in}{2.504861in}}%
\pgfpathlineto{\pgfqpoint{2.037515in}{2.323790in}}%
\pgfpathlineto{\pgfqpoint{2.029581in}{2.319823in}}%
\pgfusepath{fill}%
\end{pgfscope}%
\begin{pgfscope}%
\pgfpathrectangle{\pgfqpoint{1.432000in}{0.528000in}}{\pgfqpoint{3.696000in}{3.696000in}} %
\pgfusepath{clip}%
\pgfsetbuttcap%
\pgfsetroundjoin%
\definecolor{currentfill}{rgb}{0.146180,0.515413,0.556823}%
\pgfsetfillcolor{currentfill}%
\pgfsetlinewidth{0.000000pt}%
\definecolor{currentstroke}{rgb}{0.000000,0.000000,0.000000}%
\pgfsetstrokecolor{currentstroke}%
\pgfsetdash{}{0pt}%
\pgfpathmoveto{\pgfqpoint{2.138799in}{2.318670in}}%
\pgfpathlineto{\pgfqpoint{2.058638in}{2.398832in}}%
\pgfpathlineto{\pgfqpoint{2.055502in}{2.389423in}}%
\pgfpathlineto{\pgfqpoint{2.033548in}{2.430194in}}%
\pgfpathlineto{\pgfqpoint{2.074318in}{2.408240in}}%
\pgfpathlineto{\pgfqpoint{2.064910in}{2.405104in}}%
\pgfpathlineto{\pgfqpoint{2.145072in}{2.324943in}}%
\pgfpathlineto{\pgfqpoint{2.138799in}{2.318670in}}%
\pgfusepath{fill}%
\end{pgfscope}%
\begin{pgfscope}%
\pgfpathrectangle{\pgfqpoint{1.432000in}{0.528000in}}{\pgfqpoint{3.696000in}{3.696000in}} %
\pgfusepath{clip}%
\pgfsetbuttcap%
\pgfsetroundjoin%
\definecolor{currentfill}{rgb}{0.280868,0.160771,0.472899}%
\pgfsetfillcolor{currentfill}%
\pgfsetlinewidth{0.000000pt}%
\definecolor{currentstroke}{rgb}{0.000000,0.000000,0.000000}%
\pgfsetstrokecolor{currentstroke}%
\pgfsetdash{}{0pt}%
\pgfpathmoveto{\pgfqpoint{2.247186in}{2.318670in}}%
\pgfpathlineto{\pgfqpoint{2.167025in}{2.398832in}}%
\pgfpathlineto{\pgfqpoint{2.163889in}{2.389423in}}%
\pgfpathlineto{\pgfqpoint{2.141935in}{2.430194in}}%
\pgfpathlineto{\pgfqpoint{2.182706in}{2.408240in}}%
\pgfpathlineto{\pgfqpoint{2.173297in}{2.405104in}}%
\pgfpathlineto{\pgfqpoint{2.253459in}{2.324943in}}%
\pgfpathlineto{\pgfqpoint{2.247186in}{2.318670in}}%
\pgfusepath{fill}%
\end{pgfscope}%
\begin{pgfscope}%
\pgfpathrectangle{\pgfqpoint{1.432000in}{0.528000in}}{\pgfqpoint{3.696000in}{3.696000in}} %
\pgfusepath{clip}%
\pgfsetbuttcap%
\pgfsetroundjoin%
\definecolor{currentfill}{rgb}{0.273006,0.204520,0.501721}%
\pgfsetfillcolor{currentfill}%
\pgfsetlinewidth{0.000000pt}%
\definecolor{currentstroke}{rgb}{0.000000,0.000000,0.000000}%
\pgfsetstrokecolor{currentstroke}%
\pgfsetdash{}{0pt}%
\pgfpathmoveto{\pgfqpoint{2.356726in}{2.317839in}}%
\pgfpathlineto{\pgfqpoint{2.175655in}{2.408375in}}%
\pgfpathlineto{\pgfqpoint{2.175655in}{2.398458in}}%
\pgfpathlineto{\pgfqpoint{2.141935in}{2.430194in}}%
\pgfpathlineto{\pgfqpoint{2.187556in}{2.422260in}}%
\pgfpathlineto{\pgfqpoint{2.179622in}{2.416309in}}%
\pgfpathlineto{\pgfqpoint{2.360693in}{2.325773in}}%
\pgfpathlineto{\pgfqpoint{2.356726in}{2.317839in}}%
\pgfusepath{fill}%
\end{pgfscope}%
\begin{pgfscope}%
\pgfpathrectangle{\pgfqpoint{1.432000in}{0.528000in}}{\pgfqpoint{3.696000in}{3.696000in}} %
\pgfusepath{clip}%
\pgfsetbuttcap%
\pgfsetroundjoin%
\definecolor{currentfill}{rgb}{0.190631,0.407061,0.556089}%
\pgfsetfillcolor{currentfill}%
\pgfsetlinewidth{0.000000pt}%
\definecolor{currentstroke}{rgb}{0.000000,0.000000,0.000000}%
\pgfsetstrokecolor{currentstroke}%
\pgfsetdash{}{0pt}%
\pgfpathmoveto{\pgfqpoint{2.465113in}{2.317839in}}%
\pgfpathlineto{\pgfqpoint{2.284042in}{2.408375in}}%
\pgfpathlineto{\pgfqpoint{2.284042in}{2.398458in}}%
\pgfpathlineto{\pgfqpoint{2.250323in}{2.430194in}}%
\pgfpathlineto{\pgfqpoint{2.295943in}{2.422260in}}%
\pgfpathlineto{\pgfqpoint{2.288009in}{2.416309in}}%
\pgfpathlineto{\pgfqpoint{2.469080in}{2.325773in}}%
\pgfpathlineto{\pgfqpoint{2.465113in}{2.317839in}}%
\pgfusepath{fill}%
\end{pgfscope}%
\begin{pgfscope}%
\pgfpathrectangle{\pgfqpoint{1.432000in}{0.528000in}}{\pgfqpoint{3.696000in}{3.696000in}} %
\pgfusepath{clip}%
\pgfsetbuttcap%
\pgfsetroundjoin%
\definecolor{currentfill}{rgb}{0.278791,0.062145,0.386592}%
\pgfsetfillcolor{currentfill}%
\pgfsetlinewidth{0.000000pt}%
\definecolor{currentstroke}{rgb}{0.000000,0.000000,0.000000}%
\pgfsetstrokecolor{currentstroke}%
\pgfsetdash{}{0pt}%
\pgfpathmoveto{\pgfqpoint{2.463961in}{2.318670in}}%
\pgfpathlineto{\pgfqpoint{2.275412in}{2.507219in}}%
\pgfpathlineto{\pgfqpoint{2.272276in}{2.497811in}}%
\pgfpathlineto{\pgfqpoint{2.250323in}{2.538581in}}%
\pgfpathlineto{\pgfqpoint{2.291093in}{2.516628in}}%
\pgfpathlineto{\pgfqpoint{2.281684in}{2.513491in}}%
\pgfpathlineto{\pgfqpoint{2.470233in}{2.324943in}}%
\pgfpathlineto{\pgfqpoint{2.463961in}{2.318670in}}%
\pgfusepath{fill}%
\end{pgfscope}%
\begin{pgfscope}%
\pgfpathrectangle{\pgfqpoint{1.432000in}{0.528000in}}{\pgfqpoint{3.696000in}{3.696000in}} %
\pgfusepath{clip}%
\pgfsetbuttcap%
\pgfsetroundjoin%
\definecolor{currentfill}{rgb}{0.169646,0.456262,0.558030}%
\pgfsetfillcolor{currentfill}%
\pgfsetlinewidth{0.000000pt}%
\definecolor{currentstroke}{rgb}{0.000000,0.000000,0.000000}%
\pgfsetstrokecolor{currentstroke}%
\pgfsetdash{}{0pt}%
\pgfpathmoveto{\pgfqpoint{2.573500in}{2.317839in}}%
\pgfpathlineto{\pgfqpoint{2.392429in}{2.408375in}}%
\pgfpathlineto{\pgfqpoint{2.392429in}{2.398458in}}%
\pgfpathlineto{\pgfqpoint{2.358710in}{2.430194in}}%
\pgfpathlineto{\pgfqpoint{2.404330in}{2.422260in}}%
\pgfpathlineto{\pgfqpoint{2.396396in}{2.416309in}}%
\pgfpathlineto{\pgfqpoint{2.577467in}{2.325773in}}%
\pgfpathlineto{\pgfqpoint{2.573500in}{2.317839in}}%
\pgfusepath{fill}%
\end{pgfscope}%
\begin{pgfscope}%
\pgfpathrectangle{\pgfqpoint{1.432000in}{0.528000in}}{\pgfqpoint{3.696000in}{3.696000in}} %
\pgfusepath{clip}%
\pgfsetbuttcap%
\pgfsetroundjoin%
\definecolor{currentfill}{rgb}{0.281887,0.150881,0.465405}%
\pgfsetfillcolor{currentfill}%
\pgfsetlinewidth{0.000000pt}%
\definecolor{currentstroke}{rgb}{0.000000,0.000000,0.000000}%
\pgfsetstrokecolor{currentstroke}%
\pgfsetdash{}{0pt}%
\pgfpathmoveto{\pgfqpoint{2.572348in}{2.318670in}}%
\pgfpathlineto{\pgfqpoint{2.383799in}{2.507219in}}%
\pgfpathlineto{\pgfqpoint{2.380663in}{2.497811in}}%
\pgfpathlineto{\pgfqpoint{2.358710in}{2.538581in}}%
\pgfpathlineto{\pgfqpoint{2.399480in}{2.516628in}}%
\pgfpathlineto{\pgfqpoint{2.390071in}{2.513491in}}%
\pgfpathlineto{\pgfqpoint{2.578620in}{2.324943in}}%
\pgfpathlineto{\pgfqpoint{2.572348in}{2.318670in}}%
\pgfusepath{fill}%
\end{pgfscope}%
\begin{pgfscope}%
\pgfpathrectangle{\pgfqpoint{1.432000in}{0.528000in}}{\pgfqpoint{3.696000in}{3.696000in}} %
\pgfusepath{clip}%
\pgfsetbuttcap%
\pgfsetroundjoin%
\definecolor{currentfill}{rgb}{0.235526,0.309527,0.542944}%
\pgfsetfillcolor{currentfill}%
\pgfsetlinewidth{0.000000pt}%
\definecolor{currentstroke}{rgb}{0.000000,0.000000,0.000000}%
\pgfsetstrokecolor{currentstroke}%
\pgfsetdash{}{0pt}%
\pgfpathmoveto{\pgfqpoint{2.681887in}{2.317839in}}%
\pgfpathlineto{\pgfqpoint{2.500816in}{2.408375in}}%
\pgfpathlineto{\pgfqpoint{2.500816in}{2.398458in}}%
\pgfpathlineto{\pgfqpoint{2.467097in}{2.430194in}}%
\pgfpathlineto{\pgfqpoint{2.512717in}{2.422260in}}%
\pgfpathlineto{\pgfqpoint{2.504783in}{2.416309in}}%
\pgfpathlineto{\pgfqpoint{2.685854in}{2.325773in}}%
\pgfpathlineto{\pgfqpoint{2.681887in}{2.317839in}}%
\pgfusepath{fill}%
\end{pgfscope}%
\begin{pgfscope}%
\pgfpathrectangle{\pgfqpoint{1.432000in}{0.528000in}}{\pgfqpoint{3.696000in}{3.696000in}} %
\pgfusepath{clip}%
\pgfsetbuttcap%
\pgfsetroundjoin%
\definecolor{currentfill}{rgb}{0.278826,0.175490,0.483397}%
\pgfsetfillcolor{currentfill}%
\pgfsetlinewidth{0.000000pt}%
\definecolor{currentstroke}{rgb}{0.000000,0.000000,0.000000}%
\pgfsetstrokecolor{currentstroke}%
\pgfsetdash{}{0pt}%
\pgfpathmoveto{\pgfqpoint{2.680735in}{2.318670in}}%
\pgfpathlineto{\pgfqpoint{2.600573in}{2.398832in}}%
\pgfpathlineto{\pgfqpoint{2.597437in}{2.389423in}}%
\pgfpathlineto{\pgfqpoint{2.575484in}{2.430194in}}%
\pgfpathlineto{\pgfqpoint{2.616254in}{2.408240in}}%
\pgfpathlineto{\pgfqpoint{2.606845in}{2.405104in}}%
\pgfpathlineto{\pgfqpoint{2.687007in}{2.324943in}}%
\pgfpathlineto{\pgfqpoint{2.680735in}{2.318670in}}%
\pgfusepath{fill}%
\end{pgfscope}%
\begin{pgfscope}%
\pgfpathrectangle{\pgfqpoint{1.432000in}{0.528000in}}{\pgfqpoint{3.696000in}{3.696000in}} %
\pgfusepath{clip}%
\pgfsetbuttcap%
\pgfsetroundjoin%
\definecolor{currentfill}{rgb}{0.283197,0.115680,0.436115}%
\pgfsetfillcolor{currentfill}%
\pgfsetlinewidth{0.000000pt}%
\definecolor{currentstroke}{rgb}{0.000000,0.000000,0.000000}%
\pgfsetstrokecolor{currentstroke}%
\pgfsetdash{}{0pt}%
\pgfpathmoveto{\pgfqpoint{2.790275in}{2.317839in}}%
\pgfpathlineto{\pgfqpoint{2.609203in}{2.408375in}}%
\pgfpathlineto{\pgfqpoint{2.609203in}{2.398458in}}%
\pgfpathlineto{\pgfqpoint{2.575484in}{2.430194in}}%
\pgfpathlineto{\pgfqpoint{2.621104in}{2.422260in}}%
\pgfpathlineto{\pgfqpoint{2.613170in}{2.416309in}}%
\pgfpathlineto{\pgfqpoint{2.794242in}{2.325773in}}%
\pgfpathlineto{\pgfqpoint{2.790275in}{2.317839in}}%
\pgfusepath{fill}%
\end{pgfscope}%
\begin{pgfscope}%
\pgfpathrectangle{\pgfqpoint{1.432000in}{0.528000in}}{\pgfqpoint{3.696000in}{3.696000in}} %
\pgfusepath{clip}%
\pgfsetbuttcap%
\pgfsetroundjoin%
\definecolor{currentfill}{rgb}{0.282290,0.145912,0.461510}%
\pgfsetfillcolor{currentfill}%
\pgfsetlinewidth{0.000000pt}%
\definecolor{currentstroke}{rgb}{0.000000,0.000000,0.000000}%
\pgfsetstrokecolor{currentstroke}%
\pgfsetdash{}{0pt}%
\pgfpathmoveto{\pgfqpoint{2.789122in}{2.318670in}}%
\pgfpathlineto{\pgfqpoint{2.708960in}{2.398832in}}%
\pgfpathlineto{\pgfqpoint{2.705824in}{2.389423in}}%
\pgfpathlineto{\pgfqpoint{2.683871in}{2.430194in}}%
\pgfpathlineto{\pgfqpoint{2.724641in}{2.408240in}}%
\pgfpathlineto{\pgfqpoint{2.715233in}{2.405104in}}%
\pgfpathlineto{\pgfqpoint{2.795394in}{2.324943in}}%
\pgfpathlineto{\pgfqpoint{2.789122in}{2.318670in}}%
\pgfusepath{fill}%
\end{pgfscope}%
\begin{pgfscope}%
\pgfpathrectangle{\pgfqpoint{1.432000in}{0.528000in}}{\pgfqpoint{3.696000in}{3.696000in}} %
\pgfusepath{clip}%
\pgfsetbuttcap%
\pgfsetroundjoin%
\definecolor{currentfill}{rgb}{0.280267,0.073417,0.397163}%
\pgfsetfillcolor{currentfill}%
\pgfsetlinewidth{0.000000pt}%
\definecolor{currentstroke}{rgb}{0.000000,0.000000,0.000000}%
\pgfsetstrokecolor{currentstroke}%
\pgfsetdash{}{0pt}%
\pgfpathmoveto{\pgfqpoint{2.788291in}{2.319823in}}%
\pgfpathlineto{\pgfqpoint{2.697755in}{2.500894in}}%
\pgfpathlineto{\pgfqpoint{2.691805in}{2.492961in}}%
\pgfpathlineto{\pgfqpoint{2.683871in}{2.538581in}}%
\pgfpathlineto{\pgfqpoint{2.715607in}{2.504861in}}%
\pgfpathlineto{\pgfqpoint{2.705689in}{2.504861in}}%
\pgfpathlineto{\pgfqpoint{2.796225in}{2.323790in}}%
\pgfpathlineto{\pgfqpoint{2.788291in}{2.319823in}}%
\pgfusepath{fill}%
\end{pgfscope}%
\begin{pgfscope}%
\pgfpathrectangle{\pgfqpoint{1.432000in}{0.528000in}}{\pgfqpoint{3.696000in}{3.696000in}} %
\pgfusepath{clip}%
\pgfsetbuttcap%
\pgfsetroundjoin%
\definecolor{currentfill}{rgb}{0.123463,0.581687,0.547445}%
\pgfsetfillcolor{currentfill}%
\pgfsetlinewidth{0.000000pt}%
\definecolor{currentstroke}{rgb}{0.000000,0.000000,0.000000}%
\pgfsetstrokecolor{currentstroke}%
\pgfsetdash{}{0pt}%
\pgfpathmoveto{\pgfqpoint{2.896678in}{2.319823in}}%
\pgfpathlineto{\pgfqpoint{2.806142in}{2.500894in}}%
\pgfpathlineto{\pgfqpoint{2.800192in}{2.492961in}}%
\pgfpathlineto{\pgfqpoint{2.792258in}{2.538581in}}%
\pgfpathlineto{\pgfqpoint{2.823994in}{2.504861in}}%
\pgfpathlineto{\pgfqpoint{2.814076in}{2.504861in}}%
\pgfpathlineto{\pgfqpoint{2.904612in}{2.323790in}}%
\pgfpathlineto{\pgfqpoint{2.896678in}{2.319823in}}%
\pgfusepath{fill}%
\end{pgfscope}%
\begin{pgfscope}%
\pgfpathrectangle{\pgfqpoint{1.432000in}{0.528000in}}{\pgfqpoint{3.696000in}{3.696000in}} %
\pgfusepath{clip}%
\pgfsetbuttcap%
\pgfsetroundjoin%
\definecolor{currentfill}{rgb}{0.157851,0.683765,0.501686}%
\pgfsetfillcolor{currentfill}%
\pgfsetlinewidth{0.000000pt}%
\definecolor{currentstroke}{rgb}{0.000000,0.000000,0.000000}%
\pgfsetstrokecolor{currentstroke}%
\pgfsetdash{}{0pt}%
\pgfpathmoveto{\pgfqpoint{3.005065in}{2.319823in}}%
\pgfpathlineto{\pgfqpoint{2.914530in}{2.500894in}}%
\pgfpathlineto{\pgfqpoint{2.908579in}{2.492961in}}%
\pgfpathlineto{\pgfqpoint{2.900645in}{2.538581in}}%
\pgfpathlineto{\pgfqpoint{2.932381in}{2.504861in}}%
\pgfpathlineto{\pgfqpoint{2.922463in}{2.504861in}}%
\pgfpathlineto{\pgfqpoint{3.012999in}{2.323790in}}%
\pgfpathlineto{\pgfqpoint{3.005065in}{2.319823in}}%
\pgfusepath{fill}%
\end{pgfscope}%
\begin{pgfscope}%
\pgfpathrectangle{\pgfqpoint{1.432000in}{0.528000in}}{\pgfqpoint{3.696000in}{3.696000in}} %
\pgfusepath{clip}%
\pgfsetbuttcap%
\pgfsetroundjoin%
\definecolor{currentfill}{rgb}{0.210503,0.363727,0.552206}%
\pgfsetfillcolor{currentfill}%
\pgfsetlinewidth{0.000000pt}%
\definecolor{currentstroke}{rgb}{0.000000,0.000000,0.000000}%
\pgfsetstrokecolor{currentstroke}%
\pgfsetdash{}{0pt}%
\pgfpathmoveto{\pgfqpoint{3.113452in}{2.319823in}}%
\pgfpathlineto{\pgfqpoint{3.022917in}{2.500894in}}%
\pgfpathlineto{\pgfqpoint{3.016966in}{2.492961in}}%
\pgfpathlineto{\pgfqpoint{3.009032in}{2.538581in}}%
\pgfpathlineto{\pgfqpoint{3.040768in}{2.504861in}}%
\pgfpathlineto{\pgfqpoint{3.030851in}{2.504861in}}%
\pgfpathlineto{\pgfqpoint{3.121386in}{2.323790in}}%
\pgfpathlineto{\pgfqpoint{3.113452in}{2.319823in}}%
\pgfusepath{fill}%
\end{pgfscope}%
\begin{pgfscope}%
\pgfpathrectangle{\pgfqpoint{1.432000in}{0.528000in}}{\pgfqpoint{3.696000in}{3.696000in}} %
\pgfusepath{clip}%
\pgfsetbuttcap%
\pgfsetroundjoin%
\definecolor{currentfill}{rgb}{0.274952,0.037752,0.364543}%
\pgfsetfillcolor{currentfill}%
\pgfsetlinewidth{0.000000pt}%
\definecolor{currentstroke}{rgb}{0.000000,0.000000,0.000000}%
\pgfsetstrokecolor{currentstroke}%
\pgfsetdash{}{0pt}%
\pgfpathmoveto{\pgfqpoint{3.113212in}{2.320404in}}%
\pgfpathlineto{\pgfqpoint{3.017447in}{2.607697in}}%
\pgfpathlineto{\pgfqpoint{3.010435in}{2.600684in}}%
\pgfpathlineto{\pgfqpoint{3.009032in}{2.646968in}}%
\pgfpathlineto{\pgfqpoint{3.035680in}{2.609099in}}%
\pgfpathlineto{\pgfqpoint{3.025863in}{2.610502in}}%
\pgfpathlineto{\pgfqpoint{3.121627in}{2.323209in}}%
\pgfpathlineto{\pgfqpoint{3.113212in}{2.320404in}}%
\pgfusepath{fill}%
\end{pgfscope}%
\begin{pgfscope}%
\pgfpathrectangle{\pgfqpoint{1.432000in}{0.528000in}}{\pgfqpoint{3.696000in}{3.696000in}} %
\pgfusepath{clip}%
\pgfsetbuttcap%
\pgfsetroundjoin%
\definecolor{currentfill}{rgb}{0.279574,0.170599,0.479997}%
\pgfsetfillcolor{currentfill}%
\pgfsetlinewidth{0.000000pt}%
\definecolor{currentstroke}{rgb}{0.000000,0.000000,0.000000}%
\pgfsetstrokecolor{currentstroke}%
\pgfsetdash{}{0pt}%
\pgfpathmoveto{\pgfqpoint{3.222670in}{2.318670in}}%
\pgfpathlineto{\pgfqpoint{3.034122in}{2.507219in}}%
\pgfpathlineto{\pgfqpoint{3.030985in}{2.497811in}}%
\pgfpathlineto{\pgfqpoint{3.009032in}{2.538581in}}%
\pgfpathlineto{\pgfqpoint{3.049802in}{2.516628in}}%
\pgfpathlineto{\pgfqpoint{3.040394in}{2.513491in}}%
\pgfpathlineto{\pgfqpoint{3.228943in}{2.324943in}}%
\pgfpathlineto{\pgfqpoint{3.222670in}{2.318670in}}%
\pgfusepath{fill}%
\end{pgfscope}%
\begin{pgfscope}%
\pgfpathrectangle{\pgfqpoint{1.432000in}{0.528000in}}{\pgfqpoint{3.696000in}{3.696000in}} %
\pgfusepath{clip}%
\pgfsetbuttcap%
\pgfsetroundjoin%
\definecolor{currentfill}{rgb}{0.223925,0.334994,0.548053}%
\pgfsetfillcolor{currentfill}%
\pgfsetlinewidth{0.000000pt}%
\definecolor{currentstroke}{rgb}{0.000000,0.000000,0.000000}%
\pgfsetstrokecolor{currentstroke}%
\pgfsetdash{}{0pt}%
\pgfpathmoveto{\pgfqpoint{3.331057in}{2.318670in}}%
\pgfpathlineto{\pgfqpoint{3.142509in}{2.507219in}}%
\pgfpathlineto{\pgfqpoint{3.139372in}{2.497811in}}%
\pgfpathlineto{\pgfqpoint{3.117419in}{2.538581in}}%
\pgfpathlineto{\pgfqpoint{3.158189in}{2.516628in}}%
\pgfpathlineto{\pgfqpoint{3.148781in}{2.513491in}}%
\pgfpathlineto{\pgfqpoint{3.337330in}{2.324943in}}%
\pgfpathlineto{\pgfqpoint{3.331057in}{2.318670in}}%
\pgfusepath{fill}%
\end{pgfscope}%
\begin{pgfscope}%
\pgfpathrectangle{\pgfqpoint{1.432000in}{0.528000in}}{\pgfqpoint{3.696000in}{3.696000in}} %
\pgfusepath{clip}%
\pgfsetbuttcap%
\pgfsetroundjoin%
\definecolor{currentfill}{rgb}{0.280868,0.160771,0.472899}%
\pgfsetfillcolor{currentfill}%
\pgfsetlinewidth{0.000000pt}%
\definecolor{currentstroke}{rgb}{0.000000,0.000000,0.000000}%
\pgfsetstrokecolor{currentstroke}%
\pgfsetdash{}{0pt}%
\pgfpathmoveto{\pgfqpoint{3.330503in}{2.319346in}}%
\pgfpathlineto{\pgfqpoint{3.135871in}{2.611295in}}%
\pgfpathlineto{\pgfqpoint{3.130950in}{2.602684in}}%
\pgfpathlineto{\pgfqpoint{3.117419in}{2.646968in}}%
\pgfpathlineto{\pgfqpoint{3.153092in}{2.617445in}}%
\pgfpathlineto{\pgfqpoint{3.143252in}{2.616215in}}%
\pgfpathlineto{\pgfqpoint{3.337884in}{2.324267in}}%
\pgfpathlineto{\pgfqpoint{3.330503in}{2.319346in}}%
\pgfusepath{fill}%
\end{pgfscope}%
\begin{pgfscope}%
\pgfpathrectangle{\pgfqpoint{1.432000in}{0.528000in}}{\pgfqpoint{3.696000in}{3.696000in}} %
\pgfusepath{clip}%
\pgfsetbuttcap%
\pgfsetroundjoin%
\definecolor{currentfill}{rgb}{0.162142,0.474838,0.558140}%
\pgfsetfillcolor{currentfill}%
\pgfsetlinewidth{0.000000pt}%
\definecolor{currentstroke}{rgb}{0.000000,0.000000,0.000000}%
\pgfsetstrokecolor{currentstroke}%
\pgfsetdash{}{0pt}%
\pgfpathmoveto{\pgfqpoint{3.439444in}{2.318670in}}%
\pgfpathlineto{\pgfqpoint{3.250896in}{2.507219in}}%
\pgfpathlineto{\pgfqpoint{3.247760in}{2.497811in}}%
\pgfpathlineto{\pgfqpoint{3.225806in}{2.538581in}}%
\pgfpathlineto{\pgfqpoint{3.266577in}{2.516628in}}%
\pgfpathlineto{\pgfqpoint{3.257168in}{2.513491in}}%
\pgfpathlineto{\pgfqpoint{3.445717in}{2.324943in}}%
\pgfpathlineto{\pgfqpoint{3.439444in}{2.318670in}}%
\pgfusepath{fill}%
\end{pgfscope}%
\begin{pgfscope}%
\pgfpathrectangle{\pgfqpoint{1.432000in}{0.528000in}}{\pgfqpoint{3.696000in}{3.696000in}} %
\pgfusepath{clip}%
\pgfsetbuttcap%
\pgfsetroundjoin%
\definecolor{currentfill}{rgb}{0.253935,0.265254,0.529983}%
\pgfsetfillcolor{currentfill}%
\pgfsetlinewidth{0.000000pt}%
\definecolor{currentstroke}{rgb}{0.000000,0.000000,0.000000}%
\pgfsetstrokecolor{currentstroke}%
\pgfsetdash{}{0pt}%
\pgfpathmoveto{\pgfqpoint{3.548508in}{2.318116in}}%
\pgfpathlineto{\pgfqpoint{3.256559in}{2.512748in}}%
\pgfpathlineto{\pgfqpoint{3.255329in}{2.502908in}}%
\pgfpathlineto{\pgfqpoint{3.225806in}{2.538581in}}%
\pgfpathlineto{\pgfqpoint{3.270090in}{2.525050in}}%
\pgfpathlineto{\pgfqpoint{3.261479in}{2.520129in}}%
\pgfpathlineto{\pgfqpoint{3.553428in}{2.325497in}}%
\pgfpathlineto{\pgfqpoint{3.548508in}{2.318116in}}%
\pgfusepath{fill}%
\end{pgfscope}%
\begin{pgfscope}%
\pgfpathrectangle{\pgfqpoint{1.432000in}{0.528000in}}{\pgfqpoint{3.696000in}{3.696000in}} %
\pgfusepath{clip}%
\pgfsetbuttcap%
\pgfsetroundjoin%
\definecolor{currentfill}{rgb}{0.267968,0.223549,0.512008}%
\pgfsetfillcolor{currentfill}%
\pgfsetlinewidth{0.000000pt}%
\definecolor{currentstroke}{rgb}{0.000000,0.000000,0.000000}%
\pgfsetstrokecolor{currentstroke}%
\pgfsetdash{}{0pt}%
\pgfpathmoveto{\pgfqpoint{3.547832in}{2.318670in}}%
\pgfpathlineto{\pgfqpoint{3.359283in}{2.507219in}}%
\pgfpathlineto{\pgfqpoint{3.356147in}{2.497811in}}%
\pgfpathlineto{\pgfqpoint{3.334194in}{2.538581in}}%
\pgfpathlineto{\pgfqpoint{3.374964in}{2.516628in}}%
\pgfpathlineto{\pgfqpoint{3.365555in}{2.513491in}}%
\pgfpathlineto{\pgfqpoint{3.554104in}{2.324943in}}%
\pgfpathlineto{\pgfqpoint{3.547832in}{2.318670in}}%
\pgfusepath{fill}%
\end{pgfscope}%
\begin{pgfscope}%
\pgfpathrectangle{\pgfqpoint{1.432000in}{0.528000in}}{\pgfqpoint{3.696000in}{3.696000in}} %
\pgfusepath{clip}%
\pgfsetbuttcap%
\pgfsetroundjoin%
\definecolor{currentfill}{rgb}{0.150148,0.676631,0.506589}%
\pgfsetfillcolor{currentfill}%
\pgfsetlinewidth{0.000000pt}%
\definecolor{currentstroke}{rgb}{0.000000,0.000000,0.000000}%
\pgfsetstrokecolor{currentstroke}%
\pgfsetdash{}{0pt}%
\pgfpathmoveto{\pgfqpoint{3.656895in}{2.318116in}}%
\pgfpathlineto{\pgfqpoint{3.364946in}{2.512748in}}%
\pgfpathlineto{\pgfqpoint{3.363716in}{2.502908in}}%
\pgfpathlineto{\pgfqpoint{3.334194in}{2.538581in}}%
\pgfpathlineto{\pgfqpoint{3.378477in}{2.525050in}}%
\pgfpathlineto{\pgfqpoint{3.369867in}{2.520129in}}%
\pgfpathlineto{\pgfqpoint{3.661815in}{2.325497in}}%
\pgfpathlineto{\pgfqpoint{3.656895in}{2.318116in}}%
\pgfusepath{fill}%
\end{pgfscope}%
\begin{pgfscope}%
\pgfpathrectangle{\pgfqpoint{1.432000in}{0.528000in}}{\pgfqpoint{3.696000in}{3.696000in}} %
\pgfusepath{clip}%
\pgfsetbuttcap%
\pgfsetroundjoin%
\definecolor{currentfill}{rgb}{0.565498,0.842430,0.262877}%
\pgfsetfillcolor{currentfill}%
\pgfsetlinewidth{0.000000pt}%
\definecolor{currentstroke}{rgb}{0.000000,0.000000,0.000000}%
\pgfsetstrokecolor{currentstroke}%
\pgfsetdash{}{0pt}%
\pgfpathmoveto{\pgfqpoint{3.765282in}{2.318116in}}%
\pgfpathlineto{\pgfqpoint{3.473333in}{2.512748in}}%
\pgfpathlineto{\pgfqpoint{3.472103in}{2.502908in}}%
\pgfpathlineto{\pgfqpoint{3.442581in}{2.538581in}}%
\pgfpathlineto{\pgfqpoint{3.486864in}{2.525050in}}%
\pgfpathlineto{\pgfqpoint{3.478254in}{2.520129in}}%
\pgfpathlineto{\pgfqpoint{3.770202in}{2.325497in}}%
\pgfpathlineto{\pgfqpoint{3.765282in}{2.318116in}}%
\pgfusepath{fill}%
\end{pgfscope}%
\begin{pgfscope}%
\pgfpathrectangle{\pgfqpoint{1.432000in}{0.528000in}}{\pgfqpoint{3.696000in}{3.696000in}} %
\pgfusepath{clip}%
\pgfsetbuttcap%
\pgfsetroundjoin%
\definecolor{currentfill}{rgb}{0.575563,0.844566,0.256415}%
\pgfsetfillcolor{currentfill}%
\pgfsetlinewidth{0.000000pt}%
\definecolor{currentstroke}{rgb}{0.000000,0.000000,0.000000}%
\pgfsetstrokecolor{currentstroke}%
\pgfsetdash{}{0pt}%
\pgfpathmoveto{\pgfqpoint{3.873669in}{2.318116in}}%
\pgfpathlineto{\pgfqpoint{3.581720in}{2.512748in}}%
\pgfpathlineto{\pgfqpoint{3.580490in}{2.502908in}}%
\pgfpathlineto{\pgfqpoint{3.550968in}{2.538581in}}%
\pgfpathlineto{\pgfqpoint{3.595251in}{2.525050in}}%
\pgfpathlineto{\pgfqpoint{3.586641in}{2.520129in}}%
\pgfpathlineto{\pgfqpoint{3.878589in}{2.325497in}}%
\pgfpathlineto{\pgfqpoint{3.873669in}{2.318116in}}%
\pgfusepath{fill}%
\end{pgfscope}%
\begin{pgfscope}%
\pgfpathrectangle{\pgfqpoint{1.432000in}{0.528000in}}{\pgfqpoint{3.696000in}{3.696000in}} %
\pgfusepath{clip}%
\pgfsetbuttcap%
\pgfsetroundjoin%
\definecolor{currentfill}{rgb}{0.267004,0.004874,0.329415}%
\pgfsetfillcolor{currentfill}%
\pgfsetlinewidth{0.000000pt}%
\definecolor{currentstroke}{rgb}{0.000000,0.000000,0.000000}%
\pgfsetstrokecolor{currentstroke}%
\pgfsetdash{}{0pt}%
\pgfpathmoveto{\pgfqpoint{3.982533in}{2.317839in}}%
\pgfpathlineto{\pgfqpoint{3.584687in}{2.516762in}}%
\pgfpathlineto{\pgfqpoint{3.584687in}{2.506845in}}%
\pgfpathlineto{\pgfqpoint{3.550968in}{2.538581in}}%
\pgfpathlineto{\pgfqpoint{3.596588in}{2.530647in}}%
\pgfpathlineto{\pgfqpoint{3.588654in}{2.524696in}}%
\pgfpathlineto{\pgfqpoint{3.986500in}{2.325773in}}%
\pgfpathlineto{\pgfqpoint{3.982533in}{2.317839in}}%
\pgfusepath{fill}%
\end{pgfscope}%
\begin{pgfscope}%
\pgfpathrectangle{\pgfqpoint{1.432000in}{0.528000in}}{\pgfqpoint{3.696000in}{3.696000in}} %
\pgfusepath{clip}%
\pgfsetbuttcap%
\pgfsetroundjoin%
\definecolor{currentfill}{rgb}{0.585678,0.846661,0.249897}%
\pgfsetfillcolor{currentfill}%
\pgfsetlinewidth{0.000000pt}%
\definecolor{currentstroke}{rgb}{0.000000,0.000000,0.000000}%
\pgfsetstrokecolor{currentstroke}%
\pgfsetdash{}{0pt}%
\pgfpathmoveto{\pgfqpoint{3.982056in}{2.318116in}}%
\pgfpathlineto{\pgfqpoint{3.690107in}{2.512748in}}%
\pgfpathlineto{\pgfqpoint{3.688877in}{2.502908in}}%
\pgfpathlineto{\pgfqpoint{3.659355in}{2.538581in}}%
\pgfpathlineto{\pgfqpoint{3.703639in}{2.525050in}}%
\pgfpathlineto{\pgfqpoint{3.695028in}{2.520129in}}%
\pgfpathlineto{\pgfqpoint{3.986976in}{2.325497in}}%
\pgfpathlineto{\pgfqpoint{3.982056in}{2.318116in}}%
\pgfusepath{fill}%
\end{pgfscope}%
\begin{pgfscope}%
\pgfpathrectangle{\pgfqpoint{1.432000in}{0.528000in}}{\pgfqpoint{3.696000in}{3.696000in}} %
\pgfusepath{clip}%
\pgfsetbuttcap%
\pgfsetroundjoin%
\definecolor{currentfill}{rgb}{0.283091,0.110553,0.431554}%
\pgfsetfillcolor{currentfill}%
\pgfsetlinewidth{0.000000pt}%
\definecolor{currentstroke}{rgb}{0.000000,0.000000,0.000000}%
\pgfsetstrokecolor{currentstroke}%
\pgfsetdash{}{0pt}%
\pgfpathmoveto{\pgfqpoint{4.090920in}{2.317839in}}%
\pgfpathlineto{\pgfqpoint{3.693074in}{2.516762in}}%
\pgfpathlineto{\pgfqpoint{3.693074in}{2.506845in}}%
\pgfpathlineto{\pgfqpoint{3.659355in}{2.538581in}}%
\pgfpathlineto{\pgfqpoint{3.704975in}{2.530647in}}%
\pgfpathlineto{\pgfqpoint{3.697041in}{2.524696in}}%
\pgfpathlineto{\pgfqpoint{4.094887in}{2.325773in}}%
\pgfpathlineto{\pgfqpoint{4.090920in}{2.317839in}}%
\pgfusepath{fill}%
\end{pgfscope}%
\begin{pgfscope}%
\pgfpathrectangle{\pgfqpoint{1.432000in}{0.528000in}}{\pgfqpoint{3.696000in}{3.696000in}} %
\pgfusepath{clip}%
\pgfsetbuttcap%
\pgfsetroundjoin%
\definecolor{currentfill}{rgb}{0.555484,0.840254,0.269281}%
\pgfsetfillcolor{currentfill}%
\pgfsetlinewidth{0.000000pt}%
\definecolor{currentstroke}{rgb}{0.000000,0.000000,0.000000}%
\pgfsetstrokecolor{currentstroke}%
\pgfsetdash{}{0pt}%
\pgfpathmoveto{\pgfqpoint{4.090443in}{2.318116in}}%
\pgfpathlineto{\pgfqpoint{3.798495in}{2.512748in}}%
\pgfpathlineto{\pgfqpoint{3.797264in}{2.502908in}}%
\pgfpathlineto{\pgfqpoint{3.767742in}{2.538581in}}%
\pgfpathlineto{\pgfqpoint{3.812026in}{2.525050in}}%
\pgfpathlineto{\pgfqpoint{3.803415in}{2.520129in}}%
\pgfpathlineto{\pgfqpoint{4.095363in}{2.325497in}}%
\pgfpathlineto{\pgfqpoint{4.090443in}{2.318116in}}%
\pgfusepath{fill}%
\end{pgfscope}%
\begin{pgfscope}%
\pgfpathrectangle{\pgfqpoint{1.432000in}{0.528000in}}{\pgfqpoint{3.696000in}{3.696000in}} %
\pgfusepath{clip}%
\pgfsetbuttcap%
\pgfsetroundjoin%
\definecolor{currentfill}{rgb}{0.279566,0.067836,0.391917}%
\pgfsetfillcolor{currentfill}%
\pgfsetlinewidth{0.000000pt}%
\definecolor{currentstroke}{rgb}{0.000000,0.000000,0.000000}%
\pgfsetstrokecolor{currentstroke}%
\pgfsetdash{}{0pt}%
\pgfpathmoveto{\pgfqpoint{4.199307in}{2.317839in}}%
\pgfpathlineto{\pgfqpoint{3.801461in}{2.516762in}}%
\pgfpathlineto{\pgfqpoint{3.801461in}{2.506845in}}%
\pgfpathlineto{\pgfqpoint{3.767742in}{2.538581in}}%
\pgfpathlineto{\pgfqpoint{3.813362in}{2.530647in}}%
\pgfpathlineto{\pgfqpoint{3.805428in}{2.524696in}}%
\pgfpathlineto{\pgfqpoint{4.203274in}{2.325773in}}%
\pgfpathlineto{\pgfqpoint{4.199307in}{2.317839in}}%
\pgfusepath{fill}%
\end{pgfscope}%
\begin{pgfscope}%
\pgfpathrectangle{\pgfqpoint{1.432000in}{0.528000in}}{\pgfqpoint{3.696000in}{3.696000in}} %
\pgfusepath{clip}%
\pgfsetbuttcap%
\pgfsetroundjoin%
\definecolor{currentfill}{rgb}{0.741388,0.873449,0.149561}%
\pgfsetfillcolor{currentfill}%
\pgfsetlinewidth{0.000000pt}%
\definecolor{currentstroke}{rgb}{0.000000,0.000000,0.000000}%
\pgfsetstrokecolor{currentstroke}%
\pgfsetdash{}{0pt}%
\pgfpathmoveto{\pgfqpoint{4.198830in}{2.318116in}}%
\pgfpathlineto{\pgfqpoint{3.906882in}{2.512748in}}%
\pgfpathlineto{\pgfqpoint{3.905652in}{2.502908in}}%
\pgfpathlineto{\pgfqpoint{3.876129in}{2.538581in}}%
\pgfpathlineto{\pgfqpoint{3.920413in}{2.525050in}}%
\pgfpathlineto{\pgfqpoint{3.911802in}{2.520129in}}%
\pgfpathlineto{\pgfqpoint{4.203751in}{2.325497in}}%
\pgfpathlineto{\pgfqpoint{4.198830in}{2.318116in}}%
\pgfusepath{fill}%
\end{pgfscope}%
\begin{pgfscope}%
\pgfpathrectangle{\pgfqpoint{1.432000in}{0.528000in}}{\pgfqpoint{3.696000in}{3.696000in}} %
\pgfusepath{clip}%
\pgfsetbuttcap%
\pgfsetroundjoin%
\definecolor{currentfill}{rgb}{0.273809,0.031497,0.358853}%
\pgfsetfillcolor{currentfill}%
\pgfsetlinewidth{0.000000pt}%
\definecolor{currentstroke}{rgb}{0.000000,0.000000,0.000000}%
\pgfsetstrokecolor{currentstroke}%
\pgfsetdash{}{0pt}%
\pgfpathmoveto{\pgfqpoint{4.307694in}{2.317839in}}%
\pgfpathlineto{\pgfqpoint{3.909848in}{2.516762in}}%
\pgfpathlineto{\pgfqpoint{3.909848in}{2.506845in}}%
\pgfpathlineto{\pgfqpoint{3.876129in}{2.538581in}}%
\pgfpathlineto{\pgfqpoint{3.921749in}{2.530647in}}%
\pgfpathlineto{\pgfqpoint{3.913815in}{2.524696in}}%
\pgfpathlineto{\pgfqpoint{4.311661in}{2.325773in}}%
\pgfpathlineto{\pgfqpoint{4.307694in}{2.317839in}}%
\pgfusepath{fill}%
\end{pgfscope}%
\begin{pgfscope}%
\pgfpathrectangle{\pgfqpoint{1.432000in}{0.528000in}}{\pgfqpoint{3.696000in}{3.696000in}} %
\pgfusepath{clip}%
\pgfsetbuttcap%
\pgfsetroundjoin%
\definecolor{currentfill}{rgb}{0.720391,0.870350,0.162603}%
\pgfsetfillcolor{currentfill}%
\pgfsetlinewidth{0.000000pt}%
\definecolor{currentstroke}{rgb}{0.000000,0.000000,0.000000}%
\pgfsetstrokecolor{currentstroke}%
\pgfsetdash{}{0pt}%
\pgfpathmoveto{\pgfqpoint{4.307217in}{2.318116in}}%
\pgfpathlineto{\pgfqpoint{4.015269in}{2.512748in}}%
\pgfpathlineto{\pgfqpoint{4.014039in}{2.502908in}}%
\pgfpathlineto{\pgfqpoint{3.984516in}{2.538581in}}%
\pgfpathlineto{\pgfqpoint{4.028800in}{2.525050in}}%
\pgfpathlineto{\pgfqpoint{4.020189in}{2.520129in}}%
\pgfpathlineto{\pgfqpoint{4.312138in}{2.325497in}}%
\pgfpathlineto{\pgfqpoint{4.307217in}{2.318116in}}%
\pgfusepath{fill}%
\end{pgfscope}%
\begin{pgfscope}%
\pgfpathrectangle{\pgfqpoint{1.432000in}{0.528000in}}{\pgfqpoint{3.696000in}{3.696000in}} %
\pgfusepath{clip}%
\pgfsetbuttcap%
\pgfsetroundjoin%
\definecolor{currentfill}{rgb}{0.208030,0.718701,0.472873}%
\pgfsetfillcolor{currentfill}%
\pgfsetlinewidth{0.000000pt}%
\definecolor{currentstroke}{rgb}{0.000000,0.000000,0.000000}%
\pgfsetstrokecolor{currentstroke}%
\pgfsetdash{}{0pt}%
\pgfpathmoveto{\pgfqpoint{4.415604in}{2.318116in}}%
\pgfpathlineto{\pgfqpoint{4.123656in}{2.512748in}}%
\pgfpathlineto{\pgfqpoint{4.122426in}{2.502908in}}%
\pgfpathlineto{\pgfqpoint{4.092903in}{2.538581in}}%
\pgfpathlineto{\pgfqpoint{4.137187in}{2.525050in}}%
\pgfpathlineto{\pgfqpoint{4.128576in}{2.520129in}}%
\pgfpathlineto{\pgfqpoint{4.420525in}{2.325497in}}%
\pgfpathlineto{\pgfqpoint{4.415604in}{2.318116in}}%
\pgfusepath{fill}%
\end{pgfscope}%
\begin{pgfscope}%
\pgfpathrectangle{\pgfqpoint{1.432000in}{0.528000in}}{\pgfqpoint{3.696000in}{3.696000in}} %
\pgfusepath{clip}%
\pgfsetbuttcap%
\pgfsetroundjoin%
\definecolor{currentfill}{rgb}{0.260571,0.246922,0.522828}%
\pgfsetfillcolor{currentfill}%
\pgfsetlinewidth{0.000000pt}%
\definecolor{currentstroke}{rgb}{0.000000,0.000000,0.000000}%
\pgfsetstrokecolor{currentstroke}%
\pgfsetdash{}{0pt}%
\pgfpathmoveto{\pgfqpoint{4.414928in}{2.318670in}}%
\pgfpathlineto{\pgfqpoint{4.226380in}{2.507219in}}%
\pgfpathlineto{\pgfqpoint{4.223243in}{2.497811in}}%
\pgfpathlineto{\pgfqpoint{4.201290in}{2.538581in}}%
\pgfpathlineto{\pgfqpoint{4.242060in}{2.516628in}}%
\pgfpathlineto{\pgfqpoint{4.232652in}{2.513491in}}%
\pgfpathlineto{\pgfqpoint{4.421201in}{2.324943in}}%
\pgfpathlineto{\pgfqpoint{4.414928in}{2.318670in}}%
\pgfusepath{fill}%
\end{pgfscope}%
\begin{pgfscope}%
\pgfpathrectangle{\pgfqpoint{1.432000in}{0.528000in}}{\pgfqpoint{3.696000in}{3.696000in}} %
\pgfusepath{clip}%
\pgfsetbuttcap%
\pgfsetroundjoin%
\definecolor{currentfill}{rgb}{0.273809,0.031497,0.358853}%
\pgfsetfillcolor{currentfill}%
\pgfsetlinewidth{0.000000pt}%
\definecolor{currentstroke}{rgb}{0.000000,0.000000,0.000000}%
\pgfsetstrokecolor{currentstroke}%
\pgfsetdash{}{0pt}%
\pgfpathmoveto{\pgfqpoint{4.414928in}{2.318670in}}%
\pgfpathlineto{\pgfqpoint{4.117993in}{2.615606in}}%
\pgfpathlineto{\pgfqpoint{4.114856in}{2.606198in}}%
\pgfpathlineto{\pgfqpoint{4.092903in}{2.646968in}}%
\pgfpathlineto{\pgfqpoint{4.133673in}{2.625015in}}%
\pgfpathlineto{\pgfqpoint{4.124265in}{2.621878in}}%
\pgfpathlineto{\pgfqpoint{4.421201in}{2.324943in}}%
\pgfpathlineto{\pgfqpoint{4.414928in}{2.318670in}}%
\pgfusepath{fill}%
\end{pgfscope}%
\begin{pgfscope}%
\pgfpathrectangle{\pgfqpoint{1.432000in}{0.528000in}}{\pgfqpoint{3.696000in}{3.696000in}} %
\pgfusepath{clip}%
\pgfsetbuttcap%
\pgfsetroundjoin%
\definecolor{currentfill}{rgb}{0.216210,0.351535,0.550627}%
\pgfsetfillcolor{currentfill}%
\pgfsetlinewidth{0.000000pt}%
\definecolor{currentstroke}{rgb}{0.000000,0.000000,0.000000}%
\pgfsetstrokecolor{currentstroke}%
\pgfsetdash{}{0pt}%
\pgfpathmoveto{\pgfqpoint{4.523991in}{2.318116in}}%
\pgfpathlineto{\pgfqpoint{4.232043in}{2.512748in}}%
\pgfpathlineto{\pgfqpoint{4.230813in}{2.502908in}}%
\pgfpathlineto{\pgfqpoint{4.201290in}{2.538581in}}%
\pgfpathlineto{\pgfqpoint{4.245574in}{2.525050in}}%
\pgfpathlineto{\pgfqpoint{4.236963in}{2.520129in}}%
\pgfpathlineto{\pgfqpoint{4.528912in}{2.325497in}}%
\pgfpathlineto{\pgfqpoint{4.523991in}{2.318116in}}%
\pgfusepath{fill}%
\end{pgfscope}%
\begin{pgfscope}%
\pgfpathrectangle{\pgfqpoint{1.432000in}{0.528000in}}{\pgfqpoint{3.696000in}{3.696000in}} %
\pgfusepath{clip}%
\pgfsetbuttcap%
\pgfsetroundjoin%
\definecolor{currentfill}{rgb}{0.121380,0.629492,0.531973}%
\pgfsetfillcolor{currentfill}%
\pgfsetlinewidth{0.000000pt}%
\definecolor{currentstroke}{rgb}{0.000000,0.000000,0.000000}%
\pgfsetstrokecolor{currentstroke}%
\pgfsetdash{}{0pt}%
\pgfpathmoveto{\pgfqpoint{4.523315in}{2.318670in}}%
\pgfpathlineto{\pgfqpoint{4.334767in}{2.507219in}}%
\pgfpathlineto{\pgfqpoint{4.331631in}{2.497811in}}%
\pgfpathlineto{\pgfqpoint{4.309677in}{2.538581in}}%
\pgfpathlineto{\pgfqpoint{4.350447in}{2.516628in}}%
\pgfpathlineto{\pgfqpoint{4.341039in}{2.513491in}}%
\pgfpathlineto{\pgfqpoint{4.529588in}{2.324943in}}%
\pgfpathlineto{\pgfqpoint{4.523315in}{2.318670in}}%
\pgfusepath{fill}%
\end{pgfscope}%
\begin{pgfscope}%
\pgfpathrectangle{\pgfqpoint{1.432000in}{0.528000in}}{\pgfqpoint{3.696000in}{3.696000in}} %
\pgfusepath{clip}%
\pgfsetbuttcap%
\pgfsetroundjoin%
\definecolor{currentfill}{rgb}{0.226397,0.728888,0.462789}%
\pgfsetfillcolor{currentfill}%
\pgfsetlinewidth{0.000000pt}%
\definecolor{currentstroke}{rgb}{0.000000,0.000000,0.000000}%
\pgfsetstrokecolor{currentstroke}%
\pgfsetdash{}{0pt}%
\pgfpathmoveto{\pgfqpoint{4.631703in}{2.318670in}}%
\pgfpathlineto{\pgfqpoint{4.443154in}{2.507219in}}%
\pgfpathlineto{\pgfqpoint{4.440018in}{2.497811in}}%
\pgfpathlineto{\pgfqpoint{4.418065in}{2.538581in}}%
\pgfpathlineto{\pgfqpoint{4.458835in}{2.516628in}}%
\pgfpathlineto{\pgfqpoint{4.449426in}{2.513491in}}%
\pgfpathlineto{\pgfqpoint{4.637975in}{2.324943in}}%
\pgfpathlineto{\pgfqpoint{4.631703in}{2.318670in}}%
\pgfusepath{fill}%
\end{pgfscope}%
\begin{pgfscope}%
\pgfpathrectangle{\pgfqpoint{1.432000in}{0.528000in}}{\pgfqpoint{3.696000in}{3.696000in}} %
\pgfusepath{clip}%
\pgfsetbuttcap%
\pgfsetroundjoin%
\definecolor{currentfill}{rgb}{0.277018,0.050344,0.375715}%
\pgfsetfillcolor{currentfill}%
\pgfsetlinewidth{0.000000pt}%
\definecolor{currentstroke}{rgb}{0.000000,0.000000,0.000000}%
\pgfsetstrokecolor{currentstroke}%
\pgfsetdash{}{0pt}%
\pgfpathmoveto{\pgfqpoint{4.630872in}{2.319823in}}%
\pgfpathlineto{\pgfqpoint{4.540336in}{2.500894in}}%
\pgfpathlineto{\pgfqpoint{4.534386in}{2.492961in}}%
\pgfpathlineto{\pgfqpoint{4.526452in}{2.538581in}}%
\pgfpathlineto{\pgfqpoint{4.558187in}{2.504861in}}%
\pgfpathlineto{\pgfqpoint{4.548270in}{2.504861in}}%
\pgfpathlineto{\pgfqpoint{4.638806in}{2.323790in}}%
\pgfpathlineto{\pgfqpoint{4.630872in}{2.319823in}}%
\pgfusepath{fill}%
\end{pgfscope}%
\begin{pgfscope}%
\pgfpathrectangle{\pgfqpoint{1.432000in}{0.528000in}}{\pgfqpoint{3.696000in}{3.696000in}} %
\pgfusepath{clip}%
\pgfsetbuttcap%
\pgfsetroundjoin%
\definecolor{currentfill}{rgb}{0.136408,0.541173,0.554483}%
\pgfsetfillcolor{currentfill}%
\pgfsetlinewidth{0.000000pt}%
\definecolor{currentstroke}{rgb}{0.000000,0.000000,0.000000}%
\pgfsetstrokecolor{currentstroke}%
\pgfsetdash{}{0pt}%
\pgfpathmoveto{\pgfqpoint{4.740090in}{2.318670in}}%
\pgfpathlineto{\pgfqpoint{4.551541in}{2.507219in}}%
\pgfpathlineto{\pgfqpoint{4.548405in}{2.497811in}}%
\pgfpathlineto{\pgfqpoint{4.526452in}{2.538581in}}%
\pgfpathlineto{\pgfqpoint{4.567222in}{2.516628in}}%
\pgfpathlineto{\pgfqpoint{4.557813in}{2.513491in}}%
\pgfpathlineto{\pgfqpoint{4.746362in}{2.324943in}}%
\pgfpathlineto{\pgfqpoint{4.740090in}{2.318670in}}%
\pgfusepath{fill}%
\end{pgfscope}%
\begin{pgfscope}%
\pgfpathrectangle{\pgfqpoint{1.432000in}{0.528000in}}{\pgfqpoint{3.696000in}{3.696000in}} %
\pgfusepath{clip}%
\pgfsetbuttcap%
\pgfsetroundjoin%
\definecolor{currentfill}{rgb}{0.235526,0.309527,0.542944}%
\pgfsetfillcolor{currentfill}%
\pgfsetlinewidth{0.000000pt}%
\definecolor{currentstroke}{rgb}{0.000000,0.000000,0.000000}%
\pgfsetstrokecolor{currentstroke}%
\pgfsetdash{}{0pt}%
\pgfpathmoveto{\pgfqpoint{4.739259in}{2.319823in}}%
\pgfpathlineto{\pgfqpoint{4.648723in}{2.500894in}}%
\pgfpathlineto{\pgfqpoint{4.642773in}{2.492961in}}%
\pgfpathlineto{\pgfqpoint{4.634839in}{2.538581in}}%
\pgfpathlineto{\pgfqpoint{4.666574in}{2.504861in}}%
\pgfpathlineto{\pgfqpoint{4.656657in}{2.504861in}}%
\pgfpathlineto{\pgfqpoint{4.747193in}{2.323790in}}%
\pgfpathlineto{\pgfqpoint{4.739259in}{2.319823in}}%
\pgfusepath{fill}%
\end{pgfscope}%
\begin{pgfscope}%
\pgfpathrectangle{\pgfqpoint{1.432000in}{0.528000in}}{\pgfqpoint{3.696000in}{3.696000in}} %
\pgfusepath{clip}%
\pgfsetbuttcap%
\pgfsetroundjoin%
\definecolor{currentfill}{rgb}{0.283187,0.125848,0.444960}%
\pgfsetfillcolor{currentfill}%
\pgfsetlinewidth{0.000000pt}%
\definecolor{currentstroke}{rgb}{0.000000,0.000000,0.000000}%
\pgfsetstrokecolor{currentstroke}%
\pgfsetdash{}{0pt}%
\pgfpathmoveto{\pgfqpoint{4.848477in}{2.318670in}}%
\pgfpathlineto{\pgfqpoint{4.659928in}{2.507219in}}%
\pgfpathlineto{\pgfqpoint{4.656792in}{2.497811in}}%
\pgfpathlineto{\pgfqpoint{4.634839in}{2.538581in}}%
\pgfpathlineto{\pgfqpoint{4.675609in}{2.516628in}}%
\pgfpathlineto{\pgfqpoint{4.666200in}{2.513491in}}%
\pgfpathlineto{\pgfqpoint{4.854749in}{2.324943in}}%
\pgfpathlineto{\pgfqpoint{4.848477in}{2.318670in}}%
\pgfusepath{fill}%
\end{pgfscope}%
\begin{pgfscope}%
\pgfpathrectangle{\pgfqpoint{1.432000in}{0.528000in}}{\pgfqpoint{3.696000in}{3.696000in}} %
\pgfusepath{clip}%
\pgfsetbuttcap%
\pgfsetroundjoin%
\definecolor{currentfill}{rgb}{0.121831,0.589055,0.545623}%
\pgfsetfillcolor{currentfill}%
\pgfsetlinewidth{0.000000pt}%
\definecolor{currentstroke}{rgb}{0.000000,0.000000,0.000000}%
\pgfsetstrokecolor{currentstroke}%
\pgfsetdash{}{0pt}%
\pgfpathmoveto{\pgfqpoint{4.847646in}{2.319823in}}%
\pgfpathlineto{\pgfqpoint{4.757110in}{2.500894in}}%
\pgfpathlineto{\pgfqpoint{4.751160in}{2.492961in}}%
\pgfpathlineto{\pgfqpoint{4.743226in}{2.538581in}}%
\pgfpathlineto{\pgfqpoint{4.774962in}{2.504861in}}%
\pgfpathlineto{\pgfqpoint{4.765044in}{2.504861in}}%
\pgfpathlineto{\pgfqpoint{4.855580in}{2.323790in}}%
\pgfpathlineto{\pgfqpoint{4.847646in}{2.319823in}}%
\pgfusepath{fill}%
\end{pgfscope}%
\begin{pgfscope}%
\pgfpathrectangle{\pgfqpoint{1.432000in}{0.528000in}}{\pgfqpoint{3.696000in}{3.696000in}} %
\pgfusepath{clip}%
\pgfsetbuttcap%
\pgfsetroundjoin%
\definecolor{currentfill}{rgb}{0.204903,0.375746,0.553533}%
\pgfsetfillcolor{currentfill}%
\pgfsetlinewidth{0.000000pt}%
\definecolor{currentstroke}{rgb}{0.000000,0.000000,0.000000}%
\pgfsetstrokecolor{currentstroke}%
\pgfsetdash{}{0pt}%
\pgfpathmoveto{\pgfqpoint{4.956033in}{2.319823in}}%
\pgfpathlineto{\pgfqpoint{4.865497in}{2.500894in}}%
\pgfpathlineto{\pgfqpoint{4.859547in}{2.492961in}}%
\pgfpathlineto{\pgfqpoint{4.851613in}{2.538581in}}%
\pgfpathlineto{\pgfqpoint{4.883349in}{2.504861in}}%
\pgfpathlineto{\pgfqpoint{4.873431in}{2.504861in}}%
\pgfpathlineto{\pgfqpoint{4.963967in}{2.323790in}}%
\pgfpathlineto{\pgfqpoint{4.956033in}{2.319823in}}%
\pgfusepath{fill}%
\end{pgfscope}%
\begin{pgfscope}%
\pgfpathrectangle{\pgfqpoint{1.432000in}{0.528000in}}{\pgfqpoint{3.696000in}{3.696000in}} %
\pgfusepath{clip}%
\pgfsetbuttcap%
\pgfsetroundjoin%
\definecolor{currentfill}{rgb}{0.201239,0.383670,0.554294}%
\pgfsetfillcolor{currentfill}%
\pgfsetlinewidth{0.000000pt}%
\definecolor{currentstroke}{rgb}{0.000000,0.000000,0.000000}%
\pgfsetstrokecolor{currentstroke}%
\pgfsetdash{}{0pt}%
\pgfpathmoveto{\pgfqpoint{4.955565in}{2.321806in}}%
\pgfpathlineto{\pgfqpoint{4.955565in}{2.498664in}}%
\pgfpathlineto{\pgfqpoint{4.946694in}{2.494229in}}%
\pgfpathlineto{\pgfqpoint{4.960000in}{2.538581in}}%
\pgfpathlineto{\pgfqpoint{4.973306in}{2.494229in}}%
\pgfpathlineto{\pgfqpoint{4.964435in}{2.498664in}}%
\pgfpathlineto{\pgfqpoint{4.964435in}{2.321806in}}%
\pgfpathlineto{\pgfqpoint{4.955565in}{2.321806in}}%
\pgfusepath{fill}%
\end{pgfscope}%
\begin{pgfscope}%
\pgfpathrectangle{\pgfqpoint{1.432000in}{0.528000in}}{\pgfqpoint{3.696000in}{3.696000in}} %
\pgfusepath{clip}%
\pgfsetbuttcap%
\pgfsetroundjoin%
\definecolor{currentfill}{rgb}{0.283229,0.120777,0.440584}%
\pgfsetfillcolor{currentfill}%
\pgfsetlinewidth{0.000000pt}%
\definecolor{currentstroke}{rgb}{0.000000,0.000000,0.000000}%
\pgfsetstrokecolor{currentstroke}%
\pgfsetdash{}{0pt}%
\pgfpathmoveto{\pgfqpoint{4.955565in}{2.321806in}}%
\pgfpathlineto{\pgfqpoint{4.955565in}{2.607051in}}%
\pgfpathlineto{\pgfqpoint{4.946694in}{2.602616in}}%
\pgfpathlineto{\pgfqpoint{4.960000in}{2.646968in}}%
\pgfpathlineto{\pgfqpoint{4.973306in}{2.602616in}}%
\pgfpathlineto{\pgfqpoint{4.964435in}{2.607051in}}%
\pgfpathlineto{\pgfqpoint{4.964435in}{2.321806in}}%
\pgfpathlineto{\pgfqpoint{4.955565in}{2.321806in}}%
\pgfusepath{fill}%
\end{pgfscope}%
\begin{pgfscope}%
\pgfpathrectangle{\pgfqpoint{1.432000in}{0.528000in}}{\pgfqpoint{3.696000in}{3.696000in}} %
\pgfusepath{clip}%
\pgfsetbuttcap%
\pgfsetroundjoin%
\definecolor{currentfill}{rgb}{0.239374,0.735588,0.455688}%
\pgfsetfillcolor{currentfill}%
\pgfsetlinewidth{0.000000pt}%
\definecolor{currentstroke}{rgb}{0.000000,0.000000,0.000000}%
\pgfsetstrokecolor{currentstroke}%
\pgfsetdash{}{0pt}%
\pgfpathmoveto{\pgfqpoint{1.595565in}{2.430194in}}%
\pgfpathlineto{\pgfqpoint{1.595565in}{2.498664in}}%
\pgfpathlineto{\pgfqpoint{1.586694in}{2.494229in}}%
\pgfpathlineto{\pgfqpoint{1.600000in}{2.538581in}}%
\pgfpathlineto{\pgfqpoint{1.613306in}{2.494229in}}%
\pgfpathlineto{\pgfqpoint{1.604435in}{2.498664in}}%
\pgfpathlineto{\pgfqpoint{1.604435in}{2.430194in}}%
\pgfpathlineto{\pgfqpoint{1.595565in}{2.430194in}}%
\pgfusepath{fill}%
\end{pgfscope}%
\begin{pgfscope}%
\pgfpathrectangle{\pgfqpoint{1.432000in}{0.528000in}}{\pgfqpoint{3.696000in}{3.696000in}} %
\pgfusepath{clip}%
\pgfsetbuttcap%
\pgfsetroundjoin%
\definecolor{currentfill}{rgb}{0.140210,0.665859,0.513427}%
\pgfsetfillcolor{currentfill}%
\pgfsetlinewidth{0.000000pt}%
\definecolor{currentstroke}{rgb}{0.000000,0.000000,0.000000}%
\pgfsetstrokecolor{currentstroke}%
\pgfsetdash{}{0pt}%
\pgfpathmoveto{\pgfqpoint{1.703952in}{2.430194in}}%
\pgfpathlineto{\pgfqpoint{1.703952in}{2.498664in}}%
\pgfpathlineto{\pgfqpoint{1.695081in}{2.494229in}}%
\pgfpathlineto{\pgfqpoint{1.708387in}{2.538581in}}%
\pgfpathlineto{\pgfqpoint{1.721693in}{2.494229in}}%
\pgfpathlineto{\pgfqpoint{1.712822in}{2.498664in}}%
\pgfpathlineto{\pgfqpoint{1.712822in}{2.430194in}}%
\pgfpathlineto{\pgfqpoint{1.703952in}{2.430194in}}%
\pgfusepath{fill}%
\end{pgfscope}%
\begin{pgfscope}%
\pgfpathrectangle{\pgfqpoint{1.432000in}{0.528000in}}{\pgfqpoint{3.696000in}{3.696000in}} %
\pgfusepath{clip}%
\pgfsetbuttcap%
\pgfsetroundjoin%
\definecolor{currentfill}{rgb}{0.216210,0.351535,0.550627}%
\pgfsetfillcolor{currentfill}%
\pgfsetlinewidth{0.000000pt}%
\definecolor{currentstroke}{rgb}{0.000000,0.000000,0.000000}%
\pgfsetstrokecolor{currentstroke}%
\pgfsetdash{}{0pt}%
\pgfpathmoveto{\pgfqpoint{1.813638in}{2.427057in}}%
\pgfpathlineto{\pgfqpoint{1.733476in}{2.507219in}}%
\pgfpathlineto{\pgfqpoint{1.730340in}{2.497811in}}%
\pgfpathlineto{\pgfqpoint{1.708387in}{2.538581in}}%
\pgfpathlineto{\pgfqpoint{1.749157in}{2.516628in}}%
\pgfpathlineto{\pgfqpoint{1.739749in}{2.513491in}}%
\pgfpathlineto{\pgfqpoint{1.819910in}{2.433330in}}%
\pgfpathlineto{\pgfqpoint{1.813638in}{2.427057in}}%
\pgfusepath{fill}%
\end{pgfscope}%
\begin{pgfscope}%
\pgfpathrectangle{\pgfqpoint{1.432000in}{0.528000in}}{\pgfqpoint{3.696000in}{3.696000in}} %
\pgfusepath{clip}%
\pgfsetbuttcap%
\pgfsetroundjoin%
\definecolor{currentfill}{rgb}{0.192357,0.403199,0.555836}%
\pgfsetfillcolor{currentfill}%
\pgfsetlinewidth{0.000000pt}%
\definecolor{currentstroke}{rgb}{0.000000,0.000000,0.000000}%
\pgfsetstrokecolor{currentstroke}%
\pgfsetdash{}{0pt}%
\pgfpathmoveto{\pgfqpoint{1.812339in}{2.430194in}}%
\pgfpathlineto{\pgfqpoint{1.812339in}{2.498664in}}%
\pgfpathlineto{\pgfqpoint{1.803469in}{2.494229in}}%
\pgfpathlineto{\pgfqpoint{1.816774in}{2.538581in}}%
\pgfpathlineto{\pgfqpoint{1.830080in}{2.494229in}}%
\pgfpathlineto{\pgfqpoint{1.821209in}{2.498664in}}%
\pgfpathlineto{\pgfqpoint{1.821209in}{2.430194in}}%
\pgfpathlineto{\pgfqpoint{1.812339in}{2.430194in}}%
\pgfusepath{fill}%
\end{pgfscope}%
\begin{pgfscope}%
\pgfpathrectangle{\pgfqpoint{1.432000in}{0.528000in}}{\pgfqpoint{3.696000in}{3.696000in}} %
\pgfusepath{clip}%
\pgfsetbuttcap%
\pgfsetroundjoin%
\definecolor{currentfill}{rgb}{0.140536,0.530132,0.555659}%
\pgfsetfillcolor{currentfill}%
\pgfsetlinewidth{0.000000pt}%
\definecolor{currentstroke}{rgb}{0.000000,0.000000,0.000000}%
\pgfsetstrokecolor{currentstroke}%
\pgfsetdash{}{0pt}%
\pgfpathmoveto{\pgfqpoint{1.922025in}{2.427057in}}%
\pgfpathlineto{\pgfqpoint{1.841863in}{2.507219in}}%
\pgfpathlineto{\pgfqpoint{1.838727in}{2.497811in}}%
\pgfpathlineto{\pgfqpoint{1.816774in}{2.538581in}}%
\pgfpathlineto{\pgfqpoint{1.857544in}{2.516628in}}%
\pgfpathlineto{\pgfqpoint{1.848136in}{2.513491in}}%
\pgfpathlineto{\pgfqpoint{1.928297in}{2.433330in}}%
\pgfpathlineto{\pgfqpoint{1.922025in}{2.427057in}}%
\pgfusepath{fill}%
\end{pgfscope}%
\begin{pgfscope}%
\pgfpathrectangle{\pgfqpoint{1.432000in}{0.528000in}}{\pgfqpoint{3.696000in}{3.696000in}} %
\pgfusepath{clip}%
\pgfsetbuttcap%
\pgfsetroundjoin%
\definecolor{currentfill}{rgb}{0.270595,0.214069,0.507052}%
\pgfsetfillcolor{currentfill}%
\pgfsetlinewidth{0.000000pt}%
\definecolor{currentstroke}{rgb}{0.000000,0.000000,0.000000}%
\pgfsetstrokecolor{currentstroke}%
\pgfsetdash{}{0pt}%
\pgfpathmoveto{\pgfqpoint{1.920726in}{2.430194in}}%
\pgfpathlineto{\pgfqpoint{1.920726in}{2.498664in}}%
\pgfpathlineto{\pgfqpoint{1.911856in}{2.494229in}}%
\pgfpathlineto{\pgfqpoint{1.925161in}{2.538581in}}%
\pgfpathlineto{\pgfqpoint{1.938467in}{2.494229in}}%
\pgfpathlineto{\pgfqpoint{1.929596in}{2.498664in}}%
\pgfpathlineto{\pgfqpoint{1.929596in}{2.430194in}}%
\pgfpathlineto{\pgfqpoint{1.920726in}{2.430194in}}%
\pgfusepath{fill}%
\end{pgfscope}%
\begin{pgfscope}%
\pgfpathrectangle{\pgfqpoint{1.432000in}{0.528000in}}{\pgfqpoint{3.696000in}{3.696000in}} %
\pgfusepath{clip}%
\pgfsetbuttcap%
\pgfsetroundjoin%
\definecolor{currentfill}{rgb}{0.126453,0.570633,0.549841}%
\pgfsetfillcolor{currentfill}%
\pgfsetlinewidth{0.000000pt}%
\definecolor{currentstroke}{rgb}{0.000000,0.000000,0.000000}%
\pgfsetstrokecolor{currentstroke}%
\pgfsetdash{}{0pt}%
\pgfpathmoveto{\pgfqpoint{2.030412in}{2.427057in}}%
\pgfpathlineto{\pgfqpoint{1.950251in}{2.507219in}}%
\pgfpathlineto{\pgfqpoint{1.947114in}{2.497811in}}%
\pgfpathlineto{\pgfqpoint{1.925161in}{2.538581in}}%
\pgfpathlineto{\pgfqpoint{1.965931in}{2.516628in}}%
\pgfpathlineto{\pgfqpoint{1.956523in}{2.513491in}}%
\pgfpathlineto{\pgfqpoint{2.036685in}{2.433330in}}%
\pgfpathlineto{\pgfqpoint{2.030412in}{2.427057in}}%
\pgfusepath{fill}%
\end{pgfscope}%
\begin{pgfscope}%
\pgfpathrectangle{\pgfqpoint{1.432000in}{0.528000in}}{\pgfqpoint{3.696000in}{3.696000in}} %
\pgfusepath{clip}%
\pgfsetbuttcap%
\pgfsetroundjoin%
\definecolor{currentfill}{rgb}{0.171176,0.452530,0.557965}%
\pgfsetfillcolor{currentfill}%
\pgfsetlinewidth{0.000000pt}%
\definecolor{currentstroke}{rgb}{0.000000,0.000000,0.000000}%
\pgfsetstrokecolor{currentstroke}%
\pgfsetdash{}{0pt}%
\pgfpathmoveto{\pgfqpoint{2.138799in}{2.427057in}}%
\pgfpathlineto{\pgfqpoint{2.058638in}{2.507219in}}%
\pgfpathlineto{\pgfqpoint{2.055502in}{2.497811in}}%
\pgfpathlineto{\pgfqpoint{2.033548in}{2.538581in}}%
\pgfpathlineto{\pgfqpoint{2.074318in}{2.516628in}}%
\pgfpathlineto{\pgfqpoint{2.064910in}{2.513491in}}%
\pgfpathlineto{\pgfqpoint{2.145072in}{2.433330in}}%
\pgfpathlineto{\pgfqpoint{2.138799in}{2.427057in}}%
\pgfusepath{fill}%
\end{pgfscope}%
\begin{pgfscope}%
\pgfpathrectangle{\pgfqpoint{1.432000in}{0.528000in}}{\pgfqpoint{3.696000in}{3.696000in}} %
\pgfusepath{clip}%
\pgfsetbuttcap%
\pgfsetroundjoin%
\definecolor{currentfill}{rgb}{0.260571,0.246922,0.522828}%
\pgfsetfillcolor{currentfill}%
\pgfsetlinewidth{0.000000pt}%
\definecolor{currentstroke}{rgb}{0.000000,0.000000,0.000000}%
\pgfsetstrokecolor{currentstroke}%
\pgfsetdash{}{0pt}%
\pgfpathmoveto{\pgfqpoint{2.247186in}{2.427057in}}%
\pgfpathlineto{\pgfqpoint{2.167025in}{2.507219in}}%
\pgfpathlineto{\pgfqpoint{2.163889in}{2.497811in}}%
\pgfpathlineto{\pgfqpoint{2.141935in}{2.538581in}}%
\pgfpathlineto{\pgfqpoint{2.182706in}{2.516628in}}%
\pgfpathlineto{\pgfqpoint{2.173297in}{2.513491in}}%
\pgfpathlineto{\pgfqpoint{2.253459in}{2.433330in}}%
\pgfpathlineto{\pgfqpoint{2.247186in}{2.427057in}}%
\pgfusepath{fill}%
\end{pgfscope}%
\begin{pgfscope}%
\pgfpathrectangle{\pgfqpoint{1.432000in}{0.528000in}}{\pgfqpoint{3.696000in}{3.696000in}} %
\pgfusepath{clip}%
\pgfsetbuttcap%
\pgfsetroundjoin%
\definecolor{currentfill}{rgb}{0.271305,0.019942,0.347269}%
\pgfsetfillcolor{currentfill}%
\pgfsetlinewidth{0.000000pt}%
\definecolor{currentstroke}{rgb}{0.000000,0.000000,0.000000}%
\pgfsetstrokecolor{currentstroke}%
\pgfsetdash{}{0pt}%
\pgfpathmoveto{\pgfqpoint{2.246356in}{2.428210in}}%
\pgfpathlineto{\pgfqpoint{2.155820in}{2.609282in}}%
\pgfpathlineto{\pgfqpoint{2.149869in}{2.601348in}}%
\pgfpathlineto{\pgfqpoint{2.141935in}{2.646968in}}%
\pgfpathlineto{\pgfqpoint{2.173671in}{2.613249in}}%
\pgfpathlineto{\pgfqpoint{2.163754in}{2.613249in}}%
\pgfpathlineto{\pgfqpoint{2.254290in}{2.432177in}}%
\pgfpathlineto{\pgfqpoint{2.246356in}{2.428210in}}%
\pgfusepath{fill}%
\end{pgfscope}%
\begin{pgfscope}%
\pgfpathrectangle{\pgfqpoint{1.432000in}{0.528000in}}{\pgfqpoint{3.696000in}{3.696000in}} %
\pgfusepath{clip}%
\pgfsetbuttcap%
\pgfsetroundjoin%
\definecolor{currentfill}{rgb}{0.283091,0.110553,0.431554}%
\pgfsetfillcolor{currentfill}%
\pgfsetlinewidth{0.000000pt}%
\definecolor{currentstroke}{rgb}{0.000000,0.000000,0.000000}%
\pgfsetstrokecolor{currentstroke}%
\pgfsetdash{}{0pt}%
\pgfpathmoveto{\pgfqpoint{2.356726in}{2.426227in}}%
\pgfpathlineto{\pgfqpoint{2.175655in}{2.516762in}}%
\pgfpathlineto{\pgfqpoint{2.175655in}{2.506845in}}%
\pgfpathlineto{\pgfqpoint{2.141935in}{2.538581in}}%
\pgfpathlineto{\pgfqpoint{2.187556in}{2.530647in}}%
\pgfpathlineto{\pgfqpoint{2.179622in}{2.524696in}}%
\pgfpathlineto{\pgfqpoint{2.360693in}{2.434161in}}%
\pgfpathlineto{\pgfqpoint{2.356726in}{2.426227in}}%
\pgfusepath{fill}%
\end{pgfscope}%
\begin{pgfscope}%
\pgfpathrectangle{\pgfqpoint{1.432000in}{0.528000in}}{\pgfqpoint{3.696000in}{3.696000in}} %
\pgfusepath{clip}%
\pgfsetbuttcap%
\pgfsetroundjoin%
\definecolor{currentfill}{rgb}{0.267004,0.004874,0.329415}%
\pgfsetfillcolor{currentfill}%
\pgfsetlinewidth{0.000000pt}%
\definecolor{currentstroke}{rgb}{0.000000,0.000000,0.000000}%
\pgfsetstrokecolor{currentstroke}%
\pgfsetdash{}{0pt}%
\pgfpathmoveto{\pgfqpoint{2.355574in}{2.427057in}}%
\pgfpathlineto{\pgfqpoint{2.275412in}{2.507219in}}%
\pgfpathlineto{\pgfqpoint{2.272276in}{2.497811in}}%
\pgfpathlineto{\pgfqpoint{2.250323in}{2.538581in}}%
\pgfpathlineto{\pgfqpoint{2.291093in}{2.516628in}}%
\pgfpathlineto{\pgfqpoint{2.281684in}{2.513491in}}%
\pgfpathlineto{\pgfqpoint{2.361846in}{2.433330in}}%
\pgfpathlineto{\pgfqpoint{2.355574in}{2.427057in}}%
\pgfusepath{fill}%
\end{pgfscope}%
\begin{pgfscope}%
\pgfpathrectangle{\pgfqpoint{1.432000in}{0.528000in}}{\pgfqpoint{3.696000in}{3.696000in}} %
\pgfusepath{clip}%
\pgfsetbuttcap%
\pgfsetroundjoin%
\definecolor{currentfill}{rgb}{0.283072,0.130895,0.449241}%
\pgfsetfillcolor{currentfill}%
\pgfsetlinewidth{0.000000pt}%
\definecolor{currentstroke}{rgb}{0.000000,0.000000,0.000000}%
\pgfsetstrokecolor{currentstroke}%
\pgfsetdash{}{0pt}%
\pgfpathmoveto{\pgfqpoint{2.465113in}{2.426227in}}%
\pgfpathlineto{\pgfqpoint{2.284042in}{2.516762in}}%
\pgfpathlineto{\pgfqpoint{2.284042in}{2.506845in}}%
\pgfpathlineto{\pgfqpoint{2.250323in}{2.538581in}}%
\pgfpathlineto{\pgfqpoint{2.295943in}{2.530647in}}%
\pgfpathlineto{\pgfqpoint{2.288009in}{2.524696in}}%
\pgfpathlineto{\pgfqpoint{2.469080in}{2.434161in}}%
\pgfpathlineto{\pgfqpoint{2.465113in}{2.426227in}}%
\pgfusepath{fill}%
\end{pgfscope}%
\begin{pgfscope}%
\pgfpathrectangle{\pgfqpoint{1.432000in}{0.528000in}}{\pgfqpoint{3.696000in}{3.696000in}} %
\pgfusepath{clip}%
\pgfsetbuttcap%
\pgfsetroundjoin%
\definecolor{currentfill}{rgb}{0.266580,0.228262,0.514349}%
\pgfsetfillcolor{currentfill}%
\pgfsetlinewidth{0.000000pt}%
\definecolor{currentstroke}{rgb}{0.000000,0.000000,0.000000}%
\pgfsetstrokecolor{currentstroke}%
\pgfsetdash{}{0pt}%
\pgfpathmoveto{\pgfqpoint{2.463961in}{2.427057in}}%
\pgfpathlineto{\pgfqpoint{2.275412in}{2.615606in}}%
\pgfpathlineto{\pgfqpoint{2.272276in}{2.606198in}}%
\pgfpathlineto{\pgfqpoint{2.250323in}{2.646968in}}%
\pgfpathlineto{\pgfqpoint{2.291093in}{2.625015in}}%
\pgfpathlineto{\pgfqpoint{2.281684in}{2.621878in}}%
\pgfpathlineto{\pgfqpoint{2.470233in}{2.433330in}}%
\pgfpathlineto{\pgfqpoint{2.463961in}{2.427057in}}%
\pgfusepath{fill}%
\end{pgfscope}%
\begin{pgfscope}%
\pgfpathrectangle{\pgfqpoint{1.432000in}{0.528000in}}{\pgfqpoint{3.696000in}{3.696000in}} %
\pgfusepath{clip}%
\pgfsetbuttcap%
\pgfsetroundjoin%
\definecolor{currentfill}{rgb}{0.221989,0.339161,0.548752}%
\pgfsetfillcolor{currentfill}%
\pgfsetlinewidth{0.000000pt}%
\definecolor{currentstroke}{rgb}{0.000000,0.000000,0.000000}%
\pgfsetstrokecolor{currentstroke}%
\pgfsetdash{}{0pt}%
\pgfpathmoveto{\pgfqpoint{2.573500in}{2.426227in}}%
\pgfpathlineto{\pgfqpoint{2.392429in}{2.516762in}}%
\pgfpathlineto{\pgfqpoint{2.392429in}{2.506845in}}%
\pgfpathlineto{\pgfqpoint{2.358710in}{2.538581in}}%
\pgfpathlineto{\pgfqpoint{2.404330in}{2.530647in}}%
\pgfpathlineto{\pgfqpoint{2.396396in}{2.524696in}}%
\pgfpathlineto{\pgfqpoint{2.577467in}{2.434161in}}%
\pgfpathlineto{\pgfqpoint{2.573500in}{2.426227in}}%
\pgfusepath{fill}%
\end{pgfscope}%
\begin{pgfscope}%
\pgfpathrectangle{\pgfqpoint{1.432000in}{0.528000in}}{\pgfqpoint{3.696000in}{3.696000in}} %
\pgfusepath{clip}%
\pgfsetbuttcap%
\pgfsetroundjoin%
\definecolor{currentfill}{rgb}{0.275191,0.194905,0.496005}%
\pgfsetfillcolor{currentfill}%
\pgfsetlinewidth{0.000000pt}%
\definecolor{currentstroke}{rgb}{0.000000,0.000000,0.000000}%
\pgfsetstrokecolor{currentstroke}%
\pgfsetdash{}{0pt}%
\pgfpathmoveto{\pgfqpoint{2.572348in}{2.427057in}}%
\pgfpathlineto{\pgfqpoint{2.383799in}{2.615606in}}%
\pgfpathlineto{\pgfqpoint{2.380663in}{2.606198in}}%
\pgfpathlineto{\pgfqpoint{2.358710in}{2.646968in}}%
\pgfpathlineto{\pgfqpoint{2.399480in}{2.625015in}}%
\pgfpathlineto{\pgfqpoint{2.390071in}{2.621878in}}%
\pgfpathlineto{\pgfqpoint{2.578620in}{2.433330in}}%
\pgfpathlineto{\pgfqpoint{2.572348in}{2.427057in}}%
\pgfusepath{fill}%
\end{pgfscope}%
\begin{pgfscope}%
\pgfpathrectangle{\pgfqpoint{1.432000in}{0.528000in}}{\pgfqpoint{3.696000in}{3.696000in}} %
\pgfusepath{clip}%
\pgfsetbuttcap%
\pgfsetroundjoin%
\definecolor{currentfill}{rgb}{0.241237,0.296485,0.539709}%
\pgfsetfillcolor{currentfill}%
\pgfsetlinewidth{0.000000pt}%
\definecolor{currentstroke}{rgb}{0.000000,0.000000,0.000000}%
\pgfsetstrokecolor{currentstroke}%
\pgfsetdash{}{0pt}%
\pgfpathmoveto{\pgfqpoint{2.681887in}{2.426227in}}%
\pgfpathlineto{\pgfqpoint{2.500816in}{2.516762in}}%
\pgfpathlineto{\pgfqpoint{2.500816in}{2.506845in}}%
\pgfpathlineto{\pgfqpoint{2.467097in}{2.538581in}}%
\pgfpathlineto{\pgfqpoint{2.512717in}{2.530647in}}%
\pgfpathlineto{\pgfqpoint{2.504783in}{2.524696in}}%
\pgfpathlineto{\pgfqpoint{2.685854in}{2.434161in}}%
\pgfpathlineto{\pgfqpoint{2.681887in}{2.426227in}}%
\pgfusepath{fill}%
\end{pgfscope}%
\begin{pgfscope}%
\pgfpathrectangle{\pgfqpoint{1.432000in}{0.528000in}}{\pgfqpoint{3.696000in}{3.696000in}} %
\pgfusepath{clip}%
\pgfsetbuttcap%
\pgfsetroundjoin%
\definecolor{currentfill}{rgb}{0.274952,0.037752,0.364543}%
\pgfsetfillcolor{currentfill}%
\pgfsetlinewidth{0.000000pt}%
\definecolor{currentstroke}{rgb}{0.000000,0.000000,0.000000}%
\pgfsetstrokecolor{currentstroke}%
\pgfsetdash{}{0pt}%
\pgfpathmoveto{\pgfqpoint{2.680735in}{2.427057in}}%
\pgfpathlineto{\pgfqpoint{2.600573in}{2.507219in}}%
\pgfpathlineto{\pgfqpoint{2.597437in}{2.497811in}}%
\pgfpathlineto{\pgfqpoint{2.575484in}{2.538581in}}%
\pgfpathlineto{\pgfqpoint{2.616254in}{2.516628in}}%
\pgfpathlineto{\pgfqpoint{2.606845in}{2.513491in}}%
\pgfpathlineto{\pgfqpoint{2.687007in}{2.433330in}}%
\pgfpathlineto{\pgfqpoint{2.680735in}{2.427057in}}%
\pgfusepath{fill}%
\end{pgfscope}%
\begin{pgfscope}%
\pgfpathrectangle{\pgfqpoint{1.432000in}{0.528000in}}{\pgfqpoint{3.696000in}{3.696000in}} %
\pgfusepath{clip}%
\pgfsetbuttcap%
\pgfsetroundjoin%
\definecolor{currentfill}{rgb}{0.281446,0.084320,0.407414}%
\pgfsetfillcolor{currentfill}%
\pgfsetlinewidth{0.000000pt}%
\definecolor{currentstroke}{rgb}{0.000000,0.000000,0.000000}%
\pgfsetstrokecolor{currentstroke}%
\pgfsetdash{}{0pt}%
\pgfpathmoveto{\pgfqpoint{2.680735in}{2.427057in}}%
\pgfpathlineto{\pgfqpoint{2.492186in}{2.615606in}}%
\pgfpathlineto{\pgfqpoint{2.489050in}{2.606198in}}%
\pgfpathlineto{\pgfqpoint{2.467097in}{2.646968in}}%
\pgfpathlineto{\pgfqpoint{2.507867in}{2.625015in}}%
\pgfpathlineto{\pgfqpoint{2.498458in}{2.621878in}}%
\pgfpathlineto{\pgfqpoint{2.687007in}{2.433330in}}%
\pgfpathlineto{\pgfqpoint{2.680735in}{2.427057in}}%
\pgfusepath{fill}%
\end{pgfscope}%
\begin{pgfscope}%
\pgfpathrectangle{\pgfqpoint{1.432000in}{0.528000in}}{\pgfqpoint{3.696000in}{3.696000in}} %
\pgfusepath{clip}%
\pgfsetbuttcap%
\pgfsetroundjoin%
\definecolor{currentfill}{rgb}{0.274952,0.037752,0.364543}%
\pgfsetfillcolor{currentfill}%
\pgfsetlinewidth{0.000000pt}%
\definecolor{currentstroke}{rgb}{0.000000,0.000000,0.000000}%
\pgfsetstrokecolor{currentstroke}%
\pgfsetdash{}{0pt}%
\pgfpathmoveto{\pgfqpoint{2.790275in}{2.426227in}}%
\pgfpathlineto{\pgfqpoint{2.609203in}{2.516762in}}%
\pgfpathlineto{\pgfqpoint{2.609203in}{2.506845in}}%
\pgfpathlineto{\pgfqpoint{2.575484in}{2.538581in}}%
\pgfpathlineto{\pgfqpoint{2.621104in}{2.530647in}}%
\pgfpathlineto{\pgfqpoint{2.613170in}{2.524696in}}%
\pgfpathlineto{\pgfqpoint{2.794242in}{2.434161in}}%
\pgfpathlineto{\pgfqpoint{2.790275in}{2.426227in}}%
\pgfusepath{fill}%
\end{pgfscope}%
\begin{pgfscope}%
\pgfpathrectangle{\pgfqpoint{1.432000in}{0.528000in}}{\pgfqpoint{3.696000in}{3.696000in}} %
\pgfusepath{clip}%
\pgfsetbuttcap%
\pgfsetroundjoin%
\definecolor{currentfill}{rgb}{0.210503,0.363727,0.552206}%
\pgfsetfillcolor{currentfill}%
\pgfsetlinewidth{0.000000pt}%
\definecolor{currentstroke}{rgb}{0.000000,0.000000,0.000000}%
\pgfsetstrokecolor{currentstroke}%
\pgfsetdash{}{0pt}%
\pgfpathmoveto{\pgfqpoint{2.789122in}{2.427057in}}%
\pgfpathlineto{\pgfqpoint{2.708960in}{2.507219in}}%
\pgfpathlineto{\pgfqpoint{2.705824in}{2.497811in}}%
\pgfpathlineto{\pgfqpoint{2.683871in}{2.538581in}}%
\pgfpathlineto{\pgfqpoint{2.724641in}{2.516628in}}%
\pgfpathlineto{\pgfqpoint{2.715233in}{2.513491in}}%
\pgfpathlineto{\pgfqpoint{2.795394in}{2.433330in}}%
\pgfpathlineto{\pgfqpoint{2.789122in}{2.427057in}}%
\pgfusepath{fill}%
\end{pgfscope}%
\begin{pgfscope}%
\pgfpathrectangle{\pgfqpoint{1.432000in}{0.528000in}}{\pgfqpoint{3.696000in}{3.696000in}} %
\pgfusepath{clip}%
\pgfsetbuttcap%
\pgfsetroundjoin%
\definecolor{currentfill}{rgb}{0.276022,0.044167,0.370164}%
\pgfsetfillcolor{currentfill}%
\pgfsetlinewidth{0.000000pt}%
\definecolor{currentstroke}{rgb}{0.000000,0.000000,0.000000}%
\pgfsetstrokecolor{currentstroke}%
\pgfsetdash{}{0pt}%
\pgfpathmoveto{\pgfqpoint{2.788291in}{2.428210in}}%
\pgfpathlineto{\pgfqpoint{2.697755in}{2.609282in}}%
\pgfpathlineto{\pgfqpoint{2.691805in}{2.601348in}}%
\pgfpathlineto{\pgfqpoint{2.683871in}{2.646968in}}%
\pgfpathlineto{\pgfqpoint{2.715607in}{2.613249in}}%
\pgfpathlineto{\pgfqpoint{2.705689in}{2.613249in}}%
\pgfpathlineto{\pgfqpoint{2.796225in}{2.432177in}}%
\pgfpathlineto{\pgfqpoint{2.788291in}{2.428210in}}%
\pgfusepath{fill}%
\end{pgfscope}%
\begin{pgfscope}%
\pgfpathrectangle{\pgfqpoint{1.432000in}{0.528000in}}{\pgfqpoint{3.696000in}{3.696000in}} %
\pgfusepath{clip}%
\pgfsetbuttcap%
\pgfsetroundjoin%
\definecolor{currentfill}{rgb}{0.125394,0.574318,0.549086}%
\pgfsetfillcolor{currentfill}%
\pgfsetlinewidth{0.000000pt}%
\definecolor{currentstroke}{rgb}{0.000000,0.000000,0.000000}%
\pgfsetstrokecolor{currentstroke}%
\pgfsetdash{}{0pt}%
\pgfpathmoveto{\pgfqpoint{2.896678in}{2.428210in}}%
\pgfpathlineto{\pgfqpoint{2.806142in}{2.609282in}}%
\pgfpathlineto{\pgfqpoint{2.800192in}{2.601348in}}%
\pgfpathlineto{\pgfqpoint{2.792258in}{2.646968in}}%
\pgfpathlineto{\pgfqpoint{2.823994in}{2.613249in}}%
\pgfpathlineto{\pgfqpoint{2.814076in}{2.613249in}}%
\pgfpathlineto{\pgfqpoint{2.904612in}{2.432177in}}%
\pgfpathlineto{\pgfqpoint{2.896678in}{2.428210in}}%
\pgfusepath{fill}%
\end{pgfscope}%
\begin{pgfscope}%
\pgfpathrectangle{\pgfqpoint{1.432000in}{0.528000in}}{\pgfqpoint{3.696000in}{3.696000in}} %
\pgfusepath{clip}%
\pgfsetbuttcap%
\pgfsetroundjoin%
\definecolor{currentfill}{rgb}{0.144759,0.519093,0.556572}%
\pgfsetfillcolor{currentfill}%
\pgfsetlinewidth{0.000000pt}%
\definecolor{currentstroke}{rgb}{0.000000,0.000000,0.000000}%
\pgfsetstrokecolor{currentstroke}%
\pgfsetdash{}{0pt}%
\pgfpathmoveto{\pgfqpoint{3.005065in}{2.428210in}}%
\pgfpathlineto{\pgfqpoint{2.914530in}{2.609282in}}%
\pgfpathlineto{\pgfqpoint{2.908579in}{2.601348in}}%
\pgfpathlineto{\pgfqpoint{2.900645in}{2.646968in}}%
\pgfpathlineto{\pgfqpoint{2.932381in}{2.613249in}}%
\pgfpathlineto{\pgfqpoint{2.922463in}{2.613249in}}%
\pgfpathlineto{\pgfqpoint{3.012999in}{2.432177in}}%
\pgfpathlineto{\pgfqpoint{3.005065in}{2.428210in}}%
\pgfusepath{fill}%
\end{pgfscope}%
\begin{pgfscope}%
\pgfpathrectangle{\pgfqpoint{1.432000in}{0.528000in}}{\pgfqpoint{3.696000in}{3.696000in}} %
\pgfusepath{clip}%
\pgfsetbuttcap%
\pgfsetroundjoin%
\definecolor{currentfill}{rgb}{0.246811,0.283237,0.535941}%
\pgfsetfillcolor{currentfill}%
\pgfsetlinewidth{0.000000pt}%
\definecolor{currentstroke}{rgb}{0.000000,0.000000,0.000000}%
\pgfsetstrokecolor{currentstroke}%
\pgfsetdash{}{0pt}%
\pgfpathmoveto{\pgfqpoint{3.113452in}{2.428210in}}%
\pgfpathlineto{\pgfqpoint{3.022917in}{2.609282in}}%
\pgfpathlineto{\pgfqpoint{3.016966in}{2.601348in}}%
\pgfpathlineto{\pgfqpoint{3.009032in}{2.646968in}}%
\pgfpathlineto{\pgfqpoint{3.040768in}{2.613249in}}%
\pgfpathlineto{\pgfqpoint{3.030851in}{2.613249in}}%
\pgfpathlineto{\pgfqpoint{3.121386in}{2.432177in}}%
\pgfpathlineto{\pgfqpoint{3.113452in}{2.428210in}}%
\pgfusepath{fill}%
\end{pgfscope}%
\begin{pgfscope}%
\pgfpathrectangle{\pgfqpoint{1.432000in}{0.528000in}}{\pgfqpoint{3.696000in}{3.696000in}} %
\pgfusepath{clip}%
\pgfsetbuttcap%
\pgfsetroundjoin%
\definecolor{currentfill}{rgb}{0.280267,0.073417,0.397163}%
\pgfsetfillcolor{currentfill}%
\pgfsetlinewidth{0.000000pt}%
\definecolor{currentstroke}{rgb}{0.000000,0.000000,0.000000}%
\pgfsetstrokecolor{currentstroke}%
\pgfsetdash{}{0pt}%
\pgfpathmoveto{\pgfqpoint{3.222670in}{2.427057in}}%
\pgfpathlineto{\pgfqpoint{3.034122in}{2.615606in}}%
\pgfpathlineto{\pgfqpoint{3.030985in}{2.606198in}}%
\pgfpathlineto{\pgfqpoint{3.009032in}{2.646968in}}%
\pgfpathlineto{\pgfqpoint{3.049802in}{2.625015in}}%
\pgfpathlineto{\pgfqpoint{3.040394in}{2.621878in}}%
\pgfpathlineto{\pgfqpoint{3.228943in}{2.433330in}}%
\pgfpathlineto{\pgfqpoint{3.222670in}{2.427057in}}%
\pgfusepath{fill}%
\end{pgfscope}%
\begin{pgfscope}%
\pgfpathrectangle{\pgfqpoint{1.432000in}{0.528000in}}{\pgfqpoint{3.696000in}{3.696000in}} %
\pgfusepath{clip}%
\pgfsetbuttcap%
\pgfsetroundjoin%
\definecolor{currentfill}{rgb}{0.282327,0.094955,0.417331}%
\pgfsetfillcolor{currentfill}%
\pgfsetlinewidth{0.000000pt}%
\definecolor{currentstroke}{rgb}{0.000000,0.000000,0.000000}%
\pgfsetstrokecolor{currentstroke}%
\pgfsetdash{}{0pt}%
\pgfpathmoveto{\pgfqpoint{3.221839in}{2.428210in}}%
\pgfpathlineto{\pgfqpoint{3.131304in}{2.609282in}}%
\pgfpathlineto{\pgfqpoint{3.125353in}{2.601348in}}%
\pgfpathlineto{\pgfqpoint{3.117419in}{2.646968in}}%
\pgfpathlineto{\pgfqpoint{3.149155in}{2.613249in}}%
\pgfpathlineto{\pgfqpoint{3.139238in}{2.613249in}}%
\pgfpathlineto{\pgfqpoint{3.229773in}{2.432177in}}%
\pgfpathlineto{\pgfqpoint{3.221839in}{2.428210in}}%
\pgfusepath{fill}%
\end{pgfscope}%
\begin{pgfscope}%
\pgfpathrectangle{\pgfqpoint{1.432000in}{0.528000in}}{\pgfqpoint{3.696000in}{3.696000in}} %
\pgfusepath{clip}%
\pgfsetbuttcap%
\pgfsetroundjoin%
\definecolor{currentfill}{rgb}{0.165117,0.467423,0.558141}%
\pgfsetfillcolor{currentfill}%
\pgfsetlinewidth{0.000000pt}%
\definecolor{currentstroke}{rgb}{0.000000,0.000000,0.000000}%
\pgfsetstrokecolor{currentstroke}%
\pgfsetdash{}{0pt}%
\pgfpathmoveto{\pgfqpoint{3.331057in}{2.427057in}}%
\pgfpathlineto{\pgfqpoint{3.142509in}{2.615606in}}%
\pgfpathlineto{\pgfqpoint{3.139372in}{2.606198in}}%
\pgfpathlineto{\pgfqpoint{3.117419in}{2.646968in}}%
\pgfpathlineto{\pgfqpoint{3.158189in}{2.625015in}}%
\pgfpathlineto{\pgfqpoint{3.148781in}{2.621878in}}%
\pgfpathlineto{\pgfqpoint{3.337330in}{2.433330in}}%
\pgfpathlineto{\pgfqpoint{3.331057in}{2.427057in}}%
\pgfusepath{fill}%
\end{pgfscope}%
\begin{pgfscope}%
\pgfpathrectangle{\pgfqpoint{1.432000in}{0.528000in}}{\pgfqpoint{3.696000in}{3.696000in}} %
\pgfusepath{clip}%
\pgfsetbuttcap%
\pgfsetroundjoin%
\definecolor{currentfill}{rgb}{0.277018,0.050344,0.375715}%
\pgfsetfillcolor{currentfill}%
\pgfsetlinewidth{0.000000pt}%
\definecolor{currentstroke}{rgb}{0.000000,0.000000,0.000000}%
\pgfsetstrokecolor{currentstroke}%
\pgfsetdash{}{0pt}%
\pgfpathmoveto{\pgfqpoint{3.330503in}{2.427733in}}%
\pgfpathlineto{\pgfqpoint{3.135871in}{2.719682in}}%
\pgfpathlineto{\pgfqpoint{3.130950in}{2.711071in}}%
\pgfpathlineto{\pgfqpoint{3.117419in}{2.755355in}}%
\pgfpathlineto{\pgfqpoint{3.153092in}{2.725832in}}%
\pgfpathlineto{\pgfqpoint{3.143252in}{2.724602in}}%
\pgfpathlineto{\pgfqpoint{3.337884in}{2.432654in}}%
\pgfpathlineto{\pgfqpoint{3.330503in}{2.427733in}}%
\pgfusepath{fill}%
\end{pgfscope}%
\begin{pgfscope}%
\pgfpathrectangle{\pgfqpoint{1.432000in}{0.528000in}}{\pgfqpoint{3.696000in}{3.696000in}} %
\pgfusepath{clip}%
\pgfsetbuttcap%
\pgfsetroundjoin%
\definecolor{currentfill}{rgb}{0.163625,0.471133,0.558148}%
\pgfsetfillcolor{currentfill}%
\pgfsetlinewidth{0.000000pt}%
\definecolor{currentstroke}{rgb}{0.000000,0.000000,0.000000}%
\pgfsetstrokecolor{currentstroke}%
\pgfsetdash{}{0pt}%
\pgfpathmoveto{\pgfqpoint{3.439444in}{2.427057in}}%
\pgfpathlineto{\pgfqpoint{3.250896in}{2.615606in}}%
\pgfpathlineto{\pgfqpoint{3.247760in}{2.606198in}}%
\pgfpathlineto{\pgfqpoint{3.225806in}{2.646968in}}%
\pgfpathlineto{\pgfqpoint{3.266577in}{2.625015in}}%
\pgfpathlineto{\pgfqpoint{3.257168in}{2.621878in}}%
\pgfpathlineto{\pgfqpoint{3.445717in}{2.433330in}}%
\pgfpathlineto{\pgfqpoint{3.439444in}{2.427057in}}%
\pgfusepath{fill}%
\end{pgfscope}%
\begin{pgfscope}%
\pgfpathrectangle{\pgfqpoint{1.432000in}{0.528000in}}{\pgfqpoint{3.696000in}{3.696000in}} %
\pgfusepath{clip}%
\pgfsetbuttcap%
\pgfsetroundjoin%
\definecolor{currentfill}{rgb}{0.195860,0.395433,0.555276}%
\pgfsetfillcolor{currentfill}%
\pgfsetlinewidth{0.000000pt}%
\definecolor{currentstroke}{rgb}{0.000000,0.000000,0.000000}%
\pgfsetstrokecolor{currentstroke}%
\pgfsetdash{}{0pt}%
\pgfpathmoveto{\pgfqpoint{3.548508in}{2.426503in}}%
\pgfpathlineto{\pgfqpoint{3.256559in}{2.621136in}}%
\pgfpathlineto{\pgfqpoint{3.255329in}{2.611295in}}%
\pgfpathlineto{\pgfqpoint{3.225806in}{2.646968in}}%
\pgfpathlineto{\pgfqpoint{3.270090in}{2.633437in}}%
\pgfpathlineto{\pgfqpoint{3.261479in}{2.628516in}}%
\pgfpathlineto{\pgfqpoint{3.553428in}{2.433884in}}%
\pgfpathlineto{\pgfqpoint{3.548508in}{2.426503in}}%
\pgfusepath{fill}%
\end{pgfscope}%
\begin{pgfscope}%
\pgfpathrectangle{\pgfqpoint{1.432000in}{0.528000in}}{\pgfqpoint{3.696000in}{3.696000in}} %
\pgfusepath{clip}%
\pgfsetbuttcap%
\pgfsetroundjoin%
\definecolor{currentfill}{rgb}{0.208030,0.718701,0.472873}%
\pgfsetfillcolor{currentfill}%
\pgfsetlinewidth{0.000000pt}%
\definecolor{currentstroke}{rgb}{0.000000,0.000000,0.000000}%
\pgfsetstrokecolor{currentstroke}%
\pgfsetdash{}{0pt}%
\pgfpathmoveto{\pgfqpoint{3.656895in}{2.426503in}}%
\pgfpathlineto{\pgfqpoint{3.364946in}{2.621136in}}%
\pgfpathlineto{\pgfqpoint{3.363716in}{2.611295in}}%
\pgfpathlineto{\pgfqpoint{3.334194in}{2.646968in}}%
\pgfpathlineto{\pgfqpoint{3.378477in}{2.633437in}}%
\pgfpathlineto{\pgfqpoint{3.369867in}{2.628516in}}%
\pgfpathlineto{\pgfqpoint{3.661815in}{2.433884in}}%
\pgfpathlineto{\pgfqpoint{3.656895in}{2.426503in}}%
\pgfusepath{fill}%
\end{pgfscope}%
\begin{pgfscope}%
\pgfpathrectangle{\pgfqpoint{1.432000in}{0.528000in}}{\pgfqpoint{3.696000in}{3.696000in}} %
\pgfusepath{clip}%
\pgfsetbuttcap%
\pgfsetroundjoin%
\definecolor{currentfill}{rgb}{0.352360,0.783011,0.392636}%
\pgfsetfillcolor{currentfill}%
\pgfsetlinewidth{0.000000pt}%
\definecolor{currentstroke}{rgb}{0.000000,0.000000,0.000000}%
\pgfsetstrokecolor{currentstroke}%
\pgfsetdash{}{0pt}%
\pgfpathmoveto{\pgfqpoint{3.765282in}{2.426503in}}%
\pgfpathlineto{\pgfqpoint{3.473333in}{2.621136in}}%
\pgfpathlineto{\pgfqpoint{3.472103in}{2.611295in}}%
\pgfpathlineto{\pgfqpoint{3.442581in}{2.646968in}}%
\pgfpathlineto{\pgfqpoint{3.486864in}{2.633437in}}%
\pgfpathlineto{\pgfqpoint{3.478254in}{2.628516in}}%
\pgfpathlineto{\pgfqpoint{3.770202in}{2.433884in}}%
\pgfpathlineto{\pgfqpoint{3.765282in}{2.426503in}}%
\pgfusepath{fill}%
\end{pgfscope}%
\begin{pgfscope}%
\pgfpathrectangle{\pgfqpoint{1.432000in}{0.528000in}}{\pgfqpoint{3.696000in}{3.696000in}} %
\pgfusepath{clip}%
\pgfsetbuttcap%
\pgfsetroundjoin%
\definecolor{currentfill}{rgb}{0.278791,0.062145,0.386592}%
\pgfsetfillcolor{currentfill}%
\pgfsetlinewidth{0.000000pt}%
\definecolor{currentstroke}{rgb}{0.000000,0.000000,0.000000}%
\pgfsetstrokecolor{currentstroke}%
\pgfsetdash{}{0pt}%
\pgfpathmoveto{\pgfqpoint{3.764606in}{2.427057in}}%
\pgfpathlineto{\pgfqpoint{3.467670in}{2.723993in}}%
\pgfpathlineto{\pgfqpoint{3.464534in}{2.714585in}}%
\pgfpathlineto{\pgfqpoint{3.442581in}{2.755355in}}%
\pgfpathlineto{\pgfqpoint{3.483351in}{2.733402in}}%
\pgfpathlineto{\pgfqpoint{3.473942in}{2.730266in}}%
\pgfpathlineto{\pgfqpoint{3.770878in}{2.433330in}}%
\pgfpathlineto{\pgfqpoint{3.764606in}{2.427057in}}%
\pgfusepath{fill}%
\end{pgfscope}%
\begin{pgfscope}%
\pgfpathrectangle{\pgfqpoint{1.432000in}{0.528000in}}{\pgfqpoint{3.696000in}{3.696000in}} %
\pgfusepath{clip}%
\pgfsetbuttcap%
\pgfsetroundjoin%
\definecolor{currentfill}{rgb}{0.377779,0.791781,0.377939}%
\pgfsetfillcolor{currentfill}%
\pgfsetlinewidth{0.000000pt}%
\definecolor{currentstroke}{rgb}{0.000000,0.000000,0.000000}%
\pgfsetstrokecolor{currentstroke}%
\pgfsetdash{}{0pt}%
\pgfpathmoveto{\pgfqpoint{3.873669in}{2.426503in}}%
\pgfpathlineto{\pgfqpoint{3.581720in}{2.621136in}}%
\pgfpathlineto{\pgfqpoint{3.580490in}{2.611295in}}%
\pgfpathlineto{\pgfqpoint{3.550968in}{2.646968in}}%
\pgfpathlineto{\pgfqpoint{3.595251in}{2.633437in}}%
\pgfpathlineto{\pgfqpoint{3.586641in}{2.628516in}}%
\pgfpathlineto{\pgfqpoint{3.878589in}{2.433884in}}%
\pgfpathlineto{\pgfqpoint{3.873669in}{2.426503in}}%
\pgfusepath{fill}%
\end{pgfscope}%
\begin{pgfscope}%
\pgfpathrectangle{\pgfqpoint{1.432000in}{0.528000in}}{\pgfqpoint{3.696000in}{3.696000in}} %
\pgfusepath{clip}%
\pgfsetbuttcap%
\pgfsetroundjoin%
\definecolor{currentfill}{rgb}{0.267004,0.004874,0.329415}%
\pgfsetfillcolor{currentfill}%
\pgfsetlinewidth{0.000000pt}%
\definecolor{currentstroke}{rgb}{0.000000,0.000000,0.000000}%
\pgfsetstrokecolor{currentstroke}%
\pgfsetdash{}{0pt}%
\pgfpathmoveto{\pgfqpoint{3.982533in}{2.426227in}}%
\pgfpathlineto{\pgfqpoint{3.584687in}{2.625149in}}%
\pgfpathlineto{\pgfqpoint{3.584687in}{2.615232in}}%
\pgfpathlineto{\pgfqpoint{3.550968in}{2.646968in}}%
\pgfpathlineto{\pgfqpoint{3.596588in}{2.639034in}}%
\pgfpathlineto{\pgfqpoint{3.588654in}{2.633083in}}%
\pgfpathlineto{\pgfqpoint{3.986500in}{2.434161in}}%
\pgfpathlineto{\pgfqpoint{3.982533in}{2.426227in}}%
\pgfusepath{fill}%
\end{pgfscope}%
\begin{pgfscope}%
\pgfpathrectangle{\pgfqpoint{1.432000in}{0.528000in}}{\pgfqpoint{3.696000in}{3.696000in}} %
\pgfusepath{clip}%
\pgfsetbuttcap%
\pgfsetroundjoin%
\definecolor{currentfill}{rgb}{0.616293,0.852709,0.230052}%
\pgfsetfillcolor{currentfill}%
\pgfsetlinewidth{0.000000pt}%
\definecolor{currentstroke}{rgb}{0.000000,0.000000,0.000000}%
\pgfsetstrokecolor{currentstroke}%
\pgfsetdash{}{0pt}%
\pgfpathmoveto{\pgfqpoint{3.982056in}{2.426503in}}%
\pgfpathlineto{\pgfqpoint{3.690107in}{2.621136in}}%
\pgfpathlineto{\pgfqpoint{3.688877in}{2.611295in}}%
\pgfpathlineto{\pgfqpoint{3.659355in}{2.646968in}}%
\pgfpathlineto{\pgfqpoint{3.703639in}{2.633437in}}%
\pgfpathlineto{\pgfqpoint{3.695028in}{2.628516in}}%
\pgfpathlineto{\pgfqpoint{3.986976in}{2.433884in}}%
\pgfpathlineto{\pgfqpoint{3.982056in}{2.426503in}}%
\pgfusepath{fill}%
\end{pgfscope}%
\begin{pgfscope}%
\pgfpathrectangle{\pgfqpoint{1.432000in}{0.528000in}}{\pgfqpoint{3.696000in}{3.696000in}} %
\pgfusepath{clip}%
\pgfsetbuttcap%
\pgfsetroundjoin%
\definecolor{currentfill}{rgb}{0.866013,0.889868,0.095953}%
\pgfsetfillcolor{currentfill}%
\pgfsetlinewidth{0.000000pt}%
\definecolor{currentstroke}{rgb}{0.000000,0.000000,0.000000}%
\pgfsetstrokecolor{currentstroke}%
\pgfsetdash{}{0pt}%
\pgfpathmoveto{\pgfqpoint{4.090443in}{2.426503in}}%
\pgfpathlineto{\pgfqpoint{3.798495in}{2.621136in}}%
\pgfpathlineto{\pgfqpoint{3.797264in}{2.611295in}}%
\pgfpathlineto{\pgfqpoint{3.767742in}{2.646968in}}%
\pgfpathlineto{\pgfqpoint{3.812026in}{2.633437in}}%
\pgfpathlineto{\pgfqpoint{3.803415in}{2.628516in}}%
\pgfpathlineto{\pgfqpoint{4.095363in}{2.433884in}}%
\pgfpathlineto{\pgfqpoint{4.090443in}{2.426503in}}%
\pgfusepath{fill}%
\end{pgfscope}%
\begin{pgfscope}%
\pgfpathrectangle{\pgfqpoint{1.432000in}{0.528000in}}{\pgfqpoint{3.696000in}{3.696000in}} %
\pgfusepath{clip}%
\pgfsetbuttcap%
\pgfsetroundjoin%
\definecolor{currentfill}{rgb}{0.269944,0.014625,0.341379}%
\pgfsetfillcolor{currentfill}%
\pgfsetlinewidth{0.000000pt}%
\definecolor{currentstroke}{rgb}{0.000000,0.000000,0.000000}%
\pgfsetstrokecolor{currentstroke}%
\pgfsetdash{}{0pt}%
\pgfpathmoveto{\pgfqpoint{4.199307in}{2.426227in}}%
\pgfpathlineto{\pgfqpoint{3.801461in}{2.625149in}}%
\pgfpathlineto{\pgfqpoint{3.801461in}{2.615232in}}%
\pgfpathlineto{\pgfqpoint{3.767742in}{2.646968in}}%
\pgfpathlineto{\pgfqpoint{3.813362in}{2.639034in}}%
\pgfpathlineto{\pgfqpoint{3.805428in}{2.633083in}}%
\pgfpathlineto{\pgfqpoint{4.203274in}{2.434161in}}%
\pgfpathlineto{\pgfqpoint{4.199307in}{2.426227in}}%
\pgfusepath{fill}%
\end{pgfscope}%
\begin{pgfscope}%
\pgfpathrectangle{\pgfqpoint{1.432000in}{0.528000in}}{\pgfqpoint{3.696000in}{3.696000in}} %
\pgfusepath{clip}%
\pgfsetbuttcap%
\pgfsetroundjoin%
\definecolor{currentfill}{rgb}{0.896320,0.893616,0.096335}%
\pgfsetfillcolor{currentfill}%
\pgfsetlinewidth{0.000000pt}%
\definecolor{currentstroke}{rgb}{0.000000,0.000000,0.000000}%
\pgfsetstrokecolor{currentstroke}%
\pgfsetdash{}{0pt}%
\pgfpathmoveto{\pgfqpoint{4.198830in}{2.426503in}}%
\pgfpathlineto{\pgfqpoint{3.906882in}{2.621136in}}%
\pgfpathlineto{\pgfqpoint{3.905652in}{2.611295in}}%
\pgfpathlineto{\pgfqpoint{3.876129in}{2.646968in}}%
\pgfpathlineto{\pgfqpoint{3.920413in}{2.633437in}}%
\pgfpathlineto{\pgfqpoint{3.911802in}{2.628516in}}%
\pgfpathlineto{\pgfqpoint{4.203751in}{2.433884in}}%
\pgfpathlineto{\pgfqpoint{4.198830in}{2.426503in}}%
\pgfusepath{fill}%
\end{pgfscope}%
\begin{pgfscope}%
\pgfpathrectangle{\pgfqpoint{1.432000in}{0.528000in}}{\pgfqpoint{3.696000in}{3.696000in}} %
\pgfusepath{clip}%
\pgfsetbuttcap%
\pgfsetroundjoin%
\definecolor{currentfill}{rgb}{0.993248,0.906157,0.143936}%
\pgfsetfillcolor{currentfill}%
\pgfsetlinewidth{0.000000pt}%
\definecolor{currentstroke}{rgb}{0.000000,0.000000,0.000000}%
\pgfsetstrokecolor{currentstroke}%
\pgfsetdash{}{0pt}%
\pgfpathmoveto{\pgfqpoint{4.307217in}{2.426503in}}%
\pgfpathlineto{\pgfqpoint{4.015269in}{2.621136in}}%
\pgfpathlineto{\pgfqpoint{4.014039in}{2.611295in}}%
\pgfpathlineto{\pgfqpoint{3.984516in}{2.646968in}}%
\pgfpathlineto{\pgfqpoint{4.028800in}{2.633437in}}%
\pgfpathlineto{\pgfqpoint{4.020189in}{2.628516in}}%
\pgfpathlineto{\pgfqpoint{4.312138in}{2.433884in}}%
\pgfpathlineto{\pgfqpoint{4.307217in}{2.426503in}}%
\pgfusepath{fill}%
\end{pgfscope}%
\begin{pgfscope}%
\pgfpathrectangle{\pgfqpoint{1.432000in}{0.528000in}}{\pgfqpoint{3.696000in}{3.696000in}} %
\pgfusepath{clip}%
\pgfsetbuttcap%
\pgfsetroundjoin%
\definecolor{currentfill}{rgb}{0.226397,0.728888,0.462789}%
\pgfsetfillcolor{currentfill}%
\pgfsetlinewidth{0.000000pt}%
\definecolor{currentstroke}{rgb}{0.000000,0.000000,0.000000}%
\pgfsetstrokecolor{currentstroke}%
\pgfsetdash{}{0pt}%
\pgfpathmoveto{\pgfqpoint{4.415604in}{2.426503in}}%
\pgfpathlineto{\pgfqpoint{4.123656in}{2.621136in}}%
\pgfpathlineto{\pgfqpoint{4.122426in}{2.611295in}}%
\pgfpathlineto{\pgfqpoint{4.092903in}{2.646968in}}%
\pgfpathlineto{\pgfqpoint{4.137187in}{2.633437in}}%
\pgfpathlineto{\pgfqpoint{4.128576in}{2.628516in}}%
\pgfpathlineto{\pgfqpoint{4.420525in}{2.433884in}}%
\pgfpathlineto{\pgfqpoint{4.415604in}{2.426503in}}%
\pgfusepath{fill}%
\end{pgfscope}%
\begin{pgfscope}%
\pgfpathrectangle{\pgfqpoint{1.432000in}{0.528000in}}{\pgfqpoint{3.696000in}{3.696000in}} %
\pgfusepath{clip}%
\pgfsetbuttcap%
\pgfsetroundjoin%
\definecolor{currentfill}{rgb}{0.278012,0.180367,0.486697}%
\pgfsetfillcolor{currentfill}%
\pgfsetlinewidth{0.000000pt}%
\definecolor{currentstroke}{rgb}{0.000000,0.000000,0.000000}%
\pgfsetstrokecolor{currentstroke}%
\pgfsetdash{}{0pt}%
\pgfpathmoveto{\pgfqpoint{4.414928in}{2.427057in}}%
\pgfpathlineto{\pgfqpoint{4.226380in}{2.615606in}}%
\pgfpathlineto{\pgfqpoint{4.223243in}{2.606198in}}%
\pgfpathlineto{\pgfqpoint{4.201290in}{2.646968in}}%
\pgfpathlineto{\pgfqpoint{4.242060in}{2.625015in}}%
\pgfpathlineto{\pgfqpoint{4.232652in}{2.621878in}}%
\pgfpathlineto{\pgfqpoint{4.421201in}{2.433330in}}%
\pgfpathlineto{\pgfqpoint{4.414928in}{2.427057in}}%
\pgfusepath{fill}%
\end{pgfscope}%
\begin{pgfscope}%
\pgfpathrectangle{\pgfqpoint{1.432000in}{0.528000in}}{\pgfqpoint{3.696000in}{3.696000in}} %
\pgfusepath{clip}%
\pgfsetbuttcap%
\pgfsetroundjoin%
\definecolor{currentfill}{rgb}{0.283197,0.115680,0.436115}%
\pgfsetfillcolor{currentfill}%
\pgfsetlinewidth{0.000000pt}%
\definecolor{currentstroke}{rgb}{0.000000,0.000000,0.000000}%
\pgfsetstrokecolor{currentstroke}%
\pgfsetdash{}{0pt}%
\pgfpathmoveto{\pgfqpoint{4.414928in}{2.427057in}}%
\pgfpathlineto{\pgfqpoint{4.117993in}{2.723993in}}%
\pgfpathlineto{\pgfqpoint{4.114856in}{2.714585in}}%
\pgfpathlineto{\pgfqpoint{4.092903in}{2.755355in}}%
\pgfpathlineto{\pgfqpoint{4.133673in}{2.733402in}}%
\pgfpathlineto{\pgfqpoint{4.124265in}{2.730266in}}%
\pgfpathlineto{\pgfqpoint{4.421201in}{2.433330in}}%
\pgfpathlineto{\pgfqpoint{4.414928in}{2.427057in}}%
\pgfusepath{fill}%
\end{pgfscope}%
\begin{pgfscope}%
\pgfpathrectangle{\pgfqpoint{1.432000in}{0.528000in}}{\pgfqpoint{3.696000in}{3.696000in}} %
\pgfusepath{clip}%
\pgfsetbuttcap%
\pgfsetroundjoin%
\definecolor{currentfill}{rgb}{0.203063,0.379716,0.553925}%
\pgfsetfillcolor{currentfill}%
\pgfsetlinewidth{0.000000pt}%
\definecolor{currentstroke}{rgb}{0.000000,0.000000,0.000000}%
\pgfsetstrokecolor{currentstroke}%
\pgfsetdash{}{0pt}%
\pgfpathmoveto{\pgfqpoint{4.523991in}{2.426503in}}%
\pgfpathlineto{\pgfqpoint{4.232043in}{2.621136in}}%
\pgfpathlineto{\pgfqpoint{4.230813in}{2.611295in}}%
\pgfpathlineto{\pgfqpoint{4.201290in}{2.646968in}}%
\pgfpathlineto{\pgfqpoint{4.245574in}{2.633437in}}%
\pgfpathlineto{\pgfqpoint{4.236963in}{2.628516in}}%
\pgfpathlineto{\pgfqpoint{4.528912in}{2.433884in}}%
\pgfpathlineto{\pgfqpoint{4.523991in}{2.426503in}}%
\pgfusepath{fill}%
\end{pgfscope}%
\begin{pgfscope}%
\pgfpathrectangle{\pgfqpoint{1.432000in}{0.528000in}}{\pgfqpoint{3.696000in}{3.696000in}} %
\pgfusepath{clip}%
\pgfsetbuttcap%
\pgfsetroundjoin%
\definecolor{currentfill}{rgb}{0.125394,0.574318,0.549086}%
\pgfsetfillcolor{currentfill}%
\pgfsetlinewidth{0.000000pt}%
\definecolor{currentstroke}{rgb}{0.000000,0.000000,0.000000}%
\pgfsetstrokecolor{currentstroke}%
\pgfsetdash{}{0pt}%
\pgfpathmoveto{\pgfqpoint{4.523315in}{2.427057in}}%
\pgfpathlineto{\pgfqpoint{4.334767in}{2.615606in}}%
\pgfpathlineto{\pgfqpoint{4.331631in}{2.606198in}}%
\pgfpathlineto{\pgfqpoint{4.309677in}{2.646968in}}%
\pgfpathlineto{\pgfqpoint{4.350447in}{2.625015in}}%
\pgfpathlineto{\pgfqpoint{4.341039in}{2.621878in}}%
\pgfpathlineto{\pgfqpoint{4.529588in}{2.433330in}}%
\pgfpathlineto{\pgfqpoint{4.523315in}{2.427057in}}%
\pgfusepath{fill}%
\end{pgfscope}%
\begin{pgfscope}%
\pgfpathrectangle{\pgfqpoint{1.432000in}{0.528000in}}{\pgfqpoint{3.696000in}{3.696000in}} %
\pgfusepath{clip}%
\pgfsetbuttcap%
\pgfsetroundjoin%
\definecolor{currentfill}{rgb}{0.269944,0.014625,0.341379}%
\pgfsetfillcolor{currentfill}%
\pgfsetlinewidth{0.000000pt}%
\definecolor{currentstroke}{rgb}{0.000000,0.000000,0.000000}%
\pgfsetstrokecolor{currentstroke}%
\pgfsetdash{}{0pt}%
\pgfpathmoveto{\pgfqpoint{4.523315in}{2.427057in}}%
\pgfpathlineto{\pgfqpoint{4.226380in}{2.723993in}}%
\pgfpathlineto{\pgfqpoint{4.223243in}{2.714585in}}%
\pgfpathlineto{\pgfqpoint{4.201290in}{2.755355in}}%
\pgfpathlineto{\pgfqpoint{4.242060in}{2.733402in}}%
\pgfpathlineto{\pgfqpoint{4.232652in}{2.730266in}}%
\pgfpathlineto{\pgfqpoint{4.529588in}{2.433330in}}%
\pgfpathlineto{\pgfqpoint{4.523315in}{2.427057in}}%
\pgfusepath{fill}%
\end{pgfscope}%
\begin{pgfscope}%
\pgfpathrectangle{\pgfqpoint{1.432000in}{0.528000in}}{\pgfqpoint{3.696000in}{3.696000in}} %
\pgfusepath{clip}%
\pgfsetbuttcap%
\pgfsetroundjoin%
\definecolor{currentfill}{rgb}{0.266941,0.748751,0.440573}%
\pgfsetfillcolor{currentfill}%
\pgfsetlinewidth{0.000000pt}%
\definecolor{currentstroke}{rgb}{0.000000,0.000000,0.000000}%
\pgfsetstrokecolor{currentstroke}%
\pgfsetdash{}{0pt}%
\pgfpathmoveto{\pgfqpoint{4.631703in}{2.427057in}}%
\pgfpathlineto{\pgfqpoint{4.443154in}{2.615606in}}%
\pgfpathlineto{\pgfqpoint{4.440018in}{2.606198in}}%
\pgfpathlineto{\pgfqpoint{4.418065in}{2.646968in}}%
\pgfpathlineto{\pgfqpoint{4.458835in}{2.625015in}}%
\pgfpathlineto{\pgfqpoint{4.449426in}{2.621878in}}%
\pgfpathlineto{\pgfqpoint{4.637975in}{2.433330in}}%
\pgfpathlineto{\pgfqpoint{4.631703in}{2.427057in}}%
\pgfusepath{fill}%
\end{pgfscope}%
\begin{pgfscope}%
\pgfpathrectangle{\pgfqpoint{1.432000in}{0.528000in}}{\pgfqpoint{3.696000in}{3.696000in}} %
\pgfusepath{clip}%
\pgfsetbuttcap%
\pgfsetroundjoin%
\definecolor{currentfill}{rgb}{0.269944,0.014625,0.341379}%
\pgfsetfillcolor{currentfill}%
\pgfsetlinewidth{0.000000pt}%
\definecolor{currentstroke}{rgb}{0.000000,0.000000,0.000000}%
\pgfsetstrokecolor{currentstroke}%
\pgfsetdash{}{0pt}%
\pgfpathmoveto{\pgfqpoint{4.630872in}{2.428210in}}%
\pgfpathlineto{\pgfqpoint{4.540336in}{2.609282in}}%
\pgfpathlineto{\pgfqpoint{4.534386in}{2.601348in}}%
\pgfpathlineto{\pgfqpoint{4.526452in}{2.646968in}}%
\pgfpathlineto{\pgfqpoint{4.558187in}{2.613249in}}%
\pgfpathlineto{\pgfqpoint{4.548270in}{2.613249in}}%
\pgfpathlineto{\pgfqpoint{4.638806in}{2.432177in}}%
\pgfpathlineto{\pgfqpoint{4.630872in}{2.428210in}}%
\pgfusepath{fill}%
\end{pgfscope}%
\begin{pgfscope}%
\pgfpathrectangle{\pgfqpoint{1.432000in}{0.528000in}}{\pgfqpoint{3.696000in}{3.696000in}} %
\pgfusepath{clip}%
\pgfsetbuttcap%
\pgfsetroundjoin%
\definecolor{currentfill}{rgb}{0.154815,0.493313,0.557840}%
\pgfsetfillcolor{currentfill}%
\pgfsetlinewidth{0.000000pt}%
\definecolor{currentstroke}{rgb}{0.000000,0.000000,0.000000}%
\pgfsetstrokecolor{currentstroke}%
\pgfsetdash{}{0pt}%
\pgfpathmoveto{\pgfqpoint{4.740090in}{2.427057in}}%
\pgfpathlineto{\pgfqpoint{4.551541in}{2.615606in}}%
\pgfpathlineto{\pgfqpoint{4.548405in}{2.606198in}}%
\pgfpathlineto{\pgfqpoint{4.526452in}{2.646968in}}%
\pgfpathlineto{\pgfqpoint{4.567222in}{2.625015in}}%
\pgfpathlineto{\pgfqpoint{4.557813in}{2.621878in}}%
\pgfpathlineto{\pgfqpoint{4.746362in}{2.433330in}}%
\pgfpathlineto{\pgfqpoint{4.740090in}{2.427057in}}%
\pgfusepath{fill}%
\end{pgfscope}%
\begin{pgfscope}%
\pgfpathrectangle{\pgfqpoint{1.432000in}{0.528000in}}{\pgfqpoint{3.696000in}{3.696000in}} %
\pgfusepath{clip}%
\pgfsetbuttcap%
\pgfsetroundjoin%
\definecolor{currentfill}{rgb}{0.199430,0.387607,0.554642}%
\pgfsetfillcolor{currentfill}%
\pgfsetlinewidth{0.000000pt}%
\definecolor{currentstroke}{rgb}{0.000000,0.000000,0.000000}%
\pgfsetstrokecolor{currentstroke}%
\pgfsetdash{}{0pt}%
\pgfpathmoveto{\pgfqpoint{4.739259in}{2.428210in}}%
\pgfpathlineto{\pgfqpoint{4.648723in}{2.609282in}}%
\pgfpathlineto{\pgfqpoint{4.642773in}{2.601348in}}%
\pgfpathlineto{\pgfqpoint{4.634839in}{2.646968in}}%
\pgfpathlineto{\pgfqpoint{4.666574in}{2.613249in}}%
\pgfpathlineto{\pgfqpoint{4.656657in}{2.613249in}}%
\pgfpathlineto{\pgfqpoint{4.747193in}{2.432177in}}%
\pgfpathlineto{\pgfqpoint{4.739259in}{2.428210in}}%
\pgfusepath{fill}%
\end{pgfscope}%
\begin{pgfscope}%
\pgfpathrectangle{\pgfqpoint{1.432000in}{0.528000in}}{\pgfqpoint{3.696000in}{3.696000in}} %
\pgfusepath{clip}%
\pgfsetbuttcap%
\pgfsetroundjoin%
\definecolor{currentfill}{rgb}{0.136408,0.541173,0.554483}%
\pgfsetfillcolor{currentfill}%
\pgfsetlinewidth{0.000000pt}%
\definecolor{currentstroke}{rgb}{0.000000,0.000000,0.000000}%
\pgfsetstrokecolor{currentstroke}%
\pgfsetdash{}{0pt}%
\pgfpathmoveto{\pgfqpoint{4.847646in}{2.428210in}}%
\pgfpathlineto{\pgfqpoint{4.757110in}{2.609282in}}%
\pgfpathlineto{\pgfqpoint{4.751160in}{2.601348in}}%
\pgfpathlineto{\pgfqpoint{4.743226in}{2.646968in}}%
\pgfpathlineto{\pgfqpoint{4.774962in}{2.613249in}}%
\pgfpathlineto{\pgfqpoint{4.765044in}{2.613249in}}%
\pgfpathlineto{\pgfqpoint{4.855580in}{2.432177in}}%
\pgfpathlineto{\pgfqpoint{4.847646in}{2.428210in}}%
\pgfusepath{fill}%
\end{pgfscope}%
\begin{pgfscope}%
\pgfpathrectangle{\pgfqpoint{1.432000in}{0.528000in}}{\pgfqpoint{3.696000in}{3.696000in}} %
\pgfusepath{clip}%
\pgfsetbuttcap%
\pgfsetroundjoin%
\definecolor{currentfill}{rgb}{0.274952,0.037752,0.364543}%
\pgfsetfillcolor{currentfill}%
\pgfsetlinewidth{0.000000pt}%
\definecolor{currentstroke}{rgb}{0.000000,0.000000,0.000000}%
\pgfsetstrokecolor{currentstroke}%
\pgfsetdash{}{0pt}%
\pgfpathmoveto{\pgfqpoint{4.847405in}{2.428791in}}%
\pgfpathlineto{\pgfqpoint{4.751641in}{2.716084in}}%
\pgfpathlineto{\pgfqpoint{4.744628in}{2.709071in}}%
\pgfpathlineto{\pgfqpoint{4.743226in}{2.755355in}}%
\pgfpathlineto{\pgfqpoint{4.769874in}{2.717486in}}%
\pgfpathlineto{\pgfqpoint{4.760056in}{2.718889in}}%
\pgfpathlineto{\pgfqpoint{4.855821in}{2.431596in}}%
\pgfpathlineto{\pgfqpoint{4.847405in}{2.428791in}}%
\pgfusepath{fill}%
\end{pgfscope}%
\begin{pgfscope}%
\pgfpathrectangle{\pgfqpoint{1.432000in}{0.528000in}}{\pgfqpoint{3.696000in}{3.696000in}} %
\pgfusepath{clip}%
\pgfsetbuttcap%
\pgfsetroundjoin%
\definecolor{currentfill}{rgb}{0.255645,0.260703,0.528312}%
\pgfsetfillcolor{currentfill}%
\pgfsetlinewidth{0.000000pt}%
\definecolor{currentstroke}{rgb}{0.000000,0.000000,0.000000}%
\pgfsetstrokecolor{currentstroke}%
\pgfsetdash{}{0pt}%
\pgfpathmoveto{\pgfqpoint{4.956033in}{2.428210in}}%
\pgfpathlineto{\pgfqpoint{4.865497in}{2.609282in}}%
\pgfpathlineto{\pgfqpoint{4.859547in}{2.601348in}}%
\pgfpathlineto{\pgfqpoint{4.851613in}{2.646968in}}%
\pgfpathlineto{\pgfqpoint{4.883349in}{2.613249in}}%
\pgfpathlineto{\pgfqpoint{4.873431in}{2.613249in}}%
\pgfpathlineto{\pgfqpoint{4.963967in}{2.432177in}}%
\pgfpathlineto{\pgfqpoint{4.956033in}{2.428210in}}%
\pgfusepath{fill}%
\end{pgfscope}%
\begin{pgfscope}%
\pgfpathrectangle{\pgfqpoint{1.432000in}{0.528000in}}{\pgfqpoint{3.696000in}{3.696000in}} %
\pgfusepath{clip}%
\pgfsetbuttcap%
\pgfsetroundjoin%
\definecolor{currentfill}{rgb}{0.179019,0.433756,0.557430}%
\pgfsetfillcolor{currentfill}%
\pgfsetlinewidth{0.000000pt}%
\definecolor{currentstroke}{rgb}{0.000000,0.000000,0.000000}%
\pgfsetstrokecolor{currentstroke}%
\pgfsetdash{}{0pt}%
\pgfpathmoveto{\pgfqpoint{4.955565in}{2.430194in}}%
\pgfpathlineto{\pgfqpoint{4.955565in}{2.607051in}}%
\pgfpathlineto{\pgfqpoint{4.946694in}{2.602616in}}%
\pgfpathlineto{\pgfqpoint{4.960000in}{2.646968in}}%
\pgfpathlineto{\pgfqpoint{4.973306in}{2.602616in}}%
\pgfpathlineto{\pgfqpoint{4.964435in}{2.607051in}}%
\pgfpathlineto{\pgfqpoint{4.964435in}{2.430194in}}%
\pgfpathlineto{\pgfqpoint{4.955565in}{2.430194in}}%
\pgfusepath{fill}%
\end{pgfscope}%
\begin{pgfscope}%
\pgfpathrectangle{\pgfqpoint{1.432000in}{0.528000in}}{\pgfqpoint{3.696000in}{3.696000in}} %
\pgfusepath{clip}%
\pgfsetbuttcap%
\pgfsetroundjoin%
\definecolor{currentfill}{rgb}{0.271305,0.019942,0.347269}%
\pgfsetfillcolor{currentfill}%
\pgfsetlinewidth{0.000000pt}%
\definecolor{currentstroke}{rgb}{0.000000,0.000000,0.000000}%
\pgfsetstrokecolor{currentstroke}%
\pgfsetdash{}{0pt}%
\pgfpathmoveto{\pgfqpoint{4.955792in}{2.428791in}}%
\pgfpathlineto{\pgfqpoint{4.860028in}{2.716084in}}%
\pgfpathlineto{\pgfqpoint{4.853015in}{2.709071in}}%
\pgfpathlineto{\pgfqpoint{4.851613in}{2.755355in}}%
\pgfpathlineto{\pgfqpoint{4.878261in}{2.717486in}}%
\pgfpathlineto{\pgfqpoint{4.868443in}{2.718889in}}%
\pgfpathlineto{\pgfqpoint{4.964208in}{2.431596in}}%
\pgfpathlineto{\pgfqpoint{4.955792in}{2.428791in}}%
\pgfusepath{fill}%
\end{pgfscope}%
\begin{pgfscope}%
\pgfpathrectangle{\pgfqpoint{1.432000in}{0.528000in}}{\pgfqpoint{3.696000in}{3.696000in}} %
\pgfusepath{clip}%
\pgfsetbuttcap%
\pgfsetroundjoin%
\definecolor{currentfill}{rgb}{0.277018,0.050344,0.375715}%
\pgfsetfillcolor{currentfill}%
\pgfsetlinewidth{0.000000pt}%
\definecolor{currentstroke}{rgb}{0.000000,0.000000,0.000000}%
\pgfsetstrokecolor{currentstroke}%
\pgfsetdash{}{0pt}%
\pgfpathmoveto{\pgfqpoint{4.955565in}{2.430194in}}%
\pgfpathlineto{\pgfqpoint{4.955565in}{2.715438in}}%
\pgfpathlineto{\pgfqpoint{4.946694in}{2.711003in}}%
\pgfpathlineto{\pgfqpoint{4.960000in}{2.755355in}}%
\pgfpathlineto{\pgfqpoint{4.973306in}{2.711003in}}%
\pgfpathlineto{\pgfqpoint{4.964435in}{2.715438in}}%
\pgfpathlineto{\pgfqpoint{4.964435in}{2.430194in}}%
\pgfpathlineto{\pgfqpoint{4.955565in}{2.430194in}}%
\pgfusepath{fill}%
\end{pgfscope}%
\begin{pgfscope}%
\pgfpathrectangle{\pgfqpoint{1.432000in}{0.528000in}}{\pgfqpoint{3.696000in}{3.696000in}} %
\pgfusepath{clip}%
\pgfsetbuttcap%
\pgfsetroundjoin%
\definecolor{currentfill}{rgb}{0.278826,0.175490,0.483397}%
\pgfsetfillcolor{currentfill}%
\pgfsetlinewidth{0.000000pt}%
\definecolor{currentstroke}{rgb}{0.000000,0.000000,0.000000}%
\pgfsetstrokecolor{currentstroke}%
\pgfsetdash{}{0pt}%
\pgfpathmoveto{\pgfqpoint{1.604435in}{2.538581in}}%
\pgfpathlineto{\pgfqpoint{1.602218in}{2.542422in}}%
\pgfpathlineto{\pgfqpoint{1.597782in}{2.542422in}}%
\pgfpathlineto{\pgfqpoint{1.595565in}{2.538581in}}%
\pgfpathlineto{\pgfqpoint{1.597782in}{2.534740in}}%
\pgfpathlineto{\pgfqpoint{1.602218in}{2.534740in}}%
\pgfpathlineto{\pgfqpoint{1.604435in}{2.538581in}}%
\pgfpathlineto{\pgfqpoint{1.602218in}{2.542422in}}%
\pgfusepath{fill}%
\end{pgfscope}%
\begin{pgfscope}%
\pgfpathrectangle{\pgfqpoint{1.432000in}{0.528000in}}{\pgfqpoint{3.696000in}{3.696000in}} %
\pgfusepath{clip}%
\pgfsetbuttcap%
\pgfsetroundjoin%
\definecolor{currentfill}{rgb}{0.188923,0.410910,0.556326}%
\pgfsetfillcolor{currentfill}%
\pgfsetlinewidth{0.000000pt}%
\definecolor{currentstroke}{rgb}{0.000000,0.000000,0.000000}%
\pgfsetstrokecolor{currentstroke}%
\pgfsetdash{}{0pt}%
\pgfpathmoveto{\pgfqpoint{1.595565in}{2.538581in}}%
\pgfpathlineto{\pgfqpoint{1.595565in}{2.607051in}}%
\pgfpathlineto{\pgfqpoint{1.586694in}{2.602616in}}%
\pgfpathlineto{\pgfqpoint{1.600000in}{2.646968in}}%
\pgfpathlineto{\pgfqpoint{1.613306in}{2.602616in}}%
\pgfpathlineto{\pgfqpoint{1.604435in}{2.607051in}}%
\pgfpathlineto{\pgfqpoint{1.604435in}{2.538581in}}%
\pgfpathlineto{\pgfqpoint{1.595565in}{2.538581in}}%
\pgfusepath{fill}%
\end{pgfscope}%
\begin{pgfscope}%
\pgfpathrectangle{\pgfqpoint{1.432000in}{0.528000in}}{\pgfqpoint{3.696000in}{3.696000in}} %
\pgfusepath{clip}%
\pgfsetbuttcap%
\pgfsetroundjoin%
\definecolor{currentfill}{rgb}{0.119512,0.607464,0.540218}%
\pgfsetfillcolor{currentfill}%
\pgfsetlinewidth{0.000000pt}%
\definecolor{currentstroke}{rgb}{0.000000,0.000000,0.000000}%
\pgfsetstrokecolor{currentstroke}%
\pgfsetdash{}{0pt}%
\pgfpathmoveto{\pgfqpoint{1.703952in}{2.538581in}}%
\pgfpathlineto{\pgfqpoint{1.703952in}{2.607051in}}%
\pgfpathlineto{\pgfqpoint{1.695081in}{2.602616in}}%
\pgfpathlineto{\pgfqpoint{1.708387in}{2.646968in}}%
\pgfpathlineto{\pgfqpoint{1.721693in}{2.602616in}}%
\pgfpathlineto{\pgfqpoint{1.712822in}{2.607051in}}%
\pgfpathlineto{\pgfqpoint{1.712822in}{2.538581in}}%
\pgfpathlineto{\pgfqpoint{1.703952in}{2.538581in}}%
\pgfusepath{fill}%
\end{pgfscope}%
\begin{pgfscope}%
\pgfpathrectangle{\pgfqpoint{1.432000in}{0.528000in}}{\pgfqpoint{3.696000in}{3.696000in}} %
\pgfusepath{clip}%
\pgfsetbuttcap%
\pgfsetroundjoin%
\definecolor{currentfill}{rgb}{0.127568,0.566949,0.550556}%
\pgfsetfillcolor{currentfill}%
\pgfsetlinewidth{0.000000pt}%
\definecolor{currentstroke}{rgb}{0.000000,0.000000,0.000000}%
\pgfsetstrokecolor{currentstroke}%
\pgfsetdash{}{0pt}%
\pgfpathmoveto{\pgfqpoint{1.812339in}{2.538581in}}%
\pgfpathlineto{\pgfqpoint{1.812339in}{2.607051in}}%
\pgfpathlineto{\pgfqpoint{1.803469in}{2.602616in}}%
\pgfpathlineto{\pgfqpoint{1.816774in}{2.646968in}}%
\pgfpathlineto{\pgfqpoint{1.830080in}{2.602616in}}%
\pgfpathlineto{\pgfqpoint{1.821209in}{2.607051in}}%
\pgfpathlineto{\pgfqpoint{1.821209in}{2.538581in}}%
\pgfpathlineto{\pgfqpoint{1.812339in}{2.538581in}}%
\pgfusepath{fill}%
\end{pgfscope}%
\begin{pgfscope}%
\pgfpathrectangle{\pgfqpoint{1.432000in}{0.528000in}}{\pgfqpoint{3.696000in}{3.696000in}} %
\pgfusepath{clip}%
\pgfsetbuttcap%
\pgfsetroundjoin%
\definecolor{currentfill}{rgb}{0.277134,0.185228,0.489898}%
\pgfsetfillcolor{currentfill}%
\pgfsetlinewidth{0.000000pt}%
\definecolor{currentstroke}{rgb}{0.000000,0.000000,0.000000}%
\pgfsetstrokecolor{currentstroke}%
\pgfsetdash{}{0pt}%
\pgfpathmoveto{\pgfqpoint{1.922025in}{2.535444in}}%
\pgfpathlineto{\pgfqpoint{1.841863in}{2.615606in}}%
\pgfpathlineto{\pgfqpoint{1.838727in}{2.606198in}}%
\pgfpathlineto{\pgfqpoint{1.816774in}{2.646968in}}%
\pgfpathlineto{\pgfqpoint{1.857544in}{2.625015in}}%
\pgfpathlineto{\pgfqpoint{1.848136in}{2.621878in}}%
\pgfpathlineto{\pgfqpoint{1.928297in}{2.541717in}}%
\pgfpathlineto{\pgfqpoint{1.922025in}{2.535444in}}%
\pgfusepath{fill}%
\end{pgfscope}%
\begin{pgfscope}%
\pgfpathrectangle{\pgfqpoint{1.432000in}{0.528000in}}{\pgfqpoint{3.696000in}{3.696000in}} %
\pgfusepath{clip}%
\pgfsetbuttcap%
\pgfsetroundjoin%
\definecolor{currentfill}{rgb}{0.194100,0.399323,0.555565}%
\pgfsetfillcolor{currentfill}%
\pgfsetlinewidth{0.000000pt}%
\definecolor{currentstroke}{rgb}{0.000000,0.000000,0.000000}%
\pgfsetstrokecolor{currentstroke}%
\pgfsetdash{}{0pt}%
\pgfpathmoveto{\pgfqpoint{1.920726in}{2.538581in}}%
\pgfpathlineto{\pgfqpoint{1.920726in}{2.607051in}}%
\pgfpathlineto{\pgfqpoint{1.911856in}{2.602616in}}%
\pgfpathlineto{\pgfqpoint{1.925161in}{2.646968in}}%
\pgfpathlineto{\pgfqpoint{1.938467in}{2.602616in}}%
\pgfpathlineto{\pgfqpoint{1.929596in}{2.607051in}}%
\pgfpathlineto{\pgfqpoint{1.929596in}{2.538581in}}%
\pgfpathlineto{\pgfqpoint{1.920726in}{2.538581in}}%
\pgfusepath{fill}%
\end{pgfscope}%
\begin{pgfscope}%
\pgfpathrectangle{\pgfqpoint{1.432000in}{0.528000in}}{\pgfqpoint{3.696000in}{3.696000in}} %
\pgfusepath{clip}%
\pgfsetbuttcap%
\pgfsetroundjoin%
\definecolor{currentfill}{rgb}{0.168126,0.459988,0.558082}%
\pgfsetfillcolor{currentfill}%
\pgfsetlinewidth{0.000000pt}%
\definecolor{currentstroke}{rgb}{0.000000,0.000000,0.000000}%
\pgfsetstrokecolor{currentstroke}%
\pgfsetdash{}{0pt}%
\pgfpathmoveto{\pgfqpoint{2.030412in}{2.535444in}}%
\pgfpathlineto{\pgfqpoint{1.950251in}{2.615606in}}%
\pgfpathlineto{\pgfqpoint{1.947114in}{2.606198in}}%
\pgfpathlineto{\pgfqpoint{1.925161in}{2.646968in}}%
\pgfpathlineto{\pgfqpoint{1.965931in}{2.625015in}}%
\pgfpathlineto{\pgfqpoint{1.956523in}{2.621878in}}%
\pgfpathlineto{\pgfqpoint{2.036685in}{2.541717in}}%
\pgfpathlineto{\pgfqpoint{2.030412in}{2.535444in}}%
\pgfusepath{fill}%
\end{pgfscope}%
\begin{pgfscope}%
\pgfpathrectangle{\pgfqpoint{1.432000in}{0.528000in}}{\pgfqpoint{3.696000in}{3.696000in}} %
\pgfusepath{clip}%
\pgfsetbuttcap%
\pgfsetroundjoin%
\definecolor{currentfill}{rgb}{0.282910,0.105393,0.426902}%
\pgfsetfillcolor{currentfill}%
\pgfsetlinewidth{0.000000pt}%
\definecolor{currentstroke}{rgb}{0.000000,0.000000,0.000000}%
\pgfsetstrokecolor{currentstroke}%
\pgfsetdash{}{0pt}%
\pgfpathmoveto{\pgfqpoint{2.029113in}{2.538581in}}%
\pgfpathlineto{\pgfqpoint{2.029113in}{2.607051in}}%
\pgfpathlineto{\pgfqpoint{2.020243in}{2.602616in}}%
\pgfpathlineto{\pgfqpoint{2.033548in}{2.646968in}}%
\pgfpathlineto{\pgfqpoint{2.046854in}{2.602616in}}%
\pgfpathlineto{\pgfqpoint{2.037984in}{2.607051in}}%
\pgfpathlineto{\pgfqpoint{2.037984in}{2.538581in}}%
\pgfpathlineto{\pgfqpoint{2.029113in}{2.538581in}}%
\pgfusepath{fill}%
\end{pgfscope}%
\begin{pgfscope}%
\pgfpathrectangle{\pgfqpoint{1.432000in}{0.528000in}}{\pgfqpoint{3.696000in}{3.696000in}} %
\pgfusepath{clip}%
\pgfsetbuttcap%
\pgfsetroundjoin%
\definecolor{currentfill}{rgb}{0.154815,0.493313,0.557840}%
\pgfsetfillcolor{currentfill}%
\pgfsetlinewidth{0.000000pt}%
\definecolor{currentstroke}{rgb}{0.000000,0.000000,0.000000}%
\pgfsetstrokecolor{currentstroke}%
\pgfsetdash{}{0pt}%
\pgfpathmoveto{\pgfqpoint{2.138799in}{2.535444in}}%
\pgfpathlineto{\pgfqpoint{2.058638in}{2.615606in}}%
\pgfpathlineto{\pgfqpoint{2.055502in}{2.606198in}}%
\pgfpathlineto{\pgfqpoint{2.033548in}{2.646968in}}%
\pgfpathlineto{\pgfqpoint{2.074318in}{2.625015in}}%
\pgfpathlineto{\pgfqpoint{2.064910in}{2.621878in}}%
\pgfpathlineto{\pgfqpoint{2.145072in}{2.541717in}}%
\pgfpathlineto{\pgfqpoint{2.138799in}{2.535444in}}%
\pgfusepath{fill}%
\end{pgfscope}%
\begin{pgfscope}%
\pgfpathrectangle{\pgfqpoint{1.432000in}{0.528000in}}{\pgfqpoint{3.696000in}{3.696000in}} %
\pgfusepath{clip}%
\pgfsetbuttcap%
\pgfsetroundjoin%
\definecolor{currentfill}{rgb}{0.243113,0.292092,0.538516}%
\pgfsetfillcolor{currentfill}%
\pgfsetlinewidth{0.000000pt}%
\definecolor{currentstroke}{rgb}{0.000000,0.000000,0.000000}%
\pgfsetstrokecolor{currentstroke}%
\pgfsetdash{}{0pt}%
\pgfpathmoveto{\pgfqpoint{2.247186in}{2.535444in}}%
\pgfpathlineto{\pgfqpoint{2.167025in}{2.615606in}}%
\pgfpathlineto{\pgfqpoint{2.163889in}{2.606198in}}%
\pgfpathlineto{\pgfqpoint{2.141935in}{2.646968in}}%
\pgfpathlineto{\pgfqpoint{2.182706in}{2.625015in}}%
\pgfpathlineto{\pgfqpoint{2.173297in}{2.621878in}}%
\pgfpathlineto{\pgfqpoint{2.253459in}{2.541717in}}%
\pgfpathlineto{\pgfqpoint{2.247186in}{2.535444in}}%
\pgfusepath{fill}%
\end{pgfscope}%
\begin{pgfscope}%
\pgfpathrectangle{\pgfqpoint{1.432000in}{0.528000in}}{\pgfqpoint{3.696000in}{3.696000in}} %
\pgfusepath{clip}%
\pgfsetbuttcap%
\pgfsetroundjoin%
\definecolor{currentfill}{rgb}{0.282884,0.135920,0.453427}%
\pgfsetfillcolor{currentfill}%
\pgfsetlinewidth{0.000000pt}%
\definecolor{currentstroke}{rgb}{0.000000,0.000000,0.000000}%
\pgfsetstrokecolor{currentstroke}%
\pgfsetdash{}{0pt}%
\pgfpathmoveto{\pgfqpoint{2.356726in}{2.534614in}}%
\pgfpathlineto{\pgfqpoint{2.175655in}{2.625149in}}%
\pgfpathlineto{\pgfqpoint{2.175655in}{2.615232in}}%
\pgfpathlineto{\pgfqpoint{2.141935in}{2.646968in}}%
\pgfpathlineto{\pgfqpoint{2.187556in}{2.639034in}}%
\pgfpathlineto{\pgfqpoint{2.179622in}{2.633083in}}%
\pgfpathlineto{\pgfqpoint{2.360693in}{2.542548in}}%
\pgfpathlineto{\pgfqpoint{2.356726in}{2.534614in}}%
\pgfusepath{fill}%
\end{pgfscope}%
\begin{pgfscope}%
\pgfpathrectangle{\pgfqpoint{1.432000in}{0.528000in}}{\pgfqpoint{3.696000in}{3.696000in}} %
\pgfusepath{clip}%
\pgfsetbuttcap%
\pgfsetroundjoin%
\definecolor{currentfill}{rgb}{0.277134,0.185228,0.489898}%
\pgfsetfillcolor{currentfill}%
\pgfsetlinewidth{0.000000pt}%
\definecolor{currentstroke}{rgb}{0.000000,0.000000,0.000000}%
\pgfsetstrokecolor{currentstroke}%
\pgfsetdash{}{0pt}%
\pgfpathmoveto{\pgfqpoint{2.465113in}{2.534614in}}%
\pgfpathlineto{\pgfqpoint{2.284042in}{2.625149in}}%
\pgfpathlineto{\pgfqpoint{2.284042in}{2.615232in}}%
\pgfpathlineto{\pgfqpoint{2.250323in}{2.646968in}}%
\pgfpathlineto{\pgfqpoint{2.295943in}{2.639034in}}%
\pgfpathlineto{\pgfqpoint{2.288009in}{2.633083in}}%
\pgfpathlineto{\pgfqpoint{2.469080in}{2.542548in}}%
\pgfpathlineto{\pgfqpoint{2.465113in}{2.534614in}}%
\pgfusepath{fill}%
\end{pgfscope}%
\begin{pgfscope}%
\pgfpathrectangle{\pgfqpoint{1.432000in}{0.528000in}}{\pgfqpoint{3.696000in}{3.696000in}} %
\pgfusepath{clip}%
\pgfsetbuttcap%
\pgfsetroundjoin%
\definecolor{currentfill}{rgb}{0.272594,0.025563,0.353093}%
\pgfsetfillcolor{currentfill}%
\pgfsetlinewidth{0.000000pt}%
\definecolor{currentstroke}{rgb}{0.000000,0.000000,0.000000}%
\pgfsetstrokecolor{currentstroke}%
\pgfsetdash{}{0pt}%
\pgfpathmoveto{\pgfqpoint{2.463961in}{2.535444in}}%
\pgfpathlineto{\pgfqpoint{2.275412in}{2.723993in}}%
\pgfpathlineto{\pgfqpoint{2.272276in}{2.714585in}}%
\pgfpathlineto{\pgfqpoint{2.250323in}{2.755355in}}%
\pgfpathlineto{\pgfqpoint{2.291093in}{2.733402in}}%
\pgfpathlineto{\pgfqpoint{2.281684in}{2.730266in}}%
\pgfpathlineto{\pgfqpoint{2.470233in}{2.541717in}}%
\pgfpathlineto{\pgfqpoint{2.463961in}{2.535444in}}%
\pgfusepath{fill}%
\end{pgfscope}%
\begin{pgfscope}%
\pgfpathrectangle{\pgfqpoint{1.432000in}{0.528000in}}{\pgfqpoint{3.696000in}{3.696000in}} %
\pgfusepath{clip}%
\pgfsetbuttcap%
\pgfsetroundjoin%
\definecolor{currentfill}{rgb}{0.203063,0.379716,0.553925}%
\pgfsetfillcolor{currentfill}%
\pgfsetlinewidth{0.000000pt}%
\definecolor{currentstroke}{rgb}{0.000000,0.000000,0.000000}%
\pgfsetstrokecolor{currentstroke}%
\pgfsetdash{}{0pt}%
\pgfpathmoveto{\pgfqpoint{2.573500in}{2.534614in}}%
\pgfpathlineto{\pgfqpoint{2.392429in}{2.625149in}}%
\pgfpathlineto{\pgfqpoint{2.392429in}{2.615232in}}%
\pgfpathlineto{\pgfqpoint{2.358710in}{2.646968in}}%
\pgfpathlineto{\pgfqpoint{2.404330in}{2.639034in}}%
\pgfpathlineto{\pgfqpoint{2.396396in}{2.633083in}}%
\pgfpathlineto{\pgfqpoint{2.577467in}{2.542548in}}%
\pgfpathlineto{\pgfqpoint{2.573500in}{2.534614in}}%
\pgfusepath{fill}%
\end{pgfscope}%
\begin{pgfscope}%
\pgfpathrectangle{\pgfqpoint{1.432000in}{0.528000in}}{\pgfqpoint{3.696000in}{3.696000in}} %
\pgfusepath{clip}%
\pgfsetbuttcap%
\pgfsetroundjoin%
\definecolor{currentfill}{rgb}{0.274952,0.037752,0.364543}%
\pgfsetfillcolor{currentfill}%
\pgfsetlinewidth{0.000000pt}%
\definecolor{currentstroke}{rgb}{0.000000,0.000000,0.000000}%
\pgfsetstrokecolor{currentstroke}%
\pgfsetdash{}{0pt}%
\pgfpathmoveto{\pgfqpoint{2.572348in}{2.535444in}}%
\pgfpathlineto{\pgfqpoint{2.383799in}{2.723993in}}%
\pgfpathlineto{\pgfqpoint{2.380663in}{2.714585in}}%
\pgfpathlineto{\pgfqpoint{2.358710in}{2.755355in}}%
\pgfpathlineto{\pgfqpoint{2.399480in}{2.733402in}}%
\pgfpathlineto{\pgfqpoint{2.390071in}{2.730266in}}%
\pgfpathlineto{\pgfqpoint{2.578620in}{2.541717in}}%
\pgfpathlineto{\pgfqpoint{2.572348in}{2.535444in}}%
\pgfusepath{fill}%
\end{pgfscope}%
\begin{pgfscope}%
\pgfpathrectangle{\pgfqpoint{1.432000in}{0.528000in}}{\pgfqpoint{3.696000in}{3.696000in}} %
\pgfusepath{clip}%
\pgfsetbuttcap%
\pgfsetroundjoin%
\definecolor{currentfill}{rgb}{0.253935,0.265254,0.529983}%
\pgfsetfillcolor{currentfill}%
\pgfsetlinewidth{0.000000pt}%
\definecolor{currentstroke}{rgb}{0.000000,0.000000,0.000000}%
\pgfsetstrokecolor{currentstroke}%
\pgfsetdash{}{0pt}%
\pgfpathmoveto{\pgfqpoint{2.681887in}{2.534614in}}%
\pgfpathlineto{\pgfqpoint{2.500816in}{2.625149in}}%
\pgfpathlineto{\pgfqpoint{2.500816in}{2.615232in}}%
\pgfpathlineto{\pgfqpoint{2.467097in}{2.646968in}}%
\pgfpathlineto{\pgfqpoint{2.512717in}{2.639034in}}%
\pgfpathlineto{\pgfqpoint{2.504783in}{2.633083in}}%
\pgfpathlineto{\pgfqpoint{2.685854in}{2.542548in}}%
\pgfpathlineto{\pgfqpoint{2.681887in}{2.534614in}}%
\pgfusepath{fill}%
\end{pgfscope}%
\begin{pgfscope}%
\pgfpathrectangle{\pgfqpoint{1.432000in}{0.528000in}}{\pgfqpoint{3.696000in}{3.696000in}} %
\pgfusepath{clip}%
\pgfsetbuttcap%
\pgfsetroundjoin%
\definecolor{currentfill}{rgb}{0.281924,0.089666,0.412415}%
\pgfsetfillcolor{currentfill}%
\pgfsetlinewidth{0.000000pt}%
\definecolor{currentstroke}{rgb}{0.000000,0.000000,0.000000}%
\pgfsetstrokecolor{currentstroke}%
\pgfsetdash{}{0pt}%
\pgfpathmoveto{\pgfqpoint{2.680735in}{2.535444in}}%
\pgfpathlineto{\pgfqpoint{2.492186in}{2.723993in}}%
\pgfpathlineto{\pgfqpoint{2.489050in}{2.714585in}}%
\pgfpathlineto{\pgfqpoint{2.467097in}{2.755355in}}%
\pgfpathlineto{\pgfqpoint{2.507867in}{2.733402in}}%
\pgfpathlineto{\pgfqpoint{2.498458in}{2.730266in}}%
\pgfpathlineto{\pgfqpoint{2.687007in}{2.541717in}}%
\pgfpathlineto{\pgfqpoint{2.680735in}{2.535444in}}%
\pgfusepath{fill}%
\end{pgfscope}%
\begin{pgfscope}%
\pgfpathrectangle{\pgfqpoint{1.432000in}{0.528000in}}{\pgfqpoint{3.696000in}{3.696000in}} %
\pgfusepath{clip}%
\pgfsetbuttcap%
\pgfsetroundjoin%
\definecolor{currentfill}{rgb}{0.246811,0.283237,0.535941}%
\pgfsetfillcolor{currentfill}%
\pgfsetlinewidth{0.000000pt}%
\definecolor{currentstroke}{rgb}{0.000000,0.000000,0.000000}%
\pgfsetstrokecolor{currentstroke}%
\pgfsetdash{}{0pt}%
\pgfpathmoveto{\pgfqpoint{2.789122in}{2.535444in}}%
\pgfpathlineto{\pgfqpoint{2.708960in}{2.615606in}}%
\pgfpathlineto{\pgfqpoint{2.705824in}{2.606198in}}%
\pgfpathlineto{\pgfqpoint{2.683871in}{2.646968in}}%
\pgfpathlineto{\pgfqpoint{2.724641in}{2.625015in}}%
\pgfpathlineto{\pgfqpoint{2.715233in}{2.621878in}}%
\pgfpathlineto{\pgfqpoint{2.795394in}{2.541717in}}%
\pgfpathlineto{\pgfqpoint{2.789122in}{2.535444in}}%
\pgfusepath{fill}%
\end{pgfscope}%
\begin{pgfscope}%
\pgfpathrectangle{\pgfqpoint{1.432000in}{0.528000in}}{\pgfqpoint{3.696000in}{3.696000in}} %
\pgfusepath{clip}%
\pgfsetbuttcap%
\pgfsetroundjoin%
\definecolor{currentfill}{rgb}{0.282623,0.140926,0.457517}%
\pgfsetfillcolor{currentfill}%
\pgfsetlinewidth{0.000000pt}%
\definecolor{currentstroke}{rgb}{0.000000,0.000000,0.000000}%
\pgfsetstrokecolor{currentstroke}%
\pgfsetdash{}{0pt}%
\pgfpathmoveto{\pgfqpoint{2.788291in}{2.536597in}}%
\pgfpathlineto{\pgfqpoint{2.697755in}{2.717669in}}%
\pgfpathlineto{\pgfqpoint{2.691805in}{2.709735in}}%
\pgfpathlineto{\pgfqpoint{2.683871in}{2.755355in}}%
\pgfpathlineto{\pgfqpoint{2.715607in}{2.721636in}}%
\pgfpathlineto{\pgfqpoint{2.705689in}{2.721636in}}%
\pgfpathlineto{\pgfqpoint{2.796225in}{2.540564in}}%
\pgfpathlineto{\pgfqpoint{2.788291in}{2.536597in}}%
\pgfusepath{fill}%
\end{pgfscope}%
\begin{pgfscope}%
\pgfpathrectangle{\pgfqpoint{1.432000in}{0.528000in}}{\pgfqpoint{3.696000in}{3.696000in}} %
\pgfusepath{clip}%
\pgfsetbuttcap%
\pgfsetroundjoin%
\definecolor{currentfill}{rgb}{0.172719,0.448791,0.557885}%
\pgfsetfillcolor{currentfill}%
\pgfsetlinewidth{0.000000pt}%
\definecolor{currentstroke}{rgb}{0.000000,0.000000,0.000000}%
\pgfsetstrokecolor{currentstroke}%
\pgfsetdash{}{0pt}%
\pgfpathmoveto{\pgfqpoint{2.896678in}{2.536597in}}%
\pgfpathlineto{\pgfqpoint{2.806142in}{2.717669in}}%
\pgfpathlineto{\pgfqpoint{2.800192in}{2.709735in}}%
\pgfpathlineto{\pgfqpoint{2.792258in}{2.755355in}}%
\pgfpathlineto{\pgfqpoint{2.823994in}{2.721636in}}%
\pgfpathlineto{\pgfqpoint{2.814076in}{2.721636in}}%
\pgfpathlineto{\pgfqpoint{2.904612in}{2.540564in}}%
\pgfpathlineto{\pgfqpoint{2.896678in}{2.536597in}}%
\pgfusepath{fill}%
\end{pgfscope}%
\begin{pgfscope}%
\pgfpathrectangle{\pgfqpoint{1.432000in}{0.528000in}}{\pgfqpoint{3.696000in}{3.696000in}} %
\pgfusepath{clip}%
\pgfsetbuttcap%
\pgfsetroundjoin%
\definecolor{currentfill}{rgb}{0.273006,0.204520,0.501721}%
\pgfsetfillcolor{currentfill}%
\pgfsetlinewidth{0.000000pt}%
\definecolor{currentstroke}{rgb}{0.000000,0.000000,0.000000}%
\pgfsetstrokecolor{currentstroke}%
\pgfsetdash{}{0pt}%
\pgfpathmoveto{\pgfqpoint{2.896210in}{2.538581in}}%
\pgfpathlineto{\pgfqpoint{2.896210in}{2.715438in}}%
\pgfpathlineto{\pgfqpoint{2.887340in}{2.711003in}}%
\pgfpathlineto{\pgfqpoint{2.900645in}{2.755355in}}%
\pgfpathlineto{\pgfqpoint{2.913951in}{2.711003in}}%
\pgfpathlineto{\pgfqpoint{2.905080in}{2.715438in}}%
\pgfpathlineto{\pgfqpoint{2.905080in}{2.538581in}}%
\pgfpathlineto{\pgfqpoint{2.896210in}{2.538581in}}%
\pgfusepath{fill}%
\end{pgfscope}%
\begin{pgfscope}%
\pgfpathrectangle{\pgfqpoint{1.432000in}{0.528000in}}{\pgfqpoint{3.696000in}{3.696000in}} %
\pgfusepath{clip}%
\pgfsetbuttcap%
\pgfsetroundjoin%
\definecolor{currentfill}{rgb}{0.280894,0.078907,0.402329}%
\pgfsetfillcolor{currentfill}%
\pgfsetlinewidth{0.000000pt}%
\definecolor{currentstroke}{rgb}{0.000000,0.000000,0.000000}%
\pgfsetstrokecolor{currentstroke}%
\pgfsetdash{}{0pt}%
\pgfpathmoveto{\pgfqpoint{3.005065in}{2.536597in}}%
\pgfpathlineto{\pgfqpoint{2.914530in}{2.717669in}}%
\pgfpathlineto{\pgfqpoint{2.908579in}{2.709735in}}%
\pgfpathlineto{\pgfqpoint{2.900645in}{2.755355in}}%
\pgfpathlineto{\pgfqpoint{2.932381in}{2.721636in}}%
\pgfpathlineto{\pgfqpoint{2.922463in}{2.721636in}}%
\pgfpathlineto{\pgfqpoint{3.012999in}{2.540564in}}%
\pgfpathlineto{\pgfqpoint{3.005065in}{2.536597in}}%
\pgfusepath{fill}%
\end{pgfscope}%
\begin{pgfscope}%
\pgfpathrectangle{\pgfqpoint{1.432000in}{0.528000in}}{\pgfqpoint{3.696000in}{3.696000in}} %
\pgfusepath{clip}%
\pgfsetbuttcap%
\pgfsetroundjoin%
\definecolor{currentfill}{rgb}{0.282290,0.145912,0.461510}%
\pgfsetfillcolor{currentfill}%
\pgfsetlinewidth{0.000000pt}%
\definecolor{currentstroke}{rgb}{0.000000,0.000000,0.000000}%
\pgfsetstrokecolor{currentstroke}%
\pgfsetdash{}{0pt}%
\pgfpathmoveto{\pgfqpoint{3.004597in}{2.538581in}}%
\pgfpathlineto{\pgfqpoint{3.004597in}{2.715438in}}%
\pgfpathlineto{\pgfqpoint{2.995727in}{2.711003in}}%
\pgfpathlineto{\pgfqpoint{3.009032in}{2.755355in}}%
\pgfpathlineto{\pgfqpoint{3.022338in}{2.711003in}}%
\pgfpathlineto{\pgfqpoint{3.013467in}{2.715438in}}%
\pgfpathlineto{\pgfqpoint{3.013467in}{2.538581in}}%
\pgfpathlineto{\pgfqpoint{3.004597in}{2.538581in}}%
\pgfusepath{fill}%
\end{pgfscope}%
\begin{pgfscope}%
\pgfpathrectangle{\pgfqpoint{1.432000in}{0.528000in}}{\pgfqpoint{3.696000in}{3.696000in}} %
\pgfusepath{clip}%
\pgfsetbuttcap%
\pgfsetroundjoin%
\definecolor{currentfill}{rgb}{0.276194,0.190074,0.493001}%
\pgfsetfillcolor{currentfill}%
\pgfsetlinewidth{0.000000pt}%
\definecolor{currentstroke}{rgb}{0.000000,0.000000,0.000000}%
\pgfsetstrokecolor{currentstroke}%
\pgfsetdash{}{0pt}%
\pgfpathmoveto{\pgfqpoint{3.113452in}{2.536597in}}%
\pgfpathlineto{\pgfqpoint{3.022917in}{2.717669in}}%
\pgfpathlineto{\pgfqpoint{3.016966in}{2.709735in}}%
\pgfpathlineto{\pgfqpoint{3.009032in}{2.755355in}}%
\pgfpathlineto{\pgfqpoint{3.040768in}{2.721636in}}%
\pgfpathlineto{\pgfqpoint{3.030851in}{2.721636in}}%
\pgfpathlineto{\pgfqpoint{3.121386in}{2.540564in}}%
\pgfpathlineto{\pgfqpoint{3.113452in}{2.536597in}}%
\pgfusepath{fill}%
\end{pgfscope}%
\begin{pgfscope}%
\pgfpathrectangle{\pgfqpoint{1.432000in}{0.528000in}}{\pgfqpoint{3.696000in}{3.696000in}} %
\pgfusepath{clip}%
\pgfsetbuttcap%
\pgfsetroundjoin%
\definecolor{currentfill}{rgb}{0.233603,0.313828,0.543914}%
\pgfsetfillcolor{currentfill}%
\pgfsetlinewidth{0.000000pt}%
\definecolor{currentstroke}{rgb}{0.000000,0.000000,0.000000}%
\pgfsetstrokecolor{currentstroke}%
\pgfsetdash{}{0pt}%
\pgfpathmoveto{\pgfqpoint{3.221839in}{2.536597in}}%
\pgfpathlineto{\pgfqpoint{3.131304in}{2.717669in}}%
\pgfpathlineto{\pgfqpoint{3.125353in}{2.709735in}}%
\pgfpathlineto{\pgfqpoint{3.117419in}{2.755355in}}%
\pgfpathlineto{\pgfqpoint{3.149155in}{2.721636in}}%
\pgfpathlineto{\pgfqpoint{3.139238in}{2.721636in}}%
\pgfpathlineto{\pgfqpoint{3.229773in}{2.540564in}}%
\pgfpathlineto{\pgfqpoint{3.221839in}{2.536597in}}%
\pgfusepath{fill}%
\end{pgfscope}%
\begin{pgfscope}%
\pgfpathrectangle{\pgfqpoint{1.432000in}{0.528000in}}{\pgfqpoint{3.696000in}{3.696000in}} %
\pgfusepath{clip}%
\pgfsetbuttcap%
\pgfsetroundjoin%
\definecolor{currentfill}{rgb}{0.195860,0.395433,0.555276}%
\pgfsetfillcolor{currentfill}%
\pgfsetlinewidth{0.000000pt}%
\definecolor{currentstroke}{rgb}{0.000000,0.000000,0.000000}%
\pgfsetstrokecolor{currentstroke}%
\pgfsetdash{}{0pt}%
\pgfpathmoveto{\pgfqpoint{3.331057in}{2.535444in}}%
\pgfpathlineto{\pgfqpoint{3.142509in}{2.723993in}}%
\pgfpathlineto{\pgfqpoint{3.139372in}{2.714585in}}%
\pgfpathlineto{\pgfqpoint{3.117419in}{2.755355in}}%
\pgfpathlineto{\pgfqpoint{3.158189in}{2.733402in}}%
\pgfpathlineto{\pgfqpoint{3.148781in}{2.730266in}}%
\pgfpathlineto{\pgfqpoint{3.337330in}{2.541717in}}%
\pgfpathlineto{\pgfqpoint{3.331057in}{2.535444in}}%
\pgfusepath{fill}%
\end{pgfscope}%
\begin{pgfscope}%
\pgfpathrectangle{\pgfqpoint{1.432000in}{0.528000in}}{\pgfqpoint{3.696000in}{3.696000in}} %
\pgfusepath{clip}%
\pgfsetbuttcap%
\pgfsetroundjoin%
\definecolor{currentfill}{rgb}{0.169646,0.456262,0.558030}%
\pgfsetfillcolor{currentfill}%
\pgfsetlinewidth{0.000000pt}%
\definecolor{currentstroke}{rgb}{0.000000,0.000000,0.000000}%
\pgfsetstrokecolor{currentstroke}%
\pgfsetdash{}{0pt}%
\pgfpathmoveto{\pgfqpoint{3.439444in}{2.535444in}}%
\pgfpathlineto{\pgfqpoint{3.250896in}{2.723993in}}%
\pgfpathlineto{\pgfqpoint{3.247760in}{2.714585in}}%
\pgfpathlineto{\pgfqpoint{3.225806in}{2.755355in}}%
\pgfpathlineto{\pgfqpoint{3.266577in}{2.733402in}}%
\pgfpathlineto{\pgfqpoint{3.257168in}{2.730266in}}%
\pgfpathlineto{\pgfqpoint{3.445717in}{2.541717in}}%
\pgfpathlineto{\pgfqpoint{3.439444in}{2.535444in}}%
\pgfusepath{fill}%
\end{pgfscope}%
\begin{pgfscope}%
\pgfpathrectangle{\pgfqpoint{1.432000in}{0.528000in}}{\pgfqpoint{3.696000in}{3.696000in}} %
\pgfusepath{clip}%
\pgfsetbuttcap%
\pgfsetroundjoin%
\definecolor{currentfill}{rgb}{0.172719,0.448791,0.557885}%
\pgfsetfillcolor{currentfill}%
\pgfsetlinewidth{0.000000pt}%
\definecolor{currentstroke}{rgb}{0.000000,0.000000,0.000000}%
\pgfsetstrokecolor{currentstroke}%
\pgfsetdash{}{0pt}%
\pgfpathmoveto{\pgfqpoint{3.548508in}{2.534890in}}%
\pgfpathlineto{\pgfqpoint{3.256559in}{2.729523in}}%
\pgfpathlineto{\pgfqpoint{3.255329in}{2.719682in}}%
\pgfpathlineto{\pgfqpoint{3.225806in}{2.755355in}}%
\pgfpathlineto{\pgfqpoint{3.270090in}{2.741824in}}%
\pgfpathlineto{\pgfqpoint{3.261479in}{2.736903in}}%
\pgfpathlineto{\pgfqpoint{3.553428in}{2.542271in}}%
\pgfpathlineto{\pgfqpoint{3.548508in}{2.534890in}}%
\pgfusepath{fill}%
\end{pgfscope}%
\begin{pgfscope}%
\pgfpathrectangle{\pgfqpoint{1.432000in}{0.528000in}}{\pgfqpoint{3.696000in}{3.696000in}} %
\pgfusepath{clip}%
\pgfsetbuttcap%
\pgfsetroundjoin%
\definecolor{currentfill}{rgb}{0.175707,0.697900,0.491033}%
\pgfsetfillcolor{currentfill}%
\pgfsetlinewidth{0.000000pt}%
\definecolor{currentstroke}{rgb}{0.000000,0.000000,0.000000}%
\pgfsetstrokecolor{currentstroke}%
\pgfsetdash{}{0pt}%
\pgfpathmoveto{\pgfqpoint{3.656895in}{2.534890in}}%
\pgfpathlineto{\pgfqpoint{3.364946in}{2.729523in}}%
\pgfpathlineto{\pgfqpoint{3.363716in}{2.719682in}}%
\pgfpathlineto{\pgfqpoint{3.334194in}{2.755355in}}%
\pgfpathlineto{\pgfqpoint{3.378477in}{2.741824in}}%
\pgfpathlineto{\pgfqpoint{3.369867in}{2.736903in}}%
\pgfpathlineto{\pgfqpoint{3.661815in}{2.542271in}}%
\pgfpathlineto{\pgfqpoint{3.656895in}{2.534890in}}%
\pgfusepath{fill}%
\end{pgfscope}%
\begin{pgfscope}%
\pgfpathrectangle{\pgfqpoint{1.432000in}{0.528000in}}{\pgfqpoint{3.696000in}{3.696000in}} %
\pgfusepath{clip}%
\pgfsetbuttcap%
\pgfsetroundjoin%
\definecolor{currentfill}{rgb}{0.131172,0.555899,0.552459}%
\pgfsetfillcolor{currentfill}%
\pgfsetlinewidth{0.000000pt}%
\definecolor{currentstroke}{rgb}{0.000000,0.000000,0.000000}%
\pgfsetstrokecolor{currentstroke}%
\pgfsetdash{}{0pt}%
\pgfpathmoveto{\pgfqpoint{3.765282in}{2.534890in}}%
\pgfpathlineto{\pgfqpoint{3.473333in}{2.729523in}}%
\pgfpathlineto{\pgfqpoint{3.472103in}{2.719682in}}%
\pgfpathlineto{\pgfqpoint{3.442581in}{2.755355in}}%
\pgfpathlineto{\pgfqpoint{3.486864in}{2.741824in}}%
\pgfpathlineto{\pgfqpoint{3.478254in}{2.736903in}}%
\pgfpathlineto{\pgfqpoint{3.770202in}{2.542271in}}%
\pgfpathlineto{\pgfqpoint{3.765282in}{2.534890in}}%
\pgfusepath{fill}%
\end{pgfscope}%
\begin{pgfscope}%
\pgfpathrectangle{\pgfqpoint{1.432000in}{0.528000in}}{\pgfqpoint{3.696000in}{3.696000in}} %
\pgfusepath{clip}%
\pgfsetbuttcap%
\pgfsetroundjoin%
\definecolor{currentfill}{rgb}{0.243113,0.292092,0.538516}%
\pgfsetfillcolor{currentfill}%
\pgfsetlinewidth{0.000000pt}%
\definecolor{currentstroke}{rgb}{0.000000,0.000000,0.000000}%
\pgfsetstrokecolor{currentstroke}%
\pgfsetdash{}{0pt}%
\pgfpathmoveto{\pgfqpoint{3.764606in}{2.535444in}}%
\pgfpathlineto{\pgfqpoint{3.467670in}{2.832380in}}%
\pgfpathlineto{\pgfqpoint{3.464534in}{2.822972in}}%
\pgfpathlineto{\pgfqpoint{3.442581in}{2.863742in}}%
\pgfpathlineto{\pgfqpoint{3.483351in}{2.841789in}}%
\pgfpathlineto{\pgfqpoint{3.473942in}{2.838653in}}%
\pgfpathlineto{\pgfqpoint{3.770878in}{2.541717in}}%
\pgfpathlineto{\pgfqpoint{3.764606in}{2.535444in}}%
\pgfusepath{fill}%
\end{pgfscope}%
\begin{pgfscope}%
\pgfpathrectangle{\pgfqpoint{1.432000in}{0.528000in}}{\pgfqpoint{3.696000in}{3.696000in}} %
\pgfusepath{clip}%
\pgfsetbuttcap%
\pgfsetroundjoin%
\definecolor{currentfill}{rgb}{0.124395,0.578002,0.548287}%
\pgfsetfillcolor{currentfill}%
\pgfsetlinewidth{0.000000pt}%
\definecolor{currentstroke}{rgb}{0.000000,0.000000,0.000000}%
\pgfsetstrokecolor{currentstroke}%
\pgfsetdash{}{0pt}%
\pgfpathmoveto{\pgfqpoint{3.873669in}{2.534890in}}%
\pgfpathlineto{\pgfqpoint{3.581720in}{2.729523in}}%
\pgfpathlineto{\pgfqpoint{3.580490in}{2.719682in}}%
\pgfpathlineto{\pgfqpoint{3.550968in}{2.755355in}}%
\pgfpathlineto{\pgfqpoint{3.595251in}{2.741824in}}%
\pgfpathlineto{\pgfqpoint{3.586641in}{2.736903in}}%
\pgfpathlineto{\pgfqpoint{3.878589in}{2.542271in}}%
\pgfpathlineto{\pgfqpoint{3.873669in}{2.534890in}}%
\pgfusepath{fill}%
\end{pgfscope}%
\begin{pgfscope}%
\pgfpathrectangle{\pgfqpoint{1.432000in}{0.528000in}}{\pgfqpoint{3.696000in}{3.696000in}} %
\pgfusepath{clip}%
\pgfsetbuttcap%
\pgfsetroundjoin%
\definecolor{currentfill}{rgb}{0.248629,0.278775,0.534556}%
\pgfsetfillcolor{currentfill}%
\pgfsetlinewidth{0.000000pt}%
\definecolor{currentstroke}{rgb}{0.000000,0.000000,0.000000}%
\pgfsetstrokecolor{currentstroke}%
\pgfsetdash{}{0pt}%
\pgfpathmoveto{\pgfqpoint{3.872993in}{2.535444in}}%
\pgfpathlineto{\pgfqpoint{3.576057in}{2.832380in}}%
\pgfpathlineto{\pgfqpoint{3.572921in}{2.822972in}}%
\pgfpathlineto{\pgfqpoint{3.550968in}{2.863742in}}%
\pgfpathlineto{\pgfqpoint{3.591738in}{2.841789in}}%
\pgfpathlineto{\pgfqpoint{3.582329in}{2.838653in}}%
\pgfpathlineto{\pgfqpoint{3.879265in}{2.541717in}}%
\pgfpathlineto{\pgfqpoint{3.872993in}{2.535444in}}%
\pgfusepath{fill}%
\end{pgfscope}%
\begin{pgfscope}%
\pgfpathrectangle{\pgfqpoint{1.432000in}{0.528000in}}{\pgfqpoint{3.696000in}{3.696000in}} %
\pgfusepath{clip}%
\pgfsetbuttcap%
\pgfsetroundjoin%
\definecolor{currentfill}{rgb}{0.119738,0.603785,0.541400}%
\pgfsetfillcolor{currentfill}%
\pgfsetlinewidth{0.000000pt}%
\definecolor{currentstroke}{rgb}{0.000000,0.000000,0.000000}%
\pgfsetstrokecolor{currentstroke}%
\pgfsetdash{}{0pt}%
\pgfpathmoveto{\pgfqpoint{3.982056in}{2.534890in}}%
\pgfpathlineto{\pgfqpoint{3.690107in}{2.729523in}}%
\pgfpathlineto{\pgfqpoint{3.688877in}{2.719682in}}%
\pgfpathlineto{\pgfqpoint{3.659355in}{2.755355in}}%
\pgfpathlineto{\pgfqpoint{3.703639in}{2.741824in}}%
\pgfpathlineto{\pgfqpoint{3.695028in}{2.736903in}}%
\pgfpathlineto{\pgfqpoint{3.986976in}{2.542271in}}%
\pgfpathlineto{\pgfqpoint{3.982056in}{2.534890in}}%
\pgfusepath{fill}%
\end{pgfscope}%
\begin{pgfscope}%
\pgfpathrectangle{\pgfqpoint{1.432000in}{0.528000in}}{\pgfqpoint{3.696000in}{3.696000in}} %
\pgfusepath{clip}%
\pgfsetbuttcap%
\pgfsetroundjoin%
\definecolor{currentfill}{rgb}{0.268510,0.009605,0.335427}%
\pgfsetfillcolor{currentfill}%
\pgfsetlinewidth{0.000000pt}%
\definecolor{currentstroke}{rgb}{0.000000,0.000000,0.000000}%
\pgfsetstrokecolor{currentstroke}%
\pgfsetdash{}{0pt}%
\pgfpathmoveto{\pgfqpoint{3.981855in}{2.535032in}}%
\pgfpathlineto{\pgfqpoint{3.580240in}{2.836244in}}%
\pgfpathlineto{\pgfqpoint{3.578466in}{2.826486in}}%
\pgfpathlineto{\pgfqpoint{3.550968in}{2.863742in}}%
\pgfpathlineto{\pgfqpoint{3.594433in}{2.847775in}}%
\pgfpathlineto{\pgfqpoint{3.585562in}{2.843340in}}%
\pgfpathlineto{\pgfqpoint{3.987177in}{2.542129in}}%
\pgfpathlineto{\pgfqpoint{3.981855in}{2.535032in}}%
\pgfusepath{fill}%
\end{pgfscope}%
\begin{pgfscope}%
\pgfpathrectangle{\pgfqpoint{1.432000in}{0.528000in}}{\pgfqpoint{3.696000in}{3.696000in}} %
\pgfusepath{clip}%
\pgfsetbuttcap%
\pgfsetroundjoin%
\definecolor{currentfill}{rgb}{0.282884,0.135920,0.453427}%
\pgfsetfillcolor{currentfill}%
\pgfsetlinewidth{0.000000pt}%
\definecolor{currentstroke}{rgb}{0.000000,0.000000,0.000000}%
\pgfsetstrokecolor{currentstroke}%
\pgfsetdash{}{0pt}%
\pgfpathmoveto{\pgfqpoint{3.981380in}{2.535444in}}%
\pgfpathlineto{\pgfqpoint{3.684444in}{2.832380in}}%
\pgfpathlineto{\pgfqpoint{3.681308in}{2.822972in}}%
\pgfpathlineto{\pgfqpoint{3.659355in}{2.863742in}}%
\pgfpathlineto{\pgfqpoint{3.700125in}{2.841789in}}%
\pgfpathlineto{\pgfqpoint{3.690716in}{2.838653in}}%
\pgfpathlineto{\pgfqpoint{3.987652in}{2.541717in}}%
\pgfpathlineto{\pgfqpoint{3.981380in}{2.535444in}}%
\pgfusepath{fill}%
\end{pgfscope}%
\begin{pgfscope}%
\pgfpathrectangle{\pgfqpoint{1.432000in}{0.528000in}}{\pgfqpoint{3.696000in}{3.696000in}} %
\pgfusepath{clip}%
\pgfsetbuttcap%
\pgfsetroundjoin%
\definecolor{currentfill}{rgb}{0.281446,0.084320,0.407414}%
\pgfsetfillcolor{currentfill}%
\pgfsetlinewidth{0.000000pt}%
\definecolor{currentstroke}{rgb}{0.000000,0.000000,0.000000}%
\pgfsetstrokecolor{currentstroke}%
\pgfsetdash{}{0pt}%
\pgfpathmoveto{\pgfqpoint{4.090920in}{2.534614in}}%
\pgfpathlineto{\pgfqpoint{3.693074in}{2.733537in}}%
\pgfpathlineto{\pgfqpoint{3.693074in}{2.723619in}}%
\pgfpathlineto{\pgfqpoint{3.659355in}{2.755355in}}%
\pgfpathlineto{\pgfqpoint{3.704975in}{2.747421in}}%
\pgfpathlineto{\pgfqpoint{3.697041in}{2.741470in}}%
\pgfpathlineto{\pgfqpoint{4.094887in}{2.542548in}}%
\pgfpathlineto{\pgfqpoint{4.090920in}{2.534614in}}%
\pgfusepath{fill}%
\end{pgfscope}%
\begin{pgfscope}%
\pgfpathrectangle{\pgfqpoint{1.432000in}{0.528000in}}{\pgfqpoint{3.696000in}{3.696000in}} %
\pgfusepath{clip}%
\pgfsetbuttcap%
\pgfsetroundjoin%
\definecolor{currentfill}{rgb}{0.121380,0.629492,0.531973}%
\pgfsetfillcolor{currentfill}%
\pgfsetlinewidth{0.000000pt}%
\definecolor{currentstroke}{rgb}{0.000000,0.000000,0.000000}%
\pgfsetstrokecolor{currentstroke}%
\pgfsetdash{}{0pt}%
\pgfpathmoveto{\pgfqpoint{4.090443in}{2.534890in}}%
\pgfpathlineto{\pgfqpoint{3.798495in}{2.729523in}}%
\pgfpathlineto{\pgfqpoint{3.797264in}{2.719682in}}%
\pgfpathlineto{\pgfqpoint{3.767742in}{2.755355in}}%
\pgfpathlineto{\pgfqpoint{3.812026in}{2.741824in}}%
\pgfpathlineto{\pgfqpoint{3.803415in}{2.736903in}}%
\pgfpathlineto{\pgfqpoint{4.095363in}{2.542271in}}%
\pgfpathlineto{\pgfqpoint{4.090443in}{2.534890in}}%
\pgfusepath{fill}%
\end{pgfscope}%
\begin{pgfscope}%
\pgfpathrectangle{\pgfqpoint{1.432000in}{0.528000in}}{\pgfqpoint{3.696000in}{3.696000in}} %
\pgfusepath{clip}%
\pgfsetbuttcap%
\pgfsetroundjoin%
\definecolor{currentfill}{rgb}{0.278791,0.062145,0.386592}%
\pgfsetfillcolor{currentfill}%
\pgfsetlinewidth{0.000000pt}%
\definecolor{currentstroke}{rgb}{0.000000,0.000000,0.000000}%
\pgfsetstrokecolor{currentstroke}%
\pgfsetdash{}{0pt}%
\pgfpathmoveto{\pgfqpoint{4.090242in}{2.535032in}}%
\pgfpathlineto{\pgfqpoint{3.688627in}{2.836244in}}%
\pgfpathlineto{\pgfqpoint{3.686853in}{2.826486in}}%
\pgfpathlineto{\pgfqpoint{3.659355in}{2.863742in}}%
\pgfpathlineto{\pgfqpoint{3.702820in}{2.847775in}}%
\pgfpathlineto{\pgfqpoint{3.693949in}{2.843340in}}%
\pgfpathlineto{\pgfqpoint{4.095564in}{2.542129in}}%
\pgfpathlineto{\pgfqpoint{4.090242in}{2.535032in}}%
\pgfusepath{fill}%
\end{pgfscope}%
\begin{pgfscope}%
\pgfpathrectangle{\pgfqpoint{1.432000in}{0.528000in}}{\pgfqpoint{3.696000in}{3.696000in}} %
\pgfusepath{clip}%
\pgfsetbuttcap%
\pgfsetroundjoin%
\definecolor{currentfill}{rgb}{0.277941,0.056324,0.381191}%
\pgfsetfillcolor{currentfill}%
\pgfsetlinewidth{0.000000pt}%
\definecolor{currentstroke}{rgb}{0.000000,0.000000,0.000000}%
\pgfsetstrokecolor{currentstroke}%
\pgfsetdash{}{0pt}%
\pgfpathmoveto{\pgfqpoint{4.089767in}{2.535444in}}%
\pgfpathlineto{\pgfqpoint{3.792831in}{2.832380in}}%
\pgfpathlineto{\pgfqpoint{3.789695in}{2.822972in}}%
\pgfpathlineto{\pgfqpoint{3.767742in}{2.863742in}}%
\pgfpathlineto{\pgfqpoint{3.808512in}{2.841789in}}%
\pgfpathlineto{\pgfqpoint{3.799104in}{2.838653in}}%
\pgfpathlineto{\pgfqpoint{4.096039in}{2.541717in}}%
\pgfpathlineto{\pgfqpoint{4.089767in}{2.535444in}}%
\pgfusepath{fill}%
\end{pgfscope}%
\begin{pgfscope}%
\pgfpathrectangle{\pgfqpoint{1.432000in}{0.528000in}}{\pgfqpoint{3.696000in}{3.696000in}} %
\pgfusepath{clip}%
\pgfsetbuttcap%
\pgfsetroundjoin%
\definecolor{currentfill}{rgb}{0.283229,0.120777,0.440584}%
\pgfsetfillcolor{currentfill}%
\pgfsetlinewidth{0.000000pt}%
\definecolor{currentstroke}{rgb}{0.000000,0.000000,0.000000}%
\pgfsetstrokecolor{currentstroke}%
\pgfsetdash{}{0pt}%
\pgfpathmoveto{\pgfqpoint{4.199307in}{2.534614in}}%
\pgfpathlineto{\pgfqpoint{3.801461in}{2.733537in}}%
\pgfpathlineto{\pgfqpoint{3.801461in}{2.723619in}}%
\pgfpathlineto{\pgfqpoint{3.767742in}{2.755355in}}%
\pgfpathlineto{\pgfqpoint{3.813362in}{2.747421in}}%
\pgfpathlineto{\pgfqpoint{3.805428in}{2.741470in}}%
\pgfpathlineto{\pgfqpoint{4.203274in}{2.542548in}}%
\pgfpathlineto{\pgfqpoint{4.199307in}{2.534614in}}%
\pgfusepath{fill}%
\end{pgfscope}%
\begin{pgfscope}%
\pgfpathrectangle{\pgfqpoint{1.432000in}{0.528000in}}{\pgfqpoint{3.696000in}{3.696000in}} %
\pgfusepath{clip}%
\pgfsetbuttcap%
\pgfsetroundjoin%
\definecolor{currentfill}{rgb}{0.166383,0.690856,0.496502}%
\pgfsetfillcolor{currentfill}%
\pgfsetlinewidth{0.000000pt}%
\definecolor{currentstroke}{rgb}{0.000000,0.000000,0.000000}%
\pgfsetstrokecolor{currentstroke}%
\pgfsetdash{}{0pt}%
\pgfpathmoveto{\pgfqpoint{4.198830in}{2.534890in}}%
\pgfpathlineto{\pgfqpoint{3.906882in}{2.729523in}}%
\pgfpathlineto{\pgfqpoint{3.905652in}{2.719682in}}%
\pgfpathlineto{\pgfqpoint{3.876129in}{2.755355in}}%
\pgfpathlineto{\pgfqpoint{3.920413in}{2.741824in}}%
\pgfpathlineto{\pgfqpoint{3.911802in}{2.736903in}}%
\pgfpathlineto{\pgfqpoint{4.203751in}{2.542271in}}%
\pgfpathlineto{\pgfqpoint{4.198830in}{2.534890in}}%
\pgfusepath{fill}%
\end{pgfscope}%
\begin{pgfscope}%
\pgfpathrectangle{\pgfqpoint{1.432000in}{0.528000in}}{\pgfqpoint{3.696000in}{3.696000in}} %
\pgfusepath{clip}%
\pgfsetbuttcap%
\pgfsetroundjoin%
\definecolor{currentfill}{rgb}{0.274952,0.037752,0.364543}%
\pgfsetfillcolor{currentfill}%
\pgfsetlinewidth{0.000000pt}%
\definecolor{currentstroke}{rgb}{0.000000,0.000000,0.000000}%
\pgfsetstrokecolor{currentstroke}%
\pgfsetdash{}{0pt}%
\pgfpathmoveto{\pgfqpoint{4.198629in}{2.535032in}}%
\pgfpathlineto{\pgfqpoint{3.797014in}{2.836244in}}%
\pgfpathlineto{\pgfqpoint{3.795240in}{2.826486in}}%
\pgfpathlineto{\pgfqpoint{3.767742in}{2.863742in}}%
\pgfpathlineto{\pgfqpoint{3.811207in}{2.847775in}}%
\pgfpathlineto{\pgfqpoint{3.802336in}{2.843340in}}%
\pgfpathlineto{\pgfqpoint{4.203951in}{2.542129in}}%
\pgfpathlineto{\pgfqpoint{4.198629in}{2.535032in}}%
\pgfusepath{fill}%
\end{pgfscope}%
\begin{pgfscope}%
\pgfpathrectangle{\pgfqpoint{1.432000in}{0.528000in}}{\pgfqpoint{3.696000in}{3.696000in}} %
\pgfusepath{clip}%
\pgfsetbuttcap%
\pgfsetroundjoin%
\definecolor{currentfill}{rgb}{0.283197,0.115680,0.436115}%
\pgfsetfillcolor{currentfill}%
\pgfsetlinewidth{0.000000pt}%
\definecolor{currentstroke}{rgb}{0.000000,0.000000,0.000000}%
\pgfsetstrokecolor{currentstroke}%
\pgfsetdash{}{0pt}%
\pgfpathmoveto{\pgfqpoint{4.198154in}{2.535444in}}%
\pgfpathlineto{\pgfqpoint{3.901218in}{2.832380in}}%
\pgfpathlineto{\pgfqpoint{3.898082in}{2.822972in}}%
\pgfpathlineto{\pgfqpoint{3.876129in}{2.863742in}}%
\pgfpathlineto{\pgfqpoint{3.916899in}{2.841789in}}%
\pgfpathlineto{\pgfqpoint{3.907491in}{2.838653in}}%
\pgfpathlineto{\pgfqpoint{4.204426in}{2.541717in}}%
\pgfpathlineto{\pgfqpoint{4.198154in}{2.535444in}}%
\pgfusepath{fill}%
\end{pgfscope}%
\begin{pgfscope}%
\pgfpathrectangle{\pgfqpoint{1.432000in}{0.528000in}}{\pgfqpoint{3.696000in}{3.696000in}} %
\pgfusepath{clip}%
\pgfsetbuttcap%
\pgfsetroundjoin%
\definecolor{currentfill}{rgb}{0.477504,0.821444,0.318195}%
\pgfsetfillcolor{currentfill}%
\pgfsetlinewidth{0.000000pt}%
\definecolor{currentstroke}{rgb}{0.000000,0.000000,0.000000}%
\pgfsetstrokecolor{currentstroke}%
\pgfsetdash{}{0pt}%
\pgfpathmoveto{\pgfqpoint{4.307217in}{2.534890in}}%
\pgfpathlineto{\pgfqpoint{4.015269in}{2.729523in}}%
\pgfpathlineto{\pgfqpoint{4.014039in}{2.719682in}}%
\pgfpathlineto{\pgfqpoint{3.984516in}{2.755355in}}%
\pgfpathlineto{\pgfqpoint{4.028800in}{2.741824in}}%
\pgfpathlineto{\pgfqpoint{4.020189in}{2.736903in}}%
\pgfpathlineto{\pgfqpoint{4.312138in}{2.542271in}}%
\pgfpathlineto{\pgfqpoint{4.307217in}{2.534890in}}%
\pgfusepath{fill}%
\end{pgfscope}%
\begin{pgfscope}%
\pgfpathrectangle{\pgfqpoint{1.432000in}{0.528000in}}{\pgfqpoint{3.696000in}{3.696000in}} %
\pgfusepath{clip}%
\pgfsetbuttcap%
\pgfsetroundjoin%
\definecolor{currentfill}{rgb}{0.277134,0.185228,0.489898}%
\pgfsetfillcolor{currentfill}%
\pgfsetlinewidth{0.000000pt}%
\definecolor{currentstroke}{rgb}{0.000000,0.000000,0.000000}%
\pgfsetstrokecolor{currentstroke}%
\pgfsetdash{}{0pt}%
\pgfpathmoveto{\pgfqpoint{4.306541in}{2.535444in}}%
\pgfpathlineto{\pgfqpoint{4.009605in}{2.832380in}}%
\pgfpathlineto{\pgfqpoint{4.006469in}{2.822972in}}%
\pgfpathlineto{\pgfqpoint{3.984516in}{2.863742in}}%
\pgfpathlineto{\pgfqpoint{4.025286in}{2.841789in}}%
\pgfpathlineto{\pgfqpoint{4.015878in}{2.838653in}}%
\pgfpathlineto{\pgfqpoint{4.312814in}{2.541717in}}%
\pgfpathlineto{\pgfqpoint{4.306541in}{2.535444in}}%
\pgfusepath{fill}%
\end{pgfscope}%
\begin{pgfscope}%
\pgfpathrectangle{\pgfqpoint{1.432000in}{0.528000in}}{\pgfqpoint{3.696000in}{3.696000in}} %
\pgfusepath{clip}%
\pgfsetbuttcap%
\pgfsetroundjoin%
\definecolor{currentfill}{rgb}{0.143303,0.669459,0.511215}%
\pgfsetfillcolor{currentfill}%
\pgfsetlinewidth{0.000000pt}%
\definecolor{currentstroke}{rgb}{0.000000,0.000000,0.000000}%
\pgfsetstrokecolor{currentstroke}%
\pgfsetdash{}{0pt}%
\pgfpathmoveto{\pgfqpoint{4.415604in}{2.534890in}}%
\pgfpathlineto{\pgfqpoint{4.123656in}{2.729523in}}%
\pgfpathlineto{\pgfqpoint{4.122426in}{2.719682in}}%
\pgfpathlineto{\pgfqpoint{4.092903in}{2.755355in}}%
\pgfpathlineto{\pgfqpoint{4.137187in}{2.741824in}}%
\pgfpathlineto{\pgfqpoint{4.128576in}{2.736903in}}%
\pgfpathlineto{\pgfqpoint{4.420525in}{2.542271in}}%
\pgfpathlineto{\pgfqpoint{4.415604in}{2.534890in}}%
\pgfusepath{fill}%
\end{pgfscope}%
\begin{pgfscope}%
\pgfpathrectangle{\pgfqpoint{1.432000in}{0.528000in}}{\pgfqpoint{3.696000in}{3.696000in}} %
\pgfusepath{clip}%
\pgfsetbuttcap%
\pgfsetroundjoin%
\definecolor{currentfill}{rgb}{0.271828,0.209303,0.504434}%
\pgfsetfillcolor{currentfill}%
\pgfsetlinewidth{0.000000pt}%
\definecolor{currentstroke}{rgb}{0.000000,0.000000,0.000000}%
\pgfsetstrokecolor{currentstroke}%
\pgfsetdash{}{0pt}%
\pgfpathmoveto{\pgfqpoint{4.414928in}{2.535444in}}%
\pgfpathlineto{\pgfqpoint{4.226380in}{2.723993in}}%
\pgfpathlineto{\pgfqpoint{4.223243in}{2.714585in}}%
\pgfpathlineto{\pgfqpoint{4.201290in}{2.755355in}}%
\pgfpathlineto{\pgfqpoint{4.242060in}{2.733402in}}%
\pgfpathlineto{\pgfqpoint{4.232652in}{2.730266in}}%
\pgfpathlineto{\pgfqpoint{4.421201in}{2.541717in}}%
\pgfpathlineto{\pgfqpoint{4.414928in}{2.535444in}}%
\pgfusepath{fill}%
\end{pgfscope}%
\begin{pgfscope}%
\pgfpathrectangle{\pgfqpoint{1.432000in}{0.528000in}}{\pgfqpoint{3.696000in}{3.696000in}} %
\pgfusepath{clip}%
\pgfsetbuttcap%
\pgfsetroundjoin%
\definecolor{currentfill}{rgb}{0.237441,0.305202,0.541921}%
\pgfsetfillcolor{currentfill}%
\pgfsetlinewidth{0.000000pt}%
\definecolor{currentstroke}{rgb}{0.000000,0.000000,0.000000}%
\pgfsetstrokecolor{currentstroke}%
\pgfsetdash{}{0pt}%
\pgfpathmoveto{\pgfqpoint{4.414928in}{2.535444in}}%
\pgfpathlineto{\pgfqpoint{4.117993in}{2.832380in}}%
\pgfpathlineto{\pgfqpoint{4.114856in}{2.822972in}}%
\pgfpathlineto{\pgfqpoint{4.092903in}{2.863742in}}%
\pgfpathlineto{\pgfqpoint{4.133673in}{2.841789in}}%
\pgfpathlineto{\pgfqpoint{4.124265in}{2.838653in}}%
\pgfpathlineto{\pgfqpoint{4.421201in}{2.541717in}}%
\pgfpathlineto{\pgfqpoint{4.414928in}{2.535444in}}%
\pgfusepath{fill}%
\end{pgfscope}%
\begin{pgfscope}%
\pgfpathrectangle{\pgfqpoint{1.432000in}{0.528000in}}{\pgfqpoint{3.696000in}{3.696000in}} %
\pgfusepath{clip}%
\pgfsetbuttcap%
\pgfsetroundjoin%
\definecolor{currentfill}{rgb}{0.273809,0.031497,0.358853}%
\pgfsetfillcolor{currentfill}%
\pgfsetlinewidth{0.000000pt}%
\definecolor{currentstroke}{rgb}{0.000000,0.000000,0.000000}%
\pgfsetstrokecolor{currentstroke}%
\pgfsetdash{}{0pt}%
\pgfpathmoveto{\pgfqpoint{4.414374in}{2.536120in}}%
\pgfpathlineto{\pgfqpoint{4.219742in}{2.828069in}}%
\pgfpathlineto{\pgfqpoint{4.214821in}{2.819458in}}%
\pgfpathlineto{\pgfqpoint{4.201290in}{2.863742in}}%
\pgfpathlineto{\pgfqpoint{4.236963in}{2.834219in}}%
\pgfpathlineto{\pgfqpoint{4.227122in}{2.832989in}}%
\pgfpathlineto{\pgfqpoint{4.421755in}{2.541041in}}%
\pgfpathlineto{\pgfqpoint{4.414374in}{2.536120in}}%
\pgfusepath{fill}%
\end{pgfscope}%
\begin{pgfscope}%
\pgfpathrectangle{\pgfqpoint{1.432000in}{0.528000in}}{\pgfqpoint{3.696000in}{3.696000in}} %
\pgfusepath{clip}%
\pgfsetbuttcap%
\pgfsetroundjoin%
\definecolor{currentfill}{rgb}{0.206756,0.371758,0.553117}%
\pgfsetfillcolor{currentfill}%
\pgfsetlinewidth{0.000000pt}%
\definecolor{currentstroke}{rgb}{0.000000,0.000000,0.000000}%
\pgfsetstrokecolor{currentstroke}%
\pgfsetdash{}{0pt}%
\pgfpathmoveto{\pgfqpoint{4.523991in}{2.534890in}}%
\pgfpathlineto{\pgfqpoint{4.232043in}{2.729523in}}%
\pgfpathlineto{\pgfqpoint{4.230813in}{2.719682in}}%
\pgfpathlineto{\pgfqpoint{4.201290in}{2.755355in}}%
\pgfpathlineto{\pgfqpoint{4.245574in}{2.741824in}}%
\pgfpathlineto{\pgfqpoint{4.236963in}{2.736903in}}%
\pgfpathlineto{\pgfqpoint{4.528912in}{2.542271in}}%
\pgfpathlineto{\pgfqpoint{4.523991in}{2.534890in}}%
\pgfusepath{fill}%
\end{pgfscope}%
\begin{pgfscope}%
\pgfpathrectangle{\pgfqpoint{1.432000in}{0.528000in}}{\pgfqpoint{3.696000in}{3.696000in}} %
\pgfusepath{clip}%
\pgfsetbuttcap%
\pgfsetroundjoin%
\definecolor{currentfill}{rgb}{0.154815,0.493313,0.557840}%
\pgfsetfillcolor{currentfill}%
\pgfsetlinewidth{0.000000pt}%
\definecolor{currentstroke}{rgb}{0.000000,0.000000,0.000000}%
\pgfsetstrokecolor{currentstroke}%
\pgfsetdash{}{0pt}%
\pgfpathmoveto{\pgfqpoint{4.523315in}{2.535444in}}%
\pgfpathlineto{\pgfqpoint{4.334767in}{2.723993in}}%
\pgfpathlineto{\pgfqpoint{4.331631in}{2.714585in}}%
\pgfpathlineto{\pgfqpoint{4.309677in}{2.755355in}}%
\pgfpathlineto{\pgfqpoint{4.350447in}{2.733402in}}%
\pgfpathlineto{\pgfqpoint{4.341039in}{2.730266in}}%
\pgfpathlineto{\pgfqpoint{4.529588in}{2.541717in}}%
\pgfpathlineto{\pgfqpoint{4.523315in}{2.535444in}}%
\pgfusepath{fill}%
\end{pgfscope}%
\begin{pgfscope}%
\pgfpathrectangle{\pgfqpoint{1.432000in}{0.528000in}}{\pgfqpoint{3.696000in}{3.696000in}} %
\pgfusepath{clip}%
\pgfsetbuttcap%
\pgfsetroundjoin%
\definecolor{currentfill}{rgb}{0.270595,0.214069,0.507052}%
\pgfsetfillcolor{currentfill}%
\pgfsetlinewidth{0.000000pt}%
\definecolor{currentstroke}{rgb}{0.000000,0.000000,0.000000}%
\pgfsetstrokecolor{currentstroke}%
\pgfsetdash{}{0pt}%
\pgfpathmoveto{\pgfqpoint{4.523315in}{2.535444in}}%
\pgfpathlineto{\pgfqpoint{4.226380in}{2.832380in}}%
\pgfpathlineto{\pgfqpoint{4.223243in}{2.822972in}}%
\pgfpathlineto{\pgfqpoint{4.201290in}{2.863742in}}%
\pgfpathlineto{\pgfqpoint{4.242060in}{2.841789in}}%
\pgfpathlineto{\pgfqpoint{4.232652in}{2.838653in}}%
\pgfpathlineto{\pgfqpoint{4.529588in}{2.541717in}}%
\pgfpathlineto{\pgfqpoint{4.523315in}{2.535444in}}%
\pgfusepath{fill}%
\end{pgfscope}%
\begin{pgfscope}%
\pgfpathrectangle{\pgfqpoint{1.432000in}{0.528000in}}{\pgfqpoint{3.696000in}{3.696000in}} %
\pgfusepath{clip}%
\pgfsetbuttcap%
\pgfsetroundjoin%
\definecolor{currentfill}{rgb}{0.280868,0.160771,0.472899}%
\pgfsetfillcolor{currentfill}%
\pgfsetlinewidth{0.000000pt}%
\definecolor{currentstroke}{rgb}{0.000000,0.000000,0.000000}%
\pgfsetstrokecolor{currentstroke}%
\pgfsetdash{}{0pt}%
\pgfpathmoveto{\pgfqpoint{4.522761in}{2.536120in}}%
\pgfpathlineto{\pgfqpoint{4.328129in}{2.828069in}}%
\pgfpathlineto{\pgfqpoint{4.323209in}{2.819458in}}%
\pgfpathlineto{\pgfqpoint{4.309677in}{2.863742in}}%
\pgfpathlineto{\pgfqpoint{4.345350in}{2.834219in}}%
\pgfpathlineto{\pgfqpoint{4.335510in}{2.832989in}}%
\pgfpathlineto{\pgfqpoint{4.530142in}{2.541041in}}%
\pgfpathlineto{\pgfqpoint{4.522761in}{2.536120in}}%
\pgfusepath{fill}%
\end{pgfscope}%
\begin{pgfscope}%
\pgfpathrectangle{\pgfqpoint{1.432000in}{0.528000in}}{\pgfqpoint{3.696000in}{3.696000in}} %
\pgfusepath{clip}%
\pgfsetbuttcap%
\pgfsetroundjoin%
\definecolor{currentfill}{rgb}{0.121380,0.629492,0.531973}%
\pgfsetfillcolor{currentfill}%
\pgfsetlinewidth{0.000000pt}%
\definecolor{currentstroke}{rgb}{0.000000,0.000000,0.000000}%
\pgfsetstrokecolor{currentstroke}%
\pgfsetdash{}{0pt}%
\pgfpathmoveto{\pgfqpoint{4.631703in}{2.535444in}}%
\pgfpathlineto{\pgfqpoint{4.443154in}{2.723993in}}%
\pgfpathlineto{\pgfqpoint{4.440018in}{2.714585in}}%
\pgfpathlineto{\pgfqpoint{4.418065in}{2.755355in}}%
\pgfpathlineto{\pgfqpoint{4.458835in}{2.733402in}}%
\pgfpathlineto{\pgfqpoint{4.449426in}{2.730266in}}%
\pgfpathlineto{\pgfqpoint{4.637975in}{2.541717in}}%
\pgfpathlineto{\pgfqpoint{4.631703in}{2.535444in}}%
\pgfusepath{fill}%
\end{pgfscope}%
\begin{pgfscope}%
\pgfpathrectangle{\pgfqpoint{1.432000in}{0.528000in}}{\pgfqpoint{3.696000in}{3.696000in}} %
\pgfusepath{clip}%
\pgfsetbuttcap%
\pgfsetroundjoin%
\definecolor{currentfill}{rgb}{0.277018,0.050344,0.375715}%
\pgfsetfillcolor{currentfill}%
\pgfsetlinewidth{0.000000pt}%
\definecolor{currentstroke}{rgb}{0.000000,0.000000,0.000000}%
\pgfsetstrokecolor{currentstroke}%
\pgfsetdash{}{0pt}%
\pgfpathmoveto{\pgfqpoint{4.630872in}{2.536597in}}%
\pgfpathlineto{\pgfqpoint{4.540336in}{2.717669in}}%
\pgfpathlineto{\pgfqpoint{4.534386in}{2.709735in}}%
\pgfpathlineto{\pgfqpoint{4.526452in}{2.755355in}}%
\pgfpathlineto{\pgfqpoint{4.558187in}{2.721636in}}%
\pgfpathlineto{\pgfqpoint{4.548270in}{2.721636in}}%
\pgfpathlineto{\pgfqpoint{4.638806in}{2.540564in}}%
\pgfpathlineto{\pgfqpoint{4.630872in}{2.536597in}}%
\pgfusepath{fill}%
\end{pgfscope}%
\begin{pgfscope}%
\pgfpathrectangle{\pgfqpoint{1.432000in}{0.528000in}}{\pgfqpoint{3.696000in}{3.696000in}} %
\pgfusepath{clip}%
\pgfsetbuttcap%
\pgfsetroundjoin%
\definecolor{currentfill}{rgb}{0.243113,0.292092,0.538516}%
\pgfsetfillcolor{currentfill}%
\pgfsetlinewidth{0.000000pt}%
\definecolor{currentstroke}{rgb}{0.000000,0.000000,0.000000}%
\pgfsetstrokecolor{currentstroke}%
\pgfsetdash{}{0pt}%
\pgfpathmoveto{\pgfqpoint{4.631148in}{2.536120in}}%
\pgfpathlineto{\pgfqpoint{4.436516in}{2.828069in}}%
\pgfpathlineto{\pgfqpoint{4.431596in}{2.819458in}}%
\pgfpathlineto{\pgfqpoint{4.418065in}{2.863742in}}%
\pgfpathlineto{\pgfqpoint{4.453738in}{2.834219in}}%
\pgfpathlineto{\pgfqpoint{4.443897in}{2.832989in}}%
\pgfpathlineto{\pgfqpoint{4.638529in}{2.541041in}}%
\pgfpathlineto{\pgfqpoint{4.631148in}{2.536120in}}%
\pgfusepath{fill}%
\end{pgfscope}%
\begin{pgfscope}%
\pgfpathrectangle{\pgfqpoint{1.432000in}{0.528000in}}{\pgfqpoint{3.696000in}{3.696000in}} %
\pgfusepath{clip}%
\pgfsetbuttcap%
\pgfsetroundjoin%
\definecolor{currentfill}{rgb}{0.185556,0.418570,0.556753}%
\pgfsetfillcolor{currentfill}%
\pgfsetlinewidth{0.000000pt}%
\definecolor{currentstroke}{rgb}{0.000000,0.000000,0.000000}%
\pgfsetstrokecolor{currentstroke}%
\pgfsetdash{}{0pt}%
\pgfpathmoveto{\pgfqpoint{4.740090in}{2.535444in}}%
\pgfpathlineto{\pgfqpoint{4.551541in}{2.723993in}}%
\pgfpathlineto{\pgfqpoint{4.548405in}{2.714585in}}%
\pgfpathlineto{\pgfqpoint{4.526452in}{2.755355in}}%
\pgfpathlineto{\pgfqpoint{4.567222in}{2.733402in}}%
\pgfpathlineto{\pgfqpoint{4.557813in}{2.730266in}}%
\pgfpathlineto{\pgfqpoint{4.746362in}{2.541717in}}%
\pgfpathlineto{\pgfqpoint{4.740090in}{2.535444in}}%
\pgfusepath{fill}%
\end{pgfscope}%
\begin{pgfscope}%
\pgfpathrectangle{\pgfqpoint{1.432000in}{0.528000in}}{\pgfqpoint{3.696000in}{3.696000in}} %
\pgfusepath{clip}%
\pgfsetbuttcap%
\pgfsetroundjoin%
\definecolor{currentfill}{rgb}{0.204903,0.375746,0.553533}%
\pgfsetfillcolor{currentfill}%
\pgfsetlinewidth{0.000000pt}%
\definecolor{currentstroke}{rgb}{0.000000,0.000000,0.000000}%
\pgfsetstrokecolor{currentstroke}%
\pgfsetdash{}{0pt}%
\pgfpathmoveto{\pgfqpoint{4.739259in}{2.536597in}}%
\pgfpathlineto{\pgfqpoint{4.648723in}{2.717669in}}%
\pgfpathlineto{\pgfqpoint{4.642773in}{2.709735in}}%
\pgfpathlineto{\pgfqpoint{4.634839in}{2.755355in}}%
\pgfpathlineto{\pgfqpoint{4.666574in}{2.721636in}}%
\pgfpathlineto{\pgfqpoint{4.656657in}{2.721636in}}%
\pgfpathlineto{\pgfqpoint{4.747193in}{2.540564in}}%
\pgfpathlineto{\pgfqpoint{4.739259in}{2.536597in}}%
\pgfusepath{fill}%
\end{pgfscope}%
\begin{pgfscope}%
\pgfpathrectangle{\pgfqpoint{1.432000in}{0.528000in}}{\pgfqpoint{3.696000in}{3.696000in}} %
\pgfusepath{clip}%
\pgfsetbuttcap%
\pgfsetroundjoin%
\definecolor{currentfill}{rgb}{0.278826,0.175490,0.483397}%
\pgfsetfillcolor{currentfill}%
\pgfsetlinewidth{0.000000pt}%
\definecolor{currentstroke}{rgb}{0.000000,0.000000,0.000000}%
\pgfsetstrokecolor{currentstroke}%
\pgfsetdash{}{0pt}%
\pgfpathmoveto{\pgfqpoint{4.739535in}{2.536120in}}%
\pgfpathlineto{\pgfqpoint{4.544903in}{2.828069in}}%
\pgfpathlineto{\pgfqpoint{4.539983in}{2.819458in}}%
\pgfpathlineto{\pgfqpoint{4.526452in}{2.863742in}}%
\pgfpathlineto{\pgfqpoint{4.562125in}{2.834219in}}%
\pgfpathlineto{\pgfqpoint{4.552284in}{2.832989in}}%
\pgfpathlineto{\pgfqpoint{4.746916in}{2.541041in}}%
\pgfpathlineto{\pgfqpoint{4.739535in}{2.536120in}}%
\pgfusepath{fill}%
\end{pgfscope}%
\begin{pgfscope}%
\pgfpathrectangle{\pgfqpoint{1.432000in}{0.528000in}}{\pgfqpoint{3.696000in}{3.696000in}} %
\pgfusepath{clip}%
\pgfsetbuttcap%
\pgfsetroundjoin%
\definecolor{currentfill}{rgb}{0.282327,0.094955,0.417331}%
\pgfsetfillcolor{currentfill}%
\pgfsetlinewidth{0.000000pt}%
\definecolor{currentstroke}{rgb}{0.000000,0.000000,0.000000}%
\pgfsetstrokecolor{currentstroke}%
\pgfsetdash{}{0pt}%
\pgfpathmoveto{\pgfqpoint{4.739018in}{2.537178in}}%
\pgfpathlineto{\pgfqpoint{4.643254in}{2.824471in}}%
\pgfpathlineto{\pgfqpoint{4.636241in}{2.817458in}}%
\pgfpathlineto{\pgfqpoint{4.634839in}{2.863742in}}%
\pgfpathlineto{\pgfqpoint{4.661487in}{2.825874in}}%
\pgfpathlineto{\pgfqpoint{4.651669in}{2.827276in}}%
\pgfpathlineto{\pgfqpoint{4.747433in}{2.539983in}}%
\pgfpathlineto{\pgfqpoint{4.739018in}{2.537178in}}%
\pgfusepath{fill}%
\end{pgfscope}%
\begin{pgfscope}%
\pgfpathrectangle{\pgfqpoint{1.432000in}{0.528000in}}{\pgfqpoint{3.696000in}{3.696000in}} %
\pgfusepath{clip}%
\pgfsetbuttcap%
\pgfsetroundjoin%
\definecolor{currentfill}{rgb}{0.151918,0.500685,0.557587}%
\pgfsetfillcolor{currentfill}%
\pgfsetlinewidth{0.000000pt}%
\definecolor{currentstroke}{rgb}{0.000000,0.000000,0.000000}%
\pgfsetstrokecolor{currentstroke}%
\pgfsetdash{}{0pt}%
\pgfpathmoveto{\pgfqpoint{4.847646in}{2.536597in}}%
\pgfpathlineto{\pgfqpoint{4.757110in}{2.717669in}}%
\pgfpathlineto{\pgfqpoint{4.751160in}{2.709735in}}%
\pgfpathlineto{\pgfqpoint{4.743226in}{2.755355in}}%
\pgfpathlineto{\pgfqpoint{4.774962in}{2.721636in}}%
\pgfpathlineto{\pgfqpoint{4.765044in}{2.721636in}}%
\pgfpathlineto{\pgfqpoint{4.855580in}{2.540564in}}%
\pgfpathlineto{\pgfqpoint{4.847646in}{2.536597in}}%
\pgfusepath{fill}%
\end{pgfscope}%
\begin{pgfscope}%
\pgfpathrectangle{\pgfqpoint{1.432000in}{0.528000in}}{\pgfqpoint{3.696000in}{3.696000in}} %
\pgfusepath{clip}%
\pgfsetbuttcap%
\pgfsetroundjoin%
\definecolor{currentfill}{rgb}{0.266580,0.228262,0.514349}%
\pgfsetfillcolor{currentfill}%
\pgfsetlinewidth{0.000000pt}%
\definecolor{currentstroke}{rgb}{0.000000,0.000000,0.000000}%
\pgfsetstrokecolor{currentstroke}%
\pgfsetdash{}{0pt}%
\pgfpathmoveto{\pgfqpoint{4.847405in}{2.537178in}}%
\pgfpathlineto{\pgfqpoint{4.751641in}{2.824471in}}%
\pgfpathlineto{\pgfqpoint{4.744628in}{2.817458in}}%
\pgfpathlineto{\pgfqpoint{4.743226in}{2.863742in}}%
\pgfpathlineto{\pgfqpoint{4.769874in}{2.825874in}}%
\pgfpathlineto{\pgfqpoint{4.760056in}{2.827276in}}%
\pgfpathlineto{\pgfqpoint{4.855821in}{2.539983in}}%
\pgfpathlineto{\pgfqpoint{4.847405in}{2.537178in}}%
\pgfusepath{fill}%
\end{pgfscope}%
\begin{pgfscope}%
\pgfpathrectangle{\pgfqpoint{1.432000in}{0.528000in}}{\pgfqpoint{3.696000in}{3.696000in}} %
\pgfusepath{clip}%
\pgfsetbuttcap%
\pgfsetroundjoin%
\definecolor{currentfill}{rgb}{0.216210,0.351535,0.550627}%
\pgfsetfillcolor{currentfill}%
\pgfsetlinewidth{0.000000pt}%
\definecolor{currentstroke}{rgb}{0.000000,0.000000,0.000000}%
\pgfsetstrokecolor{currentstroke}%
\pgfsetdash{}{0pt}%
\pgfpathmoveto{\pgfqpoint{4.956033in}{2.536597in}}%
\pgfpathlineto{\pgfqpoint{4.865497in}{2.717669in}}%
\pgfpathlineto{\pgfqpoint{4.859547in}{2.709735in}}%
\pgfpathlineto{\pgfqpoint{4.851613in}{2.755355in}}%
\pgfpathlineto{\pgfqpoint{4.883349in}{2.721636in}}%
\pgfpathlineto{\pgfqpoint{4.873431in}{2.721636in}}%
\pgfpathlineto{\pgfqpoint{4.963967in}{2.540564in}}%
\pgfpathlineto{\pgfqpoint{4.956033in}{2.536597in}}%
\pgfusepath{fill}%
\end{pgfscope}%
\begin{pgfscope}%
\pgfpathrectangle{\pgfqpoint{1.432000in}{0.528000in}}{\pgfqpoint{3.696000in}{3.696000in}} %
\pgfusepath{clip}%
\pgfsetbuttcap%
\pgfsetroundjoin%
\definecolor{currentfill}{rgb}{0.185556,0.418570,0.556753}%
\pgfsetfillcolor{currentfill}%
\pgfsetlinewidth{0.000000pt}%
\definecolor{currentstroke}{rgb}{0.000000,0.000000,0.000000}%
\pgfsetstrokecolor{currentstroke}%
\pgfsetdash{}{0pt}%
\pgfpathmoveto{\pgfqpoint{4.955565in}{2.538581in}}%
\pgfpathlineto{\pgfqpoint{4.955565in}{2.715438in}}%
\pgfpathlineto{\pgfqpoint{4.946694in}{2.711003in}}%
\pgfpathlineto{\pgfqpoint{4.960000in}{2.755355in}}%
\pgfpathlineto{\pgfqpoint{4.973306in}{2.711003in}}%
\pgfpathlineto{\pgfqpoint{4.964435in}{2.715438in}}%
\pgfpathlineto{\pgfqpoint{4.964435in}{2.538581in}}%
\pgfpathlineto{\pgfqpoint{4.955565in}{2.538581in}}%
\pgfusepath{fill}%
\end{pgfscope}%
\begin{pgfscope}%
\pgfpathrectangle{\pgfqpoint{1.432000in}{0.528000in}}{\pgfqpoint{3.696000in}{3.696000in}} %
\pgfusepath{clip}%
\pgfsetbuttcap%
\pgfsetroundjoin%
\definecolor{currentfill}{rgb}{0.281412,0.155834,0.469201}%
\pgfsetfillcolor{currentfill}%
\pgfsetlinewidth{0.000000pt}%
\definecolor{currentstroke}{rgb}{0.000000,0.000000,0.000000}%
\pgfsetstrokecolor{currentstroke}%
\pgfsetdash{}{0pt}%
\pgfpathmoveto{\pgfqpoint{4.955792in}{2.537178in}}%
\pgfpathlineto{\pgfqpoint{4.860028in}{2.824471in}}%
\pgfpathlineto{\pgfqpoint{4.853015in}{2.817458in}}%
\pgfpathlineto{\pgfqpoint{4.851613in}{2.863742in}}%
\pgfpathlineto{\pgfqpoint{4.878261in}{2.825874in}}%
\pgfpathlineto{\pgfqpoint{4.868443in}{2.827276in}}%
\pgfpathlineto{\pgfqpoint{4.964208in}{2.539983in}}%
\pgfpathlineto{\pgfqpoint{4.955792in}{2.537178in}}%
\pgfusepath{fill}%
\end{pgfscope}%
\begin{pgfscope}%
\pgfpathrectangle{\pgfqpoint{1.432000in}{0.528000in}}{\pgfqpoint{3.696000in}{3.696000in}} %
\pgfusepath{clip}%
\pgfsetbuttcap%
\pgfsetroundjoin%
\definecolor{currentfill}{rgb}{0.282910,0.105393,0.426902}%
\pgfsetfillcolor{currentfill}%
\pgfsetlinewidth{0.000000pt}%
\definecolor{currentstroke}{rgb}{0.000000,0.000000,0.000000}%
\pgfsetstrokecolor{currentstroke}%
\pgfsetdash{}{0pt}%
\pgfpathmoveto{\pgfqpoint{4.955565in}{2.538581in}}%
\pgfpathlineto{\pgfqpoint{4.955565in}{2.823825in}}%
\pgfpathlineto{\pgfqpoint{4.946694in}{2.819390in}}%
\pgfpathlineto{\pgfqpoint{4.960000in}{2.863742in}}%
\pgfpathlineto{\pgfqpoint{4.973306in}{2.819390in}}%
\pgfpathlineto{\pgfqpoint{4.964435in}{2.823825in}}%
\pgfpathlineto{\pgfqpoint{4.964435in}{2.538581in}}%
\pgfpathlineto{\pgfqpoint{4.955565in}{2.538581in}}%
\pgfusepath{fill}%
\end{pgfscope}%
\begin{pgfscope}%
\pgfpathrectangle{\pgfqpoint{1.432000in}{0.528000in}}{\pgfqpoint{3.696000in}{3.696000in}} %
\pgfusepath{clip}%
\pgfsetbuttcap%
\pgfsetroundjoin%
\definecolor{currentfill}{rgb}{0.128087,0.647749,0.523491}%
\pgfsetfillcolor{currentfill}%
\pgfsetlinewidth{0.000000pt}%
\definecolor{currentstroke}{rgb}{0.000000,0.000000,0.000000}%
\pgfsetstrokecolor{currentstroke}%
\pgfsetdash{}{0pt}%
\pgfpathmoveto{\pgfqpoint{1.604435in}{2.646968in}}%
\pgfpathlineto{\pgfqpoint{1.602218in}{2.650809in}}%
\pgfpathlineto{\pgfqpoint{1.597782in}{2.650809in}}%
\pgfpathlineto{\pgfqpoint{1.595565in}{2.646968in}}%
\pgfpathlineto{\pgfqpoint{1.597782in}{2.643127in}}%
\pgfpathlineto{\pgfqpoint{1.602218in}{2.643127in}}%
\pgfpathlineto{\pgfqpoint{1.604435in}{2.646968in}}%
\pgfpathlineto{\pgfqpoint{1.602218in}{2.650809in}}%
\pgfusepath{fill}%
\end{pgfscope}%
\begin{pgfscope}%
\pgfpathrectangle{\pgfqpoint{1.432000in}{0.528000in}}{\pgfqpoint{3.696000in}{3.696000in}} %
\pgfusepath{clip}%
\pgfsetbuttcap%
\pgfsetroundjoin%
\definecolor{currentfill}{rgb}{0.162142,0.474838,0.558140}%
\pgfsetfillcolor{currentfill}%
\pgfsetlinewidth{0.000000pt}%
\definecolor{currentstroke}{rgb}{0.000000,0.000000,0.000000}%
\pgfsetstrokecolor{currentstroke}%
\pgfsetdash{}{0pt}%
\pgfpathmoveto{\pgfqpoint{1.712822in}{2.646968in}}%
\pgfpathlineto{\pgfqpoint{1.710605in}{2.650809in}}%
\pgfpathlineto{\pgfqpoint{1.706169in}{2.650809in}}%
\pgfpathlineto{\pgfqpoint{1.703952in}{2.646968in}}%
\pgfpathlineto{\pgfqpoint{1.706169in}{2.643127in}}%
\pgfpathlineto{\pgfqpoint{1.710605in}{2.643127in}}%
\pgfpathlineto{\pgfqpoint{1.712822in}{2.646968in}}%
\pgfpathlineto{\pgfqpoint{1.710605in}{2.650809in}}%
\pgfusepath{fill}%
\end{pgfscope}%
\begin{pgfscope}%
\pgfpathrectangle{\pgfqpoint{1.432000in}{0.528000in}}{\pgfqpoint{3.696000in}{3.696000in}} %
\pgfusepath{clip}%
\pgfsetbuttcap%
\pgfsetroundjoin%
\definecolor{currentfill}{rgb}{0.280267,0.073417,0.397163}%
\pgfsetfillcolor{currentfill}%
\pgfsetlinewidth{0.000000pt}%
\definecolor{currentstroke}{rgb}{0.000000,0.000000,0.000000}%
\pgfsetstrokecolor{currentstroke}%
\pgfsetdash{}{0pt}%
\pgfpathmoveto{\pgfqpoint{1.703952in}{2.646968in}}%
\pgfpathlineto{\pgfqpoint{1.703952in}{2.715438in}}%
\pgfpathlineto{\pgfqpoint{1.695081in}{2.711003in}}%
\pgfpathlineto{\pgfqpoint{1.708387in}{2.755355in}}%
\pgfpathlineto{\pgfqpoint{1.721693in}{2.711003in}}%
\pgfpathlineto{\pgfqpoint{1.712822in}{2.715438in}}%
\pgfpathlineto{\pgfqpoint{1.712822in}{2.646968in}}%
\pgfpathlineto{\pgfqpoint{1.703952in}{2.646968in}}%
\pgfusepath{fill}%
\end{pgfscope}%
\begin{pgfscope}%
\pgfpathrectangle{\pgfqpoint{1.432000in}{0.528000in}}{\pgfqpoint{3.696000in}{3.696000in}} %
\pgfusepath{clip}%
\pgfsetbuttcap%
\pgfsetroundjoin%
\definecolor{currentfill}{rgb}{0.282910,0.105393,0.426902}%
\pgfsetfillcolor{currentfill}%
\pgfsetlinewidth{0.000000pt}%
\definecolor{currentstroke}{rgb}{0.000000,0.000000,0.000000}%
\pgfsetstrokecolor{currentstroke}%
\pgfsetdash{}{0pt}%
\pgfpathmoveto{\pgfqpoint{1.821209in}{2.646968in}}%
\pgfpathlineto{\pgfqpoint{1.818992in}{2.650809in}}%
\pgfpathlineto{\pgfqpoint{1.814557in}{2.650809in}}%
\pgfpathlineto{\pgfqpoint{1.812339in}{2.646968in}}%
\pgfpathlineto{\pgfqpoint{1.814557in}{2.643127in}}%
\pgfpathlineto{\pgfqpoint{1.818992in}{2.643127in}}%
\pgfpathlineto{\pgfqpoint{1.821209in}{2.646968in}}%
\pgfpathlineto{\pgfqpoint{1.818992in}{2.650809in}}%
\pgfusepath{fill}%
\end{pgfscope}%
\begin{pgfscope}%
\pgfpathrectangle{\pgfqpoint{1.432000in}{0.528000in}}{\pgfqpoint{3.696000in}{3.696000in}} %
\pgfusepath{clip}%
\pgfsetbuttcap%
\pgfsetroundjoin%
\definecolor{currentfill}{rgb}{0.262138,0.242286,0.520837}%
\pgfsetfillcolor{currentfill}%
\pgfsetlinewidth{0.000000pt}%
\definecolor{currentstroke}{rgb}{0.000000,0.000000,0.000000}%
\pgfsetstrokecolor{currentstroke}%
\pgfsetdash{}{0pt}%
\pgfpathmoveto{\pgfqpoint{1.812339in}{2.646968in}}%
\pgfpathlineto{\pgfqpoint{1.812339in}{2.715438in}}%
\pgfpathlineto{\pgfqpoint{1.803469in}{2.711003in}}%
\pgfpathlineto{\pgfqpoint{1.816774in}{2.755355in}}%
\pgfpathlineto{\pgfqpoint{1.830080in}{2.711003in}}%
\pgfpathlineto{\pgfqpoint{1.821209in}{2.715438in}}%
\pgfpathlineto{\pgfqpoint{1.821209in}{2.646968in}}%
\pgfpathlineto{\pgfqpoint{1.812339in}{2.646968in}}%
\pgfusepath{fill}%
\end{pgfscope}%
\begin{pgfscope}%
\pgfpathrectangle{\pgfqpoint{1.432000in}{0.528000in}}{\pgfqpoint{3.696000in}{3.696000in}} %
\pgfusepath{clip}%
\pgfsetbuttcap%
\pgfsetroundjoin%
\definecolor{currentfill}{rgb}{0.248629,0.278775,0.534556}%
\pgfsetfillcolor{currentfill}%
\pgfsetlinewidth{0.000000pt}%
\definecolor{currentstroke}{rgb}{0.000000,0.000000,0.000000}%
\pgfsetstrokecolor{currentstroke}%
\pgfsetdash{}{0pt}%
\pgfpathmoveto{\pgfqpoint{1.920726in}{2.646968in}}%
\pgfpathlineto{\pgfqpoint{1.920726in}{2.715438in}}%
\pgfpathlineto{\pgfqpoint{1.911856in}{2.711003in}}%
\pgfpathlineto{\pgfqpoint{1.925161in}{2.755355in}}%
\pgfpathlineto{\pgfqpoint{1.938467in}{2.711003in}}%
\pgfpathlineto{\pgfqpoint{1.929596in}{2.715438in}}%
\pgfpathlineto{\pgfqpoint{1.929596in}{2.646968in}}%
\pgfpathlineto{\pgfqpoint{1.920726in}{2.646968in}}%
\pgfusepath{fill}%
\end{pgfscope}%
\begin{pgfscope}%
\pgfpathrectangle{\pgfqpoint{1.432000in}{0.528000in}}{\pgfqpoint{3.696000in}{3.696000in}} %
\pgfusepath{clip}%
\pgfsetbuttcap%
\pgfsetroundjoin%
\definecolor{currentfill}{rgb}{0.277018,0.050344,0.375715}%
\pgfsetfillcolor{currentfill}%
\pgfsetlinewidth{0.000000pt}%
\definecolor{currentstroke}{rgb}{0.000000,0.000000,0.000000}%
\pgfsetstrokecolor{currentstroke}%
\pgfsetdash{}{0pt}%
\pgfpathmoveto{\pgfqpoint{2.030412in}{2.643832in}}%
\pgfpathlineto{\pgfqpoint{1.950251in}{2.723993in}}%
\pgfpathlineto{\pgfqpoint{1.947114in}{2.714585in}}%
\pgfpathlineto{\pgfqpoint{1.925161in}{2.755355in}}%
\pgfpathlineto{\pgfqpoint{1.965931in}{2.733402in}}%
\pgfpathlineto{\pgfqpoint{1.956523in}{2.730266in}}%
\pgfpathlineto{\pgfqpoint{2.036685in}{2.650104in}}%
\pgfpathlineto{\pgfqpoint{2.030412in}{2.643832in}}%
\pgfusepath{fill}%
\end{pgfscope}%
\begin{pgfscope}%
\pgfpathrectangle{\pgfqpoint{1.432000in}{0.528000in}}{\pgfqpoint{3.696000in}{3.696000in}} %
\pgfusepath{clip}%
\pgfsetbuttcap%
\pgfsetroundjoin%
\definecolor{currentfill}{rgb}{0.275191,0.194905,0.496005}%
\pgfsetfillcolor{currentfill}%
\pgfsetlinewidth{0.000000pt}%
\definecolor{currentstroke}{rgb}{0.000000,0.000000,0.000000}%
\pgfsetstrokecolor{currentstroke}%
\pgfsetdash{}{0pt}%
\pgfpathmoveto{\pgfqpoint{2.029113in}{2.646968in}}%
\pgfpathlineto{\pgfqpoint{2.029113in}{2.715438in}}%
\pgfpathlineto{\pgfqpoint{2.020243in}{2.711003in}}%
\pgfpathlineto{\pgfqpoint{2.033548in}{2.755355in}}%
\pgfpathlineto{\pgfqpoint{2.046854in}{2.711003in}}%
\pgfpathlineto{\pgfqpoint{2.037984in}{2.715438in}}%
\pgfpathlineto{\pgfqpoint{2.037984in}{2.646968in}}%
\pgfpathlineto{\pgfqpoint{2.029113in}{2.646968in}}%
\pgfusepath{fill}%
\end{pgfscope}%
\begin{pgfscope}%
\pgfpathrectangle{\pgfqpoint{1.432000in}{0.528000in}}{\pgfqpoint{3.696000in}{3.696000in}} %
\pgfusepath{clip}%
\pgfsetbuttcap%
\pgfsetroundjoin%
\definecolor{currentfill}{rgb}{0.216210,0.351535,0.550627}%
\pgfsetfillcolor{currentfill}%
\pgfsetlinewidth{0.000000pt}%
\definecolor{currentstroke}{rgb}{0.000000,0.000000,0.000000}%
\pgfsetstrokecolor{currentstroke}%
\pgfsetdash{}{0pt}%
\pgfpathmoveto{\pgfqpoint{2.138799in}{2.643832in}}%
\pgfpathlineto{\pgfqpoint{2.058638in}{2.723993in}}%
\pgfpathlineto{\pgfqpoint{2.055502in}{2.714585in}}%
\pgfpathlineto{\pgfqpoint{2.033548in}{2.755355in}}%
\pgfpathlineto{\pgfqpoint{2.074318in}{2.733402in}}%
\pgfpathlineto{\pgfqpoint{2.064910in}{2.730266in}}%
\pgfpathlineto{\pgfqpoint{2.145072in}{2.650104in}}%
\pgfpathlineto{\pgfqpoint{2.138799in}{2.643832in}}%
\pgfusepath{fill}%
\end{pgfscope}%
\begin{pgfscope}%
\pgfpathrectangle{\pgfqpoint{1.432000in}{0.528000in}}{\pgfqpoint{3.696000in}{3.696000in}} %
\pgfusepath{clip}%
\pgfsetbuttcap%
\pgfsetroundjoin%
\definecolor{currentfill}{rgb}{0.204903,0.375746,0.553533}%
\pgfsetfillcolor{currentfill}%
\pgfsetlinewidth{0.000000pt}%
\definecolor{currentstroke}{rgb}{0.000000,0.000000,0.000000}%
\pgfsetstrokecolor{currentstroke}%
\pgfsetdash{}{0pt}%
\pgfpathmoveto{\pgfqpoint{2.247186in}{2.643832in}}%
\pgfpathlineto{\pgfqpoint{2.167025in}{2.723993in}}%
\pgfpathlineto{\pgfqpoint{2.163889in}{2.714585in}}%
\pgfpathlineto{\pgfqpoint{2.141935in}{2.755355in}}%
\pgfpathlineto{\pgfqpoint{2.182706in}{2.733402in}}%
\pgfpathlineto{\pgfqpoint{2.173297in}{2.730266in}}%
\pgfpathlineto{\pgfqpoint{2.253459in}{2.650104in}}%
\pgfpathlineto{\pgfqpoint{2.247186in}{2.643832in}}%
\pgfusepath{fill}%
\end{pgfscope}%
\begin{pgfscope}%
\pgfpathrectangle{\pgfqpoint{1.432000in}{0.528000in}}{\pgfqpoint{3.696000in}{3.696000in}} %
\pgfusepath{clip}%
\pgfsetbuttcap%
\pgfsetroundjoin%
\definecolor{currentfill}{rgb}{0.276022,0.044167,0.370164}%
\pgfsetfillcolor{currentfill}%
\pgfsetlinewidth{0.000000pt}%
\definecolor{currentstroke}{rgb}{0.000000,0.000000,0.000000}%
\pgfsetstrokecolor{currentstroke}%
\pgfsetdash{}{0pt}%
\pgfpathmoveto{\pgfqpoint{2.356726in}{2.643001in}}%
\pgfpathlineto{\pgfqpoint{2.175655in}{2.733537in}}%
\pgfpathlineto{\pgfqpoint{2.175655in}{2.723619in}}%
\pgfpathlineto{\pgfqpoint{2.141935in}{2.755355in}}%
\pgfpathlineto{\pgfqpoint{2.187556in}{2.747421in}}%
\pgfpathlineto{\pgfqpoint{2.179622in}{2.741470in}}%
\pgfpathlineto{\pgfqpoint{2.360693in}{2.650935in}}%
\pgfpathlineto{\pgfqpoint{2.356726in}{2.643001in}}%
\pgfusepath{fill}%
\end{pgfscope}%
\begin{pgfscope}%
\pgfpathrectangle{\pgfqpoint{1.432000in}{0.528000in}}{\pgfqpoint{3.696000in}{3.696000in}} %
\pgfusepath{clip}%
\pgfsetbuttcap%
\pgfsetroundjoin%
\definecolor{currentfill}{rgb}{0.246811,0.283237,0.535941}%
\pgfsetfillcolor{currentfill}%
\pgfsetlinewidth{0.000000pt}%
\definecolor{currentstroke}{rgb}{0.000000,0.000000,0.000000}%
\pgfsetstrokecolor{currentstroke}%
\pgfsetdash{}{0pt}%
\pgfpathmoveto{\pgfqpoint{2.355574in}{2.643832in}}%
\pgfpathlineto{\pgfqpoint{2.275412in}{2.723993in}}%
\pgfpathlineto{\pgfqpoint{2.272276in}{2.714585in}}%
\pgfpathlineto{\pgfqpoint{2.250323in}{2.755355in}}%
\pgfpathlineto{\pgfqpoint{2.291093in}{2.733402in}}%
\pgfpathlineto{\pgfqpoint{2.281684in}{2.730266in}}%
\pgfpathlineto{\pgfqpoint{2.361846in}{2.650104in}}%
\pgfpathlineto{\pgfqpoint{2.355574in}{2.643832in}}%
\pgfusepath{fill}%
\end{pgfscope}%
\begin{pgfscope}%
\pgfpathrectangle{\pgfqpoint{1.432000in}{0.528000in}}{\pgfqpoint{3.696000in}{3.696000in}} %
\pgfusepath{clip}%
\pgfsetbuttcap%
\pgfsetroundjoin%
\definecolor{currentfill}{rgb}{0.273006,0.204520,0.501721}%
\pgfsetfillcolor{currentfill}%
\pgfsetlinewidth{0.000000pt}%
\definecolor{currentstroke}{rgb}{0.000000,0.000000,0.000000}%
\pgfsetstrokecolor{currentstroke}%
\pgfsetdash{}{0pt}%
\pgfpathmoveto{\pgfqpoint{2.465113in}{2.643001in}}%
\pgfpathlineto{\pgfqpoint{2.284042in}{2.733537in}}%
\pgfpathlineto{\pgfqpoint{2.284042in}{2.723619in}}%
\pgfpathlineto{\pgfqpoint{2.250323in}{2.755355in}}%
\pgfpathlineto{\pgfqpoint{2.295943in}{2.747421in}}%
\pgfpathlineto{\pgfqpoint{2.288009in}{2.741470in}}%
\pgfpathlineto{\pgfqpoint{2.469080in}{2.650935in}}%
\pgfpathlineto{\pgfqpoint{2.465113in}{2.643001in}}%
\pgfusepath{fill}%
\end{pgfscope}%
\begin{pgfscope}%
\pgfpathrectangle{\pgfqpoint{1.432000in}{0.528000in}}{\pgfqpoint{3.696000in}{3.696000in}} %
\pgfusepath{clip}%
\pgfsetbuttcap%
\pgfsetroundjoin%
\definecolor{currentfill}{rgb}{0.199430,0.387607,0.554642}%
\pgfsetfillcolor{currentfill}%
\pgfsetlinewidth{0.000000pt}%
\definecolor{currentstroke}{rgb}{0.000000,0.000000,0.000000}%
\pgfsetstrokecolor{currentstroke}%
\pgfsetdash{}{0pt}%
\pgfpathmoveto{\pgfqpoint{2.573500in}{2.643001in}}%
\pgfpathlineto{\pgfqpoint{2.392429in}{2.733537in}}%
\pgfpathlineto{\pgfqpoint{2.392429in}{2.723619in}}%
\pgfpathlineto{\pgfqpoint{2.358710in}{2.755355in}}%
\pgfpathlineto{\pgfqpoint{2.404330in}{2.747421in}}%
\pgfpathlineto{\pgfqpoint{2.396396in}{2.741470in}}%
\pgfpathlineto{\pgfqpoint{2.577467in}{2.650935in}}%
\pgfpathlineto{\pgfqpoint{2.573500in}{2.643001in}}%
\pgfusepath{fill}%
\end{pgfscope}%
\begin{pgfscope}%
\pgfpathrectangle{\pgfqpoint{1.432000in}{0.528000in}}{\pgfqpoint{3.696000in}{3.696000in}} %
\pgfusepath{clip}%
\pgfsetbuttcap%
\pgfsetroundjoin%
\definecolor{currentfill}{rgb}{0.194100,0.399323,0.555565}%
\pgfsetfillcolor{currentfill}%
\pgfsetlinewidth{0.000000pt}%
\definecolor{currentstroke}{rgb}{0.000000,0.000000,0.000000}%
\pgfsetstrokecolor{currentstroke}%
\pgfsetdash{}{0pt}%
\pgfpathmoveto{\pgfqpoint{2.681887in}{2.643001in}}%
\pgfpathlineto{\pgfqpoint{2.500816in}{2.733537in}}%
\pgfpathlineto{\pgfqpoint{2.500816in}{2.723619in}}%
\pgfpathlineto{\pgfqpoint{2.467097in}{2.755355in}}%
\pgfpathlineto{\pgfqpoint{2.512717in}{2.747421in}}%
\pgfpathlineto{\pgfqpoint{2.504783in}{2.741470in}}%
\pgfpathlineto{\pgfqpoint{2.685854in}{2.650935in}}%
\pgfpathlineto{\pgfqpoint{2.681887in}{2.643001in}}%
\pgfusepath{fill}%
\end{pgfscope}%
\begin{pgfscope}%
\pgfpathrectangle{\pgfqpoint{1.432000in}{0.528000in}}{\pgfqpoint{3.696000in}{3.696000in}} %
\pgfusepath{clip}%
\pgfsetbuttcap%
\pgfsetroundjoin%
\definecolor{currentfill}{rgb}{0.280894,0.078907,0.402329}%
\pgfsetfillcolor{currentfill}%
\pgfsetlinewidth{0.000000pt}%
\definecolor{currentstroke}{rgb}{0.000000,0.000000,0.000000}%
\pgfsetstrokecolor{currentstroke}%
\pgfsetdash{}{0pt}%
\pgfpathmoveto{\pgfqpoint{2.680735in}{2.643832in}}%
\pgfpathlineto{\pgfqpoint{2.600573in}{2.723993in}}%
\pgfpathlineto{\pgfqpoint{2.597437in}{2.714585in}}%
\pgfpathlineto{\pgfqpoint{2.575484in}{2.755355in}}%
\pgfpathlineto{\pgfqpoint{2.616254in}{2.733402in}}%
\pgfpathlineto{\pgfqpoint{2.606845in}{2.730266in}}%
\pgfpathlineto{\pgfqpoint{2.687007in}{2.650104in}}%
\pgfpathlineto{\pgfqpoint{2.680735in}{2.643832in}}%
\pgfusepath{fill}%
\end{pgfscope}%
\begin{pgfscope}%
\pgfpathrectangle{\pgfqpoint{1.432000in}{0.528000in}}{\pgfqpoint{3.696000in}{3.696000in}} %
\pgfusepath{clip}%
\pgfsetbuttcap%
\pgfsetroundjoin%
\definecolor{currentfill}{rgb}{0.283072,0.130895,0.449241}%
\pgfsetfillcolor{currentfill}%
\pgfsetlinewidth{0.000000pt}%
\definecolor{currentstroke}{rgb}{0.000000,0.000000,0.000000}%
\pgfsetstrokecolor{currentstroke}%
\pgfsetdash{}{0pt}%
\pgfpathmoveto{\pgfqpoint{2.790275in}{2.643001in}}%
\pgfpathlineto{\pgfqpoint{2.609203in}{2.733537in}}%
\pgfpathlineto{\pgfqpoint{2.609203in}{2.723619in}}%
\pgfpathlineto{\pgfqpoint{2.575484in}{2.755355in}}%
\pgfpathlineto{\pgfqpoint{2.621104in}{2.747421in}}%
\pgfpathlineto{\pgfqpoint{2.613170in}{2.741470in}}%
\pgfpathlineto{\pgfqpoint{2.794242in}{2.650935in}}%
\pgfpathlineto{\pgfqpoint{2.790275in}{2.643001in}}%
\pgfusepath{fill}%
\end{pgfscope}%
\begin{pgfscope}%
\pgfpathrectangle{\pgfqpoint{1.432000in}{0.528000in}}{\pgfqpoint{3.696000in}{3.696000in}} %
\pgfusepath{clip}%
\pgfsetbuttcap%
\pgfsetroundjoin%
\definecolor{currentfill}{rgb}{0.206756,0.371758,0.553117}%
\pgfsetfillcolor{currentfill}%
\pgfsetlinewidth{0.000000pt}%
\definecolor{currentstroke}{rgb}{0.000000,0.000000,0.000000}%
\pgfsetstrokecolor{currentstroke}%
\pgfsetdash{}{0pt}%
\pgfpathmoveto{\pgfqpoint{2.789122in}{2.643832in}}%
\pgfpathlineto{\pgfqpoint{2.708960in}{2.723993in}}%
\pgfpathlineto{\pgfqpoint{2.705824in}{2.714585in}}%
\pgfpathlineto{\pgfqpoint{2.683871in}{2.755355in}}%
\pgfpathlineto{\pgfqpoint{2.724641in}{2.733402in}}%
\pgfpathlineto{\pgfqpoint{2.715233in}{2.730266in}}%
\pgfpathlineto{\pgfqpoint{2.795394in}{2.650104in}}%
\pgfpathlineto{\pgfqpoint{2.789122in}{2.643832in}}%
\pgfusepath{fill}%
\end{pgfscope}%
\begin{pgfscope}%
\pgfpathrectangle{\pgfqpoint{1.432000in}{0.528000in}}{\pgfqpoint{3.696000in}{3.696000in}} %
\pgfusepath{clip}%
\pgfsetbuttcap%
\pgfsetroundjoin%
\definecolor{currentfill}{rgb}{0.279566,0.067836,0.391917}%
\pgfsetfillcolor{currentfill}%
\pgfsetlinewidth{0.000000pt}%
\definecolor{currentstroke}{rgb}{0.000000,0.000000,0.000000}%
\pgfsetstrokecolor{currentstroke}%
\pgfsetdash{}{0pt}%
\pgfpathmoveto{\pgfqpoint{2.897509in}{2.643832in}}%
\pgfpathlineto{\pgfqpoint{2.817347in}{2.723993in}}%
\pgfpathlineto{\pgfqpoint{2.814211in}{2.714585in}}%
\pgfpathlineto{\pgfqpoint{2.792258in}{2.755355in}}%
\pgfpathlineto{\pgfqpoint{2.833028in}{2.733402in}}%
\pgfpathlineto{\pgfqpoint{2.823620in}{2.730266in}}%
\pgfpathlineto{\pgfqpoint{2.903781in}{2.650104in}}%
\pgfpathlineto{\pgfqpoint{2.897509in}{2.643832in}}%
\pgfusepath{fill}%
\end{pgfscope}%
\begin{pgfscope}%
\pgfpathrectangle{\pgfqpoint{1.432000in}{0.528000in}}{\pgfqpoint{3.696000in}{3.696000in}} %
\pgfusepath{clip}%
\pgfsetbuttcap%
\pgfsetroundjoin%
\definecolor{currentfill}{rgb}{0.282327,0.094955,0.417331}%
\pgfsetfillcolor{currentfill}%
\pgfsetlinewidth{0.000000pt}%
\definecolor{currentstroke}{rgb}{0.000000,0.000000,0.000000}%
\pgfsetstrokecolor{currentstroke}%
\pgfsetdash{}{0pt}%
\pgfpathmoveto{\pgfqpoint{2.896210in}{2.646968in}}%
\pgfpathlineto{\pgfqpoint{2.896210in}{2.715438in}}%
\pgfpathlineto{\pgfqpoint{2.887340in}{2.711003in}}%
\pgfpathlineto{\pgfqpoint{2.900645in}{2.755355in}}%
\pgfpathlineto{\pgfqpoint{2.913951in}{2.711003in}}%
\pgfpathlineto{\pgfqpoint{2.905080in}{2.715438in}}%
\pgfpathlineto{\pgfqpoint{2.905080in}{2.646968in}}%
\pgfpathlineto{\pgfqpoint{2.896210in}{2.646968in}}%
\pgfusepath{fill}%
\end{pgfscope}%
\begin{pgfscope}%
\pgfpathrectangle{\pgfqpoint{1.432000in}{0.528000in}}{\pgfqpoint{3.696000in}{3.696000in}} %
\pgfusepath{clip}%
\pgfsetbuttcap%
\pgfsetroundjoin%
\definecolor{currentfill}{rgb}{0.277134,0.185228,0.489898}%
\pgfsetfillcolor{currentfill}%
\pgfsetlinewidth{0.000000pt}%
\definecolor{currentstroke}{rgb}{0.000000,0.000000,0.000000}%
\pgfsetstrokecolor{currentstroke}%
\pgfsetdash{}{0pt}%
\pgfpathmoveto{\pgfqpoint{2.896678in}{2.644984in}}%
\pgfpathlineto{\pgfqpoint{2.806142in}{2.826056in}}%
\pgfpathlineto{\pgfqpoint{2.800192in}{2.818122in}}%
\pgfpathlineto{\pgfqpoint{2.792258in}{2.863742in}}%
\pgfpathlineto{\pgfqpoint{2.823994in}{2.830023in}}%
\pgfpathlineto{\pgfqpoint{2.814076in}{2.830023in}}%
\pgfpathlineto{\pgfqpoint{2.904612in}{2.648951in}}%
\pgfpathlineto{\pgfqpoint{2.896678in}{2.644984in}}%
\pgfusepath{fill}%
\end{pgfscope}%
\begin{pgfscope}%
\pgfpathrectangle{\pgfqpoint{1.432000in}{0.528000in}}{\pgfqpoint{3.696000in}{3.696000in}} %
\pgfusepath{clip}%
\pgfsetbuttcap%
\pgfsetroundjoin%
\definecolor{currentfill}{rgb}{0.279574,0.170599,0.479997}%
\pgfsetfillcolor{currentfill}%
\pgfsetlinewidth{0.000000pt}%
\definecolor{currentstroke}{rgb}{0.000000,0.000000,0.000000}%
\pgfsetstrokecolor{currentstroke}%
\pgfsetdash{}{0pt}%
\pgfpathmoveto{\pgfqpoint{2.896210in}{2.646968in}}%
\pgfpathlineto{\pgfqpoint{2.896210in}{2.823825in}}%
\pgfpathlineto{\pgfqpoint{2.887340in}{2.819390in}}%
\pgfpathlineto{\pgfqpoint{2.900645in}{2.863742in}}%
\pgfpathlineto{\pgfqpoint{2.913951in}{2.819390in}}%
\pgfpathlineto{\pgfqpoint{2.905080in}{2.823825in}}%
\pgfpathlineto{\pgfqpoint{2.905080in}{2.646968in}}%
\pgfpathlineto{\pgfqpoint{2.896210in}{2.646968in}}%
\pgfusepath{fill}%
\end{pgfscope}%
\begin{pgfscope}%
\pgfpathrectangle{\pgfqpoint{1.432000in}{0.528000in}}{\pgfqpoint{3.696000in}{3.696000in}} %
\pgfusepath{clip}%
\pgfsetbuttcap%
\pgfsetroundjoin%
\definecolor{currentfill}{rgb}{0.255645,0.260703,0.528312}%
\pgfsetfillcolor{currentfill}%
\pgfsetlinewidth{0.000000pt}%
\definecolor{currentstroke}{rgb}{0.000000,0.000000,0.000000}%
\pgfsetstrokecolor{currentstroke}%
\pgfsetdash{}{0pt}%
\pgfpathmoveto{\pgfqpoint{3.004597in}{2.646968in}}%
\pgfpathlineto{\pgfqpoint{3.004597in}{2.823825in}}%
\pgfpathlineto{\pgfqpoint{2.995727in}{2.819390in}}%
\pgfpathlineto{\pgfqpoint{3.009032in}{2.863742in}}%
\pgfpathlineto{\pgfqpoint{3.022338in}{2.819390in}}%
\pgfpathlineto{\pgfqpoint{3.013467in}{2.823825in}}%
\pgfpathlineto{\pgfqpoint{3.013467in}{2.646968in}}%
\pgfpathlineto{\pgfqpoint{3.004597in}{2.646968in}}%
\pgfusepath{fill}%
\end{pgfscope}%
\begin{pgfscope}%
\pgfpathrectangle{\pgfqpoint{1.432000in}{0.528000in}}{\pgfqpoint{3.696000in}{3.696000in}} %
\pgfusepath{clip}%
\pgfsetbuttcap%
\pgfsetroundjoin%
\definecolor{currentfill}{rgb}{0.282656,0.100196,0.422160}%
\pgfsetfillcolor{currentfill}%
\pgfsetlinewidth{0.000000pt}%
\definecolor{currentstroke}{rgb}{0.000000,0.000000,0.000000}%
\pgfsetstrokecolor{currentstroke}%
\pgfsetdash{}{0pt}%
\pgfpathmoveto{\pgfqpoint{3.113452in}{2.644984in}}%
\pgfpathlineto{\pgfqpoint{3.022917in}{2.826056in}}%
\pgfpathlineto{\pgfqpoint{3.016966in}{2.818122in}}%
\pgfpathlineto{\pgfqpoint{3.009032in}{2.863742in}}%
\pgfpathlineto{\pgfqpoint{3.040768in}{2.830023in}}%
\pgfpathlineto{\pgfqpoint{3.030851in}{2.830023in}}%
\pgfpathlineto{\pgfqpoint{3.121386in}{2.648951in}}%
\pgfpathlineto{\pgfqpoint{3.113452in}{2.644984in}}%
\pgfusepath{fill}%
\end{pgfscope}%
\begin{pgfscope}%
\pgfpathrectangle{\pgfqpoint{1.432000in}{0.528000in}}{\pgfqpoint{3.696000in}{3.696000in}} %
\pgfusepath{clip}%
\pgfsetbuttcap%
\pgfsetroundjoin%
\definecolor{currentfill}{rgb}{0.208623,0.367752,0.552675}%
\pgfsetfillcolor{currentfill}%
\pgfsetlinewidth{0.000000pt}%
\definecolor{currentstroke}{rgb}{0.000000,0.000000,0.000000}%
\pgfsetstrokecolor{currentstroke}%
\pgfsetdash{}{0pt}%
\pgfpathmoveto{\pgfqpoint{3.221839in}{2.644984in}}%
\pgfpathlineto{\pgfqpoint{3.131304in}{2.826056in}}%
\pgfpathlineto{\pgfqpoint{3.125353in}{2.818122in}}%
\pgfpathlineto{\pgfqpoint{3.117419in}{2.863742in}}%
\pgfpathlineto{\pgfqpoint{3.149155in}{2.830023in}}%
\pgfpathlineto{\pgfqpoint{3.139238in}{2.830023in}}%
\pgfpathlineto{\pgfqpoint{3.229773in}{2.648951in}}%
\pgfpathlineto{\pgfqpoint{3.221839in}{2.644984in}}%
\pgfusepath{fill}%
\end{pgfscope}%
\begin{pgfscope}%
\pgfpathrectangle{\pgfqpoint{1.432000in}{0.528000in}}{\pgfqpoint{3.696000in}{3.696000in}} %
\pgfusepath{clip}%
\pgfsetbuttcap%
\pgfsetroundjoin%
\definecolor{currentfill}{rgb}{0.216210,0.351535,0.550627}%
\pgfsetfillcolor{currentfill}%
\pgfsetlinewidth{0.000000pt}%
\definecolor{currentstroke}{rgb}{0.000000,0.000000,0.000000}%
\pgfsetstrokecolor{currentstroke}%
\pgfsetdash{}{0pt}%
\pgfpathmoveto{\pgfqpoint{3.331057in}{2.643832in}}%
\pgfpathlineto{\pgfqpoint{3.142509in}{2.832380in}}%
\pgfpathlineto{\pgfqpoint{3.139372in}{2.822972in}}%
\pgfpathlineto{\pgfqpoint{3.117419in}{2.863742in}}%
\pgfpathlineto{\pgfqpoint{3.158189in}{2.841789in}}%
\pgfpathlineto{\pgfqpoint{3.148781in}{2.838653in}}%
\pgfpathlineto{\pgfqpoint{3.337330in}{2.650104in}}%
\pgfpathlineto{\pgfqpoint{3.331057in}{2.643832in}}%
\pgfusepath{fill}%
\end{pgfscope}%
\begin{pgfscope}%
\pgfpathrectangle{\pgfqpoint{1.432000in}{0.528000in}}{\pgfqpoint{3.696000in}{3.696000in}} %
\pgfusepath{clip}%
\pgfsetbuttcap%
\pgfsetroundjoin%
\definecolor{currentfill}{rgb}{0.157729,0.485932,0.558013}%
\pgfsetfillcolor{currentfill}%
\pgfsetlinewidth{0.000000pt}%
\definecolor{currentstroke}{rgb}{0.000000,0.000000,0.000000}%
\pgfsetstrokecolor{currentstroke}%
\pgfsetdash{}{0pt}%
\pgfpathmoveto{\pgfqpoint{3.439444in}{2.643832in}}%
\pgfpathlineto{\pgfqpoint{3.250896in}{2.832380in}}%
\pgfpathlineto{\pgfqpoint{3.247760in}{2.822972in}}%
\pgfpathlineto{\pgfqpoint{3.225806in}{2.863742in}}%
\pgfpathlineto{\pgfqpoint{3.266577in}{2.841789in}}%
\pgfpathlineto{\pgfqpoint{3.257168in}{2.838653in}}%
\pgfpathlineto{\pgfqpoint{3.445717in}{2.650104in}}%
\pgfpathlineto{\pgfqpoint{3.439444in}{2.643832in}}%
\pgfusepath{fill}%
\end{pgfscope}%
\begin{pgfscope}%
\pgfpathrectangle{\pgfqpoint{1.432000in}{0.528000in}}{\pgfqpoint{3.696000in}{3.696000in}} %
\pgfusepath{clip}%
\pgfsetbuttcap%
\pgfsetroundjoin%
\definecolor{currentfill}{rgb}{0.151918,0.500685,0.557587}%
\pgfsetfillcolor{currentfill}%
\pgfsetlinewidth{0.000000pt}%
\definecolor{currentstroke}{rgb}{0.000000,0.000000,0.000000}%
\pgfsetstrokecolor{currentstroke}%
\pgfsetdash{}{0pt}%
\pgfpathmoveto{\pgfqpoint{3.548508in}{2.643277in}}%
\pgfpathlineto{\pgfqpoint{3.256559in}{2.837910in}}%
\pgfpathlineto{\pgfqpoint{3.255329in}{2.828069in}}%
\pgfpathlineto{\pgfqpoint{3.225806in}{2.863742in}}%
\pgfpathlineto{\pgfqpoint{3.270090in}{2.850211in}}%
\pgfpathlineto{\pgfqpoint{3.261479in}{2.845290in}}%
\pgfpathlineto{\pgfqpoint{3.553428in}{2.650658in}}%
\pgfpathlineto{\pgfqpoint{3.548508in}{2.643277in}}%
\pgfusepath{fill}%
\end{pgfscope}%
\begin{pgfscope}%
\pgfpathrectangle{\pgfqpoint{1.432000in}{0.528000in}}{\pgfqpoint{3.696000in}{3.696000in}} %
\pgfusepath{clip}%
\pgfsetbuttcap%
\pgfsetroundjoin%
\definecolor{currentfill}{rgb}{0.232815,0.732247,0.459277}%
\pgfsetfillcolor{currentfill}%
\pgfsetlinewidth{0.000000pt}%
\definecolor{currentstroke}{rgb}{0.000000,0.000000,0.000000}%
\pgfsetstrokecolor{currentstroke}%
\pgfsetdash{}{0pt}%
\pgfpathmoveto{\pgfqpoint{3.656895in}{2.643277in}}%
\pgfpathlineto{\pgfqpoint{3.364946in}{2.837910in}}%
\pgfpathlineto{\pgfqpoint{3.363716in}{2.828069in}}%
\pgfpathlineto{\pgfqpoint{3.334194in}{2.863742in}}%
\pgfpathlineto{\pgfqpoint{3.378477in}{2.850211in}}%
\pgfpathlineto{\pgfqpoint{3.369867in}{2.845290in}}%
\pgfpathlineto{\pgfqpoint{3.661815in}{2.650658in}}%
\pgfpathlineto{\pgfqpoint{3.656895in}{2.643277in}}%
\pgfusepath{fill}%
\end{pgfscope}%
\begin{pgfscope}%
\pgfpathrectangle{\pgfqpoint{1.432000in}{0.528000in}}{\pgfqpoint{3.696000in}{3.696000in}} %
\pgfusepath{clip}%
\pgfsetbuttcap%
\pgfsetroundjoin%
\definecolor{currentfill}{rgb}{0.214298,0.355619,0.551184}%
\pgfsetfillcolor{currentfill}%
\pgfsetlinewidth{0.000000pt}%
\definecolor{currentstroke}{rgb}{0.000000,0.000000,0.000000}%
\pgfsetstrokecolor{currentstroke}%
\pgfsetdash{}{0pt}%
\pgfpathmoveto{\pgfqpoint{3.765282in}{2.643277in}}%
\pgfpathlineto{\pgfqpoint{3.473333in}{2.837910in}}%
\pgfpathlineto{\pgfqpoint{3.472103in}{2.828069in}}%
\pgfpathlineto{\pgfqpoint{3.442581in}{2.863742in}}%
\pgfpathlineto{\pgfqpoint{3.486864in}{2.850211in}}%
\pgfpathlineto{\pgfqpoint{3.478254in}{2.845290in}}%
\pgfpathlineto{\pgfqpoint{3.770202in}{2.650658in}}%
\pgfpathlineto{\pgfqpoint{3.765282in}{2.643277in}}%
\pgfusepath{fill}%
\end{pgfscope}%
\begin{pgfscope}%
\pgfpathrectangle{\pgfqpoint{1.432000in}{0.528000in}}{\pgfqpoint{3.696000in}{3.696000in}} %
\pgfusepath{clip}%
\pgfsetbuttcap%
\pgfsetroundjoin%
\definecolor{currentfill}{rgb}{0.194100,0.399323,0.555565}%
\pgfsetfillcolor{currentfill}%
\pgfsetlinewidth{0.000000pt}%
\definecolor{currentstroke}{rgb}{0.000000,0.000000,0.000000}%
\pgfsetstrokecolor{currentstroke}%
\pgfsetdash{}{0pt}%
\pgfpathmoveto{\pgfqpoint{3.764606in}{2.643832in}}%
\pgfpathlineto{\pgfqpoint{3.467670in}{2.940767in}}%
\pgfpathlineto{\pgfqpoint{3.464534in}{2.931359in}}%
\pgfpathlineto{\pgfqpoint{3.442581in}{2.972129in}}%
\pgfpathlineto{\pgfqpoint{3.483351in}{2.950176in}}%
\pgfpathlineto{\pgfqpoint{3.473942in}{2.947040in}}%
\pgfpathlineto{\pgfqpoint{3.770878in}{2.650104in}}%
\pgfpathlineto{\pgfqpoint{3.764606in}{2.643832in}}%
\pgfusepath{fill}%
\end{pgfscope}%
\begin{pgfscope}%
\pgfpathrectangle{\pgfqpoint{1.432000in}{0.528000in}}{\pgfqpoint{3.696000in}{3.696000in}} %
\pgfusepath{clip}%
\pgfsetbuttcap%
\pgfsetroundjoin%
\definecolor{currentfill}{rgb}{0.252194,0.269783,0.531579}%
\pgfsetfillcolor{currentfill}%
\pgfsetlinewidth{0.000000pt}%
\definecolor{currentstroke}{rgb}{0.000000,0.000000,0.000000}%
\pgfsetstrokecolor{currentstroke}%
\pgfsetdash{}{0pt}%
\pgfpathmoveto{\pgfqpoint{3.873669in}{2.643277in}}%
\pgfpathlineto{\pgfqpoint{3.581720in}{2.837910in}}%
\pgfpathlineto{\pgfqpoint{3.580490in}{2.828069in}}%
\pgfpathlineto{\pgfqpoint{3.550968in}{2.863742in}}%
\pgfpathlineto{\pgfqpoint{3.595251in}{2.850211in}}%
\pgfpathlineto{\pgfqpoint{3.586641in}{2.845290in}}%
\pgfpathlineto{\pgfqpoint{3.878589in}{2.650658in}}%
\pgfpathlineto{\pgfqpoint{3.873669in}{2.643277in}}%
\pgfusepath{fill}%
\end{pgfscope}%
\begin{pgfscope}%
\pgfpathrectangle{\pgfqpoint{1.432000in}{0.528000in}}{\pgfqpoint{3.696000in}{3.696000in}} %
\pgfusepath{clip}%
\pgfsetbuttcap%
\pgfsetroundjoin%
\definecolor{currentfill}{rgb}{0.157729,0.485932,0.558013}%
\pgfsetfillcolor{currentfill}%
\pgfsetlinewidth{0.000000pt}%
\definecolor{currentstroke}{rgb}{0.000000,0.000000,0.000000}%
\pgfsetstrokecolor{currentstroke}%
\pgfsetdash{}{0pt}%
\pgfpathmoveto{\pgfqpoint{3.872993in}{2.643832in}}%
\pgfpathlineto{\pgfqpoint{3.576057in}{2.940767in}}%
\pgfpathlineto{\pgfqpoint{3.572921in}{2.931359in}}%
\pgfpathlineto{\pgfqpoint{3.550968in}{2.972129in}}%
\pgfpathlineto{\pgfqpoint{3.591738in}{2.950176in}}%
\pgfpathlineto{\pgfqpoint{3.582329in}{2.947040in}}%
\pgfpathlineto{\pgfqpoint{3.879265in}{2.650104in}}%
\pgfpathlineto{\pgfqpoint{3.872993in}{2.643832in}}%
\pgfusepath{fill}%
\end{pgfscope}%
\begin{pgfscope}%
\pgfpathrectangle{\pgfqpoint{1.432000in}{0.528000in}}{\pgfqpoint{3.696000in}{3.696000in}} %
\pgfusepath{clip}%
\pgfsetbuttcap%
\pgfsetroundjoin%
\definecolor{currentfill}{rgb}{0.269308,0.218818,0.509577}%
\pgfsetfillcolor{currentfill}%
\pgfsetlinewidth{0.000000pt}%
\definecolor{currentstroke}{rgb}{0.000000,0.000000,0.000000}%
\pgfsetstrokecolor{currentstroke}%
\pgfsetdash{}{0pt}%
\pgfpathmoveto{\pgfqpoint{3.982056in}{2.643277in}}%
\pgfpathlineto{\pgfqpoint{3.690107in}{2.837910in}}%
\pgfpathlineto{\pgfqpoint{3.688877in}{2.828069in}}%
\pgfpathlineto{\pgfqpoint{3.659355in}{2.863742in}}%
\pgfpathlineto{\pgfqpoint{3.703639in}{2.850211in}}%
\pgfpathlineto{\pgfqpoint{3.695028in}{2.845290in}}%
\pgfpathlineto{\pgfqpoint{3.986976in}{2.650658in}}%
\pgfpathlineto{\pgfqpoint{3.982056in}{2.643277in}}%
\pgfusepath{fill}%
\end{pgfscope}%
\begin{pgfscope}%
\pgfpathrectangle{\pgfqpoint{1.432000in}{0.528000in}}{\pgfqpoint{3.696000in}{3.696000in}} %
\pgfusepath{clip}%
\pgfsetbuttcap%
\pgfsetroundjoin%
\definecolor{currentfill}{rgb}{0.281446,0.084320,0.407414}%
\pgfsetfillcolor{currentfill}%
\pgfsetlinewidth{0.000000pt}%
\definecolor{currentstroke}{rgb}{0.000000,0.000000,0.000000}%
\pgfsetstrokecolor{currentstroke}%
\pgfsetdash{}{0pt}%
\pgfpathmoveto{\pgfqpoint{3.981855in}{2.643420in}}%
\pgfpathlineto{\pgfqpoint{3.580240in}{2.944631in}}%
\pgfpathlineto{\pgfqpoint{3.578466in}{2.934873in}}%
\pgfpathlineto{\pgfqpoint{3.550968in}{2.972129in}}%
\pgfpathlineto{\pgfqpoint{3.594433in}{2.956162in}}%
\pgfpathlineto{\pgfqpoint{3.585562in}{2.951727in}}%
\pgfpathlineto{\pgfqpoint{3.987177in}{2.650516in}}%
\pgfpathlineto{\pgfqpoint{3.981855in}{2.643420in}}%
\pgfusepath{fill}%
\end{pgfscope}%
\begin{pgfscope}%
\pgfpathrectangle{\pgfqpoint{1.432000in}{0.528000in}}{\pgfqpoint{3.696000in}{3.696000in}} %
\pgfusepath{clip}%
\pgfsetbuttcap%
\pgfsetroundjoin%
\definecolor{currentfill}{rgb}{0.199430,0.387607,0.554642}%
\pgfsetfillcolor{currentfill}%
\pgfsetlinewidth{0.000000pt}%
\definecolor{currentstroke}{rgb}{0.000000,0.000000,0.000000}%
\pgfsetstrokecolor{currentstroke}%
\pgfsetdash{}{0pt}%
\pgfpathmoveto{\pgfqpoint{3.981380in}{2.643832in}}%
\pgfpathlineto{\pgfqpoint{3.684444in}{2.940767in}}%
\pgfpathlineto{\pgfqpoint{3.681308in}{2.931359in}}%
\pgfpathlineto{\pgfqpoint{3.659355in}{2.972129in}}%
\pgfpathlineto{\pgfqpoint{3.700125in}{2.950176in}}%
\pgfpathlineto{\pgfqpoint{3.690716in}{2.947040in}}%
\pgfpathlineto{\pgfqpoint{3.987652in}{2.650104in}}%
\pgfpathlineto{\pgfqpoint{3.981380in}{2.643832in}}%
\pgfusepath{fill}%
\end{pgfscope}%
\begin{pgfscope}%
\pgfpathrectangle{\pgfqpoint{1.432000in}{0.528000in}}{\pgfqpoint{3.696000in}{3.696000in}} %
\pgfusepath{clip}%
\pgfsetbuttcap%
\pgfsetroundjoin%
\definecolor{currentfill}{rgb}{0.276022,0.044167,0.370164}%
\pgfsetfillcolor{currentfill}%
\pgfsetlinewidth{0.000000pt}%
\definecolor{currentstroke}{rgb}{0.000000,0.000000,0.000000}%
\pgfsetstrokecolor{currentstroke}%
\pgfsetdash{}{0pt}%
\pgfpathmoveto{\pgfqpoint{4.090920in}{2.643001in}}%
\pgfpathlineto{\pgfqpoint{3.693074in}{2.841924in}}%
\pgfpathlineto{\pgfqpoint{3.693074in}{2.832006in}}%
\pgfpathlineto{\pgfqpoint{3.659355in}{2.863742in}}%
\pgfpathlineto{\pgfqpoint{3.704975in}{2.855808in}}%
\pgfpathlineto{\pgfqpoint{3.697041in}{2.849858in}}%
\pgfpathlineto{\pgfqpoint{4.094887in}{2.650935in}}%
\pgfpathlineto{\pgfqpoint{4.090920in}{2.643001in}}%
\pgfusepath{fill}%
\end{pgfscope}%
\begin{pgfscope}%
\pgfpathrectangle{\pgfqpoint{1.432000in}{0.528000in}}{\pgfqpoint{3.696000in}{3.696000in}} %
\pgfusepath{clip}%
\pgfsetbuttcap%
\pgfsetroundjoin%
\definecolor{currentfill}{rgb}{0.227802,0.326594,0.546532}%
\pgfsetfillcolor{currentfill}%
\pgfsetlinewidth{0.000000pt}%
\definecolor{currentstroke}{rgb}{0.000000,0.000000,0.000000}%
\pgfsetstrokecolor{currentstroke}%
\pgfsetdash{}{0pt}%
\pgfpathmoveto{\pgfqpoint{4.090443in}{2.643277in}}%
\pgfpathlineto{\pgfqpoint{3.798495in}{2.837910in}}%
\pgfpathlineto{\pgfqpoint{3.797264in}{2.828069in}}%
\pgfpathlineto{\pgfqpoint{3.767742in}{2.863742in}}%
\pgfpathlineto{\pgfqpoint{3.812026in}{2.850211in}}%
\pgfpathlineto{\pgfqpoint{3.803415in}{2.845290in}}%
\pgfpathlineto{\pgfqpoint{4.095363in}{2.650658in}}%
\pgfpathlineto{\pgfqpoint{4.090443in}{2.643277in}}%
\pgfusepath{fill}%
\end{pgfscope}%
\begin{pgfscope}%
\pgfpathrectangle{\pgfqpoint{1.432000in}{0.528000in}}{\pgfqpoint{3.696000in}{3.696000in}} %
\pgfusepath{clip}%
\pgfsetbuttcap%
\pgfsetroundjoin%
\definecolor{currentfill}{rgb}{0.270595,0.214069,0.507052}%
\pgfsetfillcolor{currentfill}%
\pgfsetlinewidth{0.000000pt}%
\definecolor{currentstroke}{rgb}{0.000000,0.000000,0.000000}%
\pgfsetstrokecolor{currentstroke}%
\pgfsetdash{}{0pt}%
\pgfpathmoveto{\pgfqpoint{4.090242in}{2.643420in}}%
\pgfpathlineto{\pgfqpoint{3.688627in}{2.944631in}}%
\pgfpathlineto{\pgfqpoint{3.686853in}{2.934873in}}%
\pgfpathlineto{\pgfqpoint{3.659355in}{2.972129in}}%
\pgfpathlineto{\pgfqpoint{3.702820in}{2.956162in}}%
\pgfpathlineto{\pgfqpoint{3.693949in}{2.951727in}}%
\pgfpathlineto{\pgfqpoint{4.095564in}{2.650516in}}%
\pgfpathlineto{\pgfqpoint{4.090242in}{2.643420in}}%
\pgfusepath{fill}%
\end{pgfscope}%
\begin{pgfscope}%
\pgfpathrectangle{\pgfqpoint{1.432000in}{0.528000in}}{\pgfqpoint{3.696000in}{3.696000in}} %
\pgfusepath{clip}%
\pgfsetbuttcap%
\pgfsetroundjoin%
\definecolor{currentfill}{rgb}{0.231674,0.318106,0.544834}%
\pgfsetfillcolor{currentfill}%
\pgfsetlinewidth{0.000000pt}%
\definecolor{currentstroke}{rgb}{0.000000,0.000000,0.000000}%
\pgfsetstrokecolor{currentstroke}%
\pgfsetdash{}{0pt}%
\pgfpathmoveto{\pgfqpoint{4.089767in}{2.643832in}}%
\pgfpathlineto{\pgfqpoint{3.792831in}{2.940767in}}%
\pgfpathlineto{\pgfqpoint{3.789695in}{2.931359in}}%
\pgfpathlineto{\pgfqpoint{3.767742in}{2.972129in}}%
\pgfpathlineto{\pgfqpoint{3.808512in}{2.950176in}}%
\pgfpathlineto{\pgfqpoint{3.799104in}{2.947040in}}%
\pgfpathlineto{\pgfqpoint{4.096039in}{2.650104in}}%
\pgfpathlineto{\pgfqpoint{4.089767in}{2.643832in}}%
\pgfusepath{fill}%
\end{pgfscope}%
\begin{pgfscope}%
\pgfpathrectangle{\pgfqpoint{1.432000in}{0.528000in}}{\pgfqpoint{3.696000in}{3.696000in}} %
\pgfusepath{clip}%
\pgfsetbuttcap%
\pgfsetroundjoin%
\definecolor{currentfill}{rgb}{0.269944,0.014625,0.341379}%
\pgfsetfillcolor{currentfill}%
\pgfsetlinewidth{0.000000pt}%
\definecolor{currentstroke}{rgb}{0.000000,0.000000,0.000000}%
\pgfsetstrokecolor{currentstroke}%
\pgfsetdash{}{0pt}%
\pgfpathmoveto{\pgfqpoint{4.199307in}{2.643001in}}%
\pgfpathlineto{\pgfqpoint{3.801461in}{2.841924in}}%
\pgfpathlineto{\pgfqpoint{3.801461in}{2.832006in}}%
\pgfpathlineto{\pgfqpoint{3.767742in}{2.863742in}}%
\pgfpathlineto{\pgfqpoint{3.813362in}{2.855808in}}%
\pgfpathlineto{\pgfqpoint{3.805428in}{2.849858in}}%
\pgfpathlineto{\pgfqpoint{4.203274in}{2.650935in}}%
\pgfpathlineto{\pgfqpoint{4.199307in}{2.643001in}}%
\pgfusepath{fill}%
\end{pgfscope}%
\begin{pgfscope}%
\pgfpathrectangle{\pgfqpoint{1.432000in}{0.528000in}}{\pgfqpoint{3.696000in}{3.696000in}} %
\pgfusepath{clip}%
\pgfsetbuttcap%
\pgfsetroundjoin%
\definecolor{currentfill}{rgb}{0.174274,0.445044,0.557792}%
\pgfsetfillcolor{currentfill}%
\pgfsetlinewidth{0.000000pt}%
\definecolor{currentstroke}{rgb}{0.000000,0.000000,0.000000}%
\pgfsetstrokecolor{currentstroke}%
\pgfsetdash{}{0pt}%
\pgfpathmoveto{\pgfqpoint{4.198830in}{2.643277in}}%
\pgfpathlineto{\pgfqpoint{3.906882in}{2.837910in}}%
\pgfpathlineto{\pgfqpoint{3.905652in}{2.828069in}}%
\pgfpathlineto{\pgfqpoint{3.876129in}{2.863742in}}%
\pgfpathlineto{\pgfqpoint{3.920413in}{2.850211in}}%
\pgfpathlineto{\pgfqpoint{3.911802in}{2.845290in}}%
\pgfpathlineto{\pgfqpoint{4.203751in}{2.650658in}}%
\pgfpathlineto{\pgfqpoint{4.198830in}{2.643277in}}%
\pgfusepath{fill}%
\end{pgfscope}%
\begin{pgfscope}%
\pgfpathrectangle{\pgfqpoint{1.432000in}{0.528000in}}{\pgfqpoint{3.696000in}{3.696000in}} %
\pgfusepath{clip}%
\pgfsetbuttcap%
\pgfsetroundjoin%
\definecolor{currentfill}{rgb}{0.279566,0.067836,0.391917}%
\pgfsetfillcolor{currentfill}%
\pgfsetlinewidth{0.000000pt}%
\definecolor{currentstroke}{rgb}{0.000000,0.000000,0.000000}%
\pgfsetstrokecolor{currentstroke}%
\pgfsetdash{}{0pt}%
\pgfpathmoveto{\pgfqpoint{4.198629in}{2.643420in}}%
\pgfpathlineto{\pgfqpoint{3.797014in}{2.944631in}}%
\pgfpathlineto{\pgfqpoint{3.795240in}{2.934873in}}%
\pgfpathlineto{\pgfqpoint{3.767742in}{2.972129in}}%
\pgfpathlineto{\pgfqpoint{3.811207in}{2.956162in}}%
\pgfpathlineto{\pgfqpoint{3.802336in}{2.951727in}}%
\pgfpathlineto{\pgfqpoint{4.203951in}{2.650516in}}%
\pgfpathlineto{\pgfqpoint{4.198629in}{2.643420in}}%
\pgfusepath{fill}%
\end{pgfscope}%
\begin{pgfscope}%
\pgfpathrectangle{\pgfqpoint{1.432000in}{0.528000in}}{\pgfqpoint{3.696000in}{3.696000in}} %
\pgfusepath{clip}%
\pgfsetbuttcap%
\pgfsetroundjoin%
\definecolor{currentfill}{rgb}{0.169646,0.456262,0.558030}%
\pgfsetfillcolor{currentfill}%
\pgfsetlinewidth{0.000000pt}%
\definecolor{currentstroke}{rgb}{0.000000,0.000000,0.000000}%
\pgfsetstrokecolor{currentstroke}%
\pgfsetdash{}{0pt}%
\pgfpathmoveto{\pgfqpoint{4.198154in}{2.643832in}}%
\pgfpathlineto{\pgfqpoint{3.901218in}{2.940767in}}%
\pgfpathlineto{\pgfqpoint{3.898082in}{2.931359in}}%
\pgfpathlineto{\pgfqpoint{3.876129in}{2.972129in}}%
\pgfpathlineto{\pgfqpoint{3.916899in}{2.950176in}}%
\pgfpathlineto{\pgfqpoint{3.907491in}{2.947040in}}%
\pgfpathlineto{\pgfqpoint{4.204426in}{2.650104in}}%
\pgfpathlineto{\pgfqpoint{4.198154in}{2.643832in}}%
\pgfusepath{fill}%
\end{pgfscope}%
\begin{pgfscope}%
\pgfpathrectangle{\pgfqpoint{1.432000in}{0.528000in}}{\pgfqpoint{3.696000in}{3.696000in}} %
\pgfusepath{clip}%
\pgfsetbuttcap%
\pgfsetroundjoin%
\definecolor{currentfill}{rgb}{0.126453,0.570633,0.549841}%
\pgfsetfillcolor{currentfill}%
\pgfsetlinewidth{0.000000pt}%
\definecolor{currentstroke}{rgb}{0.000000,0.000000,0.000000}%
\pgfsetstrokecolor{currentstroke}%
\pgfsetdash{}{0pt}%
\pgfpathmoveto{\pgfqpoint{4.307217in}{2.643277in}}%
\pgfpathlineto{\pgfqpoint{4.015269in}{2.837910in}}%
\pgfpathlineto{\pgfqpoint{4.014039in}{2.828069in}}%
\pgfpathlineto{\pgfqpoint{3.984516in}{2.863742in}}%
\pgfpathlineto{\pgfqpoint{4.028800in}{2.850211in}}%
\pgfpathlineto{\pgfqpoint{4.020189in}{2.845290in}}%
\pgfpathlineto{\pgfqpoint{4.312138in}{2.650658in}}%
\pgfpathlineto{\pgfqpoint{4.307217in}{2.643277in}}%
\pgfusepath{fill}%
\end{pgfscope}%
\begin{pgfscope}%
\pgfpathrectangle{\pgfqpoint{1.432000in}{0.528000in}}{\pgfqpoint{3.696000in}{3.696000in}} %
\pgfusepath{clip}%
\pgfsetbuttcap%
\pgfsetroundjoin%
\definecolor{currentfill}{rgb}{0.146180,0.515413,0.556823}%
\pgfsetfillcolor{currentfill}%
\pgfsetlinewidth{0.000000pt}%
\definecolor{currentstroke}{rgb}{0.000000,0.000000,0.000000}%
\pgfsetstrokecolor{currentstroke}%
\pgfsetdash{}{0pt}%
\pgfpathmoveto{\pgfqpoint{4.306541in}{2.643832in}}%
\pgfpathlineto{\pgfqpoint{4.009605in}{2.940767in}}%
\pgfpathlineto{\pgfqpoint{4.006469in}{2.931359in}}%
\pgfpathlineto{\pgfqpoint{3.984516in}{2.972129in}}%
\pgfpathlineto{\pgfqpoint{4.025286in}{2.950176in}}%
\pgfpathlineto{\pgfqpoint{4.015878in}{2.947040in}}%
\pgfpathlineto{\pgfqpoint{4.312814in}{2.650104in}}%
\pgfpathlineto{\pgfqpoint{4.306541in}{2.643832in}}%
\pgfusepath{fill}%
\end{pgfscope}%
\begin{pgfscope}%
\pgfpathrectangle{\pgfqpoint{1.432000in}{0.528000in}}{\pgfqpoint{3.696000in}{3.696000in}} %
\pgfusepath{clip}%
\pgfsetbuttcap%
\pgfsetroundjoin%
\definecolor{currentfill}{rgb}{0.188923,0.410910,0.556326}%
\pgfsetfillcolor{currentfill}%
\pgfsetlinewidth{0.000000pt}%
\definecolor{currentstroke}{rgb}{0.000000,0.000000,0.000000}%
\pgfsetstrokecolor{currentstroke}%
\pgfsetdash{}{0pt}%
\pgfpathmoveto{\pgfqpoint{4.415604in}{2.643277in}}%
\pgfpathlineto{\pgfqpoint{4.123656in}{2.837910in}}%
\pgfpathlineto{\pgfqpoint{4.122426in}{2.828069in}}%
\pgfpathlineto{\pgfqpoint{4.092903in}{2.863742in}}%
\pgfpathlineto{\pgfqpoint{4.137187in}{2.850211in}}%
\pgfpathlineto{\pgfqpoint{4.128576in}{2.845290in}}%
\pgfpathlineto{\pgfqpoint{4.420525in}{2.650658in}}%
\pgfpathlineto{\pgfqpoint{4.415604in}{2.643277in}}%
\pgfusepath{fill}%
\end{pgfscope}%
\begin{pgfscope}%
\pgfpathrectangle{\pgfqpoint{1.432000in}{0.528000in}}{\pgfqpoint{3.696000in}{3.696000in}} %
\pgfusepath{clip}%
\pgfsetbuttcap%
\pgfsetroundjoin%
\definecolor{currentfill}{rgb}{0.283072,0.130895,0.449241}%
\pgfsetfillcolor{currentfill}%
\pgfsetlinewidth{0.000000pt}%
\definecolor{currentstroke}{rgb}{0.000000,0.000000,0.000000}%
\pgfsetstrokecolor{currentstroke}%
\pgfsetdash{}{0pt}%
\pgfpathmoveto{\pgfqpoint{4.414928in}{2.643832in}}%
\pgfpathlineto{\pgfqpoint{4.226380in}{2.832380in}}%
\pgfpathlineto{\pgfqpoint{4.223243in}{2.822972in}}%
\pgfpathlineto{\pgfqpoint{4.201290in}{2.863742in}}%
\pgfpathlineto{\pgfqpoint{4.242060in}{2.841789in}}%
\pgfpathlineto{\pgfqpoint{4.232652in}{2.838653in}}%
\pgfpathlineto{\pgfqpoint{4.421201in}{2.650104in}}%
\pgfpathlineto{\pgfqpoint{4.414928in}{2.643832in}}%
\pgfusepath{fill}%
\end{pgfscope}%
\begin{pgfscope}%
\pgfpathrectangle{\pgfqpoint{1.432000in}{0.528000in}}{\pgfqpoint{3.696000in}{3.696000in}} %
\pgfusepath{clip}%
\pgfsetbuttcap%
\pgfsetroundjoin%
\definecolor{currentfill}{rgb}{0.165117,0.467423,0.558141}%
\pgfsetfillcolor{currentfill}%
\pgfsetlinewidth{0.000000pt}%
\definecolor{currentstroke}{rgb}{0.000000,0.000000,0.000000}%
\pgfsetstrokecolor{currentstroke}%
\pgfsetdash{}{0pt}%
\pgfpathmoveto{\pgfqpoint{4.414928in}{2.643832in}}%
\pgfpathlineto{\pgfqpoint{4.117993in}{2.940767in}}%
\pgfpathlineto{\pgfqpoint{4.114856in}{2.931359in}}%
\pgfpathlineto{\pgfqpoint{4.092903in}{2.972129in}}%
\pgfpathlineto{\pgfqpoint{4.133673in}{2.950176in}}%
\pgfpathlineto{\pgfqpoint{4.124265in}{2.947040in}}%
\pgfpathlineto{\pgfqpoint{4.421201in}{2.650104in}}%
\pgfpathlineto{\pgfqpoint{4.414928in}{2.643832in}}%
\pgfusepath{fill}%
\end{pgfscope}%
\begin{pgfscope}%
\pgfpathrectangle{\pgfqpoint{1.432000in}{0.528000in}}{\pgfqpoint{3.696000in}{3.696000in}} %
\pgfusepath{clip}%
\pgfsetbuttcap%
\pgfsetroundjoin%
\definecolor{currentfill}{rgb}{0.280255,0.165693,0.476498}%
\pgfsetfillcolor{currentfill}%
\pgfsetlinewidth{0.000000pt}%
\definecolor{currentstroke}{rgb}{0.000000,0.000000,0.000000}%
\pgfsetstrokecolor{currentstroke}%
\pgfsetdash{}{0pt}%
\pgfpathmoveto{\pgfqpoint{4.414374in}{2.644508in}}%
\pgfpathlineto{\pgfqpoint{4.219742in}{2.936456in}}%
\pgfpathlineto{\pgfqpoint{4.214821in}{2.927845in}}%
\pgfpathlineto{\pgfqpoint{4.201290in}{2.972129in}}%
\pgfpathlineto{\pgfqpoint{4.236963in}{2.942607in}}%
\pgfpathlineto{\pgfqpoint{4.227122in}{2.941376in}}%
\pgfpathlineto{\pgfqpoint{4.421755in}{2.649428in}}%
\pgfpathlineto{\pgfqpoint{4.414374in}{2.644508in}}%
\pgfusepath{fill}%
\end{pgfscope}%
\begin{pgfscope}%
\pgfpathrectangle{\pgfqpoint{1.432000in}{0.528000in}}{\pgfqpoint{3.696000in}{3.696000in}} %
\pgfusepath{clip}%
\pgfsetbuttcap%
\pgfsetroundjoin%
\definecolor{currentfill}{rgb}{0.279574,0.170599,0.479997}%
\pgfsetfillcolor{currentfill}%
\pgfsetlinewidth{0.000000pt}%
\definecolor{currentstroke}{rgb}{0.000000,0.000000,0.000000}%
\pgfsetstrokecolor{currentstroke}%
\pgfsetdash{}{0pt}%
\pgfpathmoveto{\pgfqpoint{4.523991in}{2.643277in}}%
\pgfpathlineto{\pgfqpoint{4.232043in}{2.837910in}}%
\pgfpathlineto{\pgfqpoint{4.230813in}{2.828069in}}%
\pgfpathlineto{\pgfqpoint{4.201290in}{2.863742in}}%
\pgfpathlineto{\pgfqpoint{4.245574in}{2.850211in}}%
\pgfpathlineto{\pgfqpoint{4.236963in}{2.845290in}}%
\pgfpathlineto{\pgfqpoint{4.528912in}{2.650658in}}%
\pgfpathlineto{\pgfqpoint{4.523991in}{2.643277in}}%
\pgfusepath{fill}%
\end{pgfscope}%
\begin{pgfscope}%
\pgfpathrectangle{\pgfqpoint{1.432000in}{0.528000in}}{\pgfqpoint{3.696000in}{3.696000in}} %
\pgfusepath{clip}%
\pgfsetbuttcap%
\pgfsetroundjoin%
\definecolor{currentfill}{rgb}{0.239346,0.300855,0.540844}%
\pgfsetfillcolor{currentfill}%
\pgfsetlinewidth{0.000000pt}%
\definecolor{currentstroke}{rgb}{0.000000,0.000000,0.000000}%
\pgfsetstrokecolor{currentstroke}%
\pgfsetdash{}{0pt}%
\pgfpathmoveto{\pgfqpoint{4.523315in}{2.643832in}}%
\pgfpathlineto{\pgfqpoint{4.334767in}{2.832380in}}%
\pgfpathlineto{\pgfqpoint{4.331631in}{2.822972in}}%
\pgfpathlineto{\pgfqpoint{4.309677in}{2.863742in}}%
\pgfpathlineto{\pgfqpoint{4.350447in}{2.841789in}}%
\pgfpathlineto{\pgfqpoint{4.341039in}{2.838653in}}%
\pgfpathlineto{\pgfqpoint{4.529588in}{2.650104in}}%
\pgfpathlineto{\pgfqpoint{4.523315in}{2.643832in}}%
\pgfusepath{fill}%
\end{pgfscope}%
\begin{pgfscope}%
\pgfpathrectangle{\pgfqpoint{1.432000in}{0.528000in}}{\pgfqpoint{3.696000in}{3.696000in}} %
\pgfusepath{clip}%
\pgfsetbuttcap%
\pgfsetroundjoin%
\definecolor{currentfill}{rgb}{0.239346,0.300855,0.540844}%
\pgfsetfillcolor{currentfill}%
\pgfsetlinewidth{0.000000pt}%
\definecolor{currentstroke}{rgb}{0.000000,0.000000,0.000000}%
\pgfsetstrokecolor{currentstroke}%
\pgfsetdash{}{0pt}%
\pgfpathmoveto{\pgfqpoint{4.523315in}{2.643832in}}%
\pgfpathlineto{\pgfqpoint{4.226380in}{2.940767in}}%
\pgfpathlineto{\pgfqpoint{4.223243in}{2.931359in}}%
\pgfpathlineto{\pgfqpoint{4.201290in}{2.972129in}}%
\pgfpathlineto{\pgfqpoint{4.242060in}{2.950176in}}%
\pgfpathlineto{\pgfqpoint{4.232652in}{2.947040in}}%
\pgfpathlineto{\pgfqpoint{4.529588in}{2.650104in}}%
\pgfpathlineto{\pgfqpoint{4.523315in}{2.643832in}}%
\pgfusepath{fill}%
\end{pgfscope}%
\begin{pgfscope}%
\pgfpathrectangle{\pgfqpoint{1.432000in}{0.528000in}}{\pgfqpoint{3.696000in}{3.696000in}} %
\pgfusepath{clip}%
\pgfsetbuttcap%
\pgfsetroundjoin%
\definecolor{currentfill}{rgb}{0.208623,0.367752,0.552675}%
\pgfsetfillcolor{currentfill}%
\pgfsetlinewidth{0.000000pt}%
\definecolor{currentstroke}{rgb}{0.000000,0.000000,0.000000}%
\pgfsetstrokecolor{currentstroke}%
\pgfsetdash{}{0pt}%
\pgfpathmoveto{\pgfqpoint{4.522761in}{2.644508in}}%
\pgfpathlineto{\pgfqpoint{4.328129in}{2.936456in}}%
\pgfpathlineto{\pgfqpoint{4.323209in}{2.927845in}}%
\pgfpathlineto{\pgfqpoint{4.309677in}{2.972129in}}%
\pgfpathlineto{\pgfqpoint{4.345350in}{2.942607in}}%
\pgfpathlineto{\pgfqpoint{4.335510in}{2.941376in}}%
\pgfpathlineto{\pgfqpoint{4.530142in}{2.649428in}}%
\pgfpathlineto{\pgfqpoint{4.522761in}{2.644508in}}%
\pgfusepath{fill}%
\end{pgfscope}%
\begin{pgfscope}%
\pgfpathrectangle{\pgfqpoint{1.432000in}{0.528000in}}{\pgfqpoint{3.696000in}{3.696000in}} %
\pgfusepath{clip}%
\pgfsetbuttcap%
\pgfsetroundjoin%
\definecolor{currentfill}{rgb}{0.163625,0.471133,0.558148}%
\pgfsetfillcolor{currentfill}%
\pgfsetlinewidth{0.000000pt}%
\definecolor{currentstroke}{rgb}{0.000000,0.000000,0.000000}%
\pgfsetstrokecolor{currentstroke}%
\pgfsetdash{}{0pt}%
\pgfpathmoveto{\pgfqpoint{4.631703in}{2.643832in}}%
\pgfpathlineto{\pgfqpoint{4.443154in}{2.832380in}}%
\pgfpathlineto{\pgfqpoint{4.440018in}{2.822972in}}%
\pgfpathlineto{\pgfqpoint{4.418065in}{2.863742in}}%
\pgfpathlineto{\pgfqpoint{4.458835in}{2.841789in}}%
\pgfpathlineto{\pgfqpoint{4.449426in}{2.838653in}}%
\pgfpathlineto{\pgfqpoint{4.637975in}{2.650104in}}%
\pgfpathlineto{\pgfqpoint{4.631703in}{2.643832in}}%
\pgfusepath{fill}%
\end{pgfscope}%
\begin{pgfscope}%
\pgfpathrectangle{\pgfqpoint{1.432000in}{0.528000in}}{\pgfqpoint{3.696000in}{3.696000in}} %
\pgfusepath{clip}%
\pgfsetbuttcap%
\pgfsetroundjoin%
\definecolor{currentfill}{rgb}{0.168126,0.459988,0.558082}%
\pgfsetfillcolor{currentfill}%
\pgfsetlinewidth{0.000000pt}%
\definecolor{currentstroke}{rgb}{0.000000,0.000000,0.000000}%
\pgfsetstrokecolor{currentstroke}%
\pgfsetdash{}{0pt}%
\pgfpathmoveto{\pgfqpoint{4.631148in}{2.644508in}}%
\pgfpathlineto{\pgfqpoint{4.436516in}{2.936456in}}%
\pgfpathlineto{\pgfqpoint{4.431596in}{2.927845in}}%
\pgfpathlineto{\pgfqpoint{4.418065in}{2.972129in}}%
\pgfpathlineto{\pgfqpoint{4.453738in}{2.942607in}}%
\pgfpathlineto{\pgfqpoint{4.443897in}{2.941376in}}%
\pgfpathlineto{\pgfqpoint{4.638529in}{2.649428in}}%
\pgfpathlineto{\pgfqpoint{4.631148in}{2.644508in}}%
\pgfusepath{fill}%
\end{pgfscope}%
\begin{pgfscope}%
\pgfpathrectangle{\pgfqpoint{1.432000in}{0.528000in}}{\pgfqpoint{3.696000in}{3.696000in}} %
\pgfusepath{clip}%
\pgfsetbuttcap%
\pgfsetroundjoin%
\definecolor{currentfill}{rgb}{0.241237,0.296485,0.539709}%
\pgfsetfillcolor{currentfill}%
\pgfsetlinewidth{0.000000pt}%
\definecolor{currentstroke}{rgb}{0.000000,0.000000,0.000000}%
\pgfsetstrokecolor{currentstroke}%
\pgfsetdash{}{0pt}%
\pgfpathmoveto{\pgfqpoint{4.740090in}{2.643832in}}%
\pgfpathlineto{\pgfqpoint{4.551541in}{2.832380in}}%
\pgfpathlineto{\pgfqpoint{4.548405in}{2.822972in}}%
\pgfpathlineto{\pgfqpoint{4.526452in}{2.863742in}}%
\pgfpathlineto{\pgfqpoint{4.567222in}{2.841789in}}%
\pgfpathlineto{\pgfqpoint{4.557813in}{2.838653in}}%
\pgfpathlineto{\pgfqpoint{4.746362in}{2.650104in}}%
\pgfpathlineto{\pgfqpoint{4.740090in}{2.643832in}}%
\pgfusepath{fill}%
\end{pgfscope}%
\begin{pgfscope}%
\pgfpathrectangle{\pgfqpoint{1.432000in}{0.528000in}}{\pgfqpoint{3.696000in}{3.696000in}} %
\pgfusepath{clip}%
\pgfsetbuttcap%
\pgfsetroundjoin%
\definecolor{currentfill}{rgb}{0.252194,0.269783,0.531579}%
\pgfsetfillcolor{currentfill}%
\pgfsetlinewidth{0.000000pt}%
\definecolor{currentstroke}{rgb}{0.000000,0.000000,0.000000}%
\pgfsetstrokecolor{currentstroke}%
\pgfsetdash{}{0pt}%
\pgfpathmoveto{\pgfqpoint{4.739259in}{2.644984in}}%
\pgfpathlineto{\pgfqpoint{4.648723in}{2.826056in}}%
\pgfpathlineto{\pgfqpoint{4.642773in}{2.818122in}}%
\pgfpathlineto{\pgfqpoint{4.634839in}{2.863742in}}%
\pgfpathlineto{\pgfqpoint{4.666574in}{2.830023in}}%
\pgfpathlineto{\pgfqpoint{4.656657in}{2.830023in}}%
\pgfpathlineto{\pgfqpoint{4.747193in}{2.648951in}}%
\pgfpathlineto{\pgfqpoint{4.739259in}{2.644984in}}%
\pgfusepath{fill}%
\end{pgfscope}%
\begin{pgfscope}%
\pgfpathrectangle{\pgfqpoint{1.432000in}{0.528000in}}{\pgfqpoint{3.696000in}{3.696000in}} %
\pgfusepath{clip}%
\pgfsetbuttcap%
\pgfsetroundjoin%
\definecolor{currentfill}{rgb}{0.250425,0.274290,0.533103}%
\pgfsetfillcolor{currentfill}%
\pgfsetlinewidth{0.000000pt}%
\definecolor{currentstroke}{rgb}{0.000000,0.000000,0.000000}%
\pgfsetstrokecolor{currentstroke}%
\pgfsetdash{}{0pt}%
\pgfpathmoveto{\pgfqpoint{4.739535in}{2.644508in}}%
\pgfpathlineto{\pgfqpoint{4.544903in}{2.936456in}}%
\pgfpathlineto{\pgfqpoint{4.539983in}{2.927845in}}%
\pgfpathlineto{\pgfqpoint{4.526452in}{2.972129in}}%
\pgfpathlineto{\pgfqpoint{4.562125in}{2.942607in}}%
\pgfpathlineto{\pgfqpoint{4.552284in}{2.941376in}}%
\pgfpathlineto{\pgfqpoint{4.746916in}{2.649428in}}%
\pgfpathlineto{\pgfqpoint{4.739535in}{2.644508in}}%
\pgfusepath{fill}%
\end{pgfscope}%
\begin{pgfscope}%
\pgfpathrectangle{\pgfqpoint{1.432000in}{0.528000in}}{\pgfqpoint{3.696000in}{3.696000in}} %
\pgfusepath{clip}%
\pgfsetbuttcap%
\pgfsetroundjoin%
\definecolor{currentfill}{rgb}{0.241237,0.296485,0.539709}%
\pgfsetfillcolor{currentfill}%
\pgfsetlinewidth{0.000000pt}%
\definecolor{currentstroke}{rgb}{0.000000,0.000000,0.000000}%
\pgfsetstrokecolor{currentstroke}%
\pgfsetdash{}{0pt}%
\pgfpathmoveto{\pgfqpoint{4.739018in}{2.645565in}}%
\pgfpathlineto{\pgfqpoint{4.643254in}{2.932858in}}%
\pgfpathlineto{\pgfqpoint{4.636241in}{2.925845in}}%
\pgfpathlineto{\pgfqpoint{4.634839in}{2.972129in}}%
\pgfpathlineto{\pgfqpoint{4.661487in}{2.934261in}}%
\pgfpathlineto{\pgfqpoint{4.651669in}{2.935663in}}%
\pgfpathlineto{\pgfqpoint{4.747433in}{2.648370in}}%
\pgfpathlineto{\pgfqpoint{4.739018in}{2.645565in}}%
\pgfusepath{fill}%
\end{pgfscope}%
\begin{pgfscope}%
\pgfpathrectangle{\pgfqpoint{1.432000in}{0.528000in}}{\pgfqpoint{3.696000in}{3.696000in}} %
\pgfusepath{clip}%
\pgfsetbuttcap%
\pgfsetroundjoin%
\definecolor{currentfill}{rgb}{0.204903,0.375746,0.553533}%
\pgfsetfillcolor{currentfill}%
\pgfsetlinewidth{0.000000pt}%
\definecolor{currentstroke}{rgb}{0.000000,0.000000,0.000000}%
\pgfsetstrokecolor{currentstroke}%
\pgfsetdash{}{0pt}%
\pgfpathmoveto{\pgfqpoint{4.847646in}{2.644984in}}%
\pgfpathlineto{\pgfqpoint{4.757110in}{2.826056in}}%
\pgfpathlineto{\pgfqpoint{4.751160in}{2.818122in}}%
\pgfpathlineto{\pgfqpoint{4.743226in}{2.863742in}}%
\pgfpathlineto{\pgfqpoint{4.774962in}{2.830023in}}%
\pgfpathlineto{\pgfqpoint{4.765044in}{2.830023in}}%
\pgfpathlineto{\pgfqpoint{4.855580in}{2.648951in}}%
\pgfpathlineto{\pgfqpoint{4.847646in}{2.644984in}}%
\pgfusepath{fill}%
\end{pgfscope}%
\begin{pgfscope}%
\pgfpathrectangle{\pgfqpoint{1.432000in}{0.528000in}}{\pgfqpoint{3.696000in}{3.696000in}} %
\pgfusepath{clip}%
\pgfsetbuttcap%
\pgfsetroundjoin%
\definecolor{currentfill}{rgb}{0.171176,0.452530,0.557965}%
\pgfsetfillcolor{currentfill}%
\pgfsetlinewidth{0.000000pt}%
\definecolor{currentstroke}{rgb}{0.000000,0.000000,0.000000}%
\pgfsetstrokecolor{currentstroke}%
\pgfsetdash{}{0pt}%
\pgfpathmoveto{\pgfqpoint{4.847405in}{2.645565in}}%
\pgfpathlineto{\pgfqpoint{4.751641in}{2.932858in}}%
\pgfpathlineto{\pgfqpoint{4.744628in}{2.925845in}}%
\pgfpathlineto{\pgfqpoint{4.743226in}{2.972129in}}%
\pgfpathlineto{\pgfqpoint{4.769874in}{2.934261in}}%
\pgfpathlineto{\pgfqpoint{4.760056in}{2.935663in}}%
\pgfpathlineto{\pgfqpoint{4.855821in}{2.648370in}}%
\pgfpathlineto{\pgfqpoint{4.847405in}{2.645565in}}%
\pgfusepath{fill}%
\end{pgfscope}%
\begin{pgfscope}%
\pgfpathrectangle{\pgfqpoint{1.432000in}{0.528000in}}{\pgfqpoint{3.696000in}{3.696000in}} %
\pgfusepath{clip}%
\pgfsetbuttcap%
\pgfsetroundjoin%
\definecolor{currentfill}{rgb}{0.278826,0.175490,0.483397}%
\pgfsetfillcolor{currentfill}%
\pgfsetlinewidth{0.000000pt}%
\definecolor{currentstroke}{rgb}{0.000000,0.000000,0.000000}%
\pgfsetstrokecolor{currentstroke}%
\pgfsetdash{}{0pt}%
\pgfpathmoveto{\pgfqpoint{4.956033in}{2.644984in}}%
\pgfpathlineto{\pgfqpoint{4.865497in}{2.826056in}}%
\pgfpathlineto{\pgfqpoint{4.859547in}{2.818122in}}%
\pgfpathlineto{\pgfqpoint{4.851613in}{2.863742in}}%
\pgfpathlineto{\pgfqpoint{4.883349in}{2.830023in}}%
\pgfpathlineto{\pgfqpoint{4.873431in}{2.830023in}}%
\pgfpathlineto{\pgfqpoint{4.963967in}{2.648951in}}%
\pgfpathlineto{\pgfqpoint{4.956033in}{2.644984in}}%
\pgfusepath{fill}%
\end{pgfscope}%
\begin{pgfscope}%
\pgfpathrectangle{\pgfqpoint{1.432000in}{0.528000in}}{\pgfqpoint{3.696000in}{3.696000in}} %
\pgfusepath{clip}%
\pgfsetbuttcap%
\pgfsetroundjoin%
\definecolor{currentfill}{rgb}{0.267968,0.223549,0.512008}%
\pgfsetfillcolor{currentfill}%
\pgfsetlinewidth{0.000000pt}%
\definecolor{currentstroke}{rgb}{0.000000,0.000000,0.000000}%
\pgfsetstrokecolor{currentstroke}%
\pgfsetdash{}{0pt}%
\pgfpathmoveto{\pgfqpoint{4.955565in}{2.646968in}}%
\pgfpathlineto{\pgfqpoint{4.955565in}{2.823825in}}%
\pgfpathlineto{\pgfqpoint{4.946694in}{2.819390in}}%
\pgfpathlineto{\pgfqpoint{4.960000in}{2.863742in}}%
\pgfpathlineto{\pgfqpoint{4.973306in}{2.819390in}}%
\pgfpathlineto{\pgfqpoint{4.964435in}{2.823825in}}%
\pgfpathlineto{\pgfqpoint{4.964435in}{2.646968in}}%
\pgfpathlineto{\pgfqpoint{4.955565in}{2.646968in}}%
\pgfusepath{fill}%
\end{pgfscope}%
\begin{pgfscope}%
\pgfpathrectangle{\pgfqpoint{1.432000in}{0.528000in}}{\pgfqpoint{3.696000in}{3.696000in}} %
\pgfusepath{clip}%
\pgfsetbuttcap%
\pgfsetroundjoin%
\definecolor{currentfill}{rgb}{0.278826,0.175490,0.483397}%
\pgfsetfillcolor{currentfill}%
\pgfsetlinewidth{0.000000pt}%
\definecolor{currentstroke}{rgb}{0.000000,0.000000,0.000000}%
\pgfsetstrokecolor{currentstroke}%
\pgfsetdash{}{0pt}%
\pgfpathmoveto{\pgfqpoint{4.955792in}{2.645565in}}%
\pgfpathlineto{\pgfqpoint{4.860028in}{2.932858in}}%
\pgfpathlineto{\pgfqpoint{4.853015in}{2.925845in}}%
\pgfpathlineto{\pgfqpoint{4.851613in}{2.972129in}}%
\pgfpathlineto{\pgfqpoint{4.878261in}{2.934261in}}%
\pgfpathlineto{\pgfqpoint{4.868443in}{2.935663in}}%
\pgfpathlineto{\pgfqpoint{4.964208in}{2.648370in}}%
\pgfpathlineto{\pgfqpoint{4.955792in}{2.645565in}}%
\pgfusepath{fill}%
\end{pgfscope}%
\begin{pgfscope}%
\pgfpathrectangle{\pgfqpoint{1.432000in}{0.528000in}}{\pgfqpoint{3.696000in}{3.696000in}} %
\pgfusepath{clip}%
\pgfsetbuttcap%
\pgfsetroundjoin%
\definecolor{currentfill}{rgb}{0.258965,0.251537,0.524736}%
\pgfsetfillcolor{currentfill}%
\pgfsetlinewidth{0.000000pt}%
\definecolor{currentstroke}{rgb}{0.000000,0.000000,0.000000}%
\pgfsetstrokecolor{currentstroke}%
\pgfsetdash{}{0pt}%
\pgfpathmoveto{\pgfqpoint{4.955565in}{2.646968in}}%
\pgfpathlineto{\pgfqpoint{4.955565in}{2.932212in}}%
\pgfpathlineto{\pgfqpoint{4.946694in}{2.927777in}}%
\pgfpathlineto{\pgfqpoint{4.960000in}{2.972129in}}%
\pgfpathlineto{\pgfqpoint{4.973306in}{2.927777in}}%
\pgfpathlineto{\pgfqpoint{4.964435in}{2.932212in}}%
\pgfpathlineto{\pgfqpoint{4.964435in}{2.646968in}}%
\pgfpathlineto{\pgfqpoint{4.955565in}{2.646968in}}%
\pgfusepath{fill}%
\end{pgfscope}%
\begin{pgfscope}%
\pgfpathrectangle{\pgfqpoint{1.432000in}{0.528000in}}{\pgfqpoint{3.696000in}{3.696000in}} %
\pgfusepath{clip}%
\pgfsetbuttcap%
\pgfsetroundjoin%
\definecolor{currentfill}{rgb}{0.150148,0.676631,0.506589}%
\pgfsetfillcolor{currentfill}%
\pgfsetlinewidth{0.000000pt}%
\definecolor{currentstroke}{rgb}{0.000000,0.000000,0.000000}%
\pgfsetstrokecolor{currentstroke}%
\pgfsetdash{}{0pt}%
\pgfpathmoveto{\pgfqpoint{1.604435in}{2.755355in}}%
\pgfpathlineto{\pgfqpoint{1.602218in}{2.759196in}}%
\pgfpathlineto{\pgfqpoint{1.597782in}{2.759196in}}%
\pgfpathlineto{\pgfqpoint{1.595565in}{2.755355in}}%
\pgfpathlineto{\pgfqpoint{1.597782in}{2.751514in}}%
\pgfpathlineto{\pgfqpoint{1.602218in}{2.751514in}}%
\pgfpathlineto{\pgfqpoint{1.604435in}{2.755355in}}%
\pgfpathlineto{\pgfqpoint{1.602218in}{2.759196in}}%
\pgfusepath{fill}%
\end{pgfscope}%
\begin{pgfscope}%
\pgfpathrectangle{\pgfqpoint{1.432000in}{0.528000in}}{\pgfqpoint{3.696000in}{3.696000in}} %
\pgfusepath{clip}%
\pgfsetbuttcap%
\pgfsetroundjoin%
\definecolor{currentfill}{rgb}{0.128087,0.647749,0.523491}%
\pgfsetfillcolor{currentfill}%
\pgfsetlinewidth{0.000000pt}%
\definecolor{currentstroke}{rgb}{0.000000,0.000000,0.000000}%
\pgfsetstrokecolor{currentstroke}%
\pgfsetdash{}{0pt}%
\pgfpathmoveto{\pgfqpoint{1.712822in}{2.755355in}}%
\pgfpathlineto{\pgfqpoint{1.710605in}{2.759196in}}%
\pgfpathlineto{\pgfqpoint{1.706169in}{2.759196in}}%
\pgfpathlineto{\pgfqpoint{1.703952in}{2.755355in}}%
\pgfpathlineto{\pgfqpoint{1.706169in}{2.751514in}}%
\pgfpathlineto{\pgfqpoint{1.710605in}{2.751514in}}%
\pgfpathlineto{\pgfqpoint{1.712822in}{2.755355in}}%
\pgfpathlineto{\pgfqpoint{1.710605in}{2.759196in}}%
\pgfusepath{fill}%
\end{pgfscope}%
\begin{pgfscope}%
\pgfpathrectangle{\pgfqpoint{1.432000in}{0.528000in}}{\pgfqpoint{3.696000in}{3.696000in}} %
\pgfusepath{clip}%
\pgfsetbuttcap%
\pgfsetroundjoin%
\definecolor{currentfill}{rgb}{0.131172,0.555899,0.552459}%
\pgfsetfillcolor{currentfill}%
\pgfsetlinewidth{0.000000pt}%
\definecolor{currentstroke}{rgb}{0.000000,0.000000,0.000000}%
\pgfsetstrokecolor{currentstroke}%
\pgfsetdash{}{0pt}%
\pgfpathmoveto{\pgfqpoint{1.821209in}{2.755355in}}%
\pgfpathlineto{\pgfqpoint{1.818992in}{2.759196in}}%
\pgfpathlineto{\pgfqpoint{1.814557in}{2.759196in}}%
\pgfpathlineto{\pgfqpoint{1.812339in}{2.755355in}}%
\pgfpathlineto{\pgfqpoint{1.814557in}{2.751514in}}%
\pgfpathlineto{\pgfqpoint{1.818992in}{2.751514in}}%
\pgfpathlineto{\pgfqpoint{1.821209in}{2.755355in}}%
\pgfpathlineto{\pgfqpoint{1.818992in}{2.759196in}}%
\pgfusepath{fill}%
\end{pgfscope}%
\begin{pgfscope}%
\pgfpathrectangle{\pgfqpoint{1.432000in}{0.528000in}}{\pgfqpoint{3.696000in}{3.696000in}} %
\pgfusepath{clip}%
\pgfsetbuttcap%
\pgfsetroundjoin%
\definecolor{currentfill}{rgb}{0.250425,0.274290,0.533103}%
\pgfsetfillcolor{currentfill}%
\pgfsetlinewidth{0.000000pt}%
\definecolor{currentstroke}{rgb}{0.000000,0.000000,0.000000}%
\pgfsetstrokecolor{currentstroke}%
\pgfsetdash{}{0pt}%
\pgfpathmoveto{\pgfqpoint{1.929596in}{2.755355in}}%
\pgfpathlineto{\pgfqpoint{1.927379in}{2.759196in}}%
\pgfpathlineto{\pgfqpoint{1.922944in}{2.759196in}}%
\pgfpathlineto{\pgfqpoint{1.920726in}{2.755355in}}%
\pgfpathlineto{\pgfqpoint{1.922944in}{2.751514in}}%
\pgfpathlineto{\pgfqpoint{1.927379in}{2.751514in}}%
\pgfpathlineto{\pgfqpoint{1.929596in}{2.755355in}}%
\pgfpathlineto{\pgfqpoint{1.927379in}{2.759196in}}%
\pgfusepath{fill}%
\end{pgfscope}%
\begin{pgfscope}%
\pgfpathrectangle{\pgfqpoint{1.432000in}{0.528000in}}{\pgfqpoint{3.696000in}{3.696000in}} %
\pgfusepath{clip}%
\pgfsetbuttcap%
\pgfsetroundjoin%
\definecolor{currentfill}{rgb}{0.273809,0.031497,0.358853}%
\pgfsetfillcolor{currentfill}%
\pgfsetlinewidth{0.000000pt}%
\definecolor{currentstroke}{rgb}{0.000000,0.000000,0.000000}%
\pgfsetstrokecolor{currentstroke}%
\pgfsetdash{}{0pt}%
\pgfpathmoveto{\pgfqpoint{1.920726in}{2.755355in}}%
\pgfpathlineto{\pgfqpoint{1.920726in}{2.823825in}}%
\pgfpathlineto{\pgfqpoint{1.911856in}{2.819390in}}%
\pgfpathlineto{\pgfqpoint{1.925161in}{2.863742in}}%
\pgfpathlineto{\pgfqpoint{1.938467in}{2.819390in}}%
\pgfpathlineto{\pgfqpoint{1.929596in}{2.823825in}}%
\pgfpathlineto{\pgfqpoint{1.929596in}{2.755355in}}%
\pgfpathlineto{\pgfqpoint{1.920726in}{2.755355in}}%
\pgfusepath{fill}%
\end{pgfscope}%
\begin{pgfscope}%
\pgfpathrectangle{\pgfqpoint{1.432000in}{0.528000in}}{\pgfqpoint{3.696000in}{3.696000in}} %
\pgfusepath{clip}%
\pgfsetbuttcap%
\pgfsetroundjoin%
\definecolor{currentfill}{rgb}{0.282656,0.100196,0.422160}%
\pgfsetfillcolor{currentfill}%
\pgfsetlinewidth{0.000000pt}%
\definecolor{currentstroke}{rgb}{0.000000,0.000000,0.000000}%
\pgfsetstrokecolor{currentstroke}%
\pgfsetdash{}{0pt}%
\pgfpathmoveto{\pgfqpoint{2.037984in}{2.755355in}}%
\pgfpathlineto{\pgfqpoint{2.035766in}{2.759196in}}%
\pgfpathlineto{\pgfqpoint{2.031331in}{2.759196in}}%
\pgfpathlineto{\pgfqpoint{2.029113in}{2.755355in}}%
\pgfpathlineto{\pgfqpoint{2.031331in}{2.751514in}}%
\pgfpathlineto{\pgfqpoint{2.035766in}{2.751514in}}%
\pgfpathlineto{\pgfqpoint{2.037984in}{2.755355in}}%
\pgfpathlineto{\pgfqpoint{2.035766in}{2.759196in}}%
\pgfusepath{fill}%
\end{pgfscope}%
\begin{pgfscope}%
\pgfpathrectangle{\pgfqpoint{1.432000in}{0.528000in}}{\pgfqpoint{3.696000in}{3.696000in}} %
\pgfusepath{clip}%
\pgfsetbuttcap%
\pgfsetroundjoin%
\definecolor{currentfill}{rgb}{0.281412,0.155834,0.469201}%
\pgfsetfillcolor{currentfill}%
\pgfsetlinewidth{0.000000pt}%
\definecolor{currentstroke}{rgb}{0.000000,0.000000,0.000000}%
\pgfsetstrokecolor{currentstroke}%
\pgfsetdash{}{0pt}%
\pgfpathmoveto{\pgfqpoint{2.029113in}{2.755355in}}%
\pgfpathlineto{\pgfqpoint{2.029113in}{2.823825in}}%
\pgfpathlineto{\pgfqpoint{2.020243in}{2.819390in}}%
\pgfpathlineto{\pgfqpoint{2.033548in}{2.863742in}}%
\pgfpathlineto{\pgfqpoint{2.046854in}{2.819390in}}%
\pgfpathlineto{\pgfqpoint{2.037984in}{2.823825in}}%
\pgfpathlineto{\pgfqpoint{2.037984in}{2.755355in}}%
\pgfpathlineto{\pgfqpoint{2.029113in}{2.755355in}}%
\pgfusepath{fill}%
\end{pgfscope}%
\begin{pgfscope}%
\pgfpathrectangle{\pgfqpoint{1.432000in}{0.528000in}}{\pgfqpoint{3.696000in}{3.696000in}} %
\pgfusepath{clip}%
\pgfsetbuttcap%
\pgfsetroundjoin%
\definecolor{currentfill}{rgb}{0.277018,0.050344,0.375715}%
\pgfsetfillcolor{currentfill}%
\pgfsetlinewidth{0.000000pt}%
\definecolor{currentstroke}{rgb}{0.000000,0.000000,0.000000}%
\pgfsetstrokecolor{currentstroke}%
\pgfsetdash{}{0pt}%
\pgfpathmoveto{\pgfqpoint{2.141935in}{2.750920in}}%
\pgfpathlineto{\pgfqpoint{2.073465in}{2.750920in}}%
\pgfpathlineto{\pgfqpoint{2.077900in}{2.742049in}}%
\pgfpathlineto{\pgfqpoint{2.033548in}{2.755355in}}%
\pgfpathlineto{\pgfqpoint{2.077900in}{2.768660in}}%
\pgfpathlineto{\pgfqpoint{2.073465in}{2.759790in}}%
\pgfpathlineto{\pgfqpoint{2.141935in}{2.759790in}}%
\pgfpathlineto{\pgfqpoint{2.141935in}{2.750920in}}%
\pgfusepath{fill}%
\end{pgfscope}%
\begin{pgfscope}%
\pgfpathrectangle{\pgfqpoint{1.432000in}{0.528000in}}{\pgfqpoint{3.696000in}{3.696000in}} %
\pgfusepath{clip}%
\pgfsetbuttcap%
\pgfsetroundjoin%
\definecolor{currentfill}{rgb}{0.274952,0.037752,0.364543}%
\pgfsetfillcolor{currentfill}%
\pgfsetlinewidth{0.000000pt}%
\definecolor{currentstroke}{rgb}{0.000000,0.000000,0.000000}%
\pgfsetstrokecolor{currentstroke}%
\pgfsetdash{}{0pt}%
\pgfpathmoveto{\pgfqpoint{2.138799in}{2.752219in}}%
\pgfpathlineto{\pgfqpoint{2.058638in}{2.832380in}}%
\pgfpathlineto{\pgfqpoint{2.055502in}{2.822972in}}%
\pgfpathlineto{\pgfqpoint{2.033548in}{2.863742in}}%
\pgfpathlineto{\pgfqpoint{2.074318in}{2.841789in}}%
\pgfpathlineto{\pgfqpoint{2.064910in}{2.838653in}}%
\pgfpathlineto{\pgfqpoint{2.145072in}{2.758491in}}%
\pgfpathlineto{\pgfqpoint{2.138799in}{2.752219in}}%
\pgfusepath{fill}%
\end{pgfscope}%
\begin{pgfscope}%
\pgfpathrectangle{\pgfqpoint{1.432000in}{0.528000in}}{\pgfqpoint{3.696000in}{3.696000in}} %
\pgfusepath{clip}%
\pgfsetbuttcap%
\pgfsetroundjoin%
\definecolor{currentfill}{rgb}{0.273809,0.031497,0.358853}%
\pgfsetfillcolor{currentfill}%
\pgfsetlinewidth{0.000000pt}%
\definecolor{currentstroke}{rgb}{0.000000,0.000000,0.000000}%
\pgfsetstrokecolor{currentstroke}%
\pgfsetdash{}{0pt}%
\pgfpathmoveto{\pgfqpoint{2.137500in}{2.755355in}}%
\pgfpathlineto{\pgfqpoint{2.137500in}{2.823825in}}%
\pgfpathlineto{\pgfqpoint{2.128630in}{2.819390in}}%
\pgfpathlineto{\pgfqpoint{2.141935in}{2.863742in}}%
\pgfpathlineto{\pgfqpoint{2.155241in}{2.819390in}}%
\pgfpathlineto{\pgfqpoint{2.146371in}{2.823825in}}%
\pgfpathlineto{\pgfqpoint{2.146371in}{2.755355in}}%
\pgfpathlineto{\pgfqpoint{2.137500in}{2.755355in}}%
\pgfusepath{fill}%
\end{pgfscope}%
\begin{pgfscope}%
\pgfpathrectangle{\pgfqpoint{1.432000in}{0.528000in}}{\pgfqpoint{3.696000in}{3.696000in}} %
\pgfusepath{clip}%
\pgfsetbuttcap%
\pgfsetroundjoin%
\definecolor{currentfill}{rgb}{0.268510,0.009605,0.335427}%
\pgfsetfillcolor{currentfill}%
\pgfsetlinewidth{0.000000pt}%
\definecolor{currentstroke}{rgb}{0.000000,0.000000,0.000000}%
\pgfsetstrokecolor{currentstroke}%
\pgfsetdash{}{0pt}%
\pgfpathmoveto{\pgfqpoint{2.250323in}{2.750920in}}%
\pgfpathlineto{\pgfqpoint{2.181852in}{2.750920in}}%
\pgfpathlineto{\pgfqpoint{2.186287in}{2.742049in}}%
\pgfpathlineto{\pgfqpoint{2.141935in}{2.755355in}}%
\pgfpathlineto{\pgfqpoint{2.186287in}{2.768660in}}%
\pgfpathlineto{\pgfqpoint{2.181852in}{2.759790in}}%
\pgfpathlineto{\pgfqpoint{2.250323in}{2.759790in}}%
\pgfpathlineto{\pgfqpoint{2.250323in}{2.750920in}}%
\pgfusepath{fill}%
\end{pgfscope}%
\begin{pgfscope}%
\pgfpathrectangle{\pgfqpoint{1.432000in}{0.528000in}}{\pgfqpoint{3.696000in}{3.696000in}} %
\pgfusepath{clip}%
\pgfsetbuttcap%
\pgfsetroundjoin%
\definecolor{currentfill}{rgb}{0.218130,0.347432,0.550038}%
\pgfsetfillcolor{currentfill}%
\pgfsetlinewidth{0.000000pt}%
\definecolor{currentstroke}{rgb}{0.000000,0.000000,0.000000}%
\pgfsetstrokecolor{currentstroke}%
\pgfsetdash{}{0pt}%
\pgfpathmoveto{\pgfqpoint{2.247186in}{2.752219in}}%
\pgfpathlineto{\pgfqpoint{2.167025in}{2.832380in}}%
\pgfpathlineto{\pgfqpoint{2.163889in}{2.822972in}}%
\pgfpathlineto{\pgfqpoint{2.141935in}{2.863742in}}%
\pgfpathlineto{\pgfqpoint{2.182706in}{2.841789in}}%
\pgfpathlineto{\pgfqpoint{2.173297in}{2.838653in}}%
\pgfpathlineto{\pgfqpoint{2.253459in}{2.758491in}}%
\pgfpathlineto{\pgfqpoint{2.247186in}{2.752219in}}%
\pgfusepath{fill}%
\end{pgfscope}%
\begin{pgfscope}%
\pgfpathrectangle{\pgfqpoint{1.432000in}{0.528000in}}{\pgfqpoint{3.696000in}{3.696000in}} %
\pgfusepath{clip}%
\pgfsetbuttcap%
\pgfsetroundjoin%
\definecolor{currentfill}{rgb}{0.126453,0.570633,0.549841}%
\pgfsetfillcolor{currentfill}%
\pgfsetlinewidth{0.000000pt}%
\definecolor{currentstroke}{rgb}{0.000000,0.000000,0.000000}%
\pgfsetstrokecolor{currentstroke}%
\pgfsetdash{}{0pt}%
\pgfpathmoveto{\pgfqpoint{2.355574in}{2.752219in}}%
\pgfpathlineto{\pgfqpoint{2.275412in}{2.832380in}}%
\pgfpathlineto{\pgfqpoint{2.272276in}{2.822972in}}%
\pgfpathlineto{\pgfqpoint{2.250323in}{2.863742in}}%
\pgfpathlineto{\pgfqpoint{2.291093in}{2.841789in}}%
\pgfpathlineto{\pgfqpoint{2.281684in}{2.838653in}}%
\pgfpathlineto{\pgfqpoint{2.361846in}{2.758491in}}%
\pgfpathlineto{\pgfqpoint{2.355574in}{2.752219in}}%
\pgfusepath{fill}%
\end{pgfscope}%
\begin{pgfscope}%
\pgfpathrectangle{\pgfqpoint{1.432000in}{0.528000in}}{\pgfqpoint{3.696000in}{3.696000in}} %
\pgfusepath{clip}%
\pgfsetbuttcap%
\pgfsetroundjoin%
\definecolor{currentfill}{rgb}{0.175841,0.441290,0.557685}%
\pgfsetfillcolor{currentfill}%
\pgfsetlinewidth{0.000000pt}%
\definecolor{currentstroke}{rgb}{0.000000,0.000000,0.000000}%
\pgfsetstrokecolor{currentstroke}%
\pgfsetdash{}{0pt}%
\pgfpathmoveto{\pgfqpoint{2.463961in}{2.752219in}}%
\pgfpathlineto{\pgfqpoint{2.383799in}{2.832380in}}%
\pgfpathlineto{\pgfqpoint{2.380663in}{2.822972in}}%
\pgfpathlineto{\pgfqpoint{2.358710in}{2.863742in}}%
\pgfpathlineto{\pgfqpoint{2.399480in}{2.841789in}}%
\pgfpathlineto{\pgfqpoint{2.390071in}{2.838653in}}%
\pgfpathlineto{\pgfqpoint{2.470233in}{2.758491in}}%
\pgfpathlineto{\pgfqpoint{2.463961in}{2.752219in}}%
\pgfusepath{fill}%
\end{pgfscope}%
\begin{pgfscope}%
\pgfpathrectangle{\pgfqpoint{1.432000in}{0.528000in}}{\pgfqpoint{3.696000in}{3.696000in}} %
\pgfusepath{clip}%
\pgfsetbuttcap%
\pgfsetroundjoin%
\definecolor{currentfill}{rgb}{0.274952,0.037752,0.364543}%
\pgfsetfillcolor{currentfill}%
\pgfsetlinewidth{0.000000pt}%
\definecolor{currentstroke}{rgb}{0.000000,0.000000,0.000000}%
\pgfsetstrokecolor{currentstroke}%
\pgfsetdash{}{0pt}%
\pgfpathmoveto{\pgfqpoint{2.573500in}{2.751388in}}%
\pgfpathlineto{\pgfqpoint{2.392429in}{2.841924in}}%
\pgfpathlineto{\pgfqpoint{2.392429in}{2.832006in}}%
\pgfpathlineto{\pgfqpoint{2.358710in}{2.863742in}}%
\pgfpathlineto{\pgfqpoint{2.404330in}{2.855808in}}%
\pgfpathlineto{\pgfqpoint{2.396396in}{2.849858in}}%
\pgfpathlineto{\pgfqpoint{2.577467in}{2.759322in}}%
\pgfpathlineto{\pgfqpoint{2.573500in}{2.751388in}}%
\pgfusepath{fill}%
\end{pgfscope}%
\begin{pgfscope}%
\pgfpathrectangle{\pgfqpoint{1.432000in}{0.528000in}}{\pgfqpoint{3.696000in}{3.696000in}} %
\pgfusepath{clip}%
\pgfsetbuttcap%
\pgfsetroundjoin%
\definecolor{currentfill}{rgb}{0.218130,0.347432,0.550038}%
\pgfsetfillcolor{currentfill}%
\pgfsetlinewidth{0.000000pt}%
\definecolor{currentstroke}{rgb}{0.000000,0.000000,0.000000}%
\pgfsetstrokecolor{currentstroke}%
\pgfsetdash{}{0pt}%
\pgfpathmoveto{\pgfqpoint{2.572348in}{2.752219in}}%
\pgfpathlineto{\pgfqpoint{2.492186in}{2.832380in}}%
\pgfpathlineto{\pgfqpoint{2.489050in}{2.822972in}}%
\pgfpathlineto{\pgfqpoint{2.467097in}{2.863742in}}%
\pgfpathlineto{\pgfqpoint{2.507867in}{2.841789in}}%
\pgfpathlineto{\pgfqpoint{2.498458in}{2.838653in}}%
\pgfpathlineto{\pgfqpoint{2.578620in}{2.758491in}}%
\pgfpathlineto{\pgfqpoint{2.572348in}{2.752219in}}%
\pgfusepath{fill}%
\end{pgfscope}%
\begin{pgfscope}%
\pgfpathrectangle{\pgfqpoint{1.432000in}{0.528000in}}{\pgfqpoint{3.696000in}{3.696000in}} %
\pgfusepath{clip}%
\pgfsetbuttcap%
\pgfsetroundjoin%
\definecolor{currentfill}{rgb}{0.119512,0.607464,0.540218}%
\pgfsetfillcolor{currentfill}%
\pgfsetlinewidth{0.000000pt}%
\definecolor{currentstroke}{rgb}{0.000000,0.000000,0.000000}%
\pgfsetstrokecolor{currentstroke}%
\pgfsetdash{}{0pt}%
\pgfpathmoveto{\pgfqpoint{2.680735in}{2.752219in}}%
\pgfpathlineto{\pgfqpoint{2.600573in}{2.832380in}}%
\pgfpathlineto{\pgfqpoint{2.597437in}{2.822972in}}%
\pgfpathlineto{\pgfqpoint{2.575484in}{2.863742in}}%
\pgfpathlineto{\pgfqpoint{2.616254in}{2.841789in}}%
\pgfpathlineto{\pgfqpoint{2.606845in}{2.838653in}}%
\pgfpathlineto{\pgfqpoint{2.687007in}{2.758491in}}%
\pgfpathlineto{\pgfqpoint{2.680735in}{2.752219in}}%
\pgfusepath{fill}%
\end{pgfscope}%
\begin{pgfscope}%
\pgfpathrectangle{\pgfqpoint{1.432000in}{0.528000in}}{\pgfqpoint{3.696000in}{3.696000in}} %
\pgfusepath{clip}%
\pgfsetbuttcap%
\pgfsetroundjoin%
\definecolor{currentfill}{rgb}{0.180653,0.701402,0.488189}%
\pgfsetfillcolor{currentfill}%
\pgfsetlinewidth{0.000000pt}%
\definecolor{currentstroke}{rgb}{0.000000,0.000000,0.000000}%
\pgfsetstrokecolor{currentstroke}%
\pgfsetdash{}{0pt}%
\pgfpathmoveto{\pgfqpoint{2.789122in}{2.752219in}}%
\pgfpathlineto{\pgfqpoint{2.708960in}{2.832380in}}%
\pgfpathlineto{\pgfqpoint{2.705824in}{2.822972in}}%
\pgfpathlineto{\pgfqpoint{2.683871in}{2.863742in}}%
\pgfpathlineto{\pgfqpoint{2.724641in}{2.841789in}}%
\pgfpathlineto{\pgfqpoint{2.715233in}{2.838653in}}%
\pgfpathlineto{\pgfqpoint{2.795394in}{2.758491in}}%
\pgfpathlineto{\pgfqpoint{2.789122in}{2.752219in}}%
\pgfusepath{fill}%
\end{pgfscope}%
\begin{pgfscope}%
\pgfpathrectangle{\pgfqpoint{1.432000in}{0.528000in}}{\pgfqpoint{3.696000in}{3.696000in}} %
\pgfusepath{clip}%
\pgfsetbuttcap%
\pgfsetroundjoin%
\definecolor{currentfill}{rgb}{0.206756,0.371758,0.553117}%
\pgfsetfillcolor{currentfill}%
\pgfsetlinewidth{0.000000pt}%
\definecolor{currentstroke}{rgb}{0.000000,0.000000,0.000000}%
\pgfsetstrokecolor{currentstroke}%
\pgfsetdash{}{0pt}%
\pgfpathmoveto{\pgfqpoint{2.897509in}{2.752219in}}%
\pgfpathlineto{\pgfqpoint{2.817347in}{2.832380in}}%
\pgfpathlineto{\pgfqpoint{2.814211in}{2.822972in}}%
\pgfpathlineto{\pgfqpoint{2.792258in}{2.863742in}}%
\pgfpathlineto{\pgfqpoint{2.833028in}{2.841789in}}%
\pgfpathlineto{\pgfqpoint{2.823620in}{2.838653in}}%
\pgfpathlineto{\pgfqpoint{2.903781in}{2.758491in}}%
\pgfpathlineto{\pgfqpoint{2.897509in}{2.752219in}}%
\pgfusepath{fill}%
\end{pgfscope}%
\begin{pgfscope}%
\pgfpathrectangle{\pgfqpoint{1.432000in}{0.528000in}}{\pgfqpoint{3.696000in}{3.696000in}} %
\pgfusepath{clip}%
\pgfsetbuttcap%
\pgfsetroundjoin%
\definecolor{currentfill}{rgb}{0.265145,0.232956,0.516599}%
\pgfsetfillcolor{currentfill}%
\pgfsetlinewidth{0.000000pt}%
\definecolor{currentstroke}{rgb}{0.000000,0.000000,0.000000}%
\pgfsetstrokecolor{currentstroke}%
\pgfsetdash{}{0pt}%
\pgfpathmoveto{\pgfqpoint{2.896210in}{2.755355in}}%
\pgfpathlineto{\pgfqpoint{2.896210in}{2.823825in}}%
\pgfpathlineto{\pgfqpoint{2.887340in}{2.819390in}}%
\pgfpathlineto{\pgfqpoint{2.900645in}{2.863742in}}%
\pgfpathlineto{\pgfqpoint{2.913951in}{2.819390in}}%
\pgfpathlineto{\pgfqpoint{2.905080in}{2.823825in}}%
\pgfpathlineto{\pgfqpoint{2.905080in}{2.755355in}}%
\pgfpathlineto{\pgfqpoint{2.896210in}{2.755355in}}%
\pgfusepath{fill}%
\end{pgfscope}%
\begin{pgfscope}%
\pgfpathrectangle{\pgfqpoint{1.432000in}{0.528000in}}{\pgfqpoint{3.696000in}{3.696000in}} %
\pgfusepath{clip}%
\pgfsetbuttcap%
\pgfsetroundjoin%
\definecolor{currentfill}{rgb}{0.278826,0.175490,0.483397}%
\pgfsetfillcolor{currentfill}%
\pgfsetlinewidth{0.000000pt}%
\definecolor{currentstroke}{rgb}{0.000000,0.000000,0.000000}%
\pgfsetstrokecolor{currentstroke}%
\pgfsetdash{}{0pt}%
\pgfpathmoveto{\pgfqpoint{3.004597in}{2.755355in}}%
\pgfpathlineto{\pgfqpoint{3.004597in}{2.823825in}}%
\pgfpathlineto{\pgfqpoint{2.995727in}{2.819390in}}%
\pgfpathlineto{\pgfqpoint{3.009032in}{2.863742in}}%
\pgfpathlineto{\pgfqpoint{3.022338in}{2.819390in}}%
\pgfpathlineto{\pgfqpoint{3.013467in}{2.823825in}}%
\pgfpathlineto{\pgfqpoint{3.013467in}{2.755355in}}%
\pgfpathlineto{\pgfqpoint{3.004597in}{2.755355in}}%
\pgfusepath{fill}%
\end{pgfscope}%
\begin{pgfscope}%
\pgfpathrectangle{\pgfqpoint{1.432000in}{0.528000in}}{\pgfqpoint{3.696000in}{3.696000in}} %
\pgfusepath{clip}%
\pgfsetbuttcap%
\pgfsetroundjoin%
\definecolor{currentfill}{rgb}{0.218130,0.347432,0.550038}%
\pgfsetfillcolor{currentfill}%
\pgfsetlinewidth{0.000000pt}%
\definecolor{currentstroke}{rgb}{0.000000,0.000000,0.000000}%
\pgfsetstrokecolor{currentstroke}%
\pgfsetdash{}{0pt}%
\pgfpathmoveto{\pgfqpoint{3.004597in}{2.755355in}}%
\pgfpathlineto{\pgfqpoint{3.004597in}{2.932212in}}%
\pgfpathlineto{\pgfqpoint{2.995727in}{2.927777in}}%
\pgfpathlineto{\pgfqpoint{3.009032in}{2.972129in}}%
\pgfpathlineto{\pgfqpoint{3.022338in}{2.927777in}}%
\pgfpathlineto{\pgfqpoint{3.013467in}{2.932212in}}%
\pgfpathlineto{\pgfqpoint{3.013467in}{2.755355in}}%
\pgfpathlineto{\pgfqpoint{3.004597in}{2.755355in}}%
\pgfusepath{fill}%
\end{pgfscope}%
\begin{pgfscope}%
\pgfpathrectangle{\pgfqpoint{1.432000in}{0.528000in}}{\pgfqpoint{3.696000in}{3.696000in}} %
\pgfusepath{clip}%
\pgfsetbuttcap%
\pgfsetroundjoin%
\definecolor{currentfill}{rgb}{0.273006,0.204520,0.501721}%
\pgfsetfillcolor{currentfill}%
\pgfsetlinewidth{0.000000pt}%
\definecolor{currentstroke}{rgb}{0.000000,0.000000,0.000000}%
\pgfsetstrokecolor{currentstroke}%
\pgfsetdash{}{0pt}%
\pgfpathmoveto{\pgfqpoint{3.112984in}{2.755355in}}%
\pgfpathlineto{\pgfqpoint{3.112984in}{2.932212in}}%
\pgfpathlineto{\pgfqpoint{3.104114in}{2.927777in}}%
\pgfpathlineto{\pgfqpoint{3.117419in}{2.972129in}}%
\pgfpathlineto{\pgfqpoint{3.130725in}{2.927777in}}%
\pgfpathlineto{\pgfqpoint{3.121855in}{2.932212in}}%
\pgfpathlineto{\pgfqpoint{3.121855in}{2.755355in}}%
\pgfpathlineto{\pgfqpoint{3.112984in}{2.755355in}}%
\pgfusepath{fill}%
\end{pgfscope}%
\begin{pgfscope}%
\pgfpathrectangle{\pgfqpoint{1.432000in}{0.528000in}}{\pgfqpoint{3.696000in}{3.696000in}} %
\pgfusepath{clip}%
\pgfsetbuttcap%
\pgfsetroundjoin%
\definecolor{currentfill}{rgb}{0.227802,0.326594,0.546532}%
\pgfsetfillcolor{currentfill}%
\pgfsetlinewidth{0.000000pt}%
\definecolor{currentstroke}{rgb}{0.000000,0.000000,0.000000}%
\pgfsetstrokecolor{currentstroke}%
\pgfsetdash{}{0pt}%
\pgfpathmoveto{\pgfqpoint{3.221839in}{2.753371in}}%
\pgfpathlineto{\pgfqpoint{3.131304in}{2.934443in}}%
\pgfpathlineto{\pgfqpoint{3.125353in}{2.926509in}}%
\pgfpathlineto{\pgfqpoint{3.117419in}{2.972129in}}%
\pgfpathlineto{\pgfqpoint{3.149155in}{2.938410in}}%
\pgfpathlineto{\pgfqpoint{3.139238in}{2.938410in}}%
\pgfpathlineto{\pgfqpoint{3.229773in}{2.757338in}}%
\pgfpathlineto{\pgfqpoint{3.221839in}{2.753371in}}%
\pgfusepath{fill}%
\end{pgfscope}%
\begin{pgfscope}%
\pgfpathrectangle{\pgfqpoint{1.432000in}{0.528000in}}{\pgfqpoint{3.696000in}{3.696000in}} %
\pgfusepath{clip}%
\pgfsetbuttcap%
\pgfsetroundjoin%
\definecolor{currentfill}{rgb}{0.279574,0.170599,0.479997}%
\pgfsetfillcolor{currentfill}%
\pgfsetlinewidth{0.000000pt}%
\definecolor{currentstroke}{rgb}{0.000000,0.000000,0.000000}%
\pgfsetstrokecolor{currentstroke}%
\pgfsetdash{}{0pt}%
\pgfpathmoveto{\pgfqpoint{3.330227in}{2.753371in}}%
\pgfpathlineto{\pgfqpoint{3.239691in}{2.934443in}}%
\pgfpathlineto{\pgfqpoint{3.233740in}{2.926509in}}%
\pgfpathlineto{\pgfqpoint{3.225806in}{2.972129in}}%
\pgfpathlineto{\pgfqpoint{3.257542in}{2.938410in}}%
\pgfpathlineto{\pgfqpoint{3.247625in}{2.938410in}}%
\pgfpathlineto{\pgfqpoint{3.338161in}{2.757338in}}%
\pgfpathlineto{\pgfqpoint{3.330227in}{2.753371in}}%
\pgfusepath{fill}%
\end{pgfscope}%
\begin{pgfscope}%
\pgfpathrectangle{\pgfqpoint{1.432000in}{0.528000in}}{\pgfqpoint{3.696000in}{3.696000in}} %
\pgfusepath{clip}%
\pgfsetbuttcap%
\pgfsetroundjoin%
\definecolor{currentfill}{rgb}{0.143343,0.522773,0.556295}%
\pgfsetfillcolor{currentfill}%
\pgfsetlinewidth{0.000000pt}%
\definecolor{currentstroke}{rgb}{0.000000,0.000000,0.000000}%
\pgfsetstrokecolor{currentstroke}%
\pgfsetdash{}{0pt}%
\pgfpathmoveto{\pgfqpoint{3.439444in}{2.752219in}}%
\pgfpathlineto{\pgfqpoint{3.250896in}{2.940767in}}%
\pgfpathlineto{\pgfqpoint{3.247760in}{2.931359in}}%
\pgfpathlineto{\pgfqpoint{3.225806in}{2.972129in}}%
\pgfpathlineto{\pgfqpoint{3.266577in}{2.950176in}}%
\pgfpathlineto{\pgfqpoint{3.257168in}{2.947040in}}%
\pgfpathlineto{\pgfqpoint{3.445717in}{2.758491in}}%
\pgfpathlineto{\pgfqpoint{3.439444in}{2.752219in}}%
\pgfusepath{fill}%
\end{pgfscope}%
\begin{pgfscope}%
\pgfpathrectangle{\pgfqpoint{1.432000in}{0.528000in}}{\pgfqpoint{3.696000in}{3.696000in}} %
\pgfusepath{clip}%
\pgfsetbuttcap%
\pgfsetroundjoin%
\definecolor{currentfill}{rgb}{0.253935,0.265254,0.529983}%
\pgfsetfillcolor{currentfill}%
\pgfsetlinewidth{0.000000pt}%
\definecolor{currentstroke}{rgb}{0.000000,0.000000,0.000000}%
\pgfsetstrokecolor{currentstroke}%
\pgfsetdash{}{0pt}%
\pgfpathmoveto{\pgfqpoint{3.548508in}{2.751665in}}%
\pgfpathlineto{\pgfqpoint{3.256559in}{2.946297in}}%
\pgfpathlineto{\pgfqpoint{3.255329in}{2.936456in}}%
\pgfpathlineto{\pgfqpoint{3.225806in}{2.972129in}}%
\pgfpathlineto{\pgfqpoint{3.270090in}{2.958598in}}%
\pgfpathlineto{\pgfqpoint{3.261479in}{2.953677in}}%
\pgfpathlineto{\pgfqpoint{3.553428in}{2.759045in}}%
\pgfpathlineto{\pgfqpoint{3.548508in}{2.751665in}}%
\pgfusepath{fill}%
\end{pgfscope}%
\begin{pgfscope}%
\pgfpathrectangle{\pgfqpoint{1.432000in}{0.528000in}}{\pgfqpoint{3.696000in}{3.696000in}} %
\pgfusepath{clip}%
\pgfsetbuttcap%
\pgfsetroundjoin%
\definecolor{currentfill}{rgb}{0.237441,0.305202,0.541921}%
\pgfsetfillcolor{currentfill}%
\pgfsetlinewidth{0.000000pt}%
\definecolor{currentstroke}{rgb}{0.000000,0.000000,0.000000}%
\pgfsetstrokecolor{currentstroke}%
\pgfsetdash{}{0pt}%
\pgfpathmoveto{\pgfqpoint{3.547832in}{2.752219in}}%
\pgfpathlineto{\pgfqpoint{3.359283in}{2.940767in}}%
\pgfpathlineto{\pgfqpoint{3.356147in}{2.931359in}}%
\pgfpathlineto{\pgfqpoint{3.334194in}{2.972129in}}%
\pgfpathlineto{\pgfqpoint{3.374964in}{2.950176in}}%
\pgfpathlineto{\pgfqpoint{3.365555in}{2.947040in}}%
\pgfpathlineto{\pgfqpoint{3.554104in}{2.758491in}}%
\pgfpathlineto{\pgfqpoint{3.547832in}{2.752219in}}%
\pgfusepath{fill}%
\end{pgfscope}%
\begin{pgfscope}%
\pgfpathrectangle{\pgfqpoint{1.432000in}{0.528000in}}{\pgfqpoint{3.696000in}{3.696000in}} %
\pgfusepath{clip}%
\pgfsetbuttcap%
\pgfsetroundjoin%
\definecolor{currentfill}{rgb}{0.157851,0.683765,0.501686}%
\pgfsetfillcolor{currentfill}%
\pgfsetlinewidth{0.000000pt}%
\definecolor{currentstroke}{rgb}{0.000000,0.000000,0.000000}%
\pgfsetstrokecolor{currentstroke}%
\pgfsetdash{}{0pt}%
\pgfpathmoveto{\pgfqpoint{3.656895in}{2.751665in}}%
\pgfpathlineto{\pgfqpoint{3.364946in}{2.946297in}}%
\pgfpathlineto{\pgfqpoint{3.363716in}{2.936456in}}%
\pgfpathlineto{\pgfqpoint{3.334194in}{2.972129in}}%
\pgfpathlineto{\pgfqpoint{3.378477in}{2.958598in}}%
\pgfpathlineto{\pgfqpoint{3.369867in}{2.953677in}}%
\pgfpathlineto{\pgfqpoint{3.661815in}{2.759045in}}%
\pgfpathlineto{\pgfqpoint{3.656895in}{2.751665in}}%
\pgfusepath{fill}%
\end{pgfscope}%
\begin{pgfscope}%
\pgfpathrectangle{\pgfqpoint{1.432000in}{0.528000in}}{\pgfqpoint{3.696000in}{3.696000in}} %
\pgfusepath{clip}%
\pgfsetbuttcap%
\pgfsetroundjoin%
\definecolor{currentfill}{rgb}{0.204903,0.375746,0.553533}%
\pgfsetfillcolor{currentfill}%
\pgfsetlinewidth{0.000000pt}%
\definecolor{currentstroke}{rgb}{0.000000,0.000000,0.000000}%
\pgfsetstrokecolor{currentstroke}%
\pgfsetdash{}{0pt}%
\pgfpathmoveto{\pgfqpoint{3.765282in}{2.751665in}}%
\pgfpathlineto{\pgfqpoint{3.473333in}{2.946297in}}%
\pgfpathlineto{\pgfqpoint{3.472103in}{2.936456in}}%
\pgfpathlineto{\pgfqpoint{3.442581in}{2.972129in}}%
\pgfpathlineto{\pgfqpoint{3.486864in}{2.958598in}}%
\pgfpathlineto{\pgfqpoint{3.478254in}{2.953677in}}%
\pgfpathlineto{\pgfqpoint{3.770202in}{2.759045in}}%
\pgfpathlineto{\pgfqpoint{3.765282in}{2.751665in}}%
\pgfusepath{fill}%
\end{pgfscope}%
\begin{pgfscope}%
\pgfpathrectangle{\pgfqpoint{1.432000in}{0.528000in}}{\pgfqpoint{3.696000in}{3.696000in}} %
\pgfusepath{clip}%
\pgfsetbuttcap%
\pgfsetroundjoin%
\definecolor{currentfill}{rgb}{0.179019,0.433756,0.557430}%
\pgfsetfillcolor{currentfill}%
\pgfsetlinewidth{0.000000pt}%
\definecolor{currentstroke}{rgb}{0.000000,0.000000,0.000000}%
\pgfsetstrokecolor{currentstroke}%
\pgfsetdash{}{0pt}%
\pgfpathmoveto{\pgfqpoint{3.764606in}{2.752219in}}%
\pgfpathlineto{\pgfqpoint{3.467670in}{3.049155in}}%
\pgfpathlineto{\pgfqpoint{3.464534in}{3.039746in}}%
\pgfpathlineto{\pgfqpoint{3.442581in}{3.080516in}}%
\pgfpathlineto{\pgfqpoint{3.483351in}{3.058563in}}%
\pgfpathlineto{\pgfqpoint{3.473942in}{3.055427in}}%
\pgfpathlineto{\pgfqpoint{3.770878in}{2.758491in}}%
\pgfpathlineto{\pgfqpoint{3.764606in}{2.752219in}}%
\pgfusepath{fill}%
\end{pgfscope}%
\begin{pgfscope}%
\pgfpathrectangle{\pgfqpoint{1.432000in}{0.528000in}}{\pgfqpoint{3.696000in}{3.696000in}} %
\pgfusepath{clip}%
\pgfsetbuttcap%
\pgfsetroundjoin%
\definecolor{currentfill}{rgb}{0.276022,0.044167,0.370164}%
\pgfsetfillcolor{currentfill}%
\pgfsetlinewidth{0.000000pt}%
\definecolor{currentstroke}{rgb}{0.000000,0.000000,0.000000}%
\pgfsetstrokecolor{currentstroke}%
\pgfsetdash{}{0pt}%
\pgfpathmoveto{\pgfqpoint{3.873669in}{2.751665in}}%
\pgfpathlineto{\pgfqpoint{3.581720in}{2.946297in}}%
\pgfpathlineto{\pgfqpoint{3.580490in}{2.936456in}}%
\pgfpathlineto{\pgfqpoint{3.550968in}{2.972129in}}%
\pgfpathlineto{\pgfqpoint{3.595251in}{2.958598in}}%
\pgfpathlineto{\pgfqpoint{3.586641in}{2.953677in}}%
\pgfpathlineto{\pgfqpoint{3.878589in}{2.759045in}}%
\pgfpathlineto{\pgfqpoint{3.873669in}{2.751665in}}%
\pgfusepath{fill}%
\end{pgfscope}%
\begin{pgfscope}%
\pgfpathrectangle{\pgfqpoint{1.432000in}{0.528000in}}{\pgfqpoint{3.696000in}{3.696000in}} %
\pgfusepath{clip}%
\pgfsetbuttcap%
\pgfsetroundjoin%
\definecolor{currentfill}{rgb}{0.119738,0.603785,0.541400}%
\pgfsetfillcolor{currentfill}%
\pgfsetlinewidth{0.000000pt}%
\definecolor{currentstroke}{rgb}{0.000000,0.000000,0.000000}%
\pgfsetstrokecolor{currentstroke}%
\pgfsetdash{}{0pt}%
\pgfpathmoveto{\pgfqpoint{3.872993in}{2.752219in}}%
\pgfpathlineto{\pgfqpoint{3.576057in}{3.049155in}}%
\pgfpathlineto{\pgfqpoint{3.572921in}{3.039746in}}%
\pgfpathlineto{\pgfqpoint{3.550968in}{3.080516in}}%
\pgfpathlineto{\pgfqpoint{3.591738in}{3.058563in}}%
\pgfpathlineto{\pgfqpoint{3.582329in}{3.055427in}}%
\pgfpathlineto{\pgfqpoint{3.879265in}{2.758491in}}%
\pgfpathlineto{\pgfqpoint{3.872993in}{2.752219in}}%
\pgfusepath{fill}%
\end{pgfscope}%
\begin{pgfscope}%
\pgfpathrectangle{\pgfqpoint{1.432000in}{0.528000in}}{\pgfqpoint{3.696000in}{3.696000in}} %
\pgfusepath{clip}%
\pgfsetbuttcap%
\pgfsetroundjoin%
\definecolor{currentfill}{rgb}{0.283091,0.110553,0.431554}%
\pgfsetfillcolor{currentfill}%
\pgfsetlinewidth{0.000000pt}%
\definecolor{currentstroke}{rgb}{0.000000,0.000000,0.000000}%
\pgfsetstrokecolor{currentstroke}%
\pgfsetdash{}{0pt}%
\pgfpathmoveto{\pgfqpoint{3.981855in}{2.751807in}}%
\pgfpathlineto{\pgfqpoint{3.580240in}{3.053018in}}%
\pgfpathlineto{\pgfqpoint{3.578466in}{3.043260in}}%
\pgfpathlineto{\pgfqpoint{3.550968in}{3.080516in}}%
\pgfpathlineto{\pgfqpoint{3.594433in}{3.064549in}}%
\pgfpathlineto{\pgfqpoint{3.585562in}{3.060114in}}%
\pgfpathlineto{\pgfqpoint{3.987177in}{2.758903in}}%
\pgfpathlineto{\pgfqpoint{3.981855in}{2.751807in}}%
\pgfusepath{fill}%
\end{pgfscope}%
\begin{pgfscope}%
\pgfpathrectangle{\pgfqpoint{1.432000in}{0.528000in}}{\pgfqpoint{3.696000in}{3.696000in}} %
\pgfusepath{clip}%
\pgfsetbuttcap%
\pgfsetroundjoin%
\definecolor{currentfill}{rgb}{0.140536,0.530132,0.555659}%
\pgfsetfillcolor{currentfill}%
\pgfsetlinewidth{0.000000pt}%
\definecolor{currentstroke}{rgb}{0.000000,0.000000,0.000000}%
\pgfsetstrokecolor{currentstroke}%
\pgfsetdash{}{0pt}%
\pgfpathmoveto{\pgfqpoint{3.981380in}{2.752219in}}%
\pgfpathlineto{\pgfqpoint{3.684444in}{3.049155in}}%
\pgfpathlineto{\pgfqpoint{3.681308in}{3.039746in}}%
\pgfpathlineto{\pgfqpoint{3.659355in}{3.080516in}}%
\pgfpathlineto{\pgfqpoint{3.700125in}{3.058563in}}%
\pgfpathlineto{\pgfqpoint{3.690716in}{3.055427in}}%
\pgfpathlineto{\pgfqpoint{3.987652in}{2.758491in}}%
\pgfpathlineto{\pgfqpoint{3.981380in}{2.752219in}}%
\pgfusepath{fill}%
\end{pgfscope}%
\begin{pgfscope}%
\pgfpathrectangle{\pgfqpoint{1.432000in}{0.528000in}}{\pgfqpoint{3.696000in}{3.696000in}} %
\pgfusepath{clip}%
\pgfsetbuttcap%
\pgfsetroundjoin%
\definecolor{currentfill}{rgb}{0.276194,0.190074,0.493001}%
\pgfsetfillcolor{currentfill}%
\pgfsetlinewidth{0.000000pt}%
\definecolor{currentstroke}{rgb}{0.000000,0.000000,0.000000}%
\pgfsetstrokecolor{currentstroke}%
\pgfsetdash{}{0pt}%
\pgfpathmoveto{\pgfqpoint{4.090242in}{2.751807in}}%
\pgfpathlineto{\pgfqpoint{3.688627in}{3.053018in}}%
\pgfpathlineto{\pgfqpoint{3.686853in}{3.043260in}}%
\pgfpathlineto{\pgfqpoint{3.659355in}{3.080516in}}%
\pgfpathlineto{\pgfqpoint{3.702820in}{3.064549in}}%
\pgfpathlineto{\pgfqpoint{3.693949in}{3.060114in}}%
\pgfpathlineto{\pgfqpoint{4.095564in}{2.758903in}}%
\pgfpathlineto{\pgfqpoint{4.090242in}{2.751807in}}%
\pgfusepath{fill}%
\end{pgfscope}%
\begin{pgfscope}%
\pgfpathrectangle{\pgfqpoint{1.432000in}{0.528000in}}{\pgfqpoint{3.696000in}{3.696000in}} %
\pgfusepath{clip}%
\pgfsetbuttcap%
\pgfsetroundjoin%
\definecolor{currentfill}{rgb}{0.175841,0.441290,0.557685}%
\pgfsetfillcolor{currentfill}%
\pgfsetlinewidth{0.000000pt}%
\definecolor{currentstroke}{rgb}{0.000000,0.000000,0.000000}%
\pgfsetstrokecolor{currentstroke}%
\pgfsetdash{}{0pt}%
\pgfpathmoveto{\pgfqpoint{4.089767in}{2.752219in}}%
\pgfpathlineto{\pgfqpoint{3.792831in}{3.049155in}}%
\pgfpathlineto{\pgfqpoint{3.789695in}{3.039746in}}%
\pgfpathlineto{\pgfqpoint{3.767742in}{3.080516in}}%
\pgfpathlineto{\pgfqpoint{3.808512in}{3.058563in}}%
\pgfpathlineto{\pgfqpoint{3.799104in}{3.055427in}}%
\pgfpathlineto{\pgfqpoint{4.096039in}{2.758491in}}%
\pgfpathlineto{\pgfqpoint{4.089767in}{2.752219in}}%
\pgfusepath{fill}%
\end{pgfscope}%
\begin{pgfscope}%
\pgfpathrectangle{\pgfqpoint{1.432000in}{0.528000in}}{\pgfqpoint{3.696000in}{3.696000in}} %
\pgfusepath{clip}%
\pgfsetbuttcap%
\pgfsetroundjoin%
\definecolor{currentfill}{rgb}{0.283072,0.130895,0.449241}%
\pgfsetfillcolor{currentfill}%
\pgfsetlinewidth{0.000000pt}%
\definecolor{currentstroke}{rgb}{0.000000,0.000000,0.000000}%
\pgfsetstrokecolor{currentstroke}%
\pgfsetdash{}{0pt}%
\pgfpathmoveto{\pgfqpoint{4.198830in}{2.751665in}}%
\pgfpathlineto{\pgfqpoint{3.906882in}{2.946297in}}%
\pgfpathlineto{\pgfqpoint{3.905652in}{2.936456in}}%
\pgfpathlineto{\pgfqpoint{3.876129in}{2.972129in}}%
\pgfpathlineto{\pgfqpoint{3.920413in}{2.958598in}}%
\pgfpathlineto{\pgfqpoint{3.911802in}{2.953677in}}%
\pgfpathlineto{\pgfqpoint{4.203751in}{2.759045in}}%
\pgfpathlineto{\pgfqpoint{4.198830in}{2.751665in}}%
\pgfusepath{fill}%
\end{pgfscope}%
\begin{pgfscope}%
\pgfpathrectangle{\pgfqpoint{1.432000in}{0.528000in}}{\pgfqpoint{3.696000in}{3.696000in}} %
\pgfusepath{clip}%
\pgfsetbuttcap%
\pgfsetroundjoin%
\definecolor{currentfill}{rgb}{0.273006,0.204520,0.501721}%
\pgfsetfillcolor{currentfill}%
\pgfsetlinewidth{0.000000pt}%
\definecolor{currentstroke}{rgb}{0.000000,0.000000,0.000000}%
\pgfsetstrokecolor{currentstroke}%
\pgfsetdash{}{0pt}%
\pgfpathmoveto{\pgfqpoint{4.198629in}{2.751807in}}%
\pgfpathlineto{\pgfqpoint{3.797014in}{3.053018in}}%
\pgfpathlineto{\pgfqpoint{3.795240in}{3.043260in}}%
\pgfpathlineto{\pgfqpoint{3.767742in}{3.080516in}}%
\pgfpathlineto{\pgfqpoint{3.811207in}{3.064549in}}%
\pgfpathlineto{\pgfqpoint{3.802336in}{3.060114in}}%
\pgfpathlineto{\pgfqpoint{4.203951in}{2.758903in}}%
\pgfpathlineto{\pgfqpoint{4.198629in}{2.751807in}}%
\pgfusepath{fill}%
\end{pgfscope}%
\begin{pgfscope}%
\pgfpathrectangle{\pgfqpoint{1.432000in}{0.528000in}}{\pgfqpoint{3.696000in}{3.696000in}} %
\pgfusepath{clip}%
\pgfsetbuttcap%
\pgfsetroundjoin%
\definecolor{currentfill}{rgb}{0.126453,0.570633,0.549841}%
\pgfsetfillcolor{currentfill}%
\pgfsetlinewidth{0.000000pt}%
\definecolor{currentstroke}{rgb}{0.000000,0.000000,0.000000}%
\pgfsetstrokecolor{currentstroke}%
\pgfsetdash{}{0pt}%
\pgfpathmoveto{\pgfqpoint{4.198154in}{2.752219in}}%
\pgfpathlineto{\pgfqpoint{3.901218in}{3.049155in}}%
\pgfpathlineto{\pgfqpoint{3.898082in}{3.039746in}}%
\pgfpathlineto{\pgfqpoint{3.876129in}{3.080516in}}%
\pgfpathlineto{\pgfqpoint{3.916899in}{3.058563in}}%
\pgfpathlineto{\pgfqpoint{3.907491in}{3.055427in}}%
\pgfpathlineto{\pgfqpoint{4.204426in}{2.758491in}}%
\pgfpathlineto{\pgfqpoint{4.198154in}{2.752219in}}%
\pgfusepath{fill}%
\end{pgfscope}%
\begin{pgfscope}%
\pgfpathrectangle{\pgfqpoint{1.432000in}{0.528000in}}{\pgfqpoint{3.696000in}{3.696000in}} %
\pgfusepath{clip}%
\pgfsetbuttcap%
\pgfsetroundjoin%
\definecolor{currentfill}{rgb}{0.199430,0.387607,0.554642}%
\pgfsetfillcolor{currentfill}%
\pgfsetlinewidth{0.000000pt}%
\definecolor{currentstroke}{rgb}{0.000000,0.000000,0.000000}%
\pgfsetstrokecolor{currentstroke}%
\pgfsetdash{}{0pt}%
\pgfpathmoveto{\pgfqpoint{4.307217in}{2.751665in}}%
\pgfpathlineto{\pgfqpoint{4.015269in}{2.946297in}}%
\pgfpathlineto{\pgfqpoint{4.014039in}{2.936456in}}%
\pgfpathlineto{\pgfqpoint{3.984516in}{2.972129in}}%
\pgfpathlineto{\pgfqpoint{4.028800in}{2.958598in}}%
\pgfpathlineto{\pgfqpoint{4.020189in}{2.953677in}}%
\pgfpathlineto{\pgfqpoint{4.312138in}{2.759045in}}%
\pgfpathlineto{\pgfqpoint{4.307217in}{2.751665in}}%
\pgfusepath{fill}%
\end{pgfscope}%
\begin{pgfscope}%
\pgfpathrectangle{\pgfqpoint{1.432000in}{0.528000in}}{\pgfqpoint{3.696000in}{3.696000in}} %
\pgfusepath{clip}%
\pgfsetbuttcap%
\pgfsetroundjoin%
\definecolor{currentfill}{rgb}{0.122606,0.585371,0.546557}%
\pgfsetfillcolor{currentfill}%
\pgfsetlinewidth{0.000000pt}%
\definecolor{currentstroke}{rgb}{0.000000,0.000000,0.000000}%
\pgfsetstrokecolor{currentstroke}%
\pgfsetdash{}{0pt}%
\pgfpathmoveto{\pgfqpoint{4.306541in}{2.752219in}}%
\pgfpathlineto{\pgfqpoint{4.009605in}{3.049155in}}%
\pgfpathlineto{\pgfqpoint{4.006469in}{3.039746in}}%
\pgfpathlineto{\pgfqpoint{3.984516in}{3.080516in}}%
\pgfpathlineto{\pgfqpoint{4.025286in}{3.058563in}}%
\pgfpathlineto{\pgfqpoint{4.015878in}{3.055427in}}%
\pgfpathlineto{\pgfqpoint{4.312814in}{2.758491in}}%
\pgfpathlineto{\pgfqpoint{4.306541in}{2.752219in}}%
\pgfusepath{fill}%
\end{pgfscope}%
\begin{pgfscope}%
\pgfpathrectangle{\pgfqpoint{1.432000in}{0.528000in}}{\pgfqpoint{3.696000in}{3.696000in}} %
\pgfusepath{clip}%
\pgfsetbuttcap%
\pgfsetroundjoin%
\definecolor{currentfill}{rgb}{0.277134,0.185228,0.489898}%
\pgfsetfillcolor{currentfill}%
\pgfsetlinewidth{0.000000pt}%
\definecolor{currentstroke}{rgb}{0.000000,0.000000,0.000000}%
\pgfsetstrokecolor{currentstroke}%
\pgfsetdash{}{0pt}%
\pgfpathmoveto{\pgfqpoint{4.415604in}{2.751665in}}%
\pgfpathlineto{\pgfqpoint{4.123656in}{2.946297in}}%
\pgfpathlineto{\pgfqpoint{4.122426in}{2.936456in}}%
\pgfpathlineto{\pgfqpoint{4.092903in}{2.972129in}}%
\pgfpathlineto{\pgfqpoint{4.137187in}{2.958598in}}%
\pgfpathlineto{\pgfqpoint{4.128576in}{2.953677in}}%
\pgfpathlineto{\pgfqpoint{4.420525in}{2.759045in}}%
\pgfpathlineto{\pgfqpoint{4.415604in}{2.751665in}}%
\pgfusepath{fill}%
\end{pgfscope}%
\begin{pgfscope}%
\pgfpathrectangle{\pgfqpoint{1.432000in}{0.528000in}}{\pgfqpoint{3.696000in}{3.696000in}} %
\pgfusepath{clip}%
\pgfsetbuttcap%
\pgfsetroundjoin%
\definecolor{currentfill}{rgb}{0.271305,0.019942,0.347269}%
\pgfsetfillcolor{currentfill}%
\pgfsetlinewidth{0.000000pt}%
\definecolor{currentstroke}{rgb}{0.000000,0.000000,0.000000}%
\pgfsetstrokecolor{currentstroke}%
\pgfsetdash{}{0pt}%
\pgfpathmoveto{\pgfqpoint{4.414928in}{2.752219in}}%
\pgfpathlineto{\pgfqpoint{4.226380in}{2.940767in}}%
\pgfpathlineto{\pgfqpoint{4.223243in}{2.931359in}}%
\pgfpathlineto{\pgfqpoint{4.201290in}{2.972129in}}%
\pgfpathlineto{\pgfqpoint{4.242060in}{2.950176in}}%
\pgfpathlineto{\pgfqpoint{4.232652in}{2.947040in}}%
\pgfpathlineto{\pgfqpoint{4.421201in}{2.758491in}}%
\pgfpathlineto{\pgfqpoint{4.414928in}{2.752219in}}%
\pgfusepath{fill}%
\end{pgfscope}%
\begin{pgfscope}%
\pgfpathrectangle{\pgfqpoint{1.432000in}{0.528000in}}{\pgfqpoint{3.696000in}{3.696000in}} %
\pgfusepath{clip}%
\pgfsetbuttcap%
\pgfsetroundjoin%
\definecolor{currentfill}{rgb}{0.143343,0.522773,0.556295}%
\pgfsetfillcolor{currentfill}%
\pgfsetlinewidth{0.000000pt}%
\definecolor{currentstroke}{rgb}{0.000000,0.000000,0.000000}%
\pgfsetstrokecolor{currentstroke}%
\pgfsetdash{}{0pt}%
\pgfpathmoveto{\pgfqpoint{4.414928in}{2.752219in}}%
\pgfpathlineto{\pgfqpoint{4.117993in}{3.049155in}}%
\pgfpathlineto{\pgfqpoint{4.114856in}{3.039746in}}%
\pgfpathlineto{\pgfqpoint{4.092903in}{3.080516in}}%
\pgfpathlineto{\pgfqpoint{4.133673in}{3.058563in}}%
\pgfpathlineto{\pgfqpoint{4.124265in}{3.055427in}}%
\pgfpathlineto{\pgfqpoint{4.421201in}{2.758491in}}%
\pgfpathlineto{\pgfqpoint{4.414928in}{2.752219in}}%
\pgfusepath{fill}%
\end{pgfscope}%
\begin{pgfscope}%
\pgfpathrectangle{\pgfqpoint{1.432000in}{0.528000in}}{\pgfqpoint{3.696000in}{3.696000in}} %
\pgfusepath{clip}%
\pgfsetbuttcap%
\pgfsetroundjoin%
\definecolor{currentfill}{rgb}{0.267968,0.223549,0.512008}%
\pgfsetfillcolor{currentfill}%
\pgfsetlinewidth{0.000000pt}%
\definecolor{currentstroke}{rgb}{0.000000,0.000000,0.000000}%
\pgfsetstrokecolor{currentstroke}%
\pgfsetdash{}{0pt}%
\pgfpathmoveto{\pgfqpoint{4.414374in}{2.752895in}}%
\pgfpathlineto{\pgfqpoint{4.219742in}{3.044843in}}%
\pgfpathlineto{\pgfqpoint{4.214821in}{3.036232in}}%
\pgfpathlineto{\pgfqpoint{4.201290in}{3.080516in}}%
\pgfpathlineto{\pgfqpoint{4.236963in}{3.050994in}}%
\pgfpathlineto{\pgfqpoint{4.227122in}{3.049764in}}%
\pgfpathlineto{\pgfqpoint{4.421755in}{2.757815in}}%
\pgfpathlineto{\pgfqpoint{4.414374in}{2.752895in}}%
\pgfusepath{fill}%
\end{pgfscope}%
\begin{pgfscope}%
\pgfpathrectangle{\pgfqpoint{1.432000in}{0.528000in}}{\pgfqpoint{3.696000in}{3.696000in}} %
\pgfusepath{clip}%
\pgfsetbuttcap%
\pgfsetroundjoin%
\definecolor{currentfill}{rgb}{0.281446,0.084320,0.407414}%
\pgfsetfillcolor{currentfill}%
\pgfsetlinewidth{0.000000pt}%
\definecolor{currentstroke}{rgb}{0.000000,0.000000,0.000000}%
\pgfsetstrokecolor{currentstroke}%
\pgfsetdash{}{0pt}%
\pgfpathmoveto{\pgfqpoint{4.523991in}{2.751665in}}%
\pgfpathlineto{\pgfqpoint{4.232043in}{2.946297in}}%
\pgfpathlineto{\pgfqpoint{4.230813in}{2.936456in}}%
\pgfpathlineto{\pgfqpoint{4.201290in}{2.972129in}}%
\pgfpathlineto{\pgfqpoint{4.245574in}{2.958598in}}%
\pgfpathlineto{\pgfqpoint{4.236963in}{2.953677in}}%
\pgfpathlineto{\pgfqpoint{4.528912in}{2.759045in}}%
\pgfpathlineto{\pgfqpoint{4.523991in}{2.751665in}}%
\pgfusepath{fill}%
\end{pgfscope}%
\begin{pgfscope}%
\pgfpathrectangle{\pgfqpoint{1.432000in}{0.528000in}}{\pgfqpoint{3.696000in}{3.696000in}} %
\pgfusepath{clip}%
\pgfsetbuttcap%
\pgfsetroundjoin%
\definecolor{currentfill}{rgb}{0.278826,0.175490,0.483397}%
\pgfsetfillcolor{currentfill}%
\pgfsetlinewidth{0.000000pt}%
\definecolor{currentstroke}{rgb}{0.000000,0.000000,0.000000}%
\pgfsetstrokecolor{currentstroke}%
\pgfsetdash{}{0pt}%
\pgfpathmoveto{\pgfqpoint{4.523315in}{2.752219in}}%
\pgfpathlineto{\pgfqpoint{4.334767in}{2.940767in}}%
\pgfpathlineto{\pgfqpoint{4.331631in}{2.931359in}}%
\pgfpathlineto{\pgfqpoint{4.309677in}{2.972129in}}%
\pgfpathlineto{\pgfqpoint{4.350447in}{2.950176in}}%
\pgfpathlineto{\pgfqpoint{4.341039in}{2.947040in}}%
\pgfpathlineto{\pgfqpoint{4.529588in}{2.758491in}}%
\pgfpathlineto{\pgfqpoint{4.523315in}{2.752219in}}%
\pgfusepath{fill}%
\end{pgfscope}%
\begin{pgfscope}%
\pgfpathrectangle{\pgfqpoint{1.432000in}{0.528000in}}{\pgfqpoint{3.696000in}{3.696000in}} %
\pgfusepath{clip}%
\pgfsetbuttcap%
\pgfsetroundjoin%
\definecolor{currentfill}{rgb}{0.208623,0.367752,0.552675}%
\pgfsetfillcolor{currentfill}%
\pgfsetlinewidth{0.000000pt}%
\definecolor{currentstroke}{rgb}{0.000000,0.000000,0.000000}%
\pgfsetstrokecolor{currentstroke}%
\pgfsetdash{}{0pt}%
\pgfpathmoveto{\pgfqpoint{4.523315in}{2.752219in}}%
\pgfpathlineto{\pgfqpoint{4.226380in}{3.049155in}}%
\pgfpathlineto{\pgfqpoint{4.223243in}{3.039746in}}%
\pgfpathlineto{\pgfqpoint{4.201290in}{3.080516in}}%
\pgfpathlineto{\pgfqpoint{4.242060in}{3.058563in}}%
\pgfpathlineto{\pgfqpoint{4.232652in}{3.055427in}}%
\pgfpathlineto{\pgfqpoint{4.529588in}{2.758491in}}%
\pgfpathlineto{\pgfqpoint{4.523315in}{2.752219in}}%
\pgfusepath{fill}%
\end{pgfscope}%
\begin{pgfscope}%
\pgfpathrectangle{\pgfqpoint{1.432000in}{0.528000in}}{\pgfqpoint{3.696000in}{3.696000in}} %
\pgfusepath{clip}%
\pgfsetbuttcap%
\pgfsetroundjoin%
\definecolor{currentfill}{rgb}{0.127568,0.566949,0.550556}%
\pgfsetfillcolor{currentfill}%
\pgfsetlinewidth{0.000000pt}%
\definecolor{currentstroke}{rgb}{0.000000,0.000000,0.000000}%
\pgfsetstrokecolor{currentstroke}%
\pgfsetdash{}{0pt}%
\pgfpathmoveto{\pgfqpoint{4.522761in}{2.752895in}}%
\pgfpathlineto{\pgfqpoint{4.328129in}{3.044843in}}%
\pgfpathlineto{\pgfqpoint{4.323209in}{3.036232in}}%
\pgfpathlineto{\pgfqpoint{4.309677in}{3.080516in}}%
\pgfpathlineto{\pgfqpoint{4.345350in}{3.050994in}}%
\pgfpathlineto{\pgfqpoint{4.335510in}{3.049764in}}%
\pgfpathlineto{\pgfqpoint{4.530142in}{2.757815in}}%
\pgfpathlineto{\pgfqpoint{4.522761in}{2.752895in}}%
\pgfusepath{fill}%
\end{pgfscope}%
\begin{pgfscope}%
\pgfpathrectangle{\pgfqpoint{1.432000in}{0.528000in}}{\pgfqpoint{3.696000in}{3.696000in}} %
\pgfusepath{clip}%
\pgfsetbuttcap%
\pgfsetroundjoin%
\definecolor{currentfill}{rgb}{0.282656,0.100196,0.422160}%
\pgfsetfillcolor{currentfill}%
\pgfsetlinewidth{0.000000pt}%
\definecolor{currentstroke}{rgb}{0.000000,0.000000,0.000000}%
\pgfsetstrokecolor{currentstroke}%
\pgfsetdash{}{0pt}%
\pgfpathmoveto{\pgfqpoint{4.631703in}{2.752219in}}%
\pgfpathlineto{\pgfqpoint{4.443154in}{2.940767in}}%
\pgfpathlineto{\pgfqpoint{4.440018in}{2.931359in}}%
\pgfpathlineto{\pgfqpoint{4.418065in}{2.972129in}}%
\pgfpathlineto{\pgfqpoint{4.458835in}{2.950176in}}%
\pgfpathlineto{\pgfqpoint{4.449426in}{2.947040in}}%
\pgfpathlineto{\pgfqpoint{4.637975in}{2.758491in}}%
\pgfpathlineto{\pgfqpoint{4.631703in}{2.752219in}}%
\pgfusepath{fill}%
\end{pgfscope}%
\begin{pgfscope}%
\pgfpathrectangle{\pgfqpoint{1.432000in}{0.528000in}}{\pgfqpoint{3.696000in}{3.696000in}} %
\pgfusepath{clip}%
\pgfsetbuttcap%
\pgfsetroundjoin%
\definecolor{currentfill}{rgb}{0.281477,0.755203,0.432552}%
\pgfsetfillcolor{currentfill}%
\pgfsetlinewidth{0.000000pt}%
\definecolor{currentstroke}{rgb}{0.000000,0.000000,0.000000}%
\pgfsetstrokecolor{currentstroke}%
\pgfsetdash{}{0pt}%
\pgfpathmoveto{\pgfqpoint{4.631148in}{2.752895in}}%
\pgfpathlineto{\pgfqpoint{4.436516in}{3.044843in}}%
\pgfpathlineto{\pgfqpoint{4.431596in}{3.036232in}}%
\pgfpathlineto{\pgfqpoint{4.418065in}{3.080516in}}%
\pgfpathlineto{\pgfqpoint{4.453738in}{3.050994in}}%
\pgfpathlineto{\pgfqpoint{4.443897in}{3.049764in}}%
\pgfpathlineto{\pgfqpoint{4.638529in}{2.757815in}}%
\pgfpathlineto{\pgfqpoint{4.631148in}{2.752895in}}%
\pgfusepath{fill}%
\end{pgfscope}%
\begin{pgfscope}%
\pgfpathrectangle{\pgfqpoint{1.432000in}{0.528000in}}{\pgfqpoint{3.696000in}{3.696000in}} %
\pgfusepath{clip}%
\pgfsetbuttcap%
\pgfsetroundjoin%
\definecolor{currentfill}{rgb}{0.272594,0.025563,0.353093}%
\pgfsetfillcolor{currentfill}%
\pgfsetlinewidth{0.000000pt}%
\definecolor{currentstroke}{rgb}{0.000000,0.000000,0.000000}%
\pgfsetstrokecolor{currentstroke}%
\pgfsetdash{}{0pt}%
\pgfpathmoveto{\pgfqpoint{4.630631in}{2.753952in}}%
\pgfpathlineto{\pgfqpoint{4.534867in}{3.041245in}}%
\pgfpathlineto{\pgfqpoint{4.527854in}{3.034233in}}%
\pgfpathlineto{\pgfqpoint{4.526452in}{3.080516in}}%
\pgfpathlineto{\pgfqpoint{4.553100in}{3.042648in}}%
\pgfpathlineto{\pgfqpoint{4.543282in}{3.044050in}}%
\pgfpathlineto{\pgfqpoint{4.639046in}{2.756757in}}%
\pgfpathlineto{\pgfqpoint{4.630631in}{2.753952in}}%
\pgfusepath{fill}%
\end{pgfscope}%
\begin{pgfscope}%
\pgfpathrectangle{\pgfqpoint{1.432000in}{0.528000in}}{\pgfqpoint{3.696000in}{3.696000in}} %
\pgfusepath{clip}%
\pgfsetbuttcap%
\pgfsetroundjoin%
\definecolor{currentfill}{rgb}{0.281924,0.089666,0.412415}%
\pgfsetfillcolor{currentfill}%
\pgfsetlinewidth{0.000000pt}%
\definecolor{currentstroke}{rgb}{0.000000,0.000000,0.000000}%
\pgfsetstrokecolor{currentstroke}%
\pgfsetdash{}{0pt}%
\pgfpathmoveto{\pgfqpoint{4.740090in}{2.752219in}}%
\pgfpathlineto{\pgfqpoint{4.551541in}{2.940767in}}%
\pgfpathlineto{\pgfqpoint{4.548405in}{2.931359in}}%
\pgfpathlineto{\pgfqpoint{4.526452in}{2.972129in}}%
\pgfpathlineto{\pgfqpoint{4.567222in}{2.950176in}}%
\pgfpathlineto{\pgfqpoint{4.557813in}{2.947040in}}%
\pgfpathlineto{\pgfqpoint{4.746362in}{2.758491in}}%
\pgfpathlineto{\pgfqpoint{4.740090in}{2.752219in}}%
\pgfusepath{fill}%
\end{pgfscope}%
\begin{pgfscope}%
\pgfpathrectangle{\pgfqpoint{1.432000in}{0.528000in}}{\pgfqpoint{3.696000in}{3.696000in}} %
\pgfusepath{clip}%
\pgfsetbuttcap%
\pgfsetroundjoin%
\definecolor{currentfill}{rgb}{0.283197,0.115680,0.436115}%
\pgfsetfillcolor{currentfill}%
\pgfsetlinewidth{0.000000pt}%
\definecolor{currentstroke}{rgb}{0.000000,0.000000,0.000000}%
\pgfsetstrokecolor{currentstroke}%
\pgfsetdash{}{0pt}%
\pgfpathmoveto{\pgfqpoint{4.739259in}{2.753371in}}%
\pgfpathlineto{\pgfqpoint{4.648723in}{2.934443in}}%
\pgfpathlineto{\pgfqpoint{4.642773in}{2.926509in}}%
\pgfpathlineto{\pgfqpoint{4.634839in}{2.972129in}}%
\pgfpathlineto{\pgfqpoint{4.666574in}{2.938410in}}%
\pgfpathlineto{\pgfqpoint{4.656657in}{2.938410in}}%
\pgfpathlineto{\pgfqpoint{4.747193in}{2.757338in}}%
\pgfpathlineto{\pgfqpoint{4.739259in}{2.753371in}}%
\pgfusepath{fill}%
\end{pgfscope}%
\begin{pgfscope}%
\pgfpathrectangle{\pgfqpoint{1.432000in}{0.528000in}}{\pgfqpoint{3.696000in}{3.696000in}} %
\pgfusepath{clip}%
\pgfsetbuttcap%
\pgfsetroundjoin%
\definecolor{currentfill}{rgb}{0.136408,0.541173,0.554483}%
\pgfsetfillcolor{currentfill}%
\pgfsetlinewidth{0.000000pt}%
\definecolor{currentstroke}{rgb}{0.000000,0.000000,0.000000}%
\pgfsetstrokecolor{currentstroke}%
\pgfsetdash{}{0pt}%
\pgfpathmoveto{\pgfqpoint{4.739535in}{2.752895in}}%
\pgfpathlineto{\pgfqpoint{4.544903in}{3.044843in}}%
\pgfpathlineto{\pgfqpoint{4.539983in}{3.036232in}}%
\pgfpathlineto{\pgfqpoint{4.526452in}{3.080516in}}%
\pgfpathlineto{\pgfqpoint{4.562125in}{3.050994in}}%
\pgfpathlineto{\pgfqpoint{4.552284in}{3.049764in}}%
\pgfpathlineto{\pgfqpoint{4.746916in}{2.757815in}}%
\pgfpathlineto{\pgfqpoint{4.739535in}{2.752895in}}%
\pgfusepath{fill}%
\end{pgfscope}%
\begin{pgfscope}%
\pgfpathrectangle{\pgfqpoint{1.432000in}{0.528000in}}{\pgfqpoint{3.696000in}{3.696000in}} %
\pgfusepath{clip}%
\pgfsetbuttcap%
\pgfsetroundjoin%
\definecolor{currentfill}{rgb}{0.153364,0.497000,0.557724}%
\pgfsetfillcolor{currentfill}%
\pgfsetlinewidth{0.000000pt}%
\definecolor{currentstroke}{rgb}{0.000000,0.000000,0.000000}%
\pgfsetstrokecolor{currentstroke}%
\pgfsetdash{}{0pt}%
\pgfpathmoveto{\pgfqpoint{4.739018in}{2.753952in}}%
\pgfpathlineto{\pgfqpoint{4.643254in}{3.041245in}}%
\pgfpathlineto{\pgfqpoint{4.636241in}{3.034233in}}%
\pgfpathlineto{\pgfqpoint{4.634839in}{3.080516in}}%
\pgfpathlineto{\pgfqpoint{4.661487in}{3.042648in}}%
\pgfpathlineto{\pgfqpoint{4.651669in}{3.044050in}}%
\pgfpathlineto{\pgfqpoint{4.747433in}{2.756757in}}%
\pgfpathlineto{\pgfqpoint{4.739018in}{2.753952in}}%
\pgfusepath{fill}%
\end{pgfscope}%
\begin{pgfscope}%
\pgfpathrectangle{\pgfqpoint{1.432000in}{0.528000in}}{\pgfqpoint{3.696000in}{3.696000in}} %
\pgfusepath{clip}%
\pgfsetbuttcap%
\pgfsetroundjoin%
\definecolor{currentfill}{rgb}{0.279574,0.170599,0.479997}%
\pgfsetfillcolor{currentfill}%
\pgfsetlinewidth{0.000000pt}%
\definecolor{currentstroke}{rgb}{0.000000,0.000000,0.000000}%
\pgfsetstrokecolor{currentstroke}%
\pgfsetdash{}{0pt}%
\pgfpathmoveto{\pgfqpoint{4.847646in}{2.753371in}}%
\pgfpathlineto{\pgfqpoint{4.757110in}{2.934443in}}%
\pgfpathlineto{\pgfqpoint{4.751160in}{2.926509in}}%
\pgfpathlineto{\pgfqpoint{4.743226in}{2.972129in}}%
\pgfpathlineto{\pgfqpoint{4.774962in}{2.938410in}}%
\pgfpathlineto{\pgfqpoint{4.765044in}{2.938410in}}%
\pgfpathlineto{\pgfqpoint{4.855580in}{2.757338in}}%
\pgfpathlineto{\pgfqpoint{4.847646in}{2.753371in}}%
\pgfusepath{fill}%
\end{pgfscope}%
\begin{pgfscope}%
\pgfpathrectangle{\pgfqpoint{1.432000in}{0.528000in}}{\pgfqpoint{3.696000in}{3.696000in}} %
\pgfusepath{clip}%
\pgfsetbuttcap%
\pgfsetroundjoin%
\definecolor{currentfill}{rgb}{0.268510,0.009605,0.335427}%
\pgfsetfillcolor{currentfill}%
\pgfsetlinewidth{0.000000pt}%
\definecolor{currentstroke}{rgb}{0.000000,0.000000,0.000000}%
\pgfsetstrokecolor{currentstroke}%
\pgfsetdash{}{0pt}%
\pgfpathmoveto{\pgfqpoint{4.847923in}{2.752895in}}%
\pgfpathlineto{\pgfqpoint{4.653290in}{3.044843in}}%
\pgfpathlineto{\pgfqpoint{4.648370in}{3.036232in}}%
\pgfpathlineto{\pgfqpoint{4.634839in}{3.080516in}}%
\pgfpathlineto{\pgfqpoint{4.670512in}{3.050994in}}%
\pgfpathlineto{\pgfqpoint{4.660671in}{3.049764in}}%
\pgfpathlineto{\pgfqpoint{4.855303in}{2.757815in}}%
\pgfpathlineto{\pgfqpoint{4.847923in}{2.752895in}}%
\pgfusepath{fill}%
\end{pgfscope}%
\begin{pgfscope}%
\pgfpathrectangle{\pgfqpoint{1.432000in}{0.528000in}}{\pgfqpoint{3.696000in}{3.696000in}} %
\pgfusepath{clip}%
\pgfsetbuttcap%
\pgfsetroundjoin%
\definecolor{currentfill}{rgb}{0.134692,0.658636,0.517649}%
\pgfsetfillcolor{currentfill}%
\pgfsetlinewidth{0.000000pt}%
\definecolor{currentstroke}{rgb}{0.000000,0.000000,0.000000}%
\pgfsetstrokecolor{currentstroke}%
\pgfsetdash{}{0pt}%
\pgfpathmoveto{\pgfqpoint{4.847405in}{2.753952in}}%
\pgfpathlineto{\pgfqpoint{4.751641in}{3.041245in}}%
\pgfpathlineto{\pgfqpoint{4.744628in}{3.034233in}}%
\pgfpathlineto{\pgfqpoint{4.743226in}{3.080516in}}%
\pgfpathlineto{\pgfqpoint{4.769874in}{3.042648in}}%
\pgfpathlineto{\pgfqpoint{4.760056in}{3.044050in}}%
\pgfpathlineto{\pgfqpoint{4.855821in}{2.756757in}}%
\pgfpathlineto{\pgfqpoint{4.847405in}{2.753952in}}%
\pgfusepath{fill}%
\end{pgfscope}%
\begin{pgfscope}%
\pgfpathrectangle{\pgfqpoint{1.432000in}{0.528000in}}{\pgfqpoint{3.696000in}{3.696000in}} %
\pgfusepath{clip}%
\pgfsetbuttcap%
\pgfsetroundjoin%
\definecolor{currentfill}{rgb}{0.267004,0.004874,0.329415}%
\pgfsetfillcolor{currentfill}%
\pgfsetlinewidth{0.000000pt}%
\definecolor{currentstroke}{rgb}{0.000000,0.000000,0.000000}%
\pgfsetstrokecolor{currentstroke}%
\pgfsetdash{}{0pt}%
\pgfpathmoveto{\pgfqpoint{4.956033in}{2.753371in}}%
\pgfpathlineto{\pgfqpoint{4.865497in}{2.934443in}}%
\pgfpathlineto{\pgfqpoint{4.859547in}{2.926509in}}%
\pgfpathlineto{\pgfqpoint{4.851613in}{2.972129in}}%
\pgfpathlineto{\pgfqpoint{4.883349in}{2.938410in}}%
\pgfpathlineto{\pgfqpoint{4.873431in}{2.938410in}}%
\pgfpathlineto{\pgfqpoint{4.963967in}{2.757338in}}%
\pgfpathlineto{\pgfqpoint{4.956033in}{2.753371in}}%
\pgfusepath{fill}%
\end{pgfscope}%
\begin{pgfscope}%
\pgfpathrectangle{\pgfqpoint{1.432000in}{0.528000in}}{\pgfqpoint{3.696000in}{3.696000in}} %
\pgfusepath{clip}%
\pgfsetbuttcap%
\pgfsetroundjoin%
\definecolor{currentfill}{rgb}{0.281446,0.084320,0.407414}%
\pgfsetfillcolor{currentfill}%
\pgfsetlinewidth{0.000000pt}%
\definecolor{currentstroke}{rgb}{0.000000,0.000000,0.000000}%
\pgfsetstrokecolor{currentstroke}%
\pgfsetdash{}{0pt}%
\pgfpathmoveto{\pgfqpoint{4.955565in}{2.755355in}}%
\pgfpathlineto{\pgfqpoint{4.955565in}{2.932212in}}%
\pgfpathlineto{\pgfqpoint{4.946694in}{2.927777in}}%
\pgfpathlineto{\pgfqpoint{4.960000in}{2.972129in}}%
\pgfpathlineto{\pgfqpoint{4.973306in}{2.927777in}}%
\pgfpathlineto{\pgfqpoint{4.964435in}{2.932212in}}%
\pgfpathlineto{\pgfqpoint{4.964435in}{2.755355in}}%
\pgfpathlineto{\pgfqpoint{4.955565in}{2.755355in}}%
\pgfusepath{fill}%
\end{pgfscope}%
\begin{pgfscope}%
\pgfpathrectangle{\pgfqpoint{1.432000in}{0.528000in}}{\pgfqpoint{3.696000in}{3.696000in}} %
\pgfusepath{clip}%
\pgfsetbuttcap%
\pgfsetroundjoin%
\definecolor{currentfill}{rgb}{0.197636,0.391528,0.554969}%
\pgfsetfillcolor{currentfill}%
\pgfsetlinewidth{0.000000pt}%
\definecolor{currentstroke}{rgb}{0.000000,0.000000,0.000000}%
\pgfsetstrokecolor{currentstroke}%
\pgfsetdash{}{0pt}%
\pgfpathmoveto{\pgfqpoint{4.955792in}{2.753952in}}%
\pgfpathlineto{\pgfqpoint{4.860028in}{3.041245in}}%
\pgfpathlineto{\pgfqpoint{4.853015in}{3.034233in}}%
\pgfpathlineto{\pgfqpoint{4.851613in}{3.080516in}}%
\pgfpathlineto{\pgfqpoint{4.878261in}{3.042648in}}%
\pgfpathlineto{\pgfqpoint{4.868443in}{3.044050in}}%
\pgfpathlineto{\pgfqpoint{4.964208in}{2.756757in}}%
\pgfpathlineto{\pgfqpoint{4.955792in}{2.753952in}}%
\pgfusepath{fill}%
\end{pgfscope}%
\begin{pgfscope}%
\pgfpathrectangle{\pgfqpoint{1.432000in}{0.528000in}}{\pgfqpoint{3.696000in}{3.696000in}} %
\pgfusepath{clip}%
\pgfsetbuttcap%
\pgfsetroundjoin%
\definecolor{currentfill}{rgb}{0.206756,0.371758,0.553117}%
\pgfsetfillcolor{currentfill}%
\pgfsetlinewidth{0.000000pt}%
\definecolor{currentstroke}{rgb}{0.000000,0.000000,0.000000}%
\pgfsetstrokecolor{currentstroke}%
\pgfsetdash{}{0pt}%
\pgfpathmoveto{\pgfqpoint{4.955565in}{2.755355in}}%
\pgfpathlineto{\pgfqpoint{4.955565in}{3.040599in}}%
\pgfpathlineto{\pgfqpoint{4.946694in}{3.036164in}}%
\pgfpathlineto{\pgfqpoint{4.960000in}{3.080516in}}%
\pgfpathlineto{\pgfqpoint{4.973306in}{3.036164in}}%
\pgfpathlineto{\pgfqpoint{4.964435in}{3.040599in}}%
\pgfpathlineto{\pgfqpoint{4.964435in}{2.755355in}}%
\pgfpathlineto{\pgfqpoint{4.955565in}{2.755355in}}%
\pgfusepath{fill}%
\end{pgfscope}%
\begin{pgfscope}%
\pgfpathrectangle{\pgfqpoint{1.432000in}{0.528000in}}{\pgfqpoint{3.696000in}{3.696000in}} %
\pgfusepath{clip}%
\pgfsetbuttcap%
\pgfsetroundjoin%
\definecolor{currentfill}{rgb}{0.282623,0.140926,0.457517}%
\pgfsetfillcolor{currentfill}%
\pgfsetlinewidth{0.000000pt}%
\definecolor{currentstroke}{rgb}{0.000000,0.000000,0.000000}%
\pgfsetstrokecolor{currentstroke}%
\pgfsetdash{}{0pt}%
\pgfpathmoveto{\pgfqpoint{1.604435in}{2.863742in}}%
\pgfpathlineto{\pgfqpoint{1.604435in}{2.795272in}}%
\pgfpathlineto{\pgfqpoint{1.613306in}{2.799707in}}%
\pgfpathlineto{\pgfqpoint{1.600000in}{2.755355in}}%
\pgfpathlineto{\pgfqpoint{1.586694in}{2.799707in}}%
\pgfpathlineto{\pgfqpoint{1.595565in}{2.795272in}}%
\pgfpathlineto{\pgfqpoint{1.595565in}{2.863742in}}%
\pgfpathlineto{\pgfqpoint{1.604435in}{2.863742in}}%
\pgfusepath{fill}%
\end{pgfscope}%
\begin{pgfscope}%
\pgfpathrectangle{\pgfqpoint{1.432000in}{0.528000in}}{\pgfqpoint{3.696000in}{3.696000in}} %
\pgfusepath{clip}%
\pgfsetbuttcap%
\pgfsetroundjoin%
\definecolor{currentfill}{rgb}{0.165117,0.467423,0.558141}%
\pgfsetfillcolor{currentfill}%
\pgfsetlinewidth{0.000000pt}%
\definecolor{currentstroke}{rgb}{0.000000,0.000000,0.000000}%
\pgfsetstrokecolor{currentstroke}%
\pgfsetdash{}{0pt}%
\pgfpathmoveto{\pgfqpoint{1.604435in}{2.863742in}}%
\pgfpathlineto{\pgfqpoint{1.602218in}{2.867583in}}%
\pgfpathlineto{\pgfqpoint{1.597782in}{2.867583in}}%
\pgfpathlineto{\pgfqpoint{1.595565in}{2.863742in}}%
\pgfpathlineto{\pgfqpoint{1.597782in}{2.859901in}}%
\pgfpathlineto{\pgfqpoint{1.602218in}{2.859901in}}%
\pgfpathlineto{\pgfqpoint{1.604435in}{2.863742in}}%
\pgfpathlineto{\pgfqpoint{1.602218in}{2.867583in}}%
\pgfusepath{fill}%
\end{pgfscope}%
\begin{pgfscope}%
\pgfpathrectangle{\pgfqpoint{1.432000in}{0.528000in}}{\pgfqpoint{3.696000in}{3.696000in}} %
\pgfusepath{clip}%
\pgfsetbuttcap%
\pgfsetroundjoin%
\definecolor{currentfill}{rgb}{0.125394,0.574318,0.549086}%
\pgfsetfillcolor{currentfill}%
\pgfsetlinewidth{0.000000pt}%
\definecolor{currentstroke}{rgb}{0.000000,0.000000,0.000000}%
\pgfsetstrokecolor{currentstroke}%
\pgfsetdash{}{0pt}%
\pgfpathmoveto{\pgfqpoint{1.712822in}{2.863742in}}%
\pgfpathlineto{\pgfqpoint{1.710605in}{2.867583in}}%
\pgfpathlineto{\pgfqpoint{1.706169in}{2.867583in}}%
\pgfpathlineto{\pgfqpoint{1.703952in}{2.863742in}}%
\pgfpathlineto{\pgfqpoint{1.706169in}{2.859901in}}%
\pgfpathlineto{\pgfqpoint{1.710605in}{2.859901in}}%
\pgfpathlineto{\pgfqpoint{1.712822in}{2.863742in}}%
\pgfpathlineto{\pgfqpoint{1.710605in}{2.867583in}}%
\pgfusepath{fill}%
\end{pgfscope}%
\begin{pgfscope}%
\pgfpathrectangle{\pgfqpoint{1.432000in}{0.528000in}}{\pgfqpoint{3.696000in}{3.696000in}} %
\pgfusepath{clip}%
\pgfsetbuttcap%
\pgfsetroundjoin%
\definecolor{currentfill}{rgb}{0.121380,0.629492,0.531973}%
\pgfsetfillcolor{currentfill}%
\pgfsetlinewidth{0.000000pt}%
\definecolor{currentstroke}{rgb}{0.000000,0.000000,0.000000}%
\pgfsetstrokecolor{currentstroke}%
\pgfsetdash{}{0pt}%
\pgfpathmoveto{\pgfqpoint{1.821209in}{2.863742in}}%
\pgfpathlineto{\pgfqpoint{1.818992in}{2.867583in}}%
\pgfpathlineto{\pgfqpoint{1.814557in}{2.867583in}}%
\pgfpathlineto{\pgfqpoint{1.812339in}{2.863742in}}%
\pgfpathlineto{\pgfqpoint{1.814557in}{2.859901in}}%
\pgfpathlineto{\pgfqpoint{1.818992in}{2.859901in}}%
\pgfpathlineto{\pgfqpoint{1.821209in}{2.863742in}}%
\pgfpathlineto{\pgfqpoint{1.818992in}{2.867583in}}%
\pgfusepath{fill}%
\end{pgfscope}%
\begin{pgfscope}%
\pgfpathrectangle{\pgfqpoint{1.432000in}{0.528000in}}{\pgfqpoint{3.696000in}{3.696000in}} %
\pgfusepath{clip}%
\pgfsetbuttcap%
\pgfsetroundjoin%
\definecolor{currentfill}{rgb}{0.119512,0.607464,0.540218}%
\pgfsetfillcolor{currentfill}%
\pgfsetlinewidth{0.000000pt}%
\definecolor{currentstroke}{rgb}{0.000000,0.000000,0.000000}%
\pgfsetstrokecolor{currentstroke}%
\pgfsetdash{}{0pt}%
\pgfpathmoveto{\pgfqpoint{1.929596in}{2.863742in}}%
\pgfpathlineto{\pgfqpoint{1.927379in}{2.867583in}}%
\pgfpathlineto{\pgfqpoint{1.922944in}{2.867583in}}%
\pgfpathlineto{\pgfqpoint{1.920726in}{2.863742in}}%
\pgfpathlineto{\pgfqpoint{1.922944in}{2.859901in}}%
\pgfpathlineto{\pgfqpoint{1.927379in}{2.859901in}}%
\pgfpathlineto{\pgfqpoint{1.929596in}{2.863742in}}%
\pgfpathlineto{\pgfqpoint{1.927379in}{2.867583in}}%
\pgfusepath{fill}%
\end{pgfscope}%
\begin{pgfscope}%
\pgfpathrectangle{\pgfqpoint{1.432000in}{0.528000in}}{\pgfqpoint{3.696000in}{3.696000in}} %
\pgfusepath{clip}%
\pgfsetbuttcap%
\pgfsetroundjoin%
\definecolor{currentfill}{rgb}{0.179019,0.433756,0.557430}%
\pgfsetfillcolor{currentfill}%
\pgfsetlinewidth{0.000000pt}%
\definecolor{currentstroke}{rgb}{0.000000,0.000000,0.000000}%
\pgfsetstrokecolor{currentstroke}%
\pgfsetdash{}{0pt}%
\pgfpathmoveto{\pgfqpoint{2.037984in}{2.863742in}}%
\pgfpathlineto{\pgfqpoint{2.035766in}{2.867583in}}%
\pgfpathlineto{\pgfqpoint{2.031331in}{2.867583in}}%
\pgfpathlineto{\pgfqpoint{2.029113in}{2.863742in}}%
\pgfpathlineto{\pgfqpoint{2.031331in}{2.859901in}}%
\pgfpathlineto{\pgfqpoint{2.035766in}{2.859901in}}%
\pgfpathlineto{\pgfqpoint{2.037984in}{2.863742in}}%
\pgfpathlineto{\pgfqpoint{2.035766in}{2.867583in}}%
\pgfusepath{fill}%
\end{pgfscope}%
\begin{pgfscope}%
\pgfpathrectangle{\pgfqpoint{1.432000in}{0.528000in}}{\pgfqpoint{3.696000in}{3.696000in}} %
\pgfusepath{clip}%
\pgfsetbuttcap%
\pgfsetroundjoin%
\definecolor{currentfill}{rgb}{0.267968,0.223549,0.512008}%
\pgfsetfillcolor{currentfill}%
\pgfsetlinewidth{0.000000pt}%
\definecolor{currentstroke}{rgb}{0.000000,0.000000,0.000000}%
\pgfsetstrokecolor{currentstroke}%
\pgfsetdash{}{0pt}%
\pgfpathmoveto{\pgfqpoint{2.141935in}{2.859307in}}%
\pgfpathlineto{\pgfqpoint{2.073465in}{2.859307in}}%
\pgfpathlineto{\pgfqpoint{2.077900in}{2.850436in}}%
\pgfpathlineto{\pgfqpoint{2.033548in}{2.863742in}}%
\pgfpathlineto{\pgfqpoint{2.077900in}{2.877048in}}%
\pgfpathlineto{\pgfqpoint{2.073465in}{2.868177in}}%
\pgfpathlineto{\pgfqpoint{2.141935in}{2.868177in}}%
\pgfpathlineto{\pgfqpoint{2.141935in}{2.859307in}}%
\pgfusepath{fill}%
\end{pgfscope}%
\begin{pgfscope}%
\pgfpathrectangle{\pgfqpoint{1.432000in}{0.528000in}}{\pgfqpoint{3.696000in}{3.696000in}} %
\pgfusepath{clip}%
\pgfsetbuttcap%
\pgfsetroundjoin%
\definecolor{currentfill}{rgb}{0.280267,0.073417,0.397163}%
\pgfsetfillcolor{currentfill}%
\pgfsetlinewidth{0.000000pt}%
\definecolor{currentstroke}{rgb}{0.000000,0.000000,0.000000}%
\pgfsetstrokecolor{currentstroke}%
\pgfsetdash{}{0pt}%
\pgfpathmoveto{\pgfqpoint{2.146371in}{2.863742in}}%
\pgfpathlineto{\pgfqpoint{2.144153in}{2.867583in}}%
\pgfpathlineto{\pgfqpoint{2.139718in}{2.867583in}}%
\pgfpathlineto{\pgfqpoint{2.137500in}{2.863742in}}%
\pgfpathlineto{\pgfqpoint{2.139718in}{2.859901in}}%
\pgfpathlineto{\pgfqpoint{2.144153in}{2.859901in}}%
\pgfpathlineto{\pgfqpoint{2.146371in}{2.863742in}}%
\pgfpathlineto{\pgfqpoint{2.144153in}{2.867583in}}%
\pgfusepath{fill}%
\end{pgfscope}%
\begin{pgfscope}%
\pgfpathrectangle{\pgfqpoint{1.432000in}{0.528000in}}{\pgfqpoint{3.696000in}{3.696000in}} %
\pgfusepath{clip}%
\pgfsetbuttcap%
\pgfsetroundjoin%
\definecolor{currentfill}{rgb}{0.195860,0.395433,0.555276}%
\pgfsetfillcolor{currentfill}%
\pgfsetlinewidth{0.000000pt}%
\definecolor{currentstroke}{rgb}{0.000000,0.000000,0.000000}%
\pgfsetstrokecolor{currentstroke}%
\pgfsetdash{}{0pt}%
\pgfpathmoveto{\pgfqpoint{2.250323in}{2.859307in}}%
\pgfpathlineto{\pgfqpoint{2.181852in}{2.859307in}}%
\pgfpathlineto{\pgfqpoint{2.186287in}{2.850436in}}%
\pgfpathlineto{\pgfqpoint{2.141935in}{2.863742in}}%
\pgfpathlineto{\pgfqpoint{2.186287in}{2.877048in}}%
\pgfpathlineto{\pgfqpoint{2.181852in}{2.868177in}}%
\pgfpathlineto{\pgfqpoint{2.250323in}{2.868177in}}%
\pgfpathlineto{\pgfqpoint{2.250323in}{2.859307in}}%
\pgfusepath{fill}%
\end{pgfscope}%
\begin{pgfscope}%
\pgfpathrectangle{\pgfqpoint{1.432000in}{0.528000in}}{\pgfqpoint{3.696000in}{3.696000in}} %
\pgfusepath{clip}%
\pgfsetbuttcap%
\pgfsetroundjoin%
\definecolor{currentfill}{rgb}{0.188923,0.410910,0.556326}%
\pgfsetfillcolor{currentfill}%
\pgfsetlinewidth{0.000000pt}%
\definecolor{currentstroke}{rgb}{0.000000,0.000000,0.000000}%
\pgfsetstrokecolor{currentstroke}%
\pgfsetdash{}{0pt}%
\pgfpathmoveto{\pgfqpoint{2.355574in}{2.860606in}}%
\pgfpathlineto{\pgfqpoint{2.275412in}{2.940767in}}%
\pgfpathlineto{\pgfqpoint{2.272276in}{2.931359in}}%
\pgfpathlineto{\pgfqpoint{2.250323in}{2.972129in}}%
\pgfpathlineto{\pgfqpoint{2.291093in}{2.950176in}}%
\pgfpathlineto{\pgfqpoint{2.281684in}{2.947040in}}%
\pgfpathlineto{\pgfqpoint{2.361846in}{2.866878in}}%
\pgfpathlineto{\pgfqpoint{2.355574in}{2.860606in}}%
\pgfusepath{fill}%
\end{pgfscope}%
\begin{pgfscope}%
\pgfpathrectangle{\pgfqpoint{1.432000in}{0.528000in}}{\pgfqpoint{3.696000in}{3.696000in}} %
\pgfusepath{clip}%
\pgfsetbuttcap%
\pgfsetroundjoin%
\definecolor{currentfill}{rgb}{0.258965,0.251537,0.524736}%
\pgfsetfillcolor{currentfill}%
\pgfsetlinewidth{0.000000pt}%
\definecolor{currentstroke}{rgb}{0.000000,0.000000,0.000000}%
\pgfsetstrokecolor{currentstroke}%
\pgfsetdash{}{0pt}%
\pgfpathmoveto{\pgfqpoint{2.463961in}{2.860606in}}%
\pgfpathlineto{\pgfqpoint{2.383799in}{2.940767in}}%
\pgfpathlineto{\pgfqpoint{2.380663in}{2.931359in}}%
\pgfpathlineto{\pgfqpoint{2.358710in}{2.972129in}}%
\pgfpathlineto{\pgfqpoint{2.399480in}{2.950176in}}%
\pgfpathlineto{\pgfqpoint{2.390071in}{2.947040in}}%
\pgfpathlineto{\pgfqpoint{2.470233in}{2.866878in}}%
\pgfpathlineto{\pgfqpoint{2.463961in}{2.860606in}}%
\pgfusepath{fill}%
\end{pgfscope}%
\begin{pgfscope}%
\pgfpathrectangle{\pgfqpoint{1.432000in}{0.528000in}}{\pgfqpoint{3.696000in}{3.696000in}} %
\pgfusepath{clip}%
\pgfsetbuttcap%
\pgfsetroundjoin%
\definecolor{currentfill}{rgb}{0.274128,0.199721,0.498911}%
\pgfsetfillcolor{currentfill}%
\pgfsetlinewidth{0.000000pt}%
\definecolor{currentstroke}{rgb}{0.000000,0.000000,0.000000}%
\pgfsetstrokecolor{currentstroke}%
\pgfsetdash{}{0pt}%
\pgfpathmoveto{\pgfqpoint{2.575484in}{2.859307in}}%
\pgfpathlineto{\pgfqpoint{2.507014in}{2.859307in}}%
\pgfpathlineto{\pgfqpoint{2.511449in}{2.850436in}}%
\pgfpathlineto{\pgfqpoint{2.467097in}{2.863742in}}%
\pgfpathlineto{\pgfqpoint{2.511449in}{2.877048in}}%
\pgfpathlineto{\pgfqpoint{2.507014in}{2.868177in}}%
\pgfpathlineto{\pgfqpoint{2.575484in}{2.868177in}}%
\pgfpathlineto{\pgfqpoint{2.575484in}{2.859307in}}%
\pgfusepath{fill}%
\end{pgfscope}%
\begin{pgfscope}%
\pgfpathrectangle{\pgfqpoint{1.432000in}{0.528000in}}{\pgfqpoint{3.696000in}{3.696000in}} %
\pgfusepath{clip}%
\pgfsetbuttcap%
\pgfsetroundjoin%
\definecolor{currentfill}{rgb}{0.280868,0.160771,0.472899}%
\pgfsetfillcolor{currentfill}%
\pgfsetlinewidth{0.000000pt}%
\definecolor{currentstroke}{rgb}{0.000000,0.000000,0.000000}%
\pgfsetstrokecolor{currentstroke}%
\pgfsetdash{}{0pt}%
\pgfpathmoveto{\pgfqpoint{2.572348in}{2.860606in}}%
\pgfpathlineto{\pgfqpoint{2.492186in}{2.940767in}}%
\pgfpathlineto{\pgfqpoint{2.489050in}{2.931359in}}%
\pgfpathlineto{\pgfqpoint{2.467097in}{2.972129in}}%
\pgfpathlineto{\pgfqpoint{2.507867in}{2.950176in}}%
\pgfpathlineto{\pgfqpoint{2.498458in}{2.947040in}}%
\pgfpathlineto{\pgfqpoint{2.578620in}{2.866878in}}%
\pgfpathlineto{\pgfqpoint{2.572348in}{2.860606in}}%
\pgfusepath{fill}%
\end{pgfscope}%
\begin{pgfscope}%
\pgfpathrectangle{\pgfqpoint{1.432000in}{0.528000in}}{\pgfqpoint{3.696000in}{3.696000in}} %
\pgfusepath{clip}%
\pgfsetbuttcap%
\pgfsetroundjoin%
\definecolor{currentfill}{rgb}{0.278826,0.175490,0.483397}%
\pgfsetfillcolor{currentfill}%
\pgfsetlinewidth{0.000000pt}%
\definecolor{currentstroke}{rgb}{0.000000,0.000000,0.000000}%
\pgfsetstrokecolor{currentstroke}%
\pgfsetdash{}{0pt}%
\pgfpathmoveto{\pgfqpoint{2.683871in}{2.859307in}}%
\pgfpathlineto{\pgfqpoint{2.615401in}{2.859307in}}%
\pgfpathlineto{\pgfqpoint{2.619836in}{2.850436in}}%
\pgfpathlineto{\pgfqpoint{2.575484in}{2.863742in}}%
\pgfpathlineto{\pgfqpoint{2.619836in}{2.877048in}}%
\pgfpathlineto{\pgfqpoint{2.615401in}{2.868177in}}%
\pgfpathlineto{\pgfqpoint{2.683871in}{2.868177in}}%
\pgfpathlineto{\pgfqpoint{2.683871in}{2.859307in}}%
\pgfusepath{fill}%
\end{pgfscope}%
\begin{pgfscope}%
\pgfpathrectangle{\pgfqpoint{1.432000in}{0.528000in}}{\pgfqpoint{3.696000in}{3.696000in}} %
\pgfusepath{clip}%
\pgfsetbuttcap%
\pgfsetroundjoin%
\definecolor{currentfill}{rgb}{0.231674,0.318106,0.544834}%
\pgfsetfillcolor{currentfill}%
\pgfsetlinewidth{0.000000pt}%
\definecolor{currentstroke}{rgb}{0.000000,0.000000,0.000000}%
\pgfsetstrokecolor{currentstroke}%
\pgfsetdash{}{0pt}%
\pgfpathmoveto{\pgfqpoint{2.680735in}{2.860606in}}%
\pgfpathlineto{\pgfqpoint{2.600573in}{2.940767in}}%
\pgfpathlineto{\pgfqpoint{2.597437in}{2.931359in}}%
\pgfpathlineto{\pgfqpoint{2.575484in}{2.972129in}}%
\pgfpathlineto{\pgfqpoint{2.616254in}{2.950176in}}%
\pgfpathlineto{\pgfqpoint{2.606845in}{2.947040in}}%
\pgfpathlineto{\pgfqpoint{2.687007in}{2.866878in}}%
\pgfpathlineto{\pgfqpoint{2.680735in}{2.860606in}}%
\pgfusepath{fill}%
\end{pgfscope}%
\begin{pgfscope}%
\pgfpathrectangle{\pgfqpoint{1.432000in}{0.528000in}}{\pgfqpoint{3.696000in}{3.696000in}} %
\pgfusepath{clip}%
\pgfsetbuttcap%
\pgfsetroundjoin%
\definecolor{currentfill}{rgb}{0.121380,0.629492,0.531973}%
\pgfsetfillcolor{currentfill}%
\pgfsetlinewidth{0.000000pt}%
\definecolor{currentstroke}{rgb}{0.000000,0.000000,0.000000}%
\pgfsetstrokecolor{currentstroke}%
\pgfsetdash{}{0pt}%
\pgfpathmoveto{\pgfqpoint{2.789122in}{2.860606in}}%
\pgfpathlineto{\pgfqpoint{2.708960in}{2.940767in}}%
\pgfpathlineto{\pgfqpoint{2.705824in}{2.931359in}}%
\pgfpathlineto{\pgfqpoint{2.683871in}{2.972129in}}%
\pgfpathlineto{\pgfqpoint{2.724641in}{2.950176in}}%
\pgfpathlineto{\pgfqpoint{2.715233in}{2.947040in}}%
\pgfpathlineto{\pgfqpoint{2.795394in}{2.866878in}}%
\pgfpathlineto{\pgfqpoint{2.789122in}{2.860606in}}%
\pgfusepath{fill}%
\end{pgfscope}%
\begin{pgfscope}%
\pgfpathrectangle{\pgfqpoint{1.432000in}{0.528000in}}{\pgfqpoint{3.696000in}{3.696000in}} %
\pgfusepath{clip}%
\pgfsetbuttcap%
\pgfsetroundjoin%
\definecolor{currentfill}{rgb}{0.190631,0.407061,0.556089}%
\pgfsetfillcolor{currentfill}%
\pgfsetlinewidth{0.000000pt}%
\definecolor{currentstroke}{rgb}{0.000000,0.000000,0.000000}%
\pgfsetstrokecolor{currentstroke}%
\pgfsetdash{}{0pt}%
\pgfpathmoveto{\pgfqpoint{2.897509in}{2.860606in}}%
\pgfpathlineto{\pgfqpoint{2.817347in}{2.940767in}}%
\pgfpathlineto{\pgfqpoint{2.814211in}{2.931359in}}%
\pgfpathlineto{\pgfqpoint{2.792258in}{2.972129in}}%
\pgfpathlineto{\pgfqpoint{2.833028in}{2.950176in}}%
\pgfpathlineto{\pgfqpoint{2.823620in}{2.947040in}}%
\pgfpathlineto{\pgfqpoint{2.903781in}{2.866878in}}%
\pgfpathlineto{\pgfqpoint{2.897509in}{2.860606in}}%
\pgfusepath{fill}%
\end{pgfscope}%
\begin{pgfscope}%
\pgfpathrectangle{\pgfqpoint{1.432000in}{0.528000in}}{\pgfqpoint{3.696000in}{3.696000in}} %
\pgfusepath{clip}%
\pgfsetbuttcap%
\pgfsetroundjoin%
\definecolor{currentfill}{rgb}{0.225863,0.330805,0.547314}%
\pgfsetfillcolor{currentfill}%
\pgfsetlinewidth{0.000000pt}%
\definecolor{currentstroke}{rgb}{0.000000,0.000000,0.000000}%
\pgfsetstrokecolor{currentstroke}%
\pgfsetdash{}{0pt}%
\pgfpathmoveto{\pgfqpoint{2.896210in}{2.863742in}}%
\pgfpathlineto{\pgfqpoint{2.896210in}{2.932212in}}%
\pgfpathlineto{\pgfqpoint{2.887340in}{2.927777in}}%
\pgfpathlineto{\pgfqpoint{2.900645in}{2.972129in}}%
\pgfpathlineto{\pgfqpoint{2.913951in}{2.927777in}}%
\pgfpathlineto{\pgfqpoint{2.905080in}{2.932212in}}%
\pgfpathlineto{\pgfqpoint{2.905080in}{2.863742in}}%
\pgfpathlineto{\pgfqpoint{2.896210in}{2.863742in}}%
\pgfusepath{fill}%
\end{pgfscope}%
\begin{pgfscope}%
\pgfpathrectangle{\pgfqpoint{1.432000in}{0.528000in}}{\pgfqpoint{3.696000in}{3.696000in}} %
\pgfusepath{clip}%
\pgfsetbuttcap%
\pgfsetroundjoin%
\definecolor{currentfill}{rgb}{0.235526,0.309527,0.542944}%
\pgfsetfillcolor{currentfill}%
\pgfsetlinewidth{0.000000pt}%
\definecolor{currentstroke}{rgb}{0.000000,0.000000,0.000000}%
\pgfsetstrokecolor{currentstroke}%
\pgfsetdash{}{0pt}%
\pgfpathmoveto{\pgfqpoint{3.004597in}{2.863742in}}%
\pgfpathlineto{\pgfqpoint{3.004597in}{2.932212in}}%
\pgfpathlineto{\pgfqpoint{2.995727in}{2.927777in}}%
\pgfpathlineto{\pgfqpoint{3.009032in}{2.972129in}}%
\pgfpathlineto{\pgfqpoint{3.022338in}{2.927777in}}%
\pgfpathlineto{\pgfqpoint{3.013467in}{2.932212in}}%
\pgfpathlineto{\pgfqpoint{3.013467in}{2.863742in}}%
\pgfpathlineto{\pgfqpoint{3.004597in}{2.863742in}}%
\pgfusepath{fill}%
\end{pgfscope}%
\begin{pgfscope}%
\pgfpathrectangle{\pgfqpoint{1.432000in}{0.528000in}}{\pgfqpoint{3.696000in}{3.696000in}} %
\pgfusepath{clip}%
\pgfsetbuttcap%
\pgfsetroundjoin%
\definecolor{currentfill}{rgb}{0.243113,0.292092,0.538516}%
\pgfsetfillcolor{currentfill}%
\pgfsetlinewidth{0.000000pt}%
\definecolor{currentstroke}{rgb}{0.000000,0.000000,0.000000}%
\pgfsetstrokecolor{currentstroke}%
\pgfsetdash{}{0pt}%
\pgfpathmoveto{\pgfqpoint{3.004597in}{2.863742in}}%
\pgfpathlineto{\pgfqpoint{3.004597in}{3.040599in}}%
\pgfpathlineto{\pgfqpoint{2.995727in}{3.036164in}}%
\pgfpathlineto{\pgfqpoint{3.009032in}{3.080516in}}%
\pgfpathlineto{\pgfqpoint{3.022338in}{3.036164in}}%
\pgfpathlineto{\pgfqpoint{3.013467in}{3.040599in}}%
\pgfpathlineto{\pgfqpoint{3.013467in}{2.863742in}}%
\pgfpathlineto{\pgfqpoint{3.004597in}{2.863742in}}%
\pgfusepath{fill}%
\end{pgfscope}%
\begin{pgfscope}%
\pgfpathrectangle{\pgfqpoint{1.432000in}{0.528000in}}{\pgfqpoint{3.696000in}{3.696000in}} %
\pgfusepath{clip}%
\pgfsetbuttcap%
\pgfsetroundjoin%
\definecolor{currentfill}{rgb}{0.269944,0.014625,0.341379}%
\pgfsetfillcolor{currentfill}%
\pgfsetlinewidth{0.000000pt}%
\definecolor{currentstroke}{rgb}{0.000000,0.000000,0.000000}%
\pgfsetstrokecolor{currentstroke}%
\pgfsetdash{}{0pt}%
\pgfpathmoveto{\pgfqpoint{3.112984in}{2.863742in}}%
\pgfpathlineto{\pgfqpoint{3.112984in}{2.932212in}}%
\pgfpathlineto{\pgfqpoint{3.104114in}{2.927777in}}%
\pgfpathlineto{\pgfqpoint{3.117419in}{2.972129in}}%
\pgfpathlineto{\pgfqpoint{3.130725in}{2.927777in}}%
\pgfpathlineto{\pgfqpoint{3.121855in}{2.932212in}}%
\pgfpathlineto{\pgfqpoint{3.121855in}{2.863742in}}%
\pgfpathlineto{\pgfqpoint{3.112984in}{2.863742in}}%
\pgfusepath{fill}%
\end{pgfscope}%
\begin{pgfscope}%
\pgfpathrectangle{\pgfqpoint{1.432000in}{0.528000in}}{\pgfqpoint{3.696000in}{3.696000in}} %
\pgfusepath{clip}%
\pgfsetbuttcap%
\pgfsetroundjoin%
\definecolor{currentfill}{rgb}{0.216210,0.351535,0.550627}%
\pgfsetfillcolor{currentfill}%
\pgfsetlinewidth{0.000000pt}%
\definecolor{currentstroke}{rgb}{0.000000,0.000000,0.000000}%
\pgfsetstrokecolor{currentstroke}%
\pgfsetdash{}{0pt}%
\pgfpathmoveto{\pgfqpoint{3.112984in}{2.863742in}}%
\pgfpathlineto{\pgfqpoint{3.112984in}{3.040599in}}%
\pgfpathlineto{\pgfqpoint{3.104114in}{3.036164in}}%
\pgfpathlineto{\pgfqpoint{3.117419in}{3.080516in}}%
\pgfpathlineto{\pgfqpoint{3.130725in}{3.036164in}}%
\pgfpathlineto{\pgfqpoint{3.121855in}{3.040599in}}%
\pgfpathlineto{\pgfqpoint{3.121855in}{2.863742in}}%
\pgfpathlineto{\pgfqpoint{3.112984in}{2.863742in}}%
\pgfusepath{fill}%
\end{pgfscope}%
\begin{pgfscope}%
\pgfpathrectangle{\pgfqpoint{1.432000in}{0.528000in}}{\pgfqpoint{3.696000in}{3.696000in}} %
\pgfusepath{clip}%
\pgfsetbuttcap%
\pgfsetroundjoin%
\definecolor{currentfill}{rgb}{0.273006,0.204520,0.501721}%
\pgfsetfillcolor{currentfill}%
\pgfsetlinewidth{0.000000pt}%
\definecolor{currentstroke}{rgb}{0.000000,0.000000,0.000000}%
\pgfsetstrokecolor{currentstroke}%
\pgfsetdash{}{0pt}%
\pgfpathmoveto{\pgfqpoint{3.221839in}{2.861758in}}%
\pgfpathlineto{\pgfqpoint{3.131304in}{3.042830in}}%
\pgfpathlineto{\pgfqpoint{3.125353in}{3.034896in}}%
\pgfpathlineto{\pgfqpoint{3.117419in}{3.080516in}}%
\pgfpathlineto{\pgfqpoint{3.149155in}{3.046797in}}%
\pgfpathlineto{\pgfqpoint{3.139238in}{3.046797in}}%
\pgfpathlineto{\pgfqpoint{3.229773in}{2.865725in}}%
\pgfpathlineto{\pgfqpoint{3.221839in}{2.861758in}}%
\pgfusepath{fill}%
\end{pgfscope}%
\begin{pgfscope}%
\pgfpathrectangle{\pgfqpoint{1.432000in}{0.528000in}}{\pgfqpoint{3.696000in}{3.696000in}} %
\pgfusepath{clip}%
\pgfsetbuttcap%
\pgfsetroundjoin%
\definecolor{currentfill}{rgb}{0.223925,0.334994,0.548053}%
\pgfsetfillcolor{currentfill}%
\pgfsetlinewidth{0.000000pt}%
\definecolor{currentstroke}{rgb}{0.000000,0.000000,0.000000}%
\pgfsetstrokecolor{currentstroke}%
\pgfsetdash{}{0pt}%
\pgfpathmoveto{\pgfqpoint{3.330227in}{2.861758in}}%
\pgfpathlineto{\pgfqpoint{3.239691in}{3.042830in}}%
\pgfpathlineto{\pgfqpoint{3.233740in}{3.034896in}}%
\pgfpathlineto{\pgfqpoint{3.225806in}{3.080516in}}%
\pgfpathlineto{\pgfqpoint{3.257542in}{3.046797in}}%
\pgfpathlineto{\pgfqpoint{3.247625in}{3.046797in}}%
\pgfpathlineto{\pgfqpoint{3.338161in}{2.865725in}}%
\pgfpathlineto{\pgfqpoint{3.330227in}{2.861758in}}%
\pgfusepath{fill}%
\end{pgfscope}%
\begin{pgfscope}%
\pgfpathrectangle{\pgfqpoint{1.432000in}{0.528000in}}{\pgfqpoint{3.696000in}{3.696000in}} %
\pgfusepath{clip}%
\pgfsetbuttcap%
\pgfsetroundjoin%
\definecolor{currentfill}{rgb}{0.122606,0.585371,0.546557}%
\pgfsetfillcolor{currentfill}%
\pgfsetlinewidth{0.000000pt}%
\definecolor{currentstroke}{rgb}{0.000000,0.000000,0.000000}%
\pgfsetstrokecolor{currentstroke}%
\pgfsetdash{}{0pt}%
\pgfpathmoveto{\pgfqpoint{3.439444in}{2.860606in}}%
\pgfpathlineto{\pgfqpoint{3.250896in}{3.049155in}}%
\pgfpathlineto{\pgfqpoint{3.247760in}{3.039746in}}%
\pgfpathlineto{\pgfqpoint{3.225806in}{3.080516in}}%
\pgfpathlineto{\pgfqpoint{3.266577in}{3.058563in}}%
\pgfpathlineto{\pgfqpoint{3.257168in}{3.055427in}}%
\pgfpathlineto{\pgfqpoint{3.445717in}{2.866878in}}%
\pgfpathlineto{\pgfqpoint{3.439444in}{2.860606in}}%
\pgfusepath{fill}%
\end{pgfscope}%
\begin{pgfscope}%
\pgfpathrectangle{\pgfqpoint{1.432000in}{0.528000in}}{\pgfqpoint{3.696000in}{3.696000in}} %
\pgfusepath{clip}%
\pgfsetbuttcap%
\pgfsetroundjoin%
\definecolor{currentfill}{rgb}{0.126453,0.570633,0.549841}%
\pgfsetfillcolor{currentfill}%
\pgfsetlinewidth{0.000000pt}%
\definecolor{currentstroke}{rgb}{0.000000,0.000000,0.000000}%
\pgfsetstrokecolor{currentstroke}%
\pgfsetdash{}{0pt}%
\pgfpathmoveto{\pgfqpoint{3.547832in}{2.860606in}}%
\pgfpathlineto{\pgfqpoint{3.359283in}{3.049155in}}%
\pgfpathlineto{\pgfqpoint{3.356147in}{3.039746in}}%
\pgfpathlineto{\pgfqpoint{3.334194in}{3.080516in}}%
\pgfpathlineto{\pgfqpoint{3.374964in}{3.058563in}}%
\pgfpathlineto{\pgfqpoint{3.365555in}{3.055427in}}%
\pgfpathlineto{\pgfqpoint{3.554104in}{2.866878in}}%
\pgfpathlineto{\pgfqpoint{3.547832in}{2.860606in}}%
\pgfusepath{fill}%
\end{pgfscope}%
\begin{pgfscope}%
\pgfpathrectangle{\pgfqpoint{1.432000in}{0.528000in}}{\pgfqpoint{3.696000in}{3.696000in}} %
\pgfusepath{clip}%
\pgfsetbuttcap%
\pgfsetroundjoin%
\definecolor{currentfill}{rgb}{0.129933,0.559582,0.551864}%
\pgfsetfillcolor{currentfill}%
\pgfsetlinewidth{0.000000pt}%
\definecolor{currentstroke}{rgb}{0.000000,0.000000,0.000000}%
\pgfsetstrokecolor{currentstroke}%
\pgfsetdash{}{0pt}%
\pgfpathmoveto{\pgfqpoint{3.656895in}{2.860052in}}%
\pgfpathlineto{\pgfqpoint{3.364946in}{3.054684in}}%
\pgfpathlineto{\pgfqpoint{3.363716in}{3.044843in}}%
\pgfpathlineto{\pgfqpoint{3.334194in}{3.080516in}}%
\pgfpathlineto{\pgfqpoint{3.378477in}{3.066985in}}%
\pgfpathlineto{\pgfqpoint{3.369867in}{3.062065in}}%
\pgfpathlineto{\pgfqpoint{3.661815in}{2.867432in}}%
\pgfpathlineto{\pgfqpoint{3.656895in}{2.860052in}}%
\pgfusepath{fill}%
\end{pgfscope}%
\begin{pgfscope}%
\pgfpathrectangle{\pgfqpoint{1.432000in}{0.528000in}}{\pgfqpoint{3.696000in}{3.696000in}} %
\pgfusepath{clip}%
\pgfsetbuttcap%
\pgfsetroundjoin%
\definecolor{currentfill}{rgb}{0.147607,0.511733,0.557049}%
\pgfsetfillcolor{currentfill}%
\pgfsetlinewidth{0.000000pt}%
\definecolor{currentstroke}{rgb}{0.000000,0.000000,0.000000}%
\pgfsetstrokecolor{currentstroke}%
\pgfsetdash{}{0pt}%
\pgfpathmoveto{\pgfqpoint{3.765282in}{2.860052in}}%
\pgfpathlineto{\pgfqpoint{3.473333in}{3.054684in}}%
\pgfpathlineto{\pgfqpoint{3.472103in}{3.044843in}}%
\pgfpathlineto{\pgfqpoint{3.442581in}{3.080516in}}%
\pgfpathlineto{\pgfqpoint{3.486864in}{3.066985in}}%
\pgfpathlineto{\pgfqpoint{3.478254in}{3.062065in}}%
\pgfpathlineto{\pgfqpoint{3.770202in}{2.867432in}}%
\pgfpathlineto{\pgfqpoint{3.765282in}{2.860052in}}%
\pgfusepath{fill}%
\end{pgfscope}%
\begin{pgfscope}%
\pgfpathrectangle{\pgfqpoint{1.432000in}{0.528000in}}{\pgfqpoint{3.696000in}{3.696000in}} %
\pgfusepath{clip}%
\pgfsetbuttcap%
\pgfsetroundjoin%
\definecolor{currentfill}{rgb}{0.227802,0.326594,0.546532}%
\pgfsetfillcolor{currentfill}%
\pgfsetlinewidth{0.000000pt}%
\definecolor{currentstroke}{rgb}{0.000000,0.000000,0.000000}%
\pgfsetstrokecolor{currentstroke}%
\pgfsetdash{}{0pt}%
\pgfpathmoveto{\pgfqpoint{3.764606in}{2.860606in}}%
\pgfpathlineto{\pgfqpoint{3.467670in}{3.157542in}}%
\pgfpathlineto{\pgfqpoint{3.464534in}{3.148133in}}%
\pgfpathlineto{\pgfqpoint{3.442581in}{3.188903in}}%
\pgfpathlineto{\pgfqpoint{3.483351in}{3.166950in}}%
\pgfpathlineto{\pgfqpoint{3.473942in}{3.163814in}}%
\pgfpathlineto{\pgfqpoint{3.770878in}{2.866878in}}%
\pgfpathlineto{\pgfqpoint{3.764606in}{2.860606in}}%
\pgfusepath{fill}%
\end{pgfscope}%
\begin{pgfscope}%
\pgfpathrectangle{\pgfqpoint{1.432000in}{0.528000in}}{\pgfqpoint{3.696000in}{3.696000in}} %
\pgfusepath{clip}%
\pgfsetbuttcap%
\pgfsetroundjoin%
\definecolor{currentfill}{rgb}{0.271305,0.019942,0.347269}%
\pgfsetfillcolor{currentfill}%
\pgfsetlinewidth{0.000000pt}%
\definecolor{currentstroke}{rgb}{0.000000,0.000000,0.000000}%
\pgfsetstrokecolor{currentstroke}%
\pgfsetdash{}{0pt}%
\pgfpathmoveto{\pgfqpoint{3.873669in}{2.860052in}}%
\pgfpathlineto{\pgfqpoint{3.581720in}{3.054684in}}%
\pgfpathlineto{\pgfqpoint{3.580490in}{3.044843in}}%
\pgfpathlineto{\pgfqpoint{3.550968in}{3.080516in}}%
\pgfpathlineto{\pgfqpoint{3.595251in}{3.066985in}}%
\pgfpathlineto{\pgfqpoint{3.586641in}{3.062065in}}%
\pgfpathlineto{\pgfqpoint{3.878589in}{2.867432in}}%
\pgfpathlineto{\pgfqpoint{3.873669in}{2.860052in}}%
\pgfusepath{fill}%
\end{pgfscope}%
\begin{pgfscope}%
\pgfpathrectangle{\pgfqpoint{1.432000in}{0.528000in}}{\pgfqpoint{3.696000in}{3.696000in}} %
\pgfusepath{clip}%
\pgfsetbuttcap%
\pgfsetroundjoin%
\definecolor{currentfill}{rgb}{0.175707,0.697900,0.491033}%
\pgfsetfillcolor{currentfill}%
\pgfsetlinewidth{0.000000pt}%
\definecolor{currentstroke}{rgb}{0.000000,0.000000,0.000000}%
\pgfsetstrokecolor{currentstroke}%
\pgfsetdash{}{0pt}%
\pgfpathmoveto{\pgfqpoint{3.872993in}{2.860606in}}%
\pgfpathlineto{\pgfqpoint{3.576057in}{3.157542in}}%
\pgfpathlineto{\pgfqpoint{3.572921in}{3.148133in}}%
\pgfpathlineto{\pgfqpoint{3.550968in}{3.188903in}}%
\pgfpathlineto{\pgfqpoint{3.591738in}{3.166950in}}%
\pgfpathlineto{\pgfqpoint{3.582329in}{3.163814in}}%
\pgfpathlineto{\pgfqpoint{3.879265in}{2.866878in}}%
\pgfpathlineto{\pgfqpoint{3.872993in}{2.860606in}}%
\pgfusepath{fill}%
\end{pgfscope}%
\begin{pgfscope}%
\pgfpathrectangle{\pgfqpoint{1.432000in}{0.528000in}}{\pgfqpoint{3.696000in}{3.696000in}} %
\pgfusepath{clip}%
\pgfsetbuttcap%
\pgfsetroundjoin%
\definecolor{currentfill}{rgb}{0.276194,0.190074,0.493001}%
\pgfsetfillcolor{currentfill}%
\pgfsetlinewidth{0.000000pt}%
\definecolor{currentstroke}{rgb}{0.000000,0.000000,0.000000}%
\pgfsetstrokecolor{currentstroke}%
\pgfsetdash{}{0pt}%
\pgfpathmoveto{\pgfqpoint{3.981855in}{2.860194in}}%
\pgfpathlineto{\pgfqpoint{3.580240in}{3.161405in}}%
\pgfpathlineto{\pgfqpoint{3.578466in}{3.151648in}}%
\pgfpathlineto{\pgfqpoint{3.550968in}{3.188903in}}%
\pgfpathlineto{\pgfqpoint{3.594433in}{3.172937in}}%
\pgfpathlineto{\pgfqpoint{3.585562in}{3.168501in}}%
\pgfpathlineto{\pgfqpoint{3.987177in}{2.867290in}}%
\pgfpathlineto{\pgfqpoint{3.981855in}{2.860194in}}%
\pgfusepath{fill}%
\end{pgfscope}%
\begin{pgfscope}%
\pgfpathrectangle{\pgfqpoint{1.432000in}{0.528000in}}{\pgfqpoint{3.696000in}{3.696000in}} %
\pgfusepath{clip}%
\pgfsetbuttcap%
\pgfsetroundjoin%
\definecolor{currentfill}{rgb}{0.125394,0.574318,0.549086}%
\pgfsetfillcolor{currentfill}%
\pgfsetlinewidth{0.000000pt}%
\definecolor{currentstroke}{rgb}{0.000000,0.000000,0.000000}%
\pgfsetstrokecolor{currentstroke}%
\pgfsetdash{}{0pt}%
\pgfpathmoveto{\pgfqpoint{3.981380in}{2.860606in}}%
\pgfpathlineto{\pgfqpoint{3.684444in}{3.157542in}}%
\pgfpathlineto{\pgfqpoint{3.681308in}{3.148133in}}%
\pgfpathlineto{\pgfqpoint{3.659355in}{3.188903in}}%
\pgfpathlineto{\pgfqpoint{3.700125in}{3.166950in}}%
\pgfpathlineto{\pgfqpoint{3.690716in}{3.163814in}}%
\pgfpathlineto{\pgfqpoint{3.987652in}{2.866878in}}%
\pgfpathlineto{\pgfqpoint{3.981380in}{2.860606in}}%
\pgfusepath{fill}%
\end{pgfscope}%
\begin{pgfscope}%
\pgfpathrectangle{\pgfqpoint{1.432000in}{0.528000in}}{\pgfqpoint{3.696000in}{3.696000in}} %
\pgfusepath{clip}%
\pgfsetbuttcap%
\pgfsetroundjoin%
\definecolor{currentfill}{rgb}{0.263663,0.237631,0.518762}%
\pgfsetfillcolor{currentfill}%
\pgfsetlinewidth{0.000000pt}%
\definecolor{currentstroke}{rgb}{0.000000,0.000000,0.000000}%
\pgfsetstrokecolor{currentstroke}%
\pgfsetdash{}{0pt}%
\pgfpathmoveto{\pgfqpoint{4.090242in}{2.860194in}}%
\pgfpathlineto{\pgfqpoint{3.688627in}{3.161405in}}%
\pgfpathlineto{\pgfqpoint{3.686853in}{3.151648in}}%
\pgfpathlineto{\pgfqpoint{3.659355in}{3.188903in}}%
\pgfpathlineto{\pgfqpoint{3.702820in}{3.172937in}}%
\pgfpathlineto{\pgfqpoint{3.693949in}{3.168501in}}%
\pgfpathlineto{\pgfqpoint{4.095564in}{2.867290in}}%
\pgfpathlineto{\pgfqpoint{4.090242in}{2.860194in}}%
\pgfusepath{fill}%
\end{pgfscope}%
\begin{pgfscope}%
\pgfpathrectangle{\pgfqpoint{1.432000in}{0.528000in}}{\pgfqpoint{3.696000in}{3.696000in}} %
\pgfusepath{clip}%
\pgfsetbuttcap%
\pgfsetroundjoin%
\definecolor{currentfill}{rgb}{0.151918,0.500685,0.557587}%
\pgfsetfillcolor{currentfill}%
\pgfsetlinewidth{0.000000pt}%
\definecolor{currentstroke}{rgb}{0.000000,0.000000,0.000000}%
\pgfsetstrokecolor{currentstroke}%
\pgfsetdash{}{0pt}%
\pgfpathmoveto{\pgfqpoint{4.089767in}{2.860606in}}%
\pgfpathlineto{\pgfqpoint{3.792831in}{3.157542in}}%
\pgfpathlineto{\pgfqpoint{3.789695in}{3.148133in}}%
\pgfpathlineto{\pgfqpoint{3.767742in}{3.188903in}}%
\pgfpathlineto{\pgfqpoint{3.808512in}{3.166950in}}%
\pgfpathlineto{\pgfqpoint{3.799104in}{3.163814in}}%
\pgfpathlineto{\pgfqpoint{4.096039in}{2.866878in}}%
\pgfpathlineto{\pgfqpoint{4.089767in}{2.860606in}}%
\pgfusepath{fill}%
\end{pgfscope}%
\begin{pgfscope}%
\pgfpathrectangle{\pgfqpoint{1.432000in}{0.528000in}}{\pgfqpoint{3.696000in}{3.696000in}} %
\pgfusepath{clip}%
\pgfsetbuttcap%
\pgfsetroundjoin%
\definecolor{currentfill}{rgb}{0.265145,0.232956,0.516599}%
\pgfsetfillcolor{currentfill}%
\pgfsetlinewidth{0.000000pt}%
\definecolor{currentstroke}{rgb}{0.000000,0.000000,0.000000}%
\pgfsetstrokecolor{currentstroke}%
\pgfsetdash{}{0pt}%
\pgfpathmoveto{\pgfqpoint{4.198629in}{2.860194in}}%
\pgfpathlineto{\pgfqpoint{3.797014in}{3.161405in}}%
\pgfpathlineto{\pgfqpoint{3.795240in}{3.151648in}}%
\pgfpathlineto{\pgfqpoint{3.767742in}{3.188903in}}%
\pgfpathlineto{\pgfqpoint{3.811207in}{3.172937in}}%
\pgfpathlineto{\pgfqpoint{3.802336in}{3.168501in}}%
\pgfpathlineto{\pgfqpoint{4.203951in}{2.867290in}}%
\pgfpathlineto{\pgfqpoint{4.198629in}{2.860194in}}%
\pgfusepath{fill}%
\end{pgfscope}%
\begin{pgfscope}%
\pgfpathrectangle{\pgfqpoint{1.432000in}{0.528000in}}{\pgfqpoint{3.696000in}{3.696000in}} %
\pgfusepath{clip}%
\pgfsetbuttcap%
\pgfsetroundjoin%
\definecolor{currentfill}{rgb}{0.126453,0.570633,0.549841}%
\pgfsetfillcolor{currentfill}%
\pgfsetlinewidth{0.000000pt}%
\definecolor{currentstroke}{rgb}{0.000000,0.000000,0.000000}%
\pgfsetstrokecolor{currentstroke}%
\pgfsetdash{}{0pt}%
\pgfpathmoveto{\pgfqpoint{4.198154in}{2.860606in}}%
\pgfpathlineto{\pgfqpoint{3.901218in}{3.157542in}}%
\pgfpathlineto{\pgfqpoint{3.898082in}{3.148133in}}%
\pgfpathlineto{\pgfqpoint{3.876129in}{3.188903in}}%
\pgfpathlineto{\pgfqpoint{3.916899in}{3.166950in}}%
\pgfpathlineto{\pgfqpoint{3.907491in}{3.163814in}}%
\pgfpathlineto{\pgfqpoint{4.204426in}{2.866878in}}%
\pgfpathlineto{\pgfqpoint{4.198154in}{2.860606in}}%
\pgfusepath{fill}%
\end{pgfscope}%
\begin{pgfscope}%
\pgfpathrectangle{\pgfqpoint{1.432000in}{0.528000in}}{\pgfqpoint{3.696000in}{3.696000in}} %
\pgfusepath{clip}%
\pgfsetbuttcap%
\pgfsetroundjoin%
\definecolor{currentfill}{rgb}{0.278791,0.062145,0.386592}%
\pgfsetfillcolor{currentfill}%
\pgfsetlinewidth{0.000000pt}%
\definecolor{currentstroke}{rgb}{0.000000,0.000000,0.000000}%
\pgfsetstrokecolor{currentstroke}%
\pgfsetdash{}{0pt}%
\pgfpathmoveto{\pgfqpoint{4.307217in}{2.860052in}}%
\pgfpathlineto{\pgfqpoint{4.015269in}{3.054684in}}%
\pgfpathlineto{\pgfqpoint{4.014039in}{3.044843in}}%
\pgfpathlineto{\pgfqpoint{3.984516in}{3.080516in}}%
\pgfpathlineto{\pgfqpoint{4.028800in}{3.066985in}}%
\pgfpathlineto{\pgfqpoint{4.020189in}{3.062065in}}%
\pgfpathlineto{\pgfqpoint{4.312138in}{2.867432in}}%
\pgfpathlineto{\pgfqpoint{4.307217in}{2.860052in}}%
\pgfusepath{fill}%
\end{pgfscope}%
\begin{pgfscope}%
\pgfpathrectangle{\pgfqpoint{1.432000in}{0.528000in}}{\pgfqpoint{3.696000in}{3.696000in}} %
\pgfusepath{clip}%
\pgfsetbuttcap%
\pgfsetroundjoin%
\definecolor{currentfill}{rgb}{0.170948,0.694384,0.493803}%
\pgfsetfillcolor{currentfill}%
\pgfsetlinewidth{0.000000pt}%
\definecolor{currentstroke}{rgb}{0.000000,0.000000,0.000000}%
\pgfsetstrokecolor{currentstroke}%
\pgfsetdash{}{0pt}%
\pgfpathmoveto{\pgfqpoint{4.306541in}{2.860606in}}%
\pgfpathlineto{\pgfqpoint{4.009605in}{3.157542in}}%
\pgfpathlineto{\pgfqpoint{4.006469in}{3.148133in}}%
\pgfpathlineto{\pgfqpoint{3.984516in}{3.188903in}}%
\pgfpathlineto{\pgfqpoint{4.025286in}{3.166950in}}%
\pgfpathlineto{\pgfqpoint{4.015878in}{3.163814in}}%
\pgfpathlineto{\pgfqpoint{4.312814in}{2.866878in}}%
\pgfpathlineto{\pgfqpoint{4.306541in}{2.860606in}}%
\pgfusepath{fill}%
\end{pgfscope}%
\begin{pgfscope}%
\pgfpathrectangle{\pgfqpoint{1.432000in}{0.528000in}}{\pgfqpoint{3.696000in}{3.696000in}} %
\pgfusepath{clip}%
\pgfsetbuttcap%
\pgfsetroundjoin%
\definecolor{currentfill}{rgb}{0.274952,0.037752,0.364543}%
\pgfsetfillcolor{currentfill}%
\pgfsetlinewidth{0.000000pt}%
\definecolor{currentstroke}{rgb}{0.000000,0.000000,0.000000}%
\pgfsetstrokecolor{currentstroke}%
\pgfsetdash{}{0pt}%
\pgfpathmoveto{\pgfqpoint{4.305987in}{2.861282in}}%
\pgfpathlineto{\pgfqpoint{4.111355in}{3.153230in}}%
\pgfpathlineto{\pgfqpoint{4.106434in}{3.144620in}}%
\pgfpathlineto{\pgfqpoint{4.092903in}{3.188903in}}%
\pgfpathlineto{\pgfqpoint{4.128576in}{3.159381in}}%
\pgfpathlineto{\pgfqpoint{4.118735in}{3.158151in}}%
\pgfpathlineto{\pgfqpoint{4.313368in}{2.866202in}}%
\pgfpathlineto{\pgfqpoint{4.305987in}{2.861282in}}%
\pgfusepath{fill}%
\end{pgfscope}%
\begin{pgfscope}%
\pgfpathrectangle{\pgfqpoint{1.432000in}{0.528000in}}{\pgfqpoint{3.696000in}{3.696000in}} %
\pgfusepath{clip}%
\pgfsetbuttcap%
\pgfsetroundjoin%
\definecolor{currentfill}{rgb}{0.119423,0.611141,0.538982}%
\pgfsetfillcolor{currentfill}%
\pgfsetlinewidth{0.000000pt}%
\definecolor{currentstroke}{rgb}{0.000000,0.000000,0.000000}%
\pgfsetstrokecolor{currentstroke}%
\pgfsetdash{}{0pt}%
\pgfpathmoveto{\pgfqpoint{4.414928in}{2.860606in}}%
\pgfpathlineto{\pgfqpoint{4.117993in}{3.157542in}}%
\pgfpathlineto{\pgfqpoint{4.114856in}{3.148133in}}%
\pgfpathlineto{\pgfqpoint{4.092903in}{3.188903in}}%
\pgfpathlineto{\pgfqpoint{4.133673in}{3.166950in}}%
\pgfpathlineto{\pgfqpoint{4.124265in}{3.163814in}}%
\pgfpathlineto{\pgfqpoint{4.421201in}{2.866878in}}%
\pgfpathlineto{\pgfqpoint{4.414928in}{2.860606in}}%
\pgfusepath{fill}%
\end{pgfscope}%
\begin{pgfscope}%
\pgfpathrectangle{\pgfqpoint{1.432000in}{0.528000in}}{\pgfqpoint{3.696000in}{3.696000in}} %
\pgfusepath{clip}%
\pgfsetbuttcap%
\pgfsetroundjoin%
\definecolor{currentfill}{rgb}{0.208623,0.367752,0.552675}%
\pgfsetfillcolor{currentfill}%
\pgfsetlinewidth{0.000000pt}%
\definecolor{currentstroke}{rgb}{0.000000,0.000000,0.000000}%
\pgfsetstrokecolor{currentstroke}%
\pgfsetdash{}{0pt}%
\pgfpathmoveto{\pgfqpoint{4.414374in}{2.861282in}}%
\pgfpathlineto{\pgfqpoint{4.219742in}{3.153230in}}%
\pgfpathlineto{\pgfqpoint{4.214821in}{3.144620in}}%
\pgfpathlineto{\pgfqpoint{4.201290in}{3.188903in}}%
\pgfpathlineto{\pgfqpoint{4.236963in}{3.159381in}}%
\pgfpathlineto{\pgfqpoint{4.227122in}{3.158151in}}%
\pgfpathlineto{\pgfqpoint{4.421755in}{2.866202in}}%
\pgfpathlineto{\pgfqpoint{4.414374in}{2.861282in}}%
\pgfusepath{fill}%
\end{pgfscope}%
\begin{pgfscope}%
\pgfpathrectangle{\pgfqpoint{1.432000in}{0.528000in}}{\pgfqpoint{3.696000in}{3.696000in}} %
\pgfusepath{clip}%
\pgfsetbuttcap%
\pgfsetroundjoin%
\definecolor{currentfill}{rgb}{0.187231,0.414746,0.556547}%
\pgfsetfillcolor{currentfill}%
\pgfsetlinewidth{0.000000pt}%
\definecolor{currentstroke}{rgb}{0.000000,0.000000,0.000000}%
\pgfsetstrokecolor{currentstroke}%
\pgfsetdash{}{0pt}%
\pgfpathmoveto{\pgfqpoint{4.523315in}{2.860606in}}%
\pgfpathlineto{\pgfqpoint{4.226380in}{3.157542in}}%
\pgfpathlineto{\pgfqpoint{4.223243in}{3.148133in}}%
\pgfpathlineto{\pgfqpoint{4.201290in}{3.188903in}}%
\pgfpathlineto{\pgfqpoint{4.242060in}{3.166950in}}%
\pgfpathlineto{\pgfqpoint{4.232652in}{3.163814in}}%
\pgfpathlineto{\pgfqpoint{4.529588in}{2.866878in}}%
\pgfpathlineto{\pgfqpoint{4.523315in}{2.860606in}}%
\pgfusepath{fill}%
\end{pgfscope}%
\begin{pgfscope}%
\pgfpathrectangle{\pgfqpoint{1.432000in}{0.528000in}}{\pgfqpoint{3.696000in}{3.696000in}} %
\pgfusepath{clip}%
\pgfsetbuttcap%
\pgfsetroundjoin%
\definecolor{currentfill}{rgb}{0.259857,0.745492,0.444467}%
\pgfsetfillcolor{currentfill}%
\pgfsetlinewidth{0.000000pt}%
\definecolor{currentstroke}{rgb}{0.000000,0.000000,0.000000}%
\pgfsetstrokecolor{currentstroke}%
\pgfsetdash{}{0pt}%
\pgfpathmoveto{\pgfqpoint{4.522761in}{2.861282in}}%
\pgfpathlineto{\pgfqpoint{4.328129in}{3.153230in}}%
\pgfpathlineto{\pgfqpoint{4.323209in}{3.144620in}}%
\pgfpathlineto{\pgfqpoint{4.309677in}{3.188903in}}%
\pgfpathlineto{\pgfqpoint{4.345350in}{3.159381in}}%
\pgfpathlineto{\pgfqpoint{4.335510in}{3.158151in}}%
\pgfpathlineto{\pgfqpoint{4.530142in}{2.866202in}}%
\pgfpathlineto{\pgfqpoint{4.522761in}{2.861282in}}%
\pgfusepath{fill}%
\end{pgfscope}%
\begin{pgfscope}%
\pgfpathrectangle{\pgfqpoint{1.432000in}{0.528000in}}{\pgfqpoint{3.696000in}{3.696000in}} %
\pgfusepath{clip}%
\pgfsetbuttcap%
\pgfsetroundjoin%
\definecolor{currentfill}{rgb}{0.616293,0.852709,0.230052}%
\pgfsetfillcolor{currentfill}%
\pgfsetlinewidth{0.000000pt}%
\definecolor{currentstroke}{rgb}{0.000000,0.000000,0.000000}%
\pgfsetstrokecolor{currentstroke}%
\pgfsetdash{}{0pt}%
\pgfpathmoveto{\pgfqpoint{4.631148in}{2.861282in}}%
\pgfpathlineto{\pgfqpoint{4.436516in}{3.153230in}}%
\pgfpathlineto{\pgfqpoint{4.431596in}{3.144620in}}%
\pgfpathlineto{\pgfqpoint{4.418065in}{3.188903in}}%
\pgfpathlineto{\pgfqpoint{4.453738in}{3.159381in}}%
\pgfpathlineto{\pgfqpoint{4.443897in}{3.158151in}}%
\pgfpathlineto{\pgfqpoint{4.638529in}{2.866202in}}%
\pgfpathlineto{\pgfqpoint{4.631148in}{2.861282in}}%
\pgfusepath{fill}%
\end{pgfscope}%
\begin{pgfscope}%
\pgfpathrectangle{\pgfqpoint{1.432000in}{0.528000in}}{\pgfqpoint{3.696000in}{3.696000in}} %
\pgfusepath{clip}%
\pgfsetbuttcap%
\pgfsetroundjoin%
\definecolor{currentfill}{rgb}{0.275191,0.194905,0.496005}%
\pgfsetfillcolor{currentfill}%
\pgfsetlinewidth{0.000000pt}%
\definecolor{currentstroke}{rgb}{0.000000,0.000000,0.000000}%
\pgfsetstrokecolor{currentstroke}%
\pgfsetdash{}{0pt}%
\pgfpathmoveto{\pgfqpoint{4.630631in}{2.862339in}}%
\pgfpathlineto{\pgfqpoint{4.534867in}{3.149632in}}%
\pgfpathlineto{\pgfqpoint{4.527854in}{3.142620in}}%
\pgfpathlineto{\pgfqpoint{4.526452in}{3.188903in}}%
\pgfpathlineto{\pgfqpoint{4.553100in}{3.151035in}}%
\pgfpathlineto{\pgfqpoint{4.543282in}{3.152437in}}%
\pgfpathlineto{\pgfqpoint{4.639046in}{2.865144in}}%
\pgfpathlineto{\pgfqpoint{4.630631in}{2.862339in}}%
\pgfusepath{fill}%
\end{pgfscope}%
\begin{pgfscope}%
\pgfpathrectangle{\pgfqpoint{1.432000in}{0.528000in}}{\pgfqpoint{3.696000in}{3.696000in}} %
\pgfusepath{clip}%
\pgfsetbuttcap%
\pgfsetroundjoin%
\definecolor{currentfill}{rgb}{0.122312,0.633153,0.530398}%
\pgfsetfillcolor{currentfill}%
\pgfsetlinewidth{0.000000pt}%
\definecolor{currentstroke}{rgb}{0.000000,0.000000,0.000000}%
\pgfsetstrokecolor{currentstroke}%
\pgfsetdash{}{0pt}%
\pgfpathmoveto{\pgfqpoint{4.739535in}{2.861282in}}%
\pgfpathlineto{\pgfqpoint{4.544903in}{3.153230in}}%
\pgfpathlineto{\pgfqpoint{4.539983in}{3.144620in}}%
\pgfpathlineto{\pgfqpoint{4.526452in}{3.188903in}}%
\pgfpathlineto{\pgfqpoint{4.562125in}{3.159381in}}%
\pgfpathlineto{\pgfqpoint{4.552284in}{3.158151in}}%
\pgfpathlineto{\pgfqpoint{4.746916in}{2.866202in}}%
\pgfpathlineto{\pgfqpoint{4.739535in}{2.861282in}}%
\pgfusepath{fill}%
\end{pgfscope}%
\begin{pgfscope}%
\pgfpathrectangle{\pgfqpoint{1.432000in}{0.528000in}}{\pgfqpoint{3.696000in}{3.696000in}} %
\pgfusepath{clip}%
\pgfsetbuttcap%
\pgfsetroundjoin%
\definecolor{currentfill}{rgb}{0.121148,0.592739,0.544641}%
\pgfsetfillcolor{currentfill}%
\pgfsetlinewidth{0.000000pt}%
\definecolor{currentstroke}{rgb}{0.000000,0.000000,0.000000}%
\pgfsetstrokecolor{currentstroke}%
\pgfsetdash{}{0pt}%
\pgfpathmoveto{\pgfqpoint{4.739018in}{2.862339in}}%
\pgfpathlineto{\pgfqpoint{4.643254in}{3.149632in}}%
\pgfpathlineto{\pgfqpoint{4.636241in}{3.142620in}}%
\pgfpathlineto{\pgfqpoint{4.634839in}{3.188903in}}%
\pgfpathlineto{\pgfqpoint{4.661487in}{3.151035in}}%
\pgfpathlineto{\pgfqpoint{4.651669in}{3.152437in}}%
\pgfpathlineto{\pgfqpoint{4.747433in}{2.865144in}}%
\pgfpathlineto{\pgfqpoint{4.739018in}{2.862339in}}%
\pgfusepath{fill}%
\end{pgfscope}%
\begin{pgfscope}%
\pgfpathrectangle{\pgfqpoint{1.432000in}{0.528000in}}{\pgfqpoint{3.696000in}{3.696000in}} %
\pgfusepath{clip}%
\pgfsetbuttcap%
\pgfsetroundjoin%
\definecolor{currentfill}{rgb}{0.120638,0.625828,0.533488}%
\pgfsetfillcolor{currentfill}%
\pgfsetlinewidth{0.000000pt}%
\definecolor{currentstroke}{rgb}{0.000000,0.000000,0.000000}%
\pgfsetstrokecolor{currentstroke}%
\pgfsetdash{}{0pt}%
\pgfpathmoveto{\pgfqpoint{4.847405in}{2.862339in}}%
\pgfpathlineto{\pgfqpoint{4.751641in}{3.149632in}}%
\pgfpathlineto{\pgfqpoint{4.744628in}{3.142620in}}%
\pgfpathlineto{\pgfqpoint{4.743226in}{3.188903in}}%
\pgfpathlineto{\pgfqpoint{4.769874in}{3.151035in}}%
\pgfpathlineto{\pgfqpoint{4.760056in}{3.152437in}}%
\pgfpathlineto{\pgfqpoint{4.855821in}{2.865144in}}%
\pgfpathlineto{\pgfqpoint{4.847405in}{2.862339in}}%
\pgfusepath{fill}%
\end{pgfscope}%
\begin{pgfscope}%
\pgfpathrectangle{\pgfqpoint{1.432000in}{0.528000in}}{\pgfqpoint{3.696000in}{3.696000in}} %
\pgfusepath{clip}%
\pgfsetbuttcap%
\pgfsetroundjoin%
\definecolor{currentfill}{rgb}{0.283072,0.130895,0.449241}%
\pgfsetfillcolor{currentfill}%
\pgfsetlinewidth{0.000000pt}%
\definecolor{currentstroke}{rgb}{0.000000,0.000000,0.000000}%
\pgfsetstrokecolor{currentstroke}%
\pgfsetdash{}{0pt}%
\pgfpathmoveto{\pgfqpoint{4.847178in}{2.863742in}}%
\pgfpathlineto{\pgfqpoint{4.847178in}{3.148986in}}%
\pgfpathlineto{\pgfqpoint{4.838307in}{3.144551in}}%
\pgfpathlineto{\pgfqpoint{4.851613in}{3.188903in}}%
\pgfpathlineto{\pgfqpoint{4.864919in}{3.144551in}}%
\pgfpathlineto{\pgfqpoint{4.856048in}{3.148986in}}%
\pgfpathlineto{\pgfqpoint{4.856048in}{2.863742in}}%
\pgfpathlineto{\pgfqpoint{4.847178in}{2.863742in}}%
\pgfusepath{fill}%
\end{pgfscope}%
\begin{pgfscope}%
\pgfpathrectangle{\pgfqpoint{1.432000in}{0.528000in}}{\pgfqpoint{3.696000in}{3.696000in}} %
\pgfusepath{clip}%
\pgfsetbuttcap%
\pgfsetroundjoin%
\definecolor{currentfill}{rgb}{0.283072,0.130895,0.449241}%
\pgfsetfillcolor{currentfill}%
\pgfsetlinewidth{0.000000pt}%
\definecolor{currentstroke}{rgb}{0.000000,0.000000,0.000000}%
\pgfsetstrokecolor{currentstroke}%
\pgfsetdash{}{0pt}%
\pgfpathmoveto{\pgfqpoint{4.847310in}{2.862666in}}%
\pgfpathlineto{\pgfqpoint{4.748604in}{3.257490in}}%
\pgfpathlineto{\pgfqpoint{4.741074in}{3.251035in}}%
\pgfpathlineto{\pgfqpoint{4.743226in}{3.297290in}}%
\pgfpathlineto{\pgfqpoint{4.766891in}{3.257490in}}%
\pgfpathlineto{\pgfqpoint{4.757210in}{3.259641in}}%
\pgfpathlineto{\pgfqpoint{4.855916in}{2.864818in}}%
\pgfpathlineto{\pgfqpoint{4.847310in}{2.862666in}}%
\pgfusepath{fill}%
\end{pgfscope}%
\begin{pgfscope}%
\pgfpathrectangle{\pgfqpoint{1.432000in}{0.528000in}}{\pgfqpoint{3.696000in}{3.696000in}} %
\pgfusepath{clip}%
\pgfsetbuttcap%
\pgfsetroundjoin%
\definecolor{currentfill}{rgb}{0.269308,0.218818,0.509577}%
\pgfsetfillcolor{currentfill}%
\pgfsetlinewidth{0.000000pt}%
\definecolor{currentstroke}{rgb}{0.000000,0.000000,0.000000}%
\pgfsetstrokecolor{currentstroke}%
\pgfsetdash{}{0pt}%
\pgfpathmoveto{\pgfqpoint{4.955792in}{2.862339in}}%
\pgfpathlineto{\pgfqpoint{4.860028in}{3.149632in}}%
\pgfpathlineto{\pgfqpoint{4.853015in}{3.142620in}}%
\pgfpathlineto{\pgfqpoint{4.851613in}{3.188903in}}%
\pgfpathlineto{\pgfqpoint{4.878261in}{3.151035in}}%
\pgfpathlineto{\pgfqpoint{4.868443in}{3.152437in}}%
\pgfpathlineto{\pgfqpoint{4.964208in}{2.865144in}}%
\pgfpathlineto{\pgfqpoint{4.955792in}{2.862339in}}%
\pgfusepath{fill}%
\end{pgfscope}%
\begin{pgfscope}%
\pgfpathrectangle{\pgfqpoint{1.432000in}{0.528000in}}{\pgfqpoint{3.696000in}{3.696000in}} %
\pgfusepath{clip}%
\pgfsetbuttcap%
\pgfsetroundjoin%
\definecolor{currentfill}{rgb}{0.206756,0.371758,0.553117}%
\pgfsetfillcolor{currentfill}%
\pgfsetlinewidth{0.000000pt}%
\definecolor{currentstroke}{rgb}{0.000000,0.000000,0.000000}%
\pgfsetstrokecolor{currentstroke}%
\pgfsetdash{}{0pt}%
\pgfpathmoveto{\pgfqpoint{4.955565in}{2.863742in}}%
\pgfpathlineto{\pgfqpoint{4.955565in}{3.148986in}}%
\pgfpathlineto{\pgfqpoint{4.946694in}{3.144551in}}%
\pgfpathlineto{\pgfqpoint{4.960000in}{3.188903in}}%
\pgfpathlineto{\pgfqpoint{4.973306in}{3.144551in}}%
\pgfpathlineto{\pgfqpoint{4.964435in}{3.148986in}}%
\pgfpathlineto{\pgfqpoint{4.964435in}{2.863742in}}%
\pgfpathlineto{\pgfqpoint{4.955565in}{2.863742in}}%
\pgfusepath{fill}%
\end{pgfscope}%
\begin{pgfscope}%
\pgfpathrectangle{\pgfqpoint{1.432000in}{0.528000in}}{\pgfqpoint{3.696000in}{3.696000in}} %
\pgfusepath{clip}%
\pgfsetbuttcap%
\pgfsetroundjoin%
\definecolor{currentfill}{rgb}{0.266580,0.228262,0.514349}%
\pgfsetfillcolor{currentfill}%
\pgfsetlinewidth{0.000000pt}%
\definecolor{currentstroke}{rgb}{0.000000,0.000000,0.000000}%
\pgfsetstrokecolor{currentstroke}%
\pgfsetdash{}{0pt}%
\pgfpathmoveto{\pgfqpoint{4.955565in}{2.863742in}}%
\pgfpathlineto{\pgfqpoint{4.955565in}{3.257374in}}%
\pgfpathlineto{\pgfqpoint{4.946694in}{3.252938in}}%
\pgfpathlineto{\pgfqpoint{4.960000in}{3.297290in}}%
\pgfpathlineto{\pgfqpoint{4.973306in}{3.252938in}}%
\pgfpathlineto{\pgfqpoint{4.964435in}{3.257374in}}%
\pgfpathlineto{\pgfqpoint{4.964435in}{2.863742in}}%
\pgfpathlineto{\pgfqpoint{4.955565in}{2.863742in}}%
\pgfusepath{fill}%
\end{pgfscope}%
\begin{pgfscope}%
\pgfpathrectangle{\pgfqpoint{1.432000in}{0.528000in}}{\pgfqpoint{3.696000in}{3.696000in}} %
\pgfusepath{clip}%
\pgfsetbuttcap%
\pgfsetroundjoin%
\definecolor{currentfill}{rgb}{0.280255,0.165693,0.476498}%
\pgfsetfillcolor{currentfill}%
\pgfsetlinewidth{0.000000pt}%
\definecolor{currentstroke}{rgb}{0.000000,0.000000,0.000000}%
\pgfsetstrokecolor{currentstroke}%
\pgfsetdash{}{0pt}%
\pgfpathmoveto{\pgfqpoint{1.604435in}{2.972129in}}%
\pgfpathlineto{\pgfqpoint{1.604435in}{2.903659in}}%
\pgfpathlineto{\pgfqpoint{1.613306in}{2.908094in}}%
\pgfpathlineto{\pgfqpoint{1.600000in}{2.863742in}}%
\pgfpathlineto{\pgfqpoint{1.586694in}{2.908094in}}%
\pgfpathlineto{\pgfqpoint{1.595565in}{2.903659in}}%
\pgfpathlineto{\pgfqpoint{1.595565in}{2.972129in}}%
\pgfpathlineto{\pgfqpoint{1.604435in}{2.972129in}}%
\pgfusepath{fill}%
\end{pgfscope}%
\begin{pgfscope}%
\pgfpathrectangle{\pgfqpoint{1.432000in}{0.528000in}}{\pgfqpoint{3.696000in}{3.696000in}} %
\pgfusepath{clip}%
\pgfsetbuttcap%
\pgfsetroundjoin%
\definecolor{currentfill}{rgb}{0.235526,0.309527,0.542944}%
\pgfsetfillcolor{currentfill}%
\pgfsetlinewidth{0.000000pt}%
\definecolor{currentstroke}{rgb}{0.000000,0.000000,0.000000}%
\pgfsetstrokecolor{currentstroke}%
\pgfsetdash{}{0pt}%
\pgfpathmoveto{\pgfqpoint{1.604435in}{2.972129in}}%
\pgfpathlineto{\pgfqpoint{1.602218in}{2.975970in}}%
\pgfpathlineto{\pgfqpoint{1.597782in}{2.975970in}}%
\pgfpathlineto{\pgfqpoint{1.595565in}{2.972129in}}%
\pgfpathlineto{\pgfqpoint{1.597782in}{2.968288in}}%
\pgfpathlineto{\pgfqpoint{1.602218in}{2.968288in}}%
\pgfpathlineto{\pgfqpoint{1.604435in}{2.972129in}}%
\pgfpathlineto{\pgfqpoint{1.602218in}{2.975970in}}%
\pgfusepath{fill}%
\end{pgfscope}%
\begin{pgfscope}%
\pgfpathrectangle{\pgfqpoint{1.432000in}{0.528000in}}{\pgfqpoint{3.696000in}{3.696000in}} %
\pgfusepath{clip}%
\pgfsetbuttcap%
\pgfsetroundjoin%
\definecolor{currentfill}{rgb}{0.185556,0.418570,0.556753}%
\pgfsetfillcolor{currentfill}%
\pgfsetlinewidth{0.000000pt}%
\definecolor{currentstroke}{rgb}{0.000000,0.000000,0.000000}%
\pgfsetstrokecolor{currentstroke}%
\pgfsetdash{}{0pt}%
\pgfpathmoveto{\pgfqpoint{1.712822in}{2.972129in}}%
\pgfpathlineto{\pgfqpoint{1.710605in}{2.975970in}}%
\pgfpathlineto{\pgfqpoint{1.706169in}{2.975970in}}%
\pgfpathlineto{\pgfqpoint{1.703952in}{2.972129in}}%
\pgfpathlineto{\pgfqpoint{1.706169in}{2.968288in}}%
\pgfpathlineto{\pgfqpoint{1.710605in}{2.968288in}}%
\pgfpathlineto{\pgfqpoint{1.712822in}{2.972129in}}%
\pgfpathlineto{\pgfqpoint{1.710605in}{2.975970in}}%
\pgfusepath{fill}%
\end{pgfscope}%
\begin{pgfscope}%
\pgfpathrectangle{\pgfqpoint{1.432000in}{0.528000in}}{\pgfqpoint{3.696000in}{3.696000in}} %
\pgfusepath{clip}%
\pgfsetbuttcap%
\pgfsetroundjoin%
\definecolor{currentfill}{rgb}{0.277941,0.056324,0.381191}%
\pgfsetfillcolor{currentfill}%
\pgfsetlinewidth{0.000000pt}%
\definecolor{currentstroke}{rgb}{0.000000,0.000000,0.000000}%
\pgfsetstrokecolor{currentstroke}%
\pgfsetdash{}{0pt}%
\pgfpathmoveto{\pgfqpoint{1.708387in}{2.976564in}}%
\pgfpathlineto{\pgfqpoint{1.776857in}{2.976564in}}%
\pgfpathlineto{\pgfqpoint{1.772422in}{2.985435in}}%
\pgfpathlineto{\pgfqpoint{1.816774in}{2.972129in}}%
\pgfpathlineto{\pgfqpoint{1.772422in}{2.958823in}}%
\pgfpathlineto{\pgfqpoint{1.776857in}{2.967694in}}%
\pgfpathlineto{\pgfqpoint{1.708387in}{2.967694in}}%
\pgfpathlineto{\pgfqpoint{1.708387in}{2.976564in}}%
\pgfusepath{fill}%
\end{pgfscope}%
\begin{pgfscope}%
\pgfpathrectangle{\pgfqpoint{1.432000in}{0.528000in}}{\pgfqpoint{3.696000in}{3.696000in}} %
\pgfusepath{clip}%
\pgfsetbuttcap%
\pgfsetroundjoin%
\definecolor{currentfill}{rgb}{0.132444,0.552216,0.553018}%
\pgfsetfillcolor{currentfill}%
\pgfsetlinewidth{0.000000pt}%
\definecolor{currentstroke}{rgb}{0.000000,0.000000,0.000000}%
\pgfsetstrokecolor{currentstroke}%
\pgfsetdash{}{0pt}%
\pgfpathmoveto{\pgfqpoint{1.821209in}{2.972129in}}%
\pgfpathlineto{\pgfqpoint{1.818992in}{2.975970in}}%
\pgfpathlineto{\pgfqpoint{1.814557in}{2.975970in}}%
\pgfpathlineto{\pgfqpoint{1.812339in}{2.972129in}}%
\pgfpathlineto{\pgfqpoint{1.814557in}{2.968288in}}%
\pgfpathlineto{\pgfqpoint{1.818992in}{2.968288in}}%
\pgfpathlineto{\pgfqpoint{1.821209in}{2.972129in}}%
\pgfpathlineto{\pgfqpoint{1.818992in}{2.975970in}}%
\pgfusepath{fill}%
\end{pgfscope}%
\begin{pgfscope}%
\pgfpathrectangle{\pgfqpoint{1.432000in}{0.528000in}}{\pgfqpoint{3.696000in}{3.696000in}} %
\pgfusepath{clip}%
\pgfsetbuttcap%
\pgfsetroundjoin%
\definecolor{currentfill}{rgb}{0.146616,0.673050,0.508936}%
\pgfsetfillcolor{currentfill}%
\pgfsetlinewidth{0.000000pt}%
\definecolor{currentstroke}{rgb}{0.000000,0.000000,0.000000}%
\pgfsetstrokecolor{currentstroke}%
\pgfsetdash{}{0pt}%
\pgfpathmoveto{\pgfqpoint{1.929596in}{2.972129in}}%
\pgfpathlineto{\pgfqpoint{1.927379in}{2.975970in}}%
\pgfpathlineto{\pgfqpoint{1.922944in}{2.975970in}}%
\pgfpathlineto{\pgfqpoint{1.920726in}{2.972129in}}%
\pgfpathlineto{\pgfqpoint{1.922944in}{2.968288in}}%
\pgfpathlineto{\pgfqpoint{1.927379in}{2.968288in}}%
\pgfpathlineto{\pgfqpoint{1.929596in}{2.972129in}}%
\pgfpathlineto{\pgfqpoint{1.927379in}{2.975970in}}%
\pgfusepath{fill}%
\end{pgfscope}%
\begin{pgfscope}%
\pgfpathrectangle{\pgfqpoint{1.432000in}{0.528000in}}{\pgfqpoint{3.696000in}{3.696000in}} %
\pgfusepath{clip}%
\pgfsetbuttcap%
\pgfsetroundjoin%
\definecolor{currentfill}{rgb}{0.120565,0.596422,0.543611}%
\pgfsetfillcolor{currentfill}%
\pgfsetlinewidth{0.000000pt}%
\definecolor{currentstroke}{rgb}{0.000000,0.000000,0.000000}%
\pgfsetstrokecolor{currentstroke}%
\pgfsetdash{}{0pt}%
\pgfpathmoveto{\pgfqpoint{2.037984in}{2.972129in}}%
\pgfpathlineto{\pgfqpoint{2.035766in}{2.975970in}}%
\pgfpathlineto{\pgfqpoint{2.031331in}{2.975970in}}%
\pgfpathlineto{\pgfqpoint{2.029113in}{2.972129in}}%
\pgfpathlineto{\pgfqpoint{2.031331in}{2.968288in}}%
\pgfpathlineto{\pgfqpoint{2.035766in}{2.968288in}}%
\pgfpathlineto{\pgfqpoint{2.037984in}{2.972129in}}%
\pgfpathlineto{\pgfqpoint{2.035766in}{2.975970in}}%
\pgfusepath{fill}%
\end{pgfscope}%
\begin{pgfscope}%
\pgfpathrectangle{\pgfqpoint{1.432000in}{0.528000in}}{\pgfqpoint{3.696000in}{3.696000in}} %
\pgfusepath{clip}%
\pgfsetbuttcap%
\pgfsetroundjoin%
\definecolor{currentfill}{rgb}{0.262138,0.242286,0.520837}%
\pgfsetfillcolor{currentfill}%
\pgfsetlinewidth{0.000000pt}%
\definecolor{currentstroke}{rgb}{0.000000,0.000000,0.000000}%
\pgfsetstrokecolor{currentstroke}%
\pgfsetdash{}{0pt}%
\pgfpathmoveto{\pgfqpoint{2.141935in}{2.967694in}}%
\pgfpathlineto{\pgfqpoint{2.073465in}{2.967694in}}%
\pgfpathlineto{\pgfqpoint{2.077900in}{2.958823in}}%
\pgfpathlineto{\pgfqpoint{2.033548in}{2.972129in}}%
\pgfpathlineto{\pgfqpoint{2.077900in}{2.985435in}}%
\pgfpathlineto{\pgfqpoint{2.073465in}{2.976564in}}%
\pgfpathlineto{\pgfqpoint{2.141935in}{2.976564in}}%
\pgfpathlineto{\pgfqpoint{2.141935in}{2.967694in}}%
\pgfusepath{fill}%
\end{pgfscope}%
\begin{pgfscope}%
\pgfpathrectangle{\pgfqpoint{1.432000in}{0.528000in}}{\pgfqpoint{3.696000in}{3.696000in}} %
\pgfusepath{clip}%
\pgfsetbuttcap%
\pgfsetroundjoin%
\definecolor{currentfill}{rgb}{0.223925,0.334994,0.548053}%
\pgfsetfillcolor{currentfill}%
\pgfsetlinewidth{0.000000pt}%
\definecolor{currentstroke}{rgb}{0.000000,0.000000,0.000000}%
\pgfsetstrokecolor{currentstroke}%
\pgfsetdash{}{0pt}%
\pgfpathmoveto{\pgfqpoint{2.146371in}{2.972129in}}%
\pgfpathlineto{\pgfqpoint{2.144153in}{2.975970in}}%
\pgfpathlineto{\pgfqpoint{2.139718in}{2.975970in}}%
\pgfpathlineto{\pgfqpoint{2.137500in}{2.972129in}}%
\pgfpathlineto{\pgfqpoint{2.139718in}{2.968288in}}%
\pgfpathlineto{\pgfqpoint{2.144153in}{2.968288in}}%
\pgfpathlineto{\pgfqpoint{2.146371in}{2.972129in}}%
\pgfpathlineto{\pgfqpoint{2.144153in}{2.975970in}}%
\pgfusepath{fill}%
\end{pgfscope}%
\begin{pgfscope}%
\pgfpathrectangle{\pgfqpoint{1.432000in}{0.528000in}}{\pgfqpoint{3.696000in}{3.696000in}} %
\pgfusepath{clip}%
\pgfsetbuttcap%
\pgfsetroundjoin%
\definecolor{currentfill}{rgb}{0.136408,0.541173,0.554483}%
\pgfsetfillcolor{currentfill}%
\pgfsetlinewidth{0.000000pt}%
\definecolor{currentstroke}{rgb}{0.000000,0.000000,0.000000}%
\pgfsetstrokecolor{currentstroke}%
\pgfsetdash{}{0pt}%
\pgfpathmoveto{\pgfqpoint{2.250323in}{2.967694in}}%
\pgfpathlineto{\pgfqpoint{2.181852in}{2.967694in}}%
\pgfpathlineto{\pgfqpoint{2.186287in}{2.958823in}}%
\pgfpathlineto{\pgfqpoint{2.141935in}{2.972129in}}%
\pgfpathlineto{\pgfqpoint{2.186287in}{2.985435in}}%
\pgfpathlineto{\pgfqpoint{2.181852in}{2.976564in}}%
\pgfpathlineto{\pgfqpoint{2.250323in}{2.976564in}}%
\pgfpathlineto{\pgfqpoint{2.250323in}{2.967694in}}%
\pgfusepath{fill}%
\end{pgfscope}%
\begin{pgfscope}%
\pgfpathrectangle{\pgfqpoint{1.432000in}{0.528000in}}{\pgfqpoint{3.696000in}{3.696000in}} %
\pgfusepath{clip}%
\pgfsetbuttcap%
\pgfsetroundjoin%
\definecolor{currentfill}{rgb}{0.151918,0.500685,0.557587}%
\pgfsetfillcolor{currentfill}%
\pgfsetlinewidth{0.000000pt}%
\definecolor{currentstroke}{rgb}{0.000000,0.000000,0.000000}%
\pgfsetstrokecolor{currentstroke}%
\pgfsetdash{}{0pt}%
\pgfpathmoveto{\pgfqpoint{2.358710in}{2.967694in}}%
\pgfpathlineto{\pgfqpoint{2.290239in}{2.967694in}}%
\pgfpathlineto{\pgfqpoint{2.294675in}{2.958823in}}%
\pgfpathlineto{\pgfqpoint{2.250323in}{2.972129in}}%
\pgfpathlineto{\pgfqpoint{2.294675in}{2.985435in}}%
\pgfpathlineto{\pgfqpoint{2.290239in}{2.976564in}}%
\pgfpathlineto{\pgfqpoint{2.358710in}{2.976564in}}%
\pgfpathlineto{\pgfqpoint{2.358710in}{2.967694in}}%
\pgfusepath{fill}%
\end{pgfscope}%
\begin{pgfscope}%
\pgfpathrectangle{\pgfqpoint{1.432000in}{0.528000in}}{\pgfqpoint{3.696000in}{3.696000in}} %
\pgfusepath{clip}%
\pgfsetbuttcap%
\pgfsetroundjoin%
\definecolor{currentfill}{rgb}{0.187231,0.414746,0.556547}%
\pgfsetfillcolor{currentfill}%
\pgfsetlinewidth{0.000000pt}%
\definecolor{currentstroke}{rgb}{0.000000,0.000000,0.000000}%
\pgfsetstrokecolor{currentstroke}%
\pgfsetdash{}{0pt}%
\pgfpathmoveto{\pgfqpoint{2.467097in}{2.967694in}}%
\pgfpathlineto{\pgfqpoint{2.398626in}{2.967694in}}%
\pgfpathlineto{\pgfqpoint{2.403062in}{2.958823in}}%
\pgfpathlineto{\pgfqpoint{2.358710in}{2.972129in}}%
\pgfpathlineto{\pgfqpoint{2.403062in}{2.985435in}}%
\pgfpathlineto{\pgfqpoint{2.398626in}{2.976564in}}%
\pgfpathlineto{\pgfqpoint{2.467097in}{2.976564in}}%
\pgfpathlineto{\pgfqpoint{2.467097in}{2.967694in}}%
\pgfusepath{fill}%
\end{pgfscope}%
\begin{pgfscope}%
\pgfpathrectangle{\pgfqpoint{1.432000in}{0.528000in}}{\pgfqpoint{3.696000in}{3.696000in}} %
\pgfusepath{clip}%
\pgfsetbuttcap%
\pgfsetroundjoin%
\definecolor{currentfill}{rgb}{0.151918,0.500685,0.557587}%
\pgfsetfillcolor{currentfill}%
\pgfsetlinewidth{0.000000pt}%
\definecolor{currentstroke}{rgb}{0.000000,0.000000,0.000000}%
\pgfsetstrokecolor{currentstroke}%
\pgfsetdash{}{0pt}%
\pgfpathmoveto{\pgfqpoint{2.575484in}{2.967694in}}%
\pgfpathlineto{\pgfqpoint{2.507014in}{2.967694in}}%
\pgfpathlineto{\pgfqpoint{2.511449in}{2.958823in}}%
\pgfpathlineto{\pgfqpoint{2.467097in}{2.972129in}}%
\pgfpathlineto{\pgfqpoint{2.511449in}{2.985435in}}%
\pgfpathlineto{\pgfqpoint{2.507014in}{2.976564in}}%
\pgfpathlineto{\pgfqpoint{2.575484in}{2.976564in}}%
\pgfpathlineto{\pgfqpoint{2.575484in}{2.967694in}}%
\pgfusepath{fill}%
\end{pgfscope}%
\begin{pgfscope}%
\pgfpathrectangle{\pgfqpoint{1.432000in}{0.528000in}}{\pgfqpoint{3.696000in}{3.696000in}} %
\pgfusepath{clip}%
\pgfsetbuttcap%
\pgfsetroundjoin%
\definecolor{currentfill}{rgb}{0.199430,0.387607,0.554642}%
\pgfsetfillcolor{currentfill}%
\pgfsetlinewidth{0.000000pt}%
\definecolor{currentstroke}{rgb}{0.000000,0.000000,0.000000}%
\pgfsetstrokecolor{currentstroke}%
\pgfsetdash{}{0pt}%
\pgfpathmoveto{\pgfqpoint{2.683871in}{2.967694in}}%
\pgfpathlineto{\pgfqpoint{2.615401in}{2.967694in}}%
\pgfpathlineto{\pgfqpoint{2.619836in}{2.958823in}}%
\pgfpathlineto{\pgfqpoint{2.575484in}{2.972129in}}%
\pgfpathlineto{\pgfqpoint{2.619836in}{2.985435in}}%
\pgfpathlineto{\pgfqpoint{2.615401in}{2.976564in}}%
\pgfpathlineto{\pgfqpoint{2.683871in}{2.976564in}}%
\pgfpathlineto{\pgfqpoint{2.683871in}{2.967694in}}%
\pgfusepath{fill}%
\end{pgfscope}%
\begin{pgfscope}%
\pgfpathrectangle{\pgfqpoint{1.432000in}{0.528000in}}{\pgfqpoint{3.696000in}{3.696000in}} %
\pgfusepath{clip}%
\pgfsetbuttcap%
\pgfsetroundjoin%
\definecolor{currentfill}{rgb}{0.280255,0.165693,0.476498}%
\pgfsetfillcolor{currentfill}%
\pgfsetlinewidth{0.000000pt}%
\definecolor{currentstroke}{rgb}{0.000000,0.000000,0.000000}%
\pgfsetstrokecolor{currentstroke}%
\pgfsetdash{}{0pt}%
\pgfpathmoveto{\pgfqpoint{2.680735in}{2.968993in}}%
\pgfpathlineto{\pgfqpoint{2.600573in}{3.049155in}}%
\pgfpathlineto{\pgfqpoint{2.597437in}{3.039746in}}%
\pgfpathlineto{\pgfqpoint{2.575484in}{3.080516in}}%
\pgfpathlineto{\pgfqpoint{2.616254in}{3.058563in}}%
\pgfpathlineto{\pgfqpoint{2.606845in}{3.055427in}}%
\pgfpathlineto{\pgfqpoint{2.687007in}{2.975265in}}%
\pgfpathlineto{\pgfqpoint{2.680735in}{2.968993in}}%
\pgfusepath{fill}%
\end{pgfscope}%
\begin{pgfscope}%
\pgfpathrectangle{\pgfqpoint{1.432000in}{0.528000in}}{\pgfqpoint{3.696000in}{3.696000in}} %
\pgfusepath{clip}%
\pgfsetbuttcap%
\pgfsetroundjoin%
\definecolor{currentfill}{rgb}{0.275191,0.194905,0.496005}%
\pgfsetfillcolor{currentfill}%
\pgfsetlinewidth{0.000000pt}%
\definecolor{currentstroke}{rgb}{0.000000,0.000000,0.000000}%
\pgfsetstrokecolor{currentstroke}%
\pgfsetdash{}{0pt}%
\pgfpathmoveto{\pgfqpoint{2.789122in}{2.968993in}}%
\pgfpathlineto{\pgfqpoint{2.708960in}{3.049155in}}%
\pgfpathlineto{\pgfqpoint{2.705824in}{3.039746in}}%
\pgfpathlineto{\pgfqpoint{2.683871in}{3.080516in}}%
\pgfpathlineto{\pgfqpoint{2.724641in}{3.058563in}}%
\pgfpathlineto{\pgfqpoint{2.715233in}{3.055427in}}%
\pgfpathlineto{\pgfqpoint{2.795394in}{2.975265in}}%
\pgfpathlineto{\pgfqpoint{2.789122in}{2.968993in}}%
\pgfusepath{fill}%
\end{pgfscope}%
\begin{pgfscope}%
\pgfpathrectangle{\pgfqpoint{1.432000in}{0.528000in}}{\pgfqpoint{3.696000in}{3.696000in}} %
\pgfusepath{clip}%
\pgfsetbuttcap%
\pgfsetroundjoin%
\definecolor{currentfill}{rgb}{0.278826,0.175490,0.483397}%
\pgfsetfillcolor{currentfill}%
\pgfsetlinewidth{0.000000pt}%
\definecolor{currentstroke}{rgb}{0.000000,0.000000,0.000000}%
\pgfsetstrokecolor{currentstroke}%
\pgfsetdash{}{0pt}%
\pgfpathmoveto{\pgfqpoint{2.787823in}{2.972129in}}%
\pgfpathlineto{\pgfqpoint{2.787823in}{3.040599in}}%
\pgfpathlineto{\pgfqpoint{2.778952in}{3.036164in}}%
\pgfpathlineto{\pgfqpoint{2.792258in}{3.080516in}}%
\pgfpathlineto{\pgfqpoint{2.805564in}{3.036164in}}%
\pgfpathlineto{\pgfqpoint{2.796693in}{3.040599in}}%
\pgfpathlineto{\pgfqpoint{2.796693in}{2.972129in}}%
\pgfpathlineto{\pgfqpoint{2.787823in}{2.972129in}}%
\pgfusepath{fill}%
\end{pgfscope}%
\begin{pgfscope}%
\pgfpathrectangle{\pgfqpoint{1.432000in}{0.528000in}}{\pgfqpoint{3.696000in}{3.696000in}} %
\pgfusepath{clip}%
\pgfsetbuttcap%
\pgfsetroundjoin%
\definecolor{currentfill}{rgb}{0.268510,0.009605,0.335427}%
\pgfsetfillcolor{currentfill}%
\pgfsetlinewidth{0.000000pt}%
\definecolor{currentstroke}{rgb}{0.000000,0.000000,0.000000}%
\pgfsetstrokecolor{currentstroke}%
\pgfsetdash{}{0pt}%
\pgfpathmoveto{\pgfqpoint{2.897509in}{2.968993in}}%
\pgfpathlineto{\pgfqpoint{2.817347in}{3.049155in}}%
\pgfpathlineto{\pgfqpoint{2.814211in}{3.039746in}}%
\pgfpathlineto{\pgfqpoint{2.792258in}{3.080516in}}%
\pgfpathlineto{\pgfqpoint{2.833028in}{3.058563in}}%
\pgfpathlineto{\pgfqpoint{2.823620in}{3.055427in}}%
\pgfpathlineto{\pgfqpoint{2.903781in}{2.975265in}}%
\pgfpathlineto{\pgfqpoint{2.897509in}{2.968993in}}%
\pgfusepath{fill}%
\end{pgfscope}%
\begin{pgfscope}%
\pgfpathrectangle{\pgfqpoint{1.432000in}{0.528000in}}{\pgfqpoint{3.696000in}{3.696000in}} %
\pgfusepath{clip}%
\pgfsetbuttcap%
\pgfsetroundjoin%
\definecolor{currentfill}{rgb}{0.187231,0.414746,0.556547}%
\pgfsetfillcolor{currentfill}%
\pgfsetlinewidth{0.000000pt}%
\definecolor{currentstroke}{rgb}{0.000000,0.000000,0.000000}%
\pgfsetstrokecolor{currentstroke}%
\pgfsetdash{}{0pt}%
\pgfpathmoveto{\pgfqpoint{2.896210in}{2.972129in}}%
\pgfpathlineto{\pgfqpoint{2.896210in}{3.040599in}}%
\pgfpathlineto{\pgfqpoint{2.887340in}{3.036164in}}%
\pgfpathlineto{\pgfqpoint{2.900645in}{3.080516in}}%
\pgfpathlineto{\pgfqpoint{2.913951in}{3.036164in}}%
\pgfpathlineto{\pgfqpoint{2.905080in}{3.040599in}}%
\pgfpathlineto{\pgfqpoint{2.905080in}{2.972129in}}%
\pgfpathlineto{\pgfqpoint{2.896210in}{2.972129in}}%
\pgfusepath{fill}%
\end{pgfscope}%
\begin{pgfscope}%
\pgfpathrectangle{\pgfqpoint{1.432000in}{0.528000in}}{\pgfqpoint{3.696000in}{3.696000in}} %
\pgfusepath{clip}%
\pgfsetbuttcap%
\pgfsetroundjoin%
\definecolor{currentfill}{rgb}{0.203063,0.379716,0.553925}%
\pgfsetfillcolor{currentfill}%
\pgfsetlinewidth{0.000000pt}%
\definecolor{currentstroke}{rgb}{0.000000,0.000000,0.000000}%
\pgfsetstrokecolor{currentstroke}%
\pgfsetdash{}{0pt}%
\pgfpathmoveto{\pgfqpoint{3.004597in}{2.972129in}}%
\pgfpathlineto{\pgfqpoint{3.004597in}{3.040599in}}%
\pgfpathlineto{\pgfqpoint{2.995727in}{3.036164in}}%
\pgfpathlineto{\pgfqpoint{3.009032in}{3.080516in}}%
\pgfpathlineto{\pgfqpoint{3.022338in}{3.036164in}}%
\pgfpathlineto{\pgfqpoint{3.013467in}{3.040599in}}%
\pgfpathlineto{\pgfqpoint{3.013467in}{2.972129in}}%
\pgfpathlineto{\pgfqpoint{3.004597in}{2.972129in}}%
\pgfusepath{fill}%
\end{pgfscope}%
\begin{pgfscope}%
\pgfpathrectangle{\pgfqpoint{1.432000in}{0.528000in}}{\pgfqpoint{3.696000in}{3.696000in}} %
\pgfusepath{clip}%
\pgfsetbuttcap%
\pgfsetroundjoin%
\definecolor{currentfill}{rgb}{0.277018,0.050344,0.375715}%
\pgfsetfillcolor{currentfill}%
\pgfsetlinewidth{0.000000pt}%
\definecolor{currentstroke}{rgb}{0.000000,0.000000,0.000000}%
\pgfsetstrokecolor{currentstroke}%
\pgfsetdash{}{0pt}%
\pgfpathmoveto{\pgfqpoint{3.004597in}{2.972129in}}%
\pgfpathlineto{\pgfqpoint{3.004597in}{3.148986in}}%
\pgfpathlineto{\pgfqpoint{2.995727in}{3.144551in}}%
\pgfpathlineto{\pgfqpoint{3.009032in}{3.188903in}}%
\pgfpathlineto{\pgfqpoint{3.022338in}{3.144551in}}%
\pgfpathlineto{\pgfqpoint{3.013467in}{3.148986in}}%
\pgfpathlineto{\pgfqpoint{3.013467in}{2.972129in}}%
\pgfpathlineto{\pgfqpoint{3.004597in}{2.972129in}}%
\pgfusepath{fill}%
\end{pgfscope}%
\begin{pgfscope}%
\pgfpathrectangle{\pgfqpoint{1.432000in}{0.528000in}}{\pgfqpoint{3.696000in}{3.696000in}} %
\pgfusepath{clip}%
\pgfsetbuttcap%
\pgfsetroundjoin%
\definecolor{currentfill}{rgb}{0.280267,0.073417,0.397163}%
\pgfsetfillcolor{currentfill}%
\pgfsetlinewidth{0.000000pt}%
\definecolor{currentstroke}{rgb}{0.000000,0.000000,0.000000}%
\pgfsetstrokecolor{currentstroke}%
\pgfsetdash{}{0pt}%
\pgfpathmoveto{\pgfqpoint{3.112984in}{2.972129in}}%
\pgfpathlineto{\pgfqpoint{3.112984in}{3.040599in}}%
\pgfpathlineto{\pgfqpoint{3.104114in}{3.036164in}}%
\pgfpathlineto{\pgfqpoint{3.117419in}{3.080516in}}%
\pgfpathlineto{\pgfqpoint{3.130725in}{3.036164in}}%
\pgfpathlineto{\pgfqpoint{3.121855in}{3.040599in}}%
\pgfpathlineto{\pgfqpoint{3.121855in}{2.972129in}}%
\pgfpathlineto{\pgfqpoint{3.112984in}{2.972129in}}%
\pgfusepath{fill}%
\end{pgfscope}%
\begin{pgfscope}%
\pgfpathrectangle{\pgfqpoint{1.432000in}{0.528000in}}{\pgfqpoint{3.696000in}{3.696000in}} %
\pgfusepath{clip}%
\pgfsetbuttcap%
\pgfsetroundjoin%
\definecolor{currentfill}{rgb}{0.248629,0.278775,0.534556}%
\pgfsetfillcolor{currentfill}%
\pgfsetlinewidth{0.000000pt}%
\definecolor{currentstroke}{rgb}{0.000000,0.000000,0.000000}%
\pgfsetstrokecolor{currentstroke}%
\pgfsetdash{}{0pt}%
\pgfpathmoveto{\pgfqpoint{3.112984in}{2.972129in}}%
\pgfpathlineto{\pgfqpoint{3.112984in}{3.148986in}}%
\pgfpathlineto{\pgfqpoint{3.104114in}{3.144551in}}%
\pgfpathlineto{\pgfqpoint{3.117419in}{3.188903in}}%
\pgfpathlineto{\pgfqpoint{3.130725in}{3.144551in}}%
\pgfpathlineto{\pgfqpoint{3.121855in}{3.148986in}}%
\pgfpathlineto{\pgfqpoint{3.121855in}{2.972129in}}%
\pgfpathlineto{\pgfqpoint{3.112984in}{2.972129in}}%
\pgfusepath{fill}%
\end{pgfscope}%
\begin{pgfscope}%
\pgfpathrectangle{\pgfqpoint{1.432000in}{0.528000in}}{\pgfqpoint{3.696000in}{3.696000in}} %
\pgfusepath{clip}%
\pgfsetbuttcap%
\pgfsetroundjoin%
\definecolor{currentfill}{rgb}{0.278791,0.062145,0.386592}%
\pgfsetfillcolor{currentfill}%
\pgfsetlinewidth{0.000000pt}%
\definecolor{currentstroke}{rgb}{0.000000,0.000000,0.000000}%
\pgfsetstrokecolor{currentstroke}%
\pgfsetdash{}{0pt}%
\pgfpathmoveto{\pgfqpoint{3.222670in}{2.968993in}}%
\pgfpathlineto{\pgfqpoint{3.142509in}{3.049155in}}%
\pgfpathlineto{\pgfqpoint{3.139372in}{3.039746in}}%
\pgfpathlineto{\pgfqpoint{3.117419in}{3.080516in}}%
\pgfpathlineto{\pgfqpoint{3.158189in}{3.058563in}}%
\pgfpathlineto{\pgfqpoint{3.148781in}{3.055427in}}%
\pgfpathlineto{\pgfqpoint{3.228943in}{2.975265in}}%
\pgfpathlineto{\pgfqpoint{3.222670in}{2.968993in}}%
\pgfusepath{fill}%
\end{pgfscope}%
\begin{pgfscope}%
\pgfpathrectangle{\pgfqpoint{1.432000in}{0.528000in}}{\pgfqpoint{3.696000in}{3.696000in}} %
\pgfusepath{clip}%
\pgfsetbuttcap%
\pgfsetroundjoin%
\definecolor{currentfill}{rgb}{0.283229,0.120777,0.440584}%
\pgfsetfillcolor{currentfill}%
\pgfsetlinewidth{0.000000pt}%
\definecolor{currentstroke}{rgb}{0.000000,0.000000,0.000000}%
\pgfsetstrokecolor{currentstroke}%
\pgfsetdash{}{0pt}%
\pgfpathmoveto{\pgfqpoint{3.221839in}{2.970146in}}%
\pgfpathlineto{\pgfqpoint{3.131304in}{3.151217in}}%
\pgfpathlineto{\pgfqpoint{3.125353in}{3.143283in}}%
\pgfpathlineto{\pgfqpoint{3.117419in}{3.188903in}}%
\pgfpathlineto{\pgfqpoint{3.149155in}{3.155184in}}%
\pgfpathlineto{\pgfqpoint{3.139238in}{3.155184in}}%
\pgfpathlineto{\pgfqpoint{3.229773in}{2.974113in}}%
\pgfpathlineto{\pgfqpoint{3.221839in}{2.970146in}}%
\pgfusepath{fill}%
\end{pgfscope}%
\begin{pgfscope}%
\pgfpathrectangle{\pgfqpoint{1.432000in}{0.528000in}}{\pgfqpoint{3.696000in}{3.696000in}} %
\pgfusepath{clip}%
\pgfsetbuttcap%
\pgfsetroundjoin%
\definecolor{currentfill}{rgb}{0.169646,0.456262,0.558030}%
\pgfsetfillcolor{currentfill}%
\pgfsetlinewidth{0.000000pt}%
\definecolor{currentstroke}{rgb}{0.000000,0.000000,0.000000}%
\pgfsetstrokecolor{currentstroke}%
\pgfsetdash{}{0pt}%
\pgfpathmoveto{\pgfqpoint{3.330227in}{2.970146in}}%
\pgfpathlineto{\pgfqpoint{3.239691in}{3.151217in}}%
\pgfpathlineto{\pgfqpoint{3.233740in}{3.143283in}}%
\pgfpathlineto{\pgfqpoint{3.225806in}{3.188903in}}%
\pgfpathlineto{\pgfqpoint{3.257542in}{3.155184in}}%
\pgfpathlineto{\pgfqpoint{3.247625in}{3.155184in}}%
\pgfpathlineto{\pgfqpoint{3.338161in}{2.974113in}}%
\pgfpathlineto{\pgfqpoint{3.330227in}{2.970146in}}%
\pgfusepath{fill}%
\end{pgfscope}%
\begin{pgfscope}%
\pgfpathrectangle{\pgfqpoint{1.432000in}{0.528000in}}{\pgfqpoint{3.696000in}{3.696000in}} %
\pgfusepath{clip}%
\pgfsetbuttcap%
\pgfsetroundjoin%
\definecolor{currentfill}{rgb}{0.151918,0.500685,0.557587}%
\pgfsetfillcolor{currentfill}%
\pgfsetlinewidth{0.000000pt}%
\definecolor{currentstroke}{rgb}{0.000000,0.000000,0.000000}%
\pgfsetstrokecolor{currentstroke}%
\pgfsetdash{}{0pt}%
\pgfpathmoveto{\pgfqpoint{3.439444in}{2.968993in}}%
\pgfpathlineto{\pgfqpoint{3.250896in}{3.157542in}}%
\pgfpathlineto{\pgfqpoint{3.247760in}{3.148133in}}%
\pgfpathlineto{\pgfqpoint{3.225806in}{3.188903in}}%
\pgfpathlineto{\pgfqpoint{3.266577in}{3.166950in}}%
\pgfpathlineto{\pgfqpoint{3.257168in}{3.163814in}}%
\pgfpathlineto{\pgfqpoint{3.445717in}{2.975265in}}%
\pgfpathlineto{\pgfqpoint{3.439444in}{2.968993in}}%
\pgfusepath{fill}%
\end{pgfscope}%
\begin{pgfscope}%
\pgfpathrectangle{\pgfqpoint{1.432000in}{0.528000in}}{\pgfqpoint{3.696000in}{3.696000in}} %
\pgfusepath{clip}%
\pgfsetbuttcap%
\pgfsetroundjoin%
\definecolor{currentfill}{rgb}{0.120638,0.625828,0.533488}%
\pgfsetfillcolor{currentfill}%
\pgfsetlinewidth{0.000000pt}%
\definecolor{currentstroke}{rgb}{0.000000,0.000000,0.000000}%
\pgfsetstrokecolor{currentstroke}%
\pgfsetdash{}{0pt}%
\pgfpathmoveto{\pgfqpoint{3.547832in}{2.968993in}}%
\pgfpathlineto{\pgfqpoint{3.359283in}{3.157542in}}%
\pgfpathlineto{\pgfqpoint{3.356147in}{3.148133in}}%
\pgfpathlineto{\pgfqpoint{3.334194in}{3.188903in}}%
\pgfpathlineto{\pgfqpoint{3.374964in}{3.166950in}}%
\pgfpathlineto{\pgfqpoint{3.365555in}{3.163814in}}%
\pgfpathlineto{\pgfqpoint{3.554104in}{2.975265in}}%
\pgfpathlineto{\pgfqpoint{3.547832in}{2.968993in}}%
\pgfusepath{fill}%
\end{pgfscope}%
\begin{pgfscope}%
\pgfpathrectangle{\pgfqpoint{1.432000in}{0.528000in}}{\pgfqpoint{3.696000in}{3.696000in}} %
\pgfusepath{clip}%
\pgfsetbuttcap%
\pgfsetroundjoin%
\definecolor{currentfill}{rgb}{0.190631,0.407061,0.556089}%
\pgfsetfillcolor{currentfill}%
\pgfsetlinewidth{0.000000pt}%
\definecolor{currentstroke}{rgb}{0.000000,0.000000,0.000000}%
\pgfsetstrokecolor{currentstroke}%
\pgfsetdash{}{0pt}%
\pgfpathmoveto{\pgfqpoint{3.656895in}{2.968439in}}%
\pgfpathlineto{\pgfqpoint{3.364946in}{3.163071in}}%
\pgfpathlineto{\pgfqpoint{3.363716in}{3.153230in}}%
\pgfpathlineto{\pgfqpoint{3.334194in}{3.188903in}}%
\pgfpathlineto{\pgfqpoint{3.378477in}{3.175372in}}%
\pgfpathlineto{\pgfqpoint{3.369867in}{3.170452in}}%
\pgfpathlineto{\pgfqpoint{3.661815in}{2.975819in}}%
\pgfpathlineto{\pgfqpoint{3.656895in}{2.968439in}}%
\pgfusepath{fill}%
\end{pgfscope}%
\begin{pgfscope}%
\pgfpathrectangle{\pgfqpoint{1.432000in}{0.528000in}}{\pgfqpoint{3.696000in}{3.696000in}} %
\pgfusepath{clip}%
\pgfsetbuttcap%
\pgfsetroundjoin%
\definecolor{currentfill}{rgb}{0.282290,0.145912,0.461510}%
\pgfsetfillcolor{currentfill}%
\pgfsetlinewidth{0.000000pt}%
\definecolor{currentstroke}{rgb}{0.000000,0.000000,0.000000}%
\pgfsetstrokecolor{currentstroke}%
\pgfsetdash{}{0pt}%
\pgfpathmoveto{\pgfqpoint{3.656219in}{2.968993in}}%
\pgfpathlineto{\pgfqpoint{3.467670in}{3.157542in}}%
\pgfpathlineto{\pgfqpoint{3.464534in}{3.148133in}}%
\pgfpathlineto{\pgfqpoint{3.442581in}{3.188903in}}%
\pgfpathlineto{\pgfqpoint{3.483351in}{3.166950in}}%
\pgfpathlineto{\pgfqpoint{3.473942in}{3.163814in}}%
\pgfpathlineto{\pgfqpoint{3.662491in}{2.975265in}}%
\pgfpathlineto{\pgfqpoint{3.656219in}{2.968993in}}%
\pgfusepath{fill}%
\end{pgfscope}%
\begin{pgfscope}%
\pgfpathrectangle{\pgfqpoint{1.432000in}{0.528000in}}{\pgfqpoint{3.696000in}{3.696000in}} %
\pgfusepath{clip}%
\pgfsetbuttcap%
\pgfsetroundjoin%
\definecolor{currentfill}{rgb}{0.143343,0.522773,0.556295}%
\pgfsetfillcolor{currentfill}%
\pgfsetlinewidth{0.000000pt}%
\definecolor{currentstroke}{rgb}{0.000000,0.000000,0.000000}%
\pgfsetstrokecolor{currentstroke}%
\pgfsetdash{}{0pt}%
\pgfpathmoveto{\pgfqpoint{3.765282in}{2.968439in}}%
\pgfpathlineto{\pgfqpoint{3.473333in}{3.163071in}}%
\pgfpathlineto{\pgfqpoint{3.472103in}{3.153230in}}%
\pgfpathlineto{\pgfqpoint{3.442581in}{3.188903in}}%
\pgfpathlineto{\pgfqpoint{3.486864in}{3.175372in}}%
\pgfpathlineto{\pgfqpoint{3.478254in}{3.170452in}}%
\pgfpathlineto{\pgfqpoint{3.770202in}{2.975819in}}%
\pgfpathlineto{\pgfqpoint{3.765282in}{2.968439in}}%
\pgfusepath{fill}%
\end{pgfscope}%
\begin{pgfscope}%
\pgfpathrectangle{\pgfqpoint{1.432000in}{0.528000in}}{\pgfqpoint{3.696000in}{3.696000in}} %
\pgfusepath{clip}%
\pgfsetbuttcap%
\pgfsetroundjoin%
\definecolor{currentfill}{rgb}{0.272594,0.025563,0.353093}%
\pgfsetfillcolor{currentfill}%
\pgfsetlinewidth{0.000000pt}%
\definecolor{currentstroke}{rgb}{0.000000,0.000000,0.000000}%
\pgfsetstrokecolor{currentstroke}%
\pgfsetdash{}{0pt}%
\pgfpathmoveto{\pgfqpoint{3.764606in}{2.968993in}}%
\pgfpathlineto{\pgfqpoint{3.467670in}{3.265929in}}%
\pgfpathlineto{\pgfqpoint{3.464534in}{3.256520in}}%
\pgfpathlineto{\pgfqpoint{3.442581in}{3.297290in}}%
\pgfpathlineto{\pgfqpoint{3.483351in}{3.275337in}}%
\pgfpathlineto{\pgfqpoint{3.473942in}{3.272201in}}%
\pgfpathlineto{\pgfqpoint{3.770878in}{2.975265in}}%
\pgfpathlineto{\pgfqpoint{3.764606in}{2.968993in}}%
\pgfusepath{fill}%
\end{pgfscope}%
\begin{pgfscope}%
\pgfpathrectangle{\pgfqpoint{1.432000in}{0.528000in}}{\pgfqpoint{3.696000in}{3.696000in}} %
\pgfusepath{clip}%
\pgfsetbuttcap%
\pgfsetroundjoin%
\definecolor{currentfill}{rgb}{0.239374,0.735588,0.455688}%
\pgfsetfillcolor{currentfill}%
\pgfsetlinewidth{0.000000pt}%
\definecolor{currentstroke}{rgb}{0.000000,0.000000,0.000000}%
\pgfsetstrokecolor{currentstroke}%
\pgfsetdash{}{0pt}%
\pgfpathmoveto{\pgfqpoint{3.872993in}{2.968993in}}%
\pgfpathlineto{\pgfqpoint{3.576057in}{3.265929in}}%
\pgfpathlineto{\pgfqpoint{3.572921in}{3.256520in}}%
\pgfpathlineto{\pgfqpoint{3.550968in}{3.297290in}}%
\pgfpathlineto{\pgfqpoint{3.591738in}{3.275337in}}%
\pgfpathlineto{\pgfqpoint{3.582329in}{3.272201in}}%
\pgfpathlineto{\pgfqpoint{3.879265in}{2.975265in}}%
\pgfpathlineto{\pgfqpoint{3.872993in}{2.968993in}}%
\pgfusepath{fill}%
\end{pgfscope}%
\begin{pgfscope}%
\pgfpathrectangle{\pgfqpoint{1.432000in}{0.528000in}}{\pgfqpoint{3.696000in}{3.696000in}} %
\pgfusepath{clip}%
\pgfsetbuttcap%
\pgfsetroundjoin%
\definecolor{currentfill}{rgb}{0.276194,0.190074,0.493001}%
\pgfsetfillcolor{currentfill}%
\pgfsetlinewidth{0.000000pt}%
\definecolor{currentstroke}{rgb}{0.000000,0.000000,0.000000}%
\pgfsetstrokecolor{currentstroke}%
\pgfsetdash{}{0pt}%
\pgfpathmoveto{\pgfqpoint{3.981855in}{2.968581in}}%
\pgfpathlineto{\pgfqpoint{3.580240in}{3.269792in}}%
\pgfpathlineto{\pgfqpoint{3.578466in}{3.260035in}}%
\pgfpathlineto{\pgfqpoint{3.550968in}{3.297290in}}%
\pgfpathlineto{\pgfqpoint{3.594433in}{3.281324in}}%
\pgfpathlineto{\pgfqpoint{3.585562in}{3.276888in}}%
\pgfpathlineto{\pgfqpoint{3.987177in}{2.975677in}}%
\pgfpathlineto{\pgfqpoint{3.981855in}{2.968581in}}%
\pgfusepath{fill}%
\end{pgfscope}%
\begin{pgfscope}%
\pgfpathrectangle{\pgfqpoint{1.432000in}{0.528000in}}{\pgfqpoint{3.696000in}{3.696000in}} %
\pgfusepath{clip}%
\pgfsetbuttcap%
\pgfsetroundjoin%
\definecolor{currentfill}{rgb}{0.129933,0.559582,0.551864}%
\pgfsetfillcolor{currentfill}%
\pgfsetlinewidth{0.000000pt}%
\definecolor{currentstroke}{rgb}{0.000000,0.000000,0.000000}%
\pgfsetstrokecolor{currentstroke}%
\pgfsetdash{}{0pt}%
\pgfpathmoveto{\pgfqpoint{3.981380in}{2.968993in}}%
\pgfpathlineto{\pgfqpoint{3.684444in}{3.265929in}}%
\pgfpathlineto{\pgfqpoint{3.681308in}{3.256520in}}%
\pgfpathlineto{\pgfqpoint{3.659355in}{3.297290in}}%
\pgfpathlineto{\pgfqpoint{3.700125in}{3.275337in}}%
\pgfpathlineto{\pgfqpoint{3.690716in}{3.272201in}}%
\pgfpathlineto{\pgfqpoint{3.987652in}{2.975265in}}%
\pgfpathlineto{\pgfqpoint{3.981380in}{2.968993in}}%
\pgfusepath{fill}%
\end{pgfscope}%
\begin{pgfscope}%
\pgfpathrectangle{\pgfqpoint{1.432000in}{0.528000in}}{\pgfqpoint{3.696000in}{3.696000in}} %
\pgfusepath{clip}%
\pgfsetbuttcap%
\pgfsetroundjoin%
\definecolor{currentfill}{rgb}{0.229739,0.322361,0.545706}%
\pgfsetfillcolor{currentfill}%
\pgfsetlinewidth{0.000000pt}%
\definecolor{currentstroke}{rgb}{0.000000,0.000000,0.000000}%
\pgfsetstrokecolor{currentstroke}%
\pgfsetdash{}{0pt}%
\pgfpathmoveto{\pgfqpoint{4.090242in}{2.968581in}}%
\pgfpathlineto{\pgfqpoint{3.688627in}{3.269792in}}%
\pgfpathlineto{\pgfqpoint{3.686853in}{3.260035in}}%
\pgfpathlineto{\pgfqpoint{3.659355in}{3.297290in}}%
\pgfpathlineto{\pgfqpoint{3.702820in}{3.281324in}}%
\pgfpathlineto{\pgfqpoint{3.693949in}{3.276888in}}%
\pgfpathlineto{\pgfqpoint{4.095564in}{2.975677in}}%
\pgfpathlineto{\pgfqpoint{4.090242in}{2.968581in}}%
\pgfusepath{fill}%
\end{pgfscope}%
\begin{pgfscope}%
\pgfpathrectangle{\pgfqpoint{1.432000in}{0.528000in}}{\pgfqpoint{3.696000in}{3.696000in}} %
\pgfusepath{clip}%
\pgfsetbuttcap%
\pgfsetroundjoin%
\definecolor{currentfill}{rgb}{0.139147,0.533812,0.555298}%
\pgfsetfillcolor{currentfill}%
\pgfsetlinewidth{0.000000pt}%
\definecolor{currentstroke}{rgb}{0.000000,0.000000,0.000000}%
\pgfsetstrokecolor{currentstroke}%
\pgfsetdash{}{0pt}%
\pgfpathmoveto{\pgfqpoint{4.089767in}{2.968993in}}%
\pgfpathlineto{\pgfqpoint{3.792831in}{3.265929in}}%
\pgfpathlineto{\pgfqpoint{3.789695in}{3.256520in}}%
\pgfpathlineto{\pgfqpoint{3.767742in}{3.297290in}}%
\pgfpathlineto{\pgfqpoint{3.808512in}{3.275337in}}%
\pgfpathlineto{\pgfqpoint{3.799104in}{3.272201in}}%
\pgfpathlineto{\pgfqpoint{4.096039in}{2.975265in}}%
\pgfpathlineto{\pgfqpoint{4.089767in}{2.968993in}}%
\pgfusepath{fill}%
\end{pgfscope}%
\begin{pgfscope}%
\pgfpathrectangle{\pgfqpoint{1.432000in}{0.528000in}}{\pgfqpoint{3.696000in}{3.696000in}} %
\pgfusepath{clip}%
\pgfsetbuttcap%
\pgfsetroundjoin%
\definecolor{currentfill}{rgb}{0.248629,0.278775,0.534556}%
\pgfsetfillcolor{currentfill}%
\pgfsetlinewidth{0.000000pt}%
\definecolor{currentstroke}{rgb}{0.000000,0.000000,0.000000}%
\pgfsetstrokecolor{currentstroke}%
\pgfsetdash{}{0pt}%
\pgfpathmoveto{\pgfqpoint{4.198629in}{2.968581in}}%
\pgfpathlineto{\pgfqpoint{3.797014in}{3.269792in}}%
\pgfpathlineto{\pgfqpoint{3.795240in}{3.260035in}}%
\pgfpathlineto{\pgfqpoint{3.767742in}{3.297290in}}%
\pgfpathlineto{\pgfqpoint{3.811207in}{3.281324in}}%
\pgfpathlineto{\pgfqpoint{3.802336in}{3.276888in}}%
\pgfpathlineto{\pgfqpoint{4.203951in}{2.975677in}}%
\pgfpathlineto{\pgfqpoint{4.198629in}{2.968581in}}%
\pgfusepath{fill}%
\end{pgfscope}%
\begin{pgfscope}%
\pgfpathrectangle{\pgfqpoint{1.432000in}{0.528000in}}{\pgfqpoint{3.696000in}{3.696000in}} %
\pgfusepath{clip}%
\pgfsetbuttcap%
\pgfsetroundjoin%
\definecolor{currentfill}{rgb}{0.150148,0.676631,0.506589}%
\pgfsetfillcolor{currentfill}%
\pgfsetlinewidth{0.000000pt}%
\definecolor{currentstroke}{rgb}{0.000000,0.000000,0.000000}%
\pgfsetstrokecolor{currentstroke}%
\pgfsetdash{}{0pt}%
\pgfpathmoveto{\pgfqpoint{4.198154in}{2.968993in}}%
\pgfpathlineto{\pgfqpoint{3.901218in}{3.265929in}}%
\pgfpathlineto{\pgfqpoint{3.898082in}{3.256520in}}%
\pgfpathlineto{\pgfqpoint{3.876129in}{3.297290in}}%
\pgfpathlineto{\pgfqpoint{3.916899in}{3.275337in}}%
\pgfpathlineto{\pgfqpoint{3.907491in}{3.272201in}}%
\pgfpathlineto{\pgfqpoint{4.204426in}{2.975265in}}%
\pgfpathlineto{\pgfqpoint{4.198154in}{2.968993in}}%
\pgfusepath{fill}%
\end{pgfscope}%
\begin{pgfscope}%
\pgfpathrectangle{\pgfqpoint{1.432000in}{0.528000in}}{\pgfqpoint{3.696000in}{3.696000in}} %
\pgfusepath{clip}%
\pgfsetbuttcap%
\pgfsetroundjoin%
\definecolor{currentfill}{rgb}{0.311925,0.767822,0.415586}%
\pgfsetfillcolor{currentfill}%
\pgfsetlinewidth{0.000000pt}%
\definecolor{currentstroke}{rgb}{0.000000,0.000000,0.000000}%
\pgfsetstrokecolor{currentstroke}%
\pgfsetdash{}{0pt}%
\pgfpathmoveto{\pgfqpoint{4.306541in}{2.968993in}}%
\pgfpathlineto{\pgfqpoint{4.009605in}{3.265929in}}%
\pgfpathlineto{\pgfqpoint{4.006469in}{3.256520in}}%
\pgfpathlineto{\pgfqpoint{3.984516in}{3.297290in}}%
\pgfpathlineto{\pgfqpoint{4.025286in}{3.275337in}}%
\pgfpathlineto{\pgfqpoint{4.015878in}{3.272201in}}%
\pgfpathlineto{\pgfqpoint{4.312814in}{2.975265in}}%
\pgfpathlineto{\pgfqpoint{4.306541in}{2.968993in}}%
\pgfusepath{fill}%
\end{pgfscope}%
\begin{pgfscope}%
\pgfpathrectangle{\pgfqpoint{1.432000in}{0.528000in}}{\pgfqpoint{3.696000in}{3.696000in}} %
\pgfusepath{clip}%
\pgfsetbuttcap%
\pgfsetroundjoin%
\definecolor{currentfill}{rgb}{0.280267,0.073417,0.397163}%
\pgfsetfillcolor{currentfill}%
\pgfsetlinewidth{0.000000pt}%
\definecolor{currentstroke}{rgb}{0.000000,0.000000,0.000000}%
\pgfsetstrokecolor{currentstroke}%
\pgfsetdash{}{0pt}%
\pgfpathmoveto{\pgfqpoint{4.305987in}{2.969669in}}%
\pgfpathlineto{\pgfqpoint{4.111355in}{3.261617in}}%
\pgfpathlineto{\pgfqpoint{4.106434in}{3.253007in}}%
\pgfpathlineto{\pgfqpoint{4.092903in}{3.297290in}}%
\pgfpathlineto{\pgfqpoint{4.128576in}{3.267768in}}%
\pgfpathlineto{\pgfqpoint{4.118735in}{3.266538in}}%
\pgfpathlineto{\pgfqpoint{4.313368in}{2.974589in}}%
\pgfpathlineto{\pgfqpoint{4.305987in}{2.969669in}}%
\pgfusepath{fill}%
\end{pgfscope}%
\begin{pgfscope}%
\pgfpathrectangle{\pgfqpoint{1.432000in}{0.528000in}}{\pgfqpoint{3.696000in}{3.696000in}} %
\pgfusepath{clip}%
\pgfsetbuttcap%
\pgfsetroundjoin%
\definecolor{currentfill}{rgb}{0.266941,0.748751,0.440573}%
\pgfsetfillcolor{currentfill}%
\pgfsetlinewidth{0.000000pt}%
\definecolor{currentstroke}{rgb}{0.000000,0.000000,0.000000}%
\pgfsetstrokecolor{currentstroke}%
\pgfsetdash{}{0pt}%
\pgfpathmoveto{\pgfqpoint{4.414928in}{2.968993in}}%
\pgfpathlineto{\pgfqpoint{4.117993in}{3.265929in}}%
\pgfpathlineto{\pgfqpoint{4.114856in}{3.256520in}}%
\pgfpathlineto{\pgfqpoint{4.092903in}{3.297290in}}%
\pgfpathlineto{\pgfqpoint{4.133673in}{3.275337in}}%
\pgfpathlineto{\pgfqpoint{4.124265in}{3.272201in}}%
\pgfpathlineto{\pgfqpoint{4.421201in}{2.975265in}}%
\pgfpathlineto{\pgfqpoint{4.414928in}{2.968993in}}%
\pgfusepath{fill}%
\end{pgfscope}%
\begin{pgfscope}%
\pgfpathrectangle{\pgfqpoint{1.432000in}{0.528000in}}{\pgfqpoint{3.696000in}{3.696000in}} %
\pgfusepath{clip}%
\pgfsetbuttcap%
\pgfsetroundjoin%
\definecolor{currentfill}{rgb}{0.227802,0.326594,0.546532}%
\pgfsetfillcolor{currentfill}%
\pgfsetlinewidth{0.000000pt}%
\definecolor{currentstroke}{rgb}{0.000000,0.000000,0.000000}%
\pgfsetstrokecolor{currentstroke}%
\pgfsetdash{}{0pt}%
\pgfpathmoveto{\pgfqpoint{4.414374in}{2.969669in}}%
\pgfpathlineto{\pgfqpoint{4.219742in}{3.261617in}}%
\pgfpathlineto{\pgfqpoint{4.214821in}{3.253007in}}%
\pgfpathlineto{\pgfqpoint{4.201290in}{3.297290in}}%
\pgfpathlineto{\pgfqpoint{4.236963in}{3.267768in}}%
\pgfpathlineto{\pgfqpoint{4.227122in}{3.266538in}}%
\pgfpathlineto{\pgfqpoint{4.421755in}{2.974589in}}%
\pgfpathlineto{\pgfqpoint{4.414374in}{2.969669in}}%
\pgfusepath{fill}%
\end{pgfscope}%
\begin{pgfscope}%
\pgfpathrectangle{\pgfqpoint{1.432000in}{0.528000in}}{\pgfqpoint{3.696000in}{3.696000in}} %
\pgfusepath{clip}%
\pgfsetbuttcap%
\pgfsetroundjoin%
\definecolor{currentfill}{rgb}{0.203063,0.379716,0.553925}%
\pgfsetfillcolor{currentfill}%
\pgfsetlinewidth{0.000000pt}%
\definecolor{currentstroke}{rgb}{0.000000,0.000000,0.000000}%
\pgfsetstrokecolor{currentstroke}%
\pgfsetdash{}{0pt}%
\pgfpathmoveto{\pgfqpoint{4.523315in}{2.968993in}}%
\pgfpathlineto{\pgfqpoint{4.226380in}{3.265929in}}%
\pgfpathlineto{\pgfqpoint{4.223243in}{3.256520in}}%
\pgfpathlineto{\pgfqpoint{4.201290in}{3.297290in}}%
\pgfpathlineto{\pgfqpoint{4.242060in}{3.275337in}}%
\pgfpathlineto{\pgfqpoint{4.232652in}{3.272201in}}%
\pgfpathlineto{\pgfqpoint{4.529588in}{2.975265in}}%
\pgfpathlineto{\pgfqpoint{4.523315in}{2.968993in}}%
\pgfusepath{fill}%
\end{pgfscope}%
\begin{pgfscope}%
\pgfpathrectangle{\pgfqpoint{1.432000in}{0.528000in}}{\pgfqpoint{3.696000in}{3.696000in}} %
\pgfusepath{clip}%
\pgfsetbuttcap%
\pgfsetroundjoin%
\definecolor{currentfill}{rgb}{0.449368,0.813768,0.335384}%
\pgfsetfillcolor{currentfill}%
\pgfsetlinewidth{0.000000pt}%
\definecolor{currentstroke}{rgb}{0.000000,0.000000,0.000000}%
\pgfsetstrokecolor{currentstroke}%
\pgfsetdash{}{0pt}%
\pgfpathmoveto{\pgfqpoint{4.522761in}{2.969669in}}%
\pgfpathlineto{\pgfqpoint{4.328129in}{3.261617in}}%
\pgfpathlineto{\pgfqpoint{4.323209in}{3.253007in}}%
\pgfpathlineto{\pgfqpoint{4.309677in}{3.297290in}}%
\pgfpathlineto{\pgfqpoint{4.345350in}{3.267768in}}%
\pgfpathlineto{\pgfqpoint{4.335510in}{3.266538in}}%
\pgfpathlineto{\pgfqpoint{4.530142in}{2.974589in}}%
\pgfpathlineto{\pgfqpoint{4.522761in}{2.969669in}}%
\pgfusepath{fill}%
\end{pgfscope}%
\begin{pgfscope}%
\pgfpathrectangle{\pgfqpoint{1.432000in}{0.528000in}}{\pgfqpoint{3.696000in}{3.696000in}} %
\pgfusepath{clip}%
\pgfsetbuttcap%
\pgfsetroundjoin%
\definecolor{currentfill}{rgb}{0.730889,0.871916,0.156029}%
\pgfsetfillcolor{currentfill}%
\pgfsetlinewidth{0.000000pt}%
\definecolor{currentstroke}{rgb}{0.000000,0.000000,0.000000}%
\pgfsetstrokecolor{currentstroke}%
\pgfsetdash{}{0pt}%
\pgfpathmoveto{\pgfqpoint{4.631148in}{2.969669in}}%
\pgfpathlineto{\pgfqpoint{4.436516in}{3.261617in}}%
\pgfpathlineto{\pgfqpoint{4.431596in}{3.253007in}}%
\pgfpathlineto{\pgfqpoint{4.418065in}{3.297290in}}%
\pgfpathlineto{\pgfqpoint{4.453738in}{3.267768in}}%
\pgfpathlineto{\pgfqpoint{4.443897in}{3.266538in}}%
\pgfpathlineto{\pgfqpoint{4.638529in}{2.974589in}}%
\pgfpathlineto{\pgfqpoint{4.631148in}{2.969669in}}%
\pgfusepath{fill}%
\end{pgfscope}%
\begin{pgfscope}%
\pgfpathrectangle{\pgfqpoint{1.432000in}{0.528000in}}{\pgfqpoint{3.696000in}{3.696000in}} %
\pgfusepath{clip}%
\pgfsetbuttcap%
\pgfsetroundjoin%
\definecolor{currentfill}{rgb}{0.199430,0.387607,0.554642}%
\pgfsetfillcolor{currentfill}%
\pgfsetlinewidth{0.000000pt}%
\definecolor{currentstroke}{rgb}{0.000000,0.000000,0.000000}%
\pgfsetstrokecolor{currentstroke}%
\pgfsetdash{}{0pt}%
\pgfpathmoveto{\pgfqpoint{4.630631in}{2.970726in}}%
\pgfpathlineto{\pgfqpoint{4.534867in}{3.258019in}}%
\pgfpathlineto{\pgfqpoint{4.527854in}{3.251007in}}%
\pgfpathlineto{\pgfqpoint{4.526452in}{3.297290in}}%
\pgfpathlineto{\pgfqpoint{4.553100in}{3.259422in}}%
\pgfpathlineto{\pgfqpoint{4.543282in}{3.260824in}}%
\pgfpathlineto{\pgfqpoint{4.639046in}{2.973532in}}%
\pgfpathlineto{\pgfqpoint{4.630631in}{2.970726in}}%
\pgfusepath{fill}%
\end{pgfscope}%
\begin{pgfscope}%
\pgfpathrectangle{\pgfqpoint{1.432000in}{0.528000in}}{\pgfqpoint{3.696000in}{3.696000in}} %
\pgfusepath{clip}%
\pgfsetbuttcap%
\pgfsetroundjoin%
\definecolor{currentfill}{rgb}{0.151918,0.500685,0.557587}%
\pgfsetfillcolor{currentfill}%
\pgfsetlinewidth{0.000000pt}%
\definecolor{currentstroke}{rgb}{0.000000,0.000000,0.000000}%
\pgfsetstrokecolor{currentstroke}%
\pgfsetdash{}{0pt}%
\pgfpathmoveto{\pgfqpoint{4.739535in}{2.969669in}}%
\pgfpathlineto{\pgfqpoint{4.544903in}{3.261617in}}%
\pgfpathlineto{\pgfqpoint{4.539983in}{3.253007in}}%
\pgfpathlineto{\pgfqpoint{4.526452in}{3.297290in}}%
\pgfpathlineto{\pgfqpoint{4.562125in}{3.267768in}}%
\pgfpathlineto{\pgfqpoint{4.552284in}{3.266538in}}%
\pgfpathlineto{\pgfqpoint{4.746916in}{2.974589in}}%
\pgfpathlineto{\pgfqpoint{4.739535in}{2.969669in}}%
\pgfusepath{fill}%
\end{pgfscope}%
\begin{pgfscope}%
\pgfpathrectangle{\pgfqpoint{1.432000in}{0.528000in}}{\pgfqpoint{3.696000in}{3.696000in}} %
\pgfusepath{clip}%
\pgfsetbuttcap%
\pgfsetroundjoin%
\definecolor{currentfill}{rgb}{0.386433,0.794644,0.372886}%
\pgfsetfillcolor{currentfill}%
\pgfsetlinewidth{0.000000pt}%
\definecolor{currentstroke}{rgb}{0.000000,0.000000,0.000000}%
\pgfsetstrokecolor{currentstroke}%
\pgfsetdash{}{0pt}%
\pgfpathmoveto{\pgfqpoint{4.739018in}{2.970726in}}%
\pgfpathlineto{\pgfqpoint{4.643254in}{3.258019in}}%
\pgfpathlineto{\pgfqpoint{4.636241in}{3.251007in}}%
\pgfpathlineto{\pgfqpoint{4.634839in}{3.297290in}}%
\pgfpathlineto{\pgfqpoint{4.661487in}{3.259422in}}%
\pgfpathlineto{\pgfqpoint{4.651669in}{3.260824in}}%
\pgfpathlineto{\pgfqpoint{4.747433in}{2.973532in}}%
\pgfpathlineto{\pgfqpoint{4.739018in}{2.970726in}}%
\pgfusepath{fill}%
\end{pgfscope}%
\begin{pgfscope}%
\pgfpathrectangle{\pgfqpoint{1.432000in}{0.528000in}}{\pgfqpoint{3.696000in}{3.696000in}} %
\pgfusepath{clip}%
\pgfsetbuttcap%
\pgfsetroundjoin%
\definecolor{currentfill}{rgb}{0.147607,0.511733,0.557049}%
\pgfsetfillcolor{currentfill}%
\pgfsetlinewidth{0.000000pt}%
\definecolor{currentstroke}{rgb}{0.000000,0.000000,0.000000}%
\pgfsetstrokecolor{currentstroke}%
\pgfsetdash{}{0pt}%
\pgfpathmoveto{\pgfqpoint{4.847405in}{2.970726in}}%
\pgfpathlineto{\pgfqpoint{4.751641in}{3.258019in}}%
\pgfpathlineto{\pgfqpoint{4.744628in}{3.251007in}}%
\pgfpathlineto{\pgfqpoint{4.743226in}{3.297290in}}%
\pgfpathlineto{\pgfqpoint{4.769874in}{3.259422in}}%
\pgfpathlineto{\pgfqpoint{4.760056in}{3.260824in}}%
\pgfpathlineto{\pgfqpoint{4.855821in}{2.973532in}}%
\pgfpathlineto{\pgfqpoint{4.847405in}{2.970726in}}%
\pgfusepath{fill}%
\end{pgfscope}%
\begin{pgfscope}%
\pgfpathrectangle{\pgfqpoint{1.432000in}{0.528000in}}{\pgfqpoint{3.696000in}{3.696000in}} %
\pgfusepath{clip}%
\pgfsetbuttcap%
\pgfsetroundjoin%
\definecolor{currentfill}{rgb}{0.282884,0.135920,0.453427}%
\pgfsetfillcolor{currentfill}%
\pgfsetlinewidth{0.000000pt}%
\definecolor{currentstroke}{rgb}{0.000000,0.000000,0.000000}%
\pgfsetstrokecolor{currentstroke}%
\pgfsetdash{}{0pt}%
\pgfpathmoveto{\pgfqpoint{4.847178in}{2.972129in}}%
\pgfpathlineto{\pgfqpoint{4.847178in}{3.257374in}}%
\pgfpathlineto{\pgfqpoint{4.838307in}{3.252938in}}%
\pgfpathlineto{\pgfqpoint{4.851613in}{3.297290in}}%
\pgfpathlineto{\pgfqpoint{4.864919in}{3.252938in}}%
\pgfpathlineto{\pgfqpoint{4.856048in}{3.257374in}}%
\pgfpathlineto{\pgfqpoint{4.856048in}{2.972129in}}%
\pgfpathlineto{\pgfqpoint{4.847178in}{2.972129in}}%
\pgfusepath{fill}%
\end{pgfscope}%
\begin{pgfscope}%
\pgfpathrectangle{\pgfqpoint{1.432000in}{0.528000in}}{\pgfqpoint{3.696000in}{3.696000in}} %
\pgfusepath{clip}%
\pgfsetbuttcap%
\pgfsetroundjoin%
\definecolor{currentfill}{rgb}{0.220057,0.343307,0.549413}%
\pgfsetfillcolor{currentfill}%
\pgfsetlinewidth{0.000000pt}%
\definecolor{currentstroke}{rgb}{0.000000,0.000000,0.000000}%
\pgfsetstrokecolor{currentstroke}%
\pgfsetdash{}{0pt}%
\pgfpathmoveto{\pgfqpoint{4.847310in}{2.971053in}}%
\pgfpathlineto{\pgfqpoint{4.748604in}{3.365877in}}%
\pgfpathlineto{\pgfqpoint{4.741074in}{3.359423in}}%
\pgfpathlineto{\pgfqpoint{4.743226in}{3.405677in}}%
\pgfpathlineto{\pgfqpoint{4.766891in}{3.365877in}}%
\pgfpathlineto{\pgfqpoint{4.757210in}{3.368028in}}%
\pgfpathlineto{\pgfqpoint{4.855916in}{2.973205in}}%
\pgfpathlineto{\pgfqpoint{4.847310in}{2.971053in}}%
\pgfusepath{fill}%
\end{pgfscope}%
\begin{pgfscope}%
\pgfpathrectangle{\pgfqpoint{1.432000in}{0.528000in}}{\pgfqpoint{3.696000in}{3.696000in}} %
\pgfusepath{clip}%
\pgfsetbuttcap%
\pgfsetroundjoin%
\definecolor{currentfill}{rgb}{0.283197,0.115680,0.436115}%
\pgfsetfillcolor{currentfill}%
\pgfsetlinewidth{0.000000pt}%
\definecolor{currentstroke}{rgb}{0.000000,0.000000,0.000000}%
\pgfsetstrokecolor{currentstroke}%
\pgfsetdash{}{0pt}%
\pgfpathmoveto{\pgfqpoint{4.847178in}{2.972129in}}%
\pgfpathlineto{\pgfqpoint{4.847178in}{3.365761in}}%
\pgfpathlineto{\pgfqpoint{4.838307in}{3.361325in}}%
\pgfpathlineto{\pgfqpoint{4.851613in}{3.405677in}}%
\pgfpathlineto{\pgfqpoint{4.864919in}{3.361325in}}%
\pgfpathlineto{\pgfqpoint{4.856048in}{3.365761in}}%
\pgfpathlineto{\pgfqpoint{4.856048in}{2.972129in}}%
\pgfpathlineto{\pgfqpoint{4.847178in}{2.972129in}}%
\pgfusepath{fill}%
\end{pgfscope}%
\begin{pgfscope}%
\pgfpathrectangle{\pgfqpoint{1.432000in}{0.528000in}}{\pgfqpoint{3.696000in}{3.696000in}} %
\pgfusepath{clip}%
\pgfsetbuttcap%
\pgfsetroundjoin%
\definecolor{currentfill}{rgb}{0.281924,0.089666,0.412415}%
\pgfsetfillcolor{currentfill}%
\pgfsetlinewidth{0.000000pt}%
\definecolor{currentstroke}{rgb}{0.000000,0.000000,0.000000}%
\pgfsetstrokecolor{currentstroke}%
\pgfsetdash{}{0pt}%
\pgfpathmoveto{\pgfqpoint{4.955792in}{2.970726in}}%
\pgfpathlineto{\pgfqpoint{4.860028in}{3.258019in}}%
\pgfpathlineto{\pgfqpoint{4.853015in}{3.251007in}}%
\pgfpathlineto{\pgfqpoint{4.851613in}{3.297290in}}%
\pgfpathlineto{\pgfqpoint{4.878261in}{3.259422in}}%
\pgfpathlineto{\pgfqpoint{4.868443in}{3.260824in}}%
\pgfpathlineto{\pgfqpoint{4.964208in}{2.973532in}}%
\pgfpathlineto{\pgfqpoint{4.955792in}{2.970726in}}%
\pgfusepath{fill}%
\end{pgfscope}%
\begin{pgfscope}%
\pgfpathrectangle{\pgfqpoint{1.432000in}{0.528000in}}{\pgfqpoint{3.696000in}{3.696000in}} %
\pgfusepath{clip}%
\pgfsetbuttcap%
\pgfsetroundjoin%
\definecolor{currentfill}{rgb}{0.194100,0.399323,0.555565}%
\pgfsetfillcolor{currentfill}%
\pgfsetlinewidth{0.000000pt}%
\definecolor{currentstroke}{rgb}{0.000000,0.000000,0.000000}%
\pgfsetstrokecolor{currentstroke}%
\pgfsetdash{}{0pt}%
\pgfpathmoveto{\pgfqpoint{4.955565in}{2.972129in}}%
\pgfpathlineto{\pgfqpoint{4.955565in}{3.257374in}}%
\pgfpathlineto{\pgfqpoint{4.946694in}{3.252938in}}%
\pgfpathlineto{\pgfqpoint{4.960000in}{3.297290in}}%
\pgfpathlineto{\pgfqpoint{4.973306in}{3.252938in}}%
\pgfpathlineto{\pgfqpoint{4.964435in}{3.257374in}}%
\pgfpathlineto{\pgfqpoint{4.964435in}{2.972129in}}%
\pgfpathlineto{\pgfqpoint{4.955565in}{2.972129in}}%
\pgfusepath{fill}%
\end{pgfscope}%
\begin{pgfscope}%
\pgfpathrectangle{\pgfqpoint{1.432000in}{0.528000in}}{\pgfqpoint{3.696000in}{3.696000in}} %
\pgfusepath{clip}%
\pgfsetbuttcap%
\pgfsetroundjoin%
\definecolor{currentfill}{rgb}{0.282327,0.094955,0.417331}%
\pgfsetfillcolor{currentfill}%
\pgfsetlinewidth{0.000000pt}%
\definecolor{currentstroke}{rgb}{0.000000,0.000000,0.000000}%
\pgfsetstrokecolor{currentstroke}%
\pgfsetdash{}{0pt}%
\pgfpathmoveto{\pgfqpoint{4.955697in}{2.971053in}}%
\pgfpathlineto{\pgfqpoint{4.856991in}{3.365877in}}%
\pgfpathlineto{\pgfqpoint{4.849462in}{3.359423in}}%
\pgfpathlineto{\pgfqpoint{4.851613in}{3.405677in}}%
\pgfpathlineto{\pgfqpoint{4.875278in}{3.365877in}}%
\pgfpathlineto{\pgfqpoint{4.865597in}{3.368028in}}%
\pgfpathlineto{\pgfqpoint{4.964303in}{2.973205in}}%
\pgfpathlineto{\pgfqpoint{4.955697in}{2.971053in}}%
\pgfusepath{fill}%
\end{pgfscope}%
\begin{pgfscope}%
\pgfpathrectangle{\pgfqpoint{1.432000in}{0.528000in}}{\pgfqpoint{3.696000in}{3.696000in}} %
\pgfusepath{clip}%
\pgfsetbuttcap%
\pgfsetroundjoin%
\definecolor{currentfill}{rgb}{0.195860,0.395433,0.555276}%
\pgfsetfillcolor{currentfill}%
\pgfsetlinewidth{0.000000pt}%
\definecolor{currentstroke}{rgb}{0.000000,0.000000,0.000000}%
\pgfsetstrokecolor{currentstroke}%
\pgfsetdash{}{0pt}%
\pgfpathmoveto{\pgfqpoint{4.955565in}{2.972129in}}%
\pgfpathlineto{\pgfqpoint{4.955565in}{3.365761in}}%
\pgfpathlineto{\pgfqpoint{4.946694in}{3.361325in}}%
\pgfpathlineto{\pgfqpoint{4.960000in}{3.405677in}}%
\pgfpathlineto{\pgfqpoint{4.973306in}{3.361325in}}%
\pgfpathlineto{\pgfqpoint{4.964435in}{3.365761in}}%
\pgfpathlineto{\pgfqpoint{4.964435in}{2.972129in}}%
\pgfpathlineto{\pgfqpoint{4.955565in}{2.972129in}}%
\pgfusepath{fill}%
\end{pgfscope}%
\begin{pgfscope}%
\pgfpathrectangle{\pgfqpoint{1.432000in}{0.528000in}}{\pgfqpoint{3.696000in}{3.696000in}} %
\pgfusepath{clip}%
\pgfsetbuttcap%
\pgfsetroundjoin%
\definecolor{currentfill}{rgb}{0.255645,0.260703,0.528312}%
\pgfsetfillcolor{currentfill}%
\pgfsetlinewidth{0.000000pt}%
\definecolor{currentstroke}{rgb}{0.000000,0.000000,0.000000}%
\pgfsetstrokecolor{currentstroke}%
\pgfsetdash{}{0pt}%
\pgfpathmoveto{\pgfqpoint{1.604435in}{3.080516in}}%
\pgfpathlineto{\pgfqpoint{1.604435in}{3.012046in}}%
\pgfpathlineto{\pgfqpoint{1.613306in}{3.016481in}}%
\pgfpathlineto{\pgfqpoint{1.600000in}{2.972129in}}%
\pgfpathlineto{\pgfqpoint{1.586694in}{3.016481in}}%
\pgfpathlineto{\pgfqpoint{1.595565in}{3.012046in}}%
\pgfpathlineto{\pgfqpoint{1.595565in}{3.080516in}}%
\pgfpathlineto{\pgfqpoint{1.604435in}{3.080516in}}%
\pgfusepath{fill}%
\end{pgfscope}%
\begin{pgfscope}%
\pgfpathrectangle{\pgfqpoint{1.432000in}{0.528000in}}{\pgfqpoint{3.696000in}{3.696000in}} %
\pgfusepath{clip}%
\pgfsetbuttcap%
\pgfsetroundjoin%
\definecolor{currentfill}{rgb}{0.280894,0.078907,0.402329}%
\pgfsetfillcolor{currentfill}%
\pgfsetlinewidth{0.000000pt}%
\definecolor{currentstroke}{rgb}{0.000000,0.000000,0.000000}%
\pgfsetstrokecolor{currentstroke}%
\pgfsetdash{}{0pt}%
\pgfpathmoveto{\pgfqpoint{1.604435in}{3.080516in}}%
\pgfpathlineto{\pgfqpoint{1.602218in}{3.084357in}}%
\pgfpathlineto{\pgfqpoint{1.597782in}{3.084357in}}%
\pgfpathlineto{\pgfqpoint{1.595565in}{3.080516in}}%
\pgfpathlineto{\pgfqpoint{1.597782in}{3.076675in}}%
\pgfpathlineto{\pgfqpoint{1.602218in}{3.076675in}}%
\pgfpathlineto{\pgfqpoint{1.604435in}{3.080516in}}%
\pgfpathlineto{\pgfqpoint{1.602218in}{3.084357in}}%
\pgfusepath{fill}%
\end{pgfscope}%
\begin{pgfscope}%
\pgfpathrectangle{\pgfqpoint{1.432000in}{0.528000in}}{\pgfqpoint{3.696000in}{3.696000in}} %
\pgfusepath{clip}%
\pgfsetbuttcap%
\pgfsetroundjoin%
\definecolor{currentfill}{rgb}{0.277018,0.050344,0.375715}%
\pgfsetfillcolor{currentfill}%
\pgfsetlinewidth{0.000000pt}%
\definecolor{currentstroke}{rgb}{0.000000,0.000000,0.000000}%
\pgfsetstrokecolor{currentstroke}%
\pgfsetdash{}{0pt}%
\pgfpathmoveto{\pgfqpoint{1.712822in}{3.080516in}}%
\pgfpathlineto{\pgfqpoint{1.712822in}{3.012046in}}%
\pgfpathlineto{\pgfqpoint{1.721693in}{3.016481in}}%
\pgfpathlineto{\pgfqpoint{1.708387in}{2.972129in}}%
\pgfpathlineto{\pgfqpoint{1.695081in}{3.016481in}}%
\pgfpathlineto{\pgfqpoint{1.703952in}{3.012046in}}%
\pgfpathlineto{\pgfqpoint{1.703952in}{3.080516in}}%
\pgfpathlineto{\pgfqpoint{1.712822in}{3.080516in}}%
\pgfusepath{fill}%
\end{pgfscope}%
\begin{pgfscope}%
\pgfpathrectangle{\pgfqpoint{1.432000in}{0.528000in}}{\pgfqpoint{3.696000in}{3.696000in}} %
\pgfusepath{clip}%
\pgfsetbuttcap%
\pgfsetroundjoin%
\definecolor{currentfill}{rgb}{0.281412,0.155834,0.469201}%
\pgfsetfillcolor{currentfill}%
\pgfsetlinewidth{0.000000pt}%
\definecolor{currentstroke}{rgb}{0.000000,0.000000,0.000000}%
\pgfsetstrokecolor{currentstroke}%
\pgfsetdash{}{0pt}%
\pgfpathmoveto{\pgfqpoint{1.712822in}{3.080516in}}%
\pgfpathlineto{\pgfqpoint{1.710605in}{3.084357in}}%
\pgfpathlineto{\pgfqpoint{1.706169in}{3.084357in}}%
\pgfpathlineto{\pgfqpoint{1.703952in}{3.080516in}}%
\pgfpathlineto{\pgfqpoint{1.706169in}{3.076675in}}%
\pgfpathlineto{\pgfqpoint{1.710605in}{3.076675in}}%
\pgfpathlineto{\pgfqpoint{1.712822in}{3.080516in}}%
\pgfpathlineto{\pgfqpoint{1.710605in}{3.084357in}}%
\pgfusepath{fill}%
\end{pgfscope}%
\begin{pgfscope}%
\pgfpathrectangle{\pgfqpoint{1.432000in}{0.528000in}}{\pgfqpoint{3.696000in}{3.696000in}} %
\pgfusepath{clip}%
\pgfsetbuttcap%
\pgfsetroundjoin%
\definecolor{currentfill}{rgb}{0.267004,0.004874,0.329415}%
\pgfsetfillcolor{currentfill}%
\pgfsetlinewidth{0.000000pt}%
\definecolor{currentstroke}{rgb}{0.000000,0.000000,0.000000}%
\pgfsetstrokecolor{currentstroke}%
\pgfsetdash{}{0pt}%
\pgfpathmoveto{\pgfqpoint{1.708387in}{3.084951in}}%
\pgfpathlineto{\pgfqpoint{1.776857in}{3.084951in}}%
\pgfpathlineto{\pgfqpoint{1.772422in}{3.093822in}}%
\pgfpathlineto{\pgfqpoint{1.816774in}{3.080516in}}%
\pgfpathlineto{\pgfqpoint{1.772422in}{3.067211in}}%
\pgfpathlineto{\pgfqpoint{1.776857in}{3.076081in}}%
\pgfpathlineto{\pgfqpoint{1.708387in}{3.076081in}}%
\pgfpathlineto{\pgfqpoint{1.708387in}{3.084951in}}%
\pgfusepath{fill}%
\end{pgfscope}%
\begin{pgfscope}%
\pgfpathrectangle{\pgfqpoint{1.432000in}{0.528000in}}{\pgfqpoint{3.696000in}{3.696000in}} %
\pgfusepath{clip}%
\pgfsetbuttcap%
\pgfsetroundjoin%
\definecolor{currentfill}{rgb}{0.257322,0.256130,0.526563}%
\pgfsetfillcolor{currentfill}%
\pgfsetlinewidth{0.000000pt}%
\definecolor{currentstroke}{rgb}{0.000000,0.000000,0.000000}%
\pgfsetstrokecolor{currentstroke}%
\pgfsetdash{}{0pt}%
\pgfpathmoveto{\pgfqpoint{1.821209in}{3.080516in}}%
\pgfpathlineto{\pgfqpoint{1.818992in}{3.084357in}}%
\pgfpathlineto{\pgfqpoint{1.814557in}{3.084357in}}%
\pgfpathlineto{\pgfqpoint{1.812339in}{3.080516in}}%
\pgfpathlineto{\pgfqpoint{1.814557in}{3.076675in}}%
\pgfpathlineto{\pgfqpoint{1.818992in}{3.076675in}}%
\pgfpathlineto{\pgfqpoint{1.821209in}{3.080516in}}%
\pgfpathlineto{\pgfqpoint{1.818992in}{3.084357in}}%
\pgfusepath{fill}%
\end{pgfscope}%
\begin{pgfscope}%
\pgfpathrectangle{\pgfqpoint{1.432000in}{0.528000in}}{\pgfqpoint{3.696000in}{3.696000in}} %
\pgfusepath{clip}%
\pgfsetbuttcap%
\pgfsetroundjoin%
\definecolor{currentfill}{rgb}{0.268510,0.009605,0.335427}%
\pgfsetfillcolor{currentfill}%
\pgfsetlinewidth{0.000000pt}%
\definecolor{currentstroke}{rgb}{0.000000,0.000000,0.000000}%
\pgfsetstrokecolor{currentstroke}%
\pgfsetdash{}{0pt}%
\pgfpathmoveto{\pgfqpoint{1.929596in}{3.080516in}}%
\pgfpathlineto{\pgfqpoint{1.929596in}{3.012046in}}%
\pgfpathlineto{\pgfqpoint{1.938467in}{3.016481in}}%
\pgfpathlineto{\pgfqpoint{1.925161in}{2.972129in}}%
\pgfpathlineto{\pgfqpoint{1.911856in}{3.016481in}}%
\pgfpathlineto{\pgfqpoint{1.920726in}{3.012046in}}%
\pgfpathlineto{\pgfqpoint{1.920726in}{3.080516in}}%
\pgfpathlineto{\pgfqpoint{1.929596in}{3.080516in}}%
\pgfusepath{fill}%
\end{pgfscope}%
\begin{pgfscope}%
\pgfpathrectangle{\pgfqpoint{1.432000in}{0.528000in}}{\pgfqpoint{3.696000in}{3.696000in}} %
\pgfusepath{clip}%
\pgfsetbuttcap%
\pgfsetroundjoin%
\definecolor{currentfill}{rgb}{0.177423,0.437527,0.557565}%
\pgfsetfillcolor{currentfill}%
\pgfsetlinewidth{0.000000pt}%
\definecolor{currentstroke}{rgb}{0.000000,0.000000,0.000000}%
\pgfsetstrokecolor{currentstroke}%
\pgfsetdash{}{0pt}%
\pgfpathmoveto{\pgfqpoint{1.929596in}{3.080516in}}%
\pgfpathlineto{\pgfqpoint{1.927379in}{3.084357in}}%
\pgfpathlineto{\pgfqpoint{1.922944in}{3.084357in}}%
\pgfpathlineto{\pgfqpoint{1.920726in}{3.080516in}}%
\pgfpathlineto{\pgfqpoint{1.922944in}{3.076675in}}%
\pgfpathlineto{\pgfqpoint{1.927379in}{3.076675in}}%
\pgfpathlineto{\pgfqpoint{1.929596in}{3.080516in}}%
\pgfpathlineto{\pgfqpoint{1.927379in}{3.084357in}}%
\pgfusepath{fill}%
\end{pgfscope}%
\begin{pgfscope}%
\pgfpathrectangle{\pgfqpoint{1.432000in}{0.528000in}}{\pgfqpoint{3.696000in}{3.696000in}} %
\pgfusepath{clip}%
\pgfsetbuttcap%
\pgfsetroundjoin%
\definecolor{currentfill}{rgb}{0.157729,0.485932,0.558013}%
\pgfsetfillcolor{currentfill}%
\pgfsetlinewidth{0.000000pt}%
\definecolor{currentstroke}{rgb}{0.000000,0.000000,0.000000}%
\pgfsetstrokecolor{currentstroke}%
\pgfsetdash{}{0pt}%
\pgfpathmoveto{\pgfqpoint{2.037984in}{3.080516in}}%
\pgfpathlineto{\pgfqpoint{2.035766in}{3.084357in}}%
\pgfpathlineto{\pgfqpoint{2.031331in}{3.084357in}}%
\pgfpathlineto{\pgfqpoint{2.029113in}{3.080516in}}%
\pgfpathlineto{\pgfqpoint{2.031331in}{3.076675in}}%
\pgfpathlineto{\pgfqpoint{2.035766in}{3.076675in}}%
\pgfpathlineto{\pgfqpoint{2.037984in}{3.080516in}}%
\pgfpathlineto{\pgfqpoint{2.035766in}{3.084357in}}%
\pgfusepath{fill}%
\end{pgfscope}%
\begin{pgfscope}%
\pgfpathrectangle{\pgfqpoint{1.432000in}{0.528000in}}{\pgfqpoint{3.696000in}{3.696000in}} %
\pgfusepath{clip}%
\pgfsetbuttcap%
\pgfsetroundjoin%
\definecolor{currentfill}{rgb}{0.279574,0.170599,0.479997}%
\pgfsetfillcolor{currentfill}%
\pgfsetlinewidth{0.000000pt}%
\definecolor{currentstroke}{rgb}{0.000000,0.000000,0.000000}%
\pgfsetstrokecolor{currentstroke}%
\pgfsetdash{}{0pt}%
\pgfpathmoveto{\pgfqpoint{2.141935in}{3.076081in}}%
\pgfpathlineto{\pgfqpoint{2.073465in}{3.076081in}}%
\pgfpathlineto{\pgfqpoint{2.077900in}{3.067211in}}%
\pgfpathlineto{\pgfqpoint{2.033548in}{3.080516in}}%
\pgfpathlineto{\pgfqpoint{2.077900in}{3.093822in}}%
\pgfpathlineto{\pgfqpoint{2.073465in}{3.084951in}}%
\pgfpathlineto{\pgfqpoint{2.141935in}{3.084951in}}%
\pgfpathlineto{\pgfqpoint{2.141935in}{3.076081in}}%
\pgfusepath{fill}%
\end{pgfscope}%
\begin{pgfscope}%
\pgfpathrectangle{\pgfqpoint{1.432000in}{0.528000in}}{\pgfqpoint{3.696000in}{3.696000in}} %
\pgfusepath{clip}%
\pgfsetbuttcap%
\pgfsetroundjoin%
\definecolor{currentfill}{rgb}{0.266580,0.228262,0.514349}%
\pgfsetfillcolor{currentfill}%
\pgfsetlinewidth{0.000000pt}%
\definecolor{currentstroke}{rgb}{0.000000,0.000000,0.000000}%
\pgfsetstrokecolor{currentstroke}%
\pgfsetdash{}{0pt}%
\pgfpathmoveto{\pgfqpoint{2.146371in}{3.080516in}}%
\pgfpathlineto{\pgfqpoint{2.144153in}{3.084357in}}%
\pgfpathlineto{\pgfqpoint{2.139718in}{3.084357in}}%
\pgfpathlineto{\pgfqpoint{2.137500in}{3.080516in}}%
\pgfpathlineto{\pgfqpoint{2.139718in}{3.076675in}}%
\pgfpathlineto{\pgfqpoint{2.144153in}{3.076675in}}%
\pgfpathlineto{\pgfqpoint{2.146371in}{3.080516in}}%
\pgfpathlineto{\pgfqpoint{2.144153in}{3.084357in}}%
\pgfusepath{fill}%
\end{pgfscope}%
\begin{pgfscope}%
\pgfpathrectangle{\pgfqpoint{1.432000in}{0.528000in}}{\pgfqpoint{3.696000in}{3.696000in}} %
\pgfusepath{clip}%
\pgfsetbuttcap%
\pgfsetroundjoin%
\definecolor{currentfill}{rgb}{0.190631,0.407061,0.556089}%
\pgfsetfillcolor{currentfill}%
\pgfsetlinewidth{0.000000pt}%
\definecolor{currentstroke}{rgb}{0.000000,0.000000,0.000000}%
\pgfsetstrokecolor{currentstroke}%
\pgfsetdash{}{0pt}%
\pgfpathmoveto{\pgfqpoint{2.250323in}{3.076081in}}%
\pgfpathlineto{\pgfqpoint{2.181852in}{3.076081in}}%
\pgfpathlineto{\pgfqpoint{2.186287in}{3.067211in}}%
\pgfpathlineto{\pgfqpoint{2.141935in}{3.080516in}}%
\pgfpathlineto{\pgfqpoint{2.186287in}{3.093822in}}%
\pgfpathlineto{\pgfqpoint{2.181852in}{3.084951in}}%
\pgfpathlineto{\pgfqpoint{2.250323in}{3.084951in}}%
\pgfpathlineto{\pgfqpoint{2.250323in}{3.076081in}}%
\pgfusepath{fill}%
\end{pgfscope}%
\begin{pgfscope}%
\pgfpathrectangle{\pgfqpoint{1.432000in}{0.528000in}}{\pgfqpoint{3.696000in}{3.696000in}} %
\pgfusepath{clip}%
\pgfsetbuttcap%
\pgfsetroundjoin%
\definecolor{currentfill}{rgb}{0.274952,0.037752,0.364543}%
\pgfsetfillcolor{currentfill}%
\pgfsetlinewidth{0.000000pt}%
\definecolor{currentstroke}{rgb}{0.000000,0.000000,0.000000}%
\pgfsetstrokecolor{currentstroke}%
\pgfsetdash{}{0pt}%
\pgfpathmoveto{\pgfqpoint{2.254758in}{3.080516in}}%
\pgfpathlineto{\pgfqpoint{2.252540in}{3.084357in}}%
\pgfpathlineto{\pgfqpoint{2.248105in}{3.084357in}}%
\pgfpathlineto{\pgfqpoint{2.245887in}{3.080516in}}%
\pgfpathlineto{\pgfqpoint{2.248105in}{3.076675in}}%
\pgfpathlineto{\pgfqpoint{2.252540in}{3.076675in}}%
\pgfpathlineto{\pgfqpoint{2.254758in}{3.080516in}}%
\pgfpathlineto{\pgfqpoint{2.252540in}{3.084357in}}%
\pgfusepath{fill}%
\end{pgfscope}%
\begin{pgfscope}%
\pgfpathrectangle{\pgfqpoint{1.432000in}{0.528000in}}{\pgfqpoint{3.696000in}{3.696000in}} %
\pgfusepath{clip}%
\pgfsetbuttcap%
\pgfsetroundjoin%
\definecolor{currentfill}{rgb}{0.120092,0.600104,0.542530}%
\pgfsetfillcolor{currentfill}%
\pgfsetlinewidth{0.000000pt}%
\definecolor{currentstroke}{rgb}{0.000000,0.000000,0.000000}%
\pgfsetstrokecolor{currentstroke}%
\pgfsetdash{}{0pt}%
\pgfpathmoveto{\pgfqpoint{2.358710in}{3.076081in}}%
\pgfpathlineto{\pgfqpoint{2.290239in}{3.076081in}}%
\pgfpathlineto{\pgfqpoint{2.294675in}{3.067211in}}%
\pgfpathlineto{\pgfqpoint{2.250323in}{3.080516in}}%
\pgfpathlineto{\pgfqpoint{2.294675in}{3.093822in}}%
\pgfpathlineto{\pgfqpoint{2.290239in}{3.084951in}}%
\pgfpathlineto{\pgfqpoint{2.358710in}{3.084951in}}%
\pgfpathlineto{\pgfqpoint{2.358710in}{3.076081in}}%
\pgfusepath{fill}%
\end{pgfscope}%
\begin{pgfscope}%
\pgfpathrectangle{\pgfqpoint{1.432000in}{0.528000in}}{\pgfqpoint{3.696000in}{3.696000in}} %
\pgfusepath{clip}%
\pgfsetbuttcap%
\pgfsetroundjoin%
\definecolor{currentfill}{rgb}{0.185783,0.704891,0.485273}%
\pgfsetfillcolor{currentfill}%
\pgfsetlinewidth{0.000000pt}%
\definecolor{currentstroke}{rgb}{0.000000,0.000000,0.000000}%
\pgfsetstrokecolor{currentstroke}%
\pgfsetdash{}{0pt}%
\pgfpathmoveto{\pgfqpoint{2.467097in}{3.076081in}}%
\pgfpathlineto{\pgfqpoint{2.398626in}{3.076081in}}%
\pgfpathlineto{\pgfqpoint{2.403062in}{3.067211in}}%
\pgfpathlineto{\pgfqpoint{2.358710in}{3.080516in}}%
\pgfpathlineto{\pgfqpoint{2.403062in}{3.093822in}}%
\pgfpathlineto{\pgfqpoint{2.398626in}{3.084951in}}%
\pgfpathlineto{\pgfqpoint{2.467097in}{3.084951in}}%
\pgfpathlineto{\pgfqpoint{2.467097in}{3.076081in}}%
\pgfusepath{fill}%
\end{pgfscope}%
\begin{pgfscope}%
\pgfpathrectangle{\pgfqpoint{1.432000in}{0.528000in}}{\pgfqpoint{3.696000in}{3.696000in}} %
\pgfusepath{clip}%
\pgfsetbuttcap%
\pgfsetroundjoin%
\definecolor{currentfill}{rgb}{0.277018,0.050344,0.375715}%
\pgfsetfillcolor{currentfill}%
\pgfsetlinewidth{0.000000pt}%
\definecolor{currentstroke}{rgb}{0.000000,0.000000,0.000000}%
\pgfsetstrokecolor{currentstroke}%
\pgfsetdash{}{0pt}%
\pgfpathmoveto{\pgfqpoint{2.575484in}{3.076081in}}%
\pgfpathlineto{\pgfqpoint{2.398626in}{3.076081in}}%
\pgfpathlineto{\pgfqpoint{2.403062in}{3.067211in}}%
\pgfpathlineto{\pgfqpoint{2.358710in}{3.080516in}}%
\pgfpathlineto{\pgfqpoint{2.403062in}{3.093822in}}%
\pgfpathlineto{\pgfqpoint{2.398626in}{3.084951in}}%
\pgfpathlineto{\pgfqpoint{2.575484in}{3.084951in}}%
\pgfpathlineto{\pgfqpoint{2.575484in}{3.076081in}}%
\pgfusepath{fill}%
\end{pgfscope}%
\begin{pgfscope}%
\pgfpathrectangle{\pgfqpoint{1.432000in}{0.528000in}}{\pgfqpoint{3.696000in}{3.696000in}} %
\pgfusepath{clip}%
\pgfsetbuttcap%
\pgfsetroundjoin%
\definecolor{currentfill}{rgb}{0.126326,0.644107,0.525311}%
\pgfsetfillcolor{currentfill}%
\pgfsetlinewidth{0.000000pt}%
\definecolor{currentstroke}{rgb}{0.000000,0.000000,0.000000}%
\pgfsetstrokecolor{currentstroke}%
\pgfsetdash{}{0pt}%
\pgfpathmoveto{\pgfqpoint{2.575484in}{3.076081in}}%
\pgfpathlineto{\pgfqpoint{2.507014in}{3.076081in}}%
\pgfpathlineto{\pgfqpoint{2.511449in}{3.067211in}}%
\pgfpathlineto{\pgfqpoint{2.467097in}{3.080516in}}%
\pgfpathlineto{\pgfqpoint{2.511449in}{3.093822in}}%
\pgfpathlineto{\pgfqpoint{2.507014in}{3.084951in}}%
\pgfpathlineto{\pgfqpoint{2.575484in}{3.084951in}}%
\pgfpathlineto{\pgfqpoint{2.575484in}{3.076081in}}%
\pgfusepath{fill}%
\end{pgfscope}%
\begin{pgfscope}%
\pgfpathrectangle{\pgfqpoint{1.432000in}{0.528000in}}{\pgfqpoint{3.696000in}{3.696000in}} %
\pgfusepath{clip}%
\pgfsetbuttcap%
\pgfsetroundjoin%
\definecolor{currentfill}{rgb}{0.206756,0.371758,0.553117}%
\pgfsetfillcolor{currentfill}%
\pgfsetlinewidth{0.000000pt}%
\definecolor{currentstroke}{rgb}{0.000000,0.000000,0.000000}%
\pgfsetstrokecolor{currentstroke}%
\pgfsetdash{}{0pt}%
\pgfpathmoveto{\pgfqpoint{2.683871in}{3.076081in}}%
\pgfpathlineto{\pgfqpoint{2.615401in}{3.076081in}}%
\pgfpathlineto{\pgfqpoint{2.619836in}{3.067211in}}%
\pgfpathlineto{\pgfqpoint{2.575484in}{3.080516in}}%
\pgfpathlineto{\pgfqpoint{2.619836in}{3.093822in}}%
\pgfpathlineto{\pgfqpoint{2.615401in}{3.084951in}}%
\pgfpathlineto{\pgfqpoint{2.683871in}{3.084951in}}%
\pgfpathlineto{\pgfqpoint{2.683871in}{3.076081in}}%
\pgfusepath{fill}%
\end{pgfscope}%
\begin{pgfscope}%
\pgfpathrectangle{\pgfqpoint{1.432000in}{0.528000in}}{\pgfqpoint{3.696000in}{3.696000in}} %
\pgfusepath{clip}%
\pgfsetbuttcap%
\pgfsetroundjoin%
\definecolor{currentfill}{rgb}{0.271305,0.019942,0.347269}%
\pgfsetfillcolor{currentfill}%
\pgfsetlinewidth{0.000000pt}%
\definecolor{currentstroke}{rgb}{0.000000,0.000000,0.000000}%
\pgfsetstrokecolor{currentstroke}%
\pgfsetdash{}{0pt}%
\pgfpathmoveto{\pgfqpoint{2.680735in}{3.077380in}}%
\pgfpathlineto{\pgfqpoint{2.600573in}{3.157542in}}%
\pgfpathlineto{\pgfqpoint{2.597437in}{3.148133in}}%
\pgfpathlineto{\pgfqpoint{2.575484in}{3.188903in}}%
\pgfpathlineto{\pgfqpoint{2.616254in}{3.166950in}}%
\pgfpathlineto{\pgfqpoint{2.606845in}{3.163814in}}%
\pgfpathlineto{\pgfqpoint{2.687007in}{3.083652in}}%
\pgfpathlineto{\pgfqpoint{2.680735in}{3.077380in}}%
\pgfusepath{fill}%
\end{pgfscope}%
\begin{pgfscope}%
\pgfpathrectangle{\pgfqpoint{1.432000in}{0.528000in}}{\pgfqpoint{3.696000in}{3.696000in}} %
\pgfusepath{clip}%
\pgfsetbuttcap%
\pgfsetroundjoin%
\definecolor{currentfill}{rgb}{0.272594,0.025563,0.353093}%
\pgfsetfillcolor{currentfill}%
\pgfsetlinewidth{0.000000pt}%
\definecolor{currentstroke}{rgb}{0.000000,0.000000,0.000000}%
\pgfsetstrokecolor{currentstroke}%
\pgfsetdash{}{0pt}%
\pgfpathmoveto{\pgfqpoint{2.679436in}{3.080516in}}%
\pgfpathlineto{\pgfqpoint{2.679436in}{3.148986in}}%
\pgfpathlineto{\pgfqpoint{2.670565in}{3.144551in}}%
\pgfpathlineto{\pgfqpoint{2.683871in}{3.188903in}}%
\pgfpathlineto{\pgfqpoint{2.697177in}{3.144551in}}%
\pgfpathlineto{\pgfqpoint{2.688306in}{3.148986in}}%
\pgfpathlineto{\pgfqpoint{2.688306in}{3.080516in}}%
\pgfpathlineto{\pgfqpoint{2.679436in}{3.080516in}}%
\pgfusepath{fill}%
\end{pgfscope}%
\begin{pgfscope}%
\pgfpathrectangle{\pgfqpoint{1.432000in}{0.528000in}}{\pgfqpoint{3.696000in}{3.696000in}} %
\pgfusepath{clip}%
\pgfsetbuttcap%
\pgfsetroundjoin%
\definecolor{currentfill}{rgb}{0.283187,0.125848,0.444960}%
\pgfsetfillcolor{currentfill}%
\pgfsetlinewidth{0.000000pt}%
\definecolor{currentstroke}{rgb}{0.000000,0.000000,0.000000}%
\pgfsetstrokecolor{currentstroke}%
\pgfsetdash{}{0pt}%
\pgfpathmoveto{\pgfqpoint{2.787823in}{3.080516in}}%
\pgfpathlineto{\pgfqpoint{2.787823in}{3.148986in}}%
\pgfpathlineto{\pgfqpoint{2.778952in}{3.144551in}}%
\pgfpathlineto{\pgfqpoint{2.792258in}{3.188903in}}%
\pgfpathlineto{\pgfqpoint{2.805564in}{3.144551in}}%
\pgfpathlineto{\pgfqpoint{2.796693in}{3.148986in}}%
\pgfpathlineto{\pgfqpoint{2.796693in}{3.080516in}}%
\pgfpathlineto{\pgfqpoint{2.787823in}{3.080516in}}%
\pgfusepath{fill}%
\end{pgfscope}%
\begin{pgfscope}%
\pgfpathrectangle{\pgfqpoint{1.432000in}{0.528000in}}{\pgfqpoint{3.696000in}{3.696000in}} %
\pgfusepath{clip}%
\pgfsetbuttcap%
\pgfsetroundjoin%
\definecolor{currentfill}{rgb}{0.146180,0.515413,0.556823}%
\pgfsetfillcolor{currentfill}%
\pgfsetlinewidth{0.000000pt}%
\definecolor{currentstroke}{rgb}{0.000000,0.000000,0.000000}%
\pgfsetstrokecolor{currentstroke}%
\pgfsetdash{}{0pt}%
\pgfpathmoveto{\pgfqpoint{2.896210in}{3.080516in}}%
\pgfpathlineto{\pgfqpoint{2.896210in}{3.148986in}}%
\pgfpathlineto{\pgfqpoint{2.887340in}{3.144551in}}%
\pgfpathlineto{\pgfqpoint{2.900645in}{3.188903in}}%
\pgfpathlineto{\pgfqpoint{2.913951in}{3.144551in}}%
\pgfpathlineto{\pgfqpoint{2.905080in}{3.148986in}}%
\pgfpathlineto{\pgfqpoint{2.905080in}{3.080516in}}%
\pgfpathlineto{\pgfqpoint{2.896210in}{3.080516in}}%
\pgfusepath{fill}%
\end{pgfscope}%
\begin{pgfscope}%
\pgfpathrectangle{\pgfqpoint{1.432000in}{0.528000in}}{\pgfqpoint{3.696000in}{3.696000in}} %
\pgfusepath{clip}%
\pgfsetbuttcap%
\pgfsetroundjoin%
\definecolor{currentfill}{rgb}{0.120092,0.600104,0.542530}%
\pgfsetfillcolor{currentfill}%
\pgfsetlinewidth{0.000000pt}%
\definecolor{currentstroke}{rgb}{0.000000,0.000000,0.000000}%
\pgfsetstrokecolor{currentstroke}%
\pgfsetdash{}{0pt}%
\pgfpathmoveto{\pgfqpoint{3.004597in}{3.080516in}}%
\pgfpathlineto{\pgfqpoint{3.004597in}{3.148986in}}%
\pgfpathlineto{\pgfqpoint{2.995727in}{3.144551in}}%
\pgfpathlineto{\pgfqpoint{3.009032in}{3.188903in}}%
\pgfpathlineto{\pgfqpoint{3.022338in}{3.144551in}}%
\pgfpathlineto{\pgfqpoint{3.013467in}{3.148986in}}%
\pgfpathlineto{\pgfqpoint{3.013467in}{3.080516in}}%
\pgfpathlineto{\pgfqpoint{3.004597in}{3.080516in}}%
\pgfusepath{fill}%
\end{pgfscope}%
\begin{pgfscope}%
\pgfpathrectangle{\pgfqpoint{1.432000in}{0.528000in}}{\pgfqpoint{3.696000in}{3.696000in}} %
\pgfusepath{clip}%
\pgfsetbuttcap%
\pgfsetroundjoin%
\definecolor{currentfill}{rgb}{0.172719,0.448791,0.557885}%
\pgfsetfillcolor{currentfill}%
\pgfsetlinewidth{0.000000pt}%
\definecolor{currentstroke}{rgb}{0.000000,0.000000,0.000000}%
\pgfsetstrokecolor{currentstroke}%
\pgfsetdash{}{0pt}%
\pgfpathmoveto{\pgfqpoint{3.112984in}{3.080516in}}%
\pgfpathlineto{\pgfqpoint{3.112984in}{3.148986in}}%
\pgfpathlineto{\pgfqpoint{3.104114in}{3.144551in}}%
\pgfpathlineto{\pgfqpoint{3.117419in}{3.188903in}}%
\pgfpathlineto{\pgfqpoint{3.130725in}{3.144551in}}%
\pgfpathlineto{\pgfqpoint{3.121855in}{3.148986in}}%
\pgfpathlineto{\pgfqpoint{3.121855in}{3.080516in}}%
\pgfpathlineto{\pgfqpoint{3.112984in}{3.080516in}}%
\pgfusepath{fill}%
\end{pgfscope}%
\begin{pgfscope}%
\pgfpathrectangle{\pgfqpoint{1.432000in}{0.528000in}}{\pgfqpoint{3.696000in}{3.696000in}} %
\pgfusepath{clip}%
\pgfsetbuttcap%
\pgfsetroundjoin%
\definecolor{currentfill}{rgb}{0.277018,0.050344,0.375715}%
\pgfsetfillcolor{currentfill}%
\pgfsetlinewidth{0.000000pt}%
\definecolor{currentstroke}{rgb}{0.000000,0.000000,0.000000}%
\pgfsetstrokecolor{currentstroke}%
\pgfsetdash{}{0pt}%
\pgfpathmoveto{\pgfqpoint{3.112984in}{3.080516in}}%
\pgfpathlineto{\pgfqpoint{3.112984in}{3.257374in}}%
\pgfpathlineto{\pgfqpoint{3.104114in}{3.252938in}}%
\pgfpathlineto{\pgfqpoint{3.117419in}{3.297290in}}%
\pgfpathlineto{\pgfqpoint{3.130725in}{3.252938in}}%
\pgfpathlineto{\pgfqpoint{3.121855in}{3.257374in}}%
\pgfpathlineto{\pgfqpoint{3.121855in}{3.080516in}}%
\pgfpathlineto{\pgfqpoint{3.112984in}{3.080516in}}%
\pgfusepath{fill}%
\end{pgfscope}%
\begin{pgfscope}%
\pgfpathrectangle{\pgfqpoint{1.432000in}{0.528000in}}{\pgfqpoint{3.696000in}{3.696000in}} %
\pgfusepath{clip}%
\pgfsetbuttcap%
\pgfsetroundjoin%
\definecolor{currentfill}{rgb}{0.271828,0.209303,0.504434}%
\pgfsetfillcolor{currentfill}%
\pgfsetlinewidth{0.000000pt}%
\definecolor{currentstroke}{rgb}{0.000000,0.000000,0.000000}%
\pgfsetstrokecolor{currentstroke}%
\pgfsetdash{}{0pt}%
\pgfpathmoveto{\pgfqpoint{3.221371in}{3.080516in}}%
\pgfpathlineto{\pgfqpoint{3.221371in}{3.257374in}}%
\pgfpathlineto{\pgfqpoint{3.212501in}{3.252938in}}%
\pgfpathlineto{\pgfqpoint{3.225806in}{3.297290in}}%
\pgfpathlineto{\pgfqpoint{3.239112in}{3.252938in}}%
\pgfpathlineto{\pgfqpoint{3.230242in}{3.257374in}}%
\pgfpathlineto{\pgfqpoint{3.230242in}{3.080516in}}%
\pgfpathlineto{\pgfqpoint{3.221371in}{3.080516in}}%
\pgfusepath{fill}%
\end{pgfscope}%
\begin{pgfscope}%
\pgfpathrectangle{\pgfqpoint{1.432000in}{0.528000in}}{\pgfqpoint{3.696000in}{3.696000in}} %
\pgfusepath{clip}%
\pgfsetbuttcap%
\pgfsetroundjoin%
\definecolor{currentfill}{rgb}{0.177423,0.437527,0.557565}%
\pgfsetfillcolor{currentfill}%
\pgfsetlinewidth{0.000000pt}%
\definecolor{currentstroke}{rgb}{0.000000,0.000000,0.000000}%
\pgfsetstrokecolor{currentstroke}%
\pgfsetdash{}{0pt}%
\pgfpathmoveto{\pgfqpoint{3.330227in}{3.078533in}}%
\pgfpathlineto{\pgfqpoint{3.239691in}{3.259604in}}%
\pgfpathlineto{\pgfqpoint{3.233740in}{3.251670in}}%
\pgfpathlineto{\pgfqpoint{3.225806in}{3.297290in}}%
\pgfpathlineto{\pgfqpoint{3.257542in}{3.263571in}}%
\pgfpathlineto{\pgfqpoint{3.247625in}{3.263571in}}%
\pgfpathlineto{\pgfqpoint{3.338161in}{3.082500in}}%
\pgfpathlineto{\pgfqpoint{3.330227in}{3.078533in}}%
\pgfusepath{fill}%
\end{pgfscope}%
\begin{pgfscope}%
\pgfpathrectangle{\pgfqpoint{1.432000in}{0.528000in}}{\pgfqpoint{3.696000in}{3.696000in}} %
\pgfusepath{clip}%
\pgfsetbuttcap%
\pgfsetroundjoin%
\definecolor{currentfill}{rgb}{0.121148,0.592739,0.544641}%
\pgfsetfillcolor{currentfill}%
\pgfsetlinewidth{0.000000pt}%
\definecolor{currentstroke}{rgb}{0.000000,0.000000,0.000000}%
\pgfsetstrokecolor{currentstroke}%
\pgfsetdash{}{0pt}%
\pgfpathmoveto{\pgfqpoint{3.438614in}{3.078533in}}%
\pgfpathlineto{\pgfqpoint{3.348078in}{3.259604in}}%
\pgfpathlineto{\pgfqpoint{3.342127in}{3.251670in}}%
\pgfpathlineto{\pgfqpoint{3.334194in}{3.297290in}}%
\pgfpathlineto{\pgfqpoint{3.365929in}{3.263571in}}%
\pgfpathlineto{\pgfqpoint{3.356012in}{3.263571in}}%
\pgfpathlineto{\pgfqpoint{3.446548in}{3.082500in}}%
\pgfpathlineto{\pgfqpoint{3.438614in}{3.078533in}}%
\pgfusepath{fill}%
\end{pgfscope}%
\begin{pgfscope}%
\pgfpathrectangle{\pgfqpoint{1.432000in}{0.528000in}}{\pgfqpoint{3.696000in}{3.696000in}} %
\pgfusepath{clip}%
\pgfsetbuttcap%
\pgfsetroundjoin%
\definecolor{currentfill}{rgb}{0.151918,0.500685,0.557587}%
\pgfsetfillcolor{currentfill}%
\pgfsetlinewidth{0.000000pt}%
\definecolor{currentstroke}{rgb}{0.000000,0.000000,0.000000}%
\pgfsetstrokecolor{currentstroke}%
\pgfsetdash{}{0pt}%
\pgfpathmoveto{\pgfqpoint{3.547832in}{3.077380in}}%
\pgfpathlineto{\pgfqpoint{3.359283in}{3.265929in}}%
\pgfpathlineto{\pgfqpoint{3.356147in}{3.256520in}}%
\pgfpathlineto{\pgfqpoint{3.334194in}{3.297290in}}%
\pgfpathlineto{\pgfqpoint{3.374964in}{3.275337in}}%
\pgfpathlineto{\pgfqpoint{3.365555in}{3.272201in}}%
\pgfpathlineto{\pgfqpoint{3.554104in}{3.083652in}}%
\pgfpathlineto{\pgfqpoint{3.547832in}{3.077380in}}%
\pgfusepath{fill}%
\end{pgfscope}%
\begin{pgfscope}%
\pgfpathrectangle{\pgfqpoint{1.432000in}{0.528000in}}{\pgfqpoint{3.696000in}{3.696000in}} %
\pgfusepath{clip}%
\pgfsetbuttcap%
\pgfsetroundjoin%
\definecolor{currentfill}{rgb}{0.185556,0.418570,0.556753}%
\pgfsetfillcolor{currentfill}%
\pgfsetlinewidth{0.000000pt}%
\definecolor{currentstroke}{rgb}{0.000000,0.000000,0.000000}%
\pgfsetstrokecolor{currentstroke}%
\pgfsetdash{}{0pt}%
\pgfpathmoveto{\pgfqpoint{3.656219in}{3.077380in}}%
\pgfpathlineto{\pgfqpoint{3.467670in}{3.265929in}}%
\pgfpathlineto{\pgfqpoint{3.464534in}{3.256520in}}%
\pgfpathlineto{\pgfqpoint{3.442581in}{3.297290in}}%
\pgfpathlineto{\pgfqpoint{3.483351in}{3.275337in}}%
\pgfpathlineto{\pgfqpoint{3.473942in}{3.272201in}}%
\pgfpathlineto{\pgfqpoint{3.662491in}{3.083652in}}%
\pgfpathlineto{\pgfqpoint{3.656219in}{3.077380in}}%
\pgfusepath{fill}%
\end{pgfscope}%
\begin{pgfscope}%
\pgfpathrectangle{\pgfqpoint{1.432000in}{0.528000in}}{\pgfqpoint{3.696000in}{3.696000in}} %
\pgfusepath{clip}%
\pgfsetbuttcap%
\pgfsetroundjoin%
\definecolor{currentfill}{rgb}{0.160665,0.478540,0.558115}%
\pgfsetfillcolor{currentfill}%
\pgfsetlinewidth{0.000000pt}%
\definecolor{currentstroke}{rgb}{0.000000,0.000000,0.000000}%
\pgfsetstrokecolor{currentstroke}%
\pgfsetdash{}{0pt}%
\pgfpathmoveto{\pgfqpoint{3.765282in}{3.076826in}}%
\pgfpathlineto{\pgfqpoint{3.473333in}{3.271458in}}%
\pgfpathlineto{\pgfqpoint{3.472103in}{3.261617in}}%
\pgfpathlineto{\pgfqpoint{3.442581in}{3.297290in}}%
\pgfpathlineto{\pgfqpoint{3.486864in}{3.283759in}}%
\pgfpathlineto{\pgfqpoint{3.478254in}{3.278839in}}%
\pgfpathlineto{\pgfqpoint{3.770202in}{3.084206in}}%
\pgfpathlineto{\pgfqpoint{3.765282in}{3.076826in}}%
\pgfusepath{fill}%
\end{pgfscope}%
\begin{pgfscope}%
\pgfpathrectangle{\pgfqpoint{1.432000in}{0.528000in}}{\pgfqpoint{3.696000in}{3.696000in}} %
\pgfusepath{clip}%
\pgfsetbuttcap%
\pgfsetroundjoin%
\definecolor{currentfill}{rgb}{0.135066,0.544853,0.554029}%
\pgfsetfillcolor{currentfill}%
\pgfsetlinewidth{0.000000pt}%
\definecolor{currentstroke}{rgb}{0.000000,0.000000,0.000000}%
\pgfsetstrokecolor{currentstroke}%
\pgfsetdash{}{0pt}%
\pgfpathmoveto{\pgfqpoint{3.872993in}{3.077380in}}%
\pgfpathlineto{\pgfqpoint{3.576057in}{3.374316in}}%
\pgfpathlineto{\pgfqpoint{3.572921in}{3.364907in}}%
\pgfpathlineto{\pgfqpoint{3.550968in}{3.405677in}}%
\pgfpathlineto{\pgfqpoint{3.591738in}{3.383724in}}%
\pgfpathlineto{\pgfqpoint{3.582329in}{3.380588in}}%
\pgfpathlineto{\pgfqpoint{3.879265in}{3.083652in}}%
\pgfpathlineto{\pgfqpoint{3.872993in}{3.077380in}}%
\pgfusepath{fill}%
\end{pgfscope}%
\begin{pgfscope}%
\pgfpathrectangle{\pgfqpoint{1.432000in}{0.528000in}}{\pgfqpoint{3.696000in}{3.696000in}} %
\pgfusepath{clip}%
\pgfsetbuttcap%
\pgfsetroundjoin%
\definecolor{currentfill}{rgb}{0.226397,0.728888,0.462789}%
\pgfsetfillcolor{currentfill}%
\pgfsetlinewidth{0.000000pt}%
\definecolor{currentstroke}{rgb}{0.000000,0.000000,0.000000}%
\pgfsetstrokecolor{currentstroke}%
\pgfsetdash{}{0pt}%
\pgfpathmoveto{\pgfqpoint{3.981380in}{3.077380in}}%
\pgfpathlineto{\pgfqpoint{3.684444in}{3.374316in}}%
\pgfpathlineto{\pgfqpoint{3.681308in}{3.364907in}}%
\pgfpathlineto{\pgfqpoint{3.659355in}{3.405677in}}%
\pgfpathlineto{\pgfqpoint{3.700125in}{3.383724in}}%
\pgfpathlineto{\pgfqpoint{3.690716in}{3.380588in}}%
\pgfpathlineto{\pgfqpoint{3.987652in}{3.083652in}}%
\pgfpathlineto{\pgfqpoint{3.981380in}{3.077380in}}%
\pgfusepath{fill}%
\end{pgfscope}%
\begin{pgfscope}%
\pgfpathrectangle{\pgfqpoint{1.432000in}{0.528000in}}{\pgfqpoint{3.696000in}{3.696000in}} %
\pgfusepath{clip}%
\pgfsetbuttcap%
\pgfsetroundjoin%
\definecolor{currentfill}{rgb}{0.282884,0.135920,0.453427}%
\pgfsetfillcolor{currentfill}%
\pgfsetlinewidth{0.000000pt}%
\definecolor{currentstroke}{rgb}{0.000000,0.000000,0.000000}%
\pgfsetstrokecolor{currentstroke}%
\pgfsetdash{}{0pt}%
\pgfpathmoveto{\pgfqpoint{4.090242in}{3.076968in}}%
\pgfpathlineto{\pgfqpoint{3.688627in}{3.378179in}}%
\pgfpathlineto{\pgfqpoint{3.686853in}{3.368422in}}%
\pgfpathlineto{\pgfqpoint{3.659355in}{3.405677in}}%
\pgfpathlineto{\pgfqpoint{3.702820in}{3.389711in}}%
\pgfpathlineto{\pgfqpoint{3.693949in}{3.385275in}}%
\pgfpathlineto{\pgfqpoint{4.095564in}{3.084064in}}%
\pgfpathlineto{\pgfqpoint{4.090242in}{3.076968in}}%
\pgfusepath{fill}%
\end{pgfscope}%
\begin{pgfscope}%
\pgfpathrectangle{\pgfqpoint{1.432000in}{0.528000in}}{\pgfqpoint{3.696000in}{3.696000in}} %
\pgfusepath{clip}%
\pgfsetbuttcap%
\pgfsetroundjoin%
\definecolor{currentfill}{rgb}{0.124780,0.640461,0.527068}%
\pgfsetfillcolor{currentfill}%
\pgfsetlinewidth{0.000000pt}%
\definecolor{currentstroke}{rgb}{0.000000,0.000000,0.000000}%
\pgfsetstrokecolor{currentstroke}%
\pgfsetdash{}{0pt}%
\pgfpathmoveto{\pgfqpoint{4.089767in}{3.077380in}}%
\pgfpathlineto{\pgfqpoint{3.792831in}{3.374316in}}%
\pgfpathlineto{\pgfqpoint{3.789695in}{3.364907in}}%
\pgfpathlineto{\pgfqpoint{3.767742in}{3.405677in}}%
\pgfpathlineto{\pgfqpoint{3.808512in}{3.383724in}}%
\pgfpathlineto{\pgfqpoint{3.799104in}{3.380588in}}%
\pgfpathlineto{\pgfqpoint{4.096039in}{3.083652in}}%
\pgfpathlineto{\pgfqpoint{4.089767in}{3.077380in}}%
\pgfusepath{fill}%
\end{pgfscope}%
\begin{pgfscope}%
\pgfpathrectangle{\pgfqpoint{1.432000in}{0.528000in}}{\pgfqpoint{3.696000in}{3.696000in}} %
\pgfusepath{clip}%
\pgfsetbuttcap%
\pgfsetroundjoin%
\definecolor{currentfill}{rgb}{0.278826,0.175490,0.483397}%
\pgfsetfillcolor{currentfill}%
\pgfsetlinewidth{0.000000pt}%
\definecolor{currentstroke}{rgb}{0.000000,0.000000,0.000000}%
\pgfsetstrokecolor{currentstroke}%
\pgfsetdash{}{0pt}%
\pgfpathmoveto{\pgfqpoint{4.198629in}{3.076968in}}%
\pgfpathlineto{\pgfqpoint{3.797014in}{3.378179in}}%
\pgfpathlineto{\pgfqpoint{3.795240in}{3.368422in}}%
\pgfpathlineto{\pgfqpoint{3.767742in}{3.405677in}}%
\pgfpathlineto{\pgfqpoint{3.811207in}{3.389711in}}%
\pgfpathlineto{\pgfqpoint{3.802336in}{3.385275in}}%
\pgfpathlineto{\pgfqpoint{4.203951in}{3.084064in}}%
\pgfpathlineto{\pgfqpoint{4.198629in}{3.076968in}}%
\pgfusepath{fill}%
\end{pgfscope}%
\begin{pgfscope}%
\pgfpathrectangle{\pgfqpoint{1.432000in}{0.528000in}}{\pgfqpoint{3.696000in}{3.696000in}} %
\pgfusepath{clip}%
\pgfsetbuttcap%
\pgfsetroundjoin%
\definecolor{currentfill}{rgb}{0.266941,0.748751,0.440573}%
\pgfsetfillcolor{currentfill}%
\pgfsetlinewidth{0.000000pt}%
\definecolor{currentstroke}{rgb}{0.000000,0.000000,0.000000}%
\pgfsetstrokecolor{currentstroke}%
\pgfsetdash{}{0pt}%
\pgfpathmoveto{\pgfqpoint{4.198154in}{3.077380in}}%
\pgfpathlineto{\pgfqpoint{3.901218in}{3.374316in}}%
\pgfpathlineto{\pgfqpoint{3.898082in}{3.364907in}}%
\pgfpathlineto{\pgfqpoint{3.876129in}{3.405677in}}%
\pgfpathlineto{\pgfqpoint{3.916899in}{3.383724in}}%
\pgfpathlineto{\pgfqpoint{3.907491in}{3.380588in}}%
\pgfpathlineto{\pgfqpoint{4.204426in}{3.083652in}}%
\pgfpathlineto{\pgfqpoint{4.198154in}{3.077380in}}%
\pgfusepath{fill}%
\end{pgfscope}%
\begin{pgfscope}%
\pgfpathrectangle{\pgfqpoint{1.432000in}{0.528000in}}{\pgfqpoint{3.696000in}{3.696000in}} %
\pgfusepath{clip}%
\pgfsetbuttcap%
\pgfsetroundjoin%
\definecolor{currentfill}{rgb}{0.506271,0.828786,0.300362}%
\pgfsetfillcolor{currentfill}%
\pgfsetlinewidth{0.000000pt}%
\definecolor{currentstroke}{rgb}{0.000000,0.000000,0.000000}%
\pgfsetstrokecolor{currentstroke}%
\pgfsetdash{}{0pt}%
\pgfpathmoveto{\pgfqpoint{4.306541in}{3.077380in}}%
\pgfpathlineto{\pgfqpoint{4.009605in}{3.374316in}}%
\pgfpathlineto{\pgfqpoint{4.006469in}{3.364907in}}%
\pgfpathlineto{\pgfqpoint{3.984516in}{3.405677in}}%
\pgfpathlineto{\pgfqpoint{4.025286in}{3.383724in}}%
\pgfpathlineto{\pgfqpoint{4.015878in}{3.380588in}}%
\pgfpathlineto{\pgfqpoint{4.312814in}{3.083652in}}%
\pgfpathlineto{\pgfqpoint{4.306541in}{3.077380in}}%
\pgfusepath{fill}%
\end{pgfscope}%
\begin{pgfscope}%
\pgfpathrectangle{\pgfqpoint{1.432000in}{0.528000in}}{\pgfqpoint{3.696000in}{3.696000in}} %
\pgfusepath{clip}%
\pgfsetbuttcap%
\pgfsetroundjoin%
\definecolor{currentfill}{rgb}{0.271305,0.019942,0.347269}%
\pgfsetfillcolor{currentfill}%
\pgfsetlinewidth{0.000000pt}%
\definecolor{currentstroke}{rgb}{0.000000,0.000000,0.000000}%
\pgfsetstrokecolor{currentstroke}%
\pgfsetdash{}{0pt}%
\pgfpathmoveto{\pgfqpoint{4.305987in}{3.078056in}}%
\pgfpathlineto{\pgfqpoint{4.111355in}{3.370004in}}%
\pgfpathlineto{\pgfqpoint{4.106434in}{3.361394in}}%
\pgfpathlineto{\pgfqpoint{4.092903in}{3.405677in}}%
\pgfpathlineto{\pgfqpoint{4.128576in}{3.376155in}}%
\pgfpathlineto{\pgfqpoint{4.118735in}{3.374925in}}%
\pgfpathlineto{\pgfqpoint{4.313368in}{3.082976in}}%
\pgfpathlineto{\pgfqpoint{4.305987in}{3.078056in}}%
\pgfusepath{fill}%
\end{pgfscope}%
\begin{pgfscope}%
\pgfpathrectangle{\pgfqpoint{1.432000in}{0.528000in}}{\pgfqpoint{3.696000in}{3.696000in}} %
\pgfusepath{clip}%
\pgfsetbuttcap%
\pgfsetroundjoin%
\definecolor{currentfill}{rgb}{0.185783,0.704891,0.485273}%
\pgfsetfillcolor{currentfill}%
\pgfsetlinewidth{0.000000pt}%
\definecolor{currentstroke}{rgb}{0.000000,0.000000,0.000000}%
\pgfsetstrokecolor{currentstroke}%
\pgfsetdash{}{0pt}%
\pgfpathmoveto{\pgfqpoint{4.414928in}{3.077380in}}%
\pgfpathlineto{\pgfqpoint{4.117993in}{3.374316in}}%
\pgfpathlineto{\pgfqpoint{4.114856in}{3.364907in}}%
\pgfpathlineto{\pgfqpoint{4.092903in}{3.405677in}}%
\pgfpathlineto{\pgfqpoint{4.133673in}{3.383724in}}%
\pgfpathlineto{\pgfqpoint{4.124265in}{3.380588in}}%
\pgfpathlineto{\pgfqpoint{4.421201in}{3.083652in}}%
\pgfpathlineto{\pgfqpoint{4.414928in}{3.077380in}}%
\pgfusepath{fill}%
\end{pgfscope}%
\begin{pgfscope}%
\pgfpathrectangle{\pgfqpoint{1.432000in}{0.528000in}}{\pgfqpoint{3.696000in}{3.696000in}} %
\pgfusepath{clip}%
\pgfsetbuttcap%
\pgfsetroundjoin%
\definecolor{currentfill}{rgb}{0.151918,0.500685,0.557587}%
\pgfsetfillcolor{currentfill}%
\pgfsetlinewidth{0.000000pt}%
\definecolor{currentstroke}{rgb}{0.000000,0.000000,0.000000}%
\pgfsetstrokecolor{currentstroke}%
\pgfsetdash{}{0pt}%
\pgfpathmoveto{\pgfqpoint{4.414374in}{3.078056in}}%
\pgfpathlineto{\pgfqpoint{4.219742in}{3.370004in}}%
\pgfpathlineto{\pgfqpoint{4.214821in}{3.361394in}}%
\pgfpathlineto{\pgfqpoint{4.201290in}{3.405677in}}%
\pgfpathlineto{\pgfqpoint{4.236963in}{3.376155in}}%
\pgfpathlineto{\pgfqpoint{4.227122in}{3.374925in}}%
\pgfpathlineto{\pgfqpoint{4.421755in}{3.082976in}}%
\pgfpathlineto{\pgfqpoint{4.414374in}{3.078056in}}%
\pgfusepath{fill}%
\end{pgfscope}%
\begin{pgfscope}%
\pgfpathrectangle{\pgfqpoint{1.432000in}{0.528000in}}{\pgfqpoint{3.696000in}{3.696000in}} %
\pgfusepath{clip}%
\pgfsetbuttcap%
\pgfsetroundjoin%
\definecolor{currentfill}{rgb}{0.257322,0.256130,0.526563}%
\pgfsetfillcolor{currentfill}%
\pgfsetlinewidth{0.000000pt}%
\definecolor{currentstroke}{rgb}{0.000000,0.000000,0.000000}%
\pgfsetstrokecolor{currentstroke}%
\pgfsetdash{}{0pt}%
\pgfpathmoveto{\pgfqpoint{4.523315in}{3.077380in}}%
\pgfpathlineto{\pgfqpoint{4.226380in}{3.374316in}}%
\pgfpathlineto{\pgfqpoint{4.223243in}{3.364907in}}%
\pgfpathlineto{\pgfqpoint{4.201290in}{3.405677in}}%
\pgfpathlineto{\pgfqpoint{4.242060in}{3.383724in}}%
\pgfpathlineto{\pgfqpoint{4.232652in}{3.380588in}}%
\pgfpathlineto{\pgfqpoint{4.529588in}{3.083652in}}%
\pgfpathlineto{\pgfqpoint{4.523315in}{3.077380in}}%
\pgfusepath{fill}%
\end{pgfscope}%
\begin{pgfscope}%
\pgfpathrectangle{\pgfqpoint{1.432000in}{0.528000in}}{\pgfqpoint{3.696000in}{3.696000in}} %
\pgfusepath{clip}%
\pgfsetbuttcap%
\pgfsetroundjoin%
\definecolor{currentfill}{rgb}{0.585678,0.846661,0.249897}%
\pgfsetfillcolor{currentfill}%
\pgfsetlinewidth{0.000000pt}%
\definecolor{currentstroke}{rgb}{0.000000,0.000000,0.000000}%
\pgfsetstrokecolor{currentstroke}%
\pgfsetdash{}{0pt}%
\pgfpathmoveto{\pgfqpoint{4.522761in}{3.078056in}}%
\pgfpathlineto{\pgfqpoint{4.328129in}{3.370004in}}%
\pgfpathlineto{\pgfqpoint{4.323209in}{3.361394in}}%
\pgfpathlineto{\pgfqpoint{4.309677in}{3.405677in}}%
\pgfpathlineto{\pgfqpoint{4.345350in}{3.376155in}}%
\pgfpathlineto{\pgfqpoint{4.335510in}{3.374925in}}%
\pgfpathlineto{\pgfqpoint{4.530142in}{3.082976in}}%
\pgfpathlineto{\pgfqpoint{4.522761in}{3.078056in}}%
\pgfusepath{fill}%
\end{pgfscope}%
\begin{pgfscope}%
\pgfpathrectangle{\pgfqpoint{1.432000in}{0.528000in}}{\pgfqpoint{3.696000in}{3.696000in}} %
\pgfusepath{clip}%
\pgfsetbuttcap%
\pgfsetroundjoin%
\definecolor{currentfill}{rgb}{0.268510,0.009605,0.335427}%
\pgfsetfillcolor{currentfill}%
\pgfsetlinewidth{0.000000pt}%
\definecolor{currentstroke}{rgb}{0.000000,0.000000,0.000000}%
\pgfsetstrokecolor{currentstroke}%
\pgfsetdash{}{0pt}%
\pgfpathmoveto{\pgfqpoint{4.522244in}{3.079114in}}%
\pgfpathlineto{\pgfqpoint{4.426480in}{3.366406in}}%
\pgfpathlineto{\pgfqpoint{4.419467in}{3.359394in}}%
\pgfpathlineto{\pgfqpoint{4.418065in}{3.405677in}}%
\pgfpathlineto{\pgfqpoint{4.444713in}{3.367809in}}%
\pgfpathlineto{\pgfqpoint{4.434895in}{3.369212in}}%
\pgfpathlineto{\pgfqpoint{4.530659in}{3.081919in}}%
\pgfpathlineto{\pgfqpoint{4.522244in}{3.079114in}}%
\pgfusepath{fill}%
\end{pgfscope}%
\begin{pgfscope}%
\pgfpathrectangle{\pgfqpoint{1.432000in}{0.528000in}}{\pgfqpoint{3.696000in}{3.696000in}} %
\pgfusepath{clip}%
\pgfsetbuttcap%
\pgfsetroundjoin%
\definecolor{currentfill}{rgb}{0.344074,0.780029,0.397381}%
\pgfsetfillcolor{currentfill}%
\pgfsetlinewidth{0.000000pt}%
\definecolor{currentstroke}{rgb}{0.000000,0.000000,0.000000}%
\pgfsetstrokecolor{currentstroke}%
\pgfsetdash{}{0pt}%
\pgfpathmoveto{\pgfqpoint{4.631148in}{3.078056in}}%
\pgfpathlineto{\pgfqpoint{4.436516in}{3.370004in}}%
\pgfpathlineto{\pgfqpoint{4.431596in}{3.361394in}}%
\pgfpathlineto{\pgfqpoint{4.418065in}{3.405677in}}%
\pgfpathlineto{\pgfqpoint{4.453738in}{3.376155in}}%
\pgfpathlineto{\pgfqpoint{4.443897in}{3.374925in}}%
\pgfpathlineto{\pgfqpoint{4.638529in}{3.082976in}}%
\pgfpathlineto{\pgfqpoint{4.631148in}{3.078056in}}%
\pgfusepath{fill}%
\end{pgfscope}%
\begin{pgfscope}%
\pgfpathrectangle{\pgfqpoint{1.432000in}{0.528000in}}{\pgfqpoint{3.696000in}{3.696000in}} %
\pgfusepath{clip}%
\pgfsetbuttcap%
\pgfsetroundjoin%
\definecolor{currentfill}{rgb}{0.129933,0.559582,0.551864}%
\pgfsetfillcolor{currentfill}%
\pgfsetlinewidth{0.000000pt}%
\definecolor{currentstroke}{rgb}{0.000000,0.000000,0.000000}%
\pgfsetstrokecolor{currentstroke}%
\pgfsetdash{}{0pt}%
\pgfpathmoveto{\pgfqpoint{4.630631in}{3.079114in}}%
\pgfpathlineto{\pgfqpoint{4.534867in}{3.366406in}}%
\pgfpathlineto{\pgfqpoint{4.527854in}{3.359394in}}%
\pgfpathlineto{\pgfqpoint{4.526452in}{3.405677in}}%
\pgfpathlineto{\pgfqpoint{4.553100in}{3.367809in}}%
\pgfpathlineto{\pgfqpoint{4.543282in}{3.369212in}}%
\pgfpathlineto{\pgfqpoint{4.639046in}{3.081919in}}%
\pgfpathlineto{\pgfqpoint{4.630631in}{3.079114in}}%
\pgfusepath{fill}%
\end{pgfscope}%
\begin{pgfscope}%
\pgfpathrectangle{\pgfqpoint{1.432000in}{0.528000in}}{\pgfqpoint{3.696000in}{3.696000in}} %
\pgfusepath{clip}%
\pgfsetbuttcap%
\pgfsetroundjoin%
\definecolor{currentfill}{rgb}{0.246811,0.283237,0.535941}%
\pgfsetfillcolor{currentfill}%
\pgfsetlinewidth{0.000000pt}%
\definecolor{currentstroke}{rgb}{0.000000,0.000000,0.000000}%
\pgfsetstrokecolor{currentstroke}%
\pgfsetdash{}{0pt}%
\pgfpathmoveto{\pgfqpoint{4.739535in}{3.078056in}}%
\pgfpathlineto{\pgfqpoint{4.544903in}{3.370004in}}%
\pgfpathlineto{\pgfqpoint{4.539983in}{3.361394in}}%
\pgfpathlineto{\pgfqpoint{4.526452in}{3.405677in}}%
\pgfpathlineto{\pgfqpoint{4.562125in}{3.376155in}}%
\pgfpathlineto{\pgfqpoint{4.552284in}{3.374925in}}%
\pgfpathlineto{\pgfqpoint{4.746916in}{3.082976in}}%
\pgfpathlineto{\pgfqpoint{4.739535in}{3.078056in}}%
\pgfusepath{fill}%
\end{pgfscope}%
\begin{pgfscope}%
\pgfpathrectangle{\pgfqpoint{1.432000in}{0.528000in}}{\pgfqpoint{3.696000in}{3.696000in}} %
\pgfusepath{clip}%
\pgfsetbuttcap%
\pgfsetroundjoin%
\definecolor{currentfill}{rgb}{0.720391,0.870350,0.162603}%
\pgfsetfillcolor{currentfill}%
\pgfsetlinewidth{0.000000pt}%
\definecolor{currentstroke}{rgb}{0.000000,0.000000,0.000000}%
\pgfsetstrokecolor{currentstroke}%
\pgfsetdash{}{0pt}%
\pgfpathmoveto{\pgfqpoint{4.739018in}{3.079114in}}%
\pgfpathlineto{\pgfqpoint{4.643254in}{3.366406in}}%
\pgfpathlineto{\pgfqpoint{4.636241in}{3.359394in}}%
\pgfpathlineto{\pgfqpoint{4.634839in}{3.405677in}}%
\pgfpathlineto{\pgfqpoint{4.661487in}{3.367809in}}%
\pgfpathlineto{\pgfqpoint{4.651669in}{3.369212in}}%
\pgfpathlineto{\pgfqpoint{4.747433in}{3.081919in}}%
\pgfpathlineto{\pgfqpoint{4.739018in}{3.079114in}}%
\pgfusepath{fill}%
\end{pgfscope}%
\begin{pgfscope}%
\pgfpathrectangle{\pgfqpoint{1.432000in}{0.528000in}}{\pgfqpoint{3.696000in}{3.696000in}} %
\pgfusepath{clip}%
\pgfsetbuttcap%
\pgfsetroundjoin%
\definecolor{currentfill}{rgb}{0.129933,0.559582,0.551864}%
\pgfsetfillcolor{currentfill}%
\pgfsetlinewidth{0.000000pt}%
\definecolor{currentstroke}{rgb}{0.000000,0.000000,0.000000}%
\pgfsetstrokecolor{currentstroke}%
\pgfsetdash{}{0pt}%
\pgfpathmoveto{\pgfqpoint{4.847405in}{3.079114in}}%
\pgfpathlineto{\pgfqpoint{4.751641in}{3.366406in}}%
\pgfpathlineto{\pgfqpoint{4.744628in}{3.359394in}}%
\pgfpathlineto{\pgfqpoint{4.743226in}{3.405677in}}%
\pgfpathlineto{\pgfqpoint{4.769874in}{3.367809in}}%
\pgfpathlineto{\pgfqpoint{4.760056in}{3.369212in}}%
\pgfpathlineto{\pgfqpoint{4.855821in}{3.081919in}}%
\pgfpathlineto{\pgfqpoint{4.847405in}{3.079114in}}%
\pgfusepath{fill}%
\end{pgfscope}%
\begin{pgfscope}%
\pgfpathrectangle{\pgfqpoint{1.432000in}{0.528000in}}{\pgfqpoint{3.696000in}{3.696000in}} %
\pgfusepath{clip}%
\pgfsetbuttcap%
\pgfsetroundjoin%
\definecolor{currentfill}{rgb}{0.244972,0.287675,0.537260}%
\pgfsetfillcolor{currentfill}%
\pgfsetlinewidth{0.000000pt}%
\definecolor{currentstroke}{rgb}{0.000000,0.000000,0.000000}%
\pgfsetstrokecolor{currentstroke}%
\pgfsetdash{}{0pt}%
\pgfpathmoveto{\pgfqpoint{4.847178in}{3.080516in}}%
\pgfpathlineto{\pgfqpoint{4.847178in}{3.365761in}}%
\pgfpathlineto{\pgfqpoint{4.838307in}{3.361325in}}%
\pgfpathlineto{\pgfqpoint{4.851613in}{3.405677in}}%
\pgfpathlineto{\pgfqpoint{4.864919in}{3.361325in}}%
\pgfpathlineto{\pgfqpoint{4.856048in}{3.365761in}}%
\pgfpathlineto{\pgfqpoint{4.856048in}{3.080516in}}%
\pgfpathlineto{\pgfqpoint{4.847178in}{3.080516in}}%
\pgfusepath{fill}%
\end{pgfscope}%
\begin{pgfscope}%
\pgfpathrectangle{\pgfqpoint{1.432000in}{0.528000in}}{\pgfqpoint{3.696000in}{3.696000in}} %
\pgfusepath{clip}%
\pgfsetbuttcap%
\pgfsetroundjoin%
\definecolor{currentfill}{rgb}{0.283229,0.120777,0.440584}%
\pgfsetfillcolor{currentfill}%
\pgfsetlinewidth{0.000000pt}%
\definecolor{currentstroke}{rgb}{0.000000,0.000000,0.000000}%
\pgfsetstrokecolor{currentstroke}%
\pgfsetdash{}{0pt}%
\pgfpathmoveto{\pgfqpoint{4.847310in}{3.079440in}}%
\pgfpathlineto{\pgfqpoint{4.748604in}{3.474264in}}%
\pgfpathlineto{\pgfqpoint{4.741074in}{3.467810in}}%
\pgfpathlineto{\pgfqpoint{4.743226in}{3.514065in}}%
\pgfpathlineto{\pgfqpoint{4.766891in}{3.474264in}}%
\pgfpathlineto{\pgfqpoint{4.757210in}{3.476415in}}%
\pgfpathlineto{\pgfqpoint{4.855916in}{3.081592in}}%
\pgfpathlineto{\pgfqpoint{4.847310in}{3.079440in}}%
\pgfusepath{fill}%
\end{pgfscope}%
\begin{pgfscope}%
\pgfpathrectangle{\pgfqpoint{1.432000in}{0.528000in}}{\pgfqpoint{3.696000in}{3.696000in}} %
\pgfusepath{clip}%
\pgfsetbuttcap%
\pgfsetroundjoin%
\definecolor{currentfill}{rgb}{0.277941,0.056324,0.381191}%
\pgfsetfillcolor{currentfill}%
\pgfsetlinewidth{0.000000pt}%
\definecolor{currentstroke}{rgb}{0.000000,0.000000,0.000000}%
\pgfsetstrokecolor{currentstroke}%
\pgfsetdash{}{0pt}%
\pgfpathmoveto{\pgfqpoint{4.847178in}{3.080516in}}%
\pgfpathlineto{\pgfqpoint{4.847178in}{3.474148in}}%
\pgfpathlineto{\pgfqpoint{4.838307in}{3.469713in}}%
\pgfpathlineto{\pgfqpoint{4.851613in}{3.514065in}}%
\pgfpathlineto{\pgfqpoint{4.864919in}{3.469713in}}%
\pgfpathlineto{\pgfqpoint{4.856048in}{3.474148in}}%
\pgfpathlineto{\pgfqpoint{4.856048in}{3.080516in}}%
\pgfpathlineto{\pgfqpoint{4.847178in}{3.080516in}}%
\pgfusepath{fill}%
\end{pgfscope}%
\begin{pgfscope}%
\pgfpathrectangle{\pgfqpoint{1.432000in}{0.528000in}}{\pgfqpoint{3.696000in}{3.696000in}} %
\pgfusepath{clip}%
\pgfsetbuttcap%
\pgfsetroundjoin%
\definecolor{currentfill}{rgb}{0.283229,0.120777,0.440584}%
\pgfsetfillcolor{currentfill}%
\pgfsetlinewidth{0.000000pt}%
\definecolor{currentstroke}{rgb}{0.000000,0.000000,0.000000}%
\pgfsetstrokecolor{currentstroke}%
\pgfsetdash{}{0pt}%
\pgfpathmoveto{\pgfqpoint{4.955792in}{3.079114in}}%
\pgfpathlineto{\pgfqpoint{4.860028in}{3.366406in}}%
\pgfpathlineto{\pgfqpoint{4.853015in}{3.359394in}}%
\pgfpathlineto{\pgfqpoint{4.851613in}{3.405677in}}%
\pgfpathlineto{\pgfqpoint{4.878261in}{3.367809in}}%
\pgfpathlineto{\pgfqpoint{4.868443in}{3.369212in}}%
\pgfpathlineto{\pgfqpoint{4.964208in}{3.081919in}}%
\pgfpathlineto{\pgfqpoint{4.955792in}{3.079114in}}%
\pgfusepath{fill}%
\end{pgfscope}%
\begin{pgfscope}%
\pgfpathrectangle{\pgfqpoint{1.432000in}{0.528000in}}{\pgfqpoint{3.696000in}{3.696000in}} %
\pgfusepath{clip}%
\pgfsetbuttcap%
\pgfsetroundjoin%
\definecolor{currentfill}{rgb}{0.159194,0.482237,0.558073}%
\pgfsetfillcolor{currentfill}%
\pgfsetlinewidth{0.000000pt}%
\definecolor{currentstroke}{rgb}{0.000000,0.000000,0.000000}%
\pgfsetstrokecolor{currentstroke}%
\pgfsetdash{}{0pt}%
\pgfpathmoveto{\pgfqpoint{4.955565in}{3.080516in}}%
\pgfpathlineto{\pgfqpoint{4.955565in}{3.365761in}}%
\pgfpathlineto{\pgfqpoint{4.946694in}{3.361325in}}%
\pgfpathlineto{\pgfqpoint{4.960000in}{3.405677in}}%
\pgfpathlineto{\pgfqpoint{4.973306in}{3.361325in}}%
\pgfpathlineto{\pgfqpoint{4.964435in}{3.365761in}}%
\pgfpathlineto{\pgfqpoint{4.964435in}{3.080516in}}%
\pgfpathlineto{\pgfqpoint{4.955565in}{3.080516in}}%
\pgfusepath{fill}%
\end{pgfscope}%
\begin{pgfscope}%
\pgfpathrectangle{\pgfqpoint{1.432000in}{0.528000in}}{\pgfqpoint{3.696000in}{3.696000in}} %
\pgfusepath{clip}%
\pgfsetbuttcap%
\pgfsetroundjoin%
\definecolor{currentfill}{rgb}{0.269944,0.014625,0.341379}%
\pgfsetfillcolor{currentfill}%
\pgfsetlinewidth{0.000000pt}%
\definecolor{currentstroke}{rgb}{0.000000,0.000000,0.000000}%
\pgfsetstrokecolor{currentstroke}%
\pgfsetdash{}{0pt}%
\pgfpathmoveto{\pgfqpoint{4.955697in}{3.079440in}}%
\pgfpathlineto{\pgfqpoint{4.856991in}{3.474264in}}%
\pgfpathlineto{\pgfqpoint{4.849462in}{3.467810in}}%
\pgfpathlineto{\pgfqpoint{4.851613in}{3.514065in}}%
\pgfpathlineto{\pgfqpoint{4.875278in}{3.474264in}}%
\pgfpathlineto{\pgfqpoint{4.865597in}{3.476415in}}%
\pgfpathlineto{\pgfqpoint{4.964303in}{3.081592in}}%
\pgfpathlineto{\pgfqpoint{4.955697in}{3.079440in}}%
\pgfusepath{fill}%
\end{pgfscope}%
\begin{pgfscope}%
\pgfpathrectangle{\pgfqpoint{1.432000in}{0.528000in}}{\pgfqpoint{3.696000in}{3.696000in}} %
\pgfusepath{clip}%
\pgfsetbuttcap%
\pgfsetroundjoin%
\definecolor{currentfill}{rgb}{0.239346,0.300855,0.540844}%
\pgfsetfillcolor{currentfill}%
\pgfsetlinewidth{0.000000pt}%
\definecolor{currentstroke}{rgb}{0.000000,0.000000,0.000000}%
\pgfsetstrokecolor{currentstroke}%
\pgfsetdash{}{0pt}%
\pgfpathmoveto{\pgfqpoint{4.955565in}{3.080516in}}%
\pgfpathlineto{\pgfqpoint{4.955565in}{3.474148in}}%
\pgfpathlineto{\pgfqpoint{4.946694in}{3.469713in}}%
\pgfpathlineto{\pgfqpoint{4.960000in}{3.514065in}}%
\pgfpathlineto{\pgfqpoint{4.973306in}{3.469713in}}%
\pgfpathlineto{\pgfqpoint{4.964435in}{3.474148in}}%
\pgfpathlineto{\pgfqpoint{4.964435in}{3.080516in}}%
\pgfpathlineto{\pgfqpoint{4.955565in}{3.080516in}}%
\pgfusepath{fill}%
\end{pgfscope}%
\begin{pgfscope}%
\pgfpathrectangle{\pgfqpoint{1.432000in}{0.528000in}}{\pgfqpoint{3.696000in}{3.696000in}} %
\pgfusepath{clip}%
\pgfsetbuttcap%
\pgfsetroundjoin%
\definecolor{currentfill}{rgb}{0.183898,0.422383,0.556944}%
\pgfsetfillcolor{currentfill}%
\pgfsetlinewidth{0.000000pt}%
\definecolor{currentstroke}{rgb}{0.000000,0.000000,0.000000}%
\pgfsetstrokecolor{currentstroke}%
\pgfsetdash{}{0pt}%
\pgfpathmoveto{\pgfqpoint{1.604435in}{3.188903in}}%
\pgfpathlineto{\pgfqpoint{1.604435in}{3.120433in}}%
\pgfpathlineto{\pgfqpoint{1.613306in}{3.124868in}}%
\pgfpathlineto{\pgfqpoint{1.600000in}{3.080516in}}%
\pgfpathlineto{\pgfqpoint{1.586694in}{3.124868in}}%
\pgfpathlineto{\pgfqpoint{1.595565in}{3.120433in}}%
\pgfpathlineto{\pgfqpoint{1.595565in}{3.188903in}}%
\pgfpathlineto{\pgfqpoint{1.604435in}{3.188903in}}%
\pgfusepath{fill}%
\end{pgfscope}%
\begin{pgfscope}%
\pgfpathrectangle{\pgfqpoint{1.432000in}{0.528000in}}{\pgfqpoint{3.696000in}{3.696000in}} %
\pgfusepath{clip}%
\pgfsetbuttcap%
\pgfsetroundjoin%
\definecolor{currentfill}{rgb}{0.276194,0.190074,0.493001}%
\pgfsetfillcolor{currentfill}%
\pgfsetlinewidth{0.000000pt}%
\definecolor{currentstroke}{rgb}{0.000000,0.000000,0.000000}%
\pgfsetstrokecolor{currentstroke}%
\pgfsetdash{}{0pt}%
\pgfpathmoveto{\pgfqpoint{1.603136in}{3.192039in}}%
\pgfpathlineto{\pgfqpoint{1.683298in}{3.111878in}}%
\pgfpathlineto{\pgfqpoint{1.686434in}{3.121286in}}%
\pgfpathlineto{\pgfqpoint{1.708387in}{3.080516in}}%
\pgfpathlineto{\pgfqpoint{1.667617in}{3.102469in}}%
\pgfpathlineto{\pgfqpoint{1.677025in}{3.105605in}}%
\pgfpathlineto{\pgfqpoint{1.596864in}{3.185767in}}%
\pgfpathlineto{\pgfqpoint{1.603136in}{3.192039in}}%
\pgfusepath{fill}%
\end{pgfscope}%
\begin{pgfscope}%
\pgfpathrectangle{\pgfqpoint{1.432000in}{0.528000in}}{\pgfqpoint{3.696000in}{3.696000in}} %
\pgfusepath{clip}%
\pgfsetbuttcap%
\pgfsetroundjoin%
\definecolor{currentfill}{rgb}{0.250425,0.274290,0.533103}%
\pgfsetfillcolor{currentfill}%
\pgfsetlinewidth{0.000000pt}%
\definecolor{currentstroke}{rgb}{0.000000,0.000000,0.000000}%
\pgfsetstrokecolor{currentstroke}%
\pgfsetdash{}{0pt}%
\pgfpathmoveto{\pgfqpoint{1.712822in}{3.188903in}}%
\pgfpathlineto{\pgfqpoint{1.712822in}{3.120433in}}%
\pgfpathlineto{\pgfqpoint{1.721693in}{3.124868in}}%
\pgfpathlineto{\pgfqpoint{1.708387in}{3.080516in}}%
\pgfpathlineto{\pgfqpoint{1.695081in}{3.124868in}}%
\pgfpathlineto{\pgfqpoint{1.703952in}{3.120433in}}%
\pgfpathlineto{\pgfqpoint{1.703952in}{3.188903in}}%
\pgfpathlineto{\pgfqpoint{1.712822in}{3.188903in}}%
\pgfusepath{fill}%
\end{pgfscope}%
\begin{pgfscope}%
\pgfpathrectangle{\pgfqpoint{1.432000in}{0.528000in}}{\pgfqpoint{3.696000in}{3.696000in}} %
\pgfusepath{clip}%
\pgfsetbuttcap%
\pgfsetroundjoin%
\definecolor{currentfill}{rgb}{0.282327,0.094955,0.417331}%
\pgfsetfillcolor{currentfill}%
\pgfsetlinewidth{0.000000pt}%
\definecolor{currentstroke}{rgb}{0.000000,0.000000,0.000000}%
\pgfsetstrokecolor{currentstroke}%
\pgfsetdash{}{0pt}%
\pgfpathmoveto{\pgfqpoint{1.711523in}{3.192039in}}%
\pgfpathlineto{\pgfqpoint{1.791685in}{3.111878in}}%
\pgfpathlineto{\pgfqpoint{1.794821in}{3.121286in}}%
\pgfpathlineto{\pgfqpoint{1.816774in}{3.080516in}}%
\pgfpathlineto{\pgfqpoint{1.776004in}{3.102469in}}%
\pgfpathlineto{\pgfqpoint{1.785413in}{3.105605in}}%
\pgfpathlineto{\pgfqpoint{1.705251in}{3.185767in}}%
\pgfpathlineto{\pgfqpoint{1.711523in}{3.192039in}}%
\pgfusepath{fill}%
\end{pgfscope}%
\begin{pgfscope}%
\pgfpathrectangle{\pgfqpoint{1.432000in}{0.528000in}}{\pgfqpoint{3.696000in}{3.696000in}} %
\pgfusepath{clip}%
\pgfsetbuttcap%
\pgfsetroundjoin%
\definecolor{currentfill}{rgb}{0.216210,0.351535,0.550627}%
\pgfsetfillcolor{currentfill}%
\pgfsetlinewidth{0.000000pt}%
\definecolor{currentstroke}{rgb}{0.000000,0.000000,0.000000}%
\pgfsetstrokecolor{currentstroke}%
\pgfsetdash{}{0pt}%
\pgfpathmoveto{\pgfqpoint{1.821209in}{3.188903in}}%
\pgfpathlineto{\pgfqpoint{1.821209in}{3.120433in}}%
\pgfpathlineto{\pgfqpoint{1.830080in}{3.124868in}}%
\pgfpathlineto{\pgfqpoint{1.816774in}{3.080516in}}%
\pgfpathlineto{\pgfqpoint{1.803469in}{3.124868in}}%
\pgfpathlineto{\pgfqpoint{1.812339in}{3.120433in}}%
\pgfpathlineto{\pgfqpoint{1.812339in}{3.188903in}}%
\pgfpathlineto{\pgfqpoint{1.821209in}{3.188903in}}%
\pgfusepath{fill}%
\end{pgfscope}%
\begin{pgfscope}%
\pgfpathrectangle{\pgfqpoint{1.432000in}{0.528000in}}{\pgfqpoint{3.696000in}{3.696000in}} %
\pgfusepath{clip}%
\pgfsetbuttcap%
\pgfsetroundjoin%
\definecolor{currentfill}{rgb}{0.175841,0.441290,0.557685}%
\pgfsetfillcolor{currentfill}%
\pgfsetlinewidth{0.000000pt}%
\definecolor{currentstroke}{rgb}{0.000000,0.000000,0.000000}%
\pgfsetstrokecolor{currentstroke}%
\pgfsetdash{}{0pt}%
\pgfpathmoveto{\pgfqpoint{1.929596in}{3.188903in}}%
\pgfpathlineto{\pgfqpoint{1.929596in}{3.120433in}}%
\pgfpathlineto{\pgfqpoint{1.938467in}{3.124868in}}%
\pgfpathlineto{\pgfqpoint{1.925161in}{3.080516in}}%
\pgfpathlineto{\pgfqpoint{1.911856in}{3.124868in}}%
\pgfpathlineto{\pgfqpoint{1.920726in}{3.120433in}}%
\pgfpathlineto{\pgfqpoint{1.920726in}{3.188903in}}%
\pgfpathlineto{\pgfqpoint{1.929596in}{3.188903in}}%
\pgfusepath{fill}%
\end{pgfscope}%
\begin{pgfscope}%
\pgfpathrectangle{\pgfqpoint{1.432000in}{0.528000in}}{\pgfqpoint{3.696000in}{3.696000in}} %
\pgfusepath{clip}%
\pgfsetbuttcap%
\pgfsetroundjoin%
\definecolor{currentfill}{rgb}{0.229739,0.322361,0.545706}%
\pgfsetfillcolor{currentfill}%
\pgfsetlinewidth{0.000000pt}%
\definecolor{currentstroke}{rgb}{0.000000,0.000000,0.000000}%
\pgfsetstrokecolor{currentstroke}%
\pgfsetdash{}{0pt}%
\pgfpathmoveto{\pgfqpoint{2.037984in}{3.188903in}}%
\pgfpathlineto{\pgfqpoint{2.037984in}{3.120433in}}%
\pgfpathlineto{\pgfqpoint{2.046854in}{3.124868in}}%
\pgfpathlineto{\pgfqpoint{2.033548in}{3.080516in}}%
\pgfpathlineto{\pgfqpoint{2.020243in}{3.124868in}}%
\pgfpathlineto{\pgfqpoint{2.029113in}{3.120433in}}%
\pgfpathlineto{\pgfqpoint{2.029113in}{3.188903in}}%
\pgfpathlineto{\pgfqpoint{2.037984in}{3.188903in}}%
\pgfusepath{fill}%
\end{pgfscope}%
\begin{pgfscope}%
\pgfpathrectangle{\pgfqpoint{1.432000in}{0.528000in}}{\pgfqpoint{3.696000in}{3.696000in}} %
\pgfusepath{clip}%
\pgfsetbuttcap%
\pgfsetroundjoin%
\definecolor{currentfill}{rgb}{0.271305,0.019942,0.347269}%
\pgfsetfillcolor{currentfill}%
\pgfsetlinewidth{0.000000pt}%
\definecolor{currentstroke}{rgb}{0.000000,0.000000,0.000000}%
\pgfsetstrokecolor{currentstroke}%
\pgfsetdash{}{0pt}%
\pgfpathmoveto{\pgfqpoint{2.037984in}{3.188903in}}%
\pgfpathlineto{\pgfqpoint{2.035766in}{3.192744in}}%
\pgfpathlineto{\pgfqpoint{2.031331in}{3.192744in}}%
\pgfpathlineto{\pgfqpoint{2.029113in}{3.188903in}}%
\pgfpathlineto{\pgfqpoint{2.031331in}{3.185062in}}%
\pgfpathlineto{\pgfqpoint{2.035766in}{3.185062in}}%
\pgfpathlineto{\pgfqpoint{2.037984in}{3.188903in}}%
\pgfpathlineto{\pgfqpoint{2.035766in}{3.192744in}}%
\pgfusepath{fill}%
\end{pgfscope}%
\begin{pgfscope}%
\pgfpathrectangle{\pgfqpoint{1.432000in}{0.528000in}}{\pgfqpoint{3.696000in}{3.696000in}} %
\pgfusepath{clip}%
\pgfsetbuttcap%
\pgfsetroundjoin%
\definecolor{currentfill}{rgb}{0.283091,0.110553,0.431554}%
\pgfsetfillcolor{currentfill}%
\pgfsetlinewidth{0.000000pt}%
\definecolor{currentstroke}{rgb}{0.000000,0.000000,0.000000}%
\pgfsetstrokecolor{currentstroke}%
\pgfsetdash{}{0pt}%
\pgfpathmoveto{\pgfqpoint{2.145072in}{3.185767in}}%
\pgfpathlineto{\pgfqpoint{2.064910in}{3.105605in}}%
\pgfpathlineto{\pgfqpoint{2.074318in}{3.102469in}}%
\pgfpathlineto{\pgfqpoint{2.033548in}{3.080516in}}%
\pgfpathlineto{\pgfqpoint{2.055502in}{3.121286in}}%
\pgfpathlineto{\pgfqpoint{2.058638in}{3.111878in}}%
\pgfpathlineto{\pgfqpoint{2.138799in}{3.192039in}}%
\pgfpathlineto{\pgfqpoint{2.145072in}{3.185767in}}%
\pgfusepath{fill}%
\end{pgfscope}%
\begin{pgfscope}%
\pgfpathrectangle{\pgfqpoint{1.432000in}{0.528000in}}{\pgfqpoint{3.696000in}{3.696000in}} %
\pgfusepath{clip}%
\pgfsetbuttcap%
\pgfsetroundjoin%
\definecolor{currentfill}{rgb}{0.282290,0.145912,0.461510}%
\pgfsetfillcolor{currentfill}%
\pgfsetlinewidth{0.000000pt}%
\definecolor{currentstroke}{rgb}{0.000000,0.000000,0.000000}%
\pgfsetstrokecolor{currentstroke}%
\pgfsetdash{}{0pt}%
\pgfpathmoveto{\pgfqpoint{2.146371in}{3.188903in}}%
\pgfpathlineto{\pgfqpoint{2.146371in}{3.120433in}}%
\pgfpathlineto{\pgfqpoint{2.155241in}{3.124868in}}%
\pgfpathlineto{\pgfqpoint{2.141935in}{3.080516in}}%
\pgfpathlineto{\pgfqpoint{2.128630in}{3.124868in}}%
\pgfpathlineto{\pgfqpoint{2.137500in}{3.120433in}}%
\pgfpathlineto{\pgfqpoint{2.137500in}{3.188903in}}%
\pgfpathlineto{\pgfqpoint{2.146371in}{3.188903in}}%
\pgfusepath{fill}%
\end{pgfscope}%
\begin{pgfscope}%
\pgfpathrectangle{\pgfqpoint{1.432000in}{0.528000in}}{\pgfqpoint{3.696000in}{3.696000in}} %
\pgfusepath{clip}%
\pgfsetbuttcap%
\pgfsetroundjoin%
\definecolor{currentfill}{rgb}{0.269944,0.014625,0.341379}%
\pgfsetfillcolor{currentfill}%
\pgfsetlinewidth{0.000000pt}%
\definecolor{currentstroke}{rgb}{0.000000,0.000000,0.000000}%
\pgfsetstrokecolor{currentstroke}%
\pgfsetdash{}{0pt}%
\pgfpathmoveto{\pgfqpoint{2.146371in}{3.188903in}}%
\pgfpathlineto{\pgfqpoint{2.144153in}{3.192744in}}%
\pgfpathlineto{\pgfqpoint{2.139718in}{3.192744in}}%
\pgfpathlineto{\pgfqpoint{2.137500in}{3.188903in}}%
\pgfpathlineto{\pgfqpoint{2.139718in}{3.185062in}}%
\pgfpathlineto{\pgfqpoint{2.144153in}{3.185062in}}%
\pgfpathlineto{\pgfqpoint{2.146371in}{3.188903in}}%
\pgfpathlineto{\pgfqpoint{2.144153in}{3.192744in}}%
\pgfusepath{fill}%
\end{pgfscope}%
\begin{pgfscope}%
\pgfpathrectangle{\pgfqpoint{1.432000in}{0.528000in}}{\pgfqpoint{3.696000in}{3.696000in}} %
\pgfusepath{clip}%
\pgfsetbuttcap%
\pgfsetroundjoin%
\definecolor{currentfill}{rgb}{0.260571,0.246922,0.522828}%
\pgfsetfillcolor{currentfill}%
\pgfsetlinewidth{0.000000pt}%
\definecolor{currentstroke}{rgb}{0.000000,0.000000,0.000000}%
\pgfsetstrokecolor{currentstroke}%
\pgfsetdash{}{0pt}%
\pgfpathmoveto{\pgfqpoint{2.253459in}{3.185767in}}%
\pgfpathlineto{\pgfqpoint{2.173297in}{3.105605in}}%
\pgfpathlineto{\pgfqpoint{2.182706in}{3.102469in}}%
\pgfpathlineto{\pgfqpoint{2.141935in}{3.080516in}}%
\pgfpathlineto{\pgfqpoint{2.163889in}{3.121286in}}%
\pgfpathlineto{\pgfqpoint{2.167025in}{3.111878in}}%
\pgfpathlineto{\pgfqpoint{2.247186in}{3.192039in}}%
\pgfpathlineto{\pgfqpoint{2.253459in}{3.185767in}}%
\pgfusepath{fill}%
\end{pgfscope}%
\begin{pgfscope}%
\pgfpathrectangle{\pgfqpoint{1.432000in}{0.528000in}}{\pgfqpoint{3.696000in}{3.696000in}} %
\pgfusepath{clip}%
\pgfsetbuttcap%
\pgfsetroundjoin%
\definecolor{currentfill}{rgb}{0.271305,0.019942,0.347269}%
\pgfsetfillcolor{currentfill}%
\pgfsetlinewidth{0.000000pt}%
\definecolor{currentstroke}{rgb}{0.000000,0.000000,0.000000}%
\pgfsetstrokecolor{currentstroke}%
\pgfsetdash{}{0pt}%
\pgfpathmoveto{\pgfqpoint{2.254758in}{3.188903in}}%
\pgfpathlineto{\pgfqpoint{2.254758in}{3.120433in}}%
\pgfpathlineto{\pgfqpoint{2.263628in}{3.124868in}}%
\pgfpathlineto{\pgfqpoint{2.250323in}{3.080516in}}%
\pgfpathlineto{\pgfqpoint{2.237017in}{3.124868in}}%
\pgfpathlineto{\pgfqpoint{2.245887in}{3.120433in}}%
\pgfpathlineto{\pgfqpoint{2.245887in}{3.188903in}}%
\pgfpathlineto{\pgfqpoint{2.254758in}{3.188903in}}%
\pgfusepath{fill}%
\end{pgfscope}%
\begin{pgfscope}%
\pgfpathrectangle{\pgfqpoint{1.432000in}{0.528000in}}{\pgfqpoint{3.696000in}{3.696000in}} %
\pgfusepath{clip}%
\pgfsetbuttcap%
\pgfsetroundjoin%
\definecolor{currentfill}{rgb}{0.277018,0.050344,0.375715}%
\pgfsetfillcolor{currentfill}%
\pgfsetlinewidth{0.000000pt}%
\definecolor{currentstroke}{rgb}{0.000000,0.000000,0.000000}%
\pgfsetstrokecolor{currentstroke}%
\pgfsetdash{}{0pt}%
\pgfpathmoveto{\pgfqpoint{2.250323in}{3.184468in}}%
\pgfpathlineto{\pgfqpoint{2.181852in}{3.184468in}}%
\pgfpathlineto{\pgfqpoint{2.186287in}{3.175598in}}%
\pgfpathlineto{\pgfqpoint{2.141935in}{3.188903in}}%
\pgfpathlineto{\pgfqpoint{2.186287in}{3.202209in}}%
\pgfpathlineto{\pgfqpoint{2.181852in}{3.193338in}}%
\pgfpathlineto{\pgfqpoint{2.250323in}{3.193338in}}%
\pgfpathlineto{\pgfqpoint{2.250323in}{3.184468in}}%
\pgfusepath{fill}%
\end{pgfscope}%
\begin{pgfscope}%
\pgfpathrectangle{\pgfqpoint{1.432000in}{0.528000in}}{\pgfqpoint{3.696000in}{3.696000in}} %
\pgfusepath{clip}%
\pgfsetbuttcap%
\pgfsetroundjoin%
\definecolor{currentfill}{rgb}{0.272594,0.025563,0.353093}%
\pgfsetfillcolor{currentfill}%
\pgfsetlinewidth{0.000000pt}%
\definecolor{currentstroke}{rgb}{0.000000,0.000000,0.000000}%
\pgfsetstrokecolor{currentstroke}%
\pgfsetdash{}{0pt}%
\pgfpathmoveto{\pgfqpoint{2.254758in}{3.188903in}}%
\pgfpathlineto{\pgfqpoint{2.252540in}{3.192744in}}%
\pgfpathlineto{\pgfqpoint{2.248105in}{3.192744in}}%
\pgfpathlineto{\pgfqpoint{2.245887in}{3.188903in}}%
\pgfpathlineto{\pgfqpoint{2.248105in}{3.185062in}}%
\pgfpathlineto{\pgfqpoint{2.252540in}{3.185062in}}%
\pgfpathlineto{\pgfqpoint{2.254758in}{3.188903in}}%
\pgfpathlineto{\pgfqpoint{2.252540in}{3.192744in}}%
\pgfusepath{fill}%
\end{pgfscope}%
\begin{pgfscope}%
\pgfpathrectangle{\pgfqpoint{1.432000in}{0.528000in}}{\pgfqpoint{3.696000in}{3.696000in}} %
\pgfusepath{clip}%
\pgfsetbuttcap%
\pgfsetroundjoin%
\definecolor{currentfill}{rgb}{0.281924,0.089666,0.412415}%
\pgfsetfillcolor{currentfill}%
\pgfsetlinewidth{0.000000pt}%
\definecolor{currentstroke}{rgb}{0.000000,0.000000,0.000000}%
\pgfsetstrokecolor{currentstroke}%
\pgfsetdash{}{0pt}%
\pgfpathmoveto{\pgfqpoint{2.361846in}{3.185767in}}%
\pgfpathlineto{\pgfqpoint{2.281684in}{3.105605in}}%
\pgfpathlineto{\pgfqpoint{2.291093in}{3.102469in}}%
\pgfpathlineto{\pgfqpoint{2.250323in}{3.080516in}}%
\pgfpathlineto{\pgfqpoint{2.272276in}{3.121286in}}%
\pgfpathlineto{\pgfqpoint{2.275412in}{3.111878in}}%
\pgfpathlineto{\pgfqpoint{2.355574in}{3.192039in}}%
\pgfpathlineto{\pgfqpoint{2.361846in}{3.185767in}}%
\pgfusepath{fill}%
\end{pgfscope}%
\begin{pgfscope}%
\pgfpathrectangle{\pgfqpoint{1.432000in}{0.528000in}}{\pgfqpoint{3.696000in}{3.696000in}} %
\pgfusepath{clip}%
\pgfsetbuttcap%
\pgfsetroundjoin%
\definecolor{currentfill}{rgb}{0.243113,0.292092,0.538516}%
\pgfsetfillcolor{currentfill}%
\pgfsetlinewidth{0.000000pt}%
\definecolor{currentstroke}{rgb}{0.000000,0.000000,0.000000}%
\pgfsetstrokecolor{currentstroke}%
\pgfsetdash{}{0pt}%
\pgfpathmoveto{\pgfqpoint{2.358710in}{3.184468in}}%
\pgfpathlineto{\pgfqpoint{2.290239in}{3.184468in}}%
\pgfpathlineto{\pgfqpoint{2.294675in}{3.175598in}}%
\pgfpathlineto{\pgfqpoint{2.250323in}{3.188903in}}%
\pgfpathlineto{\pgfqpoint{2.294675in}{3.202209in}}%
\pgfpathlineto{\pgfqpoint{2.290239in}{3.193338in}}%
\pgfpathlineto{\pgfqpoint{2.358710in}{3.193338in}}%
\pgfpathlineto{\pgfqpoint{2.358710in}{3.184468in}}%
\pgfusepath{fill}%
\end{pgfscope}%
\begin{pgfscope}%
\pgfpathrectangle{\pgfqpoint{1.432000in}{0.528000in}}{\pgfqpoint{3.696000in}{3.696000in}} %
\pgfusepath{clip}%
\pgfsetbuttcap%
\pgfsetroundjoin%
\definecolor{currentfill}{rgb}{0.126326,0.644107,0.525311}%
\pgfsetfillcolor{currentfill}%
\pgfsetlinewidth{0.000000pt}%
\definecolor{currentstroke}{rgb}{0.000000,0.000000,0.000000}%
\pgfsetstrokecolor{currentstroke}%
\pgfsetdash{}{0pt}%
\pgfpathmoveto{\pgfqpoint{2.467097in}{3.184468in}}%
\pgfpathlineto{\pgfqpoint{2.398626in}{3.184468in}}%
\pgfpathlineto{\pgfqpoint{2.403062in}{3.175598in}}%
\pgfpathlineto{\pgfqpoint{2.358710in}{3.188903in}}%
\pgfpathlineto{\pgfqpoint{2.403062in}{3.202209in}}%
\pgfpathlineto{\pgfqpoint{2.398626in}{3.193338in}}%
\pgfpathlineto{\pgfqpoint{2.467097in}{3.193338in}}%
\pgfpathlineto{\pgfqpoint{2.467097in}{3.184468in}}%
\pgfusepath{fill}%
\end{pgfscope}%
\begin{pgfscope}%
\pgfpathrectangle{\pgfqpoint{1.432000in}{0.528000in}}{\pgfqpoint{3.696000in}{3.696000in}} %
\pgfusepath{clip}%
\pgfsetbuttcap%
\pgfsetroundjoin%
\definecolor{currentfill}{rgb}{0.202219,0.715272,0.476084}%
\pgfsetfillcolor{currentfill}%
\pgfsetlinewidth{0.000000pt}%
\definecolor{currentstroke}{rgb}{0.000000,0.000000,0.000000}%
\pgfsetstrokecolor{currentstroke}%
\pgfsetdash{}{0pt}%
\pgfpathmoveto{\pgfqpoint{2.575484in}{3.184468in}}%
\pgfpathlineto{\pgfqpoint{2.507014in}{3.184468in}}%
\pgfpathlineto{\pgfqpoint{2.511449in}{3.175598in}}%
\pgfpathlineto{\pgfqpoint{2.467097in}{3.188903in}}%
\pgfpathlineto{\pgfqpoint{2.511449in}{3.202209in}}%
\pgfpathlineto{\pgfqpoint{2.507014in}{3.193338in}}%
\pgfpathlineto{\pgfqpoint{2.575484in}{3.193338in}}%
\pgfpathlineto{\pgfqpoint{2.575484in}{3.184468in}}%
\pgfusepath{fill}%
\end{pgfscope}%
\begin{pgfscope}%
\pgfpathrectangle{\pgfqpoint{1.432000in}{0.528000in}}{\pgfqpoint{3.696000in}{3.696000in}} %
\pgfusepath{clip}%
\pgfsetbuttcap%
\pgfsetroundjoin%
\definecolor{currentfill}{rgb}{0.188923,0.410910,0.556326}%
\pgfsetfillcolor{currentfill}%
\pgfsetlinewidth{0.000000pt}%
\definecolor{currentstroke}{rgb}{0.000000,0.000000,0.000000}%
\pgfsetstrokecolor{currentstroke}%
\pgfsetdash{}{0pt}%
\pgfpathmoveto{\pgfqpoint{2.683871in}{3.184468in}}%
\pgfpathlineto{\pgfqpoint{2.615401in}{3.184468in}}%
\pgfpathlineto{\pgfqpoint{2.619836in}{3.175598in}}%
\pgfpathlineto{\pgfqpoint{2.575484in}{3.188903in}}%
\pgfpathlineto{\pgfqpoint{2.619836in}{3.202209in}}%
\pgfpathlineto{\pgfqpoint{2.615401in}{3.193338in}}%
\pgfpathlineto{\pgfqpoint{2.683871in}{3.193338in}}%
\pgfpathlineto{\pgfqpoint{2.683871in}{3.184468in}}%
\pgfusepath{fill}%
\end{pgfscope}%
\begin{pgfscope}%
\pgfpathrectangle{\pgfqpoint{1.432000in}{0.528000in}}{\pgfqpoint{3.696000in}{3.696000in}} %
\pgfusepath{clip}%
\pgfsetbuttcap%
\pgfsetroundjoin%
\definecolor{currentfill}{rgb}{0.279566,0.067836,0.391917}%
\pgfsetfillcolor{currentfill}%
\pgfsetlinewidth{0.000000pt}%
\definecolor{currentstroke}{rgb}{0.000000,0.000000,0.000000}%
\pgfsetstrokecolor{currentstroke}%
\pgfsetdash{}{0pt}%
\pgfpathmoveto{\pgfqpoint{2.688306in}{3.188903in}}%
\pgfpathlineto{\pgfqpoint{2.686089in}{3.192744in}}%
\pgfpathlineto{\pgfqpoint{2.681653in}{3.192744in}}%
\pgfpathlineto{\pgfqpoint{2.679436in}{3.188903in}}%
\pgfpathlineto{\pgfqpoint{2.681653in}{3.185062in}}%
\pgfpathlineto{\pgfqpoint{2.686089in}{3.185062in}}%
\pgfpathlineto{\pgfqpoint{2.688306in}{3.188903in}}%
\pgfpathlineto{\pgfqpoint{2.686089in}{3.192744in}}%
\pgfusepath{fill}%
\end{pgfscope}%
\begin{pgfscope}%
\pgfpathrectangle{\pgfqpoint{1.432000in}{0.528000in}}{\pgfqpoint{3.696000in}{3.696000in}} %
\pgfusepath{clip}%
\pgfsetbuttcap%
\pgfsetroundjoin%
\definecolor{currentfill}{rgb}{0.267004,0.004874,0.329415}%
\pgfsetfillcolor{currentfill}%
\pgfsetlinewidth{0.000000pt}%
\definecolor{currentstroke}{rgb}{0.000000,0.000000,0.000000}%
\pgfsetstrokecolor{currentstroke}%
\pgfsetdash{}{0pt}%
\pgfpathmoveto{\pgfqpoint{2.679436in}{3.188903in}}%
\pgfpathlineto{\pgfqpoint{2.679436in}{3.257374in}}%
\pgfpathlineto{\pgfqpoint{2.670565in}{3.252938in}}%
\pgfpathlineto{\pgfqpoint{2.683871in}{3.297290in}}%
\pgfpathlineto{\pgfqpoint{2.697177in}{3.252938in}}%
\pgfpathlineto{\pgfqpoint{2.688306in}{3.257374in}}%
\pgfpathlineto{\pgfqpoint{2.688306in}{3.188903in}}%
\pgfpathlineto{\pgfqpoint{2.679436in}{3.188903in}}%
\pgfusepath{fill}%
\end{pgfscope}%
\begin{pgfscope}%
\pgfpathrectangle{\pgfqpoint{1.432000in}{0.528000in}}{\pgfqpoint{3.696000in}{3.696000in}} %
\pgfusepath{clip}%
\pgfsetbuttcap%
\pgfsetroundjoin%
\definecolor{currentfill}{rgb}{0.260571,0.246922,0.522828}%
\pgfsetfillcolor{currentfill}%
\pgfsetlinewidth{0.000000pt}%
\definecolor{currentstroke}{rgb}{0.000000,0.000000,0.000000}%
\pgfsetstrokecolor{currentstroke}%
\pgfsetdash{}{0pt}%
\pgfpathmoveto{\pgfqpoint{2.796693in}{3.188903in}}%
\pgfpathlineto{\pgfqpoint{2.794476in}{3.192744in}}%
\pgfpathlineto{\pgfqpoint{2.790040in}{3.192744in}}%
\pgfpathlineto{\pgfqpoint{2.787823in}{3.188903in}}%
\pgfpathlineto{\pgfqpoint{2.790040in}{3.185062in}}%
\pgfpathlineto{\pgfqpoint{2.794476in}{3.185062in}}%
\pgfpathlineto{\pgfqpoint{2.796693in}{3.188903in}}%
\pgfpathlineto{\pgfqpoint{2.794476in}{3.192744in}}%
\pgfusepath{fill}%
\end{pgfscope}%
\begin{pgfscope}%
\pgfpathrectangle{\pgfqpoint{1.432000in}{0.528000in}}{\pgfqpoint{3.696000in}{3.696000in}} %
\pgfusepath{clip}%
\pgfsetbuttcap%
\pgfsetroundjoin%
\definecolor{currentfill}{rgb}{0.283229,0.120777,0.440584}%
\pgfsetfillcolor{currentfill}%
\pgfsetlinewidth{0.000000pt}%
\definecolor{currentstroke}{rgb}{0.000000,0.000000,0.000000}%
\pgfsetstrokecolor{currentstroke}%
\pgfsetdash{}{0pt}%
\pgfpathmoveto{\pgfqpoint{2.787823in}{3.188903in}}%
\pgfpathlineto{\pgfqpoint{2.787823in}{3.257374in}}%
\pgfpathlineto{\pgfqpoint{2.778952in}{3.252938in}}%
\pgfpathlineto{\pgfqpoint{2.792258in}{3.297290in}}%
\pgfpathlineto{\pgfqpoint{2.805564in}{3.252938in}}%
\pgfpathlineto{\pgfqpoint{2.796693in}{3.257374in}}%
\pgfpathlineto{\pgfqpoint{2.796693in}{3.188903in}}%
\pgfpathlineto{\pgfqpoint{2.787823in}{3.188903in}}%
\pgfusepath{fill}%
\end{pgfscope}%
\begin{pgfscope}%
\pgfpathrectangle{\pgfqpoint{1.432000in}{0.528000in}}{\pgfqpoint{3.696000in}{3.696000in}} %
\pgfusepath{clip}%
\pgfsetbuttcap%
\pgfsetroundjoin%
\definecolor{currentfill}{rgb}{0.147607,0.511733,0.557049}%
\pgfsetfillcolor{currentfill}%
\pgfsetlinewidth{0.000000pt}%
\definecolor{currentstroke}{rgb}{0.000000,0.000000,0.000000}%
\pgfsetstrokecolor{currentstroke}%
\pgfsetdash{}{0pt}%
\pgfpathmoveto{\pgfqpoint{2.896210in}{3.188903in}}%
\pgfpathlineto{\pgfqpoint{2.896210in}{3.257374in}}%
\pgfpathlineto{\pgfqpoint{2.887340in}{3.252938in}}%
\pgfpathlineto{\pgfqpoint{2.900645in}{3.297290in}}%
\pgfpathlineto{\pgfqpoint{2.913951in}{3.252938in}}%
\pgfpathlineto{\pgfqpoint{2.905080in}{3.257374in}}%
\pgfpathlineto{\pgfqpoint{2.905080in}{3.188903in}}%
\pgfpathlineto{\pgfqpoint{2.896210in}{3.188903in}}%
\pgfusepath{fill}%
\end{pgfscope}%
\begin{pgfscope}%
\pgfpathrectangle{\pgfqpoint{1.432000in}{0.528000in}}{\pgfqpoint{3.696000in}{3.696000in}} %
\pgfusepath{clip}%
\pgfsetbuttcap%
\pgfsetroundjoin%
\definecolor{currentfill}{rgb}{0.126326,0.644107,0.525311}%
\pgfsetfillcolor{currentfill}%
\pgfsetlinewidth{0.000000pt}%
\definecolor{currentstroke}{rgb}{0.000000,0.000000,0.000000}%
\pgfsetstrokecolor{currentstroke}%
\pgfsetdash{}{0pt}%
\pgfpathmoveto{\pgfqpoint{3.004597in}{3.188903in}}%
\pgfpathlineto{\pgfqpoint{3.004597in}{3.257374in}}%
\pgfpathlineto{\pgfqpoint{2.995727in}{3.252938in}}%
\pgfpathlineto{\pgfqpoint{3.009032in}{3.297290in}}%
\pgfpathlineto{\pgfqpoint{3.022338in}{3.252938in}}%
\pgfpathlineto{\pgfqpoint{3.013467in}{3.257374in}}%
\pgfpathlineto{\pgfqpoint{3.013467in}{3.188903in}}%
\pgfpathlineto{\pgfqpoint{3.004597in}{3.188903in}}%
\pgfusepath{fill}%
\end{pgfscope}%
\begin{pgfscope}%
\pgfpathrectangle{\pgfqpoint{1.432000in}{0.528000in}}{\pgfqpoint{3.696000in}{3.696000in}} %
\pgfusepath{clip}%
\pgfsetbuttcap%
\pgfsetroundjoin%
\definecolor{currentfill}{rgb}{0.268510,0.009605,0.335427}%
\pgfsetfillcolor{currentfill}%
\pgfsetlinewidth{0.000000pt}%
\definecolor{currentstroke}{rgb}{0.000000,0.000000,0.000000}%
\pgfsetstrokecolor{currentstroke}%
\pgfsetdash{}{0pt}%
\pgfpathmoveto{\pgfqpoint{3.005896in}{3.192039in}}%
\pgfpathlineto{\pgfqpoint{3.086058in}{3.272201in}}%
\pgfpathlineto{\pgfqpoint{3.076649in}{3.275337in}}%
\pgfpathlineto{\pgfqpoint{3.117419in}{3.297290in}}%
\pgfpathlineto{\pgfqpoint{3.095466in}{3.256520in}}%
\pgfpathlineto{\pgfqpoint{3.092330in}{3.265929in}}%
\pgfpathlineto{\pgfqpoint{3.012168in}{3.185767in}}%
\pgfpathlineto{\pgfqpoint{3.005896in}{3.192039in}}%
\pgfusepath{fill}%
\end{pgfscope}%
\begin{pgfscope}%
\pgfpathrectangle{\pgfqpoint{1.432000in}{0.528000in}}{\pgfqpoint{3.696000in}{3.696000in}} %
\pgfusepath{clip}%
\pgfsetbuttcap%
\pgfsetroundjoin%
\definecolor{currentfill}{rgb}{0.123444,0.636809,0.528763}%
\pgfsetfillcolor{currentfill}%
\pgfsetlinewidth{0.000000pt}%
\definecolor{currentstroke}{rgb}{0.000000,0.000000,0.000000}%
\pgfsetstrokecolor{currentstroke}%
\pgfsetdash{}{0pt}%
\pgfpathmoveto{\pgfqpoint{3.112984in}{3.188903in}}%
\pgfpathlineto{\pgfqpoint{3.112984in}{3.257374in}}%
\pgfpathlineto{\pgfqpoint{3.104114in}{3.252938in}}%
\pgfpathlineto{\pgfqpoint{3.117419in}{3.297290in}}%
\pgfpathlineto{\pgfqpoint{3.130725in}{3.252938in}}%
\pgfpathlineto{\pgfqpoint{3.121855in}{3.257374in}}%
\pgfpathlineto{\pgfqpoint{3.121855in}{3.188903in}}%
\pgfpathlineto{\pgfqpoint{3.112984in}{3.188903in}}%
\pgfusepath{fill}%
\end{pgfscope}%
\begin{pgfscope}%
\pgfpathrectangle{\pgfqpoint{1.432000in}{0.528000in}}{\pgfqpoint{3.696000in}{3.696000in}} %
\pgfusepath{clip}%
\pgfsetbuttcap%
\pgfsetroundjoin%
\definecolor{currentfill}{rgb}{0.278826,0.175490,0.483397}%
\pgfsetfillcolor{currentfill}%
\pgfsetlinewidth{0.000000pt}%
\definecolor{currentstroke}{rgb}{0.000000,0.000000,0.000000}%
\pgfsetstrokecolor{currentstroke}%
\pgfsetdash{}{0pt}%
\pgfpathmoveto{\pgfqpoint{3.221371in}{3.188903in}}%
\pgfpathlineto{\pgfqpoint{3.221371in}{3.257374in}}%
\pgfpathlineto{\pgfqpoint{3.212501in}{3.252938in}}%
\pgfpathlineto{\pgfqpoint{3.225806in}{3.297290in}}%
\pgfpathlineto{\pgfqpoint{3.239112in}{3.252938in}}%
\pgfpathlineto{\pgfqpoint{3.230242in}{3.257374in}}%
\pgfpathlineto{\pgfqpoint{3.230242in}{3.188903in}}%
\pgfpathlineto{\pgfqpoint{3.221371in}{3.188903in}}%
\pgfusepath{fill}%
\end{pgfscope}%
\begin{pgfscope}%
\pgfpathrectangle{\pgfqpoint{1.432000in}{0.528000in}}{\pgfqpoint{3.696000in}{3.696000in}} %
\pgfusepath{clip}%
\pgfsetbuttcap%
\pgfsetroundjoin%
\definecolor{currentfill}{rgb}{0.282910,0.105393,0.426902}%
\pgfsetfillcolor{currentfill}%
\pgfsetlinewidth{0.000000pt}%
\definecolor{currentstroke}{rgb}{0.000000,0.000000,0.000000}%
\pgfsetstrokecolor{currentstroke}%
\pgfsetdash{}{0pt}%
\pgfpathmoveto{\pgfqpoint{3.221371in}{3.188903in}}%
\pgfpathlineto{\pgfqpoint{3.221371in}{3.365761in}}%
\pgfpathlineto{\pgfqpoint{3.212501in}{3.361325in}}%
\pgfpathlineto{\pgfqpoint{3.225806in}{3.405677in}}%
\pgfpathlineto{\pgfqpoint{3.239112in}{3.361325in}}%
\pgfpathlineto{\pgfqpoint{3.230242in}{3.365761in}}%
\pgfpathlineto{\pgfqpoint{3.230242in}{3.188903in}}%
\pgfpathlineto{\pgfqpoint{3.221371in}{3.188903in}}%
\pgfusepath{fill}%
\end{pgfscope}%
\begin{pgfscope}%
\pgfpathrectangle{\pgfqpoint{1.432000in}{0.528000in}}{\pgfqpoint{3.696000in}{3.696000in}} %
\pgfusepath{clip}%
\pgfsetbuttcap%
\pgfsetroundjoin%
\definecolor{currentfill}{rgb}{0.281446,0.084320,0.407414}%
\pgfsetfillcolor{currentfill}%
\pgfsetlinewidth{0.000000pt}%
\definecolor{currentstroke}{rgb}{0.000000,0.000000,0.000000}%
\pgfsetstrokecolor{currentstroke}%
\pgfsetdash{}{0pt}%
\pgfpathmoveto{\pgfqpoint{3.331057in}{3.185767in}}%
\pgfpathlineto{\pgfqpoint{3.250896in}{3.265929in}}%
\pgfpathlineto{\pgfqpoint{3.247760in}{3.256520in}}%
\pgfpathlineto{\pgfqpoint{3.225806in}{3.297290in}}%
\pgfpathlineto{\pgfqpoint{3.266577in}{3.275337in}}%
\pgfpathlineto{\pgfqpoint{3.257168in}{3.272201in}}%
\pgfpathlineto{\pgfqpoint{3.337330in}{3.192039in}}%
\pgfpathlineto{\pgfqpoint{3.331057in}{3.185767in}}%
\pgfusepath{fill}%
\end{pgfscope}%
\begin{pgfscope}%
\pgfpathrectangle{\pgfqpoint{1.432000in}{0.528000in}}{\pgfqpoint{3.696000in}{3.696000in}} %
\pgfusepath{clip}%
\pgfsetbuttcap%
\pgfsetroundjoin%
\definecolor{currentfill}{rgb}{0.282327,0.094955,0.417331}%
\pgfsetfillcolor{currentfill}%
\pgfsetlinewidth{0.000000pt}%
\definecolor{currentstroke}{rgb}{0.000000,0.000000,0.000000}%
\pgfsetstrokecolor{currentstroke}%
\pgfsetdash{}{0pt}%
\pgfpathmoveto{\pgfqpoint{3.330227in}{3.186920in}}%
\pgfpathlineto{\pgfqpoint{3.239691in}{3.367991in}}%
\pgfpathlineto{\pgfqpoint{3.233740in}{3.360057in}}%
\pgfpathlineto{\pgfqpoint{3.225806in}{3.405677in}}%
\pgfpathlineto{\pgfqpoint{3.257542in}{3.371958in}}%
\pgfpathlineto{\pgfqpoint{3.247625in}{3.371958in}}%
\pgfpathlineto{\pgfqpoint{3.338161in}{3.190887in}}%
\pgfpathlineto{\pgfqpoint{3.330227in}{3.186920in}}%
\pgfusepath{fill}%
\end{pgfscope}%
\begin{pgfscope}%
\pgfpathrectangle{\pgfqpoint{1.432000in}{0.528000in}}{\pgfqpoint{3.696000in}{3.696000in}} %
\pgfusepath{clip}%
\pgfsetbuttcap%
\pgfsetroundjoin%
\definecolor{currentfill}{rgb}{0.218130,0.347432,0.550038}%
\pgfsetfillcolor{currentfill}%
\pgfsetlinewidth{0.000000pt}%
\definecolor{currentstroke}{rgb}{0.000000,0.000000,0.000000}%
\pgfsetstrokecolor{currentstroke}%
\pgfsetdash{}{0pt}%
\pgfpathmoveto{\pgfqpoint{3.438614in}{3.186920in}}%
\pgfpathlineto{\pgfqpoint{3.348078in}{3.367991in}}%
\pgfpathlineto{\pgfqpoint{3.342127in}{3.360057in}}%
\pgfpathlineto{\pgfqpoint{3.334194in}{3.405677in}}%
\pgfpathlineto{\pgfqpoint{3.365929in}{3.371958in}}%
\pgfpathlineto{\pgfqpoint{3.356012in}{3.371958in}}%
\pgfpathlineto{\pgfqpoint{3.446548in}{3.190887in}}%
\pgfpathlineto{\pgfqpoint{3.438614in}{3.186920in}}%
\pgfusepath{fill}%
\end{pgfscope}%
\begin{pgfscope}%
\pgfpathrectangle{\pgfqpoint{1.432000in}{0.528000in}}{\pgfqpoint{3.696000in}{3.696000in}} %
\pgfusepath{clip}%
\pgfsetbuttcap%
\pgfsetroundjoin%
\definecolor{currentfill}{rgb}{0.185556,0.418570,0.556753}%
\pgfsetfillcolor{currentfill}%
\pgfsetlinewidth{0.000000pt}%
\definecolor{currentstroke}{rgb}{0.000000,0.000000,0.000000}%
\pgfsetstrokecolor{currentstroke}%
\pgfsetdash{}{0pt}%
\pgfpathmoveto{\pgfqpoint{3.547832in}{3.185767in}}%
\pgfpathlineto{\pgfqpoint{3.359283in}{3.374316in}}%
\pgfpathlineto{\pgfqpoint{3.356147in}{3.364907in}}%
\pgfpathlineto{\pgfqpoint{3.334194in}{3.405677in}}%
\pgfpathlineto{\pgfqpoint{3.374964in}{3.383724in}}%
\pgfpathlineto{\pgfqpoint{3.365555in}{3.380588in}}%
\pgfpathlineto{\pgfqpoint{3.554104in}{3.192039in}}%
\pgfpathlineto{\pgfqpoint{3.547832in}{3.185767in}}%
\pgfusepath{fill}%
\end{pgfscope}%
\begin{pgfscope}%
\pgfpathrectangle{\pgfqpoint{1.432000in}{0.528000in}}{\pgfqpoint{3.696000in}{3.696000in}} %
\pgfusepath{clip}%
\pgfsetbuttcap%
\pgfsetroundjoin%
\definecolor{currentfill}{rgb}{0.206756,0.371758,0.553117}%
\pgfsetfillcolor{currentfill}%
\pgfsetlinewidth{0.000000pt}%
\definecolor{currentstroke}{rgb}{0.000000,0.000000,0.000000}%
\pgfsetstrokecolor{currentstroke}%
\pgfsetdash{}{0pt}%
\pgfpathmoveto{\pgfqpoint{3.656219in}{3.185767in}}%
\pgfpathlineto{\pgfqpoint{3.467670in}{3.374316in}}%
\pgfpathlineto{\pgfqpoint{3.464534in}{3.364907in}}%
\pgfpathlineto{\pgfqpoint{3.442581in}{3.405677in}}%
\pgfpathlineto{\pgfqpoint{3.483351in}{3.383724in}}%
\pgfpathlineto{\pgfqpoint{3.473942in}{3.380588in}}%
\pgfpathlineto{\pgfqpoint{3.662491in}{3.192039in}}%
\pgfpathlineto{\pgfqpoint{3.656219in}{3.185767in}}%
\pgfusepath{fill}%
\end{pgfscope}%
\begin{pgfscope}%
\pgfpathrectangle{\pgfqpoint{1.432000in}{0.528000in}}{\pgfqpoint{3.696000in}{3.696000in}} %
\pgfusepath{clip}%
\pgfsetbuttcap%
\pgfsetroundjoin%
\definecolor{currentfill}{rgb}{0.182256,0.426184,0.557120}%
\pgfsetfillcolor{currentfill}%
\pgfsetlinewidth{0.000000pt}%
\definecolor{currentstroke}{rgb}{0.000000,0.000000,0.000000}%
\pgfsetstrokecolor{currentstroke}%
\pgfsetdash{}{0pt}%
\pgfpathmoveto{\pgfqpoint{3.765282in}{3.185213in}}%
\pgfpathlineto{\pgfqpoint{3.473333in}{3.379845in}}%
\pgfpathlineto{\pgfqpoint{3.472103in}{3.370004in}}%
\pgfpathlineto{\pgfqpoint{3.442581in}{3.405677in}}%
\pgfpathlineto{\pgfqpoint{3.486864in}{3.392146in}}%
\pgfpathlineto{\pgfqpoint{3.478254in}{3.387226in}}%
\pgfpathlineto{\pgfqpoint{3.770202in}{3.192594in}}%
\pgfpathlineto{\pgfqpoint{3.765282in}{3.185213in}}%
\pgfusepath{fill}%
\end{pgfscope}%
\begin{pgfscope}%
\pgfpathrectangle{\pgfqpoint{1.432000in}{0.528000in}}{\pgfqpoint{3.696000in}{3.696000in}} %
\pgfusepath{clip}%
\pgfsetbuttcap%
\pgfsetroundjoin%
\definecolor{currentfill}{rgb}{0.165117,0.467423,0.558141}%
\pgfsetfillcolor{currentfill}%
\pgfsetlinewidth{0.000000pt}%
\definecolor{currentstroke}{rgb}{0.000000,0.000000,0.000000}%
\pgfsetstrokecolor{currentstroke}%
\pgfsetdash{}{0pt}%
\pgfpathmoveto{\pgfqpoint{3.873669in}{3.185213in}}%
\pgfpathlineto{\pgfqpoint{3.581720in}{3.379845in}}%
\pgfpathlineto{\pgfqpoint{3.580490in}{3.370004in}}%
\pgfpathlineto{\pgfqpoint{3.550968in}{3.405677in}}%
\pgfpathlineto{\pgfqpoint{3.595251in}{3.392146in}}%
\pgfpathlineto{\pgfqpoint{3.586641in}{3.387226in}}%
\pgfpathlineto{\pgfqpoint{3.878589in}{3.192594in}}%
\pgfpathlineto{\pgfqpoint{3.873669in}{3.185213in}}%
\pgfusepath{fill}%
\end{pgfscope}%
\begin{pgfscope}%
\pgfpathrectangle{\pgfqpoint{1.432000in}{0.528000in}}{\pgfqpoint{3.696000in}{3.696000in}} %
\pgfusepath{clip}%
\pgfsetbuttcap%
\pgfsetroundjoin%
\definecolor{currentfill}{rgb}{0.280267,0.073417,0.397163}%
\pgfsetfillcolor{currentfill}%
\pgfsetlinewidth{0.000000pt}%
\definecolor{currentstroke}{rgb}{0.000000,0.000000,0.000000}%
\pgfsetstrokecolor{currentstroke}%
\pgfsetdash{}{0pt}%
\pgfpathmoveto{\pgfqpoint{3.872993in}{3.185767in}}%
\pgfpathlineto{\pgfqpoint{3.576057in}{3.482703in}}%
\pgfpathlineto{\pgfqpoint{3.572921in}{3.473294in}}%
\pgfpathlineto{\pgfqpoint{3.550968in}{3.514065in}}%
\pgfpathlineto{\pgfqpoint{3.591738in}{3.492111in}}%
\pgfpathlineto{\pgfqpoint{3.582329in}{3.488975in}}%
\pgfpathlineto{\pgfqpoint{3.879265in}{3.192039in}}%
\pgfpathlineto{\pgfqpoint{3.872993in}{3.185767in}}%
\pgfusepath{fill}%
\end{pgfscope}%
\begin{pgfscope}%
\pgfpathrectangle{\pgfqpoint{1.432000in}{0.528000in}}{\pgfqpoint{3.696000in}{3.696000in}} %
\pgfusepath{clip}%
\pgfsetbuttcap%
\pgfsetroundjoin%
\definecolor{currentfill}{rgb}{0.121148,0.592739,0.544641}%
\pgfsetfillcolor{currentfill}%
\pgfsetlinewidth{0.000000pt}%
\definecolor{currentstroke}{rgb}{0.000000,0.000000,0.000000}%
\pgfsetstrokecolor{currentstroke}%
\pgfsetdash{}{0pt}%
\pgfpathmoveto{\pgfqpoint{3.981380in}{3.185767in}}%
\pgfpathlineto{\pgfqpoint{3.684444in}{3.482703in}}%
\pgfpathlineto{\pgfqpoint{3.681308in}{3.473294in}}%
\pgfpathlineto{\pgfqpoint{3.659355in}{3.514065in}}%
\pgfpathlineto{\pgfqpoint{3.700125in}{3.492111in}}%
\pgfpathlineto{\pgfqpoint{3.690716in}{3.488975in}}%
\pgfpathlineto{\pgfqpoint{3.987652in}{3.192039in}}%
\pgfpathlineto{\pgfqpoint{3.981380in}{3.185767in}}%
\pgfusepath{fill}%
\end{pgfscope}%
\begin{pgfscope}%
\pgfpathrectangle{\pgfqpoint{1.432000in}{0.528000in}}{\pgfqpoint{3.696000in}{3.696000in}} %
\pgfusepath{clip}%
\pgfsetbuttcap%
\pgfsetroundjoin%
\definecolor{currentfill}{rgb}{0.252899,0.742211,0.448284}%
\pgfsetfillcolor{currentfill}%
\pgfsetlinewidth{0.000000pt}%
\definecolor{currentstroke}{rgb}{0.000000,0.000000,0.000000}%
\pgfsetstrokecolor{currentstroke}%
\pgfsetdash{}{0pt}%
\pgfpathmoveto{\pgfqpoint{4.089767in}{3.185767in}}%
\pgfpathlineto{\pgfqpoint{3.792831in}{3.482703in}}%
\pgfpathlineto{\pgfqpoint{3.789695in}{3.473294in}}%
\pgfpathlineto{\pgfqpoint{3.767742in}{3.514065in}}%
\pgfpathlineto{\pgfqpoint{3.808512in}{3.492111in}}%
\pgfpathlineto{\pgfqpoint{3.799104in}{3.488975in}}%
\pgfpathlineto{\pgfqpoint{4.096039in}{3.192039in}}%
\pgfpathlineto{\pgfqpoint{4.089767in}{3.185767in}}%
\pgfusepath{fill}%
\end{pgfscope}%
\begin{pgfscope}%
\pgfpathrectangle{\pgfqpoint{1.432000in}{0.528000in}}{\pgfqpoint{3.696000in}{3.696000in}} %
\pgfusepath{clip}%
\pgfsetbuttcap%
\pgfsetroundjoin%
\definecolor{currentfill}{rgb}{0.772852,0.877868,0.131109}%
\pgfsetfillcolor{currentfill}%
\pgfsetlinewidth{0.000000pt}%
\definecolor{currentstroke}{rgb}{0.000000,0.000000,0.000000}%
\pgfsetstrokecolor{currentstroke}%
\pgfsetdash{}{0pt}%
\pgfpathmoveto{\pgfqpoint{4.198154in}{3.185767in}}%
\pgfpathlineto{\pgfqpoint{3.901218in}{3.482703in}}%
\pgfpathlineto{\pgfqpoint{3.898082in}{3.473294in}}%
\pgfpathlineto{\pgfqpoint{3.876129in}{3.514065in}}%
\pgfpathlineto{\pgfqpoint{3.916899in}{3.492111in}}%
\pgfpathlineto{\pgfqpoint{3.907491in}{3.488975in}}%
\pgfpathlineto{\pgfqpoint{4.204426in}{3.192039in}}%
\pgfpathlineto{\pgfqpoint{4.198154in}{3.185767in}}%
\pgfusepath{fill}%
\end{pgfscope}%
\begin{pgfscope}%
\pgfpathrectangle{\pgfqpoint{1.432000in}{0.528000in}}{\pgfqpoint{3.696000in}{3.696000in}} %
\pgfusepath{clip}%
\pgfsetbuttcap%
\pgfsetroundjoin%
\definecolor{currentfill}{rgb}{0.335885,0.777018,0.402049}%
\pgfsetfillcolor{currentfill}%
\pgfsetlinewidth{0.000000pt}%
\definecolor{currentstroke}{rgb}{0.000000,0.000000,0.000000}%
\pgfsetstrokecolor{currentstroke}%
\pgfsetdash{}{0pt}%
\pgfpathmoveto{\pgfqpoint{4.306541in}{3.185767in}}%
\pgfpathlineto{\pgfqpoint{4.009605in}{3.482703in}}%
\pgfpathlineto{\pgfqpoint{4.006469in}{3.473294in}}%
\pgfpathlineto{\pgfqpoint{3.984516in}{3.514065in}}%
\pgfpathlineto{\pgfqpoint{4.025286in}{3.492111in}}%
\pgfpathlineto{\pgfqpoint{4.015878in}{3.488975in}}%
\pgfpathlineto{\pgfqpoint{4.312814in}{3.192039in}}%
\pgfpathlineto{\pgfqpoint{4.306541in}{3.185767in}}%
\pgfusepath{fill}%
\end{pgfscope}%
\begin{pgfscope}%
\pgfpathrectangle{\pgfqpoint{1.432000in}{0.528000in}}{\pgfqpoint{3.696000in}{3.696000in}} %
\pgfusepath{clip}%
\pgfsetbuttcap%
\pgfsetroundjoin%
\definecolor{currentfill}{rgb}{0.271828,0.209303,0.504434}%
\pgfsetfillcolor{currentfill}%
\pgfsetlinewidth{0.000000pt}%
\definecolor{currentstroke}{rgb}{0.000000,0.000000,0.000000}%
\pgfsetstrokecolor{currentstroke}%
\pgfsetdash{}{0pt}%
\pgfpathmoveto{\pgfqpoint{4.305987in}{3.186443in}}%
\pgfpathlineto{\pgfqpoint{4.111355in}{3.478392in}}%
\pgfpathlineto{\pgfqpoint{4.106434in}{3.469781in}}%
\pgfpathlineto{\pgfqpoint{4.092903in}{3.514065in}}%
\pgfpathlineto{\pgfqpoint{4.128576in}{3.484542in}}%
\pgfpathlineto{\pgfqpoint{4.118735in}{3.483312in}}%
\pgfpathlineto{\pgfqpoint{4.313368in}{3.191363in}}%
\pgfpathlineto{\pgfqpoint{4.305987in}{3.186443in}}%
\pgfusepath{fill}%
\end{pgfscope}%
\begin{pgfscope}%
\pgfpathrectangle{\pgfqpoint{1.432000in}{0.528000in}}{\pgfqpoint{3.696000in}{3.696000in}} %
\pgfusepath{clip}%
\pgfsetbuttcap%
\pgfsetroundjoin%
\definecolor{currentfill}{rgb}{0.156270,0.489624,0.557936}%
\pgfsetfillcolor{currentfill}%
\pgfsetlinewidth{0.000000pt}%
\definecolor{currentstroke}{rgb}{0.000000,0.000000,0.000000}%
\pgfsetstrokecolor{currentstroke}%
\pgfsetdash{}{0pt}%
\pgfpathmoveto{\pgfqpoint{4.414928in}{3.185767in}}%
\pgfpathlineto{\pgfqpoint{4.117993in}{3.482703in}}%
\pgfpathlineto{\pgfqpoint{4.114856in}{3.473294in}}%
\pgfpathlineto{\pgfqpoint{4.092903in}{3.514065in}}%
\pgfpathlineto{\pgfqpoint{4.133673in}{3.492111in}}%
\pgfpathlineto{\pgfqpoint{4.124265in}{3.488975in}}%
\pgfpathlineto{\pgfqpoint{4.421201in}{3.192039in}}%
\pgfpathlineto{\pgfqpoint{4.414928in}{3.185767in}}%
\pgfusepath{fill}%
\end{pgfscope}%
\begin{pgfscope}%
\pgfpathrectangle{\pgfqpoint{1.432000in}{0.528000in}}{\pgfqpoint{3.696000in}{3.696000in}} %
\pgfusepath{clip}%
\pgfsetbuttcap%
\pgfsetroundjoin%
\definecolor{currentfill}{rgb}{0.140210,0.665859,0.513427}%
\pgfsetfillcolor{currentfill}%
\pgfsetlinewidth{0.000000pt}%
\definecolor{currentstroke}{rgb}{0.000000,0.000000,0.000000}%
\pgfsetstrokecolor{currentstroke}%
\pgfsetdash{}{0pt}%
\pgfpathmoveto{\pgfqpoint{4.414374in}{3.186443in}}%
\pgfpathlineto{\pgfqpoint{4.219742in}{3.478392in}}%
\pgfpathlineto{\pgfqpoint{4.214821in}{3.469781in}}%
\pgfpathlineto{\pgfqpoint{4.201290in}{3.514065in}}%
\pgfpathlineto{\pgfqpoint{4.236963in}{3.484542in}}%
\pgfpathlineto{\pgfqpoint{4.227122in}{3.483312in}}%
\pgfpathlineto{\pgfqpoint{4.421755in}{3.191363in}}%
\pgfpathlineto{\pgfqpoint{4.414374in}{3.186443in}}%
\pgfusepath{fill}%
\end{pgfscope}%
\begin{pgfscope}%
\pgfpathrectangle{\pgfqpoint{1.432000in}{0.528000in}}{\pgfqpoint{3.696000in}{3.696000in}} %
\pgfusepath{clip}%
\pgfsetbuttcap%
\pgfsetroundjoin%
\definecolor{currentfill}{rgb}{0.377779,0.791781,0.377939}%
\pgfsetfillcolor{currentfill}%
\pgfsetlinewidth{0.000000pt}%
\definecolor{currentstroke}{rgb}{0.000000,0.000000,0.000000}%
\pgfsetstrokecolor{currentstroke}%
\pgfsetdash{}{0pt}%
\pgfpathmoveto{\pgfqpoint{4.522761in}{3.186443in}}%
\pgfpathlineto{\pgfqpoint{4.328129in}{3.478392in}}%
\pgfpathlineto{\pgfqpoint{4.323209in}{3.469781in}}%
\pgfpathlineto{\pgfqpoint{4.309677in}{3.514065in}}%
\pgfpathlineto{\pgfqpoint{4.345350in}{3.484542in}}%
\pgfpathlineto{\pgfqpoint{4.335510in}{3.483312in}}%
\pgfpathlineto{\pgfqpoint{4.530142in}{3.191363in}}%
\pgfpathlineto{\pgfqpoint{4.522761in}{3.186443in}}%
\pgfusepath{fill}%
\end{pgfscope}%
\begin{pgfscope}%
\pgfpathrectangle{\pgfqpoint{1.432000in}{0.528000in}}{\pgfqpoint{3.696000in}{3.696000in}} %
\pgfusepath{clip}%
\pgfsetbuttcap%
\pgfsetroundjoin%
\definecolor{currentfill}{rgb}{0.258965,0.251537,0.524736}%
\pgfsetfillcolor{currentfill}%
\pgfsetlinewidth{0.000000pt}%
\definecolor{currentstroke}{rgb}{0.000000,0.000000,0.000000}%
\pgfsetstrokecolor{currentstroke}%
\pgfsetdash{}{0pt}%
\pgfpathmoveto{\pgfqpoint{4.522244in}{3.187501in}}%
\pgfpathlineto{\pgfqpoint{4.426480in}{3.474794in}}%
\pgfpathlineto{\pgfqpoint{4.419467in}{3.467781in}}%
\pgfpathlineto{\pgfqpoint{4.418065in}{3.514065in}}%
\pgfpathlineto{\pgfqpoint{4.444713in}{3.476196in}}%
\pgfpathlineto{\pgfqpoint{4.434895in}{3.477599in}}%
\pgfpathlineto{\pgfqpoint{4.530659in}{3.190306in}}%
\pgfpathlineto{\pgfqpoint{4.522244in}{3.187501in}}%
\pgfusepath{fill}%
\end{pgfscope}%
\begin{pgfscope}%
\pgfpathrectangle{\pgfqpoint{1.432000in}{0.528000in}}{\pgfqpoint{3.696000in}{3.696000in}} %
\pgfusepath{clip}%
\pgfsetbuttcap%
\pgfsetroundjoin%
\definecolor{currentfill}{rgb}{0.154815,0.493313,0.557840}%
\pgfsetfillcolor{currentfill}%
\pgfsetlinewidth{0.000000pt}%
\definecolor{currentstroke}{rgb}{0.000000,0.000000,0.000000}%
\pgfsetstrokecolor{currentstroke}%
\pgfsetdash{}{0pt}%
\pgfpathmoveto{\pgfqpoint{4.631148in}{3.186443in}}%
\pgfpathlineto{\pgfqpoint{4.436516in}{3.478392in}}%
\pgfpathlineto{\pgfqpoint{4.431596in}{3.469781in}}%
\pgfpathlineto{\pgfqpoint{4.418065in}{3.514065in}}%
\pgfpathlineto{\pgfqpoint{4.453738in}{3.484542in}}%
\pgfpathlineto{\pgfqpoint{4.443897in}{3.483312in}}%
\pgfpathlineto{\pgfqpoint{4.638529in}{3.191363in}}%
\pgfpathlineto{\pgfqpoint{4.631148in}{3.186443in}}%
\pgfusepath{fill}%
\end{pgfscope}%
\begin{pgfscope}%
\pgfpathrectangle{\pgfqpoint{1.432000in}{0.528000in}}{\pgfqpoint{3.696000in}{3.696000in}} %
\pgfusepath{clip}%
\pgfsetbuttcap%
\pgfsetroundjoin%
\definecolor{currentfill}{rgb}{0.226397,0.728888,0.462789}%
\pgfsetfillcolor{currentfill}%
\pgfsetlinewidth{0.000000pt}%
\definecolor{currentstroke}{rgb}{0.000000,0.000000,0.000000}%
\pgfsetstrokecolor{currentstroke}%
\pgfsetdash{}{0pt}%
\pgfpathmoveto{\pgfqpoint{4.630631in}{3.187501in}}%
\pgfpathlineto{\pgfqpoint{4.534867in}{3.474794in}}%
\pgfpathlineto{\pgfqpoint{4.527854in}{3.467781in}}%
\pgfpathlineto{\pgfqpoint{4.526452in}{3.514065in}}%
\pgfpathlineto{\pgfqpoint{4.553100in}{3.476196in}}%
\pgfpathlineto{\pgfqpoint{4.543282in}{3.477599in}}%
\pgfpathlineto{\pgfqpoint{4.639046in}{3.190306in}}%
\pgfpathlineto{\pgfqpoint{4.630631in}{3.187501in}}%
\pgfusepath{fill}%
\end{pgfscope}%
\begin{pgfscope}%
\pgfpathrectangle{\pgfqpoint{1.432000in}{0.528000in}}{\pgfqpoint{3.696000in}{3.696000in}} %
\pgfusepath{clip}%
\pgfsetbuttcap%
\pgfsetroundjoin%
\definecolor{currentfill}{rgb}{0.783315,0.879285,0.125405}%
\pgfsetfillcolor{currentfill}%
\pgfsetlinewidth{0.000000pt}%
\definecolor{currentstroke}{rgb}{0.000000,0.000000,0.000000}%
\pgfsetstrokecolor{currentstroke}%
\pgfsetdash{}{0pt}%
\pgfpathmoveto{\pgfqpoint{4.739018in}{3.187501in}}%
\pgfpathlineto{\pgfqpoint{4.643254in}{3.474794in}}%
\pgfpathlineto{\pgfqpoint{4.636241in}{3.467781in}}%
\pgfpathlineto{\pgfqpoint{4.634839in}{3.514065in}}%
\pgfpathlineto{\pgfqpoint{4.661487in}{3.476196in}}%
\pgfpathlineto{\pgfqpoint{4.651669in}{3.477599in}}%
\pgfpathlineto{\pgfqpoint{4.747433in}{3.190306in}}%
\pgfpathlineto{\pgfqpoint{4.739018in}{3.187501in}}%
\pgfusepath{fill}%
\end{pgfscope}%
\begin{pgfscope}%
\pgfpathrectangle{\pgfqpoint{1.432000in}{0.528000in}}{\pgfqpoint{3.696000in}{3.696000in}} %
\pgfusepath{clip}%
\pgfsetbuttcap%
\pgfsetroundjoin%
\definecolor{currentfill}{rgb}{0.283072,0.130895,0.449241}%
\pgfsetfillcolor{currentfill}%
\pgfsetlinewidth{0.000000pt}%
\definecolor{currentstroke}{rgb}{0.000000,0.000000,0.000000}%
\pgfsetstrokecolor{currentstroke}%
\pgfsetdash{}{0pt}%
\pgfpathmoveto{\pgfqpoint{4.738791in}{3.188903in}}%
\pgfpathlineto{\pgfqpoint{4.738791in}{3.474148in}}%
\pgfpathlineto{\pgfqpoint{4.729920in}{3.469713in}}%
\pgfpathlineto{\pgfqpoint{4.743226in}{3.514065in}}%
\pgfpathlineto{\pgfqpoint{4.756531in}{3.469713in}}%
\pgfpathlineto{\pgfqpoint{4.747661in}{3.474148in}}%
\pgfpathlineto{\pgfqpoint{4.747661in}{3.188903in}}%
\pgfpathlineto{\pgfqpoint{4.738791in}{3.188903in}}%
\pgfusepath{fill}%
\end{pgfscope}%
\begin{pgfscope}%
\pgfpathrectangle{\pgfqpoint{1.432000in}{0.528000in}}{\pgfqpoint{3.696000in}{3.696000in}} %
\pgfusepath{clip}%
\pgfsetbuttcap%
\pgfsetroundjoin%
\definecolor{currentfill}{rgb}{0.122606,0.585371,0.546557}%
\pgfsetfillcolor{currentfill}%
\pgfsetlinewidth{0.000000pt}%
\definecolor{currentstroke}{rgb}{0.000000,0.000000,0.000000}%
\pgfsetstrokecolor{currentstroke}%
\pgfsetdash{}{0pt}%
\pgfpathmoveto{\pgfqpoint{4.847405in}{3.187501in}}%
\pgfpathlineto{\pgfqpoint{4.751641in}{3.474794in}}%
\pgfpathlineto{\pgfqpoint{4.744628in}{3.467781in}}%
\pgfpathlineto{\pgfqpoint{4.743226in}{3.514065in}}%
\pgfpathlineto{\pgfqpoint{4.769874in}{3.476196in}}%
\pgfpathlineto{\pgfqpoint{4.760056in}{3.477599in}}%
\pgfpathlineto{\pgfqpoint{4.855821in}{3.190306in}}%
\pgfpathlineto{\pgfqpoint{4.847405in}{3.187501in}}%
\pgfusepath{fill}%
\end{pgfscope}%
\begin{pgfscope}%
\pgfpathrectangle{\pgfqpoint{1.432000in}{0.528000in}}{\pgfqpoint{3.696000in}{3.696000in}} %
\pgfusepath{clip}%
\pgfsetbuttcap%
\pgfsetroundjoin%
\definecolor{currentfill}{rgb}{0.163625,0.471133,0.558148}%
\pgfsetfillcolor{currentfill}%
\pgfsetlinewidth{0.000000pt}%
\definecolor{currentstroke}{rgb}{0.000000,0.000000,0.000000}%
\pgfsetstrokecolor{currentstroke}%
\pgfsetdash{}{0pt}%
\pgfpathmoveto{\pgfqpoint{4.847178in}{3.188903in}}%
\pgfpathlineto{\pgfqpoint{4.847178in}{3.474148in}}%
\pgfpathlineto{\pgfqpoint{4.838307in}{3.469713in}}%
\pgfpathlineto{\pgfqpoint{4.851613in}{3.514065in}}%
\pgfpathlineto{\pgfqpoint{4.864919in}{3.469713in}}%
\pgfpathlineto{\pgfqpoint{4.856048in}{3.474148in}}%
\pgfpathlineto{\pgfqpoint{4.856048in}{3.188903in}}%
\pgfpathlineto{\pgfqpoint{4.847178in}{3.188903in}}%
\pgfusepath{fill}%
\end{pgfscope}%
\begin{pgfscope}%
\pgfpathrectangle{\pgfqpoint{1.432000in}{0.528000in}}{\pgfqpoint{3.696000in}{3.696000in}} %
\pgfusepath{clip}%
\pgfsetbuttcap%
\pgfsetroundjoin%
\definecolor{currentfill}{rgb}{0.283091,0.110553,0.431554}%
\pgfsetfillcolor{currentfill}%
\pgfsetlinewidth{0.000000pt}%
\definecolor{currentstroke}{rgb}{0.000000,0.000000,0.000000}%
\pgfsetstrokecolor{currentstroke}%
\pgfsetdash{}{0pt}%
\pgfpathmoveto{\pgfqpoint{4.955792in}{3.187501in}}%
\pgfpathlineto{\pgfqpoint{4.860028in}{3.474794in}}%
\pgfpathlineto{\pgfqpoint{4.853015in}{3.467781in}}%
\pgfpathlineto{\pgfqpoint{4.851613in}{3.514065in}}%
\pgfpathlineto{\pgfqpoint{4.878261in}{3.476196in}}%
\pgfpathlineto{\pgfqpoint{4.868443in}{3.477599in}}%
\pgfpathlineto{\pgfqpoint{4.964208in}{3.190306in}}%
\pgfpathlineto{\pgfqpoint{4.955792in}{3.187501in}}%
\pgfusepath{fill}%
\end{pgfscope}%
\begin{pgfscope}%
\pgfpathrectangle{\pgfqpoint{1.432000in}{0.528000in}}{\pgfqpoint{3.696000in}{3.696000in}} %
\pgfusepath{clip}%
\pgfsetbuttcap%
\pgfsetroundjoin%
\definecolor{currentfill}{rgb}{0.226397,0.728888,0.462789}%
\pgfsetfillcolor{currentfill}%
\pgfsetlinewidth{0.000000pt}%
\definecolor{currentstroke}{rgb}{0.000000,0.000000,0.000000}%
\pgfsetstrokecolor{currentstroke}%
\pgfsetdash{}{0pt}%
\pgfpathmoveto{\pgfqpoint{4.955565in}{3.188903in}}%
\pgfpathlineto{\pgfqpoint{4.955565in}{3.474148in}}%
\pgfpathlineto{\pgfqpoint{4.946694in}{3.469713in}}%
\pgfpathlineto{\pgfqpoint{4.960000in}{3.514065in}}%
\pgfpathlineto{\pgfqpoint{4.973306in}{3.469713in}}%
\pgfpathlineto{\pgfqpoint{4.964435in}{3.474148in}}%
\pgfpathlineto{\pgfqpoint{4.964435in}{3.188903in}}%
\pgfpathlineto{\pgfqpoint{4.955565in}{3.188903in}}%
\pgfusepath{fill}%
\end{pgfscope}%
\begin{pgfscope}%
\pgfpathrectangle{\pgfqpoint{1.432000in}{0.528000in}}{\pgfqpoint{3.696000in}{3.696000in}} %
\pgfusepath{clip}%
\pgfsetbuttcap%
\pgfsetroundjoin%
\definecolor{currentfill}{rgb}{0.144759,0.519093,0.556572}%
\pgfsetfillcolor{currentfill}%
\pgfsetlinewidth{0.000000pt}%
\definecolor{currentstroke}{rgb}{0.000000,0.000000,0.000000}%
\pgfsetstrokecolor{currentstroke}%
\pgfsetdash{}{0pt}%
\pgfpathmoveto{\pgfqpoint{1.604435in}{3.297290in}}%
\pgfpathlineto{\pgfqpoint{1.604435in}{3.228820in}}%
\pgfpathlineto{\pgfqpoint{1.613306in}{3.233255in}}%
\pgfpathlineto{\pgfqpoint{1.600000in}{3.188903in}}%
\pgfpathlineto{\pgfqpoint{1.586694in}{3.233255in}}%
\pgfpathlineto{\pgfqpoint{1.595565in}{3.228820in}}%
\pgfpathlineto{\pgfqpoint{1.595565in}{3.297290in}}%
\pgfpathlineto{\pgfqpoint{1.604435in}{3.297290in}}%
\pgfusepath{fill}%
\end{pgfscope}%
\begin{pgfscope}%
\pgfpathrectangle{\pgfqpoint{1.432000in}{0.528000in}}{\pgfqpoint{3.696000in}{3.696000in}} %
\pgfusepath{clip}%
\pgfsetbuttcap%
\pgfsetroundjoin%
\definecolor{currentfill}{rgb}{0.277018,0.050344,0.375715}%
\pgfsetfillcolor{currentfill}%
\pgfsetlinewidth{0.000000pt}%
\definecolor{currentstroke}{rgb}{0.000000,0.000000,0.000000}%
\pgfsetstrokecolor{currentstroke}%
\pgfsetdash{}{0pt}%
\pgfpathmoveto{\pgfqpoint{1.603136in}{3.300426in}}%
\pgfpathlineto{\pgfqpoint{1.683298in}{3.220265in}}%
\pgfpathlineto{\pgfqpoint{1.686434in}{3.229673in}}%
\pgfpathlineto{\pgfqpoint{1.708387in}{3.188903in}}%
\pgfpathlineto{\pgfqpoint{1.667617in}{3.210856in}}%
\pgfpathlineto{\pgfqpoint{1.677025in}{3.213993in}}%
\pgfpathlineto{\pgfqpoint{1.596864in}{3.294154in}}%
\pgfpathlineto{\pgfqpoint{1.603136in}{3.300426in}}%
\pgfusepath{fill}%
\end{pgfscope}%
\begin{pgfscope}%
\pgfpathrectangle{\pgfqpoint{1.432000in}{0.528000in}}{\pgfqpoint{3.696000in}{3.696000in}} %
\pgfusepath{clip}%
\pgfsetbuttcap%
\pgfsetroundjoin%
\definecolor{currentfill}{rgb}{0.212395,0.359683,0.551710}%
\pgfsetfillcolor{currentfill}%
\pgfsetlinewidth{0.000000pt}%
\definecolor{currentstroke}{rgb}{0.000000,0.000000,0.000000}%
\pgfsetstrokecolor{currentstroke}%
\pgfsetdash{}{0pt}%
\pgfpathmoveto{\pgfqpoint{1.712822in}{3.297290in}}%
\pgfpathlineto{\pgfqpoint{1.712822in}{3.228820in}}%
\pgfpathlineto{\pgfqpoint{1.721693in}{3.233255in}}%
\pgfpathlineto{\pgfqpoint{1.708387in}{3.188903in}}%
\pgfpathlineto{\pgfqpoint{1.695081in}{3.233255in}}%
\pgfpathlineto{\pgfqpoint{1.703952in}{3.228820in}}%
\pgfpathlineto{\pgfqpoint{1.703952in}{3.297290in}}%
\pgfpathlineto{\pgfqpoint{1.712822in}{3.297290in}}%
\pgfusepath{fill}%
\end{pgfscope}%
\begin{pgfscope}%
\pgfpathrectangle{\pgfqpoint{1.432000in}{0.528000in}}{\pgfqpoint{3.696000in}{3.696000in}} %
\pgfusepath{clip}%
\pgfsetbuttcap%
\pgfsetroundjoin%
\definecolor{currentfill}{rgb}{0.279574,0.170599,0.479997}%
\pgfsetfillcolor{currentfill}%
\pgfsetlinewidth{0.000000pt}%
\definecolor{currentstroke}{rgb}{0.000000,0.000000,0.000000}%
\pgfsetstrokecolor{currentstroke}%
\pgfsetdash{}{0pt}%
\pgfpathmoveto{\pgfqpoint{1.711523in}{3.300426in}}%
\pgfpathlineto{\pgfqpoint{1.791685in}{3.220265in}}%
\pgfpathlineto{\pgfqpoint{1.794821in}{3.229673in}}%
\pgfpathlineto{\pgfqpoint{1.816774in}{3.188903in}}%
\pgfpathlineto{\pgfqpoint{1.776004in}{3.210856in}}%
\pgfpathlineto{\pgfqpoint{1.785413in}{3.213993in}}%
\pgfpathlineto{\pgfqpoint{1.705251in}{3.294154in}}%
\pgfpathlineto{\pgfqpoint{1.711523in}{3.300426in}}%
\pgfusepath{fill}%
\end{pgfscope}%
\begin{pgfscope}%
\pgfpathrectangle{\pgfqpoint{1.432000in}{0.528000in}}{\pgfqpoint{3.696000in}{3.696000in}} %
\pgfusepath{clip}%
\pgfsetbuttcap%
\pgfsetroundjoin%
\definecolor{currentfill}{rgb}{0.177423,0.437527,0.557565}%
\pgfsetfillcolor{currentfill}%
\pgfsetlinewidth{0.000000pt}%
\definecolor{currentstroke}{rgb}{0.000000,0.000000,0.000000}%
\pgfsetstrokecolor{currentstroke}%
\pgfsetdash{}{0pt}%
\pgfpathmoveto{\pgfqpoint{1.821209in}{3.297290in}}%
\pgfpathlineto{\pgfqpoint{1.821209in}{3.228820in}}%
\pgfpathlineto{\pgfqpoint{1.830080in}{3.233255in}}%
\pgfpathlineto{\pgfqpoint{1.816774in}{3.188903in}}%
\pgfpathlineto{\pgfqpoint{1.803469in}{3.233255in}}%
\pgfpathlineto{\pgfqpoint{1.812339in}{3.228820in}}%
\pgfpathlineto{\pgfqpoint{1.812339in}{3.297290in}}%
\pgfpathlineto{\pgfqpoint{1.821209in}{3.297290in}}%
\pgfusepath{fill}%
\end{pgfscope}%
\begin{pgfscope}%
\pgfpathrectangle{\pgfqpoint{1.432000in}{0.528000in}}{\pgfqpoint{3.696000in}{3.696000in}} %
\pgfusepath{clip}%
\pgfsetbuttcap%
\pgfsetroundjoin%
\definecolor{currentfill}{rgb}{0.147607,0.511733,0.557049}%
\pgfsetfillcolor{currentfill}%
\pgfsetlinewidth{0.000000pt}%
\definecolor{currentstroke}{rgb}{0.000000,0.000000,0.000000}%
\pgfsetstrokecolor{currentstroke}%
\pgfsetdash{}{0pt}%
\pgfpathmoveto{\pgfqpoint{1.929596in}{3.297290in}}%
\pgfpathlineto{\pgfqpoint{1.929596in}{3.228820in}}%
\pgfpathlineto{\pgfqpoint{1.938467in}{3.233255in}}%
\pgfpathlineto{\pgfqpoint{1.925161in}{3.188903in}}%
\pgfpathlineto{\pgfqpoint{1.911856in}{3.233255in}}%
\pgfpathlineto{\pgfqpoint{1.920726in}{3.228820in}}%
\pgfpathlineto{\pgfqpoint{1.920726in}{3.297290in}}%
\pgfpathlineto{\pgfqpoint{1.929596in}{3.297290in}}%
\pgfusepath{fill}%
\end{pgfscope}%
\begin{pgfscope}%
\pgfpathrectangle{\pgfqpoint{1.432000in}{0.528000in}}{\pgfqpoint{3.696000in}{3.696000in}} %
\pgfusepath{clip}%
\pgfsetbuttcap%
\pgfsetroundjoin%
\definecolor{currentfill}{rgb}{0.128729,0.563265,0.551229}%
\pgfsetfillcolor{currentfill}%
\pgfsetlinewidth{0.000000pt}%
\definecolor{currentstroke}{rgb}{0.000000,0.000000,0.000000}%
\pgfsetstrokecolor{currentstroke}%
\pgfsetdash{}{0pt}%
\pgfpathmoveto{\pgfqpoint{2.037984in}{3.297290in}}%
\pgfpathlineto{\pgfqpoint{2.037984in}{3.228820in}}%
\pgfpathlineto{\pgfqpoint{2.046854in}{3.233255in}}%
\pgfpathlineto{\pgfqpoint{2.033548in}{3.188903in}}%
\pgfpathlineto{\pgfqpoint{2.020243in}{3.233255in}}%
\pgfpathlineto{\pgfqpoint{2.029113in}{3.228820in}}%
\pgfpathlineto{\pgfqpoint{2.029113in}{3.297290in}}%
\pgfpathlineto{\pgfqpoint{2.037984in}{3.297290in}}%
\pgfusepath{fill}%
\end{pgfscope}%
\begin{pgfscope}%
\pgfpathrectangle{\pgfqpoint{1.432000in}{0.528000in}}{\pgfqpoint{3.696000in}{3.696000in}} %
\pgfusepath{clip}%
\pgfsetbuttcap%
\pgfsetroundjoin%
\definecolor{currentfill}{rgb}{0.273006,0.204520,0.501721}%
\pgfsetfillcolor{currentfill}%
\pgfsetlinewidth{0.000000pt}%
\definecolor{currentstroke}{rgb}{0.000000,0.000000,0.000000}%
\pgfsetstrokecolor{currentstroke}%
\pgfsetdash{}{0pt}%
\pgfpathmoveto{\pgfqpoint{2.145072in}{3.294154in}}%
\pgfpathlineto{\pgfqpoint{2.064910in}{3.213993in}}%
\pgfpathlineto{\pgfqpoint{2.074318in}{3.210856in}}%
\pgfpathlineto{\pgfqpoint{2.033548in}{3.188903in}}%
\pgfpathlineto{\pgfqpoint{2.055502in}{3.229673in}}%
\pgfpathlineto{\pgfqpoint{2.058638in}{3.220265in}}%
\pgfpathlineto{\pgfqpoint{2.138799in}{3.300426in}}%
\pgfpathlineto{\pgfqpoint{2.145072in}{3.294154in}}%
\pgfusepath{fill}%
\end{pgfscope}%
\begin{pgfscope}%
\pgfpathrectangle{\pgfqpoint{1.432000in}{0.528000in}}{\pgfqpoint{3.696000in}{3.696000in}} %
\pgfusepath{clip}%
\pgfsetbuttcap%
\pgfsetroundjoin%
\definecolor{currentfill}{rgb}{0.154815,0.493313,0.557840}%
\pgfsetfillcolor{currentfill}%
\pgfsetlinewidth{0.000000pt}%
\definecolor{currentstroke}{rgb}{0.000000,0.000000,0.000000}%
\pgfsetstrokecolor{currentstroke}%
\pgfsetdash{}{0pt}%
\pgfpathmoveto{\pgfqpoint{2.146371in}{3.297290in}}%
\pgfpathlineto{\pgfqpoint{2.146371in}{3.228820in}}%
\pgfpathlineto{\pgfqpoint{2.155241in}{3.233255in}}%
\pgfpathlineto{\pgfqpoint{2.141935in}{3.188903in}}%
\pgfpathlineto{\pgfqpoint{2.128630in}{3.233255in}}%
\pgfpathlineto{\pgfqpoint{2.137500in}{3.228820in}}%
\pgfpathlineto{\pgfqpoint{2.137500in}{3.297290in}}%
\pgfpathlineto{\pgfqpoint{2.146371in}{3.297290in}}%
\pgfusepath{fill}%
\end{pgfscope}%
\begin{pgfscope}%
\pgfpathrectangle{\pgfqpoint{1.432000in}{0.528000in}}{\pgfqpoint{3.696000in}{3.696000in}} %
\pgfusepath{clip}%
\pgfsetbuttcap%
\pgfsetroundjoin%
\definecolor{currentfill}{rgb}{0.253935,0.265254,0.529983}%
\pgfsetfillcolor{currentfill}%
\pgfsetlinewidth{0.000000pt}%
\definecolor{currentstroke}{rgb}{0.000000,0.000000,0.000000}%
\pgfsetstrokecolor{currentstroke}%
\pgfsetdash{}{0pt}%
\pgfpathmoveto{\pgfqpoint{2.253459in}{3.294154in}}%
\pgfpathlineto{\pgfqpoint{2.173297in}{3.213993in}}%
\pgfpathlineto{\pgfqpoint{2.182706in}{3.210856in}}%
\pgfpathlineto{\pgfqpoint{2.141935in}{3.188903in}}%
\pgfpathlineto{\pgfqpoint{2.163889in}{3.229673in}}%
\pgfpathlineto{\pgfqpoint{2.167025in}{3.220265in}}%
\pgfpathlineto{\pgfqpoint{2.247186in}{3.300426in}}%
\pgfpathlineto{\pgfqpoint{2.253459in}{3.294154in}}%
\pgfusepath{fill}%
\end{pgfscope}%
\begin{pgfscope}%
\pgfpathrectangle{\pgfqpoint{1.432000in}{0.528000in}}{\pgfqpoint{3.696000in}{3.696000in}} %
\pgfusepath{clip}%
\pgfsetbuttcap%
\pgfsetroundjoin%
\definecolor{currentfill}{rgb}{0.267968,0.223549,0.512008}%
\pgfsetfillcolor{currentfill}%
\pgfsetlinewidth{0.000000pt}%
\definecolor{currentstroke}{rgb}{0.000000,0.000000,0.000000}%
\pgfsetstrokecolor{currentstroke}%
\pgfsetdash{}{0pt}%
\pgfpathmoveto{\pgfqpoint{2.254758in}{3.297290in}}%
\pgfpathlineto{\pgfqpoint{2.254758in}{3.228820in}}%
\pgfpathlineto{\pgfqpoint{2.263628in}{3.233255in}}%
\pgfpathlineto{\pgfqpoint{2.250323in}{3.188903in}}%
\pgfpathlineto{\pgfqpoint{2.237017in}{3.233255in}}%
\pgfpathlineto{\pgfqpoint{2.245887in}{3.228820in}}%
\pgfpathlineto{\pgfqpoint{2.245887in}{3.297290in}}%
\pgfpathlineto{\pgfqpoint{2.254758in}{3.297290in}}%
\pgfusepath{fill}%
\end{pgfscope}%
\begin{pgfscope}%
\pgfpathrectangle{\pgfqpoint{1.432000in}{0.528000in}}{\pgfqpoint{3.696000in}{3.696000in}} %
\pgfusepath{clip}%
\pgfsetbuttcap%
\pgfsetroundjoin%
\definecolor{currentfill}{rgb}{0.233603,0.313828,0.543914}%
\pgfsetfillcolor{currentfill}%
\pgfsetlinewidth{0.000000pt}%
\definecolor{currentstroke}{rgb}{0.000000,0.000000,0.000000}%
\pgfsetstrokecolor{currentstroke}%
\pgfsetdash{}{0pt}%
\pgfpathmoveto{\pgfqpoint{2.361846in}{3.294154in}}%
\pgfpathlineto{\pgfqpoint{2.281684in}{3.213993in}}%
\pgfpathlineto{\pgfqpoint{2.291093in}{3.210856in}}%
\pgfpathlineto{\pgfqpoint{2.250323in}{3.188903in}}%
\pgfpathlineto{\pgfqpoint{2.272276in}{3.229673in}}%
\pgfpathlineto{\pgfqpoint{2.275412in}{3.220265in}}%
\pgfpathlineto{\pgfqpoint{2.355574in}{3.300426in}}%
\pgfpathlineto{\pgfqpoint{2.361846in}{3.294154in}}%
\pgfusepath{fill}%
\end{pgfscope}%
\begin{pgfscope}%
\pgfpathrectangle{\pgfqpoint{1.432000in}{0.528000in}}{\pgfqpoint{3.696000in}{3.696000in}} %
\pgfusepath{clip}%
\pgfsetbuttcap%
\pgfsetroundjoin%
\definecolor{currentfill}{rgb}{0.281924,0.089666,0.412415}%
\pgfsetfillcolor{currentfill}%
\pgfsetlinewidth{0.000000pt}%
\definecolor{currentstroke}{rgb}{0.000000,0.000000,0.000000}%
\pgfsetstrokecolor{currentstroke}%
\pgfsetdash{}{0pt}%
\pgfpathmoveto{\pgfqpoint{2.470233in}{3.294154in}}%
\pgfpathlineto{\pgfqpoint{2.390071in}{3.213993in}}%
\pgfpathlineto{\pgfqpoint{2.399480in}{3.210856in}}%
\pgfpathlineto{\pgfqpoint{2.358710in}{3.188903in}}%
\pgfpathlineto{\pgfqpoint{2.380663in}{3.229673in}}%
\pgfpathlineto{\pgfqpoint{2.383799in}{3.220265in}}%
\pgfpathlineto{\pgfqpoint{2.463961in}{3.300426in}}%
\pgfpathlineto{\pgfqpoint{2.470233in}{3.294154in}}%
\pgfusepath{fill}%
\end{pgfscope}%
\begin{pgfscope}%
\pgfpathrectangle{\pgfqpoint{1.432000in}{0.528000in}}{\pgfqpoint{3.696000in}{3.696000in}} %
\pgfusepath{clip}%
\pgfsetbuttcap%
\pgfsetroundjoin%
\definecolor{currentfill}{rgb}{0.153364,0.497000,0.557724}%
\pgfsetfillcolor{currentfill}%
\pgfsetlinewidth{0.000000pt}%
\definecolor{currentstroke}{rgb}{0.000000,0.000000,0.000000}%
\pgfsetstrokecolor{currentstroke}%
\pgfsetdash{}{0pt}%
\pgfpathmoveto{\pgfqpoint{2.467097in}{3.292855in}}%
\pgfpathlineto{\pgfqpoint{2.398626in}{3.292855in}}%
\pgfpathlineto{\pgfqpoint{2.403062in}{3.283985in}}%
\pgfpathlineto{\pgfqpoint{2.358710in}{3.297290in}}%
\pgfpathlineto{\pgfqpoint{2.403062in}{3.310596in}}%
\pgfpathlineto{\pgfqpoint{2.398626in}{3.301726in}}%
\pgfpathlineto{\pgfqpoint{2.467097in}{3.301726in}}%
\pgfpathlineto{\pgfqpoint{2.467097in}{3.292855in}}%
\pgfusepath{fill}%
\end{pgfscope}%
\begin{pgfscope}%
\pgfpathrectangle{\pgfqpoint{1.432000in}{0.528000in}}{\pgfqpoint{3.696000in}{3.696000in}} %
\pgfusepath{clip}%
\pgfsetbuttcap%
\pgfsetroundjoin%
\definecolor{currentfill}{rgb}{0.180653,0.701402,0.488189}%
\pgfsetfillcolor{currentfill}%
\pgfsetlinewidth{0.000000pt}%
\definecolor{currentstroke}{rgb}{0.000000,0.000000,0.000000}%
\pgfsetstrokecolor{currentstroke}%
\pgfsetdash{}{0pt}%
\pgfpathmoveto{\pgfqpoint{2.575484in}{3.292855in}}%
\pgfpathlineto{\pgfqpoint{2.507014in}{3.292855in}}%
\pgfpathlineto{\pgfqpoint{2.511449in}{3.283985in}}%
\pgfpathlineto{\pgfqpoint{2.467097in}{3.297290in}}%
\pgfpathlineto{\pgfqpoint{2.511449in}{3.310596in}}%
\pgfpathlineto{\pgfqpoint{2.507014in}{3.301726in}}%
\pgfpathlineto{\pgfqpoint{2.575484in}{3.301726in}}%
\pgfpathlineto{\pgfqpoint{2.575484in}{3.292855in}}%
\pgfusepath{fill}%
\end{pgfscope}%
\begin{pgfscope}%
\pgfpathrectangle{\pgfqpoint{1.432000in}{0.528000in}}{\pgfqpoint{3.696000in}{3.696000in}} %
\pgfusepath{clip}%
\pgfsetbuttcap%
\pgfsetroundjoin%
\definecolor{currentfill}{rgb}{0.267004,0.004874,0.329415}%
\pgfsetfillcolor{currentfill}%
\pgfsetlinewidth{0.000000pt}%
\definecolor{currentstroke}{rgb}{0.000000,0.000000,0.000000}%
\pgfsetstrokecolor{currentstroke}%
\pgfsetdash{}{0pt}%
\pgfpathmoveto{\pgfqpoint{2.579919in}{3.297290in}}%
\pgfpathlineto{\pgfqpoint{2.577701in}{3.301131in}}%
\pgfpathlineto{\pgfqpoint{2.573266in}{3.301131in}}%
\pgfpathlineto{\pgfqpoint{2.571049in}{3.297290in}}%
\pgfpathlineto{\pgfqpoint{2.573266in}{3.293449in}}%
\pgfpathlineto{\pgfqpoint{2.577701in}{3.293449in}}%
\pgfpathlineto{\pgfqpoint{2.579919in}{3.297290in}}%
\pgfpathlineto{\pgfqpoint{2.577701in}{3.301131in}}%
\pgfusepath{fill}%
\end{pgfscope}%
\begin{pgfscope}%
\pgfpathrectangle{\pgfqpoint{1.432000in}{0.528000in}}{\pgfqpoint{3.696000in}{3.696000in}} %
\pgfusepath{clip}%
\pgfsetbuttcap%
\pgfsetroundjoin%
\definecolor{currentfill}{rgb}{0.194100,0.399323,0.555565}%
\pgfsetfillcolor{currentfill}%
\pgfsetlinewidth{0.000000pt}%
\definecolor{currentstroke}{rgb}{0.000000,0.000000,0.000000}%
\pgfsetstrokecolor{currentstroke}%
\pgfsetdash{}{0pt}%
\pgfpathmoveto{\pgfqpoint{2.683871in}{3.292855in}}%
\pgfpathlineto{\pgfqpoint{2.615401in}{3.292855in}}%
\pgfpathlineto{\pgfqpoint{2.619836in}{3.283985in}}%
\pgfpathlineto{\pgfqpoint{2.575484in}{3.297290in}}%
\pgfpathlineto{\pgfqpoint{2.619836in}{3.310596in}}%
\pgfpathlineto{\pgfqpoint{2.615401in}{3.301726in}}%
\pgfpathlineto{\pgfqpoint{2.683871in}{3.301726in}}%
\pgfpathlineto{\pgfqpoint{2.683871in}{3.292855in}}%
\pgfusepath{fill}%
\end{pgfscope}%
\begin{pgfscope}%
\pgfpathrectangle{\pgfqpoint{1.432000in}{0.528000in}}{\pgfqpoint{3.696000in}{3.696000in}} %
\pgfusepath{clip}%
\pgfsetbuttcap%
\pgfsetroundjoin%
\definecolor{currentfill}{rgb}{0.235526,0.309527,0.542944}%
\pgfsetfillcolor{currentfill}%
\pgfsetlinewidth{0.000000pt}%
\definecolor{currentstroke}{rgb}{0.000000,0.000000,0.000000}%
\pgfsetstrokecolor{currentstroke}%
\pgfsetdash{}{0pt}%
\pgfpathmoveto{\pgfqpoint{2.688306in}{3.297290in}}%
\pgfpathlineto{\pgfqpoint{2.686089in}{3.301131in}}%
\pgfpathlineto{\pgfqpoint{2.681653in}{3.301131in}}%
\pgfpathlineto{\pgfqpoint{2.679436in}{3.297290in}}%
\pgfpathlineto{\pgfqpoint{2.681653in}{3.293449in}}%
\pgfpathlineto{\pgfqpoint{2.686089in}{3.293449in}}%
\pgfpathlineto{\pgfqpoint{2.688306in}{3.297290in}}%
\pgfpathlineto{\pgfqpoint{2.686089in}{3.301131in}}%
\pgfusepath{fill}%
\end{pgfscope}%
\begin{pgfscope}%
\pgfpathrectangle{\pgfqpoint{1.432000in}{0.528000in}}{\pgfqpoint{3.696000in}{3.696000in}} %
\pgfusepath{clip}%
\pgfsetbuttcap%
\pgfsetroundjoin%
\definecolor{currentfill}{rgb}{0.194100,0.399323,0.555565}%
\pgfsetfillcolor{currentfill}%
\pgfsetlinewidth{0.000000pt}%
\definecolor{currentstroke}{rgb}{0.000000,0.000000,0.000000}%
\pgfsetstrokecolor{currentstroke}%
\pgfsetdash{}{0pt}%
\pgfpathmoveto{\pgfqpoint{2.796693in}{3.297290in}}%
\pgfpathlineto{\pgfqpoint{2.794476in}{3.301131in}}%
\pgfpathlineto{\pgfqpoint{2.790040in}{3.301131in}}%
\pgfpathlineto{\pgfqpoint{2.787823in}{3.297290in}}%
\pgfpathlineto{\pgfqpoint{2.790040in}{3.293449in}}%
\pgfpathlineto{\pgfqpoint{2.794476in}{3.293449in}}%
\pgfpathlineto{\pgfqpoint{2.796693in}{3.297290in}}%
\pgfpathlineto{\pgfqpoint{2.794476in}{3.301131in}}%
\pgfusepath{fill}%
\end{pgfscope}%
\begin{pgfscope}%
\pgfpathrectangle{\pgfqpoint{1.432000in}{0.528000in}}{\pgfqpoint{3.696000in}{3.696000in}} %
\pgfusepath{clip}%
\pgfsetbuttcap%
\pgfsetroundjoin%
\definecolor{currentfill}{rgb}{0.283072,0.130895,0.449241}%
\pgfsetfillcolor{currentfill}%
\pgfsetlinewidth{0.000000pt}%
\definecolor{currentstroke}{rgb}{0.000000,0.000000,0.000000}%
\pgfsetstrokecolor{currentstroke}%
\pgfsetdash{}{0pt}%
\pgfpathmoveto{\pgfqpoint{2.787823in}{3.297290in}}%
\pgfpathlineto{\pgfqpoint{2.787823in}{3.365761in}}%
\pgfpathlineto{\pgfqpoint{2.778952in}{3.361325in}}%
\pgfpathlineto{\pgfqpoint{2.792258in}{3.405677in}}%
\pgfpathlineto{\pgfqpoint{2.805564in}{3.361325in}}%
\pgfpathlineto{\pgfqpoint{2.796693in}{3.365761in}}%
\pgfpathlineto{\pgfqpoint{2.796693in}{3.297290in}}%
\pgfpathlineto{\pgfqpoint{2.787823in}{3.297290in}}%
\pgfusepath{fill}%
\end{pgfscope}%
\begin{pgfscope}%
\pgfpathrectangle{\pgfqpoint{1.432000in}{0.528000in}}{\pgfqpoint{3.696000in}{3.696000in}} %
\pgfusepath{clip}%
\pgfsetbuttcap%
\pgfsetroundjoin%
\definecolor{currentfill}{rgb}{0.280894,0.078907,0.402329}%
\pgfsetfillcolor{currentfill}%
\pgfsetlinewidth{0.000000pt}%
\definecolor{currentstroke}{rgb}{0.000000,0.000000,0.000000}%
\pgfsetstrokecolor{currentstroke}%
\pgfsetdash{}{0pt}%
\pgfpathmoveto{\pgfqpoint{2.905080in}{3.297290in}}%
\pgfpathlineto{\pgfqpoint{2.902863in}{3.301131in}}%
\pgfpathlineto{\pgfqpoint{2.898428in}{3.301131in}}%
\pgfpathlineto{\pgfqpoint{2.896210in}{3.297290in}}%
\pgfpathlineto{\pgfqpoint{2.898428in}{3.293449in}}%
\pgfpathlineto{\pgfqpoint{2.902863in}{3.293449in}}%
\pgfpathlineto{\pgfqpoint{2.905080in}{3.297290in}}%
\pgfpathlineto{\pgfqpoint{2.902863in}{3.301131in}}%
\pgfusepath{fill}%
\end{pgfscope}%
\begin{pgfscope}%
\pgfpathrectangle{\pgfqpoint{1.432000in}{0.528000in}}{\pgfqpoint{3.696000in}{3.696000in}} %
\pgfusepath{clip}%
\pgfsetbuttcap%
\pgfsetroundjoin%
\definecolor{currentfill}{rgb}{0.246811,0.283237,0.535941}%
\pgfsetfillcolor{currentfill}%
\pgfsetlinewidth{0.000000pt}%
\definecolor{currentstroke}{rgb}{0.000000,0.000000,0.000000}%
\pgfsetstrokecolor{currentstroke}%
\pgfsetdash{}{0pt}%
\pgfpathmoveto{\pgfqpoint{2.896210in}{3.297290in}}%
\pgfpathlineto{\pgfqpoint{2.896210in}{3.365761in}}%
\pgfpathlineto{\pgfqpoint{2.887340in}{3.361325in}}%
\pgfpathlineto{\pgfqpoint{2.900645in}{3.405677in}}%
\pgfpathlineto{\pgfqpoint{2.913951in}{3.361325in}}%
\pgfpathlineto{\pgfqpoint{2.905080in}{3.365761in}}%
\pgfpathlineto{\pgfqpoint{2.905080in}{3.297290in}}%
\pgfpathlineto{\pgfqpoint{2.896210in}{3.297290in}}%
\pgfusepath{fill}%
\end{pgfscope}%
\begin{pgfscope}%
\pgfpathrectangle{\pgfqpoint{1.432000in}{0.528000in}}{\pgfqpoint{3.696000in}{3.696000in}} %
\pgfusepath{clip}%
\pgfsetbuttcap%
\pgfsetroundjoin%
\definecolor{currentfill}{rgb}{0.201239,0.383670,0.554294}%
\pgfsetfillcolor{currentfill}%
\pgfsetlinewidth{0.000000pt}%
\definecolor{currentstroke}{rgb}{0.000000,0.000000,0.000000}%
\pgfsetstrokecolor{currentstroke}%
\pgfsetdash{}{0pt}%
\pgfpathmoveto{\pgfqpoint{3.004597in}{3.297290in}}%
\pgfpathlineto{\pgfqpoint{3.004597in}{3.365761in}}%
\pgfpathlineto{\pgfqpoint{2.995727in}{3.361325in}}%
\pgfpathlineto{\pgfqpoint{3.009032in}{3.405677in}}%
\pgfpathlineto{\pgfqpoint{3.022338in}{3.361325in}}%
\pgfpathlineto{\pgfqpoint{3.013467in}{3.365761in}}%
\pgfpathlineto{\pgfqpoint{3.013467in}{3.297290in}}%
\pgfpathlineto{\pgfqpoint{3.004597in}{3.297290in}}%
\pgfusepath{fill}%
\end{pgfscope}%
\begin{pgfscope}%
\pgfpathrectangle{\pgfqpoint{1.432000in}{0.528000in}}{\pgfqpoint{3.696000in}{3.696000in}} %
\pgfusepath{clip}%
\pgfsetbuttcap%
\pgfsetroundjoin%
\definecolor{currentfill}{rgb}{0.275191,0.194905,0.496005}%
\pgfsetfillcolor{currentfill}%
\pgfsetlinewidth{0.000000pt}%
\definecolor{currentstroke}{rgb}{0.000000,0.000000,0.000000}%
\pgfsetstrokecolor{currentstroke}%
\pgfsetdash{}{0pt}%
\pgfpathmoveto{\pgfqpoint{3.005896in}{3.300426in}}%
\pgfpathlineto{\pgfqpoint{3.086058in}{3.380588in}}%
\pgfpathlineto{\pgfqpoint{3.076649in}{3.383724in}}%
\pgfpathlineto{\pgfqpoint{3.117419in}{3.405677in}}%
\pgfpathlineto{\pgfqpoint{3.095466in}{3.364907in}}%
\pgfpathlineto{\pgfqpoint{3.092330in}{3.374316in}}%
\pgfpathlineto{\pgfqpoint{3.012168in}{3.294154in}}%
\pgfpathlineto{\pgfqpoint{3.005896in}{3.300426in}}%
\pgfusepath{fill}%
\end{pgfscope}%
\begin{pgfscope}%
\pgfpathrectangle{\pgfqpoint{1.432000in}{0.528000in}}{\pgfqpoint{3.696000in}{3.696000in}} %
\pgfusepath{clip}%
\pgfsetbuttcap%
\pgfsetroundjoin%
\definecolor{currentfill}{rgb}{0.183898,0.422383,0.556944}%
\pgfsetfillcolor{currentfill}%
\pgfsetlinewidth{0.000000pt}%
\definecolor{currentstroke}{rgb}{0.000000,0.000000,0.000000}%
\pgfsetstrokecolor{currentstroke}%
\pgfsetdash{}{0pt}%
\pgfpathmoveto{\pgfqpoint{3.112984in}{3.297290in}}%
\pgfpathlineto{\pgfqpoint{3.112984in}{3.365761in}}%
\pgfpathlineto{\pgfqpoint{3.104114in}{3.361325in}}%
\pgfpathlineto{\pgfqpoint{3.117419in}{3.405677in}}%
\pgfpathlineto{\pgfqpoint{3.130725in}{3.361325in}}%
\pgfpathlineto{\pgfqpoint{3.121855in}{3.365761in}}%
\pgfpathlineto{\pgfqpoint{3.121855in}{3.297290in}}%
\pgfpathlineto{\pgfqpoint{3.112984in}{3.297290in}}%
\pgfusepath{fill}%
\end{pgfscope}%
\begin{pgfscope}%
\pgfpathrectangle{\pgfqpoint{1.432000in}{0.528000in}}{\pgfqpoint{3.696000in}{3.696000in}} %
\pgfusepath{clip}%
\pgfsetbuttcap%
\pgfsetroundjoin%
\definecolor{currentfill}{rgb}{0.221989,0.339161,0.548752}%
\pgfsetfillcolor{currentfill}%
\pgfsetlinewidth{0.000000pt}%
\definecolor{currentstroke}{rgb}{0.000000,0.000000,0.000000}%
\pgfsetstrokecolor{currentstroke}%
\pgfsetdash{}{0pt}%
\pgfpathmoveto{\pgfqpoint{3.221371in}{3.297290in}}%
\pgfpathlineto{\pgfqpoint{3.221371in}{3.365761in}}%
\pgfpathlineto{\pgfqpoint{3.212501in}{3.361325in}}%
\pgfpathlineto{\pgfqpoint{3.225806in}{3.405677in}}%
\pgfpathlineto{\pgfqpoint{3.239112in}{3.361325in}}%
\pgfpathlineto{\pgfqpoint{3.230242in}{3.365761in}}%
\pgfpathlineto{\pgfqpoint{3.230242in}{3.297290in}}%
\pgfpathlineto{\pgfqpoint{3.221371in}{3.297290in}}%
\pgfusepath{fill}%
\end{pgfscope}%
\begin{pgfscope}%
\pgfpathrectangle{\pgfqpoint{1.432000in}{0.528000in}}{\pgfqpoint{3.696000in}{3.696000in}} %
\pgfusepath{clip}%
\pgfsetbuttcap%
\pgfsetroundjoin%
\definecolor{currentfill}{rgb}{0.279566,0.067836,0.391917}%
\pgfsetfillcolor{currentfill}%
\pgfsetlinewidth{0.000000pt}%
\definecolor{currentstroke}{rgb}{0.000000,0.000000,0.000000}%
\pgfsetstrokecolor{currentstroke}%
\pgfsetdash{}{0pt}%
\pgfpathmoveto{\pgfqpoint{3.329758in}{3.297290in}}%
\pgfpathlineto{\pgfqpoint{3.329758in}{3.474148in}}%
\pgfpathlineto{\pgfqpoint{3.320888in}{3.469713in}}%
\pgfpathlineto{\pgfqpoint{3.334194in}{3.514065in}}%
\pgfpathlineto{\pgfqpoint{3.347499in}{3.469713in}}%
\pgfpathlineto{\pgfqpoint{3.338629in}{3.474148in}}%
\pgfpathlineto{\pgfqpoint{3.338629in}{3.297290in}}%
\pgfpathlineto{\pgfqpoint{3.329758in}{3.297290in}}%
\pgfusepath{fill}%
\end{pgfscope}%
\begin{pgfscope}%
\pgfpathrectangle{\pgfqpoint{1.432000in}{0.528000in}}{\pgfqpoint{3.696000in}{3.696000in}} %
\pgfusepath{clip}%
\pgfsetbuttcap%
\pgfsetroundjoin%
\definecolor{currentfill}{rgb}{0.269944,0.014625,0.341379}%
\pgfsetfillcolor{currentfill}%
\pgfsetlinewidth{0.000000pt}%
\definecolor{currentstroke}{rgb}{0.000000,0.000000,0.000000}%
\pgfsetstrokecolor{currentstroke}%
\pgfsetdash{}{0pt}%
\pgfpathmoveto{\pgfqpoint{3.438614in}{3.295307in}}%
\pgfpathlineto{\pgfqpoint{3.348078in}{3.476378in}}%
\pgfpathlineto{\pgfqpoint{3.342127in}{3.468444in}}%
\pgfpathlineto{\pgfqpoint{3.334194in}{3.514065in}}%
\pgfpathlineto{\pgfqpoint{3.365929in}{3.480345in}}%
\pgfpathlineto{\pgfqpoint{3.356012in}{3.480345in}}%
\pgfpathlineto{\pgfqpoint{3.446548in}{3.299274in}}%
\pgfpathlineto{\pgfqpoint{3.438614in}{3.295307in}}%
\pgfusepath{fill}%
\end{pgfscope}%
\begin{pgfscope}%
\pgfpathrectangle{\pgfqpoint{1.432000in}{0.528000in}}{\pgfqpoint{3.696000in}{3.696000in}} %
\pgfusepath{clip}%
\pgfsetbuttcap%
\pgfsetroundjoin%
\definecolor{currentfill}{rgb}{0.277134,0.185228,0.489898}%
\pgfsetfillcolor{currentfill}%
\pgfsetlinewidth{0.000000pt}%
\definecolor{currentstroke}{rgb}{0.000000,0.000000,0.000000}%
\pgfsetstrokecolor{currentstroke}%
\pgfsetdash{}{0pt}%
\pgfpathmoveto{\pgfqpoint{3.547001in}{3.295307in}}%
\pgfpathlineto{\pgfqpoint{3.456465in}{3.476378in}}%
\pgfpathlineto{\pgfqpoint{3.450515in}{3.468444in}}%
\pgfpathlineto{\pgfqpoint{3.442581in}{3.514065in}}%
\pgfpathlineto{\pgfqpoint{3.474316in}{3.480345in}}%
\pgfpathlineto{\pgfqpoint{3.464399in}{3.480345in}}%
\pgfpathlineto{\pgfqpoint{3.554935in}{3.299274in}}%
\pgfpathlineto{\pgfqpoint{3.547001in}{3.295307in}}%
\pgfusepath{fill}%
\end{pgfscope}%
\begin{pgfscope}%
\pgfpathrectangle{\pgfqpoint{1.432000in}{0.528000in}}{\pgfqpoint{3.696000in}{3.696000in}} %
\pgfusepath{clip}%
\pgfsetbuttcap%
\pgfsetroundjoin%
\definecolor{currentfill}{rgb}{0.223925,0.334994,0.548053}%
\pgfsetfillcolor{currentfill}%
\pgfsetlinewidth{0.000000pt}%
\definecolor{currentstroke}{rgb}{0.000000,0.000000,0.000000}%
\pgfsetstrokecolor{currentstroke}%
\pgfsetdash{}{0pt}%
\pgfpathmoveto{\pgfqpoint{3.656219in}{3.294154in}}%
\pgfpathlineto{\pgfqpoint{3.467670in}{3.482703in}}%
\pgfpathlineto{\pgfqpoint{3.464534in}{3.473294in}}%
\pgfpathlineto{\pgfqpoint{3.442581in}{3.514065in}}%
\pgfpathlineto{\pgfqpoint{3.483351in}{3.492111in}}%
\pgfpathlineto{\pgfqpoint{3.473942in}{3.488975in}}%
\pgfpathlineto{\pgfqpoint{3.662491in}{3.300426in}}%
\pgfpathlineto{\pgfqpoint{3.656219in}{3.294154in}}%
\pgfusepath{fill}%
\end{pgfscope}%
\begin{pgfscope}%
\pgfpathrectangle{\pgfqpoint{1.432000in}{0.528000in}}{\pgfqpoint{3.696000in}{3.696000in}} %
\pgfusepath{clip}%
\pgfsetbuttcap%
\pgfsetroundjoin%
\definecolor{currentfill}{rgb}{0.258965,0.251537,0.524736}%
\pgfsetfillcolor{currentfill}%
\pgfsetlinewidth{0.000000pt}%
\definecolor{currentstroke}{rgb}{0.000000,0.000000,0.000000}%
\pgfsetstrokecolor{currentstroke}%
\pgfsetdash{}{0pt}%
\pgfpathmoveto{\pgfqpoint{3.765282in}{3.293600in}}%
\pgfpathlineto{\pgfqpoint{3.473333in}{3.488232in}}%
\pgfpathlineto{\pgfqpoint{3.472103in}{3.478392in}}%
\pgfpathlineto{\pgfqpoint{3.442581in}{3.514065in}}%
\pgfpathlineto{\pgfqpoint{3.486864in}{3.500533in}}%
\pgfpathlineto{\pgfqpoint{3.478254in}{3.495613in}}%
\pgfpathlineto{\pgfqpoint{3.770202in}{3.300981in}}%
\pgfpathlineto{\pgfqpoint{3.765282in}{3.293600in}}%
\pgfusepath{fill}%
\end{pgfscope}%
\begin{pgfscope}%
\pgfpathrectangle{\pgfqpoint{1.432000in}{0.528000in}}{\pgfqpoint{3.696000in}{3.696000in}} %
\pgfusepath{clip}%
\pgfsetbuttcap%
\pgfsetroundjoin%
\definecolor{currentfill}{rgb}{0.137339,0.662252,0.515571}%
\pgfsetfillcolor{currentfill}%
\pgfsetlinewidth{0.000000pt}%
\definecolor{currentstroke}{rgb}{0.000000,0.000000,0.000000}%
\pgfsetstrokecolor{currentstroke}%
\pgfsetdash{}{0pt}%
\pgfpathmoveto{\pgfqpoint{3.873669in}{3.293600in}}%
\pgfpathlineto{\pgfqpoint{3.581720in}{3.488232in}}%
\pgfpathlineto{\pgfqpoint{3.580490in}{3.478392in}}%
\pgfpathlineto{\pgfqpoint{3.550968in}{3.514065in}}%
\pgfpathlineto{\pgfqpoint{3.595251in}{3.500533in}}%
\pgfpathlineto{\pgfqpoint{3.586641in}{3.495613in}}%
\pgfpathlineto{\pgfqpoint{3.878589in}{3.300981in}}%
\pgfpathlineto{\pgfqpoint{3.873669in}{3.293600in}}%
\pgfusepath{fill}%
\end{pgfscope}%
\begin{pgfscope}%
\pgfpathrectangle{\pgfqpoint{1.432000in}{0.528000in}}{\pgfqpoint{3.696000in}{3.696000in}} %
\pgfusepath{clip}%
\pgfsetbuttcap%
\pgfsetroundjoin%
\definecolor{currentfill}{rgb}{0.208623,0.367752,0.552675}%
\pgfsetfillcolor{currentfill}%
\pgfsetlinewidth{0.000000pt}%
\definecolor{currentstroke}{rgb}{0.000000,0.000000,0.000000}%
\pgfsetstrokecolor{currentstroke}%
\pgfsetdash{}{0pt}%
\pgfpathmoveto{\pgfqpoint{3.982056in}{3.293600in}}%
\pgfpathlineto{\pgfqpoint{3.690107in}{3.488232in}}%
\pgfpathlineto{\pgfqpoint{3.688877in}{3.478392in}}%
\pgfpathlineto{\pgfqpoint{3.659355in}{3.514065in}}%
\pgfpathlineto{\pgfqpoint{3.703639in}{3.500533in}}%
\pgfpathlineto{\pgfqpoint{3.695028in}{3.495613in}}%
\pgfpathlineto{\pgfqpoint{3.986976in}{3.300981in}}%
\pgfpathlineto{\pgfqpoint{3.982056in}{3.293600in}}%
\pgfusepath{fill}%
\end{pgfscope}%
\begin{pgfscope}%
\pgfpathrectangle{\pgfqpoint{1.432000in}{0.528000in}}{\pgfqpoint{3.696000in}{3.696000in}} %
\pgfusepath{clip}%
\pgfsetbuttcap%
\pgfsetroundjoin%
\definecolor{currentfill}{rgb}{0.250425,0.274290,0.533103}%
\pgfsetfillcolor{currentfill}%
\pgfsetlinewidth{0.000000pt}%
\definecolor{currentstroke}{rgb}{0.000000,0.000000,0.000000}%
\pgfsetstrokecolor{currentstroke}%
\pgfsetdash{}{0pt}%
\pgfpathmoveto{\pgfqpoint{3.981380in}{3.294154in}}%
\pgfpathlineto{\pgfqpoint{3.684444in}{3.591090in}}%
\pgfpathlineto{\pgfqpoint{3.681308in}{3.581682in}}%
\pgfpathlineto{\pgfqpoint{3.659355in}{3.622452in}}%
\pgfpathlineto{\pgfqpoint{3.700125in}{3.600498in}}%
\pgfpathlineto{\pgfqpoint{3.690716in}{3.597362in}}%
\pgfpathlineto{\pgfqpoint{3.987652in}{3.300426in}}%
\pgfpathlineto{\pgfqpoint{3.981380in}{3.294154in}}%
\pgfusepath{fill}%
\end{pgfscope}%
\begin{pgfscope}%
\pgfpathrectangle{\pgfqpoint{1.432000in}{0.528000in}}{\pgfqpoint{3.696000in}{3.696000in}} %
\pgfusepath{clip}%
\pgfsetbuttcap%
\pgfsetroundjoin%
\definecolor{currentfill}{rgb}{0.279574,0.170599,0.479997}%
\pgfsetfillcolor{currentfill}%
\pgfsetlinewidth{0.000000pt}%
\definecolor{currentstroke}{rgb}{0.000000,0.000000,0.000000}%
\pgfsetstrokecolor{currentstroke}%
\pgfsetdash{}{0pt}%
\pgfpathmoveto{\pgfqpoint{4.090443in}{3.293600in}}%
\pgfpathlineto{\pgfqpoint{3.798495in}{3.488232in}}%
\pgfpathlineto{\pgfqpoint{3.797264in}{3.478392in}}%
\pgfpathlineto{\pgfqpoint{3.767742in}{3.514065in}}%
\pgfpathlineto{\pgfqpoint{3.812026in}{3.500533in}}%
\pgfpathlineto{\pgfqpoint{3.803415in}{3.495613in}}%
\pgfpathlineto{\pgfqpoint{4.095363in}{3.300981in}}%
\pgfpathlineto{\pgfqpoint{4.090443in}{3.293600in}}%
\pgfusepath{fill}%
\end{pgfscope}%
\begin{pgfscope}%
\pgfpathrectangle{\pgfqpoint{1.432000in}{0.528000in}}{\pgfqpoint{3.696000in}{3.696000in}} %
\pgfusepath{clip}%
\pgfsetbuttcap%
\pgfsetroundjoin%
\definecolor{currentfill}{rgb}{0.139147,0.533812,0.555298}%
\pgfsetfillcolor{currentfill}%
\pgfsetlinewidth{0.000000pt}%
\definecolor{currentstroke}{rgb}{0.000000,0.000000,0.000000}%
\pgfsetstrokecolor{currentstroke}%
\pgfsetdash{}{0pt}%
\pgfpathmoveto{\pgfqpoint{4.089767in}{3.294154in}}%
\pgfpathlineto{\pgfqpoint{3.792831in}{3.591090in}}%
\pgfpathlineto{\pgfqpoint{3.789695in}{3.581682in}}%
\pgfpathlineto{\pgfqpoint{3.767742in}{3.622452in}}%
\pgfpathlineto{\pgfqpoint{3.808512in}{3.600498in}}%
\pgfpathlineto{\pgfqpoint{3.799104in}{3.597362in}}%
\pgfpathlineto{\pgfqpoint{4.096039in}{3.300426in}}%
\pgfpathlineto{\pgfqpoint{4.089767in}{3.294154in}}%
\pgfusepath{fill}%
\end{pgfscope}%
\begin{pgfscope}%
\pgfpathrectangle{\pgfqpoint{1.432000in}{0.528000in}}{\pgfqpoint{3.696000in}{3.696000in}} %
\pgfusepath{clip}%
\pgfsetbuttcap%
\pgfsetroundjoin%
\definecolor{currentfill}{rgb}{0.430983,0.808473,0.346476}%
\pgfsetfillcolor{currentfill}%
\pgfsetlinewidth{0.000000pt}%
\definecolor{currentstroke}{rgb}{0.000000,0.000000,0.000000}%
\pgfsetstrokecolor{currentstroke}%
\pgfsetdash{}{0pt}%
\pgfpathmoveto{\pgfqpoint{4.198154in}{3.294154in}}%
\pgfpathlineto{\pgfqpoint{3.901218in}{3.591090in}}%
\pgfpathlineto{\pgfqpoint{3.898082in}{3.581682in}}%
\pgfpathlineto{\pgfqpoint{3.876129in}{3.622452in}}%
\pgfpathlineto{\pgfqpoint{3.916899in}{3.600498in}}%
\pgfpathlineto{\pgfqpoint{3.907491in}{3.597362in}}%
\pgfpathlineto{\pgfqpoint{4.204426in}{3.300426in}}%
\pgfpathlineto{\pgfqpoint{4.198154in}{3.294154in}}%
\pgfusepath{fill}%
\end{pgfscope}%
\begin{pgfscope}%
\pgfpathrectangle{\pgfqpoint{1.432000in}{0.528000in}}{\pgfqpoint{3.696000in}{3.696000in}} %
\pgfusepath{clip}%
\pgfsetbuttcap%
\pgfsetroundjoin%
\definecolor{currentfill}{rgb}{0.208030,0.718701,0.472873}%
\pgfsetfillcolor{currentfill}%
\pgfsetlinewidth{0.000000pt}%
\definecolor{currentstroke}{rgb}{0.000000,0.000000,0.000000}%
\pgfsetstrokecolor{currentstroke}%
\pgfsetdash{}{0pt}%
\pgfpathmoveto{\pgfqpoint{4.306541in}{3.294154in}}%
\pgfpathlineto{\pgfqpoint{4.009605in}{3.591090in}}%
\pgfpathlineto{\pgfqpoint{4.006469in}{3.581682in}}%
\pgfpathlineto{\pgfqpoint{3.984516in}{3.622452in}}%
\pgfpathlineto{\pgfqpoint{4.025286in}{3.600498in}}%
\pgfpathlineto{\pgfqpoint{4.015878in}{3.597362in}}%
\pgfpathlineto{\pgfqpoint{4.312814in}{3.300426in}}%
\pgfpathlineto{\pgfqpoint{4.306541in}{3.294154in}}%
\pgfusepath{fill}%
\end{pgfscope}%
\begin{pgfscope}%
\pgfpathrectangle{\pgfqpoint{1.432000in}{0.528000in}}{\pgfqpoint{3.696000in}{3.696000in}} %
\pgfusepath{clip}%
\pgfsetbuttcap%
\pgfsetroundjoin%
\definecolor{currentfill}{rgb}{0.277134,0.185228,0.489898}%
\pgfsetfillcolor{currentfill}%
\pgfsetlinewidth{0.000000pt}%
\definecolor{currentstroke}{rgb}{0.000000,0.000000,0.000000}%
\pgfsetstrokecolor{currentstroke}%
\pgfsetdash{}{0pt}%
\pgfpathmoveto{\pgfqpoint{4.305987in}{3.294830in}}%
\pgfpathlineto{\pgfqpoint{4.111355in}{3.586779in}}%
\pgfpathlineto{\pgfqpoint{4.106434in}{3.578168in}}%
\pgfpathlineto{\pgfqpoint{4.092903in}{3.622452in}}%
\pgfpathlineto{\pgfqpoint{4.128576in}{3.592929in}}%
\pgfpathlineto{\pgfqpoint{4.118735in}{3.591699in}}%
\pgfpathlineto{\pgfqpoint{4.313368in}{3.299751in}}%
\pgfpathlineto{\pgfqpoint{4.305987in}{3.294830in}}%
\pgfusepath{fill}%
\end{pgfscope}%
\begin{pgfscope}%
\pgfpathrectangle{\pgfqpoint{1.432000in}{0.528000in}}{\pgfqpoint{3.696000in}{3.696000in}} %
\pgfusepath{clip}%
\pgfsetbuttcap%
\pgfsetroundjoin%
\definecolor{currentfill}{rgb}{0.204903,0.375746,0.553533}%
\pgfsetfillcolor{currentfill}%
\pgfsetlinewidth{0.000000pt}%
\definecolor{currentstroke}{rgb}{0.000000,0.000000,0.000000}%
\pgfsetstrokecolor{currentstroke}%
\pgfsetdash{}{0pt}%
\pgfpathmoveto{\pgfqpoint{4.414928in}{3.294154in}}%
\pgfpathlineto{\pgfqpoint{4.117993in}{3.591090in}}%
\pgfpathlineto{\pgfqpoint{4.114856in}{3.581682in}}%
\pgfpathlineto{\pgfqpoint{4.092903in}{3.622452in}}%
\pgfpathlineto{\pgfqpoint{4.133673in}{3.600498in}}%
\pgfpathlineto{\pgfqpoint{4.124265in}{3.597362in}}%
\pgfpathlineto{\pgfqpoint{4.421201in}{3.300426in}}%
\pgfpathlineto{\pgfqpoint{4.414928in}{3.294154in}}%
\pgfusepath{fill}%
\end{pgfscope}%
\begin{pgfscope}%
\pgfpathrectangle{\pgfqpoint{1.432000in}{0.528000in}}{\pgfqpoint{3.696000in}{3.696000in}} %
\pgfusepath{clip}%
\pgfsetbuttcap%
\pgfsetroundjoin%
\definecolor{currentfill}{rgb}{0.122312,0.633153,0.530398}%
\pgfsetfillcolor{currentfill}%
\pgfsetlinewidth{0.000000pt}%
\definecolor{currentstroke}{rgb}{0.000000,0.000000,0.000000}%
\pgfsetstrokecolor{currentstroke}%
\pgfsetdash{}{0pt}%
\pgfpathmoveto{\pgfqpoint{4.414374in}{3.294830in}}%
\pgfpathlineto{\pgfqpoint{4.219742in}{3.586779in}}%
\pgfpathlineto{\pgfqpoint{4.214821in}{3.578168in}}%
\pgfpathlineto{\pgfqpoint{4.201290in}{3.622452in}}%
\pgfpathlineto{\pgfqpoint{4.236963in}{3.592929in}}%
\pgfpathlineto{\pgfqpoint{4.227122in}{3.591699in}}%
\pgfpathlineto{\pgfqpoint{4.421755in}{3.299751in}}%
\pgfpathlineto{\pgfqpoint{4.414374in}{3.294830in}}%
\pgfusepath{fill}%
\end{pgfscope}%
\begin{pgfscope}%
\pgfpathrectangle{\pgfqpoint{1.432000in}{0.528000in}}{\pgfqpoint{3.696000in}{3.696000in}} %
\pgfusepath{clip}%
\pgfsetbuttcap%
\pgfsetroundjoin%
\definecolor{currentfill}{rgb}{0.180653,0.701402,0.488189}%
\pgfsetfillcolor{currentfill}%
\pgfsetlinewidth{0.000000pt}%
\definecolor{currentstroke}{rgb}{0.000000,0.000000,0.000000}%
\pgfsetstrokecolor{currentstroke}%
\pgfsetdash{}{0pt}%
\pgfpathmoveto{\pgfqpoint{4.522761in}{3.294830in}}%
\pgfpathlineto{\pgfqpoint{4.328129in}{3.586779in}}%
\pgfpathlineto{\pgfqpoint{4.323209in}{3.578168in}}%
\pgfpathlineto{\pgfqpoint{4.309677in}{3.622452in}}%
\pgfpathlineto{\pgfqpoint{4.345350in}{3.592929in}}%
\pgfpathlineto{\pgfqpoint{4.335510in}{3.591699in}}%
\pgfpathlineto{\pgfqpoint{4.530142in}{3.299751in}}%
\pgfpathlineto{\pgfqpoint{4.522761in}{3.294830in}}%
\pgfusepath{fill}%
\end{pgfscope}%
\begin{pgfscope}%
\pgfpathrectangle{\pgfqpoint{1.432000in}{0.528000in}}{\pgfqpoint{3.696000in}{3.696000in}} %
\pgfusepath{clip}%
\pgfsetbuttcap%
\pgfsetroundjoin%
\definecolor{currentfill}{rgb}{0.244972,0.287675,0.537260}%
\pgfsetfillcolor{currentfill}%
\pgfsetlinewidth{0.000000pt}%
\definecolor{currentstroke}{rgb}{0.000000,0.000000,0.000000}%
\pgfsetstrokecolor{currentstroke}%
\pgfsetdash{}{0pt}%
\pgfpathmoveto{\pgfqpoint{4.522244in}{3.295888in}}%
\pgfpathlineto{\pgfqpoint{4.426480in}{3.583181in}}%
\pgfpathlineto{\pgfqpoint{4.419467in}{3.576168in}}%
\pgfpathlineto{\pgfqpoint{4.418065in}{3.622452in}}%
\pgfpathlineto{\pgfqpoint{4.444713in}{3.584583in}}%
\pgfpathlineto{\pgfqpoint{4.434895in}{3.585986in}}%
\pgfpathlineto{\pgfqpoint{4.530659in}{3.298693in}}%
\pgfpathlineto{\pgfqpoint{4.522244in}{3.295888in}}%
\pgfusepath{fill}%
\end{pgfscope}%
\begin{pgfscope}%
\pgfpathrectangle{\pgfqpoint{1.432000in}{0.528000in}}{\pgfqpoint{3.696000in}{3.696000in}} %
\pgfusepath{clip}%
\pgfsetbuttcap%
\pgfsetroundjoin%
\definecolor{currentfill}{rgb}{0.235526,0.309527,0.542944}%
\pgfsetfillcolor{currentfill}%
\pgfsetlinewidth{0.000000pt}%
\definecolor{currentstroke}{rgb}{0.000000,0.000000,0.000000}%
\pgfsetstrokecolor{currentstroke}%
\pgfsetdash{}{0pt}%
\pgfpathmoveto{\pgfqpoint{4.631148in}{3.294830in}}%
\pgfpathlineto{\pgfqpoint{4.436516in}{3.586779in}}%
\pgfpathlineto{\pgfqpoint{4.431596in}{3.578168in}}%
\pgfpathlineto{\pgfqpoint{4.418065in}{3.622452in}}%
\pgfpathlineto{\pgfqpoint{4.453738in}{3.592929in}}%
\pgfpathlineto{\pgfqpoint{4.443897in}{3.591699in}}%
\pgfpathlineto{\pgfqpoint{4.638529in}{3.299751in}}%
\pgfpathlineto{\pgfqpoint{4.631148in}{3.294830in}}%
\pgfusepath{fill}%
\end{pgfscope}%
\begin{pgfscope}%
\pgfpathrectangle{\pgfqpoint{1.432000in}{0.528000in}}{\pgfqpoint{3.696000in}{3.696000in}} %
\pgfusepath{clip}%
\pgfsetbuttcap%
\pgfsetroundjoin%
\definecolor{currentfill}{rgb}{0.175707,0.697900,0.491033}%
\pgfsetfillcolor{currentfill}%
\pgfsetlinewidth{0.000000pt}%
\definecolor{currentstroke}{rgb}{0.000000,0.000000,0.000000}%
\pgfsetstrokecolor{currentstroke}%
\pgfsetdash{}{0pt}%
\pgfpathmoveto{\pgfqpoint{4.630631in}{3.295888in}}%
\pgfpathlineto{\pgfqpoint{4.534867in}{3.583181in}}%
\pgfpathlineto{\pgfqpoint{4.527854in}{3.576168in}}%
\pgfpathlineto{\pgfqpoint{4.526452in}{3.622452in}}%
\pgfpathlineto{\pgfqpoint{4.553100in}{3.584583in}}%
\pgfpathlineto{\pgfqpoint{4.543282in}{3.585986in}}%
\pgfpathlineto{\pgfqpoint{4.639046in}{3.298693in}}%
\pgfpathlineto{\pgfqpoint{4.630631in}{3.295888in}}%
\pgfusepath{fill}%
\end{pgfscope}%
\begin{pgfscope}%
\pgfpathrectangle{\pgfqpoint{1.432000in}{0.528000in}}{\pgfqpoint{3.696000in}{3.696000in}} %
\pgfusepath{clip}%
\pgfsetbuttcap%
\pgfsetroundjoin%
\definecolor{currentfill}{rgb}{0.185783,0.704891,0.485273}%
\pgfsetfillcolor{currentfill}%
\pgfsetlinewidth{0.000000pt}%
\definecolor{currentstroke}{rgb}{0.000000,0.000000,0.000000}%
\pgfsetstrokecolor{currentstroke}%
\pgfsetdash{}{0pt}%
\pgfpathmoveto{\pgfqpoint{4.739018in}{3.295888in}}%
\pgfpathlineto{\pgfqpoint{4.643254in}{3.583181in}}%
\pgfpathlineto{\pgfqpoint{4.636241in}{3.576168in}}%
\pgfpathlineto{\pgfqpoint{4.634839in}{3.622452in}}%
\pgfpathlineto{\pgfqpoint{4.661487in}{3.584583in}}%
\pgfpathlineto{\pgfqpoint{4.651669in}{3.585986in}}%
\pgfpathlineto{\pgfqpoint{4.747433in}{3.298693in}}%
\pgfpathlineto{\pgfqpoint{4.739018in}{3.295888in}}%
\pgfusepath{fill}%
\end{pgfscope}%
\begin{pgfscope}%
\pgfpathrectangle{\pgfqpoint{1.432000in}{0.528000in}}{\pgfqpoint{3.696000in}{3.696000in}} %
\pgfusepath{clip}%
\pgfsetbuttcap%
\pgfsetroundjoin%
\definecolor{currentfill}{rgb}{0.265145,0.232956,0.516599}%
\pgfsetfillcolor{currentfill}%
\pgfsetlinewidth{0.000000pt}%
\definecolor{currentstroke}{rgb}{0.000000,0.000000,0.000000}%
\pgfsetstrokecolor{currentstroke}%
\pgfsetdash{}{0pt}%
\pgfpathmoveto{\pgfqpoint{4.738791in}{3.297290in}}%
\pgfpathlineto{\pgfqpoint{4.738791in}{3.582535in}}%
\pgfpathlineto{\pgfqpoint{4.729920in}{3.578100in}}%
\pgfpathlineto{\pgfqpoint{4.743226in}{3.622452in}}%
\pgfpathlineto{\pgfqpoint{4.756531in}{3.578100in}}%
\pgfpathlineto{\pgfqpoint{4.747661in}{3.582535in}}%
\pgfpathlineto{\pgfqpoint{4.747661in}{3.297290in}}%
\pgfpathlineto{\pgfqpoint{4.738791in}{3.297290in}}%
\pgfusepath{fill}%
\end{pgfscope}%
\begin{pgfscope}%
\pgfpathrectangle{\pgfqpoint{1.432000in}{0.528000in}}{\pgfqpoint{3.696000in}{3.696000in}} %
\pgfusepath{clip}%
\pgfsetbuttcap%
\pgfsetroundjoin%
\definecolor{currentfill}{rgb}{0.218130,0.347432,0.550038}%
\pgfsetfillcolor{currentfill}%
\pgfsetlinewidth{0.000000pt}%
\definecolor{currentstroke}{rgb}{0.000000,0.000000,0.000000}%
\pgfsetstrokecolor{currentstroke}%
\pgfsetdash{}{0pt}%
\pgfpathmoveto{\pgfqpoint{4.847405in}{3.295888in}}%
\pgfpathlineto{\pgfqpoint{4.751641in}{3.583181in}}%
\pgfpathlineto{\pgfqpoint{4.744628in}{3.576168in}}%
\pgfpathlineto{\pgfqpoint{4.743226in}{3.622452in}}%
\pgfpathlineto{\pgfqpoint{4.769874in}{3.584583in}}%
\pgfpathlineto{\pgfqpoint{4.760056in}{3.585986in}}%
\pgfpathlineto{\pgfqpoint{4.855821in}{3.298693in}}%
\pgfpathlineto{\pgfqpoint{4.847405in}{3.295888in}}%
\pgfusepath{fill}%
\end{pgfscope}%
\begin{pgfscope}%
\pgfpathrectangle{\pgfqpoint{1.432000in}{0.528000in}}{\pgfqpoint{3.696000in}{3.696000in}} %
\pgfusepath{clip}%
\pgfsetbuttcap%
\pgfsetroundjoin%
\definecolor{currentfill}{rgb}{0.123463,0.581687,0.547445}%
\pgfsetfillcolor{currentfill}%
\pgfsetlinewidth{0.000000pt}%
\definecolor{currentstroke}{rgb}{0.000000,0.000000,0.000000}%
\pgfsetstrokecolor{currentstroke}%
\pgfsetdash{}{0pt}%
\pgfpathmoveto{\pgfqpoint{4.847178in}{3.297290in}}%
\pgfpathlineto{\pgfqpoint{4.847178in}{3.582535in}}%
\pgfpathlineto{\pgfqpoint{4.838307in}{3.578100in}}%
\pgfpathlineto{\pgfqpoint{4.851613in}{3.622452in}}%
\pgfpathlineto{\pgfqpoint{4.864919in}{3.578100in}}%
\pgfpathlineto{\pgfqpoint{4.856048in}{3.582535in}}%
\pgfpathlineto{\pgfqpoint{4.856048in}{3.297290in}}%
\pgfpathlineto{\pgfqpoint{4.847178in}{3.297290in}}%
\pgfusepath{fill}%
\end{pgfscope}%
\begin{pgfscope}%
\pgfpathrectangle{\pgfqpoint{1.432000in}{0.528000in}}{\pgfqpoint{3.696000in}{3.696000in}} %
\pgfusepath{clip}%
\pgfsetbuttcap%
\pgfsetroundjoin%
\definecolor{currentfill}{rgb}{0.421908,0.805774,0.351910}%
\pgfsetfillcolor{currentfill}%
\pgfsetlinewidth{0.000000pt}%
\definecolor{currentstroke}{rgb}{0.000000,0.000000,0.000000}%
\pgfsetstrokecolor{currentstroke}%
\pgfsetdash{}{0pt}%
\pgfpathmoveto{\pgfqpoint{4.955565in}{3.297290in}}%
\pgfpathlineto{\pgfqpoint{4.955565in}{3.582535in}}%
\pgfpathlineto{\pgfqpoint{4.946694in}{3.578100in}}%
\pgfpathlineto{\pgfqpoint{4.960000in}{3.622452in}}%
\pgfpathlineto{\pgfqpoint{4.973306in}{3.578100in}}%
\pgfpathlineto{\pgfqpoint{4.964435in}{3.582535in}}%
\pgfpathlineto{\pgfqpoint{4.964435in}{3.297290in}}%
\pgfpathlineto{\pgfqpoint{4.955565in}{3.297290in}}%
\pgfusepath{fill}%
\end{pgfscope}%
\begin{pgfscope}%
\pgfpathrectangle{\pgfqpoint{1.432000in}{0.528000in}}{\pgfqpoint{3.696000in}{3.696000in}} %
\pgfusepath{clip}%
\pgfsetbuttcap%
\pgfsetroundjoin%
\definecolor{currentfill}{rgb}{0.269944,0.014625,0.341379}%
\pgfsetfillcolor{currentfill}%
\pgfsetlinewidth{0.000000pt}%
\definecolor{currentstroke}{rgb}{0.000000,0.000000,0.000000}%
\pgfsetstrokecolor{currentstroke}%
\pgfsetdash{}{0pt}%
\pgfpathmoveto{\pgfqpoint{1.604435in}{3.405677in}}%
\pgfpathlineto{\pgfqpoint{1.604435in}{3.228820in}}%
\pgfpathlineto{\pgfqpoint{1.613306in}{3.233255in}}%
\pgfpathlineto{\pgfqpoint{1.600000in}{3.188903in}}%
\pgfpathlineto{\pgfqpoint{1.586694in}{3.233255in}}%
\pgfpathlineto{\pgfqpoint{1.595565in}{3.228820in}}%
\pgfpathlineto{\pgfqpoint{1.595565in}{3.405677in}}%
\pgfpathlineto{\pgfqpoint{1.604435in}{3.405677in}}%
\pgfusepath{fill}%
\end{pgfscope}%
\begin{pgfscope}%
\pgfpathrectangle{\pgfqpoint{1.432000in}{0.528000in}}{\pgfqpoint{3.696000in}{3.696000in}} %
\pgfusepath{clip}%
\pgfsetbuttcap%
\pgfsetroundjoin%
\definecolor{currentfill}{rgb}{0.235526,0.309527,0.542944}%
\pgfsetfillcolor{currentfill}%
\pgfsetlinewidth{0.000000pt}%
\definecolor{currentstroke}{rgb}{0.000000,0.000000,0.000000}%
\pgfsetstrokecolor{currentstroke}%
\pgfsetdash{}{0pt}%
\pgfpathmoveto{\pgfqpoint{1.604435in}{3.405677in}}%
\pgfpathlineto{\pgfqpoint{1.604435in}{3.337207in}}%
\pgfpathlineto{\pgfqpoint{1.613306in}{3.341642in}}%
\pgfpathlineto{\pgfqpoint{1.600000in}{3.297290in}}%
\pgfpathlineto{\pgfqpoint{1.586694in}{3.341642in}}%
\pgfpathlineto{\pgfqpoint{1.595565in}{3.337207in}}%
\pgfpathlineto{\pgfqpoint{1.595565in}{3.405677in}}%
\pgfpathlineto{\pgfqpoint{1.604435in}{3.405677in}}%
\pgfusepath{fill}%
\end{pgfscope}%
\begin{pgfscope}%
\pgfpathrectangle{\pgfqpoint{1.432000in}{0.528000in}}{\pgfqpoint{3.696000in}{3.696000in}} %
\pgfusepath{clip}%
\pgfsetbuttcap%
\pgfsetroundjoin%
\definecolor{currentfill}{rgb}{0.267004,0.004874,0.329415}%
\pgfsetfillcolor{currentfill}%
\pgfsetlinewidth{0.000000pt}%
\definecolor{currentstroke}{rgb}{0.000000,0.000000,0.000000}%
\pgfsetstrokecolor{currentstroke}%
\pgfsetdash{}{0pt}%
\pgfpathmoveto{\pgfqpoint{1.712354in}{3.407661in}}%
\pgfpathlineto{\pgfqpoint{1.802890in}{3.226589in}}%
\pgfpathlineto{\pgfqpoint{1.808840in}{3.234523in}}%
\pgfpathlineto{\pgfqpoint{1.816774in}{3.188903in}}%
\pgfpathlineto{\pgfqpoint{1.785038in}{3.222622in}}%
\pgfpathlineto{\pgfqpoint{1.794956in}{3.222622in}}%
\pgfpathlineto{\pgfqpoint{1.704420in}{3.403694in}}%
\pgfpathlineto{\pgfqpoint{1.712354in}{3.407661in}}%
\pgfusepath{fill}%
\end{pgfscope}%
\begin{pgfscope}%
\pgfpathrectangle{\pgfqpoint{1.432000in}{0.528000in}}{\pgfqpoint{3.696000in}{3.696000in}} %
\pgfusepath{clip}%
\pgfsetbuttcap%
\pgfsetroundjoin%
\definecolor{currentfill}{rgb}{0.266580,0.228262,0.514349}%
\pgfsetfillcolor{currentfill}%
\pgfsetlinewidth{0.000000pt}%
\definecolor{currentstroke}{rgb}{0.000000,0.000000,0.000000}%
\pgfsetstrokecolor{currentstroke}%
\pgfsetdash{}{0pt}%
\pgfpathmoveto{\pgfqpoint{1.712822in}{3.405677in}}%
\pgfpathlineto{\pgfqpoint{1.712822in}{3.337207in}}%
\pgfpathlineto{\pgfqpoint{1.721693in}{3.341642in}}%
\pgfpathlineto{\pgfqpoint{1.708387in}{3.297290in}}%
\pgfpathlineto{\pgfqpoint{1.695081in}{3.341642in}}%
\pgfpathlineto{\pgfqpoint{1.703952in}{3.337207in}}%
\pgfpathlineto{\pgfqpoint{1.703952in}{3.405677in}}%
\pgfpathlineto{\pgfqpoint{1.712822in}{3.405677in}}%
\pgfusepath{fill}%
\end{pgfscope}%
\begin{pgfscope}%
\pgfpathrectangle{\pgfqpoint{1.432000in}{0.528000in}}{\pgfqpoint{3.696000in}{3.696000in}} %
\pgfusepath{clip}%
\pgfsetbuttcap%
\pgfsetroundjoin%
\definecolor{currentfill}{rgb}{0.277018,0.050344,0.375715}%
\pgfsetfillcolor{currentfill}%
\pgfsetlinewidth{0.000000pt}%
\definecolor{currentstroke}{rgb}{0.000000,0.000000,0.000000}%
\pgfsetstrokecolor{currentstroke}%
\pgfsetdash{}{0pt}%
\pgfpathmoveto{\pgfqpoint{1.711523in}{3.408814in}}%
\pgfpathlineto{\pgfqpoint{1.791685in}{3.328652in}}%
\pgfpathlineto{\pgfqpoint{1.794821in}{3.338060in}}%
\pgfpathlineto{\pgfqpoint{1.816774in}{3.297290in}}%
\pgfpathlineto{\pgfqpoint{1.776004in}{3.319243in}}%
\pgfpathlineto{\pgfqpoint{1.785413in}{3.322380in}}%
\pgfpathlineto{\pgfqpoint{1.705251in}{3.402541in}}%
\pgfpathlineto{\pgfqpoint{1.711523in}{3.408814in}}%
\pgfusepath{fill}%
\end{pgfscope}%
\begin{pgfscope}%
\pgfpathrectangle{\pgfqpoint{1.432000in}{0.528000in}}{\pgfqpoint{3.696000in}{3.696000in}} %
\pgfusepath{clip}%
\pgfsetbuttcap%
\pgfsetroundjoin%
\definecolor{currentfill}{rgb}{0.241237,0.296485,0.539709}%
\pgfsetfillcolor{currentfill}%
\pgfsetlinewidth{0.000000pt}%
\definecolor{currentstroke}{rgb}{0.000000,0.000000,0.000000}%
\pgfsetstrokecolor{currentstroke}%
\pgfsetdash{}{0pt}%
\pgfpathmoveto{\pgfqpoint{1.821209in}{3.405677in}}%
\pgfpathlineto{\pgfqpoint{1.821209in}{3.337207in}}%
\pgfpathlineto{\pgfqpoint{1.830080in}{3.341642in}}%
\pgfpathlineto{\pgfqpoint{1.816774in}{3.297290in}}%
\pgfpathlineto{\pgfqpoint{1.803469in}{3.341642in}}%
\pgfpathlineto{\pgfqpoint{1.812339in}{3.337207in}}%
\pgfpathlineto{\pgfqpoint{1.812339in}{3.405677in}}%
\pgfpathlineto{\pgfqpoint{1.821209in}{3.405677in}}%
\pgfusepath{fill}%
\end{pgfscope}%
\begin{pgfscope}%
\pgfpathrectangle{\pgfqpoint{1.432000in}{0.528000in}}{\pgfqpoint{3.696000in}{3.696000in}} %
\pgfusepath{clip}%
\pgfsetbuttcap%
\pgfsetroundjoin%
\definecolor{currentfill}{rgb}{0.282884,0.135920,0.453427}%
\pgfsetfillcolor{currentfill}%
\pgfsetlinewidth{0.000000pt}%
\definecolor{currentstroke}{rgb}{0.000000,0.000000,0.000000}%
\pgfsetstrokecolor{currentstroke}%
\pgfsetdash{}{0pt}%
\pgfpathmoveto{\pgfqpoint{1.929596in}{3.405677in}}%
\pgfpathlineto{\pgfqpoint{1.929596in}{3.228820in}}%
\pgfpathlineto{\pgfqpoint{1.938467in}{3.233255in}}%
\pgfpathlineto{\pgfqpoint{1.925161in}{3.188903in}}%
\pgfpathlineto{\pgfqpoint{1.911856in}{3.233255in}}%
\pgfpathlineto{\pgfqpoint{1.920726in}{3.228820in}}%
\pgfpathlineto{\pgfqpoint{1.920726in}{3.405677in}}%
\pgfpathlineto{\pgfqpoint{1.929596in}{3.405677in}}%
\pgfusepath{fill}%
\end{pgfscope}%
\begin{pgfscope}%
\pgfpathrectangle{\pgfqpoint{1.432000in}{0.528000in}}{\pgfqpoint{3.696000in}{3.696000in}} %
\pgfusepath{clip}%
\pgfsetbuttcap%
\pgfsetroundjoin%
\definecolor{currentfill}{rgb}{0.195860,0.395433,0.555276}%
\pgfsetfillcolor{currentfill}%
\pgfsetlinewidth{0.000000pt}%
\definecolor{currentstroke}{rgb}{0.000000,0.000000,0.000000}%
\pgfsetstrokecolor{currentstroke}%
\pgfsetdash{}{0pt}%
\pgfpathmoveto{\pgfqpoint{1.929596in}{3.405677in}}%
\pgfpathlineto{\pgfqpoint{1.929596in}{3.337207in}}%
\pgfpathlineto{\pgfqpoint{1.938467in}{3.341642in}}%
\pgfpathlineto{\pgfqpoint{1.925161in}{3.297290in}}%
\pgfpathlineto{\pgfqpoint{1.911856in}{3.341642in}}%
\pgfpathlineto{\pgfqpoint{1.920726in}{3.337207in}}%
\pgfpathlineto{\pgfqpoint{1.920726in}{3.405677in}}%
\pgfpathlineto{\pgfqpoint{1.929596in}{3.405677in}}%
\pgfusepath{fill}%
\end{pgfscope}%
\begin{pgfscope}%
\pgfpathrectangle{\pgfqpoint{1.432000in}{0.528000in}}{\pgfqpoint{3.696000in}{3.696000in}} %
\pgfusepath{clip}%
\pgfsetbuttcap%
\pgfsetroundjoin%
\definecolor{currentfill}{rgb}{0.150148,0.676631,0.506589}%
\pgfsetfillcolor{currentfill}%
\pgfsetlinewidth{0.000000pt}%
\definecolor{currentstroke}{rgb}{0.000000,0.000000,0.000000}%
\pgfsetstrokecolor{currentstroke}%
\pgfsetdash{}{0pt}%
\pgfpathmoveto{\pgfqpoint{2.037984in}{3.405677in}}%
\pgfpathlineto{\pgfqpoint{2.037984in}{3.337207in}}%
\pgfpathlineto{\pgfqpoint{2.046854in}{3.341642in}}%
\pgfpathlineto{\pgfqpoint{2.033548in}{3.297290in}}%
\pgfpathlineto{\pgfqpoint{2.020243in}{3.341642in}}%
\pgfpathlineto{\pgfqpoint{2.029113in}{3.337207in}}%
\pgfpathlineto{\pgfqpoint{2.029113in}{3.405677in}}%
\pgfpathlineto{\pgfqpoint{2.037984in}{3.405677in}}%
\pgfusepath{fill}%
\end{pgfscope}%
\begin{pgfscope}%
\pgfpathrectangle{\pgfqpoint{1.432000in}{0.528000in}}{\pgfqpoint{3.696000in}{3.696000in}} %
\pgfusepath{clip}%
\pgfsetbuttcap%
\pgfsetroundjoin%
\definecolor{currentfill}{rgb}{0.565498,0.842430,0.262877}%
\pgfsetfillcolor{currentfill}%
\pgfsetlinewidth{0.000000pt}%
\definecolor{currentstroke}{rgb}{0.000000,0.000000,0.000000}%
\pgfsetstrokecolor{currentstroke}%
\pgfsetdash{}{0pt}%
\pgfpathmoveto{\pgfqpoint{2.146371in}{3.405677in}}%
\pgfpathlineto{\pgfqpoint{2.146371in}{3.337207in}}%
\pgfpathlineto{\pgfqpoint{2.155241in}{3.341642in}}%
\pgfpathlineto{\pgfqpoint{2.141935in}{3.297290in}}%
\pgfpathlineto{\pgfqpoint{2.128630in}{3.341642in}}%
\pgfpathlineto{\pgfqpoint{2.137500in}{3.337207in}}%
\pgfpathlineto{\pgfqpoint{2.137500in}{3.405677in}}%
\pgfpathlineto{\pgfqpoint{2.146371in}{3.405677in}}%
\pgfusepath{fill}%
\end{pgfscope}%
\begin{pgfscope}%
\pgfpathrectangle{\pgfqpoint{1.432000in}{0.528000in}}{\pgfqpoint{3.696000in}{3.696000in}} %
\pgfusepath{clip}%
\pgfsetbuttcap%
\pgfsetroundjoin%
\definecolor{currentfill}{rgb}{0.131172,0.555899,0.552459}%
\pgfsetfillcolor{currentfill}%
\pgfsetlinewidth{0.000000pt}%
\definecolor{currentstroke}{rgb}{0.000000,0.000000,0.000000}%
\pgfsetstrokecolor{currentstroke}%
\pgfsetdash{}{0pt}%
\pgfpathmoveto{\pgfqpoint{2.254758in}{3.405677in}}%
\pgfpathlineto{\pgfqpoint{2.254758in}{3.337207in}}%
\pgfpathlineto{\pgfqpoint{2.263628in}{3.341642in}}%
\pgfpathlineto{\pgfqpoint{2.250323in}{3.297290in}}%
\pgfpathlineto{\pgfqpoint{2.237017in}{3.341642in}}%
\pgfpathlineto{\pgfqpoint{2.245887in}{3.337207in}}%
\pgfpathlineto{\pgfqpoint{2.245887in}{3.405677in}}%
\pgfpathlineto{\pgfqpoint{2.254758in}{3.405677in}}%
\pgfusepath{fill}%
\end{pgfscope}%
\begin{pgfscope}%
\pgfpathrectangle{\pgfqpoint{1.432000in}{0.528000in}}{\pgfqpoint{3.696000in}{3.696000in}} %
\pgfusepath{clip}%
\pgfsetbuttcap%
\pgfsetroundjoin%
\definecolor{currentfill}{rgb}{0.278012,0.180367,0.486697}%
\pgfsetfillcolor{currentfill}%
\pgfsetlinewidth{0.000000pt}%
\definecolor{currentstroke}{rgb}{0.000000,0.000000,0.000000}%
\pgfsetstrokecolor{currentstroke}%
\pgfsetdash{}{0pt}%
\pgfpathmoveto{\pgfqpoint{2.361846in}{3.402541in}}%
\pgfpathlineto{\pgfqpoint{2.281684in}{3.322380in}}%
\pgfpathlineto{\pgfqpoint{2.291093in}{3.319243in}}%
\pgfpathlineto{\pgfqpoint{2.250323in}{3.297290in}}%
\pgfpathlineto{\pgfqpoint{2.272276in}{3.338060in}}%
\pgfpathlineto{\pgfqpoint{2.275412in}{3.328652in}}%
\pgfpathlineto{\pgfqpoint{2.355574in}{3.408814in}}%
\pgfpathlineto{\pgfqpoint{2.361846in}{3.402541in}}%
\pgfusepath{fill}%
\end{pgfscope}%
\begin{pgfscope}%
\pgfpathrectangle{\pgfqpoint{1.432000in}{0.528000in}}{\pgfqpoint{3.696000in}{3.696000in}} %
\pgfusepath{clip}%
\pgfsetbuttcap%
\pgfsetroundjoin%
\definecolor{currentfill}{rgb}{0.282623,0.140926,0.457517}%
\pgfsetfillcolor{currentfill}%
\pgfsetlinewidth{0.000000pt}%
\definecolor{currentstroke}{rgb}{0.000000,0.000000,0.000000}%
\pgfsetstrokecolor{currentstroke}%
\pgfsetdash{}{0pt}%
\pgfpathmoveto{\pgfqpoint{2.363145in}{3.405677in}}%
\pgfpathlineto{\pgfqpoint{2.363145in}{3.337207in}}%
\pgfpathlineto{\pgfqpoint{2.372015in}{3.341642in}}%
\pgfpathlineto{\pgfqpoint{2.358710in}{3.297290in}}%
\pgfpathlineto{\pgfqpoint{2.345404in}{3.341642in}}%
\pgfpathlineto{\pgfqpoint{2.354274in}{3.337207in}}%
\pgfpathlineto{\pgfqpoint{2.354274in}{3.405677in}}%
\pgfpathlineto{\pgfqpoint{2.363145in}{3.405677in}}%
\pgfusepath{fill}%
\end{pgfscope}%
\begin{pgfscope}%
\pgfpathrectangle{\pgfqpoint{1.432000in}{0.528000in}}{\pgfqpoint{3.696000in}{3.696000in}} %
\pgfusepath{clip}%
\pgfsetbuttcap%
\pgfsetroundjoin%
\definecolor{currentfill}{rgb}{0.283091,0.110553,0.431554}%
\pgfsetfillcolor{currentfill}%
\pgfsetlinewidth{0.000000pt}%
\definecolor{currentstroke}{rgb}{0.000000,0.000000,0.000000}%
\pgfsetstrokecolor{currentstroke}%
\pgfsetdash{}{0pt}%
\pgfpathmoveto{\pgfqpoint{2.363145in}{3.405677in}}%
\pgfpathlineto{\pgfqpoint{2.360927in}{3.409518in}}%
\pgfpathlineto{\pgfqpoint{2.356492in}{3.409518in}}%
\pgfpathlineto{\pgfqpoint{2.354274in}{3.405677in}}%
\pgfpathlineto{\pgfqpoint{2.356492in}{3.401836in}}%
\pgfpathlineto{\pgfqpoint{2.360927in}{3.401836in}}%
\pgfpathlineto{\pgfqpoint{2.363145in}{3.405677in}}%
\pgfpathlineto{\pgfqpoint{2.360927in}{3.409518in}}%
\pgfusepath{fill}%
\end{pgfscope}%
\begin{pgfscope}%
\pgfpathrectangle{\pgfqpoint{1.432000in}{0.528000in}}{\pgfqpoint{3.696000in}{3.696000in}} %
\pgfusepath{clip}%
\pgfsetbuttcap%
\pgfsetroundjoin%
\definecolor{currentfill}{rgb}{0.175841,0.441290,0.557685}%
\pgfsetfillcolor{currentfill}%
\pgfsetlinewidth{0.000000pt}%
\definecolor{currentstroke}{rgb}{0.000000,0.000000,0.000000}%
\pgfsetstrokecolor{currentstroke}%
\pgfsetdash{}{0pt}%
\pgfpathmoveto{\pgfqpoint{2.471532in}{3.405677in}}%
\pgfpathlineto{\pgfqpoint{2.469314in}{3.409518in}}%
\pgfpathlineto{\pgfqpoint{2.464879in}{3.409518in}}%
\pgfpathlineto{\pgfqpoint{2.462662in}{3.405677in}}%
\pgfpathlineto{\pgfqpoint{2.464879in}{3.401836in}}%
\pgfpathlineto{\pgfqpoint{2.469314in}{3.401836in}}%
\pgfpathlineto{\pgfqpoint{2.471532in}{3.405677in}}%
\pgfpathlineto{\pgfqpoint{2.469314in}{3.409518in}}%
\pgfusepath{fill}%
\end{pgfscope}%
\begin{pgfscope}%
\pgfpathrectangle{\pgfqpoint{1.432000in}{0.528000in}}{\pgfqpoint{3.696000in}{3.696000in}} %
\pgfusepath{clip}%
\pgfsetbuttcap%
\pgfsetroundjoin%
\definecolor{currentfill}{rgb}{0.274149,0.751988,0.436601}%
\pgfsetfillcolor{currentfill}%
\pgfsetlinewidth{0.000000pt}%
\definecolor{currentstroke}{rgb}{0.000000,0.000000,0.000000}%
\pgfsetstrokecolor{currentstroke}%
\pgfsetdash{}{0pt}%
\pgfpathmoveto{\pgfqpoint{2.579919in}{3.405677in}}%
\pgfpathlineto{\pgfqpoint{2.577701in}{3.409518in}}%
\pgfpathlineto{\pgfqpoint{2.573266in}{3.409518in}}%
\pgfpathlineto{\pgfqpoint{2.571049in}{3.405677in}}%
\pgfpathlineto{\pgfqpoint{2.573266in}{3.401836in}}%
\pgfpathlineto{\pgfqpoint{2.577701in}{3.401836in}}%
\pgfpathlineto{\pgfqpoint{2.579919in}{3.405677in}}%
\pgfpathlineto{\pgfqpoint{2.577701in}{3.409518in}}%
\pgfusepath{fill}%
\end{pgfscope}%
\begin{pgfscope}%
\pgfpathrectangle{\pgfqpoint{1.432000in}{0.528000in}}{\pgfqpoint{3.696000in}{3.696000in}} %
\pgfusepath{clip}%
\pgfsetbuttcap%
\pgfsetroundjoin%
\definecolor{currentfill}{rgb}{0.525776,0.833491,0.288127}%
\pgfsetfillcolor{currentfill}%
\pgfsetlinewidth{0.000000pt}%
\definecolor{currentstroke}{rgb}{0.000000,0.000000,0.000000}%
\pgfsetstrokecolor{currentstroke}%
\pgfsetdash{}{0pt}%
\pgfpathmoveto{\pgfqpoint{2.688306in}{3.405677in}}%
\pgfpathlineto{\pgfqpoint{2.686089in}{3.409518in}}%
\pgfpathlineto{\pgfqpoint{2.681653in}{3.409518in}}%
\pgfpathlineto{\pgfqpoint{2.679436in}{3.405677in}}%
\pgfpathlineto{\pgfqpoint{2.681653in}{3.401836in}}%
\pgfpathlineto{\pgfqpoint{2.686089in}{3.401836in}}%
\pgfpathlineto{\pgfqpoint{2.688306in}{3.405677in}}%
\pgfpathlineto{\pgfqpoint{2.686089in}{3.409518in}}%
\pgfusepath{fill}%
\end{pgfscope}%
\begin{pgfscope}%
\pgfpathrectangle{\pgfqpoint{1.432000in}{0.528000in}}{\pgfqpoint{3.696000in}{3.696000in}} %
\pgfusepath{clip}%
\pgfsetbuttcap%
\pgfsetroundjoin%
\definecolor{currentfill}{rgb}{0.185783,0.704891,0.485273}%
\pgfsetfillcolor{currentfill}%
\pgfsetlinewidth{0.000000pt}%
\definecolor{currentstroke}{rgb}{0.000000,0.000000,0.000000}%
\pgfsetstrokecolor{currentstroke}%
\pgfsetdash{}{0pt}%
\pgfpathmoveto{\pgfqpoint{2.796693in}{3.405677in}}%
\pgfpathlineto{\pgfqpoint{2.794476in}{3.409518in}}%
\pgfpathlineto{\pgfqpoint{2.790040in}{3.409518in}}%
\pgfpathlineto{\pgfqpoint{2.787823in}{3.405677in}}%
\pgfpathlineto{\pgfqpoint{2.790040in}{3.401836in}}%
\pgfpathlineto{\pgfqpoint{2.794476in}{3.401836in}}%
\pgfpathlineto{\pgfqpoint{2.796693in}{3.405677in}}%
\pgfpathlineto{\pgfqpoint{2.794476in}{3.409518in}}%
\pgfusepath{fill}%
\end{pgfscope}%
\begin{pgfscope}%
\pgfpathrectangle{\pgfqpoint{1.432000in}{0.528000in}}{\pgfqpoint{3.696000in}{3.696000in}} %
\pgfusepath{clip}%
\pgfsetbuttcap%
\pgfsetroundjoin%
\definecolor{currentfill}{rgb}{0.163625,0.471133,0.558148}%
\pgfsetfillcolor{currentfill}%
\pgfsetlinewidth{0.000000pt}%
\definecolor{currentstroke}{rgb}{0.000000,0.000000,0.000000}%
\pgfsetstrokecolor{currentstroke}%
\pgfsetdash{}{0pt}%
\pgfpathmoveto{\pgfqpoint{2.905080in}{3.405677in}}%
\pgfpathlineto{\pgfqpoint{2.902863in}{3.409518in}}%
\pgfpathlineto{\pgfqpoint{2.898428in}{3.409518in}}%
\pgfpathlineto{\pgfqpoint{2.896210in}{3.405677in}}%
\pgfpathlineto{\pgfqpoint{2.898428in}{3.401836in}}%
\pgfpathlineto{\pgfqpoint{2.902863in}{3.401836in}}%
\pgfpathlineto{\pgfqpoint{2.905080in}{3.405677in}}%
\pgfpathlineto{\pgfqpoint{2.902863in}{3.409518in}}%
\pgfusepath{fill}%
\end{pgfscope}%
\begin{pgfscope}%
\pgfpathrectangle{\pgfqpoint{1.432000in}{0.528000in}}{\pgfqpoint{3.696000in}{3.696000in}} %
\pgfusepath{clip}%
\pgfsetbuttcap%
\pgfsetroundjoin%
\definecolor{currentfill}{rgb}{0.283197,0.115680,0.436115}%
\pgfsetfillcolor{currentfill}%
\pgfsetlinewidth{0.000000pt}%
\definecolor{currentstroke}{rgb}{0.000000,0.000000,0.000000}%
\pgfsetstrokecolor{currentstroke}%
\pgfsetdash{}{0pt}%
\pgfpathmoveto{\pgfqpoint{2.896210in}{3.405677in}}%
\pgfpathlineto{\pgfqpoint{2.896210in}{3.474148in}}%
\pgfpathlineto{\pgfqpoint{2.887340in}{3.469713in}}%
\pgfpathlineto{\pgfqpoint{2.900645in}{3.514065in}}%
\pgfpathlineto{\pgfqpoint{2.913951in}{3.469713in}}%
\pgfpathlineto{\pgfqpoint{2.905080in}{3.474148in}}%
\pgfpathlineto{\pgfqpoint{2.905080in}{3.405677in}}%
\pgfpathlineto{\pgfqpoint{2.896210in}{3.405677in}}%
\pgfusepath{fill}%
\end{pgfscope}%
\begin{pgfscope}%
\pgfpathrectangle{\pgfqpoint{1.432000in}{0.528000in}}{\pgfqpoint{3.696000in}{3.696000in}} %
\pgfusepath{clip}%
\pgfsetbuttcap%
\pgfsetroundjoin%
\definecolor{currentfill}{rgb}{0.252194,0.269783,0.531579}%
\pgfsetfillcolor{currentfill}%
\pgfsetlinewidth{0.000000pt}%
\definecolor{currentstroke}{rgb}{0.000000,0.000000,0.000000}%
\pgfsetstrokecolor{currentstroke}%
\pgfsetdash{}{0pt}%
\pgfpathmoveto{\pgfqpoint{3.013467in}{3.405677in}}%
\pgfpathlineto{\pgfqpoint{3.011250in}{3.409518in}}%
\pgfpathlineto{\pgfqpoint{3.006815in}{3.409518in}}%
\pgfpathlineto{\pgfqpoint{3.004597in}{3.405677in}}%
\pgfpathlineto{\pgfqpoint{3.006815in}{3.401836in}}%
\pgfpathlineto{\pgfqpoint{3.011250in}{3.401836in}}%
\pgfpathlineto{\pgfqpoint{3.013467in}{3.405677in}}%
\pgfpathlineto{\pgfqpoint{3.011250in}{3.409518in}}%
\pgfusepath{fill}%
\end{pgfscope}%
\begin{pgfscope}%
\pgfpathrectangle{\pgfqpoint{1.432000in}{0.528000in}}{\pgfqpoint{3.696000in}{3.696000in}} %
\pgfusepath{clip}%
\pgfsetbuttcap%
\pgfsetroundjoin%
\definecolor{currentfill}{rgb}{0.276022,0.044167,0.370164}%
\pgfsetfillcolor{currentfill}%
\pgfsetlinewidth{0.000000pt}%
\definecolor{currentstroke}{rgb}{0.000000,0.000000,0.000000}%
\pgfsetstrokecolor{currentstroke}%
\pgfsetdash{}{0pt}%
\pgfpathmoveto{\pgfqpoint{3.009032in}{3.410113in}}%
\pgfpathlineto{\pgfqpoint{3.077503in}{3.410113in}}%
\pgfpathlineto{\pgfqpoint{3.073067in}{3.418983in}}%
\pgfpathlineto{\pgfqpoint{3.117419in}{3.405677in}}%
\pgfpathlineto{\pgfqpoint{3.073067in}{3.392372in}}%
\pgfpathlineto{\pgfqpoint{3.077503in}{3.401242in}}%
\pgfpathlineto{\pgfqpoint{3.009032in}{3.401242in}}%
\pgfpathlineto{\pgfqpoint{3.009032in}{3.410113in}}%
\pgfusepath{fill}%
\end{pgfscope}%
\begin{pgfscope}%
\pgfpathrectangle{\pgfqpoint{1.432000in}{0.528000in}}{\pgfqpoint{3.696000in}{3.696000in}} %
\pgfusepath{clip}%
\pgfsetbuttcap%
\pgfsetroundjoin%
\definecolor{currentfill}{rgb}{0.283072,0.130895,0.449241}%
\pgfsetfillcolor{currentfill}%
\pgfsetlinewidth{0.000000pt}%
\definecolor{currentstroke}{rgb}{0.000000,0.000000,0.000000}%
\pgfsetstrokecolor{currentstroke}%
\pgfsetdash{}{0pt}%
\pgfpathmoveto{\pgfqpoint{3.004597in}{3.405677in}}%
\pgfpathlineto{\pgfqpoint{3.004597in}{3.474148in}}%
\pgfpathlineto{\pgfqpoint{2.995727in}{3.469713in}}%
\pgfpathlineto{\pgfqpoint{3.009032in}{3.514065in}}%
\pgfpathlineto{\pgfqpoint{3.022338in}{3.469713in}}%
\pgfpathlineto{\pgfqpoint{3.013467in}{3.474148in}}%
\pgfpathlineto{\pgfqpoint{3.013467in}{3.405677in}}%
\pgfpathlineto{\pgfqpoint{3.004597in}{3.405677in}}%
\pgfusepath{fill}%
\end{pgfscope}%
\begin{pgfscope}%
\pgfpathrectangle{\pgfqpoint{1.432000in}{0.528000in}}{\pgfqpoint{3.696000in}{3.696000in}} %
\pgfusepath{clip}%
\pgfsetbuttcap%
\pgfsetroundjoin%
\definecolor{currentfill}{rgb}{0.268510,0.009605,0.335427}%
\pgfsetfillcolor{currentfill}%
\pgfsetlinewidth{0.000000pt}%
\definecolor{currentstroke}{rgb}{0.000000,0.000000,0.000000}%
\pgfsetstrokecolor{currentstroke}%
\pgfsetdash{}{0pt}%
\pgfpathmoveto{\pgfqpoint{3.005896in}{3.408814in}}%
\pgfpathlineto{\pgfqpoint{3.086058in}{3.488975in}}%
\pgfpathlineto{\pgfqpoint{3.076649in}{3.492111in}}%
\pgfpathlineto{\pgfqpoint{3.117419in}{3.514065in}}%
\pgfpathlineto{\pgfqpoint{3.095466in}{3.473294in}}%
\pgfpathlineto{\pgfqpoint{3.092330in}{3.482703in}}%
\pgfpathlineto{\pgfqpoint{3.012168in}{3.402541in}}%
\pgfpathlineto{\pgfqpoint{3.005896in}{3.408814in}}%
\pgfusepath{fill}%
\end{pgfscope}%
\begin{pgfscope}%
\pgfpathrectangle{\pgfqpoint{1.432000in}{0.528000in}}{\pgfqpoint{3.696000in}{3.696000in}} %
\pgfusepath{clip}%
\pgfsetbuttcap%
\pgfsetroundjoin%
\definecolor{currentfill}{rgb}{0.282910,0.105393,0.426902}%
\pgfsetfillcolor{currentfill}%
\pgfsetlinewidth{0.000000pt}%
\definecolor{currentstroke}{rgb}{0.000000,0.000000,0.000000}%
\pgfsetstrokecolor{currentstroke}%
\pgfsetdash{}{0pt}%
\pgfpathmoveto{\pgfqpoint{3.121855in}{3.405677in}}%
\pgfpathlineto{\pgfqpoint{3.119637in}{3.409518in}}%
\pgfpathlineto{\pgfqpoint{3.115202in}{3.409518in}}%
\pgfpathlineto{\pgfqpoint{3.112984in}{3.405677in}}%
\pgfpathlineto{\pgfqpoint{3.115202in}{3.401836in}}%
\pgfpathlineto{\pgfqpoint{3.119637in}{3.401836in}}%
\pgfpathlineto{\pgfqpoint{3.121855in}{3.405677in}}%
\pgfpathlineto{\pgfqpoint{3.119637in}{3.409518in}}%
\pgfusepath{fill}%
\end{pgfscope}%
\begin{pgfscope}%
\pgfpathrectangle{\pgfqpoint{1.432000in}{0.528000in}}{\pgfqpoint{3.696000in}{3.696000in}} %
\pgfusepath{clip}%
\pgfsetbuttcap%
\pgfsetroundjoin%
\definecolor{currentfill}{rgb}{0.280267,0.073417,0.397163}%
\pgfsetfillcolor{currentfill}%
\pgfsetlinewidth{0.000000pt}%
\definecolor{currentstroke}{rgb}{0.000000,0.000000,0.000000}%
\pgfsetstrokecolor{currentstroke}%
\pgfsetdash{}{0pt}%
\pgfpathmoveto{\pgfqpoint{3.112984in}{3.405677in}}%
\pgfpathlineto{\pgfqpoint{3.112984in}{3.474148in}}%
\pgfpathlineto{\pgfqpoint{3.104114in}{3.469713in}}%
\pgfpathlineto{\pgfqpoint{3.117419in}{3.514065in}}%
\pgfpathlineto{\pgfqpoint{3.130725in}{3.469713in}}%
\pgfpathlineto{\pgfqpoint{3.121855in}{3.474148in}}%
\pgfpathlineto{\pgfqpoint{3.121855in}{3.405677in}}%
\pgfpathlineto{\pgfqpoint{3.112984in}{3.405677in}}%
\pgfusepath{fill}%
\end{pgfscope}%
\begin{pgfscope}%
\pgfpathrectangle{\pgfqpoint{1.432000in}{0.528000in}}{\pgfqpoint{3.696000in}{3.696000in}} %
\pgfusepath{clip}%
\pgfsetbuttcap%
\pgfsetroundjoin%
\definecolor{currentfill}{rgb}{0.280894,0.078907,0.402329}%
\pgfsetfillcolor{currentfill}%
\pgfsetlinewidth{0.000000pt}%
\definecolor{currentstroke}{rgb}{0.000000,0.000000,0.000000}%
\pgfsetstrokecolor{currentstroke}%
\pgfsetdash{}{0pt}%
\pgfpathmoveto{\pgfqpoint{3.114283in}{3.408814in}}%
\pgfpathlineto{\pgfqpoint{3.194445in}{3.488975in}}%
\pgfpathlineto{\pgfqpoint{3.185036in}{3.492111in}}%
\pgfpathlineto{\pgfqpoint{3.225806in}{3.514065in}}%
\pgfpathlineto{\pgfqpoint{3.203853in}{3.473294in}}%
\pgfpathlineto{\pgfqpoint{3.200717in}{3.482703in}}%
\pgfpathlineto{\pgfqpoint{3.120556in}{3.402541in}}%
\pgfpathlineto{\pgfqpoint{3.114283in}{3.408814in}}%
\pgfusepath{fill}%
\end{pgfscope}%
\begin{pgfscope}%
\pgfpathrectangle{\pgfqpoint{1.432000in}{0.528000in}}{\pgfqpoint{3.696000in}{3.696000in}} %
\pgfusepath{clip}%
\pgfsetbuttcap%
\pgfsetroundjoin%
\definecolor{currentfill}{rgb}{0.282884,0.135920,0.453427}%
\pgfsetfillcolor{currentfill}%
\pgfsetlinewidth{0.000000pt}%
\definecolor{currentstroke}{rgb}{0.000000,0.000000,0.000000}%
\pgfsetstrokecolor{currentstroke}%
\pgfsetdash{}{0pt}%
\pgfpathmoveto{\pgfqpoint{3.221371in}{3.405677in}}%
\pgfpathlineto{\pgfqpoint{3.221371in}{3.474148in}}%
\pgfpathlineto{\pgfqpoint{3.212501in}{3.469713in}}%
\pgfpathlineto{\pgfqpoint{3.225806in}{3.514065in}}%
\pgfpathlineto{\pgfqpoint{3.239112in}{3.469713in}}%
\pgfpathlineto{\pgfqpoint{3.230242in}{3.474148in}}%
\pgfpathlineto{\pgfqpoint{3.230242in}{3.405677in}}%
\pgfpathlineto{\pgfqpoint{3.221371in}{3.405677in}}%
\pgfusepath{fill}%
\end{pgfscope}%
\begin{pgfscope}%
\pgfpathrectangle{\pgfqpoint{1.432000in}{0.528000in}}{\pgfqpoint{3.696000in}{3.696000in}} %
\pgfusepath{clip}%
\pgfsetbuttcap%
\pgfsetroundjoin%
\definecolor{currentfill}{rgb}{0.283197,0.115680,0.436115}%
\pgfsetfillcolor{currentfill}%
\pgfsetlinewidth{0.000000pt}%
\definecolor{currentstroke}{rgb}{0.000000,0.000000,0.000000}%
\pgfsetstrokecolor{currentstroke}%
\pgfsetdash{}{0pt}%
\pgfpathmoveto{\pgfqpoint{3.329758in}{3.405677in}}%
\pgfpathlineto{\pgfqpoint{3.329758in}{3.474148in}}%
\pgfpathlineto{\pgfqpoint{3.320888in}{3.469713in}}%
\pgfpathlineto{\pgfqpoint{3.334194in}{3.514065in}}%
\pgfpathlineto{\pgfqpoint{3.347499in}{3.469713in}}%
\pgfpathlineto{\pgfqpoint{3.338629in}{3.474148in}}%
\pgfpathlineto{\pgfqpoint{3.338629in}{3.405677in}}%
\pgfpathlineto{\pgfqpoint{3.329758in}{3.405677in}}%
\pgfusepath{fill}%
\end{pgfscope}%
\begin{pgfscope}%
\pgfpathrectangle{\pgfqpoint{1.432000in}{0.528000in}}{\pgfqpoint{3.696000in}{3.696000in}} %
\pgfusepath{clip}%
\pgfsetbuttcap%
\pgfsetroundjoin%
\definecolor{currentfill}{rgb}{0.282327,0.094955,0.417331}%
\pgfsetfillcolor{currentfill}%
\pgfsetlinewidth{0.000000pt}%
\definecolor{currentstroke}{rgb}{0.000000,0.000000,0.000000}%
\pgfsetstrokecolor{currentstroke}%
\pgfsetdash{}{0pt}%
\pgfpathmoveto{\pgfqpoint{3.656219in}{3.402541in}}%
\pgfpathlineto{\pgfqpoint{3.467670in}{3.591090in}}%
\pgfpathlineto{\pgfqpoint{3.464534in}{3.581682in}}%
\pgfpathlineto{\pgfqpoint{3.442581in}{3.622452in}}%
\pgfpathlineto{\pgfqpoint{3.483351in}{3.600498in}}%
\pgfpathlineto{\pgfqpoint{3.473942in}{3.597362in}}%
\pgfpathlineto{\pgfqpoint{3.662491in}{3.408814in}}%
\pgfpathlineto{\pgfqpoint{3.656219in}{3.402541in}}%
\pgfusepath{fill}%
\end{pgfscope}%
\begin{pgfscope}%
\pgfpathrectangle{\pgfqpoint{1.432000in}{0.528000in}}{\pgfqpoint{3.696000in}{3.696000in}} %
\pgfusepath{clip}%
\pgfsetbuttcap%
\pgfsetroundjoin%
\definecolor{currentfill}{rgb}{0.248629,0.278775,0.534556}%
\pgfsetfillcolor{currentfill}%
\pgfsetlinewidth{0.000000pt}%
\definecolor{currentstroke}{rgb}{0.000000,0.000000,0.000000}%
\pgfsetstrokecolor{currentstroke}%
\pgfsetdash{}{0pt}%
\pgfpathmoveto{\pgfqpoint{3.764606in}{3.402541in}}%
\pgfpathlineto{\pgfqpoint{3.576057in}{3.591090in}}%
\pgfpathlineto{\pgfqpoint{3.572921in}{3.581682in}}%
\pgfpathlineto{\pgfqpoint{3.550968in}{3.622452in}}%
\pgfpathlineto{\pgfqpoint{3.591738in}{3.600498in}}%
\pgfpathlineto{\pgfqpoint{3.582329in}{3.597362in}}%
\pgfpathlineto{\pgfqpoint{3.770878in}{3.408814in}}%
\pgfpathlineto{\pgfqpoint{3.764606in}{3.402541in}}%
\pgfusepath{fill}%
\end{pgfscope}%
\begin{pgfscope}%
\pgfpathrectangle{\pgfqpoint{1.432000in}{0.528000in}}{\pgfqpoint{3.696000in}{3.696000in}} %
\pgfusepath{clip}%
\pgfsetbuttcap%
\pgfsetroundjoin%
\definecolor{currentfill}{rgb}{0.166617,0.463708,0.558119}%
\pgfsetfillcolor{currentfill}%
\pgfsetlinewidth{0.000000pt}%
\definecolor{currentstroke}{rgb}{0.000000,0.000000,0.000000}%
\pgfsetstrokecolor{currentstroke}%
\pgfsetdash{}{0pt}%
\pgfpathmoveto{\pgfqpoint{3.873669in}{3.401987in}}%
\pgfpathlineto{\pgfqpoint{3.581720in}{3.596619in}}%
\pgfpathlineto{\pgfqpoint{3.580490in}{3.586779in}}%
\pgfpathlineto{\pgfqpoint{3.550968in}{3.622452in}}%
\pgfpathlineto{\pgfqpoint{3.595251in}{3.608920in}}%
\pgfpathlineto{\pgfqpoint{3.586641in}{3.604000in}}%
\pgfpathlineto{\pgfqpoint{3.878589in}{3.409368in}}%
\pgfpathlineto{\pgfqpoint{3.873669in}{3.401987in}}%
\pgfusepath{fill}%
\end{pgfscope}%
\begin{pgfscope}%
\pgfpathrectangle{\pgfqpoint{1.432000in}{0.528000in}}{\pgfqpoint{3.696000in}{3.696000in}} %
\pgfusepath{clip}%
\pgfsetbuttcap%
\pgfsetroundjoin%
\definecolor{currentfill}{rgb}{0.119699,0.618490,0.536347}%
\pgfsetfillcolor{currentfill}%
\pgfsetlinewidth{0.000000pt}%
\definecolor{currentstroke}{rgb}{0.000000,0.000000,0.000000}%
\pgfsetstrokecolor{currentstroke}%
\pgfsetdash{}{0pt}%
\pgfpathmoveto{\pgfqpoint{3.982056in}{3.401987in}}%
\pgfpathlineto{\pgfqpoint{3.690107in}{3.596619in}}%
\pgfpathlineto{\pgfqpoint{3.688877in}{3.586779in}}%
\pgfpathlineto{\pgfqpoint{3.659355in}{3.622452in}}%
\pgfpathlineto{\pgfqpoint{3.703639in}{3.608920in}}%
\pgfpathlineto{\pgfqpoint{3.695028in}{3.604000in}}%
\pgfpathlineto{\pgfqpoint{3.986976in}{3.409368in}}%
\pgfpathlineto{\pgfqpoint{3.982056in}{3.401987in}}%
\pgfusepath{fill}%
\end{pgfscope}%
\begin{pgfscope}%
\pgfpathrectangle{\pgfqpoint{1.432000in}{0.528000in}}{\pgfqpoint{3.696000in}{3.696000in}} %
\pgfusepath{clip}%
\pgfsetbuttcap%
\pgfsetroundjoin%
\definecolor{currentfill}{rgb}{0.192357,0.403199,0.555836}%
\pgfsetfillcolor{currentfill}%
\pgfsetlinewidth{0.000000pt}%
\definecolor{currentstroke}{rgb}{0.000000,0.000000,0.000000}%
\pgfsetstrokecolor{currentstroke}%
\pgfsetdash{}{0pt}%
\pgfpathmoveto{\pgfqpoint{4.090443in}{3.401987in}}%
\pgfpathlineto{\pgfqpoint{3.798495in}{3.596619in}}%
\pgfpathlineto{\pgfqpoint{3.797264in}{3.586779in}}%
\pgfpathlineto{\pgfqpoint{3.767742in}{3.622452in}}%
\pgfpathlineto{\pgfqpoint{3.812026in}{3.608920in}}%
\pgfpathlineto{\pgfqpoint{3.803415in}{3.604000in}}%
\pgfpathlineto{\pgfqpoint{4.095363in}{3.409368in}}%
\pgfpathlineto{\pgfqpoint{4.090443in}{3.401987in}}%
\pgfusepath{fill}%
\end{pgfscope}%
\begin{pgfscope}%
\pgfpathrectangle{\pgfqpoint{1.432000in}{0.528000in}}{\pgfqpoint{3.696000in}{3.696000in}} %
\pgfusepath{clip}%
\pgfsetbuttcap%
\pgfsetroundjoin%
\definecolor{currentfill}{rgb}{0.250425,0.274290,0.533103}%
\pgfsetfillcolor{currentfill}%
\pgfsetlinewidth{0.000000pt}%
\definecolor{currentstroke}{rgb}{0.000000,0.000000,0.000000}%
\pgfsetstrokecolor{currentstroke}%
\pgfsetdash{}{0pt}%
\pgfpathmoveto{\pgfqpoint{4.089767in}{3.402541in}}%
\pgfpathlineto{\pgfqpoint{3.792831in}{3.699477in}}%
\pgfpathlineto{\pgfqpoint{3.789695in}{3.690069in}}%
\pgfpathlineto{\pgfqpoint{3.767742in}{3.730839in}}%
\pgfpathlineto{\pgfqpoint{3.808512in}{3.708886in}}%
\pgfpathlineto{\pgfqpoint{3.799104in}{3.705749in}}%
\pgfpathlineto{\pgfqpoint{4.096039in}{3.408814in}}%
\pgfpathlineto{\pgfqpoint{4.089767in}{3.402541in}}%
\pgfusepath{fill}%
\end{pgfscope}%
\begin{pgfscope}%
\pgfpathrectangle{\pgfqpoint{1.432000in}{0.528000in}}{\pgfqpoint{3.696000in}{3.696000in}} %
\pgfusepath{clip}%
\pgfsetbuttcap%
\pgfsetroundjoin%
\definecolor{currentfill}{rgb}{0.269944,0.014625,0.341379}%
\pgfsetfillcolor{currentfill}%
\pgfsetlinewidth{0.000000pt}%
\definecolor{currentstroke}{rgb}{0.000000,0.000000,0.000000}%
\pgfsetstrokecolor{currentstroke}%
\pgfsetdash{}{0pt}%
\pgfpathmoveto{\pgfqpoint{4.198830in}{3.401987in}}%
\pgfpathlineto{\pgfqpoint{3.906882in}{3.596619in}}%
\pgfpathlineto{\pgfqpoint{3.905652in}{3.586779in}}%
\pgfpathlineto{\pgfqpoint{3.876129in}{3.622452in}}%
\pgfpathlineto{\pgfqpoint{3.920413in}{3.608920in}}%
\pgfpathlineto{\pgfqpoint{3.911802in}{3.604000in}}%
\pgfpathlineto{\pgfqpoint{4.203751in}{3.409368in}}%
\pgfpathlineto{\pgfqpoint{4.198830in}{3.401987in}}%
\pgfusepath{fill}%
\end{pgfscope}%
\begin{pgfscope}%
\pgfpathrectangle{\pgfqpoint{1.432000in}{0.528000in}}{\pgfqpoint{3.696000in}{3.696000in}} %
\pgfusepath{clip}%
\pgfsetbuttcap%
\pgfsetroundjoin%
\definecolor{currentfill}{rgb}{0.122606,0.585371,0.546557}%
\pgfsetfillcolor{currentfill}%
\pgfsetlinewidth{0.000000pt}%
\definecolor{currentstroke}{rgb}{0.000000,0.000000,0.000000}%
\pgfsetstrokecolor{currentstroke}%
\pgfsetdash{}{0pt}%
\pgfpathmoveto{\pgfqpoint{4.198154in}{3.402541in}}%
\pgfpathlineto{\pgfqpoint{3.901218in}{3.699477in}}%
\pgfpathlineto{\pgfqpoint{3.898082in}{3.690069in}}%
\pgfpathlineto{\pgfqpoint{3.876129in}{3.730839in}}%
\pgfpathlineto{\pgfqpoint{3.916899in}{3.708886in}}%
\pgfpathlineto{\pgfqpoint{3.907491in}{3.705749in}}%
\pgfpathlineto{\pgfqpoint{4.204426in}{3.408814in}}%
\pgfpathlineto{\pgfqpoint{4.198154in}{3.402541in}}%
\pgfusepath{fill}%
\end{pgfscope}%
\begin{pgfscope}%
\pgfpathrectangle{\pgfqpoint{1.432000in}{0.528000in}}{\pgfqpoint{3.696000in}{3.696000in}} %
\pgfusepath{clip}%
\pgfsetbuttcap%
\pgfsetroundjoin%
\definecolor{currentfill}{rgb}{0.120638,0.625828,0.533488}%
\pgfsetfillcolor{currentfill}%
\pgfsetlinewidth{0.000000pt}%
\definecolor{currentstroke}{rgb}{0.000000,0.000000,0.000000}%
\pgfsetstrokecolor{currentstroke}%
\pgfsetdash{}{0pt}%
\pgfpathmoveto{\pgfqpoint{4.306541in}{3.402541in}}%
\pgfpathlineto{\pgfqpoint{4.009605in}{3.699477in}}%
\pgfpathlineto{\pgfqpoint{4.006469in}{3.690069in}}%
\pgfpathlineto{\pgfqpoint{3.984516in}{3.730839in}}%
\pgfpathlineto{\pgfqpoint{4.025286in}{3.708886in}}%
\pgfpathlineto{\pgfqpoint{4.015878in}{3.705749in}}%
\pgfpathlineto{\pgfqpoint{4.312814in}{3.408814in}}%
\pgfpathlineto{\pgfqpoint{4.306541in}{3.402541in}}%
\pgfusepath{fill}%
\end{pgfscope}%
\begin{pgfscope}%
\pgfpathrectangle{\pgfqpoint{1.432000in}{0.528000in}}{\pgfqpoint{3.696000in}{3.696000in}} %
\pgfusepath{clip}%
\pgfsetbuttcap%
\pgfsetroundjoin%
\definecolor{currentfill}{rgb}{0.273809,0.031497,0.358853}%
\pgfsetfillcolor{currentfill}%
\pgfsetlinewidth{0.000000pt}%
\definecolor{currentstroke}{rgb}{0.000000,0.000000,0.000000}%
\pgfsetstrokecolor{currentstroke}%
\pgfsetdash{}{0pt}%
\pgfpathmoveto{\pgfqpoint{4.305987in}{3.403217in}}%
\pgfpathlineto{\pgfqpoint{4.111355in}{3.695166in}}%
\pgfpathlineto{\pgfqpoint{4.106434in}{3.686555in}}%
\pgfpathlineto{\pgfqpoint{4.092903in}{3.730839in}}%
\pgfpathlineto{\pgfqpoint{4.128576in}{3.701316in}}%
\pgfpathlineto{\pgfqpoint{4.118735in}{3.700086in}}%
\pgfpathlineto{\pgfqpoint{4.313368in}{3.408138in}}%
\pgfpathlineto{\pgfqpoint{4.305987in}{3.403217in}}%
\pgfusepath{fill}%
\end{pgfscope}%
\begin{pgfscope}%
\pgfpathrectangle{\pgfqpoint{1.432000in}{0.528000in}}{\pgfqpoint{3.696000in}{3.696000in}} %
\pgfusepath{clip}%
\pgfsetbuttcap%
\pgfsetroundjoin%
\definecolor{currentfill}{rgb}{0.208623,0.367752,0.552675}%
\pgfsetfillcolor{currentfill}%
\pgfsetlinewidth{0.000000pt}%
\definecolor{currentstroke}{rgb}{0.000000,0.000000,0.000000}%
\pgfsetstrokecolor{currentstroke}%
\pgfsetdash{}{0pt}%
\pgfpathmoveto{\pgfqpoint{4.414928in}{3.402541in}}%
\pgfpathlineto{\pgfqpoint{4.117993in}{3.699477in}}%
\pgfpathlineto{\pgfqpoint{4.114856in}{3.690069in}}%
\pgfpathlineto{\pgfqpoint{4.092903in}{3.730839in}}%
\pgfpathlineto{\pgfqpoint{4.133673in}{3.708886in}}%
\pgfpathlineto{\pgfqpoint{4.124265in}{3.705749in}}%
\pgfpathlineto{\pgfqpoint{4.421201in}{3.408814in}}%
\pgfpathlineto{\pgfqpoint{4.414928in}{3.402541in}}%
\pgfusepath{fill}%
\end{pgfscope}%
\begin{pgfscope}%
\pgfpathrectangle{\pgfqpoint{1.432000in}{0.528000in}}{\pgfqpoint{3.696000in}{3.696000in}} %
\pgfusepath{clip}%
\pgfsetbuttcap%
\pgfsetroundjoin%
\definecolor{currentfill}{rgb}{0.166617,0.463708,0.558119}%
\pgfsetfillcolor{currentfill}%
\pgfsetlinewidth{0.000000pt}%
\definecolor{currentstroke}{rgb}{0.000000,0.000000,0.000000}%
\pgfsetstrokecolor{currentstroke}%
\pgfsetdash{}{0pt}%
\pgfpathmoveto{\pgfqpoint{4.414374in}{3.403217in}}%
\pgfpathlineto{\pgfqpoint{4.219742in}{3.695166in}}%
\pgfpathlineto{\pgfqpoint{4.214821in}{3.686555in}}%
\pgfpathlineto{\pgfqpoint{4.201290in}{3.730839in}}%
\pgfpathlineto{\pgfqpoint{4.236963in}{3.701316in}}%
\pgfpathlineto{\pgfqpoint{4.227122in}{3.700086in}}%
\pgfpathlineto{\pgfqpoint{4.421755in}{3.408138in}}%
\pgfpathlineto{\pgfqpoint{4.414374in}{3.403217in}}%
\pgfusepath{fill}%
\end{pgfscope}%
\begin{pgfscope}%
\pgfpathrectangle{\pgfqpoint{1.432000in}{0.528000in}}{\pgfqpoint{3.696000in}{3.696000in}} %
\pgfusepath{clip}%
\pgfsetbuttcap%
\pgfsetroundjoin%
\definecolor{currentfill}{rgb}{0.149039,0.508051,0.557250}%
\pgfsetfillcolor{currentfill}%
\pgfsetlinewidth{0.000000pt}%
\definecolor{currentstroke}{rgb}{0.000000,0.000000,0.000000}%
\pgfsetstrokecolor{currentstroke}%
\pgfsetdash{}{0pt}%
\pgfpathmoveto{\pgfqpoint{4.522761in}{3.403217in}}%
\pgfpathlineto{\pgfqpoint{4.328129in}{3.695166in}}%
\pgfpathlineto{\pgfqpoint{4.323209in}{3.686555in}}%
\pgfpathlineto{\pgfqpoint{4.309677in}{3.730839in}}%
\pgfpathlineto{\pgfqpoint{4.345350in}{3.701316in}}%
\pgfpathlineto{\pgfqpoint{4.335510in}{3.700086in}}%
\pgfpathlineto{\pgfqpoint{4.530142in}{3.408138in}}%
\pgfpathlineto{\pgfqpoint{4.522761in}{3.403217in}}%
\pgfusepath{fill}%
\end{pgfscope}%
\begin{pgfscope}%
\pgfpathrectangle{\pgfqpoint{1.432000in}{0.528000in}}{\pgfqpoint{3.696000in}{3.696000in}} %
\pgfusepath{clip}%
\pgfsetbuttcap%
\pgfsetroundjoin%
\definecolor{currentfill}{rgb}{0.267968,0.223549,0.512008}%
\pgfsetfillcolor{currentfill}%
\pgfsetlinewidth{0.000000pt}%
\definecolor{currentstroke}{rgb}{0.000000,0.000000,0.000000}%
\pgfsetstrokecolor{currentstroke}%
\pgfsetdash{}{0pt}%
\pgfpathmoveto{\pgfqpoint{4.522244in}{3.404275in}}%
\pgfpathlineto{\pgfqpoint{4.426480in}{3.691568in}}%
\pgfpathlineto{\pgfqpoint{4.419467in}{3.684555in}}%
\pgfpathlineto{\pgfqpoint{4.418065in}{3.730839in}}%
\pgfpathlineto{\pgfqpoint{4.444713in}{3.692970in}}%
\pgfpathlineto{\pgfqpoint{4.434895in}{3.694373in}}%
\pgfpathlineto{\pgfqpoint{4.530659in}{3.407080in}}%
\pgfpathlineto{\pgfqpoint{4.522244in}{3.404275in}}%
\pgfusepath{fill}%
\end{pgfscope}%
\begin{pgfscope}%
\pgfpathrectangle{\pgfqpoint{1.432000in}{0.528000in}}{\pgfqpoint{3.696000in}{3.696000in}} %
\pgfusepath{clip}%
\pgfsetbuttcap%
\pgfsetroundjoin%
\definecolor{currentfill}{rgb}{0.276194,0.190074,0.493001}%
\pgfsetfillcolor{currentfill}%
\pgfsetlinewidth{0.000000pt}%
\definecolor{currentstroke}{rgb}{0.000000,0.000000,0.000000}%
\pgfsetstrokecolor{currentstroke}%
\pgfsetdash{}{0pt}%
\pgfpathmoveto{\pgfqpoint{4.631148in}{3.403217in}}%
\pgfpathlineto{\pgfqpoint{4.436516in}{3.695166in}}%
\pgfpathlineto{\pgfqpoint{4.431596in}{3.686555in}}%
\pgfpathlineto{\pgfqpoint{4.418065in}{3.730839in}}%
\pgfpathlineto{\pgfqpoint{4.453738in}{3.701316in}}%
\pgfpathlineto{\pgfqpoint{4.443897in}{3.700086in}}%
\pgfpathlineto{\pgfqpoint{4.638529in}{3.408138in}}%
\pgfpathlineto{\pgfqpoint{4.631148in}{3.403217in}}%
\pgfusepath{fill}%
\end{pgfscope}%
\begin{pgfscope}%
\pgfpathrectangle{\pgfqpoint{1.432000in}{0.528000in}}{\pgfqpoint{3.696000in}{3.696000in}} %
\pgfusepath{clip}%
\pgfsetbuttcap%
\pgfsetroundjoin%
\definecolor{currentfill}{rgb}{0.149039,0.508051,0.557250}%
\pgfsetfillcolor{currentfill}%
\pgfsetlinewidth{0.000000pt}%
\definecolor{currentstroke}{rgb}{0.000000,0.000000,0.000000}%
\pgfsetstrokecolor{currentstroke}%
\pgfsetdash{}{0pt}%
\pgfpathmoveto{\pgfqpoint{4.630631in}{3.404275in}}%
\pgfpathlineto{\pgfqpoint{4.534867in}{3.691568in}}%
\pgfpathlineto{\pgfqpoint{4.527854in}{3.684555in}}%
\pgfpathlineto{\pgfqpoint{4.526452in}{3.730839in}}%
\pgfpathlineto{\pgfqpoint{4.553100in}{3.692970in}}%
\pgfpathlineto{\pgfqpoint{4.543282in}{3.694373in}}%
\pgfpathlineto{\pgfqpoint{4.639046in}{3.407080in}}%
\pgfpathlineto{\pgfqpoint{4.630631in}{3.404275in}}%
\pgfusepath{fill}%
\end{pgfscope}%
\begin{pgfscope}%
\pgfpathrectangle{\pgfqpoint{1.432000in}{0.528000in}}{\pgfqpoint{3.696000in}{3.696000in}} %
\pgfusepath{clip}%
\pgfsetbuttcap%
\pgfsetroundjoin%
\definecolor{currentfill}{rgb}{0.282327,0.094955,0.417331}%
\pgfsetfillcolor{currentfill}%
\pgfsetlinewidth{0.000000pt}%
\definecolor{currentstroke}{rgb}{0.000000,0.000000,0.000000}%
\pgfsetstrokecolor{currentstroke}%
\pgfsetdash{}{0pt}%
\pgfpathmoveto{\pgfqpoint{4.739259in}{3.403694in}}%
\pgfpathlineto{\pgfqpoint{4.648723in}{3.584765in}}%
\pgfpathlineto{\pgfqpoint{4.642773in}{3.576832in}}%
\pgfpathlineto{\pgfqpoint{4.634839in}{3.622452in}}%
\pgfpathlineto{\pgfqpoint{4.666574in}{3.588732in}}%
\pgfpathlineto{\pgfqpoint{4.656657in}{3.588732in}}%
\pgfpathlineto{\pgfqpoint{4.747193in}{3.407661in}}%
\pgfpathlineto{\pgfqpoint{4.739259in}{3.403694in}}%
\pgfusepath{fill}%
\end{pgfscope}%
\begin{pgfscope}%
\pgfpathrectangle{\pgfqpoint{1.432000in}{0.528000in}}{\pgfqpoint{3.696000in}{3.696000in}} %
\pgfusepath{clip}%
\pgfsetbuttcap%
\pgfsetroundjoin%
\definecolor{currentfill}{rgb}{0.203063,0.379716,0.553925}%
\pgfsetfillcolor{currentfill}%
\pgfsetlinewidth{0.000000pt}%
\definecolor{currentstroke}{rgb}{0.000000,0.000000,0.000000}%
\pgfsetstrokecolor{currentstroke}%
\pgfsetdash{}{0pt}%
\pgfpathmoveto{\pgfqpoint{4.739018in}{3.404275in}}%
\pgfpathlineto{\pgfqpoint{4.643254in}{3.691568in}}%
\pgfpathlineto{\pgfqpoint{4.636241in}{3.684555in}}%
\pgfpathlineto{\pgfqpoint{4.634839in}{3.730839in}}%
\pgfpathlineto{\pgfqpoint{4.661487in}{3.692970in}}%
\pgfpathlineto{\pgfqpoint{4.651669in}{3.694373in}}%
\pgfpathlineto{\pgfqpoint{4.747433in}{3.407080in}}%
\pgfpathlineto{\pgfqpoint{4.739018in}{3.404275in}}%
\pgfusepath{fill}%
\end{pgfscope}%
\begin{pgfscope}%
\pgfpathrectangle{\pgfqpoint{1.432000in}{0.528000in}}{\pgfqpoint{3.696000in}{3.696000in}} %
\pgfusepath{clip}%
\pgfsetbuttcap%
\pgfsetroundjoin%
\definecolor{currentfill}{rgb}{0.278791,0.062145,0.386592}%
\pgfsetfillcolor{currentfill}%
\pgfsetlinewidth{0.000000pt}%
\definecolor{currentstroke}{rgb}{0.000000,0.000000,0.000000}%
\pgfsetstrokecolor{currentstroke}%
\pgfsetdash{}{0pt}%
\pgfpathmoveto{\pgfqpoint{4.738791in}{3.405677in}}%
\pgfpathlineto{\pgfqpoint{4.738791in}{3.690922in}}%
\pgfpathlineto{\pgfqpoint{4.729920in}{3.686487in}}%
\pgfpathlineto{\pgfqpoint{4.743226in}{3.730839in}}%
\pgfpathlineto{\pgfqpoint{4.756531in}{3.686487in}}%
\pgfpathlineto{\pgfqpoint{4.747661in}{3.690922in}}%
\pgfpathlineto{\pgfqpoint{4.747661in}{3.405677in}}%
\pgfpathlineto{\pgfqpoint{4.738791in}{3.405677in}}%
\pgfusepath{fill}%
\end{pgfscope}%
\begin{pgfscope}%
\pgfpathrectangle{\pgfqpoint{1.432000in}{0.528000in}}{\pgfqpoint{3.696000in}{3.696000in}} %
\pgfusepath{clip}%
\pgfsetbuttcap%
\pgfsetroundjoin%
\definecolor{currentfill}{rgb}{0.278791,0.062145,0.386592}%
\pgfsetfillcolor{currentfill}%
\pgfsetlinewidth{0.000000pt}%
\definecolor{currentstroke}{rgb}{0.000000,0.000000,0.000000}%
\pgfsetstrokecolor{currentstroke}%
\pgfsetdash{}{0pt}%
\pgfpathmoveto{\pgfqpoint{4.847178in}{3.405677in}}%
\pgfpathlineto{\pgfqpoint{4.847178in}{3.582535in}}%
\pgfpathlineto{\pgfqpoint{4.838307in}{3.578100in}}%
\pgfpathlineto{\pgfqpoint{4.851613in}{3.622452in}}%
\pgfpathlineto{\pgfqpoint{4.864919in}{3.578100in}}%
\pgfpathlineto{\pgfqpoint{4.856048in}{3.582535in}}%
\pgfpathlineto{\pgfqpoint{4.856048in}{3.405677in}}%
\pgfpathlineto{\pgfqpoint{4.847178in}{3.405677in}}%
\pgfusepath{fill}%
\end{pgfscope}%
\begin{pgfscope}%
\pgfpathrectangle{\pgfqpoint{1.432000in}{0.528000in}}{\pgfqpoint{3.696000in}{3.696000in}} %
\pgfusepath{clip}%
\pgfsetbuttcap%
\pgfsetroundjoin%
\definecolor{currentfill}{rgb}{0.281412,0.155834,0.469201}%
\pgfsetfillcolor{currentfill}%
\pgfsetlinewidth{0.000000pt}%
\definecolor{currentstroke}{rgb}{0.000000,0.000000,0.000000}%
\pgfsetstrokecolor{currentstroke}%
\pgfsetdash{}{0pt}%
\pgfpathmoveto{\pgfqpoint{4.847405in}{3.404275in}}%
\pgfpathlineto{\pgfqpoint{4.751641in}{3.691568in}}%
\pgfpathlineto{\pgfqpoint{4.744628in}{3.684555in}}%
\pgfpathlineto{\pgfqpoint{4.743226in}{3.730839in}}%
\pgfpathlineto{\pgfqpoint{4.769874in}{3.692970in}}%
\pgfpathlineto{\pgfqpoint{4.760056in}{3.694373in}}%
\pgfpathlineto{\pgfqpoint{4.855821in}{3.407080in}}%
\pgfpathlineto{\pgfqpoint{4.847405in}{3.404275in}}%
\pgfusepath{fill}%
\end{pgfscope}%
\begin{pgfscope}%
\pgfpathrectangle{\pgfqpoint{1.432000in}{0.528000in}}{\pgfqpoint{3.696000in}{3.696000in}} %
\pgfusepath{clip}%
\pgfsetbuttcap%
\pgfsetroundjoin%
\definecolor{currentfill}{rgb}{0.235526,0.309527,0.542944}%
\pgfsetfillcolor{currentfill}%
\pgfsetlinewidth{0.000000pt}%
\definecolor{currentstroke}{rgb}{0.000000,0.000000,0.000000}%
\pgfsetstrokecolor{currentstroke}%
\pgfsetdash{}{0pt}%
\pgfpathmoveto{\pgfqpoint{4.847178in}{3.405677in}}%
\pgfpathlineto{\pgfqpoint{4.847178in}{3.690922in}}%
\pgfpathlineto{\pgfqpoint{4.838307in}{3.686487in}}%
\pgfpathlineto{\pgfqpoint{4.851613in}{3.730839in}}%
\pgfpathlineto{\pgfqpoint{4.864919in}{3.686487in}}%
\pgfpathlineto{\pgfqpoint{4.856048in}{3.690922in}}%
\pgfpathlineto{\pgfqpoint{4.856048in}{3.405677in}}%
\pgfpathlineto{\pgfqpoint{4.847178in}{3.405677in}}%
\pgfusepath{fill}%
\end{pgfscope}%
\begin{pgfscope}%
\pgfpathrectangle{\pgfqpoint{1.432000in}{0.528000in}}{\pgfqpoint{3.696000in}{3.696000in}} %
\pgfusepath{clip}%
\pgfsetbuttcap%
\pgfsetroundjoin%
\definecolor{currentfill}{rgb}{0.279574,0.170599,0.479997}%
\pgfsetfillcolor{currentfill}%
\pgfsetlinewidth{0.000000pt}%
\definecolor{currentstroke}{rgb}{0.000000,0.000000,0.000000}%
\pgfsetstrokecolor{currentstroke}%
\pgfsetdash{}{0pt}%
\pgfpathmoveto{\pgfqpoint{4.955565in}{3.405677in}}%
\pgfpathlineto{\pgfqpoint{4.955565in}{3.582535in}}%
\pgfpathlineto{\pgfqpoint{4.946694in}{3.578100in}}%
\pgfpathlineto{\pgfqpoint{4.960000in}{3.622452in}}%
\pgfpathlineto{\pgfqpoint{4.973306in}{3.578100in}}%
\pgfpathlineto{\pgfqpoint{4.964435in}{3.582535in}}%
\pgfpathlineto{\pgfqpoint{4.964435in}{3.405677in}}%
\pgfpathlineto{\pgfqpoint{4.955565in}{3.405677in}}%
\pgfusepath{fill}%
\end{pgfscope}%
\begin{pgfscope}%
\pgfpathrectangle{\pgfqpoint{1.432000in}{0.528000in}}{\pgfqpoint{3.696000in}{3.696000in}} %
\pgfusepath{clip}%
\pgfsetbuttcap%
\pgfsetroundjoin%
\definecolor{currentfill}{rgb}{0.225863,0.330805,0.547314}%
\pgfsetfillcolor{currentfill}%
\pgfsetlinewidth{0.000000pt}%
\definecolor{currentstroke}{rgb}{0.000000,0.000000,0.000000}%
\pgfsetstrokecolor{currentstroke}%
\pgfsetdash{}{0pt}%
\pgfpathmoveto{\pgfqpoint{4.955565in}{3.405677in}}%
\pgfpathlineto{\pgfqpoint{4.955565in}{3.690922in}}%
\pgfpathlineto{\pgfqpoint{4.946694in}{3.686487in}}%
\pgfpathlineto{\pgfqpoint{4.960000in}{3.730839in}}%
\pgfpathlineto{\pgfqpoint{4.973306in}{3.686487in}}%
\pgfpathlineto{\pgfqpoint{4.964435in}{3.690922in}}%
\pgfpathlineto{\pgfqpoint{4.964435in}{3.405677in}}%
\pgfpathlineto{\pgfqpoint{4.955565in}{3.405677in}}%
\pgfusepath{fill}%
\end{pgfscope}%
\begin{pgfscope}%
\pgfpathrectangle{\pgfqpoint{1.432000in}{0.528000in}}{\pgfqpoint{3.696000in}{3.696000in}} %
\pgfusepath{clip}%
\pgfsetbuttcap%
\pgfsetroundjoin%
\definecolor{currentfill}{rgb}{0.252194,0.269783,0.531579}%
\pgfsetfillcolor{currentfill}%
\pgfsetlinewidth{0.000000pt}%
\definecolor{currentstroke}{rgb}{0.000000,0.000000,0.000000}%
\pgfsetstrokecolor{currentstroke}%
\pgfsetdash{}{0pt}%
\pgfpathmoveto{\pgfqpoint{1.604435in}{3.514065in}}%
\pgfpathlineto{\pgfqpoint{1.604435in}{3.337207in}}%
\pgfpathlineto{\pgfqpoint{1.613306in}{3.341642in}}%
\pgfpathlineto{\pgfqpoint{1.600000in}{3.297290in}}%
\pgfpathlineto{\pgfqpoint{1.586694in}{3.341642in}}%
\pgfpathlineto{\pgfqpoint{1.595565in}{3.337207in}}%
\pgfpathlineto{\pgfqpoint{1.595565in}{3.514065in}}%
\pgfpathlineto{\pgfqpoint{1.604435in}{3.514065in}}%
\pgfusepath{fill}%
\end{pgfscope}%
\begin{pgfscope}%
\pgfpathrectangle{\pgfqpoint{1.432000in}{0.528000in}}{\pgfqpoint{3.696000in}{3.696000in}} %
\pgfusepath{clip}%
\pgfsetbuttcap%
\pgfsetroundjoin%
\definecolor{currentfill}{rgb}{0.283091,0.110553,0.431554}%
\pgfsetfillcolor{currentfill}%
\pgfsetlinewidth{0.000000pt}%
\definecolor{currentstroke}{rgb}{0.000000,0.000000,0.000000}%
\pgfsetstrokecolor{currentstroke}%
\pgfsetdash{}{0pt}%
\pgfpathmoveto{\pgfqpoint{1.604435in}{3.514065in}}%
\pgfpathlineto{\pgfqpoint{1.604435in}{3.445594in}}%
\pgfpathlineto{\pgfqpoint{1.613306in}{3.450029in}}%
\pgfpathlineto{\pgfqpoint{1.600000in}{3.405677in}}%
\pgfpathlineto{\pgfqpoint{1.586694in}{3.450029in}}%
\pgfpathlineto{\pgfqpoint{1.595565in}{3.445594in}}%
\pgfpathlineto{\pgfqpoint{1.595565in}{3.514065in}}%
\pgfpathlineto{\pgfqpoint{1.604435in}{3.514065in}}%
\pgfusepath{fill}%
\end{pgfscope}%
\begin{pgfscope}%
\pgfpathrectangle{\pgfqpoint{1.432000in}{0.528000in}}{\pgfqpoint{3.696000in}{3.696000in}} %
\pgfusepath{clip}%
\pgfsetbuttcap%
\pgfsetroundjoin%
\definecolor{currentfill}{rgb}{0.281412,0.155834,0.469201}%
\pgfsetfillcolor{currentfill}%
\pgfsetlinewidth{0.000000pt}%
\definecolor{currentstroke}{rgb}{0.000000,0.000000,0.000000}%
\pgfsetstrokecolor{currentstroke}%
\pgfsetdash{}{0pt}%
\pgfpathmoveto{\pgfqpoint{1.712822in}{3.514065in}}%
\pgfpathlineto{\pgfqpoint{1.712822in}{3.337207in}}%
\pgfpathlineto{\pgfqpoint{1.721693in}{3.341642in}}%
\pgfpathlineto{\pgfqpoint{1.708387in}{3.297290in}}%
\pgfpathlineto{\pgfqpoint{1.695081in}{3.341642in}}%
\pgfpathlineto{\pgfqpoint{1.703952in}{3.337207in}}%
\pgfpathlineto{\pgfqpoint{1.703952in}{3.514065in}}%
\pgfpathlineto{\pgfqpoint{1.712822in}{3.514065in}}%
\pgfusepath{fill}%
\end{pgfscope}%
\begin{pgfscope}%
\pgfpathrectangle{\pgfqpoint{1.432000in}{0.528000in}}{\pgfqpoint{3.696000in}{3.696000in}} %
\pgfusepath{clip}%
\pgfsetbuttcap%
\pgfsetroundjoin%
\definecolor{currentfill}{rgb}{0.271305,0.019942,0.347269}%
\pgfsetfillcolor{currentfill}%
\pgfsetlinewidth{0.000000pt}%
\definecolor{currentstroke}{rgb}{0.000000,0.000000,0.000000}%
\pgfsetstrokecolor{currentstroke}%
\pgfsetdash{}{0pt}%
\pgfpathmoveto{\pgfqpoint{1.712354in}{3.516048in}}%
\pgfpathlineto{\pgfqpoint{1.802890in}{3.334976in}}%
\pgfpathlineto{\pgfqpoint{1.808840in}{3.342910in}}%
\pgfpathlineto{\pgfqpoint{1.816774in}{3.297290in}}%
\pgfpathlineto{\pgfqpoint{1.785038in}{3.331010in}}%
\pgfpathlineto{\pgfqpoint{1.794956in}{3.331010in}}%
\pgfpathlineto{\pgfqpoint{1.704420in}{3.512081in}}%
\pgfpathlineto{\pgfqpoint{1.712354in}{3.516048in}}%
\pgfusepath{fill}%
\end{pgfscope}%
\begin{pgfscope}%
\pgfpathrectangle{\pgfqpoint{1.432000in}{0.528000in}}{\pgfqpoint{3.696000in}{3.696000in}} %
\pgfusepath{clip}%
\pgfsetbuttcap%
\pgfsetroundjoin%
\definecolor{currentfill}{rgb}{0.268510,0.009605,0.335427}%
\pgfsetfillcolor{currentfill}%
\pgfsetlinewidth{0.000000pt}%
\definecolor{currentstroke}{rgb}{0.000000,0.000000,0.000000}%
\pgfsetstrokecolor{currentstroke}%
\pgfsetdash{}{0pt}%
\pgfpathmoveto{\pgfqpoint{1.712822in}{3.514065in}}%
\pgfpathlineto{\pgfqpoint{1.712822in}{3.445594in}}%
\pgfpathlineto{\pgfqpoint{1.721693in}{3.450029in}}%
\pgfpathlineto{\pgfqpoint{1.708387in}{3.405677in}}%
\pgfpathlineto{\pgfqpoint{1.695081in}{3.450029in}}%
\pgfpathlineto{\pgfqpoint{1.703952in}{3.445594in}}%
\pgfpathlineto{\pgfqpoint{1.703952in}{3.514065in}}%
\pgfpathlineto{\pgfqpoint{1.712822in}{3.514065in}}%
\pgfusepath{fill}%
\end{pgfscope}%
\begin{pgfscope}%
\pgfpathrectangle{\pgfqpoint{1.432000in}{0.528000in}}{\pgfqpoint{3.696000in}{3.696000in}} %
\pgfusepath{clip}%
\pgfsetbuttcap%
\pgfsetroundjoin%
\definecolor{currentfill}{rgb}{0.281412,0.155834,0.469201}%
\pgfsetfillcolor{currentfill}%
\pgfsetlinewidth{0.000000pt}%
\definecolor{currentstroke}{rgb}{0.000000,0.000000,0.000000}%
\pgfsetstrokecolor{currentstroke}%
\pgfsetdash{}{0pt}%
\pgfpathmoveto{\pgfqpoint{1.821209in}{3.514065in}}%
\pgfpathlineto{\pgfqpoint{1.821209in}{3.337207in}}%
\pgfpathlineto{\pgfqpoint{1.830080in}{3.341642in}}%
\pgfpathlineto{\pgfqpoint{1.816774in}{3.297290in}}%
\pgfpathlineto{\pgfqpoint{1.803469in}{3.341642in}}%
\pgfpathlineto{\pgfqpoint{1.812339in}{3.337207in}}%
\pgfpathlineto{\pgfqpoint{1.812339in}{3.514065in}}%
\pgfpathlineto{\pgfqpoint{1.821209in}{3.514065in}}%
\pgfusepath{fill}%
\end{pgfscope}%
\begin{pgfscope}%
\pgfpathrectangle{\pgfqpoint{1.432000in}{0.528000in}}{\pgfqpoint{3.696000in}{3.696000in}} %
\pgfusepath{clip}%
\pgfsetbuttcap%
\pgfsetroundjoin%
\definecolor{currentfill}{rgb}{0.283197,0.115680,0.436115}%
\pgfsetfillcolor{currentfill}%
\pgfsetlinewidth{0.000000pt}%
\definecolor{currentstroke}{rgb}{0.000000,0.000000,0.000000}%
\pgfsetstrokecolor{currentstroke}%
\pgfsetdash{}{0pt}%
\pgfpathmoveto{\pgfqpoint{1.821209in}{3.514065in}}%
\pgfpathlineto{\pgfqpoint{1.821209in}{3.445594in}}%
\pgfpathlineto{\pgfqpoint{1.830080in}{3.450029in}}%
\pgfpathlineto{\pgfqpoint{1.816774in}{3.405677in}}%
\pgfpathlineto{\pgfqpoint{1.803469in}{3.450029in}}%
\pgfpathlineto{\pgfqpoint{1.812339in}{3.445594in}}%
\pgfpathlineto{\pgfqpoint{1.812339in}{3.514065in}}%
\pgfpathlineto{\pgfqpoint{1.821209in}{3.514065in}}%
\pgfusepath{fill}%
\end{pgfscope}%
\begin{pgfscope}%
\pgfpathrectangle{\pgfqpoint{1.432000in}{0.528000in}}{\pgfqpoint{3.696000in}{3.696000in}} %
\pgfusepath{clip}%
\pgfsetbuttcap%
\pgfsetroundjoin%
\definecolor{currentfill}{rgb}{0.260571,0.246922,0.522828}%
\pgfsetfillcolor{currentfill}%
\pgfsetlinewidth{0.000000pt}%
\definecolor{currentstroke}{rgb}{0.000000,0.000000,0.000000}%
\pgfsetstrokecolor{currentstroke}%
\pgfsetdash{}{0pt}%
\pgfpathmoveto{\pgfqpoint{1.929596in}{3.514065in}}%
\pgfpathlineto{\pgfqpoint{1.929596in}{3.337207in}}%
\pgfpathlineto{\pgfqpoint{1.938467in}{3.341642in}}%
\pgfpathlineto{\pgfqpoint{1.925161in}{3.297290in}}%
\pgfpathlineto{\pgfqpoint{1.911856in}{3.341642in}}%
\pgfpathlineto{\pgfqpoint{1.920726in}{3.337207in}}%
\pgfpathlineto{\pgfqpoint{1.920726in}{3.514065in}}%
\pgfpathlineto{\pgfqpoint{1.929596in}{3.514065in}}%
\pgfusepath{fill}%
\end{pgfscope}%
\begin{pgfscope}%
\pgfpathrectangle{\pgfqpoint{1.432000in}{0.528000in}}{\pgfqpoint{3.696000in}{3.696000in}} %
\pgfusepath{clip}%
\pgfsetbuttcap%
\pgfsetroundjoin%
\definecolor{currentfill}{rgb}{0.262138,0.242286,0.520837}%
\pgfsetfillcolor{currentfill}%
\pgfsetlinewidth{0.000000pt}%
\definecolor{currentstroke}{rgb}{0.000000,0.000000,0.000000}%
\pgfsetstrokecolor{currentstroke}%
\pgfsetdash{}{0pt}%
\pgfpathmoveto{\pgfqpoint{1.929596in}{3.514065in}}%
\pgfpathlineto{\pgfqpoint{1.929596in}{3.445594in}}%
\pgfpathlineto{\pgfqpoint{1.938467in}{3.450029in}}%
\pgfpathlineto{\pgfqpoint{1.925161in}{3.405677in}}%
\pgfpathlineto{\pgfqpoint{1.911856in}{3.450029in}}%
\pgfpathlineto{\pgfqpoint{1.920726in}{3.445594in}}%
\pgfpathlineto{\pgfqpoint{1.920726in}{3.514065in}}%
\pgfpathlineto{\pgfqpoint{1.929596in}{3.514065in}}%
\pgfusepath{fill}%
\end{pgfscope}%
\begin{pgfscope}%
\pgfpathrectangle{\pgfqpoint{1.432000in}{0.528000in}}{\pgfqpoint{3.696000in}{3.696000in}} %
\pgfusepath{clip}%
\pgfsetbuttcap%
\pgfsetroundjoin%
\definecolor{currentfill}{rgb}{0.281924,0.089666,0.412415}%
\pgfsetfillcolor{currentfill}%
\pgfsetlinewidth{0.000000pt}%
\definecolor{currentstroke}{rgb}{0.000000,0.000000,0.000000}%
\pgfsetstrokecolor{currentstroke}%
\pgfsetdash{}{0pt}%
\pgfpathmoveto{\pgfqpoint{2.037984in}{3.514065in}}%
\pgfpathlineto{\pgfqpoint{2.037984in}{3.337207in}}%
\pgfpathlineto{\pgfqpoint{2.046854in}{3.341642in}}%
\pgfpathlineto{\pgfqpoint{2.033548in}{3.297290in}}%
\pgfpathlineto{\pgfqpoint{2.020243in}{3.341642in}}%
\pgfpathlineto{\pgfqpoint{2.029113in}{3.337207in}}%
\pgfpathlineto{\pgfqpoint{2.029113in}{3.514065in}}%
\pgfpathlineto{\pgfqpoint{2.037984in}{3.514065in}}%
\pgfusepath{fill}%
\end{pgfscope}%
\begin{pgfscope}%
\pgfpathrectangle{\pgfqpoint{1.432000in}{0.528000in}}{\pgfqpoint{3.696000in}{3.696000in}} %
\pgfusepath{clip}%
\pgfsetbuttcap%
\pgfsetroundjoin%
\definecolor{currentfill}{rgb}{0.122606,0.585371,0.546557}%
\pgfsetfillcolor{currentfill}%
\pgfsetlinewidth{0.000000pt}%
\definecolor{currentstroke}{rgb}{0.000000,0.000000,0.000000}%
\pgfsetstrokecolor{currentstroke}%
\pgfsetdash{}{0pt}%
\pgfpathmoveto{\pgfqpoint{2.037984in}{3.514065in}}%
\pgfpathlineto{\pgfqpoint{2.037984in}{3.445594in}}%
\pgfpathlineto{\pgfqpoint{2.046854in}{3.450029in}}%
\pgfpathlineto{\pgfqpoint{2.033548in}{3.405677in}}%
\pgfpathlineto{\pgfqpoint{2.020243in}{3.450029in}}%
\pgfpathlineto{\pgfqpoint{2.029113in}{3.445594in}}%
\pgfpathlineto{\pgfqpoint{2.029113in}{3.514065in}}%
\pgfpathlineto{\pgfqpoint{2.037984in}{3.514065in}}%
\pgfusepath{fill}%
\end{pgfscope}%
\begin{pgfscope}%
\pgfpathrectangle{\pgfqpoint{1.432000in}{0.528000in}}{\pgfqpoint{3.696000in}{3.696000in}} %
\pgfusepath{clip}%
\pgfsetbuttcap%
\pgfsetroundjoin%
\definecolor{currentfill}{rgb}{0.377779,0.791781,0.377939}%
\pgfsetfillcolor{currentfill}%
\pgfsetlinewidth{0.000000pt}%
\definecolor{currentstroke}{rgb}{0.000000,0.000000,0.000000}%
\pgfsetstrokecolor{currentstroke}%
\pgfsetdash{}{0pt}%
\pgfpathmoveto{\pgfqpoint{2.146371in}{3.514065in}}%
\pgfpathlineto{\pgfqpoint{2.146371in}{3.445594in}}%
\pgfpathlineto{\pgfqpoint{2.155241in}{3.450029in}}%
\pgfpathlineto{\pgfqpoint{2.141935in}{3.405677in}}%
\pgfpathlineto{\pgfqpoint{2.128630in}{3.450029in}}%
\pgfpathlineto{\pgfqpoint{2.137500in}{3.445594in}}%
\pgfpathlineto{\pgfqpoint{2.137500in}{3.514065in}}%
\pgfpathlineto{\pgfqpoint{2.146371in}{3.514065in}}%
\pgfusepath{fill}%
\end{pgfscope}%
\begin{pgfscope}%
\pgfpathrectangle{\pgfqpoint{1.432000in}{0.528000in}}{\pgfqpoint{3.696000in}{3.696000in}} %
\pgfusepath{clip}%
\pgfsetbuttcap%
\pgfsetroundjoin%
\definecolor{currentfill}{rgb}{0.120081,0.622161,0.534946}%
\pgfsetfillcolor{currentfill}%
\pgfsetlinewidth{0.000000pt}%
\definecolor{currentstroke}{rgb}{0.000000,0.000000,0.000000}%
\pgfsetstrokecolor{currentstroke}%
\pgfsetdash{}{0pt}%
\pgfpathmoveto{\pgfqpoint{2.254758in}{3.514065in}}%
\pgfpathlineto{\pgfqpoint{2.254758in}{3.445594in}}%
\pgfpathlineto{\pgfqpoint{2.263628in}{3.450029in}}%
\pgfpathlineto{\pgfqpoint{2.250323in}{3.405677in}}%
\pgfpathlineto{\pgfqpoint{2.237017in}{3.450029in}}%
\pgfpathlineto{\pgfqpoint{2.245887in}{3.445594in}}%
\pgfpathlineto{\pgfqpoint{2.245887in}{3.514065in}}%
\pgfpathlineto{\pgfqpoint{2.254758in}{3.514065in}}%
\pgfusepath{fill}%
\end{pgfscope}%
\begin{pgfscope}%
\pgfpathrectangle{\pgfqpoint{1.432000in}{0.528000in}}{\pgfqpoint{3.696000in}{3.696000in}} %
\pgfusepath{clip}%
\pgfsetbuttcap%
\pgfsetroundjoin%
\definecolor{currentfill}{rgb}{0.162142,0.474838,0.558140}%
\pgfsetfillcolor{currentfill}%
\pgfsetlinewidth{0.000000pt}%
\definecolor{currentstroke}{rgb}{0.000000,0.000000,0.000000}%
\pgfsetstrokecolor{currentstroke}%
\pgfsetdash{}{0pt}%
\pgfpathmoveto{\pgfqpoint{2.363145in}{3.514065in}}%
\pgfpathlineto{\pgfqpoint{2.363145in}{3.445594in}}%
\pgfpathlineto{\pgfqpoint{2.372015in}{3.450029in}}%
\pgfpathlineto{\pgfqpoint{2.358710in}{3.405677in}}%
\pgfpathlineto{\pgfqpoint{2.345404in}{3.450029in}}%
\pgfpathlineto{\pgfqpoint{2.354274in}{3.445594in}}%
\pgfpathlineto{\pgfqpoint{2.354274in}{3.514065in}}%
\pgfpathlineto{\pgfqpoint{2.363145in}{3.514065in}}%
\pgfusepath{fill}%
\end{pgfscope}%
\begin{pgfscope}%
\pgfpathrectangle{\pgfqpoint{1.432000in}{0.528000in}}{\pgfqpoint{3.696000in}{3.696000in}} %
\pgfusepath{clip}%
\pgfsetbuttcap%
\pgfsetroundjoin%
\definecolor{currentfill}{rgb}{0.253935,0.265254,0.529983}%
\pgfsetfillcolor{currentfill}%
\pgfsetlinewidth{0.000000pt}%
\definecolor{currentstroke}{rgb}{0.000000,0.000000,0.000000}%
\pgfsetstrokecolor{currentstroke}%
\pgfsetdash{}{0pt}%
\pgfpathmoveto{\pgfqpoint{2.471532in}{3.514065in}}%
\pgfpathlineto{\pgfqpoint{2.471532in}{3.445594in}}%
\pgfpathlineto{\pgfqpoint{2.480402in}{3.450029in}}%
\pgfpathlineto{\pgfqpoint{2.467097in}{3.405677in}}%
\pgfpathlineto{\pgfqpoint{2.453791in}{3.450029in}}%
\pgfpathlineto{\pgfqpoint{2.462662in}{3.445594in}}%
\pgfpathlineto{\pgfqpoint{2.462662in}{3.514065in}}%
\pgfpathlineto{\pgfqpoint{2.471532in}{3.514065in}}%
\pgfusepath{fill}%
\end{pgfscope}%
\begin{pgfscope}%
\pgfpathrectangle{\pgfqpoint{1.432000in}{0.528000in}}{\pgfqpoint{3.696000in}{3.696000in}} %
\pgfusepath{clip}%
\pgfsetbuttcap%
\pgfsetroundjoin%
\definecolor{currentfill}{rgb}{0.277134,0.185228,0.489898}%
\pgfsetfillcolor{currentfill}%
\pgfsetlinewidth{0.000000pt}%
\definecolor{currentstroke}{rgb}{0.000000,0.000000,0.000000}%
\pgfsetstrokecolor{currentstroke}%
\pgfsetdash{}{0pt}%
\pgfpathmoveto{\pgfqpoint{2.471532in}{3.514065in}}%
\pgfpathlineto{\pgfqpoint{2.469314in}{3.517906in}}%
\pgfpathlineto{\pgfqpoint{2.464879in}{3.517906in}}%
\pgfpathlineto{\pgfqpoint{2.462662in}{3.514065in}}%
\pgfpathlineto{\pgfqpoint{2.464879in}{3.510224in}}%
\pgfpathlineto{\pgfqpoint{2.469314in}{3.510224in}}%
\pgfpathlineto{\pgfqpoint{2.471532in}{3.514065in}}%
\pgfpathlineto{\pgfqpoint{2.469314in}{3.517906in}}%
\pgfusepath{fill}%
\end{pgfscope}%
\begin{pgfscope}%
\pgfpathrectangle{\pgfqpoint{1.432000in}{0.528000in}}{\pgfqpoint{3.696000in}{3.696000in}} %
\pgfusepath{clip}%
\pgfsetbuttcap%
\pgfsetroundjoin%
\definecolor{currentfill}{rgb}{0.226397,0.728888,0.462789}%
\pgfsetfillcolor{currentfill}%
\pgfsetlinewidth{0.000000pt}%
\definecolor{currentstroke}{rgb}{0.000000,0.000000,0.000000}%
\pgfsetstrokecolor{currentstroke}%
\pgfsetdash{}{0pt}%
\pgfpathmoveto{\pgfqpoint{2.579919in}{3.514065in}}%
\pgfpathlineto{\pgfqpoint{2.577701in}{3.517906in}}%
\pgfpathlineto{\pgfqpoint{2.573266in}{3.517906in}}%
\pgfpathlineto{\pgfqpoint{2.571049in}{3.514065in}}%
\pgfpathlineto{\pgfqpoint{2.573266in}{3.510224in}}%
\pgfpathlineto{\pgfqpoint{2.577701in}{3.510224in}}%
\pgfpathlineto{\pgfqpoint{2.579919in}{3.514065in}}%
\pgfpathlineto{\pgfqpoint{2.577701in}{3.517906in}}%
\pgfusepath{fill}%
\end{pgfscope}%
\begin{pgfscope}%
\pgfpathrectangle{\pgfqpoint{1.432000in}{0.528000in}}{\pgfqpoint{3.696000in}{3.696000in}} %
\pgfusepath{clip}%
\pgfsetbuttcap%
\pgfsetroundjoin%
\definecolor{currentfill}{rgb}{0.647257,0.858400,0.209861}%
\pgfsetfillcolor{currentfill}%
\pgfsetlinewidth{0.000000pt}%
\definecolor{currentstroke}{rgb}{0.000000,0.000000,0.000000}%
\pgfsetstrokecolor{currentstroke}%
\pgfsetdash{}{0pt}%
\pgfpathmoveto{\pgfqpoint{2.688306in}{3.514065in}}%
\pgfpathlineto{\pgfqpoint{2.686089in}{3.517906in}}%
\pgfpathlineto{\pgfqpoint{2.681653in}{3.517906in}}%
\pgfpathlineto{\pgfqpoint{2.679436in}{3.514065in}}%
\pgfpathlineto{\pgfqpoint{2.681653in}{3.510224in}}%
\pgfpathlineto{\pgfqpoint{2.686089in}{3.510224in}}%
\pgfpathlineto{\pgfqpoint{2.688306in}{3.514065in}}%
\pgfpathlineto{\pgfqpoint{2.686089in}{3.517906in}}%
\pgfusepath{fill}%
\end{pgfscope}%
\begin{pgfscope}%
\pgfpathrectangle{\pgfqpoint{1.432000in}{0.528000in}}{\pgfqpoint{3.696000in}{3.696000in}} %
\pgfusepath{clip}%
\pgfsetbuttcap%
\pgfsetroundjoin%
\definecolor{currentfill}{rgb}{0.657642,0.860219,0.203082}%
\pgfsetfillcolor{currentfill}%
\pgfsetlinewidth{0.000000pt}%
\definecolor{currentstroke}{rgb}{0.000000,0.000000,0.000000}%
\pgfsetstrokecolor{currentstroke}%
\pgfsetdash{}{0pt}%
\pgfpathmoveto{\pgfqpoint{2.796693in}{3.514065in}}%
\pgfpathlineto{\pgfqpoint{2.794476in}{3.517906in}}%
\pgfpathlineto{\pgfqpoint{2.790040in}{3.517906in}}%
\pgfpathlineto{\pgfqpoint{2.787823in}{3.514065in}}%
\pgfpathlineto{\pgfqpoint{2.790040in}{3.510224in}}%
\pgfpathlineto{\pgfqpoint{2.794476in}{3.510224in}}%
\pgfpathlineto{\pgfqpoint{2.796693in}{3.514065in}}%
\pgfpathlineto{\pgfqpoint{2.794476in}{3.517906in}}%
\pgfusepath{fill}%
\end{pgfscope}%
\begin{pgfscope}%
\pgfpathrectangle{\pgfqpoint{1.432000in}{0.528000in}}{\pgfqpoint{3.696000in}{3.696000in}} %
\pgfusepath{clip}%
\pgfsetbuttcap%
\pgfsetroundjoin%
\definecolor{currentfill}{rgb}{0.140210,0.665859,0.513427}%
\pgfsetfillcolor{currentfill}%
\pgfsetlinewidth{0.000000pt}%
\definecolor{currentstroke}{rgb}{0.000000,0.000000,0.000000}%
\pgfsetstrokecolor{currentstroke}%
\pgfsetdash{}{0pt}%
\pgfpathmoveto{\pgfqpoint{2.905080in}{3.514065in}}%
\pgfpathlineto{\pgfqpoint{2.902863in}{3.517906in}}%
\pgfpathlineto{\pgfqpoint{2.898428in}{3.517906in}}%
\pgfpathlineto{\pgfqpoint{2.896210in}{3.514065in}}%
\pgfpathlineto{\pgfqpoint{2.898428in}{3.510224in}}%
\pgfpathlineto{\pgfqpoint{2.902863in}{3.510224in}}%
\pgfpathlineto{\pgfqpoint{2.905080in}{3.514065in}}%
\pgfpathlineto{\pgfqpoint{2.902863in}{3.517906in}}%
\pgfusepath{fill}%
\end{pgfscope}%
\begin{pgfscope}%
\pgfpathrectangle{\pgfqpoint{1.432000in}{0.528000in}}{\pgfqpoint{3.696000in}{3.696000in}} %
\pgfusepath{clip}%
\pgfsetbuttcap%
\pgfsetroundjoin%
\definecolor{currentfill}{rgb}{0.179019,0.433756,0.557430}%
\pgfsetfillcolor{currentfill}%
\pgfsetlinewidth{0.000000pt}%
\definecolor{currentstroke}{rgb}{0.000000,0.000000,0.000000}%
\pgfsetstrokecolor{currentstroke}%
\pgfsetdash{}{0pt}%
\pgfpathmoveto{\pgfqpoint{3.013467in}{3.514065in}}%
\pgfpathlineto{\pgfqpoint{3.011250in}{3.517906in}}%
\pgfpathlineto{\pgfqpoint{3.006815in}{3.517906in}}%
\pgfpathlineto{\pgfqpoint{3.004597in}{3.514065in}}%
\pgfpathlineto{\pgfqpoint{3.006815in}{3.510224in}}%
\pgfpathlineto{\pgfqpoint{3.011250in}{3.510224in}}%
\pgfpathlineto{\pgfqpoint{3.013467in}{3.514065in}}%
\pgfpathlineto{\pgfqpoint{3.011250in}{3.517906in}}%
\pgfusepath{fill}%
\end{pgfscope}%
\begin{pgfscope}%
\pgfpathrectangle{\pgfqpoint{1.432000in}{0.528000in}}{\pgfqpoint{3.696000in}{3.696000in}} %
\pgfusepath{clip}%
\pgfsetbuttcap%
\pgfsetroundjoin%
\definecolor{currentfill}{rgb}{0.270595,0.214069,0.507052}%
\pgfsetfillcolor{currentfill}%
\pgfsetlinewidth{0.000000pt}%
\definecolor{currentstroke}{rgb}{0.000000,0.000000,0.000000}%
\pgfsetstrokecolor{currentstroke}%
\pgfsetdash{}{0pt}%
\pgfpathmoveto{\pgfqpoint{3.009032in}{3.518500in}}%
\pgfpathlineto{\pgfqpoint{3.077503in}{3.518500in}}%
\pgfpathlineto{\pgfqpoint{3.073067in}{3.527370in}}%
\pgfpathlineto{\pgfqpoint{3.117419in}{3.514065in}}%
\pgfpathlineto{\pgfqpoint{3.073067in}{3.500759in}}%
\pgfpathlineto{\pgfqpoint{3.077503in}{3.509629in}}%
\pgfpathlineto{\pgfqpoint{3.009032in}{3.509629in}}%
\pgfpathlineto{\pgfqpoint{3.009032in}{3.518500in}}%
\pgfusepath{fill}%
\end{pgfscope}%
\begin{pgfscope}%
\pgfpathrectangle{\pgfqpoint{1.432000in}{0.528000in}}{\pgfqpoint{3.696000in}{3.696000in}} %
\pgfusepath{clip}%
\pgfsetbuttcap%
\pgfsetroundjoin%
\definecolor{currentfill}{rgb}{0.257322,0.256130,0.526563}%
\pgfsetfillcolor{currentfill}%
\pgfsetlinewidth{0.000000pt}%
\definecolor{currentstroke}{rgb}{0.000000,0.000000,0.000000}%
\pgfsetstrokecolor{currentstroke}%
\pgfsetdash{}{0pt}%
\pgfpathmoveto{\pgfqpoint{3.121855in}{3.514065in}}%
\pgfpathlineto{\pgfqpoint{3.119637in}{3.517906in}}%
\pgfpathlineto{\pgfqpoint{3.115202in}{3.517906in}}%
\pgfpathlineto{\pgfqpoint{3.112984in}{3.514065in}}%
\pgfpathlineto{\pgfqpoint{3.115202in}{3.510224in}}%
\pgfpathlineto{\pgfqpoint{3.119637in}{3.510224in}}%
\pgfpathlineto{\pgfqpoint{3.121855in}{3.514065in}}%
\pgfpathlineto{\pgfqpoint{3.119637in}{3.517906in}}%
\pgfusepath{fill}%
\end{pgfscope}%
\begin{pgfscope}%
\pgfpathrectangle{\pgfqpoint{1.432000in}{0.528000in}}{\pgfqpoint{3.696000in}{3.696000in}} %
\pgfusepath{clip}%
\pgfsetbuttcap%
\pgfsetroundjoin%
\definecolor{currentfill}{rgb}{0.267968,0.223549,0.512008}%
\pgfsetfillcolor{currentfill}%
\pgfsetlinewidth{0.000000pt}%
\definecolor{currentstroke}{rgb}{0.000000,0.000000,0.000000}%
\pgfsetstrokecolor{currentstroke}%
\pgfsetdash{}{0pt}%
\pgfpathmoveto{\pgfqpoint{3.117419in}{3.518500in}}%
\pgfpathlineto{\pgfqpoint{3.185890in}{3.518500in}}%
\pgfpathlineto{\pgfqpoint{3.181454in}{3.527370in}}%
\pgfpathlineto{\pgfqpoint{3.225806in}{3.514065in}}%
\pgfpathlineto{\pgfqpoint{3.181454in}{3.500759in}}%
\pgfpathlineto{\pgfqpoint{3.185890in}{3.509629in}}%
\pgfpathlineto{\pgfqpoint{3.117419in}{3.509629in}}%
\pgfpathlineto{\pgfqpoint{3.117419in}{3.518500in}}%
\pgfusepath{fill}%
\end{pgfscope}%
\begin{pgfscope}%
\pgfpathrectangle{\pgfqpoint{1.432000in}{0.528000in}}{\pgfqpoint{3.696000in}{3.696000in}} %
\pgfusepath{clip}%
\pgfsetbuttcap%
\pgfsetroundjoin%
\definecolor{currentfill}{rgb}{0.277134,0.185228,0.489898}%
\pgfsetfillcolor{currentfill}%
\pgfsetlinewidth{0.000000pt}%
\definecolor{currentstroke}{rgb}{0.000000,0.000000,0.000000}%
\pgfsetstrokecolor{currentstroke}%
\pgfsetdash{}{0pt}%
\pgfpathmoveto{\pgfqpoint{3.230242in}{3.514065in}}%
\pgfpathlineto{\pgfqpoint{3.228024in}{3.517906in}}%
\pgfpathlineto{\pgfqpoint{3.223589in}{3.517906in}}%
\pgfpathlineto{\pgfqpoint{3.221371in}{3.514065in}}%
\pgfpathlineto{\pgfqpoint{3.223589in}{3.510224in}}%
\pgfpathlineto{\pgfqpoint{3.228024in}{3.510224in}}%
\pgfpathlineto{\pgfqpoint{3.230242in}{3.514065in}}%
\pgfpathlineto{\pgfqpoint{3.228024in}{3.517906in}}%
\pgfusepath{fill}%
\end{pgfscope}%
\begin{pgfscope}%
\pgfpathrectangle{\pgfqpoint{1.432000in}{0.528000in}}{\pgfqpoint{3.696000in}{3.696000in}} %
\pgfusepath{clip}%
\pgfsetbuttcap%
\pgfsetroundjoin%
\definecolor{currentfill}{rgb}{0.268510,0.009605,0.335427}%
\pgfsetfillcolor{currentfill}%
\pgfsetlinewidth{0.000000pt}%
\definecolor{currentstroke}{rgb}{0.000000,0.000000,0.000000}%
\pgfsetstrokecolor{currentstroke}%
\pgfsetdash{}{0pt}%
\pgfpathmoveto{\pgfqpoint{3.547832in}{3.510928in}}%
\pgfpathlineto{\pgfqpoint{3.467670in}{3.591090in}}%
\pgfpathlineto{\pgfqpoint{3.464534in}{3.581682in}}%
\pgfpathlineto{\pgfqpoint{3.442581in}{3.622452in}}%
\pgfpathlineto{\pgfqpoint{3.483351in}{3.600498in}}%
\pgfpathlineto{\pgfqpoint{3.473942in}{3.597362in}}%
\pgfpathlineto{\pgfqpoint{3.554104in}{3.517201in}}%
\pgfpathlineto{\pgfqpoint{3.547832in}{3.510928in}}%
\pgfusepath{fill}%
\end{pgfscope}%
\begin{pgfscope}%
\pgfpathrectangle{\pgfqpoint{1.432000in}{0.528000in}}{\pgfqpoint{3.696000in}{3.696000in}} %
\pgfusepath{clip}%
\pgfsetbuttcap%
\pgfsetroundjoin%
\definecolor{currentfill}{rgb}{0.281887,0.150881,0.465405}%
\pgfsetfillcolor{currentfill}%
\pgfsetlinewidth{0.000000pt}%
\definecolor{currentstroke}{rgb}{0.000000,0.000000,0.000000}%
\pgfsetstrokecolor{currentstroke}%
\pgfsetdash{}{0pt}%
\pgfpathmoveto{\pgfqpoint{3.657371in}{3.510098in}}%
\pgfpathlineto{\pgfqpoint{3.476300in}{3.600633in}}%
\pgfpathlineto{\pgfqpoint{3.476300in}{3.590716in}}%
\pgfpathlineto{\pgfqpoint{3.442581in}{3.622452in}}%
\pgfpathlineto{\pgfqpoint{3.488201in}{3.614518in}}%
\pgfpathlineto{\pgfqpoint{3.480267in}{3.608567in}}%
\pgfpathlineto{\pgfqpoint{3.661338in}{3.518031in}}%
\pgfpathlineto{\pgfqpoint{3.657371in}{3.510098in}}%
\pgfusepath{fill}%
\end{pgfscope}%
\begin{pgfscope}%
\pgfpathrectangle{\pgfqpoint{1.432000in}{0.528000in}}{\pgfqpoint{3.696000in}{3.696000in}} %
\pgfusepath{clip}%
\pgfsetbuttcap%
\pgfsetroundjoin%
\definecolor{currentfill}{rgb}{0.267004,0.004874,0.329415}%
\pgfsetfillcolor{currentfill}%
\pgfsetlinewidth{0.000000pt}%
\definecolor{currentstroke}{rgb}{0.000000,0.000000,0.000000}%
\pgfsetstrokecolor{currentstroke}%
\pgfsetdash{}{0pt}%
\pgfpathmoveto{\pgfqpoint{3.764606in}{3.510928in}}%
\pgfpathlineto{\pgfqpoint{3.576057in}{3.699477in}}%
\pgfpathlineto{\pgfqpoint{3.572921in}{3.690069in}}%
\pgfpathlineto{\pgfqpoint{3.550968in}{3.730839in}}%
\pgfpathlineto{\pgfqpoint{3.591738in}{3.708886in}}%
\pgfpathlineto{\pgfqpoint{3.582329in}{3.705749in}}%
\pgfpathlineto{\pgfqpoint{3.770878in}{3.517201in}}%
\pgfpathlineto{\pgfqpoint{3.764606in}{3.510928in}}%
\pgfusepath{fill}%
\end{pgfscope}%
\begin{pgfscope}%
\pgfpathrectangle{\pgfqpoint{1.432000in}{0.528000in}}{\pgfqpoint{3.696000in}{3.696000in}} %
\pgfusepath{clip}%
\pgfsetbuttcap%
\pgfsetroundjoin%
\definecolor{currentfill}{rgb}{0.252194,0.269783,0.531579}%
\pgfsetfillcolor{currentfill}%
\pgfsetlinewidth{0.000000pt}%
\definecolor{currentstroke}{rgb}{0.000000,0.000000,0.000000}%
\pgfsetstrokecolor{currentstroke}%
\pgfsetdash{}{0pt}%
\pgfpathmoveto{\pgfqpoint{3.872993in}{3.510928in}}%
\pgfpathlineto{\pgfqpoint{3.684444in}{3.699477in}}%
\pgfpathlineto{\pgfqpoint{3.681308in}{3.690069in}}%
\pgfpathlineto{\pgfqpoint{3.659355in}{3.730839in}}%
\pgfpathlineto{\pgfqpoint{3.700125in}{3.708886in}}%
\pgfpathlineto{\pgfqpoint{3.690716in}{3.705749in}}%
\pgfpathlineto{\pgfqpoint{3.879265in}{3.517201in}}%
\pgfpathlineto{\pgfqpoint{3.872993in}{3.510928in}}%
\pgfusepath{fill}%
\end{pgfscope}%
\begin{pgfscope}%
\pgfpathrectangle{\pgfqpoint{1.432000in}{0.528000in}}{\pgfqpoint{3.696000in}{3.696000in}} %
\pgfusepath{clip}%
\pgfsetbuttcap%
\pgfsetroundjoin%
\definecolor{currentfill}{rgb}{0.255645,0.260703,0.528312}%
\pgfsetfillcolor{currentfill}%
\pgfsetlinewidth{0.000000pt}%
\definecolor{currentstroke}{rgb}{0.000000,0.000000,0.000000}%
\pgfsetstrokecolor{currentstroke}%
\pgfsetdash{}{0pt}%
\pgfpathmoveto{\pgfqpoint{3.982056in}{3.510374in}}%
\pgfpathlineto{\pgfqpoint{3.690107in}{3.705007in}}%
\pgfpathlineto{\pgfqpoint{3.688877in}{3.695166in}}%
\pgfpathlineto{\pgfqpoint{3.659355in}{3.730839in}}%
\pgfpathlineto{\pgfqpoint{3.703639in}{3.717308in}}%
\pgfpathlineto{\pgfqpoint{3.695028in}{3.712387in}}%
\pgfpathlineto{\pgfqpoint{3.986976in}{3.517755in}}%
\pgfpathlineto{\pgfqpoint{3.982056in}{3.510374in}}%
\pgfusepath{fill}%
\end{pgfscope}%
\begin{pgfscope}%
\pgfpathrectangle{\pgfqpoint{1.432000in}{0.528000in}}{\pgfqpoint{3.696000in}{3.696000in}} %
\pgfusepath{clip}%
\pgfsetbuttcap%
\pgfsetroundjoin%
\definecolor{currentfill}{rgb}{0.165117,0.467423,0.558141}%
\pgfsetfillcolor{currentfill}%
\pgfsetlinewidth{0.000000pt}%
\definecolor{currentstroke}{rgb}{0.000000,0.000000,0.000000}%
\pgfsetstrokecolor{currentstroke}%
\pgfsetdash{}{0pt}%
\pgfpathmoveto{\pgfqpoint{4.090443in}{3.510374in}}%
\pgfpathlineto{\pgfqpoint{3.798495in}{3.705007in}}%
\pgfpathlineto{\pgfqpoint{3.797264in}{3.695166in}}%
\pgfpathlineto{\pgfqpoint{3.767742in}{3.730839in}}%
\pgfpathlineto{\pgfqpoint{3.812026in}{3.717308in}}%
\pgfpathlineto{\pgfqpoint{3.803415in}{3.712387in}}%
\pgfpathlineto{\pgfqpoint{4.095363in}{3.517755in}}%
\pgfpathlineto{\pgfqpoint{4.090443in}{3.510374in}}%
\pgfusepath{fill}%
\end{pgfscope}%
\begin{pgfscope}%
\pgfpathrectangle{\pgfqpoint{1.432000in}{0.528000in}}{\pgfqpoint{3.696000in}{3.696000in}} %
\pgfusepath{clip}%
\pgfsetbuttcap%
\pgfsetroundjoin%
\definecolor{currentfill}{rgb}{0.223925,0.334994,0.548053}%
\pgfsetfillcolor{currentfill}%
\pgfsetlinewidth{0.000000pt}%
\definecolor{currentstroke}{rgb}{0.000000,0.000000,0.000000}%
\pgfsetstrokecolor{currentstroke}%
\pgfsetdash{}{0pt}%
\pgfpathmoveto{\pgfqpoint{4.198830in}{3.510374in}}%
\pgfpathlineto{\pgfqpoint{3.906882in}{3.705007in}}%
\pgfpathlineto{\pgfqpoint{3.905652in}{3.695166in}}%
\pgfpathlineto{\pgfqpoint{3.876129in}{3.730839in}}%
\pgfpathlineto{\pgfqpoint{3.920413in}{3.717308in}}%
\pgfpathlineto{\pgfqpoint{3.911802in}{3.712387in}}%
\pgfpathlineto{\pgfqpoint{4.203751in}{3.517755in}}%
\pgfpathlineto{\pgfqpoint{4.198830in}{3.510374in}}%
\pgfusepath{fill}%
\end{pgfscope}%
\begin{pgfscope}%
\pgfpathrectangle{\pgfqpoint{1.432000in}{0.528000in}}{\pgfqpoint{3.696000in}{3.696000in}} %
\pgfusepath{clip}%
\pgfsetbuttcap%
\pgfsetroundjoin%
\definecolor{currentfill}{rgb}{0.281924,0.089666,0.412415}%
\pgfsetfillcolor{currentfill}%
\pgfsetlinewidth{0.000000pt}%
\definecolor{currentstroke}{rgb}{0.000000,0.000000,0.000000}%
\pgfsetstrokecolor{currentstroke}%
\pgfsetdash{}{0pt}%
\pgfpathmoveto{\pgfqpoint{4.198154in}{3.510928in}}%
\pgfpathlineto{\pgfqpoint{3.901218in}{3.807864in}}%
\pgfpathlineto{\pgfqpoint{3.898082in}{3.798456in}}%
\pgfpathlineto{\pgfqpoint{3.876129in}{3.839226in}}%
\pgfpathlineto{\pgfqpoint{3.916899in}{3.817273in}}%
\pgfpathlineto{\pgfqpoint{3.907491in}{3.814137in}}%
\pgfpathlineto{\pgfqpoint{4.204426in}{3.517201in}}%
\pgfpathlineto{\pgfqpoint{4.198154in}{3.510928in}}%
\pgfusepath{fill}%
\end{pgfscope}%
\begin{pgfscope}%
\pgfpathrectangle{\pgfqpoint{1.432000in}{0.528000in}}{\pgfqpoint{3.696000in}{3.696000in}} %
\pgfusepath{clip}%
\pgfsetbuttcap%
\pgfsetroundjoin%
\definecolor{currentfill}{rgb}{0.281887,0.150881,0.465405}%
\pgfsetfillcolor{currentfill}%
\pgfsetlinewidth{0.000000pt}%
\definecolor{currentstroke}{rgb}{0.000000,0.000000,0.000000}%
\pgfsetstrokecolor{currentstroke}%
\pgfsetdash{}{0pt}%
\pgfpathmoveto{\pgfqpoint{4.307217in}{3.510374in}}%
\pgfpathlineto{\pgfqpoint{4.015269in}{3.705007in}}%
\pgfpathlineto{\pgfqpoint{4.014039in}{3.695166in}}%
\pgfpathlineto{\pgfqpoint{3.984516in}{3.730839in}}%
\pgfpathlineto{\pgfqpoint{4.028800in}{3.717308in}}%
\pgfpathlineto{\pgfqpoint{4.020189in}{3.712387in}}%
\pgfpathlineto{\pgfqpoint{4.312138in}{3.517755in}}%
\pgfpathlineto{\pgfqpoint{4.307217in}{3.510374in}}%
\pgfusepath{fill}%
\end{pgfscope}%
\begin{pgfscope}%
\pgfpathrectangle{\pgfqpoint{1.432000in}{0.528000in}}{\pgfqpoint{3.696000in}{3.696000in}} %
\pgfusepath{clip}%
\pgfsetbuttcap%
\pgfsetroundjoin%
\definecolor{currentfill}{rgb}{0.190631,0.407061,0.556089}%
\pgfsetfillcolor{currentfill}%
\pgfsetlinewidth{0.000000pt}%
\definecolor{currentstroke}{rgb}{0.000000,0.000000,0.000000}%
\pgfsetstrokecolor{currentstroke}%
\pgfsetdash{}{0pt}%
\pgfpathmoveto{\pgfqpoint{4.306541in}{3.510928in}}%
\pgfpathlineto{\pgfqpoint{4.009605in}{3.807864in}}%
\pgfpathlineto{\pgfqpoint{4.006469in}{3.798456in}}%
\pgfpathlineto{\pgfqpoint{3.984516in}{3.839226in}}%
\pgfpathlineto{\pgfqpoint{4.025286in}{3.817273in}}%
\pgfpathlineto{\pgfqpoint{4.015878in}{3.814137in}}%
\pgfpathlineto{\pgfqpoint{4.312814in}{3.517201in}}%
\pgfpathlineto{\pgfqpoint{4.306541in}{3.510928in}}%
\pgfusepath{fill}%
\end{pgfscope}%
\begin{pgfscope}%
\pgfpathrectangle{\pgfqpoint{1.432000in}{0.528000in}}{\pgfqpoint{3.696000in}{3.696000in}} %
\pgfusepath{clip}%
\pgfsetbuttcap%
\pgfsetroundjoin%
\definecolor{currentfill}{rgb}{0.277941,0.056324,0.381191}%
\pgfsetfillcolor{currentfill}%
\pgfsetlinewidth{0.000000pt}%
\definecolor{currentstroke}{rgb}{0.000000,0.000000,0.000000}%
\pgfsetstrokecolor{currentstroke}%
\pgfsetdash{}{0pt}%
\pgfpathmoveto{\pgfqpoint{4.415604in}{3.510374in}}%
\pgfpathlineto{\pgfqpoint{4.123656in}{3.705007in}}%
\pgfpathlineto{\pgfqpoint{4.122426in}{3.695166in}}%
\pgfpathlineto{\pgfqpoint{4.092903in}{3.730839in}}%
\pgfpathlineto{\pgfqpoint{4.137187in}{3.717308in}}%
\pgfpathlineto{\pgfqpoint{4.128576in}{3.712387in}}%
\pgfpathlineto{\pgfqpoint{4.420525in}{3.517755in}}%
\pgfpathlineto{\pgfqpoint{4.415604in}{3.510374in}}%
\pgfusepath{fill}%
\end{pgfscope}%
\begin{pgfscope}%
\pgfpathrectangle{\pgfqpoint{1.432000in}{0.528000in}}{\pgfqpoint{3.696000in}{3.696000in}} %
\pgfusepath{clip}%
\pgfsetbuttcap%
\pgfsetroundjoin%
\definecolor{currentfill}{rgb}{0.279574,0.170599,0.479997}%
\pgfsetfillcolor{currentfill}%
\pgfsetlinewidth{0.000000pt}%
\definecolor{currentstroke}{rgb}{0.000000,0.000000,0.000000}%
\pgfsetstrokecolor{currentstroke}%
\pgfsetdash{}{0pt}%
\pgfpathmoveto{\pgfqpoint{4.414928in}{3.510928in}}%
\pgfpathlineto{\pgfqpoint{4.226380in}{3.699477in}}%
\pgfpathlineto{\pgfqpoint{4.223243in}{3.690069in}}%
\pgfpathlineto{\pgfqpoint{4.201290in}{3.730839in}}%
\pgfpathlineto{\pgfqpoint{4.242060in}{3.708886in}}%
\pgfpathlineto{\pgfqpoint{4.232652in}{3.705749in}}%
\pgfpathlineto{\pgfqpoint{4.421201in}{3.517201in}}%
\pgfpathlineto{\pgfqpoint{4.414928in}{3.510928in}}%
\pgfusepath{fill}%
\end{pgfscope}%
\begin{pgfscope}%
\pgfpathrectangle{\pgfqpoint{1.432000in}{0.528000in}}{\pgfqpoint{3.696000in}{3.696000in}} %
\pgfusepath{clip}%
\pgfsetbuttcap%
\pgfsetroundjoin%
\definecolor{currentfill}{rgb}{0.273006,0.204520,0.501721}%
\pgfsetfillcolor{currentfill}%
\pgfsetlinewidth{0.000000pt}%
\definecolor{currentstroke}{rgb}{0.000000,0.000000,0.000000}%
\pgfsetstrokecolor{currentstroke}%
\pgfsetdash{}{0pt}%
\pgfpathmoveto{\pgfqpoint{4.414928in}{3.510928in}}%
\pgfpathlineto{\pgfqpoint{4.117993in}{3.807864in}}%
\pgfpathlineto{\pgfqpoint{4.114856in}{3.798456in}}%
\pgfpathlineto{\pgfqpoint{4.092903in}{3.839226in}}%
\pgfpathlineto{\pgfqpoint{4.133673in}{3.817273in}}%
\pgfpathlineto{\pgfqpoint{4.124265in}{3.814137in}}%
\pgfpathlineto{\pgfqpoint{4.421201in}{3.517201in}}%
\pgfpathlineto{\pgfqpoint{4.414928in}{3.510928in}}%
\pgfusepath{fill}%
\end{pgfscope}%
\begin{pgfscope}%
\pgfpathrectangle{\pgfqpoint{1.432000in}{0.528000in}}{\pgfqpoint{3.696000in}{3.696000in}} %
\pgfusepath{clip}%
\pgfsetbuttcap%
\pgfsetroundjoin%
\definecolor{currentfill}{rgb}{0.274128,0.199721,0.498911}%
\pgfsetfillcolor{currentfill}%
\pgfsetlinewidth{0.000000pt}%
\definecolor{currentstroke}{rgb}{0.000000,0.000000,0.000000}%
\pgfsetstrokecolor{currentstroke}%
\pgfsetdash{}{0pt}%
\pgfpathmoveto{\pgfqpoint{4.414374in}{3.511604in}}%
\pgfpathlineto{\pgfqpoint{4.219742in}{3.803553in}}%
\pgfpathlineto{\pgfqpoint{4.214821in}{3.794942in}}%
\pgfpathlineto{\pgfqpoint{4.201290in}{3.839226in}}%
\pgfpathlineto{\pgfqpoint{4.236963in}{3.809703in}}%
\pgfpathlineto{\pgfqpoint{4.227122in}{3.808473in}}%
\pgfpathlineto{\pgfqpoint{4.421755in}{3.516525in}}%
\pgfpathlineto{\pgfqpoint{4.414374in}{3.511604in}}%
\pgfusepath{fill}%
\end{pgfscope}%
\begin{pgfscope}%
\pgfpathrectangle{\pgfqpoint{1.432000in}{0.528000in}}{\pgfqpoint{3.696000in}{3.696000in}} %
\pgfusepath{clip}%
\pgfsetbuttcap%
\pgfsetroundjoin%
\definecolor{currentfill}{rgb}{0.276194,0.190074,0.493001}%
\pgfsetfillcolor{currentfill}%
\pgfsetlinewidth{0.000000pt}%
\definecolor{currentstroke}{rgb}{0.000000,0.000000,0.000000}%
\pgfsetstrokecolor{currentstroke}%
\pgfsetdash{}{0pt}%
\pgfpathmoveto{\pgfqpoint{4.523315in}{3.510928in}}%
\pgfpathlineto{\pgfqpoint{4.334767in}{3.699477in}}%
\pgfpathlineto{\pgfqpoint{4.331631in}{3.690069in}}%
\pgfpathlineto{\pgfqpoint{4.309677in}{3.730839in}}%
\pgfpathlineto{\pgfqpoint{4.350447in}{3.708886in}}%
\pgfpathlineto{\pgfqpoint{4.341039in}{3.705749in}}%
\pgfpathlineto{\pgfqpoint{4.529588in}{3.517201in}}%
\pgfpathlineto{\pgfqpoint{4.523315in}{3.510928in}}%
\pgfusepath{fill}%
\end{pgfscope}%
\begin{pgfscope}%
\pgfpathrectangle{\pgfqpoint{1.432000in}{0.528000in}}{\pgfqpoint{3.696000in}{3.696000in}} %
\pgfusepath{clip}%
\pgfsetbuttcap%
\pgfsetroundjoin%
\definecolor{currentfill}{rgb}{0.270595,0.214069,0.507052}%
\pgfsetfillcolor{currentfill}%
\pgfsetlinewidth{0.000000pt}%
\definecolor{currentstroke}{rgb}{0.000000,0.000000,0.000000}%
\pgfsetstrokecolor{currentstroke}%
\pgfsetdash{}{0pt}%
\pgfpathmoveto{\pgfqpoint{4.522761in}{3.511604in}}%
\pgfpathlineto{\pgfqpoint{4.328129in}{3.803553in}}%
\pgfpathlineto{\pgfqpoint{4.323209in}{3.794942in}}%
\pgfpathlineto{\pgfqpoint{4.309677in}{3.839226in}}%
\pgfpathlineto{\pgfqpoint{4.345350in}{3.809703in}}%
\pgfpathlineto{\pgfqpoint{4.335510in}{3.808473in}}%
\pgfpathlineto{\pgfqpoint{4.530142in}{3.516525in}}%
\pgfpathlineto{\pgfqpoint{4.522761in}{3.511604in}}%
\pgfusepath{fill}%
\end{pgfscope}%
\begin{pgfscope}%
\pgfpathrectangle{\pgfqpoint{1.432000in}{0.528000in}}{\pgfqpoint{3.696000in}{3.696000in}} %
\pgfusepath{clip}%
\pgfsetbuttcap%
\pgfsetroundjoin%
\definecolor{currentfill}{rgb}{0.280894,0.078907,0.402329}%
\pgfsetfillcolor{currentfill}%
\pgfsetlinewidth{0.000000pt}%
\definecolor{currentstroke}{rgb}{0.000000,0.000000,0.000000}%
\pgfsetstrokecolor{currentstroke}%
\pgfsetdash{}{0pt}%
\pgfpathmoveto{\pgfqpoint{4.522244in}{3.512662in}}%
\pgfpathlineto{\pgfqpoint{4.426480in}{3.799955in}}%
\pgfpathlineto{\pgfqpoint{4.419467in}{3.792942in}}%
\pgfpathlineto{\pgfqpoint{4.418065in}{3.839226in}}%
\pgfpathlineto{\pgfqpoint{4.444713in}{3.801357in}}%
\pgfpathlineto{\pgfqpoint{4.434895in}{3.802760in}}%
\pgfpathlineto{\pgfqpoint{4.530659in}{3.515467in}}%
\pgfpathlineto{\pgfqpoint{4.522244in}{3.512662in}}%
\pgfusepath{fill}%
\end{pgfscope}%
\begin{pgfscope}%
\pgfpathrectangle{\pgfqpoint{1.432000in}{0.528000in}}{\pgfqpoint{3.696000in}{3.696000in}} %
\pgfusepath{clip}%
\pgfsetbuttcap%
\pgfsetroundjoin%
\definecolor{currentfill}{rgb}{0.273809,0.031497,0.358853}%
\pgfsetfillcolor{currentfill}%
\pgfsetlinewidth{0.000000pt}%
\definecolor{currentstroke}{rgb}{0.000000,0.000000,0.000000}%
\pgfsetstrokecolor{currentstroke}%
\pgfsetdash{}{0pt}%
\pgfpathmoveto{\pgfqpoint{4.631703in}{3.510928in}}%
\pgfpathlineto{\pgfqpoint{4.443154in}{3.699477in}}%
\pgfpathlineto{\pgfqpoint{4.440018in}{3.690069in}}%
\pgfpathlineto{\pgfqpoint{4.418065in}{3.730839in}}%
\pgfpathlineto{\pgfqpoint{4.458835in}{3.708886in}}%
\pgfpathlineto{\pgfqpoint{4.449426in}{3.705749in}}%
\pgfpathlineto{\pgfqpoint{4.637975in}{3.517201in}}%
\pgfpathlineto{\pgfqpoint{4.631703in}{3.510928in}}%
\pgfusepath{fill}%
\end{pgfscope}%
\begin{pgfscope}%
\pgfpathrectangle{\pgfqpoint{1.432000in}{0.528000in}}{\pgfqpoint{3.696000in}{3.696000in}} %
\pgfusepath{clip}%
\pgfsetbuttcap%
\pgfsetroundjoin%
\definecolor{currentfill}{rgb}{0.267968,0.223549,0.512008}%
\pgfsetfillcolor{currentfill}%
\pgfsetlinewidth{0.000000pt}%
\definecolor{currentstroke}{rgb}{0.000000,0.000000,0.000000}%
\pgfsetstrokecolor{currentstroke}%
\pgfsetdash{}{0pt}%
\pgfpathmoveto{\pgfqpoint{4.630872in}{3.512081in}}%
\pgfpathlineto{\pgfqpoint{4.540336in}{3.693153in}}%
\pgfpathlineto{\pgfqpoint{4.534386in}{3.685219in}}%
\pgfpathlineto{\pgfqpoint{4.526452in}{3.730839in}}%
\pgfpathlineto{\pgfqpoint{4.558187in}{3.697120in}}%
\pgfpathlineto{\pgfqpoint{4.548270in}{3.697120in}}%
\pgfpathlineto{\pgfqpoint{4.638806in}{3.516048in}}%
\pgfpathlineto{\pgfqpoint{4.630872in}{3.512081in}}%
\pgfusepath{fill}%
\end{pgfscope}%
\begin{pgfscope}%
\pgfpathrectangle{\pgfqpoint{1.432000in}{0.528000in}}{\pgfqpoint{3.696000in}{3.696000in}} %
\pgfusepath{clip}%
\pgfsetbuttcap%
\pgfsetroundjoin%
\definecolor{currentfill}{rgb}{0.265145,0.232956,0.516599}%
\pgfsetfillcolor{currentfill}%
\pgfsetlinewidth{0.000000pt}%
\definecolor{currentstroke}{rgb}{0.000000,0.000000,0.000000}%
\pgfsetstrokecolor{currentstroke}%
\pgfsetdash{}{0pt}%
\pgfpathmoveto{\pgfqpoint{4.630631in}{3.512662in}}%
\pgfpathlineto{\pgfqpoint{4.534867in}{3.799955in}}%
\pgfpathlineto{\pgfqpoint{4.527854in}{3.792942in}}%
\pgfpathlineto{\pgfqpoint{4.526452in}{3.839226in}}%
\pgfpathlineto{\pgfqpoint{4.553100in}{3.801357in}}%
\pgfpathlineto{\pgfqpoint{4.543282in}{3.802760in}}%
\pgfpathlineto{\pgfqpoint{4.639046in}{3.515467in}}%
\pgfpathlineto{\pgfqpoint{4.630631in}{3.512662in}}%
\pgfusepath{fill}%
\end{pgfscope}%
\begin{pgfscope}%
\pgfpathrectangle{\pgfqpoint{1.432000in}{0.528000in}}{\pgfqpoint{3.696000in}{3.696000in}} %
\pgfusepath{clip}%
\pgfsetbuttcap%
\pgfsetroundjoin%
\definecolor{currentfill}{rgb}{0.199430,0.387607,0.554642}%
\pgfsetfillcolor{currentfill}%
\pgfsetlinewidth{0.000000pt}%
\definecolor{currentstroke}{rgb}{0.000000,0.000000,0.000000}%
\pgfsetstrokecolor{currentstroke}%
\pgfsetdash{}{0pt}%
\pgfpathmoveto{\pgfqpoint{4.739259in}{3.512081in}}%
\pgfpathlineto{\pgfqpoint{4.648723in}{3.693153in}}%
\pgfpathlineto{\pgfqpoint{4.642773in}{3.685219in}}%
\pgfpathlineto{\pgfqpoint{4.634839in}{3.730839in}}%
\pgfpathlineto{\pgfqpoint{4.666574in}{3.697120in}}%
\pgfpathlineto{\pgfqpoint{4.656657in}{3.697120in}}%
\pgfpathlineto{\pgfqpoint{4.747193in}{3.516048in}}%
\pgfpathlineto{\pgfqpoint{4.739259in}{3.512081in}}%
\pgfusepath{fill}%
\end{pgfscope}%
\begin{pgfscope}%
\pgfpathrectangle{\pgfqpoint{1.432000in}{0.528000in}}{\pgfqpoint{3.696000in}{3.696000in}} %
\pgfusepath{clip}%
\pgfsetbuttcap%
\pgfsetroundjoin%
\definecolor{currentfill}{rgb}{0.281887,0.150881,0.465405}%
\pgfsetfillcolor{currentfill}%
\pgfsetlinewidth{0.000000pt}%
\definecolor{currentstroke}{rgb}{0.000000,0.000000,0.000000}%
\pgfsetstrokecolor{currentstroke}%
\pgfsetdash{}{0pt}%
\pgfpathmoveto{\pgfqpoint{4.738791in}{3.514065in}}%
\pgfpathlineto{\pgfqpoint{4.738791in}{3.690922in}}%
\pgfpathlineto{\pgfqpoint{4.729920in}{3.686487in}}%
\pgfpathlineto{\pgfqpoint{4.743226in}{3.730839in}}%
\pgfpathlineto{\pgfqpoint{4.756531in}{3.686487in}}%
\pgfpathlineto{\pgfqpoint{4.747661in}{3.690922in}}%
\pgfpathlineto{\pgfqpoint{4.747661in}{3.514065in}}%
\pgfpathlineto{\pgfqpoint{4.738791in}{3.514065in}}%
\pgfusepath{fill}%
\end{pgfscope}%
\begin{pgfscope}%
\pgfpathrectangle{\pgfqpoint{1.432000in}{0.528000in}}{\pgfqpoint{3.696000in}{3.696000in}} %
\pgfusepath{clip}%
\pgfsetbuttcap%
\pgfsetroundjoin%
\definecolor{currentfill}{rgb}{0.283187,0.125848,0.444960}%
\pgfsetfillcolor{currentfill}%
\pgfsetlinewidth{0.000000pt}%
\definecolor{currentstroke}{rgb}{0.000000,0.000000,0.000000}%
\pgfsetstrokecolor{currentstroke}%
\pgfsetdash{}{0pt}%
\pgfpathmoveto{\pgfqpoint{4.847646in}{3.512081in}}%
\pgfpathlineto{\pgfqpoint{4.757110in}{3.693153in}}%
\pgfpathlineto{\pgfqpoint{4.751160in}{3.685219in}}%
\pgfpathlineto{\pgfqpoint{4.743226in}{3.730839in}}%
\pgfpathlineto{\pgfqpoint{4.774962in}{3.697120in}}%
\pgfpathlineto{\pgfqpoint{4.765044in}{3.697120in}}%
\pgfpathlineto{\pgfqpoint{4.855580in}{3.516048in}}%
\pgfpathlineto{\pgfqpoint{4.847646in}{3.512081in}}%
\pgfusepath{fill}%
\end{pgfscope}%
\begin{pgfscope}%
\pgfpathrectangle{\pgfqpoint{1.432000in}{0.528000in}}{\pgfqpoint{3.696000in}{3.696000in}} %
\pgfusepath{clip}%
\pgfsetbuttcap%
\pgfsetroundjoin%
\definecolor{currentfill}{rgb}{0.206756,0.371758,0.553117}%
\pgfsetfillcolor{currentfill}%
\pgfsetlinewidth{0.000000pt}%
\definecolor{currentstroke}{rgb}{0.000000,0.000000,0.000000}%
\pgfsetstrokecolor{currentstroke}%
\pgfsetdash{}{0pt}%
\pgfpathmoveto{\pgfqpoint{4.847178in}{3.514065in}}%
\pgfpathlineto{\pgfqpoint{4.847178in}{3.690922in}}%
\pgfpathlineto{\pgfqpoint{4.838307in}{3.686487in}}%
\pgfpathlineto{\pgfqpoint{4.851613in}{3.730839in}}%
\pgfpathlineto{\pgfqpoint{4.864919in}{3.686487in}}%
\pgfpathlineto{\pgfqpoint{4.856048in}{3.690922in}}%
\pgfpathlineto{\pgfqpoint{4.856048in}{3.514065in}}%
\pgfpathlineto{\pgfqpoint{4.847178in}{3.514065in}}%
\pgfusepath{fill}%
\end{pgfscope}%
\begin{pgfscope}%
\pgfpathrectangle{\pgfqpoint{1.432000in}{0.528000in}}{\pgfqpoint{3.696000in}{3.696000in}} %
\pgfusepath{clip}%
\pgfsetbuttcap%
\pgfsetroundjoin%
\definecolor{currentfill}{rgb}{0.133743,0.548535,0.553541}%
\pgfsetfillcolor{currentfill}%
\pgfsetlinewidth{0.000000pt}%
\definecolor{currentstroke}{rgb}{0.000000,0.000000,0.000000}%
\pgfsetstrokecolor{currentstroke}%
\pgfsetdash{}{0pt}%
\pgfpathmoveto{\pgfqpoint{4.955565in}{3.514065in}}%
\pgfpathlineto{\pgfqpoint{4.955565in}{3.690922in}}%
\pgfpathlineto{\pgfqpoint{4.946694in}{3.686487in}}%
\pgfpathlineto{\pgfqpoint{4.960000in}{3.730839in}}%
\pgfpathlineto{\pgfqpoint{4.973306in}{3.686487in}}%
\pgfpathlineto{\pgfqpoint{4.964435in}{3.690922in}}%
\pgfpathlineto{\pgfqpoint{4.964435in}{3.514065in}}%
\pgfpathlineto{\pgfqpoint{4.955565in}{3.514065in}}%
\pgfusepath{fill}%
\end{pgfscope}%
\begin{pgfscope}%
\pgfpathrectangle{\pgfqpoint{1.432000in}{0.528000in}}{\pgfqpoint{3.696000in}{3.696000in}} %
\pgfusepath{clip}%
\pgfsetbuttcap%
\pgfsetroundjoin%
\definecolor{currentfill}{rgb}{0.241237,0.296485,0.539709}%
\pgfsetfillcolor{currentfill}%
\pgfsetlinewidth{0.000000pt}%
\definecolor{currentstroke}{rgb}{0.000000,0.000000,0.000000}%
\pgfsetstrokecolor{currentstroke}%
\pgfsetdash{}{0pt}%
\pgfpathmoveto{\pgfqpoint{1.604435in}{3.622452in}}%
\pgfpathlineto{\pgfqpoint{1.604435in}{3.445594in}}%
\pgfpathlineto{\pgfqpoint{1.613306in}{3.450029in}}%
\pgfpathlineto{\pgfqpoint{1.600000in}{3.405677in}}%
\pgfpathlineto{\pgfqpoint{1.586694in}{3.450029in}}%
\pgfpathlineto{\pgfqpoint{1.595565in}{3.445594in}}%
\pgfpathlineto{\pgfqpoint{1.595565in}{3.622452in}}%
\pgfpathlineto{\pgfqpoint{1.604435in}{3.622452in}}%
\pgfusepath{fill}%
\end{pgfscope}%
\begin{pgfscope}%
\pgfpathrectangle{\pgfqpoint{1.432000in}{0.528000in}}{\pgfqpoint{3.696000in}{3.696000in}} %
\pgfusepath{clip}%
\pgfsetbuttcap%
\pgfsetroundjoin%
\definecolor{currentfill}{rgb}{0.280868,0.160771,0.472899}%
\pgfsetfillcolor{currentfill}%
\pgfsetlinewidth{0.000000pt}%
\definecolor{currentstroke}{rgb}{0.000000,0.000000,0.000000}%
\pgfsetstrokecolor{currentstroke}%
\pgfsetdash{}{0pt}%
\pgfpathmoveto{\pgfqpoint{1.604435in}{3.622452in}}%
\pgfpathlineto{\pgfqpoint{1.604435in}{3.553981in}}%
\pgfpathlineto{\pgfqpoint{1.613306in}{3.558417in}}%
\pgfpathlineto{\pgfqpoint{1.600000in}{3.514065in}}%
\pgfpathlineto{\pgfqpoint{1.586694in}{3.558417in}}%
\pgfpathlineto{\pgfqpoint{1.595565in}{3.553981in}}%
\pgfpathlineto{\pgfqpoint{1.595565in}{3.622452in}}%
\pgfpathlineto{\pgfqpoint{1.604435in}{3.622452in}}%
\pgfusepath{fill}%
\end{pgfscope}%
\begin{pgfscope}%
\pgfpathrectangle{\pgfqpoint{1.432000in}{0.528000in}}{\pgfqpoint{3.696000in}{3.696000in}} %
\pgfusepath{clip}%
\pgfsetbuttcap%
\pgfsetroundjoin%
\definecolor{currentfill}{rgb}{0.282623,0.140926,0.457517}%
\pgfsetfillcolor{currentfill}%
\pgfsetlinewidth{0.000000pt}%
\definecolor{currentstroke}{rgb}{0.000000,0.000000,0.000000}%
\pgfsetstrokecolor{currentstroke}%
\pgfsetdash{}{0pt}%
\pgfpathmoveto{\pgfqpoint{1.712822in}{3.622452in}}%
\pgfpathlineto{\pgfqpoint{1.712822in}{3.445594in}}%
\pgfpathlineto{\pgfqpoint{1.721693in}{3.450029in}}%
\pgfpathlineto{\pgfqpoint{1.708387in}{3.405677in}}%
\pgfpathlineto{\pgfqpoint{1.695081in}{3.450029in}}%
\pgfpathlineto{\pgfqpoint{1.703952in}{3.445594in}}%
\pgfpathlineto{\pgfqpoint{1.703952in}{3.622452in}}%
\pgfpathlineto{\pgfqpoint{1.712822in}{3.622452in}}%
\pgfusepath{fill}%
\end{pgfscope}%
\begin{pgfscope}%
\pgfpathrectangle{\pgfqpoint{1.432000in}{0.528000in}}{\pgfqpoint{3.696000in}{3.696000in}} %
\pgfusepath{clip}%
\pgfsetbuttcap%
\pgfsetroundjoin%
\definecolor{currentfill}{rgb}{0.278826,0.175490,0.483397}%
\pgfsetfillcolor{currentfill}%
\pgfsetlinewidth{0.000000pt}%
\definecolor{currentstroke}{rgb}{0.000000,0.000000,0.000000}%
\pgfsetstrokecolor{currentstroke}%
\pgfsetdash{}{0pt}%
\pgfpathmoveto{\pgfqpoint{1.821209in}{3.622452in}}%
\pgfpathlineto{\pgfqpoint{1.821209in}{3.445594in}}%
\pgfpathlineto{\pgfqpoint{1.830080in}{3.450029in}}%
\pgfpathlineto{\pgfqpoint{1.816774in}{3.405677in}}%
\pgfpathlineto{\pgfqpoint{1.803469in}{3.450029in}}%
\pgfpathlineto{\pgfqpoint{1.812339in}{3.445594in}}%
\pgfpathlineto{\pgfqpoint{1.812339in}{3.622452in}}%
\pgfpathlineto{\pgfqpoint{1.821209in}{3.622452in}}%
\pgfusepath{fill}%
\end{pgfscope}%
\begin{pgfscope}%
\pgfpathrectangle{\pgfqpoint{1.432000in}{0.528000in}}{\pgfqpoint{3.696000in}{3.696000in}} %
\pgfusepath{clip}%
\pgfsetbuttcap%
\pgfsetroundjoin%
\definecolor{currentfill}{rgb}{0.280894,0.078907,0.402329}%
\pgfsetfillcolor{currentfill}%
\pgfsetlinewidth{0.000000pt}%
\definecolor{currentstroke}{rgb}{0.000000,0.000000,0.000000}%
\pgfsetstrokecolor{currentstroke}%
\pgfsetdash{}{0pt}%
\pgfpathmoveto{\pgfqpoint{1.821209in}{3.622452in}}%
\pgfpathlineto{\pgfqpoint{1.821209in}{3.553981in}}%
\pgfpathlineto{\pgfqpoint{1.830080in}{3.558417in}}%
\pgfpathlineto{\pgfqpoint{1.816774in}{3.514065in}}%
\pgfpathlineto{\pgfqpoint{1.803469in}{3.558417in}}%
\pgfpathlineto{\pgfqpoint{1.812339in}{3.553981in}}%
\pgfpathlineto{\pgfqpoint{1.812339in}{3.622452in}}%
\pgfpathlineto{\pgfqpoint{1.821209in}{3.622452in}}%
\pgfusepath{fill}%
\end{pgfscope}%
\begin{pgfscope}%
\pgfpathrectangle{\pgfqpoint{1.432000in}{0.528000in}}{\pgfqpoint{3.696000in}{3.696000in}} %
\pgfusepath{clip}%
\pgfsetbuttcap%
\pgfsetroundjoin%
\definecolor{currentfill}{rgb}{0.255645,0.260703,0.528312}%
\pgfsetfillcolor{currentfill}%
\pgfsetlinewidth{0.000000pt}%
\definecolor{currentstroke}{rgb}{0.000000,0.000000,0.000000}%
\pgfsetstrokecolor{currentstroke}%
\pgfsetdash{}{0pt}%
\pgfpathmoveto{\pgfqpoint{1.929596in}{3.622452in}}%
\pgfpathlineto{\pgfqpoint{1.929596in}{3.445594in}}%
\pgfpathlineto{\pgfqpoint{1.938467in}{3.450029in}}%
\pgfpathlineto{\pgfqpoint{1.925161in}{3.405677in}}%
\pgfpathlineto{\pgfqpoint{1.911856in}{3.450029in}}%
\pgfpathlineto{\pgfqpoint{1.920726in}{3.445594in}}%
\pgfpathlineto{\pgfqpoint{1.920726in}{3.622452in}}%
\pgfpathlineto{\pgfqpoint{1.929596in}{3.622452in}}%
\pgfusepath{fill}%
\end{pgfscope}%
\begin{pgfscope}%
\pgfpathrectangle{\pgfqpoint{1.432000in}{0.528000in}}{\pgfqpoint{3.696000in}{3.696000in}} %
\pgfusepath{clip}%
\pgfsetbuttcap%
\pgfsetroundjoin%
\definecolor{currentfill}{rgb}{0.282623,0.140926,0.457517}%
\pgfsetfillcolor{currentfill}%
\pgfsetlinewidth{0.000000pt}%
\definecolor{currentstroke}{rgb}{0.000000,0.000000,0.000000}%
\pgfsetstrokecolor{currentstroke}%
\pgfsetdash{}{0pt}%
\pgfpathmoveto{\pgfqpoint{1.929596in}{3.622452in}}%
\pgfpathlineto{\pgfqpoint{1.929596in}{3.553981in}}%
\pgfpathlineto{\pgfqpoint{1.938467in}{3.558417in}}%
\pgfpathlineto{\pgfqpoint{1.925161in}{3.514065in}}%
\pgfpathlineto{\pgfqpoint{1.911856in}{3.558417in}}%
\pgfpathlineto{\pgfqpoint{1.920726in}{3.553981in}}%
\pgfpathlineto{\pgfqpoint{1.920726in}{3.622452in}}%
\pgfpathlineto{\pgfqpoint{1.929596in}{3.622452in}}%
\pgfusepath{fill}%
\end{pgfscope}%
\begin{pgfscope}%
\pgfpathrectangle{\pgfqpoint{1.432000in}{0.528000in}}{\pgfqpoint{3.696000in}{3.696000in}} %
\pgfusepath{clip}%
\pgfsetbuttcap%
\pgfsetroundjoin%
\definecolor{currentfill}{rgb}{0.171176,0.452530,0.557965}%
\pgfsetfillcolor{currentfill}%
\pgfsetlinewidth{0.000000pt}%
\definecolor{currentstroke}{rgb}{0.000000,0.000000,0.000000}%
\pgfsetstrokecolor{currentstroke}%
\pgfsetdash{}{0pt}%
\pgfpathmoveto{\pgfqpoint{2.037984in}{3.622452in}}%
\pgfpathlineto{\pgfqpoint{2.037984in}{3.553981in}}%
\pgfpathlineto{\pgfqpoint{2.046854in}{3.558417in}}%
\pgfpathlineto{\pgfqpoint{2.033548in}{3.514065in}}%
\pgfpathlineto{\pgfqpoint{2.020243in}{3.558417in}}%
\pgfpathlineto{\pgfqpoint{2.029113in}{3.553981in}}%
\pgfpathlineto{\pgfqpoint{2.029113in}{3.622452in}}%
\pgfpathlineto{\pgfqpoint{2.037984in}{3.622452in}}%
\pgfusepath{fill}%
\end{pgfscope}%
\begin{pgfscope}%
\pgfpathrectangle{\pgfqpoint{1.432000in}{0.528000in}}{\pgfqpoint{3.696000in}{3.696000in}} %
\pgfusepath{clip}%
\pgfsetbuttcap%
\pgfsetroundjoin%
\definecolor{currentfill}{rgb}{0.120638,0.625828,0.533488}%
\pgfsetfillcolor{currentfill}%
\pgfsetlinewidth{0.000000pt}%
\definecolor{currentstroke}{rgb}{0.000000,0.000000,0.000000}%
\pgfsetstrokecolor{currentstroke}%
\pgfsetdash{}{0pt}%
\pgfpathmoveto{\pgfqpoint{2.146371in}{3.622452in}}%
\pgfpathlineto{\pgfqpoint{2.146371in}{3.553981in}}%
\pgfpathlineto{\pgfqpoint{2.155241in}{3.558417in}}%
\pgfpathlineto{\pgfqpoint{2.141935in}{3.514065in}}%
\pgfpathlineto{\pgfqpoint{2.128630in}{3.558417in}}%
\pgfpathlineto{\pgfqpoint{2.137500in}{3.553981in}}%
\pgfpathlineto{\pgfqpoint{2.137500in}{3.622452in}}%
\pgfpathlineto{\pgfqpoint{2.146371in}{3.622452in}}%
\pgfusepath{fill}%
\end{pgfscope}%
\begin{pgfscope}%
\pgfpathrectangle{\pgfqpoint{1.432000in}{0.528000in}}{\pgfqpoint{3.696000in}{3.696000in}} %
\pgfusepath{clip}%
\pgfsetbuttcap%
\pgfsetroundjoin%
\definecolor{currentfill}{rgb}{0.147607,0.511733,0.557049}%
\pgfsetfillcolor{currentfill}%
\pgfsetlinewidth{0.000000pt}%
\definecolor{currentstroke}{rgb}{0.000000,0.000000,0.000000}%
\pgfsetstrokecolor{currentstroke}%
\pgfsetdash{}{0pt}%
\pgfpathmoveto{\pgfqpoint{2.254758in}{3.622452in}}%
\pgfpathlineto{\pgfqpoint{2.254758in}{3.553981in}}%
\pgfpathlineto{\pgfqpoint{2.263628in}{3.558417in}}%
\pgfpathlineto{\pgfqpoint{2.250323in}{3.514065in}}%
\pgfpathlineto{\pgfqpoint{2.237017in}{3.558417in}}%
\pgfpathlineto{\pgfqpoint{2.245887in}{3.553981in}}%
\pgfpathlineto{\pgfqpoint{2.245887in}{3.622452in}}%
\pgfpathlineto{\pgfqpoint{2.254758in}{3.622452in}}%
\pgfusepath{fill}%
\end{pgfscope}%
\begin{pgfscope}%
\pgfpathrectangle{\pgfqpoint{1.432000in}{0.528000in}}{\pgfqpoint{3.696000in}{3.696000in}} %
\pgfusepath{clip}%
\pgfsetbuttcap%
\pgfsetroundjoin%
\definecolor{currentfill}{rgb}{0.120565,0.596422,0.543611}%
\pgfsetfillcolor{currentfill}%
\pgfsetlinewidth{0.000000pt}%
\definecolor{currentstroke}{rgb}{0.000000,0.000000,0.000000}%
\pgfsetstrokecolor{currentstroke}%
\pgfsetdash{}{0pt}%
\pgfpathmoveto{\pgfqpoint{2.363145in}{3.622452in}}%
\pgfpathlineto{\pgfqpoint{2.363145in}{3.553981in}}%
\pgfpathlineto{\pgfqpoint{2.372015in}{3.558417in}}%
\pgfpathlineto{\pgfqpoint{2.358710in}{3.514065in}}%
\pgfpathlineto{\pgfqpoint{2.345404in}{3.558417in}}%
\pgfpathlineto{\pgfqpoint{2.354274in}{3.553981in}}%
\pgfpathlineto{\pgfqpoint{2.354274in}{3.622452in}}%
\pgfpathlineto{\pgfqpoint{2.363145in}{3.622452in}}%
\pgfusepath{fill}%
\end{pgfscope}%
\begin{pgfscope}%
\pgfpathrectangle{\pgfqpoint{1.432000in}{0.528000in}}{\pgfqpoint{3.696000in}{3.696000in}} %
\pgfusepath{clip}%
\pgfsetbuttcap%
\pgfsetroundjoin%
\definecolor{currentfill}{rgb}{0.212395,0.359683,0.551710}%
\pgfsetfillcolor{currentfill}%
\pgfsetlinewidth{0.000000pt}%
\definecolor{currentstroke}{rgb}{0.000000,0.000000,0.000000}%
\pgfsetstrokecolor{currentstroke}%
\pgfsetdash{}{0pt}%
\pgfpathmoveto{\pgfqpoint{2.471532in}{3.622452in}}%
\pgfpathlineto{\pgfqpoint{2.471532in}{3.553981in}}%
\pgfpathlineto{\pgfqpoint{2.480402in}{3.558417in}}%
\pgfpathlineto{\pgfqpoint{2.467097in}{3.514065in}}%
\pgfpathlineto{\pgfqpoint{2.453791in}{3.558417in}}%
\pgfpathlineto{\pgfqpoint{2.462662in}{3.553981in}}%
\pgfpathlineto{\pgfqpoint{2.462662in}{3.622452in}}%
\pgfpathlineto{\pgfqpoint{2.471532in}{3.622452in}}%
\pgfusepath{fill}%
\end{pgfscope}%
\begin{pgfscope}%
\pgfpathrectangle{\pgfqpoint{1.432000in}{0.528000in}}{\pgfqpoint{3.696000in}{3.696000in}} %
\pgfusepath{clip}%
\pgfsetbuttcap%
\pgfsetroundjoin%
\definecolor{currentfill}{rgb}{0.277018,0.050344,0.375715}%
\pgfsetfillcolor{currentfill}%
\pgfsetlinewidth{0.000000pt}%
\definecolor{currentstroke}{rgb}{0.000000,0.000000,0.000000}%
\pgfsetstrokecolor{currentstroke}%
\pgfsetdash{}{0pt}%
\pgfpathmoveto{\pgfqpoint{2.471532in}{3.622452in}}%
\pgfpathlineto{\pgfqpoint{2.469314in}{3.626293in}}%
\pgfpathlineto{\pgfqpoint{2.464879in}{3.626293in}}%
\pgfpathlineto{\pgfqpoint{2.462662in}{3.622452in}}%
\pgfpathlineto{\pgfqpoint{2.464879in}{3.618611in}}%
\pgfpathlineto{\pgfqpoint{2.469314in}{3.618611in}}%
\pgfpathlineto{\pgfqpoint{2.471532in}{3.622452in}}%
\pgfpathlineto{\pgfqpoint{2.469314in}{3.626293in}}%
\pgfusepath{fill}%
\end{pgfscope}%
\begin{pgfscope}%
\pgfpathrectangle{\pgfqpoint{1.432000in}{0.528000in}}{\pgfqpoint{3.696000in}{3.696000in}} %
\pgfusepath{clip}%
\pgfsetbuttcap%
\pgfsetroundjoin%
\definecolor{currentfill}{rgb}{0.202219,0.715272,0.476084}%
\pgfsetfillcolor{currentfill}%
\pgfsetlinewidth{0.000000pt}%
\definecolor{currentstroke}{rgb}{0.000000,0.000000,0.000000}%
\pgfsetstrokecolor{currentstroke}%
\pgfsetdash{}{0pt}%
\pgfpathmoveto{\pgfqpoint{2.579919in}{3.622452in}}%
\pgfpathlineto{\pgfqpoint{2.577701in}{3.626293in}}%
\pgfpathlineto{\pgfqpoint{2.573266in}{3.626293in}}%
\pgfpathlineto{\pgfqpoint{2.571049in}{3.622452in}}%
\pgfpathlineto{\pgfqpoint{2.573266in}{3.618611in}}%
\pgfpathlineto{\pgfqpoint{2.577701in}{3.618611in}}%
\pgfpathlineto{\pgfqpoint{2.579919in}{3.622452in}}%
\pgfpathlineto{\pgfqpoint{2.577701in}{3.626293in}}%
\pgfusepath{fill}%
\end{pgfscope}%
\begin{pgfscope}%
\pgfpathrectangle{\pgfqpoint{1.432000in}{0.528000in}}{\pgfqpoint{3.696000in}{3.696000in}} %
\pgfusepath{clip}%
\pgfsetbuttcap%
\pgfsetroundjoin%
\definecolor{currentfill}{rgb}{0.496615,0.826376,0.306377}%
\pgfsetfillcolor{currentfill}%
\pgfsetlinewidth{0.000000pt}%
\definecolor{currentstroke}{rgb}{0.000000,0.000000,0.000000}%
\pgfsetstrokecolor{currentstroke}%
\pgfsetdash{}{0pt}%
\pgfpathmoveto{\pgfqpoint{2.688306in}{3.622452in}}%
\pgfpathlineto{\pgfqpoint{2.686089in}{3.626293in}}%
\pgfpathlineto{\pgfqpoint{2.681653in}{3.626293in}}%
\pgfpathlineto{\pgfqpoint{2.679436in}{3.622452in}}%
\pgfpathlineto{\pgfqpoint{2.681653in}{3.618611in}}%
\pgfpathlineto{\pgfqpoint{2.686089in}{3.618611in}}%
\pgfpathlineto{\pgfqpoint{2.688306in}{3.622452in}}%
\pgfpathlineto{\pgfqpoint{2.686089in}{3.626293in}}%
\pgfusepath{fill}%
\end{pgfscope}%
\begin{pgfscope}%
\pgfpathrectangle{\pgfqpoint{1.432000in}{0.528000in}}{\pgfqpoint{3.696000in}{3.696000in}} %
\pgfusepath{clip}%
\pgfsetbuttcap%
\pgfsetroundjoin%
\definecolor{currentfill}{rgb}{0.266941,0.748751,0.440573}%
\pgfsetfillcolor{currentfill}%
\pgfsetlinewidth{0.000000pt}%
\definecolor{currentstroke}{rgb}{0.000000,0.000000,0.000000}%
\pgfsetstrokecolor{currentstroke}%
\pgfsetdash{}{0pt}%
\pgfpathmoveto{\pgfqpoint{2.796693in}{3.622452in}}%
\pgfpathlineto{\pgfqpoint{2.794476in}{3.626293in}}%
\pgfpathlineto{\pgfqpoint{2.790040in}{3.626293in}}%
\pgfpathlineto{\pgfqpoint{2.787823in}{3.622452in}}%
\pgfpathlineto{\pgfqpoint{2.790040in}{3.618611in}}%
\pgfpathlineto{\pgfqpoint{2.794476in}{3.618611in}}%
\pgfpathlineto{\pgfqpoint{2.796693in}{3.622452in}}%
\pgfpathlineto{\pgfqpoint{2.794476in}{3.626293in}}%
\pgfusepath{fill}%
\end{pgfscope}%
\begin{pgfscope}%
\pgfpathrectangle{\pgfqpoint{1.432000in}{0.528000in}}{\pgfqpoint{3.696000in}{3.696000in}} %
\pgfusepath{clip}%
\pgfsetbuttcap%
\pgfsetroundjoin%
\definecolor{currentfill}{rgb}{0.282884,0.135920,0.453427}%
\pgfsetfillcolor{currentfill}%
\pgfsetlinewidth{0.000000pt}%
\definecolor{currentstroke}{rgb}{0.000000,0.000000,0.000000}%
\pgfsetstrokecolor{currentstroke}%
\pgfsetdash{}{0pt}%
\pgfpathmoveto{\pgfqpoint{2.792258in}{3.626887in}}%
\pgfpathlineto{\pgfqpoint{2.860728in}{3.626887in}}%
\pgfpathlineto{\pgfqpoint{2.856293in}{3.635757in}}%
\pgfpathlineto{\pgfqpoint{2.900645in}{3.622452in}}%
\pgfpathlineto{\pgfqpoint{2.856293in}{3.609146in}}%
\pgfpathlineto{\pgfqpoint{2.860728in}{3.618016in}}%
\pgfpathlineto{\pgfqpoint{2.792258in}{3.618016in}}%
\pgfpathlineto{\pgfqpoint{2.792258in}{3.626887in}}%
\pgfusepath{fill}%
\end{pgfscope}%
\begin{pgfscope}%
\pgfpathrectangle{\pgfqpoint{1.432000in}{0.528000in}}{\pgfqpoint{3.696000in}{3.696000in}} %
\pgfusepath{clip}%
\pgfsetbuttcap%
\pgfsetroundjoin%
\definecolor{currentfill}{rgb}{0.190631,0.407061,0.556089}%
\pgfsetfillcolor{currentfill}%
\pgfsetlinewidth{0.000000pt}%
\definecolor{currentstroke}{rgb}{0.000000,0.000000,0.000000}%
\pgfsetstrokecolor{currentstroke}%
\pgfsetdash{}{0pt}%
\pgfpathmoveto{\pgfqpoint{2.905080in}{3.622452in}}%
\pgfpathlineto{\pgfqpoint{2.902863in}{3.626293in}}%
\pgfpathlineto{\pgfqpoint{2.898428in}{3.626293in}}%
\pgfpathlineto{\pgfqpoint{2.896210in}{3.622452in}}%
\pgfpathlineto{\pgfqpoint{2.898428in}{3.618611in}}%
\pgfpathlineto{\pgfqpoint{2.902863in}{3.618611in}}%
\pgfpathlineto{\pgfqpoint{2.905080in}{3.622452in}}%
\pgfpathlineto{\pgfqpoint{2.902863in}{3.626293in}}%
\pgfusepath{fill}%
\end{pgfscope}%
\begin{pgfscope}%
\pgfpathrectangle{\pgfqpoint{1.432000in}{0.528000in}}{\pgfqpoint{3.696000in}{3.696000in}} %
\pgfusepath{clip}%
\pgfsetbuttcap%
\pgfsetroundjoin%
\definecolor{currentfill}{rgb}{0.237441,0.305202,0.541921}%
\pgfsetfillcolor{currentfill}%
\pgfsetlinewidth{0.000000pt}%
\definecolor{currentstroke}{rgb}{0.000000,0.000000,0.000000}%
\pgfsetstrokecolor{currentstroke}%
\pgfsetdash{}{0pt}%
\pgfpathmoveto{\pgfqpoint{2.900645in}{3.626887in}}%
\pgfpathlineto{\pgfqpoint{2.969115in}{3.626887in}}%
\pgfpathlineto{\pgfqpoint{2.964680in}{3.635757in}}%
\pgfpathlineto{\pgfqpoint{3.009032in}{3.622452in}}%
\pgfpathlineto{\pgfqpoint{2.964680in}{3.609146in}}%
\pgfpathlineto{\pgfqpoint{2.969115in}{3.618016in}}%
\pgfpathlineto{\pgfqpoint{2.900645in}{3.618016in}}%
\pgfpathlineto{\pgfqpoint{2.900645in}{3.626887in}}%
\pgfusepath{fill}%
\end{pgfscope}%
\begin{pgfscope}%
\pgfpathrectangle{\pgfqpoint{1.432000in}{0.528000in}}{\pgfqpoint{3.696000in}{3.696000in}} %
\pgfusepath{clip}%
\pgfsetbuttcap%
\pgfsetroundjoin%
\definecolor{currentfill}{rgb}{0.263663,0.237631,0.518762}%
\pgfsetfillcolor{currentfill}%
\pgfsetlinewidth{0.000000pt}%
\definecolor{currentstroke}{rgb}{0.000000,0.000000,0.000000}%
\pgfsetstrokecolor{currentstroke}%
\pgfsetdash{}{0pt}%
\pgfpathmoveto{\pgfqpoint{3.013467in}{3.622452in}}%
\pgfpathlineto{\pgfqpoint{3.011250in}{3.626293in}}%
\pgfpathlineto{\pgfqpoint{3.006815in}{3.626293in}}%
\pgfpathlineto{\pgfqpoint{3.004597in}{3.622452in}}%
\pgfpathlineto{\pgfqpoint{3.006815in}{3.618611in}}%
\pgfpathlineto{\pgfqpoint{3.011250in}{3.618611in}}%
\pgfpathlineto{\pgfqpoint{3.013467in}{3.622452in}}%
\pgfpathlineto{\pgfqpoint{3.011250in}{3.626293in}}%
\pgfusepath{fill}%
\end{pgfscope}%
\begin{pgfscope}%
\pgfpathrectangle{\pgfqpoint{1.432000in}{0.528000in}}{\pgfqpoint{3.696000in}{3.696000in}} %
\pgfusepath{clip}%
\pgfsetbuttcap%
\pgfsetroundjoin%
\definecolor{currentfill}{rgb}{0.188923,0.410910,0.556326}%
\pgfsetfillcolor{currentfill}%
\pgfsetlinewidth{0.000000pt}%
\definecolor{currentstroke}{rgb}{0.000000,0.000000,0.000000}%
\pgfsetstrokecolor{currentstroke}%
\pgfsetdash{}{0pt}%
\pgfpathmoveto{\pgfqpoint{3.009032in}{3.626887in}}%
\pgfpathlineto{\pgfqpoint{3.077503in}{3.626887in}}%
\pgfpathlineto{\pgfqpoint{3.073067in}{3.635757in}}%
\pgfpathlineto{\pgfqpoint{3.117419in}{3.622452in}}%
\pgfpathlineto{\pgfqpoint{3.073067in}{3.609146in}}%
\pgfpathlineto{\pgfqpoint{3.077503in}{3.618016in}}%
\pgfpathlineto{\pgfqpoint{3.009032in}{3.618016in}}%
\pgfpathlineto{\pgfqpoint{3.009032in}{3.626887in}}%
\pgfusepath{fill}%
\end{pgfscope}%
\begin{pgfscope}%
\pgfpathrectangle{\pgfqpoint{1.432000in}{0.528000in}}{\pgfqpoint{3.696000in}{3.696000in}} %
\pgfusepath{clip}%
\pgfsetbuttcap%
\pgfsetroundjoin%
\definecolor{currentfill}{rgb}{0.273809,0.031497,0.358853}%
\pgfsetfillcolor{currentfill}%
\pgfsetlinewidth{0.000000pt}%
\definecolor{currentstroke}{rgb}{0.000000,0.000000,0.000000}%
\pgfsetstrokecolor{currentstroke}%
\pgfsetdash{}{0pt}%
\pgfpathmoveto{\pgfqpoint{3.121855in}{3.622452in}}%
\pgfpathlineto{\pgfqpoint{3.119637in}{3.626293in}}%
\pgfpathlineto{\pgfqpoint{3.115202in}{3.626293in}}%
\pgfpathlineto{\pgfqpoint{3.112984in}{3.622452in}}%
\pgfpathlineto{\pgfqpoint{3.115202in}{3.618611in}}%
\pgfpathlineto{\pgfqpoint{3.119637in}{3.618611in}}%
\pgfpathlineto{\pgfqpoint{3.121855in}{3.622452in}}%
\pgfpathlineto{\pgfqpoint{3.119637in}{3.626293in}}%
\pgfusepath{fill}%
\end{pgfscope}%
\begin{pgfscope}%
\pgfpathrectangle{\pgfqpoint{1.432000in}{0.528000in}}{\pgfqpoint{3.696000in}{3.696000in}} %
\pgfusepath{clip}%
\pgfsetbuttcap%
\pgfsetroundjoin%
\definecolor{currentfill}{rgb}{0.199430,0.387607,0.554642}%
\pgfsetfillcolor{currentfill}%
\pgfsetlinewidth{0.000000pt}%
\definecolor{currentstroke}{rgb}{0.000000,0.000000,0.000000}%
\pgfsetstrokecolor{currentstroke}%
\pgfsetdash{}{0pt}%
\pgfpathmoveto{\pgfqpoint{3.117419in}{3.626887in}}%
\pgfpathlineto{\pgfqpoint{3.185890in}{3.626887in}}%
\pgfpathlineto{\pgfqpoint{3.181454in}{3.635757in}}%
\pgfpathlineto{\pgfqpoint{3.225806in}{3.622452in}}%
\pgfpathlineto{\pgfqpoint{3.181454in}{3.609146in}}%
\pgfpathlineto{\pgfqpoint{3.185890in}{3.618016in}}%
\pgfpathlineto{\pgfqpoint{3.117419in}{3.618016in}}%
\pgfpathlineto{\pgfqpoint{3.117419in}{3.626887in}}%
\pgfusepath{fill}%
\end{pgfscope}%
\begin{pgfscope}%
\pgfpathrectangle{\pgfqpoint{1.432000in}{0.528000in}}{\pgfqpoint{3.696000in}{3.696000in}} %
\pgfusepath{clip}%
\pgfsetbuttcap%
\pgfsetroundjoin%
\definecolor{currentfill}{rgb}{0.273809,0.031497,0.358853}%
\pgfsetfillcolor{currentfill}%
\pgfsetlinewidth{0.000000pt}%
\definecolor{currentstroke}{rgb}{0.000000,0.000000,0.000000}%
\pgfsetstrokecolor{currentstroke}%
\pgfsetdash{}{0pt}%
\pgfpathmoveto{\pgfqpoint{3.230242in}{3.622452in}}%
\pgfpathlineto{\pgfqpoint{3.228024in}{3.626293in}}%
\pgfpathlineto{\pgfqpoint{3.223589in}{3.626293in}}%
\pgfpathlineto{\pgfqpoint{3.221371in}{3.622452in}}%
\pgfpathlineto{\pgfqpoint{3.223589in}{3.618611in}}%
\pgfpathlineto{\pgfqpoint{3.228024in}{3.618611in}}%
\pgfpathlineto{\pgfqpoint{3.230242in}{3.622452in}}%
\pgfpathlineto{\pgfqpoint{3.228024in}{3.626293in}}%
\pgfusepath{fill}%
\end{pgfscope}%
\begin{pgfscope}%
\pgfpathrectangle{\pgfqpoint{1.432000in}{0.528000in}}{\pgfqpoint{3.696000in}{3.696000in}} %
\pgfusepath{clip}%
\pgfsetbuttcap%
\pgfsetroundjoin%
\definecolor{currentfill}{rgb}{0.248629,0.278775,0.534556}%
\pgfsetfillcolor{currentfill}%
\pgfsetlinewidth{0.000000pt}%
\definecolor{currentstroke}{rgb}{0.000000,0.000000,0.000000}%
\pgfsetstrokecolor{currentstroke}%
\pgfsetdash{}{0pt}%
\pgfpathmoveto{\pgfqpoint{3.225806in}{3.626887in}}%
\pgfpathlineto{\pgfqpoint{3.294277in}{3.626887in}}%
\pgfpathlineto{\pgfqpoint{3.289842in}{3.635757in}}%
\pgfpathlineto{\pgfqpoint{3.334194in}{3.622452in}}%
\pgfpathlineto{\pgfqpoint{3.289842in}{3.609146in}}%
\pgfpathlineto{\pgfqpoint{3.294277in}{3.618016in}}%
\pgfpathlineto{\pgfqpoint{3.225806in}{3.618016in}}%
\pgfpathlineto{\pgfqpoint{3.225806in}{3.626887in}}%
\pgfusepath{fill}%
\end{pgfscope}%
\begin{pgfscope}%
\pgfpathrectangle{\pgfqpoint{1.432000in}{0.528000in}}{\pgfqpoint{3.696000in}{3.696000in}} %
\pgfusepath{clip}%
\pgfsetbuttcap%
\pgfsetroundjoin%
\definecolor{currentfill}{rgb}{0.274128,0.199721,0.498911}%
\pgfsetfillcolor{currentfill}%
\pgfsetlinewidth{0.000000pt}%
\definecolor{currentstroke}{rgb}{0.000000,0.000000,0.000000}%
\pgfsetstrokecolor{currentstroke}%
\pgfsetdash{}{0pt}%
\pgfpathmoveto{\pgfqpoint{3.338629in}{3.622452in}}%
\pgfpathlineto{\pgfqpoint{3.336411in}{3.626293in}}%
\pgfpathlineto{\pgfqpoint{3.331976in}{3.626293in}}%
\pgfpathlineto{\pgfqpoint{3.329758in}{3.622452in}}%
\pgfpathlineto{\pgfqpoint{3.331976in}{3.618611in}}%
\pgfpathlineto{\pgfqpoint{3.336411in}{3.618611in}}%
\pgfpathlineto{\pgfqpoint{3.338629in}{3.622452in}}%
\pgfpathlineto{\pgfqpoint{3.336411in}{3.626293in}}%
\pgfusepath{fill}%
\end{pgfscope}%
\begin{pgfscope}%
\pgfpathrectangle{\pgfqpoint{1.432000in}{0.528000in}}{\pgfqpoint{3.696000in}{3.696000in}} %
\pgfusepath{clip}%
\pgfsetbuttcap%
\pgfsetroundjoin%
\definecolor{currentfill}{rgb}{0.276022,0.044167,0.370164}%
\pgfsetfillcolor{currentfill}%
\pgfsetlinewidth{0.000000pt}%
\definecolor{currentstroke}{rgb}{0.000000,0.000000,0.000000}%
\pgfsetstrokecolor{currentstroke}%
\pgfsetdash{}{0pt}%
\pgfpathmoveto{\pgfqpoint{3.656219in}{3.619315in}}%
\pgfpathlineto{\pgfqpoint{3.576057in}{3.699477in}}%
\pgfpathlineto{\pgfqpoint{3.572921in}{3.690069in}}%
\pgfpathlineto{\pgfqpoint{3.550968in}{3.730839in}}%
\pgfpathlineto{\pgfqpoint{3.591738in}{3.708886in}}%
\pgfpathlineto{\pgfqpoint{3.582329in}{3.705749in}}%
\pgfpathlineto{\pgfqpoint{3.662491in}{3.625588in}}%
\pgfpathlineto{\pgfqpoint{3.656219in}{3.619315in}}%
\pgfusepath{fill}%
\end{pgfscope}%
\begin{pgfscope}%
\pgfpathrectangle{\pgfqpoint{1.432000in}{0.528000in}}{\pgfqpoint{3.696000in}{3.696000in}} %
\pgfusepath{clip}%
\pgfsetbuttcap%
\pgfsetroundjoin%
\definecolor{currentfill}{rgb}{0.243113,0.292092,0.538516}%
\pgfsetfillcolor{currentfill}%
\pgfsetlinewidth{0.000000pt}%
\definecolor{currentstroke}{rgb}{0.000000,0.000000,0.000000}%
\pgfsetstrokecolor{currentstroke}%
\pgfsetdash{}{0pt}%
\pgfpathmoveto{\pgfqpoint{3.765758in}{3.618485in}}%
\pgfpathlineto{\pgfqpoint{3.584687in}{3.709020in}}%
\pgfpathlineto{\pgfqpoint{3.584687in}{3.699103in}}%
\pgfpathlineto{\pgfqpoint{3.550968in}{3.730839in}}%
\pgfpathlineto{\pgfqpoint{3.596588in}{3.722905in}}%
\pgfpathlineto{\pgfqpoint{3.588654in}{3.716954in}}%
\pgfpathlineto{\pgfqpoint{3.769725in}{3.626419in}}%
\pgfpathlineto{\pgfqpoint{3.765758in}{3.618485in}}%
\pgfusepath{fill}%
\end{pgfscope}%
\begin{pgfscope}%
\pgfpathrectangle{\pgfqpoint{1.432000in}{0.528000in}}{\pgfqpoint{3.696000in}{3.696000in}} %
\pgfusepath{clip}%
\pgfsetbuttcap%
\pgfsetroundjoin%
\definecolor{currentfill}{rgb}{0.281924,0.089666,0.412415}%
\pgfsetfillcolor{currentfill}%
\pgfsetlinewidth{0.000000pt}%
\definecolor{currentstroke}{rgb}{0.000000,0.000000,0.000000}%
\pgfsetstrokecolor{currentstroke}%
\pgfsetdash{}{0pt}%
\pgfpathmoveto{\pgfqpoint{3.874146in}{3.618485in}}%
\pgfpathlineto{\pgfqpoint{3.693074in}{3.709020in}}%
\pgfpathlineto{\pgfqpoint{3.693074in}{3.699103in}}%
\pgfpathlineto{\pgfqpoint{3.659355in}{3.730839in}}%
\pgfpathlineto{\pgfqpoint{3.704975in}{3.722905in}}%
\pgfpathlineto{\pgfqpoint{3.697041in}{3.716954in}}%
\pgfpathlineto{\pgfqpoint{3.878113in}{3.626419in}}%
\pgfpathlineto{\pgfqpoint{3.874146in}{3.618485in}}%
\pgfusepath{fill}%
\end{pgfscope}%
\begin{pgfscope}%
\pgfpathrectangle{\pgfqpoint{1.432000in}{0.528000in}}{\pgfqpoint{3.696000in}{3.696000in}} %
\pgfusepath{clip}%
\pgfsetbuttcap%
\pgfsetroundjoin%
\definecolor{currentfill}{rgb}{0.233603,0.313828,0.543914}%
\pgfsetfillcolor{currentfill}%
\pgfsetlinewidth{0.000000pt}%
\definecolor{currentstroke}{rgb}{0.000000,0.000000,0.000000}%
\pgfsetstrokecolor{currentstroke}%
\pgfsetdash{}{0pt}%
\pgfpathmoveto{\pgfqpoint{3.981380in}{3.619315in}}%
\pgfpathlineto{\pgfqpoint{3.792831in}{3.807864in}}%
\pgfpathlineto{\pgfqpoint{3.789695in}{3.798456in}}%
\pgfpathlineto{\pgfqpoint{3.767742in}{3.839226in}}%
\pgfpathlineto{\pgfqpoint{3.808512in}{3.817273in}}%
\pgfpathlineto{\pgfqpoint{3.799104in}{3.814137in}}%
\pgfpathlineto{\pgfqpoint{3.987652in}{3.625588in}}%
\pgfpathlineto{\pgfqpoint{3.981380in}{3.619315in}}%
\pgfusepath{fill}%
\end{pgfscope}%
\begin{pgfscope}%
\pgfpathrectangle{\pgfqpoint{1.432000in}{0.528000in}}{\pgfqpoint{3.696000in}{3.696000in}} %
\pgfusepath{clip}%
\pgfsetbuttcap%
\pgfsetroundjoin%
\definecolor{currentfill}{rgb}{0.280894,0.078907,0.402329}%
\pgfsetfillcolor{currentfill}%
\pgfsetlinewidth{0.000000pt}%
\definecolor{currentstroke}{rgb}{0.000000,0.000000,0.000000}%
\pgfsetstrokecolor{currentstroke}%
\pgfsetdash{}{0pt}%
\pgfpathmoveto{\pgfqpoint{4.090443in}{3.618761in}}%
\pgfpathlineto{\pgfqpoint{3.798495in}{3.813394in}}%
\pgfpathlineto{\pgfqpoint{3.797264in}{3.803553in}}%
\pgfpathlineto{\pgfqpoint{3.767742in}{3.839226in}}%
\pgfpathlineto{\pgfqpoint{3.812026in}{3.825695in}}%
\pgfpathlineto{\pgfqpoint{3.803415in}{3.820774in}}%
\pgfpathlineto{\pgfqpoint{4.095363in}{3.626142in}}%
\pgfpathlineto{\pgfqpoint{4.090443in}{3.618761in}}%
\pgfusepath{fill}%
\end{pgfscope}%
\begin{pgfscope}%
\pgfpathrectangle{\pgfqpoint{1.432000in}{0.528000in}}{\pgfqpoint{3.696000in}{3.696000in}} %
\pgfusepath{clip}%
\pgfsetbuttcap%
\pgfsetroundjoin%
\definecolor{currentfill}{rgb}{0.282656,0.100196,0.422160}%
\pgfsetfillcolor{currentfill}%
\pgfsetlinewidth{0.000000pt}%
\definecolor{currentstroke}{rgb}{0.000000,0.000000,0.000000}%
\pgfsetstrokecolor{currentstroke}%
\pgfsetdash{}{0pt}%
\pgfpathmoveto{\pgfqpoint{4.089767in}{3.619315in}}%
\pgfpathlineto{\pgfqpoint{3.901218in}{3.807864in}}%
\pgfpathlineto{\pgfqpoint{3.898082in}{3.798456in}}%
\pgfpathlineto{\pgfqpoint{3.876129in}{3.839226in}}%
\pgfpathlineto{\pgfqpoint{3.916899in}{3.817273in}}%
\pgfpathlineto{\pgfqpoint{3.907491in}{3.814137in}}%
\pgfpathlineto{\pgfqpoint{4.096039in}{3.625588in}}%
\pgfpathlineto{\pgfqpoint{4.089767in}{3.619315in}}%
\pgfusepath{fill}%
\end{pgfscope}%
\begin{pgfscope}%
\pgfpathrectangle{\pgfqpoint{1.432000in}{0.528000in}}{\pgfqpoint{3.696000in}{3.696000in}} %
\pgfusepath{clip}%
\pgfsetbuttcap%
\pgfsetroundjoin%
\definecolor{currentfill}{rgb}{0.146180,0.515413,0.556823}%
\pgfsetfillcolor{currentfill}%
\pgfsetlinewidth{0.000000pt}%
\definecolor{currentstroke}{rgb}{0.000000,0.000000,0.000000}%
\pgfsetstrokecolor{currentstroke}%
\pgfsetdash{}{0pt}%
\pgfpathmoveto{\pgfqpoint{4.198830in}{3.618761in}}%
\pgfpathlineto{\pgfqpoint{3.906882in}{3.813394in}}%
\pgfpathlineto{\pgfqpoint{3.905652in}{3.803553in}}%
\pgfpathlineto{\pgfqpoint{3.876129in}{3.839226in}}%
\pgfpathlineto{\pgfqpoint{3.920413in}{3.825695in}}%
\pgfpathlineto{\pgfqpoint{3.911802in}{3.820774in}}%
\pgfpathlineto{\pgfqpoint{4.203751in}{3.626142in}}%
\pgfpathlineto{\pgfqpoint{4.198830in}{3.618761in}}%
\pgfusepath{fill}%
\end{pgfscope}%
\begin{pgfscope}%
\pgfpathrectangle{\pgfqpoint{1.432000in}{0.528000in}}{\pgfqpoint{3.696000in}{3.696000in}} %
\pgfusepath{clip}%
\pgfsetbuttcap%
\pgfsetroundjoin%
\definecolor{currentfill}{rgb}{0.156270,0.489624,0.557936}%
\pgfsetfillcolor{currentfill}%
\pgfsetlinewidth{0.000000pt}%
\definecolor{currentstroke}{rgb}{0.000000,0.000000,0.000000}%
\pgfsetstrokecolor{currentstroke}%
\pgfsetdash{}{0pt}%
\pgfpathmoveto{\pgfqpoint{4.307217in}{3.618761in}}%
\pgfpathlineto{\pgfqpoint{4.015269in}{3.813394in}}%
\pgfpathlineto{\pgfqpoint{4.014039in}{3.803553in}}%
\pgfpathlineto{\pgfqpoint{3.984516in}{3.839226in}}%
\pgfpathlineto{\pgfqpoint{4.028800in}{3.825695in}}%
\pgfpathlineto{\pgfqpoint{4.020189in}{3.820774in}}%
\pgfpathlineto{\pgfqpoint{4.312138in}{3.626142in}}%
\pgfpathlineto{\pgfqpoint{4.307217in}{3.618761in}}%
\pgfusepath{fill}%
\end{pgfscope}%
\begin{pgfscope}%
\pgfpathrectangle{\pgfqpoint{1.432000in}{0.528000in}}{\pgfqpoint{3.696000in}{3.696000in}} %
\pgfusepath{clip}%
\pgfsetbuttcap%
\pgfsetroundjoin%
\definecolor{currentfill}{rgb}{0.277941,0.056324,0.381191}%
\pgfsetfillcolor{currentfill}%
\pgfsetlinewidth{0.000000pt}%
\definecolor{currentstroke}{rgb}{0.000000,0.000000,0.000000}%
\pgfsetstrokecolor{currentstroke}%
\pgfsetdash{}{0pt}%
\pgfpathmoveto{\pgfqpoint{4.306541in}{3.619315in}}%
\pgfpathlineto{\pgfqpoint{4.117993in}{3.807864in}}%
\pgfpathlineto{\pgfqpoint{4.114856in}{3.798456in}}%
\pgfpathlineto{\pgfqpoint{4.092903in}{3.839226in}}%
\pgfpathlineto{\pgfqpoint{4.133673in}{3.817273in}}%
\pgfpathlineto{\pgfqpoint{4.124265in}{3.814137in}}%
\pgfpathlineto{\pgfqpoint{4.312814in}{3.625588in}}%
\pgfpathlineto{\pgfqpoint{4.306541in}{3.619315in}}%
\pgfusepath{fill}%
\end{pgfscope}%
\begin{pgfscope}%
\pgfpathrectangle{\pgfqpoint{1.432000in}{0.528000in}}{\pgfqpoint{3.696000in}{3.696000in}} %
\pgfusepath{clip}%
\pgfsetbuttcap%
\pgfsetroundjoin%
\definecolor{currentfill}{rgb}{0.237441,0.305202,0.541921}%
\pgfsetfillcolor{currentfill}%
\pgfsetlinewidth{0.000000pt}%
\definecolor{currentstroke}{rgb}{0.000000,0.000000,0.000000}%
\pgfsetstrokecolor{currentstroke}%
\pgfsetdash{}{0pt}%
\pgfpathmoveto{\pgfqpoint{4.415604in}{3.618761in}}%
\pgfpathlineto{\pgfqpoint{4.123656in}{3.813394in}}%
\pgfpathlineto{\pgfqpoint{4.122426in}{3.803553in}}%
\pgfpathlineto{\pgfqpoint{4.092903in}{3.839226in}}%
\pgfpathlineto{\pgfqpoint{4.137187in}{3.825695in}}%
\pgfpathlineto{\pgfqpoint{4.128576in}{3.820774in}}%
\pgfpathlineto{\pgfqpoint{4.420525in}{3.626142in}}%
\pgfpathlineto{\pgfqpoint{4.415604in}{3.618761in}}%
\pgfusepath{fill}%
\end{pgfscope}%
\begin{pgfscope}%
\pgfpathrectangle{\pgfqpoint{1.432000in}{0.528000in}}{\pgfqpoint{3.696000in}{3.696000in}} %
\pgfusepath{clip}%
\pgfsetbuttcap%
\pgfsetroundjoin%
\definecolor{currentfill}{rgb}{0.201239,0.383670,0.554294}%
\pgfsetfillcolor{currentfill}%
\pgfsetlinewidth{0.000000pt}%
\definecolor{currentstroke}{rgb}{0.000000,0.000000,0.000000}%
\pgfsetstrokecolor{currentstroke}%
\pgfsetdash{}{0pt}%
\pgfpathmoveto{\pgfqpoint{4.414928in}{3.619315in}}%
\pgfpathlineto{\pgfqpoint{4.226380in}{3.807864in}}%
\pgfpathlineto{\pgfqpoint{4.223243in}{3.798456in}}%
\pgfpathlineto{\pgfqpoint{4.201290in}{3.839226in}}%
\pgfpathlineto{\pgfqpoint{4.242060in}{3.817273in}}%
\pgfpathlineto{\pgfqpoint{4.232652in}{3.814137in}}%
\pgfpathlineto{\pgfqpoint{4.421201in}{3.625588in}}%
\pgfpathlineto{\pgfqpoint{4.414928in}{3.619315in}}%
\pgfusepath{fill}%
\end{pgfscope}%
\begin{pgfscope}%
\pgfpathrectangle{\pgfqpoint{1.432000in}{0.528000in}}{\pgfqpoint{3.696000in}{3.696000in}} %
\pgfusepath{clip}%
\pgfsetbuttcap%
\pgfsetroundjoin%
\definecolor{currentfill}{rgb}{0.144759,0.519093,0.556572}%
\pgfsetfillcolor{currentfill}%
\pgfsetlinewidth{0.000000pt}%
\definecolor{currentstroke}{rgb}{0.000000,0.000000,0.000000}%
\pgfsetstrokecolor{currentstroke}%
\pgfsetdash{}{0pt}%
\pgfpathmoveto{\pgfqpoint{4.523315in}{3.619315in}}%
\pgfpathlineto{\pgfqpoint{4.334767in}{3.807864in}}%
\pgfpathlineto{\pgfqpoint{4.331631in}{3.798456in}}%
\pgfpathlineto{\pgfqpoint{4.309677in}{3.839226in}}%
\pgfpathlineto{\pgfqpoint{4.350447in}{3.817273in}}%
\pgfpathlineto{\pgfqpoint{4.341039in}{3.814137in}}%
\pgfpathlineto{\pgfqpoint{4.529588in}{3.625588in}}%
\pgfpathlineto{\pgfqpoint{4.523315in}{3.619315in}}%
\pgfusepath{fill}%
\end{pgfscope}%
\begin{pgfscope}%
\pgfpathrectangle{\pgfqpoint{1.432000in}{0.528000in}}{\pgfqpoint{3.696000in}{3.696000in}} %
\pgfusepath{clip}%
\pgfsetbuttcap%
\pgfsetroundjoin%
\definecolor{currentfill}{rgb}{0.283197,0.115680,0.436115}%
\pgfsetfillcolor{currentfill}%
\pgfsetlinewidth{0.000000pt}%
\definecolor{currentstroke}{rgb}{0.000000,0.000000,0.000000}%
\pgfsetstrokecolor{currentstroke}%
\pgfsetdash{}{0pt}%
\pgfpathmoveto{\pgfqpoint{4.522485in}{3.620468in}}%
\pgfpathlineto{\pgfqpoint{4.431949in}{3.801540in}}%
\pgfpathlineto{\pgfqpoint{4.425998in}{3.793606in}}%
\pgfpathlineto{\pgfqpoint{4.418065in}{3.839226in}}%
\pgfpathlineto{\pgfqpoint{4.449800in}{3.805507in}}%
\pgfpathlineto{\pgfqpoint{4.439883in}{3.805507in}}%
\pgfpathlineto{\pgfqpoint{4.530419in}{3.624435in}}%
\pgfpathlineto{\pgfqpoint{4.522485in}{3.620468in}}%
\pgfusepath{fill}%
\end{pgfscope}%
\begin{pgfscope}%
\pgfpathrectangle{\pgfqpoint{1.432000in}{0.528000in}}{\pgfqpoint{3.696000in}{3.696000in}} %
\pgfusepath{clip}%
\pgfsetbuttcap%
\pgfsetroundjoin%
\definecolor{currentfill}{rgb}{0.266580,0.228262,0.514349}%
\pgfsetfillcolor{currentfill}%
\pgfsetlinewidth{0.000000pt}%
\definecolor{currentstroke}{rgb}{0.000000,0.000000,0.000000}%
\pgfsetstrokecolor{currentstroke}%
\pgfsetdash{}{0pt}%
\pgfpathmoveto{\pgfqpoint{4.631703in}{3.619315in}}%
\pgfpathlineto{\pgfqpoint{4.443154in}{3.807864in}}%
\pgfpathlineto{\pgfqpoint{4.440018in}{3.798456in}}%
\pgfpathlineto{\pgfqpoint{4.418065in}{3.839226in}}%
\pgfpathlineto{\pgfqpoint{4.458835in}{3.817273in}}%
\pgfpathlineto{\pgfqpoint{4.449426in}{3.814137in}}%
\pgfpathlineto{\pgfqpoint{4.637975in}{3.625588in}}%
\pgfpathlineto{\pgfqpoint{4.631703in}{3.619315in}}%
\pgfusepath{fill}%
\end{pgfscope}%
\begin{pgfscope}%
\pgfpathrectangle{\pgfqpoint{1.432000in}{0.528000in}}{\pgfqpoint{3.696000in}{3.696000in}} %
\pgfusepath{clip}%
\pgfsetbuttcap%
\pgfsetroundjoin%
\definecolor{currentfill}{rgb}{0.185556,0.418570,0.556753}%
\pgfsetfillcolor{currentfill}%
\pgfsetlinewidth{0.000000pt}%
\definecolor{currentstroke}{rgb}{0.000000,0.000000,0.000000}%
\pgfsetstrokecolor{currentstroke}%
\pgfsetdash{}{0pt}%
\pgfpathmoveto{\pgfqpoint{4.630872in}{3.620468in}}%
\pgfpathlineto{\pgfqpoint{4.540336in}{3.801540in}}%
\pgfpathlineto{\pgfqpoint{4.534386in}{3.793606in}}%
\pgfpathlineto{\pgfqpoint{4.526452in}{3.839226in}}%
\pgfpathlineto{\pgfqpoint{4.558187in}{3.805507in}}%
\pgfpathlineto{\pgfqpoint{4.548270in}{3.805507in}}%
\pgfpathlineto{\pgfqpoint{4.638806in}{3.624435in}}%
\pgfpathlineto{\pgfqpoint{4.630872in}{3.620468in}}%
\pgfusepath{fill}%
\end{pgfscope}%
\begin{pgfscope}%
\pgfpathrectangle{\pgfqpoint{1.432000in}{0.528000in}}{\pgfqpoint{3.696000in}{3.696000in}} %
\pgfusepath{clip}%
\pgfsetbuttcap%
\pgfsetroundjoin%
\definecolor{currentfill}{rgb}{0.153364,0.497000,0.557724}%
\pgfsetfillcolor{currentfill}%
\pgfsetlinewidth{0.000000pt}%
\definecolor{currentstroke}{rgb}{0.000000,0.000000,0.000000}%
\pgfsetstrokecolor{currentstroke}%
\pgfsetdash{}{0pt}%
\pgfpathmoveto{\pgfqpoint{4.739259in}{3.620468in}}%
\pgfpathlineto{\pgfqpoint{4.648723in}{3.801540in}}%
\pgfpathlineto{\pgfqpoint{4.642773in}{3.793606in}}%
\pgfpathlineto{\pgfqpoint{4.634839in}{3.839226in}}%
\pgfpathlineto{\pgfqpoint{4.666574in}{3.805507in}}%
\pgfpathlineto{\pgfqpoint{4.656657in}{3.805507in}}%
\pgfpathlineto{\pgfqpoint{4.747193in}{3.624435in}}%
\pgfpathlineto{\pgfqpoint{4.739259in}{3.620468in}}%
\pgfusepath{fill}%
\end{pgfscope}%
\begin{pgfscope}%
\pgfpathrectangle{\pgfqpoint{1.432000in}{0.528000in}}{\pgfqpoint{3.696000in}{3.696000in}} %
\pgfusepath{clip}%
\pgfsetbuttcap%
\pgfsetroundjoin%
\definecolor{currentfill}{rgb}{0.231674,0.318106,0.544834}%
\pgfsetfillcolor{currentfill}%
\pgfsetlinewidth{0.000000pt}%
\definecolor{currentstroke}{rgb}{0.000000,0.000000,0.000000}%
\pgfsetstrokecolor{currentstroke}%
\pgfsetdash{}{0pt}%
\pgfpathmoveto{\pgfqpoint{4.847646in}{3.620468in}}%
\pgfpathlineto{\pgfqpoint{4.757110in}{3.801540in}}%
\pgfpathlineto{\pgfqpoint{4.751160in}{3.793606in}}%
\pgfpathlineto{\pgfqpoint{4.743226in}{3.839226in}}%
\pgfpathlineto{\pgfqpoint{4.774962in}{3.805507in}}%
\pgfpathlineto{\pgfqpoint{4.765044in}{3.805507in}}%
\pgfpathlineto{\pgfqpoint{4.855580in}{3.624435in}}%
\pgfpathlineto{\pgfqpoint{4.847646in}{3.620468in}}%
\pgfusepath{fill}%
\end{pgfscope}%
\begin{pgfscope}%
\pgfpathrectangle{\pgfqpoint{1.432000in}{0.528000in}}{\pgfqpoint{3.696000in}{3.696000in}} %
\pgfusepath{clip}%
\pgfsetbuttcap%
\pgfsetroundjoin%
\definecolor{currentfill}{rgb}{0.281412,0.155834,0.469201}%
\pgfsetfillcolor{currentfill}%
\pgfsetlinewidth{0.000000pt}%
\definecolor{currentstroke}{rgb}{0.000000,0.000000,0.000000}%
\pgfsetstrokecolor{currentstroke}%
\pgfsetdash{}{0pt}%
\pgfpathmoveto{\pgfqpoint{4.847178in}{3.622452in}}%
\pgfpathlineto{\pgfqpoint{4.847178in}{3.799309in}}%
\pgfpathlineto{\pgfqpoint{4.838307in}{3.794874in}}%
\pgfpathlineto{\pgfqpoint{4.851613in}{3.839226in}}%
\pgfpathlineto{\pgfqpoint{4.864919in}{3.794874in}}%
\pgfpathlineto{\pgfqpoint{4.856048in}{3.799309in}}%
\pgfpathlineto{\pgfqpoint{4.856048in}{3.622452in}}%
\pgfpathlineto{\pgfqpoint{4.847178in}{3.622452in}}%
\pgfusepath{fill}%
\end{pgfscope}%
\begin{pgfscope}%
\pgfpathrectangle{\pgfqpoint{1.432000in}{0.528000in}}{\pgfqpoint{3.696000in}{3.696000in}} %
\pgfusepath{clip}%
\pgfsetbuttcap%
\pgfsetroundjoin%
\definecolor{currentfill}{rgb}{0.273809,0.031497,0.358853}%
\pgfsetfillcolor{currentfill}%
\pgfsetlinewidth{0.000000pt}%
\definecolor{currentstroke}{rgb}{0.000000,0.000000,0.000000}%
\pgfsetstrokecolor{currentstroke}%
\pgfsetdash{}{0pt}%
\pgfpathmoveto{\pgfqpoint{4.956033in}{3.620468in}}%
\pgfpathlineto{\pgfqpoint{4.865497in}{3.801540in}}%
\pgfpathlineto{\pgfqpoint{4.859547in}{3.793606in}}%
\pgfpathlineto{\pgfqpoint{4.851613in}{3.839226in}}%
\pgfpathlineto{\pgfqpoint{4.883349in}{3.805507in}}%
\pgfpathlineto{\pgfqpoint{4.873431in}{3.805507in}}%
\pgfpathlineto{\pgfqpoint{4.963967in}{3.624435in}}%
\pgfpathlineto{\pgfqpoint{4.956033in}{3.620468in}}%
\pgfusepath{fill}%
\end{pgfscope}%
\begin{pgfscope}%
\pgfpathrectangle{\pgfqpoint{1.432000in}{0.528000in}}{\pgfqpoint{3.696000in}{3.696000in}} %
\pgfusepath{clip}%
\pgfsetbuttcap%
\pgfsetroundjoin%
\definecolor{currentfill}{rgb}{0.159194,0.482237,0.558073}%
\pgfsetfillcolor{currentfill}%
\pgfsetlinewidth{0.000000pt}%
\definecolor{currentstroke}{rgb}{0.000000,0.000000,0.000000}%
\pgfsetstrokecolor{currentstroke}%
\pgfsetdash{}{0pt}%
\pgfpathmoveto{\pgfqpoint{4.955565in}{3.622452in}}%
\pgfpathlineto{\pgfqpoint{4.955565in}{3.799309in}}%
\pgfpathlineto{\pgfqpoint{4.946694in}{3.794874in}}%
\pgfpathlineto{\pgfqpoint{4.960000in}{3.839226in}}%
\pgfpathlineto{\pgfqpoint{4.973306in}{3.794874in}}%
\pgfpathlineto{\pgfqpoint{4.964435in}{3.799309in}}%
\pgfpathlineto{\pgfqpoint{4.964435in}{3.622452in}}%
\pgfpathlineto{\pgfqpoint{4.955565in}{3.622452in}}%
\pgfusepath{fill}%
\end{pgfscope}%
\begin{pgfscope}%
\pgfpathrectangle{\pgfqpoint{1.432000in}{0.528000in}}{\pgfqpoint{3.696000in}{3.696000in}} %
\pgfusepath{clip}%
\pgfsetbuttcap%
\pgfsetroundjoin%
\definecolor{currentfill}{rgb}{0.281446,0.084320,0.407414}%
\pgfsetfillcolor{currentfill}%
\pgfsetlinewidth{0.000000pt}%
\definecolor{currentstroke}{rgb}{0.000000,0.000000,0.000000}%
\pgfsetstrokecolor{currentstroke}%
\pgfsetdash{}{0pt}%
\pgfpathmoveto{\pgfqpoint{1.604435in}{3.730839in}}%
\pgfpathlineto{\pgfqpoint{1.604435in}{3.553981in}}%
\pgfpathlineto{\pgfqpoint{1.613306in}{3.558417in}}%
\pgfpathlineto{\pgfqpoint{1.600000in}{3.514065in}}%
\pgfpathlineto{\pgfqpoint{1.586694in}{3.558417in}}%
\pgfpathlineto{\pgfqpoint{1.595565in}{3.553981in}}%
\pgfpathlineto{\pgfqpoint{1.595565in}{3.730839in}}%
\pgfpathlineto{\pgfqpoint{1.604435in}{3.730839in}}%
\pgfusepath{fill}%
\end{pgfscope}%
\begin{pgfscope}%
\pgfpathrectangle{\pgfqpoint{1.432000in}{0.528000in}}{\pgfqpoint{3.696000in}{3.696000in}} %
\pgfusepath{clip}%
\pgfsetbuttcap%
\pgfsetroundjoin%
\definecolor{currentfill}{rgb}{0.229739,0.322361,0.545706}%
\pgfsetfillcolor{currentfill}%
\pgfsetlinewidth{0.000000pt}%
\definecolor{currentstroke}{rgb}{0.000000,0.000000,0.000000}%
\pgfsetstrokecolor{currentstroke}%
\pgfsetdash{}{0pt}%
\pgfpathmoveto{\pgfqpoint{1.604435in}{3.730839in}}%
\pgfpathlineto{\pgfqpoint{1.604435in}{3.662368in}}%
\pgfpathlineto{\pgfqpoint{1.613306in}{3.666804in}}%
\pgfpathlineto{\pgfqpoint{1.600000in}{3.622452in}}%
\pgfpathlineto{\pgfqpoint{1.586694in}{3.666804in}}%
\pgfpathlineto{\pgfqpoint{1.595565in}{3.662368in}}%
\pgfpathlineto{\pgfqpoint{1.595565in}{3.730839in}}%
\pgfpathlineto{\pgfqpoint{1.604435in}{3.730839in}}%
\pgfusepath{fill}%
\end{pgfscope}%
\begin{pgfscope}%
\pgfpathrectangle{\pgfqpoint{1.432000in}{0.528000in}}{\pgfqpoint{3.696000in}{3.696000in}} %
\pgfusepath{clip}%
\pgfsetbuttcap%
\pgfsetroundjoin%
\definecolor{currentfill}{rgb}{0.282327,0.094955,0.417331}%
\pgfsetfillcolor{currentfill}%
\pgfsetlinewidth{0.000000pt}%
\definecolor{currentstroke}{rgb}{0.000000,0.000000,0.000000}%
\pgfsetstrokecolor{currentstroke}%
\pgfsetdash{}{0pt}%
\pgfpathmoveto{\pgfqpoint{1.712822in}{3.730839in}}%
\pgfpathlineto{\pgfqpoint{1.712822in}{3.553981in}}%
\pgfpathlineto{\pgfqpoint{1.721693in}{3.558417in}}%
\pgfpathlineto{\pgfqpoint{1.708387in}{3.514065in}}%
\pgfpathlineto{\pgfqpoint{1.695081in}{3.558417in}}%
\pgfpathlineto{\pgfqpoint{1.703952in}{3.553981in}}%
\pgfpathlineto{\pgfqpoint{1.703952in}{3.730839in}}%
\pgfpathlineto{\pgfqpoint{1.712822in}{3.730839in}}%
\pgfusepath{fill}%
\end{pgfscope}%
\begin{pgfscope}%
\pgfpathrectangle{\pgfqpoint{1.432000in}{0.528000in}}{\pgfqpoint{3.696000in}{3.696000in}} %
\pgfusepath{clip}%
\pgfsetbuttcap%
\pgfsetroundjoin%
\definecolor{currentfill}{rgb}{0.274952,0.037752,0.364543}%
\pgfsetfillcolor{currentfill}%
\pgfsetlinewidth{0.000000pt}%
\definecolor{currentstroke}{rgb}{0.000000,0.000000,0.000000}%
\pgfsetstrokecolor{currentstroke}%
\pgfsetdash{}{0pt}%
\pgfpathmoveto{\pgfqpoint{1.712822in}{3.730839in}}%
\pgfpathlineto{\pgfqpoint{1.712822in}{3.662368in}}%
\pgfpathlineto{\pgfqpoint{1.721693in}{3.666804in}}%
\pgfpathlineto{\pgfqpoint{1.708387in}{3.622452in}}%
\pgfpathlineto{\pgfqpoint{1.695081in}{3.666804in}}%
\pgfpathlineto{\pgfqpoint{1.703952in}{3.662368in}}%
\pgfpathlineto{\pgfqpoint{1.703952in}{3.730839in}}%
\pgfpathlineto{\pgfqpoint{1.712822in}{3.730839in}}%
\pgfusepath{fill}%
\end{pgfscope}%
\begin{pgfscope}%
\pgfpathrectangle{\pgfqpoint{1.432000in}{0.528000in}}{\pgfqpoint{3.696000in}{3.696000in}} %
\pgfusepath{clip}%
\pgfsetbuttcap%
\pgfsetroundjoin%
\definecolor{currentfill}{rgb}{0.280894,0.078907,0.402329}%
\pgfsetfillcolor{currentfill}%
\pgfsetlinewidth{0.000000pt}%
\definecolor{currentstroke}{rgb}{0.000000,0.000000,0.000000}%
\pgfsetstrokecolor{currentstroke}%
\pgfsetdash{}{0pt}%
\pgfpathmoveto{\pgfqpoint{1.821209in}{3.730839in}}%
\pgfpathlineto{\pgfqpoint{1.821209in}{3.553981in}}%
\pgfpathlineto{\pgfqpoint{1.830080in}{3.558417in}}%
\pgfpathlineto{\pgfqpoint{1.816774in}{3.514065in}}%
\pgfpathlineto{\pgfqpoint{1.803469in}{3.558417in}}%
\pgfpathlineto{\pgfqpoint{1.812339in}{3.553981in}}%
\pgfpathlineto{\pgfqpoint{1.812339in}{3.730839in}}%
\pgfpathlineto{\pgfqpoint{1.821209in}{3.730839in}}%
\pgfusepath{fill}%
\end{pgfscope}%
\begin{pgfscope}%
\pgfpathrectangle{\pgfqpoint{1.432000in}{0.528000in}}{\pgfqpoint{3.696000in}{3.696000in}} %
\pgfusepath{clip}%
\pgfsetbuttcap%
\pgfsetroundjoin%
\definecolor{currentfill}{rgb}{0.279574,0.170599,0.479997}%
\pgfsetfillcolor{currentfill}%
\pgfsetlinewidth{0.000000pt}%
\definecolor{currentstroke}{rgb}{0.000000,0.000000,0.000000}%
\pgfsetstrokecolor{currentstroke}%
\pgfsetdash{}{0pt}%
\pgfpathmoveto{\pgfqpoint{1.929596in}{3.730839in}}%
\pgfpathlineto{\pgfqpoint{1.929596in}{3.553981in}}%
\pgfpathlineto{\pgfqpoint{1.938467in}{3.558417in}}%
\pgfpathlineto{\pgfqpoint{1.925161in}{3.514065in}}%
\pgfpathlineto{\pgfqpoint{1.911856in}{3.558417in}}%
\pgfpathlineto{\pgfqpoint{1.920726in}{3.553981in}}%
\pgfpathlineto{\pgfqpoint{1.920726in}{3.730839in}}%
\pgfpathlineto{\pgfqpoint{1.929596in}{3.730839in}}%
\pgfusepath{fill}%
\end{pgfscope}%
\begin{pgfscope}%
\pgfpathrectangle{\pgfqpoint{1.432000in}{0.528000in}}{\pgfqpoint{3.696000in}{3.696000in}} %
\pgfusepath{clip}%
\pgfsetbuttcap%
\pgfsetroundjoin%
\definecolor{currentfill}{rgb}{0.276022,0.044167,0.370164}%
\pgfsetfillcolor{currentfill}%
\pgfsetlinewidth{0.000000pt}%
\definecolor{currentstroke}{rgb}{0.000000,0.000000,0.000000}%
\pgfsetstrokecolor{currentstroke}%
\pgfsetdash{}{0pt}%
\pgfpathmoveto{\pgfqpoint{2.037984in}{3.730839in}}%
\pgfpathlineto{\pgfqpoint{2.037984in}{3.553981in}}%
\pgfpathlineto{\pgfqpoint{2.046854in}{3.558417in}}%
\pgfpathlineto{\pgfqpoint{2.033548in}{3.514065in}}%
\pgfpathlineto{\pgfqpoint{2.020243in}{3.558417in}}%
\pgfpathlineto{\pgfqpoint{2.029113in}{3.553981in}}%
\pgfpathlineto{\pgfqpoint{2.029113in}{3.730839in}}%
\pgfpathlineto{\pgfqpoint{2.037984in}{3.730839in}}%
\pgfusepath{fill}%
\end{pgfscope}%
\begin{pgfscope}%
\pgfpathrectangle{\pgfqpoint{1.432000in}{0.528000in}}{\pgfqpoint{3.696000in}{3.696000in}} %
\pgfusepath{clip}%
\pgfsetbuttcap%
\pgfsetroundjoin%
\definecolor{currentfill}{rgb}{0.282327,0.094955,0.417331}%
\pgfsetfillcolor{currentfill}%
\pgfsetlinewidth{0.000000pt}%
\definecolor{currentstroke}{rgb}{0.000000,0.000000,0.000000}%
\pgfsetstrokecolor{currentstroke}%
\pgfsetdash{}{0pt}%
\pgfpathmoveto{\pgfqpoint{2.037984in}{3.730839in}}%
\pgfpathlineto{\pgfqpoint{2.037984in}{3.662368in}}%
\pgfpathlineto{\pgfqpoint{2.046854in}{3.666804in}}%
\pgfpathlineto{\pgfqpoint{2.033548in}{3.622452in}}%
\pgfpathlineto{\pgfqpoint{2.020243in}{3.666804in}}%
\pgfpathlineto{\pgfqpoint{2.029113in}{3.662368in}}%
\pgfpathlineto{\pgfqpoint{2.029113in}{3.730839in}}%
\pgfpathlineto{\pgfqpoint{2.037984in}{3.730839in}}%
\pgfusepath{fill}%
\end{pgfscope}%
\begin{pgfscope}%
\pgfpathrectangle{\pgfqpoint{1.432000in}{0.528000in}}{\pgfqpoint{3.696000in}{3.696000in}} %
\pgfusepath{clip}%
\pgfsetbuttcap%
\pgfsetroundjoin%
\definecolor{currentfill}{rgb}{0.265145,0.232956,0.516599}%
\pgfsetfillcolor{currentfill}%
\pgfsetlinewidth{0.000000pt}%
\definecolor{currentstroke}{rgb}{0.000000,0.000000,0.000000}%
\pgfsetstrokecolor{currentstroke}%
\pgfsetdash{}{0pt}%
\pgfpathmoveto{\pgfqpoint{2.146371in}{3.730839in}}%
\pgfpathlineto{\pgfqpoint{2.146371in}{3.662368in}}%
\pgfpathlineto{\pgfqpoint{2.155241in}{3.666804in}}%
\pgfpathlineto{\pgfqpoint{2.141935in}{3.622452in}}%
\pgfpathlineto{\pgfqpoint{2.128630in}{3.666804in}}%
\pgfpathlineto{\pgfqpoint{2.137500in}{3.662368in}}%
\pgfpathlineto{\pgfqpoint{2.137500in}{3.730839in}}%
\pgfpathlineto{\pgfqpoint{2.146371in}{3.730839in}}%
\pgfusepath{fill}%
\end{pgfscope}%
\begin{pgfscope}%
\pgfpathrectangle{\pgfqpoint{1.432000in}{0.528000in}}{\pgfqpoint{3.696000in}{3.696000in}} %
\pgfusepath{clip}%
\pgfsetbuttcap%
\pgfsetroundjoin%
\definecolor{currentfill}{rgb}{0.208623,0.367752,0.552675}%
\pgfsetfillcolor{currentfill}%
\pgfsetlinewidth{0.000000pt}%
\definecolor{currentstroke}{rgb}{0.000000,0.000000,0.000000}%
\pgfsetstrokecolor{currentstroke}%
\pgfsetdash{}{0pt}%
\pgfpathmoveto{\pgfqpoint{2.254758in}{3.730839in}}%
\pgfpathlineto{\pgfqpoint{2.254758in}{3.662368in}}%
\pgfpathlineto{\pgfqpoint{2.263628in}{3.666804in}}%
\pgfpathlineto{\pgfqpoint{2.250323in}{3.622452in}}%
\pgfpathlineto{\pgfqpoint{2.237017in}{3.666804in}}%
\pgfpathlineto{\pgfqpoint{2.245887in}{3.662368in}}%
\pgfpathlineto{\pgfqpoint{2.245887in}{3.730839in}}%
\pgfpathlineto{\pgfqpoint{2.254758in}{3.730839in}}%
\pgfusepath{fill}%
\end{pgfscope}%
\begin{pgfscope}%
\pgfpathrectangle{\pgfqpoint{1.432000in}{0.528000in}}{\pgfqpoint{3.696000in}{3.696000in}} %
\pgfusepath{clip}%
\pgfsetbuttcap%
\pgfsetroundjoin%
\definecolor{currentfill}{rgb}{0.151918,0.500685,0.557587}%
\pgfsetfillcolor{currentfill}%
\pgfsetlinewidth{0.000000pt}%
\definecolor{currentstroke}{rgb}{0.000000,0.000000,0.000000}%
\pgfsetstrokecolor{currentstroke}%
\pgfsetdash{}{0pt}%
\pgfpathmoveto{\pgfqpoint{2.363145in}{3.730839in}}%
\pgfpathlineto{\pgfqpoint{2.363145in}{3.662368in}}%
\pgfpathlineto{\pgfqpoint{2.372015in}{3.666804in}}%
\pgfpathlineto{\pgfqpoint{2.358710in}{3.622452in}}%
\pgfpathlineto{\pgfqpoint{2.345404in}{3.666804in}}%
\pgfpathlineto{\pgfqpoint{2.354274in}{3.662368in}}%
\pgfpathlineto{\pgfqpoint{2.354274in}{3.730839in}}%
\pgfpathlineto{\pgfqpoint{2.363145in}{3.730839in}}%
\pgfusepath{fill}%
\end{pgfscope}%
\begin{pgfscope}%
\pgfpathrectangle{\pgfqpoint{1.432000in}{0.528000in}}{\pgfqpoint{3.696000in}{3.696000in}} %
\pgfusepath{clip}%
\pgfsetbuttcap%
\pgfsetroundjoin%
\definecolor{currentfill}{rgb}{0.146180,0.515413,0.556823}%
\pgfsetfillcolor{currentfill}%
\pgfsetlinewidth{0.000000pt}%
\definecolor{currentstroke}{rgb}{0.000000,0.000000,0.000000}%
\pgfsetstrokecolor{currentstroke}%
\pgfsetdash{}{0pt}%
\pgfpathmoveto{\pgfqpoint{2.471532in}{3.730839in}}%
\pgfpathlineto{\pgfqpoint{2.471532in}{3.662368in}}%
\pgfpathlineto{\pgfqpoint{2.480402in}{3.666804in}}%
\pgfpathlineto{\pgfqpoint{2.467097in}{3.622452in}}%
\pgfpathlineto{\pgfqpoint{2.453791in}{3.666804in}}%
\pgfpathlineto{\pgfqpoint{2.462662in}{3.662368in}}%
\pgfpathlineto{\pgfqpoint{2.462662in}{3.730839in}}%
\pgfpathlineto{\pgfqpoint{2.471532in}{3.730839in}}%
\pgfusepath{fill}%
\end{pgfscope}%
\begin{pgfscope}%
\pgfpathrectangle{\pgfqpoint{1.432000in}{0.528000in}}{\pgfqpoint{3.696000in}{3.696000in}} %
\pgfusepath{clip}%
\pgfsetbuttcap%
\pgfsetroundjoin%
\definecolor{currentfill}{rgb}{0.144759,0.519093,0.556572}%
\pgfsetfillcolor{currentfill}%
\pgfsetlinewidth{0.000000pt}%
\definecolor{currentstroke}{rgb}{0.000000,0.000000,0.000000}%
\pgfsetstrokecolor{currentstroke}%
\pgfsetdash{}{0pt}%
\pgfpathmoveto{\pgfqpoint{2.579919in}{3.730839in}}%
\pgfpathlineto{\pgfqpoint{2.577701in}{3.734680in}}%
\pgfpathlineto{\pgfqpoint{2.573266in}{3.734680in}}%
\pgfpathlineto{\pgfqpoint{2.571049in}{3.730839in}}%
\pgfpathlineto{\pgfqpoint{2.573266in}{3.726998in}}%
\pgfpathlineto{\pgfqpoint{2.577701in}{3.726998in}}%
\pgfpathlineto{\pgfqpoint{2.579919in}{3.730839in}}%
\pgfpathlineto{\pgfqpoint{2.577701in}{3.734680in}}%
\pgfusepath{fill}%
\end{pgfscope}%
\begin{pgfscope}%
\pgfpathrectangle{\pgfqpoint{1.432000in}{0.528000in}}{\pgfqpoint{3.696000in}{3.696000in}} %
\pgfusepath{clip}%
\pgfsetbuttcap%
\pgfsetroundjoin%
\definecolor{currentfill}{rgb}{0.119512,0.607464,0.540218}%
\pgfsetfillcolor{currentfill}%
\pgfsetlinewidth{0.000000pt}%
\definecolor{currentstroke}{rgb}{0.000000,0.000000,0.000000}%
\pgfsetstrokecolor{currentstroke}%
\pgfsetdash{}{0pt}%
\pgfpathmoveto{\pgfqpoint{2.688306in}{3.730839in}}%
\pgfpathlineto{\pgfqpoint{2.686089in}{3.734680in}}%
\pgfpathlineto{\pgfqpoint{2.681653in}{3.734680in}}%
\pgfpathlineto{\pgfqpoint{2.679436in}{3.730839in}}%
\pgfpathlineto{\pgfqpoint{2.681653in}{3.726998in}}%
\pgfpathlineto{\pgfqpoint{2.686089in}{3.726998in}}%
\pgfpathlineto{\pgfqpoint{2.688306in}{3.730839in}}%
\pgfpathlineto{\pgfqpoint{2.686089in}{3.734680in}}%
\pgfusepath{fill}%
\end{pgfscope}%
\begin{pgfscope}%
\pgfpathrectangle{\pgfqpoint{1.432000in}{0.528000in}}{\pgfqpoint{3.696000in}{3.696000in}} %
\pgfusepath{clip}%
\pgfsetbuttcap%
\pgfsetroundjoin%
\definecolor{currentfill}{rgb}{0.150476,0.504369,0.557430}%
\pgfsetfillcolor{currentfill}%
\pgfsetlinewidth{0.000000pt}%
\definecolor{currentstroke}{rgb}{0.000000,0.000000,0.000000}%
\pgfsetstrokecolor{currentstroke}%
\pgfsetdash{}{0pt}%
\pgfpathmoveto{\pgfqpoint{2.796693in}{3.730839in}}%
\pgfpathlineto{\pgfqpoint{2.794476in}{3.734680in}}%
\pgfpathlineto{\pgfqpoint{2.790040in}{3.734680in}}%
\pgfpathlineto{\pgfqpoint{2.787823in}{3.730839in}}%
\pgfpathlineto{\pgfqpoint{2.790040in}{3.726998in}}%
\pgfpathlineto{\pgfqpoint{2.794476in}{3.726998in}}%
\pgfpathlineto{\pgfqpoint{2.796693in}{3.730839in}}%
\pgfpathlineto{\pgfqpoint{2.794476in}{3.734680in}}%
\pgfusepath{fill}%
\end{pgfscope}%
\begin{pgfscope}%
\pgfpathrectangle{\pgfqpoint{1.432000in}{0.528000in}}{\pgfqpoint{3.696000in}{3.696000in}} %
\pgfusepath{clip}%
\pgfsetbuttcap%
\pgfsetroundjoin%
\definecolor{currentfill}{rgb}{0.283229,0.120777,0.440584}%
\pgfsetfillcolor{currentfill}%
\pgfsetlinewidth{0.000000pt}%
\definecolor{currentstroke}{rgb}{0.000000,0.000000,0.000000}%
\pgfsetstrokecolor{currentstroke}%
\pgfsetdash{}{0pt}%
\pgfpathmoveto{\pgfqpoint{2.792258in}{3.735274in}}%
\pgfpathlineto{\pgfqpoint{2.860728in}{3.735274in}}%
\pgfpathlineto{\pgfqpoint{2.856293in}{3.744144in}}%
\pgfpathlineto{\pgfqpoint{2.900645in}{3.730839in}}%
\pgfpathlineto{\pgfqpoint{2.856293in}{3.717533in}}%
\pgfpathlineto{\pgfqpoint{2.860728in}{3.726404in}}%
\pgfpathlineto{\pgfqpoint{2.792258in}{3.726404in}}%
\pgfpathlineto{\pgfqpoint{2.792258in}{3.735274in}}%
\pgfusepath{fill}%
\end{pgfscope}%
\begin{pgfscope}%
\pgfpathrectangle{\pgfqpoint{1.432000in}{0.528000in}}{\pgfqpoint{3.696000in}{3.696000in}} %
\pgfusepath{clip}%
\pgfsetbuttcap%
\pgfsetroundjoin%
\definecolor{currentfill}{rgb}{0.273006,0.204520,0.501721}%
\pgfsetfillcolor{currentfill}%
\pgfsetlinewidth{0.000000pt}%
\definecolor{currentstroke}{rgb}{0.000000,0.000000,0.000000}%
\pgfsetstrokecolor{currentstroke}%
\pgfsetdash{}{0pt}%
\pgfpathmoveto{\pgfqpoint{2.905080in}{3.730839in}}%
\pgfpathlineto{\pgfqpoint{2.902863in}{3.734680in}}%
\pgfpathlineto{\pgfqpoint{2.898428in}{3.734680in}}%
\pgfpathlineto{\pgfqpoint{2.896210in}{3.730839in}}%
\pgfpathlineto{\pgfqpoint{2.898428in}{3.726998in}}%
\pgfpathlineto{\pgfqpoint{2.902863in}{3.726998in}}%
\pgfpathlineto{\pgfqpoint{2.905080in}{3.730839in}}%
\pgfpathlineto{\pgfqpoint{2.902863in}{3.734680in}}%
\pgfusepath{fill}%
\end{pgfscope}%
\begin{pgfscope}%
\pgfpathrectangle{\pgfqpoint{1.432000in}{0.528000in}}{\pgfqpoint{3.696000in}{3.696000in}} %
\pgfusepath{clip}%
\pgfsetbuttcap%
\pgfsetroundjoin%
\definecolor{currentfill}{rgb}{0.214298,0.355619,0.551184}%
\pgfsetfillcolor{currentfill}%
\pgfsetlinewidth{0.000000pt}%
\definecolor{currentstroke}{rgb}{0.000000,0.000000,0.000000}%
\pgfsetstrokecolor{currentstroke}%
\pgfsetdash{}{0pt}%
\pgfpathmoveto{\pgfqpoint{2.900645in}{3.735274in}}%
\pgfpathlineto{\pgfqpoint{2.969115in}{3.735274in}}%
\pgfpathlineto{\pgfqpoint{2.964680in}{3.744144in}}%
\pgfpathlineto{\pgfqpoint{3.009032in}{3.730839in}}%
\pgfpathlineto{\pgfqpoint{2.964680in}{3.717533in}}%
\pgfpathlineto{\pgfqpoint{2.969115in}{3.726404in}}%
\pgfpathlineto{\pgfqpoint{2.900645in}{3.726404in}}%
\pgfpathlineto{\pgfqpoint{2.900645in}{3.735274in}}%
\pgfusepath{fill}%
\end{pgfscope}%
\begin{pgfscope}%
\pgfpathrectangle{\pgfqpoint{1.432000in}{0.528000in}}{\pgfqpoint{3.696000in}{3.696000in}} %
\pgfusepath{clip}%
\pgfsetbuttcap%
\pgfsetroundjoin%
\definecolor{currentfill}{rgb}{0.183898,0.422383,0.556944}%
\pgfsetfillcolor{currentfill}%
\pgfsetlinewidth{0.000000pt}%
\definecolor{currentstroke}{rgb}{0.000000,0.000000,0.000000}%
\pgfsetstrokecolor{currentstroke}%
\pgfsetdash{}{0pt}%
\pgfpathmoveto{\pgfqpoint{3.009032in}{3.735274in}}%
\pgfpathlineto{\pgfqpoint{3.077503in}{3.735274in}}%
\pgfpathlineto{\pgfqpoint{3.073067in}{3.744144in}}%
\pgfpathlineto{\pgfqpoint{3.117419in}{3.730839in}}%
\pgfpathlineto{\pgfqpoint{3.073067in}{3.717533in}}%
\pgfpathlineto{\pgfqpoint{3.077503in}{3.726404in}}%
\pgfpathlineto{\pgfqpoint{3.009032in}{3.726404in}}%
\pgfpathlineto{\pgfqpoint{3.009032in}{3.735274in}}%
\pgfusepath{fill}%
\end{pgfscope}%
\begin{pgfscope}%
\pgfpathrectangle{\pgfqpoint{1.432000in}{0.528000in}}{\pgfqpoint{3.696000in}{3.696000in}} %
\pgfusepath{clip}%
\pgfsetbuttcap%
\pgfsetroundjoin%
\definecolor{currentfill}{rgb}{0.171176,0.452530,0.557965}%
\pgfsetfillcolor{currentfill}%
\pgfsetlinewidth{0.000000pt}%
\definecolor{currentstroke}{rgb}{0.000000,0.000000,0.000000}%
\pgfsetstrokecolor{currentstroke}%
\pgfsetdash{}{0pt}%
\pgfpathmoveto{\pgfqpoint{3.117419in}{3.735274in}}%
\pgfpathlineto{\pgfqpoint{3.185890in}{3.735274in}}%
\pgfpathlineto{\pgfqpoint{3.181454in}{3.744144in}}%
\pgfpathlineto{\pgfqpoint{3.225806in}{3.730839in}}%
\pgfpathlineto{\pgfqpoint{3.181454in}{3.717533in}}%
\pgfpathlineto{\pgfqpoint{3.185890in}{3.726404in}}%
\pgfpathlineto{\pgfqpoint{3.117419in}{3.726404in}}%
\pgfpathlineto{\pgfqpoint{3.117419in}{3.735274in}}%
\pgfusepath{fill}%
\end{pgfscope}%
\begin{pgfscope}%
\pgfpathrectangle{\pgfqpoint{1.432000in}{0.528000in}}{\pgfqpoint{3.696000in}{3.696000in}} %
\pgfusepath{clip}%
\pgfsetbuttcap%
\pgfsetroundjoin%
\definecolor{currentfill}{rgb}{0.210503,0.363727,0.552206}%
\pgfsetfillcolor{currentfill}%
\pgfsetlinewidth{0.000000pt}%
\definecolor{currentstroke}{rgb}{0.000000,0.000000,0.000000}%
\pgfsetstrokecolor{currentstroke}%
\pgfsetdash{}{0pt}%
\pgfpathmoveto{\pgfqpoint{3.225806in}{3.735274in}}%
\pgfpathlineto{\pgfqpoint{3.294277in}{3.735274in}}%
\pgfpathlineto{\pgfqpoint{3.289842in}{3.744144in}}%
\pgfpathlineto{\pgfqpoint{3.334194in}{3.730839in}}%
\pgfpathlineto{\pgfqpoint{3.289842in}{3.717533in}}%
\pgfpathlineto{\pgfqpoint{3.294277in}{3.726404in}}%
\pgfpathlineto{\pgfqpoint{3.225806in}{3.726404in}}%
\pgfpathlineto{\pgfqpoint{3.225806in}{3.735274in}}%
\pgfusepath{fill}%
\end{pgfscope}%
\begin{pgfscope}%
\pgfpathrectangle{\pgfqpoint{1.432000in}{0.528000in}}{\pgfqpoint{3.696000in}{3.696000in}} %
\pgfusepath{clip}%
\pgfsetbuttcap%
\pgfsetroundjoin%
\definecolor{currentfill}{rgb}{0.283229,0.120777,0.440584}%
\pgfsetfillcolor{currentfill}%
\pgfsetlinewidth{0.000000pt}%
\definecolor{currentstroke}{rgb}{0.000000,0.000000,0.000000}%
\pgfsetstrokecolor{currentstroke}%
\pgfsetdash{}{0pt}%
\pgfpathmoveto{\pgfqpoint{3.338629in}{3.730839in}}%
\pgfpathlineto{\pgfqpoint{3.336411in}{3.734680in}}%
\pgfpathlineto{\pgfqpoint{3.331976in}{3.734680in}}%
\pgfpathlineto{\pgfqpoint{3.329758in}{3.730839in}}%
\pgfpathlineto{\pgfqpoint{3.331976in}{3.726998in}}%
\pgfpathlineto{\pgfqpoint{3.336411in}{3.726998in}}%
\pgfpathlineto{\pgfqpoint{3.338629in}{3.730839in}}%
\pgfpathlineto{\pgfqpoint{3.336411in}{3.734680in}}%
\pgfusepath{fill}%
\end{pgfscope}%
\begin{pgfscope}%
\pgfpathrectangle{\pgfqpoint{1.432000in}{0.528000in}}{\pgfqpoint{3.696000in}{3.696000in}} %
\pgfusepath{clip}%
\pgfsetbuttcap%
\pgfsetroundjoin%
\definecolor{currentfill}{rgb}{0.273809,0.031497,0.358853}%
\pgfsetfillcolor{currentfill}%
\pgfsetlinewidth{0.000000pt}%
\definecolor{currentstroke}{rgb}{0.000000,0.000000,0.000000}%
\pgfsetstrokecolor{currentstroke}%
\pgfsetdash{}{0pt}%
\pgfpathmoveto{\pgfqpoint{3.447016in}{3.730839in}}%
\pgfpathlineto{\pgfqpoint{3.444798in}{3.734680in}}%
\pgfpathlineto{\pgfqpoint{3.440363in}{3.734680in}}%
\pgfpathlineto{\pgfqpoint{3.438145in}{3.730839in}}%
\pgfpathlineto{\pgfqpoint{3.440363in}{3.726998in}}%
\pgfpathlineto{\pgfqpoint{3.444798in}{3.726998in}}%
\pgfpathlineto{\pgfqpoint{3.447016in}{3.730839in}}%
\pgfpathlineto{\pgfqpoint{3.444798in}{3.734680in}}%
\pgfusepath{fill}%
\end{pgfscope}%
\begin{pgfscope}%
\pgfpathrectangle{\pgfqpoint{1.432000in}{0.528000in}}{\pgfqpoint{3.696000in}{3.696000in}} %
\pgfusepath{clip}%
\pgfsetbuttcap%
\pgfsetroundjoin%
\definecolor{currentfill}{rgb}{0.280267,0.073417,0.397163}%
\pgfsetfillcolor{currentfill}%
\pgfsetlinewidth{0.000000pt}%
\definecolor{currentstroke}{rgb}{0.000000,0.000000,0.000000}%
\pgfsetstrokecolor{currentstroke}%
\pgfsetdash{}{0pt}%
\pgfpathmoveto{\pgfqpoint{3.550968in}{3.726404in}}%
\pgfpathlineto{\pgfqpoint{3.482497in}{3.726404in}}%
\pgfpathlineto{\pgfqpoint{3.486933in}{3.717533in}}%
\pgfpathlineto{\pgfqpoint{3.442581in}{3.730839in}}%
\pgfpathlineto{\pgfqpoint{3.486933in}{3.744144in}}%
\pgfpathlineto{\pgfqpoint{3.482497in}{3.735274in}}%
\pgfpathlineto{\pgfqpoint{3.550968in}{3.735274in}}%
\pgfpathlineto{\pgfqpoint{3.550968in}{3.726404in}}%
\pgfusepath{fill}%
\end{pgfscope}%
\begin{pgfscope}%
\pgfpathrectangle{\pgfqpoint{1.432000in}{0.528000in}}{\pgfqpoint{3.696000in}{3.696000in}} %
\pgfusepath{clip}%
\pgfsetbuttcap%
\pgfsetroundjoin%
\definecolor{currentfill}{rgb}{0.280267,0.073417,0.397163}%
\pgfsetfillcolor{currentfill}%
\pgfsetlinewidth{0.000000pt}%
\definecolor{currentstroke}{rgb}{0.000000,0.000000,0.000000}%
\pgfsetstrokecolor{currentstroke}%
\pgfsetdash{}{0pt}%
\pgfpathmoveto{\pgfqpoint{3.656219in}{3.727703in}}%
\pgfpathlineto{\pgfqpoint{3.576057in}{3.807864in}}%
\pgfpathlineto{\pgfqpoint{3.572921in}{3.798456in}}%
\pgfpathlineto{\pgfqpoint{3.550968in}{3.839226in}}%
\pgfpathlineto{\pgfqpoint{3.591738in}{3.817273in}}%
\pgfpathlineto{\pgfqpoint{3.582329in}{3.814137in}}%
\pgfpathlineto{\pgfqpoint{3.662491in}{3.733975in}}%
\pgfpathlineto{\pgfqpoint{3.656219in}{3.727703in}}%
\pgfusepath{fill}%
\end{pgfscope}%
\begin{pgfscope}%
\pgfpathrectangle{\pgfqpoint{1.432000in}{0.528000in}}{\pgfqpoint{3.696000in}{3.696000in}} %
\pgfusepath{clip}%
\pgfsetbuttcap%
\pgfsetroundjoin%
\definecolor{currentfill}{rgb}{0.278012,0.180367,0.486697}%
\pgfsetfillcolor{currentfill}%
\pgfsetlinewidth{0.000000pt}%
\definecolor{currentstroke}{rgb}{0.000000,0.000000,0.000000}%
\pgfsetstrokecolor{currentstroke}%
\pgfsetdash{}{0pt}%
\pgfpathmoveto{\pgfqpoint{3.764606in}{3.727703in}}%
\pgfpathlineto{\pgfqpoint{3.684444in}{3.807864in}}%
\pgfpathlineto{\pgfqpoint{3.681308in}{3.798456in}}%
\pgfpathlineto{\pgfqpoint{3.659355in}{3.839226in}}%
\pgfpathlineto{\pgfqpoint{3.700125in}{3.817273in}}%
\pgfpathlineto{\pgfqpoint{3.690716in}{3.814137in}}%
\pgfpathlineto{\pgfqpoint{3.770878in}{3.733975in}}%
\pgfpathlineto{\pgfqpoint{3.764606in}{3.727703in}}%
\pgfusepath{fill}%
\end{pgfscope}%
\begin{pgfscope}%
\pgfpathrectangle{\pgfqpoint{1.432000in}{0.528000in}}{\pgfqpoint{3.696000in}{3.696000in}} %
\pgfusepath{clip}%
\pgfsetbuttcap%
\pgfsetroundjoin%
\definecolor{currentfill}{rgb}{0.208623,0.367752,0.552675}%
\pgfsetfillcolor{currentfill}%
\pgfsetlinewidth{0.000000pt}%
\definecolor{currentstroke}{rgb}{0.000000,0.000000,0.000000}%
\pgfsetstrokecolor{currentstroke}%
\pgfsetdash{}{0pt}%
\pgfpathmoveto{\pgfqpoint{3.874146in}{3.726872in}}%
\pgfpathlineto{\pgfqpoint{3.693074in}{3.817408in}}%
\pgfpathlineto{\pgfqpoint{3.693074in}{3.807490in}}%
\pgfpathlineto{\pgfqpoint{3.659355in}{3.839226in}}%
\pgfpathlineto{\pgfqpoint{3.704975in}{3.831292in}}%
\pgfpathlineto{\pgfqpoint{3.697041in}{3.825341in}}%
\pgfpathlineto{\pgfqpoint{3.878113in}{3.734806in}}%
\pgfpathlineto{\pgfqpoint{3.874146in}{3.726872in}}%
\pgfusepath{fill}%
\end{pgfscope}%
\begin{pgfscope}%
\pgfpathrectangle{\pgfqpoint{1.432000in}{0.528000in}}{\pgfqpoint{3.696000in}{3.696000in}} %
\pgfusepath{clip}%
\pgfsetbuttcap%
\pgfsetroundjoin%
\definecolor{currentfill}{rgb}{0.177423,0.437527,0.557565}%
\pgfsetfillcolor{currentfill}%
\pgfsetlinewidth{0.000000pt}%
\definecolor{currentstroke}{rgb}{0.000000,0.000000,0.000000}%
\pgfsetstrokecolor{currentstroke}%
\pgfsetdash{}{0pt}%
\pgfpathmoveto{\pgfqpoint{3.982533in}{3.726872in}}%
\pgfpathlineto{\pgfqpoint{3.801461in}{3.817408in}}%
\pgfpathlineto{\pgfqpoint{3.801461in}{3.807490in}}%
\pgfpathlineto{\pgfqpoint{3.767742in}{3.839226in}}%
\pgfpathlineto{\pgfqpoint{3.813362in}{3.831292in}}%
\pgfpathlineto{\pgfqpoint{3.805428in}{3.825341in}}%
\pgfpathlineto{\pgfqpoint{3.986500in}{3.734806in}}%
\pgfpathlineto{\pgfqpoint{3.982533in}{3.726872in}}%
\pgfusepath{fill}%
\end{pgfscope}%
\begin{pgfscope}%
\pgfpathrectangle{\pgfqpoint{1.432000in}{0.528000in}}{\pgfqpoint{3.696000in}{3.696000in}} %
\pgfusepath{clip}%
\pgfsetbuttcap%
\pgfsetroundjoin%
\definecolor{currentfill}{rgb}{0.262138,0.242286,0.520837}%
\pgfsetfillcolor{currentfill}%
\pgfsetlinewidth{0.000000pt}%
\definecolor{currentstroke}{rgb}{0.000000,0.000000,0.000000}%
\pgfsetstrokecolor{currentstroke}%
\pgfsetdash{}{0pt}%
\pgfpathmoveto{\pgfqpoint{4.089767in}{3.727703in}}%
\pgfpathlineto{\pgfqpoint{3.901218in}{3.916251in}}%
\pgfpathlineto{\pgfqpoint{3.898082in}{3.906843in}}%
\pgfpathlineto{\pgfqpoint{3.876129in}{3.947613in}}%
\pgfpathlineto{\pgfqpoint{3.916899in}{3.925660in}}%
\pgfpathlineto{\pgfqpoint{3.907491in}{3.922524in}}%
\pgfpathlineto{\pgfqpoint{4.096039in}{3.733975in}}%
\pgfpathlineto{\pgfqpoint{4.089767in}{3.727703in}}%
\pgfusepath{fill}%
\end{pgfscope}%
\begin{pgfscope}%
\pgfpathrectangle{\pgfqpoint{1.432000in}{0.528000in}}{\pgfqpoint{3.696000in}{3.696000in}} %
\pgfusepath{clip}%
\pgfsetbuttcap%
\pgfsetroundjoin%
\definecolor{currentfill}{rgb}{0.282910,0.105393,0.426902}%
\pgfsetfillcolor{currentfill}%
\pgfsetlinewidth{0.000000pt}%
\definecolor{currentstroke}{rgb}{0.000000,0.000000,0.000000}%
\pgfsetstrokecolor{currentstroke}%
\pgfsetdash{}{0pt}%
\pgfpathmoveto{\pgfqpoint{4.198830in}{3.727148in}}%
\pgfpathlineto{\pgfqpoint{3.906882in}{3.921781in}}%
\pgfpathlineto{\pgfqpoint{3.905652in}{3.911940in}}%
\pgfpathlineto{\pgfqpoint{3.876129in}{3.947613in}}%
\pgfpathlineto{\pgfqpoint{3.920413in}{3.934082in}}%
\pgfpathlineto{\pgfqpoint{3.911802in}{3.929161in}}%
\pgfpathlineto{\pgfqpoint{4.203751in}{3.734529in}}%
\pgfpathlineto{\pgfqpoint{4.198830in}{3.727148in}}%
\pgfusepath{fill}%
\end{pgfscope}%
\begin{pgfscope}%
\pgfpathrectangle{\pgfqpoint{1.432000in}{0.528000in}}{\pgfqpoint{3.696000in}{3.696000in}} %
\pgfusepath{clip}%
\pgfsetbuttcap%
\pgfsetroundjoin%
\definecolor{currentfill}{rgb}{0.252194,0.269783,0.531579}%
\pgfsetfillcolor{currentfill}%
\pgfsetlinewidth{0.000000pt}%
\definecolor{currentstroke}{rgb}{0.000000,0.000000,0.000000}%
\pgfsetstrokecolor{currentstroke}%
\pgfsetdash{}{0pt}%
\pgfpathmoveto{\pgfqpoint{4.198154in}{3.727703in}}%
\pgfpathlineto{\pgfqpoint{4.009605in}{3.916251in}}%
\pgfpathlineto{\pgfqpoint{4.006469in}{3.906843in}}%
\pgfpathlineto{\pgfqpoint{3.984516in}{3.947613in}}%
\pgfpathlineto{\pgfqpoint{4.025286in}{3.925660in}}%
\pgfpathlineto{\pgfqpoint{4.015878in}{3.922524in}}%
\pgfpathlineto{\pgfqpoint{4.204426in}{3.733975in}}%
\pgfpathlineto{\pgfqpoint{4.198154in}{3.727703in}}%
\pgfusepath{fill}%
\end{pgfscope}%
\begin{pgfscope}%
\pgfpathrectangle{\pgfqpoint{1.432000in}{0.528000in}}{\pgfqpoint{3.696000in}{3.696000in}} %
\pgfusepath{clip}%
\pgfsetbuttcap%
\pgfsetroundjoin%
\definecolor{currentfill}{rgb}{0.260571,0.246922,0.522828}%
\pgfsetfillcolor{currentfill}%
\pgfsetlinewidth{0.000000pt}%
\definecolor{currentstroke}{rgb}{0.000000,0.000000,0.000000}%
\pgfsetstrokecolor{currentstroke}%
\pgfsetdash{}{0pt}%
\pgfpathmoveto{\pgfqpoint{4.307217in}{3.727148in}}%
\pgfpathlineto{\pgfqpoint{4.015269in}{3.921781in}}%
\pgfpathlineto{\pgfqpoint{4.014039in}{3.911940in}}%
\pgfpathlineto{\pgfqpoint{3.984516in}{3.947613in}}%
\pgfpathlineto{\pgfqpoint{4.028800in}{3.934082in}}%
\pgfpathlineto{\pgfqpoint{4.020189in}{3.929161in}}%
\pgfpathlineto{\pgfqpoint{4.312138in}{3.734529in}}%
\pgfpathlineto{\pgfqpoint{4.307217in}{3.727148in}}%
\pgfusepath{fill}%
\end{pgfscope}%
\begin{pgfscope}%
\pgfpathrectangle{\pgfqpoint{1.432000in}{0.528000in}}{\pgfqpoint{3.696000in}{3.696000in}} %
\pgfusepath{clip}%
\pgfsetbuttcap%
\pgfsetroundjoin%
\definecolor{currentfill}{rgb}{0.271828,0.209303,0.504434}%
\pgfsetfillcolor{currentfill}%
\pgfsetlinewidth{0.000000pt}%
\definecolor{currentstroke}{rgb}{0.000000,0.000000,0.000000}%
\pgfsetstrokecolor{currentstroke}%
\pgfsetdash{}{0pt}%
\pgfpathmoveto{\pgfqpoint{4.306541in}{3.727703in}}%
\pgfpathlineto{\pgfqpoint{4.117993in}{3.916251in}}%
\pgfpathlineto{\pgfqpoint{4.114856in}{3.906843in}}%
\pgfpathlineto{\pgfqpoint{4.092903in}{3.947613in}}%
\pgfpathlineto{\pgfqpoint{4.133673in}{3.925660in}}%
\pgfpathlineto{\pgfqpoint{4.124265in}{3.922524in}}%
\pgfpathlineto{\pgfqpoint{4.312814in}{3.733975in}}%
\pgfpathlineto{\pgfqpoint{4.306541in}{3.727703in}}%
\pgfusepath{fill}%
\end{pgfscope}%
\begin{pgfscope}%
\pgfpathrectangle{\pgfqpoint{1.432000in}{0.528000in}}{\pgfqpoint{3.696000in}{3.696000in}} %
\pgfusepath{clip}%
\pgfsetbuttcap%
\pgfsetroundjoin%
\definecolor{currentfill}{rgb}{0.257322,0.256130,0.526563}%
\pgfsetfillcolor{currentfill}%
\pgfsetlinewidth{0.000000pt}%
\definecolor{currentstroke}{rgb}{0.000000,0.000000,0.000000}%
\pgfsetstrokecolor{currentstroke}%
\pgfsetdash{}{0pt}%
\pgfpathmoveto{\pgfqpoint{4.415604in}{3.727148in}}%
\pgfpathlineto{\pgfqpoint{4.123656in}{3.921781in}}%
\pgfpathlineto{\pgfqpoint{4.122426in}{3.911940in}}%
\pgfpathlineto{\pgfqpoint{4.092903in}{3.947613in}}%
\pgfpathlineto{\pgfqpoint{4.137187in}{3.934082in}}%
\pgfpathlineto{\pgfqpoint{4.128576in}{3.929161in}}%
\pgfpathlineto{\pgfqpoint{4.420525in}{3.734529in}}%
\pgfpathlineto{\pgfqpoint{4.415604in}{3.727148in}}%
\pgfusepath{fill}%
\end{pgfscope}%
\begin{pgfscope}%
\pgfpathrectangle{\pgfqpoint{1.432000in}{0.528000in}}{\pgfqpoint{3.696000in}{3.696000in}} %
\pgfusepath{clip}%
\pgfsetbuttcap%
\pgfsetroundjoin%
\definecolor{currentfill}{rgb}{0.270595,0.214069,0.507052}%
\pgfsetfillcolor{currentfill}%
\pgfsetlinewidth{0.000000pt}%
\definecolor{currentstroke}{rgb}{0.000000,0.000000,0.000000}%
\pgfsetstrokecolor{currentstroke}%
\pgfsetdash{}{0pt}%
\pgfpathmoveto{\pgfqpoint{4.414928in}{3.727703in}}%
\pgfpathlineto{\pgfqpoint{4.226380in}{3.916251in}}%
\pgfpathlineto{\pgfqpoint{4.223243in}{3.906843in}}%
\pgfpathlineto{\pgfqpoint{4.201290in}{3.947613in}}%
\pgfpathlineto{\pgfqpoint{4.242060in}{3.925660in}}%
\pgfpathlineto{\pgfqpoint{4.232652in}{3.922524in}}%
\pgfpathlineto{\pgfqpoint{4.421201in}{3.733975in}}%
\pgfpathlineto{\pgfqpoint{4.414928in}{3.727703in}}%
\pgfusepath{fill}%
\end{pgfscope}%
\begin{pgfscope}%
\pgfpathrectangle{\pgfqpoint{1.432000in}{0.528000in}}{\pgfqpoint{3.696000in}{3.696000in}} %
\pgfusepath{clip}%
\pgfsetbuttcap%
\pgfsetroundjoin%
\definecolor{currentfill}{rgb}{0.282910,0.105393,0.426902}%
\pgfsetfillcolor{currentfill}%
\pgfsetlinewidth{0.000000pt}%
\definecolor{currentstroke}{rgb}{0.000000,0.000000,0.000000}%
\pgfsetstrokecolor{currentstroke}%
\pgfsetdash{}{0pt}%
\pgfpathmoveto{\pgfqpoint{4.523991in}{3.727148in}}%
\pgfpathlineto{\pgfqpoint{4.232043in}{3.921781in}}%
\pgfpathlineto{\pgfqpoint{4.230813in}{3.911940in}}%
\pgfpathlineto{\pgfqpoint{4.201290in}{3.947613in}}%
\pgfpathlineto{\pgfqpoint{4.245574in}{3.934082in}}%
\pgfpathlineto{\pgfqpoint{4.236963in}{3.929161in}}%
\pgfpathlineto{\pgfqpoint{4.528912in}{3.734529in}}%
\pgfpathlineto{\pgfqpoint{4.523991in}{3.727148in}}%
\pgfusepath{fill}%
\end{pgfscope}%
\begin{pgfscope}%
\pgfpathrectangle{\pgfqpoint{1.432000in}{0.528000in}}{\pgfqpoint{3.696000in}{3.696000in}} %
\pgfusepath{clip}%
\pgfsetbuttcap%
\pgfsetroundjoin%
\definecolor{currentfill}{rgb}{0.183898,0.422383,0.556944}%
\pgfsetfillcolor{currentfill}%
\pgfsetlinewidth{0.000000pt}%
\definecolor{currentstroke}{rgb}{0.000000,0.000000,0.000000}%
\pgfsetstrokecolor{currentstroke}%
\pgfsetdash{}{0pt}%
\pgfpathmoveto{\pgfqpoint{4.523315in}{3.727703in}}%
\pgfpathlineto{\pgfqpoint{4.334767in}{3.916251in}}%
\pgfpathlineto{\pgfqpoint{4.331631in}{3.906843in}}%
\pgfpathlineto{\pgfqpoint{4.309677in}{3.947613in}}%
\pgfpathlineto{\pgfqpoint{4.350447in}{3.925660in}}%
\pgfpathlineto{\pgfqpoint{4.341039in}{3.922524in}}%
\pgfpathlineto{\pgfqpoint{4.529588in}{3.733975in}}%
\pgfpathlineto{\pgfqpoint{4.523315in}{3.727703in}}%
\pgfusepath{fill}%
\end{pgfscope}%
\begin{pgfscope}%
\pgfpathrectangle{\pgfqpoint{1.432000in}{0.528000in}}{\pgfqpoint{3.696000in}{3.696000in}} %
\pgfusepath{clip}%
\pgfsetbuttcap%
\pgfsetroundjoin%
\definecolor{currentfill}{rgb}{0.243113,0.292092,0.538516}%
\pgfsetfillcolor{currentfill}%
\pgfsetlinewidth{0.000000pt}%
\definecolor{currentstroke}{rgb}{0.000000,0.000000,0.000000}%
\pgfsetstrokecolor{currentstroke}%
\pgfsetdash{}{0pt}%
\pgfpathmoveto{\pgfqpoint{4.631703in}{3.727703in}}%
\pgfpathlineto{\pgfqpoint{4.443154in}{3.916251in}}%
\pgfpathlineto{\pgfqpoint{4.440018in}{3.906843in}}%
\pgfpathlineto{\pgfqpoint{4.418065in}{3.947613in}}%
\pgfpathlineto{\pgfqpoint{4.458835in}{3.925660in}}%
\pgfpathlineto{\pgfqpoint{4.449426in}{3.922524in}}%
\pgfpathlineto{\pgfqpoint{4.637975in}{3.733975in}}%
\pgfpathlineto{\pgfqpoint{4.631703in}{3.727703in}}%
\pgfusepath{fill}%
\end{pgfscope}%
\begin{pgfscope}%
\pgfpathrectangle{\pgfqpoint{1.432000in}{0.528000in}}{\pgfqpoint{3.696000in}{3.696000in}} %
\pgfusepath{clip}%
\pgfsetbuttcap%
\pgfsetroundjoin%
\definecolor{currentfill}{rgb}{0.276022,0.044167,0.370164}%
\pgfsetfillcolor{currentfill}%
\pgfsetlinewidth{0.000000pt}%
\definecolor{currentstroke}{rgb}{0.000000,0.000000,0.000000}%
\pgfsetstrokecolor{currentstroke}%
\pgfsetdash{}{0pt}%
\pgfpathmoveto{\pgfqpoint{4.630872in}{3.728855in}}%
\pgfpathlineto{\pgfqpoint{4.540336in}{3.909927in}}%
\pgfpathlineto{\pgfqpoint{4.534386in}{3.901993in}}%
\pgfpathlineto{\pgfqpoint{4.526452in}{3.947613in}}%
\pgfpathlineto{\pgfqpoint{4.558187in}{3.913894in}}%
\pgfpathlineto{\pgfqpoint{4.548270in}{3.913894in}}%
\pgfpathlineto{\pgfqpoint{4.638806in}{3.732822in}}%
\pgfpathlineto{\pgfqpoint{4.630872in}{3.728855in}}%
\pgfusepath{fill}%
\end{pgfscope}%
\begin{pgfscope}%
\pgfpathrectangle{\pgfqpoint{1.432000in}{0.528000in}}{\pgfqpoint{3.696000in}{3.696000in}} %
\pgfusepath{clip}%
\pgfsetbuttcap%
\pgfsetroundjoin%
\definecolor{currentfill}{rgb}{0.269944,0.014625,0.341379}%
\pgfsetfillcolor{currentfill}%
\pgfsetlinewidth{0.000000pt}%
\definecolor{currentstroke}{rgb}{0.000000,0.000000,0.000000}%
\pgfsetstrokecolor{currentstroke}%
\pgfsetdash{}{0pt}%
\pgfpathmoveto{\pgfqpoint{4.740090in}{3.727703in}}%
\pgfpathlineto{\pgfqpoint{4.659928in}{3.807864in}}%
\pgfpathlineto{\pgfqpoint{4.656792in}{3.798456in}}%
\pgfpathlineto{\pgfqpoint{4.634839in}{3.839226in}}%
\pgfpathlineto{\pgfqpoint{4.675609in}{3.817273in}}%
\pgfpathlineto{\pgfqpoint{4.666200in}{3.814137in}}%
\pgfpathlineto{\pgfqpoint{4.746362in}{3.733975in}}%
\pgfpathlineto{\pgfqpoint{4.740090in}{3.727703in}}%
\pgfusepath{fill}%
\end{pgfscope}%
\begin{pgfscope}%
\pgfpathrectangle{\pgfqpoint{1.432000in}{0.528000in}}{\pgfqpoint{3.696000in}{3.696000in}} %
\pgfusepath{clip}%
\pgfsetbuttcap%
\pgfsetroundjoin%
\definecolor{currentfill}{rgb}{0.272594,0.025563,0.353093}%
\pgfsetfillcolor{currentfill}%
\pgfsetlinewidth{0.000000pt}%
\definecolor{currentstroke}{rgb}{0.000000,0.000000,0.000000}%
\pgfsetstrokecolor{currentstroke}%
\pgfsetdash{}{0pt}%
\pgfpathmoveto{\pgfqpoint{4.740090in}{3.727703in}}%
\pgfpathlineto{\pgfqpoint{4.551541in}{3.916251in}}%
\pgfpathlineto{\pgfqpoint{4.548405in}{3.906843in}}%
\pgfpathlineto{\pgfqpoint{4.526452in}{3.947613in}}%
\pgfpathlineto{\pgfqpoint{4.567222in}{3.925660in}}%
\pgfpathlineto{\pgfqpoint{4.557813in}{3.922524in}}%
\pgfpathlineto{\pgfqpoint{4.746362in}{3.733975in}}%
\pgfpathlineto{\pgfqpoint{4.740090in}{3.727703in}}%
\pgfusepath{fill}%
\end{pgfscope}%
\begin{pgfscope}%
\pgfpathrectangle{\pgfqpoint{1.432000in}{0.528000in}}{\pgfqpoint{3.696000in}{3.696000in}} %
\pgfusepath{clip}%
\pgfsetbuttcap%
\pgfsetroundjoin%
\definecolor{currentfill}{rgb}{0.257322,0.256130,0.526563}%
\pgfsetfillcolor{currentfill}%
\pgfsetlinewidth{0.000000pt}%
\definecolor{currentstroke}{rgb}{0.000000,0.000000,0.000000}%
\pgfsetstrokecolor{currentstroke}%
\pgfsetdash{}{0pt}%
\pgfpathmoveto{\pgfqpoint{4.739259in}{3.728855in}}%
\pgfpathlineto{\pgfqpoint{4.648723in}{3.909927in}}%
\pgfpathlineto{\pgfqpoint{4.642773in}{3.901993in}}%
\pgfpathlineto{\pgfqpoint{4.634839in}{3.947613in}}%
\pgfpathlineto{\pgfqpoint{4.666574in}{3.913894in}}%
\pgfpathlineto{\pgfqpoint{4.656657in}{3.913894in}}%
\pgfpathlineto{\pgfqpoint{4.747193in}{3.732822in}}%
\pgfpathlineto{\pgfqpoint{4.739259in}{3.728855in}}%
\pgfusepath{fill}%
\end{pgfscope}%
\begin{pgfscope}%
\pgfpathrectangle{\pgfqpoint{1.432000in}{0.528000in}}{\pgfqpoint{3.696000in}{3.696000in}} %
\pgfusepath{clip}%
\pgfsetbuttcap%
\pgfsetroundjoin%
\definecolor{currentfill}{rgb}{0.253935,0.265254,0.529983}%
\pgfsetfillcolor{currentfill}%
\pgfsetlinewidth{0.000000pt}%
\definecolor{currentstroke}{rgb}{0.000000,0.000000,0.000000}%
\pgfsetstrokecolor{currentstroke}%
\pgfsetdash{}{0pt}%
\pgfpathmoveto{\pgfqpoint{4.848477in}{3.727703in}}%
\pgfpathlineto{\pgfqpoint{4.768315in}{3.807864in}}%
\pgfpathlineto{\pgfqpoint{4.765179in}{3.798456in}}%
\pgfpathlineto{\pgfqpoint{4.743226in}{3.839226in}}%
\pgfpathlineto{\pgfqpoint{4.783996in}{3.817273in}}%
\pgfpathlineto{\pgfqpoint{4.774587in}{3.814137in}}%
\pgfpathlineto{\pgfqpoint{4.854749in}{3.733975in}}%
\pgfpathlineto{\pgfqpoint{4.848477in}{3.727703in}}%
\pgfusepath{fill}%
\end{pgfscope}%
\begin{pgfscope}%
\pgfpathrectangle{\pgfqpoint{1.432000in}{0.528000in}}{\pgfqpoint{3.696000in}{3.696000in}} %
\pgfusepath{clip}%
\pgfsetbuttcap%
\pgfsetroundjoin%
\definecolor{currentfill}{rgb}{0.283187,0.125848,0.444960}%
\pgfsetfillcolor{currentfill}%
\pgfsetlinewidth{0.000000pt}%
\definecolor{currentstroke}{rgb}{0.000000,0.000000,0.000000}%
\pgfsetstrokecolor{currentstroke}%
\pgfsetdash{}{0pt}%
\pgfpathmoveto{\pgfqpoint{4.847178in}{3.730839in}}%
\pgfpathlineto{\pgfqpoint{4.847178in}{3.799309in}}%
\pgfpathlineto{\pgfqpoint{4.838307in}{3.794874in}}%
\pgfpathlineto{\pgfqpoint{4.851613in}{3.839226in}}%
\pgfpathlineto{\pgfqpoint{4.864919in}{3.794874in}}%
\pgfpathlineto{\pgfqpoint{4.856048in}{3.799309in}}%
\pgfpathlineto{\pgfqpoint{4.856048in}{3.730839in}}%
\pgfpathlineto{\pgfqpoint{4.847178in}{3.730839in}}%
\pgfusepath{fill}%
\end{pgfscope}%
\begin{pgfscope}%
\pgfpathrectangle{\pgfqpoint{1.432000in}{0.528000in}}{\pgfqpoint{3.696000in}{3.696000in}} %
\pgfusepath{clip}%
\pgfsetbuttcap%
\pgfsetroundjoin%
\definecolor{currentfill}{rgb}{0.250425,0.274290,0.533103}%
\pgfsetfillcolor{currentfill}%
\pgfsetlinewidth{0.000000pt}%
\definecolor{currentstroke}{rgb}{0.000000,0.000000,0.000000}%
\pgfsetstrokecolor{currentstroke}%
\pgfsetdash{}{0pt}%
\pgfpathmoveto{\pgfqpoint{4.955565in}{3.730839in}}%
\pgfpathlineto{\pgfqpoint{4.955565in}{3.799309in}}%
\pgfpathlineto{\pgfqpoint{4.946694in}{3.794874in}}%
\pgfpathlineto{\pgfqpoint{4.960000in}{3.839226in}}%
\pgfpathlineto{\pgfqpoint{4.973306in}{3.794874in}}%
\pgfpathlineto{\pgfqpoint{4.964435in}{3.799309in}}%
\pgfpathlineto{\pgfqpoint{4.964435in}{3.730839in}}%
\pgfpathlineto{\pgfqpoint{4.955565in}{3.730839in}}%
\pgfusepath{fill}%
\end{pgfscope}%
\begin{pgfscope}%
\pgfpathrectangle{\pgfqpoint{1.432000in}{0.528000in}}{\pgfqpoint{3.696000in}{3.696000in}} %
\pgfusepath{clip}%
\pgfsetbuttcap%
\pgfsetroundjoin%
\definecolor{currentfill}{rgb}{0.280267,0.073417,0.397163}%
\pgfsetfillcolor{currentfill}%
\pgfsetlinewidth{0.000000pt}%
\definecolor{currentstroke}{rgb}{0.000000,0.000000,0.000000}%
\pgfsetstrokecolor{currentstroke}%
\pgfsetdash{}{0pt}%
\pgfpathmoveto{\pgfqpoint{4.955565in}{3.730839in}}%
\pgfpathlineto{\pgfqpoint{4.955565in}{3.907696in}}%
\pgfpathlineto{\pgfqpoint{4.946694in}{3.903261in}}%
\pgfpathlineto{\pgfqpoint{4.960000in}{3.947613in}}%
\pgfpathlineto{\pgfqpoint{4.973306in}{3.903261in}}%
\pgfpathlineto{\pgfqpoint{4.964435in}{3.907696in}}%
\pgfpathlineto{\pgfqpoint{4.964435in}{3.730839in}}%
\pgfpathlineto{\pgfqpoint{4.955565in}{3.730839in}}%
\pgfusepath{fill}%
\end{pgfscope}%
\begin{pgfscope}%
\pgfpathrectangle{\pgfqpoint{1.432000in}{0.528000in}}{\pgfqpoint{3.696000in}{3.696000in}} %
\pgfusepath{clip}%
\pgfsetbuttcap%
\pgfsetroundjoin%
\definecolor{currentfill}{rgb}{0.174274,0.445044,0.557792}%
\pgfsetfillcolor{currentfill}%
\pgfsetlinewidth{0.000000pt}%
\definecolor{currentstroke}{rgb}{0.000000,0.000000,0.000000}%
\pgfsetstrokecolor{currentstroke}%
\pgfsetdash{}{0pt}%
\pgfpathmoveto{\pgfqpoint{1.604435in}{3.839226in}}%
\pgfpathlineto{\pgfqpoint{1.604435in}{3.770756in}}%
\pgfpathlineto{\pgfqpoint{1.613306in}{3.775191in}}%
\pgfpathlineto{\pgfqpoint{1.600000in}{3.730839in}}%
\pgfpathlineto{\pgfqpoint{1.586694in}{3.775191in}}%
\pgfpathlineto{\pgfqpoint{1.595565in}{3.770756in}}%
\pgfpathlineto{\pgfqpoint{1.595565in}{3.839226in}}%
\pgfpathlineto{\pgfqpoint{1.604435in}{3.839226in}}%
\pgfusepath{fill}%
\end{pgfscope}%
\begin{pgfscope}%
\pgfpathrectangle{\pgfqpoint{1.432000in}{0.528000in}}{\pgfqpoint{3.696000in}{3.696000in}} %
\pgfusepath{clip}%
\pgfsetbuttcap%
\pgfsetroundjoin%
\definecolor{currentfill}{rgb}{0.273809,0.031497,0.358853}%
\pgfsetfillcolor{currentfill}%
\pgfsetlinewidth{0.000000pt}%
\definecolor{currentstroke}{rgb}{0.000000,0.000000,0.000000}%
\pgfsetstrokecolor{currentstroke}%
\pgfsetdash{}{0pt}%
\pgfpathmoveto{\pgfqpoint{1.712822in}{3.839226in}}%
\pgfpathlineto{\pgfqpoint{1.712822in}{3.662368in}}%
\pgfpathlineto{\pgfqpoint{1.721693in}{3.666804in}}%
\pgfpathlineto{\pgfqpoint{1.708387in}{3.622452in}}%
\pgfpathlineto{\pgfqpoint{1.695081in}{3.666804in}}%
\pgfpathlineto{\pgfqpoint{1.703952in}{3.662368in}}%
\pgfpathlineto{\pgfqpoint{1.703952in}{3.839226in}}%
\pgfpathlineto{\pgfqpoint{1.712822in}{3.839226in}}%
\pgfusepath{fill}%
\end{pgfscope}%
\begin{pgfscope}%
\pgfpathrectangle{\pgfqpoint{1.432000in}{0.528000in}}{\pgfqpoint{3.696000in}{3.696000in}} %
\pgfusepath{clip}%
\pgfsetbuttcap%
\pgfsetroundjoin%
\definecolor{currentfill}{rgb}{0.277941,0.056324,0.381191}%
\pgfsetfillcolor{currentfill}%
\pgfsetlinewidth{0.000000pt}%
\definecolor{currentstroke}{rgb}{0.000000,0.000000,0.000000}%
\pgfsetstrokecolor{currentstroke}%
\pgfsetdash{}{0pt}%
\pgfpathmoveto{\pgfqpoint{1.821209in}{3.839226in}}%
\pgfpathlineto{\pgfqpoint{1.821209in}{3.662368in}}%
\pgfpathlineto{\pgfqpoint{1.830080in}{3.666804in}}%
\pgfpathlineto{\pgfqpoint{1.816774in}{3.622452in}}%
\pgfpathlineto{\pgfqpoint{1.803469in}{3.666804in}}%
\pgfpathlineto{\pgfqpoint{1.812339in}{3.662368in}}%
\pgfpathlineto{\pgfqpoint{1.812339in}{3.839226in}}%
\pgfpathlineto{\pgfqpoint{1.821209in}{3.839226in}}%
\pgfusepath{fill}%
\end{pgfscope}%
\begin{pgfscope}%
\pgfpathrectangle{\pgfqpoint{1.432000in}{0.528000in}}{\pgfqpoint{3.696000in}{3.696000in}} %
\pgfusepath{clip}%
\pgfsetbuttcap%
\pgfsetroundjoin%
\definecolor{currentfill}{rgb}{0.274952,0.037752,0.364543}%
\pgfsetfillcolor{currentfill}%
\pgfsetlinewidth{0.000000pt}%
\definecolor{currentstroke}{rgb}{0.000000,0.000000,0.000000}%
\pgfsetstrokecolor{currentstroke}%
\pgfsetdash{}{0pt}%
\pgfpathmoveto{\pgfqpoint{1.929596in}{3.839226in}}%
\pgfpathlineto{\pgfqpoint{1.929596in}{3.662368in}}%
\pgfpathlineto{\pgfqpoint{1.938467in}{3.666804in}}%
\pgfpathlineto{\pgfqpoint{1.925161in}{3.622452in}}%
\pgfpathlineto{\pgfqpoint{1.911856in}{3.666804in}}%
\pgfpathlineto{\pgfqpoint{1.920726in}{3.662368in}}%
\pgfpathlineto{\pgfqpoint{1.920726in}{3.839226in}}%
\pgfpathlineto{\pgfqpoint{1.929596in}{3.839226in}}%
\pgfusepath{fill}%
\end{pgfscope}%
\begin{pgfscope}%
\pgfpathrectangle{\pgfqpoint{1.432000in}{0.528000in}}{\pgfqpoint{3.696000in}{3.696000in}} %
\pgfusepath{clip}%
\pgfsetbuttcap%
\pgfsetroundjoin%
\definecolor{currentfill}{rgb}{0.271305,0.019942,0.347269}%
\pgfsetfillcolor{currentfill}%
\pgfsetlinewidth{0.000000pt}%
\definecolor{currentstroke}{rgb}{0.000000,0.000000,0.000000}%
\pgfsetstrokecolor{currentstroke}%
\pgfsetdash{}{0pt}%
\pgfpathmoveto{\pgfqpoint{2.037984in}{3.839226in}}%
\pgfpathlineto{\pgfqpoint{2.037984in}{3.662368in}}%
\pgfpathlineto{\pgfqpoint{2.046854in}{3.666804in}}%
\pgfpathlineto{\pgfqpoint{2.033548in}{3.622452in}}%
\pgfpathlineto{\pgfqpoint{2.020243in}{3.666804in}}%
\pgfpathlineto{\pgfqpoint{2.029113in}{3.662368in}}%
\pgfpathlineto{\pgfqpoint{2.029113in}{3.839226in}}%
\pgfpathlineto{\pgfqpoint{2.037984in}{3.839226in}}%
\pgfusepath{fill}%
\end{pgfscope}%
\begin{pgfscope}%
\pgfpathrectangle{\pgfqpoint{1.432000in}{0.528000in}}{\pgfqpoint{3.696000in}{3.696000in}} %
\pgfusepath{clip}%
\pgfsetbuttcap%
\pgfsetroundjoin%
\definecolor{currentfill}{rgb}{0.277018,0.050344,0.375715}%
\pgfsetfillcolor{currentfill}%
\pgfsetlinewidth{0.000000pt}%
\definecolor{currentstroke}{rgb}{0.000000,0.000000,0.000000}%
\pgfsetstrokecolor{currentstroke}%
\pgfsetdash{}{0pt}%
\pgfpathmoveto{\pgfqpoint{2.254758in}{3.839226in}}%
\pgfpathlineto{\pgfqpoint{2.254758in}{3.770756in}}%
\pgfpathlineto{\pgfqpoint{2.263628in}{3.775191in}}%
\pgfpathlineto{\pgfqpoint{2.250323in}{3.730839in}}%
\pgfpathlineto{\pgfqpoint{2.237017in}{3.775191in}}%
\pgfpathlineto{\pgfqpoint{2.245887in}{3.770756in}}%
\pgfpathlineto{\pgfqpoint{2.245887in}{3.839226in}}%
\pgfpathlineto{\pgfqpoint{2.254758in}{3.839226in}}%
\pgfusepath{fill}%
\end{pgfscope}%
\begin{pgfscope}%
\pgfpathrectangle{\pgfqpoint{1.432000in}{0.528000in}}{\pgfqpoint{3.696000in}{3.696000in}} %
\pgfusepath{clip}%
\pgfsetbuttcap%
\pgfsetroundjoin%
\definecolor{currentfill}{rgb}{0.277941,0.056324,0.381191}%
\pgfsetfillcolor{currentfill}%
\pgfsetlinewidth{0.000000pt}%
\definecolor{currentstroke}{rgb}{0.000000,0.000000,0.000000}%
\pgfsetstrokecolor{currentstroke}%
\pgfsetdash{}{0pt}%
\pgfpathmoveto{\pgfqpoint{2.253459in}{3.842362in}}%
\pgfpathlineto{\pgfqpoint{2.333620in}{3.762200in}}%
\pgfpathlineto{\pgfqpoint{2.336757in}{3.771609in}}%
\pgfpathlineto{\pgfqpoint{2.358710in}{3.730839in}}%
\pgfpathlineto{\pgfqpoint{2.317940in}{3.752792in}}%
\pgfpathlineto{\pgfqpoint{2.327348in}{3.755928in}}%
\pgfpathlineto{\pgfqpoint{2.247186in}{3.836090in}}%
\pgfpathlineto{\pgfqpoint{2.253459in}{3.842362in}}%
\pgfusepath{fill}%
\end{pgfscope}%
\begin{pgfscope}%
\pgfpathrectangle{\pgfqpoint{1.432000in}{0.528000in}}{\pgfqpoint{3.696000in}{3.696000in}} %
\pgfusepath{clip}%
\pgfsetbuttcap%
\pgfsetroundjoin%
\definecolor{currentfill}{rgb}{0.281887,0.150881,0.465405}%
\pgfsetfillcolor{currentfill}%
\pgfsetlinewidth{0.000000pt}%
\definecolor{currentstroke}{rgb}{0.000000,0.000000,0.000000}%
\pgfsetstrokecolor{currentstroke}%
\pgfsetdash{}{0pt}%
\pgfpathmoveto{\pgfqpoint{2.363145in}{3.839226in}}%
\pgfpathlineto{\pgfqpoint{2.363145in}{3.770756in}}%
\pgfpathlineto{\pgfqpoint{2.372015in}{3.775191in}}%
\pgfpathlineto{\pgfqpoint{2.358710in}{3.730839in}}%
\pgfpathlineto{\pgfqpoint{2.345404in}{3.775191in}}%
\pgfpathlineto{\pgfqpoint{2.354274in}{3.770756in}}%
\pgfpathlineto{\pgfqpoint{2.354274in}{3.839226in}}%
\pgfpathlineto{\pgfqpoint{2.363145in}{3.839226in}}%
\pgfusepath{fill}%
\end{pgfscope}%
\begin{pgfscope}%
\pgfpathrectangle{\pgfqpoint{1.432000in}{0.528000in}}{\pgfqpoint{3.696000in}{3.696000in}} %
\pgfusepath{clip}%
\pgfsetbuttcap%
\pgfsetroundjoin%
\definecolor{currentfill}{rgb}{0.282884,0.135920,0.453427}%
\pgfsetfillcolor{currentfill}%
\pgfsetlinewidth{0.000000pt}%
\definecolor{currentstroke}{rgb}{0.000000,0.000000,0.000000}%
\pgfsetstrokecolor{currentstroke}%
\pgfsetdash{}{0pt}%
\pgfpathmoveto{\pgfqpoint{2.361846in}{3.842362in}}%
\pgfpathlineto{\pgfqpoint{2.442007in}{3.762200in}}%
\pgfpathlineto{\pgfqpoint{2.445144in}{3.771609in}}%
\pgfpathlineto{\pgfqpoint{2.467097in}{3.730839in}}%
\pgfpathlineto{\pgfqpoint{2.426327in}{3.752792in}}%
\pgfpathlineto{\pgfqpoint{2.435735in}{3.755928in}}%
\pgfpathlineto{\pgfqpoint{2.355574in}{3.836090in}}%
\pgfpathlineto{\pgfqpoint{2.361846in}{3.842362in}}%
\pgfusepath{fill}%
\end{pgfscope}%
\begin{pgfscope}%
\pgfpathrectangle{\pgfqpoint{1.432000in}{0.528000in}}{\pgfqpoint{3.696000in}{3.696000in}} %
\pgfusepath{clip}%
\pgfsetbuttcap%
\pgfsetroundjoin%
\definecolor{currentfill}{rgb}{0.239346,0.300855,0.540844}%
\pgfsetfillcolor{currentfill}%
\pgfsetlinewidth{0.000000pt}%
\definecolor{currentstroke}{rgb}{0.000000,0.000000,0.000000}%
\pgfsetstrokecolor{currentstroke}%
\pgfsetdash{}{0pt}%
\pgfpathmoveto{\pgfqpoint{2.471532in}{3.839226in}}%
\pgfpathlineto{\pgfqpoint{2.471532in}{3.770756in}}%
\pgfpathlineto{\pgfqpoint{2.480402in}{3.775191in}}%
\pgfpathlineto{\pgfqpoint{2.467097in}{3.730839in}}%
\pgfpathlineto{\pgfqpoint{2.453791in}{3.775191in}}%
\pgfpathlineto{\pgfqpoint{2.462662in}{3.770756in}}%
\pgfpathlineto{\pgfqpoint{2.462662in}{3.839226in}}%
\pgfpathlineto{\pgfqpoint{2.471532in}{3.839226in}}%
\pgfusepath{fill}%
\end{pgfscope}%
\begin{pgfscope}%
\pgfpathrectangle{\pgfqpoint{1.432000in}{0.528000in}}{\pgfqpoint{3.696000in}{3.696000in}} %
\pgfusepath{clip}%
\pgfsetbuttcap%
\pgfsetroundjoin%
\definecolor{currentfill}{rgb}{0.269944,0.014625,0.341379}%
\pgfsetfillcolor{currentfill}%
\pgfsetlinewidth{0.000000pt}%
\definecolor{currentstroke}{rgb}{0.000000,0.000000,0.000000}%
\pgfsetstrokecolor{currentstroke}%
\pgfsetdash{}{0pt}%
\pgfpathmoveto{\pgfqpoint{2.579919in}{3.839226in}}%
\pgfpathlineto{\pgfqpoint{2.579919in}{3.770756in}}%
\pgfpathlineto{\pgfqpoint{2.588789in}{3.775191in}}%
\pgfpathlineto{\pgfqpoint{2.575484in}{3.730839in}}%
\pgfpathlineto{\pgfqpoint{2.562178in}{3.775191in}}%
\pgfpathlineto{\pgfqpoint{2.571049in}{3.770756in}}%
\pgfpathlineto{\pgfqpoint{2.571049in}{3.839226in}}%
\pgfpathlineto{\pgfqpoint{2.579919in}{3.839226in}}%
\pgfusepath{fill}%
\end{pgfscope}%
\begin{pgfscope}%
\pgfpathrectangle{\pgfqpoint{1.432000in}{0.528000in}}{\pgfqpoint{3.696000in}{3.696000in}} %
\pgfusepath{clip}%
\pgfsetbuttcap%
\pgfsetroundjoin%
\definecolor{currentfill}{rgb}{0.182256,0.426184,0.557120}%
\pgfsetfillcolor{currentfill}%
\pgfsetlinewidth{0.000000pt}%
\definecolor{currentstroke}{rgb}{0.000000,0.000000,0.000000}%
\pgfsetstrokecolor{currentstroke}%
\pgfsetdash{}{0pt}%
\pgfpathmoveto{\pgfqpoint{2.579919in}{3.839226in}}%
\pgfpathlineto{\pgfqpoint{2.577701in}{3.843067in}}%
\pgfpathlineto{\pgfqpoint{2.573266in}{3.843067in}}%
\pgfpathlineto{\pgfqpoint{2.571049in}{3.839226in}}%
\pgfpathlineto{\pgfqpoint{2.573266in}{3.835385in}}%
\pgfpathlineto{\pgfqpoint{2.577701in}{3.835385in}}%
\pgfpathlineto{\pgfqpoint{2.579919in}{3.839226in}}%
\pgfpathlineto{\pgfqpoint{2.577701in}{3.843067in}}%
\pgfusepath{fill}%
\end{pgfscope}%
\begin{pgfscope}%
\pgfpathrectangle{\pgfqpoint{1.432000in}{0.528000in}}{\pgfqpoint{3.696000in}{3.696000in}} %
\pgfusepath{clip}%
\pgfsetbuttcap%
\pgfsetroundjoin%
\definecolor{currentfill}{rgb}{0.282656,0.100196,0.422160}%
\pgfsetfillcolor{currentfill}%
\pgfsetlinewidth{0.000000pt}%
\definecolor{currentstroke}{rgb}{0.000000,0.000000,0.000000}%
\pgfsetstrokecolor{currentstroke}%
\pgfsetdash{}{0pt}%
\pgfpathmoveto{\pgfqpoint{2.688306in}{3.839226in}}%
\pgfpathlineto{\pgfqpoint{2.688306in}{3.770756in}}%
\pgfpathlineto{\pgfqpoint{2.697177in}{3.775191in}}%
\pgfpathlineto{\pgfqpoint{2.683871in}{3.730839in}}%
\pgfpathlineto{\pgfqpoint{2.670565in}{3.775191in}}%
\pgfpathlineto{\pgfqpoint{2.679436in}{3.770756in}}%
\pgfpathlineto{\pgfqpoint{2.679436in}{3.839226in}}%
\pgfpathlineto{\pgfqpoint{2.688306in}{3.839226in}}%
\pgfusepath{fill}%
\end{pgfscope}%
\begin{pgfscope}%
\pgfpathrectangle{\pgfqpoint{1.432000in}{0.528000in}}{\pgfqpoint{3.696000in}{3.696000in}} %
\pgfusepath{clip}%
\pgfsetbuttcap%
\pgfsetroundjoin%
\definecolor{currentfill}{rgb}{0.214298,0.355619,0.551184}%
\pgfsetfillcolor{currentfill}%
\pgfsetlinewidth{0.000000pt}%
\definecolor{currentstroke}{rgb}{0.000000,0.000000,0.000000}%
\pgfsetstrokecolor{currentstroke}%
\pgfsetdash{}{0pt}%
\pgfpathmoveto{\pgfqpoint{2.688306in}{3.839226in}}%
\pgfpathlineto{\pgfqpoint{2.686089in}{3.843067in}}%
\pgfpathlineto{\pgfqpoint{2.681653in}{3.843067in}}%
\pgfpathlineto{\pgfqpoint{2.679436in}{3.839226in}}%
\pgfpathlineto{\pgfqpoint{2.681653in}{3.835385in}}%
\pgfpathlineto{\pgfqpoint{2.686089in}{3.835385in}}%
\pgfpathlineto{\pgfqpoint{2.688306in}{3.839226in}}%
\pgfpathlineto{\pgfqpoint{2.686089in}{3.843067in}}%
\pgfusepath{fill}%
\end{pgfscope}%
\begin{pgfscope}%
\pgfpathrectangle{\pgfqpoint{1.432000in}{0.528000in}}{\pgfqpoint{3.696000in}{3.696000in}} %
\pgfusepath{clip}%
\pgfsetbuttcap%
\pgfsetroundjoin%
\definecolor{currentfill}{rgb}{0.272594,0.025563,0.353093}%
\pgfsetfillcolor{currentfill}%
\pgfsetlinewidth{0.000000pt}%
\definecolor{currentstroke}{rgb}{0.000000,0.000000,0.000000}%
\pgfsetstrokecolor{currentstroke}%
\pgfsetdash{}{0pt}%
\pgfpathmoveto{\pgfqpoint{2.796693in}{3.839226in}}%
\pgfpathlineto{\pgfqpoint{2.796693in}{3.770756in}}%
\pgfpathlineto{\pgfqpoint{2.805564in}{3.775191in}}%
\pgfpathlineto{\pgfqpoint{2.792258in}{3.730839in}}%
\pgfpathlineto{\pgfqpoint{2.778952in}{3.775191in}}%
\pgfpathlineto{\pgfqpoint{2.787823in}{3.770756in}}%
\pgfpathlineto{\pgfqpoint{2.787823in}{3.839226in}}%
\pgfpathlineto{\pgfqpoint{2.796693in}{3.839226in}}%
\pgfusepath{fill}%
\end{pgfscope}%
\begin{pgfscope}%
\pgfpathrectangle{\pgfqpoint{1.432000in}{0.528000in}}{\pgfqpoint{3.696000in}{3.696000in}} %
\pgfusepath{clip}%
\pgfsetbuttcap%
\pgfsetroundjoin%
\definecolor{currentfill}{rgb}{0.271828,0.209303,0.504434}%
\pgfsetfillcolor{currentfill}%
\pgfsetlinewidth{0.000000pt}%
\definecolor{currentstroke}{rgb}{0.000000,0.000000,0.000000}%
\pgfsetstrokecolor{currentstroke}%
\pgfsetdash{}{0pt}%
\pgfpathmoveto{\pgfqpoint{2.796693in}{3.839226in}}%
\pgfpathlineto{\pgfqpoint{2.794476in}{3.843067in}}%
\pgfpathlineto{\pgfqpoint{2.790040in}{3.843067in}}%
\pgfpathlineto{\pgfqpoint{2.787823in}{3.839226in}}%
\pgfpathlineto{\pgfqpoint{2.790040in}{3.835385in}}%
\pgfpathlineto{\pgfqpoint{2.794476in}{3.835385in}}%
\pgfpathlineto{\pgfqpoint{2.796693in}{3.839226in}}%
\pgfpathlineto{\pgfqpoint{2.794476in}{3.843067in}}%
\pgfusepath{fill}%
\end{pgfscope}%
\begin{pgfscope}%
\pgfpathrectangle{\pgfqpoint{1.432000in}{0.528000in}}{\pgfqpoint{3.696000in}{3.696000in}} %
\pgfusepath{clip}%
\pgfsetbuttcap%
\pgfsetroundjoin%
\definecolor{currentfill}{rgb}{0.277941,0.056324,0.381191}%
\pgfsetfillcolor{currentfill}%
\pgfsetlinewidth{0.000000pt}%
\definecolor{currentstroke}{rgb}{0.000000,0.000000,0.000000}%
\pgfsetstrokecolor{currentstroke}%
\pgfsetdash{}{0pt}%
\pgfpathmoveto{\pgfqpoint{2.903781in}{3.842362in}}%
\pgfpathlineto{\pgfqpoint{2.983943in}{3.762200in}}%
\pgfpathlineto{\pgfqpoint{2.987079in}{3.771609in}}%
\pgfpathlineto{\pgfqpoint{3.009032in}{3.730839in}}%
\pgfpathlineto{\pgfqpoint{2.968262in}{3.752792in}}%
\pgfpathlineto{\pgfqpoint{2.977671in}{3.755928in}}%
\pgfpathlineto{\pgfqpoint{2.897509in}{3.836090in}}%
\pgfpathlineto{\pgfqpoint{2.903781in}{3.842362in}}%
\pgfusepath{fill}%
\end{pgfscope}%
\begin{pgfscope}%
\pgfpathrectangle{\pgfqpoint{1.432000in}{0.528000in}}{\pgfqpoint{3.696000in}{3.696000in}} %
\pgfusepath{clip}%
\pgfsetbuttcap%
\pgfsetroundjoin%
\definecolor{currentfill}{rgb}{0.272594,0.025563,0.353093}%
\pgfsetfillcolor{currentfill}%
\pgfsetlinewidth{0.000000pt}%
\definecolor{currentstroke}{rgb}{0.000000,0.000000,0.000000}%
\pgfsetstrokecolor{currentstroke}%
\pgfsetdash{}{0pt}%
\pgfpathmoveto{\pgfqpoint{2.905080in}{3.839226in}}%
\pgfpathlineto{\pgfqpoint{2.902863in}{3.843067in}}%
\pgfpathlineto{\pgfqpoint{2.898428in}{3.843067in}}%
\pgfpathlineto{\pgfqpoint{2.896210in}{3.839226in}}%
\pgfpathlineto{\pgfqpoint{2.898428in}{3.835385in}}%
\pgfpathlineto{\pgfqpoint{2.902863in}{3.835385in}}%
\pgfpathlineto{\pgfqpoint{2.905080in}{3.839226in}}%
\pgfpathlineto{\pgfqpoint{2.902863in}{3.843067in}}%
\pgfusepath{fill}%
\end{pgfscope}%
\begin{pgfscope}%
\pgfpathrectangle{\pgfqpoint{1.432000in}{0.528000in}}{\pgfqpoint{3.696000in}{3.696000in}} %
\pgfusepath{clip}%
\pgfsetbuttcap%
\pgfsetroundjoin%
\definecolor{currentfill}{rgb}{0.279574,0.170599,0.479997}%
\pgfsetfillcolor{currentfill}%
\pgfsetlinewidth{0.000000pt}%
\definecolor{currentstroke}{rgb}{0.000000,0.000000,0.000000}%
\pgfsetstrokecolor{currentstroke}%
\pgfsetdash{}{0pt}%
\pgfpathmoveto{\pgfqpoint{2.900645in}{3.843661in}}%
\pgfpathlineto{\pgfqpoint{2.969115in}{3.843661in}}%
\pgfpathlineto{\pgfqpoint{2.964680in}{3.852531in}}%
\pgfpathlineto{\pgfqpoint{3.009032in}{3.839226in}}%
\pgfpathlineto{\pgfqpoint{2.964680in}{3.825920in}}%
\pgfpathlineto{\pgfqpoint{2.969115in}{3.834791in}}%
\pgfpathlineto{\pgfqpoint{2.900645in}{3.834791in}}%
\pgfpathlineto{\pgfqpoint{2.900645in}{3.843661in}}%
\pgfusepath{fill}%
\end{pgfscope}%
\begin{pgfscope}%
\pgfpathrectangle{\pgfqpoint{1.432000in}{0.528000in}}{\pgfqpoint{3.696000in}{3.696000in}} %
\pgfusepath{clip}%
\pgfsetbuttcap%
\pgfsetroundjoin%
\definecolor{currentfill}{rgb}{0.272594,0.025563,0.353093}%
\pgfsetfillcolor{currentfill}%
\pgfsetlinewidth{0.000000pt}%
\definecolor{currentstroke}{rgb}{0.000000,0.000000,0.000000}%
\pgfsetstrokecolor{currentstroke}%
\pgfsetdash{}{0pt}%
\pgfpathmoveto{\pgfqpoint{3.012168in}{3.842362in}}%
\pgfpathlineto{\pgfqpoint{3.092330in}{3.762200in}}%
\pgfpathlineto{\pgfqpoint{3.095466in}{3.771609in}}%
\pgfpathlineto{\pgfqpoint{3.117419in}{3.730839in}}%
\pgfpathlineto{\pgfqpoint{3.076649in}{3.752792in}}%
\pgfpathlineto{\pgfqpoint{3.086058in}{3.755928in}}%
\pgfpathlineto{\pgfqpoint{3.005896in}{3.836090in}}%
\pgfpathlineto{\pgfqpoint{3.012168in}{3.842362in}}%
\pgfusepath{fill}%
\end{pgfscope}%
\begin{pgfscope}%
\pgfpathrectangle{\pgfqpoint{1.432000in}{0.528000in}}{\pgfqpoint{3.696000in}{3.696000in}} %
\pgfusepath{clip}%
\pgfsetbuttcap%
\pgfsetroundjoin%
\definecolor{currentfill}{rgb}{0.262138,0.242286,0.520837}%
\pgfsetfillcolor{currentfill}%
\pgfsetlinewidth{0.000000pt}%
\definecolor{currentstroke}{rgb}{0.000000,0.000000,0.000000}%
\pgfsetstrokecolor{currentstroke}%
\pgfsetdash{}{0pt}%
\pgfpathmoveto{\pgfqpoint{3.009032in}{3.843661in}}%
\pgfpathlineto{\pgfqpoint{3.077503in}{3.843661in}}%
\pgfpathlineto{\pgfqpoint{3.073067in}{3.852531in}}%
\pgfpathlineto{\pgfqpoint{3.117419in}{3.839226in}}%
\pgfpathlineto{\pgfqpoint{3.073067in}{3.825920in}}%
\pgfpathlineto{\pgfqpoint{3.077503in}{3.834791in}}%
\pgfpathlineto{\pgfqpoint{3.009032in}{3.834791in}}%
\pgfpathlineto{\pgfqpoint{3.009032in}{3.843661in}}%
\pgfusepath{fill}%
\end{pgfscope}%
\begin{pgfscope}%
\pgfpathrectangle{\pgfqpoint{1.432000in}{0.528000in}}{\pgfqpoint{3.696000in}{3.696000in}} %
\pgfusepath{clip}%
\pgfsetbuttcap%
\pgfsetroundjoin%
\definecolor{currentfill}{rgb}{0.279566,0.067836,0.391917}%
\pgfsetfillcolor{currentfill}%
\pgfsetlinewidth{0.000000pt}%
\definecolor{currentstroke}{rgb}{0.000000,0.000000,0.000000}%
\pgfsetstrokecolor{currentstroke}%
\pgfsetdash{}{0pt}%
\pgfpathmoveto{\pgfqpoint{3.120556in}{3.842362in}}%
\pgfpathlineto{\pgfqpoint{3.200717in}{3.762200in}}%
\pgfpathlineto{\pgfqpoint{3.203853in}{3.771609in}}%
\pgfpathlineto{\pgfqpoint{3.225806in}{3.730839in}}%
\pgfpathlineto{\pgfqpoint{3.185036in}{3.752792in}}%
\pgfpathlineto{\pgfqpoint{3.194445in}{3.755928in}}%
\pgfpathlineto{\pgfqpoint{3.114283in}{3.836090in}}%
\pgfpathlineto{\pgfqpoint{3.120556in}{3.842362in}}%
\pgfusepath{fill}%
\end{pgfscope}%
\begin{pgfscope}%
\pgfpathrectangle{\pgfqpoint{1.432000in}{0.528000in}}{\pgfqpoint{3.696000in}{3.696000in}} %
\pgfusepath{clip}%
\pgfsetbuttcap%
\pgfsetroundjoin%
\definecolor{currentfill}{rgb}{0.271828,0.209303,0.504434}%
\pgfsetfillcolor{currentfill}%
\pgfsetlinewidth{0.000000pt}%
\definecolor{currentstroke}{rgb}{0.000000,0.000000,0.000000}%
\pgfsetstrokecolor{currentstroke}%
\pgfsetdash{}{0pt}%
\pgfpathmoveto{\pgfqpoint{3.117419in}{3.843661in}}%
\pgfpathlineto{\pgfqpoint{3.185890in}{3.843661in}}%
\pgfpathlineto{\pgfqpoint{3.181454in}{3.852531in}}%
\pgfpathlineto{\pgfqpoint{3.225806in}{3.839226in}}%
\pgfpathlineto{\pgfqpoint{3.181454in}{3.825920in}}%
\pgfpathlineto{\pgfqpoint{3.185890in}{3.834791in}}%
\pgfpathlineto{\pgfqpoint{3.117419in}{3.834791in}}%
\pgfpathlineto{\pgfqpoint{3.117419in}{3.843661in}}%
\pgfusepath{fill}%
\end{pgfscope}%
\begin{pgfscope}%
\pgfpathrectangle{\pgfqpoint{1.432000in}{0.528000in}}{\pgfqpoint{3.696000in}{3.696000in}} %
\pgfusepath{clip}%
\pgfsetbuttcap%
\pgfsetroundjoin%
\definecolor{currentfill}{rgb}{0.283197,0.115680,0.436115}%
\pgfsetfillcolor{currentfill}%
\pgfsetlinewidth{0.000000pt}%
\definecolor{currentstroke}{rgb}{0.000000,0.000000,0.000000}%
\pgfsetstrokecolor{currentstroke}%
\pgfsetdash{}{0pt}%
\pgfpathmoveto{\pgfqpoint{3.228943in}{3.842362in}}%
\pgfpathlineto{\pgfqpoint{3.309104in}{3.762200in}}%
\pgfpathlineto{\pgfqpoint{3.312240in}{3.771609in}}%
\pgfpathlineto{\pgfqpoint{3.334194in}{3.730839in}}%
\pgfpathlineto{\pgfqpoint{3.293423in}{3.752792in}}%
\pgfpathlineto{\pgfqpoint{3.302832in}{3.755928in}}%
\pgfpathlineto{\pgfqpoint{3.222670in}{3.836090in}}%
\pgfpathlineto{\pgfqpoint{3.228943in}{3.842362in}}%
\pgfusepath{fill}%
\end{pgfscope}%
\begin{pgfscope}%
\pgfpathrectangle{\pgfqpoint{1.432000in}{0.528000in}}{\pgfqpoint{3.696000in}{3.696000in}} %
\pgfusepath{clip}%
\pgfsetbuttcap%
\pgfsetroundjoin%
\definecolor{currentfill}{rgb}{0.278791,0.062145,0.386592}%
\pgfsetfillcolor{currentfill}%
\pgfsetlinewidth{0.000000pt}%
\definecolor{currentstroke}{rgb}{0.000000,0.000000,0.000000}%
\pgfsetstrokecolor{currentstroke}%
\pgfsetdash{}{0pt}%
\pgfpathmoveto{\pgfqpoint{3.225806in}{3.843661in}}%
\pgfpathlineto{\pgfqpoint{3.294277in}{3.843661in}}%
\pgfpathlineto{\pgfqpoint{3.289842in}{3.852531in}}%
\pgfpathlineto{\pgfqpoint{3.334194in}{3.839226in}}%
\pgfpathlineto{\pgfqpoint{3.289842in}{3.825920in}}%
\pgfpathlineto{\pgfqpoint{3.294277in}{3.834791in}}%
\pgfpathlineto{\pgfqpoint{3.225806in}{3.834791in}}%
\pgfpathlineto{\pgfqpoint{3.225806in}{3.843661in}}%
\pgfusepath{fill}%
\end{pgfscope}%
\begin{pgfscope}%
\pgfpathrectangle{\pgfqpoint{1.432000in}{0.528000in}}{\pgfqpoint{3.696000in}{3.696000in}} %
\pgfusepath{clip}%
\pgfsetbuttcap%
\pgfsetroundjoin%
\definecolor{currentfill}{rgb}{0.276022,0.044167,0.370164}%
\pgfsetfillcolor{currentfill}%
\pgfsetlinewidth{0.000000pt}%
\definecolor{currentstroke}{rgb}{0.000000,0.000000,0.000000}%
\pgfsetstrokecolor{currentstroke}%
\pgfsetdash{}{0pt}%
\pgfpathmoveto{\pgfqpoint{3.447016in}{3.839226in}}%
\pgfpathlineto{\pgfqpoint{3.444798in}{3.843067in}}%
\pgfpathlineto{\pgfqpoint{3.440363in}{3.843067in}}%
\pgfpathlineto{\pgfqpoint{3.438145in}{3.839226in}}%
\pgfpathlineto{\pgfqpoint{3.440363in}{3.835385in}}%
\pgfpathlineto{\pgfqpoint{3.444798in}{3.835385in}}%
\pgfpathlineto{\pgfqpoint{3.447016in}{3.839226in}}%
\pgfpathlineto{\pgfqpoint{3.444798in}{3.843067in}}%
\pgfusepath{fill}%
\end{pgfscope}%
\begin{pgfscope}%
\pgfpathrectangle{\pgfqpoint{1.432000in}{0.528000in}}{\pgfqpoint{3.696000in}{3.696000in}} %
\pgfusepath{clip}%
\pgfsetbuttcap%
\pgfsetroundjoin%
\definecolor{currentfill}{rgb}{0.283229,0.120777,0.440584}%
\pgfsetfillcolor{currentfill}%
\pgfsetlinewidth{0.000000pt}%
\definecolor{currentstroke}{rgb}{0.000000,0.000000,0.000000}%
\pgfsetstrokecolor{currentstroke}%
\pgfsetdash{}{0pt}%
\pgfpathmoveto{\pgfqpoint{3.555403in}{3.839226in}}%
\pgfpathlineto{\pgfqpoint{3.553185in}{3.843067in}}%
\pgfpathlineto{\pgfqpoint{3.548750in}{3.843067in}}%
\pgfpathlineto{\pgfqpoint{3.546533in}{3.839226in}}%
\pgfpathlineto{\pgfqpoint{3.548750in}{3.835385in}}%
\pgfpathlineto{\pgfqpoint{3.553185in}{3.835385in}}%
\pgfpathlineto{\pgfqpoint{3.555403in}{3.839226in}}%
\pgfpathlineto{\pgfqpoint{3.553185in}{3.843067in}}%
\pgfusepath{fill}%
\end{pgfscope}%
\begin{pgfscope}%
\pgfpathrectangle{\pgfqpoint{1.432000in}{0.528000in}}{\pgfqpoint{3.696000in}{3.696000in}} %
\pgfusepath{clip}%
\pgfsetbuttcap%
\pgfsetroundjoin%
\definecolor{currentfill}{rgb}{0.197636,0.391528,0.554969}%
\pgfsetfillcolor{currentfill}%
\pgfsetlinewidth{0.000000pt}%
\definecolor{currentstroke}{rgb}{0.000000,0.000000,0.000000}%
\pgfsetstrokecolor{currentstroke}%
\pgfsetdash{}{0pt}%
\pgfpathmoveto{\pgfqpoint{3.659355in}{3.834791in}}%
\pgfpathlineto{\pgfqpoint{3.590885in}{3.834791in}}%
\pgfpathlineto{\pgfqpoint{3.595320in}{3.825920in}}%
\pgfpathlineto{\pgfqpoint{3.550968in}{3.839226in}}%
\pgfpathlineto{\pgfqpoint{3.595320in}{3.852531in}}%
\pgfpathlineto{\pgfqpoint{3.590885in}{3.843661in}}%
\pgfpathlineto{\pgfqpoint{3.659355in}{3.843661in}}%
\pgfpathlineto{\pgfqpoint{3.659355in}{3.834791in}}%
\pgfusepath{fill}%
\end{pgfscope}%
\begin{pgfscope}%
\pgfpathrectangle{\pgfqpoint{1.432000in}{0.528000in}}{\pgfqpoint{3.696000in}{3.696000in}} %
\pgfusepath{clip}%
\pgfsetbuttcap%
\pgfsetroundjoin%
\definecolor{currentfill}{rgb}{0.275191,0.194905,0.496005}%
\pgfsetfillcolor{currentfill}%
\pgfsetlinewidth{0.000000pt}%
\definecolor{currentstroke}{rgb}{0.000000,0.000000,0.000000}%
\pgfsetstrokecolor{currentstroke}%
\pgfsetdash{}{0pt}%
\pgfpathmoveto{\pgfqpoint{3.764606in}{3.836090in}}%
\pgfpathlineto{\pgfqpoint{3.684444in}{3.916251in}}%
\pgfpathlineto{\pgfqpoint{3.681308in}{3.906843in}}%
\pgfpathlineto{\pgfqpoint{3.659355in}{3.947613in}}%
\pgfpathlineto{\pgfqpoint{3.700125in}{3.925660in}}%
\pgfpathlineto{\pgfqpoint{3.690716in}{3.922524in}}%
\pgfpathlineto{\pgfqpoint{3.770878in}{3.842362in}}%
\pgfpathlineto{\pgfqpoint{3.764606in}{3.836090in}}%
\pgfusepath{fill}%
\end{pgfscope}%
\begin{pgfscope}%
\pgfpathrectangle{\pgfqpoint{1.432000in}{0.528000in}}{\pgfqpoint{3.696000in}{3.696000in}} %
\pgfusepath{clip}%
\pgfsetbuttcap%
\pgfsetroundjoin%
\definecolor{currentfill}{rgb}{0.229739,0.322361,0.545706}%
\pgfsetfillcolor{currentfill}%
\pgfsetlinewidth{0.000000pt}%
\definecolor{currentstroke}{rgb}{0.000000,0.000000,0.000000}%
\pgfsetstrokecolor{currentstroke}%
\pgfsetdash{}{0pt}%
\pgfpathmoveto{\pgfqpoint{3.872993in}{3.836090in}}%
\pgfpathlineto{\pgfqpoint{3.792831in}{3.916251in}}%
\pgfpathlineto{\pgfqpoint{3.789695in}{3.906843in}}%
\pgfpathlineto{\pgfqpoint{3.767742in}{3.947613in}}%
\pgfpathlineto{\pgfqpoint{3.808512in}{3.925660in}}%
\pgfpathlineto{\pgfqpoint{3.799104in}{3.922524in}}%
\pgfpathlineto{\pgfqpoint{3.879265in}{3.842362in}}%
\pgfpathlineto{\pgfqpoint{3.872993in}{3.836090in}}%
\pgfusepath{fill}%
\end{pgfscope}%
\begin{pgfscope}%
\pgfpathrectangle{\pgfqpoint{1.432000in}{0.528000in}}{\pgfqpoint{3.696000in}{3.696000in}} %
\pgfusepath{clip}%
\pgfsetbuttcap%
\pgfsetroundjoin%
\definecolor{currentfill}{rgb}{0.165117,0.467423,0.558141}%
\pgfsetfillcolor{currentfill}%
\pgfsetlinewidth{0.000000pt}%
\definecolor{currentstroke}{rgb}{0.000000,0.000000,0.000000}%
\pgfsetstrokecolor{currentstroke}%
\pgfsetdash{}{0pt}%
\pgfpathmoveto{\pgfqpoint{3.982533in}{3.835259in}}%
\pgfpathlineto{\pgfqpoint{3.801461in}{3.925795in}}%
\pgfpathlineto{\pgfqpoint{3.801461in}{3.915877in}}%
\pgfpathlineto{\pgfqpoint{3.767742in}{3.947613in}}%
\pgfpathlineto{\pgfqpoint{3.813362in}{3.939679in}}%
\pgfpathlineto{\pgfqpoint{3.805428in}{3.933729in}}%
\pgfpathlineto{\pgfqpoint{3.986500in}{3.843193in}}%
\pgfpathlineto{\pgfqpoint{3.982533in}{3.835259in}}%
\pgfusepath{fill}%
\end{pgfscope}%
\begin{pgfscope}%
\pgfpathrectangle{\pgfqpoint{1.432000in}{0.528000in}}{\pgfqpoint{3.696000in}{3.696000in}} %
\pgfusepath{clip}%
\pgfsetbuttcap%
\pgfsetroundjoin%
\definecolor{currentfill}{rgb}{0.165117,0.467423,0.558141}%
\pgfsetfillcolor{currentfill}%
\pgfsetlinewidth{0.000000pt}%
\definecolor{currentstroke}{rgb}{0.000000,0.000000,0.000000}%
\pgfsetstrokecolor{currentstroke}%
\pgfsetdash{}{0pt}%
\pgfpathmoveto{\pgfqpoint{4.090920in}{3.835259in}}%
\pgfpathlineto{\pgfqpoint{3.909848in}{3.925795in}}%
\pgfpathlineto{\pgfqpoint{3.909848in}{3.915877in}}%
\pgfpathlineto{\pgfqpoint{3.876129in}{3.947613in}}%
\pgfpathlineto{\pgfqpoint{3.921749in}{3.939679in}}%
\pgfpathlineto{\pgfqpoint{3.913815in}{3.933729in}}%
\pgfpathlineto{\pgfqpoint{4.094887in}{3.843193in}}%
\pgfpathlineto{\pgfqpoint{4.090920in}{3.835259in}}%
\pgfusepath{fill}%
\end{pgfscope}%
\begin{pgfscope}%
\pgfpathrectangle{\pgfqpoint{1.432000in}{0.528000in}}{\pgfqpoint{3.696000in}{3.696000in}} %
\pgfusepath{clip}%
\pgfsetbuttcap%
\pgfsetroundjoin%
\definecolor{currentfill}{rgb}{0.277941,0.056324,0.381191}%
\pgfsetfillcolor{currentfill}%
\pgfsetlinewidth{0.000000pt}%
\definecolor{currentstroke}{rgb}{0.000000,0.000000,0.000000}%
\pgfsetstrokecolor{currentstroke}%
\pgfsetdash{}{0pt}%
\pgfpathmoveto{\pgfqpoint{4.199888in}{3.835018in}}%
\pgfpathlineto{\pgfqpoint{3.912595in}{3.930783in}}%
\pgfpathlineto{\pgfqpoint{3.913997in}{3.920965in}}%
\pgfpathlineto{\pgfqpoint{3.876129in}{3.947613in}}%
\pgfpathlineto{\pgfqpoint{3.922413in}{3.946210in}}%
\pgfpathlineto{\pgfqpoint{3.915400in}{3.939198in}}%
\pgfpathlineto{\pgfqpoint{4.202693in}{3.843433in}}%
\pgfpathlineto{\pgfqpoint{4.199888in}{3.835018in}}%
\pgfusepath{fill}%
\end{pgfscope}%
\begin{pgfscope}%
\pgfpathrectangle{\pgfqpoint{1.432000in}{0.528000in}}{\pgfqpoint{3.696000in}{3.696000in}} %
\pgfusepath{clip}%
\pgfsetbuttcap%
\pgfsetroundjoin%
\definecolor{currentfill}{rgb}{0.250425,0.274290,0.533103}%
\pgfsetfillcolor{currentfill}%
\pgfsetlinewidth{0.000000pt}%
\definecolor{currentstroke}{rgb}{0.000000,0.000000,0.000000}%
\pgfsetstrokecolor{currentstroke}%
\pgfsetdash{}{0pt}%
\pgfpathmoveto{\pgfqpoint{4.199307in}{3.835259in}}%
\pgfpathlineto{\pgfqpoint{4.018235in}{3.925795in}}%
\pgfpathlineto{\pgfqpoint{4.018235in}{3.915877in}}%
\pgfpathlineto{\pgfqpoint{3.984516in}{3.947613in}}%
\pgfpathlineto{\pgfqpoint{4.030136in}{3.939679in}}%
\pgfpathlineto{\pgfqpoint{4.022202in}{3.933729in}}%
\pgfpathlineto{\pgfqpoint{4.203274in}{3.843193in}}%
\pgfpathlineto{\pgfqpoint{4.199307in}{3.835259in}}%
\pgfusepath{fill}%
\end{pgfscope}%
\begin{pgfscope}%
\pgfpathrectangle{\pgfqpoint{1.432000in}{0.528000in}}{\pgfqpoint{3.696000in}{3.696000in}} %
\pgfusepath{clip}%
\pgfsetbuttcap%
\pgfsetroundjoin%
\definecolor{currentfill}{rgb}{0.283091,0.110553,0.431554}%
\pgfsetfillcolor{currentfill}%
\pgfsetlinewidth{0.000000pt}%
\definecolor{currentstroke}{rgb}{0.000000,0.000000,0.000000}%
\pgfsetstrokecolor{currentstroke}%
\pgfsetdash{}{0pt}%
\pgfpathmoveto{\pgfqpoint{4.308275in}{3.835018in}}%
\pgfpathlineto{\pgfqpoint{4.020982in}{3.930783in}}%
\pgfpathlineto{\pgfqpoint{4.022385in}{3.920965in}}%
\pgfpathlineto{\pgfqpoint{3.984516in}{3.947613in}}%
\pgfpathlineto{\pgfqpoint{4.030800in}{3.946210in}}%
\pgfpathlineto{\pgfqpoint{4.023787in}{3.939198in}}%
\pgfpathlineto{\pgfqpoint{4.311080in}{3.843433in}}%
\pgfpathlineto{\pgfqpoint{4.308275in}{3.835018in}}%
\pgfusepath{fill}%
\end{pgfscope}%
\begin{pgfscope}%
\pgfpathrectangle{\pgfqpoint{1.432000in}{0.528000in}}{\pgfqpoint{3.696000in}{3.696000in}} %
\pgfusepath{clip}%
\pgfsetbuttcap%
\pgfsetroundjoin%
\definecolor{currentfill}{rgb}{0.273809,0.031497,0.358853}%
\pgfsetfillcolor{currentfill}%
\pgfsetlinewidth{0.000000pt}%
\definecolor{currentstroke}{rgb}{0.000000,0.000000,0.000000}%
\pgfsetstrokecolor{currentstroke}%
\pgfsetdash{}{0pt}%
\pgfpathmoveto{\pgfqpoint{4.307694in}{3.835259in}}%
\pgfpathlineto{\pgfqpoint{4.126622in}{3.925795in}}%
\pgfpathlineto{\pgfqpoint{4.126622in}{3.915877in}}%
\pgfpathlineto{\pgfqpoint{4.092903in}{3.947613in}}%
\pgfpathlineto{\pgfqpoint{4.138523in}{3.939679in}}%
\pgfpathlineto{\pgfqpoint{4.130589in}{3.933729in}}%
\pgfpathlineto{\pgfqpoint{4.311661in}{3.843193in}}%
\pgfpathlineto{\pgfqpoint{4.307694in}{3.835259in}}%
\pgfusepath{fill}%
\end{pgfscope}%
\begin{pgfscope}%
\pgfpathrectangle{\pgfqpoint{1.432000in}{0.528000in}}{\pgfqpoint{3.696000in}{3.696000in}} %
\pgfusepath{clip}%
\pgfsetbuttcap%
\pgfsetroundjoin%
\definecolor{currentfill}{rgb}{0.281887,0.150881,0.465405}%
\pgfsetfillcolor{currentfill}%
\pgfsetlinewidth{0.000000pt}%
\definecolor{currentstroke}{rgb}{0.000000,0.000000,0.000000}%
\pgfsetstrokecolor{currentstroke}%
\pgfsetdash{}{0pt}%
\pgfpathmoveto{\pgfqpoint{4.416662in}{3.835018in}}%
\pgfpathlineto{\pgfqpoint{4.129369in}{3.930783in}}%
\pgfpathlineto{\pgfqpoint{4.130772in}{3.920965in}}%
\pgfpathlineto{\pgfqpoint{4.092903in}{3.947613in}}%
\pgfpathlineto{\pgfqpoint{4.139187in}{3.946210in}}%
\pgfpathlineto{\pgfqpoint{4.132174in}{3.939198in}}%
\pgfpathlineto{\pgfqpoint{4.419467in}{3.843433in}}%
\pgfpathlineto{\pgfqpoint{4.416662in}{3.835018in}}%
\pgfusepath{fill}%
\end{pgfscope}%
\begin{pgfscope}%
\pgfpathrectangle{\pgfqpoint{1.432000in}{0.528000in}}{\pgfqpoint{3.696000in}{3.696000in}} %
\pgfusepath{clip}%
\pgfsetbuttcap%
\pgfsetroundjoin%
\definecolor{currentfill}{rgb}{0.278791,0.062145,0.386592}%
\pgfsetfillcolor{currentfill}%
\pgfsetlinewidth{0.000000pt}%
\definecolor{currentstroke}{rgb}{0.000000,0.000000,0.000000}%
\pgfsetstrokecolor{currentstroke}%
\pgfsetdash{}{0pt}%
\pgfpathmoveto{\pgfqpoint{4.416081in}{3.835259in}}%
\pgfpathlineto{\pgfqpoint{4.235010in}{3.925795in}}%
\pgfpathlineto{\pgfqpoint{4.235010in}{3.915877in}}%
\pgfpathlineto{\pgfqpoint{4.201290in}{3.947613in}}%
\pgfpathlineto{\pgfqpoint{4.246910in}{3.939679in}}%
\pgfpathlineto{\pgfqpoint{4.238976in}{3.933729in}}%
\pgfpathlineto{\pgfqpoint{4.420048in}{3.843193in}}%
\pgfpathlineto{\pgfqpoint{4.416081in}{3.835259in}}%
\pgfusepath{fill}%
\end{pgfscope}%
\begin{pgfscope}%
\pgfpathrectangle{\pgfqpoint{1.432000in}{0.528000in}}{\pgfqpoint{3.696000in}{3.696000in}} %
\pgfusepath{clip}%
\pgfsetbuttcap%
\pgfsetroundjoin%
\definecolor{currentfill}{rgb}{0.272594,0.025563,0.353093}%
\pgfsetfillcolor{currentfill}%
\pgfsetlinewidth{0.000000pt}%
\definecolor{currentstroke}{rgb}{0.000000,0.000000,0.000000}%
\pgfsetstrokecolor{currentstroke}%
\pgfsetdash{}{0pt}%
\pgfpathmoveto{\pgfqpoint{4.525049in}{3.835018in}}%
\pgfpathlineto{\pgfqpoint{4.237756in}{3.930783in}}%
\pgfpathlineto{\pgfqpoint{4.239159in}{3.920965in}}%
\pgfpathlineto{\pgfqpoint{4.201290in}{3.947613in}}%
\pgfpathlineto{\pgfqpoint{4.247574in}{3.946210in}}%
\pgfpathlineto{\pgfqpoint{4.240561in}{3.939198in}}%
\pgfpathlineto{\pgfqpoint{4.527854in}{3.843433in}}%
\pgfpathlineto{\pgfqpoint{4.525049in}{3.835018in}}%
\pgfusepath{fill}%
\end{pgfscope}%
\begin{pgfscope}%
\pgfpathrectangle{\pgfqpoint{1.432000in}{0.528000in}}{\pgfqpoint{3.696000in}{3.696000in}} %
\pgfusepath{clip}%
\pgfsetbuttcap%
\pgfsetroundjoin%
\definecolor{currentfill}{rgb}{0.246811,0.283237,0.535941}%
\pgfsetfillcolor{currentfill}%
\pgfsetlinewidth{0.000000pt}%
\definecolor{currentstroke}{rgb}{0.000000,0.000000,0.000000}%
\pgfsetstrokecolor{currentstroke}%
\pgfsetdash{}{0pt}%
\pgfpathmoveto{\pgfqpoint{4.524468in}{3.835259in}}%
\pgfpathlineto{\pgfqpoint{4.343397in}{3.925795in}}%
\pgfpathlineto{\pgfqpoint{4.343397in}{3.915877in}}%
\pgfpathlineto{\pgfqpoint{4.309677in}{3.947613in}}%
\pgfpathlineto{\pgfqpoint{4.355297in}{3.939679in}}%
\pgfpathlineto{\pgfqpoint{4.347364in}{3.933729in}}%
\pgfpathlineto{\pgfqpoint{4.528435in}{3.843193in}}%
\pgfpathlineto{\pgfqpoint{4.524468in}{3.835259in}}%
\pgfusepath{fill}%
\end{pgfscope}%
\begin{pgfscope}%
\pgfpathrectangle{\pgfqpoint{1.432000in}{0.528000in}}{\pgfqpoint{3.696000in}{3.696000in}} %
\pgfusepath{clip}%
\pgfsetbuttcap%
\pgfsetroundjoin%
\definecolor{currentfill}{rgb}{0.227802,0.326594,0.546532}%
\pgfsetfillcolor{currentfill}%
\pgfsetlinewidth{0.000000pt}%
\definecolor{currentstroke}{rgb}{0.000000,0.000000,0.000000}%
\pgfsetstrokecolor{currentstroke}%
\pgfsetdash{}{0pt}%
\pgfpathmoveto{\pgfqpoint{4.632855in}{3.835259in}}%
\pgfpathlineto{\pgfqpoint{4.451784in}{3.925795in}}%
\pgfpathlineto{\pgfqpoint{4.451784in}{3.915877in}}%
\pgfpathlineto{\pgfqpoint{4.418065in}{3.947613in}}%
\pgfpathlineto{\pgfqpoint{4.463685in}{3.939679in}}%
\pgfpathlineto{\pgfqpoint{4.455751in}{3.933729in}}%
\pgfpathlineto{\pgfqpoint{4.636822in}{3.843193in}}%
\pgfpathlineto{\pgfqpoint{4.632855in}{3.835259in}}%
\pgfusepath{fill}%
\end{pgfscope}%
\begin{pgfscope}%
\pgfpathrectangle{\pgfqpoint{1.432000in}{0.528000in}}{\pgfqpoint{3.696000in}{3.696000in}} %
\pgfusepath{clip}%
\pgfsetbuttcap%
\pgfsetroundjoin%
\definecolor{currentfill}{rgb}{0.282656,0.100196,0.422160}%
\pgfsetfillcolor{currentfill}%
\pgfsetlinewidth{0.000000pt}%
\definecolor{currentstroke}{rgb}{0.000000,0.000000,0.000000}%
\pgfsetstrokecolor{currentstroke}%
\pgfsetdash{}{0pt}%
\pgfpathmoveto{\pgfqpoint{4.631703in}{3.836090in}}%
\pgfpathlineto{\pgfqpoint{4.551541in}{3.916251in}}%
\pgfpathlineto{\pgfqpoint{4.548405in}{3.906843in}}%
\pgfpathlineto{\pgfqpoint{4.526452in}{3.947613in}}%
\pgfpathlineto{\pgfqpoint{4.567222in}{3.925660in}}%
\pgfpathlineto{\pgfqpoint{4.557813in}{3.922524in}}%
\pgfpathlineto{\pgfqpoint{4.637975in}{3.842362in}}%
\pgfpathlineto{\pgfqpoint{4.631703in}{3.836090in}}%
\pgfusepath{fill}%
\end{pgfscope}%
\begin{pgfscope}%
\pgfpathrectangle{\pgfqpoint{1.432000in}{0.528000in}}{\pgfqpoint{3.696000in}{3.696000in}} %
\pgfusepath{clip}%
\pgfsetbuttcap%
\pgfsetroundjoin%
\definecolor{currentfill}{rgb}{0.277941,0.056324,0.381191}%
\pgfsetfillcolor{currentfill}%
\pgfsetlinewidth{0.000000pt}%
\definecolor{currentstroke}{rgb}{0.000000,0.000000,0.000000}%
\pgfsetstrokecolor{currentstroke}%
\pgfsetdash{}{0pt}%
\pgfpathmoveto{\pgfqpoint{4.741242in}{3.835259in}}%
\pgfpathlineto{\pgfqpoint{4.560171in}{3.925795in}}%
\pgfpathlineto{\pgfqpoint{4.560171in}{3.915877in}}%
\pgfpathlineto{\pgfqpoint{4.526452in}{3.947613in}}%
\pgfpathlineto{\pgfqpoint{4.572072in}{3.939679in}}%
\pgfpathlineto{\pgfqpoint{4.564138in}{3.933729in}}%
\pgfpathlineto{\pgfqpoint{4.745209in}{3.843193in}}%
\pgfpathlineto{\pgfqpoint{4.741242in}{3.835259in}}%
\pgfusepath{fill}%
\end{pgfscope}%
\begin{pgfscope}%
\pgfpathrectangle{\pgfqpoint{1.432000in}{0.528000in}}{\pgfqpoint{3.696000in}{3.696000in}} %
\pgfusepath{clip}%
\pgfsetbuttcap%
\pgfsetroundjoin%
\definecolor{currentfill}{rgb}{0.244972,0.287675,0.537260}%
\pgfsetfillcolor{currentfill}%
\pgfsetlinewidth{0.000000pt}%
\definecolor{currentstroke}{rgb}{0.000000,0.000000,0.000000}%
\pgfsetstrokecolor{currentstroke}%
\pgfsetdash{}{0pt}%
\pgfpathmoveto{\pgfqpoint{4.740090in}{3.836090in}}%
\pgfpathlineto{\pgfqpoint{4.659928in}{3.916251in}}%
\pgfpathlineto{\pgfqpoint{4.656792in}{3.906843in}}%
\pgfpathlineto{\pgfqpoint{4.634839in}{3.947613in}}%
\pgfpathlineto{\pgfqpoint{4.675609in}{3.925660in}}%
\pgfpathlineto{\pgfqpoint{4.666200in}{3.922524in}}%
\pgfpathlineto{\pgfqpoint{4.746362in}{3.842362in}}%
\pgfpathlineto{\pgfqpoint{4.740090in}{3.836090in}}%
\pgfusepath{fill}%
\end{pgfscope}%
\begin{pgfscope}%
\pgfpathrectangle{\pgfqpoint{1.432000in}{0.528000in}}{\pgfqpoint{3.696000in}{3.696000in}} %
\pgfusepath{clip}%
\pgfsetbuttcap%
\pgfsetroundjoin%
\definecolor{currentfill}{rgb}{0.239346,0.300855,0.540844}%
\pgfsetfillcolor{currentfill}%
\pgfsetlinewidth{0.000000pt}%
\definecolor{currentstroke}{rgb}{0.000000,0.000000,0.000000}%
\pgfsetstrokecolor{currentstroke}%
\pgfsetdash{}{0pt}%
\pgfpathmoveto{\pgfqpoint{4.848477in}{3.836090in}}%
\pgfpathlineto{\pgfqpoint{4.768315in}{3.916251in}}%
\pgfpathlineto{\pgfqpoint{4.765179in}{3.906843in}}%
\pgfpathlineto{\pgfqpoint{4.743226in}{3.947613in}}%
\pgfpathlineto{\pgfqpoint{4.783996in}{3.925660in}}%
\pgfpathlineto{\pgfqpoint{4.774587in}{3.922524in}}%
\pgfpathlineto{\pgfqpoint{4.854749in}{3.842362in}}%
\pgfpathlineto{\pgfqpoint{4.848477in}{3.836090in}}%
\pgfusepath{fill}%
\end{pgfscope}%
\begin{pgfscope}%
\pgfpathrectangle{\pgfqpoint{1.432000in}{0.528000in}}{\pgfqpoint{3.696000in}{3.696000in}} %
\pgfusepath{clip}%
\pgfsetbuttcap%
\pgfsetroundjoin%
\definecolor{currentfill}{rgb}{0.278826,0.175490,0.483397}%
\pgfsetfillcolor{currentfill}%
\pgfsetlinewidth{0.000000pt}%
\definecolor{currentstroke}{rgb}{0.000000,0.000000,0.000000}%
\pgfsetstrokecolor{currentstroke}%
\pgfsetdash{}{0pt}%
\pgfpathmoveto{\pgfqpoint{4.847178in}{3.839226in}}%
\pgfpathlineto{\pgfqpoint{4.847178in}{3.907696in}}%
\pgfpathlineto{\pgfqpoint{4.838307in}{3.903261in}}%
\pgfpathlineto{\pgfqpoint{4.851613in}{3.947613in}}%
\pgfpathlineto{\pgfqpoint{4.864919in}{3.903261in}}%
\pgfpathlineto{\pgfqpoint{4.856048in}{3.907696in}}%
\pgfpathlineto{\pgfqpoint{4.856048in}{3.839226in}}%
\pgfpathlineto{\pgfqpoint{4.847178in}{3.839226in}}%
\pgfusepath{fill}%
\end{pgfscope}%
\begin{pgfscope}%
\pgfpathrectangle{\pgfqpoint{1.432000in}{0.528000in}}{\pgfqpoint{3.696000in}{3.696000in}} %
\pgfusepath{clip}%
\pgfsetbuttcap%
\pgfsetroundjoin%
\definecolor{currentfill}{rgb}{0.195860,0.395433,0.555276}%
\pgfsetfillcolor{currentfill}%
\pgfsetlinewidth{0.000000pt}%
\definecolor{currentstroke}{rgb}{0.000000,0.000000,0.000000}%
\pgfsetstrokecolor{currentstroke}%
\pgfsetdash{}{0pt}%
\pgfpathmoveto{\pgfqpoint{4.955565in}{3.839226in}}%
\pgfpathlineto{\pgfqpoint{4.955565in}{3.907696in}}%
\pgfpathlineto{\pgfqpoint{4.946694in}{3.903261in}}%
\pgfpathlineto{\pgfqpoint{4.960000in}{3.947613in}}%
\pgfpathlineto{\pgfqpoint{4.973306in}{3.903261in}}%
\pgfpathlineto{\pgfqpoint{4.964435in}{3.907696in}}%
\pgfpathlineto{\pgfqpoint{4.964435in}{3.839226in}}%
\pgfpathlineto{\pgfqpoint{4.955565in}{3.839226in}}%
\pgfusepath{fill}%
\end{pgfscope}%
\begin{pgfscope}%
\pgfpathrectangle{\pgfqpoint{1.432000in}{0.528000in}}{\pgfqpoint{3.696000in}{3.696000in}} %
\pgfusepath{clip}%
\pgfsetbuttcap%
\pgfsetroundjoin%
\definecolor{currentfill}{rgb}{0.204903,0.375746,0.553533}%
\pgfsetfillcolor{currentfill}%
\pgfsetlinewidth{0.000000pt}%
\definecolor{currentstroke}{rgb}{0.000000,0.000000,0.000000}%
\pgfsetstrokecolor{currentstroke}%
\pgfsetdash{}{0pt}%
\pgfpathmoveto{\pgfqpoint{1.604435in}{3.947613in}}%
\pgfpathlineto{\pgfqpoint{1.604435in}{3.879143in}}%
\pgfpathlineto{\pgfqpoint{1.613306in}{3.883578in}}%
\pgfpathlineto{\pgfqpoint{1.600000in}{3.839226in}}%
\pgfpathlineto{\pgfqpoint{1.586694in}{3.883578in}}%
\pgfpathlineto{\pgfqpoint{1.595565in}{3.879143in}}%
\pgfpathlineto{\pgfqpoint{1.595565in}{3.947613in}}%
\pgfpathlineto{\pgfqpoint{1.604435in}{3.947613in}}%
\pgfusepath{fill}%
\end{pgfscope}%
\begin{pgfscope}%
\pgfpathrectangle{\pgfqpoint{1.432000in}{0.528000in}}{\pgfqpoint{3.696000in}{3.696000in}} %
\pgfusepath{clip}%
\pgfsetbuttcap%
\pgfsetroundjoin%
\definecolor{currentfill}{rgb}{0.283187,0.125848,0.444960}%
\pgfsetfillcolor{currentfill}%
\pgfsetlinewidth{0.000000pt}%
\definecolor{currentstroke}{rgb}{0.000000,0.000000,0.000000}%
\pgfsetstrokecolor{currentstroke}%
\pgfsetdash{}{0pt}%
\pgfpathmoveto{\pgfqpoint{1.712822in}{3.947613in}}%
\pgfpathlineto{\pgfqpoint{1.712822in}{3.770756in}}%
\pgfpathlineto{\pgfqpoint{1.721693in}{3.775191in}}%
\pgfpathlineto{\pgfqpoint{1.708387in}{3.730839in}}%
\pgfpathlineto{\pgfqpoint{1.695081in}{3.775191in}}%
\pgfpathlineto{\pgfqpoint{1.703952in}{3.770756in}}%
\pgfpathlineto{\pgfqpoint{1.703952in}{3.947613in}}%
\pgfpathlineto{\pgfqpoint{1.712822in}{3.947613in}}%
\pgfusepath{fill}%
\end{pgfscope}%
\begin{pgfscope}%
\pgfpathrectangle{\pgfqpoint{1.432000in}{0.528000in}}{\pgfqpoint{3.696000in}{3.696000in}} %
\pgfusepath{clip}%
\pgfsetbuttcap%
\pgfsetroundjoin%
\definecolor{currentfill}{rgb}{0.277134,0.185228,0.489898}%
\pgfsetfillcolor{currentfill}%
\pgfsetlinewidth{0.000000pt}%
\definecolor{currentstroke}{rgb}{0.000000,0.000000,0.000000}%
\pgfsetstrokecolor{currentstroke}%
\pgfsetdash{}{0pt}%
\pgfpathmoveto{\pgfqpoint{1.712822in}{3.947613in}}%
\pgfpathlineto{\pgfqpoint{1.712822in}{3.879143in}}%
\pgfpathlineto{\pgfqpoint{1.721693in}{3.883578in}}%
\pgfpathlineto{\pgfqpoint{1.708387in}{3.839226in}}%
\pgfpathlineto{\pgfqpoint{1.695081in}{3.883578in}}%
\pgfpathlineto{\pgfqpoint{1.703952in}{3.879143in}}%
\pgfpathlineto{\pgfqpoint{1.703952in}{3.947613in}}%
\pgfpathlineto{\pgfqpoint{1.712822in}{3.947613in}}%
\pgfusepath{fill}%
\end{pgfscope}%
\begin{pgfscope}%
\pgfpathrectangle{\pgfqpoint{1.432000in}{0.528000in}}{\pgfqpoint{3.696000in}{3.696000in}} %
\pgfusepath{clip}%
\pgfsetbuttcap%
\pgfsetroundjoin%
\definecolor{currentfill}{rgb}{0.279566,0.067836,0.391917}%
\pgfsetfillcolor{currentfill}%
\pgfsetlinewidth{0.000000pt}%
\definecolor{currentstroke}{rgb}{0.000000,0.000000,0.000000}%
\pgfsetstrokecolor{currentstroke}%
\pgfsetdash{}{0pt}%
\pgfpathmoveto{\pgfqpoint{1.821209in}{3.947613in}}%
\pgfpathlineto{\pgfqpoint{1.821209in}{3.879143in}}%
\pgfpathlineto{\pgfqpoint{1.830080in}{3.883578in}}%
\pgfpathlineto{\pgfqpoint{1.816774in}{3.839226in}}%
\pgfpathlineto{\pgfqpoint{1.803469in}{3.883578in}}%
\pgfpathlineto{\pgfqpoint{1.812339in}{3.879143in}}%
\pgfpathlineto{\pgfqpoint{1.812339in}{3.947613in}}%
\pgfpathlineto{\pgfqpoint{1.821209in}{3.947613in}}%
\pgfusepath{fill}%
\end{pgfscope}%
\begin{pgfscope}%
\pgfpathrectangle{\pgfqpoint{1.432000in}{0.528000in}}{\pgfqpoint{3.696000in}{3.696000in}} %
\pgfusepath{clip}%
\pgfsetbuttcap%
\pgfsetroundjoin%
\definecolor{currentfill}{rgb}{0.268510,0.009605,0.335427}%
\pgfsetfillcolor{currentfill}%
\pgfsetlinewidth{0.000000pt}%
\definecolor{currentstroke}{rgb}{0.000000,0.000000,0.000000}%
\pgfsetstrokecolor{currentstroke}%
\pgfsetdash{}{0pt}%
\pgfpathmoveto{\pgfqpoint{2.361846in}{3.950749in}}%
\pgfpathlineto{\pgfqpoint{2.442007in}{3.870587in}}%
\pgfpathlineto{\pgfqpoint{2.445144in}{3.879996in}}%
\pgfpathlineto{\pgfqpoint{2.467097in}{3.839226in}}%
\pgfpathlineto{\pgfqpoint{2.426327in}{3.861179in}}%
\pgfpathlineto{\pgfqpoint{2.435735in}{3.864315in}}%
\pgfpathlineto{\pgfqpoint{2.355574in}{3.944477in}}%
\pgfpathlineto{\pgfqpoint{2.361846in}{3.950749in}}%
\pgfusepath{fill}%
\end{pgfscope}%
\begin{pgfscope}%
\pgfpathrectangle{\pgfqpoint{1.432000in}{0.528000in}}{\pgfqpoint{3.696000in}{3.696000in}} %
\pgfusepath{clip}%
\pgfsetbuttcap%
\pgfsetroundjoin%
\definecolor{currentfill}{rgb}{0.267004,0.004874,0.329415}%
\pgfsetfillcolor{currentfill}%
\pgfsetlinewidth{0.000000pt}%
\definecolor{currentstroke}{rgb}{0.000000,0.000000,0.000000}%
\pgfsetstrokecolor{currentstroke}%
\pgfsetdash{}{0pt}%
\pgfpathmoveto{\pgfqpoint{2.471532in}{3.947613in}}%
\pgfpathlineto{\pgfqpoint{2.471532in}{3.879143in}}%
\pgfpathlineto{\pgfqpoint{2.480402in}{3.883578in}}%
\pgfpathlineto{\pgfqpoint{2.467097in}{3.839226in}}%
\pgfpathlineto{\pgfqpoint{2.453791in}{3.883578in}}%
\pgfpathlineto{\pgfqpoint{2.462662in}{3.879143in}}%
\pgfpathlineto{\pgfqpoint{2.462662in}{3.947613in}}%
\pgfpathlineto{\pgfqpoint{2.471532in}{3.947613in}}%
\pgfusepath{fill}%
\end{pgfscope}%
\begin{pgfscope}%
\pgfpathrectangle{\pgfqpoint{1.432000in}{0.528000in}}{\pgfqpoint{3.696000in}{3.696000in}} %
\pgfusepath{clip}%
\pgfsetbuttcap%
\pgfsetroundjoin%
\definecolor{currentfill}{rgb}{0.280267,0.073417,0.397163}%
\pgfsetfillcolor{currentfill}%
\pgfsetlinewidth{0.000000pt}%
\definecolor{currentstroke}{rgb}{0.000000,0.000000,0.000000}%
\pgfsetstrokecolor{currentstroke}%
\pgfsetdash{}{0pt}%
\pgfpathmoveto{\pgfqpoint{2.471532in}{3.947613in}}%
\pgfpathlineto{\pgfqpoint{2.469314in}{3.951454in}}%
\pgfpathlineto{\pgfqpoint{2.464879in}{3.951454in}}%
\pgfpathlineto{\pgfqpoint{2.462662in}{3.947613in}}%
\pgfpathlineto{\pgfqpoint{2.464879in}{3.943772in}}%
\pgfpathlineto{\pgfqpoint{2.469314in}{3.943772in}}%
\pgfpathlineto{\pgfqpoint{2.471532in}{3.947613in}}%
\pgfpathlineto{\pgfqpoint{2.469314in}{3.951454in}}%
\pgfusepath{fill}%
\end{pgfscope}%
\begin{pgfscope}%
\pgfpathrectangle{\pgfqpoint{1.432000in}{0.528000in}}{\pgfqpoint{3.696000in}{3.696000in}} %
\pgfusepath{clip}%
\pgfsetbuttcap%
\pgfsetroundjoin%
\definecolor{currentfill}{rgb}{0.218130,0.347432,0.550038}%
\pgfsetfillcolor{currentfill}%
\pgfsetlinewidth{0.000000pt}%
\definecolor{currentstroke}{rgb}{0.000000,0.000000,0.000000}%
\pgfsetstrokecolor{currentstroke}%
\pgfsetdash{}{0pt}%
\pgfpathmoveto{\pgfqpoint{2.579919in}{3.947613in}}%
\pgfpathlineto{\pgfqpoint{2.577701in}{3.951454in}}%
\pgfpathlineto{\pgfqpoint{2.573266in}{3.951454in}}%
\pgfpathlineto{\pgfqpoint{2.571049in}{3.947613in}}%
\pgfpathlineto{\pgfqpoint{2.573266in}{3.943772in}}%
\pgfpathlineto{\pgfqpoint{2.577701in}{3.943772in}}%
\pgfpathlineto{\pgfqpoint{2.579919in}{3.947613in}}%
\pgfpathlineto{\pgfqpoint{2.577701in}{3.951454in}}%
\pgfusepath{fill}%
\end{pgfscope}%
\begin{pgfscope}%
\pgfpathrectangle{\pgfqpoint{1.432000in}{0.528000in}}{\pgfqpoint{3.696000in}{3.696000in}} %
\pgfusepath{clip}%
\pgfsetbuttcap%
\pgfsetroundjoin%
\definecolor{currentfill}{rgb}{0.280894,0.078907,0.402329}%
\pgfsetfillcolor{currentfill}%
\pgfsetlinewidth{0.000000pt}%
\definecolor{currentstroke}{rgb}{0.000000,0.000000,0.000000}%
\pgfsetstrokecolor{currentstroke}%
\pgfsetdash{}{0pt}%
\pgfpathmoveto{\pgfqpoint{2.688306in}{3.947613in}}%
\pgfpathlineto{\pgfqpoint{2.688306in}{3.879143in}}%
\pgfpathlineto{\pgfqpoint{2.697177in}{3.883578in}}%
\pgfpathlineto{\pgfqpoint{2.683871in}{3.839226in}}%
\pgfpathlineto{\pgfqpoint{2.670565in}{3.883578in}}%
\pgfpathlineto{\pgfqpoint{2.679436in}{3.879143in}}%
\pgfpathlineto{\pgfqpoint{2.679436in}{3.947613in}}%
\pgfpathlineto{\pgfqpoint{2.688306in}{3.947613in}}%
\pgfusepath{fill}%
\end{pgfscope}%
\begin{pgfscope}%
\pgfpathrectangle{\pgfqpoint{1.432000in}{0.528000in}}{\pgfqpoint{3.696000in}{3.696000in}} %
\pgfusepath{clip}%
\pgfsetbuttcap%
\pgfsetroundjoin%
\definecolor{currentfill}{rgb}{0.241237,0.296485,0.539709}%
\pgfsetfillcolor{currentfill}%
\pgfsetlinewidth{0.000000pt}%
\definecolor{currentstroke}{rgb}{0.000000,0.000000,0.000000}%
\pgfsetstrokecolor{currentstroke}%
\pgfsetdash{}{0pt}%
\pgfpathmoveto{\pgfqpoint{2.688306in}{3.947613in}}%
\pgfpathlineto{\pgfqpoint{2.686089in}{3.951454in}}%
\pgfpathlineto{\pgfqpoint{2.681653in}{3.951454in}}%
\pgfpathlineto{\pgfqpoint{2.679436in}{3.947613in}}%
\pgfpathlineto{\pgfqpoint{2.681653in}{3.943772in}}%
\pgfpathlineto{\pgfqpoint{2.686089in}{3.943772in}}%
\pgfpathlineto{\pgfqpoint{2.688306in}{3.947613in}}%
\pgfpathlineto{\pgfqpoint{2.686089in}{3.951454in}}%
\pgfusepath{fill}%
\end{pgfscope}%
\begin{pgfscope}%
\pgfpathrectangle{\pgfqpoint{1.432000in}{0.528000in}}{\pgfqpoint{3.696000in}{3.696000in}} %
\pgfusepath{clip}%
\pgfsetbuttcap%
\pgfsetroundjoin%
\definecolor{currentfill}{rgb}{0.276022,0.044167,0.370164}%
\pgfsetfillcolor{currentfill}%
\pgfsetlinewidth{0.000000pt}%
\definecolor{currentstroke}{rgb}{0.000000,0.000000,0.000000}%
\pgfsetstrokecolor{currentstroke}%
\pgfsetdash{}{0pt}%
\pgfpathmoveto{\pgfqpoint{2.796693in}{3.947613in}}%
\pgfpathlineto{\pgfqpoint{2.796693in}{3.879143in}}%
\pgfpathlineto{\pgfqpoint{2.805564in}{3.883578in}}%
\pgfpathlineto{\pgfqpoint{2.792258in}{3.839226in}}%
\pgfpathlineto{\pgfqpoint{2.778952in}{3.883578in}}%
\pgfpathlineto{\pgfqpoint{2.787823in}{3.879143in}}%
\pgfpathlineto{\pgfqpoint{2.787823in}{3.947613in}}%
\pgfpathlineto{\pgfqpoint{2.796693in}{3.947613in}}%
\pgfusepath{fill}%
\end{pgfscope}%
\begin{pgfscope}%
\pgfpathrectangle{\pgfqpoint{1.432000in}{0.528000in}}{\pgfqpoint{3.696000in}{3.696000in}} %
\pgfusepath{clip}%
\pgfsetbuttcap%
\pgfsetroundjoin%
\definecolor{currentfill}{rgb}{0.283229,0.120777,0.440584}%
\pgfsetfillcolor{currentfill}%
\pgfsetlinewidth{0.000000pt}%
\definecolor{currentstroke}{rgb}{0.000000,0.000000,0.000000}%
\pgfsetstrokecolor{currentstroke}%
\pgfsetdash{}{0pt}%
\pgfpathmoveto{\pgfqpoint{2.796693in}{3.947613in}}%
\pgfpathlineto{\pgfqpoint{2.794476in}{3.951454in}}%
\pgfpathlineto{\pgfqpoint{2.790040in}{3.951454in}}%
\pgfpathlineto{\pgfqpoint{2.787823in}{3.947613in}}%
\pgfpathlineto{\pgfqpoint{2.790040in}{3.943772in}}%
\pgfpathlineto{\pgfqpoint{2.794476in}{3.943772in}}%
\pgfpathlineto{\pgfqpoint{2.796693in}{3.947613in}}%
\pgfpathlineto{\pgfqpoint{2.794476in}{3.951454in}}%
\pgfusepath{fill}%
\end{pgfscope}%
\begin{pgfscope}%
\pgfpathrectangle{\pgfqpoint{1.432000in}{0.528000in}}{\pgfqpoint{3.696000in}{3.696000in}} %
\pgfusepath{clip}%
\pgfsetbuttcap%
\pgfsetroundjoin%
\definecolor{currentfill}{rgb}{0.272594,0.025563,0.353093}%
\pgfsetfillcolor{currentfill}%
\pgfsetlinewidth{0.000000pt}%
\definecolor{currentstroke}{rgb}{0.000000,0.000000,0.000000}%
\pgfsetstrokecolor{currentstroke}%
\pgfsetdash{}{0pt}%
\pgfpathmoveto{\pgfqpoint{2.903781in}{3.950749in}}%
\pgfpathlineto{\pgfqpoint{2.983943in}{3.870587in}}%
\pgfpathlineto{\pgfqpoint{2.987079in}{3.879996in}}%
\pgfpathlineto{\pgfqpoint{3.009032in}{3.839226in}}%
\pgfpathlineto{\pgfqpoint{2.968262in}{3.861179in}}%
\pgfpathlineto{\pgfqpoint{2.977671in}{3.864315in}}%
\pgfpathlineto{\pgfqpoint{2.897509in}{3.944477in}}%
\pgfpathlineto{\pgfqpoint{2.903781in}{3.950749in}}%
\pgfusepath{fill}%
\end{pgfscope}%
\begin{pgfscope}%
\pgfpathrectangle{\pgfqpoint{1.432000in}{0.528000in}}{\pgfqpoint{3.696000in}{3.696000in}} %
\pgfusepath{clip}%
\pgfsetbuttcap%
\pgfsetroundjoin%
\definecolor{currentfill}{rgb}{0.281924,0.089666,0.412415}%
\pgfsetfillcolor{currentfill}%
\pgfsetlinewidth{0.000000pt}%
\definecolor{currentstroke}{rgb}{0.000000,0.000000,0.000000}%
\pgfsetstrokecolor{currentstroke}%
\pgfsetdash{}{0pt}%
\pgfpathmoveto{\pgfqpoint{2.900645in}{3.952048in}}%
\pgfpathlineto{\pgfqpoint{2.969115in}{3.952048in}}%
\pgfpathlineto{\pgfqpoint{2.964680in}{3.960919in}}%
\pgfpathlineto{\pgfqpoint{3.009032in}{3.947613in}}%
\pgfpathlineto{\pgfqpoint{2.964680in}{3.934307in}}%
\pgfpathlineto{\pgfqpoint{2.969115in}{3.943178in}}%
\pgfpathlineto{\pgfqpoint{2.900645in}{3.943178in}}%
\pgfpathlineto{\pgfqpoint{2.900645in}{3.952048in}}%
\pgfusepath{fill}%
\end{pgfscope}%
\begin{pgfscope}%
\pgfpathrectangle{\pgfqpoint{1.432000in}{0.528000in}}{\pgfqpoint{3.696000in}{3.696000in}} %
\pgfusepath{clip}%
\pgfsetbuttcap%
\pgfsetroundjoin%
\definecolor{currentfill}{rgb}{0.267968,0.223549,0.512008}%
\pgfsetfillcolor{currentfill}%
\pgfsetlinewidth{0.000000pt}%
\definecolor{currentstroke}{rgb}{0.000000,0.000000,0.000000}%
\pgfsetstrokecolor{currentstroke}%
\pgfsetdash{}{0pt}%
\pgfpathmoveto{\pgfqpoint{3.009032in}{3.952048in}}%
\pgfpathlineto{\pgfqpoint{3.077503in}{3.952048in}}%
\pgfpathlineto{\pgfqpoint{3.073067in}{3.960919in}}%
\pgfpathlineto{\pgfqpoint{3.117419in}{3.947613in}}%
\pgfpathlineto{\pgfqpoint{3.073067in}{3.934307in}}%
\pgfpathlineto{\pgfqpoint{3.077503in}{3.943178in}}%
\pgfpathlineto{\pgfqpoint{3.009032in}{3.943178in}}%
\pgfpathlineto{\pgfqpoint{3.009032in}{3.952048in}}%
\pgfusepath{fill}%
\end{pgfscope}%
\begin{pgfscope}%
\pgfpathrectangle{\pgfqpoint{1.432000in}{0.528000in}}{\pgfqpoint{3.696000in}{3.696000in}} %
\pgfusepath{clip}%
\pgfsetbuttcap%
\pgfsetroundjoin%
\definecolor{currentfill}{rgb}{0.282910,0.105393,0.426902}%
\pgfsetfillcolor{currentfill}%
\pgfsetlinewidth{0.000000pt}%
\definecolor{currentstroke}{rgb}{0.000000,0.000000,0.000000}%
\pgfsetstrokecolor{currentstroke}%
\pgfsetdash{}{0pt}%
\pgfpathmoveto{\pgfqpoint{3.120556in}{3.950749in}}%
\pgfpathlineto{\pgfqpoint{3.200717in}{3.870587in}}%
\pgfpathlineto{\pgfqpoint{3.203853in}{3.879996in}}%
\pgfpathlineto{\pgfqpoint{3.225806in}{3.839226in}}%
\pgfpathlineto{\pgfqpoint{3.185036in}{3.861179in}}%
\pgfpathlineto{\pgfqpoint{3.194445in}{3.864315in}}%
\pgfpathlineto{\pgfqpoint{3.114283in}{3.944477in}}%
\pgfpathlineto{\pgfqpoint{3.120556in}{3.950749in}}%
\pgfusepath{fill}%
\end{pgfscope}%
\begin{pgfscope}%
\pgfpathrectangle{\pgfqpoint{1.432000in}{0.528000in}}{\pgfqpoint{3.696000in}{3.696000in}} %
\pgfusepath{clip}%
\pgfsetbuttcap%
\pgfsetroundjoin%
\definecolor{currentfill}{rgb}{0.278826,0.175490,0.483397}%
\pgfsetfillcolor{currentfill}%
\pgfsetlinewidth{0.000000pt}%
\definecolor{currentstroke}{rgb}{0.000000,0.000000,0.000000}%
\pgfsetstrokecolor{currentstroke}%
\pgfsetdash{}{0pt}%
\pgfpathmoveto{\pgfqpoint{3.117419in}{3.952048in}}%
\pgfpathlineto{\pgfqpoint{3.185890in}{3.952048in}}%
\pgfpathlineto{\pgfqpoint{3.181454in}{3.960919in}}%
\pgfpathlineto{\pgfqpoint{3.225806in}{3.947613in}}%
\pgfpathlineto{\pgfqpoint{3.181454in}{3.934307in}}%
\pgfpathlineto{\pgfqpoint{3.185890in}{3.943178in}}%
\pgfpathlineto{\pgfqpoint{3.117419in}{3.943178in}}%
\pgfpathlineto{\pgfqpoint{3.117419in}{3.952048in}}%
\pgfusepath{fill}%
\end{pgfscope}%
\begin{pgfscope}%
\pgfpathrectangle{\pgfqpoint{1.432000in}{0.528000in}}{\pgfqpoint{3.696000in}{3.696000in}} %
\pgfusepath{clip}%
\pgfsetbuttcap%
\pgfsetroundjoin%
\definecolor{currentfill}{rgb}{0.262138,0.242286,0.520837}%
\pgfsetfillcolor{currentfill}%
\pgfsetlinewidth{0.000000pt}%
\definecolor{currentstroke}{rgb}{0.000000,0.000000,0.000000}%
\pgfsetstrokecolor{currentstroke}%
\pgfsetdash{}{0pt}%
\pgfpathmoveto{\pgfqpoint{3.228943in}{3.950749in}}%
\pgfpathlineto{\pgfqpoint{3.309104in}{3.870587in}}%
\pgfpathlineto{\pgfqpoint{3.312240in}{3.879996in}}%
\pgfpathlineto{\pgfqpoint{3.334194in}{3.839226in}}%
\pgfpathlineto{\pgfqpoint{3.293423in}{3.861179in}}%
\pgfpathlineto{\pgfqpoint{3.302832in}{3.864315in}}%
\pgfpathlineto{\pgfqpoint{3.222670in}{3.944477in}}%
\pgfpathlineto{\pgfqpoint{3.228943in}{3.950749in}}%
\pgfusepath{fill}%
\end{pgfscope}%
\begin{pgfscope}%
\pgfpathrectangle{\pgfqpoint{1.432000in}{0.528000in}}{\pgfqpoint{3.696000in}{3.696000in}} %
\pgfusepath{clip}%
\pgfsetbuttcap%
\pgfsetroundjoin%
\definecolor{currentfill}{rgb}{0.277018,0.050344,0.375715}%
\pgfsetfillcolor{currentfill}%
\pgfsetlinewidth{0.000000pt}%
\definecolor{currentstroke}{rgb}{0.000000,0.000000,0.000000}%
\pgfsetstrokecolor{currentstroke}%
\pgfsetdash{}{0pt}%
\pgfpathmoveto{\pgfqpoint{3.225806in}{3.952048in}}%
\pgfpathlineto{\pgfqpoint{3.294277in}{3.952048in}}%
\pgfpathlineto{\pgfqpoint{3.289842in}{3.960919in}}%
\pgfpathlineto{\pgfqpoint{3.334194in}{3.947613in}}%
\pgfpathlineto{\pgfqpoint{3.289842in}{3.934307in}}%
\pgfpathlineto{\pgfqpoint{3.294277in}{3.943178in}}%
\pgfpathlineto{\pgfqpoint{3.225806in}{3.943178in}}%
\pgfpathlineto{\pgfqpoint{3.225806in}{3.952048in}}%
\pgfusepath{fill}%
\end{pgfscope}%
\begin{pgfscope}%
\pgfpathrectangle{\pgfqpoint{1.432000in}{0.528000in}}{\pgfqpoint{3.696000in}{3.696000in}} %
\pgfusepath{clip}%
\pgfsetbuttcap%
\pgfsetroundjoin%
\definecolor{currentfill}{rgb}{0.274952,0.037752,0.364543}%
\pgfsetfillcolor{currentfill}%
\pgfsetlinewidth{0.000000pt}%
\definecolor{currentstroke}{rgb}{0.000000,0.000000,0.000000}%
\pgfsetstrokecolor{currentstroke}%
\pgfsetdash{}{0pt}%
\pgfpathmoveto{\pgfqpoint{3.338629in}{3.947613in}}%
\pgfpathlineto{\pgfqpoint{3.338629in}{3.879143in}}%
\pgfpathlineto{\pgfqpoint{3.347499in}{3.883578in}}%
\pgfpathlineto{\pgfqpoint{3.334194in}{3.839226in}}%
\pgfpathlineto{\pgfqpoint{3.320888in}{3.883578in}}%
\pgfpathlineto{\pgfqpoint{3.329758in}{3.879143in}}%
\pgfpathlineto{\pgfqpoint{3.329758in}{3.947613in}}%
\pgfpathlineto{\pgfqpoint{3.338629in}{3.947613in}}%
\pgfusepath{fill}%
\end{pgfscope}%
\begin{pgfscope}%
\pgfpathrectangle{\pgfqpoint{1.432000in}{0.528000in}}{\pgfqpoint{3.696000in}{3.696000in}} %
\pgfusepath{clip}%
\pgfsetbuttcap%
\pgfsetroundjoin%
\definecolor{currentfill}{rgb}{0.283229,0.120777,0.440584}%
\pgfsetfillcolor{currentfill}%
\pgfsetlinewidth{0.000000pt}%
\definecolor{currentstroke}{rgb}{0.000000,0.000000,0.000000}%
\pgfsetstrokecolor{currentstroke}%
\pgfsetdash{}{0pt}%
\pgfpathmoveto{\pgfqpoint{3.337330in}{3.950749in}}%
\pgfpathlineto{\pgfqpoint{3.417491in}{3.870587in}}%
\pgfpathlineto{\pgfqpoint{3.420628in}{3.879996in}}%
\pgfpathlineto{\pgfqpoint{3.442581in}{3.839226in}}%
\pgfpathlineto{\pgfqpoint{3.401811in}{3.861179in}}%
\pgfpathlineto{\pgfqpoint{3.411219in}{3.864315in}}%
\pgfpathlineto{\pgfqpoint{3.331057in}{3.944477in}}%
\pgfpathlineto{\pgfqpoint{3.337330in}{3.950749in}}%
\pgfusepath{fill}%
\end{pgfscope}%
\begin{pgfscope}%
\pgfpathrectangle{\pgfqpoint{1.432000in}{0.528000in}}{\pgfqpoint{3.696000in}{3.696000in}} %
\pgfusepath{clip}%
\pgfsetbuttcap%
\pgfsetroundjoin%
\definecolor{currentfill}{rgb}{0.243113,0.292092,0.538516}%
\pgfsetfillcolor{currentfill}%
\pgfsetlinewidth{0.000000pt}%
\definecolor{currentstroke}{rgb}{0.000000,0.000000,0.000000}%
\pgfsetstrokecolor{currentstroke}%
\pgfsetdash{}{0pt}%
\pgfpathmoveto{\pgfqpoint{3.447016in}{3.947613in}}%
\pgfpathlineto{\pgfqpoint{3.447016in}{3.879143in}}%
\pgfpathlineto{\pgfqpoint{3.455886in}{3.883578in}}%
\pgfpathlineto{\pgfqpoint{3.442581in}{3.839226in}}%
\pgfpathlineto{\pgfqpoint{3.429275in}{3.883578in}}%
\pgfpathlineto{\pgfqpoint{3.438145in}{3.879143in}}%
\pgfpathlineto{\pgfqpoint{3.438145in}{3.947613in}}%
\pgfpathlineto{\pgfqpoint{3.447016in}{3.947613in}}%
\pgfusepath{fill}%
\end{pgfscope}%
\begin{pgfscope}%
\pgfpathrectangle{\pgfqpoint{1.432000in}{0.528000in}}{\pgfqpoint{3.696000in}{3.696000in}} %
\pgfusepath{clip}%
\pgfsetbuttcap%
\pgfsetroundjoin%
\definecolor{currentfill}{rgb}{0.267004,0.004874,0.329415}%
\pgfsetfillcolor{currentfill}%
\pgfsetlinewidth{0.000000pt}%
\definecolor{currentstroke}{rgb}{0.000000,0.000000,0.000000}%
\pgfsetstrokecolor{currentstroke}%
\pgfsetdash{}{0pt}%
\pgfpathmoveto{\pgfqpoint{3.447016in}{3.947613in}}%
\pgfpathlineto{\pgfqpoint{3.444798in}{3.951454in}}%
\pgfpathlineto{\pgfqpoint{3.440363in}{3.951454in}}%
\pgfpathlineto{\pgfqpoint{3.438145in}{3.947613in}}%
\pgfpathlineto{\pgfqpoint{3.440363in}{3.943772in}}%
\pgfpathlineto{\pgfqpoint{3.444798in}{3.943772in}}%
\pgfpathlineto{\pgfqpoint{3.447016in}{3.947613in}}%
\pgfpathlineto{\pgfqpoint{3.444798in}{3.951454in}}%
\pgfusepath{fill}%
\end{pgfscope}%
\begin{pgfscope}%
\pgfpathrectangle{\pgfqpoint{1.432000in}{0.528000in}}{\pgfqpoint{3.696000in}{3.696000in}} %
\pgfusepath{clip}%
\pgfsetbuttcap%
\pgfsetroundjoin%
\definecolor{currentfill}{rgb}{0.177423,0.437527,0.557565}%
\pgfsetfillcolor{currentfill}%
\pgfsetlinewidth{0.000000pt}%
\definecolor{currentstroke}{rgb}{0.000000,0.000000,0.000000}%
\pgfsetstrokecolor{currentstroke}%
\pgfsetdash{}{0pt}%
\pgfpathmoveto{\pgfqpoint{3.555403in}{3.947613in}}%
\pgfpathlineto{\pgfqpoint{3.553185in}{3.951454in}}%
\pgfpathlineto{\pgfqpoint{3.548750in}{3.951454in}}%
\pgfpathlineto{\pgfqpoint{3.546533in}{3.947613in}}%
\pgfpathlineto{\pgfqpoint{3.548750in}{3.943772in}}%
\pgfpathlineto{\pgfqpoint{3.553185in}{3.943772in}}%
\pgfpathlineto{\pgfqpoint{3.555403in}{3.947613in}}%
\pgfpathlineto{\pgfqpoint{3.553185in}{3.951454in}}%
\pgfusepath{fill}%
\end{pgfscope}%
\begin{pgfscope}%
\pgfpathrectangle{\pgfqpoint{1.432000in}{0.528000in}}{\pgfqpoint{3.696000in}{3.696000in}} %
\pgfusepath{clip}%
\pgfsetbuttcap%
\pgfsetroundjoin%
\definecolor{currentfill}{rgb}{0.190631,0.407061,0.556089}%
\pgfsetfillcolor{currentfill}%
\pgfsetlinewidth{0.000000pt}%
\definecolor{currentstroke}{rgb}{0.000000,0.000000,0.000000}%
\pgfsetstrokecolor{currentstroke}%
\pgfsetdash{}{0pt}%
\pgfpathmoveto{\pgfqpoint{3.659355in}{3.943178in}}%
\pgfpathlineto{\pgfqpoint{3.590885in}{3.943178in}}%
\pgfpathlineto{\pgfqpoint{3.595320in}{3.934307in}}%
\pgfpathlineto{\pgfqpoint{3.550968in}{3.947613in}}%
\pgfpathlineto{\pgfqpoint{3.595320in}{3.960919in}}%
\pgfpathlineto{\pgfqpoint{3.590885in}{3.952048in}}%
\pgfpathlineto{\pgfqpoint{3.659355in}{3.952048in}}%
\pgfpathlineto{\pgfqpoint{3.659355in}{3.943178in}}%
\pgfusepath{fill}%
\end{pgfscope}%
\begin{pgfscope}%
\pgfpathrectangle{\pgfqpoint{1.432000in}{0.528000in}}{\pgfqpoint{3.696000in}{3.696000in}} %
\pgfusepath{clip}%
\pgfsetbuttcap%
\pgfsetroundjoin%
\definecolor{currentfill}{rgb}{0.149039,0.508051,0.557250}%
\pgfsetfillcolor{currentfill}%
\pgfsetlinewidth{0.000000pt}%
\definecolor{currentstroke}{rgb}{0.000000,0.000000,0.000000}%
\pgfsetstrokecolor{currentstroke}%
\pgfsetdash{}{0pt}%
\pgfpathmoveto{\pgfqpoint{3.767742in}{3.943178in}}%
\pgfpathlineto{\pgfqpoint{3.699272in}{3.943178in}}%
\pgfpathlineto{\pgfqpoint{3.703707in}{3.934307in}}%
\pgfpathlineto{\pgfqpoint{3.659355in}{3.947613in}}%
\pgfpathlineto{\pgfqpoint{3.703707in}{3.960919in}}%
\pgfpathlineto{\pgfqpoint{3.699272in}{3.952048in}}%
\pgfpathlineto{\pgfqpoint{3.767742in}{3.952048in}}%
\pgfpathlineto{\pgfqpoint{3.767742in}{3.943178in}}%
\pgfusepath{fill}%
\end{pgfscope}%
\begin{pgfscope}%
\pgfpathrectangle{\pgfqpoint{1.432000in}{0.528000in}}{\pgfqpoint{3.696000in}{3.696000in}} %
\pgfusepath{clip}%
\pgfsetbuttcap%
\pgfsetroundjoin%
\definecolor{currentfill}{rgb}{0.273809,0.031497,0.358853}%
\pgfsetfillcolor{currentfill}%
\pgfsetlinewidth{0.000000pt}%
\definecolor{currentstroke}{rgb}{0.000000,0.000000,0.000000}%
\pgfsetstrokecolor{currentstroke}%
\pgfsetdash{}{0pt}%
\pgfpathmoveto{\pgfqpoint{3.876129in}{3.943178in}}%
\pgfpathlineto{\pgfqpoint{3.807659in}{3.943178in}}%
\pgfpathlineto{\pgfqpoint{3.812094in}{3.934307in}}%
\pgfpathlineto{\pgfqpoint{3.767742in}{3.947613in}}%
\pgfpathlineto{\pgfqpoint{3.812094in}{3.960919in}}%
\pgfpathlineto{\pgfqpoint{3.807659in}{3.952048in}}%
\pgfpathlineto{\pgfqpoint{3.876129in}{3.952048in}}%
\pgfpathlineto{\pgfqpoint{3.876129in}{3.943178in}}%
\pgfusepath{fill}%
\end{pgfscope}%
\begin{pgfscope}%
\pgfpathrectangle{\pgfqpoint{1.432000in}{0.528000in}}{\pgfqpoint{3.696000in}{3.696000in}} %
\pgfusepath{clip}%
\pgfsetbuttcap%
\pgfsetroundjoin%
\definecolor{currentfill}{rgb}{0.281887,0.150881,0.465405}%
\pgfsetfillcolor{currentfill}%
\pgfsetlinewidth{0.000000pt}%
\definecolor{currentstroke}{rgb}{0.000000,0.000000,0.000000}%
\pgfsetstrokecolor{currentstroke}%
\pgfsetdash{}{0pt}%
\pgfpathmoveto{\pgfqpoint{3.872993in}{3.944477in}}%
\pgfpathlineto{\pgfqpoint{3.792831in}{4.024638in}}%
\pgfpathlineto{\pgfqpoint{3.789695in}{4.015230in}}%
\pgfpathlineto{\pgfqpoint{3.767742in}{4.056000in}}%
\pgfpathlineto{\pgfqpoint{3.808512in}{4.034047in}}%
\pgfpathlineto{\pgfqpoint{3.799104in}{4.030911in}}%
\pgfpathlineto{\pgfqpoint{3.879265in}{3.950749in}}%
\pgfpathlineto{\pgfqpoint{3.872993in}{3.944477in}}%
\pgfusepath{fill}%
\end{pgfscope}%
\begin{pgfscope}%
\pgfpathrectangle{\pgfqpoint{1.432000in}{0.528000in}}{\pgfqpoint{3.696000in}{3.696000in}} %
\pgfusepath{clip}%
\pgfsetbuttcap%
\pgfsetroundjoin%
\definecolor{currentfill}{rgb}{0.187231,0.414746,0.556547}%
\pgfsetfillcolor{currentfill}%
\pgfsetlinewidth{0.000000pt}%
\definecolor{currentstroke}{rgb}{0.000000,0.000000,0.000000}%
\pgfsetstrokecolor{currentstroke}%
\pgfsetdash{}{0pt}%
\pgfpathmoveto{\pgfqpoint{4.090920in}{3.943646in}}%
\pgfpathlineto{\pgfqpoint{3.909848in}{4.034182in}}%
\pgfpathlineto{\pgfqpoint{3.909848in}{4.024264in}}%
\pgfpathlineto{\pgfqpoint{3.876129in}{4.056000in}}%
\pgfpathlineto{\pgfqpoint{3.921749in}{4.048066in}}%
\pgfpathlineto{\pgfqpoint{3.913815in}{4.042116in}}%
\pgfpathlineto{\pgfqpoint{4.094887in}{3.951580in}}%
\pgfpathlineto{\pgfqpoint{4.090920in}{3.943646in}}%
\pgfusepath{fill}%
\end{pgfscope}%
\begin{pgfscope}%
\pgfpathrectangle{\pgfqpoint{1.432000in}{0.528000in}}{\pgfqpoint{3.696000in}{3.696000in}} %
\pgfusepath{clip}%
\pgfsetbuttcap%
\pgfsetroundjoin%
\definecolor{currentfill}{rgb}{0.190631,0.407061,0.556089}%
\pgfsetfillcolor{currentfill}%
\pgfsetlinewidth{0.000000pt}%
\definecolor{currentstroke}{rgb}{0.000000,0.000000,0.000000}%
\pgfsetstrokecolor{currentstroke}%
\pgfsetdash{}{0pt}%
\pgfpathmoveto{\pgfqpoint{4.199307in}{3.943646in}}%
\pgfpathlineto{\pgfqpoint{4.018235in}{4.034182in}}%
\pgfpathlineto{\pgfqpoint{4.018235in}{4.024264in}}%
\pgfpathlineto{\pgfqpoint{3.984516in}{4.056000in}}%
\pgfpathlineto{\pgfqpoint{4.030136in}{4.048066in}}%
\pgfpathlineto{\pgfqpoint{4.022202in}{4.042116in}}%
\pgfpathlineto{\pgfqpoint{4.203274in}{3.951580in}}%
\pgfpathlineto{\pgfqpoint{4.199307in}{3.943646in}}%
\pgfusepath{fill}%
\end{pgfscope}%
\begin{pgfscope}%
\pgfpathrectangle{\pgfqpoint{1.432000in}{0.528000in}}{\pgfqpoint{3.696000in}{3.696000in}} %
\pgfusepath{clip}%
\pgfsetbuttcap%
\pgfsetroundjoin%
\definecolor{currentfill}{rgb}{0.269308,0.218818,0.509577}%
\pgfsetfillcolor{currentfill}%
\pgfsetlinewidth{0.000000pt}%
\definecolor{currentstroke}{rgb}{0.000000,0.000000,0.000000}%
\pgfsetstrokecolor{currentstroke}%
\pgfsetdash{}{0pt}%
\pgfpathmoveto{\pgfqpoint{4.307694in}{3.943646in}}%
\pgfpathlineto{\pgfqpoint{4.126622in}{4.034182in}}%
\pgfpathlineto{\pgfqpoint{4.126622in}{4.024264in}}%
\pgfpathlineto{\pgfqpoint{4.092903in}{4.056000in}}%
\pgfpathlineto{\pgfqpoint{4.138523in}{4.048066in}}%
\pgfpathlineto{\pgfqpoint{4.130589in}{4.042116in}}%
\pgfpathlineto{\pgfqpoint{4.311661in}{3.951580in}}%
\pgfpathlineto{\pgfqpoint{4.307694in}{3.943646in}}%
\pgfusepath{fill}%
\end{pgfscope}%
\begin{pgfscope}%
\pgfpathrectangle{\pgfqpoint{1.432000in}{0.528000in}}{\pgfqpoint{3.696000in}{3.696000in}} %
\pgfusepath{clip}%
\pgfsetbuttcap%
\pgfsetroundjoin%
\definecolor{currentfill}{rgb}{0.279566,0.067836,0.391917}%
\pgfsetfillcolor{currentfill}%
\pgfsetlinewidth{0.000000pt}%
\definecolor{currentstroke}{rgb}{0.000000,0.000000,0.000000}%
\pgfsetstrokecolor{currentstroke}%
\pgfsetdash{}{0pt}%
\pgfpathmoveto{\pgfqpoint{4.416662in}{3.943405in}}%
\pgfpathlineto{\pgfqpoint{4.129369in}{4.039170in}}%
\pgfpathlineto{\pgfqpoint{4.130772in}{4.029352in}}%
\pgfpathlineto{\pgfqpoint{4.092903in}{4.056000in}}%
\pgfpathlineto{\pgfqpoint{4.139187in}{4.054597in}}%
\pgfpathlineto{\pgfqpoint{4.132174in}{4.047585in}}%
\pgfpathlineto{\pgfqpoint{4.419467in}{3.951821in}}%
\pgfpathlineto{\pgfqpoint{4.416662in}{3.943405in}}%
\pgfusepath{fill}%
\end{pgfscope}%
\begin{pgfscope}%
\pgfpathrectangle{\pgfqpoint{1.432000in}{0.528000in}}{\pgfqpoint{3.696000in}{3.696000in}} %
\pgfusepath{clip}%
\pgfsetbuttcap%
\pgfsetroundjoin%
\definecolor{currentfill}{rgb}{0.280267,0.073417,0.397163}%
\pgfsetfillcolor{currentfill}%
\pgfsetlinewidth{0.000000pt}%
\definecolor{currentstroke}{rgb}{0.000000,0.000000,0.000000}%
\pgfsetstrokecolor{currentstroke}%
\pgfsetdash{}{0pt}%
\pgfpathmoveto{\pgfqpoint{4.416081in}{3.943646in}}%
\pgfpathlineto{\pgfqpoint{4.235010in}{4.034182in}}%
\pgfpathlineto{\pgfqpoint{4.235010in}{4.024264in}}%
\pgfpathlineto{\pgfqpoint{4.201290in}{4.056000in}}%
\pgfpathlineto{\pgfqpoint{4.246910in}{4.048066in}}%
\pgfpathlineto{\pgfqpoint{4.238976in}{4.042116in}}%
\pgfpathlineto{\pgfqpoint{4.420048in}{3.951580in}}%
\pgfpathlineto{\pgfqpoint{4.416081in}{3.943646in}}%
\pgfusepath{fill}%
\end{pgfscope}%
\begin{pgfscope}%
\pgfpathrectangle{\pgfqpoint{1.432000in}{0.528000in}}{\pgfqpoint{3.696000in}{3.696000in}} %
\pgfusepath{clip}%
\pgfsetbuttcap%
\pgfsetroundjoin%
\definecolor{currentfill}{rgb}{0.283091,0.110553,0.431554}%
\pgfsetfillcolor{currentfill}%
\pgfsetlinewidth{0.000000pt}%
\definecolor{currentstroke}{rgb}{0.000000,0.000000,0.000000}%
\pgfsetstrokecolor{currentstroke}%
\pgfsetdash{}{0pt}%
\pgfpathmoveto{\pgfqpoint{4.525049in}{3.943405in}}%
\pgfpathlineto{\pgfqpoint{4.237756in}{4.039170in}}%
\pgfpathlineto{\pgfqpoint{4.239159in}{4.029352in}}%
\pgfpathlineto{\pgfqpoint{4.201290in}{4.056000in}}%
\pgfpathlineto{\pgfqpoint{4.247574in}{4.054597in}}%
\pgfpathlineto{\pgfqpoint{4.240561in}{4.047585in}}%
\pgfpathlineto{\pgfqpoint{4.527854in}{3.951821in}}%
\pgfpathlineto{\pgfqpoint{4.525049in}{3.943405in}}%
\pgfusepath{fill}%
\end{pgfscope}%
\begin{pgfscope}%
\pgfpathrectangle{\pgfqpoint{1.432000in}{0.528000in}}{\pgfqpoint{3.696000in}{3.696000in}} %
\pgfusepath{clip}%
\pgfsetbuttcap%
\pgfsetroundjoin%
\definecolor{currentfill}{rgb}{0.274128,0.199721,0.498911}%
\pgfsetfillcolor{currentfill}%
\pgfsetlinewidth{0.000000pt}%
\definecolor{currentstroke}{rgb}{0.000000,0.000000,0.000000}%
\pgfsetstrokecolor{currentstroke}%
\pgfsetdash{}{0pt}%
\pgfpathmoveto{\pgfqpoint{4.524468in}{3.943646in}}%
\pgfpathlineto{\pgfqpoint{4.343397in}{4.034182in}}%
\pgfpathlineto{\pgfqpoint{4.343397in}{4.024264in}}%
\pgfpathlineto{\pgfqpoint{4.309677in}{4.056000in}}%
\pgfpathlineto{\pgfqpoint{4.355297in}{4.048066in}}%
\pgfpathlineto{\pgfqpoint{4.347364in}{4.042116in}}%
\pgfpathlineto{\pgfqpoint{4.528435in}{3.951580in}}%
\pgfpathlineto{\pgfqpoint{4.524468in}{3.943646in}}%
\pgfusepath{fill}%
\end{pgfscope}%
\begin{pgfscope}%
\pgfpathrectangle{\pgfqpoint{1.432000in}{0.528000in}}{\pgfqpoint{3.696000in}{3.696000in}} %
\pgfusepath{clip}%
\pgfsetbuttcap%
\pgfsetroundjoin%
\definecolor{currentfill}{rgb}{0.273006,0.204520,0.501721}%
\pgfsetfillcolor{currentfill}%
\pgfsetlinewidth{0.000000pt}%
\definecolor{currentstroke}{rgb}{0.000000,0.000000,0.000000}%
\pgfsetstrokecolor{currentstroke}%
\pgfsetdash{}{0pt}%
\pgfpathmoveto{\pgfqpoint{4.632855in}{3.943646in}}%
\pgfpathlineto{\pgfqpoint{4.451784in}{4.034182in}}%
\pgfpathlineto{\pgfqpoint{4.451784in}{4.024264in}}%
\pgfpathlineto{\pgfqpoint{4.418065in}{4.056000in}}%
\pgfpathlineto{\pgfqpoint{4.463685in}{4.048066in}}%
\pgfpathlineto{\pgfqpoint{4.455751in}{4.042116in}}%
\pgfpathlineto{\pgfqpoint{4.636822in}{3.951580in}}%
\pgfpathlineto{\pgfqpoint{4.632855in}{3.943646in}}%
\pgfusepath{fill}%
\end{pgfscope}%
\begin{pgfscope}%
\pgfpathrectangle{\pgfqpoint{1.432000in}{0.528000in}}{\pgfqpoint{3.696000in}{3.696000in}} %
\pgfusepath{clip}%
\pgfsetbuttcap%
\pgfsetroundjoin%
\definecolor{currentfill}{rgb}{0.272594,0.025563,0.353093}%
\pgfsetfillcolor{currentfill}%
\pgfsetlinewidth{0.000000pt}%
\definecolor{currentstroke}{rgb}{0.000000,0.000000,0.000000}%
\pgfsetstrokecolor{currentstroke}%
\pgfsetdash{}{0pt}%
\pgfpathmoveto{\pgfqpoint{4.631703in}{3.944477in}}%
\pgfpathlineto{\pgfqpoint{4.551541in}{4.024638in}}%
\pgfpathlineto{\pgfqpoint{4.548405in}{4.015230in}}%
\pgfpathlineto{\pgfqpoint{4.526452in}{4.056000in}}%
\pgfpathlineto{\pgfqpoint{4.567222in}{4.034047in}}%
\pgfpathlineto{\pgfqpoint{4.557813in}{4.030911in}}%
\pgfpathlineto{\pgfqpoint{4.637975in}{3.950749in}}%
\pgfpathlineto{\pgfqpoint{4.631703in}{3.944477in}}%
\pgfusepath{fill}%
\end{pgfscope}%
\begin{pgfscope}%
\pgfpathrectangle{\pgfqpoint{1.432000in}{0.528000in}}{\pgfqpoint{3.696000in}{3.696000in}} %
\pgfusepath{clip}%
\pgfsetbuttcap%
\pgfsetroundjoin%
\definecolor{currentfill}{rgb}{0.276194,0.190074,0.493001}%
\pgfsetfillcolor{currentfill}%
\pgfsetlinewidth{0.000000pt}%
\definecolor{currentstroke}{rgb}{0.000000,0.000000,0.000000}%
\pgfsetstrokecolor{currentstroke}%
\pgfsetdash{}{0pt}%
\pgfpathmoveto{\pgfqpoint{4.740090in}{3.944477in}}%
\pgfpathlineto{\pgfqpoint{4.659928in}{4.024638in}}%
\pgfpathlineto{\pgfqpoint{4.656792in}{4.015230in}}%
\pgfpathlineto{\pgfqpoint{4.634839in}{4.056000in}}%
\pgfpathlineto{\pgfqpoint{4.675609in}{4.034047in}}%
\pgfpathlineto{\pgfqpoint{4.666200in}{4.030911in}}%
\pgfpathlineto{\pgfqpoint{4.746362in}{3.950749in}}%
\pgfpathlineto{\pgfqpoint{4.740090in}{3.944477in}}%
\pgfusepath{fill}%
\end{pgfscope}%
\begin{pgfscope}%
\pgfpathrectangle{\pgfqpoint{1.432000in}{0.528000in}}{\pgfqpoint{3.696000in}{3.696000in}} %
\pgfusepath{clip}%
\pgfsetbuttcap%
\pgfsetroundjoin%
\definecolor{currentfill}{rgb}{0.282290,0.145912,0.461510}%
\pgfsetfillcolor{currentfill}%
\pgfsetlinewidth{0.000000pt}%
\definecolor{currentstroke}{rgb}{0.000000,0.000000,0.000000}%
\pgfsetstrokecolor{currentstroke}%
\pgfsetdash{}{0pt}%
\pgfpathmoveto{\pgfqpoint{4.848477in}{3.944477in}}%
\pgfpathlineto{\pgfqpoint{4.768315in}{4.024638in}}%
\pgfpathlineto{\pgfqpoint{4.765179in}{4.015230in}}%
\pgfpathlineto{\pgfqpoint{4.743226in}{4.056000in}}%
\pgfpathlineto{\pgfqpoint{4.783996in}{4.034047in}}%
\pgfpathlineto{\pgfqpoint{4.774587in}{4.030911in}}%
\pgfpathlineto{\pgfqpoint{4.854749in}{3.950749in}}%
\pgfpathlineto{\pgfqpoint{4.848477in}{3.944477in}}%
\pgfusepath{fill}%
\end{pgfscope}%
\begin{pgfscope}%
\pgfpathrectangle{\pgfqpoint{1.432000in}{0.528000in}}{\pgfqpoint{3.696000in}{3.696000in}} %
\pgfusepath{clip}%
\pgfsetbuttcap%
\pgfsetroundjoin%
\definecolor{currentfill}{rgb}{0.279566,0.067836,0.391917}%
\pgfsetfillcolor{currentfill}%
\pgfsetlinewidth{0.000000pt}%
\definecolor{currentstroke}{rgb}{0.000000,0.000000,0.000000}%
\pgfsetstrokecolor{currentstroke}%
\pgfsetdash{}{0pt}%
\pgfpathmoveto{\pgfqpoint{4.847178in}{3.947613in}}%
\pgfpathlineto{\pgfqpoint{4.847178in}{4.016083in}}%
\pgfpathlineto{\pgfqpoint{4.838307in}{4.011648in}}%
\pgfpathlineto{\pgfqpoint{4.851613in}{4.056000in}}%
\pgfpathlineto{\pgfqpoint{4.864919in}{4.011648in}}%
\pgfpathlineto{\pgfqpoint{4.856048in}{4.016083in}}%
\pgfpathlineto{\pgfqpoint{4.856048in}{3.947613in}}%
\pgfpathlineto{\pgfqpoint{4.847178in}{3.947613in}}%
\pgfusepath{fill}%
\end{pgfscope}%
\begin{pgfscope}%
\pgfpathrectangle{\pgfqpoint{1.432000in}{0.528000in}}{\pgfqpoint{3.696000in}{3.696000in}} %
\pgfusepath{clip}%
\pgfsetbuttcap%
\pgfsetroundjoin%
\definecolor{currentfill}{rgb}{0.274952,0.037752,0.364543}%
\pgfsetfillcolor{currentfill}%
\pgfsetlinewidth{0.000000pt}%
\definecolor{currentstroke}{rgb}{0.000000,0.000000,0.000000}%
\pgfsetstrokecolor{currentstroke}%
\pgfsetdash{}{0pt}%
\pgfpathmoveto{\pgfqpoint{4.964435in}{3.947613in}}%
\pgfpathlineto{\pgfqpoint{4.962218in}{3.951454in}}%
\pgfpathlineto{\pgfqpoint{4.957782in}{3.951454in}}%
\pgfpathlineto{\pgfqpoint{4.955565in}{3.947613in}}%
\pgfpathlineto{\pgfqpoint{4.957782in}{3.943772in}}%
\pgfpathlineto{\pgfqpoint{4.962218in}{3.943772in}}%
\pgfpathlineto{\pgfqpoint{4.964435in}{3.947613in}}%
\pgfpathlineto{\pgfqpoint{4.962218in}{3.951454in}}%
\pgfusepath{fill}%
\end{pgfscope}%
\begin{pgfscope}%
\pgfpathrectangle{\pgfqpoint{1.432000in}{0.528000in}}{\pgfqpoint{3.696000in}{3.696000in}} %
\pgfusepath{clip}%
\pgfsetbuttcap%
\pgfsetroundjoin%
\definecolor{currentfill}{rgb}{0.274128,0.199721,0.498911}%
\pgfsetfillcolor{currentfill}%
\pgfsetlinewidth{0.000000pt}%
\definecolor{currentstroke}{rgb}{0.000000,0.000000,0.000000}%
\pgfsetstrokecolor{currentstroke}%
\pgfsetdash{}{0pt}%
\pgfpathmoveto{\pgfqpoint{4.955565in}{3.947613in}}%
\pgfpathlineto{\pgfqpoint{4.955565in}{4.016083in}}%
\pgfpathlineto{\pgfqpoint{4.946694in}{4.011648in}}%
\pgfpathlineto{\pgfqpoint{4.960000in}{4.056000in}}%
\pgfpathlineto{\pgfqpoint{4.973306in}{4.011648in}}%
\pgfpathlineto{\pgfqpoint{4.964435in}{4.016083in}}%
\pgfpathlineto{\pgfqpoint{4.964435in}{3.947613in}}%
\pgfpathlineto{\pgfqpoint{4.955565in}{3.947613in}}%
\pgfusepath{fill}%
\end{pgfscope}%
\begin{pgfscope}%
\pgfpathrectangle{\pgfqpoint{1.432000in}{0.528000in}}{\pgfqpoint{3.696000in}{3.696000in}} %
\pgfusepath{clip}%
\pgfsetbuttcap%
\pgfsetroundjoin%
\definecolor{currentfill}{rgb}{0.266580,0.228262,0.514349}%
\pgfsetfillcolor{currentfill}%
\pgfsetlinewidth{0.000000pt}%
\definecolor{currentstroke}{rgb}{0.000000,0.000000,0.000000}%
\pgfsetstrokecolor{currentstroke}%
\pgfsetdash{}{0pt}%
\pgfpathmoveto{\pgfqpoint{1.604435in}{4.056000in}}%
\pgfpathlineto{\pgfqpoint{1.604435in}{3.987530in}}%
\pgfpathlineto{\pgfqpoint{1.613306in}{3.991965in}}%
\pgfpathlineto{\pgfqpoint{1.600000in}{3.947613in}}%
\pgfpathlineto{\pgfqpoint{1.586694in}{3.991965in}}%
\pgfpathlineto{\pgfqpoint{1.595565in}{3.987530in}}%
\pgfpathlineto{\pgfqpoint{1.595565in}{4.056000in}}%
\pgfpathlineto{\pgfqpoint{1.604435in}{4.056000in}}%
\pgfusepath{fill}%
\end{pgfscope}%
\begin{pgfscope}%
\pgfpathrectangle{\pgfqpoint{1.432000in}{0.528000in}}{\pgfqpoint{3.696000in}{3.696000in}} %
\pgfusepath{clip}%
\pgfsetbuttcap%
\pgfsetroundjoin%
\definecolor{currentfill}{rgb}{0.260571,0.246922,0.522828}%
\pgfsetfillcolor{currentfill}%
\pgfsetlinewidth{0.000000pt}%
\definecolor{currentstroke}{rgb}{0.000000,0.000000,0.000000}%
\pgfsetstrokecolor{currentstroke}%
\pgfsetdash{}{0pt}%
\pgfpathmoveto{\pgfqpoint{1.604435in}{4.056000in}}%
\pgfpathlineto{\pgfqpoint{1.602218in}{4.059841in}}%
\pgfpathlineto{\pgfqpoint{1.597782in}{4.059841in}}%
\pgfpathlineto{\pgfqpoint{1.595565in}{4.056000in}}%
\pgfpathlineto{\pgfqpoint{1.597782in}{4.052159in}}%
\pgfpathlineto{\pgfqpoint{1.602218in}{4.052159in}}%
\pgfpathlineto{\pgfqpoint{1.604435in}{4.056000in}}%
\pgfpathlineto{\pgfqpoint{1.602218in}{4.059841in}}%
\pgfusepath{fill}%
\end{pgfscope}%
\begin{pgfscope}%
\pgfpathrectangle{\pgfqpoint{1.432000in}{0.528000in}}{\pgfqpoint{3.696000in}{3.696000in}} %
\pgfusepath{clip}%
\pgfsetbuttcap%
\pgfsetroundjoin%
\definecolor{currentfill}{rgb}{0.283229,0.120777,0.440584}%
\pgfsetfillcolor{currentfill}%
\pgfsetlinewidth{0.000000pt}%
\definecolor{currentstroke}{rgb}{0.000000,0.000000,0.000000}%
\pgfsetstrokecolor{currentstroke}%
\pgfsetdash{}{0pt}%
\pgfpathmoveto{\pgfqpoint{1.712822in}{4.056000in}}%
\pgfpathlineto{\pgfqpoint{1.712822in}{3.987530in}}%
\pgfpathlineto{\pgfqpoint{1.721693in}{3.991965in}}%
\pgfpathlineto{\pgfqpoint{1.708387in}{3.947613in}}%
\pgfpathlineto{\pgfqpoint{1.695081in}{3.991965in}}%
\pgfpathlineto{\pgfqpoint{1.703952in}{3.987530in}}%
\pgfpathlineto{\pgfqpoint{1.703952in}{4.056000in}}%
\pgfpathlineto{\pgfqpoint{1.712822in}{4.056000in}}%
\pgfusepath{fill}%
\end{pgfscope}%
\begin{pgfscope}%
\pgfpathrectangle{\pgfqpoint{1.432000in}{0.528000in}}{\pgfqpoint{3.696000in}{3.696000in}} %
\pgfusepath{clip}%
\pgfsetbuttcap%
\pgfsetroundjoin%
\definecolor{currentfill}{rgb}{0.270595,0.214069,0.507052}%
\pgfsetfillcolor{currentfill}%
\pgfsetlinewidth{0.000000pt}%
\definecolor{currentstroke}{rgb}{0.000000,0.000000,0.000000}%
\pgfsetstrokecolor{currentstroke}%
\pgfsetdash{}{0pt}%
\pgfpathmoveto{\pgfqpoint{1.712822in}{4.056000in}}%
\pgfpathlineto{\pgfqpoint{1.710605in}{4.059841in}}%
\pgfpathlineto{\pgfqpoint{1.706169in}{4.059841in}}%
\pgfpathlineto{\pgfqpoint{1.703952in}{4.056000in}}%
\pgfpathlineto{\pgfqpoint{1.706169in}{4.052159in}}%
\pgfpathlineto{\pgfqpoint{1.710605in}{4.052159in}}%
\pgfpathlineto{\pgfqpoint{1.712822in}{4.056000in}}%
\pgfpathlineto{\pgfqpoint{1.710605in}{4.059841in}}%
\pgfusepath{fill}%
\end{pgfscope}%
\begin{pgfscope}%
\pgfpathrectangle{\pgfqpoint{1.432000in}{0.528000in}}{\pgfqpoint{3.696000in}{3.696000in}} %
\pgfusepath{clip}%
\pgfsetbuttcap%
\pgfsetroundjoin%
\definecolor{currentfill}{rgb}{0.273809,0.031497,0.358853}%
\pgfsetfillcolor{currentfill}%
\pgfsetlinewidth{0.000000pt}%
\definecolor{currentstroke}{rgb}{0.000000,0.000000,0.000000}%
\pgfsetstrokecolor{currentstroke}%
\pgfsetdash{}{0pt}%
\pgfpathmoveto{\pgfqpoint{1.821209in}{4.056000in}}%
\pgfpathlineto{\pgfqpoint{1.821209in}{3.987530in}}%
\pgfpathlineto{\pgfqpoint{1.830080in}{3.991965in}}%
\pgfpathlineto{\pgfqpoint{1.816774in}{3.947613in}}%
\pgfpathlineto{\pgfqpoint{1.803469in}{3.991965in}}%
\pgfpathlineto{\pgfqpoint{1.812339in}{3.987530in}}%
\pgfpathlineto{\pgfqpoint{1.812339in}{4.056000in}}%
\pgfpathlineto{\pgfqpoint{1.821209in}{4.056000in}}%
\pgfusepath{fill}%
\end{pgfscope}%
\begin{pgfscope}%
\pgfpathrectangle{\pgfqpoint{1.432000in}{0.528000in}}{\pgfqpoint{3.696000in}{3.696000in}} %
\pgfusepath{clip}%
\pgfsetbuttcap%
\pgfsetroundjoin%
\definecolor{currentfill}{rgb}{0.262138,0.242286,0.520837}%
\pgfsetfillcolor{currentfill}%
\pgfsetlinewidth{0.000000pt}%
\definecolor{currentstroke}{rgb}{0.000000,0.000000,0.000000}%
\pgfsetstrokecolor{currentstroke}%
\pgfsetdash{}{0pt}%
\pgfpathmoveto{\pgfqpoint{1.821209in}{4.056000in}}%
\pgfpathlineto{\pgfqpoint{1.818992in}{4.059841in}}%
\pgfpathlineto{\pgfqpoint{1.814557in}{4.059841in}}%
\pgfpathlineto{\pgfqpoint{1.812339in}{4.056000in}}%
\pgfpathlineto{\pgfqpoint{1.814557in}{4.052159in}}%
\pgfpathlineto{\pgfqpoint{1.818992in}{4.052159in}}%
\pgfpathlineto{\pgfqpoint{1.821209in}{4.056000in}}%
\pgfpathlineto{\pgfqpoint{1.818992in}{4.059841in}}%
\pgfusepath{fill}%
\end{pgfscope}%
\begin{pgfscope}%
\pgfpathrectangle{\pgfqpoint{1.432000in}{0.528000in}}{\pgfqpoint{3.696000in}{3.696000in}} %
\pgfusepath{clip}%
\pgfsetbuttcap%
\pgfsetroundjoin%
\definecolor{currentfill}{rgb}{0.269944,0.014625,0.341379}%
\pgfsetfillcolor{currentfill}%
\pgfsetlinewidth{0.000000pt}%
\definecolor{currentstroke}{rgb}{0.000000,0.000000,0.000000}%
\pgfsetstrokecolor{currentstroke}%
\pgfsetdash{}{0pt}%
\pgfpathmoveto{\pgfqpoint{1.928297in}{4.059136in}}%
\pgfpathlineto{\pgfqpoint{2.008459in}{3.978975in}}%
\pgfpathlineto{\pgfqpoint{2.011595in}{3.988383in}}%
\pgfpathlineto{\pgfqpoint{2.033548in}{3.947613in}}%
\pgfpathlineto{\pgfqpoint{1.992778in}{3.969566in}}%
\pgfpathlineto{\pgfqpoint{2.002187in}{3.972702in}}%
\pgfpathlineto{\pgfqpoint{1.922025in}{4.052864in}}%
\pgfpathlineto{\pgfqpoint{1.928297in}{4.059136in}}%
\pgfusepath{fill}%
\end{pgfscope}%
\begin{pgfscope}%
\pgfpathrectangle{\pgfqpoint{1.432000in}{0.528000in}}{\pgfqpoint{3.696000in}{3.696000in}} %
\pgfusepath{clip}%
\pgfsetbuttcap%
\pgfsetroundjoin%
\definecolor{currentfill}{rgb}{0.276022,0.044167,0.370164}%
\pgfsetfillcolor{currentfill}%
\pgfsetlinewidth{0.000000pt}%
\definecolor{currentstroke}{rgb}{0.000000,0.000000,0.000000}%
\pgfsetstrokecolor{currentstroke}%
\pgfsetdash{}{0pt}%
\pgfpathmoveto{\pgfqpoint{1.925161in}{4.060435in}}%
\pgfpathlineto{\pgfqpoint{1.993632in}{4.060435in}}%
\pgfpathlineto{\pgfqpoint{1.989196in}{4.069306in}}%
\pgfpathlineto{\pgfqpoint{2.033548in}{4.056000in}}%
\pgfpathlineto{\pgfqpoint{1.989196in}{4.042694in}}%
\pgfpathlineto{\pgfqpoint{1.993632in}{4.051565in}}%
\pgfpathlineto{\pgfqpoint{1.925161in}{4.051565in}}%
\pgfpathlineto{\pgfqpoint{1.925161in}{4.060435in}}%
\pgfusepath{fill}%
\end{pgfscope}%
\begin{pgfscope}%
\pgfpathrectangle{\pgfqpoint{1.432000in}{0.528000in}}{\pgfqpoint{3.696000in}{3.696000in}} %
\pgfusepath{clip}%
\pgfsetbuttcap%
\pgfsetroundjoin%
\definecolor{currentfill}{rgb}{0.271305,0.019942,0.347269}%
\pgfsetfillcolor{currentfill}%
\pgfsetlinewidth{0.000000pt}%
\definecolor{currentstroke}{rgb}{0.000000,0.000000,0.000000}%
\pgfsetstrokecolor{currentstroke}%
\pgfsetdash{}{0pt}%
\pgfpathmoveto{\pgfqpoint{2.036685in}{4.059136in}}%
\pgfpathlineto{\pgfqpoint{2.116846in}{3.978975in}}%
\pgfpathlineto{\pgfqpoint{2.119982in}{3.988383in}}%
\pgfpathlineto{\pgfqpoint{2.141935in}{3.947613in}}%
\pgfpathlineto{\pgfqpoint{2.101165in}{3.969566in}}%
\pgfpathlineto{\pgfqpoint{2.110574in}{3.972702in}}%
\pgfpathlineto{\pgfqpoint{2.030412in}{4.052864in}}%
\pgfpathlineto{\pgfqpoint{2.036685in}{4.059136in}}%
\pgfusepath{fill}%
\end{pgfscope}%
\begin{pgfscope}%
\pgfpathrectangle{\pgfqpoint{1.432000in}{0.528000in}}{\pgfqpoint{3.696000in}{3.696000in}} %
\pgfusepath{clip}%
\pgfsetbuttcap%
\pgfsetroundjoin%
\definecolor{currentfill}{rgb}{0.263663,0.237631,0.518762}%
\pgfsetfillcolor{currentfill}%
\pgfsetlinewidth{0.000000pt}%
\definecolor{currentstroke}{rgb}{0.000000,0.000000,0.000000}%
\pgfsetstrokecolor{currentstroke}%
\pgfsetdash{}{0pt}%
\pgfpathmoveto{\pgfqpoint{2.033548in}{4.060435in}}%
\pgfpathlineto{\pgfqpoint{2.102019in}{4.060435in}}%
\pgfpathlineto{\pgfqpoint{2.097583in}{4.069306in}}%
\pgfpathlineto{\pgfqpoint{2.141935in}{4.056000in}}%
\pgfpathlineto{\pgfqpoint{2.097583in}{4.042694in}}%
\pgfpathlineto{\pgfqpoint{2.102019in}{4.051565in}}%
\pgfpathlineto{\pgfqpoint{2.033548in}{4.051565in}}%
\pgfpathlineto{\pgfqpoint{2.033548in}{4.060435in}}%
\pgfusepath{fill}%
\end{pgfscope}%
\begin{pgfscope}%
\pgfpathrectangle{\pgfqpoint{1.432000in}{0.528000in}}{\pgfqpoint{3.696000in}{3.696000in}} %
\pgfusepath{clip}%
\pgfsetbuttcap%
\pgfsetroundjoin%
\definecolor{currentfill}{rgb}{0.265145,0.232956,0.516599}%
\pgfsetfillcolor{currentfill}%
\pgfsetlinewidth{0.000000pt}%
\definecolor{currentstroke}{rgb}{0.000000,0.000000,0.000000}%
\pgfsetstrokecolor{currentstroke}%
\pgfsetdash{}{0pt}%
\pgfpathmoveto{\pgfqpoint{2.141935in}{4.060435in}}%
\pgfpathlineto{\pgfqpoint{2.210406in}{4.060435in}}%
\pgfpathlineto{\pgfqpoint{2.205971in}{4.069306in}}%
\pgfpathlineto{\pgfqpoint{2.250323in}{4.056000in}}%
\pgfpathlineto{\pgfqpoint{2.205971in}{4.042694in}}%
\pgfpathlineto{\pgfqpoint{2.210406in}{4.051565in}}%
\pgfpathlineto{\pgfqpoint{2.141935in}{4.051565in}}%
\pgfpathlineto{\pgfqpoint{2.141935in}{4.060435in}}%
\pgfusepath{fill}%
\end{pgfscope}%
\begin{pgfscope}%
\pgfpathrectangle{\pgfqpoint{1.432000in}{0.528000in}}{\pgfqpoint{3.696000in}{3.696000in}} %
\pgfusepath{clip}%
\pgfsetbuttcap%
\pgfsetroundjoin%
\definecolor{currentfill}{rgb}{0.281887,0.150881,0.465405}%
\pgfsetfillcolor{currentfill}%
\pgfsetlinewidth{0.000000pt}%
\definecolor{currentstroke}{rgb}{0.000000,0.000000,0.000000}%
\pgfsetstrokecolor{currentstroke}%
\pgfsetdash{}{0pt}%
\pgfpathmoveto{\pgfqpoint{2.250323in}{4.060435in}}%
\pgfpathlineto{\pgfqpoint{2.318793in}{4.060435in}}%
\pgfpathlineto{\pgfqpoint{2.314358in}{4.069306in}}%
\pgfpathlineto{\pgfqpoint{2.358710in}{4.056000in}}%
\pgfpathlineto{\pgfqpoint{2.314358in}{4.042694in}}%
\pgfpathlineto{\pgfqpoint{2.318793in}{4.051565in}}%
\pgfpathlineto{\pgfqpoint{2.250323in}{4.051565in}}%
\pgfpathlineto{\pgfqpoint{2.250323in}{4.060435in}}%
\pgfusepath{fill}%
\end{pgfscope}%
\begin{pgfscope}%
\pgfpathrectangle{\pgfqpoint{1.432000in}{0.528000in}}{\pgfqpoint{3.696000in}{3.696000in}} %
\pgfusepath{clip}%
\pgfsetbuttcap%
\pgfsetroundjoin%
\definecolor{currentfill}{rgb}{0.277018,0.050344,0.375715}%
\pgfsetfillcolor{currentfill}%
\pgfsetlinewidth{0.000000pt}%
\definecolor{currentstroke}{rgb}{0.000000,0.000000,0.000000}%
\pgfsetstrokecolor{currentstroke}%
\pgfsetdash{}{0pt}%
\pgfpathmoveto{\pgfqpoint{2.363145in}{4.056000in}}%
\pgfpathlineto{\pgfqpoint{2.360927in}{4.059841in}}%
\pgfpathlineto{\pgfqpoint{2.356492in}{4.059841in}}%
\pgfpathlineto{\pgfqpoint{2.354274in}{4.056000in}}%
\pgfpathlineto{\pgfqpoint{2.356492in}{4.052159in}}%
\pgfpathlineto{\pgfqpoint{2.360927in}{4.052159in}}%
\pgfpathlineto{\pgfqpoint{2.363145in}{4.056000in}}%
\pgfpathlineto{\pgfqpoint{2.360927in}{4.059841in}}%
\pgfusepath{fill}%
\end{pgfscope}%
\begin{pgfscope}%
\pgfpathrectangle{\pgfqpoint{1.432000in}{0.528000in}}{\pgfqpoint{3.696000in}{3.696000in}} %
\pgfusepath{clip}%
\pgfsetbuttcap%
\pgfsetroundjoin%
\definecolor{currentfill}{rgb}{0.276022,0.044167,0.370164}%
\pgfsetfillcolor{currentfill}%
\pgfsetlinewidth{0.000000pt}%
\definecolor{currentstroke}{rgb}{0.000000,0.000000,0.000000}%
\pgfsetstrokecolor{currentstroke}%
\pgfsetdash{}{0pt}%
\pgfpathmoveto{\pgfqpoint{2.358710in}{4.060435in}}%
\pgfpathlineto{\pgfqpoint{2.427180in}{4.060435in}}%
\pgfpathlineto{\pgfqpoint{2.422745in}{4.069306in}}%
\pgfpathlineto{\pgfqpoint{2.467097in}{4.056000in}}%
\pgfpathlineto{\pgfqpoint{2.422745in}{4.042694in}}%
\pgfpathlineto{\pgfqpoint{2.427180in}{4.051565in}}%
\pgfpathlineto{\pgfqpoint{2.358710in}{4.051565in}}%
\pgfpathlineto{\pgfqpoint{2.358710in}{4.060435in}}%
\pgfusepath{fill}%
\end{pgfscope}%
\begin{pgfscope}%
\pgfpathrectangle{\pgfqpoint{1.432000in}{0.528000in}}{\pgfqpoint{3.696000in}{3.696000in}} %
\pgfusepath{clip}%
\pgfsetbuttcap%
\pgfsetroundjoin%
\definecolor{currentfill}{rgb}{0.278012,0.180367,0.486697}%
\pgfsetfillcolor{currentfill}%
\pgfsetlinewidth{0.000000pt}%
\definecolor{currentstroke}{rgb}{0.000000,0.000000,0.000000}%
\pgfsetstrokecolor{currentstroke}%
\pgfsetdash{}{0pt}%
\pgfpathmoveto{\pgfqpoint{2.471532in}{4.056000in}}%
\pgfpathlineto{\pgfqpoint{2.469314in}{4.059841in}}%
\pgfpathlineto{\pgfqpoint{2.464879in}{4.059841in}}%
\pgfpathlineto{\pgfqpoint{2.462662in}{4.056000in}}%
\pgfpathlineto{\pgfqpoint{2.464879in}{4.052159in}}%
\pgfpathlineto{\pgfqpoint{2.469314in}{4.052159in}}%
\pgfpathlineto{\pgfqpoint{2.471532in}{4.056000in}}%
\pgfpathlineto{\pgfqpoint{2.469314in}{4.059841in}}%
\pgfusepath{fill}%
\end{pgfscope}%
\begin{pgfscope}%
\pgfpathrectangle{\pgfqpoint{1.432000in}{0.528000in}}{\pgfqpoint{3.696000in}{3.696000in}} %
\pgfusepath{clip}%
\pgfsetbuttcap%
\pgfsetroundjoin%
\definecolor{currentfill}{rgb}{0.279574,0.170599,0.479997}%
\pgfsetfillcolor{currentfill}%
\pgfsetlinewidth{0.000000pt}%
\definecolor{currentstroke}{rgb}{0.000000,0.000000,0.000000}%
\pgfsetstrokecolor{currentstroke}%
\pgfsetdash{}{0pt}%
\pgfpathmoveto{\pgfqpoint{2.579919in}{4.056000in}}%
\pgfpathlineto{\pgfqpoint{2.577701in}{4.059841in}}%
\pgfpathlineto{\pgfqpoint{2.573266in}{4.059841in}}%
\pgfpathlineto{\pgfqpoint{2.571049in}{4.056000in}}%
\pgfpathlineto{\pgfqpoint{2.573266in}{4.052159in}}%
\pgfpathlineto{\pgfqpoint{2.577701in}{4.052159in}}%
\pgfpathlineto{\pgfqpoint{2.579919in}{4.056000in}}%
\pgfpathlineto{\pgfqpoint{2.577701in}{4.059841in}}%
\pgfusepath{fill}%
\end{pgfscope}%
\begin{pgfscope}%
\pgfpathrectangle{\pgfqpoint{1.432000in}{0.528000in}}{\pgfqpoint{3.696000in}{3.696000in}} %
\pgfusepath{clip}%
\pgfsetbuttcap%
\pgfsetroundjoin%
\definecolor{currentfill}{rgb}{0.281887,0.150881,0.465405}%
\pgfsetfillcolor{currentfill}%
\pgfsetlinewidth{0.000000pt}%
\definecolor{currentstroke}{rgb}{0.000000,0.000000,0.000000}%
\pgfsetstrokecolor{currentstroke}%
\pgfsetdash{}{0pt}%
\pgfpathmoveto{\pgfqpoint{2.688306in}{4.056000in}}%
\pgfpathlineto{\pgfqpoint{2.686089in}{4.059841in}}%
\pgfpathlineto{\pgfqpoint{2.681653in}{4.059841in}}%
\pgfpathlineto{\pgfqpoint{2.679436in}{4.056000in}}%
\pgfpathlineto{\pgfqpoint{2.681653in}{4.052159in}}%
\pgfpathlineto{\pgfqpoint{2.686089in}{4.052159in}}%
\pgfpathlineto{\pgfqpoint{2.688306in}{4.056000in}}%
\pgfpathlineto{\pgfqpoint{2.686089in}{4.059841in}}%
\pgfusepath{fill}%
\end{pgfscope}%
\begin{pgfscope}%
\pgfpathrectangle{\pgfqpoint{1.432000in}{0.528000in}}{\pgfqpoint{3.696000in}{3.696000in}} %
\pgfusepath{clip}%
\pgfsetbuttcap%
\pgfsetroundjoin%
\definecolor{currentfill}{rgb}{0.280267,0.073417,0.397163}%
\pgfsetfillcolor{currentfill}%
\pgfsetlinewidth{0.000000pt}%
\definecolor{currentstroke}{rgb}{0.000000,0.000000,0.000000}%
\pgfsetstrokecolor{currentstroke}%
\pgfsetdash{}{0pt}%
\pgfpathmoveto{\pgfqpoint{2.796693in}{4.056000in}}%
\pgfpathlineto{\pgfqpoint{2.794476in}{4.059841in}}%
\pgfpathlineto{\pgfqpoint{2.790040in}{4.059841in}}%
\pgfpathlineto{\pgfqpoint{2.787823in}{4.056000in}}%
\pgfpathlineto{\pgfqpoint{2.790040in}{4.052159in}}%
\pgfpathlineto{\pgfqpoint{2.794476in}{4.052159in}}%
\pgfpathlineto{\pgfqpoint{2.796693in}{4.056000in}}%
\pgfpathlineto{\pgfqpoint{2.794476in}{4.059841in}}%
\pgfusepath{fill}%
\end{pgfscope}%
\begin{pgfscope}%
\pgfpathrectangle{\pgfqpoint{1.432000in}{0.528000in}}{\pgfqpoint{3.696000in}{3.696000in}} %
\pgfusepath{clip}%
\pgfsetbuttcap%
\pgfsetroundjoin%
\definecolor{currentfill}{rgb}{0.283091,0.110553,0.431554}%
\pgfsetfillcolor{currentfill}%
\pgfsetlinewidth{0.000000pt}%
\definecolor{currentstroke}{rgb}{0.000000,0.000000,0.000000}%
\pgfsetstrokecolor{currentstroke}%
\pgfsetdash{}{0pt}%
\pgfpathmoveto{\pgfqpoint{2.900645in}{4.060435in}}%
\pgfpathlineto{\pgfqpoint{2.969115in}{4.060435in}}%
\pgfpathlineto{\pgfqpoint{2.964680in}{4.069306in}}%
\pgfpathlineto{\pgfqpoint{3.009032in}{4.056000in}}%
\pgfpathlineto{\pgfqpoint{2.964680in}{4.042694in}}%
\pgfpathlineto{\pgfqpoint{2.969115in}{4.051565in}}%
\pgfpathlineto{\pgfqpoint{2.900645in}{4.051565in}}%
\pgfpathlineto{\pgfqpoint{2.900645in}{4.060435in}}%
\pgfusepath{fill}%
\end{pgfscope}%
\begin{pgfscope}%
\pgfpathrectangle{\pgfqpoint{1.432000in}{0.528000in}}{\pgfqpoint{3.696000in}{3.696000in}} %
\pgfusepath{clip}%
\pgfsetbuttcap%
\pgfsetroundjoin%
\definecolor{currentfill}{rgb}{0.282656,0.100196,0.422160}%
\pgfsetfillcolor{currentfill}%
\pgfsetlinewidth{0.000000pt}%
\definecolor{currentstroke}{rgb}{0.000000,0.000000,0.000000}%
\pgfsetstrokecolor{currentstroke}%
\pgfsetdash{}{0pt}%
\pgfpathmoveto{\pgfqpoint{3.009032in}{4.060435in}}%
\pgfpathlineto{\pgfqpoint{3.077503in}{4.060435in}}%
\pgfpathlineto{\pgfqpoint{3.073067in}{4.069306in}}%
\pgfpathlineto{\pgfqpoint{3.117419in}{4.056000in}}%
\pgfpathlineto{\pgfqpoint{3.073067in}{4.042694in}}%
\pgfpathlineto{\pgfqpoint{3.077503in}{4.051565in}}%
\pgfpathlineto{\pgfqpoint{3.009032in}{4.051565in}}%
\pgfpathlineto{\pgfqpoint{3.009032in}{4.060435in}}%
\pgfusepath{fill}%
\end{pgfscope}%
\begin{pgfscope}%
\pgfpathrectangle{\pgfqpoint{1.432000in}{0.528000in}}{\pgfqpoint{3.696000in}{3.696000in}} %
\pgfusepath{clip}%
\pgfsetbuttcap%
\pgfsetroundjoin%
\definecolor{currentfill}{rgb}{0.280267,0.073417,0.397163}%
\pgfsetfillcolor{currentfill}%
\pgfsetlinewidth{0.000000pt}%
\definecolor{currentstroke}{rgb}{0.000000,0.000000,0.000000}%
\pgfsetstrokecolor{currentstroke}%
\pgfsetdash{}{0pt}%
\pgfpathmoveto{\pgfqpoint{3.121855in}{4.056000in}}%
\pgfpathlineto{\pgfqpoint{3.119637in}{4.059841in}}%
\pgfpathlineto{\pgfqpoint{3.115202in}{4.059841in}}%
\pgfpathlineto{\pgfqpoint{3.112984in}{4.056000in}}%
\pgfpathlineto{\pgfqpoint{3.115202in}{4.052159in}}%
\pgfpathlineto{\pgfqpoint{3.119637in}{4.052159in}}%
\pgfpathlineto{\pgfqpoint{3.121855in}{4.056000in}}%
\pgfpathlineto{\pgfqpoint{3.119637in}{4.059841in}}%
\pgfusepath{fill}%
\end{pgfscope}%
\begin{pgfscope}%
\pgfpathrectangle{\pgfqpoint{1.432000in}{0.528000in}}{\pgfqpoint{3.696000in}{3.696000in}} %
\pgfusepath{clip}%
\pgfsetbuttcap%
\pgfsetroundjoin%
\definecolor{currentfill}{rgb}{0.265145,0.232956,0.516599}%
\pgfsetfillcolor{currentfill}%
\pgfsetlinewidth{0.000000pt}%
\definecolor{currentstroke}{rgb}{0.000000,0.000000,0.000000}%
\pgfsetstrokecolor{currentstroke}%
\pgfsetdash{}{0pt}%
\pgfpathmoveto{\pgfqpoint{3.117419in}{4.060435in}}%
\pgfpathlineto{\pgfqpoint{3.185890in}{4.060435in}}%
\pgfpathlineto{\pgfqpoint{3.181454in}{4.069306in}}%
\pgfpathlineto{\pgfqpoint{3.225806in}{4.056000in}}%
\pgfpathlineto{\pgfqpoint{3.181454in}{4.042694in}}%
\pgfpathlineto{\pgfqpoint{3.185890in}{4.051565in}}%
\pgfpathlineto{\pgfqpoint{3.117419in}{4.051565in}}%
\pgfpathlineto{\pgfqpoint{3.117419in}{4.060435in}}%
\pgfusepath{fill}%
\end{pgfscope}%
\begin{pgfscope}%
\pgfpathrectangle{\pgfqpoint{1.432000in}{0.528000in}}{\pgfqpoint{3.696000in}{3.696000in}} %
\pgfusepath{clip}%
\pgfsetbuttcap%
\pgfsetroundjoin%
\definecolor{currentfill}{rgb}{0.273809,0.031497,0.358853}%
\pgfsetfillcolor{currentfill}%
\pgfsetlinewidth{0.000000pt}%
\definecolor{currentstroke}{rgb}{0.000000,0.000000,0.000000}%
\pgfsetstrokecolor{currentstroke}%
\pgfsetdash{}{0pt}%
\pgfpathmoveto{\pgfqpoint{3.228943in}{4.059136in}}%
\pgfpathlineto{\pgfqpoint{3.309104in}{3.978975in}}%
\pgfpathlineto{\pgfqpoint{3.312240in}{3.988383in}}%
\pgfpathlineto{\pgfqpoint{3.334194in}{3.947613in}}%
\pgfpathlineto{\pgfqpoint{3.293423in}{3.969566in}}%
\pgfpathlineto{\pgfqpoint{3.302832in}{3.972702in}}%
\pgfpathlineto{\pgfqpoint{3.222670in}{4.052864in}}%
\pgfpathlineto{\pgfqpoint{3.228943in}{4.059136in}}%
\pgfusepath{fill}%
\end{pgfscope}%
\begin{pgfscope}%
\pgfpathrectangle{\pgfqpoint{1.432000in}{0.528000in}}{\pgfqpoint{3.696000in}{3.696000in}} %
\pgfusepath{clip}%
\pgfsetbuttcap%
\pgfsetroundjoin%
\definecolor{currentfill}{rgb}{0.283091,0.110553,0.431554}%
\pgfsetfillcolor{currentfill}%
\pgfsetlinewidth{0.000000pt}%
\definecolor{currentstroke}{rgb}{0.000000,0.000000,0.000000}%
\pgfsetstrokecolor{currentstroke}%
\pgfsetdash{}{0pt}%
\pgfpathmoveto{\pgfqpoint{3.230242in}{4.056000in}}%
\pgfpathlineto{\pgfqpoint{3.228024in}{4.059841in}}%
\pgfpathlineto{\pgfqpoint{3.223589in}{4.059841in}}%
\pgfpathlineto{\pgfqpoint{3.221371in}{4.056000in}}%
\pgfpathlineto{\pgfqpoint{3.223589in}{4.052159in}}%
\pgfpathlineto{\pgfqpoint{3.228024in}{4.052159in}}%
\pgfpathlineto{\pgfqpoint{3.230242in}{4.056000in}}%
\pgfpathlineto{\pgfqpoint{3.228024in}{4.059841in}}%
\pgfusepath{fill}%
\end{pgfscope}%
\begin{pgfscope}%
\pgfpathrectangle{\pgfqpoint{1.432000in}{0.528000in}}{\pgfqpoint{3.696000in}{3.696000in}} %
\pgfusepath{clip}%
\pgfsetbuttcap%
\pgfsetroundjoin%
\definecolor{currentfill}{rgb}{0.258965,0.251537,0.524736}%
\pgfsetfillcolor{currentfill}%
\pgfsetlinewidth{0.000000pt}%
\definecolor{currentstroke}{rgb}{0.000000,0.000000,0.000000}%
\pgfsetstrokecolor{currentstroke}%
\pgfsetdash{}{0pt}%
\pgfpathmoveto{\pgfqpoint{3.225806in}{4.060435in}}%
\pgfpathlineto{\pgfqpoint{3.294277in}{4.060435in}}%
\pgfpathlineto{\pgfqpoint{3.289842in}{4.069306in}}%
\pgfpathlineto{\pgfqpoint{3.334194in}{4.056000in}}%
\pgfpathlineto{\pgfqpoint{3.289842in}{4.042694in}}%
\pgfpathlineto{\pgfqpoint{3.294277in}{4.051565in}}%
\pgfpathlineto{\pgfqpoint{3.225806in}{4.051565in}}%
\pgfpathlineto{\pgfqpoint{3.225806in}{4.060435in}}%
\pgfusepath{fill}%
\end{pgfscope}%
\begin{pgfscope}%
\pgfpathrectangle{\pgfqpoint{1.432000in}{0.528000in}}{\pgfqpoint{3.696000in}{3.696000in}} %
\pgfusepath{clip}%
\pgfsetbuttcap%
\pgfsetroundjoin%
\definecolor{currentfill}{rgb}{0.277134,0.185228,0.489898}%
\pgfsetfillcolor{currentfill}%
\pgfsetlinewidth{0.000000pt}%
\definecolor{currentstroke}{rgb}{0.000000,0.000000,0.000000}%
\pgfsetstrokecolor{currentstroke}%
\pgfsetdash{}{0pt}%
\pgfpathmoveto{\pgfqpoint{3.337330in}{4.059136in}}%
\pgfpathlineto{\pgfqpoint{3.417491in}{3.978975in}}%
\pgfpathlineto{\pgfqpoint{3.420628in}{3.988383in}}%
\pgfpathlineto{\pgfqpoint{3.442581in}{3.947613in}}%
\pgfpathlineto{\pgfqpoint{3.401811in}{3.969566in}}%
\pgfpathlineto{\pgfqpoint{3.411219in}{3.972702in}}%
\pgfpathlineto{\pgfqpoint{3.331057in}{4.052864in}}%
\pgfpathlineto{\pgfqpoint{3.337330in}{4.059136in}}%
\pgfusepath{fill}%
\end{pgfscope}%
\begin{pgfscope}%
\pgfpathrectangle{\pgfqpoint{1.432000in}{0.528000in}}{\pgfqpoint{3.696000in}{3.696000in}} %
\pgfusepath{clip}%
\pgfsetbuttcap%
\pgfsetroundjoin%
\definecolor{currentfill}{rgb}{0.283197,0.115680,0.436115}%
\pgfsetfillcolor{currentfill}%
\pgfsetlinewidth{0.000000pt}%
\definecolor{currentstroke}{rgb}{0.000000,0.000000,0.000000}%
\pgfsetstrokecolor{currentstroke}%
\pgfsetdash{}{0pt}%
\pgfpathmoveto{\pgfqpoint{3.338629in}{4.056000in}}%
\pgfpathlineto{\pgfqpoint{3.336411in}{4.059841in}}%
\pgfpathlineto{\pgfqpoint{3.331976in}{4.059841in}}%
\pgfpathlineto{\pgfqpoint{3.329758in}{4.056000in}}%
\pgfpathlineto{\pgfqpoint{3.331976in}{4.052159in}}%
\pgfpathlineto{\pgfqpoint{3.336411in}{4.052159in}}%
\pgfpathlineto{\pgfqpoint{3.338629in}{4.056000in}}%
\pgfpathlineto{\pgfqpoint{3.336411in}{4.059841in}}%
\pgfusepath{fill}%
\end{pgfscope}%
\begin{pgfscope}%
\pgfpathrectangle{\pgfqpoint{1.432000in}{0.528000in}}{\pgfqpoint{3.696000in}{3.696000in}} %
\pgfusepath{clip}%
\pgfsetbuttcap%
\pgfsetroundjoin%
\definecolor{currentfill}{rgb}{0.282656,0.100196,0.422160}%
\pgfsetfillcolor{currentfill}%
\pgfsetlinewidth{0.000000pt}%
\definecolor{currentstroke}{rgb}{0.000000,0.000000,0.000000}%
\pgfsetstrokecolor{currentstroke}%
\pgfsetdash{}{0pt}%
\pgfpathmoveto{\pgfqpoint{3.334194in}{4.060435in}}%
\pgfpathlineto{\pgfqpoint{3.402664in}{4.060435in}}%
\pgfpathlineto{\pgfqpoint{3.398229in}{4.069306in}}%
\pgfpathlineto{\pgfqpoint{3.442581in}{4.056000in}}%
\pgfpathlineto{\pgfqpoint{3.398229in}{4.042694in}}%
\pgfpathlineto{\pgfqpoint{3.402664in}{4.051565in}}%
\pgfpathlineto{\pgfqpoint{3.334194in}{4.051565in}}%
\pgfpathlineto{\pgfqpoint{3.334194in}{4.060435in}}%
\pgfusepath{fill}%
\end{pgfscope}%
\begin{pgfscope}%
\pgfpathrectangle{\pgfqpoint{1.432000in}{0.528000in}}{\pgfqpoint{3.696000in}{3.696000in}} %
\pgfusepath{clip}%
\pgfsetbuttcap%
\pgfsetroundjoin%
\definecolor{currentfill}{rgb}{0.233603,0.313828,0.543914}%
\pgfsetfillcolor{currentfill}%
\pgfsetlinewidth{0.000000pt}%
\definecolor{currentstroke}{rgb}{0.000000,0.000000,0.000000}%
\pgfsetstrokecolor{currentstroke}%
\pgfsetdash{}{0pt}%
\pgfpathmoveto{\pgfqpoint{3.447016in}{4.056000in}}%
\pgfpathlineto{\pgfqpoint{3.447016in}{3.987530in}}%
\pgfpathlineto{\pgfqpoint{3.455886in}{3.991965in}}%
\pgfpathlineto{\pgfqpoint{3.442581in}{3.947613in}}%
\pgfpathlineto{\pgfqpoint{3.429275in}{3.991965in}}%
\pgfpathlineto{\pgfqpoint{3.438145in}{3.987530in}}%
\pgfpathlineto{\pgfqpoint{3.438145in}{4.056000in}}%
\pgfpathlineto{\pgfqpoint{3.447016in}{4.056000in}}%
\pgfusepath{fill}%
\end{pgfscope}%
\begin{pgfscope}%
\pgfpathrectangle{\pgfqpoint{1.432000in}{0.528000in}}{\pgfqpoint{3.696000in}{3.696000in}} %
\pgfusepath{clip}%
\pgfsetbuttcap%
\pgfsetroundjoin%
\definecolor{currentfill}{rgb}{0.208623,0.367752,0.552675}%
\pgfsetfillcolor{currentfill}%
\pgfsetlinewidth{0.000000pt}%
\definecolor{currentstroke}{rgb}{0.000000,0.000000,0.000000}%
\pgfsetstrokecolor{currentstroke}%
\pgfsetdash{}{0pt}%
\pgfpathmoveto{\pgfqpoint{3.447016in}{4.056000in}}%
\pgfpathlineto{\pgfqpoint{3.444798in}{4.059841in}}%
\pgfpathlineto{\pgfqpoint{3.440363in}{4.059841in}}%
\pgfpathlineto{\pgfqpoint{3.438145in}{4.056000in}}%
\pgfpathlineto{\pgfqpoint{3.440363in}{4.052159in}}%
\pgfpathlineto{\pgfqpoint{3.444798in}{4.052159in}}%
\pgfpathlineto{\pgfqpoint{3.447016in}{4.056000in}}%
\pgfpathlineto{\pgfqpoint{3.444798in}{4.059841in}}%
\pgfusepath{fill}%
\end{pgfscope}%
\begin{pgfscope}%
\pgfpathrectangle{\pgfqpoint{1.432000in}{0.528000in}}{\pgfqpoint{3.696000in}{3.696000in}} %
\pgfusepath{clip}%
\pgfsetbuttcap%
\pgfsetroundjoin%
\definecolor{currentfill}{rgb}{0.128087,0.647749,0.523491}%
\pgfsetfillcolor{currentfill}%
\pgfsetlinewidth{0.000000pt}%
\definecolor{currentstroke}{rgb}{0.000000,0.000000,0.000000}%
\pgfsetstrokecolor{currentstroke}%
\pgfsetdash{}{0pt}%
\pgfpathmoveto{\pgfqpoint{3.555403in}{4.056000in}}%
\pgfpathlineto{\pgfqpoint{3.553185in}{4.059841in}}%
\pgfpathlineto{\pgfqpoint{3.548750in}{4.059841in}}%
\pgfpathlineto{\pgfqpoint{3.546533in}{4.056000in}}%
\pgfpathlineto{\pgfqpoint{3.548750in}{4.052159in}}%
\pgfpathlineto{\pgfqpoint{3.553185in}{4.052159in}}%
\pgfpathlineto{\pgfqpoint{3.555403in}{4.056000in}}%
\pgfpathlineto{\pgfqpoint{3.553185in}{4.059841in}}%
\pgfusepath{fill}%
\end{pgfscope}%
\begin{pgfscope}%
\pgfpathrectangle{\pgfqpoint{1.432000in}{0.528000in}}{\pgfqpoint{3.696000in}{3.696000in}} %
\pgfusepath{clip}%
\pgfsetbuttcap%
\pgfsetroundjoin%
\definecolor{currentfill}{rgb}{0.147607,0.511733,0.557049}%
\pgfsetfillcolor{currentfill}%
\pgfsetlinewidth{0.000000pt}%
\definecolor{currentstroke}{rgb}{0.000000,0.000000,0.000000}%
\pgfsetstrokecolor{currentstroke}%
\pgfsetdash{}{0pt}%
\pgfpathmoveto{\pgfqpoint{3.663790in}{4.056000in}}%
\pgfpathlineto{\pgfqpoint{3.661572in}{4.059841in}}%
\pgfpathlineto{\pgfqpoint{3.657137in}{4.059841in}}%
\pgfpathlineto{\pgfqpoint{3.654920in}{4.056000in}}%
\pgfpathlineto{\pgfqpoint{3.657137in}{4.052159in}}%
\pgfpathlineto{\pgfqpoint{3.661572in}{4.052159in}}%
\pgfpathlineto{\pgfqpoint{3.663790in}{4.056000in}}%
\pgfpathlineto{\pgfqpoint{3.661572in}{4.059841in}}%
\pgfusepath{fill}%
\end{pgfscope}%
\begin{pgfscope}%
\pgfpathrectangle{\pgfqpoint{1.432000in}{0.528000in}}{\pgfqpoint{3.696000in}{3.696000in}} %
\pgfusepath{clip}%
\pgfsetbuttcap%
\pgfsetroundjoin%
\definecolor{currentfill}{rgb}{0.266580,0.228262,0.514349}%
\pgfsetfillcolor{currentfill}%
\pgfsetlinewidth{0.000000pt}%
\definecolor{currentstroke}{rgb}{0.000000,0.000000,0.000000}%
\pgfsetstrokecolor{currentstroke}%
\pgfsetdash{}{0pt}%
\pgfpathmoveto{\pgfqpoint{3.767742in}{4.051565in}}%
\pgfpathlineto{\pgfqpoint{3.699272in}{4.051565in}}%
\pgfpathlineto{\pgfqpoint{3.703707in}{4.042694in}}%
\pgfpathlineto{\pgfqpoint{3.659355in}{4.056000in}}%
\pgfpathlineto{\pgfqpoint{3.703707in}{4.069306in}}%
\pgfpathlineto{\pgfqpoint{3.699272in}{4.060435in}}%
\pgfpathlineto{\pgfqpoint{3.767742in}{4.060435in}}%
\pgfpathlineto{\pgfqpoint{3.767742in}{4.051565in}}%
\pgfusepath{fill}%
\end{pgfscope}%
\begin{pgfscope}%
\pgfpathrectangle{\pgfqpoint{1.432000in}{0.528000in}}{\pgfqpoint{3.696000in}{3.696000in}} %
\pgfusepath{clip}%
\pgfsetbuttcap%
\pgfsetroundjoin%
\definecolor{currentfill}{rgb}{0.281887,0.150881,0.465405}%
\pgfsetfillcolor{currentfill}%
\pgfsetlinewidth{0.000000pt}%
\definecolor{currentstroke}{rgb}{0.000000,0.000000,0.000000}%
\pgfsetstrokecolor{currentstroke}%
\pgfsetdash{}{0pt}%
\pgfpathmoveto{\pgfqpoint{3.772177in}{4.056000in}}%
\pgfpathlineto{\pgfqpoint{3.769960in}{4.059841in}}%
\pgfpathlineto{\pgfqpoint{3.765524in}{4.059841in}}%
\pgfpathlineto{\pgfqpoint{3.763307in}{4.056000in}}%
\pgfpathlineto{\pgfqpoint{3.765524in}{4.052159in}}%
\pgfpathlineto{\pgfqpoint{3.769960in}{4.052159in}}%
\pgfpathlineto{\pgfqpoint{3.772177in}{4.056000in}}%
\pgfpathlineto{\pgfqpoint{3.769960in}{4.059841in}}%
\pgfusepath{fill}%
\end{pgfscope}%
\begin{pgfscope}%
\pgfpathrectangle{\pgfqpoint{1.432000in}{0.528000in}}{\pgfqpoint{3.696000in}{3.696000in}} %
\pgfusepath{clip}%
\pgfsetbuttcap%
\pgfsetroundjoin%
\definecolor{currentfill}{rgb}{0.243113,0.292092,0.538516}%
\pgfsetfillcolor{currentfill}%
\pgfsetlinewidth{0.000000pt}%
\definecolor{currentstroke}{rgb}{0.000000,0.000000,0.000000}%
\pgfsetstrokecolor{currentstroke}%
\pgfsetdash{}{0pt}%
\pgfpathmoveto{\pgfqpoint{3.876129in}{4.051565in}}%
\pgfpathlineto{\pgfqpoint{3.807659in}{4.051565in}}%
\pgfpathlineto{\pgfqpoint{3.812094in}{4.042694in}}%
\pgfpathlineto{\pgfqpoint{3.767742in}{4.056000in}}%
\pgfpathlineto{\pgfqpoint{3.812094in}{4.069306in}}%
\pgfpathlineto{\pgfqpoint{3.807659in}{4.060435in}}%
\pgfpathlineto{\pgfqpoint{3.876129in}{4.060435in}}%
\pgfpathlineto{\pgfqpoint{3.876129in}{4.051565in}}%
\pgfusepath{fill}%
\end{pgfscope}%
\begin{pgfscope}%
\pgfpathrectangle{\pgfqpoint{1.432000in}{0.528000in}}{\pgfqpoint{3.696000in}{3.696000in}} %
\pgfusepath{clip}%
\pgfsetbuttcap%
\pgfsetroundjoin%
\definecolor{currentfill}{rgb}{0.277018,0.050344,0.375715}%
\pgfsetfillcolor{currentfill}%
\pgfsetlinewidth{0.000000pt}%
\definecolor{currentstroke}{rgb}{0.000000,0.000000,0.000000}%
\pgfsetstrokecolor{currentstroke}%
\pgfsetdash{}{0pt}%
\pgfpathmoveto{\pgfqpoint{3.984516in}{4.051565in}}%
\pgfpathlineto{\pgfqpoint{3.807659in}{4.051565in}}%
\pgfpathlineto{\pgfqpoint{3.812094in}{4.042694in}}%
\pgfpathlineto{\pgfqpoint{3.767742in}{4.056000in}}%
\pgfpathlineto{\pgfqpoint{3.812094in}{4.069306in}}%
\pgfpathlineto{\pgfqpoint{3.807659in}{4.060435in}}%
\pgfpathlineto{\pgfqpoint{3.984516in}{4.060435in}}%
\pgfpathlineto{\pgfqpoint{3.984516in}{4.051565in}}%
\pgfusepath{fill}%
\end{pgfscope}%
\begin{pgfscope}%
\pgfpathrectangle{\pgfqpoint{1.432000in}{0.528000in}}{\pgfqpoint{3.696000in}{3.696000in}} %
\pgfusepath{clip}%
\pgfsetbuttcap%
\pgfsetroundjoin%
\definecolor{currentfill}{rgb}{0.279566,0.067836,0.391917}%
\pgfsetfillcolor{currentfill}%
\pgfsetlinewidth{0.000000pt}%
\definecolor{currentstroke}{rgb}{0.000000,0.000000,0.000000}%
\pgfsetstrokecolor{currentstroke}%
\pgfsetdash{}{0pt}%
\pgfpathmoveto{\pgfqpoint{3.984516in}{4.051565in}}%
\pgfpathlineto{\pgfqpoint{3.916046in}{4.051565in}}%
\pgfpathlineto{\pgfqpoint{3.920481in}{4.042694in}}%
\pgfpathlineto{\pgfqpoint{3.876129in}{4.056000in}}%
\pgfpathlineto{\pgfqpoint{3.920481in}{4.069306in}}%
\pgfpathlineto{\pgfqpoint{3.916046in}{4.060435in}}%
\pgfpathlineto{\pgfqpoint{3.984516in}{4.060435in}}%
\pgfpathlineto{\pgfqpoint{3.984516in}{4.051565in}}%
\pgfusepath{fill}%
\end{pgfscope}%
\begin{pgfscope}%
\pgfpathrectangle{\pgfqpoint{1.432000in}{0.528000in}}{\pgfqpoint{3.696000in}{3.696000in}} %
\pgfusepath{clip}%
\pgfsetbuttcap%
\pgfsetroundjoin%
\definecolor{currentfill}{rgb}{0.180629,0.429975,0.557282}%
\pgfsetfillcolor{currentfill}%
\pgfsetlinewidth{0.000000pt}%
\definecolor{currentstroke}{rgb}{0.000000,0.000000,0.000000}%
\pgfsetstrokecolor{currentstroke}%
\pgfsetdash{}{0pt}%
\pgfpathmoveto{\pgfqpoint{4.092903in}{4.051565in}}%
\pgfpathlineto{\pgfqpoint{3.916046in}{4.051565in}}%
\pgfpathlineto{\pgfqpoint{3.920481in}{4.042694in}}%
\pgfpathlineto{\pgfqpoint{3.876129in}{4.056000in}}%
\pgfpathlineto{\pgfqpoint{3.920481in}{4.069306in}}%
\pgfpathlineto{\pgfqpoint{3.916046in}{4.060435in}}%
\pgfpathlineto{\pgfqpoint{4.092903in}{4.060435in}}%
\pgfpathlineto{\pgfqpoint{4.092903in}{4.051565in}}%
\pgfusepath{fill}%
\end{pgfscope}%
\begin{pgfscope}%
\pgfpathrectangle{\pgfqpoint{1.432000in}{0.528000in}}{\pgfqpoint{3.696000in}{3.696000in}} %
\pgfusepath{clip}%
\pgfsetbuttcap%
\pgfsetroundjoin%
\definecolor{currentfill}{rgb}{0.199430,0.387607,0.554642}%
\pgfsetfillcolor{currentfill}%
\pgfsetlinewidth{0.000000pt}%
\definecolor{currentstroke}{rgb}{0.000000,0.000000,0.000000}%
\pgfsetstrokecolor{currentstroke}%
\pgfsetdash{}{0pt}%
\pgfpathmoveto{\pgfqpoint{4.201290in}{4.051565in}}%
\pgfpathlineto{\pgfqpoint{4.024433in}{4.051565in}}%
\pgfpathlineto{\pgfqpoint{4.028868in}{4.042694in}}%
\pgfpathlineto{\pgfqpoint{3.984516in}{4.056000in}}%
\pgfpathlineto{\pgfqpoint{4.028868in}{4.069306in}}%
\pgfpathlineto{\pgfqpoint{4.024433in}{4.060435in}}%
\pgfpathlineto{\pgfqpoint{4.201290in}{4.060435in}}%
\pgfpathlineto{\pgfqpoint{4.201290in}{4.051565in}}%
\pgfusepath{fill}%
\end{pgfscope}%
\begin{pgfscope}%
\pgfpathrectangle{\pgfqpoint{1.432000in}{0.528000in}}{\pgfqpoint{3.696000in}{3.696000in}} %
\pgfusepath{clip}%
\pgfsetbuttcap%
\pgfsetroundjoin%
\definecolor{currentfill}{rgb}{0.266580,0.228262,0.514349}%
\pgfsetfillcolor{currentfill}%
\pgfsetlinewidth{0.000000pt}%
\definecolor{currentstroke}{rgb}{0.000000,0.000000,0.000000}%
\pgfsetstrokecolor{currentstroke}%
\pgfsetdash{}{0pt}%
\pgfpathmoveto{\pgfqpoint{4.309677in}{4.051565in}}%
\pgfpathlineto{\pgfqpoint{4.132820in}{4.051565in}}%
\pgfpathlineto{\pgfqpoint{4.137255in}{4.042694in}}%
\pgfpathlineto{\pgfqpoint{4.092903in}{4.056000in}}%
\pgfpathlineto{\pgfqpoint{4.137255in}{4.069306in}}%
\pgfpathlineto{\pgfqpoint{4.132820in}{4.060435in}}%
\pgfpathlineto{\pgfqpoint{4.309677in}{4.060435in}}%
\pgfpathlineto{\pgfqpoint{4.309677in}{4.051565in}}%
\pgfusepath{fill}%
\end{pgfscope}%
\begin{pgfscope}%
\pgfpathrectangle{\pgfqpoint{1.432000in}{0.528000in}}{\pgfqpoint{3.696000in}{3.696000in}} %
\pgfusepath{clip}%
\pgfsetbuttcap%
\pgfsetroundjoin%
\definecolor{currentfill}{rgb}{0.283072,0.130895,0.449241}%
\pgfsetfillcolor{currentfill}%
\pgfsetlinewidth{0.000000pt}%
\definecolor{currentstroke}{rgb}{0.000000,0.000000,0.000000}%
\pgfsetstrokecolor{currentstroke}%
\pgfsetdash{}{0pt}%
\pgfpathmoveto{\pgfqpoint{4.418065in}{4.051565in}}%
\pgfpathlineto{\pgfqpoint{4.132820in}{4.051565in}}%
\pgfpathlineto{\pgfqpoint{4.137255in}{4.042694in}}%
\pgfpathlineto{\pgfqpoint{4.092903in}{4.056000in}}%
\pgfpathlineto{\pgfqpoint{4.137255in}{4.069306in}}%
\pgfpathlineto{\pgfqpoint{4.132820in}{4.060435in}}%
\pgfpathlineto{\pgfqpoint{4.418065in}{4.060435in}}%
\pgfpathlineto{\pgfqpoint{4.418065in}{4.051565in}}%
\pgfusepath{fill}%
\end{pgfscope}%
\begin{pgfscope}%
\pgfpathrectangle{\pgfqpoint{1.432000in}{0.528000in}}{\pgfqpoint{3.696000in}{3.696000in}} %
\pgfusepath{clip}%
\pgfsetbuttcap%
\pgfsetroundjoin%
\definecolor{currentfill}{rgb}{0.283091,0.110553,0.431554}%
\pgfsetfillcolor{currentfill}%
\pgfsetlinewidth{0.000000pt}%
\definecolor{currentstroke}{rgb}{0.000000,0.000000,0.000000}%
\pgfsetstrokecolor{currentstroke}%
\pgfsetdash{}{0pt}%
\pgfpathmoveto{\pgfqpoint{4.418065in}{4.051565in}}%
\pgfpathlineto{\pgfqpoint{4.241207in}{4.051565in}}%
\pgfpathlineto{\pgfqpoint{4.245642in}{4.042694in}}%
\pgfpathlineto{\pgfqpoint{4.201290in}{4.056000in}}%
\pgfpathlineto{\pgfqpoint{4.245642in}{4.069306in}}%
\pgfpathlineto{\pgfqpoint{4.241207in}{4.060435in}}%
\pgfpathlineto{\pgfqpoint{4.418065in}{4.060435in}}%
\pgfpathlineto{\pgfqpoint{4.418065in}{4.051565in}}%
\pgfusepath{fill}%
\end{pgfscope}%
\begin{pgfscope}%
\pgfpathrectangle{\pgfqpoint{1.432000in}{0.528000in}}{\pgfqpoint{3.696000in}{3.696000in}} %
\pgfusepath{clip}%
\pgfsetbuttcap%
\pgfsetroundjoin%
\definecolor{currentfill}{rgb}{0.280868,0.160771,0.472899}%
\pgfsetfillcolor{currentfill}%
\pgfsetlinewidth{0.000000pt}%
\definecolor{currentstroke}{rgb}{0.000000,0.000000,0.000000}%
\pgfsetstrokecolor{currentstroke}%
\pgfsetdash{}{0pt}%
\pgfpathmoveto{\pgfqpoint{4.526452in}{4.051565in}}%
\pgfpathlineto{\pgfqpoint{4.241207in}{4.051565in}}%
\pgfpathlineto{\pgfqpoint{4.245642in}{4.042694in}}%
\pgfpathlineto{\pgfqpoint{4.201290in}{4.056000in}}%
\pgfpathlineto{\pgfqpoint{4.245642in}{4.069306in}}%
\pgfpathlineto{\pgfqpoint{4.241207in}{4.060435in}}%
\pgfpathlineto{\pgfqpoint{4.526452in}{4.060435in}}%
\pgfpathlineto{\pgfqpoint{4.526452in}{4.051565in}}%
\pgfusepath{fill}%
\end{pgfscope}%
\begin{pgfscope}%
\pgfpathrectangle{\pgfqpoint{1.432000in}{0.528000in}}{\pgfqpoint{3.696000in}{3.696000in}} %
\pgfusepath{clip}%
\pgfsetbuttcap%
\pgfsetroundjoin%
\definecolor{currentfill}{rgb}{0.263663,0.237631,0.518762}%
\pgfsetfillcolor{currentfill}%
\pgfsetlinewidth{0.000000pt}%
\definecolor{currentstroke}{rgb}{0.000000,0.000000,0.000000}%
\pgfsetstrokecolor{currentstroke}%
\pgfsetdash{}{0pt}%
\pgfpathmoveto{\pgfqpoint{4.526452in}{4.051565in}}%
\pgfpathlineto{\pgfqpoint{4.349594in}{4.051565in}}%
\pgfpathlineto{\pgfqpoint{4.354029in}{4.042694in}}%
\pgfpathlineto{\pgfqpoint{4.309677in}{4.056000in}}%
\pgfpathlineto{\pgfqpoint{4.354029in}{4.069306in}}%
\pgfpathlineto{\pgfqpoint{4.349594in}{4.060435in}}%
\pgfpathlineto{\pgfqpoint{4.526452in}{4.060435in}}%
\pgfpathlineto{\pgfqpoint{4.526452in}{4.051565in}}%
\pgfusepath{fill}%
\end{pgfscope}%
\begin{pgfscope}%
\pgfpathrectangle{\pgfqpoint{1.432000in}{0.528000in}}{\pgfqpoint{3.696000in}{3.696000in}} %
\pgfusepath{clip}%
\pgfsetbuttcap%
\pgfsetroundjoin%
\definecolor{currentfill}{rgb}{0.250425,0.274290,0.533103}%
\pgfsetfillcolor{currentfill}%
\pgfsetlinewidth{0.000000pt}%
\definecolor{currentstroke}{rgb}{0.000000,0.000000,0.000000}%
\pgfsetstrokecolor{currentstroke}%
\pgfsetdash{}{0pt}%
\pgfpathmoveto{\pgfqpoint{4.634839in}{4.051565in}}%
\pgfpathlineto{\pgfqpoint{4.457981in}{4.051565in}}%
\pgfpathlineto{\pgfqpoint{4.462417in}{4.042694in}}%
\pgfpathlineto{\pgfqpoint{4.418065in}{4.056000in}}%
\pgfpathlineto{\pgfqpoint{4.462417in}{4.069306in}}%
\pgfpathlineto{\pgfqpoint{4.457981in}{4.060435in}}%
\pgfpathlineto{\pgfqpoint{4.634839in}{4.060435in}}%
\pgfpathlineto{\pgfqpoint{4.634839in}{4.051565in}}%
\pgfusepath{fill}%
\end{pgfscope}%
\begin{pgfscope}%
\pgfpathrectangle{\pgfqpoint{1.432000in}{0.528000in}}{\pgfqpoint{3.696000in}{3.696000in}} %
\pgfusepath{clip}%
\pgfsetbuttcap%
\pgfsetroundjoin%
\definecolor{currentfill}{rgb}{0.281412,0.155834,0.469201}%
\pgfsetfillcolor{currentfill}%
\pgfsetlinewidth{0.000000pt}%
\definecolor{currentstroke}{rgb}{0.000000,0.000000,0.000000}%
\pgfsetstrokecolor{currentstroke}%
\pgfsetdash{}{0pt}%
\pgfpathmoveto{\pgfqpoint{4.634839in}{4.051565in}}%
\pgfpathlineto{\pgfqpoint{4.566368in}{4.051565in}}%
\pgfpathlineto{\pgfqpoint{4.570804in}{4.042694in}}%
\pgfpathlineto{\pgfqpoint{4.526452in}{4.056000in}}%
\pgfpathlineto{\pgfqpoint{4.570804in}{4.069306in}}%
\pgfpathlineto{\pgfqpoint{4.566368in}{4.060435in}}%
\pgfpathlineto{\pgfqpoint{4.634839in}{4.060435in}}%
\pgfpathlineto{\pgfqpoint{4.634839in}{4.051565in}}%
\pgfusepath{fill}%
\end{pgfscope}%
\begin{pgfscope}%
\pgfpathrectangle{\pgfqpoint{1.432000in}{0.528000in}}{\pgfqpoint{3.696000in}{3.696000in}} %
\pgfusepath{clip}%
\pgfsetbuttcap%
\pgfsetroundjoin%
\definecolor{currentfill}{rgb}{0.283197,0.115680,0.436115}%
\pgfsetfillcolor{currentfill}%
\pgfsetlinewidth{0.000000pt}%
\definecolor{currentstroke}{rgb}{0.000000,0.000000,0.000000}%
\pgfsetstrokecolor{currentstroke}%
\pgfsetdash{}{0pt}%
\pgfpathmoveto{\pgfqpoint{4.743226in}{4.051565in}}%
\pgfpathlineto{\pgfqpoint{4.566368in}{4.051565in}}%
\pgfpathlineto{\pgfqpoint{4.570804in}{4.042694in}}%
\pgfpathlineto{\pgfqpoint{4.526452in}{4.056000in}}%
\pgfpathlineto{\pgfqpoint{4.570804in}{4.069306in}}%
\pgfpathlineto{\pgfqpoint{4.566368in}{4.060435in}}%
\pgfpathlineto{\pgfqpoint{4.743226in}{4.060435in}}%
\pgfpathlineto{\pgfqpoint{4.743226in}{4.051565in}}%
\pgfusepath{fill}%
\end{pgfscope}%
\begin{pgfscope}%
\pgfpathrectangle{\pgfqpoint{1.432000in}{0.528000in}}{\pgfqpoint{3.696000in}{3.696000in}} %
\pgfusepath{clip}%
\pgfsetbuttcap%
\pgfsetroundjoin%
\definecolor{currentfill}{rgb}{0.231674,0.318106,0.544834}%
\pgfsetfillcolor{currentfill}%
\pgfsetlinewidth{0.000000pt}%
\definecolor{currentstroke}{rgb}{0.000000,0.000000,0.000000}%
\pgfsetstrokecolor{currentstroke}%
\pgfsetdash{}{0pt}%
\pgfpathmoveto{\pgfqpoint{4.743226in}{4.051565in}}%
\pgfpathlineto{\pgfqpoint{4.674756in}{4.051565in}}%
\pgfpathlineto{\pgfqpoint{4.679191in}{4.042694in}}%
\pgfpathlineto{\pgfqpoint{4.634839in}{4.056000in}}%
\pgfpathlineto{\pgfqpoint{4.679191in}{4.069306in}}%
\pgfpathlineto{\pgfqpoint{4.674756in}{4.060435in}}%
\pgfpathlineto{\pgfqpoint{4.743226in}{4.060435in}}%
\pgfpathlineto{\pgfqpoint{4.743226in}{4.051565in}}%
\pgfusepath{fill}%
\end{pgfscope}%
\begin{pgfscope}%
\pgfpathrectangle{\pgfqpoint{1.432000in}{0.528000in}}{\pgfqpoint{3.696000in}{3.696000in}} %
\pgfusepath{clip}%
\pgfsetbuttcap%
\pgfsetroundjoin%
\definecolor{currentfill}{rgb}{0.255645,0.260703,0.528312}%
\pgfsetfillcolor{currentfill}%
\pgfsetlinewidth{0.000000pt}%
\definecolor{currentstroke}{rgb}{0.000000,0.000000,0.000000}%
\pgfsetstrokecolor{currentstroke}%
\pgfsetdash{}{0pt}%
\pgfpathmoveto{\pgfqpoint{4.851613in}{4.051565in}}%
\pgfpathlineto{\pgfqpoint{4.783143in}{4.051565in}}%
\pgfpathlineto{\pgfqpoint{4.787578in}{4.042694in}}%
\pgfpathlineto{\pgfqpoint{4.743226in}{4.056000in}}%
\pgfpathlineto{\pgfqpoint{4.787578in}{4.069306in}}%
\pgfpathlineto{\pgfqpoint{4.783143in}{4.060435in}}%
\pgfpathlineto{\pgfqpoint{4.851613in}{4.060435in}}%
\pgfpathlineto{\pgfqpoint{4.851613in}{4.051565in}}%
\pgfusepath{fill}%
\end{pgfscope}%
\begin{pgfscope}%
\pgfpathrectangle{\pgfqpoint{1.432000in}{0.528000in}}{\pgfqpoint{3.696000in}{3.696000in}} %
\pgfusepath{clip}%
\pgfsetbuttcap%
\pgfsetroundjoin%
\definecolor{currentfill}{rgb}{0.280255,0.165693,0.476498}%
\pgfsetfillcolor{currentfill}%
\pgfsetlinewidth{0.000000pt}%
\definecolor{currentstroke}{rgb}{0.000000,0.000000,0.000000}%
\pgfsetstrokecolor{currentstroke}%
\pgfsetdash{}{0pt}%
\pgfpathmoveto{\pgfqpoint{4.856048in}{4.056000in}}%
\pgfpathlineto{\pgfqpoint{4.853831in}{4.059841in}}%
\pgfpathlineto{\pgfqpoint{4.849395in}{4.059841in}}%
\pgfpathlineto{\pgfqpoint{4.847178in}{4.056000in}}%
\pgfpathlineto{\pgfqpoint{4.849395in}{4.052159in}}%
\pgfpathlineto{\pgfqpoint{4.853831in}{4.052159in}}%
\pgfpathlineto{\pgfqpoint{4.856048in}{4.056000in}}%
\pgfpathlineto{\pgfqpoint{4.853831in}{4.059841in}}%
\pgfusepath{fill}%
\end{pgfscope}%
\begin{pgfscope}%
\pgfpathrectangle{\pgfqpoint{1.432000in}{0.528000in}}{\pgfqpoint{3.696000in}{3.696000in}} %
\pgfusepath{clip}%
\pgfsetbuttcap%
\pgfsetroundjoin%
\definecolor{currentfill}{rgb}{0.244972,0.287675,0.537260}%
\pgfsetfillcolor{currentfill}%
\pgfsetlinewidth{0.000000pt}%
\definecolor{currentstroke}{rgb}{0.000000,0.000000,0.000000}%
\pgfsetstrokecolor{currentstroke}%
\pgfsetdash{}{0pt}%
\pgfpathmoveto{\pgfqpoint{4.964435in}{4.056000in}}%
\pgfpathlineto{\pgfqpoint{4.962218in}{4.059841in}}%
\pgfpathlineto{\pgfqpoint{4.957782in}{4.059841in}}%
\pgfpathlineto{\pgfqpoint{4.955565in}{4.056000in}}%
\pgfpathlineto{\pgfqpoint{4.957782in}{4.052159in}}%
\pgfpathlineto{\pgfqpoint{4.962218in}{4.052159in}}%
\pgfpathlineto{\pgfqpoint{4.964435in}{4.056000in}}%
\pgfpathlineto{\pgfqpoint{4.962218in}{4.059841in}}%
\pgfusepath{fill}%
\end{pgfscope}%
\begin{pgfscope}%
\pgfsetbuttcap%
\pgfsetroundjoin%
\definecolor{currentfill}{rgb}{0.000000,0.000000,0.000000}%
\pgfsetfillcolor{currentfill}%
\pgfsetlinewidth{0.803000pt}%
\definecolor{currentstroke}{rgb}{0.000000,0.000000,0.000000}%
\pgfsetstrokecolor{currentstroke}%
\pgfsetdash{}{0pt}%
\pgfsys@defobject{currentmarker}{\pgfqpoint{0.000000in}{-0.048611in}}{\pgfqpoint{0.000000in}{0.000000in}}{%
\pgfpathmoveto{\pgfqpoint{0.000000in}{0.000000in}}%
\pgfpathlineto{\pgfqpoint{0.000000in}{-0.048611in}}%
\pgfusepath{stroke,fill}%
}%
\begin{pgfscope}%
\pgfsys@transformshift{1.545806in}{0.528000in}%
\pgfsys@useobject{currentmarker}{}%
\end{pgfscope}%
\end{pgfscope}%
\begin{pgfscope}%
\pgftext[x=1.545806in,y=0.430778in,,top]{\rmfamily\fontsize{10.000000}{12.000000}\selectfont \(\displaystyle 0.0\)}%
\end{pgfscope}%
\begin{pgfscope}%
\pgfsetbuttcap%
\pgfsetroundjoin%
\definecolor{currentfill}{rgb}{0.000000,0.000000,0.000000}%
\pgfsetfillcolor{currentfill}%
\pgfsetlinewidth{0.803000pt}%
\definecolor{currentstroke}{rgb}{0.000000,0.000000,0.000000}%
\pgfsetstrokecolor{currentstroke}%
\pgfsetdash{}{0pt}%
\pgfsys@defobject{currentmarker}{\pgfqpoint{0.000000in}{-0.048611in}}{\pgfqpoint{0.000000in}{0.000000in}}{%
\pgfpathmoveto{\pgfqpoint{0.000000in}{0.000000in}}%
\pgfpathlineto{\pgfqpoint{0.000000in}{-0.048611in}}%
\pgfusepath{stroke,fill}%
}%
\begin{pgfscope}%
\pgfsys@transformshift{2.239484in}{0.528000in}%
\pgfsys@useobject{currentmarker}{}%
\end{pgfscope}%
\end{pgfscope}%
\begin{pgfscope}%
\pgftext[x=2.239484in,y=0.430778in,,top]{\rmfamily\fontsize{10.000000}{12.000000}\selectfont \(\displaystyle 0.2\)}%
\end{pgfscope}%
\begin{pgfscope}%
\pgfsetbuttcap%
\pgfsetroundjoin%
\definecolor{currentfill}{rgb}{0.000000,0.000000,0.000000}%
\pgfsetfillcolor{currentfill}%
\pgfsetlinewidth{0.803000pt}%
\definecolor{currentstroke}{rgb}{0.000000,0.000000,0.000000}%
\pgfsetstrokecolor{currentstroke}%
\pgfsetdash{}{0pt}%
\pgfsys@defobject{currentmarker}{\pgfqpoint{0.000000in}{-0.048611in}}{\pgfqpoint{0.000000in}{0.000000in}}{%
\pgfpathmoveto{\pgfqpoint{0.000000in}{0.000000in}}%
\pgfpathlineto{\pgfqpoint{0.000000in}{-0.048611in}}%
\pgfusepath{stroke,fill}%
}%
\begin{pgfscope}%
\pgfsys@transformshift{2.933161in}{0.528000in}%
\pgfsys@useobject{currentmarker}{}%
\end{pgfscope}%
\end{pgfscope}%
\begin{pgfscope}%
\pgftext[x=2.933161in,y=0.430778in,,top]{\rmfamily\fontsize{10.000000}{12.000000}\selectfont \(\displaystyle 0.4\)}%
\end{pgfscope}%
\begin{pgfscope}%
\pgfsetbuttcap%
\pgfsetroundjoin%
\definecolor{currentfill}{rgb}{0.000000,0.000000,0.000000}%
\pgfsetfillcolor{currentfill}%
\pgfsetlinewidth{0.803000pt}%
\definecolor{currentstroke}{rgb}{0.000000,0.000000,0.000000}%
\pgfsetstrokecolor{currentstroke}%
\pgfsetdash{}{0pt}%
\pgfsys@defobject{currentmarker}{\pgfqpoint{0.000000in}{-0.048611in}}{\pgfqpoint{0.000000in}{0.000000in}}{%
\pgfpathmoveto{\pgfqpoint{0.000000in}{0.000000in}}%
\pgfpathlineto{\pgfqpoint{0.000000in}{-0.048611in}}%
\pgfusepath{stroke,fill}%
}%
\begin{pgfscope}%
\pgfsys@transformshift{3.626839in}{0.528000in}%
\pgfsys@useobject{currentmarker}{}%
\end{pgfscope}%
\end{pgfscope}%
\begin{pgfscope}%
\pgftext[x=3.626839in,y=0.430778in,,top]{\rmfamily\fontsize{10.000000}{12.000000}\selectfont \(\displaystyle 0.6\)}%
\end{pgfscope}%
\begin{pgfscope}%
\pgfsetbuttcap%
\pgfsetroundjoin%
\definecolor{currentfill}{rgb}{0.000000,0.000000,0.000000}%
\pgfsetfillcolor{currentfill}%
\pgfsetlinewidth{0.803000pt}%
\definecolor{currentstroke}{rgb}{0.000000,0.000000,0.000000}%
\pgfsetstrokecolor{currentstroke}%
\pgfsetdash{}{0pt}%
\pgfsys@defobject{currentmarker}{\pgfqpoint{0.000000in}{-0.048611in}}{\pgfqpoint{0.000000in}{0.000000in}}{%
\pgfpathmoveto{\pgfqpoint{0.000000in}{0.000000in}}%
\pgfpathlineto{\pgfqpoint{0.000000in}{-0.048611in}}%
\pgfusepath{stroke,fill}%
}%
\begin{pgfscope}%
\pgfsys@transformshift{4.320516in}{0.528000in}%
\pgfsys@useobject{currentmarker}{}%
\end{pgfscope}%
\end{pgfscope}%
\begin{pgfscope}%
\pgftext[x=4.320516in,y=0.430778in,,top]{\rmfamily\fontsize{10.000000}{12.000000}\selectfont \(\displaystyle 0.8\)}%
\end{pgfscope}%
\begin{pgfscope}%
\pgfsetbuttcap%
\pgfsetroundjoin%
\definecolor{currentfill}{rgb}{0.000000,0.000000,0.000000}%
\pgfsetfillcolor{currentfill}%
\pgfsetlinewidth{0.803000pt}%
\definecolor{currentstroke}{rgb}{0.000000,0.000000,0.000000}%
\pgfsetstrokecolor{currentstroke}%
\pgfsetdash{}{0pt}%
\pgfsys@defobject{currentmarker}{\pgfqpoint{0.000000in}{-0.048611in}}{\pgfqpoint{0.000000in}{0.000000in}}{%
\pgfpathmoveto{\pgfqpoint{0.000000in}{0.000000in}}%
\pgfpathlineto{\pgfqpoint{0.000000in}{-0.048611in}}%
\pgfusepath{stroke,fill}%
}%
\begin{pgfscope}%
\pgfsys@transformshift{5.014194in}{0.528000in}%
\pgfsys@useobject{currentmarker}{}%
\end{pgfscope}%
\end{pgfscope}%
\begin{pgfscope}%
\pgftext[x=5.014194in,y=0.430778in,,top]{\rmfamily\fontsize{10.000000}{12.000000}\selectfont \(\displaystyle 1.0\)}%
\end{pgfscope}%
\begin{pgfscope}%
\pgfsetbuttcap%
\pgfsetroundjoin%
\definecolor{currentfill}{rgb}{0.000000,0.000000,0.000000}%
\pgfsetfillcolor{currentfill}%
\pgfsetlinewidth{0.803000pt}%
\definecolor{currentstroke}{rgb}{0.000000,0.000000,0.000000}%
\pgfsetstrokecolor{currentstroke}%
\pgfsetdash{}{0pt}%
\pgfsys@defobject{currentmarker}{\pgfqpoint{-0.048611in}{0.000000in}}{\pgfqpoint{0.000000in}{0.000000in}}{%
\pgfpathmoveto{\pgfqpoint{0.000000in}{0.000000in}}%
\pgfpathlineto{\pgfqpoint{-0.048611in}{0.000000in}}%
\pgfusepath{stroke,fill}%
}%
\begin{pgfscope}%
\pgfsys@transformshift{1.432000in}{0.641806in}%
\pgfsys@useobject{currentmarker}{}%
\end{pgfscope}%
\end{pgfscope}%
\begin{pgfscope}%
\pgftext[x=1.157308in,y=0.593612in,left,base]{\rmfamily\fontsize{10.000000}{12.000000}\selectfont \(\displaystyle 0.0\)}%
\end{pgfscope}%
\begin{pgfscope}%
\pgfsetbuttcap%
\pgfsetroundjoin%
\definecolor{currentfill}{rgb}{0.000000,0.000000,0.000000}%
\pgfsetfillcolor{currentfill}%
\pgfsetlinewidth{0.803000pt}%
\definecolor{currentstroke}{rgb}{0.000000,0.000000,0.000000}%
\pgfsetstrokecolor{currentstroke}%
\pgfsetdash{}{0pt}%
\pgfsys@defobject{currentmarker}{\pgfqpoint{-0.048611in}{0.000000in}}{\pgfqpoint{0.000000in}{0.000000in}}{%
\pgfpathmoveto{\pgfqpoint{0.000000in}{0.000000in}}%
\pgfpathlineto{\pgfqpoint{-0.048611in}{0.000000in}}%
\pgfusepath{stroke,fill}%
}%
\begin{pgfscope}%
\pgfsys@transformshift{1.432000in}{1.335484in}%
\pgfsys@useobject{currentmarker}{}%
\end{pgfscope}%
\end{pgfscope}%
\begin{pgfscope}%
\pgftext[x=1.157308in,y=1.287289in,left,base]{\rmfamily\fontsize{10.000000}{12.000000}\selectfont \(\displaystyle 0.2\)}%
\end{pgfscope}%
\begin{pgfscope}%
\pgfsetbuttcap%
\pgfsetroundjoin%
\definecolor{currentfill}{rgb}{0.000000,0.000000,0.000000}%
\pgfsetfillcolor{currentfill}%
\pgfsetlinewidth{0.803000pt}%
\definecolor{currentstroke}{rgb}{0.000000,0.000000,0.000000}%
\pgfsetstrokecolor{currentstroke}%
\pgfsetdash{}{0pt}%
\pgfsys@defobject{currentmarker}{\pgfqpoint{-0.048611in}{0.000000in}}{\pgfqpoint{0.000000in}{0.000000in}}{%
\pgfpathmoveto{\pgfqpoint{0.000000in}{0.000000in}}%
\pgfpathlineto{\pgfqpoint{-0.048611in}{0.000000in}}%
\pgfusepath{stroke,fill}%
}%
\begin{pgfscope}%
\pgfsys@transformshift{1.432000in}{2.029161in}%
\pgfsys@useobject{currentmarker}{}%
\end{pgfscope}%
\end{pgfscope}%
\begin{pgfscope}%
\pgftext[x=1.157308in,y=1.980967in,left,base]{\rmfamily\fontsize{10.000000}{12.000000}\selectfont \(\displaystyle 0.4\)}%
\end{pgfscope}%
\begin{pgfscope}%
\pgfsetbuttcap%
\pgfsetroundjoin%
\definecolor{currentfill}{rgb}{0.000000,0.000000,0.000000}%
\pgfsetfillcolor{currentfill}%
\pgfsetlinewidth{0.803000pt}%
\definecolor{currentstroke}{rgb}{0.000000,0.000000,0.000000}%
\pgfsetstrokecolor{currentstroke}%
\pgfsetdash{}{0pt}%
\pgfsys@defobject{currentmarker}{\pgfqpoint{-0.048611in}{0.000000in}}{\pgfqpoint{0.000000in}{0.000000in}}{%
\pgfpathmoveto{\pgfqpoint{0.000000in}{0.000000in}}%
\pgfpathlineto{\pgfqpoint{-0.048611in}{0.000000in}}%
\pgfusepath{stroke,fill}%
}%
\begin{pgfscope}%
\pgfsys@transformshift{1.432000in}{2.722839in}%
\pgfsys@useobject{currentmarker}{}%
\end{pgfscope}%
\end{pgfscope}%
\begin{pgfscope}%
\pgftext[x=1.157308in,y=2.674644in,left,base]{\rmfamily\fontsize{10.000000}{12.000000}\selectfont \(\displaystyle 0.6\)}%
\end{pgfscope}%
\begin{pgfscope}%
\pgfsetbuttcap%
\pgfsetroundjoin%
\definecolor{currentfill}{rgb}{0.000000,0.000000,0.000000}%
\pgfsetfillcolor{currentfill}%
\pgfsetlinewidth{0.803000pt}%
\definecolor{currentstroke}{rgb}{0.000000,0.000000,0.000000}%
\pgfsetstrokecolor{currentstroke}%
\pgfsetdash{}{0pt}%
\pgfsys@defobject{currentmarker}{\pgfqpoint{-0.048611in}{0.000000in}}{\pgfqpoint{0.000000in}{0.000000in}}{%
\pgfpathmoveto{\pgfqpoint{0.000000in}{0.000000in}}%
\pgfpathlineto{\pgfqpoint{-0.048611in}{0.000000in}}%
\pgfusepath{stroke,fill}%
}%
\begin{pgfscope}%
\pgfsys@transformshift{1.432000in}{3.416516in}%
\pgfsys@useobject{currentmarker}{}%
\end{pgfscope}%
\end{pgfscope}%
\begin{pgfscope}%
\pgftext[x=1.157308in,y=3.368322in,left,base]{\rmfamily\fontsize{10.000000}{12.000000}\selectfont \(\displaystyle 0.8\)}%
\end{pgfscope}%
\begin{pgfscope}%
\pgfsetbuttcap%
\pgfsetroundjoin%
\definecolor{currentfill}{rgb}{0.000000,0.000000,0.000000}%
\pgfsetfillcolor{currentfill}%
\pgfsetlinewidth{0.803000pt}%
\definecolor{currentstroke}{rgb}{0.000000,0.000000,0.000000}%
\pgfsetstrokecolor{currentstroke}%
\pgfsetdash{}{0pt}%
\pgfsys@defobject{currentmarker}{\pgfqpoint{-0.048611in}{0.000000in}}{\pgfqpoint{0.000000in}{0.000000in}}{%
\pgfpathmoveto{\pgfqpoint{0.000000in}{0.000000in}}%
\pgfpathlineto{\pgfqpoint{-0.048611in}{0.000000in}}%
\pgfusepath{stroke,fill}%
}%
\begin{pgfscope}%
\pgfsys@transformshift{1.432000in}{4.110194in}%
\pgfsys@useobject{currentmarker}{}%
\end{pgfscope}%
\end{pgfscope}%
\begin{pgfscope}%
\pgftext[x=1.157308in,y=4.061999in,left,base]{\rmfamily\fontsize{10.000000}{12.000000}\selectfont \(\displaystyle 1.0\)}%
\end{pgfscope}%
\begin{pgfscope}%
\pgfsetrectcap%
\pgfsetmiterjoin%
\pgfsetlinewidth{0.803000pt}%
\definecolor{currentstroke}{rgb}{0.000000,0.000000,0.000000}%
\pgfsetstrokecolor{currentstroke}%
\pgfsetdash{}{0pt}%
\pgfpathmoveto{\pgfqpoint{1.432000in}{0.528000in}}%
\pgfpathlineto{\pgfqpoint{1.432000in}{4.224000in}}%
\pgfusepath{stroke}%
\end{pgfscope}%
\begin{pgfscope}%
\pgfsetrectcap%
\pgfsetmiterjoin%
\pgfsetlinewidth{0.803000pt}%
\definecolor{currentstroke}{rgb}{0.000000,0.000000,0.000000}%
\pgfsetstrokecolor{currentstroke}%
\pgfsetdash{}{0pt}%
\pgfpathmoveto{\pgfqpoint{5.128000in}{0.528000in}}%
\pgfpathlineto{\pgfqpoint{5.128000in}{4.224000in}}%
\pgfusepath{stroke}%
\end{pgfscope}%
\begin{pgfscope}%
\pgfsetrectcap%
\pgfsetmiterjoin%
\pgfsetlinewidth{0.803000pt}%
\definecolor{currentstroke}{rgb}{0.000000,0.000000,0.000000}%
\pgfsetstrokecolor{currentstroke}%
\pgfsetdash{}{0pt}%
\pgfpathmoveto{\pgfqpoint{1.432000in}{0.528000in}}%
\pgfpathlineto{\pgfqpoint{5.128000in}{0.528000in}}%
\pgfusepath{stroke}%
\end{pgfscope}%
\begin{pgfscope}%
\pgfsetrectcap%
\pgfsetmiterjoin%
\pgfsetlinewidth{0.803000pt}%
\definecolor{currentstroke}{rgb}{0.000000,0.000000,0.000000}%
\pgfsetstrokecolor{currentstroke}%
\pgfsetdash{}{0pt}%
\pgfpathmoveto{\pgfqpoint{1.432000in}{4.224000in}}%
\pgfpathlineto{\pgfqpoint{5.128000in}{4.224000in}}%
\pgfusepath{stroke}%
\end{pgfscope}%
\end{pgfpicture}%
\makeatother%
\endgroup%
}
\caption{An transportation example on DOTmark} \label{Fig:DOTMark}
\end{figure}

We evaluate algorithms from three aspects:
\begin{partlist}
\item Distance to the optimal primal object, where the solution of interior point methods from MOSEK is refered as a standard;
\item Running time;
\item Satisfication of constraints, where we consider the $1$-norm of $ \mu - \sume{j}{1}{n}{s_{ \cdot j }} $ and $ \nu - \sume{i}{1}{m}{s_{ i \cdot }}$ as the error to $\mu$ and $\nu$.
\end{partlist}

We first test these algorithms on randomly generated datasets, Caffarelli datasets and ellipses datasets of source size 1000 and target size 1000. Results are shown in Table \ref{Tbl:PC1000}.

\begin{table}[htbp]
\centering \footnotesize
\begin{tabular}{|c|c|c|c|c|c|c|c|c|c|c|}
\hline
& & M prim. & M dual & M int & G prim. & G dual & G int & \hyperlink{EAlg:12}{1+2} & \ref{Alg:TS} & \ref{Alg:MS} \\ \hline
& dist. & 5.59e-8 & 4.28e-9 & 0.00e0 & 4.77e-11 & 4.77e-11 & 4.77e-11 & 3.46e-2 & 4.77e-11 & 2.79e-03 \\ \cline{2-11}
& time & 3.66 & 16.21 & 10.21 & 15.1 & 15.84 & 17.6 & 166.68 & 294.19 & 2.36 \\ \cline{2-11}
& err. $\mu$ & 1.01e-7 & 4.27e-17 & 5.73e-11 & 3.76e-17 & 6.82e-17 & 6.83e-17 & 3.33e-6 & 1.22e-15 & 1.91e-7 \\ \cline{2-11}
\multirow{-4}*{rand.} & err. $\nu$ & 1.01e-7 & 7.22e-17 & 6.5e-11 & 7.00e-17 & 3.96e-17 & 4.00e-17 & 3.33e-6 & 1.27e-15 & 1.91e-7 \\ \hline
& dist. & 8.64e-8 & 5.95e-15 & 0.00e0 & 1.33e-10 & 3.13e-10 & 5.95e-15 & 1.06e-4 & 6.17e-15 & 2.54e-4 \\ \cline{2-11}
& time & 2.04 & 17.37 & 5.46 & 8.73 & 12.52 & 10.43 & 134.72 & 128.97 & 1.37 \\ \cline{2-11}
& err. $\mu$ & 8.55e-8 & 4.86e-17 & 8.96e-13 & 2.6e-17 & 4.79e-17 & 2.65e-12 & 1.12e-5 & 4.77e-18 & 3.68e-18 \\ \cline{2-11}
\multirow{-4}*{Caff.} & err. $\nu$ & 8.55e-8 & 5.20e-17 & 8.30e-13 & 7.48e-17 & 5.31e-17 & 7.48e-17 & 9.02e-6 & 9.62e-17 & 5.42e-17 \\ \hline
& dist. & 7.94e-8 & 6.11e-12 & 0.00e0 & 6.11e-12 & 6.11e-12 & 6.11e-12 & 1.18e-4 & 6.11e-12 & 2.50e-5 \\ \cline{2-11}
& time & 3.15 & 58.74 & 9.53 & 13.93 & 37.18 & 17.07 & 221.35 & 275.89 & 1.2 \\ \cline{2-11}
& err. $\mu$ & 8.47e-8 & 0.00e0 & 2.60e-12 & 0.00e0 & 0.00e0 & 0.00e0 & 1.48e-7 & 0.00e0 & 1.05e-16 \\ \cline{2-11}
\multirow{-4}*{ellip.}& err. $\nu$ & 8.47e-8 & 0.00e0 & 4.50e-12 & 0.00e0 & 0.00e0 & 0.00e0 & 1.49e-7 & 0.00e0 & 1.05e-16 \\ \hline
\end{tabular}
\caption{Numerical results for point clouds of size $ 1000 \times 1000 $} \label{Tbl:PC1000}
\end{table}

We first test these algorithms on randomly generated datasets, Caffarelli datasets and ellipses datasets of size 250, 500, 1000, 2000 respectively. Results are shown in Table \ref{Tbl:PCSize}.

\begin{table}[htbp]
\centering \footnotesize
\begin{tabular}{|c|c|c|c|c|c|c|c|c|c|c|}
\hline
& & M prim. & M dual & M int & G prim. & G dual & G int & \hyperlink{EAlg:12}{1+2} & \ref{Alg:TS} & \ref{Alg:MS} \\ \hline
& dist. & 6.17e-8 & 1.98e-12 & 0.00e0 & 1.91e-12 & 1.91e-12 & 1.81e-12 & 1.48e-2 & 1.91e-12 & 1.28e-3 \\ \cline{2-11}
\multirow{-2}*{\makecell{rand. \\ 250}}& time & 0.20 & 0.25 & 0.36 & 0.69 & 0.78 & 0.80 & 2.66 & 5.18 & 1.14 \\ \hline
& dist. & 1.21e-7 & 9.74e-10 & 0.00e0 & 9.74e-10 & 9.74e-10 & 9.74e-10 & 4.76e-2 & 9.74e-10 & 5.54e-3 \\ \cline{2-11}
\multirow{-2}*{\makecell{rand. \\ 500}}& time & 0.81 & 1.66 & 1.89 & 3.2 & 2.92 & 3.81 & 20.11 & 33.38 & 1.06 \\ \hline
& dist. & 5.59e-8 & 4.28e-9 & 0.00e0 & 4.77e-11 & 4.77e-11 & 4.77e-11 & 3.46e-2 & 4.77e-11 & 2.79e-3 \\ \cline{2-11}
\multirow{-2}*{\makecell{rand. \\ 1000}}& time & 3.66 & 16.21 & 10.21 & 15.1 & 15.84 & 17.6 & 166.68 & 294.19 & 2.35 \\ \hline
& dist. & 3.38e-7 & 3.97e-8 & 0.00e0 & 6.85e-7 & 3.91e-7 & 4.66e-10 & 6.39e-2 & 3.13e0\textsuperscript{*} & 5.37e-3 \\ \cline{2-11}
\multirow{-2}*{\makecell{rand. \\ 2000}}& time & 18.6 & 90.37 & 48.74 & 280.93 & 58.64 & 75.59 & 1172.89 & 1035.05 & 5.15 \\ \hline
& dist. & 5.34e-8 & 3.01e-11 & 0.00e0 & 3.01-11 & 3.01e-11 & 3.01e-11 & 2.01e-4 & 3.01e-11 & 1.35e-3 \\ \cline{2-11}
\multirow{-2}*{\makecell{Caff. \\ 250}}& time & 0.11 & 0.2 & 0.18 & 0.4 & 0.59 & 0.48 & 1.84 & 2.92 & 0.3 \\ \hline
& dist. & 8.49e-8 & 5.45e-13 & 0.00e0 & 5.45e-13 & 5.44e-13 & 5.44e-13 & 1.30e-4 & 5.45e-13 & 5.96e-4 \\ \cline{2-11}
\multirow{-2}*{\makecell{Caff. \\ 500}}& time & 0.45 & 1.71 & 1.16 & 1.86 & 2.52 & 2.11 & 14.31 & 15.16 & 0.64 \\ \hline
& dist. & 8.64e-8 & 5.95e-15 & 0.00e0 & 1.33e-10 & 3.12e-10 & 5.95e-15 & 1.06e-4 & 6.16e-15 & 2.54e-4 \\ \cline{2-11}
\multirow{-2}*{\makecell{Caff. \\ 1000}}& time & 2.04 & 17.38 & 5.46 & 8.73 & 12.51 & 10.43 & 134.72 & 128.97 & 1.37 \\ \hline
& dist. & 1.36e-7 & 1.54-10 & 0.00e0 & 5.30e-13 & 2.90e-10 & 2.93e-12 & 9.23e-5 & 2.59e-3\textsuperscript{*} & 3.68e-4 \\ \cline{2-11}
\multirow{-2}*{\makecell{Caff. \\ 2000}}& time & 10.42 & 136.17 & 25.99 & 38.58 & 64.5 & 47.21 & 910.79 & 717.34 & 2.43 \\ \hline
& dist. & 6.02e-8 & 5.11e-11 & 0.00e0 & 5.11e-11 & 5.11e-11 & 5.11e-11 & 1.95e-4 & 5.11e-11 & 5.28e-5 \\ \cline{2-11}
\multirow{-2}*{\makecell{ellip. \\ 250}}& time & 0.18 & 0.48 & 0.31 & 0.67 & 1.32 & 0.77 & 2.89 & 5.08 & 2.29 \\ \hline
& dist. & 7.79e-8 & 1.65e-10 & 0.00e0 & 1.65e-10 & 1.65e-10 & 1.64e-10 & 2.08e-4 & 1.65e-10 & 8.89e-6 \\ \cline{2-11}
\multirow{-2}*{\makecell{ellip. \\ 500}}& time & 0.73 & 5.74 & 1.61 & 3.08 & 6.52 & 3.64 & 31.2 & 35.11 & 0.64 \\ \hline
& dist. & 7.95e-8 & 6.11e-12 & 0.00e0 & 6.11e-12 & 6.11e-12 & 6.11e-12 & 1.17e-4 & 6.11e-12 & 2.50e-5 \\ \cline{2-11}
\multirow{-2}*{\makecell{ellip. \\ 1000}}& time & 3.15 & 58.74 & 9.53 & 13.94 & 37.19 & 17.01 & 221.35 & 275.87 & 1.2 \\ \hline
& dist. & 1.10e-7 & 4.07e-14 & 0.00e0 & 4.08e-14 & 4.08e-14 & 4.08e-14 & 7.91e-5 & 3.19e-3\textsuperscript{*} & 1.58e-5 \\ \cline{2-11}
\multirow{-2}*{\makecell{ellip. \\ 2000}}& time & 15.61 & 464.57 & 42.1 & 285.01 & 1049.3 & 74.86 & 1526.3 & 1045.14 & 1.97 \\ \hline
\end{tabular}
*: Transportation simplex method fails to converge in 20000 iterations
\caption{Numerical results for point clouds of different sizes} \label{Tbl:PCSize}
\end{table}

We then conduct tests on DOTmark. Results are shown in Table \ref{Tbl:DOT}.

\begin{table}[htbp]
\centering \footnotesize
\begin{tabular}{|c|c|c|c|c|c|c|c|c|c|c|}
\hline
class & & M prim. & M dual & M int & G prim. & G dual & G int & \hyperlink{EAlg:12}{1+2} & \ref{Alg:TS} & \ref{Alg:MS} \\ \hline
& dist. & 5.22e-8 & 1.02e-9 & 0.00e0 & 1.02e-9 & 1.02e-9 & 1.02e-9 & 6.38e-2 & 1.02e-9 & 1.54e-3 \\ \cline{2-11}
\multirow{-2}*{1} & time & 5.12 & 12.84 & 10.8 & 15.43 & 13.06 & 18.98 & 211.32 & 396.68 & 0.75 \\ \hline
& dist. & 1.93e-8 & 1.18e-10 & 0.00e0 & 1.18e-10 & 1.19e-10 & 1.19e-10 & 5.64e-2 & 1.18e-10 & 1.54e-4 \\ \cline{2-11}
\multirow{-2}*{2} & time & 5.04 & 19.83 & 11.03 & 15.43 & 13.27 & 19.23 & 212.58 & 440.19 & 0.82 \\ \hline
& dist. & 1.91e-8 & 9.92e-13 & 0.00e0 & 9.93e-13 & 9.93e-13 & 9.93e-13 & 6.49e-3 & 9.93e-13 & 1.80e-3 \\ \cline{2-11}
\multirow{-2}*{3} & time & 4.92 & 27.84 & 10.84 & 15.26 & 18.39 & 19.4 & 230.33 & 458.15 & 0.86 \\ \hline
& dist. & 4.36e-8 & 4.71e-13 & 0.00e0 & 4.71e-13 & 4.71e-13 & 4.71e-13 & 1.82e-3 & 4.71e-13 & 2.69e-4 \\ \cline{2-11}
\multirow{-2}*{4} & time & 4.92 & 29.11 & 12.49 & 15.2 & 24.87 & 19.79 & 221.99 & 462.48 & 0.92 \\ \hline
& dist. & 7.81e-9 & 5.56e-12 & 0.00e0 & 5.56e-12 & 5.57e-12 & 5.56e-12 & 1.75e-3 & 5.56e-12 & 1.09e-3 \\ \cline{2-11}
\multirow{-2}*{5} & time & 5.59 & 35.41 & 10.96 & 15.14 & 24.4 & 19.54 & 214.4 & 562.46 & 0.81 \\ \hline
& dist. & 3.79e-8 & 8.85e-10 & 0.00e0 & 8.86e-10 & 8.86e-10 & 8.86e-10 & 4.20e-3 & 8.86e-10 & 1.36e-2 \\ \cline{2-11}
\multirow{-2}*{6} & time & 5.01 & 25.06 & 10.89 & 15.32 & 19.46 & 19.48 & 217.84 & 536.3 & 0.87 \\ \hline
& dist. & 3.69e-8 & 3.03e-11 & 0.00e0 & 3.03e-11 & 3.03e-11 & 3.03e-11 & 1.86e-3 & 3.03e-11 & 1.23e-4 \\ \cline{2-11}
\multirow{-2}*{7} & time & 4.9 & 41.34 & 10.72 & 14.98 & 33.1 & 19.18 & 221.53 & 421.21 & 0.90 \\ \hline
& dist. & 4.07e-8 & 5.31e-10 & 0.00e0 & 5.31e-10 & 5.31e-10 & 5.31e-10 & 4.25e-2 & 5.31e-10 & 1.64e-1 \\ \cline{2-11}
\multirow{-2}*{8} & time & 2.69 & 3.59 & 4.02 & 12.95 & 12.90 & 13.50 & 169.20 & 405.04 & 0.36 \\ \hline
& dist. & 3.84e-9 & 2.81e-12 & 0.00e0 & 2.81e-12 & 2.81e-12 & 2.81e-12 & 7.06e-3 & 2.81e-12 & 3.64e-2 \\ \cline{2-11}
\multirow{-2}*{9} & time & 5.23 & 24.96 & 10.65 & 15.32 & 16.87 & 19.91 & 228.10 & 494.80 & 0.87 \\ \hline
& dist. & 3.71e-8 & 4.00e-9 & 0.00e0 & 4.00e-9 & 4.00e-9 & 4.00e-9 & 1.93e-3 & 4.00e-9 & 2.13e-3 \\ \cline{2-11}
\multirow{-2}*{10} & time & 3.33 & 13.66 & 5.88 & 14.65 & 17.42 & 15.37 & 235.06 & 457.52 & 0.63 \\ \hline
\end{tabular}
\caption{Numerical results on DOTmark} \label{Tbl:DOT}
\end{table}

From the numerical results, we found that all the algorithms converge, but there are still some differences. All the algorithms has an error of $\mu$ and $\nu$ below $10^{-4}$, which means the constraint are basically satisfied.

For solvers, it seems that Gurobi may always reach the truly optimal solution, while the MOSEK sometimes fails. For MOSEK, the primal simplex method are faster than the interior method and the dual simplex method. For Gurobi, simplex methods are generally faster than interior point methods. This is because an simplex methods are specified for linear programs. The bad proformance of the dual simplex may be caused by the huge amount of constraints. Gurobi being generally bettern than MOSEK because of defect of the Python interface of Gurobi.

Algorithm \hyperlink{EAlg:12}{1+2} is rather slow, with a moderate accuracy. This is because the size of problems tested are rather large, and ADMM, as a general algorithm, is not completely capable of this problem.

Algorithm \ref{Alg:TS} has a high accuracy and precision, but it is still very slow. Actually, the dynamic feature of Python accounts for this failure, which makes discrete algorithms, especially graph algorithms very slow. Actually, programs in \parencite{Schrieber2017} are implemented by Java, which has better proformance in discrete algorithms.

Algorithm \ref{Alg:MS} is very fast, but its precision is not very high, this is because the multiscale algorithm itself is an approximate algorithm.

All the codes are implemented in Python. These results are carried out on a computer with Intel Core i7-6500U (4 threads) and $\SI{7890}{\mebi\byte}$ RAM. The operating system is Arch Linux 4.14.10 64-bit and the Python environment is provided by Anaconda 4.3.30. Further environment setting can be find in \verb"environmnet.yml".

\section{Conclusion and outlook}

As for optimal transport problem, our work in this semester ends up here. For these methods we have implemented, we have the following outlooks. 

For first order methods, we hope to implement the interior point algorithm, together with Douglas-Rachford splitting in Algorithm \ref{Alg:GradPrimal}. In addition, we also want to test out second-order algorithms like Newton methods and semi-smooth Newton methods. 

For the transportation simplex method, we hope to refine the quality of basis solution to let this algorithm become faster. Besides, the way to find dual variable is very costly, if we could simplify this process, it would be much faster. Additionally, if the time have permitted, we might have tried implementing this algorithm in C++ in order to avoid intensive discrete operations in Python. 

For multiscale strategies, we hope to find a theoretical proof for the convergence. Besides, we wish to refine our propagation process, which could improve the robustness by keeping the subproblem feasible even with extreme condition such as when capacity constraints is smaller.

For numerical experiments, we hope to test all algorithms on the whole DOTmark to have a more precise investigation on the precision and efficiency of all these algorithms.

As we discussed before, transportation problem is basis for many high level applications. However, solving the problem is still tough even with these brilliant methods. We also hope to find more time and space efficient methods. This will be left for future research.

\section{Acknowledgement}

We would like to thank Samuel Gerber in the end, for his clarification of details about the article \parencite{Gerber2017}.

\printbibliography

\newpage
\cleanthanks

\DeclareRobustCommand{\authorthing}{%
\begin{tabular}{ccc}%
侯霁开 & 贾泽宇 & 李知含\\%
1600010681 & 1600010603 & 1600010653%
\end{tabular}%
}
\title{Supplementary Materials}
\author{\authorthing}

\maketitle

\section{Other first-order methods}

Besides three algorithms mentioned in Section \ref{Sec:FOM}, we have also tried some other first-order methods.

\subsection{ADMM for the dual problem}

We have implemented an alternative direction method of multipliers (ADMM) to the dual problem according to a reformulation of \eqref{Eq:Dual}
\begin{equation}
\begin{array}{ll}
\mtx{minimize} & -\sume{i}{1}{m}{ \mu_i \lambda_i } - \sume{j}{1}{n}{ \nu_j \eta_j } + \iota_+ \rbr{e}, \\
\mtx{subject to} & c_{ i j } - \lambda_i - \eta_j - e_{ i j } = 0. \crbr{ i = 1, 2, \cdots, m; j = 1, 2, \cdots, n }.
\end{array}
\end{equation}
The augmented Lagragian is
\begin{equation}
\begin{aligned}
L_{\rho} \rbr{ \lambda, \mu, e, d } &= -\sume{i}{1}{m}{ \mu_i \lambda_i } - \sume{j}{1}{n}{ \nu_j \eta_j } + \iota_+ \rbr{e} \\
&+ \sume{i}{1}{m}{\sume{j}{1}{n}{ d_{ i j } \rbr{ c_{ i j } - \lambda_i - \eta_j - e_{ i j } } }} \\
&+ \frac{\rho}{2} \sume{i}{1}{m}{\sume{j}{1}{n}{\rbr{ c_{ i j } - \lambda_i - \eta_j - e_{ i j } }^2}}.
\end{aligned}
\end{equation}
The minimization of $e$ can be done directly by solving for zero gradient and projection, while the minimization of $\lambda$ and $\mu$ can be done by solving for zero gradient. (Actually this system has one degree of freedom, which can be fixed by letting $ \sume{j}{1}{n}{\eta_j} = 0 $ and this does not influence the result)  The algorithm is listed as Algorithm \ref{Alg:ADMMDual}. Solution $s$ can be recovered by $ s = -d $ from KKT conditions.

\begin{algorithm}
\caption{ADMM for the dual problem}
\label{Alg:ADMMDual}
\begin{algorithmic}
\REQUIRE $\mu$, $\nu$, $c$, step size $\rho$, scale factor $\alpha$
\STATE $ t \slar 0 $
\STATE $ \lambda^{\rbr{t}} \slar 0 $, $ \eta^{\rbr{t}} \slar 0 $, $ s^{\rbr{t}}, e^{\rbr{t}}, d^{\rbr{t}} \slar 0 $
\WHILE{not converges}
\STATE $ \lambda^{\rbr{ t + 1 }}, \eta^{\rbr{ t + 1 }} = \argmin_{ \lambda, \eta } L_{\rho} \rbr{ \lambda, \mu, e^{\rbr{t}}, d^{\rbr{t}} } $
\STATE $ e^{\rbr{ t + 1 }} = \argmin_e L_{\rho} \rbr{ \lambda^{\rbr{ t + 1 }}, \eta^{\rbr{ t + 1 }}, e, d^{\rbr{t}} } $
\STATE $ d^{\rbr{ t + 1 }}_{ i, j } \slar d^{\rbr{t}}_{ i j } + \alpha \rho \rbr{ c_{ i j } - \lambda_i - \eta_j - e_{ i j } } $
\STATE $ e^{\rbr{ t + 1 }} \slar e^{\rbr{t}} + \alpha \rho \rbr{ s - \widetilde{s} } $
\STATE $ t \slar t + 1 $
\ENDWHILE
\STATE $ s^{\rbr{t}} \slar -d^{\rbr{t}} $
\end{algorithmic}
\end{algorithm}

However, because this algorithm introduces more variables ($ 3 m n + m + n $) and the constraints are not introduced explicitly, it suffers from heavy computation and slow convergence.

\subsection{Approximate augmented Lagragian method for the primal problem}

We have also tried augmented Lagragian method (ALM) to the primal problem directly. The augmented Lagragian is
\begin{equation}
\begin{aligned}
L_{\rho} \rbr{ s, \lambda, \eta } &= \sume{i}{1}{n}{\sume{j}{1}{m}{ c_{ i j } s_{ i j } }} + \iota_+ \rbr{s} \\
&+ \sume{i}{1}{n}{ \lambda_i \rbr{ \mu_i - \sume{j}{1}{m}{s_{ i j }} } } + \sume{j}{1}{m}{ \eta_j \rbr{ \nu_j - \sume{i}{1}{n}{s_{ i j }} } } \\
&+ \frac{\rho}{2} \sume{i}{1}{n}{\rbr{ \mu_i - \sume{j}{1}{m}{s_{ i j }} }^2} + \frac{\rho}{2} \sume{j}{1}{m}{\rbr{ \nu_j - \sume{i}{1}{n}{s_{ i j }} }^2}. \\
\end{aligned}
\end{equation}
To minimize $s$, we adopt a projection gradient step to perform a approximate minimization. The algorithm is listed as Algorithm \ref{Alg:ApproxALM}.

\begin{algorithm}
\caption{Approximate ALM for the primal problem}
\label{Alg:ApproxALM}
\begin{algorithmic}
\REQUIRE $\mu$, $\nu$, $c$, step size $\rho$, scale factor $\alpha$, gradient step size $l$
\STATE $ t \slar 0 $
\STATE $ s^{\rbr{t}} \slar 0 $, $ \lambda^{\rbr{t}} \slar 0 $, $ \eta^{\rbr{t}} \slar 0 $
\WHILE{not converges}
\STATE $ s' = s^{\rbr{t}} - l \nabla_s L_{\rho} \rbr{ s^{\rbr{t}}, \lambda^{\rbr{t}}, \eta^{\rbr{t}} } $
\STATE $ s^{\rbr{ t + 1 }} = \iota_+ \rbr{s'} $
\STATE $ \lambda^{\rbr{ t + 1 }}_i \slar \lambda^{\rbr{t}}_i + \alpha \rho \rbr{ \mu_i - \sume{j}{1}{m}{s_{ i j }} } $
\STATE $ \eta^{\rbr{ t + 1 }}_j \slar \eta^{\rbr{t}}_j + \alpha \rho \rbr{ \nu_i - \sume{i}{1}{n}{s_{ i j }} } $
\STATE $ t \slar t + 1 $
\ENDWHILE
\end{algorithmic}
\end{algorithm}

This algorithm suffers from the approximation step heavily, which makes the error of $\mu$ and $\nu$ hard to reach $10^{-4}$.

\section{Other implementation of transportation simplex method}

As mentioned in Section \ref{Sec:TS}, all the algorithms are implemented in Python, which introduces a huge disadvantages for the transportation simplex method. This is because the dynamic feature of Python, which results in slow discrete operations, especially graph algorithms. Therefore, we empoly the graph package NetworkX in order to realize some graph algorithm, and we call this implementation \textbf{Algorithm \hypertarget{EAlg:3N}{3N}}. However, this algorithm is even slower than Algorithm \ref{Alg:TS}, this is because the package does not include BFS algorithms, which is the key to the whole algorithm.

\section{Multiscale Strategies with ADMM}

In addition to Section \ref{Sec:FOM} and \ref{Sec:MS}, we give a new algorithm combine the multiscale strategy and ADMM to accelerate ADMM.

In the capacity constraint propagation, we add upper bound for every variable $\pi_{ij}$. Note that a variable equal to $0$ can also be saved as a variable with a upper bound $0$.

We save all the paths in a $m \times n$ matrix and use another matrix to save the upper bound of each variable. In the primal ADMM algorithm, we use projection step \eqref{Eq:ProjStep} to deal with the constraints $s_{ij}$. Now we can use the same method to deal with the upper bound. We modify the primal ADMM algorithm slightly into a ADMM algorithm for box constraint optimal transport problem, and combine the algorithm with the multiscale strategy we have implemented before.

Notice that many variables with upper bound $0$ should be $0$ in the optimal solution, so project these variables to $0$ may accelerate the speed of convergence, although each iteration takes the same period of time.

We save the optimal plan in a matrix because we want to use the former ADMM algorithm, so the memory cost of the multiscale strategy is not reduced. However, we have seen in the experiment result that the time cost of the multiscale algorithm are lower.

This algorithm with ADMM is listed as Algorithm \ref{Alg:MSADMM}.

\begin{algorithm} 
\caption{Multiscale strategy with ADMM} \label{Alg:MSADMM}
\begin{algorithmic}
\REQUIRE Source $ \rbr{ X, \mu } $ and target $ \rbr{ Y, \nu } $ as images or point clouds
\STATE Coarsen the datium by grid or downsampling and get two chains $\{ X_j, \mu_j \}^J_{j=0}$,$\{Y_j, \nu_j\}^J_{j=0}$
\STATE Construct a path set include all paths from $\rbr{X_0, \mu_0}$ to $\rbr{Y_0, \nu_0}$
\FOR{$j$ from $0$ to $J - 1$}
\STATE Solve the optimal transport problem with box constraints at scale $j$ by ADMM for the primal problem
\STATE Use capacity constraint propagation and refinement strategy to compute the upper bound matrix (box constraints) of paths at scale $j + 1$
\ENDFOR
\STATE Solve the optimal transport problem without capacity constraints on the path set at scale $j$
\STATE Construct the optimal solution $s$
\end{algorithmic}
\end{algorithm}

We have tested this algorithm in DOTmark dataset. We stop the ADMM algorithm if error of $\mu$ and $\nu$ are below $10^{-4}$. The results is shown in Table \ref{Tbl:ADOT}. The number of iterations refers to the optimization of the coarest scale.

\begin{table}[htbp]
\centering \footnotesize
\begin{tabular}{|c|c|c|c|c|c|c|c|c|c|c|c|}
\hline
class & 1 & 2 & 3 & 4 & 5 & 6 & 7 & 8 & 9 & 10 & average \\ \hline
Algorithm \ref{Alg:ADMMPrimal} & 9794 & 10805 & 10359 & 11867 & 11506 & 10439 &13187 & 10827 & 10646 & 12749 & 11217.9 \\ \hline 
Algorithm \ref{Alg:MSADMM} & 8549 & 11371 & 10973 & 9899 & 9118 & 9826 & 10677 & 10882 & 9848 & 10504 & 10164.7 \\ \hline
\end{tabular}
\caption{Number of iterations on DOTmark dataset} \label{Tbl:ADOT}
\end{table}

From the result, we conclude that Algorithm \ref{Alg:MSADMM} indeed accelerates ADMM.

\section{Entropy regularized methods}

Entropy regularization \parencite{Benamou2015} is an important method in solving large scale optimization problems, which makes the optimization easier and also leads to some fast algorithms. The entropy regularization term is
\begin{equation}
R \rbr{s} = \sume{i}{1}{n}{\sume{j}{1}{m}{ s_{ i j } \rbr{ \ln s_{ i j } - 1 } }}.
\end{equation}
When the regularization coefficient $ \gamma \srar +\infty $, the solution $ s^{\star} \srar \nu^{\rmut} \mu $; and when $ \gamma \srar 0 $, $s^{\star}$ tends to the real solution, as shown in Figure \ref{Fig:Gamma}, where $\gamma$ is valued 1e1, 3e0, 1e0, 3e-1, 1e-1, 3e-2, 1e-2, 3e-3 respectively.

\begin{figure}
\centering
\scalebox{0.25}{%% Creator: Matplotlib, PGF backend
%%
%% To include the figure in your LaTeX document, write
%%   \input{<filename>.pgf}
%%
%% Make sure the required packages are loaded in your preamble
%%   \usepackage{pgf}
%%
%% Figures using additional raster images can only be included by \input if
%% they are in the same directory as the main LaTeX file. For loading figures
%% from other directories you can use the `import` package
%%   \usepackage{import}
%% and then include the figures with
%%   \import{<path to file>}{<filename>.pgf}
%%
%% Matplotlib used the following preamble
%%   \usepackage{fontspec}
%%
\begingroup%
\makeatletter%
\begin{pgfpicture}%
\pgfpathrectangle{\pgfpointorigin}{\pgfqpoint{6.400000in}{4.800000in}}%
\pgfusepath{use as bounding box, clip}%
\begin{pgfscope}%
\pgfsetbuttcap%
\pgfsetmiterjoin%
\definecolor{currentfill}{rgb}{1.000000,1.000000,1.000000}%
\pgfsetfillcolor{currentfill}%
\pgfsetlinewidth{0.000000pt}%
\definecolor{currentstroke}{rgb}{1.000000,1.000000,1.000000}%
\pgfsetstrokecolor{currentstroke}%
\pgfsetdash{}{0pt}%
\pgfpathmoveto{\pgfqpoint{0.000000in}{0.000000in}}%
\pgfpathlineto{\pgfqpoint{6.400000in}{0.000000in}}%
\pgfpathlineto{\pgfqpoint{6.400000in}{4.800000in}}%
\pgfpathlineto{\pgfqpoint{0.000000in}{4.800000in}}%
\pgfpathclose%
\pgfusepath{fill}%
\end{pgfscope}%
\begin{pgfscope}%
\pgfsetbuttcap%
\pgfsetmiterjoin%
\definecolor{currentfill}{rgb}{1.000000,1.000000,1.000000}%
\pgfsetfillcolor{currentfill}%
\pgfsetlinewidth{0.000000pt}%
\definecolor{currentstroke}{rgb}{0.000000,0.000000,0.000000}%
\pgfsetstrokecolor{currentstroke}%
\pgfsetstrokeopacity{0.000000}%
\pgfsetdash{}{0pt}%
\pgfpathmoveto{\pgfqpoint{1.432000in}{0.528000in}}%
\pgfpathlineto{\pgfqpoint{5.128000in}{0.528000in}}%
\pgfpathlineto{\pgfqpoint{5.128000in}{4.224000in}}%
\pgfpathlineto{\pgfqpoint{1.432000in}{4.224000in}}%
\pgfpathclose%
\pgfusepath{fill}%
\end{pgfscope}%
\begin{pgfscope}%
\pgfpathrectangle{\pgfqpoint{1.432000in}{0.528000in}}{\pgfqpoint{3.696000in}{3.696000in}} %
\pgfusepath{clip}%
\pgfsys@transformshift{1.432000in}{0.528000in}%
\pgftext[left,bottom]{\pgfimage[interpolate=true,width=3.700000in,height=3.700000in]{Figure-0001-20180109-013526-071116-img0.png}}%
\end{pgfscope}%
\begin{pgfscope}%
\pgfsetbuttcap%
\pgfsetroundjoin%
\definecolor{currentfill}{rgb}{0.000000,0.000000,0.000000}%
\pgfsetfillcolor{currentfill}%
\pgfsetlinewidth{0.803000pt}%
\definecolor{currentstroke}{rgb}{0.000000,0.000000,0.000000}%
\pgfsetstrokecolor{currentstroke}%
\pgfsetdash{}{0pt}%
\pgfsys@defobject{currentmarker}{\pgfqpoint{0.000000in}{-0.048611in}}{\pgfqpoint{0.000000in}{0.000000in}}{%
\pgfpathmoveto{\pgfqpoint{0.000000in}{0.000000in}}%
\pgfpathlineto{\pgfqpoint{0.000000in}{-0.048611in}}%
\pgfusepath{stroke,fill}%
}%
\begin{pgfscope}%
\pgfsys@transformshift{1.450480in}{0.528000in}%
\pgfsys@useobject{currentmarker}{}%
\end{pgfscope}%
\end{pgfscope}%
\begin{pgfscope}%
\pgftext[x=1.450480in,y=0.430778in,,top]{\rmfamily\fontsize{10.000000}{12.000000}\selectfont \(\displaystyle 0\)}%
\end{pgfscope}%
\begin{pgfscope}%
\pgfsetbuttcap%
\pgfsetroundjoin%
\definecolor{currentfill}{rgb}{0.000000,0.000000,0.000000}%
\pgfsetfillcolor{currentfill}%
\pgfsetlinewidth{0.803000pt}%
\definecolor{currentstroke}{rgb}{0.000000,0.000000,0.000000}%
\pgfsetstrokecolor{currentstroke}%
\pgfsetdash{}{0pt}%
\pgfsys@defobject{currentmarker}{\pgfqpoint{0.000000in}{-0.048611in}}{\pgfqpoint{0.000000in}{0.000000in}}{%
\pgfpathmoveto{\pgfqpoint{0.000000in}{0.000000in}}%
\pgfpathlineto{\pgfqpoint{0.000000in}{-0.048611in}}%
\pgfusepath{stroke,fill}%
}%
\begin{pgfscope}%
\pgfsys@transformshift{2.189680in}{0.528000in}%
\pgfsys@useobject{currentmarker}{}%
\end{pgfscope}%
\end{pgfscope}%
\begin{pgfscope}%
\pgftext[x=2.189680in,y=0.430778in,,top]{\rmfamily\fontsize{10.000000}{12.000000}\selectfont \(\displaystyle 20\)}%
\end{pgfscope}%
\begin{pgfscope}%
\pgfsetbuttcap%
\pgfsetroundjoin%
\definecolor{currentfill}{rgb}{0.000000,0.000000,0.000000}%
\pgfsetfillcolor{currentfill}%
\pgfsetlinewidth{0.803000pt}%
\definecolor{currentstroke}{rgb}{0.000000,0.000000,0.000000}%
\pgfsetstrokecolor{currentstroke}%
\pgfsetdash{}{0pt}%
\pgfsys@defobject{currentmarker}{\pgfqpoint{0.000000in}{-0.048611in}}{\pgfqpoint{0.000000in}{0.000000in}}{%
\pgfpathmoveto{\pgfqpoint{0.000000in}{0.000000in}}%
\pgfpathlineto{\pgfqpoint{0.000000in}{-0.048611in}}%
\pgfusepath{stroke,fill}%
}%
\begin{pgfscope}%
\pgfsys@transformshift{2.928880in}{0.528000in}%
\pgfsys@useobject{currentmarker}{}%
\end{pgfscope}%
\end{pgfscope}%
\begin{pgfscope}%
\pgftext[x=2.928880in,y=0.430778in,,top]{\rmfamily\fontsize{10.000000}{12.000000}\selectfont \(\displaystyle 40\)}%
\end{pgfscope}%
\begin{pgfscope}%
\pgfsetbuttcap%
\pgfsetroundjoin%
\definecolor{currentfill}{rgb}{0.000000,0.000000,0.000000}%
\pgfsetfillcolor{currentfill}%
\pgfsetlinewidth{0.803000pt}%
\definecolor{currentstroke}{rgb}{0.000000,0.000000,0.000000}%
\pgfsetstrokecolor{currentstroke}%
\pgfsetdash{}{0pt}%
\pgfsys@defobject{currentmarker}{\pgfqpoint{0.000000in}{-0.048611in}}{\pgfqpoint{0.000000in}{0.000000in}}{%
\pgfpathmoveto{\pgfqpoint{0.000000in}{0.000000in}}%
\pgfpathlineto{\pgfqpoint{0.000000in}{-0.048611in}}%
\pgfusepath{stroke,fill}%
}%
\begin{pgfscope}%
\pgfsys@transformshift{3.668080in}{0.528000in}%
\pgfsys@useobject{currentmarker}{}%
\end{pgfscope}%
\end{pgfscope}%
\begin{pgfscope}%
\pgftext[x=3.668080in,y=0.430778in,,top]{\rmfamily\fontsize{10.000000}{12.000000}\selectfont \(\displaystyle 60\)}%
\end{pgfscope}%
\begin{pgfscope}%
\pgfsetbuttcap%
\pgfsetroundjoin%
\definecolor{currentfill}{rgb}{0.000000,0.000000,0.000000}%
\pgfsetfillcolor{currentfill}%
\pgfsetlinewidth{0.803000pt}%
\definecolor{currentstroke}{rgb}{0.000000,0.000000,0.000000}%
\pgfsetstrokecolor{currentstroke}%
\pgfsetdash{}{0pt}%
\pgfsys@defobject{currentmarker}{\pgfqpoint{0.000000in}{-0.048611in}}{\pgfqpoint{0.000000in}{0.000000in}}{%
\pgfpathmoveto{\pgfqpoint{0.000000in}{0.000000in}}%
\pgfpathlineto{\pgfqpoint{0.000000in}{-0.048611in}}%
\pgfusepath{stroke,fill}%
}%
\begin{pgfscope}%
\pgfsys@transformshift{4.407280in}{0.528000in}%
\pgfsys@useobject{currentmarker}{}%
\end{pgfscope}%
\end{pgfscope}%
\begin{pgfscope}%
\pgftext[x=4.407280in,y=0.430778in,,top]{\rmfamily\fontsize{10.000000}{12.000000}\selectfont \(\displaystyle 80\)}%
\end{pgfscope}%
\begin{pgfscope}%
\pgfsetbuttcap%
\pgfsetroundjoin%
\definecolor{currentfill}{rgb}{0.000000,0.000000,0.000000}%
\pgfsetfillcolor{currentfill}%
\pgfsetlinewidth{0.803000pt}%
\definecolor{currentstroke}{rgb}{0.000000,0.000000,0.000000}%
\pgfsetstrokecolor{currentstroke}%
\pgfsetdash{}{0pt}%
\pgfsys@defobject{currentmarker}{\pgfqpoint{-0.048611in}{0.000000in}}{\pgfqpoint{0.000000in}{0.000000in}}{%
\pgfpathmoveto{\pgfqpoint{0.000000in}{0.000000in}}%
\pgfpathlineto{\pgfqpoint{-0.048611in}{0.000000in}}%
\pgfusepath{stroke,fill}%
}%
\begin{pgfscope}%
\pgfsys@transformshift{1.432000in}{4.205520in}%
\pgfsys@useobject{currentmarker}{}%
\end{pgfscope}%
\end{pgfscope}%
\begin{pgfscope}%
\pgftext[x=1.265333in,y=4.157326in,left,base]{\rmfamily\fontsize{10.000000}{12.000000}\selectfont \(\displaystyle 0\)}%
\end{pgfscope}%
\begin{pgfscope}%
\pgfsetbuttcap%
\pgfsetroundjoin%
\definecolor{currentfill}{rgb}{0.000000,0.000000,0.000000}%
\pgfsetfillcolor{currentfill}%
\pgfsetlinewidth{0.803000pt}%
\definecolor{currentstroke}{rgb}{0.000000,0.000000,0.000000}%
\pgfsetstrokecolor{currentstroke}%
\pgfsetdash{}{0pt}%
\pgfsys@defobject{currentmarker}{\pgfqpoint{-0.048611in}{0.000000in}}{\pgfqpoint{0.000000in}{0.000000in}}{%
\pgfpathmoveto{\pgfqpoint{0.000000in}{0.000000in}}%
\pgfpathlineto{\pgfqpoint{-0.048611in}{0.000000in}}%
\pgfusepath{stroke,fill}%
}%
\begin{pgfscope}%
\pgfsys@transformshift{1.432000in}{3.466320in}%
\pgfsys@useobject{currentmarker}{}%
\end{pgfscope}%
\end{pgfscope}%
\begin{pgfscope}%
\pgftext[x=1.195888in,y=3.418126in,left,base]{\rmfamily\fontsize{10.000000}{12.000000}\selectfont \(\displaystyle 20\)}%
\end{pgfscope}%
\begin{pgfscope}%
\pgfsetbuttcap%
\pgfsetroundjoin%
\definecolor{currentfill}{rgb}{0.000000,0.000000,0.000000}%
\pgfsetfillcolor{currentfill}%
\pgfsetlinewidth{0.803000pt}%
\definecolor{currentstroke}{rgb}{0.000000,0.000000,0.000000}%
\pgfsetstrokecolor{currentstroke}%
\pgfsetdash{}{0pt}%
\pgfsys@defobject{currentmarker}{\pgfqpoint{-0.048611in}{0.000000in}}{\pgfqpoint{0.000000in}{0.000000in}}{%
\pgfpathmoveto{\pgfqpoint{0.000000in}{0.000000in}}%
\pgfpathlineto{\pgfqpoint{-0.048611in}{0.000000in}}%
\pgfusepath{stroke,fill}%
}%
\begin{pgfscope}%
\pgfsys@transformshift{1.432000in}{2.727120in}%
\pgfsys@useobject{currentmarker}{}%
\end{pgfscope}%
\end{pgfscope}%
\begin{pgfscope}%
\pgftext[x=1.195888in,y=2.678926in,left,base]{\rmfamily\fontsize{10.000000}{12.000000}\selectfont \(\displaystyle 40\)}%
\end{pgfscope}%
\begin{pgfscope}%
\pgfsetbuttcap%
\pgfsetroundjoin%
\definecolor{currentfill}{rgb}{0.000000,0.000000,0.000000}%
\pgfsetfillcolor{currentfill}%
\pgfsetlinewidth{0.803000pt}%
\definecolor{currentstroke}{rgb}{0.000000,0.000000,0.000000}%
\pgfsetstrokecolor{currentstroke}%
\pgfsetdash{}{0pt}%
\pgfsys@defobject{currentmarker}{\pgfqpoint{-0.048611in}{0.000000in}}{\pgfqpoint{0.000000in}{0.000000in}}{%
\pgfpathmoveto{\pgfqpoint{0.000000in}{0.000000in}}%
\pgfpathlineto{\pgfqpoint{-0.048611in}{0.000000in}}%
\pgfusepath{stroke,fill}%
}%
\begin{pgfscope}%
\pgfsys@transformshift{1.432000in}{1.987920in}%
\pgfsys@useobject{currentmarker}{}%
\end{pgfscope}%
\end{pgfscope}%
\begin{pgfscope}%
\pgftext[x=1.195888in,y=1.939726in,left,base]{\rmfamily\fontsize{10.000000}{12.000000}\selectfont \(\displaystyle 60\)}%
\end{pgfscope}%
\begin{pgfscope}%
\pgfsetbuttcap%
\pgfsetroundjoin%
\definecolor{currentfill}{rgb}{0.000000,0.000000,0.000000}%
\pgfsetfillcolor{currentfill}%
\pgfsetlinewidth{0.803000pt}%
\definecolor{currentstroke}{rgb}{0.000000,0.000000,0.000000}%
\pgfsetstrokecolor{currentstroke}%
\pgfsetdash{}{0pt}%
\pgfsys@defobject{currentmarker}{\pgfqpoint{-0.048611in}{0.000000in}}{\pgfqpoint{0.000000in}{0.000000in}}{%
\pgfpathmoveto{\pgfqpoint{0.000000in}{0.000000in}}%
\pgfpathlineto{\pgfqpoint{-0.048611in}{0.000000in}}%
\pgfusepath{stroke,fill}%
}%
\begin{pgfscope}%
\pgfsys@transformshift{1.432000in}{1.248720in}%
\pgfsys@useobject{currentmarker}{}%
\end{pgfscope}%
\end{pgfscope}%
\begin{pgfscope}%
\pgftext[x=1.195888in,y=1.200526in,left,base]{\rmfamily\fontsize{10.000000}{12.000000}\selectfont \(\displaystyle 80\)}%
\end{pgfscope}%
\begin{pgfscope}%
\pgfsetrectcap%
\pgfsetmiterjoin%
\pgfsetlinewidth{0.803000pt}%
\definecolor{currentstroke}{rgb}{0.000000,0.000000,0.000000}%
\pgfsetstrokecolor{currentstroke}%
\pgfsetdash{}{0pt}%
\pgfpathmoveto{\pgfqpoint{1.432000in}{0.528000in}}%
\pgfpathlineto{\pgfqpoint{1.432000in}{4.224000in}}%
\pgfusepath{stroke}%
\end{pgfscope}%
\begin{pgfscope}%
\pgfsetrectcap%
\pgfsetmiterjoin%
\pgfsetlinewidth{0.803000pt}%
\definecolor{currentstroke}{rgb}{0.000000,0.000000,0.000000}%
\pgfsetstrokecolor{currentstroke}%
\pgfsetdash{}{0pt}%
\pgfpathmoveto{\pgfqpoint{5.128000in}{0.528000in}}%
\pgfpathlineto{\pgfqpoint{5.128000in}{4.224000in}}%
\pgfusepath{stroke}%
\end{pgfscope}%
\begin{pgfscope}%
\pgfsetrectcap%
\pgfsetmiterjoin%
\pgfsetlinewidth{0.803000pt}%
\definecolor{currentstroke}{rgb}{0.000000,0.000000,0.000000}%
\pgfsetstrokecolor{currentstroke}%
\pgfsetdash{}{0pt}%
\pgfpathmoveto{\pgfqpoint{1.432000in}{0.528000in}}%
\pgfpathlineto{\pgfqpoint{5.128000in}{0.528000in}}%
\pgfusepath{stroke}%
\end{pgfscope}%
\begin{pgfscope}%
\pgfsetrectcap%
\pgfsetmiterjoin%
\pgfsetlinewidth{0.803000pt}%
\definecolor{currentstroke}{rgb}{0.000000,0.000000,0.000000}%
\pgfsetstrokecolor{currentstroke}%
\pgfsetdash{}{0pt}%
\pgfpathmoveto{\pgfqpoint{1.432000in}{4.224000in}}%
\pgfpathlineto{\pgfqpoint{5.128000in}{4.224000in}}%
\pgfusepath{stroke}%
\end{pgfscope}%
\end{pgfpicture}%
\makeatother%
\endgroup%
} 
\hspace{-0.8cm}
\scalebox{0.25}{%% Creator: Matplotlib, PGF backend
%%
%% To include the figure in your LaTeX document, write
%%   \input{<filename>.pgf}
%%
%% Make sure the required packages are loaded in your preamble
%%   \usepackage{pgf}
%%
%% Figures using additional raster images can only be included by \input if
%% they are in the same directory as the main LaTeX file. For loading figures
%% from other directories you can use the `import` package
%%   \usepackage{import}
%% and then include the figures with
%%   \import{<path to file>}{<filename>.pgf}
%%
%% Matplotlib used the following preamble
%%   \usepackage{fontspec}
%%
\begingroup%
\makeatletter%
\begin{pgfpicture}%
\pgfpathrectangle{\pgfpointorigin}{\pgfqpoint{6.400000in}{4.800000in}}%
\pgfusepath{use as bounding box, clip}%
\begin{pgfscope}%
\pgfsetbuttcap%
\pgfsetmiterjoin%
\definecolor{currentfill}{rgb}{1.000000,1.000000,1.000000}%
\pgfsetfillcolor{currentfill}%
\pgfsetlinewidth{0.000000pt}%
\definecolor{currentstroke}{rgb}{1.000000,1.000000,1.000000}%
\pgfsetstrokecolor{currentstroke}%
\pgfsetdash{}{0pt}%
\pgfpathmoveto{\pgfqpoint{0.000000in}{0.000000in}}%
\pgfpathlineto{\pgfqpoint{6.400000in}{0.000000in}}%
\pgfpathlineto{\pgfqpoint{6.400000in}{4.800000in}}%
\pgfpathlineto{\pgfqpoint{0.000000in}{4.800000in}}%
\pgfpathclose%
\pgfusepath{fill}%
\end{pgfscope}%
\begin{pgfscope}%
\pgfsetbuttcap%
\pgfsetmiterjoin%
\definecolor{currentfill}{rgb}{1.000000,1.000000,1.000000}%
\pgfsetfillcolor{currentfill}%
\pgfsetlinewidth{0.000000pt}%
\definecolor{currentstroke}{rgb}{0.000000,0.000000,0.000000}%
\pgfsetstrokecolor{currentstroke}%
\pgfsetstrokeopacity{0.000000}%
\pgfsetdash{}{0pt}%
\pgfpathmoveto{\pgfqpoint{1.432000in}{0.528000in}}%
\pgfpathlineto{\pgfqpoint{5.128000in}{0.528000in}}%
\pgfpathlineto{\pgfqpoint{5.128000in}{4.224000in}}%
\pgfpathlineto{\pgfqpoint{1.432000in}{4.224000in}}%
\pgfpathclose%
\pgfusepath{fill}%
\end{pgfscope}%
\begin{pgfscope}%
\pgfpathrectangle{\pgfqpoint{1.432000in}{0.528000in}}{\pgfqpoint{3.696000in}{3.696000in}} %
\pgfusepath{clip}%
\pgfsys@transformshift{1.432000in}{0.528000in}%
\pgftext[left,bottom]{\pgfimage[interpolate=true,width=3.700000in,height=3.700000in]{Figure-0002-20180109-013529-629617-img0.png}}%
\end{pgfscope}%
\begin{pgfscope}%
\pgfsetbuttcap%
\pgfsetroundjoin%
\definecolor{currentfill}{rgb}{0.000000,0.000000,0.000000}%
\pgfsetfillcolor{currentfill}%
\pgfsetlinewidth{0.803000pt}%
\definecolor{currentstroke}{rgb}{0.000000,0.000000,0.000000}%
\pgfsetstrokecolor{currentstroke}%
\pgfsetdash{}{0pt}%
\pgfsys@defobject{currentmarker}{\pgfqpoint{0.000000in}{-0.048611in}}{\pgfqpoint{0.000000in}{0.000000in}}{%
\pgfpathmoveto{\pgfqpoint{0.000000in}{0.000000in}}%
\pgfpathlineto{\pgfqpoint{0.000000in}{-0.048611in}}%
\pgfusepath{stroke,fill}%
}%
\begin{pgfscope}%
\pgfsys@transformshift{1.450480in}{0.528000in}%
\pgfsys@useobject{currentmarker}{}%
\end{pgfscope}%
\end{pgfscope}%
\begin{pgfscope}%
\pgftext[x=1.450480in,y=0.430778in,,top]{\rmfamily\fontsize{10.000000}{12.000000}\selectfont \(\displaystyle 0\)}%
\end{pgfscope}%
\begin{pgfscope}%
\pgfsetbuttcap%
\pgfsetroundjoin%
\definecolor{currentfill}{rgb}{0.000000,0.000000,0.000000}%
\pgfsetfillcolor{currentfill}%
\pgfsetlinewidth{0.803000pt}%
\definecolor{currentstroke}{rgb}{0.000000,0.000000,0.000000}%
\pgfsetstrokecolor{currentstroke}%
\pgfsetdash{}{0pt}%
\pgfsys@defobject{currentmarker}{\pgfqpoint{0.000000in}{-0.048611in}}{\pgfqpoint{0.000000in}{0.000000in}}{%
\pgfpathmoveto{\pgfqpoint{0.000000in}{0.000000in}}%
\pgfpathlineto{\pgfqpoint{0.000000in}{-0.048611in}}%
\pgfusepath{stroke,fill}%
}%
\begin{pgfscope}%
\pgfsys@transformshift{2.189680in}{0.528000in}%
\pgfsys@useobject{currentmarker}{}%
\end{pgfscope}%
\end{pgfscope}%
\begin{pgfscope}%
\pgftext[x=2.189680in,y=0.430778in,,top]{\rmfamily\fontsize{10.000000}{12.000000}\selectfont \(\displaystyle 20\)}%
\end{pgfscope}%
\begin{pgfscope}%
\pgfsetbuttcap%
\pgfsetroundjoin%
\definecolor{currentfill}{rgb}{0.000000,0.000000,0.000000}%
\pgfsetfillcolor{currentfill}%
\pgfsetlinewidth{0.803000pt}%
\definecolor{currentstroke}{rgb}{0.000000,0.000000,0.000000}%
\pgfsetstrokecolor{currentstroke}%
\pgfsetdash{}{0pt}%
\pgfsys@defobject{currentmarker}{\pgfqpoint{0.000000in}{-0.048611in}}{\pgfqpoint{0.000000in}{0.000000in}}{%
\pgfpathmoveto{\pgfqpoint{0.000000in}{0.000000in}}%
\pgfpathlineto{\pgfqpoint{0.000000in}{-0.048611in}}%
\pgfusepath{stroke,fill}%
}%
\begin{pgfscope}%
\pgfsys@transformshift{2.928880in}{0.528000in}%
\pgfsys@useobject{currentmarker}{}%
\end{pgfscope}%
\end{pgfscope}%
\begin{pgfscope}%
\pgftext[x=2.928880in,y=0.430778in,,top]{\rmfamily\fontsize{10.000000}{12.000000}\selectfont \(\displaystyle 40\)}%
\end{pgfscope}%
\begin{pgfscope}%
\pgfsetbuttcap%
\pgfsetroundjoin%
\definecolor{currentfill}{rgb}{0.000000,0.000000,0.000000}%
\pgfsetfillcolor{currentfill}%
\pgfsetlinewidth{0.803000pt}%
\definecolor{currentstroke}{rgb}{0.000000,0.000000,0.000000}%
\pgfsetstrokecolor{currentstroke}%
\pgfsetdash{}{0pt}%
\pgfsys@defobject{currentmarker}{\pgfqpoint{0.000000in}{-0.048611in}}{\pgfqpoint{0.000000in}{0.000000in}}{%
\pgfpathmoveto{\pgfqpoint{0.000000in}{0.000000in}}%
\pgfpathlineto{\pgfqpoint{0.000000in}{-0.048611in}}%
\pgfusepath{stroke,fill}%
}%
\begin{pgfscope}%
\pgfsys@transformshift{3.668080in}{0.528000in}%
\pgfsys@useobject{currentmarker}{}%
\end{pgfscope}%
\end{pgfscope}%
\begin{pgfscope}%
\pgftext[x=3.668080in,y=0.430778in,,top]{\rmfamily\fontsize{10.000000}{12.000000}\selectfont \(\displaystyle 60\)}%
\end{pgfscope}%
\begin{pgfscope}%
\pgfsetbuttcap%
\pgfsetroundjoin%
\definecolor{currentfill}{rgb}{0.000000,0.000000,0.000000}%
\pgfsetfillcolor{currentfill}%
\pgfsetlinewidth{0.803000pt}%
\definecolor{currentstroke}{rgb}{0.000000,0.000000,0.000000}%
\pgfsetstrokecolor{currentstroke}%
\pgfsetdash{}{0pt}%
\pgfsys@defobject{currentmarker}{\pgfqpoint{0.000000in}{-0.048611in}}{\pgfqpoint{0.000000in}{0.000000in}}{%
\pgfpathmoveto{\pgfqpoint{0.000000in}{0.000000in}}%
\pgfpathlineto{\pgfqpoint{0.000000in}{-0.048611in}}%
\pgfusepath{stroke,fill}%
}%
\begin{pgfscope}%
\pgfsys@transformshift{4.407280in}{0.528000in}%
\pgfsys@useobject{currentmarker}{}%
\end{pgfscope}%
\end{pgfscope}%
\begin{pgfscope}%
\pgftext[x=4.407280in,y=0.430778in,,top]{\rmfamily\fontsize{10.000000}{12.000000}\selectfont \(\displaystyle 80\)}%
\end{pgfscope}%
\begin{pgfscope}%
\pgfsetbuttcap%
\pgfsetroundjoin%
\definecolor{currentfill}{rgb}{0.000000,0.000000,0.000000}%
\pgfsetfillcolor{currentfill}%
\pgfsetlinewidth{0.803000pt}%
\definecolor{currentstroke}{rgb}{0.000000,0.000000,0.000000}%
\pgfsetstrokecolor{currentstroke}%
\pgfsetdash{}{0pt}%
\pgfsys@defobject{currentmarker}{\pgfqpoint{-0.048611in}{0.000000in}}{\pgfqpoint{0.000000in}{0.000000in}}{%
\pgfpathmoveto{\pgfqpoint{0.000000in}{0.000000in}}%
\pgfpathlineto{\pgfqpoint{-0.048611in}{0.000000in}}%
\pgfusepath{stroke,fill}%
}%
\begin{pgfscope}%
\pgfsys@transformshift{1.432000in}{4.205520in}%
\pgfsys@useobject{currentmarker}{}%
\end{pgfscope}%
\end{pgfscope}%
\begin{pgfscope}%
\pgftext[x=1.265333in,y=4.157326in,left,base]{\rmfamily\fontsize{10.000000}{12.000000}\selectfont \(\displaystyle 0\)}%
\end{pgfscope}%
\begin{pgfscope}%
\pgfsetbuttcap%
\pgfsetroundjoin%
\definecolor{currentfill}{rgb}{0.000000,0.000000,0.000000}%
\pgfsetfillcolor{currentfill}%
\pgfsetlinewidth{0.803000pt}%
\definecolor{currentstroke}{rgb}{0.000000,0.000000,0.000000}%
\pgfsetstrokecolor{currentstroke}%
\pgfsetdash{}{0pt}%
\pgfsys@defobject{currentmarker}{\pgfqpoint{-0.048611in}{0.000000in}}{\pgfqpoint{0.000000in}{0.000000in}}{%
\pgfpathmoveto{\pgfqpoint{0.000000in}{0.000000in}}%
\pgfpathlineto{\pgfqpoint{-0.048611in}{0.000000in}}%
\pgfusepath{stroke,fill}%
}%
\begin{pgfscope}%
\pgfsys@transformshift{1.432000in}{3.466320in}%
\pgfsys@useobject{currentmarker}{}%
\end{pgfscope}%
\end{pgfscope}%
\begin{pgfscope}%
\pgftext[x=1.195888in,y=3.418126in,left,base]{\rmfamily\fontsize{10.000000}{12.000000}\selectfont \(\displaystyle 20\)}%
\end{pgfscope}%
\begin{pgfscope}%
\pgfsetbuttcap%
\pgfsetroundjoin%
\definecolor{currentfill}{rgb}{0.000000,0.000000,0.000000}%
\pgfsetfillcolor{currentfill}%
\pgfsetlinewidth{0.803000pt}%
\definecolor{currentstroke}{rgb}{0.000000,0.000000,0.000000}%
\pgfsetstrokecolor{currentstroke}%
\pgfsetdash{}{0pt}%
\pgfsys@defobject{currentmarker}{\pgfqpoint{-0.048611in}{0.000000in}}{\pgfqpoint{0.000000in}{0.000000in}}{%
\pgfpathmoveto{\pgfqpoint{0.000000in}{0.000000in}}%
\pgfpathlineto{\pgfqpoint{-0.048611in}{0.000000in}}%
\pgfusepath{stroke,fill}%
}%
\begin{pgfscope}%
\pgfsys@transformshift{1.432000in}{2.727120in}%
\pgfsys@useobject{currentmarker}{}%
\end{pgfscope}%
\end{pgfscope}%
\begin{pgfscope}%
\pgftext[x=1.195888in,y=2.678926in,left,base]{\rmfamily\fontsize{10.000000}{12.000000}\selectfont \(\displaystyle 40\)}%
\end{pgfscope}%
\begin{pgfscope}%
\pgfsetbuttcap%
\pgfsetroundjoin%
\definecolor{currentfill}{rgb}{0.000000,0.000000,0.000000}%
\pgfsetfillcolor{currentfill}%
\pgfsetlinewidth{0.803000pt}%
\definecolor{currentstroke}{rgb}{0.000000,0.000000,0.000000}%
\pgfsetstrokecolor{currentstroke}%
\pgfsetdash{}{0pt}%
\pgfsys@defobject{currentmarker}{\pgfqpoint{-0.048611in}{0.000000in}}{\pgfqpoint{0.000000in}{0.000000in}}{%
\pgfpathmoveto{\pgfqpoint{0.000000in}{0.000000in}}%
\pgfpathlineto{\pgfqpoint{-0.048611in}{0.000000in}}%
\pgfusepath{stroke,fill}%
}%
\begin{pgfscope}%
\pgfsys@transformshift{1.432000in}{1.987920in}%
\pgfsys@useobject{currentmarker}{}%
\end{pgfscope}%
\end{pgfscope}%
\begin{pgfscope}%
\pgftext[x=1.195888in,y=1.939726in,left,base]{\rmfamily\fontsize{10.000000}{12.000000}\selectfont \(\displaystyle 60\)}%
\end{pgfscope}%
\begin{pgfscope}%
\pgfsetbuttcap%
\pgfsetroundjoin%
\definecolor{currentfill}{rgb}{0.000000,0.000000,0.000000}%
\pgfsetfillcolor{currentfill}%
\pgfsetlinewidth{0.803000pt}%
\definecolor{currentstroke}{rgb}{0.000000,0.000000,0.000000}%
\pgfsetstrokecolor{currentstroke}%
\pgfsetdash{}{0pt}%
\pgfsys@defobject{currentmarker}{\pgfqpoint{-0.048611in}{0.000000in}}{\pgfqpoint{0.000000in}{0.000000in}}{%
\pgfpathmoveto{\pgfqpoint{0.000000in}{0.000000in}}%
\pgfpathlineto{\pgfqpoint{-0.048611in}{0.000000in}}%
\pgfusepath{stroke,fill}%
}%
\begin{pgfscope}%
\pgfsys@transformshift{1.432000in}{1.248720in}%
\pgfsys@useobject{currentmarker}{}%
\end{pgfscope}%
\end{pgfscope}%
\begin{pgfscope}%
\pgftext[x=1.195888in,y=1.200526in,left,base]{\rmfamily\fontsize{10.000000}{12.000000}\selectfont \(\displaystyle 80\)}%
\end{pgfscope}%
\begin{pgfscope}%
\pgfsetrectcap%
\pgfsetmiterjoin%
\pgfsetlinewidth{0.803000pt}%
\definecolor{currentstroke}{rgb}{0.000000,0.000000,0.000000}%
\pgfsetstrokecolor{currentstroke}%
\pgfsetdash{}{0pt}%
\pgfpathmoveto{\pgfqpoint{1.432000in}{0.528000in}}%
\pgfpathlineto{\pgfqpoint{1.432000in}{4.224000in}}%
\pgfusepath{stroke}%
\end{pgfscope}%
\begin{pgfscope}%
\pgfsetrectcap%
\pgfsetmiterjoin%
\pgfsetlinewidth{0.803000pt}%
\definecolor{currentstroke}{rgb}{0.000000,0.000000,0.000000}%
\pgfsetstrokecolor{currentstroke}%
\pgfsetdash{}{0pt}%
\pgfpathmoveto{\pgfqpoint{5.128000in}{0.528000in}}%
\pgfpathlineto{\pgfqpoint{5.128000in}{4.224000in}}%
\pgfusepath{stroke}%
\end{pgfscope}%
\begin{pgfscope}%
\pgfsetrectcap%
\pgfsetmiterjoin%
\pgfsetlinewidth{0.803000pt}%
\definecolor{currentstroke}{rgb}{0.000000,0.000000,0.000000}%
\pgfsetstrokecolor{currentstroke}%
\pgfsetdash{}{0pt}%
\pgfpathmoveto{\pgfqpoint{1.432000in}{0.528000in}}%
\pgfpathlineto{\pgfqpoint{5.128000in}{0.528000in}}%
\pgfusepath{stroke}%
\end{pgfscope}%
\begin{pgfscope}%
\pgfsetrectcap%
\pgfsetmiterjoin%
\pgfsetlinewidth{0.803000pt}%
\definecolor{currentstroke}{rgb}{0.000000,0.000000,0.000000}%
\pgfsetstrokecolor{currentstroke}%
\pgfsetdash{}{0pt}%
\pgfpathmoveto{\pgfqpoint{1.432000in}{4.224000in}}%
\pgfpathlineto{\pgfqpoint{5.128000in}{4.224000in}}%
\pgfusepath{stroke}%
\end{pgfscope}%
\end{pgfpicture}%
\makeatother%
\endgroup%
}
\hspace{-0.8cm} 
\scalebox{0.25}{%% Creator: Matplotlib, PGF backend
%%
%% To include the figure in your LaTeX document, write
%%   \input{<filename>.pgf}
%%
%% Make sure the required packages are loaded in your preamble
%%   \usepackage{pgf}
%%
%% Figures using additional raster images can only be included by \input if
%% they are in the same directory as the main LaTeX file. For loading figures
%% from other directories you can use the `import` package
%%   \usepackage{import}
%% and then include the figures with
%%   \import{<path to file>}{<filename>.pgf}
%%
%% Matplotlib used the following preamble
%%   \usepackage{fontspec}
%%
\begingroup%
\makeatletter%
\begin{pgfpicture}%
\pgfpathrectangle{\pgfpointorigin}{\pgfqpoint{6.400000in}{4.800000in}}%
\pgfusepath{use as bounding box, clip}%
\begin{pgfscope}%
\pgfsetbuttcap%
\pgfsetmiterjoin%
\definecolor{currentfill}{rgb}{1.000000,1.000000,1.000000}%
\pgfsetfillcolor{currentfill}%
\pgfsetlinewidth{0.000000pt}%
\definecolor{currentstroke}{rgb}{1.000000,1.000000,1.000000}%
\pgfsetstrokecolor{currentstroke}%
\pgfsetdash{}{0pt}%
\pgfpathmoveto{\pgfqpoint{0.000000in}{0.000000in}}%
\pgfpathlineto{\pgfqpoint{6.400000in}{0.000000in}}%
\pgfpathlineto{\pgfqpoint{6.400000in}{4.800000in}}%
\pgfpathlineto{\pgfqpoint{0.000000in}{4.800000in}}%
\pgfpathclose%
\pgfusepath{fill}%
\end{pgfscope}%
\begin{pgfscope}%
\pgfsetbuttcap%
\pgfsetmiterjoin%
\definecolor{currentfill}{rgb}{1.000000,1.000000,1.000000}%
\pgfsetfillcolor{currentfill}%
\pgfsetlinewidth{0.000000pt}%
\definecolor{currentstroke}{rgb}{0.000000,0.000000,0.000000}%
\pgfsetstrokecolor{currentstroke}%
\pgfsetstrokeopacity{0.000000}%
\pgfsetdash{}{0pt}%
\pgfpathmoveto{\pgfqpoint{1.432000in}{0.528000in}}%
\pgfpathlineto{\pgfqpoint{5.128000in}{0.528000in}}%
\pgfpathlineto{\pgfqpoint{5.128000in}{4.224000in}}%
\pgfpathlineto{\pgfqpoint{1.432000in}{4.224000in}}%
\pgfpathclose%
\pgfusepath{fill}%
\end{pgfscope}%
\begin{pgfscope}%
\pgfpathrectangle{\pgfqpoint{1.432000in}{0.528000in}}{\pgfqpoint{3.696000in}{3.696000in}} %
\pgfusepath{clip}%
\pgfsys@transformshift{1.432000in}{0.528000in}%
\pgftext[left,bottom]{\pgfimage[interpolate=true,width=3.700000in,height=3.700000in]{Figure-0003-20180109-013531-985313-img0.png}}%
\end{pgfscope}%
\begin{pgfscope}%
\pgfsetbuttcap%
\pgfsetroundjoin%
\definecolor{currentfill}{rgb}{0.000000,0.000000,0.000000}%
\pgfsetfillcolor{currentfill}%
\pgfsetlinewidth{0.803000pt}%
\definecolor{currentstroke}{rgb}{0.000000,0.000000,0.000000}%
\pgfsetstrokecolor{currentstroke}%
\pgfsetdash{}{0pt}%
\pgfsys@defobject{currentmarker}{\pgfqpoint{0.000000in}{-0.048611in}}{\pgfqpoint{0.000000in}{0.000000in}}{%
\pgfpathmoveto{\pgfqpoint{0.000000in}{0.000000in}}%
\pgfpathlineto{\pgfqpoint{0.000000in}{-0.048611in}}%
\pgfusepath{stroke,fill}%
}%
\begin{pgfscope}%
\pgfsys@transformshift{1.450480in}{0.528000in}%
\pgfsys@useobject{currentmarker}{}%
\end{pgfscope}%
\end{pgfscope}%
\begin{pgfscope}%
\pgftext[x=1.450480in,y=0.430778in,,top]{\rmfamily\fontsize{10.000000}{12.000000}\selectfont \(\displaystyle 0\)}%
\end{pgfscope}%
\begin{pgfscope}%
\pgfsetbuttcap%
\pgfsetroundjoin%
\definecolor{currentfill}{rgb}{0.000000,0.000000,0.000000}%
\pgfsetfillcolor{currentfill}%
\pgfsetlinewidth{0.803000pt}%
\definecolor{currentstroke}{rgb}{0.000000,0.000000,0.000000}%
\pgfsetstrokecolor{currentstroke}%
\pgfsetdash{}{0pt}%
\pgfsys@defobject{currentmarker}{\pgfqpoint{0.000000in}{-0.048611in}}{\pgfqpoint{0.000000in}{0.000000in}}{%
\pgfpathmoveto{\pgfqpoint{0.000000in}{0.000000in}}%
\pgfpathlineto{\pgfqpoint{0.000000in}{-0.048611in}}%
\pgfusepath{stroke,fill}%
}%
\begin{pgfscope}%
\pgfsys@transformshift{2.189680in}{0.528000in}%
\pgfsys@useobject{currentmarker}{}%
\end{pgfscope}%
\end{pgfscope}%
\begin{pgfscope}%
\pgftext[x=2.189680in,y=0.430778in,,top]{\rmfamily\fontsize{10.000000}{12.000000}\selectfont \(\displaystyle 20\)}%
\end{pgfscope}%
\begin{pgfscope}%
\pgfsetbuttcap%
\pgfsetroundjoin%
\definecolor{currentfill}{rgb}{0.000000,0.000000,0.000000}%
\pgfsetfillcolor{currentfill}%
\pgfsetlinewidth{0.803000pt}%
\definecolor{currentstroke}{rgb}{0.000000,0.000000,0.000000}%
\pgfsetstrokecolor{currentstroke}%
\pgfsetdash{}{0pt}%
\pgfsys@defobject{currentmarker}{\pgfqpoint{0.000000in}{-0.048611in}}{\pgfqpoint{0.000000in}{0.000000in}}{%
\pgfpathmoveto{\pgfqpoint{0.000000in}{0.000000in}}%
\pgfpathlineto{\pgfqpoint{0.000000in}{-0.048611in}}%
\pgfusepath{stroke,fill}%
}%
\begin{pgfscope}%
\pgfsys@transformshift{2.928880in}{0.528000in}%
\pgfsys@useobject{currentmarker}{}%
\end{pgfscope}%
\end{pgfscope}%
\begin{pgfscope}%
\pgftext[x=2.928880in,y=0.430778in,,top]{\rmfamily\fontsize{10.000000}{12.000000}\selectfont \(\displaystyle 40\)}%
\end{pgfscope}%
\begin{pgfscope}%
\pgfsetbuttcap%
\pgfsetroundjoin%
\definecolor{currentfill}{rgb}{0.000000,0.000000,0.000000}%
\pgfsetfillcolor{currentfill}%
\pgfsetlinewidth{0.803000pt}%
\definecolor{currentstroke}{rgb}{0.000000,0.000000,0.000000}%
\pgfsetstrokecolor{currentstroke}%
\pgfsetdash{}{0pt}%
\pgfsys@defobject{currentmarker}{\pgfqpoint{0.000000in}{-0.048611in}}{\pgfqpoint{0.000000in}{0.000000in}}{%
\pgfpathmoveto{\pgfqpoint{0.000000in}{0.000000in}}%
\pgfpathlineto{\pgfqpoint{0.000000in}{-0.048611in}}%
\pgfusepath{stroke,fill}%
}%
\begin{pgfscope}%
\pgfsys@transformshift{3.668080in}{0.528000in}%
\pgfsys@useobject{currentmarker}{}%
\end{pgfscope}%
\end{pgfscope}%
\begin{pgfscope}%
\pgftext[x=3.668080in,y=0.430778in,,top]{\rmfamily\fontsize{10.000000}{12.000000}\selectfont \(\displaystyle 60\)}%
\end{pgfscope}%
\begin{pgfscope}%
\pgfsetbuttcap%
\pgfsetroundjoin%
\definecolor{currentfill}{rgb}{0.000000,0.000000,0.000000}%
\pgfsetfillcolor{currentfill}%
\pgfsetlinewidth{0.803000pt}%
\definecolor{currentstroke}{rgb}{0.000000,0.000000,0.000000}%
\pgfsetstrokecolor{currentstroke}%
\pgfsetdash{}{0pt}%
\pgfsys@defobject{currentmarker}{\pgfqpoint{0.000000in}{-0.048611in}}{\pgfqpoint{0.000000in}{0.000000in}}{%
\pgfpathmoveto{\pgfqpoint{0.000000in}{0.000000in}}%
\pgfpathlineto{\pgfqpoint{0.000000in}{-0.048611in}}%
\pgfusepath{stroke,fill}%
}%
\begin{pgfscope}%
\pgfsys@transformshift{4.407280in}{0.528000in}%
\pgfsys@useobject{currentmarker}{}%
\end{pgfscope}%
\end{pgfscope}%
\begin{pgfscope}%
\pgftext[x=4.407280in,y=0.430778in,,top]{\rmfamily\fontsize{10.000000}{12.000000}\selectfont \(\displaystyle 80\)}%
\end{pgfscope}%
\begin{pgfscope}%
\pgfsetbuttcap%
\pgfsetroundjoin%
\definecolor{currentfill}{rgb}{0.000000,0.000000,0.000000}%
\pgfsetfillcolor{currentfill}%
\pgfsetlinewidth{0.803000pt}%
\definecolor{currentstroke}{rgb}{0.000000,0.000000,0.000000}%
\pgfsetstrokecolor{currentstroke}%
\pgfsetdash{}{0pt}%
\pgfsys@defobject{currentmarker}{\pgfqpoint{-0.048611in}{0.000000in}}{\pgfqpoint{0.000000in}{0.000000in}}{%
\pgfpathmoveto{\pgfqpoint{0.000000in}{0.000000in}}%
\pgfpathlineto{\pgfqpoint{-0.048611in}{0.000000in}}%
\pgfusepath{stroke,fill}%
}%
\begin{pgfscope}%
\pgfsys@transformshift{1.432000in}{4.205520in}%
\pgfsys@useobject{currentmarker}{}%
\end{pgfscope}%
\end{pgfscope}%
\begin{pgfscope}%
\pgftext[x=1.265333in,y=4.157326in,left,base]{\rmfamily\fontsize{10.000000}{12.000000}\selectfont \(\displaystyle 0\)}%
\end{pgfscope}%
\begin{pgfscope}%
\pgfsetbuttcap%
\pgfsetroundjoin%
\definecolor{currentfill}{rgb}{0.000000,0.000000,0.000000}%
\pgfsetfillcolor{currentfill}%
\pgfsetlinewidth{0.803000pt}%
\definecolor{currentstroke}{rgb}{0.000000,0.000000,0.000000}%
\pgfsetstrokecolor{currentstroke}%
\pgfsetdash{}{0pt}%
\pgfsys@defobject{currentmarker}{\pgfqpoint{-0.048611in}{0.000000in}}{\pgfqpoint{0.000000in}{0.000000in}}{%
\pgfpathmoveto{\pgfqpoint{0.000000in}{0.000000in}}%
\pgfpathlineto{\pgfqpoint{-0.048611in}{0.000000in}}%
\pgfusepath{stroke,fill}%
}%
\begin{pgfscope}%
\pgfsys@transformshift{1.432000in}{3.466320in}%
\pgfsys@useobject{currentmarker}{}%
\end{pgfscope}%
\end{pgfscope}%
\begin{pgfscope}%
\pgftext[x=1.195888in,y=3.418126in,left,base]{\rmfamily\fontsize{10.000000}{12.000000}\selectfont \(\displaystyle 20\)}%
\end{pgfscope}%
\begin{pgfscope}%
\pgfsetbuttcap%
\pgfsetroundjoin%
\definecolor{currentfill}{rgb}{0.000000,0.000000,0.000000}%
\pgfsetfillcolor{currentfill}%
\pgfsetlinewidth{0.803000pt}%
\definecolor{currentstroke}{rgb}{0.000000,0.000000,0.000000}%
\pgfsetstrokecolor{currentstroke}%
\pgfsetdash{}{0pt}%
\pgfsys@defobject{currentmarker}{\pgfqpoint{-0.048611in}{0.000000in}}{\pgfqpoint{0.000000in}{0.000000in}}{%
\pgfpathmoveto{\pgfqpoint{0.000000in}{0.000000in}}%
\pgfpathlineto{\pgfqpoint{-0.048611in}{0.000000in}}%
\pgfusepath{stroke,fill}%
}%
\begin{pgfscope}%
\pgfsys@transformshift{1.432000in}{2.727120in}%
\pgfsys@useobject{currentmarker}{}%
\end{pgfscope}%
\end{pgfscope}%
\begin{pgfscope}%
\pgftext[x=1.195888in,y=2.678926in,left,base]{\rmfamily\fontsize{10.000000}{12.000000}\selectfont \(\displaystyle 40\)}%
\end{pgfscope}%
\begin{pgfscope}%
\pgfsetbuttcap%
\pgfsetroundjoin%
\definecolor{currentfill}{rgb}{0.000000,0.000000,0.000000}%
\pgfsetfillcolor{currentfill}%
\pgfsetlinewidth{0.803000pt}%
\definecolor{currentstroke}{rgb}{0.000000,0.000000,0.000000}%
\pgfsetstrokecolor{currentstroke}%
\pgfsetdash{}{0pt}%
\pgfsys@defobject{currentmarker}{\pgfqpoint{-0.048611in}{0.000000in}}{\pgfqpoint{0.000000in}{0.000000in}}{%
\pgfpathmoveto{\pgfqpoint{0.000000in}{0.000000in}}%
\pgfpathlineto{\pgfqpoint{-0.048611in}{0.000000in}}%
\pgfusepath{stroke,fill}%
}%
\begin{pgfscope}%
\pgfsys@transformshift{1.432000in}{1.987920in}%
\pgfsys@useobject{currentmarker}{}%
\end{pgfscope}%
\end{pgfscope}%
\begin{pgfscope}%
\pgftext[x=1.195888in,y=1.939726in,left,base]{\rmfamily\fontsize{10.000000}{12.000000}\selectfont \(\displaystyle 60\)}%
\end{pgfscope}%
\begin{pgfscope}%
\pgfsetbuttcap%
\pgfsetroundjoin%
\definecolor{currentfill}{rgb}{0.000000,0.000000,0.000000}%
\pgfsetfillcolor{currentfill}%
\pgfsetlinewidth{0.803000pt}%
\definecolor{currentstroke}{rgb}{0.000000,0.000000,0.000000}%
\pgfsetstrokecolor{currentstroke}%
\pgfsetdash{}{0pt}%
\pgfsys@defobject{currentmarker}{\pgfqpoint{-0.048611in}{0.000000in}}{\pgfqpoint{0.000000in}{0.000000in}}{%
\pgfpathmoveto{\pgfqpoint{0.000000in}{0.000000in}}%
\pgfpathlineto{\pgfqpoint{-0.048611in}{0.000000in}}%
\pgfusepath{stroke,fill}%
}%
\begin{pgfscope}%
\pgfsys@transformshift{1.432000in}{1.248720in}%
\pgfsys@useobject{currentmarker}{}%
\end{pgfscope}%
\end{pgfscope}%
\begin{pgfscope}%
\pgftext[x=1.195888in,y=1.200526in,left,base]{\rmfamily\fontsize{10.000000}{12.000000}\selectfont \(\displaystyle 80\)}%
\end{pgfscope}%
\begin{pgfscope}%
\pgfsetrectcap%
\pgfsetmiterjoin%
\pgfsetlinewidth{0.803000pt}%
\definecolor{currentstroke}{rgb}{0.000000,0.000000,0.000000}%
\pgfsetstrokecolor{currentstroke}%
\pgfsetdash{}{0pt}%
\pgfpathmoveto{\pgfqpoint{1.432000in}{0.528000in}}%
\pgfpathlineto{\pgfqpoint{1.432000in}{4.224000in}}%
\pgfusepath{stroke}%
\end{pgfscope}%
\begin{pgfscope}%
\pgfsetrectcap%
\pgfsetmiterjoin%
\pgfsetlinewidth{0.803000pt}%
\definecolor{currentstroke}{rgb}{0.000000,0.000000,0.000000}%
\pgfsetstrokecolor{currentstroke}%
\pgfsetdash{}{0pt}%
\pgfpathmoveto{\pgfqpoint{5.128000in}{0.528000in}}%
\pgfpathlineto{\pgfqpoint{5.128000in}{4.224000in}}%
\pgfusepath{stroke}%
\end{pgfscope}%
\begin{pgfscope}%
\pgfsetrectcap%
\pgfsetmiterjoin%
\pgfsetlinewidth{0.803000pt}%
\definecolor{currentstroke}{rgb}{0.000000,0.000000,0.000000}%
\pgfsetstrokecolor{currentstroke}%
\pgfsetdash{}{0pt}%
\pgfpathmoveto{\pgfqpoint{1.432000in}{0.528000in}}%
\pgfpathlineto{\pgfqpoint{5.128000in}{0.528000in}}%
\pgfusepath{stroke}%
\end{pgfscope}%
\begin{pgfscope}%
\pgfsetrectcap%
\pgfsetmiterjoin%
\pgfsetlinewidth{0.803000pt}%
\definecolor{currentstroke}{rgb}{0.000000,0.000000,0.000000}%
\pgfsetstrokecolor{currentstroke}%
\pgfsetdash{}{0pt}%
\pgfpathmoveto{\pgfqpoint{1.432000in}{4.224000in}}%
\pgfpathlineto{\pgfqpoint{5.128000in}{4.224000in}}%
\pgfusepath{stroke}%
\end{pgfscope}%
\end{pgfpicture}%
\makeatother%
\endgroup%
} 
\hspace{-0.8cm}
\scalebox{0.25}{%% Creator: Matplotlib, PGF backend
%%
%% To include the figure in your LaTeX document, write
%%   \input{<filename>.pgf}
%%
%% Make sure the required packages are loaded in your preamble
%%   \usepackage{pgf}
%%
%% Figures using additional raster images can only be included by \input if
%% they are in the same directory as the main LaTeX file. For loading figures
%% from other directories you can use the `import` package
%%   \usepackage{import}
%% and then include the figures with
%%   \import{<path to file>}{<filename>.pgf}
%%
%% Matplotlib used the following preamble
%%   \usepackage{fontspec}
%%
\begingroup%
\makeatletter%
\begin{pgfpicture}%
\pgfpathrectangle{\pgfpointorigin}{\pgfqpoint{6.400000in}{4.800000in}}%
\pgfusepath{use as bounding box, clip}%
\begin{pgfscope}%
\pgfsetbuttcap%
\pgfsetmiterjoin%
\definecolor{currentfill}{rgb}{1.000000,1.000000,1.000000}%
\pgfsetfillcolor{currentfill}%
\pgfsetlinewidth{0.000000pt}%
\definecolor{currentstroke}{rgb}{1.000000,1.000000,1.000000}%
\pgfsetstrokecolor{currentstroke}%
\pgfsetdash{}{0pt}%
\pgfpathmoveto{\pgfqpoint{0.000000in}{0.000000in}}%
\pgfpathlineto{\pgfqpoint{6.400000in}{0.000000in}}%
\pgfpathlineto{\pgfqpoint{6.400000in}{4.800000in}}%
\pgfpathlineto{\pgfqpoint{0.000000in}{4.800000in}}%
\pgfpathclose%
\pgfusepath{fill}%
\end{pgfscope}%
\begin{pgfscope}%
\pgfsetbuttcap%
\pgfsetmiterjoin%
\definecolor{currentfill}{rgb}{1.000000,1.000000,1.000000}%
\pgfsetfillcolor{currentfill}%
\pgfsetlinewidth{0.000000pt}%
\definecolor{currentstroke}{rgb}{0.000000,0.000000,0.000000}%
\pgfsetstrokecolor{currentstroke}%
\pgfsetstrokeopacity{0.000000}%
\pgfsetdash{}{0pt}%
\pgfpathmoveto{\pgfqpoint{1.432000in}{0.528000in}}%
\pgfpathlineto{\pgfqpoint{5.128000in}{0.528000in}}%
\pgfpathlineto{\pgfqpoint{5.128000in}{4.224000in}}%
\pgfpathlineto{\pgfqpoint{1.432000in}{4.224000in}}%
\pgfpathclose%
\pgfusepath{fill}%
\end{pgfscope}%
\begin{pgfscope}%
\pgfpathrectangle{\pgfqpoint{1.432000in}{0.528000in}}{\pgfqpoint{3.696000in}{3.696000in}} %
\pgfusepath{clip}%
\pgfsys@transformshift{1.432000in}{0.528000in}%
\pgftext[left,bottom]{\pgfimage[interpolate=true,width=3.700000in,height=3.700000in]{Figure-0004-20180109-013533-357840-img0.png}}%
\end{pgfscope}%
\begin{pgfscope}%
\pgfsetbuttcap%
\pgfsetroundjoin%
\definecolor{currentfill}{rgb}{0.000000,0.000000,0.000000}%
\pgfsetfillcolor{currentfill}%
\pgfsetlinewidth{0.803000pt}%
\definecolor{currentstroke}{rgb}{0.000000,0.000000,0.000000}%
\pgfsetstrokecolor{currentstroke}%
\pgfsetdash{}{0pt}%
\pgfsys@defobject{currentmarker}{\pgfqpoint{0.000000in}{-0.048611in}}{\pgfqpoint{0.000000in}{0.000000in}}{%
\pgfpathmoveto{\pgfqpoint{0.000000in}{0.000000in}}%
\pgfpathlineto{\pgfqpoint{0.000000in}{-0.048611in}}%
\pgfusepath{stroke,fill}%
}%
\begin{pgfscope}%
\pgfsys@transformshift{1.450480in}{0.528000in}%
\pgfsys@useobject{currentmarker}{}%
\end{pgfscope}%
\end{pgfscope}%
\begin{pgfscope}%
\pgftext[x=1.450480in,y=0.430778in,,top]{\rmfamily\fontsize{10.000000}{12.000000}\selectfont \(\displaystyle 0\)}%
\end{pgfscope}%
\begin{pgfscope}%
\pgfsetbuttcap%
\pgfsetroundjoin%
\definecolor{currentfill}{rgb}{0.000000,0.000000,0.000000}%
\pgfsetfillcolor{currentfill}%
\pgfsetlinewidth{0.803000pt}%
\definecolor{currentstroke}{rgb}{0.000000,0.000000,0.000000}%
\pgfsetstrokecolor{currentstroke}%
\pgfsetdash{}{0pt}%
\pgfsys@defobject{currentmarker}{\pgfqpoint{0.000000in}{-0.048611in}}{\pgfqpoint{0.000000in}{0.000000in}}{%
\pgfpathmoveto{\pgfqpoint{0.000000in}{0.000000in}}%
\pgfpathlineto{\pgfqpoint{0.000000in}{-0.048611in}}%
\pgfusepath{stroke,fill}%
}%
\begin{pgfscope}%
\pgfsys@transformshift{2.189680in}{0.528000in}%
\pgfsys@useobject{currentmarker}{}%
\end{pgfscope}%
\end{pgfscope}%
\begin{pgfscope}%
\pgftext[x=2.189680in,y=0.430778in,,top]{\rmfamily\fontsize{10.000000}{12.000000}\selectfont \(\displaystyle 20\)}%
\end{pgfscope}%
\begin{pgfscope}%
\pgfsetbuttcap%
\pgfsetroundjoin%
\definecolor{currentfill}{rgb}{0.000000,0.000000,0.000000}%
\pgfsetfillcolor{currentfill}%
\pgfsetlinewidth{0.803000pt}%
\definecolor{currentstroke}{rgb}{0.000000,0.000000,0.000000}%
\pgfsetstrokecolor{currentstroke}%
\pgfsetdash{}{0pt}%
\pgfsys@defobject{currentmarker}{\pgfqpoint{0.000000in}{-0.048611in}}{\pgfqpoint{0.000000in}{0.000000in}}{%
\pgfpathmoveto{\pgfqpoint{0.000000in}{0.000000in}}%
\pgfpathlineto{\pgfqpoint{0.000000in}{-0.048611in}}%
\pgfusepath{stroke,fill}%
}%
\begin{pgfscope}%
\pgfsys@transformshift{2.928880in}{0.528000in}%
\pgfsys@useobject{currentmarker}{}%
\end{pgfscope}%
\end{pgfscope}%
\begin{pgfscope}%
\pgftext[x=2.928880in,y=0.430778in,,top]{\rmfamily\fontsize{10.000000}{12.000000}\selectfont \(\displaystyle 40\)}%
\end{pgfscope}%
\begin{pgfscope}%
\pgfsetbuttcap%
\pgfsetroundjoin%
\definecolor{currentfill}{rgb}{0.000000,0.000000,0.000000}%
\pgfsetfillcolor{currentfill}%
\pgfsetlinewidth{0.803000pt}%
\definecolor{currentstroke}{rgb}{0.000000,0.000000,0.000000}%
\pgfsetstrokecolor{currentstroke}%
\pgfsetdash{}{0pt}%
\pgfsys@defobject{currentmarker}{\pgfqpoint{0.000000in}{-0.048611in}}{\pgfqpoint{0.000000in}{0.000000in}}{%
\pgfpathmoveto{\pgfqpoint{0.000000in}{0.000000in}}%
\pgfpathlineto{\pgfqpoint{0.000000in}{-0.048611in}}%
\pgfusepath{stroke,fill}%
}%
\begin{pgfscope}%
\pgfsys@transformshift{3.668080in}{0.528000in}%
\pgfsys@useobject{currentmarker}{}%
\end{pgfscope}%
\end{pgfscope}%
\begin{pgfscope}%
\pgftext[x=3.668080in,y=0.430778in,,top]{\rmfamily\fontsize{10.000000}{12.000000}\selectfont \(\displaystyle 60\)}%
\end{pgfscope}%
\begin{pgfscope}%
\pgfsetbuttcap%
\pgfsetroundjoin%
\definecolor{currentfill}{rgb}{0.000000,0.000000,0.000000}%
\pgfsetfillcolor{currentfill}%
\pgfsetlinewidth{0.803000pt}%
\definecolor{currentstroke}{rgb}{0.000000,0.000000,0.000000}%
\pgfsetstrokecolor{currentstroke}%
\pgfsetdash{}{0pt}%
\pgfsys@defobject{currentmarker}{\pgfqpoint{0.000000in}{-0.048611in}}{\pgfqpoint{0.000000in}{0.000000in}}{%
\pgfpathmoveto{\pgfqpoint{0.000000in}{0.000000in}}%
\pgfpathlineto{\pgfqpoint{0.000000in}{-0.048611in}}%
\pgfusepath{stroke,fill}%
}%
\begin{pgfscope}%
\pgfsys@transformshift{4.407280in}{0.528000in}%
\pgfsys@useobject{currentmarker}{}%
\end{pgfscope}%
\end{pgfscope}%
\begin{pgfscope}%
\pgftext[x=4.407280in,y=0.430778in,,top]{\rmfamily\fontsize{10.000000}{12.000000}\selectfont \(\displaystyle 80\)}%
\end{pgfscope}%
\begin{pgfscope}%
\pgfsetbuttcap%
\pgfsetroundjoin%
\definecolor{currentfill}{rgb}{0.000000,0.000000,0.000000}%
\pgfsetfillcolor{currentfill}%
\pgfsetlinewidth{0.803000pt}%
\definecolor{currentstroke}{rgb}{0.000000,0.000000,0.000000}%
\pgfsetstrokecolor{currentstroke}%
\pgfsetdash{}{0pt}%
\pgfsys@defobject{currentmarker}{\pgfqpoint{-0.048611in}{0.000000in}}{\pgfqpoint{0.000000in}{0.000000in}}{%
\pgfpathmoveto{\pgfqpoint{0.000000in}{0.000000in}}%
\pgfpathlineto{\pgfqpoint{-0.048611in}{0.000000in}}%
\pgfusepath{stroke,fill}%
}%
\begin{pgfscope}%
\pgfsys@transformshift{1.432000in}{4.205520in}%
\pgfsys@useobject{currentmarker}{}%
\end{pgfscope}%
\end{pgfscope}%
\begin{pgfscope}%
\pgftext[x=1.265333in,y=4.157326in,left,base]{\rmfamily\fontsize{10.000000}{12.000000}\selectfont \(\displaystyle 0\)}%
\end{pgfscope}%
\begin{pgfscope}%
\pgfsetbuttcap%
\pgfsetroundjoin%
\definecolor{currentfill}{rgb}{0.000000,0.000000,0.000000}%
\pgfsetfillcolor{currentfill}%
\pgfsetlinewidth{0.803000pt}%
\definecolor{currentstroke}{rgb}{0.000000,0.000000,0.000000}%
\pgfsetstrokecolor{currentstroke}%
\pgfsetdash{}{0pt}%
\pgfsys@defobject{currentmarker}{\pgfqpoint{-0.048611in}{0.000000in}}{\pgfqpoint{0.000000in}{0.000000in}}{%
\pgfpathmoveto{\pgfqpoint{0.000000in}{0.000000in}}%
\pgfpathlineto{\pgfqpoint{-0.048611in}{0.000000in}}%
\pgfusepath{stroke,fill}%
}%
\begin{pgfscope}%
\pgfsys@transformshift{1.432000in}{3.466320in}%
\pgfsys@useobject{currentmarker}{}%
\end{pgfscope}%
\end{pgfscope}%
\begin{pgfscope}%
\pgftext[x=1.195888in,y=3.418126in,left,base]{\rmfamily\fontsize{10.000000}{12.000000}\selectfont \(\displaystyle 20\)}%
\end{pgfscope}%
\begin{pgfscope}%
\pgfsetbuttcap%
\pgfsetroundjoin%
\definecolor{currentfill}{rgb}{0.000000,0.000000,0.000000}%
\pgfsetfillcolor{currentfill}%
\pgfsetlinewidth{0.803000pt}%
\definecolor{currentstroke}{rgb}{0.000000,0.000000,0.000000}%
\pgfsetstrokecolor{currentstroke}%
\pgfsetdash{}{0pt}%
\pgfsys@defobject{currentmarker}{\pgfqpoint{-0.048611in}{0.000000in}}{\pgfqpoint{0.000000in}{0.000000in}}{%
\pgfpathmoveto{\pgfqpoint{0.000000in}{0.000000in}}%
\pgfpathlineto{\pgfqpoint{-0.048611in}{0.000000in}}%
\pgfusepath{stroke,fill}%
}%
\begin{pgfscope}%
\pgfsys@transformshift{1.432000in}{2.727120in}%
\pgfsys@useobject{currentmarker}{}%
\end{pgfscope}%
\end{pgfscope}%
\begin{pgfscope}%
\pgftext[x=1.195888in,y=2.678926in,left,base]{\rmfamily\fontsize{10.000000}{12.000000}\selectfont \(\displaystyle 40\)}%
\end{pgfscope}%
\begin{pgfscope}%
\pgfsetbuttcap%
\pgfsetroundjoin%
\definecolor{currentfill}{rgb}{0.000000,0.000000,0.000000}%
\pgfsetfillcolor{currentfill}%
\pgfsetlinewidth{0.803000pt}%
\definecolor{currentstroke}{rgb}{0.000000,0.000000,0.000000}%
\pgfsetstrokecolor{currentstroke}%
\pgfsetdash{}{0pt}%
\pgfsys@defobject{currentmarker}{\pgfqpoint{-0.048611in}{0.000000in}}{\pgfqpoint{0.000000in}{0.000000in}}{%
\pgfpathmoveto{\pgfqpoint{0.000000in}{0.000000in}}%
\pgfpathlineto{\pgfqpoint{-0.048611in}{0.000000in}}%
\pgfusepath{stroke,fill}%
}%
\begin{pgfscope}%
\pgfsys@transformshift{1.432000in}{1.987920in}%
\pgfsys@useobject{currentmarker}{}%
\end{pgfscope}%
\end{pgfscope}%
\begin{pgfscope}%
\pgftext[x=1.195888in,y=1.939726in,left,base]{\rmfamily\fontsize{10.000000}{12.000000}\selectfont \(\displaystyle 60\)}%
\end{pgfscope}%
\begin{pgfscope}%
\pgfsetbuttcap%
\pgfsetroundjoin%
\definecolor{currentfill}{rgb}{0.000000,0.000000,0.000000}%
\pgfsetfillcolor{currentfill}%
\pgfsetlinewidth{0.803000pt}%
\definecolor{currentstroke}{rgb}{0.000000,0.000000,0.000000}%
\pgfsetstrokecolor{currentstroke}%
\pgfsetdash{}{0pt}%
\pgfsys@defobject{currentmarker}{\pgfqpoint{-0.048611in}{0.000000in}}{\pgfqpoint{0.000000in}{0.000000in}}{%
\pgfpathmoveto{\pgfqpoint{0.000000in}{0.000000in}}%
\pgfpathlineto{\pgfqpoint{-0.048611in}{0.000000in}}%
\pgfusepath{stroke,fill}%
}%
\begin{pgfscope}%
\pgfsys@transformshift{1.432000in}{1.248720in}%
\pgfsys@useobject{currentmarker}{}%
\end{pgfscope}%
\end{pgfscope}%
\begin{pgfscope}%
\pgftext[x=1.195888in,y=1.200526in,left,base]{\rmfamily\fontsize{10.000000}{12.000000}\selectfont \(\displaystyle 80\)}%
\end{pgfscope}%
\begin{pgfscope}%
\pgfsetrectcap%
\pgfsetmiterjoin%
\pgfsetlinewidth{0.803000pt}%
\definecolor{currentstroke}{rgb}{0.000000,0.000000,0.000000}%
\pgfsetstrokecolor{currentstroke}%
\pgfsetdash{}{0pt}%
\pgfpathmoveto{\pgfqpoint{1.432000in}{0.528000in}}%
\pgfpathlineto{\pgfqpoint{1.432000in}{4.224000in}}%
\pgfusepath{stroke}%
\end{pgfscope}%
\begin{pgfscope}%
\pgfsetrectcap%
\pgfsetmiterjoin%
\pgfsetlinewidth{0.803000pt}%
\definecolor{currentstroke}{rgb}{0.000000,0.000000,0.000000}%
\pgfsetstrokecolor{currentstroke}%
\pgfsetdash{}{0pt}%
\pgfpathmoveto{\pgfqpoint{5.128000in}{0.528000in}}%
\pgfpathlineto{\pgfqpoint{5.128000in}{4.224000in}}%
\pgfusepath{stroke}%
\end{pgfscope}%
\begin{pgfscope}%
\pgfsetrectcap%
\pgfsetmiterjoin%
\pgfsetlinewidth{0.803000pt}%
\definecolor{currentstroke}{rgb}{0.000000,0.000000,0.000000}%
\pgfsetstrokecolor{currentstroke}%
\pgfsetdash{}{0pt}%
\pgfpathmoveto{\pgfqpoint{1.432000in}{0.528000in}}%
\pgfpathlineto{\pgfqpoint{5.128000in}{0.528000in}}%
\pgfusepath{stroke}%
\end{pgfscope}%
\begin{pgfscope}%
\pgfsetrectcap%
\pgfsetmiterjoin%
\pgfsetlinewidth{0.803000pt}%
\definecolor{currentstroke}{rgb}{0.000000,0.000000,0.000000}%
\pgfsetstrokecolor{currentstroke}%
\pgfsetdash{}{0pt}%
\pgfpathmoveto{\pgfqpoint{1.432000in}{4.224000in}}%
\pgfpathlineto{\pgfqpoint{5.128000in}{4.224000in}}%
\pgfusepath{stroke}%
\end{pgfscope}%
\end{pgfpicture}%
\makeatother%
\endgroup%
} 
\scalebox{0.25}{%% Creator: Matplotlib, PGF backend
%%
%% To include the figure in your LaTeX document, write
%%   \input{<filename>.pgf}
%%
%% Make sure the required packages are loaded in your preamble
%%   \usepackage{pgf}
%%
%% Figures using additional raster images can only be included by \input if
%% they are in the same directory as the main LaTeX file. For loading figures
%% from other directories you can use the `import` package
%%   \usepackage{import}
%% and then include the figures with
%%   \import{<path to file>}{<filename>.pgf}
%%
%% Matplotlib used the following preamble
%%   \usepackage{fontspec}
%%
\begingroup%
\makeatletter%
\begin{pgfpicture}%
\pgfpathrectangle{\pgfpointorigin}{\pgfqpoint{6.400000in}{4.800000in}}%
\pgfusepath{use as bounding box, clip}%
\begin{pgfscope}%
\pgfsetbuttcap%
\pgfsetmiterjoin%
\definecolor{currentfill}{rgb}{1.000000,1.000000,1.000000}%
\pgfsetfillcolor{currentfill}%
\pgfsetlinewidth{0.000000pt}%
\definecolor{currentstroke}{rgb}{1.000000,1.000000,1.000000}%
\pgfsetstrokecolor{currentstroke}%
\pgfsetdash{}{0pt}%
\pgfpathmoveto{\pgfqpoint{0.000000in}{0.000000in}}%
\pgfpathlineto{\pgfqpoint{6.400000in}{0.000000in}}%
\pgfpathlineto{\pgfqpoint{6.400000in}{4.800000in}}%
\pgfpathlineto{\pgfqpoint{0.000000in}{4.800000in}}%
\pgfpathclose%
\pgfusepath{fill}%
\end{pgfscope}%
\begin{pgfscope}%
\pgfsetbuttcap%
\pgfsetmiterjoin%
\definecolor{currentfill}{rgb}{1.000000,1.000000,1.000000}%
\pgfsetfillcolor{currentfill}%
\pgfsetlinewidth{0.000000pt}%
\definecolor{currentstroke}{rgb}{0.000000,0.000000,0.000000}%
\pgfsetstrokecolor{currentstroke}%
\pgfsetstrokeopacity{0.000000}%
\pgfsetdash{}{0pt}%
\pgfpathmoveto{\pgfqpoint{1.432000in}{0.528000in}}%
\pgfpathlineto{\pgfqpoint{5.128000in}{0.528000in}}%
\pgfpathlineto{\pgfqpoint{5.128000in}{4.224000in}}%
\pgfpathlineto{\pgfqpoint{1.432000in}{4.224000in}}%
\pgfpathclose%
\pgfusepath{fill}%
\end{pgfscope}%
\begin{pgfscope}%
\pgfpathrectangle{\pgfqpoint{1.432000in}{0.528000in}}{\pgfqpoint{3.696000in}{3.696000in}} %
\pgfusepath{clip}%
\pgfsys@transformshift{1.432000in}{0.528000in}%
\pgftext[left,bottom]{\pgfimage[interpolate=true,width=3.700000in,height=3.700000in]{Figure-0005-20180109-013534-717418-img0.png}}%
\end{pgfscope}%
\begin{pgfscope}%
\pgfsetbuttcap%
\pgfsetroundjoin%
\definecolor{currentfill}{rgb}{0.000000,0.000000,0.000000}%
\pgfsetfillcolor{currentfill}%
\pgfsetlinewidth{0.803000pt}%
\definecolor{currentstroke}{rgb}{0.000000,0.000000,0.000000}%
\pgfsetstrokecolor{currentstroke}%
\pgfsetdash{}{0pt}%
\pgfsys@defobject{currentmarker}{\pgfqpoint{0.000000in}{-0.048611in}}{\pgfqpoint{0.000000in}{0.000000in}}{%
\pgfpathmoveto{\pgfqpoint{0.000000in}{0.000000in}}%
\pgfpathlineto{\pgfqpoint{0.000000in}{-0.048611in}}%
\pgfusepath{stroke,fill}%
}%
\begin{pgfscope}%
\pgfsys@transformshift{1.450480in}{0.528000in}%
\pgfsys@useobject{currentmarker}{}%
\end{pgfscope}%
\end{pgfscope}%
\begin{pgfscope}%
\pgftext[x=1.450480in,y=0.430778in,,top]{\rmfamily\fontsize{10.000000}{12.000000}\selectfont \(\displaystyle 0\)}%
\end{pgfscope}%
\begin{pgfscope}%
\pgfsetbuttcap%
\pgfsetroundjoin%
\definecolor{currentfill}{rgb}{0.000000,0.000000,0.000000}%
\pgfsetfillcolor{currentfill}%
\pgfsetlinewidth{0.803000pt}%
\definecolor{currentstroke}{rgb}{0.000000,0.000000,0.000000}%
\pgfsetstrokecolor{currentstroke}%
\pgfsetdash{}{0pt}%
\pgfsys@defobject{currentmarker}{\pgfqpoint{0.000000in}{-0.048611in}}{\pgfqpoint{0.000000in}{0.000000in}}{%
\pgfpathmoveto{\pgfqpoint{0.000000in}{0.000000in}}%
\pgfpathlineto{\pgfqpoint{0.000000in}{-0.048611in}}%
\pgfusepath{stroke,fill}%
}%
\begin{pgfscope}%
\pgfsys@transformshift{2.189680in}{0.528000in}%
\pgfsys@useobject{currentmarker}{}%
\end{pgfscope}%
\end{pgfscope}%
\begin{pgfscope}%
\pgftext[x=2.189680in,y=0.430778in,,top]{\rmfamily\fontsize{10.000000}{12.000000}\selectfont \(\displaystyle 20\)}%
\end{pgfscope}%
\begin{pgfscope}%
\pgfsetbuttcap%
\pgfsetroundjoin%
\definecolor{currentfill}{rgb}{0.000000,0.000000,0.000000}%
\pgfsetfillcolor{currentfill}%
\pgfsetlinewidth{0.803000pt}%
\definecolor{currentstroke}{rgb}{0.000000,0.000000,0.000000}%
\pgfsetstrokecolor{currentstroke}%
\pgfsetdash{}{0pt}%
\pgfsys@defobject{currentmarker}{\pgfqpoint{0.000000in}{-0.048611in}}{\pgfqpoint{0.000000in}{0.000000in}}{%
\pgfpathmoveto{\pgfqpoint{0.000000in}{0.000000in}}%
\pgfpathlineto{\pgfqpoint{0.000000in}{-0.048611in}}%
\pgfusepath{stroke,fill}%
}%
\begin{pgfscope}%
\pgfsys@transformshift{2.928880in}{0.528000in}%
\pgfsys@useobject{currentmarker}{}%
\end{pgfscope}%
\end{pgfscope}%
\begin{pgfscope}%
\pgftext[x=2.928880in,y=0.430778in,,top]{\rmfamily\fontsize{10.000000}{12.000000}\selectfont \(\displaystyle 40\)}%
\end{pgfscope}%
\begin{pgfscope}%
\pgfsetbuttcap%
\pgfsetroundjoin%
\definecolor{currentfill}{rgb}{0.000000,0.000000,0.000000}%
\pgfsetfillcolor{currentfill}%
\pgfsetlinewidth{0.803000pt}%
\definecolor{currentstroke}{rgb}{0.000000,0.000000,0.000000}%
\pgfsetstrokecolor{currentstroke}%
\pgfsetdash{}{0pt}%
\pgfsys@defobject{currentmarker}{\pgfqpoint{0.000000in}{-0.048611in}}{\pgfqpoint{0.000000in}{0.000000in}}{%
\pgfpathmoveto{\pgfqpoint{0.000000in}{0.000000in}}%
\pgfpathlineto{\pgfqpoint{0.000000in}{-0.048611in}}%
\pgfusepath{stroke,fill}%
}%
\begin{pgfscope}%
\pgfsys@transformshift{3.668080in}{0.528000in}%
\pgfsys@useobject{currentmarker}{}%
\end{pgfscope}%
\end{pgfscope}%
\begin{pgfscope}%
\pgftext[x=3.668080in,y=0.430778in,,top]{\rmfamily\fontsize{10.000000}{12.000000}\selectfont \(\displaystyle 60\)}%
\end{pgfscope}%
\begin{pgfscope}%
\pgfsetbuttcap%
\pgfsetroundjoin%
\definecolor{currentfill}{rgb}{0.000000,0.000000,0.000000}%
\pgfsetfillcolor{currentfill}%
\pgfsetlinewidth{0.803000pt}%
\definecolor{currentstroke}{rgb}{0.000000,0.000000,0.000000}%
\pgfsetstrokecolor{currentstroke}%
\pgfsetdash{}{0pt}%
\pgfsys@defobject{currentmarker}{\pgfqpoint{0.000000in}{-0.048611in}}{\pgfqpoint{0.000000in}{0.000000in}}{%
\pgfpathmoveto{\pgfqpoint{0.000000in}{0.000000in}}%
\pgfpathlineto{\pgfqpoint{0.000000in}{-0.048611in}}%
\pgfusepath{stroke,fill}%
}%
\begin{pgfscope}%
\pgfsys@transformshift{4.407280in}{0.528000in}%
\pgfsys@useobject{currentmarker}{}%
\end{pgfscope}%
\end{pgfscope}%
\begin{pgfscope}%
\pgftext[x=4.407280in,y=0.430778in,,top]{\rmfamily\fontsize{10.000000}{12.000000}\selectfont \(\displaystyle 80\)}%
\end{pgfscope}%
\begin{pgfscope}%
\pgfsetbuttcap%
\pgfsetroundjoin%
\definecolor{currentfill}{rgb}{0.000000,0.000000,0.000000}%
\pgfsetfillcolor{currentfill}%
\pgfsetlinewidth{0.803000pt}%
\definecolor{currentstroke}{rgb}{0.000000,0.000000,0.000000}%
\pgfsetstrokecolor{currentstroke}%
\pgfsetdash{}{0pt}%
\pgfsys@defobject{currentmarker}{\pgfqpoint{-0.048611in}{0.000000in}}{\pgfqpoint{0.000000in}{0.000000in}}{%
\pgfpathmoveto{\pgfqpoint{0.000000in}{0.000000in}}%
\pgfpathlineto{\pgfqpoint{-0.048611in}{0.000000in}}%
\pgfusepath{stroke,fill}%
}%
\begin{pgfscope}%
\pgfsys@transformshift{1.432000in}{4.205520in}%
\pgfsys@useobject{currentmarker}{}%
\end{pgfscope}%
\end{pgfscope}%
\begin{pgfscope}%
\pgftext[x=1.265333in,y=4.157326in,left,base]{\rmfamily\fontsize{10.000000}{12.000000}\selectfont \(\displaystyle 0\)}%
\end{pgfscope}%
\begin{pgfscope}%
\pgfsetbuttcap%
\pgfsetroundjoin%
\definecolor{currentfill}{rgb}{0.000000,0.000000,0.000000}%
\pgfsetfillcolor{currentfill}%
\pgfsetlinewidth{0.803000pt}%
\definecolor{currentstroke}{rgb}{0.000000,0.000000,0.000000}%
\pgfsetstrokecolor{currentstroke}%
\pgfsetdash{}{0pt}%
\pgfsys@defobject{currentmarker}{\pgfqpoint{-0.048611in}{0.000000in}}{\pgfqpoint{0.000000in}{0.000000in}}{%
\pgfpathmoveto{\pgfqpoint{0.000000in}{0.000000in}}%
\pgfpathlineto{\pgfqpoint{-0.048611in}{0.000000in}}%
\pgfusepath{stroke,fill}%
}%
\begin{pgfscope}%
\pgfsys@transformshift{1.432000in}{3.466320in}%
\pgfsys@useobject{currentmarker}{}%
\end{pgfscope}%
\end{pgfscope}%
\begin{pgfscope}%
\pgftext[x=1.195888in,y=3.418126in,left,base]{\rmfamily\fontsize{10.000000}{12.000000}\selectfont \(\displaystyle 20\)}%
\end{pgfscope}%
\begin{pgfscope}%
\pgfsetbuttcap%
\pgfsetroundjoin%
\definecolor{currentfill}{rgb}{0.000000,0.000000,0.000000}%
\pgfsetfillcolor{currentfill}%
\pgfsetlinewidth{0.803000pt}%
\definecolor{currentstroke}{rgb}{0.000000,0.000000,0.000000}%
\pgfsetstrokecolor{currentstroke}%
\pgfsetdash{}{0pt}%
\pgfsys@defobject{currentmarker}{\pgfqpoint{-0.048611in}{0.000000in}}{\pgfqpoint{0.000000in}{0.000000in}}{%
\pgfpathmoveto{\pgfqpoint{0.000000in}{0.000000in}}%
\pgfpathlineto{\pgfqpoint{-0.048611in}{0.000000in}}%
\pgfusepath{stroke,fill}%
}%
\begin{pgfscope}%
\pgfsys@transformshift{1.432000in}{2.727120in}%
\pgfsys@useobject{currentmarker}{}%
\end{pgfscope}%
\end{pgfscope}%
\begin{pgfscope}%
\pgftext[x=1.195888in,y=2.678926in,left,base]{\rmfamily\fontsize{10.000000}{12.000000}\selectfont \(\displaystyle 40\)}%
\end{pgfscope}%
\begin{pgfscope}%
\pgfsetbuttcap%
\pgfsetroundjoin%
\definecolor{currentfill}{rgb}{0.000000,0.000000,0.000000}%
\pgfsetfillcolor{currentfill}%
\pgfsetlinewidth{0.803000pt}%
\definecolor{currentstroke}{rgb}{0.000000,0.000000,0.000000}%
\pgfsetstrokecolor{currentstroke}%
\pgfsetdash{}{0pt}%
\pgfsys@defobject{currentmarker}{\pgfqpoint{-0.048611in}{0.000000in}}{\pgfqpoint{0.000000in}{0.000000in}}{%
\pgfpathmoveto{\pgfqpoint{0.000000in}{0.000000in}}%
\pgfpathlineto{\pgfqpoint{-0.048611in}{0.000000in}}%
\pgfusepath{stroke,fill}%
}%
\begin{pgfscope}%
\pgfsys@transformshift{1.432000in}{1.987920in}%
\pgfsys@useobject{currentmarker}{}%
\end{pgfscope}%
\end{pgfscope}%
\begin{pgfscope}%
\pgftext[x=1.195888in,y=1.939726in,left,base]{\rmfamily\fontsize{10.000000}{12.000000}\selectfont \(\displaystyle 60\)}%
\end{pgfscope}%
\begin{pgfscope}%
\pgfsetbuttcap%
\pgfsetroundjoin%
\definecolor{currentfill}{rgb}{0.000000,0.000000,0.000000}%
\pgfsetfillcolor{currentfill}%
\pgfsetlinewidth{0.803000pt}%
\definecolor{currentstroke}{rgb}{0.000000,0.000000,0.000000}%
\pgfsetstrokecolor{currentstroke}%
\pgfsetdash{}{0pt}%
\pgfsys@defobject{currentmarker}{\pgfqpoint{-0.048611in}{0.000000in}}{\pgfqpoint{0.000000in}{0.000000in}}{%
\pgfpathmoveto{\pgfqpoint{0.000000in}{0.000000in}}%
\pgfpathlineto{\pgfqpoint{-0.048611in}{0.000000in}}%
\pgfusepath{stroke,fill}%
}%
\begin{pgfscope}%
\pgfsys@transformshift{1.432000in}{1.248720in}%
\pgfsys@useobject{currentmarker}{}%
\end{pgfscope}%
\end{pgfscope}%
\begin{pgfscope}%
\pgftext[x=1.195888in,y=1.200526in,left,base]{\rmfamily\fontsize{10.000000}{12.000000}\selectfont \(\displaystyle 80\)}%
\end{pgfscope}%
\begin{pgfscope}%
\pgfsetrectcap%
\pgfsetmiterjoin%
\pgfsetlinewidth{0.803000pt}%
\definecolor{currentstroke}{rgb}{0.000000,0.000000,0.000000}%
\pgfsetstrokecolor{currentstroke}%
\pgfsetdash{}{0pt}%
\pgfpathmoveto{\pgfqpoint{1.432000in}{0.528000in}}%
\pgfpathlineto{\pgfqpoint{1.432000in}{4.224000in}}%
\pgfusepath{stroke}%
\end{pgfscope}%
\begin{pgfscope}%
\pgfsetrectcap%
\pgfsetmiterjoin%
\pgfsetlinewidth{0.803000pt}%
\definecolor{currentstroke}{rgb}{0.000000,0.000000,0.000000}%
\pgfsetstrokecolor{currentstroke}%
\pgfsetdash{}{0pt}%
\pgfpathmoveto{\pgfqpoint{5.128000in}{0.528000in}}%
\pgfpathlineto{\pgfqpoint{5.128000in}{4.224000in}}%
\pgfusepath{stroke}%
\end{pgfscope}%
\begin{pgfscope}%
\pgfsetrectcap%
\pgfsetmiterjoin%
\pgfsetlinewidth{0.803000pt}%
\definecolor{currentstroke}{rgb}{0.000000,0.000000,0.000000}%
\pgfsetstrokecolor{currentstroke}%
\pgfsetdash{}{0pt}%
\pgfpathmoveto{\pgfqpoint{1.432000in}{0.528000in}}%
\pgfpathlineto{\pgfqpoint{5.128000in}{0.528000in}}%
\pgfusepath{stroke}%
\end{pgfscope}%
\begin{pgfscope}%
\pgfsetrectcap%
\pgfsetmiterjoin%
\pgfsetlinewidth{0.803000pt}%
\definecolor{currentstroke}{rgb}{0.000000,0.000000,0.000000}%
\pgfsetstrokecolor{currentstroke}%
\pgfsetdash{}{0pt}%
\pgfpathmoveto{\pgfqpoint{1.432000in}{4.224000in}}%
\pgfpathlineto{\pgfqpoint{5.128000in}{4.224000in}}%
\pgfusepath{stroke}%
\end{pgfscope}%
\end{pgfpicture}%
\makeatother%
\endgroup%
} 
\hspace{-0.8cm}
\scalebox{0.25}{%% Creator: Matplotlib, PGF backend
%%
%% To include the figure in your LaTeX document, write
%%   \input{<filename>.pgf}
%%
%% Make sure the required packages are loaded in your preamble
%%   \usepackage{pgf}
%%
%% Figures using additional raster images can only be included by \input if
%% they are in the same directory as the main LaTeX file. For loading figures
%% from other directories you can use the `import` package
%%   \usepackage{import}
%% and then include the figures with
%%   \import{<path to file>}{<filename>.pgf}
%%
%% Matplotlib used the following preamble
%%   \usepackage{fontspec}
%%
\begingroup%
\makeatletter%
\begin{pgfpicture}%
\pgfpathrectangle{\pgfpointorigin}{\pgfqpoint{6.400000in}{4.800000in}}%
\pgfusepath{use as bounding box, clip}%
\begin{pgfscope}%
\pgfsetbuttcap%
\pgfsetmiterjoin%
\definecolor{currentfill}{rgb}{1.000000,1.000000,1.000000}%
\pgfsetfillcolor{currentfill}%
\pgfsetlinewidth{0.000000pt}%
\definecolor{currentstroke}{rgb}{1.000000,1.000000,1.000000}%
\pgfsetstrokecolor{currentstroke}%
\pgfsetdash{}{0pt}%
\pgfpathmoveto{\pgfqpoint{0.000000in}{0.000000in}}%
\pgfpathlineto{\pgfqpoint{6.400000in}{0.000000in}}%
\pgfpathlineto{\pgfqpoint{6.400000in}{4.800000in}}%
\pgfpathlineto{\pgfqpoint{0.000000in}{4.800000in}}%
\pgfpathclose%
\pgfusepath{fill}%
\end{pgfscope}%
\begin{pgfscope}%
\pgfsetbuttcap%
\pgfsetmiterjoin%
\definecolor{currentfill}{rgb}{1.000000,1.000000,1.000000}%
\pgfsetfillcolor{currentfill}%
\pgfsetlinewidth{0.000000pt}%
\definecolor{currentstroke}{rgb}{0.000000,0.000000,0.000000}%
\pgfsetstrokecolor{currentstroke}%
\pgfsetstrokeopacity{0.000000}%
\pgfsetdash{}{0pt}%
\pgfpathmoveto{\pgfqpoint{1.432000in}{0.528000in}}%
\pgfpathlineto{\pgfqpoint{5.128000in}{0.528000in}}%
\pgfpathlineto{\pgfqpoint{5.128000in}{4.224000in}}%
\pgfpathlineto{\pgfqpoint{1.432000in}{4.224000in}}%
\pgfpathclose%
\pgfusepath{fill}%
\end{pgfscope}%
\begin{pgfscope}%
\pgfpathrectangle{\pgfqpoint{1.432000in}{0.528000in}}{\pgfqpoint{3.696000in}{3.696000in}} %
\pgfusepath{clip}%
\pgfsys@transformshift{1.432000in}{0.528000in}%
\pgftext[left,bottom]{\pgfimage[interpolate=true,width=3.700000in,height=3.700000in]{Figure-0006-20180109-013536-070054-img0.png}}%
\end{pgfscope}%
\begin{pgfscope}%
\pgfsetbuttcap%
\pgfsetroundjoin%
\definecolor{currentfill}{rgb}{0.000000,0.000000,0.000000}%
\pgfsetfillcolor{currentfill}%
\pgfsetlinewidth{0.803000pt}%
\definecolor{currentstroke}{rgb}{0.000000,0.000000,0.000000}%
\pgfsetstrokecolor{currentstroke}%
\pgfsetdash{}{0pt}%
\pgfsys@defobject{currentmarker}{\pgfqpoint{0.000000in}{-0.048611in}}{\pgfqpoint{0.000000in}{0.000000in}}{%
\pgfpathmoveto{\pgfqpoint{0.000000in}{0.000000in}}%
\pgfpathlineto{\pgfqpoint{0.000000in}{-0.048611in}}%
\pgfusepath{stroke,fill}%
}%
\begin{pgfscope}%
\pgfsys@transformshift{1.450480in}{0.528000in}%
\pgfsys@useobject{currentmarker}{}%
\end{pgfscope}%
\end{pgfscope}%
\begin{pgfscope}%
\pgftext[x=1.450480in,y=0.430778in,,top]{\rmfamily\fontsize{10.000000}{12.000000}\selectfont \(\displaystyle 0\)}%
\end{pgfscope}%
\begin{pgfscope}%
\pgfsetbuttcap%
\pgfsetroundjoin%
\definecolor{currentfill}{rgb}{0.000000,0.000000,0.000000}%
\pgfsetfillcolor{currentfill}%
\pgfsetlinewidth{0.803000pt}%
\definecolor{currentstroke}{rgb}{0.000000,0.000000,0.000000}%
\pgfsetstrokecolor{currentstroke}%
\pgfsetdash{}{0pt}%
\pgfsys@defobject{currentmarker}{\pgfqpoint{0.000000in}{-0.048611in}}{\pgfqpoint{0.000000in}{0.000000in}}{%
\pgfpathmoveto{\pgfqpoint{0.000000in}{0.000000in}}%
\pgfpathlineto{\pgfqpoint{0.000000in}{-0.048611in}}%
\pgfusepath{stroke,fill}%
}%
\begin{pgfscope}%
\pgfsys@transformshift{2.189680in}{0.528000in}%
\pgfsys@useobject{currentmarker}{}%
\end{pgfscope}%
\end{pgfscope}%
\begin{pgfscope}%
\pgftext[x=2.189680in,y=0.430778in,,top]{\rmfamily\fontsize{10.000000}{12.000000}\selectfont \(\displaystyle 20\)}%
\end{pgfscope}%
\begin{pgfscope}%
\pgfsetbuttcap%
\pgfsetroundjoin%
\definecolor{currentfill}{rgb}{0.000000,0.000000,0.000000}%
\pgfsetfillcolor{currentfill}%
\pgfsetlinewidth{0.803000pt}%
\definecolor{currentstroke}{rgb}{0.000000,0.000000,0.000000}%
\pgfsetstrokecolor{currentstroke}%
\pgfsetdash{}{0pt}%
\pgfsys@defobject{currentmarker}{\pgfqpoint{0.000000in}{-0.048611in}}{\pgfqpoint{0.000000in}{0.000000in}}{%
\pgfpathmoveto{\pgfqpoint{0.000000in}{0.000000in}}%
\pgfpathlineto{\pgfqpoint{0.000000in}{-0.048611in}}%
\pgfusepath{stroke,fill}%
}%
\begin{pgfscope}%
\pgfsys@transformshift{2.928880in}{0.528000in}%
\pgfsys@useobject{currentmarker}{}%
\end{pgfscope}%
\end{pgfscope}%
\begin{pgfscope}%
\pgftext[x=2.928880in,y=0.430778in,,top]{\rmfamily\fontsize{10.000000}{12.000000}\selectfont \(\displaystyle 40\)}%
\end{pgfscope}%
\begin{pgfscope}%
\pgfsetbuttcap%
\pgfsetroundjoin%
\definecolor{currentfill}{rgb}{0.000000,0.000000,0.000000}%
\pgfsetfillcolor{currentfill}%
\pgfsetlinewidth{0.803000pt}%
\definecolor{currentstroke}{rgb}{0.000000,0.000000,0.000000}%
\pgfsetstrokecolor{currentstroke}%
\pgfsetdash{}{0pt}%
\pgfsys@defobject{currentmarker}{\pgfqpoint{0.000000in}{-0.048611in}}{\pgfqpoint{0.000000in}{0.000000in}}{%
\pgfpathmoveto{\pgfqpoint{0.000000in}{0.000000in}}%
\pgfpathlineto{\pgfqpoint{0.000000in}{-0.048611in}}%
\pgfusepath{stroke,fill}%
}%
\begin{pgfscope}%
\pgfsys@transformshift{3.668080in}{0.528000in}%
\pgfsys@useobject{currentmarker}{}%
\end{pgfscope}%
\end{pgfscope}%
\begin{pgfscope}%
\pgftext[x=3.668080in,y=0.430778in,,top]{\rmfamily\fontsize{10.000000}{12.000000}\selectfont \(\displaystyle 60\)}%
\end{pgfscope}%
\begin{pgfscope}%
\pgfsetbuttcap%
\pgfsetroundjoin%
\definecolor{currentfill}{rgb}{0.000000,0.000000,0.000000}%
\pgfsetfillcolor{currentfill}%
\pgfsetlinewidth{0.803000pt}%
\definecolor{currentstroke}{rgb}{0.000000,0.000000,0.000000}%
\pgfsetstrokecolor{currentstroke}%
\pgfsetdash{}{0pt}%
\pgfsys@defobject{currentmarker}{\pgfqpoint{0.000000in}{-0.048611in}}{\pgfqpoint{0.000000in}{0.000000in}}{%
\pgfpathmoveto{\pgfqpoint{0.000000in}{0.000000in}}%
\pgfpathlineto{\pgfqpoint{0.000000in}{-0.048611in}}%
\pgfusepath{stroke,fill}%
}%
\begin{pgfscope}%
\pgfsys@transformshift{4.407280in}{0.528000in}%
\pgfsys@useobject{currentmarker}{}%
\end{pgfscope}%
\end{pgfscope}%
\begin{pgfscope}%
\pgftext[x=4.407280in,y=0.430778in,,top]{\rmfamily\fontsize{10.000000}{12.000000}\selectfont \(\displaystyle 80\)}%
\end{pgfscope}%
\begin{pgfscope}%
\pgfsetbuttcap%
\pgfsetroundjoin%
\definecolor{currentfill}{rgb}{0.000000,0.000000,0.000000}%
\pgfsetfillcolor{currentfill}%
\pgfsetlinewidth{0.803000pt}%
\definecolor{currentstroke}{rgb}{0.000000,0.000000,0.000000}%
\pgfsetstrokecolor{currentstroke}%
\pgfsetdash{}{0pt}%
\pgfsys@defobject{currentmarker}{\pgfqpoint{-0.048611in}{0.000000in}}{\pgfqpoint{0.000000in}{0.000000in}}{%
\pgfpathmoveto{\pgfqpoint{0.000000in}{0.000000in}}%
\pgfpathlineto{\pgfqpoint{-0.048611in}{0.000000in}}%
\pgfusepath{stroke,fill}%
}%
\begin{pgfscope}%
\pgfsys@transformshift{1.432000in}{4.205520in}%
\pgfsys@useobject{currentmarker}{}%
\end{pgfscope}%
\end{pgfscope}%
\begin{pgfscope}%
\pgftext[x=1.265333in,y=4.157326in,left,base]{\rmfamily\fontsize{10.000000}{12.000000}\selectfont \(\displaystyle 0\)}%
\end{pgfscope}%
\begin{pgfscope}%
\pgfsetbuttcap%
\pgfsetroundjoin%
\definecolor{currentfill}{rgb}{0.000000,0.000000,0.000000}%
\pgfsetfillcolor{currentfill}%
\pgfsetlinewidth{0.803000pt}%
\definecolor{currentstroke}{rgb}{0.000000,0.000000,0.000000}%
\pgfsetstrokecolor{currentstroke}%
\pgfsetdash{}{0pt}%
\pgfsys@defobject{currentmarker}{\pgfqpoint{-0.048611in}{0.000000in}}{\pgfqpoint{0.000000in}{0.000000in}}{%
\pgfpathmoveto{\pgfqpoint{0.000000in}{0.000000in}}%
\pgfpathlineto{\pgfqpoint{-0.048611in}{0.000000in}}%
\pgfusepath{stroke,fill}%
}%
\begin{pgfscope}%
\pgfsys@transformshift{1.432000in}{3.466320in}%
\pgfsys@useobject{currentmarker}{}%
\end{pgfscope}%
\end{pgfscope}%
\begin{pgfscope}%
\pgftext[x=1.195888in,y=3.418126in,left,base]{\rmfamily\fontsize{10.000000}{12.000000}\selectfont \(\displaystyle 20\)}%
\end{pgfscope}%
\begin{pgfscope}%
\pgfsetbuttcap%
\pgfsetroundjoin%
\definecolor{currentfill}{rgb}{0.000000,0.000000,0.000000}%
\pgfsetfillcolor{currentfill}%
\pgfsetlinewidth{0.803000pt}%
\definecolor{currentstroke}{rgb}{0.000000,0.000000,0.000000}%
\pgfsetstrokecolor{currentstroke}%
\pgfsetdash{}{0pt}%
\pgfsys@defobject{currentmarker}{\pgfqpoint{-0.048611in}{0.000000in}}{\pgfqpoint{0.000000in}{0.000000in}}{%
\pgfpathmoveto{\pgfqpoint{0.000000in}{0.000000in}}%
\pgfpathlineto{\pgfqpoint{-0.048611in}{0.000000in}}%
\pgfusepath{stroke,fill}%
}%
\begin{pgfscope}%
\pgfsys@transformshift{1.432000in}{2.727120in}%
\pgfsys@useobject{currentmarker}{}%
\end{pgfscope}%
\end{pgfscope}%
\begin{pgfscope}%
\pgftext[x=1.195888in,y=2.678926in,left,base]{\rmfamily\fontsize{10.000000}{12.000000}\selectfont \(\displaystyle 40\)}%
\end{pgfscope}%
\begin{pgfscope}%
\pgfsetbuttcap%
\pgfsetroundjoin%
\definecolor{currentfill}{rgb}{0.000000,0.000000,0.000000}%
\pgfsetfillcolor{currentfill}%
\pgfsetlinewidth{0.803000pt}%
\definecolor{currentstroke}{rgb}{0.000000,0.000000,0.000000}%
\pgfsetstrokecolor{currentstroke}%
\pgfsetdash{}{0pt}%
\pgfsys@defobject{currentmarker}{\pgfqpoint{-0.048611in}{0.000000in}}{\pgfqpoint{0.000000in}{0.000000in}}{%
\pgfpathmoveto{\pgfqpoint{0.000000in}{0.000000in}}%
\pgfpathlineto{\pgfqpoint{-0.048611in}{0.000000in}}%
\pgfusepath{stroke,fill}%
}%
\begin{pgfscope}%
\pgfsys@transformshift{1.432000in}{1.987920in}%
\pgfsys@useobject{currentmarker}{}%
\end{pgfscope}%
\end{pgfscope}%
\begin{pgfscope}%
\pgftext[x=1.195888in,y=1.939726in,left,base]{\rmfamily\fontsize{10.000000}{12.000000}\selectfont \(\displaystyle 60\)}%
\end{pgfscope}%
\begin{pgfscope}%
\pgfsetbuttcap%
\pgfsetroundjoin%
\definecolor{currentfill}{rgb}{0.000000,0.000000,0.000000}%
\pgfsetfillcolor{currentfill}%
\pgfsetlinewidth{0.803000pt}%
\definecolor{currentstroke}{rgb}{0.000000,0.000000,0.000000}%
\pgfsetstrokecolor{currentstroke}%
\pgfsetdash{}{0pt}%
\pgfsys@defobject{currentmarker}{\pgfqpoint{-0.048611in}{0.000000in}}{\pgfqpoint{0.000000in}{0.000000in}}{%
\pgfpathmoveto{\pgfqpoint{0.000000in}{0.000000in}}%
\pgfpathlineto{\pgfqpoint{-0.048611in}{0.000000in}}%
\pgfusepath{stroke,fill}%
}%
\begin{pgfscope}%
\pgfsys@transformshift{1.432000in}{1.248720in}%
\pgfsys@useobject{currentmarker}{}%
\end{pgfscope}%
\end{pgfscope}%
\begin{pgfscope}%
\pgftext[x=1.195888in,y=1.200526in,left,base]{\rmfamily\fontsize{10.000000}{12.000000}\selectfont \(\displaystyle 80\)}%
\end{pgfscope}%
\begin{pgfscope}%
\pgfsetrectcap%
\pgfsetmiterjoin%
\pgfsetlinewidth{0.803000pt}%
\definecolor{currentstroke}{rgb}{0.000000,0.000000,0.000000}%
\pgfsetstrokecolor{currentstroke}%
\pgfsetdash{}{0pt}%
\pgfpathmoveto{\pgfqpoint{1.432000in}{0.528000in}}%
\pgfpathlineto{\pgfqpoint{1.432000in}{4.224000in}}%
\pgfusepath{stroke}%
\end{pgfscope}%
\begin{pgfscope}%
\pgfsetrectcap%
\pgfsetmiterjoin%
\pgfsetlinewidth{0.803000pt}%
\definecolor{currentstroke}{rgb}{0.000000,0.000000,0.000000}%
\pgfsetstrokecolor{currentstroke}%
\pgfsetdash{}{0pt}%
\pgfpathmoveto{\pgfqpoint{5.128000in}{0.528000in}}%
\pgfpathlineto{\pgfqpoint{5.128000in}{4.224000in}}%
\pgfusepath{stroke}%
\end{pgfscope}%
\begin{pgfscope}%
\pgfsetrectcap%
\pgfsetmiterjoin%
\pgfsetlinewidth{0.803000pt}%
\definecolor{currentstroke}{rgb}{0.000000,0.000000,0.000000}%
\pgfsetstrokecolor{currentstroke}%
\pgfsetdash{}{0pt}%
\pgfpathmoveto{\pgfqpoint{1.432000in}{0.528000in}}%
\pgfpathlineto{\pgfqpoint{5.128000in}{0.528000in}}%
\pgfusepath{stroke}%
\end{pgfscope}%
\begin{pgfscope}%
\pgfsetrectcap%
\pgfsetmiterjoin%
\pgfsetlinewidth{0.803000pt}%
\definecolor{currentstroke}{rgb}{0.000000,0.000000,0.000000}%
\pgfsetstrokecolor{currentstroke}%
\pgfsetdash{}{0pt}%
\pgfpathmoveto{\pgfqpoint{1.432000in}{4.224000in}}%
\pgfpathlineto{\pgfqpoint{5.128000in}{4.224000in}}%
\pgfusepath{stroke}%
\end{pgfscope}%
\end{pgfpicture}%
\makeatother%
\endgroup%
}
\hspace{-0.8cm}
\scalebox{0.25}{%% Creator: Matplotlib, PGF backend
%%
%% To include the figure in your LaTeX document, write
%%   \input{<filename>.pgf}
%%
%% Make sure the required packages are loaded in your preamble
%%   \usepackage{pgf}
%%
%% Figures using additional raster images can only be included by \input if
%% they are in the same directory as the main LaTeX file. For loading figures
%% from other directories you can use the `import` package
%%   \usepackage{import}
%% and then include the figures with
%%   \import{<path to file>}{<filename>.pgf}
%%
%% Matplotlib used the following preamble
%%   \usepackage{fontspec}
%%
\begingroup%
\makeatletter%
\begin{pgfpicture}%
\pgfpathrectangle{\pgfpointorigin}{\pgfqpoint{6.400000in}{4.800000in}}%
\pgfusepath{use as bounding box, clip}%
\begin{pgfscope}%
\pgfsetbuttcap%
\pgfsetmiterjoin%
\definecolor{currentfill}{rgb}{1.000000,1.000000,1.000000}%
\pgfsetfillcolor{currentfill}%
\pgfsetlinewidth{0.000000pt}%
\definecolor{currentstroke}{rgb}{1.000000,1.000000,1.000000}%
\pgfsetstrokecolor{currentstroke}%
\pgfsetdash{}{0pt}%
\pgfpathmoveto{\pgfqpoint{0.000000in}{0.000000in}}%
\pgfpathlineto{\pgfqpoint{6.400000in}{0.000000in}}%
\pgfpathlineto{\pgfqpoint{6.400000in}{4.800000in}}%
\pgfpathlineto{\pgfqpoint{0.000000in}{4.800000in}}%
\pgfpathclose%
\pgfusepath{fill}%
\end{pgfscope}%
\begin{pgfscope}%
\pgfsetbuttcap%
\pgfsetmiterjoin%
\definecolor{currentfill}{rgb}{1.000000,1.000000,1.000000}%
\pgfsetfillcolor{currentfill}%
\pgfsetlinewidth{0.000000pt}%
\definecolor{currentstroke}{rgb}{0.000000,0.000000,0.000000}%
\pgfsetstrokecolor{currentstroke}%
\pgfsetstrokeopacity{0.000000}%
\pgfsetdash{}{0pt}%
\pgfpathmoveto{\pgfqpoint{1.432000in}{0.528000in}}%
\pgfpathlineto{\pgfqpoint{5.128000in}{0.528000in}}%
\pgfpathlineto{\pgfqpoint{5.128000in}{4.224000in}}%
\pgfpathlineto{\pgfqpoint{1.432000in}{4.224000in}}%
\pgfpathclose%
\pgfusepath{fill}%
\end{pgfscope}%
\begin{pgfscope}%
\pgfpathrectangle{\pgfqpoint{1.432000in}{0.528000in}}{\pgfqpoint{3.696000in}{3.696000in}} %
\pgfusepath{clip}%
\pgfsys@transformshift{1.432000in}{0.528000in}%
\pgftext[left,bottom]{\pgfimage[interpolate=true,width=3.700000in,height=3.700000in]{Figure-0007-20180109-013537-414753-img0.png}}%
\end{pgfscope}%
\begin{pgfscope}%
\pgfsetbuttcap%
\pgfsetroundjoin%
\definecolor{currentfill}{rgb}{0.000000,0.000000,0.000000}%
\pgfsetfillcolor{currentfill}%
\pgfsetlinewidth{0.803000pt}%
\definecolor{currentstroke}{rgb}{0.000000,0.000000,0.000000}%
\pgfsetstrokecolor{currentstroke}%
\pgfsetdash{}{0pt}%
\pgfsys@defobject{currentmarker}{\pgfqpoint{0.000000in}{-0.048611in}}{\pgfqpoint{0.000000in}{0.000000in}}{%
\pgfpathmoveto{\pgfqpoint{0.000000in}{0.000000in}}%
\pgfpathlineto{\pgfqpoint{0.000000in}{-0.048611in}}%
\pgfusepath{stroke,fill}%
}%
\begin{pgfscope}%
\pgfsys@transformshift{1.450480in}{0.528000in}%
\pgfsys@useobject{currentmarker}{}%
\end{pgfscope}%
\end{pgfscope}%
\begin{pgfscope}%
\pgftext[x=1.450480in,y=0.430778in,,top]{\rmfamily\fontsize{10.000000}{12.000000}\selectfont \(\displaystyle 0\)}%
\end{pgfscope}%
\begin{pgfscope}%
\pgfsetbuttcap%
\pgfsetroundjoin%
\definecolor{currentfill}{rgb}{0.000000,0.000000,0.000000}%
\pgfsetfillcolor{currentfill}%
\pgfsetlinewidth{0.803000pt}%
\definecolor{currentstroke}{rgb}{0.000000,0.000000,0.000000}%
\pgfsetstrokecolor{currentstroke}%
\pgfsetdash{}{0pt}%
\pgfsys@defobject{currentmarker}{\pgfqpoint{0.000000in}{-0.048611in}}{\pgfqpoint{0.000000in}{0.000000in}}{%
\pgfpathmoveto{\pgfqpoint{0.000000in}{0.000000in}}%
\pgfpathlineto{\pgfqpoint{0.000000in}{-0.048611in}}%
\pgfusepath{stroke,fill}%
}%
\begin{pgfscope}%
\pgfsys@transformshift{2.189680in}{0.528000in}%
\pgfsys@useobject{currentmarker}{}%
\end{pgfscope}%
\end{pgfscope}%
\begin{pgfscope}%
\pgftext[x=2.189680in,y=0.430778in,,top]{\rmfamily\fontsize{10.000000}{12.000000}\selectfont \(\displaystyle 20\)}%
\end{pgfscope}%
\begin{pgfscope}%
\pgfsetbuttcap%
\pgfsetroundjoin%
\definecolor{currentfill}{rgb}{0.000000,0.000000,0.000000}%
\pgfsetfillcolor{currentfill}%
\pgfsetlinewidth{0.803000pt}%
\definecolor{currentstroke}{rgb}{0.000000,0.000000,0.000000}%
\pgfsetstrokecolor{currentstroke}%
\pgfsetdash{}{0pt}%
\pgfsys@defobject{currentmarker}{\pgfqpoint{0.000000in}{-0.048611in}}{\pgfqpoint{0.000000in}{0.000000in}}{%
\pgfpathmoveto{\pgfqpoint{0.000000in}{0.000000in}}%
\pgfpathlineto{\pgfqpoint{0.000000in}{-0.048611in}}%
\pgfusepath{stroke,fill}%
}%
\begin{pgfscope}%
\pgfsys@transformshift{2.928880in}{0.528000in}%
\pgfsys@useobject{currentmarker}{}%
\end{pgfscope}%
\end{pgfscope}%
\begin{pgfscope}%
\pgftext[x=2.928880in,y=0.430778in,,top]{\rmfamily\fontsize{10.000000}{12.000000}\selectfont \(\displaystyle 40\)}%
\end{pgfscope}%
\begin{pgfscope}%
\pgfsetbuttcap%
\pgfsetroundjoin%
\definecolor{currentfill}{rgb}{0.000000,0.000000,0.000000}%
\pgfsetfillcolor{currentfill}%
\pgfsetlinewidth{0.803000pt}%
\definecolor{currentstroke}{rgb}{0.000000,0.000000,0.000000}%
\pgfsetstrokecolor{currentstroke}%
\pgfsetdash{}{0pt}%
\pgfsys@defobject{currentmarker}{\pgfqpoint{0.000000in}{-0.048611in}}{\pgfqpoint{0.000000in}{0.000000in}}{%
\pgfpathmoveto{\pgfqpoint{0.000000in}{0.000000in}}%
\pgfpathlineto{\pgfqpoint{0.000000in}{-0.048611in}}%
\pgfusepath{stroke,fill}%
}%
\begin{pgfscope}%
\pgfsys@transformshift{3.668080in}{0.528000in}%
\pgfsys@useobject{currentmarker}{}%
\end{pgfscope}%
\end{pgfscope}%
\begin{pgfscope}%
\pgftext[x=3.668080in,y=0.430778in,,top]{\rmfamily\fontsize{10.000000}{12.000000}\selectfont \(\displaystyle 60\)}%
\end{pgfscope}%
\begin{pgfscope}%
\pgfsetbuttcap%
\pgfsetroundjoin%
\definecolor{currentfill}{rgb}{0.000000,0.000000,0.000000}%
\pgfsetfillcolor{currentfill}%
\pgfsetlinewidth{0.803000pt}%
\definecolor{currentstroke}{rgb}{0.000000,0.000000,0.000000}%
\pgfsetstrokecolor{currentstroke}%
\pgfsetdash{}{0pt}%
\pgfsys@defobject{currentmarker}{\pgfqpoint{0.000000in}{-0.048611in}}{\pgfqpoint{0.000000in}{0.000000in}}{%
\pgfpathmoveto{\pgfqpoint{0.000000in}{0.000000in}}%
\pgfpathlineto{\pgfqpoint{0.000000in}{-0.048611in}}%
\pgfusepath{stroke,fill}%
}%
\begin{pgfscope}%
\pgfsys@transformshift{4.407280in}{0.528000in}%
\pgfsys@useobject{currentmarker}{}%
\end{pgfscope}%
\end{pgfscope}%
\begin{pgfscope}%
\pgftext[x=4.407280in,y=0.430778in,,top]{\rmfamily\fontsize{10.000000}{12.000000}\selectfont \(\displaystyle 80\)}%
\end{pgfscope}%
\begin{pgfscope}%
\pgfsetbuttcap%
\pgfsetroundjoin%
\definecolor{currentfill}{rgb}{0.000000,0.000000,0.000000}%
\pgfsetfillcolor{currentfill}%
\pgfsetlinewidth{0.803000pt}%
\definecolor{currentstroke}{rgb}{0.000000,0.000000,0.000000}%
\pgfsetstrokecolor{currentstroke}%
\pgfsetdash{}{0pt}%
\pgfsys@defobject{currentmarker}{\pgfqpoint{-0.048611in}{0.000000in}}{\pgfqpoint{0.000000in}{0.000000in}}{%
\pgfpathmoveto{\pgfqpoint{0.000000in}{0.000000in}}%
\pgfpathlineto{\pgfqpoint{-0.048611in}{0.000000in}}%
\pgfusepath{stroke,fill}%
}%
\begin{pgfscope}%
\pgfsys@transformshift{1.432000in}{4.205520in}%
\pgfsys@useobject{currentmarker}{}%
\end{pgfscope}%
\end{pgfscope}%
\begin{pgfscope}%
\pgftext[x=1.265333in,y=4.157326in,left,base]{\rmfamily\fontsize{10.000000}{12.000000}\selectfont \(\displaystyle 0\)}%
\end{pgfscope}%
\begin{pgfscope}%
\pgfsetbuttcap%
\pgfsetroundjoin%
\definecolor{currentfill}{rgb}{0.000000,0.000000,0.000000}%
\pgfsetfillcolor{currentfill}%
\pgfsetlinewidth{0.803000pt}%
\definecolor{currentstroke}{rgb}{0.000000,0.000000,0.000000}%
\pgfsetstrokecolor{currentstroke}%
\pgfsetdash{}{0pt}%
\pgfsys@defobject{currentmarker}{\pgfqpoint{-0.048611in}{0.000000in}}{\pgfqpoint{0.000000in}{0.000000in}}{%
\pgfpathmoveto{\pgfqpoint{0.000000in}{0.000000in}}%
\pgfpathlineto{\pgfqpoint{-0.048611in}{0.000000in}}%
\pgfusepath{stroke,fill}%
}%
\begin{pgfscope}%
\pgfsys@transformshift{1.432000in}{3.466320in}%
\pgfsys@useobject{currentmarker}{}%
\end{pgfscope}%
\end{pgfscope}%
\begin{pgfscope}%
\pgftext[x=1.195888in,y=3.418126in,left,base]{\rmfamily\fontsize{10.000000}{12.000000}\selectfont \(\displaystyle 20\)}%
\end{pgfscope}%
\begin{pgfscope}%
\pgfsetbuttcap%
\pgfsetroundjoin%
\definecolor{currentfill}{rgb}{0.000000,0.000000,0.000000}%
\pgfsetfillcolor{currentfill}%
\pgfsetlinewidth{0.803000pt}%
\definecolor{currentstroke}{rgb}{0.000000,0.000000,0.000000}%
\pgfsetstrokecolor{currentstroke}%
\pgfsetdash{}{0pt}%
\pgfsys@defobject{currentmarker}{\pgfqpoint{-0.048611in}{0.000000in}}{\pgfqpoint{0.000000in}{0.000000in}}{%
\pgfpathmoveto{\pgfqpoint{0.000000in}{0.000000in}}%
\pgfpathlineto{\pgfqpoint{-0.048611in}{0.000000in}}%
\pgfusepath{stroke,fill}%
}%
\begin{pgfscope}%
\pgfsys@transformshift{1.432000in}{2.727120in}%
\pgfsys@useobject{currentmarker}{}%
\end{pgfscope}%
\end{pgfscope}%
\begin{pgfscope}%
\pgftext[x=1.195888in,y=2.678926in,left,base]{\rmfamily\fontsize{10.000000}{12.000000}\selectfont \(\displaystyle 40\)}%
\end{pgfscope}%
\begin{pgfscope}%
\pgfsetbuttcap%
\pgfsetroundjoin%
\definecolor{currentfill}{rgb}{0.000000,0.000000,0.000000}%
\pgfsetfillcolor{currentfill}%
\pgfsetlinewidth{0.803000pt}%
\definecolor{currentstroke}{rgb}{0.000000,0.000000,0.000000}%
\pgfsetstrokecolor{currentstroke}%
\pgfsetdash{}{0pt}%
\pgfsys@defobject{currentmarker}{\pgfqpoint{-0.048611in}{0.000000in}}{\pgfqpoint{0.000000in}{0.000000in}}{%
\pgfpathmoveto{\pgfqpoint{0.000000in}{0.000000in}}%
\pgfpathlineto{\pgfqpoint{-0.048611in}{0.000000in}}%
\pgfusepath{stroke,fill}%
}%
\begin{pgfscope}%
\pgfsys@transformshift{1.432000in}{1.987920in}%
\pgfsys@useobject{currentmarker}{}%
\end{pgfscope}%
\end{pgfscope}%
\begin{pgfscope}%
\pgftext[x=1.195888in,y=1.939726in,left,base]{\rmfamily\fontsize{10.000000}{12.000000}\selectfont \(\displaystyle 60\)}%
\end{pgfscope}%
\begin{pgfscope}%
\pgfsetbuttcap%
\pgfsetroundjoin%
\definecolor{currentfill}{rgb}{0.000000,0.000000,0.000000}%
\pgfsetfillcolor{currentfill}%
\pgfsetlinewidth{0.803000pt}%
\definecolor{currentstroke}{rgb}{0.000000,0.000000,0.000000}%
\pgfsetstrokecolor{currentstroke}%
\pgfsetdash{}{0pt}%
\pgfsys@defobject{currentmarker}{\pgfqpoint{-0.048611in}{0.000000in}}{\pgfqpoint{0.000000in}{0.000000in}}{%
\pgfpathmoveto{\pgfqpoint{0.000000in}{0.000000in}}%
\pgfpathlineto{\pgfqpoint{-0.048611in}{0.000000in}}%
\pgfusepath{stroke,fill}%
}%
\begin{pgfscope}%
\pgfsys@transformshift{1.432000in}{1.248720in}%
\pgfsys@useobject{currentmarker}{}%
\end{pgfscope}%
\end{pgfscope}%
\begin{pgfscope}%
\pgftext[x=1.195888in,y=1.200526in,left,base]{\rmfamily\fontsize{10.000000}{12.000000}\selectfont \(\displaystyle 80\)}%
\end{pgfscope}%
\begin{pgfscope}%
\pgfsetrectcap%
\pgfsetmiterjoin%
\pgfsetlinewidth{0.803000pt}%
\definecolor{currentstroke}{rgb}{0.000000,0.000000,0.000000}%
\pgfsetstrokecolor{currentstroke}%
\pgfsetdash{}{0pt}%
\pgfpathmoveto{\pgfqpoint{1.432000in}{0.528000in}}%
\pgfpathlineto{\pgfqpoint{1.432000in}{4.224000in}}%
\pgfusepath{stroke}%
\end{pgfscope}%
\begin{pgfscope}%
\pgfsetrectcap%
\pgfsetmiterjoin%
\pgfsetlinewidth{0.803000pt}%
\definecolor{currentstroke}{rgb}{0.000000,0.000000,0.000000}%
\pgfsetstrokecolor{currentstroke}%
\pgfsetdash{}{0pt}%
\pgfpathmoveto{\pgfqpoint{5.128000in}{0.528000in}}%
\pgfpathlineto{\pgfqpoint{5.128000in}{4.224000in}}%
\pgfusepath{stroke}%
\end{pgfscope}%
\begin{pgfscope}%
\pgfsetrectcap%
\pgfsetmiterjoin%
\pgfsetlinewidth{0.803000pt}%
\definecolor{currentstroke}{rgb}{0.000000,0.000000,0.000000}%
\pgfsetstrokecolor{currentstroke}%
\pgfsetdash{}{0pt}%
\pgfpathmoveto{\pgfqpoint{1.432000in}{0.528000in}}%
\pgfpathlineto{\pgfqpoint{5.128000in}{0.528000in}}%
\pgfusepath{stroke}%
\end{pgfscope}%
\begin{pgfscope}%
\pgfsetrectcap%
\pgfsetmiterjoin%
\pgfsetlinewidth{0.803000pt}%
\definecolor{currentstroke}{rgb}{0.000000,0.000000,0.000000}%
\pgfsetstrokecolor{currentstroke}%
\pgfsetdash{}{0pt}%
\pgfpathmoveto{\pgfqpoint{1.432000in}{4.224000in}}%
\pgfpathlineto{\pgfqpoint{5.128000in}{4.224000in}}%
\pgfusepath{stroke}%
\end{pgfscope}%
\end{pgfpicture}%
\makeatother%
\endgroup%
}
\hspace{-0.8cm} 
\scalebox{0.25}{%% Creator: Matplotlib, PGF backend
%%
%% To include the figure in your LaTeX document, write
%%   \input{<filename>.pgf}
%%
%% Make sure the required packages are loaded in your preamble
%%   \usepackage{pgf}
%%
%% Figures using additional raster images can only be included by \input if
%% they are in the same directory as the main LaTeX file. For loading figures
%% from other directories you can use the `import` package
%%   \usepackage{import}
%% and then include the figures with
%%   \import{<path to file>}{<filename>.pgf}
%%
%% Matplotlib used the following preamble
%%   \usepackage{fontspec}
%%
\begingroup%
\makeatletter%
\begin{pgfpicture}%
\pgfpathrectangle{\pgfpointorigin}{\pgfqpoint{6.400000in}{4.800000in}}%
\pgfusepath{use as bounding box, clip}%
\begin{pgfscope}%
\pgfsetbuttcap%
\pgfsetmiterjoin%
\definecolor{currentfill}{rgb}{1.000000,1.000000,1.000000}%
\pgfsetfillcolor{currentfill}%
\pgfsetlinewidth{0.000000pt}%
\definecolor{currentstroke}{rgb}{1.000000,1.000000,1.000000}%
\pgfsetstrokecolor{currentstroke}%
\pgfsetdash{}{0pt}%
\pgfpathmoveto{\pgfqpoint{0.000000in}{0.000000in}}%
\pgfpathlineto{\pgfqpoint{6.400000in}{0.000000in}}%
\pgfpathlineto{\pgfqpoint{6.400000in}{4.800000in}}%
\pgfpathlineto{\pgfqpoint{0.000000in}{4.800000in}}%
\pgfpathclose%
\pgfusepath{fill}%
\end{pgfscope}%
\begin{pgfscope}%
\pgfsetbuttcap%
\pgfsetmiterjoin%
\definecolor{currentfill}{rgb}{1.000000,1.000000,1.000000}%
\pgfsetfillcolor{currentfill}%
\pgfsetlinewidth{0.000000pt}%
\definecolor{currentstroke}{rgb}{0.000000,0.000000,0.000000}%
\pgfsetstrokecolor{currentstroke}%
\pgfsetstrokeopacity{0.000000}%
\pgfsetdash{}{0pt}%
\pgfpathmoveto{\pgfqpoint{1.432000in}{0.528000in}}%
\pgfpathlineto{\pgfqpoint{5.128000in}{0.528000in}}%
\pgfpathlineto{\pgfqpoint{5.128000in}{4.224000in}}%
\pgfpathlineto{\pgfqpoint{1.432000in}{4.224000in}}%
\pgfpathclose%
\pgfusepath{fill}%
\end{pgfscope}%
\begin{pgfscope}%
\pgfpathrectangle{\pgfqpoint{1.432000in}{0.528000in}}{\pgfqpoint{3.696000in}{3.696000in}} %
\pgfusepath{clip}%
\pgfsys@transformshift{1.432000in}{0.528000in}%
\pgftext[left,bottom]{\pgfimage[interpolate=true,width=3.700000in,height=3.700000in]{Figure-0008-20180109-014255-034295-img0.png}}%
\end{pgfscope}%
\begin{pgfscope}%
\pgfsetbuttcap%
\pgfsetroundjoin%
\definecolor{currentfill}{rgb}{0.000000,0.000000,0.000000}%
\pgfsetfillcolor{currentfill}%
\pgfsetlinewidth{0.803000pt}%
\definecolor{currentstroke}{rgb}{0.000000,0.000000,0.000000}%
\pgfsetstrokecolor{currentstroke}%
\pgfsetdash{}{0pt}%
\pgfsys@defobject{currentmarker}{\pgfqpoint{0.000000in}{-0.048611in}}{\pgfqpoint{0.000000in}{0.000000in}}{%
\pgfpathmoveto{\pgfqpoint{0.000000in}{0.000000in}}%
\pgfpathlineto{\pgfqpoint{0.000000in}{-0.048611in}}%
\pgfusepath{stroke,fill}%
}%
\begin{pgfscope}%
\pgfsys@transformshift{1.450480in}{0.528000in}%
\pgfsys@useobject{currentmarker}{}%
\end{pgfscope}%
\end{pgfscope}%
\begin{pgfscope}%
\pgftext[x=1.450480in,y=0.430778in,,top]{\rmfamily\fontsize{10.000000}{12.000000}\selectfont \(\displaystyle 0\)}%
\end{pgfscope}%
\begin{pgfscope}%
\pgfsetbuttcap%
\pgfsetroundjoin%
\definecolor{currentfill}{rgb}{0.000000,0.000000,0.000000}%
\pgfsetfillcolor{currentfill}%
\pgfsetlinewidth{0.803000pt}%
\definecolor{currentstroke}{rgb}{0.000000,0.000000,0.000000}%
\pgfsetstrokecolor{currentstroke}%
\pgfsetdash{}{0pt}%
\pgfsys@defobject{currentmarker}{\pgfqpoint{0.000000in}{-0.048611in}}{\pgfqpoint{0.000000in}{0.000000in}}{%
\pgfpathmoveto{\pgfqpoint{0.000000in}{0.000000in}}%
\pgfpathlineto{\pgfqpoint{0.000000in}{-0.048611in}}%
\pgfusepath{stroke,fill}%
}%
\begin{pgfscope}%
\pgfsys@transformshift{2.189680in}{0.528000in}%
\pgfsys@useobject{currentmarker}{}%
\end{pgfscope}%
\end{pgfscope}%
\begin{pgfscope}%
\pgftext[x=2.189680in,y=0.430778in,,top]{\rmfamily\fontsize{10.000000}{12.000000}\selectfont \(\displaystyle 20\)}%
\end{pgfscope}%
\begin{pgfscope}%
\pgfsetbuttcap%
\pgfsetroundjoin%
\definecolor{currentfill}{rgb}{0.000000,0.000000,0.000000}%
\pgfsetfillcolor{currentfill}%
\pgfsetlinewidth{0.803000pt}%
\definecolor{currentstroke}{rgb}{0.000000,0.000000,0.000000}%
\pgfsetstrokecolor{currentstroke}%
\pgfsetdash{}{0pt}%
\pgfsys@defobject{currentmarker}{\pgfqpoint{0.000000in}{-0.048611in}}{\pgfqpoint{0.000000in}{0.000000in}}{%
\pgfpathmoveto{\pgfqpoint{0.000000in}{0.000000in}}%
\pgfpathlineto{\pgfqpoint{0.000000in}{-0.048611in}}%
\pgfusepath{stroke,fill}%
}%
\begin{pgfscope}%
\pgfsys@transformshift{2.928880in}{0.528000in}%
\pgfsys@useobject{currentmarker}{}%
\end{pgfscope}%
\end{pgfscope}%
\begin{pgfscope}%
\pgftext[x=2.928880in,y=0.430778in,,top]{\rmfamily\fontsize{10.000000}{12.000000}\selectfont \(\displaystyle 40\)}%
\end{pgfscope}%
\begin{pgfscope}%
\pgfsetbuttcap%
\pgfsetroundjoin%
\definecolor{currentfill}{rgb}{0.000000,0.000000,0.000000}%
\pgfsetfillcolor{currentfill}%
\pgfsetlinewidth{0.803000pt}%
\definecolor{currentstroke}{rgb}{0.000000,0.000000,0.000000}%
\pgfsetstrokecolor{currentstroke}%
\pgfsetdash{}{0pt}%
\pgfsys@defobject{currentmarker}{\pgfqpoint{0.000000in}{-0.048611in}}{\pgfqpoint{0.000000in}{0.000000in}}{%
\pgfpathmoveto{\pgfqpoint{0.000000in}{0.000000in}}%
\pgfpathlineto{\pgfqpoint{0.000000in}{-0.048611in}}%
\pgfusepath{stroke,fill}%
}%
\begin{pgfscope}%
\pgfsys@transformshift{3.668080in}{0.528000in}%
\pgfsys@useobject{currentmarker}{}%
\end{pgfscope}%
\end{pgfscope}%
\begin{pgfscope}%
\pgftext[x=3.668080in,y=0.430778in,,top]{\rmfamily\fontsize{10.000000}{12.000000}\selectfont \(\displaystyle 60\)}%
\end{pgfscope}%
\begin{pgfscope}%
\pgfsetbuttcap%
\pgfsetroundjoin%
\definecolor{currentfill}{rgb}{0.000000,0.000000,0.000000}%
\pgfsetfillcolor{currentfill}%
\pgfsetlinewidth{0.803000pt}%
\definecolor{currentstroke}{rgb}{0.000000,0.000000,0.000000}%
\pgfsetstrokecolor{currentstroke}%
\pgfsetdash{}{0pt}%
\pgfsys@defobject{currentmarker}{\pgfqpoint{0.000000in}{-0.048611in}}{\pgfqpoint{0.000000in}{0.000000in}}{%
\pgfpathmoveto{\pgfqpoint{0.000000in}{0.000000in}}%
\pgfpathlineto{\pgfqpoint{0.000000in}{-0.048611in}}%
\pgfusepath{stroke,fill}%
}%
\begin{pgfscope}%
\pgfsys@transformshift{4.407280in}{0.528000in}%
\pgfsys@useobject{currentmarker}{}%
\end{pgfscope}%
\end{pgfscope}%
\begin{pgfscope}%
\pgftext[x=4.407280in,y=0.430778in,,top]{\rmfamily\fontsize{10.000000}{12.000000}\selectfont \(\displaystyle 80\)}%
\end{pgfscope}%
\begin{pgfscope}%
\pgfsetbuttcap%
\pgfsetroundjoin%
\definecolor{currentfill}{rgb}{0.000000,0.000000,0.000000}%
\pgfsetfillcolor{currentfill}%
\pgfsetlinewidth{0.803000pt}%
\definecolor{currentstroke}{rgb}{0.000000,0.000000,0.000000}%
\pgfsetstrokecolor{currentstroke}%
\pgfsetdash{}{0pt}%
\pgfsys@defobject{currentmarker}{\pgfqpoint{-0.048611in}{0.000000in}}{\pgfqpoint{0.000000in}{0.000000in}}{%
\pgfpathmoveto{\pgfqpoint{0.000000in}{0.000000in}}%
\pgfpathlineto{\pgfqpoint{-0.048611in}{0.000000in}}%
\pgfusepath{stroke,fill}%
}%
\begin{pgfscope}%
\pgfsys@transformshift{1.432000in}{4.205520in}%
\pgfsys@useobject{currentmarker}{}%
\end{pgfscope}%
\end{pgfscope}%
\begin{pgfscope}%
\pgftext[x=1.265333in,y=4.157326in,left,base]{\rmfamily\fontsize{10.000000}{12.000000}\selectfont \(\displaystyle 0\)}%
\end{pgfscope}%
\begin{pgfscope}%
\pgfsetbuttcap%
\pgfsetroundjoin%
\definecolor{currentfill}{rgb}{0.000000,0.000000,0.000000}%
\pgfsetfillcolor{currentfill}%
\pgfsetlinewidth{0.803000pt}%
\definecolor{currentstroke}{rgb}{0.000000,0.000000,0.000000}%
\pgfsetstrokecolor{currentstroke}%
\pgfsetdash{}{0pt}%
\pgfsys@defobject{currentmarker}{\pgfqpoint{-0.048611in}{0.000000in}}{\pgfqpoint{0.000000in}{0.000000in}}{%
\pgfpathmoveto{\pgfqpoint{0.000000in}{0.000000in}}%
\pgfpathlineto{\pgfqpoint{-0.048611in}{0.000000in}}%
\pgfusepath{stroke,fill}%
}%
\begin{pgfscope}%
\pgfsys@transformshift{1.432000in}{3.466320in}%
\pgfsys@useobject{currentmarker}{}%
\end{pgfscope}%
\end{pgfscope}%
\begin{pgfscope}%
\pgftext[x=1.195888in,y=3.418126in,left,base]{\rmfamily\fontsize{10.000000}{12.000000}\selectfont \(\displaystyle 20\)}%
\end{pgfscope}%
\begin{pgfscope}%
\pgfsetbuttcap%
\pgfsetroundjoin%
\definecolor{currentfill}{rgb}{0.000000,0.000000,0.000000}%
\pgfsetfillcolor{currentfill}%
\pgfsetlinewidth{0.803000pt}%
\definecolor{currentstroke}{rgb}{0.000000,0.000000,0.000000}%
\pgfsetstrokecolor{currentstroke}%
\pgfsetdash{}{0pt}%
\pgfsys@defobject{currentmarker}{\pgfqpoint{-0.048611in}{0.000000in}}{\pgfqpoint{0.000000in}{0.000000in}}{%
\pgfpathmoveto{\pgfqpoint{0.000000in}{0.000000in}}%
\pgfpathlineto{\pgfqpoint{-0.048611in}{0.000000in}}%
\pgfusepath{stroke,fill}%
}%
\begin{pgfscope}%
\pgfsys@transformshift{1.432000in}{2.727120in}%
\pgfsys@useobject{currentmarker}{}%
\end{pgfscope}%
\end{pgfscope}%
\begin{pgfscope}%
\pgftext[x=1.195888in,y=2.678926in,left,base]{\rmfamily\fontsize{10.000000}{12.000000}\selectfont \(\displaystyle 40\)}%
\end{pgfscope}%
\begin{pgfscope}%
\pgfsetbuttcap%
\pgfsetroundjoin%
\definecolor{currentfill}{rgb}{0.000000,0.000000,0.000000}%
\pgfsetfillcolor{currentfill}%
\pgfsetlinewidth{0.803000pt}%
\definecolor{currentstroke}{rgb}{0.000000,0.000000,0.000000}%
\pgfsetstrokecolor{currentstroke}%
\pgfsetdash{}{0pt}%
\pgfsys@defobject{currentmarker}{\pgfqpoint{-0.048611in}{0.000000in}}{\pgfqpoint{0.000000in}{0.000000in}}{%
\pgfpathmoveto{\pgfqpoint{0.000000in}{0.000000in}}%
\pgfpathlineto{\pgfqpoint{-0.048611in}{0.000000in}}%
\pgfusepath{stroke,fill}%
}%
\begin{pgfscope}%
\pgfsys@transformshift{1.432000in}{1.987920in}%
\pgfsys@useobject{currentmarker}{}%
\end{pgfscope}%
\end{pgfscope}%
\begin{pgfscope}%
\pgftext[x=1.195888in,y=1.939726in,left,base]{\rmfamily\fontsize{10.000000}{12.000000}\selectfont \(\displaystyle 60\)}%
\end{pgfscope}%
\begin{pgfscope}%
\pgfsetbuttcap%
\pgfsetroundjoin%
\definecolor{currentfill}{rgb}{0.000000,0.000000,0.000000}%
\pgfsetfillcolor{currentfill}%
\pgfsetlinewidth{0.803000pt}%
\definecolor{currentstroke}{rgb}{0.000000,0.000000,0.000000}%
\pgfsetstrokecolor{currentstroke}%
\pgfsetdash{}{0pt}%
\pgfsys@defobject{currentmarker}{\pgfqpoint{-0.048611in}{0.000000in}}{\pgfqpoint{0.000000in}{0.000000in}}{%
\pgfpathmoveto{\pgfqpoint{0.000000in}{0.000000in}}%
\pgfpathlineto{\pgfqpoint{-0.048611in}{0.000000in}}%
\pgfusepath{stroke,fill}%
}%
\begin{pgfscope}%
\pgfsys@transformshift{1.432000in}{1.248720in}%
\pgfsys@useobject{currentmarker}{}%
\end{pgfscope}%
\end{pgfscope}%
\begin{pgfscope}%
\pgftext[x=1.195888in,y=1.200526in,left,base]{\rmfamily\fontsize{10.000000}{12.000000}\selectfont \(\displaystyle 80\)}%
\end{pgfscope}%
\begin{pgfscope}%
\pgfsetrectcap%
\pgfsetmiterjoin%
\pgfsetlinewidth{0.803000pt}%
\definecolor{currentstroke}{rgb}{0.000000,0.000000,0.000000}%
\pgfsetstrokecolor{currentstroke}%
\pgfsetdash{}{0pt}%
\pgfpathmoveto{\pgfqpoint{1.432000in}{0.528000in}}%
\pgfpathlineto{\pgfqpoint{1.432000in}{4.224000in}}%
\pgfusepath{stroke}%
\end{pgfscope}%
\begin{pgfscope}%
\pgfsetrectcap%
\pgfsetmiterjoin%
\pgfsetlinewidth{0.803000pt}%
\definecolor{currentstroke}{rgb}{0.000000,0.000000,0.000000}%
\pgfsetstrokecolor{currentstroke}%
\pgfsetdash{}{0pt}%
\pgfpathmoveto{\pgfqpoint{5.128000in}{0.528000in}}%
\pgfpathlineto{\pgfqpoint{5.128000in}{4.224000in}}%
\pgfusepath{stroke}%
\end{pgfscope}%
\begin{pgfscope}%
\pgfsetrectcap%
\pgfsetmiterjoin%
\pgfsetlinewidth{0.803000pt}%
\definecolor{currentstroke}{rgb}{0.000000,0.000000,0.000000}%
\pgfsetstrokecolor{currentstroke}%
\pgfsetdash{}{0pt}%
\pgfpathmoveto{\pgfqpoint{1.432000in}{0.528000in}}%
\pgfpathlineto{\pgfqpoint{5.128000in}{0.528000in}}%
\pgfusepath{stroke}%
\end{pgfscope}%
\begin{pgfscope}%
\pgfsetrectcap%
\pgfsetmiterjoin%
\pgfsetlinewidth{0.803000pt}%
\definecolor{currentstroke}{rgb}{0.000000,0.000000,0.000000}%
\pgfsetstrokecolor{currentstroke}%
\pgfsetdash{}{0pt}%
\pgfpathmoveto{\pgfqpoint{1.432000in}{4.224000in}}%
\pgfpathlineto{\pgfqpoint{5.128000in}{4.224000in}}%
\pgfusepath{stroke}%
\end{pgfscope}%
\end{pgfpicture}%
\makeatother%
\endgroup%
}
\caption{The influence of the regularization coefficient $\gamma$} \label{Fig:Gamma}
\end{figure}

Using entropy regularization, we may modify Algorithm \ref{Alg:ADMMPrimal} to a entropy regularized version \textbf{(Algorithm \hypertarget{EAlg:1R}{1R})}. The augmented Lagragian is
\begin{equation}
\begin{aligned}
L_{ \rho \gamma } \rbr{ s, \widetilde{s}, \lambda, \eta, e } &= \sume{i}{1}{n}{\sume{j}{1}{m}{ c_{ i j } s_{ i j } }} + \iota_+ \rbr{\widetilde{s}} + \gamma R \rbr{ \widetilde{s} + \delta } \\
&+ \sume{i}{1}{n}{ \lambda_i \rbr{ \mu_i - \sume{j}{1}{m}{s_{ i j }} } } + \sume{j}{1}{m}{ \eta_j \rbr{ \nu_j - \sume{i}{1}{n}{s_{ i j }} } } + \sume{i}{1}{n}{\sume{j}{1}{m}{ e_{ i j } \rbr{ s_{ i j } - \widetilde{s}_{ i j } } }} \\
&+ \frac{\rho}{2} \sume{i}{1}{n}{\rbr{ \mu_i - \sume{j}{1}{m}{s_{ i j }} }^2} + \frac{\rho}{2} \sume{j}{1}{m}{\rbr{ \nu_j - \sume{i}{1}{n}{s_{ i j }} }^2} + \frac{\rho}{2} \sume{i}{1}{n}{\sume{j}{1}{m}{\rbr{ s_{ i j } - \widetilde{s}_{ i j } }^2}}, \\
\end{aligned}
\end{equation}
where $\delta$ is a small number (valued $10^{-6}$ in numerical experiments) to increase numerical stability. However, this algorithm is still rather slow because ADMM is used.

We have also implemented IPFP / Sinkhorn algorithm in \parencite{Benamou2015}. The algorithm is listed as Algorithm \ref{Alg:IPFP}. However, the deficiency of this algorithm lies in numerical instability: the $ \exp \rbr{ -c / \gamma } $ step is rather dangerous for small $\gamma$, which makes it impossible to reach a close solution. Note that a large regularization term leads to a large error.

\begin{algorithm}
\caption{Sinkhorn algorithm}
\label{Alg:IPFP}
\begin{algorithmic}
\REQUIRE $\mu$, $\nu$, $c$, $\gamma$
\STATE $ \xi \slar \exp \rbr{ -c / \gamma } $
\STATE $ t \slar 0 $
\STATE $ v^{\rbr{t}} \slar 1 $
\WHILE{not converges}
\STATE $ u^{\rbr{ t + 1 }}_i \slar \mu_i / \rbr{ \xi v^{\rbr{t}} }_i $
\STATE $ v^{\rbr{ t + 1 }}_j \slar \nu_j / \rbr{ \xi^{\rmut} u^{\rbr{ t + 1 }} }_j $
\STATE $ t \slar t + 1 $
\ENDWHILE
$s^{\rbr{t}}_{ i j } = u^{\rbr{t}}_i \xi_{ i j } v^{\rbr{t}}_j $
\end{algorithmic}
\end{algorithm}

	\end{document}
